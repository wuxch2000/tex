
\usepackage{tikz}
\usetikzlibrary{scopes,fadings,patterns,calc,shapes.geometric}

\usepackage{pgfmath}

\newcommand{\wuxchcal}[3]{%
\begin{tikzpicture}[scale=1,transform shape]


% calendar is :
% a         b
%     X1
% c         d
% c1        d1
%     X2
% e         f

\def\calendaredge{#1}
\def\calendarmiddlelineratio{0.8}

\coordinate (A) at (0,\calendaredge);
\coordinate (B) at (\calendaredge,\calendaredge);
\coordinate (C) at (0,\calendaredge*\calendarmiddlelineratio);
\coordinate (D) at (\calendaredge,\calendaredge*\calendarmiddlelineratio);
\coordinate (E) at (0,0);
\coordinate (F) at (\calendaredge,0);

\coordinate (C1) at ($(C)-0.03*(A)$);
\coordinate (D1) at ($(D)-0.03*(A)$);

\definecolor{calendarupsidebegin}{rgb}{0.9451 0.7137 0.7137}
\definecolor{calendarupsideend}{rgb}{0.7882 0.0078 0.0078}
\definecolor{calendarmiddlelinecolor}{rgb}{0.5 0.1333 0.1333}
\definecolor{calendargray}{gray}{0.92}

\shade[top color=calendarupsidebegin, bottom color=calendarupsideend] 
    {[rounded corners=\calendaredge*2.5](C) -- (A) -- (B)} -- (D) -- cycle;


\fill[color=calendarmiddlelinecolor] (C) -- (D) -- (D1) -- (C1) -- cycle;

\fill[color=calendargray] {[rounded corners=\calendaredge*2.5](C1) -- (E) -- (F)} -- (D1) -- cycle;

\coordinate (X1) at ($0.5*(C1)+0.5*(D1)+0.5*(A)-0.5*(C1)$);
\coordinate (X2) at ($0.5*(E)+0.5*(F)+0.5*(C1)-0.5*(E)$);


\node [scale=#1*0.4,font=\sffamily\bfseries](node-x1) at (X1) {#2};

\node [scale=#1*2,font=\sffamily\bfseries](node-x1) at (X2) {#3};

\end{tikzpicture}
}

%%% Local Variables: 
%%% mode: latex
%%% TeX-master: "testpage"
%%% End: 
