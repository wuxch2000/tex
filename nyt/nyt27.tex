\documentclass[12pt]{article}
\title{Digest of The New York Times}
\author{The New York Times}

\usepackage{config}

% \makeindex
\begin{document}
\date{}
% \thispagestyle{empty}

\tableofcontents\thispagestyle{empty}

\clearpage
\setcounter{page}{1}

\section{U.S. Strains to Stop Arms Flow}

\lettrine{J}{ust}\mycalendar{Dec.'10}{07} a week after President Bashar al-Assad of Syria assured a
top State Department official that his government was not sending sophisticated weapons to
Hezbollah, the Obama administration lodged a confidential protest accusing Syria of doing precisely
what it had denied doing.

``In our meetings last week it was stated that Syria is not transferring any `new' missiles to
Lebanese Hizballah,'' noted a cable sent by Secretary of State Hillary Rodham Clinton in February,
using an alternative spelling for the militant group. ``We are aware, however, of current Syrian
efforts to supply Hizballah with ballistic missiles. I must stress that this activity is of deep
concern to my government, and we strongly caution you against such a serious escalation.''

A senior Syrian Foreign Ministry official, a cable from the American Embassy in Damascus reported,
flatly denied the allegation. But nine months later, administration officials assert, the flow of
arms had continued to Hezbollah. According to a Pentagon official, Hezbollah's arsenal now includes
up to 50,000 rockets and missiles, including some 40 to 50 Fatah-110 missiles capable of reaching
Tel Aviv and most of Israel, and 10 Scud-D missiles. The newly fortified Hezbollah has raised fears
that any future conflict with Israel could erupt into a full-scale regional war.

The Syrian episode offers a glimpse of the United States' efforts to prevent buildups of arms --
including Scud missiles, Soviet-era tanks and antiaircraft weapons -- in some of the world's tensest
regions. Wielding surveillance photos and sales contracts, American diplomats have confronted
foreign governments about shadowy front companies, secretive banks and shippers around the globe,
according to secret State Department cables obtained by WikiLeaks and made available to several news
organizations.

American officials have tried to block a Serbian black marketer from selling sniper rifles to Yemen.
They have sought to disrupt the sale of Chinese missile technology to Pakistan, the cables show, and
questioned Indian officials about chemical industry exports that could be used to make poison gas.

But while American officials can claim some successes -- Russia appears to have deferred delivery of
the S-300 air defense system to Iran -- the diplomats' dispatches underscore how often their efforts
have been frustrated in trying to choke off trade by Syria and others, including Iran and North
Korea.

The United States is the world's largest arms supplier, and with Russia, dominates trade in the
developing world. Its role as a purveyor of weapons to certain allies -- including Israel, Saudi
Arabia and other Persian Gulf states -- has drawn criticism that it has fueled an arms race. But it
has also taken on a leading role as traffic cop in trying to halt deliveries of advanced weapons and
other arms to militants and adversaries.

According to the cables, American diplomats have repeatedly expressed concern that huge cargo planes
operated by Badr Airlines of Sudan were flying weapons from Tehran to Khartoum, Sudan, where they
were shipped to Hamas, the militant group in Gaza.

Sudan insisted that the cargo was farm equipment, but the United States asked countries in the
region to deny overflight rights to the airlines. Jordan and several other countries agreed, but
Yemen declined, a February 2009 cable reported.

Egyptian officials, who view Iran with deep wariness, privately issued a threat. Omar Suleiman, the
chief of Egypt's intelligence service, told Adm. Mike Mullen, the chairman of the Joint Chiefs of
Staff, that Iran not only was providing \$25 million a month to support Hamas but also was linked to
a Hezbollah cell trying to smuggle arms from Gaza into Egypt, according to an April 2009 cable.

``Egypt had sent the clear message to Iran that if it interferes in Egypt, Egypt will interfere in
Iran,'' noted the cable, adding that the Egyptian official said his country had trained agents for
that purpose.

North Korea has abetted the arms race in the Middle East by providing missile technology to Iran and
Syria, which then backed Hamas and Hezbollah, according to American intelligence officials and a
cable from Mrs.~Clinton. The cables tell something of an international detective story: how North
Korea's arms industry has conducted many of its transactions through the Korea Mining Development
Corporation, relied on suppliers of machinery and steel from countries including Switzerland, Japan,
China and Taiwan, passed money through Chinese and Hong Kong banks and sold weapons to other
countries.

To disrupt the transactions, American officials have prodded and protested. Diplomats raised
questions in the spring of 2009, for example, about planned purchases from North Korea of
rocket-propelled launchers by Sri Lanka and Scud missile launchers by Yemen, apparently to no avail.

In July 2009, Stuart A.~Levey, a senior United States Treasury official, warned a top official of
the People's Bank of China that ``Chinese banks have been targeted by North Korea as the main access
point into the international financial system,'' according to one cable. And in meetings in Hong
Kong that month, Mr.~Levey complained that a local businessman was helping procure luxury goods for
the North Korean leadership. (The Hong Kong bank later suggested that it had shut down the man's
accounts.)

It is the arms transactions involving Syria and Hezbollah, however, that appear to be among the
Obama administration's gravest concerns. President Obama came into office pledging to engage with
Syria, arguing that the Bush administration's efforts to isolate Syria had done nothing to wean it
from Iran or encourage Middle East peace efforts.

Even before American diplomats began talks with the Assad government, Senator John Kerry, the
Massachusetts Democrat who is the chairman of the Senate Foreign Relations Committee, prodded
Mr.~Assad in a February 2009 meeting in Damascus to make a gesture that he could take back to the
Obama administration as ``an indicator of Assad's good will.'' Mr.~Kerry told Mr.~Assad that
Mr.~Obama intended to withdraw American troops from Iraq ``as soon as possible'' and also hinted to
a senior Syrian official that the Obama administration intended to take a firm line against the
establishment of new Israeli settlements on the West Bank.

``It is not our goal for the United States to be humiliated,'' Mr.~Assad said, referring to Iraq,
according to a cable.

In March 2009, a delegation of State Department and National Security Council officials traveled to
Damascus for the first discussions, and in the next several months, each side made some modest
gestures.

The United States provided information ``regarding a potential threat to a Syrian official'' through
Syria's Washington ambassador and allowed a senior aide to George J.~Mitchell, the American Middle
East negotiator, to attend an Syrian holiday event at the Syrian Embassy, a cable reported. Syria,
for its part, allowed the Americans to reopen an English-language school and hosted a team of
American military officials to discuss how to better regulate the Syria-Iraq border.

Each side, however, wanted the other to take the first major initiative. Syria kept pressing for the
lifting of economic sanctions, which had crippled its aviation industry, and the Americans urged
Syria to curtail its support for Hezbollah and Hamas.

``The U.S.~had publicly recognized its mistakes, e.g. use of torture methods, and would continue to
take steps,'' Daniel B.~Shapiro, a senior official on the National Security Council told the Syrians
in the meeting, according to a May 2009 cable. ``But others needed to reciprocate to ensure that the
opportunity did not pass.''

By the fall, however, officials at the American Embassy in Damascus appeared concerned that military
developments were outpacing the incremental diplomacy.

``Syria's determined support of Hizballah's military build-up, particularly the steady supply of
longer-range rockets and the introduction of guided missiles could change the military balance and
produce a scenario significantly more destructive than the July-August 2006 war,'' said a November
2009 cable from the American charg\'e d'affaires in Damascus.

According to cables, Syrian leaders appeared to believe that the weapons shipments increased their
political leverage with the Israelis. But they made Lebanon even more of a tinderbox and increased
the prospect that a future conflict might include Syria.

A major worry was that Syria or Iran had provided Hezbollah with Fatah-110 missiles, with the range
to strike Tel Aviv. (A United States government official said last week that the 40 to 50 missiles
were viewed as especially threatening because they are highly accurate.) Israeli officials told
American officials in November 2009 that if war broke out, they assumed that Hezbollah would try to
launch 400 to 600 rockets at day and sustain the attacks for at least 12 months, the cables note.

In February, the White House announced that a new American ambassador would be sent to Syria after a
five-year hiatus. The next day, William J.~Burns, a State Department under secretary, met with the
Syrian leader.

During the session, Mr.~Burns repeated American concerns about weapons smuggling to Hezbollah, one
dispatch noted. Mr.~Assad replied that while he could not be Israel's policeman, no ``new'' weapons
were being sent to Hezbollah.

Soon after the meeting, though, a cable noted that the Americans received intelligence reports that
the Syrians were about to provide Hezbollah with Scud-D missiles, which are based on North Korean
technology. (Some recent intelligence reports conclude that the group has about 10 such missiles
stored in a Syrian warehouse, according to American officials. The Defense Intelligence Agency
believes that two have probably been moved to Lebanon, according to the officials, speaking on the
condition of anonymity.) The United States officials also worried about Hezbollah's vow to avenge
the death of Imad Mughniyah, a senior fighter killed in a 2008 car bombing the militant group said
was the work of the Israelis.

In a classified cable in February, Secretary Clinton directed the embassy to deliver a warning to
Faisal al-Miqdad, the deputy foreign minister. ``I know you are a strategic thinker, which is why I
want to underscore for you that, from our perspective, your operational support for Hizballah is a
strategic miscalculation that is damaging your long-term national interests.''

The Syrian official's response was dismissive, according to an American cable. He denied that any
weapons had been sent, argued that Hezbollah would not take military action if not provoked and
expressed surprise at the stern American protest. The complaint, he said, ``shows the U.S.~has not
come to a mature position (that would enable it) to differentiate between its own interests and
Israel's.''

\section{North Korea Is Sign of Chilled U.S.-China Relations}

\lettrine{P}{resident}\mycalendar{Dec.'10}{07} Obama and President Hu Jintao of China
talked by telephone on Monday about North Korea, culminating 13 days of effort by the White House to
persuade China's leaders to discuss a crisis that many experts fear could escalate into military
action.

Administration officials say they have no evidence Mr.~Hu was ducking the call, which the Chinese
knew would urge them to crack down on their unruly ally, a step Beijing clearly is highly reluctant
to take amid a leadership succession in North Korea. White House officials insisted that the long
delay was simply the result of scheduling problems.

But in Beijing, both Chinese and American officials and analysts have another explanation: the long
silence epitomizes the speed with which relations between Washington and Beijing have plunged into a
freeze. This year has witnessed the longest period of tension between the two capitals in a decade.
And if anything, both sides appear to be hardening their positions.

``The issues that used to be on the positive side of the ledger are increasingly on the negative
side of the ledger, starting with North Korea,'' Bonnie Glaser, a China scholar at the
Washington-based Center for Strategic and International Studies, said in an interview last week. ``I
don't think this is easily repairable, and I think we're going to have a fairly cold relationship
over the next two years, and potentially longer.''

Mr.~Obama came into office seeking just the opposite: a new rapprochement with a rising power whose
deep economic ties with the United States all but demand closer diplomatic ones. But the days when
the White House spoke of a ``G-2'' that would manage the world economy and more, a phrase that
preceded the first meeting between Mr.~Obama and Mr.~Hu in the depths of the financial crisis in
early 2009, are long over.

Instead, he faces a problem very similar to the one Secretary of State Hillary Rodham Clinton
described in March 2009 during a lunch with Kevin Rudd, a China expert who was then Australia's
prime minister, according to a cable recounting their conversation that was in a newly released
trove of WikiLeaks documents.

Mrs.~Clinton was said to have asked Mr.~Rudd, ``How do you deal toughly with your banker?'''

The latest bad sign is that cooperation on managing North Korea's nuclear ambitions, which began
with considerable promise in 2009, appears to have disintegrated.

On Monday, Mrs.~Clinton was scheduled to convene an emergency meeting in Washington with her
Japanese and South Korean counterparts about both the North's shelling of a South Korean island last
month, and its recent disclosure of a new nuclear facility that potentially expands its nuclear
arsenal. China, the only nation with real sway over the North Korean leadership, will not be there.

To the contrary, China's strategy on North Korea is at odds with that of Washington and its allies.
In Monday's telephone call with Mr.~Hu, the White House said, Mr.~Obama said North Korea's new
enrichment facility flouted commitments it made during the six-party talks on curbing its nuclear
program, and urged China's help in sending ``a clear message to North Korea that its provocations
are unacceptable.''

One former Chinese official with close ties to the government dismissed the American approach last
week as characteristically legalistic. The former official, who would not be named because he is not
authorized to speak on the topic, said China's strategy is to reassure the Koreans about their
security, not lecture them about diplomatic obligations.

Indeed, China's strongest public reaction to last month's shelling of South Korea has not been to
condemn the North, but to criticize Washington's response -- joint war games with South Korea that
put the American carrier George Washington and its strike force in the Yellow Sea, off China's
borders.

After Mr.~Obama's national security team met last Tuesday night, administration officials began
saying that the United States would conduct more military exercises near North Korea and China
should the North engage in further provocations. It was an unmistakable message to Beijing that
failing to rein in its ally would only increase an American military presence that China loathes.

But the lack of cooperation on North Korea only hints at the deterioration in the U.S.-China
relationship.

In another leaked diplomatic cable, the American ambassador to Beijing, Jon Huntsman, wrote last
January that the United States faced ``a challenging year ahead'' in relations with China, adding:
``We need to find ways to keep the relationship positive.''

Instead, two successive meetings between Mr.~Obama and China's top leaders in recent months have
yielded little change in China's management of its currency. A China-based attack on Google
computers early this year riveted attention on Beijing's potential for cyberwar, and provoked nasty
exchanges on the two nations' concepts of free speech. In public and private, China bitterly accuses
the United States of engineering the award in October of the Nobel Peace Prize to a Chinese
dissident, Liu Xiaobo, in an effort to undermine its government.

And those are but lowlights of a year in which Chinese officials have railed loudly and publicly
against what they consider to be American efforts to smother their rightful emergence on the global
stage.

No outsiders can peer inside the black box where Chinese policies are made. But rumors abound that
Beijing's leadership decided in the last year that the United States' faltering after the 2008
economic crisis had handed Beijing an opportunity to seize the global initiative.

``They feel the ball is at their feet,'' one well-connected Chinese political analyst, who did not
want to be named because of the sensitivity of the issue, said in an interview. ``The economic
crisis was the main thing. China could really say `no' to the U.S.''

By comparison, he said, Washington has little leverage over Chinese policies.

There are doubtless other factors at work. China's domestic Internet is being bombarded by ardent
nationalists whose sheer volume of comments may sway official policy. The government faces a
difficult succession in 2012 -- one in which the safe route for every contender for power is to hew
to solidly pro-China ideology.

Nor are Americans ready to accept a world in which their top-dog status is at considerable risk.

``A lot of America hasn't caught up with the fact that China now can say no more often,'' said
Russell Leigh Moses, a Beijing-based analyst of China's leadership. ``When you've got a 6-year-old
and he throws a whiffle ball against a window and it bounces off harmlessly, nobody thinks anything
of it.

``But when he's older, and he throws a hardball, you've got a broken window.''

The difficulty in connecting Mr.~Obama and Mr.~Hu is reminiscent of the last moment of deep tension
between the two countries in early 2001, after a Chinese fighter collided with an American spy plane
trailing just off the Chinese coast, sending the American plane and its crew into an emergency
landing on Chinese soil.

President George W.~Bush spent more than a week trying to get his counterpart, Jiang Zemin, on the
phone. The Chinese -- clearly trying to sort out how to handle the crisis -- declined to take the
call until their internal debate was over. The American crew was eventually returned; the Chinese
sent the plane back in a box, after disassembling and inspecting every component.

Several current and former American officials, as well as foreign diplomats, say they suspect that
this time, the Chinese leadership is still debating how to balance its interest in propping up North
Korea with their interest in preventing more incidents or another nuclear test, which would be North
Korea's third.

But the outside world may never know, particularly after candid discussions about North Korea
between American and Chinese officials have been revealed in WikiLeaks documents.

``The Chinese don't like talking about this stuff,'' one senior American official put it two weeks
ago, after the North revealed its new nuclear site. ``And they certainly don't like talking about it
over the phone.''

\section{On a Hunt for What Makes Gamers Keep Gaming}

\lettrine{B}{y}\mycalendar{Dec.'10}{07} the age of 21, the typical American has spent 10,000 hours
playing computer games, and endured a smaller but much drearier chunk of time listening to sermons
about this sinful habit. Why, the experts wail, are so many people wasting their lives solving
meaningless puzzles in virtual worlds?

Now some other experts -- ones who have actually played these games -- are asking more interesting
questions. Why are these virtual worlds so much more absorbing than school and work? How could these
gamers' labors be used to solve real-world puzzles? Why can't life be more like a video game?

``Gamers are engaged, focused, and happy,'' says Edward Castronova, a professor of
telecommunications at Indiana University who has studied and designed online games. ``How many
employers wish they could say that about even a tenth of their work force?

``Many activities in games are not very different from work activities. Look at information on a
screen, discern immediate objectives, choose what to click and drag.''

Jane McGonigal, a game designer and researcher at the Institute for the Future, sums up the new
argument in her coming book, ``Reality Is Broken: Why Games Make Us Better and How They Can Change
the World.'' It's a manifesto urging designers to aim high -- why not a Nobel Prize? -- with games
that solve scientific problems and promote happiness in daily life.

In the past, puzzles and games were sometimes considered useful instructional tools. The emperor
Charlemagne hired a scholar to compile ``Problems to Sharpen the Young,'' a collection of puzzles
like the old one about ferrying animals across a river (without leaving the hungry fox on the same
bank as the defenseless goat). The British credited their victory over Napoleon to the games played
on the fields of Eton.

But once puzzles and gaming went digital, once the industry's revenues rivaled Hollywood's, once
children and adults became so absorbed that they forsook even television, then the activity was
routinely denounced as ``escapism'' and an ``addiction.'' Meanwhile, a few researchers were more
interested in understanding why players were becoming so absorbed and focused. They seemed to be
achieving the state of ``flow'' that psychologists had used to describe master musicians and
champion athletes, but the gamers were getting there right away instead of having to train for
years.

One game-design consultant, Nicole Lazzaro, the president of XEODesign, recorded the facial
expressions of players and interviewed them along with their friends and relatives to identify the
crucial ingredients of a good game. One ingredient is ``hard fun,'' which Ms.~Lazzaro defines as
overcoming obstacles in pursuit of a goal. That's the same appeal of old-fashioned puzzles, but the
video games provide something new: instantaneous feedback and continual encouragement, both from the
computer and from the other players.

Players get steady rewards for little achievements as they amass points and progress to higher
levels, with the challenges becoming harder as their skill increases.

Even though they fail over and over, they remain motivated to keep going until they succeed and
experience what game researchers call ``fiero.'' The term (Italian for ``proud'') describes the
feeling that makes a gamer lift both arms above the head in triumph.

It's not a gesture you see often in classrooms or offices or on the street, but game designers like
Dr.~McGonigal are working on that. She has designed Cruel 2 B Kind, a game in which players advance
by being nice to strangers in public places, and which has been played in more than 50 cities on
four continents.

She and her husband are among the avid players of Chorewars, an online game in which they earn real
rewards (like the privilege of choosing the music for their next car ride) by doing chores at their
apartment in San Francisco. Cleaning the bathroom is worth so many points that she has sometimes hid
the toilet brush to prevent him from getting too far ahead.

Other people, working through a ``microvolunteering'' Web site called Sparked, are using a
smartphone app undertake quests for nonprofit groups like First Aid Corps, which is compiling a
worldwide map of the locations of defibrillators available for cardiac emergencies. Instead of
looking for magical healing potions in virtual worlds, these players scour buildings for
defibrillators that haven't been cataloged yet. If that defibrillator later helps save someone's
life, the player's online glory increases (along with the sense of fiero).

To properly apply gaming techniques to school and work and other institutions, there are certain
core principles to keep in mind, says Tom Chatfield, a British journalist and the author of ``Fun
Inc.: Why Gaming Will Dominate the Twenty-First Century.'' These include using an ``experience
system'' (like an avatar or a profile that levels up), creating a variety of short-term and
long-term goals, and rewarding effort continually while also providing occasional unexpected
rewards.

``One of the most profound transformations we can learn from games,'' he says, ``is how to turn the
sense that someone has `failed' into the sense that they `haven't succeeded yet.'''

Some schools are starting to borrow gamers' system of quests and rewards, and the principles could
be applied to lots of enterprises, especially colossal collaborations online. By one estimate,
Dr.~McGonigal notes, creating Wikipedia took eight years and 100 million hours of work, but that's
only half the number of hours spent in a single week by people playing World of Warcraft.

``Whoever figures out how to effectively engage them first for real work is going to reap enormous
benefits,'' Dr.~McGonigal predicts.

Researchers like Dr.~Castronova have already benefited by tracking the economic transactions and
social behavior in online games. Now that Facebook and smartphones have enabled virtual communities
to be created fairly cheaply, Dr.~Castronova is hoping to build a prototype that could be adapted by
researchers studying a variety of real-world problems.

``Social media like video games are the only research tool we've ever had that lets us do controlled
experiments on large-scale problems like global warming, terrorism and pandemics,'' Dr.~Castronova
says. ``Not everything in virtual environments maps onto real behavior, but a heck of a lot does.
Rules like `buy low, sell high' and `tall people are sexier' play out exactly the same way, whether
the environment is virtual or real.''

Dr.~Castronova envisions creating financial games to study how bubbles and panics occur, or virtual
cities to see how they respond to disasters.

``One reason that policy keeps screwing up -- think Katrina -- is because it never gets tested,'' he
says. ``In the real world, you can't create five versions of New Orleans and throw five hurricanes
at them to test different logistics. But you can do that in virtual environments.''

Well, you can do it as long as there enough players in that virtual New Orleans who are having
enough fun to keep serving as unpaid lab rats. Researchers will need the skills exhibited by Tom
Sawyer when he persuaded his friends it would be a joyous privilege to whitewash a fence.

Tom discovered, as Twain explained, ``that Work consists of whatever a body is obliged to do, and
that Play consists of whatever a body is not obliged to do.'' The ultimate challenge, when trying to
extract work from the World of Warcraft questers and other players, will be persuading them that
it's still just a game.

\section{Haunted by Attack, Few Return to Island}

\lettrine{S}{itting}\mycalendar{Dec.'10}{07} in front of his home, Chang Woon-il looked almost out
of place, a lonely human figure on a street where stray dogs wandered among the abandoned and broken
houses. Nor was he intending to stay very long. Wielding a stubby knife, Mr.~Chang, a 64-year-old
fisherman, quickly shucked oysters as he hurried to gather as much food as he could before catching
the next ferry out.

He arrived the previous day on his first visit back to this small island in the Yellow Sea since
last week, when it was hit by a deadly North Korean artillery barrage. Like most of 1,350 civilian
residents of this once sleepy fishing town, he fled to the South Korean mainland after the attack.

And like many of these newly created refugees, he says he is so fearful of another attack that he
may never be able to live here again, even though the islanders have nowhere else to go.

``It's our home, but 80 percent of us don't want to come back,'' he said, seated over a pile of
oysters that he had caught before the attack. ``We're in limbo.''

As Yeonpyeong Island's few remaining civilian residents start to pick up the pieces from the sudden
bombardment, which killed two South Korean marines and two civilian construction workers, many
wonder whether it will ever fully recover.

Many, like Mr.~Chang, are giving second thought to living on the island, which is also host to a
garrison of 1,000 marines and is just eight miles from heavily armed North Korea. But others appear
determined to stay, saying Yeonpyeong, best known for its tasty crabs until last week's attack, has
been home to their families since before the peninsula was divided when Japanese colonial control
ended in 1945.

``We have all lived here quite comfortably for all this time, so we should stay here even if we are
near North Korea,'' said Song Young-ock, 49, a ticket seller who returned two days after the attack.

The marines are mostly out of sight on the island's northern half, though a few can be seen posted
at newly erected roadblocks that keep the civilians within the town limits.

According to the town hall, Yeonpyeong's civilian population is now just 60 permanent residents and
74 town officials, many of them rotating from the mainland. The attack also drew 71 journalists and
as many as 80 activists, mostly members of anti-North Korean nationalist groups.

An eerie hush has fallen on the community, now mostly a ghost town of empty streets and homes. About
a half-dozen areas were struck by what appeared to be clusters of artillery shells or rockets, which
left blackened stretches of burned-out homes that still smell of smoke. In other places, blasts had
blown out windows or pried off corrugated roofs, tossing them like twisted ribbons onto the street.

Along one narrow street that had been the island's tiny entertainment district, the front walls of
bars had crumbled, exposing scorched innards filled with the skeletal frames of burned chairs and
melted bottles.

Across the street from one such bar, Lee In-koo, 44, stretched packing tape across his home's broken
windows and used a large chunk of broken masonry to pin his door shut. He arrived that morning to
survey the damage.

``I don't want to come back, but I may not have a choice,'' said Mr.~Lee, who was working in his
restaurant when the attack took place. ``I have to make a living somewhere.''

According to the town hall, 44 islanders and marines were wounded in the attack, which destroyed 25
homes and damaged 78 more.

There were other signs that some of those who had fled intended, at least initially, to come back. A
handwritten sign on the broken windows of a small grocery store asked passers-by not to take the
soft drinks, potato chips and other merchandise still visible inside, apparently untouched. ``I
believe in your conscientiousness,'' the sign said.

Most islanders now stay in the city of Inchon, a two-and-a-half-hour ride away by high-speed ferry.
A large recreational spa there has been turned into a makeshift shelter. Many have come back since
ferry service resumed two days after the attack, but most stay only briefly to pick up belongings,
get food and lock doors, town officials say.

The officials are trying to reach out to residents and persuade them to return. They have also asked
the central government for financial assistance, including a resumption of subsidies that were paid
to residents until the 1980s to keep them on the island, strategically located near waters claimed
by both Koreas. The town's 19 bomb shelters, unfurnished concrete bunkers that date from the 1970s,
will also be modernized.

One town official, Chang Heung-hwa, has also made more personal appeals to those who fled the
bombardment, greeting those who come back for short trips and trying to persuade them to stay. He
admits he has had little luck so far.

But he expressed confidence that some would eventually accept his offer, especially if North Korea
does not attack again and the situation calms down.

``It won't be like before, but somehow the community will survive,'' he said.

About the only people eager to come to the island have been activists, who started showing up right
after the islanders fled. Most were nationalists, who on a recent morning stood on a pier waving
South Korean flags and chanting for revenge against the North Korean dictator, Kim Jong-il.
Animal-rights activists arrived to feed and take back some of the dozens of dogs and cats left
behind by panicked islanders.

Of the small number of residents who stayed, many said they were determined to remain, even if no
one else came back to join them.

``If they don't come back, that means we get all the crabs for ourselves,'' joked Park Chul-hong, a
54-year-old fisherman with a black knit cap and a ribbed parka. But he also said he faced bankruptcy
because South Korea's navy barred him from fishing for a week after the attack -- the last and what
he called the most productive week of the annual crab season.

He remained undaunted despite having spent most of the hourlong barrage cowering with his son in a
cave near the port.

Mr.~Chang, the oyster fisherman, said the terror of the attack was enough to drive him away for
good.

``We just want to fish the sea,'' he said of his family. ``But how can we in this war zone?''

\section{Enforcement Takes On a Softer Side in China}

\lettrine{L}{ike}\mycalendar{Dec.'10}{07} an urban drill sergeant, Tang Shenbin paced on a city
square, sternly inspecting his nervous charges, issuing sotto voce commands with military authority.
He wanted the female members of chengguan -- China's burly enforcers of urban order, feared and
despised for their capricious crackdowns and penchant for violence -- to convey a certain impression
to a clutch of onlookers.

``Stand straight! Look sharp!'' Show them, he whispered, ``what pretty girls are like!''

Four barely-past-teenage girls in white gloves and identical olive jackets and pants snapped to
attention. Four pairs of black pumps lined up ruler-straight. Four prim hats perched perfectly atop
hair bound in blue and white striped bows.

``Personally, I think they are average-looking,'' Mr.~Tang said, dismissively. ``Models are
pretty.''

More than one government has tried to brush up the image of China's urban inspectors. One city
mandated that all new recruits have a college degree. Guangdong Province changed the gray-green
uniforms to a supposedly more inviting blue.

Wuhan, in central China, substituted stare-downs for strong-arming: in 2009, one report stated, 50
officers encircled a wayward snack cart, glowering steadily for a half-hour until the peddler packed
up and left.

Xindu, an urban district of 680,000 in Chengdu, has chosen major image surgery. Since 2003, the
district has supplemented its urban street police with 13 women, specifically chosen for their
looks, shapeliness and youth. The idea is to give the rough-hewn police a softer, feminine side.

Unfortunately, even Scarlett Johansson might struggle to raise China's subterranean regard for these
city squads.

And for good reason, critics would argue. Unlike the police, these officers are authorized only to
enforce city ordinances by imposing fines and other administrative penalties. But the Chinese news
media routinely portray a different reality.

In January 2008, Hubei Province inspectors beat a bystander to death after he used his cellphone to
film them breaking up a protest against a waste dump. Last year, a training manual for Beijing
inspectors, pilfered and posted online, described how to effectively thrash offenders without
drawing blood.

This year, a Shanghai watermelon peddler was left brain-damaged after a scuffle with five officers.
One violence-soaked video game, available for download online, features Chinese-trained inspectors
who assault street vendors.

``Chengguan has scarred the government,'' China Daily, a national publication, lamented last year
after yet another controversy over tactics. The paper demanded a ``truly thorough cleanup.''

Skeptics say the approach here falls far short of that. After the district advertised for eight new
female recruits in October, an editorial in The Beijing Evening News questioned whether the women
had actual duties or were simply scenic diversions. The answer appears to be a little of both. The
district's advertisement called for female applicants 18 to 22 years old, with a good figure and
``the five facial features in proper order.'' They should be above-average height -- taller than 1.6
meters or 5 feet 2 ½ inches.

Retirement at age 26 is mandatory. Officials said the job was physically too arduous for women over
25.

``Their image is the important thing,'' one unnamed district official told Rednet.com, a
quasi-governmental Web site. ``First, the candidates' external qualities will determine if they make
the cut, such as height, weight, facial features, etc.'' Next comes temperament and ``inner
qualities.''

Female chengguan are like flower vases, he said, adding, ``Besides being vases, they will have other
responsibilities.''

Zheng Lihua, the deputy director of the district's city management bureau, is not eager to endorse
that description. But he noted that height requirements were standard in many Chinese job
advertisements for both sexes. So is the demand for orderly facial features.

Whether that means good-looking is a matter of debate among Chinese. Certainly, the disabled or
disfigured need not apply. ``We can't let a lame person or a hunchback come to serve here,''
Mr.~Zheng said. ``His image would not be good.''

Liu Yi, who patrols the Baoguang Square near a monastery, is 22, apple-cheeked with a finely curved
mouth. She does not consider the stress on her appearance to be sexist, she said.

``Do you think I look sexy in this uniform?'' she asked with a wry look. Said her dimpled co-worker,
21-year-old Xu Yang, ``Our job is to present the city's image.''

They do not object to their limited tenure either, they said, because they harbor career ambitions
greater than simply shooing vendors into the alleyways where they are supposed to confine their
business. Every morning, the squad faces off against a dozen or so peddlers who dart around on foot
or bicycle, trying to sell as many buns or bowls of tofu as possible before they are run off.

``Master Wang, you have to leave. We have told you many times!'' said Ms.~Xu as one vendor fled on
foot, temporarily deserting his bicycle-drawn cart of noodle-fixings.

The officers describe their duties as more monotonous than strenuous. ``It is pretty much the same
every day,'' said Huang Jing, 20, who studies marketing in her off hours. ``Very routine.''

One reason is that female officers lack the power of their male counterparts to confiscate vendors'
goods. They can only threaten to report violators to their male supervisors. That tends to shield
them from the sudden public displays of animosity against officialdom that are common throughout
China.

This year hundreds of citizens in Kunming, the Yunnan provincial capital, rioted after false rumors
spread that chengguan officers had killed a vendor. More than a dozen police or chengguan officers
were injured in the nighttime episode; 14 government vehicles were overturned or set on fire.

Xindu has so far escaped such violence. But calm is hardly guaranteed. Just two blocks from placid
Baoguang Square, where the female officers patrolled that morning, more than 50 people gathered on a
street corner.

Officers had confiscated a motorcycle that was being repaired on the sidewalk instead of inside a
shop, as regulations require. The bike's owner was crying foul. A 15-minute standoff ensued before
the officers, grim-faced, elbowed their way to their vehicles and sped off with the motorcycle and
its owner.

Li Xuedong, 40, a coordinator attached to the male squad, remained behind, his white badge flipped
over to conceal his name. Like the female officers, the coordinators -- men age 40 or over -- play a
purely supportive role.

Unlike them, they are not schooled in maintaining a polished image.

``Sometimes we fight verbally. Sometimes we fight physically,'' Mr.~Li said matter-of-factly. ``Most
of the time it is the public who starts it.''

\section{Inside North Korea, `Business as Usual'}

\lettrine{W}{hile}\mycalendar{Dec.'10}{07} North Korea's state-run media continued to rage over the
military exercises being held off the North's coastline, saying the four days of drills that ended
Wednesday afternoon had brought the Korean Peninsula to ``the brink of war,'' much of daily life in
the secretive North appeared remarkably normal, or at least what passes for normal.

Accounts from the North reaching Seoul suggested that residents of Pyongyang, the North Korean
capital, had been calmly discussing last week's artillery duel with South Korea, foreigners living
in the city were worrying about an escalation in tensions with the South and the nation's leader was
celebrated for his legendary contributions to ``the brilliant tradition of Korean dancing art.''

Information about the reclusive North Korean state is notoriously difficult to come by, but a
proliferation of cellphones and computer links in recent years have pried the lid off to some
extent. In the past several days, a foreigner was able to get out a detailed message describing
conditions in Pyongyang based on personal observations and the reports of foreign aid workers in the
city.

Aid workers said their conversations with North Korean colleagues and clients suggested that the
artillery battle was being seen as ``business as usual, another incident of many that have been
happening over many years,'' said the foreigner, who requested anonymity for fear of angering the
North Korean government.

``Of course, people don't know about the international response to the shelling, but it sounds like
it feels like it's `more of the same' for these Koreans.''

The foreigner said some expatriates working with charities and development agencies in the North had
this response: ``We have seen crises like this before, but we're taking it very seriously. And this
is the first time there have been civilian casualties, which ups the ante.''

Meanwhile, apparently, life goes on, at least for Kim Jong-il, the ``Dear Leader'' of North Korea.
Mr.~Kim was the focus of a university symposium to mark the 20th anniversary of the publication of
his celebrated book ``The Theory of Dancing Art.''

Mr.~Kim's work ``clarifies the principled requirements and ways for carrying forward and developing
the brilliant tradition of Korean dancing art established in the period of the anti-Japanese
revolutionary struggle,'' according to a report Tuesday from the North's official news agency.

Mr.~Kim also toured a machine-tool factory and offered ``field guidance'' on how to improve
production, the Korean Central News Agency said. Mr.~Kim is heralded in the North, as was his
father, for dispensing guidance on myriad trips to factories, farms, hospitals, military bases, film
sets and sound stages.

At the Ryongsong Machine Complex, the agency reported, Mr.~Kim visited the factory dining hall where
he expressed concern for the workers' daily lives ``as their real father would do.''

He was accompanied by his sister, Kim Kyong-hee, and her husband, Jang Song-taek, both of whom are
members of Mr.~Kim's ruling inner circle. No mention was made of his youngest son and apparent heir,
Kim Jong-un.

Pyongyang, of course, has by far the best living standards in the otherwise impoverished North,
which is facing renewed famine, according to numerous reports. A recent visitor there said he had
seen many people with cellphones, including three teenagers huddled over one phone as they were,
yes, texting.

Unlike on a previous visit, he said, he managed to slip away from his government escorts and went
for a jog in the capital. He and his colleagues on an academic tour saw splashy displays of imported
goods, especially cognac, and they were treated to lengthy and bountiful banquets.

In a four-day visit, said the man, a Western scholar, there was only one brief power failure, and
the huge statue of North Korea's founding president, Kim Il-sung, remained lighted overnight in
central Pyongyang. The North is chronically short of fuel and diesel oil, and many factories have
been idled because of a lack of power and raw materials.

A beer factory was operating, however, and the visitor pronounced the Taedong River beer, a local
brand, ``very drinkable.'' Private stalls and markets in and around the capital ``appeared to be
thriving,'' he added.

After hearing complaints from factory managers and state economic planners, the scholar said any
potential reform of the hyper-socialist economy in North Korea was being hamstrung by a range of
sanctions by the United Nations, the United States and other nations. Most of the sanctions have
been imposed over the North's refusal to abandon its nuclear programs.

The foreigner living in Pyongyang also saw the bark and the bite of sanctions among North Koreans.

``Sanctions have not changed behaviors of the elite, nor have they stopped the flow of luxury goods
or cars to Pyongyang,'' the foreigner said. ``But for ordinary people, they have so very little to
work with -- tools, running water, medicines in clinics. Imports of any kind are just absent.''

The foreigner said political isolation had been effective in shutting off North Koreans to most
international commerce, from material goods to cultural influences to political ideas.

``In some ways,'' the foreigner said, ``people here are so used to being without -- without stuff,
without validation, without contact to the outside world, without a set of standards about how
things happen -- that sanctions and isolation are normal. They feed a sense of victimhood, or at
least underdog-ness, that justifies both the regime and attacks like the one we've just seen.

``That said, every day I wake up and I can't believe that my `normal' is living in a dictatorship. I
feel a mixture of sadness and frustration at what I observe.''

\section{Rabbit Ears Perk Up for Free HDTV}

\lettrine{J}{ulie}\mycalendar{Dec.'10}{07} and Anthony Bayerl of St.~Paul, Minn., love watching
prime-time shows on the sleek 50-inch television in their bedroom. They also love that they pay
nothing for the programming.

The only thing they do not love is how a low-flying plane, heavy rain or just a little too much
movement in the room can wipe out the picture.

``If someone is changing in there, it messes up your reception,'' said Ms.~Bayerl, a legislative
assistant. ``We try to stay very still when we watch television.''

The Bayerls are using an old technology that some people are giving a second chance. They pull free
TV signals out of the air with the modern equivalent of the classic rabbit-ear antenna.

Some viewers who have decided that they are no longer willing or able to pay for cable or satellite
service, including younger ones, are buying antennas and tuning in to a surprising number of free
broadcast channels. These often become part of a video diet that includes the fast-growing menu of
options available online.

The antenna reception has also led many of these converts to discover -- or rediscover -- the
frustration of weak and spotty signals. But its fans argue that it is tough to beat the price.

``My husband's best friend thinks we're big dorks for having rabbit ears and not cable,'' Ms.~Bayerl
said. But when their introductory price for cable TV and Internet access expired this year and the
bill soared to \$150, the couple halved it by cutting TV. ``It wasn't something we were willing to
pay for,'' she said.

Many pay TV customers are making the same decision. From April to September, cable and satellite
companies had a net loss of about 330,000 customers. Craig Moffett, a longtime cable analyst with
Sanford C.~Bernstein, said the consensus of the industry executives he had talked to was that most
of these so-called cord-cutters were turning to over-the-air TV. ``It looks like they're leaving for
the antenna,'' he said.

Neil Smit, president of Comcast Cable, acknowledged in a recent call with investors that some
customers had dropped cable for free signals. Company executives also said they expected business to
rebound with the economy.

Last month, Time Warner Cable fought back with a lower-cost package that it said might appeal to
people who are feeling the economic squeeze. For \$40 in New York, or \$30 in Ohio, customers can
get a slimmed-down set of channels.

To be sure, around 90 percent of American households still pay for cable or satellite television --
a figure that in recent years has been slowly and steadily rising. But American's relationship with
television has recently been in flux, in part because of the switch last June to digital broadcast
signals.

That initially gave pay TV providers a group of new subscribers who had worried that their old sets
would not pick up the new signals. But analysts say some of those subscribers have since gone back
to free signals.

Another big change is the rise of Internet video, which can ease the pain of losing favorite cable
channels.

Bradley Lautenbach, 28, who recently moved to Los Angeles to work at Disney, found enough
alternatives to allow him to turn back the technological clock on his TV.

``I've always had cable. It's the thing you do when you move to a new place: call the company and
set it up,'' he said. Not this time. Instead, he got an antenna and now watches over-the-air news
and sports, complemented by episodes of shows like ``Entourage'' that he buys from iTunes. ``I don't
miss cable at all,'' he said.

Industrywide figures on antenna sales are hard to come by, so it is difficult to tell how widely
they are being adopted. Antennas Direct, a maker of TV antennas in St.~Louis, expects to sell
500,000 this year, up from 385,000 in 2009, according to its president, Richard Schneider.

The company's digital TV antennas, like others on the market, are a far cry from the wire-hanger
versions of old. The sleek circles encased in plastic look more like mouse ears than anything
belonging to a bunny.

Mr.~Schneider said that based on customer support calls and feedback from retailers, his customers
were 20-somethings who pair over-the-air and Internet programming, people forced to make choices by
a tough economy and others who, he argues, have long been eager to sever ties with their pay-TV
provider.

``Over-the-air is the new basic cable,'' he said, arguing that free TV and Internet alternatives
``are giving people the rationale they've been looking for to end a bad relationship.''

Broadcasters, far from being troubled by the trend, believe it benefits them, according to Dennis
Wharton, a spokesman for the National Association of Broadcasters. He said broadcasters did not mind
the move to over-the-air programming because those viewers were also potential audience members for
the ads that support programming.

Modern antennas, which cost \$25 to \$150, pick up high-definition signals that can actually be
crisper than the cable or satellite version of the same program, because the pay TV companies
compress the video data.

But compared with analog broadcasts, which occasionally showed static, digital signals are less
forgiving of interference and more likely to blank out altogether. At a World Cup viewing party in a
Brooklyn apartment last summer, the hosts encouraged guests to limit trips to the kitchen and the
bathroom to avoid too many interruptions of the signal.

The new antennas do pull in more programs than your grandfather's rabbit ears, because of new
channels that broadcasters added during the transition to digital signals. The broadcasters can fit
multiple digital channels into the same frequencies that used to carry one analog channel.

In St.~Paul, for example, where Ms.~Bayerl lives, there are extra channels from ABC and NBC with
local news and weather, four public television channels and a music video channel. Big markets like
Los Angeles have 40 or more channels, according to Nielsen.

Given the new options, Chris Foster, 29, a graduate student at the University of Virginia, and his
wife decided to forgo a \$35 fee for basic cable. They watch movies and older shows through Netflix
on the Internet and use their antenna for sitcoms and the news.

Mr.~Forster concedes he misses cable news stations, but over all, he is satisfied. ``It feels more
like a step forward than a step back,'' he said.

\section{British Newspaper Finds Readers Flocking Online}

\lettrine{F}{or}\mycalendar{Dec.'10}{07} daily, or even hourly, updates on the life of Cheryl Cole,
a British television talent-show judge and sometime pop singer who used to be married to a soccer
star, there may be no better resource than the Web site of The Daily Mail.

Is the comely Ms.~Cole headed to America to join the U.S.~version of ``The X Factor''? Is she
scouting around for real estate in Los Angeles? Will a seven-year-old assault conviction deny her a
U.S.~visa? Mail Online is on the case.

Readers have noticed. According to the research firm comScore, more than 32 million people flock to
the British newspaper's site every month for the latest on Ms.~Cole, as well as news of narrower
interest. Its growth curve has been remarkable: Created only two years ago, Mail Online now has the
second-largest Web audience of any paper in the world, behind only Nytimes.com, the combined site of
The New York Times and the International Herald Tribune.

``The great thing about the Internet is that you can be late in and still catch up very quickly,''
said Martin Clarke, publisher of the site, on the Web at MailOnline.com. ``The reason we were so
late was that the revenues didn't seem that interesting. Once they did, we got into it.''

Mail Online is a striking example of how newspapers, struggling to develop strategies for the
digital future, can use free Web sites to reach huge new audiences as print circulation and
advertising decline. But it also illustrates the limitations of this strategy; for all the millions
of Web visitors, The Daily Mail's digital businesses generated a grand total of \textsterling12
million, or \$18.7 million, in sales in the latest financial year, the paper's owner, Daily Mail \&
General Trust, reported last month.

Along with the Web site of The Guardian, Mail Online has been the most prominent example of the free
approach in the hypercompetitive British newspaper market, which is emerging as an important
laboratory in the debate of free versus paid.

On the other side is News Corp., which has already moved two of its British newspapers, The Times
and News of the World, behind a paywall, and which plans to do the same for The Sun. This week the
owner of another paper, The Telegraph, said it was considering plans to charge Web readers, while
The Financial Times has done so for years.

``If there's money in it for them, then good luck to them,'' Mr.~Clarke said of the News
Corp.~papers. ``We looked at those numbers; we went through them and came to a different
conclusion.''

Mail Online's commitment to the free approach is not ideological; a large Web audience, Mr.~Clarke
said, could be used to market paid services, like the paper's iPad and iPhone applications, for
which it plans to charge users soon.

Unlike some other newspaper publishers, The Daily Mail can afford to be patient. It dominates the
most lucrative portion of the British newspaper market, the middle ground between serious
publications like The Guardian, The Times and The Telegraph and lower-end tabloids like The Sun and
The Mirror.

As newspapers nearly everywhere shed print readers, The Mail's daily October circulation of 2.13
million slipped only 1.3 percent from a year earlier, the strongest performance of any of the
national British newspapers. Daily Mail \& General Trust said last week that The Daily Mail had
posted its highest earnings ever for the latest financial year, though it did not say what part of
the company's overall profit of \textsterling247 million that represented.

Until it introduced Mail Online with content from the newspaper in June 2008, the editors worried
that giving it away on the Web would undermine the print version.

Now, while the biggest scoops are still reserved for print, most of the newspaper's articles appear
on the Web, too. But The Mail has preserved differences between the print and online editions by
keeping their staffs separate -- a strategy that runs counter to the trend at other newspapers,
where the buzzword is integration.

``Rather than talking about synergies, The Mail is being honest about this: These are very different
products, with very different audiences and very different business models,'' said Douglas McCabe,
an analyst at Enders Analysis in London.

Online, the emphasis has been on celebrity news, featuring Ms.~Cole and other A-, B- and C-list
figures whose every facial wrinkle or fashion choice is scoured for the possibility of extracting a
headline. According to comScore, the site's show business and television coverage drew 15.3 million
visitors in October, compared with 14.9 million for the news sections.

Analysts say Mail Online's traffic growth has been fueled by particularly adept use of the art of
crafting headlines so that the Web page on which they appear secures a high ranking from Google.
Google's methodology was written by engineers rather than poets, so a straightforward approach to
such search engine optimization of headlines generally works best.

Thus, a Mail Online story about a sultry Cheryl Cole showing her moody side in a gloomy new music
video for her song ``The Flood'' reads, ``Sultry Cheryl Cole shows her moody side in gloomy new
video for song The Flood.''

The site has been particularly popular among women, who account for 60 percent of its visitors,
compared with 46 percent of overall Internet users worldwide, according to comScore. Use of the
Internet is growing faster among women than men, and women tend to spend more time on individual
sites.

Celebrity news is less prominent in the print edition, which often highlights tales of immigrants,
European Union officials, single mothers or carcinogens run amok. Those are the bogeymen that stalk
Middle England, the bastion of the printed newspaper's readership.

The Mail's Web site has a broader audience. Not only does the site contain more celebrity news, it
also has more foreign coverage, as well as writing about science, history and other topics that get
shorter shrift in print, Mr.~Clarke said.

And, like Ms.~Cole, the Web site's star is on the rise in the United States; with more than 11
million monthly American visitors, it is already the sixth-biggest newspaper Web site there.

Yet huge audiences have not translated into huge advertising revenue. One reason for this, analysts
say, is the kind of advertising that predominates on the Internet. While marketers use traditional
media like television and newspapers to raise general awareness of a brand or product, when they
turn to the Web they want to close a sale. Unless consumers are nearby, they are of less interest.

For Mail Online, ``the value of the overseas audience is a fraction of the value of the U.K.
audience,'' said Alex Randall, head of online trading at Aegis Media, an agency that buys
advertising space and time.

Ad revenue from the United States remains minimal, Mr.~Clarke acknowledged, and comes in mostly via
advertising networks, which scour the Web for bargain-bin deals, rather than directly from
advertisers. Though The Mail has created a separate version of its Web site that is edited for
U.S.~users, he said his main focus was to serve the British audience.

``We're not on some mission for global domination, but clearly we do have an attraction to overseas
users, and it would be foolish not to monetize that,'' he said.

That is one way of explaining why Cheryl Cole, until she makes the move, will sometimes have to
share Mail Online's home page with Kim Kardashian.

\section{Across and Down, the Wizard Who Is Fastest of All}

\lettrine{T}{he}\mycalendar{Dec.'10}{07} cameras were set to shoot for only eight minutes.

``Oh, it won't take that long,'' Dan Feyer said, with the hint of a smile.

Hubris, anyone?

The pressure was on. Mr.~Feyer (FAY-er), 33, a soft-spoken, balding musician, had come to a photo
studio at The New York Times to demonstrate one of his odder talents.

With the clock ticking and the shutters clicking, he put pencil to crossword. Not just any puzzle,
but the Saturday one from The New York Times -- the week's hardest, notoriously clever and tricky.
Fiendish, even, some would say. A form of mental cruelty. There are people who spend hours on this
puzzle, people who give up, people who won't even touch it. And then there is Dan Feyer.

His left hand tracked the clues while his right skittered over the grid. He pressed his lips
together and grimaced. He erased, and rapidly filled in more boxes. Then he paused, erased again,
and resumed skittering. Nearly five minutes had passed and he still seemed to be working the top
left corner of the puzzle, the very beginning. He mumbled once and erased three more times. Was he
in trouble? He wrote something, looked up, put his pencil down.

Done. Five minutes, 29 seconds. Penmanship, neat as a nun's. Mr.~Feyer, in jeans, sneakers and a
black T-shirt, hadn't broken a sweat.

Who is this guy? What kind of person knows the name of Gorbachev's wife (Raisa), a synonym for
no-good (dadblasted), the Rangers coach in 1994 (Keenan), a platinum-group element (iridium) and the
meaning of objurgation (rant)?

The kind of person who whips through 20 crosswords a day (at least 20,000 in the last three years),
who won this year's American Crossword Puzzle Tournament and who has 100,000 puzzles saved on his
computer.

``I feel I want to do them all, somehow,'' Mr.~Feyer said. ``I've probably done more crosswords than
anybody in the world in the last three years. I don't know if that's something to be proud of, but
it's a claim to fame.''

He does have another life, as a pianist and music director for musical theater productions. His most
recent shows were ``With Glee,'' which ran Off Broadway in Manhattan last summer, and ``Dracula, a
Rock Opera,'' which ran in Rochester, Mich., in October.

``Music directors teach actors the music, accompany them in rehearsals and conduct the band,''
Mr.~Feyer said. ``On Broadway, the music director is the guy with the baton in the pit. Off
Broadway, it's the guy sitting at a piano conducting with his head.''

So how does that guy become a puzzle ace? Besides training like an athlete, Mr.~Feyer said, it helps
to have ``underlying brain power and a head for trivia.'' He always had high grades and test scores,
he said. He excelled at math as well as music, abilities that he thinks go together with crossword
solving.

What they all have in common, he said, is pattern recognition -- as he begins filling in a puzzle
grid, he starts recognizing what the words are likely to be, even without looking at the clues,
based on just a few letters.

``A lot of the time, crossword people are musicians,'' he said, noting that Jon Delfin, who has won
the tournament seven times, is a pianist and music director. ``Mathematicians and computer
scientists are also constructors.''

Arthur Schulman, a crossword constructor and retired psychology professor from the University of
Virginia, who taught a seminar called ``The Mind of the Puzzler,'' agreed that there is a strong
correlation between skill at word puzzles and talent for math and music. All, he said, involve
playing with symbols that in and of themselves are not meaningful. ``There's an underlying
connection, but I'm not sure what it might be,'' Professor Schulman said. ``It's finding meaning in
structure.''

Mr.~Feyer is a relative newcomer to the world of competitive crosswords, though he has liked all
sorts of puzzles since childhood, when his parents bought him books of brain teasers to make up for
his boredom at school. He grew up in San Francisco, where his father is a municipal bond lawyer, his
mother a law professor. He has two younger brothers, one a management consultant and the other an
English teacher in Bhutan. His grandfather George Feyer was a pianist, and played for decades in the
lounges of some of Manhattan's most elegant hotels.

Mr.~Feyer went to Princeton, majoring in music. He did crosswords from time to time over the years,
but he didn't get hooked on them until he saw the 2006 movie ``Wordplay'', a documentary about
crosswords, the tournament and Will Shortz, the New York Times puzzle editor and the founder and
director of the tournament.

``I didn't realize this whole puzzle world existed,'' he said.

He bought a book of crosswords, and then another, and began following crossword blogs and
downloading puzzles. Before he knew it, he had become one with the puzzle people.

In 2008 he entered his first tournament, in which hundreds of people in a hotel ballroom race to
finish a series of puzzles. He had found his niche: the sound of 700 people turning over a piece of
paper at the same time thrilled him. He finished ``50-somethingth,'' he said. But that put him at
the top of the rookie division for which he had qualified. The following year, he finished fourth.
This year he won, beating many veterans, including Tyler Hinman, the champion of five previous
tournaments.

His brain is jammed with factoids: the names of songs and rock bands that lived and died before he
was born, far-flung rivers and capitals, foreign sports equipment, dead astronomers, fallen
monarchs, extinct cars, old movies, heroes of mythology, dusty novelists and the myriad other
bevoweled wraiths that haunt the twisted minds of crossword constructors. He has learned their wily
tricks and traps, like using ``number'' in a clue that most people would take to mean ``numeral''
but that really meant ``more numb.''

He was almost stumped recently, by a clue asking for a type of wheel. The answer: wire spoke. ``I
had a heck of a time with that,'' he said. Cruel? Maybe so, he said, but he shrugged and added,
``That's what Saturday is for.''

Each morning, he finishes a half dozen newly published puzzles and a few more from the trove stashed
in his computer. The easier ones take him only two or three minutes. He does puzzles while riding
the subway and while watching TV. He may do a few before going to sleep and even take one to bed
with him. He said he now spends about an hour a day on puzzles.

``It hasn't taken over my life or anything,'' he said, then added, ``I don't think.''

Nonetheless, on his blog, Mr.~Feyer describes himself as ``a mild-mannered musician who developed an
addiction to crosswords,'' and he posts his solving times every day. There is, he says, a friendly
competition among top-ranked solvers. For a Monday puzzle in The New York Times, his fastest time by
computer was 1 minute, 22 seconds. Paper takes longer, 1 minute 58 seconds, maybe 59. His fastest
time ever was 1 minute, 9 seconds, for a Newsday puzzle. But he admits that in speed-solving, one
can lose the ``Aha!'' moment and the chance to savor a clever solution.

Another musician who works in puzzle publishing helped Mr.~Feyer land freelance editing and
proofreading work for a company that produces books of puzzles, and signed him up to write a book of
word-search puzzles. He has tried his hand at constructing puzzles, but decided that just as he is
better at playing or directing music than composing, he is better at solving puzzles than creating
them.

Even so, he has managed to sell a few puzzles, including one to the Times; it will run on a Tuesday,
a relatively easy day (the Times puzzles get harder each day, with the easiest on Monday and the
hardest on Saturday). He spent 20 hours creating it, off and on over six months. The pay for a daily
puzzle is \$200 (\$1,000 for the big Sunday one).

Given a choice, he prefers to solve crosswords on a computer. But in competitions the puzzles are
done on paper, where the clues are arranged a bit differently, so as the yearly tournament
approaches (the next one is in March), he switches to paper to stay on top of his game. Otherwise,
he said, ``you can lose precious, precious seconds looking for clues.'' He writes with a mechanical
pencil, the same kind he uses to mark up musical scores.

He plans to compete again.

``I'll definitely try to defend the title,'' he said. Winning is fun, but also, he added, ``the only
way to make money with this hobby.''

The grand prize is \$5,000. He professed mock outrage that Sudoku tournaments pay much more --
\$20,000.

``I'm not particularly good at Sudoku,'' he said.

He does not play Scrabble. That game and crosswords differ in the types of words they use.

``It would probably hurt my championship chances if I tried to memorize the Scrabble list, too,'' he
said. ``My brain is full of crossword vocabulary.''

He figures he has seen just about everything that is likely to appear in a crossword puzzle. But he
continues to collect them. ``One day,'' he said, ``I will have done them all.''

\section{Winner's Chair Remains Empty at Nobel Event}

\lettrine{I}{mprisoned}\mycalendar{Dec.'10}{11} and incommunicado in China, the Chinese writer and
dissident Liu Xiaobo was awarded the Nobel Peace Prize on Friday, his absence marked at the prize
ceremony here by an empty chair.

For the first time since the 1935 prize, when the laureate, Carl von Ossietzky, languished in a
concentration camp and Hitler forbade any sympathizers to attend the ceremony, no relative or
representative of the winner was present to accept the award or the \$1.5 million check it comes
with. Nor was Mr.~Liu able to provide a speech, even in absentia.

Guests at the ceremony in Oslo's City Hall listened instead to a recitation of his defiant yet
gentle statement to a Chinese court before his incarceration last year. ``I have no enemies and no
hatred,'' Mr.~Liu said in ``I Have No Enemies: My Final Statement to the Court,'' read aloud by the
Norwegian actress Liv Ullmann. ``Hatred can rot away at a person's intelligence and conscience.''

Through his wife, Liu Xia, Mr.~Liu sent word that he wanted to dedicate the award to the ``lost
souls'' massacred in 1989 in Tiananmen Square.

Mr.~Liu, 54, a professor, poet, essayist and campaigner for human rights, has been an irritant to
the Chinese authorities since helping resolve confrontations between the police and students in
Tiananmen Square. Mr.~Liu was detained in December 2008, after co-writing the Charter 08 call for
human rights and reform, and is currently serving an 11-year sentence for the crime of ``incitement
to the overthrow of the state power and socialist system and the people's democratic dictatorship.''

He was named this year's laureate because of his heroic and nonviolent struggles on behalf of
democracy and human rights, said Thorbjorn Jagland, chairman of the Norwegian Nobel Committee,
adding that China needed to learn that with economic power came social and political responsibility.

``We can to a certain degree say that China with its 1.3 billion people is carrying mankind's fate
on its shoulders,'' Mr.~Jagland said in a speech at the ceremony. ``If the country proves capable of
developing a social market economy with full civil rights, this will have a huge favorable impact on
the world.''

He added, ``Many will ask whether China's weakness -- for all the strength the country is currently
showing -- is not manifested in the need to imprison a man for 11 years merely for expressing his
opinions on how his country should be governed.''

Mr.~Jagland, a former prime minister, became chairman of the Nobel committee last year and seems
unafraid to use the position to make strong political statements. Last year's selection of President
Obama as the peace laureate was interpreted by many as a thinly veiled rebuke to the politics of
former President George W.~Bush, and this year's award has had broad political implications.

In an interview on Friday, Mr.~Jagland said that on several occasions this fall, he and the
Norwegian foreign minister were specifically warned by top Chinese officials not to give the award
to Mr.~Liu. But even though the committee disregarded their threats, Mr.~Jagland said, its choice
should not be interpreted as an insult to China.

Rather, Mr.~Jagland said, its reasoning should be seen as similar to that of 1964, when the prize
went to the Rev.~Dr.~Martin Luther King Jr., who was defying the authorities to fight for civil
rights in America. The prize helped nudge the United States to change, Mr.~Jagland said, and he
hoped that it would have the same effect on China.

He said it was important that other countries stand up to China, in all its might. ``The whole world
is now benefiting from China's economic growth, but we cannot only look to short-term commercial
interests,'' he said. ``We need to uphold our own values.''

But China has reacted to the award with a mix of fury and derision, dismissing it as a ``political
farce.'' It leaned hard on other countries to stay away from the Nobel ceremony (in the end, the
committee said, 16 countries with ambassadors in Norway did not attend). It also denounced members
of the Nobel committee as ``clowns''; hastily organized its own, competing peace prize; shut down
Web sites referring to Mr.~Liu; and blacked out broadcasts and closed off access to the Web sites of
foreign news media outlets like CNN and the BBC.

China also suspended bilateral trade negotiations with Norway, and has had no official contact with
the country since October, Mr.~Jagland said. And Norwegian exporters of salmon, for whom China is an
important market, say that they have been informed that, starting next week, their fish will be
subject to a slew of extra tests at Chinese ports.

With the world's attention on Oslo, the authorities were coming down hard on Mr.~Liu's supporters in
China. On Thursday in Beijing, Zhang Zuhua, a former official who helped write Charter 08, was
forced into a vehicle by police officers, according to rights advocates, and dozens of other people
were either confined to their homes or escorted out of the capital.

Blue construction panels were erected outside Mr.~Liu's apartment building, where his wife, Liu Xia,
is being detained, apparently in an attempt to block it from the view of foreign reporters gathering
outside. In a text message, one of Mr.~Liu's brothers, Liu Xiaoxuan, apologized for being unable to
speak, saying that his phone was being monitored.

About 100 pro-Liu Chinese people came to Oslo to see the ceremony, either at City Hall or on a
screen at the nearby Peace Center. One of them, a student who asked that his name not be used
because he was afraid of reprisals back home, said that only a tiny percentage of his friends in
China had even heard of Mr.~Liu, and that many were so consumed with work and content with the
country's economic success that they had little time or sympathy for political dissent.

In the sentencing appeal read at the ceremony here, Mr.~Liu told of his sorrow at being branded a
``counterrevolutionary'' and rendered essentially voiceless. ``Merely for publishing different
political views and taking part in a peaceful democracy movement, a teacher lost his lectern, a
writer lost his right to publish and a public intellectual lost the opportunity to give talks
publicly.''

But he praised China for making much progress, and praised his prison guards for treating him with
compassion and humanity. He told of how he was sustained by his wife's love -- ``the sunlight that
leaps over high walls and penetrates the iron bars of my prison window, stroking every inch of my
skin, warming every cell of my body, allowing me to keep peace, openness and brightness in my
heart.'' He went on to say, ``Even if I were crushed into powder, I would still use my ashes to
embrace you.''

Mr.~Obama issued a statement saying Mr.~Liu was ``far more deserving of this award than I was,'' and
calling for his release ``as soon as possible.''

In City Hall, the audience was moved by Ms.~Ullmann's solemn reading of Mr.~Liu's words.

``Freedom of expression is the foundation of human rights, the source of humanity and the mother of
truth,'' Mr.~Liu's statement said. As for ``China's endless literary inquisitions,'' he said, ``I
hope I will be the last victim.''

\section{Tirades Against Nobel Aim at Audience in China}

\lettrine{A}{s}\mycalendar{Dec.'10}{11} much of the world on Friday focused their eyes on the empty
seat in Oslo that starkly represented the absence of the Nobel laureate Liu Xiaobo, a lone Chinese
blogger posted the image of a chair on the country's most popular microblogging site.

Within minutes, it had been deleted by a censor's unseen hand.

That small gesture of solidarity with Mr.~Liu, who is serving an 11-year prison sentence for
``subversion of state power,'' is largely emblematic of China's sweeping effort to quash any
expression of sympathy for a man whose plight has captivated the world.

All mentions of the pageantry at the award ceremony on Friday were scrubbed from the Chinese
Internet, and the relatively small number of people who have access to overseas news outlets such as
BBC and CNN saw their television screens go black in the days leading up the ceremony. On Friday
night, the most discussed topics on Sina, the largest news portal, included plunging temperatures
and flight delays at Beijing's airport.

To those outside China, the government's response to the Norwegian committee's decision to give
Mr.~Liu the Nobel Peace Prize was remarkable for its bombast and audacity.

Beijing dispatched its diplomats to warn countries against sending envoys to the ceremony, while the
Foreign Ministry and state media issued a steady drumbeat of invective, describing the prize as a
Western plot to hold back a rising China and branding the award's supporters as ``clowns.'' On
Friday, Global Times, a populist tabloid affiliated with the party-owned People's Daily, called the
event a ``political farce'' and Oslo a ``cult center.''

But while such outbursts may have provoked snickers around the world, the stern-faced men who run
China's government may have the last laugh. Minxin Pei, a political scientist at Claremont McKenna
College in California, said those who focus solely on the damage done to Beijing's global image are
missing the point. In the end, he said, the only opinions that matter are those held by China's 1.3
billion citizens.

``After Tiananmen, China suffered three years of international isolation, but it recovered,'' he
said, referring to the violent crackdown of pro-democracy protesters in 1989. ``The regime's
approach to the Nobel was strategic. They know the world will come calling again because China and
its economy cannot be ignored for long.''

After the prize was announced, China's censors promptly took measures to stymie the spread of the
news via the Internet and text messaging, while police agents began detaining and harassing liberal
colleagues and supporters.

But once they realized they could not control the debate beyond their nation's borders from seeping
into China, Mr.~Pei and other analysts say, senior leaders decided to tailor their message to the
domestic audience. Although the Chinese government has become increasingly adept at controlling
information available to its 440 million Internet users, several people with knowledge of the
government's deliberations said the Nobel Prize presented propaganda officials with a daunting
challenge: how to smear what many ordinary Chinese see as honor without fanning interest in Mr.~Liu.

Within a few days of the announcement, China's Politburo met to complete a game plan: Mr.~Liu would
be painted as a traitor and the Nobel committee's decision would be officially labeled a ``plot by
Western enemy forces, headed by the United States,'' according to a veteran journalist at a
party-run media outlet who had knowledge of the deliberations.

Wielding rhetoric redolent of the Maoist era, a succession of commentaries soon appeared that played
on nationalistic sentiment by highlighting Mr.~Liu's affiliation with an American pro-democracy
group. Others pulled quotes from an interview he gave to a Hong Kong magazine in 1988 in which he
described colonialism as the antidote to China's problems. (Supporters say his remarks were
incendiary to make a point about China's dysfunction.)

In every article about him, Mr.~Liu was described as a criminal who had been tried and convicted by
the nation's justice system.

At the same time, the censors assiduously removed information about Mr.~Liu not approved by the
propaganda ministry, including any mention at all of Charter 08, the pro-democracy manifesto that he
helped shape and that led to his conviction.

The intense media controls appear to have had the desired effect. According to the veteran party
journalist, an official survey of university students taken since the prize was awarded found that
85 percent said they knew nothing about Mr.~Liu and Charter 08.

Although it is not clear exactly when the survey was taken, that figure was partly borne out Friday
in conversations with more than three dozen people across the capital, many of them students at two
of the country's top universities. One student said she thought the Nobel recipient was the Dalai
Lama (he won in 1989) and another insisted that the award ceremony had long since taken place. Most
said they had no idea who Mr.~Liu was, but a handful quietly voiced support for him and his ideas.

Even if they did not know his name, those who were aware that a Chinese citizen was the recipient
said they agreed with the government that his selection was a plot to embarrass China.

Xiao Feng, 27, said she thought the recipient had probably done something to harm the nation. ``I
think this year's prize is a little bit unfair,'' she said. ``From what I can tell, its purpose is
to humiliate China.''

\section{China's Army of Graduates Struggles for Jobs}

\lettrine{L}{iu}\mycalendar{Dec.'10}{31} Yang, a coal miner's daughter, arrived in the capital this
past summer with a freshly printed diploma from Datong University, \$140 in her wallet and an air of
invincibility.

Her first taste of reality came later the same day, as she lugged her bags through a ramshackle
neighborhood, not far from the Olympic Village, where tens of thousands of other young strivers cram
four to a room.

Unable to find a bed and unimpressed by the rabbit warren of slapdash buildings, Ms.~Liu scowled as
the smell of trash wafted up around her. ``Beijing isn't like this in the movies,'' she said.

Often the first from their families to finish even high school, ambitious graduates like Ms.~Liu are
part of an unprecedented wave of young people all around China who were supposed to move the
country's labor-dependent economy toward a white-collar future. In 1998, when Jiang Zemin, then the
president, announced plans to bolster higher education, Chinese universities and colleges produced
830,000 graduates a year. Last May, that number was more than six million and rising.

It is a remarkable achievement, yet for a government fixated on stability such figures are also a
cause for concern. The economy, despite its robust growth, does not generate enough good
professional jobs to absorb the influx of highly educated young adults. And many of them bear the
inflated expectations of their parents, who emptied their bank accounts to buy them the good life
that a higher education is presumed to guarantee.

``College essentially provided them with nothing,'' said Zhang Ming, a political scientist and vocal
critic of China's education system. ``For many young graduates, it's all about survival. If there
was ever an economic crisis, they could be a source of instability.''

In a kind of cruel reversal, China's old migrant class -- uneducated villagers who flocked to
factory towns to make goods for export -- are now in high demand, with spot labor shortages and
tighter government oversight driving up blue-collar wages.

But the supply of those trained in accounting, finance and computer programming now seems limitless,
and their value has plunged. Between 2003 and 2009, the average starting salary for migrant laborers
grew by nearly 80 percent; during the same period, starting pay for college graduates stayed the
same, although their wages actually decreased if inflation is taken into account.

Chinese sociologists have come up with a new term for educated young people who move in search of
work like Ms.~Liu: the ant tribe. It is a reference to their immense numbers -- at least 100,000 in
Beijing alone -- and to the fact that they often settle into crowded neighborhoods, toiling for
wages that would give even low-paid factory workers pause.

``Like ants, they gather in colonies, sometimes underground in basements, and work long and hard,''
said Zhou Xiaozheng, a sociology professor at Renmin University in Beijing.

The central government, well aware of the risks of inequitable growth, has been trying to channel
more development to inland provinces like Shanxi, Ms.~Liu's home province, where the dismantling of
state-owned industries a decade ago left a string of anemic cities.

Despite government efforts, urban residents earned on average 3.3 times more last year than those
living in the countryside. Such disparities -- and the lure of spectacular wealth in coastal cities
like Shanghai, Tianjin and Shenzhen -- keep young graduates coming.

``Compared with Beijing, my hometown in Shanxi feels like it's stuck in the 1950s,'' said Li Xudong,
25, one of Ms.~Liu's classmates, whose father is a vegetable peddler. ``If I stayed there, my life
would be empty and depressing.''

While some recent graduates find success, many are worn down by a gauntlet of challenges and
disappointments. Living conditions can be Dickensian, and grueling six-day work weeks leave little
time for anything else but sleeping, eating and doing the laundry.

But what many new arrivals find more discomfiting are the obstacles that hard work alone cannot
overcome. Their undergraduate degrees, many from the growing crop of third-tier provincial schools,
earn them little respect in the big city. And as the children of peasants or factory workers, they
lack the essential social lubricant known as guanxi, or personal connections, that greases the way
for the offspring of China's nouveau riche and the politically connected.

Emerging from the sheltered adolescence of one-child families, they quickly bump up against the
bureaucracy of population management, known as the hukou system, which denies migrants the
subsidized housing and other health and welfare benefits enjoyed by legally registered residents.

Add to this a demographic tide that has increased the ranks of China's 20-to-25-year-olds to 123
million, about 17 million more than there were just four years ago.

``China has really improved the quality of its work force, but on the other hand competition has
never been more serious,'' said Peng Xizhe, dean of Social Development and Public Policy at Fudan
University in Shanghai.

Given the glut of underemployed graduates, Mr.~Peng suggested that young people either shift to more
practical vocations like nursing and teaching or recalibrated their expectations. ``It's O.K. if
they want to try a few years seeking their fortune, but if they stay too long in places like Beijing
or Shanghai, they will find trouble for themselves and trouble for society,'' he said.

A fellow Datong University graduate, Yuan Lei, threw the first wet blanket over the exuberance of
Ms.~Liu, Mr.~Li and three friends not long after their July arrival in Beijing. Mr.~Yuan had arrived
several months earlier for an internship but was still jobless.

``If you're not the son of an official or you don't come from money, life is going to be bitter,''
he told them over bowls of 90-cent noodles, their first meal in the capital.

As the light faded and the streets became thick with young receptionists, cashiers and sales clerks
heading home, Mr.~Yuan led his friends down a dank alley and up an unsteady staircase to his room.
It was about the width of a queen-size bed, and he shared a filthy toilet with dozens of other
tenants and a common area with a communal hot plate.

Mr.~Li smiled as he took in the scene. Like most young Chinese, his life until that moment had been
coddled, chaperoned and intensely regimented. ``I'm ready to go out into the world and test
myself,'' he said.

The next five months would provide more of a test than he or the others had expected. For weeks
Mr.~Li elbowed his way through crowded job fairs but came away empty-handed. His finance degree,
recruiters told him, was useless because he was a ``waidi ren,'' an outsider, who could not be
trusted to handle cash and company secrets.

When he finally found a job selling apartments for a real estate agency, he left after less than a
week when his employer reneged on a promised salary and then fined him each day he failed to bring
in potential clients.

In the end, Mr.~Li and his friends settled for sales jobs with an instant noodle company. The
starting salary, a low \$180 a month, turned out to be partly contingent on meeting ambitious sales
figures. Wearing purple golf shirts with the words ``Lao Yun Pickled Vegetable Beef Noodles,'' they
worked 12-hour days, returning home after dark to a meal of instant noodles.

``This isn't what I want to be doing, but at least I have a job,'' said Mr.~Li, sitting in his room
one October evening. Decorated with origami birds left by a previous occupant, the room faced a
neighbor's less than two feet across an airshaft. The only personal touch was an instant noodle
poster taped over the front door for privacy.

Because he had sold only 800 cases of noodles that month, 200 short of his sales target, Mr.~Li's
paltry salary was taking a hit. And citing the arrival of winter, ``peak noodle-eating season,'' his
boss had just doubled sales quotas.

Mr.~Li worried aloud whether he would be able to marry his high school sweetheart, who had
accompanied him here, if he could not earn enough money to buy a home. Such concerns are rampant
among young Chinese men, who have been squeezed by skyrocketing real estate prices and a culture
that demands that a groom provide an apartment for his bride. ``I'm giving myself two years,'' he
said, his voice trailing off.

By November, the pressure had taken its toll on two of the others, including the irrepressible Liu
Yang. After quitting the noodle company and finding no other job, she gave up and returned home.

That left Mr.~Yuan, Mr.~Li and their girlfriends. Over dinner one night, the four of them complained
about the unkindness of Beijingers, the high cost of living and the boredom of their jobs. Still,
they all vowed to stick it out.

``Now that I see what the outside world is like, my only regret is that I didn't have more fun in
college,'' Mr.~Yuan said.

\section{Families Bear Brunt of Deployment Strains}

\lettrine{L}{ife}\mycalendar{Dec.'10}{31} changed for Shawn Eisch with a phone call last January.
His youngest brother, Brian, a soldier and single father, had just received orders to deploy from
Fort Drum, N.Y., to Afghanistan and was mulling who might take his two boys for a year. Shawn
volunteered.

So began a season of adjustments as the boys came to live in their uncle's home here. Joey, the
8-year-old, got into fistfights at his new school. His 12-year-old brother, Isaac, rebelled against
their uncle's rules. And Shawn's three children quietly resented sharing a bedroom, the family
computer and, most of all, their parents' attention with their younger cousins.

The once comfortable Eisch farmhouse suddenly felt crowded.

``It was a lot more traumatic than I ever pictured it, for them,'' Shawn, 44, said. ``And it was for
me, too.''

The work of war is very much a family affair. Nearly 6 in 10 of the troops deployed today are
married, and nearly half have children. Those families -- more than a million of them since 2001 --
have borne the brunt of the psychological and emotional strain of deployments.

Siblings and grandparents have become surrogate parents. Spouses have struggled with loneliness and
stress. Children have felt confused and abandoned during the long separations. All have felt
anxieties about the distant dangers of war.

Christina Narewski, 26, thought her husband's second deployment might be easier for her than his
first. But she awoke one night this summer feeling so anxious about his absence that she thought she
was having a heart attack and called an ambulance. And she still jumps when the doorbell rings,
worried it will be officers bearing unwanted news.

``You're afraid to answer your door,'' she said.

Social scientists are just beginning to document the rippling effects of multiple combat deployments
on families -- effects that those families themselves have intimately understood for years. A study
published in The New England Journal of Medicine in January found that wives of deployed soldiers
sought mental health services more often than other Army wives.

They were also more likely to report mental health problems, including depression, anxiety and sleep
disorder, the longer the deployments lasted.

And a paper published in the journal Pediatrics in late 2009 found that children in military
families were more likely to report anxiety than children in civilian families. The longer a parent
had been deployed in the previous three years, the researchers found, the more likely the children
were to have had difficulties in school and at home.

But those studies do not describe the myriad ways, often imperceptible to outsiders, in which
families cope with deployments every day.

For Ms.~Narewski, a mother of three, it has meant taking a grocery store job to distract her from
thinking about her husband, a staff sergeant with the First Battalion, 87th Infantry, now in
northern Afghanistan.

For Tim Sullivan, it has meant learning how to potty train, braid hair and fix dinner for his two
young children while his wife, a sergeant in a support battalion to the 1-87, is deployed.

For young Joey Eisch, it meant crying himself to sleep for days after his father, a platoon sergeant
with the battalion, left last spring. His older brother, Isaac, calm on the outside, was nervous on
the inside.

``It's pretty hard worrying if he'll come back safe,'' Isaac said. ``I think about it like 10 or
more times a day.''

Joining the Army Life

Soon after Christina and Francisco Narewski married in 2004, he applied for a job with the local
sheriff's office in Salinas, Calif. But he got tired of waiting and, after talking things over with
Christina, enlisted in the Army instead.

``We both signed up for it,'' Ms.~Narewski said. ``We knew deployments were going to come.''

That day arrived in the fall of 2007, when their third child was just 5 months old. Ms.~Narewski
missed Francisco dearly and sometimes cried just hearing his voice when he called from Iraq. But
when he returned home in October 2008, it took them weeks to feel comfortable together again, she
recalled.

``It's almost like you've forgotten how to be with each other,'' she said. ``He's been living in his
spot for 15 months. Me and the kids have our own routine. It's hard to get back to, 'Oh, you're
home.' ''

Last April, he left again, this time to Afghanistan. Ms.~Narewski, who lives in Watertown, N.Y.,
thought she was prepared. Her mother came to live with them. She signed up for exercise classes to
fill the hours. She and Francisco bought BlackBerrys with instant messaging service so they could
communicate daily. And yet.

``I've never missed him as much as I do right now,'' she said recently. ``It doesn't feel like we're
moving. It's like you're in a dream and you're trying to get something and you can't get it.''

Not all the spouses back home are women. Tim Sullivan's days have revolved almost entirely around
his two children, Austin, 4, and Leah, 2, since his wife, Sgt. Tamara Sullivan, deployed to
Afghanistan in March.

He rises each weekday at 5:30 a.m. to dress and feed them before shuttling them to day care.
Evenings are the reverse, usually ending with him dozing off in front of the television at their
rented ranch-style house in Fayetteville, N.C.

He has moved twice and changed jobs three times in recent years to accommodate his wife's military
career. But he does not mind being home with the children, he says, because his father was not,
having left the family when Mr.~Sullivan was young.

``I'm not going to put my kids through that,'' said Mr.~Sullivan, 35, who handles child support
cases for the county. ``I'm going to be there.''

He worries about lost intimacy with his wife, saying that they have had a number of arguments by
phone, usually about bill paying or child rearing. ``She tells me: 'Tim, you can't just be Daddy,
the hard person. You have to be Mommy, too,''' he said. ``I tell her it's not that easy.''

Yet he says that if she stays in the Army -- as she has said she wants to do -- he is prepared to
move again or even endure another deployment. ``I love her,'' he said. ``I'm already signed up. I
made a decision to join the life that goes with that.''

Doing What Uncle Sam Asks

Isaac and Joey Eisch have also had to adjust to their father's nomadic life. ``I don't try to get
too attached to my friends because I move around a lot,'' said Isaac, who has lived in five states
and Germany with his father. (Joey has lived in three states.) ``When I leave, it's like, hard.''

When Sergeant Eisch got divorced in 2004, he took Isaac to an Army post in Germany while Joey stayed
with his mother in Wisconsin. Soon after returning to the States in 2007, the sergeant became
worried that his ex-wife was neglecting Joey. He petitioned family court for full custody of both
boys and won.

In 2009, he transferred to Fort Drum and took the boys with him. Within months, he received orders
for Afghanistan.

After nearly 17 years in the Army with no combat deployments, Sergeant Eisch, 36, was determined to
go to war. The boys, he felt, were old enough to handle his leaving. Little did he know how hard it
would be.

When Shawn put the boys in his truck at Fort Drum to take them to Wautoma, a two-stoplight town in
central Wisconsin, Isaac clawed at the rear window ``like a caged animal,'' Sergeant Eisch said. He
still tears up at the recollection.

``I question myself every day if I'm doing the right thing for my kids,'' he said. ``I'm trying to
do my duty to my country and deploy, and do what Uncle Sam asks me to do. But what's everybody
asking my boys to do?''

Within a few weeks of arriving at his uncle's home, Joey beat up a boy so badly that the school
summoned the police. It was not the last time Shawn and his wife, Lisa, would be summoned to the
principal's office.

The boys were in pain, Shawn realized. ``There was a lot more emotion,'' he said, ``than Lisa and I
ever expected.''

Shawn, a state water conservation officer, decided he needed to set strict rules for homework and
behavior. Violations led to chores, typically stacking wood. But there were carrots, too: for Joey,
promises of going to Build-a-Bear if he obeyed his teachers; for Isaac, going hunting with his uncle
was the prize. Gradually, the calls from the principal declined, though they have not ended.

In September, Sergeant Eisch returned for midtour leave and the homecoming was as joyful as his
departure had been wrenching. Father and sons spent the first nights in hotels, visited an amusement
park, went fishing and traveled to New York City, where they saw Times Square and the Intrepid Sea,
Air and Space Museum.

But the two weeks were over in what seemed like hours. In his final days, Sergeant Eisch had prepped
the boys for his departure, but that did not make it any easier.

``Why can't we just, like, end the war?'' Isaac asked at one point.

As they waited at the airport, father and sons clung to each other. ``I'm going to have to drink
like a gallon of water to replenish these tears,'' the sergeant said. ``Be safe,'' Isaac implored
him over and over.

Sergeant Eisch said he would, and then was gone.

Despite his worries, Isaac tried to reassure himself. ``He's halfway through, and he's going to make
it,'' he said. ``With all that training he's probably not going to get shot. He knows if there's a
red dot on his chest, run. Not toward the enemy. Run, and shoot.''

But his father did not run.

Dad Comes Home

Just weeks after returning to Afghanistan, Sergeant Eisch, the senior noncommissioned officer for a
reconnaissance and sniper platoon, was involved with Afghan police officers in a major offensive
into a Taliban stronghold south of Kunduz city.

While directing fire from his armored truck, Sergeant Eisch saw a rocket-propelled grenade explode
among a group of police officers standing in a field. The Afghans scattered, leaving behind a man
writhing in pain. Sergeant Eisch ordered his medic to move their truck alongside the officer to
shield him from gunfire. Then Sergeant Eisch got out.

``I just reacted,'' he recalled. ``I seen a guy hurt and nobody was helping him, so I went out
there.''

The police officer was bleeding from several gaping wounds and seemed to have lost an eye. Sergeant
Eisch started applying tourniquets when he heard the snap of bullets and felt ``a chainsaw ripping
through my legs.'' He had been hit by machine gun fire, twice in the left leg, once in the right.

He crawled back into his truck and helped tighten tourniquets on his own legs. He was evacuated by
helicopter and taken to a military hospital where, in a morphine daze, he called Shawn.

``Are you sitting down?'' Brian asked woozily. ``I've been shot.''

Shawn hung up and went into a quiet panic. He could not tell how badly Brian had been wounded. Would
he lose his leg? He called the school and asked them to shield the boys from the news until he could
get there.

Outside school, Shawn told Isaac, Joey and his 12-year-old daughter, Anna, about Brian's injury.
Only Isaac stayed relatively calm.

But later, Shawn found Isaac in his bedroom weeping quietly while looking at a photograph showing
his father outside his tent, holding a rifle. Shawn helped him turn the photograph into a PowerPoint
presentation titled, ``I Love You Dad!''

For Shawn, a gentle and reserved man, his brother's injury brought six months of family turmoil to a
new level. Sensing his distress, Lisa urged him to go hunting, a favorite pastime. So he grabbed his
bow and went to a wooded ridge on his 40 acres of property.

To his amazement, an eight-point buck wandered by. Shawn hit the deer, the largest he had ever
killed with a bow. It seemed a good omen.

A few days later, Shawn flew with the boys, his father and Brian's twin sister, Brenda, to
Washington to visit Sergeant Eisch at Walter Reed Army Medical Center. At the entrance, they saw men
in wheelchairs with no arms and no legs. Others were burned or missing eyes. Shawn feared what the
boys would see inside Brian's room.

But Brian, giddy from painkillers, was his cheerful self. His right leg seemed almost normal. His
left leg, swollen and stapled together, looked terrible. But it was a real leg, and it was still
attached. The boys felt relieved.

Within days, Brian was wheeling himself around the hospital and cracking jokes with nurses, a
green-and-yellow Green Bay Packers cap on his head. While Joey lost himself in coloring books and
television, Isaac attended to his father's every need.

``I feel a little more grown up,'' Isaac said. ``I feel a lot more attached to him than I was when
he left.''

One doctor told Brian that he would never be able to carry a rucksack or run again because of nerve
damage in his left leg. Someone even asked him if he wanted the leg amputated, since he would
certainly be able to run with a prosthetic. Brian refused, and vowed to prove the doctor wrong. By
December, he was walking with a cane and driving.

For Shawn, too, the future had become murkier. It might be many weeks before Brian could reclaim his
sons. But he also knew how glad the boys were to have their father back in one piece.

``Brian came home,'' Shawn said one evening after visiting his brother in the hospital. ``He didn't
come home like we hoped he would come home, but he came home.''

``Every single day I think about all those families and all those kids that are not going to have a
dad come home from Afghanistan,'' he said. ``That hurts more than watching my brother try to take a
step because I know my brother will take a step and I know he's going to walk down the dock and get
in his bass boat someday.''

It was late, and he had to get the boys up the next morning to visit their father at the hospital
again. The holidays were fast approaching and the snow would soon be arriving in Wisconsin. Shawn
wondered whether he could get Isaac out hunting before the season ended.

Yeah, he thought. He probably could.

\section{For Sushi Chain, Conveyor Belts Carry Profit}

\lettrine{T}{he}\mycalendar{Dec.'10}{31} Kura ``revolving sushi'' restaurant chain has no Michelin
stars, but it has succeeded where many of Japan's more celebrated eateries fall short: turning a
profit in a punishing economy.

Efficiency is paramount at Kura: absent are the traditional sushi chefs and their painstaking
attention to detail. In their place are sushi-making robots and an emphasis on efficiency.

Absent, too, are flocks of waiters. They have been largely replaced by conveyors belts that carry
sushi to diners and remote managers who monitor Kura's 262 restaurants from three control centers
across Japan. (``We see gaps of over a meter between your sushi plates -- please fix,'' a manager
said recently by telephone to a Kura restaurant 10 miles away.)

Absent, too, are the exorbitant prices of conventional sushi restaurants. At a Kura, a sushi plate
goes for 100 yen, or about \$1.22.

Such measures are helping Kura stay afloat even though the country's once-profligate diners have
tightened their belts in response to two decades of little economic growth and stagnant wages.

Many other restaurants and dining businesses in Japan have not fared so well. After peaking at 29.7
trillion yen in 1997, the country's restaurant sector has shrunk almost every year as a weak economy
has driven businesses into price wars -- or worse, sent them belly-up. In 2009, restaurant revenue,
including from fast-food stores, fell 2.3 percent, to 23.9 trillion yen --20 percent below the peak,
according to the Foodservice Industry Research Institute, a research firm in Tokyo.

Bankruptcies have been rampant: in 2009, 674 dining businesses with liabilities of over 10 million
yen went under, the highest number in the last five years, according to Teikoku Data Bank, a credit
research company.

In November 2009, Soho's Hospitality, the company behind celebrity restaurants like Nobu and Roy's,
filed for bankruptcy. Roy's is now run by another company, while Nobu's chef, Nobu Matsuhisa, has
opened a new restaurant elsewhere in Tokyo with Robert De Niro.

Along with other low-cost restaurant chains, Kura has bucked the dining-out slump with low prices
and a dogged pursuit of efficiency. In the company's most recent fiscal year, which ended on
Oct.~31, net profit jumped 20 percent from the same period a year earlier, to 2.8 billion yen.

In the last two months alone, Kura has added seven stores.

``If you look at the restaurant business, consumers are still holding back because of employment
fears and falling incomes, and there's no signs that will change,'' said Kunihiko Tanaka, Kura's
chief executive, who opened Kura's first sushi restaurant in 1995. ``Amid these worsening
conditions, our company feels that consumer sentiment matches, or is even a tail wind'' to the Kura
business, he told shareholders earlier this year.

The travails of Japan's restaurant industry, and the changes in Japanese dining habits, may be among
most visible manifestations of how Japan's ``bubble economy'' excesses in the 1980s have given way
to frugal times since the bubble burst in 1990.

With wages weak -- average annual private sector pay has fallen 12 percent in the last decade, to
4.05 million yen, or about \$49,300, in 2009 -- the Japanese now spend less on eating out. An
average single-person household spent 163,000 yen on dining in 2009, 27 percent less than in 2000,
according to detailed budget surveys compiled by the Ministry of Internal Affairs.

In a survey by Citizen Holdings, the watchmaker, of 400 men in their 20s to 50s, the average time
spent at cafes and restaurants plunged from 7 hours and 52 minutes a week in 1990 to 2 hours and 25
minutes in 2010.

An aging population is also depressing restaurant sales. More than one-fifth of Japan's population
is already over 65, and surveys indicate that older people tend to eat out less. The population is
also shrinking, reducing the restaurants' potential customer base.

Meanwhile, Japanese companies have cut back sharply on their entertainment expenses, further hurting
restaurant sales. Total corporate spending on dining and entertainment has halved from a peak of 9.5
trillion yen in 1991 to 4.8 trillion yen in 2008, according to data from the National Tax Agency.

``The restaurant industry here is so linked to the state of the economy, and that's why we're seeing
this decline,'' said Munenori Hotta, a food service industry expert at Miyagi University in Japan.
``In this climate, even top restaurants are having to moderate their prices to keep attracting
customers,'' he said.

Japan's dining-out boom had its roots in the 1970s and 1980s, as incomes grew and rural populations
flocked to big cities. So-called family restaurants brought cheap, Western-style food to the masses
flourished in that era. So did American fast-food chains, which were considered novel at the time.
Kentucky Fried Chicken opened its first restaurant here in 1970, followed by McDonald's in 1971.

At the other end of the price range, a new generation of wealthy Japanese savored imported French
wines at lavish restaurants. By 1986, there were 503,088 restaurants across Japan, according to
records from the Internal Affairs Ministry. That was nearly double the number from 15 years earlier
-- and was more restaurants than now operate in the United States, which has more than twice the
population of Japan.

After the bubble burst in 1990, new low-cost restaurant chains that offered pizzas for as little as
400 yen, or \$4.86, started to spread across Japan, and restaurateurs spoke with alarm of
ready-made, convenience-store meals that were siphoning off sales.

In the depths of the slump, in 1995, Mr.~Tanaka started a company based on serving quality sushi on
the cheap.

His idea of using conveyor belts to offer diners a steady stream of sushi on small plates was not a
new one; an Osaka-based entrepreneur invented such a system in the late 1950s. But Mr.~Tanaka set
out to undercut his rivals with deft automation, an investment in information technology, some
creativity and an almost extreme devotion to cost-efficiency. In Japan, where labor costs are high,
that meant running his restaurants with as few workers as possible.

Instead of placing supervisors at each restaurant, Kura set up central control centers with video
links to the stores. At these centers, a small group of managers watch for everything from wayward
tuna slices to outdated posters on restaurant walls.

Each Kura store is also highly automated. Diners use a touch panel to order soup and other side
dishes, which are delivered to tables on special express conveyor belts. In the kitchen, a robot
busily makes the rice morsels for a server to top with cuts of fish that have been shipped from a
central processing plant, where workers are trained to slice tuna and mackerel accurately down to
the gram.

Diners are asked to slide finished plates into a tableside bay, where they are automatically counted
to calculate the bill, doused in cleaning fluid and flushed back to the kitchen on a stream of
water. Matrix codes on the backs of plates keep track of how long a sushi portion has been
circulating on conveyor belts; a small robotic arm disposes of any that have been out too long.

Kura spends 10 million yen to fit each new restaurant with the latest automation systems, an
investment it says pays off in labor cost savings. In all, just six servers and a minimal kitchen
staff can service a restaurant seating 196 people, said a company spokesman, Takeshi Hattori.

``Its not just about efficiency,'' Mr.~Hattori said. ``Diners love it too. For example, women say
they like clearing finished plates right away, so others can't see how much they've eaten.''

Traditional sushi chefs have not fared so well, however. While the overall market for belt-conveyor
sushi restaurants jumped 42 percent, to 428 billion yen, in 2009 compared with 2003, higher-end
sushi restaurants are on the decline, according to Fuji-Keizai, a market research firm.

``It's such a bargain at 100 yen,'' said Toshiyuki Arai, a delivery company worker dining at a Kura
restaurant with his sister and her 3-year-old son. ``A real sushi restaurant?'' he said. ``I hardly
go anymore.''

\section{Shanghai Schools' Approach Pushes Students to Top of Tests}

\lettrine{I}{n}\mycalendar{Dec.'10}{31} Li Zhen's ninth-grade mathematics class here last week, the
morning drill was geometry. Students at the middle school affiliated with Jing'An Teachers' College
were asked to explain the relative size of geometric shapes by using Euclid's theorem of
parallelograms.

``Who in this class can tell me how to demonstrate two lines are parallel without using a
proportional segment?'' Ms.~Li called out to about 40 students seated in a cramped classroom.

One by one, a series of students at this medium-size public school raised their hands. When Ms.~Li
called on them, they each stood politely by their desks and usually answered correctly. They
returned to their seats only when she told them to sit down.

Educators say this disciplined approach helps explain the announcement this month that 5,100
15-year-olds in Shanghai outperformed students from about 65 countries on an international
standardized test that measured math, science and reading competency.

American students came in between 15th and 31st place in the three categories. France and Britain
also fared poorly.

Experts said comparing scores from countries and cities of different sizes is complicated. They also
said that the Shanghai scores were not representative of China, since this fast-growing city of 20
million is relatively affluent. Still, they were impressed by the high scores from students in
Shanghai.

The results were seen as another sign of China's growing competitiveness. The United States rankings
are a ``wake-up call,'' said Arne Duncan, the secretary of education.

Although it was the first time China had taken part in the test, which was administered by the
Organization for Economic Cooperation and Development, based in Paris, the results bolstered this
country's reputation for producing students with strong math and science skills.

Many educators were also surprised by the city's strong reading scores, which measured students'
proficiency in their native Chinese.

The Shanghai students performed well, experts say, for the same reason students from other parts of
Asia -- including South Korea, Singapore and Hong Kong -- do: Their education systems are steeped in
discipline, rote learning and obsessive test preparation.

Public school students in Shanghai often remain at school until 4 p.m., watch very little television
and are restricted by Chinese law from working before the age of 16.

``Very rarely do children in other countries receive academic training as intensive as our children
do,'' said Sun Baohong, an authority on education at the Shanghai Academy of Social Sciences. ``So
if the test is on math and science, there's no doubt Chinese students will win the competition.''

But many educators say China's strength in education is also a weakness. The nation's education
system is too test-oriented, schools here stifle creativity and parental pressures often deprive
children of the joys of childhood, they say.

``These are two sides of the same coin: Chinese schools are very good at preparing their students
for standardized tests,'' Jiang Xueqin, a deputy principal at Peking University High School in
Beijing, wrote in an opinion article published in The Wall Street Journal shortly after the test
results were announced. ``For that reason, they fail to prepare them for higher education and the
knowledge economy.''

In an interview, Mr.~Jiang said Chinese schools emphasized testing too much, and produced students
who lacked curiosity and the ability to think critically or independently.

``It creates very narrow-minded students,'' he said. ``But what China needs now is entrepreneurs and
innovators.''

This is a common complaint in China. Educators say an emphasis on standardized tests is partly to
blame for the shortage of innovative start-ups in China. And executives at global companies
operating here say they have difficulty finding middle managers who can think creatively and solve
problems.

In many ways, the system is a reflection of China's Confucianist past. Children are expected to
honor and respect their parents and teachers.

``Discipline is rarely a problem,'' said Ding Yi, vice principal at the middle school affiliated
with Jing'An Teachers' College. ``The biggest challenge is a student who chronically fails to do his
homework.''

While the quality of schools varies greatly in China (rural schools often lack sufficient money, and
dropout rates can be high), schools in major cities typically produce students with strong math and
science skills.

Shanghai is believed to have the nation's best school system, and many students here gain admission
to America's most selective colleges and universities.

In Shanghai, teachers are required to have a teaching certificate and to undergo a minimum of 240
hours of training; higher-level teachers can be required to have up to 540 hours of training. There
is a system of incentives and merit pay, just like the systems in some parts of the United States.

``Within a teacher's salary package, 70 percent is basic salary,'' said Xiong Bingqi, a professor of
education at Shanghai Jiaotong University. ``The other 30 percent is called performance salary.''

Still, teacher salaries are modest, about \$750 a month before bonuses and allowances -- far less
than what accountants, lawyers or other professionals earn.

While Shanghai schools are renowned for their test preparation skills, administrators here are
trying to broaden the curriculums and extend more freedom to local districts. The Jing'An school,
one of about 150 schools in Shanghai that took part in the international test, was created 12 years
ago to raise standards in an area known for failing schools.

The principal, Zhang Renli, created an experimental school that put less emphasis on math and allows
children more free time to play and experiment. The school holds a weekly talent show, for example.

The five-story school building, which houses Grades eight and nine in a central district of
Shanghai, is rather nondescript. Students wear rumpled school uniforms, classrooms are crowded and
lunch is bused in every afternoon. But the school, which operates from 8:20 a.m. to 4 p.m. on most
days, is considered one of the city's best middle schools.

In Shanghai, most students begin studying English in first grade. Many middle school students attend
extra-credit courses after school or on Saturdays. A student at Jing'An, Zhou Han, 14, said she
entered writing and speech-making competitions and studied the erhu, a Chinese classical instrument.
She also has a math tutor.

``I'm not really good at math,'' she said. ``At first, my parents wanted me to take it, but now I
want to do it.''

\section{Suspicious Death Ignites Fury in China}

\lettrine{T}{he}\mycalendar{Dec.'10}{31} photograph is so graphic that it appears cartoonish at
first glance.

A man lies on a road with his eyes closed, blood streaming from his half-open mouth, his torso
completely crushed under the large tire of a red truck. One arm reaches out from beneath the tire.
His shoulder is a bloody pile of flesh. His head is no longer attached to the flattened spinal cord.

The man in the photograph, Qian Yunhui, 53, has become the latest Internet sensation in China, as
thousands of people viewing the image online since the weekend have accused government officials of
gruesomely killing Mr.~Qian to silence his six-year campaign to protect fellow villagers in a land
dispute. Illegal land seizures by officials are common in China, but the horrific photographs of
Mr.~Qian's death on Saturday have ignited widespread fury, forcing local officials to offer
explanations in a news conference.

It is the latest in a string of cases in which anger against the government has been fanned by the
lightning-fast spread of information online. In late October, the son of a deputy police chief in
central China drunkenly drove his car into two college students, killing one and injuring another.
His parting phrase as he drove away from the scene of the crime -- ``Sue me if you dare, my father
is Li Gang!'' -- has since become a byword for official corruption and nepotism.

Officials in the city of Yueqing in Zhejiang Province, which supervises Mr.~Qian's home village,
insist that the photographs show nothing more than an unfortunate traffic accident. They made their
case in a hastily arranged news conference on Monday afternoon, as the images of Mr.~Qian's death
continued proliferating on the Internet. Mr.~Qian's family, some Chinese reporters and residents of
Zhaiqiao Village cite the photographs as proof of foul play and a sloppy cover-up.

It is unclear who took the photographs, but they first appeared Sunday afternoon on Tianya, a
popular online forum for discussing Chinese social issues.

Within 36 hours, the initial post attracted nearly 20,000 comments. It has since been deleted.
Tianya and two other Web sites that reported on the case together got 400,000 hits, according to
Xinhua, the state news agency. The Chinese government goes to great lengths to block servers here
from accessing information it deems harmful to political stability, but censors have apparently
failed to keep up with the proliferation of blog posts related to Mr.~Qian. Once the information had
spread, higher authorities apparently found it necessary to show the public they were looking into
the matter -- officials from the nearby city of Wenzhou ordered police officers from there to go to
Yueqing to assist the investigation, Xinhua reported.

Chinese Internet users were drawn not only to the gruesome images, but also to the fact that the
land dispute involving Mr.~Qian is a common narrative in China.

In 2004, the city government approved construction of a power plant in Zhaiqiao Village. The company
building the plant got virtually all the arable land in the village, and the 4,000 or so villagers
received no compensation, according to a blog post on Tianya that was written four months ago under
Mr.~Qian's name. At the time, Mr.~Qian and other villagers went to government offices to protest the
land grab, and riot police officers beat more than 130 people and arrested 72, the post said.

Mr.~Qian, the former Communist Party representative in the village, traveled to Beijing to file a
petition with the central authorities. In the news conference on Monday, city officials said that
Mr.~Qian had been arrested, found guilty of criminal conduct and imprisoned at least twice. Mr.~Qian
continued his crusade after recently being released from prison. Before his death, he was the
overwhelming favorite of the villagers in a coming election for village chief, according to local
media reports.

Around 8:30 a.m. on Saturday, Mr.~Qian received a call on his cellphone and walked out as he was
talking, according to a report by Chinese Business News that cited Mr.~Qian's wife, Wang Zhaoyan.

An hour later, he was run over by the red truck, his body crushed beneath the left front tire. The
driver, Fei Liangyu, has been detained, according to a statement on the Yueqing city government Web
site.

Chinese news reports said another villager, Qian Chengwei, told people that he had watched as the
victim was held down in the road by several men wearing security uniforms. One of the men waved his
hand, and a truck then drove slowly over Mr.~Qian, the reports said. Villagers arriving at the scene
were immediately suspicious. They refused to allow the police to remove Mr.~Qian's body, and a
scuffle ensued.

The witness and the victim's family members were detained, according to Southern Daily, a newspaper
based in Guangdong Province. Government officials told the newspaper that the witness was a drug
user.

Local news organizations reported Tuesday that Mr.~Qian's family members have been released. Phone
calls to Mr.~Qian's home were not answered.

Internet users and Chinese reporters have continued to question the explanation by city officials,
pointing to discrepancies revealed by the photos. Why does the front of the truck show little sign
of impact or blood? Why, if Mr.~Qian had been accidentally hit while walking upright, is his body
lying completely perpendicular to the truck's tire? Why was a brand-new security camera at the
intersection where Mr.~Qian killed not working on Saturday? Who called Mr.~Qian on that fateful
morning?

``A few years ago, there were other people petitioning with my dad,'' one family member, Qian
Shuangping, told China Business Daily. ``Some of them were bought off. Some of them got scared. We
said: 'Just take some money and forget it. What if something happens to you?' My father wouldn't
listen to us.''

\section{Hu shuli}

\lettrine{O}{n}\mycalendar{Apr.'11}{27} May 12, 2008, Hu Shuli, the founding editor of the biweekly
magazine Caijing, was hosting a ceremony for scholarship recipients at a hotel in the mountains west
of Beijing. When a text message informed her that a powerful earthquake had struck the province of
Sichuan, she leaned over to the man next to her, a veteran editor named Qian Gang, who had covered
previous quakes, and asked him for a rough prediction of the damage. At least it hadn't struck while
everyone was asleep, he figured. He soon realized, however, that school was in session and ``the
casualties among students would be enormous.''

Hu set off for downtown Beijing, working the phone and e-mailing from the back seat of a car,
directing her staff to rent a satellite phone and get a crew to Sichuan. Petite, voluble, and
pugnacious--``a female Godfather,'' one of her reporters thought upon first meeting her--Hu was
determined to cover the story even though in China reporting on a disaster of this scale could be
politically hazardous. When the country suffered its previous huge quake, in 1976, the government
suppressed news of the death toll for three years.

But Hu had made her name divining the boundaries of free expression in China. In the decade since
she founded Caijing--the name means ``finance and economics''--she had sharply defied the image of
China's somnambulant press, and become, as David Ignatius, of the Washington Post, put it to me, the
country's ``avenging angel.'' Hu had endured as editor long after other tenacious Chinese
journalists had been imprisoned or silenced. She was often described in the Chinese and foreign
press as ``the most dangerous woman in China,'' and she was still in business.

Within the hour, the first Caijing journalist was on a flight to Sichuan, followed by nine more.
While Xinhua, the state-run news service, was emphasizing that the earthquake ``tugged at the
heartstrings of the Chinese Communist Party,'' Caijing was ferreting out estimates of the numbers of
the dead and wounded and noting that ``many disaster victims have yet to receive any relief
supplies.''

Schools lay in heaps of concrete and rebar, and the Central Propaganda Department, a government
agency with the power to remove editors and shut down publications, banned coverage of rescue
efforts at the schools. Several Chinese newspapers questioned why so many schools had collapsed
anyway, producing poignant stories of construction errors and the human toll. (At least fifty-three
hundred schoolchildren are believed to have died.) Hu had heard that local authorities were
criticizing papers that continued to report on the schools' collapse, but she believed that Caijing
could find a way to write about it. She thought that a story could be published if it carried the
right tone and facts. ``If it's not absolutely forbidden,'' she said, ``we do it.''

On June 9th, Caijing published a twelve-page investigative report that was cool and definitive.
According to the report, heedless economic growth, squandered public funds, and rampant neglect of
construction standards had led to the disaster. The story detailed how local cadres cut corners, but
it stopped short of assigning responsibility by name. When I asked Hu about the government's
reaction, she said, ``They got angry. Very, very angry.'' But she and Caijing were never punished.

In the world of Chinese journalists, or ``news workers,'' as they are known in Party-speak, Hu, who
is fifty-six years old, has a singular profile. She is an incurable muckraker, and in 1989 was
suspended from a reporting job because of her sympathy for the Tiananmen Square demonstrations, yet
she has gone on to cultivate first-name familiarity with some of China's most powerful Party
leaders.

Five feet two and slim, with a pixie haircut and a wardrobe of color-coördinated outfits, she is
usually heard before she is seen. In the newsroom of Caijing, a sleek and open gray brick space on
the nineteenth floor of the Prime Tower, in downtown Beijing, her arrival is heralded by the urgent
click-clack of heels down the hallway. She sweeps through the newsroom, spouting decrees and ideas,
and then heads out the door again--``as sudden and rash as a gust of wind,'' said Qian, who is now a
researcher at the University of Hong Kong.

More than one person I know likens the experience of chatting with Hu to being on the receiving end
of machine-gun fire. Some have less appetite for her intensity. Wang Lang, an old friend of Hu's and
an editor at Economic Daily, a state-run newspaper, has repeatedly declined her offers to work
together, because, he said, ``Keeping some distance is better for our friendship.'' Depending on the
point of view, being with her is either thrilling or unnerving. Her boss, Wang Boming, the chairman
of Caijing's parent company, the SEEC Media Group, told me, half jokingly, ``I'm afraid of her!''

Since 1998, when Hu established Caijing, with two computers and a borrowed conference room, she has
guided the magazine with near-perfect pitch for how much candor and provocation the regime will
tolerate. That has meant deciding what to cover--rampant corporate fraud, the government cover-up of
the SARS virus, case after case of political corruption--but also what not to cover (Falun Gong, the
Tiananmen Square anniversary). At a time when the American print media is in decline, the Chinese
press is growing, and Caijing is the first Chinese publication with the prospect of becoming a
world-class news organization. ``It's different from everything you see in China,'' Andy Xie, a
former Morgan Stanley economist who writes a column for Caijing, said. ``Its existence, in a way, is
a miracle.''

Caijing has the glossy feel and design of Fortune. It is heavy with advertising, for Cartier
watches, Chinese credit cards, Mercedes S.U.V.s. The writing can be purposefully dense, and even
\'elitist, but China's propaganda officials are more likely to clamp down on television and
mass-market newspapers, which have audiences in the millions, than they are on a magazine that sells
only two hundred thousand copies. But those copies go to many of China's most important offices in
government, finance, and academia, giving the magazine extraordinary influence. In recent years, it
has begun to expand that reach, through a pair of Web sites, in Chinese and English, that are
loosely modelled on nytimes.com. Together, the sites attract some 3.2 million unique visitors every
month. Hu writes a widely quoted column for the print edition and the Web. She also oversees a
conference series that attracts the economic leadership of the Communist Party. Caijing's newest
project, yet to be unveiled, will take direct aim at the likes of Bloomberg and Dow Jones: an
English-language wire service, in partnership with the Hong Kong tycoon Richard Li Tzar-kai, that
will distribute stories by Caijing reporters.

The first time that Sam Popkin, a political scientist at the University of California at San Diego,
who, along with his wife, the China scholar Susan Shirk, has known Hu for many years, watched Hu
report a story, it reminded him of the portrait of the Times reporter R.~W. Apple in ``The Boys on
the Bus,'' when ``Apple used to make something like a hundred calls a day,'' Popkin said. ``She is
always figuring out who in the system really has the power to know what's going on.'' Popkin added,
``She is a human USB drive. You fill her drive, and she goes on to someone else.'' Inevitably, her
competitors' memories are the clearest. Nearly two decades ago, Lin Libo, then a reporter for a
leading business newspaper, struggled to match her coverage of a round of closed-door negotiations.
He recalled, ``She even had the menu!''

In 1992, as the international editor at China Business Times, one of the country's first national
business papers, Hu began covering the work of a small number of Chinese who had trained in Western
finance and, returning from overseas, were promoting the Chinese stock markets. Many of them were
her age and were the children of powerful Chinese leaders. The group called itself the Stock
Exchange Executive Council, and rented a cluster of rooms at Beijing's Chongwenmen Hotel. The
members pulled out the beds and set up an office. At one desk was Gao Xiqing, who had earned a law
degree at Duke and worked at Richard Nixon's law firm in New York before returning to China. At
another was Wang Boming, the son of a former ambassador and vice-foreign minister; Wang had studied
finance at Columbia and worked as an economist in the research department of the New York Stock
Exchange. They enlisted the support of rising stars in the Party, such as Wang Qishan, who was the
son-in-law of a vice-premier, and Zhou Xiaochuan, a reform-minded political scion.

``I decided to interview all the top financiers in China,'' Hu recalled. She called it her
``homework,'' and James McGregor, then a Wall Street Journal reporter in Beijing, began noticing Hu
``working all these people, pumping them for information like a graduate student talking to esteemed
professors.'' Hu ended up with a string of scoops and, eventually, an incomparable Rolodex of names
destined for China's highest offices: today, Gao Xiqing is the head of China's
two-hundred-billion-dollar sovereign-wealth fund; Wang Qishan is a vice-premier and a top economic
policymaker; Zhou Xiaochuan runs China's central bank.

Many people in Beijing wonder how much those early connections have protected Hu. But she insists
that people overestimate her proximity to power. ``I don't know their birthdays,'' she said, of
high-ranking officials. ``I'm a journalist, and they treat me as a journalist.''

Hu's connections seem to serve a more subtle function. By positioning herself on the border between
insider and outsider, between Communist history and the capitalist present, between protecting
China's interests and embracing the world, she has become an invaluable interpreter. When, in the
weeks before the 2008 Olympics, the Chinese government was wound so tightly that it had begun to
look thuggish, she used an editorial to condemn clashes between the police and reporters. She
preached ``self-confidence, open-mindedness, friendliness.'' ``To use an English expression,'' she
added, referring to Chinese organizers, ``they should take it easy.'' It is a delicate role. Once,
Hu had to choose the cover photograph for a high-profile year-end edition. Editors had narrowed the
choices: a staid collage of newsy images or an edgy shot of a woman shouldering into a sandstorm,
her face shrouded in a scarf. Hu favored the provocative picture, but at the last moment she
hesitated. ``Could it get us into trouble?'' she asked, according to someone who was present. ``Is
it too negative about China?'' Others argued that it showed China's best side--its
determination--and Hu smiled. ``I can explain that,'' she said.

The Chinese press is no longer cowed into complete compliance, but it is also not yet as free as
other parts of a raucous economy. Caijing and its news values are a minority. Last September, Xinhua
published a story on its Web site detailing how China's Shenzhou VII rocket made its thirtieth orbit
of the earth. The story had plenty of gripping detail--``The dispatcher's firm voice broke the
silence on the ship.'' Unfortunately, the rocket had yet to be launched. (The news service later
apologized for posting a ``draft.'') Of China's two thousand newspapers and eight thousand
magazines, Caijing and several business newspapers are among the few publications with independent
voices and private funding. (All Chinese media are required to have a government-affiliated sponsor,
though the level of interference varies. The SEEC Media Group, which is traded on the Hong Kong
stock exchange, is controlled by fifteen individual investors.)

The Chinese leadership has been especially wary of press reform ever since Tiananmen Square. ``Never
again would China's newspapers, radio, and television be permitted to become a battle front for
bourgeois liberalism,'' President Jiang Zemin vowed, according to internal Party documents collected
by Anne-Marie Brady, a specialist on Chinese media at the University of Canterbury in New Zealand.
Chinese journalists do not face the kind of gangland killings that beset reporters in Russia, but
Reporters Without Borders, in its most recent global index of press freedom, ranks China a hundred
and sixty-seventh out of a hundred and seventy-three countries--just behind Iran and ahead of
Vietnam. Article 35 of the Chinese constitution guarantees freedom of speech and the press, but it
is no match in court against a web of laws on libel and on revealing state secrets. The Committee to
Protect Journalists 2008 report counted twenty-eight reporters in Chinese jails, more than in any
other country. (Earlier this month, Iran overtook China, for the first time in ten years.)

The Central Propaganda Department, as it's known in Chinese, operates in semi-secrecy, with no sign
on its headquarters and no listed phone number. It issues directives to editors and publishers that
outline the latest recommendations of dos and don'ts. Some boundaries are fixed; taboos include the
military, religion, ethnic disputes, and the inner workings of government. But others are flexible.
In the fall of 2005, editors enjoyed a free hand to report on a catastrophic chemical leak in the
Songhua River. Weeks later, news sites were ordered to stop reporting the case of a surgeon who had
spoken on the phone during surgery and paralyzed a patient's face. (Even revealing the contents of a
directive can be dangerous for a Chinese reporter. Shi Tao, a contributor to the Contemporary
Business News, is serving a ten-year sentence for describing a directive from his local propaganda
authorities in an e-mail that he sent abroad.)

A publication's first offense usually draws a warning ``yellow card,'' as in soccer. Three yellow
cards in one year, journalists say, and a paper or magazine is shut down. (In 2004, three hundred
and thirty-eight publications were suspended for printing ``internal'' information, according to a
report in the state press.) But it's up to editors themselves to guess how far they can go and
compute the risk of exceeding an undefined limit--an approach to censorship that the China scholar
Perry Link, a Princeton emeritus professor, has likened to ``a giant anaconda coiled in an overhead
chandelier.'' ``Normally, the great snake doesn't move,'' he wrote in the New York Review of Books
in 2002. ``It doesn't have to. It feels no need to be clear about its prohibitions. Its silent
constant message is 'You yourself decide,' after which, more often than not, everyone in its shadows
makes his or her large and small adjustments--all quite 'naturally.' ''

The first time that I took a cab to visit Hu at home, I was sure that I was lost. Unlike many of the
reporters and editors on her staff, she does not live in one of Beijing's new residential
high-rises. She and her husband, Miao Di, a film professor at Beijing's Communication University of
China, share a three-bedroom apartment in a concrete housing block with a view of an overgrown
garden. In the nineteen-fifties, when the building was a privileged residence for Party cadres, the
government assigned space in it to Hu's father. The neighborhood is China's old-media stronghold,
home to the headquarters of the state radio and to China's film-and-television censors.

Hu's drive to work takes twenty minutes, and whisks her from one century to another; by the time she
reaches the Caijing offices, she is next door to the Beijing bureau of the Wall Street Journal.
Heading to work one recent afternoon, she was running late for an unusual appointment: Hu had
decided that her top editors needed new clothes, and she had summoned a tailor. As Caijing grows in
prominence, her staff is spending more time in front of crowds or overseas. ``Foreigners always wear
suits this way,'' she said, approvingly pulling her jacket tightly around her, as she hastened to
her car. She had offered her editors a deal: Buy one new suit and the magazine would pay for
another. The tailor carried an armful of suits into a conference room, and the staff filed in for a
fitting.

``Doesn't it look baggy here?'' Hu said, tugging at the underarm of an elegant gray pin-striped
jacket being fitted to Wang Shuo, her thirty-seven-year-old managing editor. With his boss prodding
at his midsection, he wore an expression of bemused tolerance that I had seen several times on a dog
in a bathtub.

``It is rather tight already,'' Wang protested.

``He feels tight already,'' the tailor said.

``Hold on!'' Hu said. ``Think about the James Bond suit in the movies. Make it like that!''

The change implied by Hu's flamboyant internationalism runs deeper than aesthetics. A well-meaning
American professor once advised her, ``If you stay in China as a journalist, you will never really
join the international mainstream.'' She seems determined to prove him wrong.

On her mother's side, Hu comes from a line of Communist Party journalists and intellectuals. Her
grandfather Hu Zhongchi was a famous translator and editor at Shen Bao, a Shanghai paper. His elder
brother Yuzhi founded a publishing house that produced collections of Lu Xun, one of modern China's
leading writers and a family friend, as well as Chinese translations of Edgar Snow and John
Steinbeck.

Hu's mother, Hu Lingsheng, was a senior editor at Workers' Daily in Beijing. Her father, Cao Qifeng,
studied English at a missionary school before becoming an impassioned underground Communist and
taking a mid-level post in a trade union. They named their younger daughter Shula, for a Soviet war
martyr. In the nineteen-seventies, she changed her name to Shuli, a more popular name for women.

Hu has an acute understanding of China's fickle regard for intellectuals. Her great-uncle Yuzhi was
a deputy minister of culture before the Cultural Revolution. ``But we were told never to say that to
others,'' Hu told me. Her forthrightness at times worried her parents. ``I was not very disciplined.
I always spoke about what I was thinking.'' She attended Beijing's \'elite 101 Middle School, which
had once educated many offspring of Party cadres. Students had privileged access to a smattering of
banned foreign literature, including translated volumes of Kerouac, Salinger, and Solzhenitsyn,
printed in limited batches for Party \'elites. Hu would also take books from her family and hide
them under her pillow until she could swap them with friends.

The Cultural Revolution engulfed China when Hu was thirteen, and her classes were suspended. As a
prominent editor, Hu's mother was criticized at her newspaper and placed under house arrest. Her
father was shunted into a backroom job. Like others her age, Hu became a Red Guard and travelled
around the country. As the movement descended into violence, she sought refuge in books, trying to
maintain a semblance of an education. ``It was a very confusing time, because we lost all values,''
she said. A month before her sixteenth birthday, she was sent to the countryside to experience the
rural revolution.

``It was ridiculous,'' she said of what she found. Farmers had lost any incentive to work. ``They
just wanted to stay lying in the field, sometimes for two hours. I said, 'Should we start work?'
They said, 'How can you think that?' '' She went on, ``Ten years later, I realized everything was
wrong.'' Hu's sister, Cao Zuyoa, who was in a nearby village, later wrote a book, ``Out of the
Crucible,'' about how the rustication campaign forever changed members of their generation. It
``buried their Communist utopian dream,'' she wrote.

After two years, Hu joined the Army--which led, several years later, to membership in the Communist
Party--and she was assigned to a remote hospital in a rural northern area of Jiangsu Province, where
she spent the next eight years. She worked in the dining room, fed pigs, helped on the wards, and
ran a tiny broadcast booth that played music and announcements. When colleges resumed classes, in
1978, Hu secured a coveted seat at People's University in Beijing. The journalism department was not
her first choice, but it was the best that the school offered. She was a conspicuous figure on
campus: the department's only freshman girl in a military uniform. ``There was not a person in our
class who didn't know who she was,'' Miao Di, a history major from a Beijing military family who met
Hu in an English class, said. He, too, had been sent to the countryside, and they shared a sense of
disaffection. They married in 1982.

After college, Hu joined Workers' Daily, and, in 1985, after some early investigative projects, she
was assigned to a bureau in the southeastern coastal city of Xiamen. The area had been designated as
a laboratory for the growth of a free market. There she developed her skill for networking, meeting
everyone in city hall--including the mayor, with whom she played bridge. Among those she interviewed
was a promising young cadre who was the city's vice-mayor: Xi Jinping, the son of a Politburo
member. Xi was a Party loyalist with pro-market sensibilities, whose building of a successful theme
park had earned him the nickname the God of Wealth. Today, Xi is China's Vice-President and is
regarded as the heir apparent to the President.

In 1987, Hu won a fellowship from the Minnesota-based World Press Institute to spend five months in
America. The experience was a revelation. ``I spent the whole night reading the St.~Paul Pioneer
Press'' and marvelling at the size of it, she said. (Workers' Daily, at the time, was four pages.)
She met investigative reporters at the Philadelphia Inquirer and interned at USA Today. She returned
to China, and in the spring of 1989 the Tiananmen Square movement energized the Beijing press. For
weeks, papers revelled in a holiday from censorship. Many journalists, including Hu, joined the
demonstrations. As soldiers cracked down on the night of June 3rd, Hu recalls, ``I went to the
street, then went back to the office and said, 'We should cover this.' '' But the decision had
already come down: ``The newspaper decided we weren't going to publish a word about it.'' Her
involvement was costly. Many reporters who had spoken out were fired or banished to the provinces.
Miao Di thought that Hu might get arrested, but, in the end, she was suspended for eighteen months.

She used the time to write ``Behind the Scenes at American Newspapers,'' the first Chinese book to
examine the relationship between the American press and democracy, with descriptions of Watergate
and the Pentagon Papers. It was a must-read among Chinese news workers, and in it she posed a
challenge: Who among them ``could take the initiative and do something akin to what American news
organizations have done?''

In 1998, Hu received a phone call from Wang Boming, one of the hotel-room founders of the Stock
Exchange Executive Council; he was starting a magazine and he wanted her to run it. She had two
conditions: Wang would never interfere in her newsroom, and he would give her a budget of two
million yuan--about a quarter of a million dollars--to pay for serious reporting trips and salaries
that were high enough to prevent reporters from taking bribes. Wang agreed. It was no charity: he
and his reform-minded allies in the government saw the magazine as an extension of their
determination to modernize the economy.

``You need the media to play its function to disclose the facts to the public, and, in a sense, help
the government detect evils,'' Wang told me recently, in his large, cluttered office downstairs from
Caijing's headquarters. He is a classic type of the generation that received an American education
and returned to China--a chain-smoker with a thick brush of gray-flecked black hair, Ferragamo
eyeglasses, and a bilingual sense of humor. When he talks about Hu, a weary look crosses his face
that suggests he has got more than he bargained for. ``We didn't know that this level of risk would
come along with it,'' he said. But Wang also betrays a keen sense of Hu's significance to China.
``When I was studying in the States, I needed to make some money to pay my tuition, so I was working
for a newspaper in Chinatown--the China Daily News,'' he said. As a cub reporter, he relished the
chance to follow a trail wherever it led. He had felt like ``a king without a crown.''

Caijing established its tone right away. Its inaugural issue, in April, 1998, featured an explosive
cover story detailing the case of Qiong Min Yuan, a real-estate company whose share price had
quadrupled before it was charged with overstating profits. Caijing revealed that, while legions of
small-time investors lost millions, insiders had been tipped off in advance and unloaded their
shares. Regulators were incensed; they accused Caijing of flouting press restrictions, and Wang's
executives had to troop to the regulators' office to make self-criticisms.

Each story refined Hu's calculation of how far she could push. In 2001, a twenty-five-year-old
Caijing reporter, thumbing through customs records, discovered that Yinguangxia Holdings, one of
China's largest listed companies, had posted online a falsified claim of eighty-seven million
dollars in profits. The political stakes were high, because a parade of top leaders had already
visited the company to praise it. Wang Boming was so worried that Caijing would be shut down if it
went with the story that he did something he said he has never done again: he called a high-ranking
Communist Party official for approval before publishing. ``He said, 'Is that story really true or is
there any doubt?' '' Wang recalled. ``I said, 'The story is definitely true, but there is political
implication there.' He said, 'If it's true, then go ahead.' '' Hours after the story appeared, the
company's stock was suspended from trading; eventually, its executives went to jail.

The defining moment in Caijing's emergence, however, came two years later, when the reporter Cao
Haili, arriving in Hong Kong, noticed that every person on the train platform seemed to be wearing a
surgical mask. What the hell is that about? she thought, and alerted Hu. The Chinese press had been
running reports of a mysterious new virus, but health officials had promised the public that it was
contained. Newspaper editors in Guangdong Province had been privately instructed to publish
reassuring stories about the virus, and some were even told what typeface to use, one editor at the
time recalled. But these restrictions did not extend to editors outside Guangdong. ``I bought a lot
of books about breathing diseases, infections, and viruses,'' Hu said, and her staff began to find
errors in the government's statements. Meanwhile, Caijing editors tracked the Web site of the World
Health Organization, which tallied a steadily growing number of SARS cases in China, even as the
government continued to deny it. The tone of the coverge was serious and questioning without
actually accusing the government of lying.

Over the course of a month, Caijing produced a series of weekly supplements on SARS in addition to
its regular issues. In the end, the magazine brushed up against the limit. ``Caijing was planning
another issue that was going to look back on the lessons of SARS,'' David Bandurski, a researcher at
the University of Hong Kong's China Media Project, said. ``And the government essentially sent the
message 'No--this is not going to happen. This stops now.' ''

Hu has cultivated her sense for the precise moment when a sensitive subject is safe to publicize.
``You can feel her making adjustments,'' Wang Feng, a former Caijing editor, told me. ``For example,
at Monday's editorial conference she might aim at something, and the editors and reporters go ahead
and do it. And by Wednesday's editorial conference she will say, 'You know what? I've got more
information on this and we should not say that. Maybe we should aim lower.' '' In
political-corruption cases--which are acutely sensitive--Caijing's investigative reporters often
collect information for weeks or months while they wait for an opening. In many cases, once Xinhua
makes a brief announcement of an official being arrested, Caijing is ready to publish a full story.
When, on June 8th, Xinhua issued a one-sentence report saying that the mayor of Shenzhen had been
detained in a corruption probe, Caijing posted an in-depth piece twentynine minutes later.

After SARS, Caijing never entirely retreated to the confines of business journalism, though today it
benefits from the perception that it is simply policing the economy. As Caijing's scoops mounted,
banking regulators began calling reporters, looking for tips, instead of the other way around. Even
more satisfying, Western media had no choice but to follow and credit Caijing's leads. At a certain
point, the magazine's success and bravado had become self-reinforcing: it had gone so far already
that conservative branches of the government could no longer be sure which other officials supported
it.

And then Caijing had its first lesson in what happens when it goes too far. In January, 2007, its
cover story ``Whose Luneng?'' described how a group of investors had paid a pittance for control of
a giant conglomerate, with assets ranging from power plants to a sports club. The conglomerate,
Luneng, was valued at more than ten billion dollars, but a pair of little-known private companies
had paid just under five hundred and fifty million dollars for ninety-two per cent of the company,
Caijing reported. State regulators had received no notification of the sale--which was ordinarily a
legal requirement--and a tangle of overlapping boards and shareholders seemed designed to obscure
the identity of the new owners and where their money originated. Nearly half of the purchasing
capital came from an untraceable source, Caijing discovered.

After Caijing attempted to publish a brief follow-up, authorities ordered the story removed from the
Web site and the newsstands. The staff in Caijing's Shanghai office are said to have torn up issues
by hand. ``Everybody felt humiliated,'' a former editor said.

Since then, Caijing has referred occasionally to its Luneng investigation, but Hu is not eager to
discuss the case; she considers that run-in with the government the magazine's ``largest disaster.''
A person who is involved with Caijing and is familiar with the story said that revealing the attempt
to profit wildly from the privatization deal had come too close to implicating the children of
senior Party leaders--a taboo that trumps even reformists' desire for a more open press.

In 2007, the Nieman Foundation, at Harvard, gave Hu an award for ``conscience and integrity.'' The
award was well deserved, but it placed her in slightly awkward company: previous winners included a
publisher in Iran who was repeatedly summoned to court for her magazine's reporting and an editor in
Zimbabwe who had been arrested and tortured by the military.

Hu does not live the marginalized life of a samizdat editor or sign dissidents' communiqu\'es. For
all her skepticism and intensity, her writing is notably short on outrage. When she criticizes,
either in her column or in her editing choices, she uses the language of loyal opposition. Following
the conviction of a high-level official in a pension-fund scandal that erupted in 2006, she didn't
challenge the moral peril of one-party government but argued that China's weak asset-disclosure laws
let officials' relatives and colleagues profit. In a column from 2007 entitled ``Yearning for
Reform,'' she declared that ``the public's top concern is the rampant corruption and an imbalanced
power system.'' She went on, ``Some argue that pushing forward with political reform will be
destabilizing. Yet, in fact, maintaining the status quo without any reform creates a hotbed for
social turbulence.'' In other words, political reform is the way to consolidate power, not lose it.

In her office one afternoon in June, shortly after the anniversary of the Sichuan earthquake, I
asked Hu why she thought that other publications had been punished for covering the collapsed
schools, while Caijing had not. ``We never say a word in a very emotional or casual way, like 'You
lied,' '' she said. ``We try to analyze the system and say why a good idea or a good wish cannot
become reality.''

If a Chinese paper dwells on the names of specific officials who allowed the construction of unsafe
schools, it might be scoring a point for accountability, but the investigative coup also leaves the
paper vulnerable to petty political retribution. Hu said, ``We try not to give any excuses to the
cadres who don't want to get criticized.'' Ultimately, she said, the important question is not
``which person didn't use good-quality bricks fifteen years ago'' but something deeper. ``We need
further reform,'' she said. ``We need checks and balances. We need transparency. We say it this way.
No simple words. No slogans.''

That approach appeals to reformers in the government who might genuinely want to solve problems but
don't want to give up power to do so. Some Chinese journalists say that Hu's greatest skill is
playing interest groups against one another, whether by amplifying the central government's effort
to round up corrupt mayors or by letting one wing of the government thwart a rival wing's agenda.
Allow the most powerful group to endure, the theory goes, and you can do real--even
profitable--journalism. The danger, however, is that, as Caijing's profile and financial stakes
grow, the magazine can afford to take fewer risks. A reader recently posted a comment on the Web
site saying, ``Caijing has become increasingly mainstream\ldots . The flavor of critical thinking is
gone, replaced by things designed to appeal to the interests of readers and subscribers, but lacking
soul.''

Caijing has proved to be attentive to the Chinese government's views on what is unmistakably
off-limits. As ethnic violence erupted in the Xinjiang capital of Urumqi last week, the magazine
sent two reporters, whose spot-news coverage described the violence and damage but, in their early
reporting, did not venture into examining underlying causes of the unrest. Likewise, in the fifteen
months since an uprising in Tibet had exposed a well of explosive discontent facing Chinese
authorities in some of the nation's ethnic regions, Caijing had largely steered clear of the story.

I had lunch not long ago with Cheng Yizhong, a former editor-in-chief of the Southern Metropolis
Daily, one of China's liveliest papers. He became famous in China for publishing an investigation
into the death, in 2003, of Sun Zhigang, a young graphic designer who died in police custody after a
beating. Sun, reporters discovered, had been transferred to a shourong, or ``custody and
repatriation'' station. Ostensibly designed to process vagrants and runaways, the shourong system
was widely resented; it gave police the right to ask people on the street for identification and
residency papers and to detain them, with little cause. Those who were unable to pay fees could earn
their release by working at prison-run farms and factories. Sun, it was later discovered, had been
stopped on the street by police who detained him even though he insisted that his papers were in
order. The newspaper's report had triggered a wave of public outrage over the shourong system.
Followup articles that ran in the Southern Metropolis Daily and other newspapers revealed that the
system was profitable for local police and had spawned a nationwide network of seven hundred
detention camps; in at least one area, it was reported, stations bought inmates in order to maximize
revenues. In August, 2003, the scandal prompted the central government to abolish the system, an
astonishing case of the Chinese press influencing national policy. But within a year Cheng had been
detained, and two of his colleagues arrested on charges of illegally distributing bonuses awarded by
the editorial board. The charges were widely considered as retribution for the newspaper's reporting
on the Sun case and also for its previous coverage of the SARS virus. Cheng spent five months in a
detention center and now works in a low-profile media job. His two colleagues were sentenced to long
jail terms.

I asked Cheng why he and Hu had fared so differently. Caijing, he said, had achieved a stature that
put it out of reach of lower-ranking bureaucrats. But he also drew a distinction between his
campaign for radically curtailing police powers and Caijing's focus on raising government
performance. ``Caijing's topics haven't affected the fundamental ruling system, so it is relatively
safe,'' he said, adding, ``I am not criticizing Hu Shuli, but in some ways Caijing is just serving a
more powerful or relatively better interest group.''

Unsurprisingly, Hu sees it slightly differently: ``We don't think about one group or another--we
think about the whole system and whatever can achieve reform.''

The strategy of acknowledging the authority of the system and then fighting prudently to improve it
defines Caijing's brilliance and its limitations. Qian Gang, the former editor, told me, ``A flood
is ferocious, but it solves no problems. In Chinese, we say that you can bore a hole in a stone by
the steady dripping of water.''

Hu prefers a noisier metaphor. Caijing is a woodpecker, she says, forever hammering at a tree, trying not to knock it down but to make it grow straighter.



\end{document}
