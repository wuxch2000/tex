\documentclass[12pt]{article}
\title{Digest of The New York Times}
\author{The New York Times}

\usepackage{config}

% \makeindex
\begin{document}
\date{}
% \thispagestyle{empty}

% \begin{figure}
% \includegraphics*[width=0.3\textwidth]{The_New_York_Times_logo.png}
% \vspace{-5ex}
% \end{figure}
% \renewcommand\contentsname{\textsf{Digest of The New York Times}}
% {\footnotesize\textsf{\tableofcontents}}
\pagenumbering{Roman}
\cfoot{\textsf{\thepage}}
% \renewcommand\contentsname{}

\tableofcontents

\clearpage
\setcounter{page}{1}
\pagenumbering{arabic}
\cfoot{\textbf{\textrm{{\color{Goldenrod} {\large \thepage}}}}\textsf{\scriptsize
    ~of~\pageref*{LastPage}}}

\section{Home for Super-Rich Tests India's Values}

\lettrine{T}{he}\mycalendar{Oct.'10}{29} newest and most exclusive residential tower for this city's
super rich is a cantilevered sheath of steel and glass soaring 27 floors into the sky. The parking
garage fills six levels. Three helipads are on the roof. There are terraces upon terraces, airborne
swimming pools and hanging gardens in a Blade Runner-meets-Babylon edifice overlooking India's most
dynamic city.

There are nine elevators, a spa, a 50-seat theater and a grand ballroom. Hundreds of servants and
staff are expected to work inside. And now, finally, after several years of planning and
construction, the residents are about to move in.

All five of them.

The tower, known as Antilia, is the new home of India's richest person, Mukesh Ambani, whose \$27
billion fortune also ranks him among the richest people in the world. And even here in the country's
financial capital, where residents bear daily witness to the stark extremes of Indian wealth and
poverty, Mr.~Ambani's building is so spectacularly over the top that the city's already elastic
boundaries of excess and disparity are being stretched to new dimensions.

``One family is going to live in that?'' said Prahlad Kakkar, an advertising filmmaker and prominent
city resident. ``Either it is a landmark, or a symbol, or it is Mammon.'' He added: ``There is shock
and awe -- both at the same time.''

Mr.~Ambani, his wife, Nita, and their three children are expected move into the building after a
housewarming party with 200 guests scheduled for Nov.~28. For his part, Mr.~Ambani has refused to
comment about the project and required his designers, decorators and other contractors to sign
confidentiality agreements, as if a cone of silence could be erected around a skyscraper rising near
the edge of the Arabian Sea.

Predictably, and perhaps by design, the opposite has happened. Details have spilled out -- many of
them confirmed or disputed anonymously. Some reports have estimated the total residential space at
400,000 square feet, though people close to the project say the real number is a humbler 60,000
square feet. Press accounts also have estimated the value of the building at \$1 billion, a figure
disputed by people familiar with the project.

Regardless, a gawking city has greeted the new tower with a mixture of moralizing and astonishment,
envy and condemnation, all sprinkled with Freudian analysis of the most basic question: Why did he
do it?

``We are all sort of perplexed,'' said Alyque Padamsee, a long-time advertising executive and actor
in the city. ``I think people see it as a bit show-offy.''

A bit.

For decades, the Ambani family has been India's most famous corporate soap opera. The father,
Dhirubhai Ambani, was a brazen, rags-to-riches tycoon who established Reliance Industries after
rising out of the city's Dickensian tenements, known as chawls. Today, Reliance is the world's
biggest producer of polyester fibers and yarns and accounts for almost 15 percent of India's
exports, according to the company's annual report. The two sons, Mukesh and Anil, inherited and
divided the empire and have spent years feuding, including a nasty recent fight over natural gas
rights that brought a reprimand from the prime minister before India's Supreme Court settled the
case in Mukesh's favor.

Of the two brothers, Anil is the more flamboyant and outgoing, while Mukesh is regarded as more
staid -- less likely, at least, to build at 27-story house for himself. The new tower is located on
Altamount Road, the same leafy residential street in south Mumbai where the father bought his first
home after moving the family out of the tenements. Later, he purchased a 14-story apartment building
named Sea Wind, where both Mukesh and Anil have lived with their families on different floors, even
during their feud. (Their mother refereed from her own residence in the building.)

Now Mukesh is moving into a tower that makes Sea Wind seem like a guest house.

``It's kind of returning with a vengeance to where they made it into the middle class and trumping
everybody,'' said Hamish McDonald, who chronicled the family's history in his new book,
``Mahabharata in Polyester: The Making of the World's Richest Brothers and Their Feud.''

``He's sort of saying, `I'm rich and I don't care what you think,' '' Mr.~McDonald said.

Mumbai, once known as Bombay, is India's most cosmopolitan city, with a metropolitan area of roughly
20 million people. Migrants have poured into the city during the past decade, drawn by Mumbai's
reputation as India's ``city of dreams,'' where anyone can become rich. But it is also a city
infamous for its poor: a recent study found that roughly 62 percent of the population lived in
slums, including one of Asia's biggest, Dharavi, which houses more than one million people.

Real estate prices are among the highest in the world, pushing many working-class residents into
slums, even as developers have brazenly cleared land for a new generation of high-rise apartment
towers for the affluent. High-rises are considered necessary, given the city's limited land, yet the
rising towers have further insulated the rich from the teeming metropolis below. With his helipads,
which still await operating approval, Mr.~Ambani could conceivably live in Mumbai without ever
touching the ground.

``This is a gated community in the sky,'' said Gyan Prakash, author of the new book ``Mumbai
Fables.'' ``It is in a way reflective of how the rich are turning their faces away from the city.''

Along Altamount Road, which is also home to other industrialists, the reaction to the new neighbor
is mixed. Some senior citizens along the street worry about the noise from the comings and goings of
helicopters. But Utsav Unadkat and Harsh Daga, college students who grew up in the neighborhood,
stared up at the tower on a recent afternoon as if it were a dream realized.

``I heard he has a BMW service station inside,'' said Mr.~Unadkat, dragging on a cigarette
(unconfirmed). ``There's also a room where you can create artificial weather,'' Mr.~Daga added
(apparently true).

Standing nearby, Laxmi Kant Pujari, 26, a decorator's assistant, waited to carry glass samples into
the building. If his samples are selected, Mr.~Pujari, a migrant, would handle the installation -- a
task he considered an honor. ``Whether it is a beggar or an Ambani, the desire to be rich is in
everyone's heart,'' he said.

Farther down the street, Sushala Pawar admitted struggling to comprehend the difference in
Mr.~Ambani's life and her own. She cooks for a family in a nearby apartment, earning 4,000 rupees a
month, or about \$90. She sleeps on the floor of the hallway after the family has gone to bed.

``I'm a human being,'' she said. ``And Mukesh Ambani is a human being. Sometimes I feel bad that I
live on 4,000 rupees and Mukesh Ambani lives there.''

But then, nodding toward the building, she perked up.

``Maybe,'' she said, ``I could get a job there.''

\section{In 2010 Campaign, War Is Rarely Mentioned}

\lettrine{T}{he}\mycalendar{Oct.'10}{29} wars in Afghanistan and Iraq have dominated American
foreign policy for the past nine years, but debate about them is all but absent from this year's
midterm election campaigns.

From Wilmington to Cleveland to Seattle, as Republicans try to wrest control of Congress from
Democrats, the subjects barely come up. Ditto for President Obama's stump speech as he zigzags
around the country.

He often alludes to Iraq, albeit briefly. ``Because of you, there are 100,000 young men and women
who are returning home from Iraq -- because of you,'' Mr.~Obama said Monday in Providence, R.I. But
he does not mention Afghanistan on the campaign trail -- nor, for the most part, do his Republican
opponents.

Both Democrats and Republicans appear to have decided that talking about the wars is not in their
best interest. Democrats are divided on the war, and do not want to air internal divisions in a year
when they have so many other problems. Republicans are unified in support of the war in Afghanistan
and Mr.~Obama's decision to send more troops there, but see no need to stress an issue on which they
are more or less allied with him.

In any case, Republicans certainly do not want to distract attention from the economy, which is
working for them as an issue. In addition, Democrats and Republicans have spent more than \$1
trillion on the wars in Iraq and Afghanistan since 2001 -- not the ideal topic to bring up on a
campaign trail that is dominated by concern about the budget deficit and the unemployment rate.

The lack of focus on the Iraq and Afghanistan wars is a big shift from the 2008 and 2006 election
cycles. In 2006, Democrats used voter unhappiness with the Bush administration's conduct of the Iraq
war to help them gain control of Congress. In 2008, Mr.~Obama stood out from the Democratic field
for his early opposition to the Iraq war and his position that the United States should focus more
on Afghanistan, and the wars played a key part in the general election campaign until the economy
became dominant in the closing stages.

``The big strategic consideration is that the electorate is energized over jobs, not over the war
right now,'' said Peter D.~Feaver, a former national security aide to President George W.~Bush.

Even in 2006, when the war in Iraq was foremost on the minds of the voting public, Mr.~Feaver said
the White House assessment was that the economy had as big an impact on how Americans were thinking
as the Iraq war.

For the White House, the lack of a real foreign policy issue to fight with Republicans about during
this campaign season is both a blessing and a curse.

Republicans sided with Mr.~Obama's decision to increase the number of troops in Afghanistan, though
not necessarily with his call to begin bringing them home next summer. And while liberal Democrats
are unhappy with Mr.~Obama's decision to escalate the fight in Afghanistan, they have yet to
challenge the White House over the issue in a concerted way.

``I think the president is an ironic beneficiary of the success of Bush's Iraq war surge,'' said
Charles Cook, publisher of The Cook Political Report and an independent analyst of Congressional
races. At the time that Mr.~Bush ordered more troops to Iraq, many foreign policy experts argued
that it was a move bound to fail. It did not. ``So the Afghan surge is getting a honeymoon of some
time,'' Mr.~Cook said.

But Afghanistan, political analysts say, will almost certainly be a campaign issue in 2012. By then,
it will be clear whether Mr.~Obama's troop surge in Afghanistan, and his decision to begin bringing
troops home next summer, have worked. The president will probably have to fend off challenges from
both the right and the left on Afghanistan. From the right, some Republicans -- possibly backed by
the president's own military commanders -- may argue for staying the course in Afghanistan, and for
holding off on withdrawing American troops.

Meanwhile, from the left, some Democrats may push the opposite way, arguing that the United States
should abandon its troop-heavy counterinsurgency strategy in Afghanistan that seeks to protect the
population, and move instead to a counterterrorism strategy that focuses on pounding Al Qaeda and
other insurgents, using far fewer troops.

In Missouri, Tommy Sowers, a Democrat who is a former Green Beret, has made pulling out of
Afghanistan war a major part of his battle to unseat Representative Jo Ann Emerson, a Republican.

For Mr.~Obama, ``if you have any problems with your flank going into re-election, it becomes a big
problem,'' said Mr.~Cook, recalling that Senator Edward M.~Kennedy's primary challenge in 1980
helped to undermine President Jimmy Carter.

\section{China Is Said to Resume Shipping Rare Earth Minerals}

\lettrine{T}{he}\mycalendar{Oct.'10}{29} Chinese government on Thursday abruptly ended its
unannounced export embargo on crucial rare earth minerals to the United States, Europe and Japan,
four industry officials said.

The embargo, which has raised trade tensions, ended as it had begun -- with no official
acknowledgment from Beijing, or any explanation from customs agents at China's ports.

Rare earths are increasingly in demand for their use in a broad range of sophisticated electronics,
from smartphones to smart bombs.

Having blocked shipments of raw rare earth minerals to Japan since mid-September, and to the United
States and Europe since early last week, Chinese customs agents on Thursday morning allowed
shipments to resume to all three destinations, the industry officials said. They spoke only on
condition of anonymity because of the business and diplomatic delicacy of the issue.

Shipments to Japan, however, still face additional scrutiny and some delays, the officials said.

Even with containers of rare earths once again leaving China's docks, foreign buyers still face
potential shortages. As China's own industrial needs for rare earths have grown, Beijing has
repeatedly reduced its export quotas for the minerals over the last five years. So even when China
is shipping its full quotas, the outbound supply is now well below world demand.

Moreover, the tight export quotas have caused world prices to soar, even while holding steady in
China.

Officials in two departments of China's General Administration of Customs in Beijing declined to
comment on Thursday evening about the status of rare earth exports. The commerce ministry, which
handles trade policy, also had no immediate comment.

Although deposits of rare earths are found in various parts of the world, including the United
States, China produces about 95 percent of the global supply of the minerals. That is largely
because rare earth mining and processing can be so environmentally risky, creating toxic and even
radioactive wastes, that other countries have tended to avoid or abandon production. Only recently
have other nations begun scrambling to develop or expand their own mining capabilities.

The Chinese shipments resumed Thursday morning only hours before Secretary of State Hillary Rodham
Clinton raised the embargo issue at a news conference in Honolulu, where she announced plans to
visit China on Saturday to pursue the matter with Chinese officials.

Mrs.~Clinton spoke after meeting with Japan's foreign minister, Seiji Maehara, and said that the
suspension of shipments had been a ``wake-up call'' and that both countries would have to find
alternative sources of rare earth materials.

Because China is on the opposite side of the international dateline from Honolulu, it was already
midday on Thursday in China by the time Mrs.~Clinton spoke in Honolulu on Wednesday. Later, after
the New York Times Web site reported that the embargo had been lifted, an administration official
said the United States was still seeking clarification from China.

In recent weeks, senior Chinese commerce ministry officials have insisted that they had not issued
any regulations halting shipments. They have suggested at various times -- implausibly, in the view
of industry executives -- that the halt resulted from a spontaneous and simultaneous decision by the
country's 32 authorized rare earth exporters not to make shipments, whether because of a
deterioration in Sino-Japanese relations or a greater thoroughness on the part of customs
inspectors.

Under this year's quota -- 30,300 metric tons of authorized shipments -- only a few thousand metric
tons remain to be exported in 2010. Meanwhile, annual demand outside China for raw rare earths
approaches 50,000 tons, according to industry estimates.

The Chinese government assigns its quotas to the authorized exporters, who often trade those rights
like commodities. As recently as 2008, the quota rights themselves had no market value. But lately,
with rising demand, the value of the remaining quotas has soared to the point that the right to
export a single ton of rare earths from China now sells for about \$40,000, including special
Chinese taxes.

That is a sizable, additional cost for buyers of neodymium, a rare earth used to make lightweight,
powerful magnets essential to technologies including giant wind turbines, gasoline-electric cars and
Apple iPhones.

Neodymium sells for about \$40,000 a metric ton in China, having recovered from a nose-dive during
the global economic crisis. But it sells for twice that much outside the country because of the
export restrictions, according to data from Metal Pages, a database service in London.

The cost of quotas has become exorbitant for users of lanthanum, which is vital for the catalytic
converters that clean the exhaust of conventional, gasoline-powered cars. It is mostly produced here
in Baotou, a smoggy mining and steel city in China's Inner Mongolia that is the capital of China's
rare earth industry. Lanthanum sells for less than \$4,500 a ton in China, but up to 10 times that
much outside China because of the export restrictions.

Such price differences have created a big incentive for companies to move factories to China, and
many already have.

China's shipping embargo has caused much more distress in Japan than in the United States or Europe,
and not just because Japan's shipments were cut off much earlier. It is because Japan tends to be
affected more than other industrial nations by the way China sets its rare earth export quotas.

China's quotas -- and the shipping embargo -- have involved only shipments in which the material has
a rare earth content of about 50 percent or more. High-technology materials made from rare earths,
like special magnetic powders for the clean energy and electronics industries, or polishing powders
for the glass industry, are not subject to quotas and are inexpensively available.

Because the United States and Europe mainly buy highly processed rare earth powders from China, the
customs policy of blocking shipments of raw rare earths had a limited, mostly symbolic effect.
Japan, in contrast, is the biggest importer of raw rare earths and tends to process them into
industrial materials. So Japan is more dependent on the materials affected by China's tightening
quotas.

It was on Oct.~18 that the Chinese government broadened its halt in raw rare earths to include the
United States and Europe. That step enabled customs officials to take the position that they were
checking all rare earth shipments closely and were not singling out Japan.

The move also occurred only hours after Zhang Guobao, the country's top energy official, summoned
foreign reporters in Beijing. There, he delivered a blistering denunciation of the Obama
administration's decision the previous Friday to begin investigating whether China's clean energy
policies violated the World Trade Organization's free trade rules. But the exact interaction between
American policy decisions and Chinese customs enforcement actions is unclear.

For China, the embargo on rare earth shipments has provided at least some geopolitical leverage. The
halt was one of a series of measures that China took after Japan detained the captain of a Chinese
fishing trawler that collided with two Japanese patrol boats; Japan later released the trawler's
captain.

Japanese companies had been able to weather the embargo without any significant factory shutdowns
because many Japanese companies had accumulated rare earth stockpiles in the last few years. Still,
the interruption of shipments caused dismay and alarm in the Japanese business community and Japan's
government.

But China's willingness to play economic hardball could yet have long-term drawbacks, if it prompts
multinationals to reduce their reliance on manufacturing in China and spread their investments among
more countries.

\section{Obama Visit To Indonesia Will Include A Speech}

\lettrine{P}{resident}\mycalendar{Oct.'10}{29} Obama will visit one of the world's largest mosques
and deliver a major address to the people of Indonesia when he visits that island nation next month,
in a long-delayed trip that White House officials say will mix childhood reflections with outreach
to the Muslim world.

Indonesia will be the second stop for the president on a four-country, 10-day swing through Asia
that will begin Nov.~5, just days after what is expected to be a bruising election for Democrats.
The president's stay in Jakarta will undoubtedly be a personal highlight of the trip, giving him a
chance to present himself before an adoring public at a time when his popularity is down at home.
The White House is looking for an outdoor site for the speech so that average Indonesians can
attend.

``Because of his biography, in some respects, and because of his policies, in others, I think that
there's great anticipation of his visit in Indonesia,'' Ben Rhodes, a top foreign policy adviser to
the president, told reporters Thursday. ``And so he wanted to not just have, kind of, official
business; he wanted to have the opportunity to reach more Indonesian people, and that's the manner
in which we've planned the speech.''

But the visit may be a delicate one for Mr.~Obama, who has made outreach to Muslims a theme of his
presidency, even as he has sought to combat misperceptions at home that he is of the Muslim faith.
(He is Christian.) The president generated an uproar earlier this year when he defended the right of
Muslims to build a community center and prayer hall near ground zero in Manhattan, and the mosque
visit could inflame that controversy.

``It is a difficult line for him to walk,'' said Robert Malley, director of the Middle East Program
at the International Crisis Group, the Brussels-based organization that seeks to prevent and resolve
deadly conflicts. ``President Bush would visit a mosque and liberal constituency would applaud.
President Obama visits a mosque and the conservative constituency raises their eyebrows.''

Indeed, the Asia trip, which will include stops in India, Indonesia, South Korea and Japan, has
already stirred a backlash in India over the president's decision to bypass the Golden Temple, the
holiest Sikh site in that country. Visitors to the temple must cover their heads in the Sikh
tradition; local officials planning the India trip said the White House was concerned about the
imagery for Mr.~Obama, in part because Sikhs are often mistaken for Muslims.

Mr.~Rhodes said the mosque visit was standard protocol for Mr.~Obama, who has already visited
mosques during presidential trips to Egypt and Turkey. He said Mr.~Obama would use the Jakarta
speech to build on one he delivered in Cairo last year, when he called for ``a new beginning''
between the United States and the Muslim world.

But Mr.~Malley said that while Muslims viewed Mr.~Obama as ``the most friendly and the most
understanding U.S.~president they have ever witnessed,'' many are disappointed in the president's
policies, particularly what they regard as his lenient stance toward Israel. While the Indonesia
speech might revive ``the depleted enthusiasm'' among Muslims, he said, ``I think the administration
knows at this point they are not going to be judged by their rhetoric. They are going to be judged
by their results.''

\section{Despite Canadian Challenge, Europe Imposes a Ban on Seal Products}

\lettrine{T}{he}\mycalendar{Oct.'10}{29} European Union won approval on Thursday to close its
borders to seal products after a court decided to impose a ban even though a legal challenge from
Canadian seal hunters was still in progress.

The European Union's General Court, one of the highest courts in the 27-state bloc, said there was
no need to delay the ban for the duration of court proceedings. It added that it had not seen enough
evidence that the ban would damage seal hunting communities.

Seal hunters and processing firms from Canada, Norway and Greenland secured a temporary delay in the
ban this summer, saying it would slash incomes in traditional Inuit communities and lead to suicides
and substance abuse.

The ruling contested that assertion, saying that ``the applicants are simply making mere general and
abstract assertions.'' The court also said too little paperwork was presented to prove their current
financial situation, future income losses or access to unemployment benefits.

In Ottawa, Gail Shea, the Canadian fisheries minister, said she was disappointed by the court
decision.

``What we want to do is continue to express our support for the seal industry here in Canada, to the
sealers on our coasts and in the North,'' she told reporters.

The International Fund for Animal Welfare praised the decision and insisted the European Union ban
would exempt Inuit-derived seal products.

The court ruling raises the stakes in a European Union trade dispute with Canada, even as the sides
pursue free trade talks.

The ban was approved last year and originally was to take effect in August. That led the Canadian
government to challenge the ban at the World Trade Organization. A separate suit by seal hunters was
also filed at the European Union's General Court this summer.

Canada says the ban, affecting about 4.2 million euros (\$5.6 million) of business, is based on
false information and violates the European Union's trade obligations.

The indigenous Inuit people of Canada's North say an exemption for seal products derived from
traditional hunts is rhetoric that will not be honored.

Canada's main seal hunt takes place in March and April on ice floes off the Atlantic Coast and in
the Gulf of St.~Lawrence.

Seal products include fur for clothing and oil that is used in vitamin supplements.

\section{China Wrests Supercomputer Title From U.S.}

\lettrine{A}{}\mycalendar{Oct.'10}{29} Chinese scientific research center has built the fastest
supercomputer ever made, replacing the United States as maker of the swiftest machine, and giving
China bragging rights as a technology superpower.

The computer, known as Tianhe-1A, has 1.4 times the horsepower of the current top computer, which is
at a national laboratory in Tennessee, as measured by the standard test used to gauge how well the
systems handle mathematical calculations, said Jack Dongarra, a University of Tennessee computer
scientist who maintains the official supercomputer rankings.

Although the official list of the top 500 fastest machines, which comes out every six months, is not
due to be completed by Mr.~Dongarra until next week, he said the Chinese computer ``blows away the
existing No.~1 machine.'' He added, ``We don't close the books until Nov.~1, but I would say it is
unlikely we will see a system that is faster.''

Officials from the Chinese research center, the National University of Defense Technology, are
expected to reveal the computer's performance on Thursday at a conference in Beijing. The center
says it is ``under the dual supervision of the Ministry of National Defense and the Ministry of
Education.''

The race to build the fastest supercomputer has become a source of national pride as these machines
are valued for their ability to solve problems critical to national interests in areas like defense,
energy, finance and science. Supercomputing technology also finds its way into mainstream business;
oil and gas companies use it to find reservoirs and Wall Street traders use it for superquick
automated trades. Procter \& Gamble even uses supercomputers to make sure that Pringles go into cans
without breaking.

And typically, research centers with large supercomputers are magnets for top scientific talent,
adding significance to the presence of the machines well beyond just cranking through calculations.

Over the last decade, the Chinese have steadily inched up in the rankings of supercomputers.
Tianhe-1A stands as the culmination of billions of dollars in investment and scientific development,
as China has gone from a computing afterthought to a world technology superpower.

``What is scary about this is that the U.S.~dominance in high-performance computing is at risk,''
said Wu-chun Feng, a supercomputing expert and professor at Virginia Polytechnic Institute and State
University. ``One could argue that this hits the foundation of our economic future.''

Modern supercomputers are built by combining thousands of small computer servers and using software
to turn them into a single entity. In that sense, any organization with enough money and expertise
can buy what amount to off-the-shelf components and create a fast machine.

The Chinese system follows that model by linking thousands upon thousands of chips made by the
American companies Intel and Nvidia. But the secret sauce behind the system -- and the technological
achievement -- is the interconnect, or networking technology, developed by Chinese researchers that
shuttles data back and forth across the smaller computers at breakneck rates, Mr.~Dongarra said.

``That technology was built by them,'' Mr.~Dongarra said. ``They are taking supercomputing very
seriously and making a deep commitment.''

The Chinese interconnect can handle data at about twice the speed of a common interconnect called
InfiniBand used in many supercomputers.

For decades, the United States has developed most of the underlying technology that goes into the
massive supercomputers and has built the largest, fastest machines at research laboratories and
universities. Some of the top systems simulate the effects of nuclear weapons, while others predict
the weather and aid in energy research.

In 2002, the United States lost its crown as supercomputing kingpin for the first time in stunning
fashion when Japan unveiled a machine with more horsepower than the top 20 American computers
combined. The United States government responded in kind, forming groups to plot a comeback and
pouring money into supercomputing projects. The United States regained its leadership status in
2004, and has kept it, until now.

At the computing conference on Thursday in China, the researchers will discuss how they are using
the new system for scientific research in fields like astrophysics and bio-molecular modeling.
Tianhe-1A, which is housed in a building at the National Supercomputing Center in Tianjin, can
perform mathematical operations about 29 million times faster than one of the earliest
supercomputers, built in 1976.

For the record, it performs 2.5 times 10 to the 15th power mathematical operations per second.

Mr.~Dongarra said a long-running Chinese project to build chips to rival those from Intel and others
remained under way and looked promising. ``It's not quite there yet, but it will be in a year or
two,'' he said.

He also said that in November, when the list comes out, he expected a second Chinese computer to be
in the top five, culminating years of investment.

``The Japanese came out of nowhere and really caught people off guard,'' Mr.~Feng said. ``With
China, you could see this one coming.''

Steven J.~Wallach, a well-known computer designer, played down the importance of taking the top spot
on the supercomputer rankings.

``It's interesting, but it's like getting to the four-minute mile,'' Mr.~Wallach said. ``The world
didn't stop. This is just a snapshot in time.''

The research labs often spend weeks tuning their systems to perform well on the standard horsepower
test. But just because a system can hammer through trillions of calculations per second does not
mean it will do well on the specialized jobs that researchers want to use it for, Mr.~Wallach added.

The United States has plans in place to make much faster machines out of proprietary components and
to advance the software used by these systems so that they are easy for researchers to use. But
those computers remain years away, and for now, China is king.

``They want to show they are No.~1 in the world, no matter what it is,'' Mr.~Wallach said. ``I don't
blame them.''

\section{As College Fees Climb, Aid Does Too}

\lettrine{A}{s}\mycalendar{Oct.'10}{29} their state financing dwindled, four-year public
universities increased their published tuition and fees almost 8 percent this year, to an average of
\$7,605, according to the College Board's annual reports. When room and board are included, the
average in-state student at a public university now pays \$16,140 a year.

At private nonprofit colleges and universities, tuition rose 4.5 percent to an average of \$27,293,
or \$36,993 with room and board.

The good news in the 2010 ``Trends in College Pricing'' and ``Trends in Student Aid'' reports is
that fast-rising tuition costs have been accompanied by a huge increase in financial aid, which
helped keep down the actual amount students and families pay.

``In 2009-2010, students got \$28 billion in Pell grants, and that's \$10 billion more than the year
before,'' said Sandy Baum, the economist who is the lead author of the reports. ``When you look at
how much students are actually paying, on average, it is lower, after adjusting for inflation, than
five years earlier.''

In the last five years, the report said, average published tuition and fees increased by about 24
percent at public four-year colleges and universities, 17 percent at private nonprofit four-year
institutions, and 11 percent at public two-year colleges -- but in each sector, the net
inflation-adjusted price, taking into account both grants and federal tax benefits, decreased over
the period.

Almost everybody has been helped by the federal government's increased spending on education,
Ms.~Baum said, either through Pell grants, which provide an average of \$3,600 for low-income
students, or through tax credits, which go further up the income scale.

The increase in federal support this year was so large that unlike former years, government grants
surpassed institutional grants.

``I think that's an aberration,'' Ms.~Baum said. ``Pell grants are unlikely to grow so rapidly in
the coming years, and institutional grants are likely to grow, so I think the ratio will flip
back.''

This year, the report found, full-time students receive an average of about \$6,100 in grant aid and
federal tax benefits at public four-year institutions, \$16,000 at private nonprofit institutions,
and \$3,400 at public two-year colleges.

``The College Board figures are depressing and utterly predictable,'' said Terry Hartle, senior vice
president of the American Council on Education. ``When states cut funding for higher education,
tuitions go up to make up for the difference. The good new is that Pell grants will cushion the
increases for low-income students, but if you're not eligible for financial aid, it's a problem,
since very few families are seeing their income go up 8 percent this year.''

Despite the weak economy, and the number of families having trouble paying tuition, the nation's
public universities continue to award most of their institutional aid without regard to financial
need. Over all, the report found, 42 percent of the public institutions' aid is awarded on the basis
of need.

That is up from 28 percent the previous year, Ms.~Baum said, for reasons that are unclear.

``It might be that they said, look at all these kids who need money, we should be giving more to
them,'' she said. ``Or it might be that because of the recession so many more people have financial
need that more of them happen to be getting institutional aid.''

Out-of-state students at public universities this year are paying an average of \$19,595 in tuition,
with total charges of \$28,130, according to the report.

At public community colleges, published tuition and fees rose 6 percent, to an average of \$2,713.

And at for-profit institutions, the report found, tuition and fees rose 5.1 percent, to an average
of \$13,935.

Over the last decade, published tuition and fees at public four-year colleges and universities
increased each year at an average of 5.6 percent beyond the rate of inflation.

``We have to figure out how to educate students in a more cost-efficient way,'' Ms.~Baum said. ``We
haven't yet figured out how to use technology to make it cheaper. But we will.''

\section{In House Race in Texas, a Spotlight on the Hispanic Vote}

\lettrine{R}{epresentative}\mycalendar{Oct.'10}{29} Ciro D.~Rodriguez, a Democrat from the
working-class streets of south San Antonio, was buttonholing voters outside a library where early
polling was being held, asking them for support in flawless Spanish.

``Don't you worry about Canseco,'' said one Latino man about Mr.~Rodriguez's opponent, pumping the
congressman's hand and speaking, like many of Mr.~Rodriguez's constituents, in Spanish. ``We are
with you. We are with you.''

``Tell your family members to vote, too,'' Mr.~Rodriguez replied. Then he added in English: ``It's
all about turnout.''

Hispanic Democrats usually have the upper hand in the Rio Grande Valley, but this year Republicans
are backing a conservative Hispanic businessman, Francisco Canseco, who they hope will split the
Latino vote and carry the banner for white conservatives angry at President Obama's economic and
health care policies.

Mr.~Canseco, a wealthy lawyer and developer who has allied himself with the antitax movement known
as the Tea Party, has been buoyed in early voting by heavy turnout in Republican suburbs, where
anger over spending in Washington is high.

Mr.~Rodriguez, a veteran Democrat who has represented two districts here, has found himself fighting
for his political life, struggling to wake up his working-class base and get them to the polls.

Though no independent polls have been done, both candidates and several political analysts say the
race is too close to call.

The surge of support for Mr.~Canseco mirrors what has happened elsewhere in the country where Latino
Republicans have become not only champions for small-government conservatives but have weakened
Hispanic voters' allegiance to the Democrats.

In Florida, a Hispanic Republican with Tea Party support, Mark Rubio, leads in a three-way race for
Senate. And in New Mexico and Nevada, Latino Republicans are the front-runners for governor.

Most years, a Congressional race like this one would revolve around whether the incumbent had
delivered federal projects or had pushed through policies benefiting local industries or other
interests. But this is a year in which ideology trumps bringing home the bacon, and that spells
trouble for Mr.~Rodriguez, who is on the Appropriations Committee and has proved adept at obtaining
projects for his constituents.

``We haven't really talked about the local issues,'' he said in an interview. ``It is really
national politics that drives it. It's all about whether I'm close to Nancy Pelosi or not.''

Like many Republican challengers across the country, Mr.~Canseco has tried to make the race a
referendum on Congress's decisions to overhaul the health care system and to spend \$787 billion to
stimulate the economy. In one of his advertisements, Mr.~Canseco drops a copy of the health care
bill onto a piggy bank, while a large picture of Mr.~Obama is shown. ``This is your family's nest
egg,'' he says.

A banker and real estate developer from Laredo who is making his third run for Congress,
Mr.~Canseco, 61, has railed about the evils of deficit spending, repeated the Republican mantra that
only tax cuts and deregulation can create jobs and promised to preserve the Bush-era tax cuts for
the wealthy. ``We need to get government out of the hair of business,'' he said in an interview.
``It should not be the government running the economy.''

He has also tapped into the widespread fear of drug cartels that have undermined civil society in
much of northern Mexico, saying the violence there is ``bleeding over'' into the United States and
must be stopped with more law enforcement.

The two candidates are vying over a giant district, Texas' 23rd, which stretches more than 550 miles
along the border from San Antonio to El Paso, taking in 20 counties and parts of both cities.

But most of the registered voters are in San Antonio, and the major battle is between the affluent,
heavily Republican suburbs on San Antonio's northeast side and the working-class, heavily Democratic
neighborhoods on the south side.

On paper, the district has more Democrats than Republicans, but the turnout is often heavier in
Republican enclaves, depending on the year, local politicians say. That is one reason another Latino
Republican, Henry Bonilla, managed to hold on to the seat for 14 years, until a court-ordered
redrawing of the district in 2006 gave Democrats an edge. This year, local politicians say, voters
are showing up in droves at early polling sites in Republican areas, as Tea Party protests have
caught fire, in a sharp reversal of the Democratic wave in 2008.

On the stump, Mr.~Rodriguez, who is 63, finds himself on the defensive. At a get-out-the-vote event
this week, he tried to remind voters that much of the national debt resulted from former President
George W.~Bush's decision to borrow money to fight two wars while cutting taxes for the wealthy. He
pointed out that he voted against the Wall Street bailout. He also defended the health care bill,
arguing that the mandate to give insurance to everyone would save Texans money because they
currently pay for the care of the indigent through property taxes.

``It's not perfect, but my God, Texas will probably benefit more than anyone else,'' he said,
standing in front of a mariachi band in the backyard of a supporter.

On the air, the race has been ugly from the get-go. Mr.~Rodriguez has shown ads calling Mr.~Canseco
``a millionaire banker'' and ``a tax evader,'' accusing him of failing to pay \$715,000 in state
taxes and fees. The Democratic Congressional Campaign Committee paid for a similar ad, depicting
Mr.~Canseco as an outlaw in the Wild West.

Mr.~Canseco, with help from the National Republican Campaign Committee, fired back with an ad
depicting Mr.~Rodriguez as a ``career politician'' who is ``desperately clinging to power'' and who
backs higher taxes. ``The answer?'' Mr.~Canseco says. ``Throw them out. Clean house.''

In the final week, both candidates are concentrating mostly on getting their loyal supporters to the
polls rather than swaying undecided voters. Mr.~Canseco is hitting retirement homes and holding
get-out-the-vote parties in areas where he has support.

At Mr.~Canseco's headquarters on the edge of an upscale suburb, dozens of volunteers, many of them
self-described Tea Party activists, worked rows of telephones on Tuesday, calling likely Republican
voters.

Mr.~Canseco predicted the anger at Washington was so deep that Hispanic Democrats would switch
parties to support him. ``They are conservative people,'' he said, ``and because of tradition they
have been forced into the Democratic Party.''

\section{Chinese Article Seems to Chide Leader}

\lettrine{C}{hina}\mycalendar{Oct.'10}{29}'s main Communist Party newspaper bluntly rejected calls
for speedier political reform on Wednesday, publishing a front-page commentary that said any changes
in China's political system should not emulate Western democracies, but ``consolidate the party's
leadership so that the party commands the overall situation.''

The opinion article in People's Daily, signed with what appeared to be a pseudonym, appeared at
least obliquely aimed at Prime Minister Wen Jiabao. He has argued in speeches and media interviews
that China's economic progress threatens to stall without systemic reforms, including an independent
judiciary, greater oversight of government by the press and improvements in China's sharply limited
form of elections.

It also may have been directed at countering recent demands for democratic reforms by Chinese
liberal intellectuals and Communist Party elders, spurred in part by Mr.~Wen's remarks and timed to
this month's award of the Nobel Peace Prize to an imprisoned Chinese democracy advocate, Liu Xiaobo.

Mr.~Wen's comments have fueled a debate among analysts over whether he is advocating Western-style
changes in China's governing system or merely calling for more openness inside the ruling Communist
Party.

Wednesday's commentary, which closely followed the ruling party's annual planning session, ran to
1,800 words and delved into topics only occasionally discussed in the state media. The article
emphatically repeated past declarations that changes modeled on American or European political
systems were inappropriate for China. It also appeared to directly reject Mr.~Wen's warning that
economic progress and political reforms were inseparably linked.

``The idea that China's political reform is seriously lagging behind its remarkable economic
development is not only contrary to the law of objectivity but also to the objective facts,'' it
stated.

It later added: ``In promoting political reform, we shouldn't copy the Western political system
model; shouldn't engage in something like multiparty coalition government or separation of powers
among the executive, legislative and judicial branches. We should stick to our own way.''

A Chinese political historian who asked not to be named in discussing the issue said, ``Obviously,
this is a criticism of Wen.'' He later qualified his remark, saying the editorial amounted to ``a
sideways swipe,'' noting that Mr.~Wen was not explicitly named.

Still, the notion of a link is bolstered by a leaked Oct.~19 directive from Communist Party censors
that ordered Internet sites and news organizations to delete all references to a recent interview of
Mr.~Wen by CNN. In that Sept.~23 interview, Mr.~Wen said that ``the people's wishes for and needs
for democracy and freedom are irresistible.''

Mr.~Wen has made similar statements in previous years, and the party's more conservative majority
has appeared to bristle. In 2007, after Mr.~Wen publicly embraced ``universal values'' like human
rights, the state-controlled press reacted with what seemed nationalistic vigor, and the term has
since become taboo.

Some analysts said on Wednesday that the party's brusque reaction this time points to a growing
debate over the future direction of China's political system.

``It does appear to be a direct swipe at Wen's statements,'' David Shambaugh, who heads the China
Policy Program at George Washington University in Washington, said in an e-mail. ``It is more
evidence of a division of views within higher levels of the party on the scope and pace of
`democratic' reform.''

Still unclear, he said, is what democratic reform means to members of the party hierarchy. Publicly,
at least, virtually all debate on democracy in party journals and speeches has been limited to ways
of making the party bureaucracy more responsive to ordinary citizens rather than giving those
citizens a direct voice.

A Beijing scholar of the leadership, Russell Leigh Moses, called the editorial ``a reminder to
cadres that the party will set the tone and terms of the debate on political reform.''

Within the system, some are skeptical that hints of a split amount to much.

``This political reform debate remains more of a rhetorical debate than an actual policy debate,
about how to define China's democracy versus the West's,'' an editor with a party publication noted
in a recent conversation.

``Perhaps some liberal media and intellectuals once again want to make something of Wen's recent
statements,'' he said. ``But realistically, even if he is sincere, all he can do is earn a better
reputation for himself.''

\section{Smartphones Help Results at Motorola}

\lettrine{M}{otorola}\mycalendar{Oct.'10}{29}, the telecommunications equipment maker, on Thursday
posted its first year-over-year quarterly increase in sales since 2006, when its Razr phone was
still cutting-edge technology.

The company's phone division reported an operating profit for the first time in three years. It had
not expected that to happen until next quarter. The division is pulling out of a deep dive by
betting on smartphones like the Droid X.

Revenue rose 6 percent, to \$5.8 billion, in the quarter.

Motorola, based in Schaumburg, Ill., reported net income of \$109 million, or 5 cents a share. That
is up from \$12 million, or 1 cent a share, a year earlier. Excluding one-time items, Motorola
earned 16 cents a share. Analysts expected 11 cents.

However, nearly all the profits came from the soon-to-be-sold network equipment division, which
supplies cellphone companies. Without it, net income would have been only \$7 million, or less than
1 cent a share.

In July, Motorola agreed to sell the network equipment division for \$1.2 billion to Nokia Siemens
Networks, a Finnish-German joint venture. That deal is expected to close late this year.

Early next year, the electronics brand will split into two companies. The cellphone and cable box
business will be called Motorola Mobility. Police radios, bar code scanners and other products for
corporate and government customers will be under Motorola Solutions.

The split has been driven by a desire to present two simple stories to investors rather than one
complex one. It was announced in 2008, but was soon delayed as phone sales collapsed. The turnaround
sets the division up for a life of its own.

The phone division shipped 9.1 million units in the most recent quarter, of which 3.8 million were
smartphones. Both figures were an increase from the second quarter, halting a multiyear sales slide.

However, Motorola finds itself in a very different place compared to when the slide started. In
2006, it was the second-largest phone maker in the world. Now, it is not even in the top five. It is
the third-largest phone maker in North America, after Apple and Research In Motion, the maker of the
BlackBerry.

Motorola said it expected fourth-quarter earnings of 14 to 16 cents a share, excluding the network
division. Analysts were on average expecting 15 cents a share, including the network division.

\section{Marijuana Web Names Snapped Up, in Case of Legalization}

\lettrine{I}{t}\mycalendar{Oct.'10}{29} was just about a year ago that Kevin Faler came up with his
get-rich-quick marijuana scheme. No, he does not plan to sell the drug, even if Californians vote
next week to become the first state in the nation to fully legalize it. He intends to sell the
Internet real estate that could one day lead to marijuana Web sites.

Mr.~Faler, a former police officer who once worked the narcotics beat, has registered more than
1,000 marijuana-related Internet domain names, including oddities like icecreammarijuana.com and
marijuanapastry.com. And he is not the only one banking on the drug's online future. He is part of
an Internet land grab for marijuana domains by so-called domainers who hope to sell their holdings
at a profit, betting that more lenient marijuana laws will eventually drive more people to the Web
for their supplies, whether they are seeking seeds, bongs, recipes or drug-laced dog treats.

All of this has been given a fresh burst of intensity by next week's vote on Proposition 19, the
California ballot measure that would legalize up to an ounce of the drug for recreational use.
Fourteen states have already legalized medical marijuana.

``Marijuana domain name values will fly off the charts once Prop 19 passes,'' said Mr.~Faler, 49.
``I'm hoping to make enough money to buy a condo in Morocco. That's how big it's going to be.''

Mr.~Faler, who lives about 90 miles southeast of Los Angeles in Menifee, Calif., is poised to enter
the pets and marijuana market by registering domains like potfordogs.com and marijuanadogbone.com
because ``dogs get cancer too'' he says.

While it is unclear if such investments by enthusiastic newcomers will pay off, buying and selling
Internet domain names can be a profitable business. The industry is estimated to be worth billions
of dollars. A \$13 million sale is pending for sex.com. In June, slots.com sold for \$5.5 million
and dating.com for \$1.75 million. The New Jersey company that paid over \$1 million for marijuana
.com in 2004 says it has turned down five offers for more than \$2 million for the domain in the
last 12 months.

Domainers use various strategies when acquiring domain names. While Mr.~Faler tends to register
domains that struck his fancy at odd hours of the night, Jordan Zazzara of Long Island prefers the
geographer's method. With the help of a California map, Mr.~Zazzara, 21, chose ``geo-targeted''
domains, registering ones that combined the state's major city names with the words marijuana, weed,
ganja, bud and cannabis.

For between \$7 and \$10 dollars a pop, he registered 100 domains stretching between
beverlyhillsmarijuana.com and modestocannabis.com. He intends to keep them by renewing the
registration every year for a nominal fee, until they are worth at least \$5,000 each, he says.
``I'll sit on them for as long as I have to,'' he said. ``And when marijuana is an accepted thing
like alcohol, which it eventually will be, these things will be worth a lot.''

Despite the enthusiasm of speculators, whether this marijuana domain gold rush will yield much legal
tender depends in large part on politics.

In recent weeks, Proposition 19 has lost its lead in the polls -- a recent one from the Public
Policy Institute of California showed 49 percent of respondents against the measure and 44 percent
in favor -- but it is still favored by most younger voters and Democrats. In another blow to
Proposition 19 supporters, Attorney General Eric H.~Holder Jr.~announced that even if voters passed
the ballot measure, federal law enforcement officials planned to aggressively prosecute federal
marijuana laws in the state.

cHow much these things are worth is up to the political winds,'' said Michael H.~Berkens, editor of
thedomains.com, a leading online news source on the domain business.

So far, most marijuana domains are being registered and resold on the cheap. DN Journal, an online
publication that tracks domain sales, has documented just one dot-com domain containing the word
``marijuana'' that sold this year for at least \$2,000, suggesting they are not yet worth much.

Still, Mr.~Berkens thinks marijuana domains could be a good investment; he sees the political
momentum moving toward legalization and decriminalization. ``We own gaymarriage.com,'' said
Mr.~Berkens, who is also president of Worldwide Media, a company that owns some 75,000 domains, 57
of them marijuana related. ``That's another one of these politically charged hot topics, heavily
dependent on politics.''

Some in the domain business are torn for more personal reasons between a potentially lucrative
investment opportunity and the moral ambiguities of marijuana.

In late September, Shane Cultra, 41, was bidding in an online auction for the domain
smokingmarijuana.com when suddenly he stopped, midclick. ``I asked myself, do I want to be in that
business?'' said Mr.~Cultra, who runs a nursery in Champaign-Urbana, Ill., and moonlights as a
domainer specializing in plant and horticultural domains.

``There is tremendous investment opportunity there,'' Mr.~Cultra said. ``Before long, you will be
able to buy and sell marijuana on the Internet.''

But Mr.~Cultra worries about associating his name -- which he shares with an uncle who is a
Republican member of the Illinois House of Representatives -- with illegal, or morally shady,
activities. He has a rule against buying pornography domains, another potentially profitable
endeavor.

If marijuana were legal nationally, Mr.~Cultra would not hesitate to snap up marijuana domains.
``But then it will be too late,'' he said. ``The real opportunity is now.''

\section{Prepaid Service Pushes Sprint Nextel Revenue Up}

\lettrine{T}{he}\mycalendar{Oct.'10}{29} Sprint Nextel Corporation reported Wednesday that its
quarterly revenue increased for the first time in three years, as improvements in Sprint-branded and
prepaid service offset the continued flight of subscribers.

But its net loss widened compared with a year ago, when a large tax benefit lifted results.

The company, based in Overland Park, Kan., gained a net 644,000 subscribers in the quarter, compared
with a loss of 545,000 in the period a year ago. It ended September with 48.8 million subscribers.

Its quarterly loss was \$911 million, or 30 cents a share. It lost \$478 million, or 17 cents a
share, in the year-earlier period.

Revenue rose 1 percent, to \$8.15 billion, from \$8.04 billion a year ago.

Analysts surveyed by Thomson Reuters had expected a smaller loss of 28 cents a share on lower
revenue of \$8.04 billion.

Sprint continued to lose subscribers on contract-based plans, which are the most lucrative, because
of the decline of Nextel. Its phones with walkie-talkie functions have been popular with work crews
and other professionals, but the Nextel network is poorly suited to smartphones, and the core Nextel
base is eroding.

Sprint compensated for the loss of contract-signing customers by adding people on prepaid plans,
which cost less.

In recent years, Sprint made a push to use excess capacity on the Nextel network for cheap
unlimited-calling plans under its Boost Mobile brand. But even that effort is flagging and new
prepaid subscribers are being put on the Sprint network instead. The company has started talking
about converting some of the Nextel spectrum to Sprint use.

The burden of operating two incompatible wireless networks has weighed heavily on Sprint since the
2005 acquisition of Nextel Communications, widely hailed as one of the most disastrous deals in the
telecommunications industry.

In the fourth quarter, Sprint said, it expects to continue to gain subscribers.

\section{Chinese Telecom Giant in Push for U.S.~Market}

\lettrine{T}{his}\mycalendar{Oct.'10}{29} spring, an executive from a Chinese telecommunications
equipment company made an intriguing job offer to a Silicon Valley software engineer. The Chinese
company, Huawei Technologies, wanted to get into the booming market for Internet-based computing,
and it had just moved its United States research headquarters here to capture some of the best local
talent.

``How many engineers would you like for your team? Several hundred? That's not a problem,'' the
recruiter said, according to the engineer.

When the software manager turned down the offer, the Chinese executive was undeterred and asked for
the name of the engineer working under him.

The exchange underscores Huawei's bold entrance onto the world's technology stage. In the span of a
decade, it has gone from imitating others' products to taking on international rivals with its own
innovative computing and communications gear. But Huawei has largely been locked out of the United
States -- until now.

Sprint Nextel, the nation's third-largest wireless carrier, is preparing to make a decision on
buying \$3 billion in advanced wireless equipment, and Huawei is considered to be a front-runner for
the deal.

Huawei is one of many Chinese companies that are pushing into more sophisticated and lucrative
businesses. But security concerns make telecommunications a particularly delicate industry in this
country, and even the hint of a Huawei deal with Sprint has generated worries in Washington.

Some in Congress and the national security establishment fear that Huawei's close ties to the
Chinese military might allow China to tamper with American communications gear.

Last week, Senator Joseph I.~Lieberman, independent of Connecticut, and three other members of
Congress wrote a letter to Julius Genachowski, chairman of the Federal Communications Commission,
raising the specter that an equipment sale might permit the Chinese government to manipulate parts
of the communications network, making it possible to disrupt or intercept phone calls and Internet
messages.

Anticipating these hurdles, Huawei has hired a remarkable array of Washington lobbyists, lawyers,
consultants and public relations firms to help it win business in the United States. It has also
helped create Amerilink Telecom, an American distributor of Huawei products whose high-powered board
includes former Representative Richard A.~Gephardt, the former World Bank president James
D.~Wolfensohn and the one-time chief executive of Nortel Networks, William A.~Owens.

Amerilink executives say they are primarily interested in helping Huawei overcome objections that
its entry into the American market could jeopardize national security.

``We take the accusations very seriously,'' said Kevin Packingham, who recently left Sprint to
become chief executive of Amerilink. ``But regardless of the accusations, we have a model in place
that ensures the security'' of the network should Huawei win American contracts, he said.

The effort is beginning to pay off. This fall, the American Internet communications firm Clearwire
will begin testing a system based on Huawei's 4G, or fourth-generation, network technology.

The Sprint contract would be Huawei's largest American deal by far. A Sprint spokesman, Scott Sloat,
declined to discuss any potential deal. Sprint bought its last round of network equipment from
Motorola, Nortel Networks and Lucent, now part of Alcatel-Lucent.

Huawei's American drive is significant because it is China's first truly home-grown multinational
corporation. And some analysts say they believe its spectacular rise will serve as a model for other
Chinese companies seeking to compete internationally.

Huawei is now the world's second-largest telecom equipment supplier behind Ericsson of Sweden, and
with Chinese government backing, it has sewn up major deals in Asia, Africa and Latin America. In
Europe, Huawei has outmaneuvered Ericsson to supply equipment to big carriers.

Despite those successes, Huawei has struggled to break into the United States market, largely
because of the security concerns and accusations of intellectual property theft and corporate
espionage.

The company has repeatedly been linked to the People's Liberation Army of China. And over the last
decade, Huawei has been sued in the United States by two of its major competitors, Cisco Systems and
Motorola, over accusations that it stole software designs and infringed on patents.

Cisco settled its suit with Huawei soon after filing it. But in court documents filed in a lawsuit
last summer, Motorola claimed that a group of Chinese-born Motorola engineers developed contacts
with Huawei's founder and then, between about 2003 and 2007, conspired to steal technology from
Motorola by way of a dummy corporation they had set up outside the company.

The national security issue has been bubbling up for some time. In a letter in August, a group of
Republican senators wrote to the heads of four federal agencies asking questions about the risks of
Huawei's entering a deal with Sprint, whose customers include the United States military and law
enforcement agencies.

The senators, who are seeking a stringent government review of Huawei, said they were troubled by
the company's history, including evidence it had supplied communications equipment to Iran and Iraq
during Saddam Hussein's regime, possibly in violation of United Nations sanctions.

``We are concerned,'' the senators wrote, ``that Huawei's position as a supplier of Sprint Nextel
could create substantial risk for U.S.~corporations and possibly undermine U.S.~national security.''

The reservations about Huawei extend to other countries. In Europe, some competitors are now
complaining about so-called subsidies that Huawei receives from the Chinese government. And in
India, there are worries that Huawei networks could pose security risks.

Huawei denies it has ties to the Chinese military and disputes accusations of intellectual property
theft. Ross Gan, a company spokesman, says that Huawei is employee-owned and that it has grown by
developing its own technology.

``We're an innovative company driven by the business needs of customers,'' he said. In a statement,
the company added: ``Huawei has never researched, developed, manufactured or sold technologies or
products for military purposes in any country.''

Industry analysts say Huawei, based in Shenzhen, has quickly matured into a fierce competitor in one
of the most important and hotly contested technology arenas: sophisticated equipment that enhances
the delivery of voice and video over the Internet and through wireless devices.

They say Huawei is gaining, in part, because of heavy spending on research and development. Chinese
companies are generally weak in R.\&D., but Huawei has 17 research centers around the world,
including in Dallas, Moscow and Bangalore, India, and most recently in Santa Clara.

Indeed, of the company's 96,000 employees, nearly half are engaged in research and development. In
May, Huawei opened a stunning \$340 million research center in Shanghai that it says will eventually
house 8,000 engineers.

Huawei's rush to become multinational has not been entirely smooth. ``It was a huge challenge for
the company,'' said Geoff Arnold, a veteran Silicon Valley software designer who spent several years
helping the company develop a cloud computing product.

``The bean counters in Shenzhen didn't have a clue about how to operate outside of China,''
Mr.~Arnold said. ``Huawei has great difficulty understanding what is happening outside of China and
adapting their business practices.''

Ren Zhengfei, a former soldier who worked for 10 years in China's Army Engineering Corps, founded
Huawei as a reseller of telecommunications equipment in 1988.

Mr.~Ren, now 66, rarely grants interviews. But according to a biography published in China, he
insists on military-style efficiency and a ``wolf spirit'' mentality that encourages the sales force
to relentlessly attack competitors.

In 2008, worries about national security and China's weak protection of intellectual property forced
Huawei to drop its \$2.2 billion joint bid with the American firm Bain Capital to acquire 3Com, the
American networking company. Huawei also failed in other bids this year to acquire the wireless
network division of Motorola as well as 2Wire, an American maker of broadband Internet software,
according to people familiar with those deals.

Those bids collapsed, analysts say, because both Motorola and 2Wire were told that Washington was
likely to block any deals.

Analysts note that Chinese companies have been willing to buy telecommunications equipment from
American makers like Motorola, apparently setting aside any concerns about American espionage.

Peter J.~Williamson, a professor of business at Cambridge University, said that while some continued
to be bothered by Huawei's origins, its technological prowess was increasingly hard to ignore.

``The hardest market to crack is the U.S.,'' he said. ``But they've cracked Europe. And if they can
work with Vodafone, one of the biggest carriers in the world, they can work with anyone.''

\section{Testing a Bar Code Technology for Smartphones}

\lettrine{S}{haron}\mycalendar{Oct.'10}{29} Bolton spotted ``those funky tags'' -- known formally as
two-dimensional bar codes -- when she took her college-age daughter to catch a train at the
Rensselaer rail station near Albany last week.

``I looked up and saw these little black-and-white boxes on the lime green wall,'' said Ms.~Bolton,
a graphic artist from Saratoga Springs, N.Y., ``and right away I knew they were those funky tags
where you click and connect to a Web site.''

She was one of the 4,500 people daily riding Albany's buses or passing through its rail stations
that are the focus of a test of the visibility and effectiveness of 2-D bar code technology. The
promotion is being conducted by the Lamar Advertising Company, one of the country's largest outdoor
advertising businesses.

$\bullet$

Albany's transit system has been blanketed with the bar codes -- also called quick response or QR
bar codes -- which consumers can scan with their smartphone and, within seconds, connect to a Web
site, photo or video. In the Albany test, users access QRiousAlbany.com, where they can register for
a contest to win an iPad.

``Several national clients asked us about using this technology in their advertising, so we decided
to see how well it works,'' said Clifford B.~Wohl, vice president and general manager of Lamar
Transit Advertising, the part of the company dealing with transit systems.

Bar code campaigns are cropping up in other transit hubs, as well. In Denver International Airport,
for example, Colorado-based FirstBank began to offer this month a free download of an e-book to
passengers scanning the bar code on posters mounted in terminal corridors.

The posters say ``free books,'' and mobile phone users scanning the code -- a scattering of
black-and-white boxes inside a larger square -- are linked to a Web page with several e-book choices
that can be downloaded at no cost. In the first two weeks, the most popular titles were ``The Art of
War,'' ``Treasure Island'' and ``The Autobiography of Benjamin Franklin,'' according to Matt Best, a
spokesman for the bank's advertising agency, TDA Advertising \& Design, in Boulder, Colo.

Airline passengers looking to fill their waiting time can also download free crosswords starting
Nov.~1, and free Sudoku games beginning Dec.~1, Mr.~Best said. The effort, which is hosted by Clear
Channel Airports, part of Phoenix-based Clear Channel Outdoor, another large outdoor ad company, had
about 1,250 downloads in the first 17 days after it began on Oct.~1, he said. Over all, about 7,000
books and puzzles are expected to be downloaded during the five-month promotion.

CBS Outdoor Advertising is working with the Ford Motor Company and the University of Maryland, which
have placed bar-coded posters in Washington Metropolitan Area Transit Authority stations. Clicking
on Ford's bar code takes the user to information about its EcoBoost engine technology, and the
university is promoting unique aspects of its educational programs with a series of ads, including
one about the entrepreneurs it has fostered.

Jodi Senese, the CBS Outdoor executive vice president for marketing, said the company expected to
have more bar code campaigns under way after Dec.~1.

For marketers, ``this is the holy grail of advertising -- interactive media in public places,'' said
Michael Becker, North American managing director for the Mobile Marketing Association, the industry
trade group. But he said the technology had challenges. One is connectivity in places like subways,
he said, but added that ``a lot of information can be embedded in a QR code, and accessed later.''

Other drawbacks include the lack of a single industry standard for 2-D bar codes in the United
States, and the relative paucity of phones that can use 2-D. The number of users with phones
equipped with scanners is expected, however, to rise to 50 percent of all users by the end of next
year, Mr.~Becker said.

In Albany, Lamar has placed graphics on the walls, floors, kiosks and other areas of the city's two
rail stations, and on bus shelters and the ceilings of more than 50 buses run by the city's Capital
District Transportation Authority.

Lamar designed the bar code ads, Mr.~Wohl said, and spent about \$10,000 to post the advertisements
in the public locations. The transit authority, said spokesman Margo Janack, is hoping that bar code
ads will proliferate and provide new revenue streams to offset lower ridership because of recent job
losses.

``The ads look like modern art,'' she said. ``People are definitely noticing.''

$\bullet$

Whether Albany transit users are scanning the ads in great numbers is unknown. There are no reported
results so far, Mr.~Wohl said. Once the results arrive, they will help clarify the extent to which
the public recognizes the bar codes.

Ms.~Bolton said she was familiar with bar codes, but only because her graphics firm had been asked
recently to incorporate one in an educational campaign.

The most popular information sought from bar codes includes ``where to buy, what do others think,
advice on usage and nutritional information,'' said Cameron Green, spokesman for GS1, a nonprofit
association that works to establish uniform technology standards.

Still, it may take some time to establish consumer familiarity if the experience of one Albany
commuter, Abbey Greenbaum, is a guide. Ms.~Greenbaum, a regional training coordinator for the New
York state health department, noticed the Lamar bar code ad on the ceiling of the bus she takes to
work everyday.

``I had seen them in magazines. It's a cute, novelty idea,'' she said. She scanned the ad and
entered the contest for the iPad. But, she noted, ``I was the only one on the bus who did.''

\section{Suicide Bomber Injures 22 in Istanbul}

\lettrine{A}{}\mycalendar{Oct.'10}{31} male suicide bomber injured at least twenty-two people in an
attack aimed at a police unit in the busiest local and tourist district in Istanbul on Sunday
morning, the city governor's office said.

The injured, including ten civilians and twelve policemen, were taken to four nearby hospitals for
treatment.

``We do not yet have information on the aim or nature of the attack,'' Huseyin Avni Mutlu, the
Istanbul governor, said in a live televised statement. ``Nevertheless, this is an act of terror.''

Television networks reported that the police were investigating a second suspect.

The bombing occurred in Taksim Square in the Beyoglu district, popular among tourists and locals.

Police cordoned off the site of the explosion and entirely blocked pedestrian and vehicle traffic
around the neighborhood.

``I once more remind here that those who want to stir up Turkey, destroy the air of peace, stability
and security will never be tolerated,'' Recep Tayyip Erdogan, the prime minister, said in Mardin, a
southeastern town in Turkey. ``Such attacks will by no means obstruct Turkey from its unity,
brotherhood and aims for progress.''

In 2003, a local fundamentalist network loyal to Al Qaeda had killed more than 60 and injured
hundreds in attacks targeting a synagogue and the British Consulate in Beyoglu district as well as
the headquarters of the British HSBC Bank in Levent neighborhood.

Television pictures showed the body of the suicide bomber lying close to the statue of Kemal
Ataturk, the founder of modern Turkey, that marks the very center of the square in Beyoglu's Taksim
area.

Istanbul police units often keep a mobile station at the square with shuttle busses and special riot
units to ensure security in a district popular among tourists.

The Kurdistan Workers' Party, the P.K.K., has been engaged in an armed struggle against Turkish
armed forces to claim autonomy in the predominantly Kurdish southeast since the early 1980s and have
often staged attacks in busy urban areas, targeting civilians.

A one-sided ceasefire issued by the group to allow politicians to offer a permanent solution expired
today, but officials refused to comment on the P.K.K.'s possible links with the attack in Taksim.

Turkey celebrated the founding of the republic on Friday in official ceremonies throughout the
country, in the midst of continuing efforts to resolve the Kurdish armed conflict that has so far
claimed more than 40,000 lives.

\section{U.S.~Sees Complexity of Bombs as Link to Al Qaeda}

\lettrine{T}{he}\mycalendar{Oct.'10}{31} powerful bombs concealed inside cargo packages and destined
for the United States were expertly constructed and unusually sophisticated, American officials said
Saturday, further evidence that Al Qaeda's affiliate in Yemen is steadily improving its abilities to
strike on American soil.

As investigators on three continents conducted forensic analyses of two bombs shipped from Yemen and
intercepted Friday in Britain and Dubai, American officials said evidence was mounting that the top
leadership of Al Qaeda in the Arabian Peninsula, including the radical American-born cleric Anwar
al-Awlaki, was behind the attempted attacks.

Yemeni officials on Saturday announced the arrest of a young woman and her mother in connection with
the plot, which also may have involved two language schools in Yemen. The two women were not
identified, but a defense lawyer who has been in contact with the family, Abdul Rahman Barham, said
the daughter was a 22 year-old engineering student at Sana University.

Yemen's president, Ali Abdullah Saleh, said Saturday night during a news conference that Yemeni
security forces had identified her based on a tip from American officials, but he did not indicate
her suspected role.

Investigators said that the bomb discovered at the Dubai airport in the United Arab Emirates was
concealed in a Hewlett-Packard desktop printer, with high explosives packed into a printer cartridge
to avoid detection by scanners.

``The wiring of the device indicates that this was done by professionals,'' said one official
involved in the investigation, who like several officials spoke on condition of anonymity because
the inquiry was continuing. ``It was set up so that if you scan it, all the printer components would
look right.''

The bomb discovered in Britain was also hidden in a printer cartridge.

The terror plot broke publicly in dramatic fashion on Friday morning, when the two packages
containing explosives and addressed to synagogues or Jewish community centers in Chicago were found,
setting off an international dragnet and fears about packages yet to be discovered. It also led to a
tense scene in which American military jets escorted a plane to Kennedy International Airport amid
concerns -- which turned out to be unfounded -- that there might be explosives on board.

On Saturday, in news conferences in London and Yemen, and from interviews with investigators here
and abroad, the contours of the investigation began to emerge, along with new details of the frantic
hours leading to the discovery of the packages.

American officials said their operating assumption was that the two bombs were the work of Ibrahim
Hassan al-Asiri, Al Qaeda in Yemen's top bomb-maker, whose previous devices have been more
rudimentary, and also unsuccessful. Mr.~Asiri is believed to have built both the bomb sewn into the
underwear of the young Nigerian who tried to blow up a trans-Atlantic flight last Dec.~25, and the
suicide bomb that nearly killed Saudi Arabia's intelligence chief, Mohammed bin Nayef, months
earlier. (In the second episode, American officials say, Mr.~Asiri hid the explosives in a body
cavity of his brother, the suicide bomber.)

Just as in the two previous attacks, the bomb discovered in Dubai contained the explosive PETN,
according to the Dubai police and Janet Napolitano, the secretary of homeland security. This new
plot, Ms.~Napolitano said, had the ``hallmarks of Al Qaeda.''

The targets of the bombs remained in question.

Prime Minister David Cameron of Britain said on Saturday that the parcel bomb intercepted in England
was designed to explode while the plane was flying. The country's home secretary, Theresa May, said
that British investigators had also concluded the device was ``viable and could have exploded.''

``The target may have been an aircraft, and had it detonated, the aircraft could have been brought
down,'' she said.

But earlier in the day, Representative Michael McCaul of Texas, the ranking Republican on the House
homeland security intelligence subcommittee, said that federal authorities indicated to him that the
packages were probably intended to blow up the Jewish sites in Chicago rather than the cargo planes,
since they do not carry passengers.

Based on a conversation with Ms.~Napolitano, he said that authorities were also leaving open the
possibility that other packages with explosives had not yet been found. On Saturday, Deputy
Commissioner Paul J.~Browne, the New York Police Department's chief spokesman, said that no specific
threats had been made against synagogues or Jewish neighborhoods in the city, but that officers were
watching them more closely as a precaution.

It was a call from Mr.~bin Nayef, the Saudi intelligence chief, on Thursday evening to John
O.~Brennan, the White House senior counterterrorism official and former C.I.A. station chief in
Riyadh, the Saudi capital, that set off the search, according to American officials. They said
Mr.~bin Nayef also notified C.I.A. officials in Riyadh.

Saudi Arabia has sometimes been a reluctant ally in America's global campaign against radical
militants. But it sees Yemen, its impoverished next door neighbor, as a different matter. The Saudis
consider the Qaeda branch in Yemen its biggest security threat and Saudi intelligence has set up
both a web of electronic surveillance and spies to penetrate the organization.

Reviewing the evidence, American intelligence officials say they believe that the plot may have been
blessed by the highest levels of Al Qaeda's affiliate in Yemen, including Mr.~Awlaki.

``We know that Awlaki has taken a very specific interest in plotting against the United States, and
we've found that he's usually behind any attempted attack on American targets,'' said one official.

Still they cautioned that it was still early to draw any firm conclusions and they did not present
proof of Mr.~Awlaki's involvement.

This year, the C.I.A. designated Mr.~Awlaki -- an American citizen -- as a high priority for the
agency's campaign of targeted killing.

According to one official involved in the investigation, the package that was discovered in Dubai
had a woman's name and location in Sana on the return address. The package left Yemen on Thursday,
the official said, where it was flown to Doha, Qatar, and on to Dubai.

Also on Saturday, the Department of Homeland Security dispatched a cable warning that the bombs may
have been associated with two schools in Yemen -- the Yemen American Institute for
Languages-Computer Management, and the American Center for Training and Development.

That connection would echo the attempted bombing last Dec.~25; the Nigerian who was implicated had
studied at a different Sana language school before training with Al Qaeda. If language schools are
again involved, it opens the possibility that a foreign student or students may have participated in
the plot.

Security forces in Yemen were in a state of heightened alert on Saturday, as investigators
questioned cargo employees and shut down the FedEx and U.P.S. offices in Sana, the Yemeni capital.

Obama administration officials said they were discussing a range of responses to the thwarted
attack. The failed attack on Dec.~25 created an opportunity for the White House to press Yemen's
government to take more aggressive action against Qaeda operatives there, and some American
officials believe the conditions are similar now.

A thinly veiled campaign of American missile strikes in Yemen this year has achieved mixed results.
American officials said that several Qaeda operatives had been killed in the attacks, but there have
also been major setbacks, including a strike in May that accidentally killed a deputy governor in a
remote province of Yemen. That strike infuriated Yemen's president, Mr.~Saleh, and forced a
months-long halt in the American military campaign.

In recent months, the Obama administration has been debating whether to escalate its secret
offensive against the Qaeda affiliate in Yemen. The C.I.A. has a fraction of the staff in Yemen that
it currently has in Pakistan, where the spy agency is running a covert war in the country's tribal
areas, but over the course of the year the C.I.A. has sent more case officers and analysts to Sana
as part of a task force with the military's Joint Special Operations Command.

American officials have been considering sending armed drone aircraft to Yemen to replicate the
Pakistan campaign, but such a move would almost certainly require the approval of the mercurial
Mr.~Saleh.

Yemeni officials have declined to comment on details of the plot, saying only that they are
investigating. But new checkpoints appeared in the capital on Saturday, with officers checking the
identity cards of drivers and pedestrians.

\section{China's Fast Rise Leads Neighbors to Join Forces}

\lettrine{C}{hina}\mycalendar{Oct.'10}{31}'s military expansion and assertive trade policies have
set off jitters across Asia, prompting many of its neighbors to rekindle old alliances and cultivate
new ones to better defend their interests against the rising superpower.

A whirl of deal-making and diplomacy, from Tokyo to New Delhi, is giving the United States an
opportunity to reassert itself in a region where its eclipse by China has been viewed as inevitable.

President Obama's trip to the region this week, his most extensive as president, will take him to
the area's big democracies, India, Indonesia, South Korea and Japan, skirting authoritarian China.
Those countries and other neighbors have taken steps, though with varying degrees of candor, to
blunt China's assertiveness in the region.

Mr.~Obama and Prime Minister Manmohan Singh of India are expected to sign a landmark deal for
American military transport aircraft and are discussing the possible sale of jet fighters, which
would escalate the Pentagon's defense partnership with India to new heights. Japan and India are
courting Southeast Asian nations with trade agreements and talk of a ``circle of democracy.''
Vietnam has a rapidly warming rapport with its old foe, the United States, in large part because its
old friend, China, makes broad territorial claims in the South China Sea.

The deals and alliances are not intended to contain China. But they suggest a palpable shift in the
diplomatic landscape, on vivid display as leaders from 18 countries gathered this weekend under the
wavelike roof of Hanoi's futuristic convention center, not far from Ho Chi Minh's mausoleum, for a
meeting suffused by tensions between China and its neighbors.

China's escalating feud with Japan over another set of islands, in the East China Sea, stole the
meeting's headlines on Saturday, and Secretary of State Hillary Rodham Clinton proposed three-way
negotiations to resolve the issue.

Most Asian countries, even as they argue that China will inevitably replace the United States as the
top regional power, have grown concerned at how quickly that shift is occurring, and what China the
superpower may look like.

China's big trading partners are complaining more loudly that it intervenes too aggressively to keep
its currency undervalued. Its recent restrictions on exports of crucial rare earths minerals, first
to Japan and then to the United States and Europe, raised the prospect that it may use its dominant
positions in some industries as a diplomatic and political weapon.

And its rapid naval expansion, combined with a more strident defense of its claims to disputed
territories far off its shores, has persuaded Japan, South Korea, Vietnam and Singapore to reaffirm
their enthusiasm for the American security umbrella.

``The most common thing that Asian leaders have said to me in my travels over this last 20 months
is, `Thank you, we're so glad that you're playing an active role in Asia again,' '' Mrs.~Clinton
said in Hawaii, opening a seven-country tour of Asia that included a last-minute stop in China.

Few of China's neighbors voice their concerns about the country publicly, but analysts and diplomats
say they express wariness about the pace of China's military expansion and the severity of its trade
policies in private.

``Most of these countries have come to us and said, `We're really worried about China,' '' said
Kenneth G.~Lieberthal, a China adviser to President Bill Clinton who is now at the Brookings
Institution.

The Obama administration has been quick to capitalize on China's missteps. Where officials used to
speak of China as the Asian economic giant, they now speak of India and China as twin giants. And
they make clear which one they believe has a closer affinity to the United States.

``India and the United States have never mattered more to each other,'' Mrs.~Clinton said. ``As the
world's two largest democracies, we are united by common interests and common values.''

As Mr.~Obama prepares to visit India in his first stop on his tour of Asian democracies, Mr.~Singh,
India's prime minister, will have just returned from his own grand tour -- with both of them
somewhat conspicuously, if at least partly coincidentally, circling China.

None of this seems likely to lead to a cold war-style standoff. China is fully integrated into the
global economy, and all of its neighbors are eager to deepen their ties with it. China has fought no
wars since a border skirmish with Vietnam three decades ago, and it often emphasizes that it has no
intention of projecting power through the use of force.

At the same time, fears that China has become more assertive as it has grown richer are having real
consequences.

India is promoting itself throughout the region as a counterweight to China; Japan is settling a
dispute with the United States over a Marine air base; the Vietnamese are negotiating a deal to
obtain civilian nuclear technology from the United States; and the Americans, who had largely
ignored the rest of Asia as they waged wars in Afghanistan and Iraq, see an opportunity to come back
in a big way.

In July, for example, Mrs.~Clinton reassured Vietnam and the Philippines by announcing that the
United States would be willing to help resolve disputes between China and its neighbors over a
string of strategically important islands in the South China Sea.

China's foreign minister, Yang Jiechi, reacted furiously, accusing the United States of plotting
against it, according to people briefed on the meeting. Mr.~Yang went on to note that China was a
big country, staring pointedly at the foreign minister of tiny Singapore. Undaunted, Mrs.~Clinton
not only repeated the American pledge on the South China Sea in Hanoi on Saturday, but expanded it
to include the dispute with Japan.

China's rise as an authoritarian power has also revived a sense that democracies should stick
together. K.~Subrahmanyam, an influential strategic analyst in India, noted that half the world's
people now live in democracies and that of the world's six biggest powers, only China has not
accepted democracy.

``Today the problem is a rising China that is not democratic and is challenging for the No.~1
position in the world,'' he said.

Indeed, how to deal with China seems to be an abiding preoccupation of Asia's leaders. In Japan,
Prime Minister Naoto Kan and Mr.~Singh discussed China's booming economy, military expansion and
increased territorial assertiveness.

``Prime Minister Kan was keen to understand how India engages China,'' India's foreign secretary,
Nirupama Rao, told reporters. ``Our prime minister said it requires developing trust, close
engagement and a lot of patience.''

South Korea was deeply frustrated earlier this year when China blocked an explicit international
condemnation of North Korea for sinking a South Korean warship, the Cheonan. South Korea accused
North Korea of the attack, but China, a historic ally of the North, was unwilling to hold it
responsible.

India has watched nervously as China has started building ports in Sri Lanka and Pakistan, extending
rail lines toward the border of Nepal, and otherwise seeking to expand its footprint in South Asia.

India's Defense Ministry has sought military contacts with a host of Asian nations while steadily
expanding contacts and weapons procurements from the United States. The United States, American
officials said, has conducted more exercises in recent years with India than with any other nation.

Mr.~Singh's trip was part of his ``Look East'' policy, intended to broaden trade with the rest of
Asia. He has said it was not related to any frictions with China, but China is concerned. On
Thursday, People's Daily, the Communist Party newspaper, ran an opinion article asking, ``Does
India's `Look East' Policy Mean `Look to Encircle China'?''

That wary view may well reflect China's reaction to the whole panoply of developments among its
neighbors.

``The Chinese perceived the Hanoi meeting as a gang attack on them,'' said Charles Freeman, an
expert on Chinese politics and economics at the Center for Strategic and International Studies.
``There's no question that they have miscalculated their own standing in the region.''

\section{U.S.~Works to Ease China-Japan Conflict}

\lettrine{W}{ith}\mycalendar{Oct.'10}{31} tensions between China and Japan spilling out at an East
Asian summit meeting here, the United States is trying to defuse an escalating diplomatic row over
their competing claims to a cluster of small islands in the East China Sea.

On Saturday, Secretary of State Hillary Rodham Clinton proposed a three-way meeting with China and
Japan to resolve the dispute, which has raged since last month when Japan detained the captain of a
Chinese fishing vessel that struck two Japanese patrol boats near the islands.

``We have certainly encouraged both Japan and China to seek peaceful resolution of any disagreements
that they have,'' Mrs.~Clinton said at a news conference after the summit meeting ended. ``It is in
all of our interest for China and Japan to have stable, peaceful relations.''

In private conversations with Chinese and Japanese diplomats, Mrs.~Clinton ``made very clear to both
sides that we want the temperature to go down on these issues,'' a senior official said. American
officials said they were troubled by what one called a sudden, drastic increase in tensions.

As the United States, Russia and 16 Asian nations gathered in Hanoi to discuss regional cooperation,
China's aggressive maritime and territorial claims were sowing unease with several of its neighbors.

When Japan last week reasserted its sovereignty over the islands -- which it calls the Senkaku and
China calls the Diaoyu -- a senior Chinese official accused it of ruining the atmosphere of the
summit meeting.

The United States, which had been mostly a bystander in such disputes, has taken a more active role
under the Obama administration. Though it has no position on the sovereignty claims, Mrs.~Clinton
said the United States viewed the islands as protected under the terms of its defense treaty with
Japan, which means it will defend them from any foreign attack.

That statement brought a rebuke from the Chinese Foreign Ministry spokesman, Ma Zhaoxu, who said
China ``will never accept any word or deed that includes the Diaoyu Islands within the scope'' of
the treaty.

On another issue that has caused friction lately -- China's halting of shipments of strategically
important minerals to the United States, Japan and Europe -- the Chinese government seemed eager to
reassure.

In a meeting with Mrs.~Clinton, Foreign Minister Yang Jiechi gave ``very clear indications'' that
China would fulfill its contracts and be a ``reliable supplier,'' according to an American official.

``While we're pleased by the clarification received from the Chinese government,'' Mrs.~Clinton
said, ``we still think the world as a whole needs to find alternatives'' to China as a supplier of
the minerals, known as rare earth metals.

China began curtailing shipments to the United States and Europe of these minerals, which are used
to make products like cellphones and wind turbines, after the dispute with Japan and a trade
investigation by the Obama administration. Then last week, without explanation, Chinese officials
said the shipments would resume.

Japan, which released the Chinese captain under heavy pressure from Beijing, had proposed a meeting
with Chinese leaders in Hanoi to clear the air. But hopes for that were dashed when Japan's foreign
minister, Seiji Maehara, asserted Japan's control over the islands last week.

Prime Minister Wen Jiabao of China refused to meet one-on-one with Prime Minister Naoto Kan of
Japan, though Mr.~Yang said China would consider Mrs.~Clinton's proposed trilateral meeting.

In her formal remarks to the Asian leaders, Mrs.~Clinton reiterated that the United States stood
ready to help resolve another territorial dispute: one that pits China against Vietnam, the
Philippines and other countries over a string of strategically significant islands in the South
China Sea.

``The United States has a national interest in the freedom of navigation and unimpeded lawful
commerce,'' she said. ``And when disputes arise over maritime territory, we are committed to
resolving them peacefully based on customary international law.''

The administration's position angers China, which has also sparred with the United States over
currency policy and trade. Chinese officials have expressed concern that all the friction could get
in the way of a visit to the United States early next year by President Hu Jintao.

At Beijing's request, Mrs.~Clinton added a last-minute China stop to her itinerary, meeting the
state councilor for foreign affairs, Dai Bingguo, on Saturday on Hainan Island, east of Vietnam. She
pressed Mr.~Dai to use Beijing's influence on North Korea to discourage it from ``provocative'' acts
before the Group of 20 leaders' meeting in Seoul next month.

\section{Koreans Reunite at a Red Cross Gathering}

\lettrine{H}{undreds}\mycalendar{Oct.'10}{31} of South Koreans, separated from family members in
North Korea for more than half a century, were reunited with their relatives on Saturday as a Red
Cross gathering got under way at the Diamond Mountain resort in the North.

The South Koreans arrived with gym bags and suitcases bulging with goods suggested by the Red Cross,
daily amenities that are scarce in the impoverished North -- sugar, soap, underwear, aspirin,
toothpaste and toothbrushes, vitamins, wool socks, even sacks of rice. Some South Koreans also took
United States dollars, even though gifts of currency were discouraged by the organizers.

A brief exchange of gunfire between North and South Korean troops on Friday night did not delay or
cancel the event. No one was injured in the cross-border shooting, an official with the Defense
Ministry said, and analysts in Seoul said they believed that the initial shots, from the North, were
probably accidental.

The reunion that began Saturday was the first in more than a year. It was to continue until Tuesday,
and a second group of relatives is scheduled to begin another meeting from Wednesday to Friday.
About 100 families from each country will meet up with one or more relatives.

Most families became separated during the 1950-53 Korean War, essentially becoming stranded on
opposite sides of the border when the fighting stopped.

Virtually no communication channels exist for average North and South Koreans. There is no
inter-Korean travel and both governments block telephone, Internet and postal services.

Oh Seung-geun, 76, a former South Korean Air Force captain, waited 15 years to be chosen for a
reunion. He said he got a call from the Red Cross two weeks ago telling him he had been approved to
go to Diamond Mountain, and he was hoping to see two sisters, a younger brother and a cousin.

``Even if only one of them shows up,'' he said, ``my heart will be open.''

The meetings can be thrilling and joyful, or awkward and tense, say those who have attended. Both
delegations have ``minders'' who monitor conversations, and many North Koreans feel unable to speak
freely with the regime's authorities nearby. Wealthier relatives from the South can be too
persistent in asking about the daily deprivations and oppressive political life in the North.

``I know the North Korean regime, I grew up with that system,'' said Mr.~Oh, who was born in
Pyongyang, the North's capital, in 1934. ``If I talk about certain things, it could be perilous for
my family.''

The current reunion is the 18th since 2000. Nearly 21,000 South Koreans have been reunited with
relatives so far, the Red Cross said, including about 4,000 connected through video conferencing.

Meanwhile, 88,000 South Koreans are on a Red Cross waiting list. But time is running out; about half
the South Korean applicants are now over 80.

The family reunions have become a highly politicized issue for the North and the South alike. The
historian James A.~Foley, in his book about separated Korean families and cross-border reunions,
detailed the ``long history of both sides' manipulation of the divided families issue for their own
political gains.''

The South has pressed the North for larger and more frequent reunions, but negotiations broke down
last week, and the future of the reunions remained unclear.

The talks ended as North Korea, through its Red Cross team, requested donations of 500,000 tons of
rice and 300,000 tons of fertilizer. South Korean Red Cross officials said they were not authorized
to agree to so much aid, and analysts said the conservative government of President Lee Myung-bak
was unlikely to grant the request.

\section{Why Twitter's C.E.O. Demoted Himself}

\lettrine{A}{t}\mycalendar{Oct.'10}{31} the annual South by Southwest gathering of techies in
Austin, Tex., in March, conference organizers had chosen a hangar-size room to accommodate their
star speaker: Evan Williams, the co-founder of Twitter, the messaging and social networking site
that had become a digital phenomenon.

In a private moment before the doors opened, Mr.~Williams, who is famously deliberate and cautious,
snapped a photograph of the endless rows of chairs facing the stage and posted it on Twitter.

``Gulp,'' he wrote.

Later, as Mr.~Williams talked with the interviewer about building a 21st-century business, keeping
to Twitter's foundational principle (the Google-like ``be a force for good'') and fostering
corporate experimentation, members of his audience started groaning -- and leaving, one by one.

``They wanted Ev Williams; they got Ev Williams,'' a Twitter staff member said later.

It is no small irony, of course, that a man so ill at ease on the big stage is a pivotal force in a
communications revolution, one that has made it easier for people to chat, disseminate information
and mobilize locally and globally with almost anyone who has a cellphone or an Internet connection.

And Twitter has become one of the rare but fabled Web companies with a growth rate that resembles
the shape of a hockey stick. It has 175 million registered users, up from 503,000 three years ago
and 58 million just last year. It is adding about 370,000 new users a day.

It has helped transform the way that news is gathered and distributed, reshaped how public figures
from celebrities to political leaders communicate, and played a role in popular protests in Iran,
China and Moldova. It has become so muscular and ubiquitous that it now competes with the likes of
Google and Facebook for users -- and is beginning to compete with them for advertising dollars.

Yet for all its astonishing growth, Twitter has succeeded in spite of itself -- the enviable product
of a great idea and lightning-in-a-bottle viral success rather than a disciplined approach to how
it's managed.

Because of that, Twitter is on the cusp of becoming the next big, independent Internet company -- or
the next start-up to be swallowed whole by a giant like Google or, possibly, the next start-up to
run out of steam.

Now the company is trying to instill some of the rigor and sense of purpose it needs to ensure that
it is, indeed, the next big thing.

``The thing I've learned that's much different than any other time in my life is I have a team that
is really, really great,'' says Mr.~Williams, 38. ``I've been studying this stuff for a really long
time, and I've screwed up in many, many, many ways in terms of managing people and product decisions
and business, so I feel fairly confident at this point that it could scale pretty well.''

Last month, he unexpectedly announced that he had decided to step down as chief executive and give
the job to Dick Costolo, who had been Twitter's chief operating officer.

Mr.~Williams, who remains on the company's board, now focuses on product strategy. He made the
decision after conceiving and spending months working on the recent redesign of the Twitter Web
site. People who have worked with him say he excels at understanding what Internet users want and
contemplating Twitter's future, but isn't a detail-oriented task manager.

``He takes these things that everyone thinks are as big as they can get, these geeky things, and he
makes them mainstream,'' says Philip Kaplan, a co-founder of the review site Blippy and one of
Mr.~Williams's close friends.

Mr.~Costolo, meanwhile, is all about the details of making money and getting things done. This has
been his third time running a company; he sold his last one, the Web subscription service
FeedBurner, to Google in 2007.

For his part, Mr.~Williams may embody a classic Silicon Valley type -- the inspired, talented
start-up guy with good ideas, but not the one to execute a sophisticated business strategy once
things get rolling, says Steve Blank, an entrepreneurship teacher at Stanford.

And Mr.~Williams may have also earned the self-awareness and confidence to recognize exactly who he
is.

``Evan Williams is the type of entrepreneur who knows when to pivot,'' says Mr.~Blank, ``and what we
may be seeing is wonderful signs of entrepreneurial wisdom.''

TWITTER was born in 2006 as a side project.

At the time, it was an appendage of a podcasting service named Odeo, another company that
Mr.~Williams co-founded that had millions of dollars from investors.

Even the founders, though, were having a hard time getting excited about Odeo, and Mr.~Williams told
everyone who worked there to hatch new ideas. While sitting on a children's slide at a park eating
Mexican food one day, an engineer, Jack Dorsey, suggested to colleagues a simple way to send status
updates by using text messages.

Mr.~Dorsey and Twitter's third co-founder, Biz Stone, built a prototype in two weeks. During that
time, Mr.~Stone was ripping up the carpet at his Berkeley home when his cellphone vibrated in his
pocket. It was Mr.~Williams sending a message on Twitter: ``Sipping pinot noir after a massage in
Napa Valley.''

Twitter was a unique entrant on the social media scene. People could follow others without being
followed back, and all posts were public by default -- and limited to 140 characters so they could
fit inside cellphone text messages.

The founders likened Twitter to ice cream: not that useful, but ``a fun thing for family and friends
when they are not in the same place,'' Mr.~Williams says.

That is a far cry from his vision today, an about-face that is typical of Twitter's evolution. For a
long time, Twitter's founders talked about it with awe, as if it had a life of its own and they were
mere bystanders. They freely acknowledged that they had no idea how people would use it or how it
would make money.

But they thought it had potential, and in 2007 they spun it off as a separate company from Odeo,
with Mr.~Dorsey serving as Twitter's first chief executive, Mr.~Stone as creative director and
Mr.~Williams as chairman.

Mr.~Williams had dipped into his own funds to cash out Odeo's investors and subsequently gained a
controlling stake in Twitter -- but he was spending his days running yet another company, Obvious,
an incubator for start-ups, and wasn't focused on managing Twitter.

While Mr.~Stone, an outgoing showman, was friends with both Mr.~Williams and Mr.~Dorsey and
socialized with them, Mr.~Williams and Mr.~Dorsey are much quieter men whose only bond was their
work. When Mr.~Williams decided to join Twitter full time in the spring of 2008, his relationship
with Mr.~Dorsey quickly became strained as the two men competed for power.

By the end of 2008, Twitter's growth was exploding -- and things inside the company were beginning
to break down. Mr.~Williams suggested to Twitter's board that it push Mr.~Dorsey out. With the
exception of Mr.~Dorsey, the board unanimously agreed, according to several people involved in the
discussions. Mr.~Williams had run three companies, directors reasoned, so they figured that he would
do a better job.

Upon Mr.~Williams's ascent, Mr.~Dorsey became Twitter's chairman. Although that move was potentially
fraught with problems, the board wanted to ensure that Mr.~Dorsey remained close to the company
because he still owned a large stake in Twitter and he had originally come up with the idea for it,
according to two board members.

The change shocked employees and further frayed relations between Mr.~Dorsey and Mr.~Williams.
Mr.~Dorsey declined to comment for this article, but people close to him say he felt betrayed by
Mr.~Williams.

``There was a feeling that Ev wanted to take control after he realized the potential importance of
Twitter,'' says a Twitter employee who was there during the transition and requested anonymity to
protect business relationships.

``It's hard and confusing,'' Mr.~Williams says of Mr.~Dorsey's departure. ``I think there's few
cases in history where the C.E.O. steps down and is also the founder and reports to someone and that
works.''

Directors say Twitter's board meetings are amicable, and in the last couple weeks, Mr.~Dorsey has
been spotted around the offices more and has taken on a greater role in long-term strategy.

Even with Mr.~Williams as C.E.O., Twitter was growing faster than he or anyone else at the company
could handle. In 2009, Twitter ballooned to 71.3 million registered users from 5 million. The Web
site crashed often, and the ``fail whale'' -- an image of a whale that appears on the site whenever
Twitter falters -- became the butt of jokes.

Twitter was fielding dozens of calls a week from big companies, celebrities and politicians. Among
the callers were CNN, ``The Oprah Winfrey Show'' and the State Department, which asked Twitter to
delay maintenance so that Iranians protesting an election in 2009 could continue using the service.

``We were just hanging on by our fingernails to a rocket ship,'' Mr.~Williams recalls.

What the company needed was simple: people to do all the work. Yet it moved painfully slowly in
hiring, with just 110 employees by the end of 2009, even though it had raised \$150 million in
venture capital by then.

``The mistake I made was definitely underhiring, both in quantity and in experience, in several
areas, for a long time,'' Mr.~Williams says now. He attributes that mistake to the daily
distractions of running Twitter and not anticipating how big it would become.

Twitter's first office in San Francisco was classic start-up: dorm-room d\'ecor, complete with a keg
in the kitchen, a couple of big, green concrete deer and a communal table where employees ate
take-out burritos together.

Big-name chief executives would visit the company and sit on frumpy couches because there wasn't an
adequate conference room. A video crew once walked in through Twitter's unlocked front door without
permission and began recording employees.

The company's offices today have locked doors and a receptionist in a sunlit lobby, where the green
deer now stand. Trendy furniture includes plentiful conference tables, and while there is still a
keg, a chef prepares lunch for the 300 employees.

Sixty percent of those people are engineers, who have spent the last year methodically rebuilding
the software that runs Twitter and developing a system to monitor downtime.

The fail whale still appears, but not nearly as often, an important change now that Twitter sells
ads to companies like Starbucks, Ford and Microsoft. The ads can appear on the Twitter feed as
sponsored posts, or in Twitter's list of trending topics or among the suggested accounts that
Twitter recommends that its users follow. Mr.~Costolo spearheaded all of these initiatives.

Twitter finally hired a recruiter, as well as people to handle mundane but important big-company
tasks like human resources, payroll and ensuring that all of Twitter's partners use the same blue
bird logo. A whiteboard near the executives' desks lists headings like ``commit,'' ``invest'' and
``leverage.''

Mr.~Williams and his colleagues no longer liken Twitter to ice cream. They now describe it as an
information network, not a social tool, and see it as an essential way for people to communicate and
get information in real time.

Yet even though Twitter's executives say their heads are finally above water, Mr.~Williams still
describes the company as ``a 6-foot-tall sixth grader -- there's a lack of maturity, despite size
and the perception of outsiders.''

He says Twitter now has a team that can realize the company's ambitions -- a revelation coming from
someone who arrived in Silicon Valley with something to prove and volumes to learn about working
with others.

EVAN WILLIAMS grew up on a farm in Nebraska, ``90 miles and an eternity'' from Lincoln, he says. And
he didn't fit in.

``My brother was the consummate Nebraska boy -- the football star who went to the university, was
president of his fraternity, hunted with my dad all the time,'' he says. ``I just didn't feel at
home there.

``I had a fierce desire to create things, to be independent and prove myself, which caused me to
reject authority, but never in a sort of rebellious way,'' he adds. ``It was more like, `I'm going
to show you by doing it all myself.' ''

Mr.~Williams dropped out of the University of Nebraska and started a business in Lincoln, financed
by his father, designing Web sites for local businesses and making CD-ROMs about Nebraska football
and the Internet.

But it turned out that, among other problems, football fans weren't using CD-ROMs.~The business
ended up being Mr.~Williams's first failure, and he couldn't repay his father.

Mr.~Williams had devoured the early issues of Wired magazine, and California loomed in his
imagination as a place where he could truly carve out his own niche as an entrepreneur. He made his
first move west in 1997, with a marketing job at O'Reilly Media, the technology publisher in
Sebastopol, Calif.

``Ev was just very frustrated, and he had ideas for how we could do things differently and better,''
recalls Tim O'Reilly, the publisher's founder. ``He had a little bit of attitude, a chip on his
shoulder, but always with good spirit.''

Mr.~Williams left O'Reilly after seven months -- ``I was bad at working for people,'' he says. And
in January 1999, at the height of the dot-com bubble, he started his second company, Pyra Labs, with
his former girlfriend, Meg Hourihan. Paul Bausch, a friend from high school, soon joined.

Pyra made a Web-based project management tool but soon saw a different opportunity: a tool that
allowed users to easily post articles and photographs to personal blogs. That became Blogger, one of
the first Web services that automated blog publishing.

Soon after, the tech bubble burst, and Blogger was running out of money. Mr.~Williams told his five
employees, including Ms.~Hourihan and Mr.~Bausch, that he could no longer pay them and that he would
run the company alone.

But six months later, in June 2001, Blogger started making money by charging for added features, and
Mr.~Williams had a budget that allowed him to hire new workers. In 2003, Google acquired Blogger.

Several people who once worked at the company said they didn't make money on the sale because
Mr.~Williams had never submitted the paperwork needed to allocate stock options. Mr.~Williams says
that this group hadn't worked at the company long enough for their stock options to vest.

Others have a different view of Mr.~Williams's tenure at Blogger.

``I don't think he took care of the people who got him to where he was,'' says Ms.~Hourihan, who
earned millions of dollars from the sale. ``It was bitter, horrible and tough. He's not C.E.O.
material. It doesn't play to his strengths. He's a better inventor; he's better at coming up with
ideas.''

Mr.~Williams says that all successful businesspeople make enemies along the way. Yet he also says he
learned from the Blogger experience. ``I was trying to do everything myself when we were going
through hard times,'' he says. ``When it was just me, I was happier, which I think is a sign of
failure of working with people.''

In 2004, Mr.~Williams left Google, where he was still running Blogger, and planned to take time off.
Instead, he started working on Odeo with his neighbor Noah Glass. A year later, he again found
himself running a company.

MR. WILLIAMS doesn't fit the Silicon Valley stereotype. He is neither a back-slapping former frat
boy nor a socially awkward programmer most content behind a computer screen.

He is at ease with himself, and convivial and dryly funny in small settings, but he tends to be
quiet in large groups and is ambivalent about his newfound celebrity. Recently, with invitations to
Davos and the Grammys, he traded in his uniform of jeans, a bird T-shirt and a hoodie for a suit --
only to lose his luggage on the flight to Switzerland.

Last year, when his wife, Sara Morishige Williams, went into labor and wrote about it on Twitter,
CNN published the news and her photo while she was still in the hospital. But Mr.~Williams rarely
posts personal messages to the 1.3 million people who follow him.

His fingers move constantly while he talks, whether fiddling with his keychain or shredding
toothpicks at a bar, and staff members give him a drink and an espresso to loosen him up before big
public appearances.

``Often there will be a room with five people having a conversation and he says the least, but when
he does talk, everyone listens intently, and it's a gem,'' says Mr.~Kaplan, his friend.

In business, that trait can be beneficial. In 2008, Facebook tried to buy Twitter, and financiers
asked Mr.~Williams if he wanted to sell. He said he wanted to sleep on it, and the next day sent
them a long e-mail about why he wanted Twitter to stay independent.

``He's got this ability to be patient in this very productive way,'' says Bijan Sabet, who is on
Twitter's board and is a partner at Spark Capital, which invested in the company. ``It was not just
this flip e-mail but very thoughtful -- what we could accomplish by when, why there's still so much
we have left to do. It was pretty inspiring.''

But others say Mr.~Williams's methodical approach can get in the way. ``Ev is very difficult to work
with because he has a tough time making a final decision on products,'' says the C.E.O. of a Silicon
Valley social networking company who requested anonymity because the company works with Twitter.
``This all changed when Dick took over. He's very logical and knows how to make things happen.''

From Mr.~Williams's point of view, his division of labor with Mr.~Costolo, a wiry and restless
counterpoint to Mr.~Williams's reserve, is a sign of success. After failing early on to work with
others, Mr.~Williams says he has figured out how to be part of a team.

``Dick is hard-charging and very focused on urgency and executing now, and I tend to be very
contemplative,'' he says. ``My weakness is probably taking too long to make a decision, and his is
being too hasty.''

THE cumbersome details of running Twitter now fall to Mr.~Costolo. He says his biggest challenge is
ensuring that in other countries, like Japan, South Korea and Brazil, where Twitter is growing by
leaps and bounds, the company avoids the managerial mistakes it made in the United States.

That means marketing Twitter as an information network, not a social one, from the get-go; buying
enough computing power; and hiring people to sell ads in those countries. Twitter also has to prove
that it can build an advertising business in the United States.

The company, meanwhile, is trying to avoid the bureaucracy that plagues larger businesses. The topic
is important to Mr.~Williams, who says he started companies because he didn't believe in aligning
himself with institutions.

Twitter's executives talk about the ``Dunbar number'' -- the maximum number of people, generally
believed to be 150, with whom one person can have strong relationships. This effort, mind you, comes
from a company with a business model that fosters a multitude of ever-growing -- and largely
glancing -- interactions among Twitter's users.

``I've never seen a company so focused on avoiding the Dunbar number,'' says Adam Bain, who recently
joined Twitter from the News Corporation as head of global revenue. ``You can tell Ev planned it
out.''

Each time employees log on to their computers, for instance, they see a photo of a colleague, with
clues and a list of the person's hobbies, and must identify the person. And notes from every meeting
are posted for all employees to read.

Speaking to a group of new hires at an orientation session last spring, Mr.~Williams said Twitter
had three goals: to change the world, to build a business and to have fun.

``You can succeed by only building a business, and many companies do,'' he said. ``We won't consider
it success unless it's all three.''

\section{How Immigrants Create More Jobs}

\lettrine{I}{n}\mycalendar{Oct.'10}{31} the campaign season now drawing to a close, immigration and
globalization have often been described as economic threats. The truth, however, is more complex.

Over all, it turns out that the continuing arrival of immigrants to American shores is encouraging
business activity here, thereby producing more jobs, according to a new study. Its authors argue
that the easier it is to find cheap immigrant labor at home, the less likely that production will
relocate offshore.

The study, ``Immigration, Offshoring and American Jobs,'' was written by two economics professors --
Gianmarco I.~P. Ottaviano of Bocconi University in Italy and Giovanni Peri of the University of
California, Davis -- along with Greg C.~Wright, a Ph.D. candidate at Davis.

The study notes that when companies move production offshore, they pull away not only low-wage jobs
but also many related jobs, which can include high-skilled managers, tech repairmen and others. But
hiring immigrants even for low-wage jobs helps keep many kinds of jobs in the United States, the
authors say. In fact, when immigration is rising as a share of employment in an economic sector,
offshoring tends to be falling, and vice versa, the study found.

In other words, immigrants may be competing more with offshored workers than with other laborers in
America.

American economic sectors with much exposure to immigration fared better in employment growth than
more insulated sectors, even for low-skilled labor, the authors found. It's hard to prove cause and
effect in these studies, or to measure all relevant variables precisely, but at the very least, the
evidence in this study doesn't offer much support for the popular bias against immigration, and
globalization more generally.

We see the job-creating benefits of trade and immigration every day, even if we don't always
recognize them. As other papers by Professor Peri have shown, low-skilled immigrants usually fill
gaps in American labor markets and generally enhance domestic business prospects rather than destroy
jobs; this occurs because of an important phenomenon, the presence of what are known as
``complementary'' workers, namely those who add value to the work of others. An immigrant will often
take a job as a construction worker, a drywall installer or a taxi driver, for example, while a
native-born worker may end up being promoted to supervisor. And as immigrants succeed here, they
help the United States develop strong business and social networks with the rest of the world,
making it easier for us to do business with India, Brazil and most other countries, again creating
more jobs.

For all the talk of the dangers of offshoring, there is a related trend that we might call
in-shoring. Dell or Apple computers may be assembled overseas, for example, but those products aid
many American businesses at home and allow them to expand here. A cheap call center in India can
encourage a company to open up more branches to sell its products in the United States.

Those are further examples of how some laborers can complement others; it's not all about one group
of people taking jobs from another. Job creation and destruction are so intertwined that, over all,
the authors find no statistically verifiable connection between offshoring and net creation of
American jobs.

We're all worried about unemployment, but the problem is usually rooted in macroeconomic conditions,
not in immigration or offshoring. (According to a Pew study, the number of illegal immigrants from
the Caribbean and Latin America fell 22 percent from 2007 to 2009; their departure has not had much
effect on the weak United States job market.) Remember, too, that each immigrant consumes products
sold here, therefore also helping to create jobs.

When it comes to immigration, positive-sum thinking is too often absent in public discourse these
days. Debates on immigration and labor markets reflect some common human cognitive failings --
namely, that we are quicker to vilify groups of different ``others'' than we are to blame impersonal
forces.

Consider the fears that foreign competition, offshoring and immigration have destroyed large numbers
of American jobs. In reality, more workers have probably been displaced by machines -- as happens
every time computer software eliminates a task formerly performed by a clerical worker. Yet we know
that machines and computers do the economy far more good than harm and that they create more jobs
than they destroy.

Nonetheless, we find it hard to transfer this attitude to our dealings with immigrants, no matter
how logically similar ``cost-saving machines'' and ``cost-saving foreign labor'' may be in their
economic effects. Similarly, tariffs or other protectionist measures aimed at foreign nations have a
certain populist appeal, even though their economic effects may be roughly the same as those caused
by a natural disaster that closes shipping lanes or chokes off a domestic harbor.

AS a nation, we spend far too much time and energy worrying about foreigners. We also end up with
more combative international relations with our economic partners, like Mexico and China, than
reason can justify. In turn, they are more economically suspicious of us than they ought to be,
which cements a negative dynamic into place.

The current skepticism has deadlocked prospects for immigration reform, even though no one is
particularly happy with the status quo. Against that trend, we should be looking to immigration as a
creative force in our economic favor. Allowing in more immigrants, skilled and unskilled, wouldn't
just create jobs. It could increase tax revenue, help finance Social Security, bring new home buyers
and improve the business environment.

The world economy will most likely grow more open, and we should be prepared to compete. That means
recognizing the benefits -- including the employment benefits -- that immigrants bring to this
country.

\section{When the Assembly Line Moves Online}

\lettrine{D}{o}\mycalendar{Oct.'10}{31} one assigned task on your computer. It shouldn't take you
more than two seconds. Repeat 14,399 times. Congratulations! Your eight-hour work day is complete.

No such workplace yet exists, but with the fiendishly clever creation of standardized two-second
tasks, delivered to any computer connected to the Internet, it is now technically possible to set
up.

Microtask, a start-up company in Finland, has come up with the software that delivers such tasks.
The company offers to take on ``dull, repetitive work'' -- like digitizing paper forms or business
cards -- for prospective clients. As it says in a video on its Web site, ``Microtask loves the work
you hate.''

Microtask is in a position to love that work because not one of its 12 employees actually performs
it. Its software carves a given task into microscopically small pieces, like transcribing a
handwritten four-digit number in a tiny rectangle on a form. (Handwritten numbers and letters are
the bane of text-recognition software.) These tasks, stripped of identifying information about the
client or the larger task, can then be distributed online anywhere.

The approach shows how the online concept of widely distributed work has evolved since it was
pioneered by the Mechanical Turk service, introduced by Amazon.com in 2005. Mechanical Turk
resembles an online bulletin board. Businesses post income-earning opportunities, with rewards for
each task completed. Turkers, as the independent contractors are informally called, choose a task
they like and are qualified for. Recent offers included 2 cents each for finding the contact
information for 7,500 hotels and 3 cents each for answering questions about 9,400 toys.

Miriam Cherry, an associate professor of law at the University of the Pacific, tried Mechanical Turk
and says she found out for herself that the compensation was meager. ``My assistant and I tried but
we couldn't make minimum wage,'' says Professor Cherry, who presented an argument last year in the
Alabama Law Review for extending minimum-wage laws into cyberspace.

Kay Kinton, a spokeswoman for Amazon, countered that a client using Mechanical Turk might pay 50
cents for transcription of a one-minute audio clip, a rate that ``can add up quickly.''

Turkers can choose their tasks and when they want to work. Microtaskers, however, will have no say
about what tasks come flowing in. They will be full-time employees of other businesses, such as a
Finnish insurer that has put employees in an office in the Baltics to work on digitizing the
company's paper forms, using Microtask software. Microtask will keep workers focused on a single
screen, supplying everything needed to complete the task, without having to surf the Web for
additional information. That's why it can assume that one of its microtasks can be completed within
two seconds.

It's easy to see how transcribing one field on a form wouldn't take any longer than that. But how
could anyone manage to perform that same task thousands of times in quick succession?

``The grand vision is to have many kinds of different tasks,'' says Ville Miettinen, Microtask's
chief executive. ``For example, you'd do five minutes of text recognition work, followed by a few
minutes of speech transcription, and then a few minutes of comparing pairs of product images to
determine whether the two photographs depict the same product -- machines have a hard time figuring
this out.''

Such variety can't yet be offered, Mr.~Miettinen says, because the service is too new; it is only
now landing its first paying clients. Microtask is negotiating with call centers to use their
employees to do its work during lulls, he says.

CloudCrowd, based in San Francisco, also offers to distribute clients' work online. Like Microtask,
it has found ways to break work into thin slices.

``Rather than crowdsourcing, we call what we do widesourcing,'' says Mark Chatow, the company's vice
president for marketing. ``We take tasks like translation that used to be done by a single
specialist and break them into pieces so a wide range of people can handle different parts of the
work.''

CloudCrowd uses machine translation software to make a first pass. Then it sends out individual
pages of the machine's translation to garble hunters, who look for sentences containing a
nonsensical sequence. A translator with native language fluency is needed only for the sentences
tagged by the garble hunter. An editor, without foreign language expertise, then polishes the prose,
but possesses only a single page, not a chapter or the entire work.

CloudCrowd exclusively uses Facebook members who come to it for assignments; it says it has 50,000
workers in its crowd. Traditional translation costs about 20 to 25 cents a word, Mr.~Chatow, says,
but ``we're doing it for 6.7 cents a word.'' He says translators make an average of \$15 an hour and
garble hunters around \$7 an hour.

Mr.~Miettinen of Microtask says, ``Pure monetary compensation is a 20th-century concept.'' He
envisions tapping the talents of game designers who would render clickwork fun, what he calls
``game-ification.'' If successful, it could minimize complaints about pitiful pay or soul-draining
boredom.

PROFESSOR CHERRY is wary of this vision. She has studied Chinese ``gold farmers,'' who play games
for long hours on behalf of first-world gamers to earn virtual currency. ``What makes a game a game
is being able to control where and for how long one plays,'' she says. To those Chinese
gamers-for-hire, ``games are work.''

Worker control is precisely what the Microtask model has engineered out -- that's the source of its
insidious efficiency. Just as Ford's assembly lines a century ago brought work to workers who
performed a single, repetitive task, Microtask's software, via the Internet, does the same.

Every two seconds.

\section{Whimsy (and Clothes) for Sale}

\lettrine{T}{o}\mycalendar{Oct.'10}{31} understand the thinking behind Chris Lindland's company,
Betabrand, you need to keep three seemingly disparate ideas in your head at the same time: 1) It's a
challenge for Web-only businesses to sell clothing. 2) Most people want to be witty. 3) Some
shoppers go crazy for limited-edition goods. (Think Beanie Babies.)

According to Mr.~Lindland, his interpretation of these truisms is crucial to his company's success.
And because he's a lot of fun, it's worth hearing him out. After all, he's the guy who invented
Cordarounds, which are horizontal corduroy pants (and horizontal seersucker, for summer). They cost
\$90 a pair.

Mr.~Lindland, 38, is also the man who, along with his business partner Enrique Landa, created a
reversible corduroy-brocade smoking jacket -- business wear by day, Hugh Hefner by night -- that
goes for \$195, and a \$100 pair of trousers, called Disco Pants, that are made of fabric that
disperses light. Once, while in Ireland for a wedding, Mr.~Lindland found himself sussing out the
possibility of making sweaters from the wool of actual black sheep. The \$120 Black Sheep Sweater,
which comes by its color naturally, sans dye, was born.

Betabrand, based in San Francisco, started in August as an expansion of the previous five-year-old
company, also called Cordarounds, and is on track to top \$1 million in sales this year,
Mr.~Lindland says. The reason, he asserts, stems from those three fundamental truths: Betabrand
employs No.~2 (our desire to be funny -- or at least original), to trump No.~1 (our reluctance to
buy something we can't examine up close). Then the company seals the deal by exploiting No.~3.
(Because its products are made only in batches of a few hundred, you'll miss them if you don't
hurry.)

``We could never afford to make product in volume, so we adopted kind of like a Beanie Baby
approach: we'd create small collections that supremely rabid buyers would end up buying,''
Mr.~Lindland said, noting that some customers own more than 20 pairs of his signature pants.
``They're a collectors' item, oddly enough.''

It is odd, the choice to embrace horizontal stripes on a garment that covers the area of the body
that many people see as their widest asset. But this is where Mr.~Lindland's sense of humor
triumphs. Whimsy, you see, is not just the engine that drives the design of Betabrand products. It
is the force that fuels what might be called the mystique of Betabrand.com -- and of its progenitor,
Cordarounds.com. To don a Betabrand garment is to self-identify as a member of a sort of meta club
-- a group of people who understand irony, who delight in difference and who are eager to simply be
contrary.

Mr.~Lindland, who says his goal is ``not to try to create the coolest, most cohesive line of
clothing, but to create the most conversation-worthy line of clothing,'' strives to keep Betabrand
stocked with products that will generate ``new, fun, cult-y talk.'' Each week, the company
introduces a new item in limited quantities. But this approach isn't merely a lark, he says -- it is
a necessary strategy for an Internet retailer that needs to break through the clutter.

Pull up the Betabrand Web site and you will find plenty of intentionally goofy pseudo-science hyping
its wares. Cordarounds are said to be ``the quietest cords in the universe,'' while a handy diagram
tries to prove how they ``drastically lower your crotch heat index.'' Buy a product and you will
receive regular e-mail newsletters, written by Mr.~Lindland, that explain his yearning to ``make a
splash by subverting that which is vertical,'' as one stated. ``This includes our genetically
modified zebra breeding program and our recent `Say No to Longitude' P.S.A. campaign.''

``I started writing people on a weekly basis just to give them new jokes to tell about their pants
or jackets -- because we really didn't have anything else to sell,'' Mr.~Lindland said, remembering
the early days. ``Our philosophy for the newsletters -- which get a 40 percent open rate, which is
very high in the e-commerce world -- is they're 99 percent fiction and 1 percent fashion.''

``Do you know His Majesty, Haji Hassanal Bolkiah Muizzaddin Waddaulah, the Sultan of Brunei?'' one
typical newsletter implored. ``If so, it please forward on this message: Your Excellency: It has
been 13 insufferable years since Bill Gates surpassed you as the world's wealthiest man'' -- a time
when wealth came to be defined ``by stock options and sensible V-neck sweaters.'' To restore the
sultan's pre-eminence, the newsletter offers to give him a pair of horizontal corduroys valued,
conservatively, at \$983 million, adding that they were ``were washed in a bath of babies' tears,
warmed by a still-smoldering meteorite.''

BUT by far the wackiest promotional gambit that Betabrands has undertaken is its Model Citizen
campaign, started this month. Customers send photographs of themselves wearing Betabrand clothing to
the company, which in turn sends them a link they can forward to friends that puts those photos
front and center on what looks like the Betabrand home page. (In fact, it's a personalized version
made just for them.) Mr.~Lindland has long been a fan of using customer-generated photos on the
site; now the campaign lets anyone become the lead model.

``This allows us to push that community idea to a new level,'' he says, acknowledging that he also
hopes this camaraderie bolsters sales of items like the limited-edition Panther Suit -- black velvet
on one side, gold disco ball fabric on the other (jacket \$250, pants \$100, or \$325 for both).
``Our customers are starting to have this visual communication going on through funnier, more
interesting fashion photos that have a decidedly do-it-yourself quality to them.''

Who's buying this stuff? Bicyclists -- and, perhaps just as important, hundreds of biking blogs --
have adopted Betabrand's Bike to Work pants (khaki on the outside, with reflective detailing, and
priced at \$90). As for the Disco Pants, which Mr.~Lindland calls ``a nonstop sellout item for us,''
Betabrand has begun to detect ``a sort of an unholy alliance of golfers and Burning Man people who
have, like, officially made them their gear.''

``Never before,'' he says, ``have those communities been aligned over anything.''

\section{Obama Walks Fine Political Line on Terror Threat}

\lettrine{T}{rying}\mycalendar{Oct.'10}{31} to manage a terrorism threat in the middle of an
election campaign, the Obama administration is walking a political and national security tightrope.

Remembering the debates over whether President George W.~Bush sought to capitalize on the terrorism
threat in the days before the 2006 election, White House officials do not want to look as if they
are seizing on a potential catastrophe to win votes. But at the same time, they remember when
President Obama was criticized when he said nothing publicly in the three days after an attempt to
blow up an airliner last Dec.~25.

``Every president has to be able to take off the partisan hat and assume the role of nonpartisan
commander in chief when there is a security incident,'' said C.~Stewart Verdery Jr., a former
assistant secretary of homeland security under Mr.~Bush. ``The president should be the public face
of the response to send the right signals to Americans worried about our defenses, especially those
partisans who might be inclined to find fault with anything the administration does.''

David Rothkopf, a national security expert who worked in the Clinton administration, said the
president took the right action in making a swift public statement. ``For Obama to wait longer, with
real devices on their way to real places -- he would have been open to criticism,'' Mr.~Rothkopf
said. ``I think the response was measured. It was not designed to produce panic. I also think,
frankly, that the way it was done was quite tempered compared to past statements that we've seen
during an election season.''

White House officials say politics has nothing to do with the quick response, and that the scene of
fighter jets accompanying a passenger plane from Dubai as it landed in New York on Friday was the
result of an ``abundance of caution.''

``This has been handled just like any credible threat on any day,'' Robert Gibbs, the White House
press secretary, said Friday in an interview. Responding to a question from a reporter earlier in
the day about whether the administration might be hyping the threat to sway the elections, Mr.~Gibbs
said that Mr.~Obama's top counterterrorism adviser, John O.~Brennan, briefed the president initially
on Thursday night ``off of very credible terror information.''

He said that the discovery afterward of explosives on two cargo planes bound for the United States
should ``put to rest any speculation that may be out there.''

After getting slammed for what critics complained was a slow public response to the Dec.~25 plot,
administration officials, four days before the midterm elections, appeared determined not to make
that mistake this time. Mr.~Brennan said that while there were ``similarities'' between the
interception of the explosives on Friday and the attempted bombing on Dec.~25, the two episodes were
also ``very different because you're dealing with two packages as opposed to an individual.''

The White House reaction contrasted with its response to the December scare, when a Nigerian aided
by the Yemeni branch of Al Qaeda tried to blow up an airliner over Detroit. While Mr.~Obama was
briefed repeatedly during a vacation in Hawaii, he did not address the episode in public until three
days later, drawing criticism for not seeming to take it seriously enough.

This time, not only did Mr.~Obama make a public statement within hours of the news breaking, but his
staff also made sure that influential Republicans, like Representative Peter T.~King of New York,
were kept informed. Mr.~King, who was among the Republican lawmakers who expressed their
dissatisfaction last year with the information they received about the December attempt, offered a
more favorable initial reaction on Friday.

``So far, everything has worked the right way,'' he said.

But some outside experts said it was risky for a president to come out as quickly as he did before
all the facts were known. ``You're trying to look presidential and in command of all the facts and
not look impotent,'' said James Jay Carafano, a homeland security expert at the conservative
Heritage Foundation. ``But on the other hand, you don't want to step in it and do something stupid.
Quite honestly, I don't know why they had a press conference.''

Moreover, Mr.~Carafano said that Mr.~Obama failed to use his remarks on Friday to justify the troop
escalation in Afghanistan in an effort to keep the country from becoming a haven again for Al Qaeda.
``The president missed the opportunity to say, `And this is why we're in Afghanistan,' ''
Mr.~Carafano said.

But in many ways, it is Yemen, and not Afghanistan, that is increasingly being viewed as a bigger
potential terrorist threat to the United States. One senior White House official noted Friday that
the discovery of the explosives was the third terrorist attempt in less than two years that appeared
to have a connection to Yemen.

American officials say that Anwar al-Awlaki, an American-born radical cleric now hiding in Yemen,
played a direct role in the December airliner plot, and he has publicly called for more attacks on
the United States. In addition, an Army psychiatrist charged with killing 13 people at Fort Hood,
Tex., a year ago had exchanged e-mails with Mr.~Awlaki beforehand.

Juan Zarate, who was Mr.~Bush's deputy national security adviser for counterterrorism, said the
interception of the explosives was different than the December plot, which stemmed from an
intelligence and security breakdown and which challenged the administration's response.

``The administration has clearly learned the lessons that it is essential that the president and his
team demonstrate that they are taking seriously the threat and allowing the CT professionals to do
their work,'' said Mr.~Zarate, using the initials for counterterrorism. ``My only concern is that we
not overreact publicly at the highest levels every time there is a terrorist incident.''

He added that the president should not feel compelled to jump every time Al Qaeda ``says boo.''

\section{At Rally, Thousands -- Billions? -- Respond}

\lettrine{W}{ashington}\mycalendar{Oct.'10}{31} --Part circus, part satire, part holiday parade, the
crowds that flooded the National Mall for Jon Stewart and Stephen Colbert's ``Rally to Restore
Sanity and/or Fear'' on Saturday made it a political event like no other.

It was a Democratic rally without a Democratic politician, featuring instead two political
satirists, Mr.~Stewart and Mr.~Colbert, who used the stage to rib national journalists and
fear-mongering politicians, and to fake-fight each other over dueling songs about trains.

Though at no point during the show did either man plug a political candidate, a strong current of
political engagement coursed through the enormous crowd, which stretched several long blocks west of
the Capitol, an overwhelming response to a call by Mr.~Stewart on his show on the Comedy Central
network. The turnout clogged traffic, and filled subways and buses to the point of overflow.

The National Park Service did not offer a crowd estimate. But in an e-mail, one Comedy Central
executive joked that the size of the crowd was ``30 or 40 million.'' And before the event,
Mr.~Colbert offered his own guess in a Twitter message: ``Early estimate of crowd size at rally: 6
billion.''

The event, sponsored by Comedy Central, was viewed by many in the crowd as a counterweight to the
Rally to Restore Honor'' led by the Fox News host Glenn Beck near the Lincoln Memorial two months
ago. Some participants holding anti-Fox and anti-Beck signs staged a protest near a Fox News
satellite truck.

For many who came, the rally was an opportunity to take control of the political narrative, if only
for one sunny Saturday afternoon. Participants, overwhelmingly liberal, wore political buttons,
waved flags and carried signs, often with funny messages.

A cluster of women in their 50s held small white signs that read ``Shrinks for sanity.'' A man in a
fleece jacket held a sign that said, ``I can see the real America from my house.''

Others signs were absurdist. Daniel Short, 29, a visual effects artist from New York City, said the
message of his sign, taped to a cardboard tube -- ``Tights are not pants'' -- was ``just another one
of those silent majority issues.''

Political protests also came in costume. One man wore only a diaper and a sombrero, and carried a
large wooden anchor -- a depiction of an ``anchor baby,'' the name conservative talk show hosts have
given to children born in the United States to immigrant parents. ``I'm feeling a little exposed,''
he said, shortly after the rally finished.

But beyond the goofiness, the rally seemed to be channeling something deep -- a craving to be heard
and a frustration with the lack of leadership, less by President Obama than by a Democratic Party
that many participants described as timid, fearful, and failing to stand up for what they see as the
president's accomplishments.

``I'm proud of Obama, but the Democrats in Congress, they're just running for cover,'' said Ron
Harris, a lawyer from Laguna Beach, Calif., who came to celebrate his 64th birthday. ``They couldn't
sell bread to a starving mother if God was standing next to them.''

Some in the crowd expressed regret that it was comedians, not politicians, who were able to channel
their frustration.

``We don't have any place to turn,'' said Michelle Sabol, 41, a jewelry designer from Pittsburgh.
Mr.~Stewart, she said, gave voice to her feeling of frustration and isolation.

Four friends, dressed as giant tea bags in a spoof of the Tea Party, said Mr.~Stewart and
Mr.~Colbert were the only ones they felt expressed their point of view.

``For everything that's happened in the past two years, ``The Daily Show'' is how we cope,'' said
one of the tea bags, a 40-year-old from Anchorage named S.J.~Klein.

``The battle for the American mind right now is between talk show hosts and comedians,'' said Alex
Foxworthy, a 26-year-old doctoral student from Richmond, Va. ``I choose the comedians.''

Mr.~Colbert plays an arch-conservative on his show, and at the rally he played a victim of fear who
was gradually shown the light by Mr.Stewart.

``Your reasonableness is poisoning my fear,'' Mr.~Colbert said in mock exasperation.

Cable news channels dipped in and out of coverage of the rally, while Comedy Central and C-SPAN
broadcast it live in its entirety.

Mark St.~John, a 57-year-old from Indianapolis, came on a bus he had rented with more than 50 of his
friends. He wanted to express his irritation with what he felt was double-speak and
misrepresentation by conservative pundits about Mr.~Obama's initiatives.

``I'm tired of the silliness,'' he said, squeezed into a coffee shop near the rally. ``Health care
as a government takeover? That's just not what it was.''

And while few interviewed in the crowd seemed to find common ground with people on the other side of
the political divide, often referring to those with different political views as ``them,''
Mr.~St.~John socialized with many people he disagreed with politically.

``They're sane, and they're not bigots,'' he said. ``But when you feel economically threatened, you
make certain decisions.''

Some were actively canvassing. Near the National Gallery of Art, Carol Newmyer, 55, was handing out
bumper stickers that read ``Give Change a Chance.'' She planned to go to a Democratic phone bank
later in the afternoon to help with get-out-the-vote efforts.

``I want to see the heart and soul of the Democrats that elected Obama come out and vote again,''
she said.

For his part, Mr.~Stewart struck a serious note in his finale speech at the rally, making an appeal
for bipartisanship.

``We know instinctively as a people that if we are to get through the darkness and back into the
light we have to work together,'' he said.``And sometimes the light at the end of the tunnel isn't
the promised land. Sometimes it's just New Jersey. But we do it anyway, together.''

\section{Innovation: It Isn't a Matter of Left or Right}

\lettrine{A}{uthors}\mycalendar{Oct.'10}{31} quickly find a certain predictability to many of the
questions they encounter on a book tour. But a few weeks ago, during the second stop on the tour for
my new book, I found myself being interviewed in front of a Seattle audience and responding to an
opening question that I had never been asked before: ``Are you a Communist?''

The question was intended as a joke, but like the best jokes, it played on the edges of an important
and uncomfortable truth. I had just spent four years writing a book about the innovative power of
open systems that work outside of or parallel to traditional market environments: the amateur
scientists of the Enlightenment, university research labs, open source software platforms.

In my research, I analyzed 300 of the most influential innovations in science, commerce and
technology -- from the discovery of vacuums to the vacuum tube to the vacuum cleaner -- and put the
innovators of each breakthrough into one of four quadrants. First, there is the classic solo
entrepreneur, protecting innovations in order to benefit from them financially; then the amateur
individual, exploring and inventing for the love of it. Then there are the private corporations
collaborating on ideas while simultaneously competing with one another. And then there is what I
call the ``fourth quadrant'': the space of collaborative, nonproprietary innovation, exemplified in
recent years by the Internet and the Web, two groundbreaking innovations not owned by anyone.

The conventional wisdom, of course, is that market forces drive innovation, with businesses
propelled to new ideas by the promise of financial reward. And yet even in the heyday of industrial
and consumer capitalism over the last two centuries, the fourth quadrant turns out to have generated
more world-changing ideas than the competitive sphere of the marketplace. Batteries, bifocals,
neonatal incubators, birth control pills -- all originated either in amateur labs or in academic
environments.

Now-ubiquitous technology like GPS was created by public-sector agencies for its original military
use. And most of the building-block innovations that make GPS possible -- satellites themselves, or
the atomic clocks that let them coordinate their signals so precisely -- were first conceived in
nonmarket environments.

The fourth quadrant, however, is not locked in a zero-sum conflict with markets. As in the case of
GPS, this fourth space creates new platforms, which then support commercial ventures.

In the next decade, we will most likely see a wave of profitable medical products enabled by genomic
science. But that underlying scientific platform -- most importantly, the ability to sequence and
map DNA -- was almost entirely developed by a decentralized group of academic scientists working
outside the commercial sector in the 1960s and '70s.

The Internet is the ultimate example of how fourth-quadrant innovation actually supports market
developments: a platform built by a loosely affiliated group of public-sector and university
visionaries that has become one of the most powerful engines of wealth creation in modern times.

Why has the fourth quadrant been so innovative, despite the lack of traditional economic rewards?
The answer, I believe, has to do with the increased connectivity that comes from these open
environments. Ideas are free to flow from mind to mind, and to be refined and modified without
complex business development deals or patent lawyers. The incentives for innovation are lower, but
so are the barriers.

The problem with the fourth quadrant is that it doesn't fit neatly into our conventional left/right
political categories. That's why my Seattle interviewer wanted to know whether I had Communist
leanings. Because these open systems operate outside the conventional incentives of capitalism, the
mind reflexively wants to put them on the side of socialism. And yet they are as far from the
command economies that Marx and Engels helped invent as they are from ``greed is good'' capitalism.

When we champion fourth-quadrant innovation, we are not arguing for top-down bureaucracies and
central planning. Stalin would have despised Wikipedia.

My ideas about the fourth quadrant are themselves the product of open collaboration, building on
ideas generated by thinkers like Lawrence Lessig, Yochai Benkler, Jonathan Zittrain, Lisa Gansky,
Tim O'Reilly, Clay Shirky, Cory Doctorow and many others. I can say with some confidence that not
one of these people is a Communist, either. But when you try to use the standard means of
categorizing us -- are we for unfettered markets, or government intervention? -- the answer suddenly
becomes very murky.

We are all interested in the crucial kinds of value that emerge outside of traditional market
economies and ownership structures. But many of us have advised or created for-profit companies. And
we spend very little time talking about -- or arguing for -- the benevolent wisdom of Big
Government.

Consider a recent start-up called Kickstarter, which embodies many of these complex values.
Kickstarter is a site that allows individuals to fund creative projects, like movies, art
installations, albums and so on. Donors may get special gifts in return for their contributions --
signed copies of the final CD or an invitation to the opening -- but they don't own the creations
they help support. In just two years of existence, Kickstarter has raised more than \$20 million for
thousands of projects, taking a small cut of each transaction.

The economic exchange that Kickstarter enables between donors and creators works outside the
traditional logic of markets. People are ``investing'' in others not for the promise of financial
reward, but for the social rewards of supporting important work. The artists, on the other hand, are
relying on a decentralized network of support, not government grants. And somehow, in the middle of
these new models of collaboration, lies Kickstarter itself, a for-profit company that may well make
a nice return for its own investors and founders.

So, no, I am not a Communist.

BUT the problem is that we don't have a word that does justice to those of us who believe in the
generative power of the fourth quadrant. My hope is that the blurriness is only temporary, the
strange disorientation one finds when new social and economic values are being formed.

The choice shouldn't be between decentralized markets and command-and-control states. Over these
last centuries, much of the history of innovation has lived in a less formal space between those two
regimes: in the grad seminar and the coffeehouse and the hobbyist's home lab and the digital
bulletin board. The wonders of modern life did not emerge exclusively from the proprietary clash
between private firms. They also emerged from open networks.

\section{IPad Opens World to a Disabled Boy}

\lettrine{O}{wen}\mycalendar{Oct.'10}{31} CAIN depends on a respirator and struggles to make even
the slightest movements -- he has had a debilitating motor-neuron disease since infancy.

Owen, 7, does not have the strength to maneuver a computer mouse, but when a nurse propped her
boyfriend's iPad within reach in June, he did something his mother had never seen before.

He aimed his left pointer finger at an icon on the screen, touched it -- just barely -- and opened
the application Gravitarium, which plays music as users create landscapes of stars on the screen.
Over the years, Owen's parents had tried several computerized communications contraptions to give
him an escape from his disability, but the iPad was the first that worked on the first try.

``We have spent all this time keeping him alive, and now we owe him more than that,'' said his
mother, Ellen Goldstein, a vice president at the Times Square Alliance business association. ``I see
his ability to communicate and to learn as a big part of that challenge -- not all of it, but a big
part of it. And so, that's my responsibility.''

Since its debut in April, the iPad has become a popular therapeutic tool for people with
disabilities of all kinds, though no one keeps track of how many are used this way, and studies are
just getting under way to test its effectiveness, which varies widely depending on diagnosis.

A speech pathologist at Walter Reed Army Medical Center uses text-to-speech applications to give
patients a voice. Christopher Bulger, a 16-year-old in Chicago who injured his spine in a car
accident, used an iPad to surf the Internet during the early stages of his rehabilitation, when his
hands were clenched into fists. ``It was nice because you progressed from the knuckle to the finger
to using more than one knuckle on the screen,'' he said.

Parents of autistic children are using applications to teach them basic skills, like brushing teeth
and communicating better.

For a mainstream technological device like the iPad to have been instantly embraced by the disabled
is unusual. It is far more common for items designed for disabled people to be adapted for general
use, like closed-captioning on televisions in gyms or GPS devices in cars that announce directions.
Also, most mainstream devices do not come with built-ins like the iPad's closed-captioning,
magnification and audible readout functions -- which were intended to keep it simple for all users,
but also help disabled people.

``Making things less complicated can actually make a lot of money,'' said Gregg C.~Vanderheiden, an
engineering professor at the University of Wisconsin at Madison who has worked on accessibility
issues for decades.

Representative Edward J.~Markey, a Massachusetts Democrat, who wrote recently enacted legislation
that will require mobile devices to be more accessible to users with disabilities, said
approximately three-fourths of communications and video devices need to be adapted for blind and
deaf people. ``Apple,'' he said in a statement, ``is an outlier when it comes to devices that are
accessible out of the box.''

The iPad is also, generally speaking, less expensive than computers and other gadgets specifically
designed to help disabled people speak, read or write. While insurers usually do not cover the cost
of mobile devices like the iPad because they are not medical equipment, in some cases they will pay
for the applications that run on them.

In Owen's case, his grandmother bought him a \$600 iPad in August, and his parents have invested
about \$200 more in software. One day this summer, his finger dangled over the title page of ``Alice
in Wonderland'' on his iPad while his mother hovered over his shoulder in their Brooklyn home. Then,
with the tiniest of movements, and thanks to the sensitivity of the iPad's touch screen, Owen began
to turn the pages of the book. ``You are reading a book on your own, Owen!'' Ms.~Goldstein, 44,
exclaimed. ``That is completely wonderful.''

But while the sensitivity of the iPad's touch screen makes it promising for Owen, it can be
problematic for others, like Glenda Watson Hyatt, a blogger in Surrey, British Columbia, who has
cerebral palsy. ``When `flipping' screens, sometimes I flip more than one screen,'' Ms.~Hyatt wrote
in an interview conducted by e-mail. ``Or I touch what I didn't intend to.''

Still, Ms.~Hyatt said that when she was having trouble chatting with friends at a bar recently, she
pulled out her iPad to help communicate and felt normal. ``People were drawn to it because it was a
`recognized' or `known' piece of technology,'' she wrote in a blog post reviewing the device.

At the Shepherd Center, a spinal cord rehabilitation clinic in Atlanta, some teenage quadriplegics
have received iPads as gifts, but they do not work well for those who rely on a mouse stick --
basically a long pen controlled by mouth.

``It wants to see a finger,'' said John Anschutz, the manager of the assistive technology program at
Shepherd. ``It really requires the quality of skin and body mass to operate.''

For Owen Cain, whose disease is physical, not mental, the iPad has limitations, too. Moving his
finger all the way across the keypad remains a challenge, and makes writing difficult. Ms.~Goldstein
said its versatility and affordability, though, were a boon. He has been experimenting with a
variety of applications -- Proloquo2Go, which allows him to touch an icon that prompts the device to
speak things like, ``I need to go to the bathroom''; Math Magic, which helps him practice
arithmetic; and Animal Match, a memory game.

``If all you're worrying about is `I can try this program, or I can try that program, I can buy that
app or I can buy this app,' and the investment is so much lower,'' his mother said, ``then our
ability to explore or experiment with different things is so much bigger.''

When Owen was about 8 weeks old, his mother noticed his right arm drooping. It led to a crushing
diagnosis: the motor-neuron disease known as spinal muscular atrophy Type 1. A 2003 New York Times
article about spinal muscular atrophy said his parents had been told Owen would be ``paralyzed for
his life, which doctors predicted would last no more than about two years.''

Owen will turn 8 on Nov.~11. While his condition is not expected to worsen, he is extremely
sensitive to infection and once nearly died of pneumonia; three specialized therapists and a nurse
help keep him alive.

Though he cannot speak, his parents have taught him to read, write and do math. He has an impish
sense of humor and a love of ``Star Wars.'' ``He's a normal child trapped in a not normal body,''
said his father, Hamilton Cain, 45, a book editor.

Since he received the iPad, Owen has been trying to read books, and playing around with apps like
Air Guitar. And, one day, he typed out on the keypad, ``I want to be Han Solo for Halloween.''

\section{Plenty of Hype -- and Questions -- About Electric Cars}

\lettrine{E}{lectric}\mycalendar{Nov.'10}{01} cars are coming! Electric cars are coming!

The hype is virtually inescapable. Nissan has already received more than 27,000 reservations around
the world for the Leaf, its all-electric car, which is to start arriving in the United States and
Japan in December and several European markets soon thereafter. General Motors will make 10,000 Volt
cars, its plug-in hybrid, next year; they will soon hit the streets of U.S.~cities like Austin,
Texas. Tesla, the California-based maker of a plug-in sports car called the Roadster, just opened
its first Asian showroom, in Tokyo.

But for ordinary people willing to swallow the high price tag (nearly \$33,000 for the Leaf in the
United States), plenty of questions about electric cars remain. Where can the cars be charged? What
happens if they need to be fixed? How long will a charge last?

The carmakers are eager to make their first customers happy, so they are bending over backwards to
help with the charging infrastructure and to allay ``range anxiety'' -- the fear of running out of
electricity and getting stranded, because all-electric cars cannot go as far as conventional ones.
(This is not a big factor for the Volt, whose gasoline engine will kick in after the batteries are
spent.)

Companies involved in the introduction of electric vehicles readily acknowledge that it is hard to
predict how people will use their new cars in practice.

``A lot of it is going to be learn as you go,'' said Arun Banskota, the president of electric
vehicle services for NRG, a power company working to put charging stations in the Houston area.

Many of the earliest adopters are likely to be urban homeowners with their own garages -- which is
no surprise, because it is this wealthier demographic that can afford electric cars. So companies
like NRG are preparing for a proliferation of home-based charging stations, with associated
electricity payment plans.

Apartment dwellers are likely to acquire electric cars later, experts say. Steve Smaha, an
Austin-based investor in clean technology who owns a Tesla Roadster, noted that while there were
likely to be some early showcase apartment buildings that installed outlet infrastructure, most
building owners were not going to be excited about paying for such a service.

They would also need to establish metering systems that would make sure the tenant was footing the
bill for the electricity.

Given the range limitations (Nissan's Leaf is built to go about 100 miles, or 160 kilometers, on one
charge), the earliest electric vehicle is also unlikely to be the only car in the family garage. For
longer road trips, people will probably keep a second vehicle, one with a conventional engine, on
hand. (This may be more true for Americans, who are accustomed to driving long distances, than
Europeans.)

``We think there's a big market there for a car that tends to stay fairly local and does 10-, 20-,
30-mile trips,'' said Simon Sproule, a Nissan spokesman.

Many cities, from Paris to Houston, are scrambling to install outlets that will allow people to
charge (or, most likely, top off their charges) at places like Starbucks or shopping centers. But
ultimately, analysts expect that most charging will be done at home, probably at night, with the
public infrastructure providing a means to alleviate range anxiety.

``We like to say that we're in the business of providing range confidence,'' said Mr.~Banskota of
NRG. Nonetheless, he said, he expected the public charging stations to get used.

Early owners of electric cars can expect plenty of perks. The most important are hefty tax
incentives that will reduce the price tags -- as much as \$7,500 for an electric vehicle from the
U.S.~government and as much as \textsterling5,000, or \$8,000, in Britain, to name but a few
examples. Some places may offer carpool-lane privileges.

Mr.~Smaha, the Roadster owner, says he can get a parking spot close to the door at the busy Whole
Foods grocery store in Austin, which has priority parking for electric cars. He can charge his
vehicle there, though he has never used more than 5 cents' worth of electricity.

Electric vehicles should require fewer repairs, car companies say, because their engines are less
complex. Tesla's cars, for example, have ``no routine oil changes, muffler or exhaust systems, spark
plugs, pistons, or other parts associated with the internal combustion engine,'' Rachel Konrad, a
spokeswoman for Tesla Motors Europe, said in an e-mail.

In addition, owners of electric cars will enjoy ``not having that smell of gasoline on your hands or
on your clothes,'' said Mr.~Banskota of NRG.

Of course, they will presumably derive even more enjoyment from having ``fuel'' that costs far less
than that for conventional vehicles -- with the environmental benefits that this implies.

Despite the amount of attention electric cars are garnering, it is important to bear in mind that
they will remain a small minority of all vehicles on the roads for the foreseeable future.

A study released last week by J.D. Power and Associates, a market research group based in
California, said that combined sales of hybrid vehicles and all-electric ones would total just 7.3
percent of the nearly 71 million passenger vehicles sold worldwide in 2020. Consumers will balk at
concerns about the cars' range and power and the speed of recharging, as well as the high price
tags, the report said.

Nissan has said that by 2020, 10 percent of vehicles sold globally will be electric.

But Mr.~Sproule, the spokesman, noted that despite the company's confidence, Nissan still made lots
of conventional vehicles and would rigorously familiarize potential Leaf customers with the
differences between electric and gasoline-powered cars.

``Electric vehicles are not going to be suitable for everybody,'' he said.

For those who can get past the high price tags and the range limitations, Mr.~Smaha, the Tesla
owner, points out another advantage of electric vehicles, especially sporty ones: driving can again
become a joy.

During his childhood, Mr.~Smaha said, his father would take the children out for car rides on
Saturday nights in Maine -- just for fun.

Now with the Tesla, he said, ``I can drive for pure guilt-free pleasure again.''

\section{Democrats Fight to Retain Senate Majority}

\lettrine{T}{he}\mycalendar{Nov.'10}{01} battle for control of Congress rolled into a frenetic final
weekend as Democrats fought to preserve the Senate as their power center on Capitol Hill, trying to
hold off a Republican surge that could reshape the political order in Washington.

With Republicans in a strong position to capture the House, President Obama on Saturday opened a
four-state weekend swing here to rally support for Senate candidates in Connecticut, Illinois, Ohio
and Pennsylvania, hoping to build a critical firewall to protect the party's Senate majority from
Republican gains across the country.

Republicans intensified their efforts to capitalize on a favorable political environment, with Sarah
Palin making a last-minute trip to West Virginia to ask voters to elect a Republican for the
Democratic seat Senator Robert C.~Byrd held for 51 years.

The outcome of five contests considered tossups will help determine if Democrats retain control of
the Senate, according to the latest analysis of races by The New York Times, with Republicans trying
to capture Democratic-held seats in Colorado, Illinois, Nevada, Pennsylvania and Washington. Should
they sweep those, they would still need to triumph in a state like California or West Virginia,
where Democratic chances seemed to be improving.

The analysis suggests that Republicans are on the cusp of significantly expanding their presence in
the Senate, but will need almost everything to go their way on Tuesday to gain the 10 seats needed
to win control.

A sour political environment has left almost no Democratic senator on the ballot immune to forceful
challenges by Republicans. Polling shows that Republicans have a firm grip on open Democratic seats
in North Dakota and Indiana. And Senators Russ Feingold of Wisconsin and Blanche Lincoln of Arkansas
have fallen behind in their races, strategists in both parties said, while Harry Reid of Nevada and
Patty Murray of Washington are trying to survive tough fights.

No contest in the country held a higher symbolic priority for Democrats or Republicans than that of
Mr.~Reid, the majority leader. He is deadlocked in a deeply personal multimillion-dollar battle with
Sharron Angle, who has become one of the most viable candidates to ascend from the Tea Party
movement.

The president made an urgent appeal to save the Democratic majority, including his own former Senate
seat in Illinois. He returned to Chicago on Saturday evening for a rally in the neighborhood where
his political career began, telling voters along the way that this election should not be about him.

``It is absolutely critical that you go out and vote,'' Mr.~Obama said here in Philadelphia. ``This
election is not just going to set the stage for the next two years. It's going to set the stage for
the next 10, the next 20.''

A new sense of uncertainty coursed through several other races, including the one in Alaska.
Democrats began running TV advertising on Saturday to try to take advantage of a chaotic situation
where Senator Lisa Murkowski, a Republican, is running an aggressive write-in campaign to try to
overcome her loss to Joe Miller in the primary.

The Democratic candidate, Scott T.~McAdams, seen as a long shot a few weeks ago, could have a chance
if Ms.~Murkowski and Mr.~Miller split the votes of Republicans and independents.

Ms.~Palin, the former governor of Alaska, who has been campaigning for Mr.~Miller and other
candidates, arrived in Charleston, W.Va., on Saturday afternoon to rally support for the Republican
Senate candidate, John Raese, whose early momentum appears to have waned in his race against
Gov.~Joe Manchin III, a Democrat.

While Republicans continued to expand the battleground in the House, growing increasingly confident
in their quest to win control of at least one chamber of Congress, Senate Republicans sought to
contain expectations even though they will end the year in a much stronger position than they could
have imagined two years after Mr.~Obama entered the White House on a wave of popularity and
optimism.

``Our hand will be strengthened, even if we're not in the majority in January,'' said Senator John
Cornyn of Texas, chairman of the National Republican Senatorial Committee. ``It's remarkable that
we've hung together as much as we have.''

Even as Election Day draws near and huge get-out-the-vote programs are waged on both sides, millions
of voters have already cast their ballots, a phenomenon that has substantially changed how American
elections are conducted. Democrats hoped the early-voting option in many states would help close the
enthusiasm gap with Republicans, because party organizers could all but drag their voters to the
polls over a span of weeks.

``It's like a dogfight in each and every one of those states,'' said Senator Robert Menendez of New
Jersey, chairman of the Democratic Senatorial Campaign Committee. ``We are poised to surprise people
on election night, but we are in a dogfight.''

In Philadelphia, Mr.~Obama's arrival was intended to draw attention to the Senate contest pitting
Representative Joe Sestak, a Democrat, against Pat Toomey, a Republican and former congressman. In
the last two weeks, Mr.~Sestak appears to have cut into Mr.~Toomey's lead. Even though Democrats
hold a statewide voter registration edge of 1.2 million people, party strategists worry that a wave
of protest against the party in power will propel Republicans.

At a crisp morning rally, Mr.~Sestak said it was time for voters to move beyond the anger at
Washington and focus on the substantive differences between the two parties. He urged Democrats to
think about the consequences of Republicans winning control of Congress, particularly in upholding
fair trade policies and trying to repeal the health care law.

``I think it will be very tight, but I do believe there is just enough time,'' Mr.~Sestak said.
``We're going to prevail because of the common sense of Pennsylvania.''

Mr.~Toomey worked the critical Philadelphia suburbs on Saturday, appealing to independent and
Republican voters, even as his campaign released a list of prominent Democrats who are supporting
his candidacy, a sign that disaffected Democratic voters are up for grabs.

In Colorado, Democrats were hoping Senator Michael Bennet, appointed to the Senate last year, could
win his close fight with Ken Buck, a Republican, and give Democrats some insurance in their push to
retain the majority and keep some of the territory the party won in the 2008 election in the
Mountain West.

``Colorado is an independent state, and I think we are seeing that in this election, as close as it
is,'' Mr.~Bennet said in an interview as he made a final tour of the state. ``We are going to need
Democratic votes, unaffiliated votes and Republican votes, and I think we can do it. We have run a
very strong campaign at a tough time.''

Republicans appear to be in no danger of losing their party's open Senate seats in Kentucky,
Missouri, New Hampshire and Ohio, though just last year it seemed Republicans would have difficulty
keeping all of them.

The difficult climate for Democrats and the strong Republican momentum around the nation has made
simply holding on to the Senate seem like a victory even though the party stands to lose a
significant chunk of its majority, some senior lawmakers and perhaps the majority leader.

When control of the House flipped in the 1994 and 2006 elections, power in the Senate changed hands
simultaneously as the Republican surge in the first case and the Democratic tide in the second
carried across the Rotunda. With Republicans in a commanding position to take the House but a Senate
majority still in doubt, that pattern could well be broken this time.

Republicans and Democrats cite several factors for the potential disparity between the House and
Senate races. One reason is that Republicans began the election cycle needing 11 seats to take the
majority, a number reduced to 10 after Scott Brown, a Republican, won a special election in
Massachusetts in January.

Republicans now have 41 senators and would need to win 10 seats to grab control, as Democrats would
hold the edge in a 50-50 Senate because of the tie-breaking ability of Vice President Joseph
R.~Biden Jr.~Even in a highly favorable year for Republicans, 10 seats is a tall order.

``We were at a low point after two losses in a row,'' said Senator Mitch McConnell of Kentucky, the
Republican leader. ``We had more ground to make up.''

\section{Arizona Immigration Law Divides Latinos, Too}

\lettrine{A}{rizona}\mycalendar{Nov.'10}{01}'s immigration law, which politicians have debated in
the Legislature, lawyers have sparred over in the courtroom and advocates have shouted about on the
street, has found its way up a driveway in central Phoenix, through the front door and right onto
the Sotelo family's kitchen table.

That is where Efrain Sotelo, 49, a process server, and his wife, Shayne, 46, an elementary school
teacher, sat and argued on a recent Friday night. He drank beer. She sipped wine. Like many
residents across the state, they differed on State Senate Bill 1070, as the immigration law is
known.

``I try not to engage in arguments with my wife,'' Mr.~Sotelo said, looking across at her. ``When we
talk about this, we both know what's going to happen. I can't convince her, and she can't convince
me.''

On that much, they seemed to agree. ``He's much more conservative than I am,'' she said. ``He can't
be changed. I can't be changed. We both know that. But that doesn't mean we don't try.''

That such a divisive social issue would divide some families is not surprising. But what makes the
Sotelos stand out is that they are both Latinos, he a Mexican immigrant who was born in the northern
state of Chihuahua and she a descendant of Spanish immigrants who grew up in Colorado.

While polls show that a vast majority of Latinos nationwide side with Mrs.~Sotelo in opposing
Arizona's law, that opposition is not uniform. ``All Latinos are not opposed to this law -- that's
too simplistic,'' said Cecilia Menjivar, an Arizona State University sociologist. There are other
Mr.~Sotelos out there, including an Arizona state legislator, Representative Steve B.~Montenegro, a
Republican who immigrated from El Salvador and became the only Latino lawmaker to vote in favor of
the bill.

Since Gov.~Jan Brewer signed the legislation in April, polls have found that about 70 percent of
Latinos nationwide oppose the law, which allows the police to arrest people they suspected of being
illegal immigrants, a provision blocked by a federal judge at the request of the Justice Department.

Mr.~Sotelo, who said he would be following Arizona's appeal of that ruling on Monday, realizes that
his views are not popular among immigrants. He was hesitant to reply when a woman in the country
illegally asked him recently what he thought of the law.

``I said, `I don't think you want to hear what I have to say,' '' he recalled.

He thinks his adopted state has been unfairly maligned since the law passed. ``I'm a Hispanic, and I
don't have any issues walking the streets,'' he said. ``They make it seem like the police or sheriff
are out there checking everyone's papers, and that's not so.''

But his wife, who was thrilled by President Obama's challenge to the law, has a different view of
her adopted state. ``It's more of a racist state after 1070,'' she said. ``We are a magnet for
neo-Nazis.''

Mr.~Sotelo, who came here in 1972 after his father obtained a green card, is unswayed by the fierce
objections of the Mexican government to the legislation. He still recalls bitterly when he was
stopped at a checkpoint inside Mexico with his two children some years ago. The official was
checking his paperwork and, Mr.~Sotelo said, clearly angling for a bribe. Mr.~Sotelo was so furious
at the encounter that he turned around and crossed back home to the United States.

Mr.~Sotelo does not believe that the bulk of illegal immigrants are criminals, as some advocates of
the law have argued. But some percentage of them are dealing in drugs, he said, and those
lawbreakers ``make the rest of us look bad.''

Mrs.~Sotelo scoffs at that. ``It's not a perfect world, and in all groups you'll have people who
abuse the system,'' she said. ``When you're dealing with individuals, there has to be flexibility.''

Because he serves summonses for a living, owning his own business, Mr.~Sotelo tends to be the
law-and-order type. Because she has taught the children of illegal immigrants and sees how
hard-edged policies affect real people, Mrs.~Sotelo tends to be more willing to give.

``As a teacher, when the law passed, I had kids crying,'' she said. ``They felt they had to uproot
themselves from the life they had known all their lives. I saw total pain. I couldn't believe it was
2010. It was almost as if I were living through the civil rights era again.''

Her husband shook his head.

Back and forth they went, with Mr.~Sotelo endorsing a get-tough approach to illegal immigrants and
his wife talking of the need for compassion.

``Phoenix is the No.~1 city in the country for kidnappings,'' he said at one point, citing the large
number of illegal immigrants who have been held against their will by smugglers.

``That's false,'' she shot back.

``O.K., but there's a lot of kidnappings in Phoenix,'' he said.

``You and I are not being targeted,'' she said.

``Right now, it may not be us,'' he said.

Her: ``I've never feared being kidnapped.''

Him: ``I do.''

Her: ``You can sincerely say you have a fear of being kidnapped?''

Him: ``I do. Who knows?''

And so they continued in what was for the most part a good-natured exchange, although with a third
party there to intervene when the debate heated up. There was no storming off, which is not always
the case, they said.

Trying to convince Mr.~Sotelo of the error of his ways can frustrate Mrs.~Sotelo. The same goes for
him, when he tries to convince his wife how wrong she is.

In one sense, Mrs.~Sotelo has come out on top. She has won over their two children with her
arguments, her husband said, leaving him with plenty of conservative company once he heads out the
front door but isolated in the confines of their home.

\section{Europe's Plagues Came From China, Study Finds}

\lettrine{T}{he}\mycalendar{Nov.'10}{01} great waves of plague that twice devastated Europe and
changed the course of history had their origins in China, a team of medical geneticists reported
Sunday, as did a third plague outbreak that struck less harmfully in the 19th century.

And in separate research, a team of biologists reported conclusively this month that the causative
agent of the most deadly plague, the Black Death, was the bacterium known as Yersinia pestis. This
agent had always been the favored cause, but a vigorous minority of biologists and historians have
argued the Black Death differed from modern cases of plague studied in India, and therefore must
have had a different cause.

The Black Death began in Europe in 1347 and carried off an estimated 30 percent or more of the
population of Europe. For centuries the epidemic continued to strike every 10 years or so, its last
major outbreak being the Great Plague of London from 1665 to 1666. The disease is spread by rats and
transmitted to people by fleas or, in some cases, directly by breathing.

One team of biologists, led by Barbara Bramanti of the Institut Pasteur in Paris and Stephanie
Haensch of Johannes Gutenberg University in Germany, analyzed ancient DNA and proteins from plague
pits, the mass burial grounds across Europe in which the dead were interred. Writing in the journal
PLoS Pathogens this month, they say their findings put beyond doubt that the Black Death was brought
about by Yersinia pestis.

Dr.~Bramanti's team was able to distinguish two strains of the Black Death plague bacterium, which
differ both from each other and from the three principal strains in the world today. They infer that
medieval Europe must have been invaded by two different sources of Yersinia pestis. One strain
reached the port of Marseilles on France's southern coast in 1347, spread rapidly across France and
by 1349 had reached Hereford, a busy English market town and pilgrimage center near the Welsh
border.

The strain of bacterium analyzed from the bones and teeth of a Hereford plague pit dug in 1349 is
identical to that from a plague pit of 1348 in southern France, suggesting a direct route of travel.
But a plague pit in the Dutch town of Bergen op Zoom has bacteria of a different strain, which the
researchers infer arrived from Norway.

The Black Death is the middle of three great waves of plague that have hit in historical times. The
first appeared in the 6th century during the reign of the Byzantine emperor Justinian, reaching his
capital, Constantinople, on grain ships from Egypt. The Justinian plague, as historians call it, is
thought to have killed perhaps half the population of Europe and to have eased the Arab takeover of
Byzantine provinces in the Near East and Africa.

The third great wave of plague began in China's Yunnan province in 1894, emerged in Hong Kong and
then spread via shipping routes throughout the world. It reached the United States through a plague
ship from Hong Kong that docked at Hawaii, where plague broke out in December 1899, and then San
Francisco, whose plague epidemic began in March 1900.

The three plague waves have now been tied together in common family tree by a team of medical
geneticists led by Mark Achtman of University College Cork in Ireland. By looking at genetic
variations in living strains of Yersinia pestis, Dr.~Achtman's team has reconstructed a family tree
of the bacterium. By counting the number of genetic changes, which clock up at a generally steady
rate, they have dated the branch points of the tree, which enables the major branches to be
correlated with historical events.

In the issue of Nature Genetics published online Sunday, they conclude that all three of the great
waves of plague originated from China, where the root of their tree is situated. Plague would have
reached Europe across the Silk Road, they say. An epidemic of plague that reached East Africa was
probably spread by the voyages of the Chinese admiral Zheng He who led a fleet of 300 ships to
Africa in 1409.

``What's exciting is that we are able to reconstruct the historical routes of bacterial disease over
centuries,'' Dr.~Achtman said.

Lester K.~Little, an expert on the Justinian plague at Smith College, said in an interview from
Bergamo, Italy, that the epidemic was first reported by the Byzantine historian Procopius in 541
A.D. from the ancient port of Pelusium, near Suez in Egypt. Historians had assumed it arrived there
from the Red Sea or Africa, but the Chinese origin now suggested by the geneticists is possible,
Dr.~Little said.

The geneticists' work is ``immensely impressive,'' Dr.~Little said, and adds a third leg to the
studies of plague by historians and by archaeologists.

The likely origin of the plague in China has nothing to do with its people or crowded cities,
Dr.~Achtman said. The bacterium has no interest in people, whom it slaughters by accident. Its
natural hosts are various species of rodent such as marmots and voles, which are found throughout
China.

\section{Proclaimed Dead, Web Is Showing New Life}

\lettrine{T}{wenty}\mycalendar{Nov.'10}{01} autumns ago, Tim Berners-Lee, a British computer
scientist, came up with a catchy name for a revolutionary project that aimed to open the Internet to
the masses. ``The World Wide Web,'' he called it, and the image proved to be so evocative that, for
many people, the Web has become synonymous with the Internet.

But now, two decades after Mr.~Berners-Lee had his brainstorm, some people are predicting the demise
of the Web. Even though the Web is merely one of many online applications, they add, this could be
the end of the Internet as we know it.

``The Web is dead,'' Wired magazine declared in a recent cover story. ``The golden age of the Web is
coming to an end,'' wrote Josh Bernoff, an analyst at Forrester Research. The Atlantic magazine
warned of ``the closing of the digital frontier.''

The argument goes something like this: After falling in love with the openness of the Web, consumers
are recoiling from its chaos and embracing the sense of order offered by walled-off digital realms.
These include applications for mobile devices like Apple's iPad and iPhone and password-protected
social networks like Facebook, where much of what people do takes place beyond the reach of search
engines and Web browsers.

Meanwhile, advocates of openness fear that telecommunications companies want to build separate,
Balkanized ``Internets'' of their own, where they control the content and collect tolls for traffic
that passes through them. Some media companies are already putting more of their content, once
freely available, behind pay walls, and lobbying governments to crack down on the free-for-all of
illegal file-sharing.

Jonathan Zittrain, a Harvard professor of Internet law, says that the growth of walled gardens like
Apple's applications store have threatened the ``generative'' character of the Internet, which has
permitted users to build on what is already there, as with Lego toys.

``The serendipity of outside tinkering that has marked that generative era gave us the Web, instant
messaging, peer-to-peer networking, Skype, Wikipedia -- all ideas out of left field,'' he writes in
a recent book, ``The Future of the Internet and How to Stop It.'' ``Now it is disappearing, leaving
a handful of new gatekeepers in place, with us and them prisoner to their limited business plans and
to regulators who fear things that are new and disruptive.''

Are matters really so dire? For the doomsayers, there are some inconvenient truths.

Every day, about a million new devices -- computers, mobile phones, televisions and other things --
are hooked up to the Internet, according to Rod Beckstrom, chief executive of the Internet
Corporation for Assigned Names and Numbers, which oversees the Internet address system. The total
number of Internet users worldwide, about two billion, is growing by 100 million to 200 million a
year.

Most of this growth is occurring in developing countries, where the Web is dominant and applications
stores and the like have made fewer inroads. The number of Web pages has grown from 26 million in
1998 to more than a trillion today, according to Google.

The Web has been better equipped to reach new corners of the world since the recent opening up of
the domain name system to non-Western languages. North America, which once dominated the Internet,
now represents only 13.5 percent of its users, according to Internet World Stats, a Web site that
compiles such data, compared with 42 percent for Asia and 24 percent for Europe.

``Reports of the death of the Web have been greatly exaggerated,'' Mr.~Beckstrom said. ``It's going
to be alive and kicking for a long time.''

While the Web is merely one of many applications that operate over the Internet, along with e-mail,
instant messaging, peer-to-peer file-sharing services and other tools, it is the most familiar one
for many people; almost anyone, anywhere with an Internet connection and a little bit of knowledge
can view a Web page.

So as other kinds of Internet traffic have started to grow more rapidly than Web use, some
open-Internet campaigners see a threat to the Web and, more generally, the Internet as we know it.
Yet the distinctions are growing less relevant. When you visit YouTube, for example, you are using
the Web to sort through the available videos, while the video stream is delivered outside the Web,
but still via the Internet.

Even if the supposed threats have been overblown, it is clear that the Web and the Internet are
changing.

Mobile devices increasingly come with Internet access as a standard feature. Within a few years,
analysts predict, more people will connect to the Internet from smartphones than from deskbound
computers.

The popularity of applications for smartphones, often with content or features similar to those
available on open Web sites, could steer more toward private digital gardens, like those that
existed in the heyday of online services like CompuServe and Prodigy.

``A walled garden is a place where everything looks beautiful, it works well, and there are flowers
everywhere,'' said Fr\'ed\'eric Donck, director of public policy at the Internet Society in
Brussels. ``But it's not the Internet.''

Why? Applications available for Apple devices are subject to approval by the company, which rejects,
among others, those that do not meet its guidelines for taste or decency. Some European newspapers
have had to censor racy photos to make their applications conform with Apple's rules, which prohibit
displays of bare female breasts.

``People don't think of their use of devices as a political act,'' said Mr.~Bernoff, the Forrester
analyst. ``They just think about whether they are having an elegant, seamless experience. But do I
really want Apple deciding what kind of content is appropriate?''

For Internet users in countries like China or Iran, the idea that there are limits to online freedom
is nothing new. There, governments routinely block access to Web sites that feature dissenting
political views.

Advocates of an open Internet worry that official oversight is on the rise elsewhere. In Australia,
the government has proposed a system through which the Internet would be filtered to block access to
sites containing child pornography or other material that is illegal or deemed to be highly
offensive.

Movie and music companies, meanwhile, have lobbied governments to crack down on digital freeloaders
who engage in unauthorized sharing of their content. Countries like France, Britain and South Korea
have established laws authorizing the suspension of persistent copyright pirates' broadband
connections, in an effort to get more of them to become paying customers.

For advocates of openness, the nightmare outlook is one in which telecommunications companies,
allied with other corporate partners, seize control of the Internet and run it in a way that
maximizes profits, rather than openness. This concern has fueled calls for governments to impose
rules to enforce ``network neutrality,'' or equal priority to all Internet traffic, regardless of
the content.

``The Internet has become a truly global space where everyone, almost everywhere, has access to the
same information,'' said J\'er\'emie Zimmermann, co-founder of La Quadrature du Net, a group based
in Paris that campaigns against restrictions on Internet use. ``I think this is one of the most
precious things we have ever built as a civilization, and this is what is at stake now.''

Others say network neutrality is a largely American issue, rooted in a lack of competition among
broadband providers, which fuels fears that these companies might abuse their monopoly positions.

There are other signs that competition can keep openness alive. One of the most successful of the
closed systems, Apple's iPhone, is already showing signs that it might be eclipsed by other, more
interoperable rivals. In the United States, sales of smartphones using Google's more open Android
platform recently overtook sales of iPhones. Android phones also use applications, but unlike Apple,
Google does not screen them, and Android is open to competing applications stores, like one planned
by Amazon.

Even the idea that the desktop and the mobile Internet exist in two different spheres may turn out
to be merely a temporary phenomenon, some analysts say. Much of the content in mobile applications
is scoured and repackaged from the Web -- so, for now, at least, it is difficult to argue that users
of applications are really turning their backs on the Web.

``If you go with the Web, the potential mobile audience is in the billions. If you go with any of
the smartphone operators in a closed environment, it's a small fraction of that,'' said Jon von
Tetzchner, chief executive of Opera Software, a Norwegian company that develops Web browsers. ``To
me, it seems like the Web has been winning fairly big time.''

\section{Fearing Election Losses in New York, National Democrats Enlist Cuomo's Help}

\lettrine{A}{nxious}\mycalendar{Nov.'10}{01} national Democrats are pressing Attorney General Andrew
M.~Cuomo to try to stave off significant losses for the party in New York, as even more Democratic
incumbents than expected find themselves imperiled in the campaign's final days.

Nancy Pelosi, the House speaker, has spoken regularly by telephone with Mr.~Cuomo, deploying him on
a near-daily basis to campaign with the most imperiled Democratic House members.

Ms.~Pelosi made the stakes clear at a fund-raiser with donors and House colleagues at an Upper East
Side apartment on Oct.~19, according to several guests: Big losses in the state on Election Day
would be demoralizing to Democrats across the nation, given New York's Democratic complexion.

The blitz has taken Mr.~Cuomo far afield of the vote-rich large cities and inner suburbs where
Democratic candidates for statewide office normally spend the closing days of a campaign. In the
last week or so, he has campaigned in Owego, N.Y., with Representative Maurice Hinchey; in the
Finger Lakes village of Cazenovia, N.Y., where he endorsed Representative Bill Owens; and in
Bedford, N.Y., where he endorsed Representative John Hall.

By Monday, Cuomo aides said, he expects to have campaigned with seven of the Republicans' top
Democratic targets in New York.

At a time when President Obama's own approval ratings are diminished, Mr.~Cuomo, the most popular
Democrat in the state and one of the few with significant crossover appeal to Republicans and
independents, has offered vulnerable House Democrats the kind of halo they are desperately seeking.

On Tuesday, Mr.~Hall, who is locked in a tight battle with his Republican opponent, Nan Hayworth,
portrayed himself not as part of the national Democratic Party, but as a kind of local partner to
Mr.~Cuomo, who is running against Carl P.~Paladino, the Republican nominee for governor and a
Buffalo businessman.

``Voters across the state and in District 19 have a clear choice to make on Tuesday of next week,''
Mr.~Hall said. ``The Cuomo-Hall ticket, which I am proud to be on, or the Paladino-Hayworth
ticket.''

Mr.~Cuomo, a hoarder of political capital and a rising Democratic star widely presumed to have one
eye on the White House, also stands to benefit. Beyond the opportunity to collect chits from
national Democratic leaders, he is also keenly aware that his own success as governor may well rest
on whether he can persuade Washington to part with billions of dollars in additional aid to help
plug the \$8 billion budget gap he would face in Albany next year.

In Bedford, after praising Mr.~Hall as someone who would protect Social Security and Medicare,
Mr.~Cuomo quickly got to the point.

``We need someone who is going to work with the federal administration to bring back funds to New
York -- because frankly, my friends, we need the money,'' Mr.~Cuomo said.

The role of Democratic standard-bearer is not a natural one for Mr.~Cuomo, who began his campaign
stressing his independence from his own party and traditional Democratic constituencies, and
conspicuously sought endorsements from Republican officials. For most of the campaign, Mr.~Cuomo has
been cautious, making only one or two public appearances a day and sometimes none at all, drawing
criticism that he was failing to engage with Mr.~Paladino.

Even now, some Democrats privately complain that Mr.~Cuomo could be doing more, especially given the
size of his campaign fund. According to his most recent filing, Mr.~Cuomo has \$12 million in his
campaign account, far more than any other major party candidate for New York governor during the
past decade.

But with Mr.~Paladino behind in most polls, Mr.~Cuomo has been virtually besieged by requests from
donors, Democratic county leaders and fellow elected officials to become a more visible presence on
behalf of other Democrats in the state. The requests have come via dozens of phone calls, e-mails to
his aides and even the occasional buttonhole at a fund-raiser or campaign event.

``I said to him, `It would really benefit to the congressionals, and to turnout, if you would get
out in some of these districts,' '' said a prominent Democratic donor, who spoke anonymously in
order to describe a private conversation with Mr.~Cuomo. ``He said, `Absolutely.' ''

Mr.~Cuomo also has a close relationship with Ms.~Pelosi, dating to his years in Washington.
Ms.~Pelosi's daughter, Christine Pelosi, served as a special counsel to Mr.~Cuomo when he was
secretary of the Department of Housing and Urban Development in the Clinton administration.

Mr.~Cuomo and Ms.~Pelosi or their aides speak about every other day, sharing intelligence from the
ground and coordinating his appearances.

``It's absolutely critical to have him involved,'' said Representative Steve Israel, a Long Island
Democrat who is the national recruitment chairman for the Democratic Congressional Campaign
Committee. ``In this environment, you need every inch of coattail that you can get. Andrew's got
miles of coattails.''

Some Democratic officials fear that high Republican turnout will overwhelm the party even in New
York, costing Democrats not only swing seats they have gained recently, like Mr.~Owens's, but also
seats they have held for years, like Mr.~Hinchey's and the Nassau County seat held by Representative
Carolyn McCarthy.

Mr.~Cuomo was expected to campaign on Saturday with Representative Scott Murphy, who represents a
rural and mostly Republican district near Albany. On Friday, Mr.~Cuomo appeared at a rally in
Melville, N.Y., with Representative Tim Bishop, a Suffolk County Democrat representing the
predominantly Republican eastern tip of Long Island.

Richard H.~Schaffer, chairman of the Democratic Party in Suffolk County, said that he and other Long
Island Democrats had implored Mr.~Cuomo to come and that he had quickly obliged.

``We were re-emphasizing and re-emphasizing that he was strong, that he was doing a lot of good,''
Mr.~Schaffer said.

Referring to Mr.~Cuomo's comfortable lead in the polls at a time when most Democrats are on the
defensive, he added: ``You're only in this position very few times in your life. You might as well
use it to everyone's advantage.''

\section{A New Search Engine, Where Less Is More}

\lettrine{S}{tart}\mycalendar{Nov.'10}{01}-ups and big companies alike have tried to take on Google
by building a better search engine. That they have failed has not stopped brave new entrants.

The latest is Blekko, a search engine that will open to the public on Monday.

Rich Skrenta, Blekko's co-founder and chief executive, says that since Google started, the Web has
been overrun by unhelpful sites full of links and keywords that push them to the top of Google's
search results but offer little relevant information. Blekko aims to show search results from only
useful, trustworthy sites.

``The goal is to clean up Web search and get all the spam out of it,'' Mr.~Skrenta said.

Blekko's search engine scours three billion Web pages that it considers worthwhile, but it shows
only the top results on any given topic. It calls its edited lists of Web sites slashtags. The
engine also tries to weed out Web pages created by so-called content farms like Demand Media that
determine popular Web search topics and then hire people at low pay to write articles on those
topics for sites like eHow.com.

It is also drawing on a fruitful category of Web search -- vertical search engines that offer
results on specific topics. Many companies assume that Google won the contest to search the entire
Web, so they have focused on topical search. Bing from Microsoft has search pages dedicated to
travel and entertainment, and Yelp is a popular choice for searching local businesses.

People who search for a topic in one of seven categories that Blekko considers to be polluted with
spamlike search results -- health, recipes, autos, hotels, song lyrics, personal finance and
colleges -- automatically see edited results.

Users can also search for results from one site (``iPad/Amazon,'' for instance, will search for
iPads on Amazon.com), narrow searches by type (``June/people'' shows people named June) or search by
topic. ``Climate change/conservative'' shows results from right-leaning sites, and ``Obama/humor''
shows humor sites that mention the president. Blekko has made hundreds of these slashtags, and users
can create their own and revise others.

Mr.~Skrenta, who has been quietly building Blekko since 2007, has spent his career trying to improve
Web search by relying on Web users to help sift through pages.

He started the Open Directory Project, a human-edited Web directory that competed with Yahoo in the
1990s and was acquired by Netscape in 1998. He ran three search properties at AOL and helped found
Topix, the human-edited news site that was acquired in 2005 by Gannett, the Tribune Company and
Knight-Ridder.

In some cases, Blekko's top results are different from Google's and more useful. Search ``pregnancy
tips,'' for instance, and only one of the top 10 results, cdc.gov, is the same on each site.
Blekko's top results showed government sites, a nonprofit group and well-known parenting sites while
Google's included OfficialDatingResource.com.

``Google has a hard time telling whether two articles on the same topic are written by Demand Media,
which paid 50 cents for it, or whether a doctor wrote it,'' said Tim Connors, founder of PivotNorth
Capital and an investor in Blekko. ``Humans are pretty good at that.''

Still, for many other queries, the results are quite similar. Blekko's challenge is that most people
are happy with Google's search results, which comScore says account for two-thirds of search queries
in the United States.

``Most people aren't saying, `I'm just overwhelmed with content farms,' '' said Danny Sullivan,
editor in chief of Search Engine Land and an industry expert.

Google also enables people to easily search individual Web sites or set up a custom search of a
group of Web sites, though it is a more complicated process.

Blekko is also taking aim at Google's opacity about its algorithm for ranking search results. Blekko
offers data like the number of inbound links to a site, where they come from and when Blekko last
searched the content of a site.

Blekko has raised \$24 million in venture capital from prominent investors like Marc Andreessen, Ron
Conway and U.S.~Venture Partners. It plans to sell Google-like search ads associated with keywords
and slashtags.

Some start-ups that have taken on search have been folded into the big companies, like Powerset,
which Microsoft bought in 2008. Others, like Cuil, a search engine started by former Google
engineers in 2008, were flops. Blekko's slashtags could be subject to spam since anyone can edit
them, but Blekko says it will avoid that with an editor and Wikipedia-style policing by users.

``They have an interesting spin,'' Mr.~Sullivan said about Blekko. ``It might take off with a small
but loyal audience, but it won't be a Google killer.''

\section{Spielberg and DreamWorks Energize the Magic Machine Anew}

\lettrine{I}{n}\mycalendar{Nov.'10}{01} the perfect little town of Paradise, Ohio, a pretty-faced
new kid has a crush on the sweet blonde who is showing him around. By the way, the kid is also a
space alien, on the run from some other aliens who are anything but pretty.

After two years in the throes of a financial restructuring, Steven Spielberg and his DreamWorks
Studios are back with some typically Spielbergian stuff. And they are starting the next round with
the sort of fanciful, scary, sometimes heartwarming movies they know best -- and their new
distribution ally, Walt Disney Studios, needs most.

Success for DreamWorks might quickly return Disney to its former status as a ``full service'' studio
offering a wide range of action movies and dramas, after it had focused for years on animation and
family fare that paid the bills but kept the company out of the Hollywood mainstream.

It would also vindicate both a new film management team, and the decision by the Disney chief
executive, Robert A.~Iger, to dismantle the company's Miramax art film unit.

DreamWorks Studios is completely separate from DreamWorks Animation, a publicly traded animation
company whose films are still distributed by Paramount Pictures.

But DreamWorks is not starting with challenging films like its Oscar winner ``American Beauty'' or
its hit musical ``Dreamgirls.'' Instead, it plans to produce one or two outsize blockbuster-style
movies each year, and four or five smaller films that stick with Hollywood's more reliable genres.

The caution is understandable. Founded in 1994 by Mr.~Spielberg, Jeffrey Katzenberg and David
Geffen, the original DreamWorks SKG took about three years to release its first movie.

It was an international thriller called ``The Peacemaker'' with George Clooney, who had soft
credentials as an action star. It fell flat, embarrassing the new studio, with just \$69 million in
worldwide ticket sales.

This time, there are no long shots.

The inaugural film from the revamped DreamWorks, ``I Am Number Four,'' with those hormonal teenagers
and nasty aliens -- and a heavy ``Twilight'' element -- is set for release on Disney's Touchstone
banner on Feb.~18. Mr.~Spielberg is not expected to take a credit on the film, remaining in his
executive role. But neither is he taking chances with the first in a string of movies that will
inevitably have investors, business allies and the audience watching for his trademark screen magic.

``There's a lot of him in there,'' said Marti Noxon, who is among the writers of ``I Am Number
Four,'' and is perhaps best known for her work on the television series ``Buffy the Vampire
Slayer.'' More than once, said Ms.~Noxon, she found herself laboring with Mr.~Spielberg in a
conference room in his adobe-style complex on the Universal Studios lot here in an effort to get the
teenagers and aliens just right.

DreamWorks -- now owned by Mr.~Spielberg and Stacey Snider, with financial backing from Reliance Big
Entertainment of India and distribution via Disney -- carefully picked the release date. It is the
kind of winter slot that has been good to popcorn fare like ``Paul Blart: Mall Cop,'' which
generated over \$183 million at the global box office last year, and ``Percy Jackson and the
Olympians: The Lightning Thief,'' which took in \$226.4 million in February.

The director of ``I Am Number Four,'' D.~J.~Caruso, gave DreamWorks a pair of PG-13 hits,
``Disturbia'' and ``Eagle Eye,'' during its unhappy tenure as a partner of Paramount Pictures. The
producer is Michael Bay, who mixed teenagers and space creatures for DreamWorks and Paramount in the
blockbuster ``Transformers'' series. In an e-mail, Mr.~Bay said he brought the project to
Mr.~Spielberg, whom he described as a ``mentor and friend.''

That ``I Am Number Four'' should be first up from the new DreamWorks in many ways owes more to the
accidents of moviemaking than any grand design.

After finally securing about \$850 million in financing from various sources about 14 months ago,
company executives approved a half-dozen films, all of which were in production this year, according
to Ms.~Snider, who plays down the drama in starting anew.

``DreamWorks has been doing this for 16 years, and I've been with Steven for four and a half
years,'' she said.

``He's made movies for every studio in town,'' Ms.~Snider added. ``But if it feels like a debut, I
guess that's good.''

``Cowboys \& Aliens,'' the most expensive film on the new DreamWorks slate, will cost \$100 million
or more, but it is a co-production with Universal Pictures, which will release it next July.
Directed by Jon Favreau, with Harrison Ford in a lead role, it is an action-fantasy about a fight
among cowboys, Indians and aliens.

A remake of ``Fright Night,'' which is directed by Craig Gillespie (whose credits include
``Mr.~Woodcock'') and written by Ms.~Noxon, is also on the list. It puts teenagers and vampires in a
3-D horror romp, and is set for release by Disney in August.

Other films for Disney in the next year include ``Real Steel,'' which is directed by Shawn Levy
(``Night at the Museum'') and pits robot against robot in a father-and-son boxing story set in the
near future; ``The Help,'' which is directed by an untested director, Tate Taylor, and is based on a
popular novel about white Southern women and their black domestic workers in the 1960s; and ``War
Horse,'' which is directed by Mr.~Spielberg and adapted from a book and play about the friendship
between a boy and his horse during World War I.

Distributing some of these pictures will be tricky -- Disney just found out how difficult a horse
movie is to sell with its own underperforming ``Secretariat'' -- but Disney is bubbly about its
budding relationship with DreamWorks.

``Releasing the DreamWorks films allows us to have greater leverage in the market,'' said Rich Ross,
chairman of Walt Disney Studios. He added, ``When you team with DreamWorks, you don't just add
tonnage, you add great filmmakers and tremendously experienced executives.'' (It doesn't hurt that
Mr.~Ross and Ms.~Snider have been friends since attending the University of Pennsylvania together.)

``I Am Number Four,'' which fits squarely in the genre category, is based on a young adult novel
ostensibly written by Pittacus Lore, a pseudonym for Jobie Hughes and James Frey. Mr.~Frey is best
known as the author of ``A Million Little Pieces,'' a memoir about his life and drug use, elements
of which were later discredited as fiction.

In the film version, Alex Pettyfer plays one among a group of young aliens who are hiding on Earth
from an another alien race who must hunt and kill them in sequence. Mr.~Pettyfer's character is
Number Four, and Numbers One through Three are dead.

(DreamWorks is grooming Mr.~Pettyfer, a 20-year-old who made his feature film debut four years ago
in ``Alex Rider: Operation Stormbreaker,'' as the next Shia LaBeouf, a slightly older star it has
featured in several films, including the ``Transformers'' series. Plans now call for Mr.~Pettyfer to
star in a biographical film for the studio, to be produced by himself and John Palermo, about the
Formula One racing driver James Hunt.)

Speaking by telephone, Mr.~Caruso said he had been developing some planned projects for Paramount
last year, when the DreamWorks crew approached him with a script that was begun when the original
juvenile novel for ``I Am Number Four'' was still in manuscript form.

``This was an incredibly quick process,'' Mr.~Caruso said of the rush to get his film in motion once
DreamWorks had its financing in place.

Much of the movie was shot in and around Vandergrift, Pa., where Mr.~Caruso could take advantage of
Pennsylvania film production incentives, as well as a utopian small-town setting that was designed
originally by the firm of Frederick Law Olmsted, who helped lay out Central Park.

The film borrows a bit from Mr.~Spielberg, acknowledged Mr.~Caruso, particularly when the young,
Earth-bound alien shares his luminous power with a special friend, named Sarah.

``There's a beautiful moment, like `E.T.' and `Close Encounters,' '' he said.

If Mr.~Spielberg offered light, Mr.~Caruso added, Mr.~Bay brought some heat.

``Number Six is a sexy, kick-ass girl, so there's definitely a little bit of Michael Bay in there.''

So film Number One has something for everyone. ``It's got adult sort of action that can cross over
to the family,'' Mr.~Caruso said.

``I wanted to do something my kids can see.''

\section{In Ads, Candidates Make Their Final Pitches to Voters}

\lettrine{S}{harron}\mycalendar{Nov.'10}{01}, the Republican candidate for Senate in Nevada, wants
the state's voters to know that Harry Reid had his chance, but it's her turn now.

This is the final pitch Ms.~Angle makes in a political ad released Thursday and titled ``Our Turn,''
which argues that two years ago, Democrats ``promised change -- but they delivered unprecedented
spending, overwhelming debt, heartbreaking job loss,'' and a laundry list of other economic woes.

``They promised change,'' the ad concludes. ``Now, it's our turn.''

With Election Day on Tuesday, candidates across the nation are turning to political ads to make
their closing arguments -- often a last-ditch plea to win over voters by either reintroducing
themselves , tearing down their opponent one last time , or something in between. And this year,
both Democrats and Republicans are trying to harness the anti-Washington sentiment to push different
versions of the same theme: That the nation's capital must change, and with their independent voice
and close ties to their home state, the candidate currently approving this ad is just the person for
the job.

``At a very base level, once again, and ironically, we are in a `change' environment,'' said Mark
McKinnon, a Republican strategist who made ads for President George W.~Bush and Senator John McCain
of Arizona. ``Which means challengers and incumbents, Republicans and Democrats, are all arguing
that they are fighting the system and that Washington is the problem.''

Republicans especially are running as outsiders and pushing a message of bringing change to
Washington. Carly Fiorina, the Republican Senate candidate in California,has a closing ad that talks
ominously about ``the legacy of Barbara Boxer,'' the incumbent senator and Ms.~Fiorina's Democratic
opponenet, before a narrator says, ``We can change Washington, but not unless we change the people
we send there.'' And many Republicans are making a similar pitch, from Jon Runyan, who is running
for Congress in New Jersey's Third Congressional District (``If you're tired of the way this country
is run, then let's change the people who run it,'' Mr.~Runyan says) to Kristi Noem, who is running
for Congress against incumbent Democrat Stephanie Herseth Sandlin in South Dakota (``Stephanie
Herseth Sandlin -- Washington really changed her,'' a narrator warns.)

``Republicans have a tailor-made message, which is that we're not Democrats, basically,'' said
Michael Franz, an associate professor of government at Bowdoin College and co-director of the
Wesleyan Media Project. ``You always see something like that from challengers who are outsiders
trying to enter the system. You have a lot of newcomers on the Republican side who have a chance of
winning so they can make that case quite clearly, that Obama and the Democrats have had control for
two years and their policies aren't working, so we need change, we need to change that leadership.''

Democrats, too, are doing a final anti-Washington dance, promising to ``shake up Washington'' if
elected, in the case of West Virginia Governor Joe Manchin, and talking about their ties to their
home states. Earl Pomeroy, an incumbent Democratic congressman from North Dakota, for instance,
acknowledges that ``I know you're angry with Washington,'' but says that he travels ``home to North
Dakota every weekend'' and adds, ``my values are formed right here in Valley City where I grew up.''
Mr.~Manchin opens his ad by saying, ``I'm as mad as you are with what's going on in Washington,''
and distances himself from both parties.

``Every Democrat is running their own race, and hopes to be judged on their connection to their
state and district,'' said John Del Cecato, a partner at AKPD, a Democratic media firm.

Running against Washington, however, is trickier for Democrats. ``The fact is, the Democrats own the
current situation,'' said Ken Goldstein, a professor of political science at the University of
Wisconsin.

``Democrats are really trying to disqualify Republicans, and in some ways throwing anything against
the wall,'' Mr.~Goldstein said. (Case in point? The ad by Charlie Melancon, the Democratic
congressman from Louisiana who is running for Senate against Senator David Vitter, which opens by
stating, ``Our tax dollars pay David Vitter's salary, and he used it for prostitutes. You're
welcome, senator.'')

But after such a bitterly divisive and negative campaign season, an increasing number of candidates
are sounding a positive note in the final stretch. Cue the spouses, the kids, and the fuzzy
biographical spots.

Marco Rubio, the Republican candidate for Senate in Florida, is airing a special two-minute ad that
talks about his personal story -- he is the son of a bartender and a maid -- and what he believes
the election is about. Alexi Giannoulias, who has called his opponent a liar and whose opponent has
called him a mob banker, ends his bid for President Obama's former Illinois Senate seat with a
positive ad that features both the president and Michelle Obama, the first lady.

``Campaigns almost always close positive,'' Mr.~McKinnon said. ``They want their final image to be
contrary to the one painted by their opponents. And anyone who has children can't possibly be guilty
of the heinous crimes of which they've been accused.''

And then, of course, there is the meta-ad, which features a candidate dismissing the negative ads
his opponent has thrown at him all season, before saying something positive about himself.

``Obviously you want to wait to do something like that until there's been a lot of negative ads out
there,'' said Travis Ridout, an associate professor of political science at Washington State
University. ``It's a way of saying something kind of negative about your opponent, without seeming
to say something negative.''

\section{Parties Ramp Up Appeals on Last Weekend of Campaigning}

\lettrine{R}{epublicans}\mycalendar{Nov.'10}{01} made their closing argument on Sunday for capturing
control of Congress, assailing President Obama as a champion of wasteful and excessive government,
as Democrats countered that returning power to Republicans would empower corporations and the
wealthy with disastrous results for middle-class Americans.

Mr.~Obama and Vice President Joseph R.~Biden Jr.~converged on Ohio in hopes of rekindling the
passion Democrats displayed two years ago in sending the pair to the White House. But Representative
John A.~Boehner, who is poised to become speaker if Republicans win the House, offered a rebuttal as
he crisscrossed the state, warning voters not to be taken in by familiar promises of changing
Washington.

``Washington hasn't been listening to the American people; I think it's been disrespecting the
American people,'' Mr.~Boehner said, firing up Republicans in Columbus. He flashed a sign of
confidence, saying: ``We're going to have a big night on Tuesday night -- a really big night.''

Republicans are positioned to reach or surpass the number of House seats that they picked up in
1994, according to strategists and independent analysts, when the party gained 54 and ended four
decades of Democratic dominance in the House.

The ranks of vulnerable Democrats deepened, with House seats in Connecticut, Maine and Rhode Island
becoming susceptible to a potential Republican wave that could exceed the 39 seats needed to win
control. Democrats clung to hope that they could hold on to the Senate by at least a seat or two.

As the White House prepared for widespread losses on Tuesday, it sought to minimize the political
damage to the party and to Mr.~Obama's re-election chances in two years by urging its supporters to
work even harder to help narrow the gap in tight races.

``In two days, you have a chance to once again say, `Yes, we can,' '' Mr.~Obama said at a rally in
Cleveland, reprising the theme of his 2008 campaign. ``There is no doubt that this is a difficult
election.''

Former President Bill Clinton began a nine-state sweep to help preserve the Democratic majority,
rallying voters along the Eastern Seaboard from Maine to Florida. The focus on Democratic-leaning
states, including Congressional races in New York and Connecticut, along with governors' races in
Maine and New Hampshire, underscored the degree to which Democrats are almost entirely on defense
against Republicans at a time when the unemployment rate remains stubbornly high and the nation is
deeply divided over issues like health care.

The most expensive Congressional contest in the country's history, with a price tag of nearly \$4
billion, drew to a close with a presidential-style deployment of troops from Alaska to Florida
intended to get-out-the-vote, even though more than half of Americans participating in the election
have already cast early ballots.

Few states were spared a last barrage of television advertisements, many of which featured
Republicans trying to capitalize on the suggestion of change.

In California, the Republican candidate for the Senate, Carly Fiorina, said, ``We can change
Washington, but not unless we change the people we send there.'' In Nevada, the Republican nominee
for the Senate, Sharron Angle, said, ``They promised change, now it's our turn.'' In New Jersey, a
Republican candidate for Congress, John Runyan, declared: ``If you're tired of the way this country
is run, then let's change the people who run it.''

Democrats were bracing for substantial losses across the ballot, from state legislative races to
governorships to Congress, with incumbents imploring voters to give them more time to create jobs,
improve the economy and control state and federal spending. It remained an open question whether the
party's extensive get-out-the-vote operation could diminish what party leaders fear could be
widespread defeats.

``This is not a year to sit on your hands,'' said Diane Denish, the lieutenant governor of New
Mexico, a Democrat who is locked in an uphill fight to win the governor's race. ``It's more
important than ever to get off your behind and get out of the shade and get into the heat, as we say
in the oil fields.''

In the House, 29 Democratic-held seats are either leaning Republican or are all but lost to
Republican candidates, according to the latest ratings of Congressional races by The New York Times,
while 42 seats held by Democrats are seen as tossups. Even if Republicans win only half of the
tossup seats, they will have passed the threshold of 39 net seats gained they need to win control.

Democrats were fighting to preserve their majority in the Senate and fortify individual races
against the possibility of a Republican wave in House and governors' races. Five Senate contests
considered tossups will help determine if the party holds control, according to the analysis by The
Times, with Republicans trying to capture Democratic-held seats in Colorado, Illinois, Nevada,
Pennsylvania and Washington.

Senator Harry Reid of Nevada, the Democratic leader, is locked into one of the closest races in the
country against Ms.~Angle. After a yearlong stretch of combative and piercing television ads, his
closing message to voters was not an argument on issues, but rather a plea to voters to reject his
opponent.

``Harry Reid, fighting for us,'' an announcer says, with images of Mr.~Reid appearing with workers
across Nevada. ``Sharon Angle: pathological.''

Republicans were looking for fresh patches of opportunity, scouring districts that only weeks ago
would have been unthinkable. The First Congressional District of Rhode Island, which for 16 years
has been represented by Patrick J.~Kennedy, was suddenly seen as winnable terrain for Republicans.
Mitt Romney, the former governor of Massachusetts and aspiring Republican presidential candidate,
was set to visit the district on Monday.

Republicans dispatched party leaders across the country on Sunday on behalf of their candidates.
Rudolph W.~Giuliani, the former mayor of New York, campaigned in northern Wisconsin for Sean Duffy,
the Republican former television reality star who is running for the seat held by Representative
David R.~Obey, a Democrat, who is retiring after 41 years in Congress. Gov.~Tim Pawlenty of
Minnesota traveled to Iowa to build support for legislative candidates.

The most heavily-traveled state on Sunday, though, was Ohio, where Republicans are trying to win up
to six Congressional districts, one Senate contest and the governors' race. After the president and
vice president appeared together in Cleveland, Mr.~Biden stopped in Toledo for an evening rally for
Gov.~Ted Strickland, who is locked in a too-close-to-call race with John Kasich, a former Republican
congressman.

For weeks, the White House has put more emphasis on Ohio than nearly any other state, but over the
weekend, signs of a Republican revival were underscored by Mr.~Boehner's rare public appearances in
his own state. He zipped from district to district as he worked to expand the Republican advantages
in the House and to deliver a counterargument to the president.

``They have been coming here for months? Why? You might think it's to help Ted Strickland. What he
is really coming for is to help himself,'' Mr.~Boehner said Sunday evening at this final stop in
Chillicothe. ``He knows that in 2012 if he doesn't have Ted Strickland in office, his re-election
chances are seriously damaged.

``So if you want to send President Obama a message about spending and about takeovers and bailouts
and all the nonsense,'' he added, ``go out there on Tuesday and vote for John Kasich.''

\section{Russia Angers Japan With Visit to Disputed Islands}

\lettrine{P}{resident}\mycalendar{Nov.'10}{02} Dmitri A.~Medvedev flew on Monday into the south
Kuril Islands, which the Soviet Union seized from Japan at the end of World War II, making it clear
that Russia had no plans to cede the mineral-rich territory despite Japanese protests.

Mr.~Medvedev is the first Russian president to visit one of the disputed Kuril islands. The four
islands lie between the Japanese island of Hokkaido and the rest of the Kuril archipelago, which is
Russian territory.

The four islands are sparsely populated but grant access to prize fisheries and promising oil and
gas fields. Touring day-care centers and family homes, Mr.~Medvedev told residents that Russia would
invest heavily to raise living standards on the islands.

``We want people to remain here,'' he said at one stop. ``Development here is important. We will
definitely be investing money here.''

The trip immediately aggravated relations with Japan, which has long demanded that Russia return the
islands. Foreign Minister Seiji Maihara said Mr.~Medvedev's presence ``injures the feelings of the
population of Japan,'' and summoned Russia's ambassador to deliver a note of protest.

Japanese leaders warned Russia in September that such a visit would damage bilateral relations.

Tokyo is already locked in a tense dispute with China over islands in the East China Sea. In
September, its coast guard detained the captain of a Chinese fishing boat that collided with two of
its vessels. The arrest sparked anti-Japanese protests in China, which is acutely sensitive about
threats to its sovereignty.

Japanese authorities have sought to calm emotions, but opposition lawmakers want to release
videotape of the collision, which they believe shows that that the Chinese captain was at fault.

Russian officials responded angrily to the Japanese complaints on Monday. Mikhail Margelov, head of
the Federation Council's international affairs committee, called the protests ``absurd, to say the
least.''

``It is important that all our Japanese neighbors and all our partners understand that talking with
Russia from a threatening position is pointless,'' Mr.~Margelov told the Interfax news service.
``Our stance cannot be changed by pressure. I am sincerely hoping that wisdom will return to
Japanese political practices.''

But Mr.~Medvedev's visit to the disputed territory will convey a clear message domestically in a
country increasingly focused on the 2012 presidential elections.

Mr.~Medvedev is generally viewed as milder and more liberal than his mentor, Prime Minister Vladimir
V.~Putin, but he has taken a tough line on territorial disputes and chose to go to war with Georgia
over the breakaway enclave of South Ossetia. In the 1956 declaration that re-established ties
between Russia and Japan, Russia offered to return two of the four islands after the two countries
signed a peace treaty. But Japan rejected that compromise, maintaining that all four islands should
be returned, and no treaty has ever been signed.

Russia's foothold in the Kurils weakened in the 1990s, when Moscow drew down its military presence
on the islands and many Russian settlers left for the mainland.

The Far East carries ``huge'' economic importance to Russia now, both because of its oil and gas
reserves and transport links to Asian markets, said Elgena V.~Molodyakova, an expert on the region
at the Russian Academy of Sciences.

``For us, the Kuril problem is how to develop the region,'' she said. ``For the Japanese, the Kuril
problem is a territorial dispute that can agitate their society. If they take a hard line on this,
they won't succeed.''

\section{Bomb Plot Shows Key Role Played by Intelligence}

\lettrine{I}{n}\mycalendar{Nov.'10}{02} the middle of last week, a woman who claimed her name was
Hanan al-Samawi, a 22-year-old engineering student, walked into the U.P.S. office in the upscale
Hadda neighborhood of Sana, Yemen's sprawling capital city. She displayed a photocopied
identification card, and dropped off a bomb hidden inside a printer cartridge with a Chicago address
listed as the package's destination. A few blocks away, another package concealing a homemade bomb
was dropped off at a FedEx office, also seemingly headed to Chicago.

Within days, the two packages had advanced through four countries in at least four different
airplanes -- two of them carrying passengers -- before they were identified in Britain and Dubai
after an 11th-hour tip from Saudi Arabia's intelligence service set off an international terrorism
alert and a frantic hunt.

The foiling of the package plot was a significant success in an era of well-publicized intelligence
breakdowns and miscommunications. It was also a sobering reminder to officials around the world that
quick response to timely intelligence rules the day. Despite the billions of dollars governments
have spent on elaborate airport technology to guard against terrorism threats, the packages would
probably have been loaded onto planes bound for the United States, but for the Saudi tip.

The plot also points up holes in the system, particularly the security of cargo flights, that have
already caused criticism abroad and are likely to rekindle new debates in the United States.

In Qatar, officials acknowledged Sunday that one of the packages had been carried on two Qatar
Airways passenger planes, apparently having eluded the airline's cargo screening system. In Britain,
officials were embarrassed about how long it took the authorities to identify one of the packages as
a carefully concealed bomb.

American and Yemeni officials still have little hard evidence about who was involved in the thwarted
attack. On Sunday officials in Yemen discovered that Ms.~Samawi's identity had apparently been
stolen, and that she was not the same woman who dropped off the packages. Ms. Samawi was released on
bail on Sunday, and the authorities in Yemen have thus far arrested no other suspects.

It was one more piece of a carefully designed and cleverly disguised plot that investigators believe
was conceived by Al Qaeda's affiliate in Yemen, the group that American officials say might pose the
most immediate threat to American soil.

In television appearances on Sunday, John O.~Brennan, President Obama's top counterterrorism
adviser, said that American and British authorities were leaning toward the conclusion that the
packages were meant to detonate in midair, en route to their destinations in Chicago. If that turns
out to be the case, it would be a rare attack aimed at the air cargo system -- one of the
foundations of the global economy -- rather than the passenger system, which has received the most
attention from governments working to avoid a repeat of the Sept. 11, 2001, attacks.

For the most part, governments around the world had bet that it was less likely that the cargo
system would be the target of attacks, given that its flights carry few passengers.

``It is time for the shipping industry and the business community to accept the reality that more
needs to be done to secure cargo planes so that they cannot be turned into a delivery systems for
bombs targeting our country,'' Representative Ed Markey, Democrat of Massachusetts, said in a
statement.

Congress in 2007, in legislation proposed by Mr.~Markey, mandated that all air cargo be inspected
before it is loaded onto passenger planes, setting an August 2010 deadline for the requirement. But
as of the deadline, only about 65 percent of the cargo headed to the United States on passenger
planes from abroad is inspected -- and a far smaller proportion coming to the United States on
all-cargo flights is physically checked, as these planes are not subject to the mandate.

Even when the cargo is checked, air carriers in certain countries use equipment like X-ray detection
devices or a visual check by an airport worker that often cannot identify packages with bombs,
because the small amount of explosive material can be carefully hidden inside a routine electronic
device, like a computer printer.

Interviews in Washington, London and the Middle East reveal how the two bombs made their way through
several countries before the tip from Saudi intelligence officials caused them to be pulled from
airplanes.

The bomb dropped at the U.P.S. office in Sana ended up in East Midlands Airport, near Nottingham,
England, by way of Cologne, Germany. A terrorism alert from Washington provoked a search for the
package, which was found and kept from being shipped to the United States. But British authorities
took more than 20 hours to determine that it contained hidden explosives.

Theresa May, the British home secretary, told the BBC that the government would review its security
arrangements for handling air cargo entering or passing through Britain in the wake of the
printer-bomb plot, but declined to give any details.

In Britain, cargo operators are vetted and named ``trusted carriers.'' Cargo itself is not screened,
which some experts said made British airports vulnerable to terrorist exploitation. Ms.~May said
that any changes would have to take into account economic concerns. ``We're well aware of the
economic aspects of air freight transport,'' she said.

The second package -- a bomb hidden inside a Hewlett-Packard desktop printer -- was sent out
Thursday on a Qatar Airways passenger flight to Doha, the Qatari capital. There it sat for a day,
and was then flown 235 miles east to Dubai, in the United Arab Emirates, where it arrived Friday in
the local FedEx distribution warehouse.

By that time Emirati authorities had received a warning call from Britain about a suspicious package
there, and they identified the printer almost immediately, according to an official familiar with
the investigation. Investigators removed and dismantled the explosive, which had been placed into
the toner cartridge printer so carefully that all the printer's components appeared to be in place
and it might well have passed unnoticed.

A cellphone was concealed in the bottom of the printer, and the printer head was designed to
detonate the explosives.

On Sunday, officials in Qatar said in a statement that ``the explosives discovered were of a
sophisticated nature whereby they could not be detected by X-ray screening or trained sniffer
dogs.''

As for who was behind the plot, evidence remains elusive, though officials believe the bombs bear
the hallmarks of Al Qaeda in Yemen's top bomb maker. On Friday, the Department of Homeland Security
issued a cable saying that the packages might have been linked to two schools in Yemen. If true,
that would suggest that foreign students might have been involved in the plot, as in the attempted
bombing of a commercial jetliner in Detroit last Dec.~25 by a Nigerian trained in Yemen.

But American and Yemeni investigators are trying to determine whether the schools -- listed as the
Yemen-American Institute for Language-Computer Management and the American Center for Training and
Development -- even exist. There is a school in Sana called the Yemen American Language Institute,
but it is sponsored by the United States State Department. Its director, Aziz al-Hadi, said in a
telephone interview that the school ``has never used FedEx or U.P.S.'' and did not help foreigners
obtain visas. The school does not have a reputation for attracting religiously conservative
students, unlike some other language schools in Yemen. There is an American Center for Training and
Development in Egypt, but not in Yemen.

Ms.~Samawi was released partly because the shipping agent for the courier company was brought into
her interrogation and told investigators that she was not the person who had signed the shipping
manifest, said a Yemeni official, who spoke on condition of anonymity. The Yemeni authorities have
concluded that the plotters deliberately used Ms.~Samawi's name, address and telephone number to
make the shipment look legitimate. Ms.~Samawi's mother was detained Saturday as well, but family
friends said that was only because she insisted on accompanying her daughter.

``She is a very open-minded person; we cannot believe these accusations at all,'' said Siham Ahmad
Haza, 24, who described herself as a close friend of Ms.~Samawi's, and a fellow student of computer
engineering at Sana University. ``She listens to music a lot, especially Western music. She loves
foreigners, she's a balanced person.''

Ms.~Samawi has two younger sisters, and her father works as an engineer in the Ministry of
Agriculture and Water, according to family friends. The family lives in Shamlaan, on the outskirts
of Sana.

About 100 students protested at Sana University on Sunday, chanting ``Freedom, freedom for Hanan!''

\section{Time for Obama to Freshen Things Up}

\lettrine{N}{o}\mycalendar{Nov.'10}{02}-drama Obama may need some drama, or at least a shake-up.

The cool, detached, sometimes too insular president will suffer a political drubbing in midterm
elections on Tuesday. It could be fatal or a glancing blow to his presidency, depending how he
responds.

What is necessary, leading Democrats say (mostly in private), are fresh personnel, policies and
approaches; a midcourse correction, not a radical overhaul.

The early signs are not encouraging. President Barack Obama has tapped a deputy, Pete Rouse, to fill
in for Rahm Emanuel, the departing White House chief of staff; the national security adviser,
Gen.~James L. Jones, is being replaced by his No.~2, Thomas E.~Donilon; and a longtime Obama
adviser, Denis McDonough, will fill Mr.~Donilon's slot.

They are very able. Mr.~Rouse is one of those rare Washington insiders who practice the late
Gen.~George Marshall's dictum that you can accomplish a great deal if you credit others. He is a
stranger to self-promotion and has served Mr.~Obama well in the Senate and the White House.

Yet these appointments convey a signal of promoting the junior varsity after the varsity has been
shellacked.

It is not clear whether the president will reach out for new faces to fill the top levels: a
corporate executive to address the complaints that his is an anti-business administration; someone
with impeccable military credentials to assist and balance Mr.~Donilon, viewed by Republicans and
some in the armed services as a political operative; and somebody with acute political antennae
outside the Obama stratosphere.

Democrats who want Mr.~Obama to succeed say that in addition to expanding his inner circle, the
president has to be more approachable and outreaching on Capitol Hill and elsewhere.

Mr.~Obama's problems are far more than just public relations. Yet somehow this White House has to
figure out why it has done such a poor job in explaining its policies.

When Americans overwhelmingly believe that taxes have gone up, the economy is shrinking, that the
Wall Street bank rescue money will not be repaid and that the stimulus did not create jobs -- all
demonstrably false -- somebody has not been doing something right. As recently as September, a
fatigued White House botched the issue of tax cuts, due to expire for all Americans at the end of
the year.

Noting that his party has controlled Congress and the White House for two years, Paul G.~Kirk Jr., a
former Democratic senator of Massachusetts, says: ``We've had the megaphone, and they stole the
message. They've incited fear in people, and we're paying the price.''

On policies, it is not about ideology. The left-wing rap is flawed: This administration rejected
nationalizing banks and a public option on health care, and downsized automobile companies. Still,
at this stage, presidencies -- whether Republican or Democratic, and even in the absence of an
election debacle -- move to find more common ground with the other side. The first two years is
about an agenda. That is when Republican administrations made tax cuts and began military buildups.

With Republicans likely to win a majority in the House and veto power in the Senate, the White House
has to look for accord in areas like education, trade and minor energy matters. Any significant
fiscal deal is probably out of reach.

Even small accomplishments depend on whether those Republicans who are predisposed to moderately
conservative solutions will act accordingly or be intimidated by the give-no-ground wing of the
party. Watch Republican senators like Richard G.~Lugar of Indiana, Lamar Alexander of Tennessee,
Scott P.~Brown of Massachusetts, Susan M.~Collins and Olympia J.~Snowe of Maine, and a likely
newcomer, Rob Portman of Ohio. And will the common-sense conservative speaker of the House,
Representative John A.~Boehner of Ohio, be able to tame his bomb throwers?

As Mr.~Obama navigates these unfamiliar waters, he will need to lean on Joseph R.~Biden Jr.~Despite
occasional foot-in-the-mouth problems, the vice president has an intimate knowledge of the way
Washington works and commands respect -- and even affection -- from more than a few Republicans.

There is another invaluable potential ally: former President Bill Clinton, with whom Mr.~Obama has a
distant relationship. During the 2008 campaign, Mr.~Emanuel, then an Illinois congressman, advised
Mr. Obama to call Mr.~Clinton periodically, even if he put the phone in the sock drawer. It is time
to call, and stay on the line.

There are differences between what Mr.~Obama faces today and what Mr. Clinton confronted when
Democrats lost their House majority 16 years ago -- the economy is far worse now, and Mr.~Boehner
may not be a convenient foil.

Still, no one understands the office, the conditions and the politics better than the 42nd
president, who, in another remarkable display of resilience, has bounced back from the 2008 doldrums
to become the most popular politician in the United States.

Mr.~Clinton could tell Mr.~Obama that the favorite story line today -- that he was saved from the
1994 election disaster by bringing in the consultant Dick Morris and making nice with Republicans --
is largely fiction.

Mr.~Morris, who was forced to resign two years later after his dalliance with a prostitute came to
light, probably did Mr.~Clinton more harm than good with his recommendations to embrace bite-sized
issues like school uniforms and avoid confrontation with Republicans.

``Dick was an enormously destructive force,'' says Paul Begala, a Clinton confidant. The two most
important events in the Clinton revival were his masterful handling of the Oklahoma City bombing
tragedy and his success in staring down Republicans, who shut down the government for more than
three weeks.

``This was when President Clinton trusted his own instincts,'' Mr.~Begala said.

Mr.~Clinton selectively cut deals with congressional Republicans, going along with a capital-gains
tax cut and forging an accord on overhauling welfare. The stellar economic performance and balanced
budget at the end of his presidency had little to do with those deals.

Some Republicans now wish Mr.~Obama were more like Mr.~Clinton and those ``Kumbaya'' moments could
be replicated. A reminder: Four years after that 1994 election, 95 percent of House Republicans
voted to impeach Mr.~Clinton, basically for lying about sex.

\section{China Census Aims to Chart Shifting Population}

\lettrine{C}{hina}\mycalendar{Nov.'10}{02} began tallying its population on Monday for the first
time since 2000, an arduous task at best, likely to be made tougher by the need to count scores of
millions of migrant workers in the nation's big cities.

The government said it has sent more than six million census-takers out to survey 400 million
households, including the shantytowns and dormitories that often house rural men who have flooded
into the cities to work in factories and on construction projects.

In the five censuses since the Communist government took power in 1949, migrants were listed as
living where their homes were registered instead of where they actually lived. By disregarding the
hukou, as the registration regime is called, the government hopes to get its first accurate count of
city dwellers.

The last major census a decade ago counted 1.265 billion mainland Chinese citizens, of which 807
million were placed in rural areas. The latest United Nations estimate two years ago projected that
the population would reach 1.396 billion this year, and the organization's 2003 estimate projected
that by this year the population would be split about equally between cities and rural areas.

But analyses vary widely, and the sheer volume of migrants -- 160 million is the middle ground of
estimates -- are a demographic wild card that could reshape perceptions of China's population. The
2010 census is expected not only to better document the rural-to-urban migration, but to shed new
light on a number of impending demographic shifts, including a rapid fall in the number of young
people, a similarly sharp growth in the number of elderly and a decline in the size of the
workforce.

Those and other trends may lessen some of the social and economic pressures on Chinese society, such
as the furious scramble to create enough jobs for new workers. But they are likely to create others,
including rising costs for social services such as pensions and changes in the structure of the
economy.

Census officials said in a briefing last week that they are taking extra steps to encourage
cooperation from some classes of citizens who might hide from census-takers. They include
undocumented migrant laborers and families that have quietly violated a 30-year-old policy that
limits many households to one child.

As an inducement to stand up and be counted, census officials are promising that the survey results
will be confidential, and that families with more than one child who choose to register excess
children with the government get a break on the stiff fine normally levied on violators -- and let
them pay on the installment plan if need be. Hukou violators will also be offered a similar deal.

Other inducements are more subtle: to avoid scaring away the vast number of families presumed to
have hidden money from tax officials, the 18-question census from does not ask about income. Nor
does it ask about religious preferences, a touchy topic in a nation that is officially atheist, but
has a burgeoning number of worshipers in underground churches.

Most questions cover mundane topics such as people's ages, gender, literacy, education and the size
of their families and homes. One in 10 people will be asked to fill out a more detailed 45-question
survey.

``It is harder to gather accurate information this time than the previous five national censuses,''
Xing Zhihong, the deputy head of the Beijing census effort, said in an interview with China Daily, a
state-run English-language newspaper. ``Not all residents are willing to reveal personal
information.''

In true Chinese fashion, banners urging cooperation with census-takers -- in green rather than the
traditional red -- have been strung across streets for weeks. Whether the lures will produce a more
accurate count, however, is anyone's guess. In a recent online poll conducted by the Internet portal
sina.com, roughly a third of respondents said they were wary of cooperating with census-takers.

This census brings another landmark change in the head count: for the first time, the government is
counting the rising number of foreigners in the nation, albeit via a smaller eight-question survey.
The results of the census are expected to be released in April.


\section{U.S.~Markets Higher Ahead of Elections and the Fed}

\lettrine{W}{all}\mycalendar{Nov.'10}{02} Street indexes rose Monday at the beginning of a week that
will be dominated by the midterm elections in the United States and central bank meetings, including
one in which the Federal Reserve is expected to announce moves to stimulate the economy.

Shares were higher as traders hoped that a report on manufacturing activity in the United States
would mirror similar data from China that showed expansion there last month.

The state-affiliated China Federation of Logistics and Purchasing said its purchasing managers index
rose to 54.7 points in October from 53.8 September and 51.7 in August. Monthly readings have stayed
above 50, the benchmark for expansion, for 20 months.

Economists polled by Thomson Reuters expect the Institute for Supply Management's manufacturing
index, which will be released Monday, slipped to 54 in October from 54.4 a month earlier. Even with
the slight slowdown, any reading above 50 indicates the sector is expanding. Manufacturing has shown
the most consistent growth during the year as a recovery remains sluggish.

Still, trading was expected to be subdued on Monday as investors wait to see how much additional
monetary easing the Fed might deliver after its meeting on Wednesday.

``With that Fed meeting looming large, it seems unlikely that anyone is going to be willing to make
any serious commitment in either direction over the next couple of days,'' a sales trader at IG
Index, Ben Critchley, said.

In early trading, the Dow Jones industrial average rose 87.52 points, or 0.8 percent. The broader
Standard \& Poor's 500-stock index rose 8.79 points, or 0.7 percent, while the Nasdaq rose 19.58, or
0.8 percent.

In Europe, the FTSE 100 in London was 16.59 points, or 0.3 percent, higher, while the DAX in
Frankfurt rose 20.29 points, or 0.3 percent. The CAC 40 in Paris was 1.75 points, or 0.1 percent,
lower.

Earlier, most Asian markets advanced, with Chinese shares rebounding on the strong manufacturing
data. The benchmark Shanghai Composite Index climbed 2.5 percent, or 75.19 points, to 3,054.02 while
the Shenzhen Composite Index of China's smaller, second market jumped 2.9 percent to 1,341.84.

Hong Kong's Hang Seng index jumped 2.4 percent to 23,652.94. Japan's benchmark Nikkei 225 stock
average bucked the trend, falling 0.5 percent to close at 9,154.72 as the yen's continued rise
against the dollar sapped confidence in the country's potential to export its way to growth.

Any movement tied to Monday's manufacturing report could be fleeting as traders quickly turn their
attention to the midterm elections and the Fed meeting. The Bank of England and the European Central
Bank gather on Thursday, and the Bank of Japan meets on Friday.

If opinion polls are correct, then President Obama will have to work with a Republican-dominated
House of Representatives at the very least. Many think that is a recipe for policy inaction over the
two years before the next presidential elections, meaning that the Fed will have to play an even
more crucial role in sustaining the economy.

``The burden to stimulate the U.S.~economy currently falls heavily on the Fed with U.S.~politics in
a state of paralysis heading into the midterm elections,'' said Lee Hardman, an analyst at the Bank
of Tokyo-Mitsubishi UFJ.

Investors have also been assuming the Fed will start a new Treasury-buying program to help stimulate
the economy. Shares rose for much of October because investors expect the Fed will announce as early
as Wednesday that it plans to buy government debt to drive interest rates lower in an effort to
spark spending and lending.

Only in the last few days has the market rally trailed off amid questions about exactly how much the
Fed will spend to buy bonds. The Dow rose 3.1 percent in October, including a 0.1 percent drop last
week.

Lower interest rates weaken returns on debt, which would make stocks and commodities more attractive
investments since their potential return would be significantly higher. Bond prices traded in a
narrow range Monday. The yield on the benchmark 10-year Treasury note was unchanged at 2.60 percent
compared with late Friday.


\section{Candidates Advance Final Arguments}

\lettrine{R}{epublicans}\mycalendar{Nov.'10}{02} made their closing argument on Sunday for capturing
control of Congress, assailing President Obama as a champion of wasteful and excessive government,
as Democrats countered that returning power to Republicans would embolden corporations and the
wealthy with disastrous results for middle-class Americans.

Mr.~Obama and Vice President Joseph R.~Biden Jr.~converged on Ohio in hopes of rekindling the
passion Democrats displayed two years ago in sending the pair to the White House.

``In two days, you have a chance to once again say, `Yes, we can,' '' Mr.~Obama said at a rally in
Cleveland, reprising the theme of his 2008 campaign. ``There is no doubt that this is a difficult
election.''

But Representative John A.~Boehner, who is poised to become speaker if Republicans win the House,
offered a rebuttal as he crisscrossed the state, warning voters not to be taken in by familiar
promises of changing Washington.

``Washington hasn't been listening to the American people; I think it's been disrespecting the
American people,'' Mr.~Boehner said, firing up Republicans in Columbus. He flashed a sign of
confidence, saying: ``We're going to have a big night on Tuesday night -- a really big night.''

Republicans are positioned to reach or surpass the number of House seats that they picked up in
1994, according to strategists and independent analysts, when the party gained 54 and ended four
decades of Democratic dominance in the House.

The ranks of vulnerable Democrats deepened, with House seats in Connecticut, Maine and Rhode Island
becoming susceptible to a potential Republican wave that could exceed the 39 seats needed to win
control. Democrats clung to hope that they could hold on to the Senate by at least a seat or two.

As the White House prepared for widespread losses on Tuesday, it sought to minimize the political
damage to the party and to Mr. Obama's re-election chances in two years by urging its supporters to
work even harder to help narrow the gap in tight races.

Former President Bill Clinton began a nine-state sweep to help preserve the Democratic majority,
rallying voters along the Eastern Seaboard from Maine to Florida. The focus on Democratic-leaning
states, including Congressional races in New York and Connecticut and races for governor in Maine
and New Hampshire, underscored the degree to which Democrats are almost entirely on defense against
Republicans at a time when the unemployment rate remains stubbornly high and the nation is deeply
divided over issues like health care.

The most expensive Congressional contest in the country's history, with spending of nearly \$4
billion, drew to a close with a presidential-style deployment of campaign volunteers from Alaska to
Florida intended to get out the vote, even though more than half of Americans participating in the
election have already cast early ballots.

Few states were spared a last barrage of television advertisements, many of which featured
Republicans trying to capitalize on the suggestion of change.

In California, the Republican candidate for the Senate, Carly Fiorina, said, ``We can change
Washington, but not unless we change the people we send there.'' In Nevada, the Republican nominee
for the Senate, Sharron Angle, said, ``They promised change, now it's our turn.''

In New Jersey, John Runyan, a Republican candidate for the House, declared: ``If you're tired of the
way this country is run, then let's change the people who run it.''

Democrats were bracing for substantial losses across the ballot, from state legislative races to
governorships to Congress, with incumbents imploring voters to give them more time to create jobs,
improve the economy and control state and federal spending. It remained an open question whether the
party's extensive get-out-the-vote operation could diminish what party leaders fear could be
widespread defeats.

``This is not a year to sit on your hands,'' said Diane Denish, the lieutenant governor of New
Mexico, a Democrat who is in an uphill fight to win the governor's race. ``It's more important than
ever to get off your behind and get out of the shade and get into the heat, as we say in the oil
fields.''

In the House, 29 Democratic-held seats are either leaning Republican or are all but lost to
Republican candidates, according to the latest analysis of Congressional races by The New York
Times, while 42 seats held by Democrats are seen as tossups. Even if Republicans win only half of
the tossup seats, they will have passed the threshold of 39 they need to win control.

Democrats were also fighting to preserve their majority in the Senate. The five Senate contests
considered tossups will help determine if the party holds control, according to the analysis by The
Times, with Republicans trying to capture Democratic-held seats in Colorado, Illinois, Nevada,
Pennsylvania and Washington.

Senator Harry Reid of Nevada, the Democratic leader, is in one of the closest races in the country
against Ms.~Angle. After a year of combative and piercing television ads, Mr.~Reid's closing message
to voters was not an argument on issues, but rather a plea to voters to reject his opponent.

``Harry Reid, fighting for us,'' an announcer says, with images of Mr. Reid appearing with workers
across Nevada. ``Sharron Angle: pathological.''

Republicans were looking for new patches of opportunity, scouring districts that only weeks ago
would have been unthinkable. The First Congressional District of Rhode Island, which for 16 years
has been represented by Patrick J.~Kennedy, was suddenly seen as winnable terrain for Republicans.
Mitt Romney, the former governor of Massachusetts and an aspiring Republican presidential candidate,
was scheduled to visit the district on Monday.

Republicans sent party leaders across the country on Sunday on behalf of their candidates. Rudolph
W.~Giuliani, the former mayor of New York, campaigned in northern Wisconsin for Sean Duffy, a former
reality television star who is running for the seat held by David R. Obey, a Democrat who is
retiring after 41 years in Congress. Gov.~Tim Pawlenty of Minnesota traveled to Iowa to build
support for legislative candidates.

The most heavily-traveled state on Sunday, though, was Ohio, where Republicans are trying to win up
to six Congressional districts, one Senate contest and the governor's race. After the president and
vice president appeared together in Cleveland, Mr.~Biden stopped in Toledo for an evening rally for
Gov.~Ted Strickland, who is in a race that is too close to call with John Kasich, a former
Republican congressman.

For weeks, the White House has put more emphasis on Ohio than nearly any other state, but over the
weekend, signs of a Republican revival were underscored by Mr.~Boehner's rare public appearances in
his own state. He zipped from district to district as he worked to expand the Republican advantages
in the House and deliver a counterargument to the president.

``They have been coming here for months? Why? You might think it's to help Ted Strickland. What he
is really coming for is to help himself,'' Mr.~Boehner said on Sunday evening at his final stop, in
Chillicothe. ``He knows that in 2012 if he doesn't have Ted Strickland in office, his re-election
chances are seriously damaged.

``So if you want to send President Obama a message about spending and about takeovers and bailouts
and all the nonsense,'' Mr.~Boehner added, ``go out there on Tuesday and vote for John Kasich.''


\section{Europe's Plagues Came From China, Study Finds}

\lettrine{T}{he}\mycalendar{Nov.'10}{02} great waves of plague that twice devastated Europe and
changed the course of history had their origins in China, a team of medical geneticists reported
Sunday, as did a third plague outbreak that struck less harmfully in the 19th century.

And in separate research, a team of biologists reported conclusively this month that the causative
agent of the most deadly plague, the Black Death, was the bacterium known as Yersinia pestis. This
agent had always been the favored cause, but a vigorous minority of biologists and historians have
argued the Black Death differed from modern cases of plague studied in India, and therefore must
have had a different cause.

The Black Death began in Europe in 1347 and carried off an estimated 30 percent or more of the
population of Europe. For centuries the epidemic continued to strike every 10 years or so, its last
major outbreak being the Great Plague of London from 1665 to 1666. The disease is spread by rats and
transmitted to people by fleas or, in some cases, directly by breathing.

One team of biologists, led by Barbara Bramanti of the Institut Pasteur in Paris and Stephanie
Haensch of Johannes Gutenberg University in Germany, analyzed ancient DNA and proteins from plague
pits, the mass burial grounds across Europe in which the dead were interred. Writing in the journal
PLoS Pathogens this month, they say their findings put beyond doubt that the Black Death was brought
about by Yersinia pestis.

Dr.~Bramanti's team was able to distinguish two strains of the Black Death plague bacterium, which
differ both from each other and from the three principal strains in the world today. They infer that
medieval Europe must have been invaded by two different sources of Yersinia pestis. One strain
reached the port of Marseilles on France's southern coast in 1347, spread rapidly across France and
by 1349 had reached Hereford, a busy English market town and pilgrimage center near the Welsh
border.

The strain of bacterium analyzed from the bones and teeth of a Hereford plague pit dug in 1349 is
identical to that from a plague pit of 1348 in southern France, suggesting a direct route of travel.
But a plague pit in the Dutch town of Bergen op Zoom has bacteria of a different strain, which the
researchers infer arrived from Norway.

The Black Death is the middle of three great waves of plague that have hit in historical times. The
first appeared in the 6th century during the reign of the Byzantine emperor Justinian, reaching his
capital, Constantinople, on grain ships from Egypt. The Justinian plague, as historians call it, is
thought to have killed perhaps half the population of Europe and to have eased the Arab takeover of
Byzantine provinces in the Near East and Africa.

The third great wave of plague began in China's Yunnan province in 1894, emerged in Hong Kong and
then spread via shipping routes throughout the world. It reached the United States through a plague
ship from Hong Kong that docked at Hawaii, where plague broke out in December 1899, and then San
Francisco, whose plague epidemic began in March 1900.

The three plague waves have now been tied together in common family tree by a team of medical
geneticists led by Mark Achtman of University College Cork in Ireland. By looking at genetic
variations in living strains of Yersinia pestis, Dr.~Achtman's team has reconstructed a family tree
of the bacterium. By counting the number of genetic changes, which clock up at a generally steady
rate, they have dated the branch points of the tree, which enables the major branches to be
correlated with historical events.

In the issue of Nature Genetics published online Sunday, they conclude that all three of the great
waves of plague originated from China, where the root of their tree is situated. Plague would have
reached Europe across the Silk Road, they say. An epidemic of plague that reached East Africa was
probably spread by the voyages of the Chinese admiral Zheng He who led a fleet of 300 ships to
Africa in 1409.

``What's exciting is that we are able to reconstruct the historical routes of bacterial disease over
centuries,'' Dr.~Achtman said.

Lester K.~Little, an expert on the Justinian plague at Smith College, said in an interview from
Bergamo, Italy, that the epidemic was first reported by the Byzantine historian Procopius in 541
A.D. from the ancient port of Pelusium, near Suez in Egypt. Historians had assumed it arrived there
from the Red Sea or Africa, but the Chinese origin now suggested by the geneticists is possible,
Dr.~Little said.

The geneticists' work is ``immensely impressive,'' Dr.~Little said, and adds a third leg to the
studies of plague by historians and by archaeologists.

The likely origin of the plague in China has nothing to do with its people or crowded cities,
Dr.~Achtman said. The bacterium has no interest in people, whom it slaughters by accident. Its
natural hosts are various species of rodent such as marmots and voles, which are found throughout
China.


\section{Lost the Remote? Another Reason to Use an App}

\lettrine{T}{he}\mycalendar{Nov.'10}{02} standard TV remote control, lost so many times beneath sofa
cushions, may soon be lost to history.

Good riddance, says Chris Lavoie, a radio producer living in Los Angeles. Taking advantage of
advances in smartphones, Mr.~Lavoie uses an app on his iPhone to control his Apple TV set-top box
and search for things to watch.

But the phone is not a perfect solution. Its tendency to go into ``sleep'' mode can be annoying,
Mr.~Lavoie said, because it takes several steps to reactivate it before he can use, say, the pause
button. For basic functions, he still uses the remote that came with the box.

TV viewing habits are changing as more Internet and on-demand content -- YouTube videos, streaming
movies, shopping sites, Facebook photos -- flows directly onto big screens. Navigating all of that
demands more action from the viewer, including a fair amount of typing, which current remotes cannot
handle.

``Everybody realizes that the remote control is the dinosaur of the consumer electronics industry,''
said David Mercer, a television analyst at Strategy Analytics, a research and consulting firm. ``The
cable companies and the TV manufacturers are beginning to realize that they have to start moving
away from the traditional, basic remote control.''

But trying to pack too many features into a remote can make it both expensive and physically
imposing. Indeed, the remote Sony developed for its new Google-powered TV -- a wide device with more
than 75 buttons that requires the use of both hands -- looks like the controller for a hobbyist's
model airplane.

Some in the technology industry believe that a better alternative would be to simply replace the
remote with smartphone apps like the one Mr.~Lavoie uses. If you create a specialized smartphone app
to control a TV or set-top box, you can pack the phone's touch screen with virtual buttons in any
configuration you like.

There have already been successful attempts to use smartphones as remotes. Sonos, which makes
Internet-connected stereos, offers a free iPhone application that replicates every feature of its
own \$349 touch-screen remote control. Over half of Sonos customers now use the app, which links to
the stereo over a Wi-Fi network.

Several television manufacturers, like Mitsubishi and Samsung, are following suit with smartphone
remotes, and phone apps are part of both Apple and Google's TV offerings. (The phone-as-remote has
an added advantage: if you lose track of it in the room, you can always call it and make it ring.)

This particular arms race may seem like a sideshow, but the lowly remote is an important piece of
technology that has made its mark on television in its own way. When it was first developed in the
1950s, manufacturers saw the remote control -- the ``Lazy Bones,'' the earliest model, was tethered
to a Zenith TV by a long cord -- primarily as a way to adjust the volume.

But it soon became clear that the remote's true calling was to enable viewers to flip through
channels quickly. Channel browsing simply would not work if the viewer had to keep getting up from
the couch.

Technology giants like Apple and Google, along with a wave of Silicon Valley start-ups, have a
vision for the future that would make channel-surfing seem quaint. Soon, they believe, viewers will
choose from vast pools of video without distinguishing between TV broadcasts and content streamed
over the Internet.

The number of homes with Internet-connected televisions is expected to reach 43 million by 2015, up
from two million at the beginning of this year, according to Forrester Research, a technology
research firm.

``The nature of how users access content is going to evolve, and is starting to evolve,'' said John
Burke, the senior vice president and general manager for broadband at Motorola, which makes set-top
boxes for cable providers. ``Having remote controls that give the user better ability to interact
with the experience is going to be key.''

Some companies are not sold on the idea of the smartphone as the remote of the future. They are
selling a range of remotes armed with full keyboards, touch screens and motion sensors. In November,
Microsoft is planning to release Kinect, a gaming system that will also allow users to move their
hands and use their voices to control movies, sports and video content streamed through an Xbox
console, no remote necessary.

Samsung's high-end televisions can be controlled with a touch-screen device that allows a user to
pull up a virtual keyboard to type or navigate menus. It can also display the same video that is
showing on the TV, so a viewer does not miss anything while heading to the kitchen for a snack.

Samsung says it sees smartphone apps as supplements to this device; it makes its own smartphone line
under the Galaxy brand. Mitsubishi, however, sees apps as a way to replace its own hardware. It
abandoned plans for a remote control with a touch screen this year because it had determined that
such a device would add several hundred dollars to the price of the television.

Instead, the company is developing software for smartphones and tablets.

``If people used an iPad, that would be the best experience. We could never hope to recreate that,''
said Frank DeMartin, Mitsubishi's vice president for marketing.

But using a phone as a remote is not for everybody. When Chris Igoby of Washington bought his first
Android smartphone in October, he immediately found an app that controlled his television. But the
high-tech novelty factor wore off quickly. On top of that, he noticed the toll it took on his
phone's battery.

``It works the exact same; it works perfect,'' he said of the app. ``But I don't understand why
anyone would want to use their phone as a remote.''

For his part, Mr.~Lavoie supplements the iPhone app and the Apple TV remote with a universal remote
for his TiVo and other devices, so he juggles all three. Still, that's better than the eight he
would have otherwise, he says.

``If I can get down from eight to three, I think I'm doing pretty good,'' Mr.~Lavoie said.

Some companies are hesitant to embrace the app idea because they balk at relying on a device or
platform owned by a competitor, said Mr. Mercer at Strategy Analytics.

``It's about control of the technology, and as soon as you're talking about applications on other
devices, you're allowing someone else to be dictating that environment,'' he said.

The issue of control could arise in living rooms as well, once everyone is carrying a personal
remote control. Scott Baldwin, a spokesman for Samsung, said his company's mobile app could give a
houseguest the power to change the channel against his host's wishes.

There is a low-tech solution to that, he noted: ``I wouldn't invite that person over again.''


\section{Awareness: Blood Pressure Check With That Haircut?}

\lettrine{B}{arber}\mycalendar{Nov.'10}{02} shops often serve as a pipeline for health information
in African-American communities. Now, a study reports a striking success: when barbers checked their
male patrons' blood pressure on every visit, the men were far more likely to see a doctor and get
high blood pressure under control. (There was also a financial incentive: a free haircut for those
who returned with a prescription.)

The study, published on Monday in Archives of Internal Medicine, was conducted at 17 black-owned
barber shops in Dallas County, Tex., over the course of two years, ending in 2008.

Eight shops distributed pamphlets to customers found to have high blood pressure at the start of the
study; nine went much further, offering blood pressure checks and urging hypertensive customers to
see a doctor.

By the end of the study, more than half of both groups had their blood pressure under control. But
the gain was more impressive among those whose barbers checked them at each haircut: 53 percent,
from 33 percent at the start of the study, compared with 51 percent, from 40 percent, for those who
received pamphlets.

Most customers were regulars who came in every two to four weeks, said the study's lead author,
Dr.~Ronald G.~Victor, now of Cedars-Sinai Heart Institute. ``That sure puts the issue on top of your
radar screen,'' he said.


\section{Nokia Taking Less-Flashy Road to Growth}

\lettrine{O}{n}\mycalendar{Nov.'10}{02} Saturday at dawn, hundreds of farmers near Jhansi, an
agricultural center in central India, received a succinct but potent text message on their
cellphones: The current average wholesale price of 100 kilograms of tomatoes was 600 rupees.

In a country where just 7 percent of the population have access to the Internet, such real-time
market data is so valuable that the farmers are willing to pay \$1.35 a month for the information.

What's unusual about the service is the company selling it: Nokia, the Finnish cellphone leader,
which unlike its rivals -- Samsung, LG, Apple, Research in Motion and Sony Ericsson -- is leveraging
its size to focus on some of the world's poorest consumers.

Since 2009, 6.3 million people have signed up to pay Nokia for commodity data in India, China and
Indonesia. On Tuesday, Nokia plans to announce it is expanding the program, called Life Tools, part
of its Ovi mobile services business, to Nigeria.

With 152 million residents, Nigeria is Africa's most populous country and, at 29 percent
penetration, one of its least developed, mobile markets.

While media coverage of the mobile industry tends to focus on the fast-growing, lucrative smartphone
market, 77 percent of all cellphones sold in the third quarter were simpler models, capable of
little more than text messaging.

Two thirds of the globe's 4.6 billion mobile phone users live in emerging markets, where Nokia, with
a 34 percent share, according to Strategy Analytics, is the market leader. By selling valuable price
data at a low cost, Nokia is blending commercial and humanitarian goals to attract the next
generation of upwardly mobile phone users.

``For Nokia, Ovi Life Tools creates tremendous brand loyalty,'' said Wally Swain, a mobile analyst
in Bogot\'a, Colombia, at The Yankee Group. ``Farmers and their families will not want to lose this
capability. No other handset manufacturer pursues anything like this.''

So far, farmers are embracing the service, but it is still too early to say whether it is bringing
concrete benefits to Nokia, said Mary McDowell, the Nokia executive vice president in charge of the
company's nonsmartphone business, which accounts for about 11 of every 13 units Nokia sells around
the world. Most of the subscribers, she said, already had one of the 20 Nokia models are required to
subscribe to the real-time pricing service.

``The premise here is that we will be able to complement good hardware with services that will
attract and create a sticky situation with consumer,'' Ms.~McDowell said in an interview. ``This is
not only good business but also about doing good for the community.''

Anecdotally, Ms.~McDowell said the service appeared to be bringing real benefits to impoverished
farmers, many of whom often fall victim to opaque markets with unscrupulous middlemen. Nokia said
many people using the service are better off.

Dattarey Bhonge, a 27-year-old onion farmer in Barshi, an Indian village 230 miles east of Mumbai,
said he learned through Life Tools that he could earn more by selling his onions at a market in
nearby Solapur. The additional profit allowed him to buy new farm equipment, and his neighbors now
look to him as a source for reliable market data.

``I don't have to go anywhere to find the prices,'' Mr.~Dattarey said in a video provided by Nokia.
``The prices are reliable. I was cheated by my agent. Now he can't cheat me.''

In India, China and Indonesia, Nokia has entered into commercial partnerships with agricultural
extension and weather agencies, which collate, edit, package and translate weather, market news and
pricing data in more than 13 local languages.

In Nigeria, Nokia plans to work with Nimet, the Nigerian meteorological agency, and Namin, the
Nigerian Agriculture Market Information, on its service, which will cost 250 Nigerian naira, or
\$1.75 a month. For another \$1.40 a month, Nigerian mobile users can receive daily texts, with
graphics, on health and disease news, English language training or entertainment and sports news.

In Nigeria, the services are available on two Nokia models, the C1-01 and 2690.

Increasingly, developing economies with small fixed-line telecommunications networks are turning to
mobile operators to disseminate critical information to hard-to-reach rural areas. Last Thursday,
for example, the Colombian government announced a strategy called ``Vive Digital'' that included an
education and information program for small farmers over cellphones.

Mr.~Swain, the Yankee Group analyst based, said he expected Nokia to extend its own agricultural
information programs soon to South America, most likely to Brazil. Ms.~McDowell, the Nokia
executive, declined to comment on expansion plans but said the company was targeting the biggest
markets for the service.

``Nokia's competitors are focused on the small but very lucrative high-end smartphone segment and
the associated application market,'' Mr.~Swain said. Nokia's focus, he said, ``doesn't attract much
attention from Wall Street analysts but is making a tremendous difference to peoples' lives in
emerging markets.''


\section{Intel Takes a Wider Role, Making Chips for Others}

\lettrine{I}{ntel}\mycalendar{Nov.'10}{02}, for the first time, has agreed to manufacture another
company's chips at its most advanced factories. The move thrusts Intel, the largest and wealthiest
chip maker, into the contract manufacturing business dominated by Taiwanese and Chinese companies.

Starting next year, Intel will make chips for Achronix Semiconductor, a small company based in
Silicon Valley that designs a specialized type of microprocessor used to accelerate computing tasks
like shuttling network traffic and encrypting data.

The tie-up, which the two companies plan to make public on Monday, would require Intel to give up
only a small fraction of its manufacturing capacity, but it still is a major departure from Intel's
tradition of keeping its prized chip-making expertise to itself.

``We totally appreciate that our manufacturing techniques and factory network are the envy of the
chip-making industry,'' said Bill Kircos, an Intel spokesman.

Mr.~Kircos played down the scope of the deal and the notion that Intel had become a contract chip
maker. ``For proper perspective, this agreement makes up much less than 1 percent of Intel's total
capacity,'' he said.

But some analysts see the deal with Achronix as an indication that Intel intends to enter the
contract manufacturing business in a more meaningful way.

``I think this is only the beginning of Intel manufacturing for others,'' said Gus Richard, a
microprocessor industry analyst with Piper Jaffray.

Intel has long held a manufacturing lead over its competitors in the PC and server chip markets,
building chips with smaller components before rivals do. Those smaller pieces result in faster,
cheaper, lower-power products. Historically, Intel has kept its collection of multibillion-dollar
chip fabrication plants, or fabs, to itself to build the chips it designs in-house.

Meanwhile, a few other companies, most of them located in Taiwan and China, have specialized in
building chips that other companies design. These companies engage in a fierce, expensive battle to
attract customers like Apple, Nvidia and Advanced Micro Devices.

Mr.~Richard said that Intel had struggled to build a major business beyond PC and server chips, and
that working with other designers could help it simultaneously learn new markets while offsetting
the cost of its factories.

``Manufacturing is their crown jewel, and they're finding new ways to monetize it,'' he said.

He and other analysts characterized the move as a mere dip of the toe into accepting business from
other chip designers. Achronix is a minor player in the relatively modest-size market for F.P.G.A.,
or field programmable gate array, chips.

These chips fill a niche because they can be tweaked to push through specialized tasks at faster
rates than the general purpose chips founds in PCs. Companies use these customizable chips to speed
up things like routers and switches, and systems that handle digital signals and video.
Increasingly, Wall Street has turned to them to speed their financial calculations, while oil and
gas companies use the chips to power through geological data.

The worldwide market for these chips is about \$3 billion, about one-tenth of the size of Intel's
annual revenue. Intel does not currently compete in the market.

By branching into manufacturing, even at a modest level, the company can gain some experience in
building products for other designers, said Joseph Byrne, a chip industry analyst with the Linley
Group. He said that came with challenges because Intel for the first time would need to meet the
needs of external customers.

Mr.~Byrne added that the move had the chance to heighten competition in the F.P.G.A. market, is
currently dominated by two companies: Xilinx and Altera. Both companies rely on contract
manufacturers for the chips they design.

John L.~Holt, the chief executive of Achronix, said that tapping Intel as a manufacturing partner
should allow the company to make faster, cheaper products than its rivals and expand the market for
these chips.

``F.P.G.A.'s have been too expensive because the top ones cost in the \$1,000 range,'' Mr.~Holt
said. ``We will sell them for \$400.''

Mr.~Byrne said Achronix in 2009 had virtually no market share but that it could make up ground
quickly if the chips manufactured by Intel did deliver the expected cost and performance increases.

``This will greatly heighten the threat'' that Achronix poses, he said.

But Mr.~Richard of Piper Jaffray said that Xilinx and Altera remained well positioned and that it
would be a ``long stretch'' to say that
Achronix could quickly catch up.

Intel, on rare occasions, has made chips for other companies in the past, but never at its most
advanced manufacturing facilities.

\section{G.O.P. Makes Gains Toward Narrowing the Gap in the Senate}

\lettrine{R}{epublicans}\mycalendar{Nov.'10}{04} improved their lot in the Senate on Tuesday but
were foiled by Democrats in a few key states with a high number of independent voters in their bid
to capture both houses of Congress.

By adding seats in Arkansas, Illinois, Indiana, Pennsylvania, North Dakota and Wisconsin,
Republicans narrowed the gap with Democrats. They maintained control in states like Florida,
Kentucky and New Hampshire.

But in some states where they had visions of victory, Republicans came up short -- notably in
Nevada, where the majority leader, Harry Reid, eked out a win over Sharron Angle, and in Connecticut
and West Virginia. California, Delaware and Vermont were also triumphs for the Democrats.

In West Virginia, the Democratic governor, Joe Manchin III, defeated the Republican businessman John
Raese for the seat once held by the late Robert C.~Byrd. Mr.~Manchin's victory was widely seen as
vital to Democratic control of the Senate.

``Tomorrow starts the rebuilding of America,'' Mr.~Manchin told a jubilant and relieved crowd in
downtown Charleston, carrying on with his outsider appeal. ``We acknowledge that Washington is
broken,'' he said, pledging to work with any lawmakers who were willing to forget party labels and
``put America first.''

In a closely watched race in Pennsylvania, Pat Toomey, a Republican former congressman, beat
Representative Joe Sestak, a Democrat, by a narrow margin -- 51 percent to 49 percent.

Dan Coats, a Republican who left the Senate more than a decade ago for a career in lobbying,
returned Indiana's seat to his party, while John Boozman became the second Republican from Arkansas
to serve in the Senate since Reconstruction, beating the incumbent Blanche Lincoln. In another
Republican pickup, Gov.~John Hoeven of North Dakota handily defeated Tracy Potter, a Democratic
state senator, to fill an open seat now held by a Democrat.

In Kentucky, the Tea Party candidate Rand Paul prevailed over the state attorney general, Jack
Conway, leaving that seat in Republican control after a nasty midterm battle. In Florida, Marco
Rubio easily beat both a Democrat and an independent to maintain Republican control of the open seat
in Florida.

Mr.~Paul, an ophthalmologist and a son of Representative Ron Paul of Texas, was among the Tea
Party's first upset victories when he prevailed in the May primary by a large margin to fill the
seat vacated by Senator Jim Bunning, who is retiring. Mr.~Paul's general election campaign was often
rocky -- his criticism of President Obama's censure of BP after the oil spill in the Gulf of Mexico
and his questioning of some aspects of the Civil Rights Act drew fire.

But Mr.~Paul, a prolific fund-raiser, benefited from his decision to largely stop speaking about his
positions and, perhaps more important, from the backlash against his opponent, who ran
advertisements questioning Mr.~Paul's Christianity.

In Florida, Gov.~Charlie Crist, a Republican turned independent, was forced to mount a fight against
both sides of the political spectrum in his race against Representative Kendrick B.~Meek, a
Democrat, and Mr.~Rubio whose fiery star power drew support across a wide a swath of demographic
groups, according to exit polls.

In less volatile races, Senator Patrick J.~Leahy, a Vermont Democrat, held on to his seat, and in
South Carolina, the Republican Jim DeMint cruised to victory over the Democrat Alvin Greene, who --
unemployed, indicted and resistant to campaigning -- posed no threat. In Ohio, Representative Rob
Portman won the open Senate seat formerly held by his fellow Republican George V.~Voinovich,
defeating Lee Fisher, the lieutenant governor.

Yet in spite of enormous spending and an aggressive effort, Republicans were unable to make the sale
in several key states, either because those states were far more divided than largely gerrymandered
House districts where Republicans prevailed, or because the party nominated candidates who were
ultimately unable to close the deal with a more heterogeneous general electorate than the one they
faced in the primaries.

In Delaware, Chris Coons, several months ago the Democratic underdog, easily dispatched his unusual
Republican opponent, the Sarah Palin favorite Christine O'Donnell.

Yet Democrats were bedeviled in some states that Mr.~Obama won in 2008. The strong Republican
opposition demonstrated disaffection among the coveted independent voters who broke for Mr.~Obama
that year.

Further, in contrast to the undoing in the House, which grew like a wave across the country, the
Senate races unfolded bit by incredible bit, with a remarkable number too close to call when the
polls opened Tuesday.

Early in the primary season, Utah started the mini-trend of incumbents with deep party roots falling
to upstarts. It continued into Alaska and Pennsylvania, where a Republican and a Democrat,
respectively, were sent packing by their parties. In Alaska, Lisa Murkowski, the Republican and a
foe of Ms.~Palin, had to resort to a write-in campaign.

In Delaware, the story was similar: although Michael N.~Castle was not a Senate incumbent, he was a
respected and long-serving Republican congressman who was expected to prevail over Ms.~O'Donnell and
whichever Democratic nominee emerged. Instead, Democrats were able to maintain a badly needed Senate
seat, one that Vice President Joseph R. Biden Jr.~had occupied for more than 30 years -- and that
they had more or less kissed goodbye several months ago.

Some Democrats paid a price with voters for their role in passing the health care bill, which helped
to build a sense around the nation, stoked to great effect by Republicans, that a Democratic
majority was shoving costly legislation down voters' throats.

Senator Mitch McConnell of Kentucky, the Republican leader, leveraged his savvy with Senate
procedure, which helped slow things down and contribute to the image of a do-nothing Senate, and his
ability to rally his party around his vision to keep Democrats from enjoying Republican support on
legislation.

Across the country, jobs and the economy were foremost among voters' concerns. Nearly half of voters
interviewed in exit polls in Florida said they were worried that their home, or that of a relative,
would be foreclosed on. More than half said they were very worried about the condition of the
national economy over the next year.

In a stinging defeat for Democrats on Mr.~Obama's home turf, Mark Steven Kirk, a Republican and
longtime member of Congress representing Chicago's affluent northern suburbs, won a narrow victory
over his Democratic rival, Alexi Giannoulias, the state treasurer, whom the president had endorsed.

Mr.~Obama, who remains popular in Illinois, campaigned for Mr. Giannoulias as recently as last
weekend. And that was seen by many political observers here as a boon to Mr.~Giannoulias in a
contentious, nail-biting race that had been an official tossup for weeks.

``Losing is not easy,'' said Mr.~Giannoulias, who lost his composure for a moment during his remarks
at a downtown Chicago hotel. ``It's not something you expect, which is probably why I didn't write a
speech.''

\section{Tight Deadline for New Speaker to Deliver}

\lettrine{I}{n}\mycalendar{Nov.'10}{04} leading his party to midterm triumph, Representative John A.
Boehner, the next speaker of the House, is not at the endgame. He is at the beginning of the next
and harder fight.

Relying on his decades of experience with the inner workings of the House, Mr.~Boehner, of Ohio, now
has less than two years to show that the Republican Party is the antidote to what ails Washington,
with a discordant caucus, a stagnant economy, a hostile White House with veto power and the long
shadow of 1994 all looming before him.

His promises on behalf of the new House majority -- reducing the size of government, creating jobs
and fundamentally altering the way the Congress conducts its business -- are mostly as lofty as they
are unspecific, and his efforts to legislate them into reality must be done with ambitious upstarts
within his own party and a fresh crop of Tea Partiers, some of whom seem to believe that it is they,
not he, now running the show.

The demands on Mr.~Boehner from voters are many and not all consistent. There is a craving, polling
shows, to see the current system upended, but preferably without gridlock or rancor. Voters want
federal spending curtailed, but jealously guard costly entitlements. They angrily reject what is,
but have no clearly articulated vision for what should be.

Indeed, Mr.~Boehner and his party were delivered no clear mandate from voters, who, polls suggested,
were rejecting a policy agenda more than they were rallying around one. One demand resonated loudly:
the reduction of federal spending immediately, a daunting goal. Yet, among the first things that
Mr.~Boehner has said he will seek to accomplish are reversing cuts to the Medicare program and
extending the expiring Bush-era tax cuts, steps that are hard to reconcile with a commitment to
reining in the national debt.

Mr.~Boehner, who will become second in line to the presidency in January, has responded to the
contradictory forces that led to Republican victory with equally mixed messages.

He has given speeches about inclusiveness, then written Twitter messages denouncing compromise. He
is specific about the amount of spending cuts he seeks -- \$100 billion -- but says little about how
he will get there. In speeches during a whirlwind tour of Ohio over the weekend, he promised things
would be ``different'' in Washington, but then returned to the two-year theme of denigrating
President Obama.

The best guiding documents for understanding what Mr.~Boehner seeks to accomplish are the House
Republican policy document titled ``A Pledge to America'' and his recent speech to the American
Enterprise Institute in Washington. The former is a short collection of goals, some of them
near-impossible in the near term (like ending government control of Fannie Mae and Freddie Mac, the
giant mortgage companies) and a few small-bore but relatively easy to accomplish, like repealing
paperwork rules imposed on small businesses.

Mr.~Boehner and his party have also made it clear that they will immediately try to unravel the new
health care law, either by repealing pieces of it, some of which have already gone into force, or by
using the appropriations process to remove financing from its key provisions.

Mr.~Boehner will also be tested -- perhaps as early as during the lame duck session this month, when
Democrats may be eager to put the issue to bed -- on the tax cut front. Democrats would like to see
tax cuts extended for all but the highest income levels, while Republicans seek to make them
permanent for all taxpayers.

Dragging this battle out would create tremendous headaches for the Internal Revenue Service, and the
issue might be settled in the lame duck session. But Mr.~Boehner learned this year, when he
indicated a willingness to work with Democrats to hammer out a middle ground solution and was
excoriated by some party mates, that not all Republicans cotton to compromise on the issue.

Mr.~Boehner, in his speech and in the pledge, also seeks to overhaul much of the way Congress does
its business, including eliminating sprawling bills that are filled with items that have nothing to
do with the legislation's main intent, as well as requiring bills to cite constitutional authority
and ensuring more bipartisan debate on bills.

There is no doubt that Mr.~Boehner, who was among the so-called Gang of Seven in 1994, when
Republicans took control of the House and then became entangled in missteps and feuds, does not want
a repeat of history.

For instance, in 1994, when Republicans captured both the House and the Senate, Speaker Newt
Gingrich and the Senate leader, Bob Dole, both harbored presidential aspirations and worked less in
tandem than as adversaries. That division between House and Senate Republicans played to President
Bill Clinton's advantage.

In contrast, Mr.~Boehner's relationship with his Senate counterpart, Senator Mitch McConnell of
Kentucky, with whom he has worked closely in the past two years to oppose Mr.~Obama's agenda,
remains strong.

Finally, there is some opportunity for bipartisan lawmaking, as Democrats and Republicans alike
recognize, often privately. The pursuit of alternative energy sources, modest trade agreements,
changes to the Bush administration's signature education act and even adjustments to the tax code
are all within reach.

``The big, interesting question is `What message does Obama take out of this election? What path
does he choose?' '' said Mr.~McConnell, who said spending and debt were potential areas of common
ground.

``There will be plenty of Republicans willing to help reduce both,'' the senator said. ``There are
places where he has expressed interest in the past that would be similar to our own interests.''

\section{Facebook's Initial Crew Moving On}

\lettrine{F}{acebook}\mycalendar{Nov.'10}{04}, the most successful start-up of the last decade, is
only six years old, and an initial public offering is still a way off.

But a number of Facebook's early employees are giving up their stable jobs, free food and laundry
service to build their own businesses. Many of them are leaving as wealthy, either on paper or after
cashing in their ownership stakes to do what they say they like best: start companies.

Dustin Moskovitz, 26, who co-founded Facebook with his Harvard roommate Mark Zuckerberg, left his
job on Facebook's technical staff to create Asana, which makes software that helps workers
collaborate.

Another Facebook co-founder, Chris Hughes, also 26, has started Jumo, a social network for ``people
who want to change the world.''

Dave Morin, formerly the senior platform manager, is building Path, a still-secretive venture, while
Adam D'Angelo, who was Facebook's chief technology officer, and Charlie Cheever, another senior
manager, set off in 2008 and 2009 respectively to start Quora, a question-and-answer site. More than
half a dozen start-ups can trace their origins to Facebook alumni.

The departures follow a familiar pattern among other Silicon Valley successes like Yahoo, eBay and
Google. After amassing fortunes, early employees start walking out the door.

PayPal's have gone off to start YouTube, Slide and Yelp, and staked Facebook. They are known as the
PayPal Mafia. Google's former employees are called Xooglers. Mr.~Morin, who left Facebook this year,
offered this suggestion: Facebook Society. ``We're social,'' he explained.

But the Facebook Society is slightly different from the earlier alumni associations. The other
serial entrepreneurs usually cashed out before resigning.

These ex-Facebookers are leaving before any I.P.O. of the company's shares. They can do that because
Facebook shares are surprisingly liquid. The rise of exchanges like Second Market and SharesPost
over the last couple of years have allowed shareholders in private companies to sell their stakes
more easily than before. These markets function much like a stock exchanges for publicly traded
companies, although the pool of buyers and sellers is much smaller. Facebook's overall value is
around \$30 billion on the exchanges.

Last year, Facebook helped current and former employees to cash out some of their shares to a
Russian Internet company. Digital Sky Technologies, now known as Mail.ru, agreed to buy up to \$100
million in stock to increase its existing stake in Facebook.

Many of Facebook's alumni are wealthy from stock options they earned while working there. The
Facebook expatriates are not saying who among them is rich on paper only and who has actually cashed
in some holdings. But Mr.~Moskovitz owns around 6 percent of Facebook, according to the book ``The
Facebook Effect,'' and would therefore be worth about \$1.8 billion.

By no means is Mr.~Zuckerberg watching a mass exodus. The number of people leaving has been
relatively small. Larry Yu, a Facebook spokesman, said that the company's early employees tended to
be entrepreneurs at heart, and it was therefore not surprising that they had left to start their own
companies. ``We don't view attrition as a particularly prominent issue for us at this time,'' he
said.

Former Facebookers describe the company as a fabulous training ground. Mr.~Zuckerberg hammered home
the lesson of focusing on the long term by declining to accept ads on the site during its infancy or
to be acquired by other companies.

Fellow colleagues expounded on entrepreneurship. Netanel Jacobsson, who was Facebook's director for
international business development before leaving last year, said the company's start-up culture
inevitably changed as a few hundred employees grew to around 1,700 today. ``Eventually, I felt it
became too big and too corporate, and that's when I decided to leave,'' Mr.~Jacobsson said.

After taking time off to decide what to do, he began advising a social gaming company. He liked the
industry so much that he created a social gaming company of his own, PlayHopper, which is to
introduce its first product this year.

The company, which is financed from Mr.~Jacobsson's pocket, has a dozen or so employees scattered
across the globe. ``I'm back to what I really like and what I'm really passionate about -- the
growth stage of a company, and watching it take off,'' Mr.~Jacobsson said.

Getting the business off the ground at his age -- 40 -- is more complex than for other of Facebook's
spawn, who tend to be much younger, he said. For one thing, he has a wife and three children. ``It's
almost a suicide mission,'' Mr.~Jacobsson said.

Mr.~Morin, 30, said that he had always harbored entrepreneurial ambitions, even before joining
Facebook in 2006. ``My dream was always to start a company,'' Mr.~Morin said. After helping to build
two central pieces of Facebook's service, Connect and Platform, he saw an opportunity in the growing
use of smartphones and decided to capitalize on the trend before it was too late.

In February, Mr.~Morin left Facebook and began working on Path, which is to introduce its service
before the end of the year. He has assembled a team of a dozen employees who work in a high-rise
apartment building in San Francisco.

Early on, Path's team tested a service that enabled users to create and share lists online.
Mr.~Morin said the company had since changed direction, but he declined to offer details.

Facebook's former employees say that their tenure provided them a fantastic network of contacts to
tap into. Although the former Facebook workers do not meet formally, they often ask one another for
advice.

Arranging meetings with venture capitalists or angel investors is also easier when you have Facebook
on your r\'esum\'e. Former colleagues turn out to be some of the most eager investors, much like the
PayPal Mafia, whose members have a reputation for supporting one another's companies.

Matt Cohler, a former Facebook vice president who is now a venture capitalist with Benchmark
Capital, epitomizes Facebook's clubby extended family. He has invested some of his own money and
Benchmark's in several companies founded by former colleagues, including Quora and Asana.

As with many families, Facebook's relationship with its start-up offspring includes some tension.
Facebook is a tough competitor when it sees an opportunity, even if that opportunity is already the
focus of some of its former employees. Quora publicly introduced its question-and-answer service in
June. Facebook followed with a similar service a month later.

Facebook has tried to minimize conflict by having exiting employees agree to no-poaching agreements.

Many former Facebook employees acknowledge the extra pressure to succeed because of their pedigree.
If their companies flop, they know that they will be in the headlines, whereas other start-ups that
fall short may go unnoticed.

``There's a lot of expectations, so the stakes are higher if you fail,'' Mr.~Jacobsson said.

\section{In Silicon Valley, Andreessen Horowitz Starts Second Fund}

\lettrine{L}{ast}\mycalendar{Nov.'10}{04} year, when Marc Andreessen set up shop on Sand Hill Road,
the tree-lined home to Silicon Valley's venture capital firms, he was already a big name.

A Midwestern transplant, Mr.~Andreessen was a founder of Netscape, which made the first popular Web
browser, and Opsware, which Hewlett-Packard bought for \$1.6 billion. But he wanted to prove that he
could become one of the storied venture capitalists who invest in the next big thing.

In 16 months, Mr.~Andreessen's firm, Andreessen Horowitz, which he started with Ben Horowitz, also a
founder of Opsware, has earned a solid reputation among entrepreneurs because it helps founders run
their companies. It has also managed to break into the top ranks of venture capitalist firms by
investing in some of the most competitive deals, like Foursquare and Zynga.

On Wednesday, Andreessen Horowitz cemented that status when it announced that it has raised \$650
million for its second fund. The amount is unusual and all the more remarkable because the firm is
so new.

Although it is too early to judge the firm's financial success, Andreessen Horowitz represents a new
breed of venture capitalist that is financing new kinds of start-ups. These firms are shaking up an
industry in need of change because returns for the decade ended in June were negative 4.2 percent.

Venture capitalists in Silicon Valley are finding that the competition for the best deals is again
highly competitive. The valuations of start-ups are soaring, up fourfold in the last year,
Mr.~Andreessen said, which means that investors have to pay more to buy pieces of companies.

The fiercest competition is for very early-stage, or seed, investments in entrepreneurs just
starting out, which is typically for relatively small amounts of money -- \$25,000 to \$200,000.

This kind of investing has been growing rapidly while venture investing over all has slowed. In the
three months that ended in September, overall financings shrunk 7 percent, to \$4.8 billion, from
the same period a year earlier, but investments in companies raising money for the first time
ballooned 60 percent, to \$1.2 billion, according to PricewaterhouseCoopers and the National Venture
Capital Association.

Mr.~Horowitz and Mr.~Andreessen are two of the major players in this new wave of early-stage
financing, known as super angel investing. Unlike typical angel investors, who invest their own
money in fledgling companies as a hobby, super angels invest their money and other people's money as
a full-time job.

The two are also simultaneously at the forefront of a second trend -- super angels going pro and
morphing into venture capitalists by also investing in more mature companies and building a firm
instead of investing alone.

These V.C'.s, which also include firms like Floodgate, First Round Capital and True Ventures, could
reshape Silicon Valley. Meanwhile, established venture capital firms like Greylock Partners and
Kleiner Perkins Caufield \& Byers are latching onto the trend by starting seed investment funds
within their firms.

Mr.~Horowitz and Mr.~Andreessen decided to expand into a venture capital firm last year, after they
had invested \$4 million into 45 start-ups, among them Twitter; Qik, a live mobile video company;
and Aliph, the maker of Jawbone headsets.

Without an office or staff, they were spending their days holding meetings in the lounge of the
Rosewood Sand Hill hotel, a hotspot for tech types, ``ordering club soda and peanuts from the same
waitress for seven hours,'' Mr.~Andreessen said. So they raised \$300 million from outside investors
and opened an office next door to the Rosewood.

They envisioned their firm as one serving as a training camp for start-up executives. They sought
companies with the technical founder still in charge. But they wanted that person to learn how to be
a chief executive, in large part because founders are more willing to sacrifice short-term gains for
long-term ideas, they said.

The philosophy is evident from the moment someone walks into Andreessen Horowitz's lobby, which is
filled with computer programming guides and biographies of tech entrepreneurs -- a touch in keeping
with the firm's carefully crafted image.

``Conventional wisdom in venture capital has been that when the business has to scale, you bring in
the professional team,'' Mr. Horowitz said. ``We think it's easier to develop a founder into being a
great C.E.O.'' He pointed to companies where the founders remained in charge, like Amazon.com,
Microsoft and Intel.

Still, Andreessen Horowitz does not entirely eschew executive shake-ups. Several of its portfolio
companies, including Skype and Digg, have new chief executives since it invested.

The two financiers help founders learn to be executives by writing long blog posts on topics like
how to hire the right executives or how to lay people off, and talking them through issues like
confrontations with employees or meltdowns from stress.

The key, they said, is that unlike other venture capitalists with finance backgrounds, they have
worked as executives. So has their third investing partner, John O'Farrell, previously an executive
at the smart grid company Silver Spring Networks.

They are also staffing their firm with people who focus on recruiting and marketing products from
start-ups.

While other venture firms also do this, some venture capitalists have earned a poor reputation for
paying lip service to the idea that they take a hands-on role at companies when they do nothing more
than show up for board meetings.

At Andreessen Horowitz, for instance, six staff members visit college campuses to meet promising
engineering students and get to know executives in the field, building a network that portfolio
companies can pluck from. In the last year, they have placed 73 people in companies.

But the firm, and others who have hired experts like recruiters and marketers to assist their
portfolio companies, are waiting to see whether the tactic works. After all, in many cases the
personal advice from people like Mr.~Andreessen is the main draw.

That might also be the biggest key to the firm's rapid rise in Silicon Valley. ``Their names carry
weight in and of themselves,'' said Ted G. Wang, a partner at the law firm Fenwick \& West who
focuses on start-ups and venture capital. ``Who wouldn't want to have Marc or Ben's imprimatur on
your company?''

\section{BMW Raises Profit Forecast After Strong Quarter}

\lettrine{B}{ayerische}\mycalendar{Nov.'10}{04} Motoren Werke, maker of BMW as well as Mini and
Rolls-Royce cars, raised its profit forecast for 2010 Wednesday after reporting a strong profit for
the third quarter thanks to soaring sales in China and a rebound in the United States.

BMW, based in Munich, said that net profit in the third quarter was \texteuro874 million, or \$1.2
billion, compared to \texteuro78 million a year earlier, when the global automobile industry was in
crisis. Revenue in the quarter rose 36 percent, to \texteuro15.9 billion, as sales in the United
States rose 9 percent to 70,657 vehicles. In China, BMW unit sales rose 91 percent.

BMW executives cautioned that the outlook for the global economy and capital markets is still
uncertain, but said that the recovery in auto sales appears to be sustainable. ``We don't see any
change in the way markets are developing,'' Friedrich Eichiner, the chief financial officer, said
during a conference call with reporters. Profit also benefited from lower costs for production
materials, he said.

The company said it would earn a 7 percent return on operating profit from car sales for the full
year, compared to a previous forecast of 5 percent. But BMW shares dipped less than 1 percent in
early trading after investors judged the profit target to be insufficiently ambitious.

``The guidance is low,'' Adam Hull, a London-based analyst with WestLB, who has a ``reduce'' rating
on the shares, told Bloomberg News. ``They should be able to better.''

BMW is profiting from strong sales for its redesigned 5 Series model, launched earlier this year, as
well as the top-of-the-line 7 Series. Sales declined in the quarter for smaller cars including the 1
Series and Mini brand, which are older model lines.

The BMW chief executive Norbert Reithofer said that the company will continue to invest in making
smaller cars that buyers are willing to pay extra for.

``We have to become stronger in smaller autos,'' he said. ``We will do that with premium cars.''

Mr.~Reithofer said that, although U.S.~sales are recovering from a severe recession, it will still
be several years before the market for premium cars reaches pre-crisis levels. BMW could outperform
competitors because of new models, he said, including a new edition of the X3 S.U.V. due out by the
end of the year.

``BMW has an advantage from new models,'' Mr.~Reithofer said. ``That will give us further
momentum.''

BMW has expanded production capacity at its plant in Spartanburg, South Carolina, in part to
insulate the company from fluctuations in the dollar-euro exchange rate. In the United States, BMW
is also benefiting from fewer non-performing car loans as the economy improves, Mr.~Eichinger said.

\section{Shanghai Expo Sets Record With 73 Million Visitors}

\lettrine{W}{hen}\mycalendar{Nov.'10}{04} city officials here promised the biggest and best World
Expo ever, they were not just blowing smoke, as Tao Renran and 60 co-workers at a state-run garment
factory found out recently when they were asked to visit this year's Shanghai World Expo.

That odd request, they said, quickly became a threat. ``We were required to come, otherwise, they
said, they would cut our wages,'' the 46-year-old Ms.~Tao said last week, after traveling eight
hours by bus to get a one-day glimpse of the Expo.

Ms.~Tao and her co-workers had lots of company. According to tourism experts, state employees and
government bureaucrats from virtually every part of the nation were ordered to pile onto buses,
trains and planes and head to the Expo 2010 in Shanghai, this year's singular national event, which
ended on Sunday.

State-run tourist agencies had travel quotas, and state companies handed out free vouchers good for
a one-day visit, all in the hopes of helping pump up the numbers.

``I'm in charge of encouraging 5,000 workers to get on this Expo trip organized by our company,''
Chen Hao, 23, deputy chief of the labor union at a state-run steel mill in Shanxi Province in
northern China, said last week, after posing for photographs near the China Pavilion. ``I got free
tickets from the travel agency.''

This government campaign had a simple but noble objective: helping the six-month-long Shanghai Expo
reach its target of 70 million visitors, which would shatter Japan's Expo attendance record of 64
million, set in Osaka in 1970.

Breaking the record was a matter of national pride, and in a country with a history of mass
mobilizations and state propaganda, reaching the target was not a question of whether but when.

That day arrived on Oct.~24, when Expo attendance eclipsed the 70 million mark, with a week to go.
By the time the Expo closed with a glittery ceremony attended by Prime Minister Wen Jiabao, just
over 73 million visitors had passed through the Expo turnstiles; it is believed to be one of the
biggest events ever staged.

Of course, only 5.8 percent of the visitors -- about 4.2 million -- were foreigners, according to
government data.

Liu Kang, who teaches at Duke University and Jiaotong University in Shanghai, said there was never
much doubt Shanghai would meet its target.

``China always has these quotas,'' Professor Liu said Monday, as the Expo Park was being dismantled
after its 184-day run. ``And if they don't make the numbers, it's not good for those in power.''

Using state resources to achieve such lofty goals is part of the game.

It is known, for instance, that in 2008 the Beijing Olympic torch relay was masterfully
stage-managed for millions of viewers of state-run television here, with crowds bused in to line the
relay route and cheer on the torch bearers.

Soon after the torch runner passed by, the cheering crowds were ordered to get back onto their
designated buses and head to the next location along the route, where they were expected to cheer
for the cameras all over again.

Shanghai officials had even bigger ambitions for the Expo, beginning with the clearing of two square
miles for the Expo site by relocating hundreds of factories and moving tens of thousands of
families.

When the Expo opened on May 1, a sprawling multibillion-dollar global theme park came to life with
hundreds of national pavilions, including Saudi Arabia's spectacularly expensive IMAX-equipped
pavilion and Britain's design gem, known as the ``seed cathedral.''

Long before the Expo got under way, Shanghai authorities predicted that it would largely be attended
by a domestic audience. Drawing huge crowds would be easy, organizers said, because only a small
percentage of China's residents get to travel overseas.

But hopes for a record-breaking effort seemed in doubt after visitor numbers dipped to 131,000 a day
on May 3. That was far below the 380,000 daily average organizers said was necessary to break the
record.

Soon after, Expo officials reminded the media that Shanghai's 20 million residents would each be
given a free one-day Expo pass. The city also started a promotional blitz on the nation's state-run
networks. State travel agencies were pressed to deliver on their Expo quotas. And they did.

``We had to entertain lots of government tourists,'' said Ni Ni, a spokeswoman for the Jiangxi
International Travel Agency in Jiangxi Province, in western China. ``We arranged for a group of
1,000 for our local state-owned company. These kinds of trips are all covered by the government.
Each travel agency partnering with the Expo has a quota. As far as I know even the very remote areas
like Anhui, Henan and here in Jiangxi Province have all surpassed the quota.''

By early summer, it was clear that Shanghai was on track. Millions of schoolchildren began arriving.
And overflow crowds jammed the Expo grounds. Entire villages of farmers, sometimes wearing matching
Tang Dynasty-era nylon jackets, camped out.

Still, visitors to the Expo were hardly lemmings being pushed solely by government handouts. There
seemed to be a genuine interest and curiosity about seeing a World Expo, and exploring a range of
pavilions and exhibition halls.

With long lines at the American, French, German and British pavilions, which were among the most
elaborate, many visitors opted for those of Slovakia, the Maldives and North Korea (which featured a
fountain, Korean folk songs and an English sign that said ``Paradise for People'').

Plastic ferns, artificial flowers and department store mannequins dressed in national garb did not
deter visitors from fighting for positions at dozens of smaller pavilions and exhibition halls.

``This place has its own uniqueness,'' said Jing Yangfa, a retired Shanghai university professor,
while examining carpets at the exhibition hall created for Tajikistan, a country on China's
northwest border. ``This is really like a visit to Tajikistan.''

In the sweltering heat of August, and in pouring rain two weeks ago, many queues involved three-hour
waits. People lined up for four hours last week to get into Japan's pavilion, dubbed the ``purple
silkworm.'' And the line snaking around Saudi Arabia's pavilion was eight hours long.

Some desperate visitors tried to con their way into the special access line of pavilions by
pretending to be confined to a wheelchair. And there were reports that elderly women were standing
near the entrance gates offering to rent themselves out as Expo escorts for \$25 a day -- a sure way
to pass through the special access line.

Scalpers even began hawking V.I.P. tickets to the Saudi Pavilion for nearly \$150.

By then, of course, Shanghai was well on its way to meeting its target. On Oct.~16, a record 1.03
million visitors jammed the Expo park -- breaking another Japanese record set in 1970. And two days
later, the 70 millionth visitor arrived. Headlines in the state-run news media declared ``70
million-plus reasons to celebrate.''

One of the last visitors to the Expo, on the final day, was Zhao Chengui, 55, who works for a state
power company in Hubei Province.

He said he was leading a group of homemakers to the China Pavilion.

``Our department arranged for 20 honored wives of pioneering workers in our department to come
here,'' he said. ``Some of them have never been to other cities before. The company pays all the
accommodation -- everything.''

The day after the conclusion of the so-called Green Expo -- as it was called because of its emphasis
on sustainable, environmentally friendly cities -- something unexpected happened: pollution levels
in Shanghai jumped to a near six-month high.

Apparently, the state mandates to close neighboring factories had been lifted, creating a post-Expo
haze.

\section{British Military Expands Links to French Allies}

\lettrine{B}{ritain}\mycalendar{Nov.'10}{04} and France signed defense agreements on Tuesday that
promised cooperation far beyond anything achieved previously in 60 years of NATO cooperation,
including the creation of a joint expeditionary force, shared use of aircraft carriers and combined
efforts to improve the safety and effectiveness of their nuclear weapons.

The agreements signed in London by Prime Minister David Cameron of Britain and President Nicolas
Sarkozy of France were a landmark of another kind for two nations that spent centuries confronting
each other on the battlefields of Europe. While neither leader mentioned Agincourt, Trafalgar or
Waterloo, or French victories that included the Norman Conquest in 1066, both stirred a brief whiff
of the troubled history of Anglo-French relations into the mood of general bonhomie.

The agreements envisaged a new combined force available for deployment at times of international
crisis that is expected to involve about 5,000 service members from each nation, with land, sea and
air components, and rotating French and British commanders. The pacts also foresee each nation
alternating in putting a single aircraft carrier to sea, with the vessels operating as bases for
French, British and American aircraft in times of need.

The nuclear agreement was in some ways the most surprising, since it committed the two nations to
sharing some of their most carefully kept secrets. Although the two leaders emphasized that France's
``force de frappe'' and Britain's similar, submarine-based ballistic missile force would remain
separate and under the sole control of each government, they agreed to establish joint research
centers, one in France and one in Britain, to further research on their stockpiles of nuclear
warheads.

The cooperation pact was set to last 50 years and could transform the way the countries project
force, fight wars and compete for defense contracts with the United States. One goal appeared to be
to give the two militaries greater buying power to support the struggling European defense industry.

Mr.~Cameron, who has navigated deep hostilities to European integration and deep skepticism toward
France in his Conservative Party, emphasized the budgetary benefits, saying the agreements would
contribute savings of ``millions of pounds'' to Britain's plan to make deep cuts in its \$60 billion
defense budget.

Previous efforts at military cooperation between the countries have more often faltered than
succeeded. In the late 1990s, Tony Blair, then Britain's prime minister, and Jacques Chirac, then
France's president, promised deeper defense cooperation, but the understanding was undone by
differences over the Iraq war. In both countries, there are significant political forces arrayed
against anything that smacks of too close a military partnership with the age-old foe.

But after the Cameron government took office in May and began pushing for deep defense savings, it
discovered a willing partner in Mr. Sarkozy. Britain and France have the biggest defense budgets in
Europe, together accounting for more than half of all military spending in the 27-nation European
Union. Both governments took care to say that their new cooperation was not intended to isolate
Germany.

The nuclear agreement, carrying faint echoes of Britain's shared role with the United States in the
Manhattan Project, which developed the first atomic bomb, will have the two governments setting up
two joint research centers, one in France and the other in Britain.

The two countries also agreed on a shared program on spare parts, maintenance and training for the
crews of the Airbus A400M military transport aircraft, a costly, overbudget project intended to
challenge American domination of the market for heavy-lift transports. They promised to work
together on a new generation of remotely piloted surveillance aircraft.

Also on the list are shared projects to develop technologies for future nuclear submarines and
military satellites, as well as countermeasures for mines and other antisubmarine weapons.

The high notes struck by the leaders at their news conference were striking.

``Today we open a new chapter in a long history of cooperation on defense and security between
Britain and France,'' Mr.~Cameron said. Mr.~Sarkozy said the agreements showed ``a level of trust
and confidence between our two nations which is unequalled in history.''

For all that, the shadows of Nelson and Napoleon, of Henry V and Joan of Arc, seemed to hang over
the occasion, with both leaders feeling the need to gesture, at least obliquely, to the less
generous attitudes that are common among some of their compatriots.

``I would like to say, contrary to what might otherwise seem to be the case, that the clocks in
France and Britain strike the same hours, precisely,'' Mr.~Sarkozy said.

Mr.~Cameron said: ``It is about defending our national interest. It is about practical, hard-headed
cooperation between two sovereign countries.''

One concern about the new agreements that has attracted criticism among British defense experts
centers on the shared use of aircraft carriers.

With both countries planning to have only one ``flattop'' in their fleets, having them configured
for each others' aircraft has been described by the two governments as extending their ability to
deploy air power, as well saving large sums. Last month, Britain decided to have its future carrier,
due for deployment in 2020, redesigned with the catapult mechanism and arresting gear necessary to
accommodate French and American aircraft.

But British critics have said military operations that require carrier-borne aircraft could be
compromised if Britain has to rely on France's allowing its carrier to be used. The example often
cited is the 1982 Falklands war, when France opposed Britain's reconquest of the Falkland Islands in
the South Atlantic, and Argentina used French-made missiles to sink British ships.

Mr.~Sarkozy described the criticism as outdated. ``Can you imagine France sitting in our armed
chairs and saying, `It is none of our business' ?'' he said.

Likewise, Mr.~Cameron rejected suggestions that Britain would undermine its close military
relationship with Washington. Mr.~Cameron said the Obama administration would welcome the new plans.
``They'd like us to have the biggest bang for our buck that we possibly can,'' he said.

In France, Marine Le Pen, the vice president of the National Front, a far-right party, called
Mr.~Sarkozy the ``gravedigger of General de Gaulle's policy of independence.'' She went on: ``It is
clear that the objective of this accord is to shift our defense to Anglo-Saxon control, and
obviously everyone will understand that behind Great Britain there is, of course, the American big
brother.''

In London, Mr.~Cameron was chided by right-wing tabloids for trusting the French with Britain's
security, but his plans received a generally warm reception in Parliament. James Arbuthnot, a former
Conservative minister who is the chairman of the House of Commons defense committee, told
Mr.~Cameron on Monday that he had ``forgiven the French for taking off the head of my
great-great-great-great grandfather at Trafalgar,'' a reference to a captain who died in the great
naval battle in 1805. Mr.~Cameron said that was just as well, since Mr. Arbuthnot was invited to
lunch with Mr.~Sarkozy on Tuesday. ``It might have been a little bit frosty,'' he said.

\section{Another Election, Another Wave}

\lettrine{T}{wo}\mycalendar{Nov.'10}{04} years ago this week, triumphant Democrats were throwing
around the word ``realignment,'' as in the kind of Democratic majority that could endure for a
generation or more. Wednesday morning, those same Democrats awoke to find that their majority had
not lasted for even another election cycle.

The question that will dominate the conversation among Democrats in the days ahead is how it came to
this, especially since Republicans offered little to voters beyond an emphatic rejection of the
president's policies. Some Democrats believe they fell victim to the inevitable tide of midterm
elections. Others blame the economy, plain and simple, while a growing chorus accuses Mr.~Obama of
failing to communicate the party's successes.

The truth is that all these explanations probably played some role in the unraveling -- though, in
the case of Mr.~Obama's message, the failure may have deeper roots than his critics assume.

Start with the issue of inevitability. It's fair to say that Democrats set themselves up for
something of a letdown in 2010 by performing beyond any reasonable standard in 2006 and 2008.
Democratic candidates in those back-to-back ``wave elections'' picked up dozens of House seats that
had generally trended Republican, including some in states like Indiana and Virginia.

Democrats were bound to lose a significant number of those vulnerable seats during the subsequent
midterm cycle, when voters, having had their catharsis, generally feel some buyer's remorse about
the incumbent president's party.

Then there's the sluggish economy and its attendant unemployment. As recently as last summer, White
House aides were gamely hoping that voters would feel some momentum in the months leading up to the
elections. Absent that, the administration was forced to argue a negative -- that is, to claim that
the stimulus spending and bank bailouts of 2009 had saved many jobs that otherwise would have been
lost.

That brings us to the question of whether Mr.~Obama stumbled in making a case for his proposals to
right the economy and remake the health care system. The answer may be yes, but if Mr.~Obama failed
to communicate the urgency of his agenda to the voters, then that failure may have less to do with
anything he did in 2010 than with his strategy in 2008.

Mr.~Obama ran his presidential campaign much like Republicans ran this one, as a referendum on the
status quo. Sure, there was what Sarah Palin calls the ``hopey, changey stuff,'' and plenty of
standard talk about affordable health care and clean energy, both of which candidate Obama was for.
What there wasn't was much mention of what change would entail, in terms of taxation or deficits.
Ending the war in Iraq and rolling back tax cuts for the wealthy, Democrats said, would pay for
everything.

No doubt Mr.~Obama's team believed there would be plenty of time to make more complicated arguments
after the election, when Mr.~Obama would be able to reassure and cajole the nation from behind the
presidential lectern. But then a couple of things happened to derail that plan.

First, the same wave that carried Mr.~Obama to a resounding victory also brought in the largest
Democratic majorities in Congress since the 1970s. Suddenly, Democrats, especially in the House,
were giddily talking of a liberal renaissance in the land, and they were confident that the
electorate had issued a blanket endorsement for whatever new investments they might see fit to make.
They weren't inclined to sit around while Mr.~Obama went to the voters and asked permission again.

Second, the economic crisis that greeted Mr.~Obama was deeper and more jolting than anyone had
anticipated, and the White House thought it had little choice but to offer major spending proposals
immediately aimed at heading off a depression, even without the benefit of a long public hearing.
Mr.~Obama's advisers believed that voters would thank them for the hundreds of billions of dollars
in new spending once the economy began to rebound, as it inevitably would.

But the economy didn't rebound, and Mr.~Obama, beset by crises and absorbed in negotiations on
Capitol Hill, never did get around to making a sustained public argument for his most divisive
policy choices. The independent voters who had sided with Mr.~Obama by a margin of eight points over
Senator John McCain, according to exit polls in 2008, drifted inexorably into the Republican camp.

It may be the case, as a lot of Democrats now contend, that Mr.~Obama offered an inconsistent pitch
over these last several weeks, jumping from a critique of the Bush years (how Republicans drove the
national car into a ditch, and so on) to an indictment of campaign cash from outside groups. But
it's probably also the case that, by that point, public opinion had hardened. Some 62 percent of
voters in exit polls Tuesday said the country was on the wrong track.

If there is a lesson in all this for both parties, perhaps it's that merely piling up votes on
Election Day doesn't confer on you a mandate for any ambitious agenda -- unless you have presented
the voters with the difficult choices you intend to make. For Republicans, this argues against
overinterpreting the meaning of Tuesday's gains. For Mr. Obama, it probably means that the campaign
for the next agenda begins right now.

\section{Tea Party Comes to Power on an Unclear Mandate}

\lettrine{T}{he}\mycalendar{Nov.'10}{04} Tea Party movement set the agenda in its first midterms,
its energy propelling the Republican sweep in the House and capturing the mood of a significant
chunk of the electorate, with a remarkable 4 in 10 voters in exit polls expressing support for the
movement.

But in the Senate, the effect was exactly what establishment Republicans had feared: While Tea Party
energy powered some victories, concerns about Tea Party extremism also cost them what could have
been easy gains -- most notably in Nevada, where the Senate Democratic leader, Harry Reid, survived
a challenge from Sharron Angle, a Tea Party favorite.

Now, as it tries to make the transition from a protest movement to a power on Capitol Hill, the Tea
Party faces the challenge of channeling the energy it brought to the election into a governing
agenda when it has no clear mandate, a stated distaste for the inevitable compromises of
legislating, and a wary relationship with Republican leaders in Congress.

For many voters the Tea Party has been a blank screen on which they have projected all kinds of
hopes and frustrations -- not always compatible or realistic.

To many in the movement, the singular goal is to stop an expanding government in its tracks, to
``hold the line at all hazards,'' as Jennifer Stefano, a Tea Party leader in Pennsylvania, put it.

But the movement is also animated by a belief that the entire political system has become
disconnected from the practical needs and values of Americans, suggesting that its voting power
stemmed as much from a populist sense of outrage in a tough economic moment as it did from ideology.
What many of its adherents want as much as anything is for the two parties to come together to solve
problems.

That sometimes conflicting mandate was neatly captured by two interviews in Searchlight, Nev.,
hometown of Mr.~Reid, who became the Tea Party's biggest target. ``I want to see gridlock,'' said
Ronald Hanvey, who supported Ms.~Angle. ``I don't want to see any more laws.''

A few months earlier on nearly the same spot, Jeff Church, arriving at a Tea Party rally against
Mr.~Reid, complained equally about the state's Republican senator, John Ensign, and yearned for
bipartisanship. ``Why can't they get along and make some common-sense solutions?'' Mr.~Church asked.

In the Senate, the Tea Party carried to victory Marco Rubio in Florida and Rand Paul in Kentucky.
Still, it cost the Republicans some seats that they had once counted as solid, including one in
Delaware, where Christine O'Donnell, who beat an establishment candidate in the primary thanks to
strong Tea Party support, lost to Chris Coons, a Democrat once considered a long shot.

Even more painful for Republicans was the result in Nevada. Mr.~Reid, once considered the most
vulnerable Democrat, fought off Ms.~Angle, who had made headlines for her controversial and
sometimes eccentric remarks.

House races showed the same win-loss effect.

Looking ahead, the immediate focus for the movement will be on the big legislative issues facing
Congress. But as attention inevitably shifts to 2012, the Tea Party will also have the chance to
exert potentially substantial influence on the race for the Republican presidential nomination, with
a variety of potential candidates, including Sarah Palin and Senator Jim DeMint of South Carolina,
maneuvering to lead it into the next election.

For the Republican Congressional leaders, Representative John A. Boehner of Ohio and Senator Mitch
McConnell of Kentucky, the question is whether the passion of the Tea Party translates into an
agenda that can drive legislative progress in a divided capital, or whether it becomes a prod to
block Mr.~Obama and his party at every turn. They want to keep the movement's energy alive through
the next presidential election -- but not fall captive to it.

A year spent observing the contours of Tea Party America revealed an uneasy alliance within it,
between those who came to the movement with unswerving ideology, generally libertarian, and those
who say they came to it more out of frustration and a desire to feel that they were doing something
to move forward when the country seemed stuck.

The much-mocked sign at a health care town hall last summer, ``Keep your government hands off my
Medicare,'' suggested how many Tea Party supporters had come to the movement without thinking
through the specifics. While the more ideological Tea Party supporters embrace ideas like phasing
out Social Security and Medicare in favor of private savings accounts, most do not.

And just as Tea Party supporters do not always agree on what the agenda is, most Americans disagree
with many of the goals proclaimed by Tea Party candidates.

While incoming Tea Party lawmakers like Mr.~Paul have advocated sharp across-the-board cuts in
federal spending, a Pew Research Center poll last week found that a plurality of Americans
disapproved of a proposal to freeze all government spending except the part that goes to national
security. A majority disapproved of permanently extending the Bush-era tax cuts on incomes greater
than \$250,000.

A New York Times/CBS News poll last month similarly found opposition to raising the retirement age
or reducing Social Security or Medicare benefits for future retirees. And a plurality of voters
disagreed with what is perhaps the Tea Party movement's most widely supported goal: repealing the
health care overhaul passed in March.

Even where the public agrees on parts of the Republican-Tea Party agenda, there are important
qualifiers. A slight majority in the Pew poll approved of changing Social Security to allow private
accounts, but only by the same margin as when President George W.~Bush advanced the cause in 2005,
only to see it fail when people read the fine print.

The most ardent Tea Party activists expect Republicans to hew to their desires.

Even before polls closed Tuesday, FreedomWorks, the libertarian advocacy group that has helped shape
the movement from its earliest hours, put out a press release declaring, ``The success of the G.O.P.
will not merely benefit from the Tea Party vote, it will depend on it.''

David Adams, a Tea Party activist in Kentucky who ran Mr.~Paul's primary campaign, said, ``I'm
hoping for a lot of fireworks in Washington over who takes control of who.

``If Republican leaders think for a minute that they're going to suck us in and continue business as
usual,'' he said, ``they're wrong.''

Like many other Tea Party supporters, Mr.~Adams said the biggest goals are to balance the budget and
reduce the national debt. And on those, it was unclear where he was willing to compromise. He
expects lawmakers like Mr.~Paul who campaigned on a promise to balance the federal budget within a
year and pass a balanced budget amendment to the Constitution to put forward proposals to do so. It
would not be enough, Mr.~Adams said, to reach an agreement to, say, balance the budget over five
years.

Mr.~Adams's list also includes changing Social Security and Medicare, cutting the military budget,
replacing the income tax with a flat tax -- all ideas that have been raised and voted down, firmly,
before. As for repealing the health care legislation? ``I can't imagine much compromise there,'' he
said.

``You look at what the Tea Party has done over the course of this last year, we've changed the shape
of the debate,'' he said. ``We have major candidates who are winning races by large margins by
talking about making government smaller. We're getting very, very close to put-up-or-shut-up time.''

There was not much room for compromise on Ms.~Stefano's list, either. She wants the health care bill
and the estate tax to be repealed, and the Bush-era tax cuts to be made permanent.

She warned Republicans not to read too much into the Tea Party support for Republicans. ``They
should not see it as a mandate for their agenda,'' she said. ``It is a repudiation of the president
and Nancy Pelosi's view of America. As far as I'm concerned, as of Nov.~3, the Republicans are on
probation.''

But as much as the Tea Party allowed the Republicans to win in enthusiasm, it will still have a
relatively small caucus in the House and the Senate. With control of Congress split, Republicans
will have to work with Democrats to get things done. Tea Party lawmakers who refuse to go along may
find they become irrelevant -- certainly not the goal of all the noise and passion of the last two
years.

\section{In Asia, Response to Elections Is Muted}

\lettrine{R}{eaction}\mycalendar{Nov.'10}{04} across Asia to the U.S.~elections on Wednesday was
muted as attention focused on the planned tour by President Obama to the region, scheduled to begin
Friday in India.

Republicans captured control of the House of Representatives on Tuesday and expanded their presence
in the Senate. Additional Republican gains in state governorships reflected the voter discontent
with Mr.~Obama's polices on the country's economy, health care and government spending.

The election results provided few surprises, stirring little official response from governments or
media in Asia. Political observers across the region said they did not expect much change in foreign
policy.

``There has been very little time to digest the U.S.~election results,'' said Wimoar Witoelar, a
political commentator in Indonesia. ``The election results won't alter relations between the two
countries, and there is also a realization that Obama might rebound in 2012. The economic problems
in America are quite severe.''

Markets across Asia were up on Wednesday. Gains were small, however, as anticipation grew for
details to emerge from the Federal Reserve's meeting. The Fed was expected to engage in another
round of Treasury bond purchases to help stimulate the U.S.~economy.

In Japan, the government expressed concern that the Republican victories would force Mr.~Obama to
focus increasingly on domestic politics at the expense of foreign policy. Japan is confronting a
stalled economic recovery and is involved in territorial disputes with China and Russia over
different groups of islands. Earlier this week, China rejected an American offer to help mediate its
dispute with Japan over the group of islands known as Senkaku in Japanese and Diaoyu in Chinese.

``From this point on, the Obama administration will have no alternative but to spend most of its
energy in tackling internal problems,'' an unidentified government official told the Kyodo news
service.

Mr.~Obama begins a 10-day tour in four Asian nations, beginning Friday in India. It will be the
president's longest trip abroad yet and will include stops in Indonesia, South Korea and Japan.

Mr.~Obama will not visit China, but the Asian giant will play prominently in scheduled talks in the
region. Each of the countries the president is visiting has shared worries of the growing political,
economic and military prominence of China. Additionally, various Asian nations have had recent
disputes with China over unresolved territorial claims in the region and Beijing's support of North
Korea.

China played prominently in the American campaigns, with several candidates using TV campaign ads to
portray the country as partly responsible for economic woes in the United States.

Still, the Republican takeover of the House will not affect the overall relationship between the
United States and China, said Shi Yinhong, a professor of international relations at Renmin
University. There will be calls among new Republican lawmakers to press China on revaluing its
currency and the trade imbalance between the two countries, but the overall tone from the White
House will note change, Mr.~Shi said.

``The situation between the two countries is already bad,'' Mr.~Shi said. ``The anger in the House
of Representatives toward China will increase, but the overall influence on policy will be
limited.''

The House of Representatives has passed a law that would consider an undervalued Chinese currency as
an unlawful subsidy. The bill is before the Senate, but it is uncertain if it will pass.

In an editorial published on Wednesday, the state-run China Daily warned American legislators from
taking further action on the bill.

``To put it simply, if China and the U.S., the world's two largest trading powers, enter a
full-blown trade and currency war, the global trading system will suffer greatly,'' the editorial
said.

\section{Shift in Washington Stirs Economic Jitters Abroad}

\lettrine{A}{s}\mycalendar{Nov.'10}{04} Republicans prepare to assert new authority in the U.S.
Congress following the midterm elections Tuesday, the United States' allies overseas are concerned
that the political upheaval in Washington may pose fresh challenges to the global economy.

Despite pledges to curb government spending and the huge U.S.~budget deficit, Republicans are
expected to address anxiety over unemployment and flagging growth by pushing hardest for an
extension of the income tax cuts for everyone, including the rich that were passed during the
presidency of George W.~Bush -- a move that would add to the deficit and, by extension, further
weaken the U.S.~dollar.

``The rest of the world, including Asia, is looking at the United States and seeing no real
effective policy measures in bringing the economy back on track,'' said Bart Van Ark, the chief
economist at the Conference Board, which measures U.S.~economic indicators. ``That is making the
U.S.~lose its legitimacy in the global economic community as a leader in terms of providing
solutions.''

Maintaining taxes at their current relatively low rates could help lift consumer spending in the
United States, while a cheaper dollar would make U.S.~exports more competitive. But analysts said
those fixes would be only temporary and would be unlikely to reverse the waning of America's
economic might at a time when emerging markets have eclipsed industrial nations as the major drivers
of the global economy.

A weaker dollar might also dampen a recovery in European countries that have adopted harsh austerity
measures under pressure from financial markets to rein in their excessive debts.

After the administration of President Barack Obama pushed through costly changes to health care and
to the financial system, voters signaled that they want U.S.~government spending curtailed.
Representative John A.~Boehner, a Republican from Ohio and most likely the next speaker of the
House, reiterated a pledge Tuesday to reduce the size of government, create jobs and change the way
that Congress does business.

But that is no easy task: Voters also want to keep expensive entitlements and are hoping Republicans
can reverse cuts to the Medicare program and extend the expiring Bush tax cuts so that taxes do not
rise in a weakened economy at the end of the year.

But those moves, if enacted into law, would make it harder, not easier, to keep commitments to curb
the national debt and the budget deficit. Yet they probably would not create as many jobs as would
equally costly spending on big initiatives like infrastructure projects, analysts said.

From the perspective of those outside the United States, ``Republican claims to fiscal probity are a
little difficult to buy into,'' said Simon Tilford, the chief economist at the Center for European
Reform in London. ``What they're advocating would probably increase the deficit rather than effect
the dramatic reduction which they claim they want to bring about.''

There is also the risk that Congress, which will be divided next year between Republican control of
the House and a fragile Democratic majority in the Senate, will dissolve into gridlock. That would
leave the task of supporting a U.S.~economic recovery almost entirely to the Federal Reserve.

Without a healthier U.S.~economy, the shift in the economic balance of power toward emerging market
nations is expected to continue. Even if U.S.~growth were to remain at current levels, analysts
said, faster-growing markets like China, Brazil and India are now the primary drivers of the global
economy, a trend underscored in a report by the International Monetary Fund last month.

It is unclear just how much the lame-duck Congress still controlled by Democrats can push through
before the new Republican-led House takes over. But the pressure is now on the Obama administration
to cut a deal on taxes before the end of the year, even though there is no consensus on where to cut
spending, analysts said.

Moreover, if the current income tax rates are extended for the wealthy as well as for middle class
taxpayers for the next few years, the action could add an estimated one to two percentage points to
the deficit as a share of overall economic activity, according to Klaus Deutsch, a senior economist
with Deutsche Bank Research in Berlin.

``So the need to adopt a more comprehensive fiscal adjustment package within the next two to three
years becomes even more important,'' he said.

If Washington ends up adding to the deficit rather than reducing it, one result could be a further
weakening of the U.S.~dollar against the euro, the pound and other floating currencies. The dollar
has slumped more than 15 percent against the euro alone since June on concerns about the U.S.~fiscal
situation and on expectations that the Federal Reserve will announce new efforts Wednesday to pump
more money into the weakened economy by buying Treasury securities, a process known as quantitative
easing.

In times of full employment, a bigger deficit can be offset by tighter monetary policy and higher
interest rates, said C.~Fred Bergsten, director of the Peterson Institute for International
Economics in Washington. ``But now with rates at zero and the Fed about to embark on a quantitative
expansion, that will mean a bigger deficit and a weaker dollar, and that's the set of issues that
would set off the biggest alarm overseas -- especially among the Europeans.''

In the European Union -- the world's largest economic bloc -- there is concern that a weaker dollar
would make it harder to pursue an export-led recovery at a time when countries like Britain, Greece
and Ireland have embraced austerity in an effort to straighten out their finances.

Even if the dollar continues to weaken, helping U.S.~exporters, most political and economic experts
expect lawmakers from both parties to keep up pressure on China to strengthen its currency. That
would make it even harder for the Obama administration to support a more cooperative approach toward
bringing the Chinese currency more into line with that country's strong economic fundamentals.

U.S.~manufacturers have long complained that China is keeping the renminbi artificially weak, making
it harder for U.S.~exports to compete in the global marketplace.

The currency issue is part of a larger skirmish over trade, said Kenneth S.~Rogoff, a Harvard
University economist and a former I.M.F. chief economist.

``If the Chinese were to give a little, that would buy a lot of time on trade policy,'' he said.
``Asians especially are extremely nervous about seeing the U.S.~slap a tariff on Chinese imports or
some such aggressive approach, not only because Republicans are taking the House, but because the
economy is very weak.''

Until the dust settles, however, ``unpredictability is the word of the day,'' Mr.~Rogoff added. ``We
have a powerful and volatile new force,'' he added. The ``rest of world will be worried about
that.''

\section{In Republican Victories, Tide Turns Starkly}

\lettrine{S}{omewhere}\mycalendar{Nov.'10}{04} along the way, the apostle of change became its
target, engulfed by the same currents that swept him to the White House two years ago. Now,
President Obama must find a way to recalibrate with nothing less than his presidency on the line.

The verdict delivered by voters on Tuesday effectively put an end to his transformational ambitions
and left him searching for a way forward with a more circumscribed horizon of possibilities. Facing
a hostile House with subpoena power and a diminished majority in the Senate, he will have to figure
out the right blend of conciliation and confrontation to reassert authority and avoid defeat in
2012.

The most pressing question as Mr.~Obama picks through the results on Wednesday morning will be what
lessons he takes from the electoral reversals. Was this the natural and unavoidable backlash in a
time of historic economic distress, or was it a repudiation of a big-spending activist government?
Was it primarily a failure of communications as the White House has suggested lately, or was it a
fundamental disconnect with the values and priorities of the American public?

``He will read the results carefully and hear the messages the American people are sending,'' David
Axelrod, the president's senior adviser, said in an interview. ``I think he'll do that with care and
with humility. But he's also very centered, and his impulse is to focus on what we do next and how
we respond in a way that's constructive.''

Mr.~Obama remained out of sight Tuesday night, watching the returns in the White House residence
with his family, but he plans to hold a news conference on Wednesday afternoon to call on the
country to put the divisive elections behind it and forge bipartisanship for a new era, aides said.
He will soon invite leaders of both parties to meet to develop a common agenda to fix the economy
and tame the deficit.

But fresh from their victories, Republicans may have little incentive to defer to Mr.~Obama. ``The
American people have sent an unmistakable message to him tonight and the message is: change
course,'' Representative John A.~Boehner of Ohio, the putative new Republican speaker, said Tuesday
night. In the days leading up to the election, Senator Mitch McConnell of Kentucky, the Republican
leader, said ``the single most important thing we want to achieve is for President Obama to be a
one-term president.''

The viability of the new political order in Washington will get a swift test when Congress returns
to town for a lame-duck session this month. Mr.~Obama may have to give ground and agree to at least
a temporary extension of expiring tax cuts for the wealthiest Americans, not just the middle class
as he favors. He will be pressured to show that he is serious about reining in government spending.

Beyond the near term, the president appears likely to turn to more incremental initiatives in
Congress and more aggressive use of his own executive authority to advance his agenda. In a recent
interview, Mr. Obama listed areas where he thought he might be able to work with a new Congress:
reforming immigration, curbing deficit spending and overhauling education.

``The president is somebody who knows he's not going to have his way on these things, that he needs
Republicans and he has the ability to reach out to them,'' said Transportation Secretary Ray LaHood,
the most prominent Republican in the administration.

At the same time, much of Mr.~Obama's time may be spent on the defensive, fending off efforts by
House Republicans to repeal or refuse to finance the health care program he pushed through this
year. Of course, he will still have a relatively free hand in foreign policy, as he will demonstrate
by leaving town on Friday for a long-delayed trip to Asia.

Mr.~Obama finds himself in a similar position to his two most recent predecessors, Bill Clinton and
George W.~Bush, each of whom endured a midterm election that handed Congress to the opposition.
While distinguished by different factors, each of those elections was cast as a referendum on the
president. Thirty-seven percent of voters in exit polls on Tuesday said they cast their vote to
express opposition to Mr.~Obama, compared with 24 percent who said they were trying to support him
-- almost identical to the numbers for Mr.~Bush four years ago.

Mr.~Clinton responded to his party's 1994 losses by tacking to the middle and cutting deals with
Republicans on welfare while outmaneuvering them during a government shutdown. With the Iraq war
effort flailing, Mr.~Bush fired Defense Secretary Donald H.~Rumsfeld the day after the 2006
election, but battled with Democrats over war spending and other issues.

Strategists on both sides said the lessons of the past offered only limited utility. As politically
toxic as the atmosphere in Washington was in the 1990s, the two sides appear even more polarized
today. The Republicans may be more beholden to a Tea Party movement that abhors deal cutting, while
Mr.~Obama has not shown the same sort of centrist sensibilities that Mr.~Clinton did and presides in
a time of higher unemployment and deficits.

``I know President Clinton. President Clinton was an acquaintance of mine. Obama is no President
Clinton,'' said former Representative Dick Armey of Texas, who as House Republican leader squared
off against Mr. Clinton at the time and today is a prime Tea Party promoter. ``Personally, I think
he's already lost his re-election.''

That remains to be determined, but he can expect a rough two years. If nothing else, both
Mr.~Clinton and Mr.~Bush saw what can happen when the other side gets subpoena power. Legitimate
oversight and political fishing expeditions can both take their toll.

``Even when carefully managed, these investigations can be distracting to senior White House
officials,'' said W.~Neil Eggleston, who was a White House lawyer under Mr.~Clinton and later
represented an aide to Mr.~Bush during a Congressional inquiry.

Still, Mr.~Obama wields the veto pen, and his Democratic allies in the Senate will provide a
firewall against Republican initiatives. The possibility of gridlock looms. And in the White House,
there is hope that Republicans descend into fratricide between establishment and Tea Party
insurgents, while Mr.~Obama presents himself as above it all.

Former Representative Tom Davis, Republican of Virginia, said it was hard to see Mr.~Obama finding
common cause with Mr.~Boehner or Mr. McConnell, the Republican leaders.

``Obama's denigrated Boehner and McConnell by name -- not very presidential,'' Mr.~Davis said.
Moreover, both sides will have to answer to partisans on the left and the right with little interest
in compromise.

``There's going to be a lot of posturing to the base,'' Mr.~Davis said. ``I think it's going to be
ugly, at least at first.''

\section{G.O.P. Captures House, but Not Senate}

\lettrine{R}{epublicans}\mycalendar{Nov.'10}{04} captured control of the House of Representatives on
Tuesday and expanded their voice in the Senate, as discontented voters, frustrated about the
nation's continuing economic woes, turned sharply against President Obama just two years after
catapulting him into the White House.

For Mr.~Obama's fellow Democrats, who won majorities in the House and Senate in 2006, it was a
punishing defeat. Republicans picked up at least 60 seats, surpassing their gains in the so-called
Republican Revolution of 1994, and making it the largest sweep of House races since 1948. In the
Senate, Republicans nabbed at least six seats, a more modest gain. The Republican resurgence,
propelled by deep economic worries and a forceful opposition to the Democratic agenda of health care
and stimulus spending, delivered defeats to House Democrats from the Northeast to the South and
across the Midwest.

The tide swept aside dozens of lawmakers, regardless of their seniority or their voting records,
upending the balance of power for the second half of Mr.~Obama's term. A number of ousted incumbents
were centrists, including fiscal hawks in the Blue Dog Coalition, leaving the Democratic caucuses
not only diminished but more liberal.

Still, Senator Harry Reid of Nevada, the Democratic leader, narrowly prevailed and his party hung on
to control by winning hard-fought contests in California, Connecticut, Delaware and West Virginia.
Republicans picked up at least six Democratic seats, including the one formerly held by Mr.~Obama,
and the party will welcome Marco Rubio of Florida and Rand Paul of Kentucky to their ranks, two
candidates who were initially shunned by the establishment but beloved by the Tea Party movement.
``The American people's voice was heard at the ballot box,'' said Representative John A.~Boehner of
Ohio, who is positioned to become the next speaker of the House. ``We have real work to do, and this
is not the time for celebration.''

In an early morning appearance on NBC's ``Today'' show, Mr.~Reid said it was time for the parties to
cooperate. ``The message to America today is that we've got to start working together,'' he said.
``The only way we can have progress is by working together. If that means legislative compromise,
we've got to do that.''

The president, who watched the election returns with a small set of advisers at the White House,
called Mr.~Boehner shortly after midnight to offer his congratulations and to talk about the way
forward as Washington prepares for divided government. Republicans won at least 60 seats, surpassing
the 52 seats the party won in the sweep of 1994.

The most expensive midterm election campaign in the nation's history, fueled by a raft of
contributions from outside interest groups and millions in donations to candidates in both parties,
played out across a wide battleground that stretched from Alaska to Maine.

The Republican tide swept into statehouse races, too, with Democrats poised to lose the majority of
governorships, particularly those in majorpresidential swing states, like Ohio, where Gov.~Ted
Strickland was defeated.

Republicans picked up governorships in at least eight states, and Democrats lost at least nine, as
Lincoln Chafee, a former Republican senator, was elected governor of Rhode Island as an independent.

One after another, once-unassailable Democrats like Senator Russ Feingold of Wisconsin,
Representatives Ike Skelton of Missouri, John Spratt of South Carolina, Rick Boucher of Virginia and
Chet Edwards of Texas fell to little-known Republican challengers.

``Voters sent a message that change has not happened fast enough,'' said Tim Kaine, chairman of the
Democratic National Committee.

The future plans of the House Democratic leadership, beyond a lame-duck session of the current
Congress that is set to begin on Nov. 15, were not immediately clear.

The House speaker, Nancy Pelosi of California, did not immediately say whether she would remain in
the Congress after losing the speakership.

But in a statement about the election results, she was resolute in defending the policies of her
caucus -- despite the evident voter backlash -- and she said Democrats had saved the nation from
economic disaster.

``Over the last four years, the Democratic majority in the House took courageous action on behalf of
America's middle class to create jobs and save the country from the worst economic catastrophe since
the Great Depression,'' Ms.~Pelosi said, adding:

``The outcome of the election does not diminish the work we have done for the American people. We
must all strive to find common ground to support the middle class, create jobs, reduce the deficit
and move our nation forward.''

Republicans did not achieve a perfect evening, losing races in several states they had once hoped to
win, including the Senate contests in Delaware and Connecticut, because some candidates supported by
the Tea Party movement knocked out establishment candidates to win their nominations.

But Republicans did score notable victories in some tight races, like the Pennsylvania Senate
contest, where former Representative Pat Toomey defeated Representative Joe Sestak for the seat now
held by Arlen Specter, the Republican-turned-Democrat.

The outcome on Tuesday was nothing short of a remarkable comeback for Republicans two years after
they suffered a crushing defeat in the White House and four years after Democrats swept control of
the House and Senate.

It gives the party substantial leverage in terms of policy, posing new challenges to Mr.~Obama as he
faces a tough two years in his term, but also for Republicans -- led by Mr.~Boehner -- as he
suddenly finds himself in a position of responsibility, rather than being simply the outsider.

In the House, Republicans found victories in most corners of the country, including five seats each
in New York, Pennsylvania, and Ohio, at least three in Illinois, three in Florida, Tennessee and
Virginia and two each in Arkansas, Colorado and Mississippi.

Throughout the evening, in race after race, Republican challengers defeated Democratic incumbents,
despite being at significant fund-raising disadvantages.

Republican-oriented independent groups invariably came to the rescue, helping level of the playing
field, including in Florida's 24th Congressional District, in which Sandy Adams defeated
Representative Suzanne Kosmas; Virginia's 9th Congressional District, where Mr. Boucher, a 14-term
incumbent, lost to Morgan Griffith; and Texas's 17th Congressional District, in which Mr.~Edwards,
who was seeking his 11th term, succumbed to Bill Flores.

Democrats argued that the Republican triumph was far from complete, particularly in the Senate,
pointing to the preservation of Mr.~Reid, and other races.

In Delaware, Chris Coons defeated Christine O'Donnell, whose candidacy became a symbol of the
unorthodox political candidates swept onto the ballot in Republican primary contests. In West
Virginia, Gov.~Joe Manchin III, a Democrat, prevailed over an insurgent Republican rival to fill the
seat held for a half-century by Senator Robert C.~Byrd. And in California, Senator Barbara Boxer
turned back a vigorous challenge from Carly Fiorina, a Republican.

But Democrats conceded that their plans to increase voter turnout did not meet expectations, party
strategists said, and extraordinary efforts that Mr.~Obama made in the final days of the campaign
appeared to have borne little fruit.

The president flew to Charlottesville, Va., on Friday evening, for instance, in hopes of rallying
Democrats to support Representative Tom Perriello, a freshman who supported every piece of the
administration's agenda, but he was defeated despite the president's appeals to Democrats in a state
that he carried two years ago.

In governors' races, Republicans won several contests in the nation's middle.

They held onto governorships in Texas, Nebraska and South Dakota, and had seized seats now occupied
by Democrats in Tennessee, Michigan and Kansas. Sam Brownback, a United States Senator and
Republican, easily took the Kansas post that Mark Parkinson, a former Republican turned Democrat, is
leaving behind.

Before the election, Democrats held 26 governors' seats compared to 24 for the Republicans, As of
Wednesday morning, Republicans controlled 26, and Democrats just 14, with 9 races still undecided..

In New York, Attorney General Andrew M.~Cuomo, the Democrat, easily defeated the Republican, Carl
P.~Paladino, even as Republicans were expected to pick up seats in the state legislature and the
congressional delegation. In Massachusetts, Gov.~Deval Patrick won a second term.

As the election results rolled in, with Republicans picking up victories shortly after polls closed
in states across the South, East and the Midwest, the House speaker, Ms.~Pelosi, and other party
leaders made urgent appeals through television interviews that there was still time for voters in
other states to cast their ballots.

But the mood in Democratic quarters was glum, with few early signs of optimism in House or Senate
races that were called early in the evening. Surveys that were conducted with voters across the
country also provided little sense of hope for Democrats, with Republicans gaining a majority of
independents, college-educated people and suburbanites -- all groups that were part of the coalition
of voters who supported Mr.~Obama two years ago.

``We've come to take our government back,'' Mr.~Paul told cheering supporters who gathered in
Bowling Green, Ky. ``They say that the U.S. Senate is the world's most deliberative body. I'm going
to ask them to deliberate on this: The American people are unhappy with what's going on in
Washington.''

The election was a referendum on President Obama and the Democratic agenda, according to interviews
with voters that were conducted for the National Election Pool, a consortium of television networks
and The Associated Press, with a wide majority of the electorate saying that the country was
seriously off track. Nearly nine in 10 voters said they were worried about the economy and about 4
in 10 said their family's situation had worsened in the last two years.

The surveys found that voters were even more dissatisfied with Congress now than they were in 2006,
when Democrats reclaimed control from the Republicans. Preliminary results also indicated an
electorate far more conservative than four years ago, a sign of stronger turnout by people leaning
toward Republicans.

Most voters said they believed Mr.~Obama's policies would hurt the country in the long run, rather
than help it, and a large share of voters said they supported the Tea Party movement, which has
backed insurgent candidates all across the country.

The Republican winds began blowing back in January when Democrats lost the seat long held by Senator
Edward M.~Kennedy of Massachusetts, with the victory of Scott P.~Brown serving as a motivating force
for the budding Tea Party movement and a burst of inspiration for Republican candidates across the
country to step forward and challenge Democrats everywhere.

On Tuesday, the president did not leave the grounds of the White House, taking a respite from days
of campaigning across the country, so he could meet with a circle of top advisers to plot a way
forward for his administration and his own looming re-election campaign.

The White House said Mr.~Obama would hold a news conference on Wednesday to address the governing
challenges that await the new Congress.

``My hope is that I can cooperate with Republicans,'' Mr.~Obama said in a radio interview on
Tuesday. ``But obviously, the kinds of compromises that will be made depends on what Capitol Hill
looks like -- who's in charge.''

But even as the president was poised to offer a fresh commitment to bipartisanship, he spent the
final hours of the midterm campaign trying to persuade Democrats in key states to take time to vote.
From the Oval Office, Mr.~Obama conducted one radio interview after another, urging black voters in
particular to help preserve the party's majority and his agenda.

``How well I'm able to move my agenda forward over the next couple of years is going to depend on
folks back home having my back,'' Mr.~Obama said in an interview with the Chicago radio station
WGCI, in which he made an unsuccessful appeal for voters to keep his former Senate seat in
Democratic hands.

There was little Democratic terrain across the country that seemed immune to Republican
encroachment, with many of the most competitive races being waged in states that Mr.~Obama carried
strongly only two years ago. From the president's home state of Illinois to neighboring Iowa,
Wisconsin, Indiana and Ohio -- all places that were kind to the Democratic ticket in 2008 --
Republicans worked aggressively to find new opportunities.

For all the drama surrounding the final day of the midterm campaign, more than 19 million Americans
had voted before Tuesday, a trend that has grown with each election cycle over the last decade, as
32 states now offer a way for voters to practice democracy in far more convenient ways than simply
waiting in line on Election Day.

\section{More Than 100,000 Pay for British News Site}

\lettrine{T}{he}\mycalendar{Nov.'10}{04} News Corporation said on Tuesday that it had gained 105,000
paying customers for the digital versions of The Times and The Sunday Times of London since it
started charging for access to their Web sites this summer.

The company said about half of those additions were regular, active subscribers to the newspapers'
Web sites, iPad application or Amazon Kindle edition. The rest were occasional purchasers. Another
100,000 readers have activated free digital accounts that are included in print subscriptions to the
papers, the News Corporation said.

The company's initiative has been closely watched among media analysts and advertisers because The
Times and Sunday Times are among the first prominent general-interest newspapers to start charging
for their digital content. Other newspapers are also moving to introduce paid services as online
advertising falls short of publishers' hopes that it might someday replace dwindling print ad
revenue.

``These figures very clearly show that large numbers of people are willing to pay for quality
journalism in digital formats,'' said Rebekah Brooks, chief executive of News International, the
London-based arm of the News Corporation that publishes The Times papers.

The News Corporation already charges for access to The Wall Street Journal. The company, controlled
by Rupert Murdoch, also moved another British newspaper, the tabloid News of the World, behind a pay
wall.

The conventional wisdom among media analysts has been that it will be difficult to persuade readers
to pay for general news online, given the panoply of free news available on the Web.

But, some specialty publications in areas like business and finance have had modest success with
paid access.

The Financial Times, for example, says it has attracted 189,000 paying customers for its Web site,
which uses a metered model, giving online readers a limited number of free articles every month
before charges kick in.

The New York Times, which also publishes The International Herald Tribune, has said it plans to take
a similar approach when it begins charging for its Web site next year.

When the News Corporation switched to a paid model, the company estimated that the number of
visitors to The Times and Sunday Times Web sites would drop by 90 percent. In fact, traffic appears
to have fallen by somewhat less. Nielsen, the media audience measurement agency, said last week that
the average number of monthly unique visitors to the newspapers' Web sites from Britain had fallen
by 42 percent, to 1.78 million, in the third quarter, after the charges were instituted.

Many of those visitors do not go beyond the home pages.

But the News Corporation has said the newspapers will benefit despite drawing smaller audiences,
because they can sell more focused advertising, as well as generate new revenue from subscribers.

Jim Chisholm, a newspaper consultant in Lille, France, said the News Corporation's announcement left
a number of questions for advertisers unanswered, including the amount of time that users were
spending on the sites.

``Much as we all want newspapers to succeed and make money, in a market as rough and crowded as the
U.K., The Times pay wall was always going to be a tall order,'' he said.

Another question is what effect the charges are having on the print editions of The Times and Sunday
Times. Paid circulation of the daily paper fell about 3 percent from June through September, to
about 487,000, while sales of the Sunday paper rose by about half of 1 percent, to nearly 1.1
million.

Analysts say the move to paid-for Web sites could also open up other possibilities for the News
Corporation, including bundled sales of online access with subscriptions to British Sky
Broadcasting, a pay-television service with 10 million customers. The News Corporation owns 39
percent of Sky and has proposed buying full control.

\section{Apple Looks to a New Computing Era}

\lettrine{G}{etty}\mycalendar{Nov.'10}{04} Images The new MacBook Air doesn't include a DVD slot.
Remember the floppy disk? I'm willing to bet Steve Jobs does. I'm also willing to bet he remembers
when he killed it.

It was 1998, to be precise, and the murder weapon was the new iMac, a computer that was missing the
then-standard internal floppy drive.

Last month Mr.~Jobs rang the final death knell for another piece of technology: optical discs like
DVDs and CDs.

For this execution, his weapon of choice is the new MacBook Air, with a little extra help from the
iTunes store, of course.

During the unveiling of the new MacBook Air line, Mr.~Jobs said these computers were next-generation
laptops. This was reiterated in the commercials for the computers, in which a voiceover calls them
``the next generation of Macbooks.''

In other words, don't expect a DVD slot in your next Mac laptop, or your next desktop computer for
that matter.

Apple hopes to replace those discs with a fluffy white iCloud, where software, music, video and your
own personal content fly around in the air like happy seagulls at the beach.

Apple has a lot at stake with its move to this new era of computing.

The disastrous launch of MobileMe, which offers online storage and syncing between devices, was one
of the company's biggest flops of the past few years. Although the service works relatively well
now, it's still very expensive, priced at \$100 a year, and the iDisk storage software is much
slower than third-party products like ZumoCast and Box.net.

To add to the pressure, Apple is also up against Google, a company that has essentially grown up in
the cloud and built a number of successful services for this next era of software.

None of this is news to Steve Jobs. Last week a report noted that a \$1 billion, 500,000-square-foot
data center that the company has been working on for years is getting ready to open for business.
And Peter Oppenheimer, Apple's chief financial officer, recently said that the data center would
open ``any day now.''

When the North Carolina center is ready, people can expect any number of cloud-based interactions to
appear on their Mac computers.

Photos and music could begin to flow between computers without the need of pesky wires, and you
could even imagine a time when your computer's operating system isn't just a desktop, but a
cloudtop, where any kind of file saved to your machine is automatically beamed into space,
accessible on iPads, iPhones and other Apple devices.

But the company isn't in a rush to switch everyone and everything to the cloud just yet. Apple wants
to get it right this time around, and customers will likely see these changes happen slowly.

One thing is for sure: get ready to stuff those old DVDs and CDs in a box with your video tapes and
floppy disks.

\section{In Alaska, `Write-In' Leads, Giving Murkowski Hope}

\lettrine{L}{isa}\mycalendar{Nov.'10}{04} Murkowski was not quite declaring victory in her long-shot
bid to retain her Senate seat, and her closest opponent, Joe Miller, a Sarah Palin-backed Tea Party
insurgent, certainly was not conceding.

``They said you can't do it, you can't win a write-in campaign, not in Alaska, not anywhere,''
Ms.~Murkowski told supporters here late Tuesday night as ``Write-In Votes'' led in the vote count.
``Do they know Alaska?''

``No!'' the response came roaring back from the crowd.

At the same time, Robert Campbell, Mr.~Miller's campaign manager said: ``We're pretty confident
Senator Murkowski's going to get some of the write-in votes. It's also possible Spiderman will get
some, too.''

It may be weeks before all those write-in ballots are manually opened and read and it is known
whether Ms.~Murkowski pulled off the upset of the night. If Ms.~Murkowski succeeds, her victory will
be counted pretty much everywhere as a loss for Ms.~Palin, a leading light of the Tea Party movement
who will have failed to put her chosen candidate, Mr.~Miller, over the top in her home state.

The outcome of the race here will not affect the balance of power in the United States Senate. The
Democrats will keep enough seats to maintain control whatever happens, and it appeared on Wednesday
that Scott McAdams, the Democrat in the race here, would most likely not be in the running.

But, as the denouement of the latest chapter in a family feud between the Murkowskis and the Palins,
the drama playing out in Alaska continued to attract national attention.

The Murkowski campaign had used unconventional means to wage her unconventional campaign -- from
wristbands to homespun jingles and omnipresent advertisements showing how to spell her name. Now
lawyers from both Republican campaigns are standing by to contest ballots. The only standard that
Alaskan voting authorities have announced is the ability to determine ``voter intent.''

As the lead for ``Write-In Votes'' expanded through the night, Ms. Murkowski was swarmed. She
grinned. She laughed. She hugged. She pumped her fists. She embraced her husband and two teenage
sons, one of whom wore a campaign T-shirt that said, ``Too legit to quit.''

On the eve of the election, the Murkowski campaign sent out an e-mail noting that the San Francisco
Giants had just won the World Series -- and that the last time the team had won the series was in
1954, the same year Strom Thurmond became the first senator -- and the only, so far -- to win
election as a write-in.

With 99 percent of precincts reporting, the write-in category had 41 percent of the vote compared
with 34 percent for Mr.~Miller. Mr. McAdams had 24 percent of the vote.

Ms.~Murkowski's campaign staff had said earlier that if the write-in category won about 40 percent
of the vote, they would feel confident that the senator would ultimately win -- and survive any
legal challenges that might be brought by her opponents. It could be weeks before the state verifies
the names entered as write-in votes. Thousands of absentee ballots and other ballots remain to be
counted.

``We're very happy with the position we're in,'' said Kevin Sweeney, Ms. Murkowski's campaign
manager.

Mr.~Miller told his supporters that the race was not over, then left his election party with his
family relatively early in the night. He did not appear at the Egan Center in downtown Anchorage,
where candidates of various parties traditionally meet after their individual parties to watch
results and do television interviews. His campaign said he would wait for all ballots to be counted
and verified.

Elections officials have said that they would not verify names entered by write-in votes until
Nov.~18, after the two-week window during which absentee ballots are accepted. As much as 10 percent
of the ballots cast could be absentee. Absentee ballots must be postmarked by Election Day.

Mr.~Miller suffered from a string of setbacks in the final weeks of the campaign, including
revelations that he had been disciplined in a previous job for using government computers for
political purposes and then lying about it. At one point, security guards he had hired for a
campaign event handcuffed a reporter who tried to ask Mr.~Miller about the disciplinary action.

Ms.~Palin headlined a rally for Mr.~Miller last week but she offered no comment about the race
Tuesday night on her customary platforms, Twitter and Facebook. She spent the evening in New York,
working for Fox News, while voters in her home state appeared to ignore her emphatic appeals to oust
Ms.~Murkowski.

The strong showing by the senator was also measure of vindication for her father, Frank
H.~Murkowski, who was among those celebrating.

Ms.~Murkowski first took office in 2002 after her father, then governor, appointed her to the Senate
seat, which he had previously held. She later won a full term in 2004, but the fact that she had
been appointed by her father fueled longstanding resentment among many Alaskans, including
Ms.~Palin.

Ms.~Palin defeated Mr.~Murkowski in a landslide in the 2006 Republican primary for governor. It was
her endorsement this spring that helped Mr.~Miller rise to prominence this summer and eventually
stun Ms. Murkowski in the primary. Ms.~Palin attacked Ms.~Murkowski, recently calling her ``the
entitlement candidate'' and ``a disconnected liberal.''

Mr.~Miller campaigned on a platform of states' rights, sharp spending cuts and a strict antiabortion
stance. The messages resonated among many Republicans in Alaska but worried others, as well as many
Democrats, who feared cuts in federal spending. About a third of Alaska's economy is dependent on
federal spending.

``That's one of the best things I ever did,'' Mr.~Murkowski said, referring to appointing his
daughter to the senate seat. ``And I'm certainly proud of her. She's certainly capable.''

Mr.~Sweeney, who previously worked as state director for Ms. Murkowski's senate staff, took over as
Ms.~Murkowski's campaign manager after she lost the primary. He said the senator's decision to
re-enter the race was not based on a sophisticated examination of voting patterns.

``It was never about the numbers,'' Mr.~Sweeney said. ``It was about giving Alaskans a choice. We
entered the race, then put together a plan to win it.''

He said Ms.~Murkowski had been overwhelmed by the number of people who encouraged her to find a way
back into the race after the primary loss, and that as many as a thousand people volunteered to help
in the first two weeks after she announced her write-in campaign on Sept.~17. The campaign opened
new offices in various Alaska cities as recently as late October because volunteers in those cities
said they wanted to work in them. The ground-level support, coupled with hundreds of millions of
dollars spent on the campaign by Alaska Native corporations, helped educate people about how to
write-in their votes.

``I think it woke up Alaskans,'' one supporter, Trina Short, said of the primary loss. ``And Lisa,
too. Everyone woke up.''

\section{Many Voters Find Little Comfort on Ballot}

\lettrine{W}{ant}\mycalendar{Nov.'10}{04} to know what's wrong with this country? Just ask everyone.

``Anybody in business knows you can't have debt like this,'' said Marty Yanke, 76, a retired meat
processing plant supervisor from Waunakee, Wis., who said he voted for Republicans.

Walter Sankey, 57, who lives in a working-class neighborhood of Columbus and chose all Democrats,
said, ``You can't expect changes in two years for what took eight years to screw up.''

Joe Nguyen, a construction subcontractor in Orlando, Fla., said, ``Nobody is listening to the
people, especially small-business people.''

And so it went Tuesday. In urban elementary schools, country churches and suburban recreational
centers across the United States, worried voters talked of wanting to find the right mix of
political leaders to lift the nation out of its economic crisis.

The mood at the polls was more about grim resignation than anger. Certainly, issues like
immigration, war and education reform mattered to people, including Rafaela Suarez, who voted in
East Los Angeles.

``There are six kids in my house,'' Ms.~Suarez said. ``I want them to have a future.''

But with friends and relatives out of work, neighboring houses in foreclosure and their own savings
accounts to consider, voters put fixing the economy first. And if that meant voting in ways that
seemed out of the question two years ago, so be it.

Kathleen Morse, a homemaker from Epping, N.H., described herself as a centrist-leaning independent,
but this time she voted for anyone with an R next to his or her name. ``I just went with it,''
Ms.~Morse said. ``I mean, it can't be any worse.''

Marcia Rutledge, 71, supported a measure that would allow a casino in western Maine, despite her
opposition to gambling. ``I know it's addictive,'' Ms.~Rutledge said, ``but it will give some people
up in that area work.''

Even at the student union at Ohio State University, young voters were more preoccupied with the
economy than with war or social issues.

``I've got to find a job, and the way it's going I don't know that I'm going to,'' said John
Breedlove, 22, a business student who supported President Obama in 2008 but backed Republicans on
Tuesday.

The solution for many voters was to throw the bums out -- but replace them with whom? That answer
was less clear.

``I'm not voting for what I think will be a great winner,'' said Lee Esker, 35, a finance director
at a medical research center in Chicago, having just marked his ballot for the Republicans. ``I
think I'm caught between the parties, and I'm voting for the lesser of two evils.''

Gina Duncan, 52, a Democrat who voted for Gov.~Charlie Crist of Florida, a former Republican now
running as an independent for the Senate, did so reluctantly. ``I was looking for candidates who can
cross the aisle and get things done,'' Ms.~Duncan said.

To be sure, some voters stood firmly behind President Obama. ``We haven't given Obama enough time,
and we're rushing to judge his administration on a lot of issues that were out of his control,''
said Emily Pechar, 22, who voted in the Fulton County suburbs north of Atlanta. ``Obama's become a
scapegoat, and that's not appropriate.''

Others saw the president and the lawmakers who supported his ambitious early agenda, including the
health care overhaul and his stimulus plan, as clear villains. The country, many voters said, was
veering dangerously to the left and needed to return to the center.

But many voters could not find solace anywhere. Mark Munn, a boilermaker from Savannah, Ga., whose
three children attend a private Christian school, has sometimes voted for conservative Democrats,
but he has not seen one in a while. Republican candidates, Mr.~Munn said, have too many ties to
big-money interests, even though he agrees with their social policies. The Tea Party is not a good
fit, either.

``I'm just not comfortable with any of them,'' he said.

Still other voters were so hungry for change that experience did not seem to matter. Fifty-seven
percent of the registered voters surveyed as part of a New York Times/CBS News poll released late
last week said that this year they were more willing to take a chance on a candidate with little
previous political experience.

John Nicholson, 69, a registered independent who voted at the Church of God in Camden Wyoming, Del.,
preferred Christine O'Donnell, the Republican nominee for Senate, mainly because she had never been
in politics before. ``Everybody thinks she's ditzy, but she can learn like anyone else,''
Mr.~Nicholson said. ``You have to give people like that a chance.''

Unlike the 2008 election, when liberals celebrated the politics of hope and change to address the
country's problems and some conservatives began to embrace what would become the Tea Party's brand
of revolution, Tuesday's contests were underscored by a deep sense of malaise on all sides.

``Things just don't feel right in the country right now,'' said Damon Gillette, 44, a business owner
in Denver.

Adding to the grim feel in many polling places was the realization that the most expensive midterm
election in history -- a record \$3.8 billion was spent -- brought what appeared to be a new low in
civil discourse, voters said.

``These elections have been a freak show all over the country,'' said David Morgan, a retired Air
Force officer, in Beckley, W.Va. His wife found the campaigns so distasteful that she refused to
vote.

Many voters just pleaded for the bickering to stop.

Tony Perelli, 75, of Chicago's South Side, who said he had voted reluctantly for all Democrats,
considers himself a middle-of-the-road voter. But middle-of-the-road candidates were elusive.
Instead, he found himself buried in rhetoric from both ends of the political spectrum.

``I'd like to get to the bottom of what's really right for this country, and that's kind of hard
while they're all calling each other names,'' Mr.~Perelli said.

Steve Bachar, 45, an investment manager in Denver, echoed that view. He said his biggest hope for
the months ahead centered on one word: pragmatism. ``The political discourse in the course of the
past several years has gotten so that both parties are talking past each other,'' he said.

It was a feeling echoed by Carla Kaiser, who was laid off by an insurance company and relied on
unemployment checks. She is scared about the economy and politicians who might want to feed off the
fear of people like her, who are enduring hard times.

``We as Americans need to pull together,'' she said. ``We need to stop looking at the Democrats and
the Republicans and just look at the problems.''

\section{In Governor's Races, Republicans Make Gains}

\lettrine{O}{n}\mycalendar{Nov.'10}{04} an Election Day with one of the largest number of governors'
races in memory, Republicans gained governorships across the country, and particularly in the
political battlegrounds of the industrial Midwest where Democrats have dominated in recent years.

In Iowa, Michigan, Ohio, Pennsylvania and Wisconsin, Republicans seized seats that had been held by
Democrats. They also took seats now held by Democrats in other parts of the country, including
Kansas, New Mexico, Oklahoma, Tennessee and Wyoming.

In Wisconsin, a beaming Scott Walker, a Republican, took to a stage and praised all the voters who,
he said, had emerged from the woodwork to ``take our state back.''

As in so many states, much of the campaign there had focused around job losses, financial woes and
state budget troubles, and Mr.~Walker, like several of his Republican colleagues, had pledged to cut
government waste, reshape government and upend a system that he said had failed. Minutes after his
victory became clear, Mr.~Walker issued a release that declared: ``Wisconsin is open for business!''

But around the nation, the outcomes are expected to have effects that reach beyond local economic
policies or legislation drawn up in statehouses.

States are preparing to carry out their once-a-decade redrawing of political districts -- for the
House and state legislatures -- based on United States census counts collected this year, and many
of these new governors will have important roles in deciding what those maps look like.

Going into Election Day, Democrats held 26 governorships, while Republicans had 24. Following most
midterm elections after the arrival of a new president, the party in power in the White House
typically loses some governorships, but the changes on Tuesday appeared to go deeper.

With votes in many states still being counted on Tuesday night, Republicans were already holding on
to many of the seats they currently hold -- in Alabama, Arizona, Georgia, Idaho, Nebraska, Nevada,
South Carolina, South Dakota, Texas and Utah -- as well anticipating significant gains.

``People are not happy with the direction of this country,'' said Terry Branstad, a Republican and
former governor who defeated Gov.~Chet Culver of Iowa, another state where the economy seemed to
overwhelm most other issues. ``The status quo is not acceptable.''

Democrats were hoping that voters might turnout in high numbers and that efforts in the final weeks
by President Obama and other Democratic leaders might lessen the damage.

There were certainly some indications of relief for Democrats, in states that included Arkansas,
Colorado and New Hampshire.

In New York, too, Attorney General Andrew M.~Cuomo easily defeated the Republican, Carl P.~Paladino,
even though Republicans were expected to pick up seats in the State Legislature and the
Congressional delegation. In Massachusetts, Gov.~Deval Patrick, a Democrat, beat Charles Baker Jr.,
a Republican and a former chief executive of one of the state's largest health insurers. And in
Maryland, Martin O'Malley, the Democratic governor, fought off a challenge from Robert L.~Ehrlich
Jr., a Republican who had once been governor. But the Republicans' gains in the Midwest were
daunting for Democrats, in part because of the size and scope of the shift.

In Wisconsin, Mr.~Walker, the county executive of Milwaukee who has promised to shrink government,
beat Tom Barrett, the Democratic mayor of Milwaukee. Mr.~Walker equated electing Mr.~Barrett with
giving one more term to James E.~Doyle, the current governor whose popularity ratings had become
anemic.

In Michigan, where Jennifer M.~Granholm, a Democrat, was barred from seeking re-election by term
limits, Rick Snyder, a Republican who stunned the party establishment by beating better-known, more
established candidates in a primary, defeated Virg Bernero, the Democratic mayor of Lansing. The
issue in the state, which had suffered devastating economic losses even before the recession, was
the same as everywhere: jobs and money.

Among the group of new Republican political leaders emerging on Tuesday: Nikki Haley, the nation's
first Indian-American female governor, a victor in South Carolina; Susana Martinez, a Republican
district attorney who promised to end a pattern of corruption and to block illegal immigrants from
getting driver's licenses in New Mexico; and Mr.~Snyder, the former head of Gateway Inc., who was
elected governor of Michigan with the catch phrase ``one tough nerd.''

Of the 37 states voting for governor, 24 races were open seats from both parties, thanks to terms
limits and to a climate that seemed to discourage some incumbents from seeking re-election.

From Maine to Hawaii, the governors' races had been hard fought, with clear indications, leaders
from both parties said, of the same broad national climate that was testing the survival of
Democrats -- and incumbents -- for the House and Senate.

In another indication of how voters seemed in search of something, anything, entirely different from
the status quo, third-party candidates had a particularly pronounced effect on governors races in at
least five states. And in Rhode Island, Lincoln D.~Chafee, a former Republican senator who ran for
governor as an independent, won on Tuesday.

While much of the attention this season has focused on who will control Washington, the outcomes in
these governors' races were drawing particular notice because of redistricting.

The shapes of the political maps can carry lasting effects for partisan victories and losses in all
sorts of offices. Governors in at least 36 states get a say in shaping Congressional maps, and
governors in 39 states have a place in redrawing state legislative districts.

``This is the most important governors' election in 20 years,'' said Nathan Daschle, executive
director of the Democratic Governors Association, which devoted \$50 million to races this year,
three times the amount the group spent four years ago, in the last comparable election. The
Republican Governors Association spent \$102 million on this year's races.

\section{Cede Political Turf? Never! Well, Maybe.}

\lettrine{N}{o}\mycalendar{Nov.'10}{04} matter how it turns out, at least one message from the
midterm election is clear: this is not a moment of harmonic convergence.

The campaign has been all about irreconcilable differences -- on health care, on spending and
climate, on who's a lunatic and who's a tyrant. Political handicappers expect at best a divided
government, and at worst a paralyzed, feuding one: ``Survivor: Washington, D.C.,'' with gavels and
filibusters in place of bonfires and tribal headbands.

But recent research suggests that several strong but subtle psychological factors will be pushing
Democrats and Republicans in an unexpected direction -- toward engagement instead of name-calling
and nastiness. These forces, rooted in the nature of personal identity and the rhythm of one-on-one
interaction, are present when any antagonists meet; they can blunt even sharp ideological
differences.

``We've done a good job of documenting how harshly humans treat their enemies throughout
evolution,'' said Dacher Keltner, a psychologist at the University of California, Berkeley, and
author of ``Born to be Good'' (Norton, 2009). ``But we also have evidence of an early shift in human
evolution to hypersociality -- a default orientation toward trust, toward sharing resources, toward
forgiveness.''

Without putting aside their differences, he went on, humans have a ``profound capacity through which
vicious adversaries can form alliances.''

Political insiders are skeptical that this will happen soon, especially if some of the strongest
opponents of the president are elected. ``These candidates believe they're being sent to undo what's
been done,'' said Frank Luntz, a Republican pollster and consultant. ``They don't even want to be
seen'' dealing with opponents.

Still, people tend to exaggerate their differences with opponents to begin with, research suggests,
especially in the company of fellow partisans. In small groups organized around a cause, for
instance, members are prone to one-up one another; the most extreme tend to rise the most quickly,
making the group look more radical than it is.

For this and other reasons -- including media coverage of the political fringes -- Americans as a
rule overestimate the policy differences between so-called red and blue voters.

A series of recent studies demonstrates how quickly large differences can be put aside, under some
circumstances. In one, a team of psychologists had a group of college students who scored very high
on measures of patriotism read and critique an essay titled ``Beyond the Rhetoric: Understanding the
Recent Terrorist Attacks in Context,'' which argued that the 9/11 attacks were partly a response to
American policy in the Middle East.

The students judged the report harshly -- unless, prompted by the researchers, they had first
described a memory that they were proud of. This group, flush with the image of having acted with
grace or courage, was significantly more open to at least considering the case spelled out in the
essay than those who had recounted a memory of having failed to exhibit their most prized personal
quality.

Confronting an opposing political view is a threat to identity, but ``if you remind people of what
they value in some other domain of their life, it lessens the pain,'' said the lead author, Geoffrey
L.~Cohen, a social psychologist at Stanford. ``It opens them up to information that they might not
otherwise consider.''

Psychologists draw out these memories in the lab by asking people to describe a personal strength
and an occasion when it was on display. But daily life is full of triggers too, whether in an
unexpected call from an old friend, a question from a child or an evocative sight -- one's family
portrait, say, propped up under the desk lamp as a reminder of what matters most.

The effect of such affirmations seems especially pronounced in people who boast strong convictions.
In a follow-up experiment, the research team had supporters of abortion-rights act out a negotiation
with an opponent on an abortion bill. Again, participants who were prompted to recall a treasured
memory beforehand were more open to seeking areas of agreement and more respectful of their
opposite's position than those not so prompted.

And most fair-minded of all were participants who recalled such a memory and also reaffirmed their
beliefs on abortion rights before the role-playing. ``This is contrary to what many assume,''
Dr.~Cohen said. ``But the combination of the two made for the greatest distance traveled to meet an
adversary, to look for a middle ground.''

Private values -- memories, affirmations -- may well have played a role in some historic
compromises. During the negotiations in 1978 to achieve what would become the Camp David accords
between Israel and Egypt, the Israeli prime minister, Menachem Begin, appeared ready to walk away.
President Jimmy Carter, who coordinated the talks, made a personal visit to Mr.~Begin, bringing him
autographed photographs of the meeting, addressed to each of the prime minister's eight
grandchildren.

``That was it,'' Mr.~Carter said in a 1994 interview. ``He looked at those eight photographs, and
tears began to run down his cheeks -- and mine -- as he read the names. In just a few minutes he
sent his attorney general to tell me he was going to look at the negotiations again.''

One reason sworn enemies may soften after the campaigning is over and they're seated face to face is
that conversation subconsciously synchronizes people physically -- and, to some extent, mentally.
Researchers have known for some time that a person's gestures, movements and expressions tend to
mimic those of another when they interact one on one; the closer the mimicry, the warmer the
interactions.

But new research suggests that conversation partners quickly and subconsciously begin to speak alike
-- even when they don't care for one another. ``I think of it as a verbal dance,'' said James W.
Pennebaker, a psychologist at the University of Texas. ``Speaking styles match up very quickly when
people begin to talk to one another -- any two people.''

Within about 30 seconds, strangers making small talk begin speaking with similar sentence structure,
a similar tone and often the same slang. In a new study, Dr.~Pennebaker and Molly E.~Ireland, also
of the University of Texas, found that over time, the degree of similarity begins to reflect the
closeness (or coolness) of a relationship.

The researchers analyzed correspondence and writings in three famous relationships: Freud and Jung
and two poetic couples, Elizabeth Barrett and Robert Browning, and Sylvia Plath and Ted Hughes.
Using a measure called language style matching -- a calculation that takes into account the
frequency of personal pronouns and other subconscious markers of rapport -- they found that
synchrony ranged from a high of nearly 90 percent (Freud and Jung, at the peak of their
collaboration) down to 60 percent (Plath and Hughes, in a low period).

The new Congress, though short on married poets, will bring together in direct debate people with
strong beliefs who -- even if hostile -- will automatically be speaking a similar language.

``It doesn't matter whether you admire the other person, simply need information or hate their
guts,'' Dr.~Pennebaker said. ``If you're paying attention to the other person, it's going to
happen.''

That is a big if -- and one that, these days, is subject to many outside radicalizing forces, not to
mention economic uncertainty.

But if members of the new Congress are secure in their convictions and willing to pay attention to
their opponents, they just might get some business done.

\section{From Farm to Fridge to Garbage Can}

\lettrine{H}{ow}\mycalendar{Nov.'10}{04} much food does your family waste?

A lot, if you are typical. By most estimates, a quarter to half of all food produced in the United
States goes uneaten -- left in fields, spoiled in transport, thrown out at the grocery store,
scraped into the garbage or forgotten until it spoils.

A study in Tompkins County, N.Y., showed that 40 percent of food waste occurred in the home. Another
study, by the Cornell University Food and Brand Lab, found that 93 percent of respondents
acknowledged buying foods they never used.

And worries about food safety prompt many of us to throw away perfectly good food. In a study at
Oregon State University, consumers were shown three samples of iceberg lettuce, two of them with
varying degrees of light brown on the edges and at the base. Although all three were edible, and the
brown edges easily cut away, 40 percent of respondents said they would serve only the pristine
lettuce.

In his new book ``American Wasteland: How America Throws Away Nearly Half of Its Food'' (Da Capo
Press), Jonathan Bloom makes the case that curbing food waste isn't just about cleaning your plate.

``The bad news is that we're extremely wasteful,'' Mr.~Bloom said in an interview. ``The positive
side of it is that we have a real role to play here, and we can effect change. If we all reduce food
waste in our homes, we'll have a significant impact.''

Why should we care about food waste? For starters, it's expensive. Citing various studies, including
one at the University of Arizona called the Garbage Project that tracked home food waste for three
decades, Mr.~Bloom estimates that as much as 25 percent of the food we bring into our homes is
wasted. So a family of four that spends \$175 a week on groceries squanders more than \$40 worth of
food each week and \$2,275 a year.

And from a health standpoint, allowing fresh fruits, vegetables and meats to spoil in our
refrigerators increases the likelihood that we will turn to less healthful processed foods or
restaurant meals. Wasted food also takes an environmental toll. Food scraps make up about 19 percent
of the waste dumped in landfills, where it ends up rotting and producing methane, a greenhouse gas.

A major culprit, Mr.~Bloom says, is refrigerator clutter. Fresh foods and leftovers languish on
crowded shelves and eventually go bad. Mr. Bloom tells the story of discovering basil, mint and a
red onion hiding in the fridge of a friend who had just bought all three, forgetting he already had
them.

``It gets frustrating when you forget about something and discover it two weeks later,'' Mr.~Bloom
said. ``So many people these days have these massive refrigerators, and there is this sense that we
need to keep them well stocked. But there's no way you can eat all that food before it goes bad.''

Then there are chilling and food-storage problems. The ideal refrigerator temperature is 37 degrees
Fahrenheit, and the freezer should be zero degrees, says Mark Connelly, deputy technical director
for Consumer Reports, which recently conducted extensive testing on a variety of refrigerators. The
magazine found that most but not all newer models had good temperature control, although models with
digital temperature settings typically were the best.

Vegetables keep best in crisper drawers with separate humidity controls.

If food seems to be spoiling quickly in your refrigerator, check to make sure you're following the
manufacturer's care instructions. Look behind the fridge to see if coils have become caked with
dust, dirt or pet hair, which can interfere with performance.

``One of the pieces of advice we give is to go to a hardware store and buy a relatively inexpensive
thermometer,'' Mr.~Connelly said. ``Put it in the refrigerator to check the temperature to make sure
it's cold enough.''

There's an even easier way: check the ice cream. If it feels soft, that means the temperature is at
least 8 degrees Fahrenheit and you need to lower the setting. And if you're investing in a new
model, don't just think about space and style, but focus on the refrigerator that has the best sight
lines, so you can see what you're storing. Bottom-freezer units put fresh foods at eye level,
lowering the chance that they will be forgotten and left to spoil.

Mr.~Bloom also suggests ``making friends with your freezer,'' using it to store fresh foods that
would otherwise spoil before you have time to eat them.

Or invest in special produce containers with top vents and bottom strainers to keep food fresh. Buy
whole heads of lettuce, which stay fresher longer, or add a paper towel to the bottom of bagged
lettuce and vegetables to absorb liquids. Finally, plan out meals and create detailed shopping lists
so you don't buy more food than you can eat.

Don't be afraid of brown spots or mushy parts that can easily be cut away.

``Consumers want perfect foods,'' said Shirley Van Garde, the now-retired co-author of the Oregon
State study. ``They have real difficulty trying to tell the difference in quality changes and safety
spoilage. With lettuce, take off a couple of leaves, you can do some cutting and the rest of it is
still usable.''

And if you do decide to throw away food, give it a second look, Mr. Bloom advises. ``The common
attitude is `when in doubt, throw it out,''' he said. ``But I try to give the food the benefit of
the doubt.''

\section{Seeing the Natural World With a Physicist's Lens}

\lettrine{I}{f}\mycalendar{Nov.'10}{04} you've ever stumbled your way through a newly darkened movie
theater, unable to distinguish an armrest from a splayed leg or a draped coat from a child's head,
you may well question some of the design features of the human visual system. Sure, we can see lots
of colors during the day, but turn down the lights and, well, did you know that a large bucket of
popcorn can accommodate an entire woman's shoe without tipping over?

Yet for all these apparent flaws, the basic building blocks of human eyesight turn out to be
practically perfect. Scientists have learned that the fundamental units of vision, the photoreceptor
cells that carpet the retinal tissue of the eye and respond to light, are not just good or great or
phabulous at their job. They are not merely exceptionally impressive by the standards of biology,
with whatever slop and wiggle room the animate category implies. Photoreceptors operate at the
outermost boundary allowed by the laws of physics, which means they are as good as they can be,
period. Each one is designed to detect and respond to single photons of light -- the smallest
possible packages in which light comes wrapped.

``Light is quantized, and you can't count half a photon,'' said William Bialek, a professor of
physics and integrative genomics at Princeton University. ``This is as far as it goes.''

So while it can take a few minutes to adjust to the dark after being fooled by a flood of artificial
light, our eyes can indeed seize the prize, and spot a dim salting of lone photons glittering on the
horizon.

Photoreceptors exemplify the principle of optimization, an idea, gaining ever wider traction among
researchers, that certain key features of the natural world have been honed by evolution to the
highest possible peaks of performance, the legal limits of what Newton, Maxwell, Pauli, Planck et
Albert will allow. Scientists have identified and mathematically anatomized an array of cases where
optimization has left its fastidious mark, among them the superb efficiency with which bacterial
cells will close in on a food source; the precision response in a fruit fly embryo to contouring
molecules that help distinguish tail from head; and the way a shark can find its prey by measuring
micro-fluxes of electricity in the water a tremulous millionth of a volt strong -- which, as Douglas
Fields observed in Scientific American, is like detecting an electrical field generated by a
standard AA battery ``with one pole dipped in the Long Island Sound and the other pole in waters of
Jacksonville, Fla.'' In each instance, biophysicists have calculated, the system couldn't get
faster, more sensitive or more efficient without first relocating to an alternate universe with
alternate physical constants.

The tenets of optimization may even help explain phenomena on a larger scale, like the rubberiness
of our reflexes and the basic architecture of our brain.

For Dr.~Bialek and other biophysicists, optimization analysis offers the chance to identify general
principles in biology that can be encapsulated in an elegant set of equations. They can then use
those first principles to make predictions about how other living systems may behave, and even test
their predictions in real-life, wetware settings -- an exercise that can quickly mount in
quantitative complexity for even the seemingly simplest cases.

On Wednesday, Dr.~Bialek will discuss his take on biological optimization at the Graduate Center of
the City University of New York, in a public lecture fetchingly titled ``More Perfect Than We
Imagined: A Physicist's View of Life.'' Dr.~Bialek is a visiting professor at the graduate school,
where he has helped establish an ``initiative for the theoretical sciences'' devoted to the grand
emulsification of mathematics, neuroscience, condensed-matter physics, quantum computation,
computational chemistry and the occasional seminar on the physics of mousse and marshmallows.

Wherever he is perched, Dr.~Bialek seeks to train the tools of physics on biology, a discipline that
historically has favored research and experimentation over theory and computation, and that
sometimes can seem so number-averse you'd think it was an they were extensions of the humanities
department.

``Because mathematics is so central to how we think about the world, physicists often are speaking a
different language than biologists, asking different questions,'' said Dr.~Bialek, his impish,
abstractedly cerebral face and full, free-wheeling beard giving him something of a jolly professor
manner. ``Of course this can lead to conflict.''

In one optimization study, Dr.~Bialek and his colleagues considered the dynamics of a major
signaling molecule in the fruit fly embryo called bicoid.

It was known that bicoid bits were dispensed into the crown end of a fruit fly egg by the mother,
that the molecules diffused tailward during development, and that the relative concentration of
bicoid at any given spot helped determine the segmentation of a budding fruit fly's form. But how,
exactly, did the fly translate something as amorphous and borderless as a seeping oil spill into the
ordered grid of a body plan?

The researchers calculated that, to operate optimally, each cell in the developing embryo would
match the strength of its bicoid signal against an overall range of possible signal strengths,
essentially by comparing notes with its neighbors. Sure enough, experiments later showed that
embryonic fly cells perform precisely this sort of quantitative matching in response to a bicoid
stimulus package. ``It's one of those things where we could have failed dramatically,'' said Dr.
Bialek, ``but we succeeded better than we could have expected.''

Other researchers have shown that an E.~coli microbe navigating its way through a chemically chaotic
environment and over to food relies on a similar algorithm of compare-contrast-act, although in this
case the note-trading takes place between surface receptors on the bacterium's front and aft. ``The
reliability of its decision-making is so high,'' said Dr.~Bialek, ``that it couldn't do much better
if it counted every single molecule in its environment.''

Emanuel Todorov, a neuroscientist at the University of Washington, said that one way to identify
likely cases of optimization is to find biological systems that are ubiquitous, ancient and
resistant to change.

``The muscles of most species are very similar,'' he said, ``and inside every muscle fiber are the
same long, organic molecules, the same actin, myosin and troponin that latch onto each other to
generate force.'' The engine of all animal motion, he said, is close to being an optimized machine
that itself needs no forward march.

Dr.~Todorov has studied how we use our muscles, and here, too, he finds evidence of optimization at
play. He points out that our body movements are ``nonrepeatable'': we may make the same motion over
and over, but we do it slightly differently every time.

``You might say, well, the human body is sloppy,'' he said, ``but no, we're better designed than any
robot.''

In making a given motion, the brain focuses on the essential elements of the task, and ignores noise
and fluctuations en route to success. If you're trying to turn on a light switch, who cares if the
elbow is down or to the side, or your wrist wobbles -- so long as your finger reaches the targeted
switch?

Dr.~Todorov and his coworkers have modeled different motions and determined that the best approach
is the wobbly, ever-varying one. If you try to correct every minor fluctuation, he explained, not
only do you expend more energy unnecessarily, and not only do you end up fatiguing your muscles more
quickly, you also introduce more noise into the system, amplifying the fluctuations until the entire
effort is compromised.

``So we reach the counterintuitive conclusion,'' he said, ``that the optimal way to control movement
allows a certain amount of fluctuation and noise'' -- a certain lack of control.

The brain, too, seems built to tolerate bloopers and static hiss. Simon Laughlin of Cambridge
University has proposed that the brain's wiring system has been maximally miniaturized, condensed
for the sake of speed to the physical edge of signal fidelity.

According to Charles Stevens of the Salk Institute, our brains distinguish noise from signal through
redundancy of neurons and a canny averaging of what those neurons have to say.

We are like microbes trepanning for food, and why not? Bacteria have been here for nearly four
billion years. They have optimized survival. They can show us the way.

\section{The China Boom}

\lettrine{I}{n}\mycalendar{Nov.'10}{10} her ballroom dance class, Li Wanrong has learned to tango
and cha-cha. At lunch one day, she tried a strange mix of flavors -- pepperoni pizza, the spicy
sausage and oozing cheese nearly burning her tongue. Then there was that Friday night before going
clubbing for the first time when new friends gave her a makeover, and she looked in the mirror to
see an American girl smiling back wearing a little black dress, red lipstick and fierce eyeliner.

``I say `wow' a lot,'' says Ms.~Li, a freshman at Drew University, a small liberal arts school in
Madison, N.J.

Against her parents' wishes, she studied for and took the SAT in Hong Kong, a three-hour bus ride
from her home in southern China. She told them she was going there to do some shopping. Her parents
eventually came around, persuaded by her determination and a \$12,000 scholarship that would take
some of the sting out of the \$40,000 tuition at Drew, which her high school teacher had
recommended.

Describing her whirlwind transformation to college kid sometimes leaves Ms.~Li at a loss for words.
And sometimes the cultural distance seems too much, especially when facing dining options in the
cafeteria. ``Sometimes I feel when I go back to China I'll never eat a hamburger ever again,'' she
says, laughing.

Ms.~Li is part of a record wave of Chinese high school graduates enrolling in American colleges,
joining the fabric of campus life as roommates and study partners and contributing to the global
perspectives to which colleges are so eager to expose their students.

``China is going to matter greatly to all students in the 21st century,'' says Robert Weisbuch,
president of Drew, which has increased its international enrollment by 60 percent in the last five
years. ``We feel it is important to provide the opportunity for American and Chinese students to
learn from one another.''

While China's students have long filled American graduate schools, its undergraduates now represent
the fastest-growing group of international students. In 2008-9, more than 26,000 were studying in
the United States, up from about 8,000 eight years earlier, according to the Institute of
International Education.

Students are ending up not just at nationally known universities, but also at regional colleges,
state schools and even community colleges that recruit overseas. Most of these students pay full
freight (international students are not eligible for government financial aid) -- a benefit for
campuses where the economic downturn has gutted endowments or state financing.

The boom parallels China's emergence as the world's largest economy after the United States. China
is home to a growing number of middle-class parents who have saved for years to get their only child
into a top school, hoping for an advantage in a competitive job market made more so by a surge in
college graduates. Since the 1990s, China has doubled its number of higher education institutions.
More than 60 percent of high school graduates now attend a university, up from 20 percent in the
1980s. But this surge has left millions of diploma-wielding young people unable to find white-collar
work in a country still heavily reliant on low-paying manufacturing.

``The Chinese are going to invest in anything that gives them an edge, and having a U.S.~degree
certainly gives them that edge back home,'' says Peggy Blumenthal, a vice president at the Institute
of International Education. American colleges offer the chance to gain fluency in English, develop
real-world skills, and land a coveted position with a multinational corporation or government
agency.

Ding Yinghan grew up in a modest apartment with his mother, a marketing executive, and his father, a
civil servant in Beijing's work safety administration whose own mother is illiterate. A child of the
``new China,'' he is fully aware that his generation has opportunities unavailable to any before.

His parents pushed him to study hard -- and study abroad -- because they have little faith in the
Chinese education system. Sipping tea in their living room one sweltering August afternoon,
Mr.~Ding's mother, Meng Suyan, reflects on the Chinese classroom. ``In the U.S.~they focus on
creative-thinking skills, while in China they only focus on theory,'' she says. ``So what university
students learn here doesn't prepare them for the real world.''

Says Mr.~Ding: ``Chinese values require me to be a good listener, and Western values require me to
be a good speaker.''

A bespectacled whiz kid, Mr.~Ding was accepted early admission to Hamilton College in upstate New
York following a yearlong exchange program at a North Carolina public high school. Now a junior, he
is on a full scholarship, No.~1 in his class and spending this year at Dartmouth on a dual-degree
engineering program. He also founded the bridge club at Hamilton, ran the Ping-Pong team, wrote for
the student newspaper and tutored in chemistry, physics and economics for \$8.50 an hour.

His first tutoring job was freshman year, when his advanced calculus professor asked him to help
classmates struggling with the material. Over textbooks and calculators, Mr.~Ding used the
opportunity to practice his English and find commonalities with people who had never met someone
from China.

At Hamilton, he is surrounded by wealth -- some students, he says, fly to Manhattan on weekends in
helicopters, party with Champagne instead of beer, and smoke \$100 cigars. It's a new experience for
a man who gets his hair cut a few times a year because the \$15 is a lot of money for his parents,
who fret that they cannot afford to provide him with health insurance in the United States. But
sending their child to live across the world is a worthy sacrifice, says his father, Ding Dapeng.
``In China 25 years ago it was rare to even go to university, so for Yinghan to study in the U.S.~is
a real miracle.''

``Today the world is so small,'' he says. ``Only by broadening his knowledge with an international
background can Yinghan really become a global citizen.''

THE cultural exchange perhaps manifests itself most in the intimacy of the shared dorm room.

When Mariapaola La Barbera learned last summer that her roommate at Drew would come from China, her
mother was thrilled. ``She said, `They're smart people, so you'll learn from her and be focused.' ''

She shares a room with Li Wanrong. The two have tacked funky tie-dye tapestries and a poster of the
Eiffel Tower to the walls; Ms.~Li is planning to study Spanish while perfecting her English, and has
taped the words ``hola'' and ``muy bien'' next to her laptop.

``Wanrong is very brave,'' Ms.~La Barbera says. ``I give her a lot of credit for moving across the
world and being so focused.'' Still, Ms.~La Barbera, who knew no one from China, says: ``It's
different. I'm not going to lie.''

They have different groups of friends but are friendly. The roommates have taught each other words
in Mandarin and Italian, discussed the political differences between the United States and China,
and had impromptu lessons on American slang.

Ms.~Li's teachers in China had told her that American parents kick their children out of the house
when they turn 18. Ms.~La Barbera, who goes home to Staten Island every weekend, has corrected this
misconception.

``She's like a window,'' Ms.~Li says. ``I can watch her and see what Americans are like.''

As a freshman at Central Michigan University, Qi Fan realized that even Americans come from
different cultures. His roommates -- one black, one white -- spoke to him in different accents and
had social circles that largely matched their own skin color. Sometimes they would grab him out of
bed and drag him to parties where beer pong was played all night.

Mr.~Qi had learned of Central Michigan from a Chinese friend who went there, and it was talked up by
a company in China that recruits students. Originally he had considered Britain or Germany, but his
parents decided there was little point in paying for college in ``second-tier'' countries, and they
would send him to the United States ``no matter what, because it's the super power.''


But the American myth faded once he settled in. He disliked a campus culture that ``was all about
drinking,'' and wanted a high-profile school closer to New York's finance world. In his sophomore
year, Mr.~Qi transferred to the University at Albany, of the State University of New York. He says
he is happy there, makes trips to New York City in the car he just bought, and avoids any drinking
culture by living with other Chinese off campus.

Partying is an American college rite of passage, but socializing in China is usually conducted
around the table, where close friends cook, eat and play games together. The fun in standing around
a dark room filled with strangers, speakers blaring, is often lost in translation.

Frances Liu, a Yale sophomore from the bustling city of Tianjin, remembers one night freshman year
when friends started smoking marijuana. And then offered her the joint. ``They were like, `Frances,
come on,' '' she says, rolling her eyes. She declined, but the pressure to fit in meant plenty of
late nights. ``I don't want to be in a bar drunk and grinding with someone I've never met and will
never see again,'' Ms.~Liu says. ``I've tried that. I went to parties every single weekend freshman
year and realized it's not for me.''

Ms.~Liu found refuge in the Beinecke Rare Book and Manuscript Library, the towering cube of
translucent marble at Yale that holds thousands of the world's most precious written originals. Last
summer she worked there as a page, bringing requested items to researchers. But more satisfying than
the \$12 an hour was discovering treasures like the original manuscript of Edith Wharton's ``Age of
Innocence'' in the stacks and leafing through illuminated parchment from the ninth century. The
experience has given her a deep appreciation for the West's values of transparency and access to
information. ``In China, I'm used to secrecy, so being 18 and able to touch history with my bare
fingers really impressed me,'' she says.

After a year, Ms.~Liu believes she is less of the quiet-Asian-nerd stereotype that she had felt
followed her through Yale's Gothic hallways. Now she wears makeup, raises her hand in class, and has
a different perspective than her friends in China, according to whom ``I'm contaminated by American
culture and not Chinese anymore.''

That harsh assessment is heard by many Chinese undergraduates, which they say is hard to ignore. It
was in a freshman literature seminar class at Yale called ``Experiences of Being Foreign'' that Xu
Luyi began to tackle the ``pulling force westernizing me rapidly and driving me away from my own
background.''

``Somehow I was stuck in this middle zone and unable to identify with either side,'' says Ms.~Xu, a
sophomore from Shanghai. She was the only international student in the class. Rather than ignore her
``otherness,'' she dived into the course's exploration of identity construction and confusion, and
embraced the assigned readings, by immigrants and exiles. For an assignment that required that
students go somewhere that would make them feel foreign, she went to Bible study.

Where she ended up feeling most at home was in her dorm. The women in her hall would meet for tea
and cookies every few weeks to discuss college life and address girl ``drama.'' This ``women's
table,'' Ms.~Liu says, ``was a great bonding experience and also a good chance to meditate on our
experiences.''

Perhaps most unsettling to Chinese students is the robust activist culture on campus, where young
Americans find their voices on issues like war, civil rights and immigration. In China, protests are
illegal and vocal dissent forbidden, and on sensitive topics like Tibet and Taiwan a majority are in
lockstep with their government. It can be especially painful hearing Westerners condemn China after
growing up steeped in propaganda blaming the West for the suffering before Communism.

Shen Xinchao, a Rutgers junior from Shanghai, chose to attend college in the United States because
``here you can argue with professors, which is not encouraged in China,'' and choose a major rather
than test into one. ``In China, your path is almost set when you get into college on the first
day,'' he says.

But American college life presented obstacles. As a freshman, he found his campus lonely and
alienating. First, he spent a semester living in a dorm lounge because Rutgers had run out of rooms
for freshmen. Then he was paired with a roommate who challenged him over his homeland's human rights
record. ``He thought China was just a very tyrannical Communist country that has no freedom, and
that is not what life is really like there,'' says Mr.~Shen, who has moved off campus to live with
Chinese friends. ``Americans are friendly, but I just can't establish a deep relationship because
our cultural differences are too deep.''

Some Chinese students have turned activist themselves to rebut criticism of homeland policies.
Following China's crackdown on Tibet before the Beijing Olympics in summer 2008, furious groups of
Chinese students confronted protesters who were trying to disrupt the torch relay in the United
States. And on rare occasions, Chinese students have harassed pro-Tibet activists on campus, and
sought to dissuade universities from inviting the Dalai Lama to speak on their campus.

But for the most part, raised on only whispers about the student troublemakers at Tiananmen Square,
Chinese students steer clear of sit-ins, demonstrations and petitions.

``In China, we definitely don't see people marching in the streets, so it's a bit disturbing to see
the masses rallying,'' says Li Yidan, a freshman at Yale, wearing a preppy white sweater at an
off-campus cafe. ``People did that in 1989, and it ended in bloodshed.''

TO help students make the cultural leap -- as well as to internationalize their institutions --
colleges and universities are building programs that begin in China and end, hopefully, on an
American campus.

Teachers College of Columbia University has started a program for high school seniors (in China,
much of the last year is spent reviewing for a college entrance exam, though curriculum varies).
This year, the program's first, 28 students spent six months at the University of International
Relations in Beijing; 19 were found qualified to finish off the year at Columbia. The program preps
students to apply as freshmen, with a focus on English instruction, cultural immersion and
counseling, including study for the Test of English as a Foreign Language and SAT, and a tour of
campuses in the Northeast. (Total cost: about \$45,000, including room and board.)

Another new program, U.S.-Sino Pathway, aims to transition high school students into one of six
participating colleges. Northeastern University devised the curriculum, a year of for-credit courses
taken at Kaplan Inc.~branches in China and at a summer bridge session at Northeastern's Boston
campus or the University of Vermont. Kaplan handles administration, English-language instruction in
China and recruitment of students. (Total cost: about \$26,000 to \$28,000, including room and board
in the United States.)

Collaborations with for-profit education companies are beginning to gain traction as American
institutions seek to tap their in-country resources. Kaplan has branches in eight Chinese cities;
INTO University Partnerships, a British company with roots in China, similarly works with the
University of South Florida and Oregon State. Kaplan, which has been criticized for overly
aggressive recruiting in the United States, says it does not use a commission model or work with
``agents'' in China. Many Chinese hire agents to navigate the American admissions and visa maze. The
industry has mushroomed, as has its reputation for unscrupulousness, like falsifying transcripts and
making bloated promises.

The goal of U.S.-Sino Pathway, says Philomena Mantella, senior vice president for enrollment
management at Northeastern, is to help Chinese families make informed choices and to increase
readiness for the American experience. ``Finally,'' she says, ``we saw this through a global
competitive lens. British and Australian institutions were ahead of us, and we saw an opportunity to
offer a strong pathway to American universities.''

Students who complete the program's China portion apply as sophomore transfers to a consortium
college -- the others are Baylor University, Marist College, Stevens Institute of Technology and the
University of Utah. Of the 171 students who started in China, 138 were ultimately accepted into a
degree program. Nine of those who didn't meet standards chose to work on their English in another
Northeastern program, American Classroom, at its adult-education college on campus (a dozen of the
successful matriculants ended up in its degree programs as well).

The University of Vermont joined the consortium to increase its international population, which was
less than 1 percent of its undergraduates.

During Vermont's first bridge session, last summer, 29 new Chinese undergraduates absorbed American
culture by hanging out with a crowd of aging hippies at a reggae concert. They went to the Ben \&
Jerry's factory and met with the co-founder Jerry Greenfield to discuss entrepreneurship and social
justice. They also got face time with elected officials, including Vermont's governor and
Burlington's mayor, for a lesson on democracy. Among course electives: cHistory of Rock \& Roll,c
for a hearing of Elvis Presley, Bob Dylan and the Doors.

Yuan Xiecheng, who grew up amid the neon-lit skyscrapers and karaoke emporiums of Shanghai, was
eager to study abroad. He had planned to go to a Canadian university until he attended a
presentation by the chief executive officer of Kaplan China, Zhou Yong. When Mr.~Zhou announced that
students would not have to take the SAT or TOEFL or attend the final year of high school, Mr.~Yuan
leaped at the opportunity. He attended an international high school, and says he was 20 course
credits short of graduation. Instead, he took the final exam given to secure a Chinese diploma, and
enrolled in the pathway program. He is now a sophomore at Vermont.

Zhao Siwei took the same route. ``This program is super easy to enter, and it was really easy to
come here to the U.S.,'' says Ms.~Zhao, who hopes to major in film and TV at Vermont. ``I love it
here,'' she concludes. She expresses amazement, though, at her program peers' English: ``They can't
talk. They can't communicate with American people.''

Language is one of Chinese students' biggest challenges. Mr.~Yuan wishes he had had more exposure to
the vernacular. His for-credit classes at Kaplan included calculus, chemistry and American studies,
taught by instructors approved by Northeastern. But only half were Westerners, he says, and none
American. His teachers in grammar, reading and listening comprehension were Chinese, he says, and
``some of their English was not good enough.''

Once in Vermont, Mr.~Yuan worried when people smiled and asked ``What's up?'' ``It was really
awkward,'' he says, ``because I wouldn't know how to respond and while I was thinking of an answer
they would just walk away.''

Still, his English is strong enough that he joined the debate team, with its fast-clip speech and
thinking. At weekly meetings he has argued about indigenous land rights and vote buying. Presenting
an opinion in under seven minutes, as he did at his first competition, at Binghamton University, has
helped him write college papers succinctly, he says, and question the world around him. ``It's about
challenging the status quo and thinking of better solutions in a way I never thought about in
China.''

ZHOU KEHUI had an unusual adjustment to Brigham Young University. Growing up in officially atheist
China, she knew little about the Church of Jesus Christ of Latter-day Saints, with which the
university is affiliated. Mormonism is not a state-sanctioned religion, and proselytizing by its
members is illegal.

Ms.~Zhou chose Brigham Young on the recommendation of a friend of her father's, who had gone there.
Its business school also ranks highly. Her parents thought the university's honor code, with its
rules of conduct, would keep her safe and focused. Initially, however, the curfew and code, which
includes a ban on short skirts and drinking tea, left her shellshocked.

``It was really hard for me to accept the rules in the beginning,'' says Ms.~Zhou, a junior majoring
in accounting. ``I mean, where I'm from, in Fujian province, drinking tea every day is what we do.''

But few American universities offer the comfort zone she found here. Though there are only 77
Chinese undergraduates at Brigham Young, with so many Mormons doing their two years of missionary
work in Taiwan and Hong Kong, finding someone fluent in her language was easy. ``A lot of times I'd
be walking on campus when some white dude would just come up to me and start speaking Chinese,''
Ms.~Zhou says. That warmth and common experience ­-- not to mention several meetings with church
missionaries -- went a long way toward convincing her B.Y.U. was the right match.

A few months after arriving, Ms.~Zhou was baptized, which, she says, provided a support network.
That Mormonism is considered subversive at home, or that her parents were unhappy with her
conversion, gave her little pause. After all, she says, saving her soul was as logical as deciding
to go to college in the United States. ``It wasn't a hard choice to make,'' she says. ``It's
probably the best decision I've ever made in my life.''

\section{China Police Confine Prominent Artist}

\lettrine{A}{}\mycalendar{Nov.'10}{10} phalanx of Beijing police officers confined the prominent
artist and activist Ai Weiwei to his north Beijing home on Friday, a move he suggested came at the
behest of unnamed but powerful political figures in Shanghai who feared that he was about to
embarrass them.

If so, they were correct.

Mr.~Ai had planned to fly to Shanghai on Friday to prepare a Sunday goodbye party at his
million-dollar art studio meant to draw attention to its pending destruction. In telephone
interviews this week, Mr.~Ai said he built the studio only after Shanghai officials, on a campaign
to burnish the city's cultural credentials, implored him to. But in July, they ordered the finished
building demolished at the command of anonymous higher-ups.

Mr.~Ai's response was the party, to be attended by eight rock bands and up to a thousand supporters
from around China. But on Thursday night, he said, the officers came to his home and asked him not
to go to Shanghai.

On Friday, after he said he was going anyway, the officers placed him under house arrest --
reluctantly, Mr.~Ai said.

``They're sorry, very sorry,'' he said by telephone from his home. ``They say they understand me and
really agree, but this is really beyond what they can do.''

Mr.~Ai said the officers told him that ``Shanghai is very nervous'' about the party. Like Mr.~Ai,
however, they did not know precisely who in Shanghai was nervous, or how they managed to arrange his
confinement in a city 650 miles away.

Mr.~Ai said he did not even know why the unnamed Shanghai officials had ordered his studio
demolished, although he had his theories.

This is not the first run-in with the authorities for Mr.~Ai, an artistic polymath who seems to be
alternately tolerated and hectored by higher-ups. An internationally known sculptor, filmmaker,
architect and performance artist, he helped design the Bird's Nest stadium for the 2008 Beijing
Olympics, then renounced his role after deciding that Chinese leaders had politicized the Games.

He was allowed to fly to Munich last year to stage a major exhibit that excoriated the government's
handling of children's deaths in the 2008 Sichuan earthquake. Yet months before, he was so severely
beaten by the police in Chengdu, the capital of Sichuan Province, where he had gone to testify in
the trial of a fellow activist, that he needed surgery to drain blood from his brain.

Mr.~Ai's latest run-in with Shanghai officials appears to exemplify that love-hate relationship.

As he tells it, he was approached more than two years ago in Beijing by the mayor of one of
Shanghai's districts -- a government unit not unlike an American city ward -- and beseeched to build
a studio on an abandoned plot of farmland. Initially suspicious -- ``I told my assistant we're not
going to deal with government anymore,'' he said; ``there's no honesty there'' -- he relented when
the mayor flew to Beijing for a personal appeal.

Mr.~Ai said he worked closely with the district to rehabilitate an abandoned warehouse on the site,
spending about \$1 million to create a vast working space fronting on a lake with a sawtoothed roof
and sides laced with a concrete grid. Other artists began building their own adjacent studios.

Then last July, as work was wrapping up, there came a city order to tear down the warehouse.

``They said only we received the notice,'' he said. ``The other artists did not. We said, `Why?' and
they said, `Well, you should know, because of Ai Weiwei's activities.' ''

Which activities offended someone is, of course, not known. But Mr.~Ai said he suspected he rankled
officials in 2008, when his blogging on the case of Yang Jia, who murdered six Shanghai policemen
after being arrested and beaten for riding an unlicensed bicycle, created a national sensation.
Mr.~Yang was later executed. He said that officials also might resent his documentary this year on
Feng Zhenghu, a lawyer and activist who spent more than three months in Tokyo's Narita Airport after
Shanghai officials denied him entry to the country.

Whatever the reason, Mr.~Ai said, the district official who first recruited Mr.~Ai returned to
Beijing this week, apologizing profusely and promising to compensate him for the cost of the
renovation if he would leave.

``I said, `Why? It took so much effort and energy, and you didn't give us a clear reason,' '' he
said. ``But they cannot really answer these questions. So I realize it's inevitable. They'll destroy
the building.''

At the planned goodbye party for the studio, in lieu of chips and dip, Mr.~Ai planned to serve river
crabs -- a sly reference to the Mandarin word hexie, which means both river crab and harmonious.
Among critics of China's censorship regime, hexie has become a buzzword for opposition to the
government's call to create a harmonious society, free from dissent.

In short order, 800 supporters from across China made plans to attend, and eight bands volunteered
to play at the event. ``They already call it Woodstock,'' he said Wednesday in an interview. ``I
think it's nice. It shows a kind of understanding and solidarity.''

On Friday, Mr.~Ai said he thought the unnamed Shanghai powers were taken aback by the attention to
the demolition and the party and reacted in typical fashion. And by doing so, they created a piece
of performance art that called more attention to the embarrassment they were seeking to suppress.

``They put you under house arrest, or they make you disappear,'' he said. ``That's all they can do.
There's no facing the issue and discussing it; it's all a very simple treatment.

``Every dirty job has to be done by the police. Then you become a police state, because they have to
deal with every problem.

``I think they hate me,'' he said. ``But I never imagined they would destroy an entire building.''

\section{Bloomberg Questions U.S.~Inquiry Into China Trade Practices}

\lettrine{C}{riticizing}\mycalendar{Nov.'10}{10} China was a popular campaign tactic for Democrats
and Republican candidates alike in many campaigns this year, but Mayor Michael R.~Bloomberg of New
York was quick to leap to China's defense on Saturday.

``It is very dangerous for us as a society -- I'm speaking of America -- to focus on blaming others,
because what you do then is you don't focus on your own practices,'' Mr.~Bloomberg said at a news
conference here.

He spoke after assuming the chairmanship of a coalition of 40 city governments around the world
concerned about climate change.

Mr.~Bloomberg was openly skeptical of the Obama administration's decision on Oct.~15 to open a broad
investigation into whether China violated World Trade Organization rules by subsidizing its exports
of solar panels and other clean-energy products to the United States and by restricting imports.

``Let me get this straight,'' Mr.~Bloomberg said, ``there is a country on the other side of the
world that is taking their taxpayers' dollars and trying to sell, subsidize things so we can buy
them cheaper and have better products, and we're going to criticize that?''

He said the United States also subsidized ``an enormous number of industries.''

The World Trade Organization has restrictions on many kinds of subsidies. But its toughest bans are
on export subsidies, in which a government tries to use its money to help its country's companies
buy market share in another country.

White House officials have acknowledged that the United States has subsidized the research,
development and deployment of clean-energy technologies. But they have denied that the United States
was subsidizing their export.

Zhang Guobao, the director of China's National Energy Administration, strongly criticized the
American investigation of Chinese practices at a news conference on Oct.~17 and emphasized that the
United States had clean-energy subsidies; he did not draw a distinction between domestic subsidies
and export subsidies.

Mr.~Bloomberg also did not make that distinction. But he said some trade disputes could have merit.

``It isn't that I think there isn't some justification to some of these trade disputes,'' he said,
``but I think it is so dangerous for America'' to lose its focus on bigger issues. Mr.~Bloomberg
added that he believed Chinese people who had ideas for great new businesses should be allowed to
immigrate to the United States.

Mr.~Bloomberg praised China for showing a much greater interest lately in environmental protection,
even as he criticized the country for having allowed severe water pollution and other problems.

The mayor disavowed again on Saturday any interest in pursuing the presidency. He expressed sympathy
for the challenges President Obama was facing, without providing specifics, and talked repeatedly
about how attractive it was to be mayor.

In New York, the mayor's office offered a terse comment on Saturday over a remark that Rupert
Murdoch said Mr.~Bloomberg had made about Mr.~Obama after playing golf with the president over the
summer.

In an interview with The Australian Financial Review, Mr.~Murdoch said that after the outing on
Martha's Vineyard, Mr.~Bloomberg ``came back and said, `I never met in my life such an arrogant
man.' ''

Jessica Scaperotti, a spokeswoman for Mr.~Bloomberg, said in an e-mail that ``the mayor remembers
the conversation differently. As he has said many times, he believes all Americans should be rooting
for the president to succeed.''

The White House press office did not respond to requests for comment.

\section{Licensing Fees the Main Topic of Oracle Testimony}

\lettrine{L}{awrence}\mycalendar{Nov.'10}{10} J.~Ellison, Oracle's chief executive, added star power
but no fireworks to a copyright infringement trial pitting his company against its rival SAP.

He instead testified on Monday in Federal District Court about the unglamorous issue of licensing
fees, a central point in the legal battle over how much SAP should pay in damages for stealing
Oracle's software. SAP has already conceded liability.

Court watchers had hoped that the outspoken chief executive would enliven the proceedings with
comments about SAP or Hewlett-Packard, the Oracle rival now run by a former SAP executive.

Oracle is seeking \$2 billion in damages against SAP, which has acknowledged that its former
subsidiary, TomorrowNow, made illicit copies of Oracle's software and manuals. But SAP argues that
Oracle is exaggerating the potential harm and that it should pay only \$40 million.

Mr.~Ellison's testimony was the high point so far in what is one of the most closely watched legal
battles in Silicon Valley. The interest is not so much about the details of the case, given SAP's
admissions of guilt, but rather the spectacle of public bickering among technology industry titans.

Mr.~Ellison, known as a fierce adversary, has used the case to attack L\'eo Apotheker, SAP's former
chief executive and now chief executive of Hewlett-Packard. He has said in public statements that
Mr.~Apotheker oversaw a vast copyright infringement plan.

But Mr.~Apotheker's name was not mentioned during Mr.~Ellison's testimony on Monday.

Oracle has tried to serve Mr.~Apotheker with a subpoena to appear in court, but its lawyers have
been unable to find him.

In a statement last week, Oracle said that Mr.~Apotheker was apparently staying away from H.P.'s
headquarters to avoid being served.

H.P. has responded that Oracle had the opportunity to ask Mr.~Apotheker any question it wanted
during an earlier deposition.

Mr.~Ellison said in court that Oracle would have charged SAP around \$4 billion to license the
software. He explained that the price tag would be that high because it would open the door for SAP
to poach up to 30 percent of the customers Oracle got through several major acquisitions.

``If they could get that software for nothing, we'd have a hard time paying 100,000 employees,''
Mr.~Ellison said Monday in response to a question from David Boies, the prominent lawyer from the
firm Boies, Schiller \& Flexner who is representing Oracle.

Mr.~Ellison seemed at ease during his testimony before Judge Phyllis J.~Hamilton and an eight-person
jury. He sometimes disputed the assertions of the opposing lawyer, Gregory Lanier, of the law firm
Jones Day, who tried to hammer the message that the damages should be lower than Oracle contends.

He repeatedly asked Mr.~Ellison to point to any e-mails or other internal documents showing that
Oracle had ever feared that it would lose a large number of customers to SAP.

``I do not write those kinds of things down,'' Mr.~Ellison responded.

In the end, Oracle lost 358 customers, or 2 percent to 3 percent. Some of those customers, SAP has
argued, would have left anyway.

Mr.~Lanier also showed e-mails in which Oracle executives wrote of seeing few defections to
TomorrowNow, which SAP acquired in 2005.

Mr.~Ellison replied that he was very concerned about the potential threat from TomorrowNow and that
he did not necessarily agree with early observations by other Oracle executives.

In any case, Mr.~Ellison said that no one at Oracle had been aware that TomorrowNow's employees were
stealing Oracle's intellectual property until later. If he had discovered it earlier, he said, he
and his colleagues would have considered the risk far greater.

\section{Color Comes to E Ink Screens}

\lettrine{E}{-book}\mycalendar{Nov.'10}{10} readers are lightweight and use little power, but most
have a distinct disadvantage to colorful tablet computers: their black-and-white displays.

But on Tuesday at the FPD International 2010 trade show in Tokyo, a Chinese company will announce
that it will be the first to sell a color display using technology from E Ink, whose black-and-white
displays are used in 90 percent of the world's e-readers, including the Amazon Kindle, Sony Readers
and the Nook from Barnes \& Noble.

While Barnes \& Noble recently announced a color Nook and the Apple iPad has a color screen, both
devices use LCD, the technology found in televisions and monitors. The first color e-reader, from
Hanvon Technology, based in Beijing, has an E Ink display.

``Color is the next logical step for E Ink,'' said Vinita Jakhanwal, an analyst at iSuppli. ``Every
display you see, whether it's a TV or a cellphone, is in color.''

Jennifer K.~Colegrove, director of display technologies at DisplaySearch, said it was a milestone
moment. ``This is a very important development,'' Ms.~Colegrove said. ``It will bring e-readers to a
higher level.''

E Ink screens have two advantages over LCD -- they use far less battery power and they are readable
in the glare of direct sunlight.

However, the new color E Ink display, while an important technological breakthrough, is not as sharp
and colorful as LCD. Unlike an LCD screen, the colors are muted, as if one were looking at a faded
color photograph. In addition, E Ink cannot handle full-motion video. At best, it can show simple
animations.

These are reasons Amazon, Sony and the other major e-reader makers are not yet embracing it. Amazon
says it will offer color E Ink when it is ready; the company sees color as useful in cookbooks and
children's books, and it offers these books in color through its Kindle application for LCD devices.
Sony is also taking a wait-and-see approach.

``On a list of things that people want in e-readers, color always comes up,'' said Steve Haber,
president of Sony's digital reading business division. ``There's no question that color is extremely
logical. But it has to be vibrant color. We're not willing to give up the true black-and-white
reading experience.''

But Sriram K.~Peruvemba, an E Ink vice president, is not upset by the reluctance of the market
leaders to adopt his color technology. ``I'm convinced that a lot of times it takes one company to
prove the market,'' Mr.~Peruvemba said.

While barely known in this country, Hanvon is the largest seller of e-readers in China. Its founder
and chairman, Liu Yingjian, says Hanvon has a 78 percent share of the Chinese market.

Hanvon's first product using a 9.68-inch color touch screen will be available this March in China,
starting at about \$440. The price is less than an iPad in China, which sells for about \$590. It
will be positioned as a business product, with Wi-Fi and 3G wireless connectivity.

``It's possible that we'll sell this in the U.S.~as well,'' Mr.~Liu said. Hanvon sells other
products, like tablets and e-readers, to Americans online and through Fry's, a regional electronics
chain.

E Ink, based in Cambridge, Mass., was bought by Prime View Holdings of Taiwan in 2009 and was
recently renamed E Ink Holdings. To create the color image, E Ink uses its standard black-and-white
display overlaid with a color filter. As a result, battery life is the same as its black-and-white
cousins, measured in weeks rather than hours, as with the iPad. The color model from Hanvon can be
easily read in bright light, although the color filter does reduce the brightness.

The Hanvon e-reader is not intended to be a multifunction competitor to the iPad, but rather a
dedicated reading device, like the Kindle. Ms.~Colegrove of DisplaySearch said these types of
lower-cost products should continue to gain market share, growing from four million units sold
worldwide in 2009 to 14 million units by 2011. At the same time, slate-type devices like the iPad
will increase from one million in 2009 to 40 million in 2011, she predicts.

``Color is absolutely critical for E Ink,'' said James McQuivey, an analyst at Forrester Research.
``Without it, they'll either be replaced by LCD displays or other competitors.''

\section{Phone Apps Aim to Fight Harassment}

\lettrine{I}{t}\mycalendar{Nov.'10}{10} was almost a year ago that a subway ride turned Violet
Kittappa into an activist.

She was on a crowded Queens-bound W train when she noticed a middle-aged man inching toward her. She
tried to concentrate on her book and ignore him, when she felt a nudge on her leg. The man was
breathing heavily, and his face was contorted. When she looked down, she realized that he was
rubbing against her.

``This is when I flip out,'' Ms.~Kittappa wrote shortly afterward on ihollaback.org, a Web site that
encourages women to post their accounts of harassment and abuse as part of a campaign to end
practices that are seldom discussed but that many women say are pervasive.

As a movement, it seems to be gaining traction. The City Council's Committee on Women's Issues held
its first hearing on street harassment last month. At the same time, technology has enabled women to
document what previous generations could not. They can now take photographs of the men who accosted
them and publicize the pictures, which allows women to assist the police or embarrass those who have
harassed them.

Now Hollaback -- which in this case refers to the act of responding to harassment on the Web -- is
expanding its service to mobile devices. This week, the group is releasing an iPhone application
that allows users to report harassment in seconds. The data is automatically mapped, and a follow-up
e-mail from Hollaback asks for a more detailed account of what happened.

Interactive technology is particularly apt for documenting and combating street harassment, said
Emily May, the executive director of Hollaback. The more people use the application, the more
valuable the database becomes, she said. The group hopes to eventually share its information with
the authorities and use it to identify ``hot spots'' and catch offenders.

``Street harassment teaches us to be silent,'' Ms.~May said. ``It teaches us to walk on. That is the
very last thing we want to be teaching women and girls.''

The iPhone application, which cost about \$15,000 to develop, was financed largely through online
donations of \$10 or less and built by OrangeMico, a Brooklyn technology firm. An Android
application, which will follow, was created pro bono by Jill P.~Dimond, a doctoral candidate at
Georgia Tech who is researching domestic violence and technology. The applications costs 99 cents.

Ms.~May founded the group with friends five years ago and became its only paid staff member earlier
this year, coordinating 20 volunteers out of an office in her home in Brooklyn. The organization now
has chapters in six American cities, along with others in Britain, Canada and Australia.

``The Internet speeds everything up,'' Ms.~May said. ``If we as activists can't get the Internet to
speed up social change, then we're not doing our jobs.''

A parallel, and more traditional, effort is also under way. At the City Council hearing, activists
called for a study examining street harassment, an advertising campaign and the establishment of
``harassment-free zones'' around schools. The committee chairwoman, Julissa Ferreras, a Democrat
from Queens, has pledged to pursue all three.

Ms.~Ferreras called the hearing after visiting a high school in Elmhurst, Queens, where instead of
requesting computers or supplies, female students asked her to stop the mechanics in a nearby garage
from making inappropriate comments directed at them.

Their stories hit close to home for the councilwoman, who grew up in the district and still
remembers the ``speed walk'' to and from school, when she would try to avoid a group of men who
stood outside one particular bodega.

``You tense up, you try not to make eye contact, walk by as fast as you can,'' she said.

It is still an almost daily occurrence, Ms.~Ferreras said, adding that she was even harassed on her
way to the hearing.

Paul J.~Browne, the Police Department's chief spokesman, said that following or threatening a woman
would constitute a criminal offense.

``Certainly the unwanted sexual touching associated with groping is a crime, and we urge any victim
to come forward with a description of the suspect,'' he said. ``If she feels she can safely capture
his image with a cellphone camera, all the better.''

Mr.~Browne added that the department had a program involving the transit bureau and the Special
Victims Division to address harassment and forms of sexual abuse in the subways. (Last November, the
Council also had a hearing on that subject.)

Yet it is sometimes unclear when the line between free speech and harassment has been crossed, and
that makes some women hesitant to come forward, said Ms.~Kittappa, who now volunteers for Hollaback.
She said she went to the police last year and picked the man who accosted her from photographs of
known sex offenders, but he remained at large.

Ms.~May and others say the problem will not be solved solely by the police. What is required is a
shift in the culture that sometimes tolerates such behavior.

``Women don't put up with harassment in the workplace or the home,'' Ms.~May said. ``Why are they
still putting up with it on the street?''

\section{Shifting Health Costs to High Earners}

\lettrine{W}{ith}\mycalendar{Nov.'10}{10} health care costs climbing even higher during this
enrollment season, more employers are adopting a tiered system to pass on the bulk of those costs to
their employees by assigning bigger contributions to workers in top salary brackets and offering
some relief to workers who make less money.

For years, employees have seen what they pay toward health care go up as companies ask them to
contribute more to premiums and deductibles. But now, as people enroll in health plans for the
coming year, the sticker shock is more jolting than ever because so many companies are passing on to
their workers most, if not all, of the higher costs.

A worker's share of a family policy is approaching \$4,000 a year on average, and is most certainly
going to keep on rising through the next few years. For lower-salaried workers, those additional
costs have only compounded their struggle in a brutal economy.

More and more companies in the last year or so have begun signaling their recognition of the added
burden shouldered by workers in low- and middle-income jobs by varying the premiums they pay based
on salary. Consultants say the trend is likely to continue, as employers devise various ways of
spreading increased health care costs among their staff and balancing that side of the ledger
against fewer raises and other compensation.

Vanderbilt University, for instance, has adopted a wage-based benefit program for 2011 under which
premiums will remain the same for employees who make \$50,000 or less, while everyone else will pay
up to \$75 more a month. ``We're trying to help those lower-paid employees cope with hard economic
times,'' said Jerry G.~Fife, the vice chancellor for administration.

Even as companies warily eye the uncertain landscape of the new health care law, especially with the
Republican midterm election gains at the federal and state levels, they also are seeking novel ways
to deal with year-after-year increases in health care, because the share-the-pain era is coming to
an end.

Corporations had absorbed some higher costs in recent years, along with their workers, but have
recently passed all, on average, onto employees. In 2010 alone, a worker's share of the cost of a
family policy jumped an average of 14 percent from the previous year, according to a recent survey
by the Kaiser Family Foundation. In real money, that is an additional \$500 a year deducted from a
paycheck.

``It feels so much worse this year than it has in prior years,'' said Helen Darling, president of
the National Business Group on Health, which represents employers who provide health benefits.

Across the country, the percentage of workers with coverage in large companies whose premiums vary
with their wages climbed to 17 percent in 2010, up from 14 percent two years ago. About 20 percent
of employees who are covered by large companies in the Northeast, which has suffered from a
combination of high unemployment and steep medical costs, have the premiums they pay tied to their
wages, according to Kaiser.

``If health care reform hadn't happened, there would be more companies going in this direction,''
said Ms.~Darling, alluding to the interim period between the law's passage this year and 2014, when
it is expected to take full effect.

Some corporations have gone further than others in trying to spare their lowest-paid workers, even
as they increased the cost of premiums for everyone else. This year, for example, employees at Bank
of America who make \$100,000 or more a year will pay at least 14 percent more for coverage for
2011. But workers who make less will actually see their contributions decrease, although their
deductibles and co-payments will stay the same. Employees earning less than \$50,000 could see as
much as a 50 percent drop in the amount deducted from their paychecks, as compared to 2010. The bank
says it is making up the difference.

``We're obviously committed to helping our associates and their families manage rising health care
costs,'' said Kelly Sapp, a spokeswoman for the bank, which employs 250,000 people with a wide range
of salaries nationwide.

At Vanderbilt, which employs 23,000 people, including groundskeepers, janitors and health care
aides, the university was mindful that employees have gone with no or small raises for the last two
years, the university typically pays about 80 percent of the cost of coverage, with employees paying
anywhere from \$41 to \$370 a month in premiums, depending on the type of plan.

The concept of tiered plans is not new, with employers that typically offer generous benefits, like
universities, being quicker to try it. General Electric, for example, has long divided its workforce
into separate tiers to determine how much an employee has to contribute toward insurance coverage.
``Employees who are higher paid can afford to pay more,'' said a spokeswoman.

Often, companies will keep premiums steady or lower for low-income workers by asking them to pay
less of the overall increase, while high-income workers will pay more to make up for the difference,
according to Joshua Miley, a principal at HighRoads, a Woburn, Mass., health benefits management
consultant. Faced with an overall increase of 9 percent, the company might ask the lower-paid
workers to pay 4 percent more, while the higher-paid group would pay 14 percent more. ``They're
doing it on the backs of the higher paid,'' he said.

Of course, some companies may be reluctant to ask certain of their employees to pay more, especially
if workers belong to unions that have negotiated a certain level of benefits for all their members.

Other companies could be wary of carrying out a system that could be viewed as unfair since salary
may not be the best indication of household income. Some low-paid employees, whose spouse is a high
earner, may not need the help, while a single person with a higher salary could be struggling.

The other concern is not being able to move away from a wage-based benefit structure once it is in
place. ``Once you ring that bell, you can't take that out,'' Mr.~Miley said.

For most workers, however, the trend has been very clear: the increase in health care costs has
easily outstripped any rise in their incomes. Since 2005, while wages have increased 18 percent,
workers' contributions to premiums have jumped 47 percent, almost twice as fast as the rise in the
policy's overall cost, according to Kaiser.

Companies have also become increasingly creative in the ways they shift costs. Instead of simply
raising premiums or increasing the size of the deductible workers must pay before their coverage
kicks in, employers are increasingly asking their workers to pay more the cost of coverage for their
dependents, or to pay more of their share of a hospital stay or an emergency room visit. ``Employers
do a bit here and do a bit there,'' said Gary Claxton, a policy expert at Kaiser.

The result is that while employees may be paying more, they may not know how much more.
``Deductibles are straightforward, and even co-payments are straightforward,'' said Mark Rukavina,
the executive director of The Access Project, a Boston advocacy group. ``You need multiple spread
sheets to figure this out.''

During her company's open enrollment period, Marilee Fisher, for example, tried to scrutinize the
three plans being made available by her employer. She is liable for any medical bills she
accumulates when her 5-year-old son, who has Down syndrome, has therapy more times than allowed
under the specific plan. ``Every year it changes, and I don't know it,'' she said.

More companies are adopting plan designs that require employees to pay more of their own medical
bills under specific circumstances so workers are increasingly feeling the pinch. Companies ``are
taking the usual cost-trend reduction measures, but more of them are doing it,'' said Beth Umland,
director of health and benefits research at Mercer, the consulting firm.

All of this complexity has clouded exactly how much more workers are paying, consultants and others
say. ``Employers have shifted costs for the past 10 years,'' Mr.~Miley said. ``The confusion allows
them to push it further.''

\section{China Looms Large in Luxury Industry's Vision}

\lettrine{W}{hen}\mycalendar{Nov.'10}{10} Chinese shoppers stride into a Gucci store these days,
they had better be shown the highest quality of the couturier's haute de gamme.

``The first thing they say now is, `Don't treat me like a Chinese and just show me leather,'''
Patrizio di Marco, the president and chief executive of Gucci, said Tuesday. Instead, they demand
the best shoes and bags in crocodile or python -- especially ones that brandish the badge of
heritage.

If heritage is the tool fashion houses have turned to in the wake of the global financial crisis,
then the actual market the luxury industry sees guaranteeing its future is China, according to
Mr.~di Marco and other speakers at the luxury conference convened by the International Herald
Tribune. And, while China booms, the industry is turning back to basics with more mature markets,
appealing to the emotions of consumers who have become far more discerning, by dusting off legacy
products.

``Before the crisis, consumers didn't feel depressed -- people still bought on impulse,'' said
Mr.~di Marco. ``Now the impulse is pretty much gone. That's good for the industry because you have
to live through a time of crisis,'' and eventually companies start emphasizing ``what's important
because that's what consumers want.''

But China is also turning to luxury to validate its rapid evolution from a basic emerging market
into a sophisticated economic powerhouse -- and heritage products are the most desirable way to
burnish that image, according to the industry leaders speaking here.

China is now the major driver of growth for luxury goods. The industry sees the country as having
almost unlimited potential over the next 5 to 10 years.

But China's nouveau riche are also undergoing a remarkable transformation that has luxury firms
scrambling to capitalize on its own rich heritage.

``Five years ago, everyone was talking about the BRICs,'' said Michele Norsa, chief executive and
managing director of Salvatore Ferragamo Italia, using the acronym for the world's fastest-growing
emerging markets: Brazil, Russia, India and China. ``Now the difference between China and the others
is huge,'' he said.

China has leapt ahead of other emerging markets in part by investing in airports, railways and
highways that connect cities where income levels, and the demand for luxury as a badge of new
wealth, are rising rapidly.

Mr.~Norsa noted that airports like the new one in Beijing will see tens of millions of people
passing through each year. These ``new cathedrals'' are the markets of the future, he suggested.

But, like Jennifer Woo, president of the 160-year-old Hong Kong-based chain Lane Crawford, he noted
that China is a market where firms have to work hard to win customers. ``The reality is: China is
not easy,'' Ms.~Woo said.

Unlike Russia and India, where only the wealthy tend to buy up-market, China's middle class is
interested in luxury and ready to pay luxury prices.

Perhaps more important, luxury industry executives said, Chinese consumers are going out of their
way to seek designs that have a unique way of burnishing their own heritage.

``There is nothing more dangerous than creating with a formula,'' said Alber Elbaz, the artistic
director at Lanvin, who has dipped time and again into the firm's archives for inspiration.

Lanvin is one of the few houses lucky enough to have a long heritage. Tommy Hilfiger, the American
designer, lacks the decades-long history of some of his competitors. But marking 25 years in
fashion, he has turned to the American heritage to build an identity that has given his brand appeal
across a range of ages and cultures, allowing him to engineer a strong turnaround for his firm in
recent years.

At Burberry, Angela Ahrendts, the chief executive, and Christopher Bailey, the chief creative
officer, have also tapped heritage to secure a turnaround that began five years ago when the two
decided to dust off a ``rough diamond'' -- the famous Burberry trench coat -- and turn it into a
polished, must-have item that appeals across genders and generations around the world.

``You can really only build something new if you destroy the old,'' said Karl Lagerfeld, the
designer, in a conversation about the industry and his relationship with the heritage of Coco
Chanel. Chanel, he said, lost the respect of her peers in the 1960s when she dismissed jeans and
miniskirts. Today, Mr.~Lagerfeld suggested, haute designers like Mr.~Elbaz do well to heed stores
like H\&M. ``Inexpensive and very expensive have more future than what is in between,'' he said.

Stories like Burberry's have particular appeal in China. ``In the last 30 to 40 years they forgot
about their past,'' said Mr.~Norsa. ``Today they want to get in touch with it: They remember they
were an empire and aspire to be more sophisticated than they are now.''

Brands with a long family history, like Herm\`es, particularly appeal to the Chinese ``self-made
man,'' who has worked hard to obtain a higher rank and are educating themselves on the value of what
they buy.

``Five years ago, people there couldn't distinguish French wine from Italian wine,'' Mr.~Norsa
noted. ``Now they have developed a knowledge of brands in a very short time.''

\section{Court Weighs Whether Consumers Can Be Deprived of Class-Action Remedy}

\lettrine{T}{he}\mycalendar{Nov.'10}{10} most significant business case of the Supreme Court term so
far involves a \$30 charge for what was said to be a free mobile phone.

But lawyers on both sides argued on Tuesday that the court's ruling in the case could radically
reshape the handling of disputes that arise when consumers sign standard contracts that require
disputes to be settled through arbitration.

Deepak Gupta, a lawyer for a California couple who filed a class-action lawsuit against AT\&T
Mobility accusing it of fraud over the \$30 charge, said a ruling for the company would spell the
end of class actions in all sorts of cases.

``If you preclude class-wide relief,'' he told the justices, ``that will gut the state's substantive
consumer protection law because people will, in the context of small frauds, not be able to bring
those cases.''

Andrew J.~Pincus, a lawyer for the company, said a ruling against his client would sound a different
sort of death knell, this one for the arbitration provisions that were common in many standard-form
contracts.

Recent Supreme Court decisions have generally favored the enforcement of arbitration agreements and
have been wary of aspects of class-action litigations.

But it was hardly clear at Tuesday's arguments that those two trends would continue in the latest
case.

That was because it included a third, confounding element. To rule for the company, the justices
would have to reject a decision of the California Supreme Court. That court said that class-action
waivers in standard-form contracts, whether applicable to arbitration or litigation, were
unconscionable under state law.

``Are we going to tell the State of California what it has to consider unconscionable?'' Justice
Antonin Scalia asked.

The contract agreed to by the couple, Vincent and Liza Concepcion, required them to resolve their
disputes through the informal mechanism of arbitration and barred them from banding together with
others to seek class-action treatment, whether in arbitration or in traditional litigation in court.

Applying the California Supreme Court's decision, federal courts in California allowed the
Concepcions' lawsuit to proceed as a class action in court.

The company appealed to the United States Supreme Court, arguing that the effect of the class-action
ruling was to discriminate against arbitrations in violation of the Federal Arbitration Act, which
generally overrides state laws unfavorable to arbitration and requires courts to enforce arbitration
agreements unless a given state law limitation applies to all kinds of contracts.

Much of Tuesday's argument revolved around a semantic point: Is a state law prohibiting class-action
waivers one that applies to all contracts or one that specially disfavors arbitration contracts?

Justice Elena Kagan said the state's approach appeared neutral. ``Its rule applied both in the
arbitration sphere and the litigation sphere,'' she said of the California Supreme Court decision.

Justices Ruth Bader Ginsburg and Stephen G.~Breyer seemed to agree. ``The rule is the same whether
it's litigation or arbitration,'' Justice Ginsburg said.

Justice Breyer added: ``Those principles apply to litigation. They apply to arbitration. So what's
the problem?''

Mr.~Pincus said the state's approach had a disproportionate effect on arbitrations, particularly in
light of a Supreme Court decision in April that class-wide arbitrations could not be compelled
unless arbitration agreements expressly contemplated them.

Neutrality as between arbitration and litigation is, Mr.~Pincus said, a ``gerrymandered category.''

Justice Breyer acknowledged the existence of such categories, using a colorful example.

``I would guess it's like Switzerland having a law saying we only buy milk from cows who are in
pastures higher than 9,000 feet,'' he said. ``That discriminates against milk from the rest of the
continent.''

``But to say we want cows that have passed the tuberculin test doesn't,'' he added. ``So I guess we
have to look at the particular case.''

Here, he said, ``class arbitration exists,'' meaning that a state law requiring that class actions
be available across the board does not specially affect arbitrations.

``So where is the 9,000-foot cow, or whatever it is?'' Justice Breyer asked. ``Where is the
discrimination?''

In its main brief in the case, AT\&T Mobility v. Concepcion, 09-893, the company said ``no rational
business will agree'' to class-wide arbitration, which it called ``a lose-lose proposition'' with
all the cost and risk of litigation but none of the procedural protections and appellate oversight.

Mr.~Pincus added that his client's arbitration agreement was unusually generous, a point
acknowledged by the courts, and that the Concepcions would have recovered all they had lost and
perhaps more in arbitration. He added that the lower courts were wrong to allow a class action in
the case on the theory that it was needed to vindicate the rights of consumers other than the
Concepcions.

That point seemed to resonate with Justice Samuel A.~Alito Jr.

``Traditional unconscionability in California and elsewhere focuses on unfairness to the party who
is before the tribunal,'' Justice Alito said. ``So here it would be unfairness to the Concepcions,
rather than unfairness to other members of the class who are not before the court.''

Justice Kagan responded that the state's approach might be curious but was still neutral.

``It may be a good unconscionability doctrine or it may be a bad unconscionability doctrine, but
it's the state's unconscionability doctrine.''

\section{Pentagon Openings Give Obama Options}

\lettrine{W}{ith}\mycalendar{Nov.'10}{10} critical decisions ahead on the war in Afghanistan,
President Obama is about to receive an unusual opportunity to reshape the Pentagon's leadership,
naming a new defense secretary as well as several top generals and admirals in the next several
months.

It is a rare confluence of tenure calendars and personal calculations, coming midway through
Mr.~Obama's first term and on the heels of an election that challenged his domestic policies. His
choices could have lasting consequences for his national security agenda, perhaps strengthening his
hand over a military with which he has often clashed, and are likely to have an effect beyond the
next election, whether he wins or loses.

That is all the more reason that Mr.~Obama's choices are certain to face scrutiny in a narrowly
divided Senate, whose Republican leadership has declared itself intent on defeating him.

Defense Secretary Robert M.~Gates has said he plans to retire next year, while the terms of four
members of the Joint Chiefs of Staff are scheduled to end: Adm. Mike Mullen, the chairman;
Gen.~James E.~Cartwright, the vice chairman; Gen.~George W.~Casey Jr., the Army chief; and Adm. Gary
Roughead, the chief of naval operations.

Andrew J.~Bacevich, a retired Army officer who is a professor of history and international relations
at Boston University, said this round of replacements, coming after two years of difficult and
sometimes intense wrangling over how to carry on the war in Afghanistan, ``is particularly
important, and is likely to prove particularly difficult.''

``The challenge facing the president,'' Mr.~Bacevich said, ``will be to identify leaders who will
provide him with disinterested advice, informed by a concern for the national interest, and, in
doing so, to avoid either the appearance or the reality of politicizing the senior leadership.''

At the top of the new pantheon of military power, the president needs a heavyweight to succeed
Mr.~Gates, an unexpected holdover from the Bush administration who stayed longer than many expected
to become perhaps the most influential member of the Obama cabinet.

White House officials say the president is not prepared to announce any decisions on his new slate
of Pentagon and military leaders for next year.

But speculation for the top Pentagon job in recent days has included two respected veterans on
military matters, both with bipartisan credentials and hands-on experience: John J.~Hamre, a deputy
defense secretary in the Clinton administration who now leads the Center for Strategic and
International Studies while running the Defense Policy Board, an advisory panel to Mr.~Gates; and
Representative Ike Skelton of Missouri, who lost his seat last week and with it the chairmanship of
the House Armed Services Committee.

Another name certain to be on Mr.~Obama's list is Michele A.~Flournoy, currently the Pentagon's
under secretary for policy and one of the foremost national security specialists of the
up-and-coming generation. Her appointment would allow Mr.~Obama to claim another first in naming a
woman to become defense secretary, something he could also accomplish by moving Hillary Rodham
Clinton into the job from secretary of state.

Other possible candidates include Ray Mabus, the Navy secretary who formerly served as governor of
Mississippi and ambassador to Saudi Arabia; Senator Jack Reed of Rhode Island, an Army veteran; and
Richard J.~Danzig, a former Navy secretary.

It also is tricky to pick members of the Joint Chiefs, who are not only the president's senior
military advisers on questions of war but also the leaders of the individual military services, at
the fulcrum of competing missions and constituencies that are never easy to balance.

A thorough revamping of the high command is especially complex at a time of persistent challenges on
so many fronts, like the wars in Afghanistan and Iraq, the threats from Iran and North Korea and the
challenges from Russia and China, all while facing the constant risk of a terrorist attack.

The new military leadership, after years of war and economic crisis, also has to cope with strains
on military budgets, while caring for the health and morale of a force that simultaneously must be
modernized.

There are lingering strains between top civilian aides to Mr.~Obama and the military brass, over
issues as diverse as how to fight and wind down the wars in Afghanistan and Iraq, and how to allow
military service by openly gay troops.

Any commander in chief is theoretically free to replace his top civilian and military subordinates
whenever he chooses, but it rarely happens all at once.

Traditionally, a new president appoints a new defense secretary and allows the chiefs to serve out
their tours, which tend to fall more or less randomly across a president's term in office. Mr.~Obama
was the first to carry over a defense secretary who had served a president of a different party, and
Mr.~Gates's expected departure now falls coincidentally along with four members of the six-person
Joint Chiefs. (The Marines got a new commandant, Gen.~James F.~Amos, last month, and the Air Force
chief, Gen.~Norton A.~Schwartz, retires in 2012 unless he is rotated into another senior position,
which is possible.)

The changes in 2011 are so unusual, and the national security risks today so significant, that
General Cartwright, the vice chairman, offered to retire a year early, so that a new officer would
be in place for some continuity, according to Pentagon and administration officials familiar with
the discussions.

But General Cartwright -- known for his fluency in nuts-and-bolts issues like missile defense,
cyberwarfare and procurement -- has been described as one of the president's favorite officers, and
was asked to stay on. He is very likely on the list of those who would be considered for promotion
to chairman.

When Mr.~Obama had opportunities in recent weeks to replace top members of his White House inner
circle, including his chief of staff and his national security adviser, he opted to replace his
inaugural ``team of rivals,'' populated by outsiders, by promoting trusted confidants and moving
toward creating, in essence, a team of insiders.

So the questions before Mr.~Obama include trust and comfort and assurances that his policy decisions
will be executed the way he wants.

The expected candidates are all familiar to the White House, and all those on any list of suitable
officers have blue-chip r\'esum\'es but differing temperaments.

One of the first questions the president will have to answer for himself is what to do with
Gen.~David H.~Petraeus, the former Iraq commander asked to rush to Afghanistan when Mr.~Obama
relieved two commanders there in a row.

Supporters say that General Petraeus has earned the chairman's job by reason of experience,
intellect and sacrifice, but that may not satisfy some political advisers around the president, who
still resent the officers involved in the Afghanistan-Pakistan review last year. General Petraeus
also could remain longer in Afghanistan or could be offered the job of Army chief or that of supreme
allied commander in Europe, a post once held by Dwight D.~Eisenhower and so hardly a dead end.

In addition to Generals Cartwright and Petraeus, another potential candidate for chairman is Adm.
James G.~Stavridis, currently the NATO commander and one of the Navy's most intellectual officers;
he likewise could be slotted into the job of chief of naval operations.

Three other highly regarded Army officers remain in the hunt for any of the top jobs. They include
Gen.~Ray Odierno, a three-time Iraq commander now in charge of the military's Joint Forces Command;
Gen.~Martin E.~Dempsey, who had two tours in Iraq before serving as acting commander of American
forces in the Middle East and who now oversees Army training and doctrine; and Gen.~Peter
W.~Chiarelli, a two-time Iraq commander who is the Army's vice chief of staff, with a focus on
improving care for wounded soldiers.

\section{Obama Pointedly Questioned by Students in India}

\lettrine{W}{hen}\mycalendar{Nov.'10}{10} Michelle Obama, the first lady, introduced her husband to
a group of college students here on Sunday, she urged them to ask him ``tough questions.''

They did.

``What is your take or opinion about jihad?'' came the first question for President Obama at a
town-hall-style meeting at St.~Xavier's College. Next up were queries about spirituality, Gandhi,
the American midterm elections and his government's negotiations with the Taliban.

Finally, there was the question Mr.~Obama confessed he had been waiting for: Why hasn't the United
States labeled Pakistan ``a terrorist state''?

``Pakistan is an enormous country; it is a strategically important country,'' Mr.~Obama began,
meandering around to a defense of his administration. He said its policy was to ``work with the
Pakistani government in order to eradicate this extremism that we consider a cancer within the
country that can potentially engulf the country.''

The diplomatic response about the neighbor that India views with suspicion was indicative of the
fine line Mr.~Obama has walked on the topic of terrorism while here. On Saturday, his first day
here, he faced criticism in the local press when he paid homage to victims and survivors of the 2008
terrorist siege in Mumbai without mentioning that the gunmen were Pakistani or suggesting, that some
groups in Pakistan pose a terrorist threat.

While the students at St.~Xavier's, a 141-year-old Jesuit institution in this pretty seaside city,
were exceedingly polite to Mr.~Obama -- in interviews many said they admired him -- they seemed
unafraid to get straight to the point, even if Mr.~Obama did not always get straight to his.

``Well,'' he said, tackling the question about jihad, ``the phrase jihad has a lot of meanings
within Islam and is subject to a lot of different interpretations.''

He carefully avoided saying that he was opposed to jihad -- which has several meanings, including
both holy war and a personal quest for self-improvement -- and instead said: ``I think all of us
recognize that this great religion in the hands of a few extremists has been distorted to justify
violence towards innocent people that is never justified. And so, I think, one of the challenges
that we face is, how do we isolate those who have these distorted notions of religious war.''

Sunday's session, in a sunny outdoor courtyard surrounded by Gothic buildings, came on the second
day of a 10-day, four-nation swing that will also take Mr.~Obama to Indonesia, his boyhood home for
four years; South Korea; and Japan. He is spending the longest stretch, three days, in India. He
left Mumbai later Sunday for the capital, New Delhi, where he was expected to address Parliament on
Monday.

While Mr.~Obama was dancing around questions -- figuratively speaking -- on Sunday, he also
participated in some literal dancing, showing off some moves that, to the delight of photographers
traveling with him, are likely to provide iconic images of his trip. Mr.~Obama's short performance
came after student dancers doing a show for him implored him to join in.

The White House has cast the Asia trip as an economic mission that will also strengthen American
diplomatic ties with emerging democracies and established ones. On Sunday, Mr.~Obama toured a small
technology show here with the aim of showcasing American partnerships with India in expanding
agriculture.

Officials billed the college meeting as a chance for Mr.~Obama to connect with average Indians. But
for a president still bruised from the trouncing his party took in last week's elections, the
appearance was also a chance to come before the kind of sympathetic crowd he now has trouble
attracting at home.

``We call him the world king, king of the world,'' said Chetman Rawal, 20, who studies commerce at
the college. ``I think he will change the world.''

It seemed a common sentiment. In interviews, students and faculty members here uniformly spoke
kindly of Mr.~Obama, praising everything from ``his cuteness,'' as one female student said, to his
basketball skills, to his respect for ``Gandhian principles.'' On the question of how he applies
those principles, Mr.~Obama sounded a note of humility.

``I'm often frustrated by how far I fall short of their example,'' he said, referring to Mohandas
K.~Gandhi, the Rev. Dr.~Martin Luther King Jr.~and Abraham Lincoln, all of whom he said he was
studying. ``But I do think that at my best what I'm trying to do is to apply principles that
fundamentally come down to something shared in all the world's religions, which is to see yourself
in other people.''

Indians followed the American elections closely, said the Rev. Lawrie Ferrao, director of the
institute of communications at the college. But he said people here were more interested in another
election -- the one in which Mr.~Obama himself ran in 2008.

``The admiration for him with regard to his campaign, his optimism, his charismatic movement and
charismatic leadership, that I think has not faded off yet,'' Father Ferrao said. As to the
midterms, he gave an explanation Mr.~Obama himself might have offered: ``He was given an economy
which was unsustainable.''

The president himself, when questioned at St.~Xavier's about the elections, pledged some ``midcourse
corrections and adjustments,'' though he was not specific about his plans. But in his commentary on
Gandhi, he offered a lesson he had learned, one that perhaps provides some insight into how he might
be feeling these days.

``On this journey,'' Mr.~Obama said, ``you're going to experience setbacks and you have to be
persistent and stubborn, and you just have to keep on going at it. And you'll never roll the boulder
all the way up the hill, but you may get it part of the way up.''

\section{Obama Visits a Nation That Knew Him as Barry}

\lettrine{T}{he}\mycalendar{Nov.'10}{10} two houses where he spent part of his boyhood stand pretty
much the way they did when he went back to Hawaii four decades ago. The two schools he attended have
grown larger but, in spirit, remain unchanged. Some of his old friends can still be found around the
neighborhood.

Near one of his homes here, the same family still runs a wooden stall selling gado-gado, an
Indonesian salad covered in peanut sauce. Agus Salam, who took over the business from his mother
years ago, played soccer with the American boy everybody here called Barry.

``His house -- all the houses around here -- haven't changed,'' said Mr.~Salam, 56.

When President Obama visits Jakarta on Tuesday, he will find a city that, in some ways, has changed
beyond recognition. A city of one luxury hotel and one shopping mall when Mr.~Obama lived here
between 1967 and 1971, Jakarta is now the overextended and overcrowded capital of the world's fourth
most populous nation. But Jakarta's neighborhoods, including the two where Mr.~Obama lived, retain
enough of their former selves that the president would quickly find his bearings.

Jakarta regards Mr.~Obama as a local boy made good, and he remains extremely popular throughout
Indonesia. But his last-minute postponements of three previously planned visits here have clearly
sapped the enthusiasm surrounding his homecoming, even among his most ardent supporters.

``He's not as popular here as he was before,'' Mr.~Salam said.

In 1967, Indonesia was still reeling from the aftershocks of an attempted Communist coup that led to
the killing of at least 500,000 people. Suharto, the general who would rule Indonesia through the
late 1990s, was about to assume power and launch an authoritarian era called the New Order.

Mr.~Obama, his mother, Stanley Ann Dunham, and his Indonesian stepfather, Lolo Soetoro, moved into a
one-story house in a district called Menteng-Dalam. At the time, it was a new neighborhood where
natives of Jakarta, known as Betawis, lived with an increasing number of newcomers from different
corners of Java and Sumatra, the main islands in Indonesia. The area was connected to the electric
grid only a couple of years before Mr.~Obama moved in.

``It was a very poor area when the family came here,'' said Coenraad Satjakoesoemah, 79, a retired
airline manager and a neighborhood leader. ``There were still dirt roads, only a few houses and lots
of large trees.''

In Mr.~Satjakoesoemah's living room, Mr.~Obama's mother taught English to the neighborhood women,
including his wife, Djumiati. While the residents regarded Mr.~Obama's mother as a ``free spirit,''
Barry, who was chubby, was referred to as the ``boy who runs like a duck,'' said
Mrs.~Satjakoesoemah, 69.

Mr.~Obama, the couple said, attended school with children who could not afford to buy shoes.

The school -- Santo Fransiskus Asisi, a Roman Catholic school that had been founded just in 1967 --
is still located a couple of blocks away. When the 6-year-old Barry entered the school, there were
only three grades with a total of 150 students. Now, about 1,300 students from kindergarten through
high school study there, said the principal, Yustina Amirah. Mr.~Obama has spoken about growing up
here and hearing the Muslim call to prayer, but Ms.~Amirah said that since the school's founding,
everyone had hewed to the institution's official religion.

``Barry followed church services like everybody else,'' Ms.~Amirah said.

Sometime in the third grade, after his family moved to a different part of the city, Mr.~Obama
transferred to Elementary School Menteng 1, possibly the most famous primary school in Indonesia.
Founded as a Dutch colonial school in 1934, it has long drawn the children of the country's ruling
class because of its location in Menteng, traditionally the wealthiest residential neighborhood in
Jakarta.

Nowadays, though many wealthy Indonesians send their children to international schools here, the
Menteng public school still draws the children of the elite, so much so that the principal, Hasimah,
said she could ``count on one hand'' the students, out of a total of 400, who are not driven to
school every day by their parents or drivers.

A mosque was built on the school grounds in 2002, a sign of the growing influence of Islam in
Indonesia's public life. But the school four decades ago did not even have a prayer room, in keeping
with the state's secularism at the time, Ms.~Hasimah and students from the era said.

During the presidential campaign of 2008, right-wing American groups spread rumors that Mr.~Obama
had attended a radical madrasa while living here. Though most of the Menteng school's students have
always been Muslim, Rully Dasaad, 49, a former classmate, chuckled at the idea that of all schools
in the country, Menteng was equated with a madrasa.

``I was brought to school in a Cadillac,'' Mr.~Dasaad said.

But Mr.~Obama's family did not live in the exclusive Menteng district. The family stayed instead in
a far humbler neighborhood called Matraman-Dalam, on a short block of single-story, detached houses,
a stone's throw from a traditional Indonesian neighborhood of narrow, winding streets.

Though he lived in that neighborhood for only two years, Mr.~Obama left a lasting impression because
of his outgoing and sometimes rowdy personality.

``Barry was so naughty that my father even scolded him one time,'' said Sonni Gondokusumo, 49, a
former neighbor and classmate.

Mr.~Obama's family rented the guest house inside a compound belonging to a prominent physician.
There, according to the neighborhood's longtime residents, the young Obama, who had already
experienced differences in class and religion in his short stay in Indonesia, was exposed to another
aspect of Jakarta's diversity.

His nanny was an openly gay man who, in keeping with Indonesia's relaxed attitudes toward
homosexuality, carried on an affair with a local butcher, longtime residents said. The nanny later
joined a group of transvestites called Fantastic Dolls, who, like the many transvestites who remain
fixtures of Jakarta's streetscape, entertained people by dancing and playing volleyball.

In the compound, Mr.~Obama often played with the two sons of the physician's driver.

One time, recalled the elder son, Slamet Januadi, now 52, Mr.~Obama asked a group of boys whether
they wanted to grow up to be president, a soldier or a businessman. A president would own nothing
while a soldier would possess weapons and a businessmen would have money, the young Obama explained.

Mr.~Januadi and his younger brother, both of whom later joined the Indonesian military, said they
wanted to become soldiers. Another boy, a future banker, said he would become a businessman.

``Then Barry said he would become president and order the soldier to guard him and the businessman
to use his money to build him something,'' Mr.~Januadi said. ``We told him, `You cheated. You didn't
give us those details.' ''

``But we all became what we said we would,'' he said.

\section{Countering China, Obama Backs India for U.N. Council}

\lettrine{B}{y}\mycalendar{Nov.'10}{10} endorsing India for a permanent seat on the United Nations
Security Council, President Obama on Monday signaled the United States' intention to create a deeper
partnership of the world's two largest democracies that would expand commercial ties and check the
influence of an increasingly assertive China.

Mr.~Obama's announcement, made during a nationally televised address to the Indian Parliament, came
at the end of a three-day visit to India that won high marks from an Indian political establishment
once uncertain of the president's commitment to the relationship. Even as stark differences remained
between the countries on a range of tough issues, including Pakistan, trade policy, climate change
and, to some degree, Iran, Mr.~Obama spoke of India as an ``indispensable'' partner for the coming
century.

``In Asia and around the world, India is not simply emerging,'' he said during his speech in
Parliament. ``India has emerged.''

Mr.~Obama's closer embrace of India prompted a sharp warning from Pakistan, India's rival and an
uncertain ally of the United States in the war in Afghanistan, which criticized the two countries
for engaging in ``power politics'' that lacked a moral foundation.

It is also likely to set off fresh concerns in Beijing, which has had a contentious relationship
with India and has expressed alarm at American efforts to tighten alliances with Asian nations wary
of China's rising power.

But warmer ties between the United States and India, in the making for many years, come at a crucial
time for Mr.~Obama. He and Prime Minister Manmohan Singh are headed to South Korea later this week
for a meeting of the Group of 20, apparently in agreement on what is expected to be a significant
clash between the world's big powers over the United States Federal Reserve's plan to boost the
American economy by pumping \$600 billion into it.

China, Brazil and Germany have sharply criticized the move by the independent Fed, which they see as
intended to push down the value of the dollar to boost American exports. Germany's finance minister
equated the move to currency manipulation ``with the help of their central bank's printing
presses.''

But at a Monday news conference, Mr.~Obama defended the Fed's move and won backing from Mr.~Singh,
who spoke about the United States' critical importance to the global economy.

``Anything that would stimulate the underlying growth and policies of entrepreneurship in the United
States would help the cause of global prosperity,'' he said.

The good will between Mr.~Obama and Mr.~Singh, as well as the almost giddy reaction to the president
and his wife, Michelle, in the Indian press, lent a glossy sheen to a United States-India
relationship that is still evolving.

India remains deeply protective of its sovereignty, while the United States is accustomed to having
the upper hand with its foreign partners. On Monday, Mr.~Singh emphasized the need for the two
countries ``to work as equal partners in a strategic relationship.''

``For India, going back to the earliest days since independence, there has always been a very strong
attachment to strategic autonomy,'' said Teresita C.~Schaffer, director of the South Asia program at
the Center for Strategic and International Studies in Washington. ``Americans throw around the word
`ally' with gay abandon.''

Mr.~Obama arrived in India on Saturday bearing a big gift: his decision to lift longstanding export
controls on sensitive technologies, albeit with some of the specifics still unclear. And the
president also made several small-bore announcements about new collaborations between the nations on
everything from homeland security to education, agriculture and open government.

Many Indian analysts said Mr.~Obama had big shoes to fill, given the popularity here of his two
predecessors. President George W.~Bush is viewed with admiration, largely for his work securing a
civil nuclear cooperation pact. And former President Bill Clinton, who in 2000 became the first
American president to visit India in two decades, is fondly remembered for his gregarious
personality and his own speech in Parliament, credited for reviving the relationship.

The headline moment of the trip was Mr.~Obama's announcement on the United Nations seat, even though
the endorsement is seemingly as much symbolic as substantive, given the serious political obstacles
that have long stalled efforts to reform membership of the Security Council.

All the major powers have said the post-World War II structure of the Security Council, in which the
United States, Britain, France, Russia and China have permanent seats with veto power, should be
changed to reflect a different balance of power. But it could take years for any changes to be made,
partly because there is no agreement on which countries should be promoted to an enlarged Security
Council.

The United States has promised to support a promotion for Japan and now India. China is viewed as
far less eager for its Asian neighbors to acquire permanent membership in the Council.

But administration officials and independent analysts emphasized the significance of the president's
political message.

Ben Rhodes, a top foreign policy adviser to Mr.~Obama, said the endorsement was intended to send a
strong message ``in terms of how we see India on the world stage.'' Meanwhile, in Washington, even
critics who had blamed Mr.~Obama for letting the relationship with India drift reacted with praise
-- and surprise.

``It's a bold move -- no president has said that before,'' said Richard Fontaine, a former adviser
to Senator John McCain who wrote a critical report of Mr.~Obama's India policy last month for the
Center for New American Security. ``It's a recognition of India's emergence as a global power and
the United States' desire to be close to India.''

But any outreach to India is bound to cause problems for Mr.~Obama in Pakistan. In Islamabad,
Pakistan's Foreign Ministry warned that Mr.~Obama's decision would further complicate the process of
reforming the Security Council. Pakistan, the ministry said in a statement, hopes the United States
``will take a moral view and not base itself on any temporary expediency or exigencies of power
politics.''

For Mr.~Obama, the Pakistan-India-United States nexus creates a delicate dance. The Obama
administration is selling warplanes to Pakistan, a move viewed with suspicion here.

During his three-day visit, the president faced criticism for being too soft on Pakistan; during a
question and answer session with college students, one demanded to know why he had not declared
Pakistan a ``terrorist state.'' And even Mr.~Singh, standing by the president's side at a joint news
conference Monday, reiterated India's position that it could not have meaningful talks with Pakistan
until it shut down the ``terror machine'' inside its borders.

But if Mr.~Obama's cautious language on Pakistan provoked initial unease, his speech at Parliament
seemed to put the matter to rest when he called on the Pakistani government to eradicate ``safe
havens'' for terrorism groups and prosecute the perpetrators of the November 2008 terrorist attacks
in Mumbai that killed at least 168 people.

``Indians were keen to listen to two `p' words,'' said Rajiv Nayan, a strategic affairs analyst in
New Delhi. ``Permanent membership of the United Nations Security Council and, second, on Pakistan.''

\section{Chinese Animator Seeks a Global Role}

\lettrine{O}{stensibly}\mycalendar{Nov.'10}{10}, there is no connection between the 2008 vampire
movie ``Twilight'' and a Warren Buffett cartoon series meant to teach financial basics to children.

But a Chinese animation and special effects company, Xing Xing Digital, has had a hand in both.

If you have seen the eerily colored sky in fight scenes of ``Twilight,'' then you have already
glimpsed Xing Xing's postproduction work. Xing Xing, one among dozens of Chinese animation and
computer special effects companies that serve as low-cost contractors to Western filmmakers, has
also added effects to movies including ``Changeling'' and ``Tropic Thunder.''

Now, though, Xing Xing (pronounced shing shing) wants to be more than an outsource supplier to the
film industry, by developing original content for the international market.

One of those efforts involves Mr.~Buffett, who provides the voice for his cartoon counterpart in the
English-language series ``Secret Millionaires Club,'' which runs on the AOL Kids Web site. The
episodes are each a few minutes in length -- enough time, say, for Mr.~Buffett and his animated
acolytes to impart the importance of location when setting up a lemonade stand.

While Xing Xing co-produces each episode, A Squared Entertainment, an American company owned by a
longtime animation executive, Andy Heyward, and his wife Amy Moynihan Heyward, handles the scripts
and voice recordings in Los Angeles. A Squared also owns rights to the series outside China.

``The `Millionaires Club' is part of our strategy of investing in I.P. targeting children,'' said
Lifeng Wang, the 37-year-old president of Xing Xing, referring to intellectual property. ``The
danger of outsourcing effects is we get caught in a price war with other Chinese animation
studios,'' Mr.~Wang said.

A Squared and Xing Xing have struck similar deals for two other animated AOL Kids Web series. ``Gigi
and the Green Team'' features the supermodel Gisele Bundchen as a superhero fighting for the
environment, and ``Martha and Friends'' has a 10-year old Martha Stewart running an event-planning
company from a tree house.

''Gigi,'' to which Ms.~Bundchen has licensed rights but does not actively participate, started
showing in Brazil last month, whereas ``Martha,'' to which Ms.~Stewart has also licensed only rights
to her character, is planned for a December premiere on AOL Kids.

Xing Xing is also collaborating with the National Wildlife Federation on ``Wild Animal Baby
Explorers,'' an animated series introducing preschoolers to nature that has run on some public
television stations in the United States.

Like many of its compatriots here, Xing Xing is getting help for its global ambitions with the same
sort of government support that has propelled other Chinese industries like automobiles and clean
energy. The State Administration of Radio, Film and Television last year announced a policy to set
up financing for and give tax breaks to movie, television and animation projects.

The government of Jiangsu Province, in southern China, is one of the largest stakeholders in a \$45
million fund that has helped Xing Xing convert a gutted steel mill in the city of Wuxi into a
futuristic studio. It is the company's third site in China, which Mr.~Wang says will eventually
house 200 employees.

Xing Xing already employs more than 300 computer graphics programmers, artists and producers in its
Beijing headquarters, and an additional 30 employees in its branch in Anhui province.

Despite its production deals and state backing, Xing Xing knows that global success is hardly
assured.

The industry's cautionary example is the animated film ``Thru the Moebius Strip,'' about a young boy
who travels to a distant galaxy to rescue his father. China's first feature-length animation in 3-D,
the effort cost the Institute of Digital Media Technology, a Shenzhen-based animation company, \$20
million to produce. But when it was shown at the 2005 Cannes Film Festival, international
distributors showed no interest.

``Moebius'' flopped within China, too, taking in only \$3 million domestically.

China simply has too little experience catering to international audiences. While billions of
dollars in government support for the industry has helped animation studios proliferate in China --
to more than 10,000 last year, compared with only 120 in 2002 -- most still churn out low-quality
cartoons for domestic distribution.

That is why Xing Xing is among a handful of next-generation animation studios that operate their own
training schools. The Xing Xing Digital University, which enrolls over 1,500 students and aims to
recruit its best graduates, offers accredited two-year certificates for full-time students and
academic credit for students from other universities.

Even so, Xing Xing ``can't approach world-class Western animation companies in terms of employing
top-notch programmers,'' said Steven D.~Katz, executive producer at Xing Xing. ``Our biggest problem
is know-how.''

Distribution within China presents its own challenges. Consider what steps ``Secret Millionaires
Club'' must take before it can be shown on television in China, for example. In addition to being
dubbed in Mandarin, it must first be approved by the state radio, film and television
administration, which censors content deemed politically sensitive.

But Mr.~Wang, who hopes to begin showing the series here next year, anticipates the animated
Mr.~Buffett will have ``no problem'' getting past the censors to hundreds of millions of young
Chinese.

``Teaching the principles of making money,'' he said, ``is generally regarded as a healthy thing
here.''

\section{Obama Presses to Complete Free-Trade Deal With South Korea}

\lettrine{T}{he}\mycalendar{Nov.'10}{10} White House is intensifying negotiations with South Korea
on revising a free-trade agreement negotiated by the Bush administration, even though the accord
still faces opposition from Democratic politicians, labor unions and the Ford Motor Company.

The agreement was signed in 2007 but requires approval by Congress, where Democrats have said they
will not support it unless barriers to American exports of automobiles and beef are further reduced.

On Monday, the United States trade representative, Ron Kirk, and his Korean counterpart, Kim
Jong-hoon, met in Seoul, where their aides have been in face-to-face talks since last week.

And as part of his 10-day trip to Asia, President Obama is scheduled to arrive in Seoul on Thursday
to meet with President Lee Myung-bak and attend a two-day meeting with other leaders of the Group of
20 economic powers.

``Hopefully, by the time he comes we will have an agreement,'' Mr.~Lee said in an interview Saturday
in Seoul. ``We will send a very powerful and strong message about maintaining open markets and
resist the trend toward protectionism.''

Mr.~Lee said the agreement would help cement the 60-year-old alliance between the countries.

At the last G-20 leaders' meeting, held in Toronto in June, Mr.~Obama announced a goal of completing
the agreement by the Seoul meeting and submitting it to Congress early next year. Mr.~Obama, who was
critical of such trade agreements when he ran for office, stopped short of calling for the deal to
be renegotiated, but called for adjustments in areas like the auto provisions.

The Republican takeover of the House in the midterm elections added momentum to pending trade
agreements with South Korea, Colombia and Panama. Business groups like the United States Chamber of
Commerce support the trade agreements, as do financial services companies.

On Sunday, Senator Mitch McConnell of Kentucky, the Republican leader in the Senate, said that he
had discussed the trade agreements with Mr.~Obama, and mentioned them as a possible area of
bipartisan cooperation. ``The notion that we're at each other's throats all the time is simply not
correct,'' he said on CBS's ``Face the Nation.''

An important exception to corporate support for the agreement is Ford, the only one of the big three
Detroit automakers that has not gotten a bailout. On Thursday, Ford ran full-page ads in newspapers
stating: ``For every 52 cars Korea ships here, the U.S.~can only export one there.''

While the agreement would lower or eliminate tariffs on cars in both countries, other barriers would
remain, including emissions, mileage and safety requirements and tax and insurance rules that
automakers and the United Automobile Workers union say discriminate against American cars.

``American-made cars can compete and win globally, but we can't afford a future with more closed
markets to American exports,'' Ford said in the ad.

Chrysler has reservations about the agreement in its current form. General Motors, which owns a
Korean automaker, Daewoo, has remained neutral.

The White House press secretary, Robert Gibbs, said last week that the negotiators hoped to amend
provisions that were ``tilted against auto companies and autoworkers.''

The other major sticking point is beef. Shortly after taking office in 2008, Mr.~Lee moved to relax
restrictions on imports of American beef, which had been banned since 2003 because of concerns over
mad cow disease. That led to protests in Korea, with scores of injuries.

South Korea now permits imports of beef from cattle younger than 30 months, which are the least
likely to harbor the agent that causes the disease. But Senator Max Baucus, a Montana Democrat and
chairman of the Senate Finance Committee, which oversees trade, has called for lifting all
restrictions on American beef exports.

Democrats have been skeptical of claims that the deal will create jobs. In a joint letter last month
to Mr.~Obama and Mr.~Lee, 20 House members and 35 Korean lawmakers called for strengthening health,
labor and environmental standards in the agreement, echoing concerns raised by the A.F.L.-C.I.O. But
neither government seems to want to revisit those terms.

Republicans broadly support the deal the Bush administration negotiated. ``Not only does the Korean
market offer excellent growth prospects for American firms, but a free-trade agreement with Korea
will have huge foreign policy and national security benefits in Asia,'' said Clayton K.~Yeutter, who
was the trade representative from 1985 to 1989, under President Ronald Reagan.

The United States had a \$10.6 billion trade deficit in goods with South Korea last year, and
supporters of the agreement say it would help close that gap. South Korea signed a free-trade
agreement with the European Union last month, and business groups say the United States should
complete the deal to remain competitive.

Han Duk-soo, the South Korean ambassador in Washington, who is trying to build support in Congress
for the agreement, said in a recent interview that the imported share of the Korean auto market had
risen substantially over the last decade and that the United States enjoyed a sizable trade surplus
with his country in agricultural products.

\section{South Korea Drops Its Call for Apology From North}

\lettrine{I}{n}\mycalendar{Nov.'10}{10} a shift that could pave the way for new talks on the
dismantling of North Korea's nuclear weapons program, the government in the South has quietly
abandoned its demand that the North apologize for the sinking of a South Korean naval vessel, no
longer making that a condition for the nuclear talks or other future exchanges.

President Lee Myung-bak, in an interview at the Blue House, the presidential residence, said he
would instead be looking for ``sincerity in North Korea's behavior'' before returning to the
six-party talks on disarmament. When pressed about an apology as a condition, he twice declined to
say it was still a requirement.

``We lost valued lives, and that's why the issue is so sensitive to us,'' said Mr.~Lee, who came
into office in early 2008 on a pro-business and free-trade platform that also called for a new,
hard-line approach to North Korea. ``For the resumption of dialogue in any form, North Korea has to
show genuine interest and sincere behavior.''

Mr.~Lee's conservative administration had been adamant about an apology for the sinking of the
warship, the Cheonan, which killed 46 South Korean sailors in March. The six-party talks, future
government aid and a variety of inter-Korean exchanges were made contingent on a formal apology and
North Korea's punishment of those involved in the sinking.

An investigation backed by South Korea attributed the sinking to a North Korean torpedo attack. The
North has denied any involvement and called the South's inquiry ``sheer fabrication.'' In July, the
United Nations Security Council condemned what it called ``the attack'' on the Cheonan but stopped
short of blaming North Korea.

A senior official in Mr.~Lee's government, speaking on the condition of anonymity because of the
delicacy of the nuclear issue, affirmed that the South was still angry about the sinking, but that
negotiations could resume without an apology. Like the president, he also used the term ``sincere
behavior,'' but he declined to specify what that behavior might include.

A top aide in the president's office also privately confirmed the policy change, saying an apology
was ``not a precondition to resuming the talks.''

The senior government official said there was no specific time frame for new six-party negotiations,
although he acknowledged that South Korea was coming under increasing pressure to hold new talks.
China, North Korea's only major ally, is the host country for the six-party talks.

The interview with Mr.~Lee on Saturday came as South Korea prepared to host a Group of 20 economic
summit meeting on Thursday and Friday, a first for an Asian nation. Mr.~Lee is positioning himself
and his country's resilient economy as new leaders on the global stage.

The North Korean nuclear issue is already on the agenda of various bilateral meetings planned during
the G-20 gathering, the president said. The foreign ministers of China, Japan, Russia and the United
States discussed the resumption of the six-party talks on the sidelines of a recent Asian regional
summit meeting in Hanoi, Vietnam. The Koreas are the other nations in the talks, which collapsed in
April 2009 when North Korea withdrew from the negotiations.

Most political analysts in the South believe a North Korean apology for the Cheonan sinking is
highly unlikely. But Mr.~Lee said it was ``possible.''

``I think we shouldn't rule out a North Korean apology,'' he said, adding, ``I wouldn't be 100
percent sure'' about not getting one.

Mr.~Lee suggested in the interview that an apology would be a wise ``strategic decision'' by the
North, a move that could result in shipments of food and other humanitarian assistance from South
Korea.

His government has consistently refused to send large-scale aid shipments to the North, although
small deliveries of rice and flour were recently permitted after droughts and flooding in the
impoverished North over the summer.

If talks were to resume, Mr.~Lee emphasized, he would want to avoid the usual pattern of diplomacy
that has dogged the nuclear issue: concessions by the North that lead to aid deliveries and new
negotiations, which are then followed by new provocations by the North and the collapse of the
talks.

``What we need now is to ensure that the six-party talks will bring about substantive results,
rather than just talks for the sake of talks,'' Mr.~Lee said, echoing language that has lately been
used by senior American diplomats.

\section{China Rights Advocates Cannot Travel}

\lettrine{T}{wo}\mycalendar{Nov.'10}{10} prominent legal scholars and rights advocates bound for an
international law conference in London were blocked from leaving China on Tuesday on vague charges
that their departure might endanger national security, they said.

The scholars, Mo Shaoping and He Weifang, said they suspected that the government feared that they
would try to attend the Nobel Peace Prize ceremony in Oslo next month honoring the jailed dissident
Liu Xiaobo.

Both men were on the list of 143 Chinese activists, academics and celebrities that Mr.~Liu's wife
invited to the award ceremony in an Internet posting two weeks ago, noting that neither she nor her
husband were likely to be allowed to attend.

Mr.~Mo and Mr.~He said that officers who detained them said a superior had described their overseas
journey as a threat to state security. ``That's the most imbecilic thing I've heard,'' Mr.~Mo said
in an interview, adding that he had no intention of traveling to Oslo. ``I don't have a visa for
Norway, and I have a ticket to return to Beijing on Nov.~15.''

In recent weeks, the government has demonstrated its resolve to stop Chinese citizens from attending
the ceremony on Dec.~10. It has kept Mr.~Liu's wife, Liu Xia, incommunicado in her Beijing apartment
and subjected scores of other writers, academics and lawyers to varying degrees of detention or
surveillance.

Among those facing restrictions are Hua Ze, a Beijing-based filmmaker who was forcibly returned to
her hometown in southern China; Liu Suli, a Beijing bookstore owner who says he was injured during
an assault by security agents last month; and Yu Fangqiang, a human rights lawyer who was stopped at
the Hong Kong border last Friday on his way to a United Nations training session in Geneva.

At the same time, China has ramped up pressure on foreign governments, warning foreign officials to
stay away from the event next month or ``bear the consequences,'' as Cui Tiankai, China's vice
foreign minister, put it last week. On Tuesday, the Japanese foreign minister told Parliament that
Beijing had requested Japan not to send a representative to the ceremony.

Liu Xiaobo, 54, an essayist who is among China's best known advocates of political reform, is
serving an 11-year prison term for his writings, including Charter 08, a manifesto calling for human
rights, the rule of law and an end to single-party rule.

In addition to using its economic might to warn world leaders away from the ceremony, China has
waged an equally vociferous campaign at home to tarnish Mr.~Liu's reputation and delegitimize the
award in the eyes of the Chinese people.

After a brief news blackout on the announcement of the prize, China's state-controlled media began
rolling out articles and editorials describing the award as an insult to the country's criminal
justice system, a ploy to hold back China's rise and a tactic to subvert the country's political
system. Other commentaries have painted Mr.~Liu as a corrupt pawn of Western governments.

The warnings have already prompted a handful of European countries, among them France, Britain,
Austria, Sweden and the Netherlands, to announce that they would hew to established protocol and
send ambassadors.

Michael C.~Davis, a law professor and human rights expert at the Chinese University of Hong Kong,
said he thought China's effort to organize a boycott of the ceremony -- like its earlier campaign to
dissuade the Norwegian Nobel Committee from selecting Mr.~Liu -- would probably backfire. In fact,
he said Beijing's overall handling of the matter was only drawing more attention to Mr.~Liu's plight
and to the country's checkered human rights record. ``The Chinese often unintentionally turn their
enemies into heroes,'' he said.

Mr.~Mo, an outspoken advocate of legal reform who signed the Charter 08 manifesto, has so far
escaped serious persecution. He was barred from defending Mr.~Liu last year, but other lawyers from
his firm were allowed to take on the case.

Mr.~He is a prominent legal scholar at Peking University whose frequent critiques of China's
judicial system may have played a role in his recent transfer to an isolated university in western
China.

Both men were scheduled to take part in discussions in London on Wednesday about the difficulties
facing civil society lawyers in China. Speaking at a restaurant in Beijing, where they were fielding
media calls, the two said the decision to block them from the conference was nonsensical. ``This is
the Chinese government defacing its own image on the international stage,'' Mr.~He said.

\section{G-20 Event to Showcase South Korea's Arrival}

\lettrine{T}{here}\mycalendar{Nov.'10}{10} may be a cabbage farmer in the rural heartland, or an
abalone fisherman off Jeju Island, or even a bartender slinging soju in Pusan who is unaware of his
country's recent emergence as a global economic player. The Bulldozer is about to let them all know,
however, that South Korea is now at ``the center of the world.''

The Bulldozer -- the nickname is used both admiringly and derisively here -- is President Lee
Myung-bak, and he has overseen, often personally, often microscopically, the extravagant
preparations for the summit meeting of the Group of 20 major economies to be held Thursday and
Friday in Seoul.

Until South Korea was chosen to serve as G-20 host, Mr.~Lee said at the time of the announcement,
the country ``was passive in international society and did not have a say.''

``Now it will no longer be possible to discuss a global issue without including Korea,'' he said,
proclaiming that the country had moved ``away from the periphery of Asia to the center of the
world.''

South Korea is the first Asian nation to hold the gathering of the G-20 heads of government and the
first non-Group of 7 nation to be the host. It was the first former aid recipient to become an aid
donor within the Organization for Economic Cooperation and Development. Once poorer than Communist
North Korea, it now boasts the world's 13th-largest economy and has recovered from the current
economic downturn faster than any other developed nation.

``The G-20 is a landmark financial coming-out event -- a sort of Korea comeback story on the back of
the perceived embarrassment it felt from the 1997-98 financial crisis,'' said Jasper Kim, associate
professor at the Graduate School of International Studies at Ewha Womans University in Seoul. ``The
G-20 is also big for South Korea because this time the G-2 -- the U.S.~and China -- are highly
focused on it.''

Mr.~Lee, 68, who as a young chief executive turned Hyundai Construction from a small company into a
worldwide builder (hence The Bulldozer), was mayor of Seoul before becoming president in 2008 on a
pro-business platform that also promised closer ties with the United States and a harder line toward
North Korea. ``The G-20 is important to Lee Myung-bak, who wants to be remembered as an economic
president,'' said Gi-wook Shin, director of the Shorenstein Asia-Pacific Research Center at Stanford
University in California. ``He ran his presidential campaign as a `C.E.O. president.' All past
presidents have wanted to leave a legacy, and with a successful hosting of the G-20, he can claim
that he has advanced Korea to the status of a global player.''

Austerity may be the economic mantra elsewhere, but South Korea is not throwing its G-20 party on
the cheap (though it seems unlikely that Seoul's budget will surpass the \$860 million that Toronto
spent to hold the previous summit meeting, last June). The organizers have declined to estimate the
total cost, but the mayor of Seoul, Oh Se-hoon, said in an interview Tuesday the city was spending
just \$9 million extra.

Politicians and business leaders here sense that this is a chance to finally strut their stuff, to
show off the national grit and economic mettle, maybe even to lecture some larger countries,
especially Japan and China, about the merits of the South Korean model.

For Seoul, the G-20 meeting is not just about exchange-rate feuds and trade policies. New museum
shows and gallery exhibits have opened, and dance and music concerts are being staged all week.
There will even be bursts of gastrodiplomacy -- culinary demonstrations featuring Korea's fiery and
famously fermented foods. Organizers have prepared a G-20 pep song, G-20 cocktails, souvenir
knickknacks, appearances by TV stars and massive G-20 billboards featuring Kim Yu-na, the Olympic
figure skating champion.

As they are laying on the show, the organizers also are going heavy on the security. The Toronto
meeting was beset with violence and nearly a thousand arrests. Mr.~Lee and his lieutenants will be
tolerating none of that. ``They'll do whatever it takes to block any protests,'' said Mr.~Shin.
Mr.~Lee, in an interview last weekend, said he did not expect any disturbances from North Korea.
Still, the South Korean military has been placed on alert, the Defense Ministry said.

South Korea and its capital are well-credentialed in the big-event business. The country
successfully staged the 2002 World Cup soccer tournament jointly with Japan. The 1988 Olympic Summer
Games, held in what was then a fledgling South Korean democracy, were a huge success, uniting all of
the world's teams after the three previous Summer Games were marred by boycotts.

``Seoul established new standards for security, transportation and efficiency,'' said Mike Moran, a
former senior official with the U.S.~Olympic Committee. ``The Koreans demonstrated patience and a
willingness to correct systems when the Games began, and that was a first.''

Mr.~Oh said that the Olympics had ``really put us on the map,'' but that ``holding the G-20 will be
more important as a way of showing that Korea is part of the new global economic order, and part of
the flow.''

Koreans are famously fond of lists that show where they rank in relation to other countries. The
topic hardly matters. First in patent filings per gross domestic product. Lowest unemployment rate
among all G-20 countries. The Olympic gold medal in baseball. The world's highest rate of Internet
penetration. The No.~1 maker of computer chips and flat-screen televisions. A government-backed plan
that intends to lift Korean food into the ``top five rank of world cuisines'' by 2017, whatever that
means.

But the country's phoenix-like economic rebound is not necessarily obvious to all South Koreans,
many of whom have seen incomes flatten and consumer prices skyrocket. A splashy hosting of the G-20
could generate some significant domestic political capital for Mr.~Lee.

``Samsung, LG and Hyundai are world-class businesses, but there are people in the countryside still
living in 20th-century houses with a 19th-century mentality, and maybe they don't realize the
meaning and importance of the G-20,'' said Kim Seoc-woo, director of the Institute of Peace and
Cooperation, a research group in Seoul. ``This a chance for them to see Korea alone in the
spotlight.''

South Korea can sometimes seem obsessed with its image. Mr.~Lee established a Presidential Council
on National Branding to improve the country's image, which, according to the council and the Samsung
Economic Research Institute, ranks 20th worldwide.

``Hosting the G-20 certainly has something to do with nationalism and identity,'' said Mr.~Shin.
``Korea's reference point has not been other developing countries, even in the 1970s and '80s, but
advanced countries, especially Japan.

``Koreans also say their country is `small,' and it certainly looks so in Northeast Asia since its
neighbors are so huge: China, Russia, Japan,'' he said. ``But it's a top-15 country in terms of
economic size. And through the G-20, Koreans want to show the world that they have become an
advanced country.''

\section{Obama and China Play Rival Suitors to Ascendant Indonesia}

\lettrine{L}{ess}\mycalendar{Nov.'10}{10} than a day before President Obama touched down in Jakarta
on Tuesday, a high-level Chinese delegation wrapped up a three-day official visit here by announcing
that Beijing would invest \$6.6 billion in desperately needed infrastructure improvements. The
announcement's emphasis on roads, bridges and canals -- not to mention its timing -- laid down a
not-so-subtle challenge to Mr.~Obama: Show your Indonesian hosts the money.

As the United States and China step up their rivalry in Southeast Asia, Indonesia -- officially
hewing to a longstanding foreign policy of nonalignment but leaning closer to Washington --
represents by far the biggest prize in a region caught uneasily between China's rise and America's
renewed engagement.

At a news conference here with his counterpart, President Susilo Bambang Yudhoyono, Mr.~Obama said
the United States was not interested in ``containing'' China. But a day after endorsing India's
pursuit of a permanent seat on the United Nations Security Council in a move widely seen as an
attempt to check China's growing influence, Mr.~Obama poured on the charm.

The president, who spent four years as a boy here, sprinkled some Indonesian words in his speech and
reminisced about daily life back then. More pointedly, he sought to align Indonesia with the United
States on shared values, calling Indonesia a ``critical partner'' in ensuring Asia's prosperity
``primarily because it is a country that has figured out how to create a genuine democracy despite
great diversity.''

American and Chinese officials have been pursuing all 10 countries in the Association of Southeast
Asian Nations, but none more aggressively than Indonesia, the world's fourth most populous nation,
spread out across a strategically important, resource-rich archipelago and now led by a
democratically elected government impatient to raise the country's international profile.

The United States will have to contend with challenges, old and new. Despite Indonesia's enduring
suspicion of China, Beijing has been making great inroads here, economically, diplomatically and
militarily. And a newly confident Indonesia has been reasserting its independent foreign policy,
promoting what it now calls a ``dynamic equilibrium'' for the region.

``We want to maintain a strategic space from the rivalry between the United States and China,'' said
Juwono Sudarsono, Indonesia's defense minister under Mr.~Yudhoyono from 2004 to 2009. ``We can
navigate between that rivalry, from time to time giving out signals that both the United States and
China are important to us, because if we align ourselves too closely, it would be detrimental to the
core values of Indonesia's foreign policy.''

Despite Beijing's efforts, Indonesia, like the rest of Southeast Asia, except Laos, Cambodia and
Myanmar, remains closer economically and strategically to the United States, experts say.

``The Indonesians would never align explicitly with the U.S., but they would learn to play the
game,'' said Carlyle A.~Thayer, a professor at the Australian Defense Force Academy who recently
released a study on the effects of the American-Chinese rivalry on Southeast Asia.

In July, the United States lifted a ban on cooperating with Kopassus, an Indonesian special forces
unit implicated in past human rights abuses. That removed the last obstacle to normalizing military
ties that had been suspended in the 1990s because of Indonesia's occupation of East Timor. The
United States, which has provided Indonesia with \$47 million in equipment to bolster maritime
security, also was co-host of a nine-nation military exercise with Indonesia last year.

But United States policy in other parts of the Muslim world undercuts its appeal in Indonesia, the
country with the world's largest Muslim population.

``As President Obama promised in Cairo more than a year ago, he said he wants to create new
relations between the United States and the Muslim world based on mutual understanding and
respect,'' said Din Syamsuddin, the chairman of Muhammadiyah, one of Indonesia's two biggest Islamic
social and political organizations. ``But many of us, including myself, are still waiting for the
materialization of his promises. We maintain our skepticism because the foreign policy of the United
States in Afghanistan or Palestine has not really shown any change.''

Meanwhile, Mr.~Syamsuddin said, China has been reaching out to Muslim leaders in Southeast Asia. He
himself was invited to meet Jia Qinglin, a member of China's ruling circle, in Beijing in May. Prime
Minister Wen Jiabao had planned a visit this year, but had to cancel for domestic reasons.

Because of a failed 1965 Communist coup in Indonesia that is believed to have been backed by
Beijing, Indonesia and China had no diplomatic relations until 1989. But trade between the two more
than doubled from 2005 to 2009.

Since a Strategic Partnership agreement in 2005, high-level military exchanges have increased, and
Indonesian officers have trained in China. Under a series of agreements, China said it would provide
Indonesia with technical assistance in building aircraft and ships, as well as help in producing
weapons and ammunition.

But there has been almost no follow-through yet on those agreements, which instead allowed Indonesia
to use the ``China card'' to pressure the United States into deepening military ties, said Ian
J.~Storey, an expert on Southeast Asian military issues at the Institute of Southeast Asian Studies
in Singapore.

Still, Indonesia signaled recently that there were limits to how close it would get to the United
States.

In September, Mr.~Yudhoyono conspicuously skipped a meeting in New York between the United States
and the Association of Southeast Asian Nations, or Asean, when Mr.~Obama was seeking support in
pressing China for a resolution to disputes in the South China Sea. Aides said Mr.~Yudhoyono was too
busy attending to domestic issues, but his absence was interpreted differently here.

``The Indonesian government felt that the U.S.~was putting too much pressure on Indonesia and other
Asean nations to choose sides,'' said Syamsul Hadi, a political scientist at the University of
Indonesia.

\section{For Indonesia and Obama, 4th Time's the Charm}

\lettrine{I}{t}\mycalendar{Nov.'10}{10} is the presidential homecoming that wasn't meant to be.

The last time President Obama was in this nation of islands, in 1992, he holed up in a rented
beachside hut in Bali, where he swam each morning and spent afternoons writing ``Dreams From My
Father,'' the memoirs that later became a best seller. In the book, he shared memories of his life
here as a boy, ``running barefoot along a paddy field, with my feet sinking into the cool, wet mud,
part of a chain of other brown boys chasing after a tattered kite.''

On Tuesday, he returned for the kind of rock star welcome that he no longer gets in the United
States. But his long-awaited trip here is being cut short by a cloud of volcanic ash.

Indonesians have prepared three other times for a visit from the president, only to be thrice
disappointed. Last year, the White House hinted that Mr.~Obama might tuck in an Indonesia stop on a
November trip to Asia, but that did not happen. In March, Mr.~Obama and his wife and daughters
canceled a trip at the last minute so that he could stay home to shepherd his health care bill
through Congress. In June, another Indonesia trip was canceled so the president could deal with the
BP oil spill.

Now, it is ash from the eruption of Mount Merapi that is interfering with his plans, forcing
Mr.~Obama to leave the country earlier on Wednesday than he had planned for Seoul, South Korea,
where he is to attend the Group of 20 conference of world leaders. He will still deliver a speech at
the University of Indonesia, and he will squeeze in a stop at the Istiqlal Mosque, the largest in
Southeast Asia. But his visit to Kalibata Heroes Cemetery in south Jakarta, where Indonesia's war
dead are buried, is being scrapped. And as if that weren't enough, a loud thunderstorm forced his
joint news conference with President Susilo Bambang Yudhoyono indoors Tuesday evening.

Mr.~Obama is on a 10-day, four-country swing through Asia that he is using to promote economic
cooperation and strengthen ties with emerging and existing democracies. He is making outreach to the
Muslim world a major theme of his stop in Indonesia, the world's largest Muslim-majority nation. He
closed his remarks at the news conference with the Muslim salute ``salaam aleikum'' and said he
intended to reshape American relations with Muslim nations to not be ``solely focused on security
issues,'' but rather on expanded cooperation across a broad range of areas, from science to
education.

Mr.~Obama also stepped into an Israeli-Palestinian dispute over Jewish construction, criticizing
Israel for its decision to advance the approval of 1,000 new housing units in East Jerusalem during
a sensitive time in peace talks. The plight of Palestinians is a big issue in Indonesia, so much so
that President Yudhoyono mentioned it in his opening remarks, saying he had told Mr.~Obama that ``we
need a resolution on Palestine and Israel in a permanent, sustainable manner.''

Later, Mr.~Obama was asked about the settlements. ``This kind of activity is never helpful when it
comes to peace negotiations, and I'm concerned that we're not seeing each side make the extra effort
involved to get a breakthrough,'' he said, adding, ``Each of these incremental steps can end up
breaking trust.''

Aides say the speech Mr.~Obama will give on Wednesday will build on one he delivered in Cairo last
year, in which he called for ``a new beginning'' with the Muslim world. At Tuesday's news
conference, he was asked to assess progress thus far. ``I think it's an incomplete project,''
Mr.~Obama said.

Mr.~Obama spent four years, from age 6 to 10, in Indonesia, living here with his mother, Stanley Ann
Dunham, and stepfather, Lolo Soetoro. In his memoirs he writes richly of the experience. He
described the markets: ``the hawkers, the leather workers, the old women chewing betelnut and
swatting flies off their fruit with whisk brooms.'' He wrote of his introduction to the food: ``dog
meat (tough), snake meat (tougher), and roasted grasshopper (crunchy).'' And the menagerie in his
backyard: ``chickens and ducks running every which way, a big yellow dog with a baleful howl, two
birds of paradise, a white cockatoo and finally two baby crocodiles.''

Jakarta is also the place that has given rise to many of the myths and falsehoods about Mr.~Obama,
including the rumor that he is Muslim (he is Christian); that he attended a madrasa that was
connected to radical Islam (he attended two schools here, one Roman Catholic and one secular,
although most of the students were Muslim); and that he was not born in the United States (he was
born in Hawaii). And his visit here, including the trip to the mosque on Wednesday, is bound to
provoke more anti-Obama sentiment at home.

Here in Jakarta, though, the only complaints about Mr.~Obama are that it took him so long to get
here. The president said he had come to ``focus not on the past but the future.'' But Indonesians
obviously had something else in mind.

When Air Force One touched down here in a typical Southeast Asia afternoon thunderstorm, a huge
cheer went up in the State Palace complex -- not from ordinary Indonesians, but from the local press
corps, watching on television. ``Finally, he arrived!'' exulted Glenn Jos, a local television
cameraman.

After descending the steps of his plane, Mr.~Obama, in a dark suit, accompanied by his wife,
Michelle, walked the red carpet that had been laid out for him and stepped into his big black
Cadillac limousine. He poked his head out the door to give a short wave.

``Yes!'' the Indonesian reporters shouted.

Later, at a state dinner in his honor, Mr.~Obama was served bakso, nasi goreng, emping and other
Indonesian dishes he said he had loved as a boy. And in a gesture that Mr.~Obama said left him
``deeply moved,'' Mr.~Yudhoyono presented the president with a gold medal in honor of Mr.~Obama's
mother, who worked here for years as an anthropologist and a pioneer in microcredit for the poor.

Jakarta has undergone a dramatic transformation since Mr.~Obama first moved here in 1967, and
Mr.~Obama said he barely recognized it. ``It's a little disorienting,'' he said, noting that then,
the tallest building in downtown Jakarta was a shopping mall, Sarinah, that has now been eclipsed by
many modern skyscrapers. He recalled how people got around in ``little taxis, but you stood in the
back and it was very crowded,'' or on bicycle rickshaws. (Mr.~Obama used the Indonesian word for
them: becak.)

``Now,'' Mr.~Obama said, ``as president I can't even see all the traffic because they block all the
streets.''

\section{Freed Dissident in Myanmar Calls for Reconciliation}

\lettrine{O}{n}\mycalendar{Nov.'10}{15} her first full day of freedom after more than seven years of
house arrest, Myanmar's pro-democracy leader, Daw Aung San Suu Kyi, demonstrated the enduring power
of her popularity on Sunday, drawing thousands of jubilant supporters to a rally at which she
pledged to lead them in a struggle for political change.

Though she spoke of reconciliation, the event itself was a challenge to the authority and control of
the ruling military junta.

The size and enthusiasm of the crowd -- the kind of outpouring of public support that had led the
government to cut short her previous period of freedom in 2003 -- suggested that she had emerged
with her popularity and moral authority intact.

``Democracy is when the people keep a government in check,'' she told the crowd outside her party's
headquarters here in the city once known as Rangoon. ``To achieve democracy we need to create a
network, not just in our country but around the world. I will try to do that. If you do nothing you
get nothing.''

She positioned her movement as an active opposition to the military leaders but gave no specifics,
and it was unclear what steps she would take next. She took pains not to be confrontational, leaving
open the possibility of a new relationship with the generals who had imprisoned her.

``I'm going to work for national reconciliation. That is a very important thing,'' she said, adding:
``There is nobody I cannot talk to. I am prepared to talk with anyone. I have no personal grudge
toward anybody.''

Nevertheless, she began the new relationship with a flat refusal to cooperate, according to a person
close to the negotiations who spoke on the condition of anonymity.

In arranging for her release, that person said, the military had asked her to agree not to leave
Yangon and not to give public speeches. When she refused, she was asked at least to wait awhile
before speaking. She refused again and proceeded with her address on Sunday.

She spoke with the buoyancy and infectious joy that have characterized her addresses in the past,
and her exchanges with the crowd were sometimes emotional.

``I need to know what you want first,'' she said to the crowd. ``Do you know what you want?''

She pointed to a middle-age man and her aides handed him a microphone. ``We love you very much!''
the man said. ``And we need democracy!''

The microphone was passed to another man in the crowd who shouted wildly, ``Today the entire country
has been released from military slavery!''

Mrs.~Aung San Suu Kyi did not respond or even smile, but only gestured that the microphone be passed
to a woman nearby. The woman wept and cried, ``I love you more than I love myself.''

The release of Mrs.~Aung San Suu Kyi, 65, came just six days after an election engineered by the
military to give it control over a civilian Parliament and government. Though the military will
still hold power, there will be new political institutions and new officeholders who could alter the
dynamics of her interactions with the government.

Her lawyers said she had been released without conditions, but it remained unclear what role the
government expected her to play, what long-term limits it intended to set on her activities or
whether it intended to open a dialogue with her.

She said she would be willing to meet with anybody, even the leader of the junta, Senior Gen.~Than
Shwe, saying at a news conference after the rally, ``It will be very good if I can discuss with him
the issues I care about.''

In what seemed a gesture of conciliation, the main government newspaper, The New Light of Myanmar,
reported her release in positive terms Sunday morning, saying that she had been granted a pardon
because of good behavior and that the police ``stand ready to give her whatever help she needs.''

It said she was being treated with leniency because she is the daughter of the nation's founding
hero, U Aung San, a general who was assassinated in 1947, and ``viewing that peace, tranquillity and
stability will prevail and that no malice be held against each other.''

It appeared that even divisions in her party, the National League for Democracy, were beginning to
melt away as party members and former party members rallied around her.

``She belongs to the entire nation,'' said U Khin Maung Swe, the leader of an opposition party that
split with the National League over its decision to boycott the election. ``We consider her a
national leader and she does not belong to any single group or party.''

For Mrs.~Aung San Suu Kyi, the moment seemed to be one of reassessment and recalculation as she
explored issues that might have evolved during her seven-year absence.

At the news conference, she said she would consider changing her position on economic sanctions
against the country formerly called Burma, which she has supported and which have been at the center
of a policy of isolation and punishment of the junta by Western nations. ``If the people really want
sanctions to be lifted, I will consider it,'' she said. ``This is the time that Burma needs help.''

She said she had been listening to radio broadcasts up to six hours a day during her arrest in the
hope of understanding the people, and told her supporters she wanted to hear from them.

``Please let us know what you are thinking, what is on your mind,'' she said at the rally. ``I would
like to know over the last six years what changes have taken place.''

She also asked her supporters to join her campaign for change.

``I'm not going to be able to do it alone,'' she said at the news conference. ``One person alone
can't do anything as important as bringing genuine democracy to a country.''

\section{Obama and Asia-Pacific Leaders Vow to Work Toward Freer Trade}

\lettrine{P}{resident}\mycalendar{Nov.'10}{15} Obama and other Asia-Pacific leaders wrapped up a
two-day meeting on Sunday with sweeping pledges to rectify global economic imbalances and move
toward creating a regional free-trade zone, but with few concrete gains.

For Mr.~Obama, it was also the end of a 10-day diplomatic and economic journey through Asia that
yielded mixed results, ranging from his warm reception in India to criticism for failing to finalize
a free-trade pact with South Korea. He spent his last day meeting with President Dmitri A.~Medvedev
of Russia and then visiting a colossal copper Buddha that he had seen as a 6-year-old, before
heading for Washington to confront the lame-duck Congress on Monday.

Issuing what they called the ``Yokohama Vision,'' Mr.~Obama and leaders of 20 other nations and
territories in the Asia-Pacific Economic Cooperation forum agreed to take steps toward creating a
free-trade zone in the Asia-Pacific region, but set no timetable. In their joint declaration, the
leaders cited as a possible starting point the Trans-Pacific Partnership, a free-trade agreement
that currently includes four small nations -- Brunei, Chile, New Zealand and Singapore -- but that
five other nations, including the United States, are in talks to join.

The leaders, whose nations account for more than half of global economic activity, also agreed to
remove protectionist measures put in place during the current global economic crisis and to avoid
sudden, sharp moves in foreign exchange markets.

In Sunday's meeting with Mr.~Medvedev, Mr.~Obama told him that his ``top priority'' on foreign
policy for the Congressional session is ratification of the two countries' new arms control treaty,
which is stalled in the Senate.

The president also told the Russian leader that he was committed to lifting cold war-era trade
restrictions, a move that would allow Russia to join the World Trade Organization. Senior American
officials said afterward that the private meeting touched on a range of issues, from Afghanistan to
human rights. Mr.~Obama told Mr.~Medvedev that he was ``very pleased'' with the Russian leader's
``strong'' statement condemning the beating of a journalist in Moscow. The two leaders will meet
again this week at the NATO summit meeting in Lisbon, and Mr.~Obama told Mr.~Medvedev that he wanted
talk further about Afghanistan and missile defense.

After leaving Yokohama, the president flew by helicopter to Kamakura, the site of Kamakura Daibutsu,
or the Great Buddha of Kamakura. The giant statue, which experts believe dates to 1252, has 656
coils of hair and weighs 121 tons. ``It is wonderful to return to this great treasure of Japanese
culture,'' he wrote in the guest book. ``Its beauty has stayed with me for many years.''

\section{Leaders of China and Japan Meet on Summit's Sidelines}

\lettrine{T}{he}\mycalendar{Nov.'10}{15} leaders of China and Japan held their first formal talks on
Saturday since clashing two months ago over disputed islands, in what appeared to be a hastily
arranged attempt to patch up differences between the two Asian powers.

The leaders of Japan and Russia, who have also sparred recently over the Kurile Islands in the
Pacific, also met Saturday, but appeared to remain divided.

The meetings took place on the sidelines of a summit meeting of 21 Asia-Pacific nations that Japan
is hosting in this port city.

The 20-minute meeting between the Japanese prime minister, Naoto Kan, and the Chinese president, Hu
Jintao, was announced at the last minute, and appeared to yield little of substance. However,
Japanese officials called it significant that the two had met at all, after Beijing had appeared to
rebuff repeated attempts by Tokyo to bring the two together for formal talks.

Ties between the neighbors, long troubled by the bitterness over Japan's militarism in the 1930s and
1940s, hit their lowest point in years in September after Japan arrested a Chinese fishing trawler
captain near disputed islands in the East China Sea, known as the Senkaku in Japanese and Diaoyu in
Chinese. China responded by suspending midlevel diplomatic talks; cutting off exports of rare earth
metals, which are used to make electronics; and seizing four Japanese businessmen.

Japan released the captain two weeks later, but China demanded an apology and compensation; Tokyo
responded by demanding that Beijing pay for damage to a Japanese coast guard vessel that Japan said
had been rammed by the Chinese trawler.

Since then, Japanese officials had been trying to arrange talks between the leaders in an effort to
improve ties, but the most they could manage was two brief, informal encounters during multilateral
summit meetings. Diplomatic analysts have said Beijing appeared to be taking a tough stand toward
Tokyo to appease domestic public opinion within China, where the standoff incited nationalistic
protests.

At the start of Saturday's meeting, both leaders appeared stiff and mirthless, briefly shaking hands
and then exchanging formal pleasantries. The mood contrasted with that of a meeting Mr.~Kan held
earlier with President Obama, at which Mr.~Kan thanked the United States, Japan's longtime
protector, for its support during the territorial disputes with China and Russia.

While Washington remained officially neutral on China's and Japan's conflicting claims of
sovereignty over the islands, Secretary of State Hillary Rodham Clinton said the islands came under
America's treaty obligations to defend Japan in case of attack. The statement angered China, which
believes the United States' cold-war-era security alliances in Asia are being used to contain it.

Mrs.~Clinton's comment was welcomed in Japan, where concerns about China's rise have forced the
year-old Democratic Party government to seek stronger ties with the United States and reverse its
previous calls to remove an American air base from Okinawa.

``For peace and stability in this region, the presence of the United States and its military is
important,'' Mr.~Kan told reporters after meeting Mr.~Obama, ``and this recognition has been shared
by me and many people in the country, as well as our neighboring countries.''

Mr.~Kan also met with President Dmitri A.~Medvedev of Russia, whose visit this month to one of
several disputed islands between northern Japan and the Russian-controlled Kurile Islands drew
strong diplomatic protests from Japan.

Russian and Japanese news reports said that Mr.~Kan protested the visit in his meeting with
Mr.~Medvedev, who retorted that the islands were Russian territory.

``The president will decide for himself which region of Russia to visit,'' the Russian foreign
minister, Sergey V.~Lavrov, said, according to the Russian news agency RIA Novosti. ``It is our
territory, and that's how it will stay. Our Japanese colleagues, I hope, will adopt a more
appropriate attitude toward this.''

Japanese officials said the two sides agreed that cooperation was in both countries' interest, and
the dispute did not stop them from signing a deal to build a fertilizer factory in Russia.

\section{Heartfelt Moments on an Up-and-Down Global Trip}

\lettrine{I}{t}\mycalendar{Nov.'10}{15} was billed as a diplomatic and economic tour of Asia, but
President Obama will have literally circled the globe by the time he returns to Washington on Sunday
evening.

He left the White House two Fridays ago, heading east to Mumbai, India, then continued to New Delhi;
Jakarta, Indonesia; Seoul, South Korea; and Yokohama, with plans for Air Force One to refuel in
Alaska on the way home. It was a grueling 10-day excursion, Mr.~Obama's longest overseas journey as
president.

He won hearts in India with his forthright answers to college students and awkward dance moves with
disadvantaged children, and in Indonesia, where his personal biography resonated. In both countries,
he got a boost from his popular wife, Michelle.

And he took knocks, especially in Seoul, where he could not seem to shake the word ``failure'' -- as
in, failure to bring home a promised trade pact with South Korea.

Yet amid the president's diplomatic highs and lows, there were personal moments, insignificant in
the grand scheme of international relations, yet revealing just the same. Here are three:

``Namaste,'' Mr.~Obama said as he walked into a classroom at Holy Name High School on a hot Sunday
morning in Mumbai.

The school was also the scene of Mr.~Obama's much-publicized dancing. But before he got his groove
on, the president did something that appealed to his wonkish side: he toured student presentations
on the importance of wind and solar energy, rain-water harvesting and a model ``eco-village'' with a
windmill. Standing before the diorama, he drew his hands together, as if in prayer, and greeted the
children with a slight bow.

The expo was heavy with warnings about the destruction of the planet. A cardboard sea was shown
swallowing up a blue globe, with the label ``Sinking Earth.'' Oversize cigarettes were bound and
wired, marked ``RDX of Cancer,'' apparently a reference to an explosive. The children had clearly
prepared, but there was something they did not know about the president: he is an occasional smoker.

``Today,'' a teenage boy said earnestly, ``about 1.3 billion people all over the world smoke. And
when a person smokes he exhales harmful pollutants into the air. And these pollutants add to the
gases, and they also harm the smoker.'' If that continued, the boy said, the world would ``look like
this -- black burnt carbon earth.'' He pointed to a papier-m\^{a}ch\'e rock with a crying face.

``Hmmm,” Mr.~Obama said. ``You don't want to live on that.''

Childhood Memories

Mr.~Obama was so focused on India, aides say, that he barely had time to think about his next stop,
Indonesia, where he had spent four years as a boy. But aboard Air Force One from New Delhi to
Jakarta, he began thinking about the speech he was to deliver, and his childhood memories came
flooding back.

``I think it really hit him: `I'm going back to Jakarta as president,' '' said Ben Rhodes, a top
foreign policy adviser. ``That's a powerful thing.''

Mr.~Rhodes ordinarily writes Mr.~Obama's speeches when the president is overseas, but Mr.~Obama
wanted a big hand in this one. So he began jotting notes on Air Force One stationery about the
sights and sounds of his childhood: the street vendors hawking skewered beef and spicy meatballs,
the bicycle rickshaws on unpaved roads, the mango tree in front of the house where he lived with his
mother and stepfather. He gave Mr.~Rhodes the notes, in his left-handed curlicue scrawl. The adviser
could barely read them.

At the state dinner in his honor that night. Mr.~Obama -- after an elaborate arrival ceremony, a
business meeting with the Indonesian president and a joint news conference -- struggled to stay
awake during the toasts. When the meal was over, he retired to his room at the Shangri-La Hotel and
began writing once again.

Presidential trips are always fraught with logistical challenges, but the Indonesia visit was more
complex than most. A cloud of volcanic ash was headed toward Jakarta, and most of the reporters on
the trip were going to have to leave early -- before Mr.~Obama delivered the speech. Mr.~Rhodes,
aware of the time crunch, had promised an advance copy as soon as the president put his pen down. It
arrived by e-mail, at 12:30 a.m.

Weary Band Soldiers On

By the time Mr.~Obama arrived in this port city, pretty much everyone in his entourage, weary
reporters included, was cranky and ready to go home. His national security adviser, Thomas
E.~Donilon, opened a news briefing on Saturday afternoon by teasing that the White House had decided
to add three or four more stops. But with a string of ``bilats'' -- diplomatic jargon for bilateral,
or one-on-one, meetings between world leaders -- facing the president at the Asia-Pacific Economic
Cooperation forum, Mr.~Obama and his irritable press pool had little choice but to soldier on.

Such sessions follow a well-worn format. First comes the ``pool spray,'' where the two leaders
exchange pleasantries before the cameras. Mr.~Obama's meeting with Prime Minister Julia Gillard of
Australia -- their first face-to-face encounter -- on Saturday began typically enough. The pool
crammed into a small room, and Mr.~Obama opened by declaring that ``the United States does not have
a closer or better ally than Australia.'' Ms.~Gillard returned the favor, proclaiming the United
States and her country ``great mates.''

As aides to the leaders ushered the reporters out, a voice piped up from the back, ``Thank you,
Mr.~President!'' Mr.~Obama seemed puzzled; someone told him it was an Australian. ``I knew it must
have been an Australian,'' Mr.~Obama replied, ``because my folks never say thank you.''

With that, the American press corps broke out in a unified show of exaggerated good manners:
``Thaaaannk you, Mr.~President.''

\section{Summit Shows U.S.~Can Still Set Agenda, if Not Get Action}

\lettrine{I}{t}\mycalendar{Nov.'10}{15} could have been far worse.

That was the consensus among the leaders of the Group of 20 economic powers as they dispersed
Friday, following a two-day summit meeting dominated by anxieties over currency and trade frictions.
In their fifth meeting since the 2008 financial crisis, the leaders agreed in essence to a year-long
cooling-off period during which they will slowly tackle persistent economic imbalances.

``In Seoul, there was too much jostling over currencies, deficits and exports for the G-20 leaders
to make any significant breakthroughs,'' said David Shorr, who studies economic diplomacy for the
Stanley Foundation, an organization in Muscatine, Iowa, that advocates international cooperation.
``But there was also enough concern to avert a disastrous breakdown.''

The meeting still showed the power of the United States to set the agenda for international
discussion, even if the result -- charging the International Monetary Fund with analyzing the
sources and consequences of the imbalances -- was far less robust than American officials had hoped
for.

Expectations for this summit meeting had been low, particularly because officials from China and
Germany, the world's two biggest export-driven economies, had repeatedly criticized the United
States for weakening the dollar as a means of supporting a sluggish domestic recovery.

But in interviews here, officials from Europe and the United States said that weakening,
specifically the decision by the Federal Reserve to inject \$600 billion into the economy, was not a
major topic of discussion in the leaders' private meetings.

In those discussions, including at a dinner on Thursday and a lunch on Friday, there was
considerable agreement on the need to address a fundamental trade, or accounts, imbalance: some
economies are spending, buying and borrowing too much and others too little.

Jose Manuel Barroso, president of the European Commission, said in an interview here that the G-20
joint statement on the need to curb those imbalances was significant given that some countries did
not even support raising the issue just two years ago.

``This represents a quantum leap in terms of global economic governance,'' he said. ``I think very
frankly that the G20 has passed the test.''

For his part, President Obama suggested that the attention on imbalances had overshadowed
significant agreements on overhauling financial regulations and improving development assistance for
poor countries.

``Naturally there's an instinct to focus on the disagreements, because otherwise, these summits
might not be very exciting -- it's just a bunch of world leaders sitting around intervening,'' he
said at a news conference. ``And so there's a search for drama. But what's remarkable is that in
each of these successive summits we've actually made real progress.''

Even the British prime minister, David Cameron, who on Thursday said the G-20 was past its ``heroic
phase,'' called the forum essential.

``If we didn't have the G-20 there would be the real danger that countries might go off and pursue
their own interests,'' he said, adding: ``I don't accept the idea that the G-20 process isn't
working. It's a process.''

China's increasingly vocal and assertive posture was a defining theme of the meeting, as was the
candid American acknowledgment that the recovery of the United States -- essential for global
economic growth -- requires outside help.

Squabbling over highly technical language in the 22-page joint statement was so intense that a
meeting of key officials from the G-20 countries stretched into Friday morning, only hours before
the leaders were to bless the communiqu\'e. And in their closed-door meetings, leaders were not shy
about sharing their views.

``There were robust conversations,'' Mr.~Cameron said, when asked to characterize the leaders'
meetings. ``What I witnessed were relatively good-natured conversations. Yes, of course there were
pressures and tensions, but it was right that they were addressed at a forum like this instead of
slipping over into bilateral action.''

Mr.~Obama found himself defending the Fed's decision buy government bonds to lower long-term
interest rates, even though he has tried to avoid commenting on the central bank, an independent
entity.

At the leaders' dinner on Thursday, Mr.~Obama acknowledged that the Fed's monetary policy offered
the most promising course of speeding the American recovery, given the political impediments to any
additional fiscal stimulus measures, according to Mr.~Barroso.

The Obama administration made it clear that it was not seeking to dictate the terms of economic
debate, as it once could have done, but was seeking to foster international norms.

``What the IMF can do is the necessary function of playing the role of referee or of independent
arbiter,'' a senior administration official said. ``The IMF has to play that role – there's no
alternative to it.''

A common complaint about the G-20 is that it lacks enforcement mechanisms. The forum, still largely
informal, operates by consensus and peer pressure, not sanctions and rules. So the rationale for
cooperation on reducing imbalances is an argument that the benefits outweigh the costs.

The challenge ahead will be to specify how to measure broad imbalances, and then to fix them.

``You have to have numbers,'' said the senior American official, who spoke on condition of anonymity
according to ground rules set by the White House. ``This is economics. Just like a thermometer has
numbers on it, you have to have numbers, and everybody recognizes that.''

Or as Dominique Strauss-Kahn, the managing director of the I.M.F., put it: ``Now we have to see --
after six months, after one year, after two years -- if we're able to give some flesh around the
bone.''

Some observers have doubts. John Kirton, a political scientist at the University of Toronto and the
co-director of the G-20 Research Group at the university's Munk School of Global Affairs, said he
was disappointed that the leaders did not comment on the economic turbulence still roiling parts of
Europe, most notably Ireland.

``The G-20 has successfully moved toward becoming a global economic steering committee, but perhaps
at the expense of abandoning the core crisis response responsibility that gave it birth,'' he said.

\section{Burmese Dissident Is Freed After Long Detention}

\lettrine{M}{yanmar}\mycalendar{Nov.'10}{15}'s pro-democracy leader, Daw Aung San Suu Kyi, was freed
from house arrest on Saturday, setting her on the path to a possible new confrontation with the
generals who had kept her out of the public eye for 15 of the past 21 years.

As she stepped to the gate of the lakeside compound where she had been confined, she was greeted by
thousands of jubilant supporters, some of them in tears.

Waving and beaming in a long-sleeve pink shirt and a purple sarong, the Nobel Peace Prize laureate
could barely be heard over the cheering and chanting.

``We haven't seen each other for so long, I have so much to tell you,'' she said, immediately
re-establishing the bond that has made her such a challenge to the nation's military rulers. It had
been more than seven years since her last arrest, a period of near total separation from the world.

Her release, just five days after an election that recast the government with a civilian face,
suggested that the generals were confident of their position and ready to face down the devotion she
still commands both in her country and abroad.

Indeed, Mrs.~Aung San Suu Kyi, 65, faced an immediate challenge of mending fences within the
democratic opposition, which fractured over the question of participating in the election. But the
election, which drew accusations of fraud from almost all opposition parties, has also opened a new
area of discontent that her lawyers said she planned to exploit.

The scene at the gates of her compound suggested that her popularity remained strong. When the
police removed barricades from around her villa on Saturday afternoon, crowds flooded into the
street.

Mrs.~Aung San Suu Kyi spoke only briefly, saying, ``If we are united, we can get what we want.'' The
crowd then broke into the singing of the national anthem.

``She is our mother, she is our mother!'' a woman cried.

After someone handed her a flower, the crowd pleaded, ``Put it in your hair!''

She obliged.

It was the kind of outpouring she had experienced twice before on earlier releases from house
arrest, in 1995 and 2002. Both times she was detained again after testing the limits of her freedom.

On Sunday, wearing a white garland, she addressed a cheering crowd of thousands of people in front
of her party's headquarters. The remarks began with a call for freedom of speech, which she said was
the ``basis of democratic freedom,'' according to news services.

A government broadcast Saturday said Mrs.~Aung San Suu Kyi had been freed without conditions. There
appeared to be no other government statement on Saturday regarding her release.

But one of her lawyers, U Kyi Win, said that even if no formal conditions were placed on her
freedom, her movements could still be restricted, as they had been at times after her previous
releases.

Her most recent detention began in 2003 after she had drawn increasingly large and enthusiastic
crowds as she toured the country. A band of organized thugs attacked her convoy in what some people
believe was an assassination attempt, and she was sent first to prison and then back to house
arrest.

The immediate response from Western capitals to her release was one of celebration. Her freedom has
been their first demand in calling for political freedoms and respect for human rights in the nation
also known as Burma.

``She is a hero of mine,'' President Obama said, ``and a source of inspiration for all who work to
advance basic human rights in Burma and around the world.''

But Western leaders also made it clear that they would continue to assess the actions of the ruling
generals before they considered moderating a policy of isolation and economic sanctions against
them.

Saying France was paying close attention, President Nicolas Sarkozy said in a statement. ``Any
obstacle to her freedom of movement or expression would constitute a new and unacceptable denial of
her rights.''

And the United Nations secretary general, Ban Ki-moon, said he expected ``no further restrictions
will be placed on her'' and he urged the junta ``to build on today's action by releasing all
remaining political prisoners.''

The daughter of the nation's founding hero, U Aung San, who was assassinated when she was 2 years
old, she has embraced her fate of isolation and self-sacrifice. When her British husband, Michael
Aris, was dying of cancer in 1999, she refused to leave to visit him for fear that she would not be
allowed to return to her country.

As Mrs.~Aung San Suu Kyi resumes her struggle for democratic freedoms, several analysts said, she
will be re-entering a battleground more complicated and difficult than the one she had faced in the
past.

``It's certainly not going to be easy for her,'' said Thant Myint-U, a historian and former United
Nations official who has written widely on the country. ``This is a very, very different political
landscape than when she was released the last time. The country is facing a whole slew of new
challenges and opportunities.''

Her party, the National League for Democracy, won the previous election, in 1990, but the generals
annulled the result and clung to power. The victory gave Mrs.~Aung San Suu Kyi the standing to speak
as the nation's disenfranchised leader. But that result has now been superseded by the new election,
which the main military-backed party won overwhelmingly.

The National League declined to take part in the election, calling it unfair and undemocratic, and
was required to formally disband. But Mrs.~Aung San Suu Kyi was assailed for that decision by party
members who saw the vote, however flawed, as an opening.

A faction of those critics broke away, creating a new party and a challenge to her leadership.
National League leaders said the party would remain the platform for her activities, but it is no
longer the only center for opposition or independent action.

Though the newly elected Parliament is seen mainly as a mechanism for the military to try to
legitimize its control -- in what more than one analyst characterized as a charade, several generals
resigned from the military so they could take the top jobs in the new civilian government -- it
nevertheless changes the political dynamic with new structures and new personalities.

There will be new opposition parties, however small and weak; new political officeholders, however
limited their scope for independent action; and the first generation of military leaders who have
not been schooled at all in the West and who perceive the United States as their biggest strategic
threat.

A new world of charity and aid groups has also emerged during the past seven years of her house
arrest, leading to a more diffuse public arena.

``In many ways Myanmar is not the isolated, closed-off country that it was 10 or 20 years ago,''
Mr.~Thant Myint-U said. ``It's a very complex place. I think we could say for sure that this year,
these couple of years, are without a doubt the country's most important watershed in a generation.''

Billions of dollars in investment have been pouring in from China and other Asian nations, and
although the people of Myanmar still struggle in abject poverty, the ruling class is better off than
ever and the junta more self-confident.

Given these complexities, Mrs.~Aung San Suu Kyi's new freedom may be a burden as much as it is a
liberation.

``She'll be facing a mountain of expectation and challenges,'' said Aung Zaw, editor of The
Irrawaddy, a Thailand-based exile magazine.

``She is just a private citizen, but a lot of people still believe that she is the leader of the
democratic movement, not just a leader of the N.L.D.,'' he said. ``People want her to expand her
leadership.''

Those expectations were on display in the crowd outside her house on Saturday.

``I'm happier than if I won the lottery,'' one older woman exulted. ``But this is just the
beginning, not the end. The political prisoners are still in jail. Everyone needs to be released!''

\section{After Struggles, Obama Seeks Lift in Japan}

\lettrine{P}{resident}\mycalendar{Nov.'10}{15} Obama, looking to wrap up his sometimes rocky
economic tour of Asia on a high note, said Saturday that he made ``no apologies for doing whatever I
can'' to bring jobs to the United States. And he vowed to pursue trade partnerships and American
investment in the region so that he could put the United States back on a path of ``discovering,
creating and building the products that are sold all over the world.''

Mr.~Obama arrived in this bustling port city on Friday evening for the second of two back-to-back
summit meetings, the Asia-Pacific Economic Cooperation forum. In a speech on Saturday morning to a
group of business leaders here, he seemed to be aiming his message as much at a domestic audience in
the United States as to an Asian one.

``As president of the United States, I make no apologies for doing whatever I can to bring those
jobs and industries to America,'' said Mr.~Obama, who will return to Washington on Sunday. ``But
what I've also said throughout this trip is that in the 21st century, there's no need to view trade,
commerce or economic growth as zero-sum games, where one country always has to prosper at the
expense of another. If we work together, and act together, strengthening our economic ties can be a
win-win for all of our nations.''

Yokohama is the president's final stop on a 10-day, four-nation journey that took him to India,
Indonesia and South Korea. In the wake of the battering his party took in the midterm elections at
home last week, Mr.~Obama has cast his Asia trip as mission to revive the American economy and
bolster job growth.

But his policies, contentious at home, have proved contentious overseas as well, and he faced stiff
challenges during the Group of 20 gathering in Seoul, South Korea, from the leaders of China,
Britain, Germany and Brazil over currency policy and his contention that the United States could
pump money into its economy to stimulate growth before concentrating on reducing the deficit.

During a news conference on Friday afternoon in Seoul, Mr.~Obama attributed the conflict to his
administration's efforts to even out global trade imbalances, by pressing other nations to accept
numerical targets for limiting trade surpluses or deficits. In the end, the leaders of the world's
20 most affluent economies drafted a communiqu\'e that fell short of Mr.~Obama's goal, leaving most
of the work on creating ways to monitor and correct such imbalances for future meetings.

``Part of the reason that sometimes it seems as if the United States is attracting some dissent is
because we're initiating ideas,'' the president said at the news conference in Seoul. ``We're
putting them forward. The easiest thing for us to do would be to take a passive role and let things
just drift, which wouldn't cause any conflict. But we thought it was important for us to put forward
more structure to this idea of balanced and sustained growth. And some countries pushed back.''

Mr.~Obama arrived on the world stage two years ago to a fawning reception by world leaders. They
arrived at global conferences carrying copies of his memoir, hoping for autographs. They angled for
handshakes and ``bilats'' -- diplomatic jargon for one-on-one meetings. They maneuvered to get near
him in photo opportunities. But in Seoul, he grappled with questions about the role of the United
States as a world power, about his stature on the world stage in a post midterm environment and even
about his own diplomatic touch.

``It's not just a function of personal charm,'' the president said. ``It's a function of countries'
interests and seeing if we can work through to align them.''

When a reporter asked what kind of complaints he was hearing from fellow leaders, Mr.~Obama laughed
it off, asking, ``What about compliments?'' As to whether the elections at home have weakened him
overseas, he served up a one-word answer: No.

And while he told reporters before leaving Washington that his relationship with the American people
had gotten ``rockier and tougher'' over time, he said in Seoul that his relations with foreign
leaders had actually grown stronger since he took office.

``When I first came into office, people might have been interested in more photo ops because there
had been a lot of hoopla surrounding my election,'' Mr.~Obama said, adding that he now had a
``genuine friendship'' with a raft of world leaders. Still, he said, ``That doesn't mean there
aren't going to be differences.''

``It wasn't any easier to talk about currency when I had just been elected and my poll numbers were
at 65 percent than it is now,'' Mr.~Obama said. ``It was hard then and it is now.''

Despite his stumbles in Seoul, Mr.~Obama's first two stops -- India and Indonesia -- were quite
successful for him. In India, he lifted export controls that had banned American companies from
selling sensitive technologies and backed India's bid for a permanent seat on the United Nations
Security Council -- moves that foreign policy analysts in both parties credited with turning around
a relationship that had been faltering. In Indonesia, where he lived for four years as a boy, he had
a sentimental homecoming -- even though it was cut short when a cloud of volcanic ash interfered
with air travel, forcing him to leave several hours early.

In Yokohama, Mr.~Obama tried to sum up those highlights, as he sought to link America's own history
of innovation and discovery -- ``the idea that led us westward and skyward'' -- with Asia's own
economic transformation, and the economic development he had witnessed here.

``In different ways and different places over the last week, I have seen this idea alive in the
teeming, thriving democracies of Asia,'' he said. `` It gives me great confidence in the ties that
bind our people, and great hope in our ability to move toward the future together -- not as drops,
but with the strength of an ocean.''

\section{Crowds Hope for Release of Myanmar Leader}

\lettrine{S}{ecurity}\mycalendar{Nov.'10}{15} was tight around the home of Myanmar's pro-democracy
leader, Daw Aung San Suu Kyi, in anticipation of her possible release Saturday from house arrest,
where she has spent 15 of the past 20 years.

A small group of her supporters gathered on Saturday outside a barbed wire barrier reading
``Restricted Area,'' some holding her portrait, some wearing white T-shirts with the words, ``We
stand with Aung San Suu Kyi.''

Her release has been the first priority for Western nations seeking to pressure the ruling junta on
questions of human rights and political freedom, and the generals may see less need to isolate her
because long-awaited parliamentary elections are over and the parties they backed are firmly in
power.

If she is released, there may be no immediate change to long-standing policies of isolation and
sanctions against the military leadership. Western diplomats said they would assess the degree of
liberty she would be afforded and watch the behavior of the military toward her.

Mrs.~Aung San Suu Kyi's lawyer, U Nyan Win, said recently that she intended to plunge again into
political activities, no matter what restrictions might be placed on her. The possibility remained
of future confrontations that would bring further condemnation from the West.

Mrs.~Aung San Suu Kyi, 65, a Nobel Peace Prize laureate, has come to symbolize nonviolent resistance
both within and outside Myanmar, and what the playwright Vaclav Havel called ``the power of the
powerless.''

Though she long has had almost no contact with the world outside her lakeside villa, she has
remained a symbol of hope for many in Myanmar, formerly Burma, that the generals could not overcome.

``She will be linking up with the people, who very much desire her release to work for democracy and
human rights,'' said U Tin Oo, deputy leader of the party, the National League for Democracy, who
was himself released in February after seven years of detention.

``She will again lead the N.L.D.,'' he said in a telephone interview. ``She is already the
democratic leader of Burma and an icon.''

The junta has released her twice before, in 1995 and 2002, calculating that her extended absence
from public view had weakened her appeal. They were proved wrong. Huge, enthusiastic crowds greeted
her wherever she went, particularly in 2002.

In both cases, she was returned to house arrest.

This time, said Josef Silverstein, a Myanmar specialist and professor emeritus at Rutgers
University, ``I think they feel pretty confident that they so controlled the election, that there
was not much violence, a quietude, that they can take a chance on her freedom.''

But he added, ``This woman didn't go through hell to remain silent at this particular point.''

The parliamentary elections on Sunday were the first in 20 years. In the last, the National League
for Democracy won in a landslide.

The generals annulled that result and clung to power. This month's election was seen as their
attempt to gain legitimacy. Though it will be dominated by the military, the new Parliament will be
the first civilian government in the country since 1962.

The National League for Democracy refused to take part in the election, saying it was unfair and
undemocratic. As a result it was forced to disband as a political party, and Mrs.~Aung San Suu Kyi
now has no official standing as a political leader.

But not everybody accepts this.

``We at the N.L.D. still consider ourselves to be in existence,'' said Mr.~Tin Oo, the party deputy
leader. ``We still honor the result of 1990 and we will respect this. Nobody can hold an election
until the problem of the 1990 election is resolved. This was the mandate of the people.''

Mrs.~Aung San Suu Kyi's most recent term of house arrest began in 2003 after an attack on her
motorcade by government-sponsored thugs that some people believe was an assassination attempt.

Her detention was extended in August last year when an American, John Yettaw, swam across a lake
uninvited to her home, leading to a trial that convicted her of violating the terms of her
detention. The date for the conclusion of her term was set in a message to the court from the leader
of Myanmar's junta, Senior Gen.~Than Shwe.

In the past, as her terms neared completion, extensions would be imposed. But this time, it appeared
that she might be freed as promised.

The crowd outside her party headquarters on Friday buzzed with anticipation. A man rode a bicycle
near the house with a sign hanging from the handlebars with a picture of Mrs.~Aung San Suu Kyi and a
poem titled ``Until We Achieve Success.''

``Please wait peacefully,'' U Win Tin, one of her former advisers, told the crowd. ``She will come
and see all of you.''

\section{H.I.V. Discrimination Law Fails in Chinese Court}

\lettrine{I}{n}\mycalendar{Nov.'10}{15} a rare, public test of the nation's law prohibiting
discrimination against people with H.I.V., a Chinese court on Friday ruled against a man who said he
was wrongly denied a teaching job after his prospective employer learned he had the virus that
causes AIDS.

The man who filed the lawsuit, a 22-year-old college graduate, had passed a battery of written tests
and an interview when a mandatory blood test revealed his H.I.V. status, prompting the local
education bureau in the eastern city of Anqing to reject his application.

``I'm heartbroken,'' said the man, who used the alias Xiao Wu in legal papers to protect his
identity. ``I just wanted to find some justice for me and for others facing the same problem.''
Lawyers for the man said they would appeal.

In his ruling, the judge agreed with the education bureau's contention that regulations barring
H.I.V.-infected civil servants trumped a four-year-old law that was supposed to protect people with
the virus from the prejudice of employers. That measure, passed by the State Council, the
government's chief administrative body, states that ``no institution or individual shall
discriminate against people living with H.I.V., AIDS patients and their relatives.''

Li Fangping, a lawyer who argued Xiao Wu's case during a three-hour trial last month, said the
judge's decision defied logic. ``It's an example of how the legal system enhances and expands
discrimination against people who are H.I.V. positive,'' he said.

People with AIDS have increasing access to medical treatment in China, but they are widely shunned
and often barred from universities, state jobs and private corporations. The ostracism has serious
implications: in a report last year, the United Nations said fear and ignorance kept many of the
estimated 740,000 Chinese infected with H.I.V. from seeking treatment.

The government has come a long way since the 1990s, when it went to great lengths to cover up a
scandal in which thousands contracted the disease at state-run transfusion programs.

These days, people with AIDS have access to free antiretroviral drugs, and China's top leaders,
Prime Minister Wen Jiabao and President Hu Jintao, make a show of consoling people with AIDS each
World AIDS Day. The government earlier this year lifted a ban on H.I.V.-infected foreigners'
visiting China.

But AIDS advocates say they face a wealth of restrictions that make it hard to carry out grass-roots
activities. Wan Yanhai, the founder of the AIDS organization Aizhixing Institute, moved to the
United States last May, claiming government harassment had made it impossible to carry out his work.

On Thursday, Beijing Loving Source, a children's AIDS charity founded by the jailed dissident Hu
Jia, announced it was shutting down after repeated scrutiny by the tax authorities.

In a closely watched case, Tian Xi, an AIDS activist who contracted H.I.V. through a blood
transfusion, is awaiting sentencing in Henan Province on charges that his protests against the
hospital responsible for his infection resulted in property damage.

In a way, the legal travails of Xiao Wu had been a bright spot for AIDS activists, who for years had
seen a series of job-discrimination lawsuits rejected by Chinese courts before going to trial.
Domestic media coverage of the case has been sympathetic, and given the central government's laws
against discrimination, legal advocates hoped a positive outcome would set a precedent.

Last month, an H.I.V.-positive college graduate, who was encouraged by Xiao Wu, filed a similar case
in Sichuan Province.

Now advocates worry that Friday's ruling will have the opposite effect, providing legal cover for
employers who do not want to hire people with H.I.V.

``This is bad news, given that it was the first time an H.I.V.-positive person dared to stand up for
his rights,'' said Yu Fangqiang, an AIDS advocate whose organization, Beijing Yirenping, provided
free representation to the defendant. ``The entire H.I.V. community had high hopes, but now the door
appears to be shutting for people who want to use the courts to fight against discrimination.''

\section{Qing Dynasty Relic Yields Record Price at Auction}

\lettrine{A}{s}\mycalendar{Nov.'10}{15} treasure-in-the-attic stories go, the 18th-century Chinese
vase sold at a suburban auction house in outer London on Thursday night will be hard to beat.

The delicate, decorative 16-inch vase started at a not-inconsequential \$800,000, but after a
half-hour of unexpectedly spirited bidding, the gavel fell at \$69.5 million. It was the highest
price ever paid at auction for a Chinese antiquity.

Adding in the 20 percent buyer's premium levied by the auction house and Britain's value-added tax,
the total came to \$85.9 million. Auction insiders said the buyer was from mainland China and bid by
telephone.

Of the sellers, the auction house, Bainbridge's, said only that they were a brother and sister who
had found the vase ``in a dusty attic'' when they were clearing out the family home in west London,
near Heathrow Airport, after their parents died. The other Chinese knickknacks they found sold for
as little as \$65.

``They had no idea what they had,'' said Helen Porter, a spokeswoman for Bainbridge's. ``They were
hopeful, but they didn't dare believe until the hammer went down. When it did, the sister had to go
out of the room and have a breath of fresh air.''

The vase dated from the period of the emperor Qianlong, who reigned from 1735 to 1796, at the height
of the Qing dynasty. He vastly expanded China's western territories and left a legacy as a great
patron of Chinese arts, including ceramics. Experts who have examined the vase, which bore an
imperial seal, have said it was likely to have been made for one of the imperial palaces.

Ovoid in shape and predominantly pastel yellow and sky-blue in color, the vase has a narrow neck,
four enameled circular motifs known as cartouches that show colorful fish and flowers, and elaborate
perforations in the outer vase that give onto a smaller vase inside. It was believed to have been
fired in the imperial potteries in Jingdezhen, in Jiangxi Province, west of modern-day Shanghai,
which functioned for 1,000 years as the porcelain capital of China.

Ms.~Porter said the sellers had no knowledge of how the vase came to be in their parents'
possession, although they believed it had been in the family since the 1930s. One theory, according
to Ivan Macquisten, the editor of Antiques Trade Gazette, a British magazine, was that it could have
been among the treasures looted by British troops when they sacked the imperial palaces in Beijing
during the second Opium War, from 1856 to 1860.

It was one of Mr.~Macquisten's reporters who found out what little was known about the buyer.

With China's wealth rapidly rising, mainland Chinese buyers have been a major force in pushing up
the prices of Chinese antiquities, reversing, at least in small measure, the flow of Chinese
artworks to the West during the centuries before the Communist revolution in 1949 -- and the loss of
imperial treasures when the Chinese nationalists fled the Communist victory for Taiwan, taking huge
quantities of antiquities with them.

The vase's price exceeded the record for Chinese antiquities set just last month in Hong Kong, when
another Qianlong vase sold for \$34.2 million.

For Bainbridge's, the sale price of the vase represented a huge leap, putting the auction house, at
least momentarily, in a league with the blue-ribbon art houses like Sotheby's and Christie's, where
sales running into the tens of millions of dollars have become almost routine in recent decades.
Bainbridge's biggest sale before Thursday was \$160,000 for a Ming enamel piece it sold two years
ago.

The sale was held in the London suburb of Ruislip, neighboring Pinner, where the vase was found.
Pinner is best known in modern times as the place where the singers Elton John and Simon Le Bon went
to school.

As the auction house was trying to establish a selling price, Ms.~Porter said, the vase had been
taken for viewing at the Arts Club in London, where it was deposited for some time on a ``metal
table next to the kitchen.''

The auction house itself began to realize its rarity only when a consultant on Chinese ceramics,
Luan Grocholski, was called in to evaluate it. ``Luan took a long, hard look at it and could hardly
believe his eyes,'' Ms.~Porter said.

Still, Bainbridge's had set its presale estimate between \$1.3 million and \$2 million.

``We are absolutely stunned,'' Ms.~Porter said after the auction. ``This must be one of the most
important Chinese vases to be offered for sale this century. How it reached Ruislip is something we
will never know, and that it is in such fine condition is amazing.

``We're just a very typical local auction house, so as you can imagine it was something of a
surprise.''

\section{Obama Speech Marks Shift on North Korea}

\lettrine{P}{resident}\mycalendar{Nov.'10}{15} Obama said Thursday that the United States would be
willing to restart stalled disarmament talks with North Korea if that country showed a ``seriousness
of purpose,'' in what analysts called a slight softening of the stance by Washington and its allies.

Speaking after a meeting with President Lee Myung-bak of South Korea on the sidelines of the Group
of 20 summit meeting in Seoul, Mr.~Obama said the North would have to show that it was serious about
ending its nuclear weapons program before the talks -- involving the two Koreas, China, Japan,
Russia and the United States -- could resume.

He said, however, that the United States, South Korea and other nations were ready to offer hefty
economic aid if the North Koreans gave up their weapons once and for all.

``They have a choice available to them,'' Mr.~Obama said. ``At the point where it appears that they
are in fact prepared to move forward on the kind of path that all of us want to see, then we're
going to be there ready to negotiate with them.''

Starting in 2003, the six-nation talks were focused on persuading North Korea to give up its nuclear
ambitions. But the talks were fitful, and North Korea pulled out last year after international
condemnation of its test of a long-range missile. Washington has made offers before to try to
restart the negotiations, but political analysts in Seoul said there was now a less strident tone
coming from the United States and particularly South Korea. They said the moves might reflect a
desire to reduce tensions and bring North Korea back to the negotiating table now that the ailing
North Korean leader, Kim Jong-il, has positioned his youngest son as his successor.

``Now that the succession is almost completed, there is a definite mood shift toward resuming the
six-party talks,'' said Kim Hyun-wook, a professor at the Institute of Foreign Affairs and National
Security, a government-run research institute in Seoul.

The shift began a few days ago, when the South Korean president, a conservative who had taken a
hawkish stance toward the North, backed away from his insistence that the talks could not resume
unless the North apologized for the March sinking of a South Korean warship, the Cheonan, which
killed 46 sailors. South Korea said it was sunk by a North Korean torpedo; the North has denied
involvement.

On Thursday, Mr.~Lee specified his more modest requirements, saying, ``North Korea should and must
show sincerity toward the Republic of Korea and to assume responsibility for what they did to the
Cheonan.''

Analysts said there had also been recent signals by China and Russia, both longtime allies of North
Korea, that they desired a reduction of tensions in the region. On Thursday, Mr.~Lee and the Chinese
president, Hu Jintao, who was also attending the Group of 20 summit meeting, agreed to cooperate in
seeking North Korea's nuclear disarmament.

North Korea has signaled its willingness to return to the talks. It blames hostility from the United
States and South Korea for the current impasse; the two countries were among its foes during the
1950-53 Korean War and held drills in waters near North Korea after the Cheonan's sinking.

Earlier Thursday, Mr.~Obama said in a Veterans Day speech at an American Army base in Seoul that the
United States remained committed to defending South Korea, and he warned North Korea that it faced
continued isolation unless it fulfilled its commitments to give up nuclear weapons.

In a strongly worded statement made before thousands of soldiers and Marines gathered on a chilly
morning, Mr.~Obama called North Korea a starving nation whose economic failures were visible even
from space, where at night ``the brilliant lights of Seoul'' can be seen giving way ``to the utter
darkness of the North.''

During the speech, Mr.~Obama also honored the 37,000 Americans and the far larger number of South
Koreans who died in the Korean War. He led the audience in a standing ovation for 62 veterans of
that war who attended the speech at the Yongsan base, in the center of this city of gleaming
skyscrapers and modern highways.

``Gentlemen, we are honored by your presence,'' Mr.~Obama said. ``We are grateful for your service.
And the world is better off because of what you did here.''

\section{In China, Money Can Often Buy Love}

\lettrine{M}{oney}\mycalendar{Nov.'10}{15} really can buy you love in China -- or at least that
seems to be a common belief in this increasingly materialistic country.

Many personal stories seem to confirm that the ideal mate is the one who can deliver a home and a
car, among other things; sentiment is secondary.

However widespread this mercantilist spirit, not everyone thinks it is a good thing. A spate of
Chinese films, plays and television shows have raised the question: What is love in an age of
breakneck economic growth?

Many Chinese were shocked this year when a female contestant on a popular TV dating show, ``If You
Are the One,'' announced: ``I'd rather cry in a BMW than smile on a bicycle.'' But others insisted
that the contestant, Ma Nuo, now popularly known as ``the BMW woman,'' was merely expressing a
social reality.

Rocketing property prices in recent years have contributed to such feelings, with many people in
Beijing and other cities accepting the idea that a woman will pursue a relationship with a man only
if he already owns an apartment.

Feng Yuan, a 26-year-old who works in a government education company, tried to set up a friend with
a man she thought suitable.

``When she heard he didn't own an apartment, she refused even to meet him,'' recalled Ms.~Feng.
``She said, `What's the point? Without an apartment, love isn't possible.'''

Fueling these attitudes is a drumbeat of fear. After three decades of fast-paced, uneven economic
growth, there is enormous anxiety among those who feel they are being left behind, lacking the
opportunities and contacts to make big money while all around them others prosper and prices soar.

The new creed can be hard, as a 26-year-old cultural events organizer learned.

The man, who asked for anonymity to protect his privacy, earns about 4,000 renminbi, or \$600, a
month, making even a modest apartment in an unfashionable district of Beijing unaffordable. These
homes can cost about \$3,000 per square meter, or about \$280 per square foot. Housing inflation is
severe. Ten years ago, a similar apartment cost about \$345 per square meter.

Instead, he tried to impress his girlfriend of three years by saving for a year to buy an iPhone 3.
The newer iPhone 4 -- a hot status symbol -- had just gone on sale. But at about \$900, that was
beyond his means.

The phone was not enough. Last week, she left him, citing pressure from her parents to find a richer
mate.

He is heartbroken, believing, despite all, that his girlfriend truly loved him. ``Why else did she
live with me for three years?'' -- albeit in a rented apartment. Yet, he is philosophical, too.

``I understand her situation and the pressure from her family,'' he said. ``I also understand that
her parents want their daughter to find someone who can give her a better life.''

The only way to find love, he said, is to become rich. ``The most important thing for me now, is to
work and earn a living.'' he said. ``I need to grow stronger, support myself and my parents, and
then my future girlfriend can have a good life.''

Such calculations have their critics. The hard-nosed attitude of Ms.~Ma, the BMW woman, earned her a
gentle reprimand recently from the film director Zhang Yimou. In an interview in The South China
Morning Post, a Hong Kong newspaper, he urged young people to re-examine their values.

``I don't think economic advancement and our yearning for love are mutually exclusive,'' he said.

Mr.~Zhang, who turns 59 on Sunday, represents an older generation that remembers the more
egalitarian, if also poorer and more politically repressive, Maoist era, before the economic changes
that unleashed the scramble for material advancement.

His latest film, ``Under the Hawthorn Tree,'' depicts the innocent love between a teacher, Jing Qiu,
and a geologist, Lao San. Set in 1975 toward the end of the Cultural Revolution, and without a BMW
in sight, the film shows the teacher spending quite a lot of time smiling on her sweetheart's
bicycle. Love is the thing, it concludes.

Other productions have joined the debate.

``Fight the Landlord,'' a play by Sun Yue that premiered in Shanghai last month, is another ringing
defense of love in an age of materialism.

A character known as B, grilled by a potential mother-in-law about her very ordinary income, yells:
``Don't think that because I have nothing to be proud of you can insult and destroy me!''

``I have my dignity and pride,'' B says, ``and I don't want to turn love, which I value so much,
into something vulgar and pale!''

A new film, ``Color Me Love,'' celebrates the cult of materialism but also comes down, somewhat, on
the side of love. Modeled on ``The Devil Wears Prada,'' and with product placement for Herm\`es,
Versace and Diesel, it follows poor but gorgeous Fei as she arrives in Beijing to intern at a
fashion magazine.

``Fei, one day you'll understand,'' Zoe, her glamorous editor, cautions her. ``Nothing is as
important as the person you'll spend the rest of your life with.''

A tumultuous courtship with a wacky artist named Yihong ends up with the couple united in New York.
A closing shot shows her in his arms, a diamond on her finger. The real fantasy, perhaps, is love
plus money.

Ms.~Feng, who had failed to find a match for her apartmentless friend, said the demands that many
Chinese women make on prospective mates reflected weakness, not power. Lower in status, they fear
not getting what they want in life, and look to men to provide it.

``Women are very dependent,'' she said. ``I blame them. Why can't they work hard and buy a house
together with their man? But very few women today think like that.''

Few Chinese men do either, reinforcing the rules of the game. For the 26-year-old events organizer,
losing his love to money was justifiable.

``We didn't need to waste time on a relationship that was doomed to vanish,'' he said.

\section{Zagat Survey Aims to Regain Its Online Balance}

\lettrine{B}{ack}\mycalendar{Nov.'10}{15} in the early '80s, when the Zagat Survey was still just a
single sheet of legal-size paper that Nina and Tim Zagat gave to friends and colleagues, they
learned that their list of the best restaurants in New York had been reproduced -- en masse --
inside Citibank.

``I had a friend there who called me and said he'd just gotten one of these things on his desk,''
Mr.~Zagat recalls. ``Printed across the top, it said, `To all officers of the bank.' And when I
asked how many officers there were, he told me there were 3,000.''

For the Zagats, this was one of the first signs that their hobby of tabulating restaurant ratings
might have wide appeal. In fairly short order, the New York survey, and the narrow maroon books for
other major cities that the couple assembled in its wake, became indispensable for gourmands on the
go in the late 1980s and throughout the 1990s.

And up until a few years ago, the release of the coming year's Zagat (pronounced zuh-GAT) ratings
was a hotly anticipated event among the status conscious and the merely hungry -- with results that
would often go viral. When a humble Brooklyn establishment called the Grocery scored an almost
perfect Zagat score of 28 -- on a zero-to-30 scale -- for its food in 2003, for example, it made the
front page of this newspaper.

Today, however, ``viral'' is hardly the first word that pops off of people's tongues when you ask
about Zagat Survey, the company's official name. They talk about the power of the Zagat brand; that
maroon color is now trademarked, for instance. And they give the Zagats credit for having invented,
or at least popularized, user-generated content.

But in the next breath, most of them wonder why Zagat hasn't won on the Web. The review site Yelp,
for example, which made its debut in 2004, draws much more traffic.

In early 2008, at a time of high valuations for many emerging Web companies, the Zagats tried and
failed to sell their company, to the disappointment of some of the venture capitalists who had
invested in Zagat in 2000.

``We've now been at it for 10 years, so of course we'd like to have an exit,'' says Bill Ford, the
C.E.O. of the venture capital firm General Atlantic, which has invested in Zagat. ``It's my belief
that the brand has not been fully leveraged.''

Mr.~Ford, who was once a Zagat board member, is quick to note that the things that attracted General
Atlantic to Zagat in 2000 are still there today. ``They had outstanding and differentiated content,
and the opportunity to migrate it from offline to online, both on the Web and on mobile,'' he says.

Yet that migration has been neither as dexterous nor as profitable as Zagat backers had hoped.
Perhaps the biggest reason is that the Zagats have kept their ratings and their reviews behind a pay
wall on their Web site.

The cost was high in terms of lost traffic, for just as the Zagats were digging in their heels,
Google was rapidly gaining market share among Web users. Google penalizes sites that keep content
behind a pay wall. As a result, Zagat listings usually didn't appear on the first page of a Google
search result for specific restaurants. Because of that, fewer people came to Zagat.com, which meant
that it didn't have the opportunity to convert them to paid subscribers.

While online traffic comparisons are never exact, Zagat.com had 570,000 unique domestic Web visitors
in September, according to the Nielsen Company, versus 9.4 million for Yelp. The Zagats say they
actually have more than 1.2 million unique users worldwide.

None of this means Zagat is in any immediate financial danger. While the Zagats refused to provide
much specific financial information for this article, they did say their company is still
profitable.

They are also quick to point out that it's not at all clear that Yelp's strategy of having dozens of
salespeople selling local ads is superior to their approach of asking Web and mobile users to pay
for listings. Zagat, meanwhile, still sells its paper books and operates what the couple describe as
an extremely profitable unit that creates custom guides for corporate clients.

The digital wheel has also turned in a new direction, with more consumers using smartphones and
other mobile devices to find content and services through apps. While Zagat.com has met resistance
from desktop computer users when it comes time to whip out their credit cards, mobile customers have
shown more willingness to pay for apps.

So now Zagat is racing to prove that it's worth paying \$9.99 a year for its smartphone app -- and
that it can persuade enough customers to fund its existence for another three decades -- even while
Yelp and other competitors are offering their apps free.

Even after the failed sale in 2008, the Zagats profess no regrets about their Web strategy. In fact,
they think the world is finally catching up to them.

``When I think back on the business model, with our books designed to fit in a pocketbook, there we
were at the beginning with all three things that we're all talking about today,'' Ms.~Zagat says.
``Mobile, social and local. That is in the D.N.A. of Zagat.''

TIM ZAGAT and Nina Safronoff got to know one another in 1963, in a study group they joined as
first-year students at Yale Law School.

``I liked her for plenty of reasons, in part because she took such great notes but also because she
knew how to cook,'' Mr.~Zagat says.

They bonded over home-cooked meals, both good and bad.

``My mother didn't think anything was safe unless it had been frozen,'' he says. ``She would stick
whatever it was into the freezer after it was cooked. In 1959, she'd pull out the 1957 turkey dinner
and the 1953 ham. Then she'd put it in a casserole with Campbell's cream of mushroom soup.''

Ms.~Zagat interjects: ``The thought of it is making me nauseous.''

``And then,'' Mr.~Zagat says, ``we'd go down to the pizza place.''

They married during their second year in law school and eventually moved to Paris to work as
lawyers. It was there, in 1968, that they began asking friends to rate their favorite restaurants.
The couple would consolidate the lists and distribute the results to others they knew around town.

When they returned to New York, they kept the Paris list updated but were otherwise occupied with
legal work. It wasn't until 1979, when a friend from their wine-tasting club complained vociferously
about a local restaurant critic, that the Zagats suggested that the group tap its hive mind and make
its own assessment.

The Zagats queried 200 people, and each year the size of that polling group multiplied. They
copyrighted the material but didn't mind when people passed it around, and they didn't charge for a
copy of the survey, either. Soon, their hobby was costing them well into the five figures annually
for their own restaurant meals and other costs.

So they approached some New York publishers to see if they wanted to put the book out themselves.
``We were turned down by the highest-quality people,'' Mr.~Zagat said. They included his uncle, who
ran Atheneum, and editors at Scribner and Doubleday, which were clients of Ms.~Zagat's law firm.

``The publishers kept saying that people don't want to hear from people like them, they want to hear
from experts,'' Ms.~Zagat recalls. ``It's sort of amazing when you look back on it now.''

It was also an incredible stroke of luck for the couple. Had one of the publishers taken on the New
York survey, the Zagats might have earned a nice pile of royalties but wouldn't have started their
company. Instead, they printed their first book in 1982 and began hauling boxes in their station
wagon, calling on bookstores all over Manhattan.

Many owners turned up their noses, saying they didn't stock self-published books. But the Madison
Avenue Bookshop agreed to take the books on consignment. The local Doubleday chain also took a
chance on the book, and it became a best seller in its stores. The Zagats circled the city with the
Doubleday best-seller list in hand, persuading more stores to stock the guides.

Sales took off in 1985, though, when the Zagats were featured on the cover of New York magazine for
an article called ``The Food Spooks.'' Sales went from 40,000 a year to 75,000 a month, and the
couple used the money they earned on the New York book to assemble restaurant guides in other major
cities. Not long after, they quit their day jobs.

By the 1990s, Zagat Survey was also reviewing hotels, specialty retailers and other types of
businesses. It also licensed its content to the early Web services Prodigy and Compuserve. But it
wasn't until 1999 that they unveiled a full-feature site.

A year later, at the height of the Internet bubble, investors came calling. General Atlantic,
Kleiner Perkins Caufield \& Byers, and Allen \& Company lined up, as did Nathan Myhrvold, the former
Microsoft chief technology officer, and Nicholas Negroponte, co-founder of the Media Lab at M.I.T.
Altogether, they invested \$31 million. At the time, that bought about a 25 percent stake in Zagat,
which put the company's value at about \$125 million, according to the Zagats.

The couple quickly embraced the cost savings that the Web site provided and stopped distributing
paper ballots to friends and strangers. Not long after, they made a pivotal move: they began
charging Web users for ratings and reviews of each restaurant -- though they did allow
nonsubscribers to get addresses and phone numbers free.

Mr.~Myhrvold, who carries the ceremonial title of chief gastronomic officer at the company,
understood the potential outcome of the free-versus-paid decision better than the Zagats did. While
Mr.~Myhrvold was at Microsoft, the online magazine Slate, now owned by the Washington Post Company,
reported in to him.

Slate, which was free when it began in 1996, started charging for access to stories in 1998 and
reversed course in 1999. Still, Mr.~Myhrvold supported the Zagats' decision to put up a pay wall on
Zagat.com. ``If you put it all online for free, you will dramatically cannibalize your book sales,''
he says now, echoing a dilemma that has challenged other book publishers, newspapers and magazines.

And given how profitable the books were, the Zagats had little interest in losing those sales. The
custom guides for corporations were especially lucrative because the Zagats didn't have to share
revenue with bookstores or deal with returns. The sales to banks, ad agencies and others had grown
so large that well-connected New Yorkers considered it a point of pride never to pay out of pocket
for a guide. One year, Bank of America ordered five million.

But while the company maintained its margins on the paper books, it was missing the opportunity to
snare new customers who were trawling Google for information on specific restaurants.

``There is a huge degradation of your brand if you are not showing up on that first page of search
results,'' says Greg Sterling, an independent analyst who specializes in local content and
advertising. ``If you're not there, you have to have a powerful, multifaceted alternative strategy,
with public relations, advertising, a reality show or something.''

Zagat.com does come up on the first page of Google results for broader searches that don't identify
individual restaurants. But some members of the Zagat team still think the company may have fumbled
its Web strategy.

``I'm not sure we got the pricing model exactly right, in terms of how much was in front of the pay
wall, though we were very early in trying to figure it out,'' says Mr.~Ford, the Zagat investor and
former director. ``If more of the content had been free, it would have driven more trial and
exposure.''

BECAUSE the best Zagat content wasn't free, there was room for start-ups like Yelp to lure online
users to their sites, which were chock-full of user ratings of restaurants, shops and other service
providers.

Yelp also did Zagat one better by making its contributors the stars of its site. Contributors to
Zagat's books are lucky if six or eight of their words end up in the pithy summaries that Zagat
editors compile -- ``Margaritas so good you might wake up married,'' for example. On Yelp, review
writers can build a following, achieve a sort of elite status and get invitations to special parties
hosted by restaurants and bars.

All of these developments were clear by early 2008, when the Zagats decided to try to sell the
company, though they bristle at the suggestion that they were the ones pushing to put the company on
the auction block.

``I don't remember who called who,'' Mr.~Zagat says.

``A lot of bankers were calling us,'' Ms.~Zagat says. ``But we never had any pressure from
investors.''

``There were numbers being tossed around that were so high,'' Mr.~Zagat recalls. ``We said, `If you
can produce those numbers, let us know.' ''

The company hired Goldman Sachs to handle the sale. A target price of \$200 million was widely
reported at the time, though the Zagats won't confirm that figure.

According to four people who saw the numbers in 2008 but requested anonymity because they had agreed
not to speak publicly about them, Zagat Survey had about \$40 million in annual revenue at the time,
about \$30 million of that from paper books. It had cash flow of \$8 million to \$10 million, they
say.

The Zagats would not comment on those numbers, but they scoff at these same people's assertions that
the highest bid came in around \$125 million. Mr.~Zagat said it was more like \$175 million.

Whatever the price, the Zagats rejected it. They blame the burgeoning financial crisis for
undermining the sale.

For his part, Mr.~Ford wishes that the sale would have moved a bit more briskly. ``If they had been
a little more aggressive in timing, we would not have hit the wall,'' he says. ``It was an
opportunity squandered.''

Wherever the blame lies, not long after the failed sale it became clear that at least one big-name
buyer saw much more value in Yelp's approach to reviews. In late 2009, Google reportedly wanted to
pay at least \$500 million to acquire Yelp, or roughly three times what the highest bid for Zagat
had been.

The Zagats dispute that comparison, noting that no one has ever publicly confirmed the price bid for
either company. Besides, Mr.~Zagat says, the companies derive their revenues in very different ways.
His company sells content to consumers and corporations, while Yelp sells ads to local businesses.

Still, both need consumer eyeballs to succeed, and Yelp, thanks to its free content and success at
landing on the first page of Google search results, has lapped Zagat on the traffic front.

There is also a perception that Yelp is on a different trajectory. ``Yelp is a local advertising
vehicle for lots of small businesses, and there are billions of dollars out there, a big pie waiting
to be fully eaten,'' says Mr.~Sterling, the analyst. ``Zagat feels more tied to the past, so there
isn't the same perception of momentum or revenue opportunities.''

Some say Zagat has been saddled with management problems as well. Although the company has taken
stabs at diversifying and upgrading its management, many senior executives didn't last long --
including a former chief executive, Amy McIntosh, who stayed only about a year, starting in 2000.

Ms.~McIntosh declined to comment, but Mr.~Ford, the former Zagat director, said problems arose
because the Zagats like to remain in charge. ``I think at the end of the day, Tim and Nina were not
really prepared to transition the leadership of the firm to somebody else,'' Mr.~Ford said. ``What
we were hoping with Amy was to access different skills, particularly around new media.''

Mr.~and Ms.~Zagat, who are 70 and 68, respectively, say they have no plans to retire. They declined
to comment on Ms.~McIntosh's tenure specifically.

``It's an unusual business because it's a family business,'' Ms.~Zagat says. ``Our name is on the
door, and our name is a brand. We believe in flat decision-making and streamlined decision-making
across the board, and we think that's been very effective.'' They added that six of the eight
members of the current senior management team have been with Zagat more than six years.

While their son Ted once served as president, he left the company in 2007 to join Univision, the
media conglomerate. ``He left because he wanted to do his own thing,'' says Mr.~Zagat. ``I'm glad
that he's happy, but if he should change his mind, I wish he would come back.''

Ted Zagat declined to comment, citing Univision corporate policy.

Now leading the Zagat mobile efforts is Ryan Charles, 30, who says he has felt free to do what is
best for the company.

``They look to me to innovate and were very flexible and supportive of our reaching out to
Foursquare and Foodspotting even before they became big,'' he said. ``And we were on Android 6 to 12
months before our competitors.

``So they've been supportive of letting me push the envelope from both a partner standpoint and a
platform one, which allows for a real entrepreneurial feel around mobile.''

The integration of new-media ideas in old-line media companies has always been a challenge.

``Established companies rarely innovate well no matter what field you're in,'' says Merrill Brown, a
media consultant and former executive at MSNBC.com and Court TV. ``The polite response is that
they've been profitably managing a successful media brand.''

Still, Mr.~Brown points to the proliferation of food blogs and sites like MenuPages and Chowhound
and wonders why Zagat didn't start similar sites first. ``There is a massive food community on the
Web that exists today, and they're not widely considered a giant player,'' he says. ``And looking
back at what they were, they could have been a significantly more engaged digital player than they
are today.''

The Zagats aim to rectify much of this when a Web site overhaul makes its debut next year. New code
and procedures will make the free material more search-engine friendly. And free material will be
more plentiful. Earlier this month, for instance, the site introduced a clever map that tracks the
daily wanderings of many New York City food trucks. While Zagat already lets customers post their
own reviews on its site, it will add new features intended to make it more social.

The couple remain unapologetic about charging for content. ``Any discussion of possibly being free,
I think, stopped at least a year ago,'' Ms.~Zagat says.

``When Time magazine and The New York Times and practically everybody smart that looked at this
started saying that it had to be paid,'' Mr.~Zagat adds, ``it just confirmed our feelings and ended
the discussion.''

The couple say the ever-expanding market for smartphone apps gives them confidence. They also have a
deep inventory of tightly written reviews, which could give them a big boost in the app world.

``Tim and Nina were very smart in that they did something that Yelp has still not gotten right,''
says Peter Steinberg, a vice president for product development at Zagat for a short period starting
in 1999. ``They boiled it down. It's a huge pain on Yelp to read 17 reviews, and they go on and on
and on.''

THE Zagat app has another crucial advantage: users can download all of the guides into their phones.
So when the AT\&T signal fritzes out, their iPhones can still offer advice when they're hungry and
standing on a street corner. This comes in handy with listings outside the United States, too, where
roaming charges are steep.

``If anything, what the adoption of apps has shown is that people are clamoring for a simple, more
curated, contained experience,'' said Rafat Ali, who started Paidcontent.org to write about emerging
media business models and sold his company in 2008. ``Zagat kind of lost the Web game, but they do
have a chance of reinvention on mobile.''

So how many people might be willing to pay for Zagat advice? If all the book revenue were to go away
-- and it has not, for the New York survey still lingers at the bottom of the New York-area
nonfiction best-seller list, according to Nielsen BookScan -- the company would need a couple of
million people paying \$24.95 a year for Web access or \$9.95 annually for its app plus updates to
keep revenue at its current level. The company currently gets just 30 percent of its revenue from
print books.

And for publishers, making money from apps is still as much of an art as it is a science. ``What
kind of information will people pay for, given the availability of pretty-high-quality free content?
What is the unique mix of features, content and experience that need to come together?'' asks
Mr.~Sterling, the analyst. ``That's where the mystery and the art lie.''

\section{Smartphone Sales Taking Toll on G.P.S. Devices}

\lettrine{T}{he}\mycalendar{Nov.'10}{15} auto navigation device, a fixture of dashboards around the
world for the past seven years, may soon begin to disappear, industry experts say, as
satellite-tracking technology is absorbed into smartphones and automobiles.

Berg Insight, a Swedish research company that tracks the navigation industry, estimates that the
number of personal navigation devices shipped globally will peak in 2011 at 42 million, up from 40
million this year, before beginning a gradual, but inexorable decline.

Berg sees annual shipments of the devices, which fix location by communicating with global
positioning satellites, falling to 30 million in 2015.

``The reason this is happening is basically the smartphone,'' said Tom Slob, an analyst in Amsterdam
for Rabobank, a Dutch bank. ``Google and Nokia have both decided to give the technology away for
free. That has changed the business.''

Whether competition from the mobile phone industry proves to be a fatal blow to the auto navigation
market, however, remains to be seen. The industry leaders, Garmin and TomTom, are fighting back,
honing real-time traffic technologies and building in-dash navigators for automakers like Renault,
Mazda and Fiat.

``The technology is being partially absorbed into smartphones and in-dash devices,'' said Kevin
Rauckman, the Garmin chief financial officer. ``We are also seeing the industry mature very rapidly.
But that in no way means the end of the road for our business.''

Regardless of how companies like Garmin ultimately fare, the uncertainty surrounding the personal
navigation industry, whose devices began appearing on dashboards in 2003, is another example of the
way smartphones are disrupting many consumer electronics businesses.

Apple has seen sales of its iPhone and iPad, both popular Internet-connected devices, cannibalize
those of its iPod music and video players. IPod sales peaked in 2008, one year after Apple
introduced the iPhone, at 55.4 million units, according to figures released by Apple, and fell to
52.4 million last year.

Global shipments of digital cameras, another device now commonly integrated into smartphones, fell
to 125 million units in 2009 from 140 million in 2008, according to Andr\'e Malm, a senior analyst
at Berg Insight in Goteborg, Sweden.

Mr.~Malm said most of the decline probably had been caused by the global economic downturn but
smartphones, some of which now come with 12-megapixel cameras, were definitely eroding sales of
lower-end digital cameras. Nokia, which sells about 106 million smartphones a year, now claims to be
the world's largest maker of digital cameras.

``There is no doubt that the smartphone is transforming many of these markets, not just navigation
devices, but cameras and media players, too,'' Mr.~Malm said. ``These markets aren't going to
disappear, but they are going to change substantially.''

The auto navigation business was humming along until October 2008, when Google introduced Android,
an operating system for mobile devices that extended Google Maps and its navigation programs to
cellphones. Since then, 45 million people have bought GPS-enabled Android phones, according to
Gartner, and many use them in their cars.

Not to be outdone, Nokia, which sells more mobile phones than anyone, began giving away its own
navigation software, Ovi Maps, last January to its smartphone users. Nokia sells about 290,000
smartphones each day around the world, 26.5 million in the third quarter alone.

The fallout from Google's and Nokia's decisions has roiled the industry, which is dominated by
Garmin, based in Switzerland, the market leader; TomTom, based in Amsterdam; and Mitac International
of Taiwan, owner of the Mio, Navigon, Magellan and Navman brands. The three make up 68 percent of
the market, according to Berg Insight.

Sales at Garmin, which has 35 percent of the global market, fell 11.4 percent in the third quarter
to \$692.4 million. The company, thanks to \$68.6 million in a one-time tax benefit, reported a 30
percent increase in net profit during the quarter, to \$279.6 million.

Mr.~Rauckman, the Garmin chief financial officer, attributed the downturns in part to competition
from smartphones and in-dash devices, but also to the rising availability of GPS navigators in cars.
He estimated they were installed in 25 percent of the world's 700 million cars.

At TomTom, the European market leader, sales rose 3 percent, to \texteuro375 million, or \$512.4
million, in the third quarter, but the company's net profit slumped 37 percent, to \texteuro19
million from \texteuro31 million a year earlier. On Oct.~20, TomTom said it expected 2010 sales and
earnings to be unchanged from 2009.

Neither company is ceding the market to Google or Nokia.

TomTom is using anonymous location data from Vodafone cellphone customers in Europe to compile
enhanced traffic reports for a dynamic traffic management service it sells in Europe. Garmin has
diversified beyond car navigators, generating a third of sales from aviation, maritime and
fitness/outdoor devices. Garmin, TomTom and Mitac also sell navigation tools as applications for the
iPhone and BlackBerry.

Mr.~Rauckman said smartphones could not compete with specialized navigators like Garmin's Nuvi
3790LMT, which offers speech recognition and 3-D street displays.

``The user interface on our devices is much easier to use than a smartphone,'' he said. ``That is
why people are willing to pay for our products.''

At a presentation to financial analysts Thursday in Amsterdam, the TomTom chief executive, Harold
Goddijn, told investors his company was expanding its relationship with Renault, Fiat and Mazda,
said Nicolas von Stackelberg, an analyst at Macquarie Securities in Frankfurt.

``Before, the outlook was rather uncertain, but it is brightening now,'' Mr.~von Stackelberg said.
``The fear had been that nobody would need to buy a personal navigation device any more. And yes,
there are some who will use their smartphones instead, but there are others who will still want to
buy a high-end device from specialists like TomTom or do away with dangling cables and go for a
reasonably priced in-dash solution.''

Tim Flight, editor of GPSReview.net, a Web site covering the industry, said it was far too early to
write the obituary for auto navigation devices.

``A lot of GPS users are now buying their second device,'' said Mr.~Flight, who is based in
Carrabassett Valley, Maine. ``I'm sure the casual user might be satisfied with a smartphone, but
people who rely on the devices are still going to opt for a real navigator.''

\section{Animation Industry Finds a Home in Singapore}

\lettrine{M}{ade}\mycalendar{Nov.'10}{15} in Singapore'' is becoming a more common tagline for
computer-generated animation.

While this industry in Singapore is less than 10 years old and still very much in its infancy, it
has turned an important corner in the past couple years, executives and analysts say.

Christopher Chia, who until two weeks ago was chief executive of the Media Development Authority in
Singapore, said he saw local production moving up the value chain.

``In the early years, our industry could be characterized as a fee-for-service industry,'' said
Mr.~Chia, who is now a senior adviser to the Ministry of Information, Communications and the Arts.
``Somebody out there would commission a part of a show and companies here would execute.''

``But increasingly, our companies are either co-owning or owning the intellectual properties,'' he
said. ``At the same time, they're working with the bigger boys around the world to co-produce or
co-market the content.''

That is the case with Dream Defenders, developed and produced by Tiny Island Productions of
Singapore and introduced in October at the Mipcom trade show in Cannes. When completed, the
26-episode show will be the first three-dimensional series from Singapore.

Another local studio, Sparky Animation, signed a contract with the Jim Henson Co., creator of the
Muppets, to co-produce a second season of the award-winning children's series Dinosaur Train, which
is animated and computer-generated. It also participated in the first series.

Meanwhile, August Media Holdings, which is based in Singapore, recently signed a \$60 million deal
with the U.S.~media company Classic Media to develop and jointly produce 10 new shows for
television. They will be based on children's classics from Classic Media's catalog, which includes
``George of the Jungle'' and ``Mister Magoo.''

August Media was founded in March by the industry veteran Jyotirmoy Saha, who used to be co-head of
Sparky Animations. Three months ago, the company acquired the Scottish children's content producer
Red Kite Animations, founded in 1997 by Ken Anderson. Red Kite is behind The Secret Life of Benjamin
Bear, Dennis and Gnasher, 64 Zoo Lane, and The Imp.

``The real business is in developing and owning intellectual property, with media products that can
move across different platforms,'' Mr.~Saha said.

``As a company we're trying to build this business in a complete 360-degree manner,'' he added. ``We
acquired Red Kite because we wanted to move quickly and that was a good way to jump-start our kids'
content business.''

Mr.~Saha said the company was in negotiations to acquire a couple of distribution companies in
Europe that would help August Media Holdings ``control the monetization of our own content and that
of our partners.''

August Media will be one of the first tenants of Mediapolis, a 19-hectare, or 47-acre,
high-technology media park, when the first building opens there in December. The park will take as
long as 10 years to be fully developed but is envisaged as a self-contained media ecosystem, with
soundstages, digital production and broadcast facilities, and media schools.

So far, the Media Development Authority has remained coy about announcing the tenants of the first
building, which will cater primarily to the incubation of start-ups and creating prototypes of
innovations in various media sectors, including interactive digital media, film and broadcasting.

Many of the changes in the Singaporean animation landscape can be attributed to the arrival of
Lucasfilm, which opened the doors of its digital studio in Singapore in 2005. The presence of the
big Hollywood name acted like a magnet, attracting more companies and creating employment
opportunities.

``When Lucasfilm opened a studio in Singapore, Classic Media identified it as a talent center that
demanded close scrutiny from us,'' said Doug Schwalbe, executive vice president of production and
program sales for Classic Media.

With a staff of 400, Lucasfilm has outgrown its current facility. It recently announced it would
start building a state-of-the-art facility in Fusionopolis, a business park close to Mediapolis,
with the goal of moving in to the facility in 2012.

Mr.~Chia is encouraged that companies like August Media are starting to provide financing for the
industry. Over the past eight years, the Media Development Authority has played a main role in
co-producing or co-financing many of the local projects, as well giving grants of as much as
\$112,000 for the production of original animation pilots by Singapore-based animation studios.

``We did a lot of hand-holding and nurturing at the beginning, helping take companies to the
market,'' Mr.~Chia said.

``We definitely want more companies like August Media locating in Singapore and taking the private
sector's rightful place in developing content,'' he added.

To date, the authority has co-financed more than 30 animation projects, but with new entrants coming
to the Singapore market, more nurturing is needed. Mr Chia acknowledged that most of the success of
the local industry had been in television and that it still needed a big success with an animation
feature film for others in the industry to really take notice.

Several such films are in development. Sparky Animation is working on The Brothers Grimpley, which
is being co-produced with Grimpley Films in Britain and PorchLight Entertainment in the United
States.

Tiny Island Productions has announced plans to co-produce a 3-D, computer-generated animation film
based on the television series Shelldon, about the underwater adventures of a seashell, scheduled
for release in 2012. Its partner is the entertainment production company Shellhut Entertainment,
based in Thailand.

David Kwok, managing director of Tiny Island Productions, said he believed the local industry had
made tremendous progress in the past few years, and disagreed with critics who say that Singaporean
studios continue to perform lower-end jobs.

``We're not a low-cost country, like China or India, where you can tap on talent to do very manual,
low-end jobs,'' he said. ``In fact, a lot of the talent here are doing high-end work, the same type
of work you would do in Los Angeles.''

Mr.~Kwok conceded that in terms of preproduction work like script writing, ``we're not quite there
yet.''

His company is using writers based in Los Angeles for the Dream Defenders series, which depicts the
twins Zane and Zoey as the last line of defense between the real world and the nightmare creatures
of the Dreamworlds.

``But there is no reason why we cannot do it one day,'' Mr.~Kwok said. ``I think it just takes
time.''

\section{Should You Be Snuggling With Your Cellphone?}

\lettrine{W}{arning}\mycalendar{Nov.'10}{15}: Holding a cellphone against your ear may be hazardous
to your health. So may stuffing it in a pocket against your body.

I'm paraphrasing here. But the legal departments of cellphone manufacturers slip a warning about
holding the phone against your head or body into the fine print of the little slip that you toss
aside when unpacking your phone. Apple, for example, doesn't want iPhones to come closer than 5/8 of
an inch; Research In Motion, BlackBerry's manufacturer, is still more cautious: keep a distance of
about an inch.

The warnings may be missed by an awful lot of customers. The United States has 292 million wireless
numbers in use, approaching one for every adult and child, according to C.T.I.A.-The Wireless
Association, the cellphone industry's primary trade group. It says that as of June, about a quarter
of domestic households were wireless-only.

If health issues arise from ordinary use of this hardware, it would affect not just many customers
but also a huge industry. Our voice calls -- we chat on our cellphones 2.26 trillion minutes
annually, according to the C.T.I.A. -- generate \$109 billion for the wireless carriers.

The cellphone instructions-cum-warnings were brought to my attention by Devra Davis, an
epidemiologist who has worked for the University of Pittsburgh and has published a book about
cellphone radiation, ``Disconnect.'' I had assumed that radiation specialists had long ago
established that worries about low-energy radiation were unfounded. Her book, however, surveys the
scientific investigations and concludes that the question is not yet settled.

Brain cancer is a concern that Ms.~Davis takes up. Over all, there has not been a general increase
in its incidence since cellphones arrived. But the average masks an increase in brain cancer in the
20-to-29 age group and a drop for the older population.

``Most cancers have multiple causes,'' she says, but she points to laboratory research that suggests
mechanisms by which low-energy radiation could damage cells in ways that could possibly lead to
cancer.

Children are more vulnerable to radiation than adults, Ms.~Davis and other scientists point out.
Radiation that penetrates only two inches into the brain of an adult will reach much deeper into the
brains of children because their skulls are thinner and their brains contain more absorptive fluid.
No field studies have been completed to date on cellphone radiation and children, she says.

Henry Lai, a research professor in the bioengineering department at the University of Washington,
began laboratory radiation studies in 1980 and found that rats exposed to radiofrequency radiation
had damaged brain DNA. He maintains a database that holds 400 scientific papers on possible
biological effects of radiation from wireless communication. He found that 28 percent of studies
with cellphone industry funding showed some sort of effect, while 67 percent of studies without such
funding did so. ``That's not trivial,'' he said.

The unit of measurement for radiofrequency exposure is called the specific absorption rate, or SAR.
The Federal Communications Commission mandates that the SAR produced by phones be no more than 1.6
watts per kilogram. One study listed by Mr.~Lai found effects like loss of memory in rats exposed to
SAR values in the range of 0.0006 to 0.06 watts per kilogram. ``I did not expect to see effects at
low levels,'' he said.

The city of San Francisco passed an ordinance this year that requires cellphone retailers to post
SARs prominently. This angered the C.T.I.A., which announced that it would no longer schedule trade
shows in the city.

The association maintains that all F.C.C.-approved phones are perfectly safe. John Walls, the
association's vice president for public affairs, said: ``What science tells us is, `If the sign on
the highway says safe clearance is 12 feet,' it doesn't matter if your vehicle is 4 feet, 6 feet or
10 feet tall; you're going to pass through safely. The same theory applies to SAR values and
wireless devices.''

The association has set up a separate Web site, cellphonehealthfacts.com. Four attractive young
people are seen on the home page, each with a cellphone pressed against the ear -- and all four are
beaming as they listen. By this visual evidence, cellphone use seems to be correlated with elation,
not cancer.

The largest study of cellphone use and brain cancer has been the Interphone International
Case-Control Study, in which researchers in 13 developed countries (but not the United States)
participated. It interviewed brain cancer patients, 30 to 59 years old, from 2000 to 2004, then
cobbled together a control group of people who had not regularly used a cellphone.

The study concluded that using a cellphone seemed to decrease the risk of brain tumors, which the
authors acknowledged was ``implausible'' and a product of the study's methodological shortcomings.

The authors included some disturbing data in an appendix available only online. These showed that
subjects who used a cellphone 10 or more years doubled the risk of developing brain gliomas, a type
of tumor.

The 737 minutes that we talk on cellphones monthly, on average, according to the C.T.I.A., makes
today's typical user indistinguishable from the heavy user of 10 years ago. Ms.~Davis recommends
keeping a phone out of close proximity to the head or body, by using wired headsets or the phone's
speaker. Children should text rather than call, she said, and pregnant women should keep phones away
from the abdomen.

The F.C.C. concurs about the best way to avoid exposure. It is not by choosing a phone with a
marginally lower SAR, it says, but rather by holding the cellphone ``away from the head or body.''

It's advice that I find hard to put into practice myself. The comforting sight of everyone around me
with phones pressed against their ears, just like me, makes the risk seem abstract.

But Ms.~Davis, citing unsettling findings from research in Israel, France, Sweden and Finland, said,
``I do think I'm looking at an epidemic in slow motion.''

\section{For Cats, a Big Gulp With a Touch of the Tongue}

\lettrine{I}{t}\mycalendar{Nov.'10}{15} has taken four highly qualified engineers and a bunch of
integral equations to figure it out, but we now know how cats drink. The answer is: very elegantly,
and not at all the way you might suppose.

Cats lap water so fast that the human eye cannot follow what is happening, which is why the trick
had apparently escaped attention until now. With the use of high-speed photography, the neatness of
the feline solution has been captured.

The act of drinking may seem like no big deal for anyone who can fully close his mouth to create
suction, as people can. But the various species that cannot do so -- and that includes most adult
carnivores -- must resort to some other mechanism.

Dog owners are familiar with the unseemly lapping noises that ensue when their thirsty pet meets a
bowl of water. The dog is thrusting its tongue into the water, forming a crude cup with it and
hauling the liquid back into the muzzle.

Cats, both big and little, are so much classier, according to new research by Pedro M.~Reis and
Roman Stocker of the Massachusetts Institute of Technology, joined by Sunghwan Jung of the Virginia
Polytechnic Institute and Jeffrey M.~Aristoff of Princeton.

Writing in the Thursday issue of Science, the four engineers report that the cat's lapping method
depends on its instinctive ability to calculate the point at which gravitational force would
overcome inertia and cause the water to fall.

What happens is that the cat darts its tongue, curving the upper side downward so that the tip
lightly touches the surface of the water.

The tongue is then pulled upward at high speed, drawing a column of water behind it.

Just at the moment that gravity finally overcomes the rush of the water and starts to pull the
column down -- snap! The cat's jaws have closed over the jet of water and swallowed it.

The cat laps four times a second -- too fast for the human eye to see anything but a blur -- and its
tongue moves at a speed of one meter per second.

Being engineers, the cat-lapping team next tested its findings with a machine that mimicked a cat's
tongue, using a glass disk at the end of a piston to serve as the tip. After calculating things like
the Froude number and the aspect ratio, they were able to figure out how fast a cat should lap to
get the greatest amount of water into its mouth. The cats, it turns out, were way ahead of them --
they lap at just that speed.

To the scientific mind, the next obvious question is whether bigger cats should lap at different
speeds.

The engineers worked out a formula: the lapping frequency should be the weight of the cat species,
raised to the power of minus one-sixth and multiplied by 4.6. They then made friends with a curator
at Zoo New England, the nonprofit group that operates the Franklin Park Zoo in Boston and the Stone
Zoo in Stoneham, Mass., who let them videotape his big cats. Lions, leopards, jaguars and ocelots
turned out to lap at the speeds predicted by the engineers.

The animal who inspired this exercise of the engineer's art is a black cat named Cutta Cutta, who
belongs to Dr.~Stocker and his family. Cutta Cutta's name comes from the word for ``many stars'' in
Jawoyn, a language of the Australian aborigines.

Dr.~Stocker's day job at M.I.T. is applying physics to biological problems, like how plankton move
in the ocean. ``Three and a half years ago, I was watching Cutta Cutta lap over breakfast,''
Dr.~Stocker said. Naturally, he wondered what hydrodynamic problems the cat might be solving. He
consulted Dr.~Reis, an expert in fluid mechanics, and the study was under way.

At first, Dr.~Stocker and his colleagues assumed that the raspy hairs on a cat's tongue, so useful
for grooming, must also be involved in drawing water into its mouth. But the tip of the tongue,
which is smooth, turned out to be all that was needed.

The project required no financing. The robot that mimicked the cat's tongue was built for an
experiment on the International Space Station, and the engineers simply borrowed it from a
neighboring lab.

\section{Cigarette Giants in Global Fight on Tighter Rules}

\lettrine{A}{s}\mycalendar{Nov.'10}{15} sales to developing nations become ever more important to
giant tobacco companies, they are stepping up efforts around the world to fight tough restrictions
on the marketing of cigarettes.

Companies like Philip Morris International and British American Tobacco are contesting limits on ads
in Britain, bigger health warnings in South America and higher cigarette taxes in the Philippines
and Mexico. They are also spending billions on lobbying and marketing campaigns in Africa and Asia,
and in one case provided undisclosed financing for TV commercials in Australia.

The industry has ramped up its efforts in advance of a gathering in Uruguay this week of public
health officials from 171 nations, who plan to shape guidelines to enforce a global anti-smoking
treaty.

This year, Philip Morris International sued the government of Uruguay, saying its tobacco
regulations were excessive. World Health Organization officials say the suit represents an effort by
the industry to intimidate the country, as well as other nations attending the conference, that are
considering strict marketing requirements for tobacco.

Uruguay's groundbreaking law mandates that health warnings cover 80 percent of cigarette packages.
It also limits each brand, like Marlboro, to one package design, so that alternate designs don't
mislead smokers into believing the products inside are less harmful.

The lawsuit against Uruguay, filed at a World Bank affiliate in Washington, seeks unspecified
damages for lost profits.

``They're using litigation to threaten low- and middle-income countries,'' says Dr.~Douglas
Bettcher, head of the W.H.O.'s Tobacco Free Initiative. Uruguay's gross domestic product is half the
size of the company's \$66 billion in annual sales.

Peter Nixon, a vice president and spokesman for Philip Morris International, said the company was
complying with every nation's marketing laws while selling a lawful product for adult consumers.

He said the company's lawsuits were intended to combat what it felt were ``excessive'' regulations,
and to protect its trademark and commercial property rights.

Cigarette companies are aggressively recruiting new customers in developing nations, Dr.~Bettcher
said, to replace those who are quitting or dying in the United States and Europe, where smoking
rates have fallen precipitously. Worldwide cigarette sales are rising 2 percent a year.

But the number of countries adopting tougher rules, as well as the global treaty, underscore the
breadth of the battleground between tobacco and public health interests in legal and political
arenas from Latin America to Africa to Asia.

The cigarette companies work together to fight some strict policies and go their separate ways on
others. For instance, Philip Morris USA, a division of Altria Group, helped negotiate and supported
the anti-smoking legislation passed by Congress last year and did not join a lawsuit filed by
R.~J.~Reynolds, Lorillard and other tobacco companies against the Food and Drug Administration. So
far, it is not protesting the agency's new rules, proposed last week, requiring graphic images with
health warnings on cigarette packs.

But Philip Morris International, the separate company spun out of Altria in 2008 to expand the
company's presence in foreign markets, has been especially aggressive in fighting new restrictions
overseas.

It has not only sued Uruguay, but also Brazil, arguing that images the government wants to put on
cigarette packages do not accurately depict the health effects of smoking and ``vilify'' tobacco
companies. The pictures depict more grotesque health effects than the smaller labels recommended in
the United States, including one showing a fetus with the warning that smoking can cause spontaneous
abortion.

In Ireland and Norway, Philip Morris subsidiaries are suing over prohibitions on store displays.

In Australia, where the government announced a plan that would require cigarettes to be in plain
brown or white packaging to make them less attractive to buyers, a Philip Morris official directed
an opposition media campaign during the federal elections last summer, according to documents
obtained by an Australian television program, and later obtained by The New York Times.

The \$5 million campaign, purporting to come from small store owners, was also partly financed by
British American and Imperial Tobacco. The Philip Morris official approved strategies, budgets, ad
buys and media interviews, according to the documents.

Mr.~Nixon, the spokesman, said Philip Morris made no secret of its financing of that effort. ``We
have helped them, not controlled them,'' he said.

Mr.~Nixon said Philip Morris agreed that smoking was harmful and supported ``reasonable''
regulations where none exist.

``The packages definitely need health warnings, but they've got to be a reasonable size,'' he said.
``We thought 50 percent was reasonable. Once you take it up to 80 percent, there's no space for
trademarks to be shown. We thought that was going too far.''

These days in courts around the world, the tobacco giants find themselves on the defensive far more
than playing offense. The W.H.O. and its treaty encourage governments and individuals to take legal
action against cigarette corporations, which have encountered growing numbers of lawsuits from
smokers and health care systems in Brazil, Canada, Israel, Italy, Nigeria, Poland and Turkey.

But in other parts of the world, notably Indonesia, the fifth-largest cigarette market, which has
little regulation, tobacco companies market their products in ways that are prohibited elsewhere. In
Indonesia, cigarette ads run on TV and before movies; billboards dot the highways; companies appeal
to children through concerts and sports events; cartoon characters adorn packages; and stores sell
to children.

Officials in Indonesia say they depend on tobacco jobs, as well as revenue from excise taxes on
cigarettes. Indonesia gets some \$2.5 billion a year from Philip Morris International alone.

``In the U.S., they took down billboards, agreed not to sponsor music events, no longer use the
Marlboro cowboy,'' said Matthew L.~Myers, president of the Washington-based Campaign for
Tobacco-Free Kids. ``They now do all of those things overseas.''

The world's second-biggest private cigarette maker, British American Tobacco, with \$4.4 billion
profits on \$23 billion sales in the year ending June 30, is spending millions of dollars lobbying
against anti-smoking health measures, like smoke-free air policies in the European Union.

A video on the company's Web site says some of the proven methods of reducing smoking -- like taxes
and display bans -- encourage a black market in cigarettes and that, in turn, would finance drug,
sex and weapons traffickers and terrorists.

The six-minute video, in which actors play gangsters, one with an Eastern European accent,
concludes, ``Only the criminals benefit.''

The conference beginning on Monday in Punta del Este, Uruguay, will try to add specific terms to a
public health treaty known as the Framework Convention on Tobacco Control, which since 2003 has been
ratified by 171 nations. It would eventually oblige its parties to impose tighter controls on
tobacco ingredients, packaging and marketing, expand cessation programs and smoke-free spaces and
raise taxes -- proven tactics against smoking.

President George W.~Bush signed the treaty in 2004 but did not send it to the Senate, where a
two-thirds vote is needed for ratification. President Obama hopes to submit it to the Senate next
year, a White House spokesman said on Thursday.

One recommendation drawing fire from tobacco farmers would either restrict or prohibit the use of
popular additives, like licorice and chocolate, to blended tobacco products that account for more
than half of worldwide sales.

The International Tobacco Growers' Association says that could threaten the makers of burley
tobacco, an air-cured leaf that has long been sweetened with additives, costing millions of farmers
their jobs and devastating economies worldwide.

``We all know the real objective here is to eliminate tobacco consumption,'' says Roger Quarles, a
Kentucky grower and president of the association.

\section{Protecting Yourself From the Cost of Type 2 Diabetes}

\lettrine{O}{ne}\mycalendar{Nov.'10}{15} in 10 Americans has diabetes, and if present trends
continue, one in three will suffer from the disease by the year 2050, according to the Centers for
Disease Control and Prevention. Already this incurable, chronic and often debilitating illness costs
the country's health care system a staggering \$174 billion a year.

``Diabetes is the noninfectious epidemic of our time,'' said Dr.~Ronald Loeppke, vice chairman of
U.S.~Preventive Medicine, a company that offers wellness and prevention programs to employers and
individuals.

What often gets lost in the talk over rising costs, however, is just how much treating diabetes can
cost an individual patient. Even with insurance, people with Type 2 diabetes, the most common form
of the illness, often face substantial out-of-pocket expenses.

The inability to shoulder them is part of the reason only 25 percent of diabetics are getting the
care they need, many experts say. And those who do manage to obtain proper care often have to make
steep sacrifices.

Karen Christian, a 74-year-old retired Red Cross worker, sold her house near Monterey Bay in
California to move in with her daughter in Vail, Ariz., a small town close to Tucson. Told 10 years
ago that she had Type 2 diabetes, she depleted her savings on co-pays for doctor visits and the
supplies and medicines that Medicare didn't cover.

In California, she did not qualify for Medicaid or other government assistance that could help fill
the gap. ``I had enough income to get by, but not enough to manage a chronic illness,''
Ms.~Christian said. In Arizona, Medicaid does cover of most of what Medicare will not for her
treatment.

Happily, Ms.~Christian's diabetes is under control, and she is in good health. But four years later,
she's still adjusting to the move.

``I used to live half a mile away from the coast,'' she said. ``I miss the cool breezes, the fog and
my garden. I'm still getting used to the heat out here. But the trade-off is good health, and that's
worth it.''

Diabetes patients spend an average of \$6,000 annually for treatment of their disease, according to
a recent report by Consumer Reports Health. That figure includes monitoring supplies, medicines,
doctor visits, annual eye exams and other routine costs.

But the total doesn't include the costs of medical complications that often result from Type 2
diabetes, like heart disease, strokes, liver and kidney damage, eye damage and a susceptibility to
infections and poor healing that can lead to amputations. The C.D.C. estimates that diabetic
patients on average pay twice as much as those without the illness for health care.

What steps can diabetes patients and their loved ones take to bring down the cost of treating
diabetes and still receive top-quality care? Here is some money-saving advice for anyone suffering
from the disease or worried about getting it.

TRY THE OLDER DRUGS FIRST Most Type 2 diabetics are prescribed a medicine or combination of
medicines intended to help the body produce less glucose or more insulin, or to increase sensitivity
to the hormone.

While several new and expensive versions of these drugs have come on the market in recent years, a
study done by Consumer Reports Health in February 2009 found that the older, less expensive and
generic versions of these drugs are just as effective as the new drugs. And they have established
safety records, while some newer diabetes drugs, notably Avandia, have been found to increase
cardiovascular and other health risks.

A 500-milligram dose of metformin, for instance, a first-line medication available generically,
costs on average only \$18 a month, according to the report, and can be found for even less at
stores like Wal-Mart and Target, which are offering big generic drug discounts. Actos, a newer drug
with a different method of action, costs on average \$241 for a month's supply of the 30-milligram
pills.

``In the case of diabetes, newer drugs are not necessarily better,'' said Dr.~Marvin Lipman, chief
medical adviser for Consumer Reports Health and a practicing endocrinologist in Westchester County.
``The expensive drugs are third- and fourth-line drugs. If you don't get results with the less
expensive drugs, you go to those. But you shouldn't start there. The vast majority of cases can be
treated with the less expensive drugs.''

SAVE ON SUPPLIES Type 2 diabetics must monitor their blood sugar levels regularly, sometimes even
many times a day, using a home monitor, lancets and testing strips. The monitor's price can be
fairly reasonable, \$10 to \$80 depending on the model. The strips, however, usually cost 60 to 80
cents each. It adds up, often costing patients hundreds of dollars a year.

The best way to reduce the cost of supplies is to keep your blood sugar levels under control so that
you have to test less often, advised Dr.~Lipman. ``If you can get your testing down to once a day or
even three times a week, you can save money that way,'' he said.

Dr.~Lipman also suggested using lancets more than once to save costs. ``If you keep a lancet sterile
and put the cover back on, you can use it two or three times before it becomes too dull,'' he said.

GET SCREENED ANNUALLY Not all physicians automatically include a blood sugar screening in your
physical, said Dr.~Loeppke. Be sure to ask your doctor if this test is included. Early detection of
diabetes, or even prediabetic conditions, can make a huge difference in treatment, ultimately
preventing the need for medication and saving money.

Ms.~Christian didn't have the slightest idea she had diabetes until the day she volunteered at a
local health fair the Red Cross was sponsoring. ``During a break I decided to get a blood sugar
screening just for the heck of it, and next thing I knew I was in the doctor's office. My blood
sugar was very high.''

If you have any of the risk factors for diabetes -- you are Hispanic, Indian or African-American;
you have a family history of the illness; you are overweight or obese; or you had diabetes during
pregnancy -- you should be screened at least once a year, said Dr.~Loeppke.

See a doctor immediately if you experience any symptoms of the disease, like unusual thirstiness,
frequent urination, extreme bouts of hunger, frequent infections, cuts or bruises that are slow to
heal, or a tingling or numbness in your hands and feet.

ADOPT HEALTHIER HABITS The good news, said Dr.~Loeppke, is that 80 percent of Type 2 diabetes in the
United States can be prevented with three steps that do not have to cost money: stopping smoking,
eating a healthy diet and exercising regularly.

``Of course, that `says easy' and `does hard' for most people,'' Dr.~Loeppke acknowledged, ``but so
far, those are the best diabetes preventions out there.'' For more on preventing and treating
diabetes, check out the C.D.C.'s comprehensive section on the disease at cdc.gov/diabetes.

FIND SUPPORT Studies show that diabetics who actively participate in a support, monitoring or
wellness program are far more successful in maintaining normal blood sugar levels, thereby saving on
health bills. As a result, many employers and insurers offer chronic illness management programs
with access to round-the-clock nurses, nutrition and exercise advice, and other support systems.

Ask your company's benefits department or your insurance company if they offer these programs and
how you can enroll. For more information on diabetes and online support, visit the American Diabetes
Association Web site at diabetes.org.

\section{Under Pressure Over Bailout, Dublin Defends Its Finances}

\lettrine{E}{uropean}\mycalendar{Nov.'10}{15} ministers worked over the weekend on a financial
rescue plan for Ireland, as pressure mounted on Dublin to seek a bailout as the best means of
preventing the markets from spreading turbulence to other European countries, officials said on
Sunday.

The Irish government continued to insist that it did not need a bailout, arguing that it could
present a credible austerity budget next month that would satisfy investors, and that it had enough
money to finance its operations through early next year.

But analysts and investors, as well as some European officials, say the government's plan needs to
be buttressed by a promise of outside funding to counter the jumpiness in the markets, which have
pushed interest rates on Irish bonds to record highs.

``There is a risk of a self-fulfilling prophecy,'' a European diplomat said, speaking on the
condition of anonymity because of the sensitivity of the issue. ``Even a denial is seen as some sort
of affirmation that there is something to deny.''

The push to shore up Irish finances reflects the desire of some officials to get ahead of a problem
that could not only undermine Dublin's recovery efforts but also threaten other weak economies in
Europe like Portugal and Spain. Last spring, when Greece teetered on the brink of default, a series
of reassurances by the European Union failed to calm investor anxiety, and a huge bailout fund had
to be arranged at the last minute to stabilize Greece and relieve pressure on the euro.

On Friday, the storm in the markets was briefly calmed when five European ministers at the Group of
20 summit in Seoul issued supportive statements and Ireland's finance minister, Brian Lenihan,
reaffirmed the country's resolve. But European officials are concerned that more needs to be done
before Ireland presents its next budget, scheduled for Dec.~7.

The reaction of the bond markets on Monday is the next test for Ireland. Officials said they were
preparing a contingency plan in case the markets moved sharply against the country.

Preliminary talks on a rescue package had already taken place. Discussions involving European Union
ministers and senior officials continued on Sunday, said one official involved in the debate. But by
early Sunday evening, diplomats said, there were still no plans for any formal teleconference
between the finance ministers of the 16 countries that use the euro.

Officials in Brussels and Dublin said that Ireland had not made any formal application for a loan
and that without such an application, no bailout could be approved or executed.

According to a report by Barclays Capital, the European Union and the International Monetary Fund
would need to loan 80 billion to 85 billion euros, or \$109 billion to \$116 billion, to satisfy
Ireland's sovereign funding needs and to create an added buffer to help recapitalize its failed
banks.

The ``extreme tension that has been prevailing in the financial markets, especially concerning the
ability of Ireland to achieve a sustainable fiscal path by going it alone,'' meant that recourse to
European Union loans ``would, in our view, represent a sensible outturn,'' Julian Callow and Antonio
Garcia Pascual of Barclays Capital wrote in a research note on Friday evening.

Finance officials are scheduled to meet on Tuesday and Wednesday to discuss the Irish situation.

Any Irish bailout would be a delicate matter for Germany, which strongly resisted the bailout of
Greece and has been pushing to overhaul the current mechanism for European rescues to ensure that
private investors help foot the bill of any sovereign defaults. German officials, however, may be
hoping that Ireland accepts the bailout sooner rather than later to soothe jittery debt markets and
ensure stability of the euro and, in the longer term, the growth prospects of the European Union.

The first of any such payments to Ireland would come from a pot of money totaling 60 billion euros
that is guaranteed by the union's budget and set up to provide rapid assistance to countries in
``acute difficulties,'' one European official said. The official spoke on the condition of anonymity
because of the delicacy of the situation and would not speculate on how much money Ireland might
need over all.

Any additional loans would have to come from a much larger pool of money guaranteed by the euro zone
nations. That pool totals 440 billion euros, the official said. Reaching such an agreement on using
those funds might prove harder, as governments still are debating how a permanent loan system should
work. It also could take up to four weeks to draw up a support program using that pool and could
require more scrutiny from the International Monetary Fund.

The two pools of funding were set up in May after the bailout of Greece. But the official stressed
that Ireland was a very different case.

Whereas Greece's rescue came after years of concealing the true state of its finances, Ireland has a
much stronger track record in economic management. That means it was still possible that Ireland
could steady the markets by imposing tough austerity measures.

While Ireland must submit its budget by Dec.~7, it is rushing to prepare a four-year plan that will
show how it plans to cut its current deficit from 32 percent of gross domestic product to 3 percent
by 2014.

That strategy will include another round of spending cuts and is likely to spark further unrest from
a citizenry that is suffering from a third consecutive year of negative growth in the economy.

The Irish government is extremely reluctant to seek a bailout, analysts say, because of the stigma
associated with applying for aid and the risk to its political standing.

The Fianna Fail Party leads a shaky coalition that holds only a thin majority in Parliament and
could be forced to call early elections next year.

Dublin is also hoping to reassure markets that it can avoid a bailout by winning approval from the
European Union for its bailout of Anglo Irish Bank.

The European Commission, the executive arm of the union, still needs to determine whether the bank
bailout falls within European state aid rules. The commission is considering whether to approve a
6.4 billion euro portion of the bailout of Allied Irish and to agree to the overall restructuring
plan.


\end{document}
