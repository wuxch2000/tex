\documentclass[11pt,a4paper,onecolumn]{article}
\title{TITLE}
\author{}
\date{}

\usepackage{fontspec,xunicode,xltxtra}
% \setmainfont[Mapping=tex-text]{Times New Roman}
\setmainfont[Mapping=tex-text]{Arial}
\setsansfont[Mapping=tex-text]{Arial}
% \setmonofont[Mapping=tex-text]{Courier New}
\setmonofont[Mapping=tex-text]{Times New Roman}

\usepackage{xeCJK}
% \setCJKmainfont[ItalicFont={Adobe Kaiti Std}]{Adobe Song Std}
% \setCJKmainfont[ItalicFont={Adobe Kaiti Std}]{Adobe Kaiti Std}
\setCJKmainfont[ItalicFont={Adobe Kaiti Std}]{Adobe Heiti Std}
\setCJKsansfont{Adobe Heiti Std}
% \setCJKsansfont{Microsoft YaHei}
\setCJKmonofont{Adobe Heiti Std}
\punctstyle{banjiao}

\usepackage{calc}
\usepackage[]{geometry}
% \geometry{paperwidth=221mm,paperheight=148.5mm}
% \geometry{paperwidth=9.309in,paperheight=6.982in}
\geometry{paperwidth=7.2cm,paperheight=10.8cm}
% \geometry{twocolumn}
\geometry{left=5mm,right=5mm}
\geometry{top=5mm,bottom=5mm,foot=5mm}
% \geometry{columnsep=10mm}
\setlength{\emergencystretch}{3em}


\usepackage{indentfirst}

%生成PDF的链接
\usepackage{hyperref}
\hypersetup{
    % bookmarks=true,         % show bookmarks bar?
    bookmarksopen=true,
    pdfpagemode=UseNone,    % options: UseNode, UseThumbs, UseOutlines, FullScreen
    pdfstartview=FitB,
    pdfborder=1,
    pdfhighlight=/P,
    pdfauthor={wuxch},
    unicode=true,           % non-Latin characters in Acrobat’s bookmarks
    colorlinks,             % false: boxed links; true: colored links
    linkcolor=blue,         % color of internal links
    citecolor=blue,        % color of links to bibliography
    filecolor=magenta,      % color of file links
    urlcolor=cyan           % color of external links
}
\makeindex

\usepackage[dvips,dvipsnames,svgnames]{xcolor}
\definecolor{light-gray}{gray}{0.95}

\usepackage{graphicx}
\usepackage{wrapfig}
\usepackage{picinpar}

\renewcommand\contentsname{目录}
\renewcommand\listfigurename{插图}
\renewcommand\listtablename{表格}
\renewcommand\indexname{索引}
\renewcommand\figurename{图}
\renewcommand\tablename{表}

\usepackage{caption}
\renewcommand{\captionfont}{\scriptsize \sffamily}
\setlength{\abovecaptionskip}{0pt}
\setlength{\belowcaptionskip}{0pt}

\graphicspath{{fig/}}

\usepackage{fancyhdr}

% \usepackage{lastpage}
% \cfoot{\thepage\ of \pageref{LastPage}}

% 嵌入的代码显示
% \usepackage{listings}
% \lstset{language=C++, breaklines, extendedchars=false}
% \lstset{basicstyle=\ttfamily,
%         frame=single,
%         keywordstyle=\color{blue},
%         commentstyle=\color{SeaGreen},
%         stringstyle=\ttfamily,
%         showstringspaces=false,
%         tabsize=4,
%         backgroundcolor=\color{light-gray}}

\usepackage[sf]{titlesec}
\titleformat{\section}{\normalsize\sffamily\bf\color{blue}}{\textsection~\thesection}{.1em}{}
\titleformat{\subsection}{\normalsize\sffamily}{\thesubsection}{.1em}{}
\titlespacing*{\section}{0pt}{1ex}{1ex}
\titlespacing*{\subsection}{0pt}{0.2ex}{0.2ex}

\usepackage{fancyhdr}
\usepackage{lastpage}
\fancyhf{}
\lhead{}
\rhead{}
\chead{\scriptsize{\textsf{蜗居}}}
\cfoot{\scriptsize{\textsf{第 \thepage ~页,共 \pageref*{LastPage} 页}}}


% \usepackage{enumitem}
% \setitemize{label=$\bullet$,leftmargin=3em,noitemsep,topsep=0pt,parsep=0pt}
% \setenumerate{leftmargin=3em,noitemsep,topsep=0pt,parsep=0pt}

% \setlength{\parskip}{1.5ex plus 0.5ex minus 0.2ex}
\setlength{\parskip}{2.0ex plus 0.5ex minus 0.2ex}

% \setlength{\parindent}{5ex}
\setlength{\parindent}{0ex}

% \usepackage{setspace}
\linespread{1.25}

% 英文的破折号--不明显,使用自己画的线。
\newcommand{\myrule}{\hspace{0.5em}\rule[3pt]{1.6em}{0.3mm}\hspace{0.5em}}


\begin{document}

\thispagestyle{empty}
\begin{center}
\textsf{\Huge{\\[10ex]南京汉口路西延工程之我见} }
\end{center}

\begin{flushright}
    \large{——《交通工程基础》课程设计}
\end{flushright}



\begin{center}
\begin{minipage}{0.3\linewidth}
\vspace{12ex}
\begin{large}
作者:\underline{ \textit{李秀峰} }
\\[3ex]学号:\underline{ ~~~~~~~~~~ }
\end{large}
\end{minipage}
\end{center}



\pagebreak

\setcounter{page}{1}
\pagenumbering{roman}
\pagestyle{plain}
\tableofcontents

\listoffigures

\pagebreak
\setcounter{page}{1}
\pagenumbering{arabic}
\pagestyle{fancy}

\section{南京汉口路西延工程调查报告}

\subsection{汉口路西延工程的提出}

在南京市开发秦淮河以西的河西地区之后,河西被定位为生活区,老城区则定位为工作区,由于秦淮河的阻隔,河
西与老城区的跨河交通联系一直是城市交通规划中的焦点和热点。北京西路、广州路由此成为连接河西老城区的主
要通道,也是南京最拥堵的道路之一。

拓宽北京西路、广州路似乎是最直接的对策,但是面对巨大的拆迁量和沿线那些不可能拆得动的单位,专家们所能
想到的唯一的办法就是在北京西路和广州路之间重新寻找出路。

时间:

南京市早在2005年就酝酿提出汉口路西延,目的是为了缓解北京西路的交通拥堵状况。

2008年初,南京市建委正式发布《2008年南京市城市规划、建设和管理任务组织实施方案》,方案中明确表示,汉
口路西延工程将于当年开工。

2009年3月初,《南京市汉口路西延初步设计》通过专家组评审。

\subsection{汉口路西延工程的位置}

汉口路,和与其平行的北京西路,广州路沿线,为南京市内枢纽地段之一。汉口路北面的北京西路被称为“政府
路”,江苏省委、省政府、省政协和众多省级机关和南京军区以及各式招待所、宾馆密布道路两侧,北京西路向东
延伸的北京东路,则是南京市委、市政府的所在地。南侧广州路,则被称为“医院一条街”,江苏省人民医院、南京
脑科医院、南京肺病医院、南京市儿童医院等大医院集中在此。

\begin{figure}[hb]
\begin{center}
\vspace{-1ex}
\includegraphics[width=0.9\textwidth]{f1.jpg}
\caption{南京市汉口路西延工程的位置}
\label{f1}
\vspace{-3ex}
\end{center}
\end{figure}

北京西路和广州路之间,汉口路沿线,则是著名的文教区,其间云集南京大学、南京师范大学、河海大学、南京艺
术学院、江苏教育学院等高校,另有颐和路公馆区、傅抱石纪念馆等历史文化风貌区和文物保护单位,还有南京市
内最为著名的中小学和幼儿园。

\subsection{汉口路西延工程周边的交通状况}

目前,主城与河西之间虽有集庆门大街、水西门大街、汉中门大街、清凉门大街、草场门大街、定淮门大街等数条
通道,但这些通道已不能满足通行需求,清凉门大街、草场门大街的通行压力日益加大,早晚高峰时,北京西路、
广州路几乎天天堵车,为了缓解拥堵,酝酿多年的汉口路西延工程提上日程,并于去年进行了环评公示。

南京市建委副主任邹建平介绍说,汉口路西延是缓解南京交通拥堵的一大举措,从北京西路到广州路之间短短的
1.1公里距离上,还没有一条横跨秦淮河的交通通道。由此导致大量驶往河西的车流云集到北京西路、广州路、草场
门大桥,造成了上述路段出现严重的拥堵。为缓解上述城市主干道的交通压力,促进“一城三区”战略的实施,汉口
路西延建设对南京交通影响举足轻重。

\subsection{计划方案和造价}

按照原来的方案,路线东起中山路,以地面道路形式沿汉口路、汉口西路西进,在宁海路路口两侧分上下层以隧道
形式下穿西康路、河海大学、虎踞路、国防园、明城墙、石头城公园、秦淮河,过龙园东路后隧道分上下层在龙园
南路上相继接地,到达与江东北路的交叉点。工程全长约4.3公里。该工程还拟在南京大学校门口设置一处地下人行
过街通道,方便师生过街。

通过专家评审的最新设计方案,进行了较大调整,最显著的是“缩身”1公里。新方案中,汉口路西延工程西起河西龙
园南路与江东北路交叉口,向东先后穿越秦淮河、石头城公园、明城墙、国防园、虎踞路、河海大学,东止于汉口
路与上海路交叉口,全长3.3公里。南大校门口下面的地下通道在方案中也被取消,并且汉口西路通往南京大学方向
为单行线,禁止通行。这样一来,南大宿舍区与教学区将不会遭到分割,机动车也将大大减少,以保证师生出行的
安全。

按照推荐方案,汉口路西延采用双层双向四车道隧道,建成后按照“上层老城向河西,下层河西向老城”的原则通行。

而根据方案,隧道出入口终于“掀开盖头”。在河西,隧道入口位于龙园西路东侧不到90米处,出口位于龙园南路
上;在老城,隧道入口位于汉口西路南京师范大学幼儿园前,距离宁海路约110米,出口位于汉口西路上海路副食品
中心市场前,距离上海路约80米。

初步设计为隧道工程净高3米,为小型车辆专用车道,设计时速40公里。汉口路西延工程总投资20~28亿。

\subsection{工程的意义}

工程可以完善区域交通布局,缓解北京西路、广州路压力。工程完工后,老城车辆可经上海路、汉口路进入隧道,
在河西出隧道后可拐向龙园西路、龙园南路、江东北路或漓江路;河西车辆从龙园西路、龙园南路、江东北路或漓
江路进入隧道后,可在老城出隧道后拐向宁海路或上海路,而汉口路通往南京大学方向为单行线,禁止通行。此
外,交通部门还将对省公安厅门前扬州路、仙霞路进行疏通,车辆出隧道后,也可经这两条道路回到西康路。这
样,上海路、广州路、北京西路、西康路区域的交通布局将进一步完善。

市建委相关负责人说,目前北京西路和广州路之间长达1.1公里的区域,没有一条分流车辆的平行道路,这在城
市交通路线设计上是个“短腿”。每天早晚高峰时段,北京西路的草场门一带和广州路的清凉门一带交通拥堵严重,
增加一个新的交通通道,是解决目前该区域交通问题的必要途径。

市交管局相关人士也表示,这一地区的交通拥堵问题已无法通过“软件”管理解决,如广州路西口的信号灯通行
周期已放到最大,除了加快硬件建设,进行工程改造外,已无新手段可以缓堵。

\subsection{反对的理由}

据了解,南京市早在2005年就酝酿提出汉口路西延,目的是为了缓解北京西路的交通拥堵状况。但这条隧道引发了
巨大的争议,工程将对南京的几所高校实施部分拆迁,拆迁改造工程量和工程难度巨大。此外,西延工程隧道的东
出口,无论是搁置在西康路、宁海路还是上海路,都可能会让这些区域更拥堵。据悉,这项工程还曾有过“双层隧
道”的建议,但实际上如果出口道路狭窄,车流越多越麻烦。而且专家认为北京西路地区的拥堵是因为今年南京建设
地铁二号线和内环北线两大工程,同时占用了汉中路、新模范马路两条东西向主干道导致的,是不是阶段性拥堵还
有待观察。

并表示这项工程一旦开工,就可能会给宁海路、上海路、中山路、洪武路、太平北路等诸多交通要道带来交通压
力,会在未来的一年多时间内影响整个南京城区的交通。

\section{观点的阐述}

\subsection{交通拥堵}

\subsubsection{交通拥挤的原因}

汽车保有量和使用率的增加是导致交通堵塞的主要原因。由于汽车的方便,导致市区内车流日益升高,每逢高峰时
间,上班的、旅游的、购物的车流从四面八方涌入市中心。但汽车的一大缺点,就是十分浪费空间,但数量又不断
增加,导致现有道路无法负荷如此大的车流量,而造成堵塞的情形。

道路容量不足亦为造成交通拥堵的因素。现今如伦敦、罗马等许多历史悠久的都市,都是交通恶名昭彰的都市,原
因就是出在于道路容量不足,因为其市区内的道路原来大都是供马车行走的,但汽车的数量不断增加,而道路扩建
的速度又跟不上车流量增加的数度,使得市中心的道路拥挤不堪。

\subsubsection{交通拥挤带来的影响}

增加了通勤的时间,使得可用于工作(生产)的时间减少,而造成驾驶者及该区域经济上的损失。

导致驾驶者感到愤怒及烦躁,增加了他们的压力,而进一步损害其健康。

浪费燃料及污染:发动机在堵车时仍不断运转,持续消耗燃料,并且在堵塞的时候,车辆必须不断加速、减速和停
车,增加燃料的耗费,因此交通堵塞不仅浪费能源,也造成空气污染。

造成主城区的生活品质降低,而导致居民大量迁至郊区(即所谓的郊区化)。

难以应变紧急状态:当有紧急需要时,可能因为交通堵塞而难以达到目的。

随着城市经济的快速发展和人们生活水平的提高,南京市私人小汽车保有量增长迅速,由此带来交通量的增加。按
照发达国家汽车消费的一般统计学规律,当轿车的平均价格与人均GDP的比值达到2/3时,轿车进入家庭完成私人消
费的进程开始加速。由此看来,私人小汽车的数量会不断增加的,已经逐渐成为人们的必需品。这是正常的需求规
律,我们不能企图通过限制人们对小汽车的需求来实现交通的通畅。另一方面,从国家的产业政策来讲,汽车生产
作为我国第五大产业支柱,对促进国民经济起着重要作用。面对小汽车的迅猛增长,我们没有理由阻止人们对它的
拥有,但是可以从交通管理的角度引导人们更加合理的使用小汽车。

\subsection{修路对缓解城市交通压力的作用是有限的}

增加道路容量是解决交通堵塞最为基本的方法,包括拓宽或新建道路。因为当汽车使用率增高时,就需要有更多的
道路来容纳车流。不过此方法的弊端在于仅能增加道路面积,而无法根治交通堵塞,因为汽车数量并未随之减少。
甚至有时,新建道路等于是在无形之中,鼓励更多的驾驶者开车上路。

当城市交通发展到一定阶段,单纯的依靠修路是满足不了日益增长的交通需求的。现有的道路拥有量无法满足日益
增长的交通需求。为了解决这一供需矛盾,首先想到的办法是新建道路和改建道路,增加交通供给。但是,土地是
有限的,尤其是主城区内部可供新建道路使用的土地更加有限。所以,仅仅通过修路来满足交通需求是不够的。当
可以被利用的土地都用尽,这个方法甚至变成不可行的。

当然,这是一个逐步发展的过程。开始,人们总会很容易的找到空闲的土地和可以拆除的房屋等设施;随着城市建
设的不断进行和完善,新建道路的代价越来越大,所遇到的争议不断增加。这时,争论的焦点会集中在建还是不
建,建就意味着巨大的经济投入和对城市已有面貌的改变。考虑到城市发展的进程,这种改变往往是比较大的,甚
至具有一定的破坏性。

本例中的汉口路西延工程,拟经过石头城公园、国防园、明城墙、傅抱石故居、金陵女子大学旧址等民国保护建
筑,并串联河海大学、南师大等名校,途径宁海路、龙江地区等省委机关工作生活区,是南京主城区备受关注的城
建项目,因为它将给沿线带来很多并不受欢迎的改变,

而且,当斯定律(Downs Law)告诉我们:新建的道路设施会诱发新的交通量,而交通需求总是倾向于超过交通供给。
可见,西延道路带来的不仅仅是分担北京西路和广州路的交通负荷。交通修建道路并不是解决城市交通拥堵的唯一
办法,城市发展到一定阶段,必须选择其他途径来改善交通,缓解交通压力,满足交通需求。

\subsection{采取交通管理的手段解决交通问题}

交通管理的一个主要方向是交通需求管理,就是对交通的产生层面的管理。在本例中,汉口路西延工程的提出背景
是河西生活区的建成,导致和主城区之间的交通量增加,引起汉口路北面的北京西路和南面的广州路交通压力大,
拥堵严重。实际上,可以采取措施,限制进入主城区的车辆。比如,以及调高私人汽车的驾驶成本,使用经济杠
杆,在草场门和清凉门设立进入主城区的费用,或增加主城区停车的费用,以此来减少河西和主城之间的交通需求。
很多发达国家已经采取这个措施,既可以缓解城市内部的交通压力,又利于降低车辆排放污染,保护主城区的环境。

用收费以抵制驾驶者的开车意愿,而减少汽车量的方法,称为“拥堵费”,目前新加坡及伦敦等都市已开始实施,纽
约市亦计划采取。

\begin{wrapfigure}{r}{0.5\textwidth}
\vspace{-2ex}
\includegraphics[width=0.5\textwidth]{f2.jpg}
\caption{伦敦市拥堵收费示意图}
\label{f2}
\vspace{-2ex}
\end{wrapfigure}

伦敦政府发出交通白皮书,公告市民:为了限制轿车数量,减少堵车和空气污染,在2000年起提高停车费用,城市
内原有的各大公司、公共场所的免费停车场一律改为收费停车场。伦敦实施交通拥挤费,以缓解市中心日益拥挤的
车流。图\ref*{f2}中路面上标示“C”的红色圆形是告知驾驶人已经进入拥挤费的收费区域。

凡到过新加坡的人都说新加坡的道路秩序最好,不堵车。原来新加坡早在十几年前就在北京市区征收特定道路使用
费,要交了费的汽车才能驶入。那时是在闹市区四周设岗亭,汽车要交了费才能驶入,现在已改由摄像机和电脑组
成的自动收费系统,计算出应交的道路使用费并输进车主的帐户。

多年来首尔不断建路,但也不断堵车,堵得大家都有意见。1997年政府提高汽油税,提高停车费,加征道路使用费。
三管齐下,一部分车主放弃了开车上班的打算,堵车情况马上缓解。


\subsection{鼓励向可以提高效率的交通方式转移}

相较于新建道路,发展公共运输系统可以改变驾驶人的通勤方式,来减少私人汽车的使用,因此根治交通堵塞的效
果,较新建道路还来的显著。

公共运输系统可分为公共汽车和城市轨道交通系统两种,都能解决拥堵。但公共汽车的运作,仍受道路交通的影响。
而后者则由于其是独立于其他交通体系 (如道路和其他铁路) 以外,运输乘客的效率更高。

城市公共交通是最有效率的交通方式。研究表明,一个公交乘客所需的道路空间仅是小汽车所需空间的5\%。利用快
速便捷的公交系统增加对其出行者的吸引力,鼓励出行者放弃小汽车选择公交系统,或者是进入主城区前停车换乘
公交。这种交通方式之间的转移朝着提高道路资源使用效率的方向,有助于改善整个城市的交通状况。考虑到其他
限制进入主城区车辆的措施,可以为公交系统的发展提供了更好的路面环境,使整个城市交通的改善趋于良性循环。

在纽约,曼哈顿上班的人,是从家里开车到市郊地铁站或火车站再换乘地铁或火车进入市区,在市内的活动靠公共
汽车、地铁和出租汽车。曼哈顿许多街道只有持特殊牌照的车辆才能停车上落货和上落客,其他车辆停车就予罚款。

华盛顿的联邦政府官员多不住在华盛顿市内,而是住在与华盛顿特区相邻的三个州的小城镇上,如果他们每天开几
十公里车程到华盛顿上班,通向华盛顿的几十条公路都会堵车。为此,联邦政府拟定用公交工具接送代替个人开车
的计划,使部分人放弃自己开车改由公交车接送。为了使人们接受这种做法,政府同时也答应在非上下班时间内,
谁要有急事回家可由公交系统提供免费出租车回家。

城市的交通问题一般而言,集中在上下班的早晚高峰时段。如果能够吸引那些上下班使用私人小汽车的出行者使用
公交系统,可以减少高峰时间拥堵。在日本东京,人们上下班多是选择便捷的地铁,家用汽车平日存在车库,一是
只有乘地铁才能准时上班,二则是公司里总经理和董事长二人才有车位。

员工较多的大公司和单位可以通过开设单位通勤的班车满足员工上下班的需要。这样,可以减少上下班私人小汽车
的使用。

鼓励合乘也可以提高交通工具的使用效率,从而提高道路资源的使用效率,当然要制定相应的法规约束。

\subsection{结语}

综上所述,我认为城市发展到一定阶段,只通过道路设施建设来满足交通需求是不可能的。随着城市的发展,交通
量的增加,土地日益紧张,修路的代价越来越大,修路对城市面貌的改变、对城市自然景观、人文景观的破坏逐渐
变大。而且,修路会进一步刺激交通需求。花出巨大代价进行汉口路西延工程的建设,不难预见,这条路很快就会
饱和。

需要结合国内外的发展经验,将改善交通状况的重点从交通建设转移到交通管理手段上。通过管理手段,限制小汽
车进入主城区,鼓励出行者上下班选择可以提高道路资源使用效率的出行方式。总之,一味的增加道路是不可行
的,道路崇拜必将终结。

另外,最好综合考虑南京市近期的交通规划,比如说规划中的地铁4号线将连接龙江、北京西路、鼓楼、北京东路
等,它的建成对解决河西龙江地区与主城区之间的交通问题将有很大帮助。而且,尽快完善河西的各项配套设施,
减少跨河交通需求,可以减少跨河交通的压力。

\section{尾声}

对南京市汉口路西延工程的讨论不仅仅是回答“要大路还是要大学”的问题,而是思考向南京这样的城市,
发展到现在这个阶段,能不能超越道路崇拜,寻找比修路更好的办法。毕竟,能用的地已经用了,可拆的房子已经
拆了。今天,是大路和大学的选择,如果面对的是普通的民宅,也能讨论这么久,就好了。


\end{document}
