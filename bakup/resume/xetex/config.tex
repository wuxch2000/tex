\usepackage{fontspec,xunicode,xltxtra}
\setmainfont[Mapping=tex-text]{Times New Roman}
% \setmainfont[Mapping=tex-text]{Arial}
\setsansfont[Mapping=tex-text]{Arial}
\setmonofont[Mapping=tex-text]{Courier New}
% \setmonofont[Mapping=tex-text]{Times New Roman}

\usepackage{xeCJK}
\setCJKmainfont[ItalicFont={Adobe Kaiti Std}]{Adobe Song Std}
% \setCJKmainfont[ItalicFont={Adobe Kaiti Std}]{Adobe Kaiti Std}
\setCJKsansfont{Adobe Heiti Std}
% \setCJKsansfont{Microsoft YaHei}
\setCJKmonofont{Adobe Heiti Std}
\punctstyle{banjiao}

% \usepackage{indentfirst}
\usepackage[dvips,dvipsnames,svgnames]{xcolor}
\definecolor{light-gray}{gray}{0.4}
\definecolor{more-light-gray}{gray}{0.2}

\usepackage[dvipdfmx]{hyperref}  % options: dvipdfmx, pdftex 
\hypersetup{
    % bookmarks=true,         % show bookmarks bar?
    bookmarksopen=true,
    pdfpagemode=UseNone,    % options: UseNone, UseThumbs, UseOutlines, FullScreen
    pdfstartview=FitH,      % options: FitH, FitV
    pdfborder=1,
    pdfhighlight=/P,
    pdfauthor={wuxch},
    unicode=false,          % xetex should set to false
    colorlinks=true,             % false: boxed links; true: colored links
    linkcolor=black,         % color of internal links
    citecolor=green,        % color of links to bibliography
    filecolor=magenta,      % color of file links
    urlcolor=light-gray           % color of external links
}
% \makeindex


\usepackage{graphicx}
\graphicspath{{fig/}}

\usepackage{enumitem}
% \setitemize{label=$\bullet$,leftmargin=1em}
\setitemize{label={\color{blue} $\star$},leftmargin=1em}
% \setitemize{label=$\bullet$,leftmargin=3em,noitemsep,topsep=0pt,parsep=0pt}
% \setenumerate{leftmargin=3em,noitemsep,topsep=0pt,parsep=0pt}

% \setlength{\parskip}{1.5ex plus 0.5ex minus 0.2ex}
% \setlength{\parskip}{2.0ex plus 0.5ex minus 0.2ex}

% \setlength{\parindent}{5ex}
% \setlength{\parindent}{0ex}

% \usepackage{setspace}
% \linespread{1.25}

% 英文的破折号--不明显,使用自己画的线。
% \newcommand{\myrule}{\hspace{0.5em}\rule[3pt]{1.6em}{0.3mm}\hspace{0.5em}}

% 以下定义简历中的格式

% section的格式定义
\renewcommand{\sectionfont}{\sffamily \color{blue}}

\renewcommand{\namefont}{\Large \sffamily}

% 时间 职位 部门 职责
\newcommand{\ztework}[4]%
{%
\begin{minipage}{1\textwidth}
\vspace{2ex}
{\sffamily #2} \hfill {\sffamily \small #1}
\end{minipage}

\begin{minipage}{1\textwidth}
\begin{description}
\item[{\sl 部门:}]#3
\item[{\sl 职责:}]#4
\end{description}
\end{minipage}
}

%%% Local Variables: 
%%% mode: latex
%%% TeX-master: "wuxch-res9a"
%%% End: 
