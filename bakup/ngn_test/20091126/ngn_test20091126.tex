\documentclass  [11pt,twocolumn]{article}
\title{NGN C Test}
\author{}
\date{}

\usepackage[a3paper,landscape]{geometry}
\geometry{left=3em}
\geometry{columnsep=15em}
\geometry{right=15em}

\ProvidesPackage{config}

\usepackage{fontspec,xunicode,xltxtra}

\setmainfont[Mapping=tex-text,Ligatures=Common]{Adobe Garamond Pro}
\setsansfont[Mapping=tex-text,Numbers=Uppercase]{Myriad Pro}
\setmonofont[Mapping=tex-text]{Courier New}

\usepackage{xeCJK}
% \setCJKmainfont[ItalicFont={Adobe Kaiti Std}]{Adobe Song Std}
\setCJKmainfont[ItalicFont={Adobe Kaiti Std}]{Adobe Heiti Std}
% \setCJKmainfont[ItalicFont={Adobe Kaiti Std}]{Adobe Kaiti Std}
\setCJKsansfont{Adobe Heiti Std}
% \setCJKsansfont{Microsoft YaHei}
\setCJKmonofont{Adobe Heiti Std}
\punctstyle{banjiao}

\usepackage[table]{xcolor}

%生成PDF的链接
\usepackage{hyperref}
\hypersetup{
    % bookmarks=true,         % show bookmarks bar?
    bookmarksopen=true,
    pdfpagemode=UseNone,    % options: UseNode, UseThumbs, UseOutlines, FullScreen
    pdfstartview=FitB,
    pdfborder=1,
    pdfhighlight=/P,
    pdfauthor={wuxch},
    unicode=true,           % non-Latin characters in Acrobat’s bookmarks
    colorlinks,             % false: boxed links; true: colored links
    linkcolor=blue,         % color of internal links
    citecolor=blue,        % color of links to bibliography
    filecolor=magenta,      % color of file links
    urlcolor=cyan           % color of external links
}



% 表格
\usepackage{booktabs}

\usepackage{caption}
\usepackage{fancyhdr}
\usepackage{graphicx}


\usepackage{geometry}
\geometry{a4paper}
\geometry{hmargin=.1cm}
\geometry{vmargin=1cm}

\rhead{\scriptsize{\textsf{NGN软件开发一部C语言内部测验 2009 11}}}

\begin{document}
% \begin{center}
\begin{minipage}{0.9\linewidth}
\blk{部门}\blk{科室}\blk{姓名}\blk{工号}
\end{minipage}
% \end{center}

% \begin{flushright}
% {\small{\textit{考试时长:3小时,总分:100分}}}
% \end{flushright}

\section{改错题 (共计9题,60分)}
\subsection{\lookforerror{4}}
\begin{lstlisting}
BYTE *ComputeCheck(WORD16 dbHandle)
{
    WORD16 curRecNum,xorCheck;
    BYTE *p,p_check[6];
    WORD32 addCheck;

    xorCheck=0;
    addCheck=0;
    curRecNum=GetRecNum(dbHandle);
    if (curRecNum==0) return NULL;
    ......
    memcpy(p_check,&xorCheck,2);
    memcpy(p_check+2,&addCheck,4);
    return (BYTE *)&p_check;
}
\end{lstlisting}
\subsection{\lookforerrorandfix{6}}
\begin{lstlisting}
extern WORD16 GetSessionIdAndPUIFromRaw(BYTE *DataIn,PUI*ptPui);

void  DimDapProcessOtherMSG(T_InstanceData *ptInstance,
                            BYTE *DataIn,
                            T_DimCmdHdr *ptCmdHead,
                            WORD32 DataLength )
{
    T_DimCmdHdr *ptDimMsgHeader = ptCmdHead;
    PUI         tPUI;

    if(ptInstance == NULL)
    {
        return;
    }  
    if(NULL == DataIn)
    {
        return;
    }
    /* 流程跟踪 */
#if (defined(DIM_PROCTRACE) && defined(DIM_ATCA_PSS))
    GetSessionIdAndPUIFromRaw((PCHAR)DataIn, &tPUI);
    if (DIM_PROC_MSG_REQ == ptDimMsgHeader->Flags_R)
    {
        wMsgType = DIM_PROC_MSGTYPE_REQ;
    }
    else
    {
        wMsgType = DIM_PROC_MSGTYPE_ACK;
    }
    DimProcProcessReportToOmc(ptDimMsgHeader->dwAppId, 
                              &tPUI, wMsgType, 
                              ptDimMsgHeader->CmdCode,
                              DIM_PROC_MSGDIRC_ACCEPT, UNUSED_FLAG_WORD32,
                              "DimDapProcessOtherMSG: DAP Reveive One 
Message From SCTP/TCP");
#endif
}
\end{lstlisting}

\subsection{\lookforerrorandfix{6}}
\begin{lstlisting}
#define M_LocationAreaIdentification   0xb
#define MAX_CODE_LEN  1000
void Decode_LocationID(BYTE * pMsgIn)
{
    BYTE  byBytePositionIndex = 0;
    WORD16 wReturn = S_SUCCESS;
    if (pMsgIn == NULL)
    {
        return;
    }
    while(byBytePositionIndex < MAX_CODE_LEN && wReturn == S_SUCCESS)
    {
	    if(*(pMsgIn + byBytePositionIndex) == M_LocationAreaIdentification)
	    {
            byBytePositionIndex++;
            wReturn = GetIE_LocationAreaIdentification(pMsgIn + byBytePositionIndex );
            byBytePositionIndex += 5;
	    }
    }
}
\end{lstlisting}

\subsection{\lookforerrorandfix{6}}
\begin{lstlisting}
WORD32 C_SRV_StrLen(BYTE* pbStr, WORD16 wMaxStrLen)
{
    WORD32  dwLen = 0;
    if (pbStr == NULL)
    {
        dwLen = 0;
    }
    else
    {
        dwLen = strlen((CHAR *) pbStr);
        if (dwLen > wMaxStrLen)
        {
            dwLen = wMaxStrLen;
        }
    }
    return  dwLen;
}
void test()
{
    CHAR chBuf[REASON_LTH];
    BYTE bTempLenth = 0;
    bTempLenth = C_SRV_StrLen("Moved Temporarily", REASON_LTH);
    memcpy(chBuf, (BYTE*)"Moved Temporarily", bTempLenth);
    chBuf [strlen(chBuf)+1]  = '\0';
    printf("%s \n",chBuf);
}
\end{lstlisting}

\subsection{\lookforerrorandfix{10}}
\begin{lstlisting}
typedef struct tagTableStru{
	DB_HANDLE	hTable;
	char		lpTableName[TABLE_NAME_LEN];
	BYTE		btFieldsNum;
	BOOL		bSave;
}TABLE_STRU, *LP_TABLE_STRU;

#define MAX_TABLE_NUM       100
TABLE_STRU	TableStru[MAX_TABLE_NUM];

#define  MAKE_TBL_FILENAME(TableLoc, Dir, TblFileName, Postfix) \
	memset(TblFileName, 0, sizeof(TblFileName));                \
	strcat(TblFileName,TableStru[TableLoc].lpTableName);        \
	strcat(TblFileName,".");                                    \
	strcat(TblFileName,Postfix)

extern STATUS	_Tables_Remove(LPSTR Dir, LPSTR Postfix)
{
	BYTE 		FileName[80];
	DB_HANDLE 	TableLoc = 0;
	FILE     * 	dbFile;

	while (TableStru[TableLoc].hTable != INVALID_DB_HANDLE)
	{
		if (TableStru[TableLoc].bSave)
		{
  			memset(FileName, 0, sizeof(FileName));
  			MAKE_TBL_FILENAME(TableLoc, Dir, FileName, Postfix);
			if ((dbFile = fopen(FileName,"rb")) != NULL)
			{
				fclose(dbFile);
     			if (OK != remove(FileName))
				{
  					return ERROR;
				}
			}
			else
			{
                DbgMsg(INFORM_ERR,("fail to open %s!\n",FileName));
            }
			TableLoc++;
   		}
    }
	return OK;
}
\end{lstlisting}

\subsection{\lookforerrorandfix{9}}
\begin{lstlisting}
#define TEMP_BUF_LEN 64

WORD16 Syn_SetToDB(VOID *para_in )
{
	BYTE		location[TEMP_BUF_LEN]	= {0};
	BYTE		location_2[TEMP_BUF_LEN]	= {0};
	WORD32 	*	pParaIn = NULL;
	WORD32 		handle;
	WORD32 		ret = 0;
	
	*pParaIn = *(WORD32 *)para_in;
	handle = Data_GetKey(ROOT , ROOT , strlen(ROOT));            
    if(!Valid_Loc(handle))
    { 
	    ret = Data_Set( ROOT , ROOT, strlen(ROOT),
                        _DATA_OPRTYPE_NOWRITE, _DATA_TYPE_ENTRY,
                        NULL,0);
	    if(ret != DATA_SET_SUC)
        {
            return ERR_SET_DB_FAIL ;
        }
    }
	snprintf(location,TEMP_BUF_LEN,"%s%s%s",ROOT,NODE_SPLIT, SYN) ;
	handle = Data_GetKey(ROOT , location , strlen(location));            
    if(!Valid_Loc(handle))
    { 
	    ret = Data_Set( ROOT , location, strlen(location),
                        _DATA_OPRTYPE_NOWRITE, _DATA_TYPE_BRANCH, NULL, 0);

	    if(ret != DATA_SET_SUC)
        {
            return ERR_SET_DB_FAIL ;
        }
    }    
	snprintf(location_2,TEMP_BUF_LEN,"%s%s%s",location,NODE_SPLIT,SYN_ENABLE) ;
	handle = Data_GetKey(ROOT , location_2 , strlen(location_2));            
    {
	    ret = Data_Set( ROOT , location_2,strlen(location_2),
                        _DATA_OPRTYPE_NOWRITE ,_DATA_TYPE_WORD32,(LPSTR)pParaIn,sizeof(WORD32)); 
	    if(ret != DATA_SET_SUC)
        {
            return ERR_SET_DB_FAIL;
        }
    }
	return SUCC_AND_NOPARA;
}
\end{lstlisting}

\subsection{\lookforerrorandfix{6}}
\begin{lstlisting}
DIM_RESULT DimGetConfigAvpByTree(BYTE *ptr, T_ConfigAvpGroup *ptConfigAvpGroup,WORD32 dwLen)
{
    ......
    TmpIndex = 0;
    dwUsedNum =ptConfigAvpGroup->dwUsedNum<=MAX_AVPCONFIG_NUM?ptConfigAvpGroup->dwUsedNum:MAX_AVPCONFIG_NUM;
    for(i = 0; i<dwUsedNum; i++)
    {
        ResultAvpNum = 0;
        dwAvpCode = ptConfigAvpGroup->tAvpConfig[i].AvpCode;
        dwAvpNum = ptConfigAvpGroup->tAvpConfig[i].AvpNum;
        for(j = 0; 
            j<pParserGvar->tRawLocationRecord.dwRecordNum;
            j++)  /*在已经扫描过的 avp 中查找*/
        {
            if(dwAvpCode == pParserGvar->tRawLocationRecord.tRawLocation[j].dwAvpCode)
            {
                ptConfigAvpGroup->tAvpConfig[i].AvpLen[ResultAvpNum] = 
                    pParserGvar->tRawLocationRecord.tRawLocation[j].dwAvpLen;
                if(++ResultAvpNum >= MAX_AVPGET_NUM)
                {
                    bGetAll = TRUE;
                    break;
                }
            }
            if(ResultAvpNum >= dwAvpNum) /*找全,跳出找下一个*/
            {
                ptConfigAvpGroup->tAvpConfig[i].ResultAvpNum = ResultAvpNum;
                bGetAll = TRUE;
                break;
            }
        }
        if(bGetAll)
        {
            bGetAll = FALSE;
            continue;
        }
        while(pCurrentRaw < pEndRaw) /*在剩余码流中找*/
        {
            if(!NetBytesToAvpHeader((CHAR*)pCurrentRaw, &tAvpHead))
            {
                if (DIM_S_OK != DIM_PUB_SemaphoreV(ptDimSemaphore))
                {
                    return ~DIM_S_OK;
                }  	
                return DIM_E_PAR_TRANS_ENDIAN_FAIL;
            }

            /*把扫描过的 avp 保存下来*/
		    pParserGvar->tRawLocationRecord.tRawLocation[TmpIndex].
                dwAvpCode = tAvpHead.dwAvpCode;
        }
        ptConfigAvpGroup->tAvpConfig[i].ResultAvpNum = ResultAvpNum; 
    }

    ......
    return DIM_S_OK;
} 
\end{lstlisting}

\subsection{\lookforerrorandfix{5}}
\begin{lstlisting}
#define MAX_BLADE_NUM_PER_POOL   20

typedef 
{
    BOOL8  blEnabled;
    BYTE   bConfigWeight;
    WORD16 wIp;
    ......
}T_BladeNode;

typedef struct
{
    BYTE        bAliveBladeNum;
    WORD16      wIndex;
    T_BladeNode atBladeArray[MAX_BLADE_NUM_PER_POOL];    
}T_BladePool;

/* 根据选定算法选取一个可用的 Blade */
int Mc_SlbChoose(T_VServer *pVServer, T_SlbInfo *pSlbInfo)
{
    T_BladePool *p = &(pVServer->tPool);  
    WORD32      j  = 0;
    BYTE        cw = 0; /* 当前权值 */

    if( (0 == p->wNum) || (0 == p->bAliveBladeNum) )
    {
        return MCS_FAIL;
    }
    switch(pVServer->bSlbType)
    {
        /* 轮询,这种 SLB 方法需要记录上次的结果,包括 wIndex */
        case SLBTYPE_ROUNDROBIN: 
        {
            j = p->wIndex;
            do
            {
                j = (j+1) % MAX_BLADE_NUM_PER_POOL;
                if( p->atBladeArray[j].blEnabled )
                {
                    p->wIndex = j;
                    pSlbInfo->bBladeIndex = j;
                    pSlbInfo->wBIp        = p->atBladeArray[j].wIp;
                    return MCS_OK;
                }
            }while(j != p->wIndex);
            return MCS_FAIL;
        }

        /* 根据权重进行轮询,这种 SLB 方法需要使用记录上次的结果,
         * 包括CurWeight, wIndex 这种方式是先设置 CurWeigh 
         * 为最大值,然后从头到尾寻找 Weight 大于等于此值的 Blade,
         * 降低 CurWeight 为 CurWeight-gcd(S),然后再从头到尾寻找 
         * Weight 大于等于此值的 Balde */
        case SLBTYPE_WEIGHTROUNDROBIN: 
        {
            j  = p->wIndex;     /* 获取上次使用 Blade Index */
            cw = p->bCurWeight; /* 获取当前使用的 Weight */
            while(1)
            {
                j = (j+1) % MAX_BLADE_NUM_PER_POOL;
                if(0 == j) /* 表示 Pool 中第一个 Blade,
                            * 初始时,或者循环了一圈再次开始 */
                {
                    if(cw <= p->bWeightGCD)
                    {
                        cw = p->bWeightMax;
                        if(0 == cw)
                        {
                            return MCS_FAIL;
                        }
                    }
                    else
                    {
                        cw -= p->bWeightGCD;
                    }
                }
                if(  p->atBladeArray[j].blEnabled &&
                     p->atBladeArray[j].bConfigWeight >= cw)
                {/* 选定此 Blade */
                    pSlbInfo->bBladeIndex = j;
                    pSlbInfo->wBIp = p->atBladeArray[j].wIp;
                    p->wIndex = j; /*记录下当前使用的 Blade index */
                    p->bCurWeight  = cw;/*记录下当前使用的 CurWeight */
                    return MCS_OK;
                }
            } 
        }
        default:
            return MCS_FAIL;
    }
}
\end{lstlisting}

\subsection{\lookforerrorandfix{8}}
\begin{lstlisting}
int sotcpbind(struct inpcb_tcp *pcb, struct sockaddr_in *addr)
{
    struct tcb *tcb = 0;
    ipaddr_t  laddr_t;
    UINT32 laddr = 0;
    UINT16  lport=0;
    struct brs_socket *so= 0;
    int wild = 0;
    int reuseport = 0;
	
    lport  = 0;
    tcb = pcb->tcpcb;
    so = tcp_get_socket(tcb);
    reuseport = (so->so_options & SO_OPT_REUSEPORT);
	
    /* TCB 完整性检查 */
    if (!tcb || tcb->state != TCP_CLOSED || tcb->rcv_buf == 0)
    {
        return BRS_SOCKET_ERROR;
    }

    /* 只有在没有 reuse 选项打开的时候,查找 tcb 的时候才允许通配*/
    if((so->so_options & (SO_OPT_REUSEADDR|SO_OPT_REUSEPORT) )==0)
		wild = LOOKUP_WILDCARD;
    if(addr)  /* 显式调用 */
    {
        /* 获取指定的地址和端口 */
        SET_IPV4_IPADDR(&laddr_t, &(addr->sin_addr.s_addr));
        SET_IPADDR_TYPE((&laddr_t), IPV4);  /* 一定是 ipv4 */
        lport = addr->sin_port;
        laddr = addr->sin_addr.s_addr;
        /* 检查是否组播地址 */
        if(ipaddr_test(&laddr_t, IPADDR_TEST_MULTICAST))
        {
            if(so->so_options & SO_OPT_REUSEADDR)
                reuseport = (SO_OPT_REUSEADDR | SO_OPT_REUSEPORT);
        }
        /*指定了特定地址,非通配地址*/
        else if(!ipaddr_test(&laddr_t, IPADDR_TEST_UNSPECIFIED)) 
        {
            /* 检查该地址是否是本地地址 */
#if !INSTALL_VOIP 
            /* 检查该地址是否是本地地址或者 vrrp 地址*/
#if INSTALL_ALG && INSTALL_ATTACHE_VRRP
            if(IsMyAddr(&laddr_t) == 0 && IsVrrpAddr(&laddr_t) == 0 )
#else
                if(IsMyAddr(&laddr_t) == 0)
#endif
                {
                    return ESO_ADDRNOTAVAIL;
                }
#endif  
        } 
        /* 检查指定地址,端口是否有冲突 */
        if(lport)
        {
            struct tcb *t = tcp_tcb_lookup(laddr, lport, 0, 0,wild);
            if (t)
            {
                /* 找到了本地地址,端口冲突的 tcb */
                struct brs_socket *s = (struct brs_socket *)tcp_get_socket(t);
                if((reuseport & s->so_options)==0)
                { /* 没有设置 reuseport 选项 */
                    return ESO_ADDRINUSE;
                }
            }	        
        }
        /* 保存本地地址*/
        tcp_set_local_address(tcb, laddr);
    }
    /* 将 tcb 按照本地地址,端口作索引,插入到 hash 表中 */
    add_tcp_tcb_to_hash(tcb);
    return BRS_SOCKET_OK;
}
\end{lstlisting}

\blankline{1ex}
\section{填空题 (共计5题,每题4分,共20分)}

\subsection{\fillblank}
\begin{lstlisting}
char strBuff[8] = "1234567";
strncpy(strBuff, "abcd", 3);
printf("%s",strBuff);
\end{lstlisting}

\subsection{\fillblank}
\begin{lstlisting}
main(int argc, char** argv)
{
    char *str="hello world";
    printf("%d,%d,%d\n",
           sizeof(str),
           sizeof("hello world"),
           strlen(str));
}
\end{lstlisting}

\subsection{\fillblank}
\begin{lstlisting}
void GetMemory(char **p, int num) 
{ 
    *p=(char *) malloc(num); 
} 
void main(void) 
{ 
    char *str=NULL; 
    GetMemory(&str,100);
    strcpy(str,"hello world");
    printf(str); 
} 
\end{lstlisting}

\subsection{\fillblank}
\begin{lstlisting}
char *GetMemory(void) 
{ 
    char p[]="hello world"; 
    return p; 
} 
void Test(void) 
{ 
    char *str=NULL; 
    str = GetMemory( );
    printf(str); 
} 
\end{lstlisting}

\subsection{\fillblank}
\begin{lstlisting}
void GetMemory(char *p, int num) 
{ 
    p=(char *) malloc(num); 
} 
void main(void) 
{ 
    char *str=NULL; 
    GetMemory(str,100); 
    strcpy(str,"hello world"); 
    printf(str); 
} 
\end{lstlisting}

\section{编程题 (共计2题,每题10分,共计20分)}
\subsection{链表排序(从小到大),10分}
\begin{minipage}{1\columnwidth}
节点定义为:
\begin{lstlisting}
struct Node
{
    int nValue;
    struct Node* pNext;
};
\end{lstlisting}
最后一个节点的\lstinline{pNext = NULL}。

函数原型:\lstinline{Node* SortChain( Node* pHead );} 
          \lstinline{pHead}为指向链表头节点的指针。
返回值:链表头
\end{minipage}
\pagebreak
\begin{minipage}{1\columnwidth}
\vspace{40ex}
\end{minipage}

\subsection{设有一个表头指针为pHead的单链表,节点定义同题目1。试设计一个算法,通过遍历一趟链表,将链表中所有结点的链接方向逆转,要求逆转结果链表的表头指针pHead指向原链表的最后一个结点。10分}

函数原型:\lstinline{Node* SortChain( Node* pHead );} 
          \lstinline{pHead}为指向链表头节点的指针。
返回值:链表头

\end{document}
