\documentclass  [11pt,twocolumn]{article}
\title{NGN C Test}
\author{}
\date{}

\usepackage[a3paper,landscape]{geometry}
\geometry{left=3em}
\geometry{columnsep=15em}
\geometry{right=15em}

\usepackage{config}
\rhead{\scriptsize{\textsf{NGN分中心技能竞赛 2009 12}}}

\begin{document}
% \begin{center}
\begin{minipage}{0.9\linewidth}
工号后6位(\textit{请只填工号,不要填其他信息}):\rule[-1pt]{.4\textwidth}{1pt}
\end{minipage}
% \end{center}


\begin{flushright}
{\small{\textit{考试时长:3小时,总分:100分}}}
\end{flushright}

\normalsize{\sffamily{试题背景说明:}}

\begin{enumerate}
    \item 基本数据类型定义:
\begin{lstlisting}
typedef  char *             LPSTR;
typedef  unsigned   char ** LPLPSTR;
typedef  signed     char    CHAR;
typedef  unsigned   char    BOOL8;
typedef  unsigned   char    BYTE;
typedef  unsigned   short   WORD;
typedef  unsigned   long    DWORD;
typedef  unsigned   short   WORD16;
typedef  unsigned   long    WORD32;
\end{lstlisting}
    \item 数据接口调用函数原型说明:
\begin{lstlisting}
void dbCall( WORD wEvent, LPSTR pReq, LPSTR ptAck );
\end{lstlisting}
其中:\lstinline{wEvent}表示事件号,\lstinline{pReq}表示入参结构体指针,\lstinline{ptAck}表示出参结构
体指针。

入参结构体的定义一般如下:
\begin{lstlisting}
typedef struct 
{
    /* 消息类型:同步调用或异步调用,调用者必须填写该参数 */
    BYTE      bMsgType;  
    ......
}D_XXX_REQ, * LPD_XXXX_REQ;
typedef struct
{
    /* 接口调用结果-返回成功或者失败码 ,  */
    /* 接口必须返回该参数的,供调用者使用。 */
    WORD     wRetCode;
    ......
}D_XXX_ACK,* LPD_XXX_ACK;
\end{lstlisting}
\end{enumerate}

\section{改错题,每道题目至少1处错误或者隐患,需要在错误的位置指明原因,并对其进行修正(共60分)。}

\subsection{\lookforerrorandfix{6}}

\begin{lstlisting}
T_REGRET regGetCgCfg(TENUM_REG_INST_TYPE tInstType, LPVOID pData, BYTE bDir)
{
    DM_GETCGCFG_REQ tGetCGCfgReq = {0};
    DM_GETCGCFG_ACK tGetCGCfgAck; 
    T_REGRET        tRegRet = {0};

    tRegRet.retCtrl = REG_ERROR;
    if ( NULL == pData ) 
    {
        return tRegRet;
    }

    if (REG_INST_UEREG == tInstType) 
    {
        tGetCGCfgReq.wStrategy = ((T_REG_UEINST *)pData)->wCGID;
        tGetCGCfgReq.wType = SIP_REGISTER;
    }
    else if(REG_INST_THIRD == tInstType) 
    {
        tGetCGCfgReq.wStrategy =((T_REG_UEINST *)pData)->wCGDestCode;
        tGetCGCfgReq.wType = SIP_REGISTER;
    }

    tGetCGCfgReq.msgType = MSG_CALL;
    tGetCGCfgReq.bDirection = bDir;

    dbCall(DM_GETCGCFG  ,(LPSTR)&tGetCGCfgReq,(LPSTR)&tGetCGCfgAck);
    if (tGetCGCfgAck.retCODE != RC_OK) 
    {
        tRegRet.retServ = REG_NOSERV;
        return tRegRet;
    }

    switch (tGetCGCfgAck.wCGPolicy)
    {
        case DB_CALLGAPING_NEED:    /* 需要查询 GDM */
            tRegRet.retCtrl = REG_SUCCESS;
            return tRegRet;
        default:    /* DB 返回异常不影响正常流程*/
            tRegRet.retCtrl=REG_SUCCESS;
            tRegRet.retServ= 0;
            return tRegRet;
    }
}

T_REGRET regThirdSaveRtaRsp(T_REG_THIRDINST *pData, LPSTR pMsgPara)
{
    T_REGRET tRegRet = {0};

    if (pData == NULL || pMsgPara == NULL) 
    {
        tRegRet.retCtrl=REG_ERROR;
        return tRegRet;
    }
    ...... //省略
    if(MMCmd_SysPerfGlobal_IsEnable(MML_PERF_GLOBAL_CODE_CALLGAPING))
    {
        tRegRet=regGetCgCfg(REG_INST_THIRD,(LPVOID)pData,B_CALLDIR_OUT);
        if (tRegRet.retCtrl != REG_SUCCESS) 
        {
            if (tRegRet.retServ = REG_NOSERV) 
            {
                tRegRet.retCtrl = REG_SUCCESS;
                tRegRet.retServ = 0;
                return tRegRet;
            }
            else 
            {
                tRegRet.retServ = REGUE_SERVERERROR;
                return tRegRet;
            }
        }
        else 
        {
            ...... //省略
        }
    }
    return tRegRet;
}
\end{lstlisting}

\subsection{\lookforerrorandfix{7}}

\begin{lstlisting}
/*****************************************************************
* 函数名称: DataSearchByKey
* 功能描述: 根据应用传入的关键字查找实例数据区
* 输入参数: T_UNIV_HASH_KEY *pkey  输入数据的 key 值
* 输出参数: WORD32 * pdwInst    返回实例的数组的索引
*            LPVOID * ppAreaAddr 返回实例的数据区地址
* 返 回 值: 正确返回 S_OK, 其他返回则异常  
******************************************************************/
T_RESULT DataSearchByKey (T_UNIV_HASH_KEY *pkey, WORD32 * pdwInst, 
                          LPVOID * ppAreaAddr);

/*****************************************************************
* 函数名称: DataUnivIterNext
* 功能描述: 根据应用传入的索引查找该记录的下一个记录的索引
* 函数参数: WORD32 * pdwInst    输入参数作为找该索引对应记录的下一个纪录
*                            输出参数,将下一个记录的索引带出
*            LPVOID * ppAreaAddr 返回实例的数据区地址
* 返 回 值: 正确返回 S_OK, 其他返回则异常  
******************************************************************/
T_RESULT DataUnivIterNext(WORD32 * pdwInst, LPVOID * ppAreaAddr);

/*****************************************************************
* 函数名称: DataUnivIterFirst
* 功能描述: 查找数据区的第一个记录,带出索引和对应数据区的地址
* 输出参数: WORD32 * pdwInst    返回实例的数组的索引
*            LPVOID * ppAreaAddr 返回实例的数据区地址
* 返 回 值: 正确返回 S_OK, 其他返回则异常  
******************************************************************/
T_RESULT DataUnivIterFirst(WORD32 * pdwInst, LPVOID * ppAreaAddr);

void  Perform_OMC_Report(void)
{
    T_RESULT            wGlbRslt;
    WORD32                dwNodeIdx;
    T_UNIV_HASH_KEY        tuHashKey;
    T_ConnListNode    *    ptConnNode = NULL;
    T_PERFORM_INFO_SUM     *    pPerformSum    = NULL;
    T_PERFORM_NODE_INFO     *    pPerformNodeInfo = NULL;

    wGlbRslt = DataGetGlbVar(GREG_RES_PERFORM_SUM, (LPVOID*)&pPerformSum);

    wGlbRslt  =    DataIterFirst(&dwNodeIdx, (LPVOID*)&pPerformNodeInfo);
    while (wGlbRslt    == S_OK    && pPerformNodeInfo    != NULL
           && dwNodeIdx !=    UNIV_UNUSED_WORD32)
    {
        if (pPerformNodeInfo->dwValid  == PERFORM_TRUE)
        {
            ptConnNode = NULL;
            tuHashKey.dwKey    = pPerformNodeInfo->dwValue;
            wGlbRslt = DataSearchByKey (&tuHashKey, &dwNodeIdx, (LPVOID*)&ptConnNode);
            if (ptConnNode != NULL && wGlbRslt == S_OK)
            {
                pPerformSum->dwRecvPPRNum +=ptConnNode->dwRecvPPRNum;
                pPerformSum->dwRecvRTRNum +=ptConnNode->dwRecvRTRNum;
            }
            else
            {
                ; //这个地方不需要特殊处理
            }
        }

        else
        {
            ; //这个地方不需要特殊处理
        }
        wGlbRslt  =    DataIterNext(&dwNodeIdx, (LPVOID*) &pPerformNodeInfo);
    }
    // ……    省略
    return;
}
\end{lstlisting}

\subsection{\lookforerrorandfix{6}}

\begin{lstlisting}
char g_nmsbcReqBuf[MAX_REQ_SIZ];
char g_nmsbcAckBuf[MAX_REQ_SIZ];
#define RM_OAM_TEMP_BUF_LEN 256

WORD NM_SetMLIaInterfaceCmd(NM_ML_IA * ptMLIaInterface)
{
    DWORD        dwLeftDataLen = 0;
    BYTE        bTempBuf[RM_OAM_TEMP_BUF_LEN];
    BYTE         * pbTempBuf  = NULL;
    DWORD        dwTmpDataLen  = 0;
    MSG_COMM_OAM * pReqMsg      = (MSG_COMM_OAM *)g_nmsbcReqBuf;
    MSG_COMM_OAM * pAckMsg      = (MSG_COMM_OAM *)g_nmsbcAckBuf;
    NM_ML_IA     * ptMIa      = NULL;
    DWORD           dwLoopNum  = 0;

    if ( NULL == ptMLIaInterface)
    {
        return GEN_ERR;
    }

    pbTempBuf = bTempBuf;

    for(dwLoopNum = 0; dwLoopNum < NM_IA_INTERFACE_MAX; dwLoopNum++)
    {
        ptMIa = &ptMLIaInterface[dwLoopNum];
        if ( !ptMIa->blUsedFlag )
        {
            continue;
        }
        dwLeftDataLen = RM_OAM_TEMP_BUF_LEN - 1;
        dwTmpDataLen = snprintf( pbTempBuf, dwLeftDataLen,
            "ip-address ipv4 %s port %u local domain %s transport ",
            ptMIa->tLocalIA.bIpAddr, ptMIa->tLocalIA.wPort,
            ptMIa->tLocalIA.bIaDomain );
 
        pbTempBuf      += dwTmpDataLen;
        dwLeftDataLen -= dwTmpDataLen;/* 命令行的最大长度限制为255, */
                                      /* 所以这里不考虑减法溢出了   */
        dwTmpDataLen  = snprintf( pbTempBuf, dwLeftDataLen, " udp" );

        bTempBuf[RM_OAM_TEMP_BUF_LEN] = '\0';                
        if ( GEN_ERR == NM_ConstructOamMsg(bTempBuf, pReqMsg))
        {
            return GEN_ERR;
        }

        if ( SUCC_AND_NOPARA != NM_SendExecMsg(pReqMsg, pAckMsg))
        {
            return pAckMsg->ReturnCode;
        }
    }
    return NO_ERROR;
}
\end{lstlisting}

\subsection{\lookforerrorandfix{9}}

\begin{lstlisting}
/* 添加变更通知的注册表项 */
#define  DBS_NOTIFY_ADD_REG_ITEM(req,dbNames,tabName,type,event)  \
do{\
    if(req.wTableNum <= _DB_CFGCHG_REG_TABLE_NUM_MAX)\
    {\    strncpy(req.tTable[req.wTableNum].dbName,\
                dbNames,(_DB_NAME_LEN-1));\
        strncpy(req.tTable[req.wTableNum].tableName,\
               (CHAR *)tabName,(_DB_TABLE_NAME_MAX-1));\
        req.tTable[req.wTableNum].ucNotifyType    = type;\
        req.tTable[req.wTableNum].dwAppNotifyID = event;\
        req.wTableNum++;\
    }\
    else \
    {\
        XOS_SysLog(DBS_PRNLEVEL_ERROR,\
                   "[DBS]: line %d: NOTIFY_ADD_REG_ITEM req.wTableNum max,failed ! table[%s]\n", \
        __LINE__, req.wTableNum, tabName );\
    }\
}while(0)

WORD32 DBS_APP_Notify_Reg_H248()
{
    WORD32    dwJIDNo;
    JID       tJIDNotify;
    XOS_STATUS  dwRet;
    _db_t_cfgchg_reg_req   tH248Item;
    
    memset(&tH248Item,0,sizeof(_db_t_cfgchg_reg_req));
    dwRet = XOS_GetSelfJID(&tH248Item.tJID);
    if (XOS_SUCCESS != dwRet)
    {
        return RC_ERROR;
    }
    tH248Item.ucFlag     = _DB_CFGCHG_NOTIFY_REG;
    tH248Item.ucPacketId = 0;
    
    DBS_NOTIFY_ADD_REG_ITEM(tH248Item, 
                            PSS_CONFIG_DATABASE, 
                            "R_UDPPORT", 
                            _DB_CFGCHG_NOTIFY_TUPLE_TYPE, 
                            EV_COMM_UDPPORTCHG);
    dwJIDNo = XOS_ConstructJNO(JOB_TYPE_DBS_NOTIFY,1);
    dwRet = XOS_GetJIDByJNO(dwJIDNo, &tJIDNotify);
    if (XOS_SUCCESS != dwRet)
    {
        return RC_ERROR;
    }

    dwRet = XOS_SendAsynMsg(EV_CFGCHG_REG_APP_TO_DBS_REQ,
                            (BYTE *)&tH248Item, 
                            sizeof(_db_t_cfgchg_reg_ack),
                            OS_MSG_VER_DEFAULT,
                            XOS_MSG_HIGH, &tJIDNotify);
    if (XOS_SUCCESS != dwRet)
    {
        return RC_ERROR;
    }
    return RC_OK;
}
\end{lstlisting}

\subsection{\lookforerrorandfix{9}}

\begin{lstlisting}
BYTE  SdpDecodeAttribute_Qos( BYTE **ppMessage, SDP_PRECONDITION_INFO_t * ptQos )
{
    DWORD bLocation;
    BYTE  bAttr[60];
    BYTE  bCurrNum = 0; /* the number of total current status line*/
    BYTE  bDesNum  = 0; /* the number of total des status line*/
    BYTE  bConfNum = 0; /* the number of conf status line*/
    if( (ptQos == NULL) || (ppMessage == NULL)) 
    {   
        return SDP_DEC_ERROR;
    }    
    memset( ptQos, 0, sizeof(SDP_PRECONDITION_INFO_t) );    
    while(**ppMessage != SDP_SIGN_ENDOFSTRING) 
    {
        while((**ppMessage == SDP_SIGN_SPACE) || (**ppMessage == SDP_SIGN_TAB)) 
            (*ppMessage)++;
        
        bLocation = 0;
        if ('a' != **ppMessage)
        {
            continue; 
        }
        (*ppMessage)++;   
        if(**ppMessage != SDP_SIGN_EQUAL)
        {    
            return(SDP_DEC_ERROR); 
        }
        (*ppMessage)++; /* jump off the character '=' */    
        /* space disallowed */ 
        if(**ppMessage == SDP_SIGN_SPACE)
        {    
            return(SDP_DEC_ERROR); 
        }
        else
        {
            while( (**ppMessage != SDP_SIGN_CARRIAGERETURN) && 
                   (**ppMessage != SDP_SIGN_LINEFEED) )
            {
                bAttr[bLocation] = **ppMessage;
                (*ppMessage)++;
                bLocation++;
                if(**ppMessage == SDP_SIGN_COLON)
                {    
                    break;
                }
                if(**ppMessage == SDP_SIGN_ENDOFSTRING)
                {    
                    return(SDP_DEC_SUCCESS);
                }
            }
            bAttr[bLocation] = SDP_SIGN_ENDOFSTRING;
            if((**ppMessage == SDP_SIGN_CARRIAGERETURN) || 
               (**ppMessage == SDP_SIGN_LINEFEED))
            {
                SdpReadLine(ppMessage); /* 跳到下一个 a 行*/
                continue; 
            }
        }
    }
    return(SDP_DEC_SUCCESS);
}
\end{lstlisting}

\pagebreak
\subsection{\lookforerrorandfix{8}}

\begin{lstlisting}
typedef struct tagOmcMsg
{
    BYTE    ucVer;    /* 消息头版本号 */
    BYTE    ucMsgVer; /* 消息体版本号 */
    BYTE    ucMsgType;/* 消息类型 */
    BYTE    ucVerType;/* 前后台消息版本,指 V3 或者 V4 版本*/ 
    BYTE    ucPad2;   /* 填充字节 2 */ 
    WORD16  wMsgSize; /* 应用消息体大小 */
    WORD32  dwMsgId;  /* 应用消息 ID */
    WORD32  dwEndian; /* 前台字节序 */
    JID     tSender;  /* 发送方 JID */
    JID     tReceiver;/* 接收方 JID */
} T_MSGOMC_HDR;

BYTE *MsgOmcMsgInvert(void *ptMsg) 
{
    WORD32  dwHeadLen;
    T_MSGOMC_HDR *ptSrcMsg=NULL;
    BYTE    *pucDestMsg=NULL;    
    /* 获取互斥信号量资源 */
    bl8Ret  = XOS_ProcessSemP(g_dwSemId,WAIT_FOREVER);   
    if(!bl8Ret)
    {
        return NULL;
    }

    ptSrcMsg =( T_MSGOMC_HDR *)ptMsg;
    dwHeadLen = sizeof(T_MSGOMC_HDR);
    /* 重新计算本板对应版本的消息头大小和送来的消息体 */
    /* 动态申请内存 */
    pucDestMsg = XOS_GetUB(dwHeadLen + ptSrcMsg->wMsgSize); 
    /* 拷贝消息头 */
    memcpy(pucDestMsg, ptSrcMsg, dwHeadLen);
    /* 整理消息体 */
    memcpy(pucDestMsg + dwHeadLen, ptSrcMsg +dwHeadLen, ptSrcMsg ->wMsgSize); 
    /* 释放互斥信号量资源 */
    XOS_ProcessSemV(g_dwSemId); 

    return pucDestMsg;
}
\end{lstlisting}

\subsection{\lookforerrorandfix{6}}

\begin{lstlisting}
RM_ERROR_NO RM_SG_GetSignalGroupInfo( 
                    IN SIGNAL_GROUP_ACCESS_REQ *ptReq,
                    OUT SIGNAL_GROUP_ACCESS_ACK *ptAck )
{
    SIGNAL_GROUP   *ptSignalGroup         = NULL;
    NEXTHOP        *ptNextHopTemp         = NULL;
    NEXTHOP        *ptNextHopTrackTemp    = NULL;
    NEXTHOP        *ptNextHop             = NULL;
    NEXTHOP        *ptTrackNextHop        = NULL;
    NEXTHOP        *ptOtherNextHop        = NULL;
    NEXTHOP        *ptTrackOtherNextHop   = NULL;
    WORD            wRet                  = RC_OK;

    ptAck->bResult = FALSE; 
    
    if ( NULL == ptReq || NULL == ptAck )
    {
        return RM_ERROR_NULL_POINTER_ID;
    }

    memset((BYTE *)ptAck, 0, sizeof(SIGNAL_GROUP_ACCESS_ACK));
    ptSignalGroup = 
        RM_Signal_Group_GetSignalGroupByID( ptReq->wSignalGroupID );
    if ( NULL == ptSignalGroup )
    {
        return RM_ERROR_SIGNAL_GROUP_NOT_EXIST;
    }

    wRet = RM_SG_FindNextHopByID( ptSignalGroup->wTrackNextHop, 
                                  &ptNextHopTemp,
                                  &ptNextHopTrackTemp);
    if ( RC_OK == wRet )
    {
        ptNextHop      = ptNextHopTemp;
        ptTrackNextHop = ptNextHopTrackTemp;
    }

    wRet = RM_SG_FindNextHopByID( ptSignalGroup->wTrackOtherNextHop, 
                                  &ptNextHopTemp,
                                  &ptNextHopTrackTemp);
    if ( RC_OK == wRet )
    {
        ptOtherNextHop      = ptNextHopTemp;
        ptTrackOtherNextHop = ptNextHopTrackTemp;
    }
    
    ptAck->wNextHopID     = ptNextHop->wID;    
    memcpy( (BYTE *)&ptAck->tNextHopIP, 
            (BYTE *)&ptNextHop->tNextHopIp.tIpAddr,
            sizeof( RM_IPA ) );
    
    if ( NULL != ptTrackNextHop )
    {
        ptAck->wTrackNextHopID     = ptTrackOtherNextHop ->wID;
        memcpy( (BYTE *)&ptAck->tTrackNextHopIP, 
                (BYTE *)& ptTrackOtherNextHop->tNextHopIp.tIpAddr,
                sizeof( RM_IPA ) );
    }

    if ( NULL != ptTrackOtherNextHop  &&  ptOtherNextHop != NULL )
    {
        ptAck->wOtherNextHopID     = ptOtherNextHop->wID;   
        memcpy( (BYTE *)&ptAck->tOtherNextHopIP, 
                (BYTE *)&ptOtherNextHop->tNextHopIp.tIpAddr,
                sizeof( RM_IPA ) );
        ptAck->wTrackOtherNextHopID = ptTrackOtherNextHop->wID;
        memcpy( (BYTE *)&ptAck->tTrackOtherNextHopIP, 
                (BYTE *)& ptTrackOtherNextHop->tNextHopIp.tIpAddr,
                sizeof( RM_IPA ) );
    }

    ptAck->bResult = TRUE;
    
    return RM_ERROR_NO_ERROR;
}
\end{lstlisting}

\subsection{\lookforerrorandfix{9}}

\begin{lstlisting}
int filecpy(char* pFileFrom, char* pFileTo)
{
    int iFileLenth=0;
    char* pBuf=NULL;
    FILE* fpFileFrom,*fpFileTo; 
    if(strlen(pFileFrom)==0  || strlen(pFileTo)==0)
    {
        return ERROR;
    }
    if (pFileFrom != NULL|| !pFileTo!= NULL)
    {  
        return ERROR; 
    }
    fpFileFrom=fopen(pFileFrom,"rb");
    if(fpFileFrom==NULL)
    {
        return ERROR;
    }
    fseek(fpFileFrom,0,SEEK_END);
    iFileLenth =ftell(fpFileFrom);
    if(iFileLenth>0)
    {
        pBuf=(char*)malloc(iFileLenth);
        if(!pBuf) 
        {
            return ERROR; 
        }
    }
    fseek(fpFileFrom,0,SEEK_SET);
    fread(pBuf,iFileLenth,1,fpFileFrom);
    fpFileTo=fopen(pFileTo,"w+");
    if(fpFileTo==NULL)
    {
        return ERROR; 
    }
    fwrite(pBuf,iFileLenth,1,fpFileTo);
    fclose(fpFileFrom);
    fclose(fpFileTo);
    free(pBuf);
    return 0;
}
\end{lstlisting}

\section{填空题(共20分)。}

\subsection{下列程序运行后的输出结果是 \myblank{5em},\myblank{5em}。(2分)}
\begin{lstlisting}
void main() 
{ 
    char a[7]= "a0\\\0a0\0"; 
    int i,j; 
    i=sizeof(a);
    j=strlen(a); 
    printf("%d ,%d\n",i,j); 
}
\end{lstlisting}

\subsection{下列程序运行后的输出结果是 \myblank{5em},\myblank{5em}。(2分)}
\begin{lstlisting}
void main()
{
    int a[5] = {0,1,2,3,4};
    int *ptr = (int *)(&a+1);
    printf("%d,%d", *(a+1), *(ptr-1));
}
\end{lstlisting}

\subsection{定义整型变量int a;,按下列要求写出赋值表达式(说明:整数a从最低位到最高位,依次为第1到32位):(2分)}
将a的第2位和第5位 置为1,其它的值不变\myblank{15em}。

将a的第2位和第5位 置为0,其它的值不变\myblank{15em}。

\pagebreak
\subsection{不考虑内存申请失败的情况,则下列程序的运行结果是\myblank{5em}。(1分)}

\begin{lstlisting}
void main() 
{ 
    char *p1 = "name";  
    char *p2 = NULL; 
    p2 = (char*)malloc(20);
    memset (p2, 0, 20); 
    while(*p2++ = *p1++);
    printf("%s\n",p2);
    free(p2);
}
\end{lstlisting}

\subsection{\lstinline{strlen("");} \lstinline{strlen(NULL);} \lstinline{strlen("0");} \lstinline{sizeof(0);}执行结果分别为:\myblank{5em},\myblank{5em},\myblank{5em},\myblank{5em}。(4分)}

\subsection{设有:\lstinline{int a=10,b=20,c=30,d=40,m=50,n=60;}执行\lstinline{(m=a>b)&&(n=c>d)}后n的值为\myblank{5em}。(1分)}

\subsection{下列程序运行后的输出结果是 \myblank{5em},\myblank{5em}。(2分)}
\begin{lstlisting}
void getSubSystems (int i,char* pName)
{
    char* pSystems[4] ={"SIG","H248","DB","SIP"};
    pName= pSystems[i];
    return; 
}
void GetYourName(char ** ptrStr)
{
    static char sString[] = "XXX";
    *ptrStr = sString;
}
int main(int argc, char* argv[])
{
    char* pName="YYY";
    getSubSystems(3,pName); 
    printf("%s   \n",pName);
    pName = NULL;
    GetYourName(&pName);
    printf("%s   \n",pName);
    return 0;
}
\end{lstlisting}

\subsection{下列程序运行后的输出结果是 \myblank{5em},\myblank{5em},\myblank{5em}。(3分)}
\begin{lstlisting}
#define IPV4_MAX 20

char * iptostring(unsigned char *Ipv4Addr)
{
    static char bTemp[IPV4_MAX];
    snprintf( bTemp, IPV4_MAX,
              "%d.%d.%d.%d",
              Ipv4Addr[0],
              Ipv4Addr[1],
              Ipv4Addr[2],
              Ipv4Addr[3]);
    return bTemp;
}
void main()
{
    char   tOuterIP1[4] = {10,33,32,21};
    char   tOuterIP2[4] = {10,44,32,21};
    char   tOuterIP3[4] = {10,55,32,21};
    printf("%s,%s,%s",
           iptostring(&tOuterIP1),
           iptostring(&tOuterIP2),
           iptostring(&tOuterIP3));
}
\end{lstlisting}

\subsection{性能提升的方法主要有:(3分)}
\myblank{30em},

\myblank{30em},

\myblank{30em}。

\begin{minipage}{1\columnwidth}
\vspace{40ex}
\end{minipage}

\pagebreak
\begin{minipage}{1\columnwidth}
\section{编程题(共20分)。}

\subsection{给定单链表头结点,删除链表中\textit{倒数}第k个结点。(10分)}
节点定义为:
\begin{lstlisting}
struct NODE
{
    int nValue;
    struct NODE * next;
};
\end{lstlisting}
最后一个节点的\lstinline{next = NULL}。

函数原型:\lstinline{NODE * DeleteNode(NODE * pList,int k);}

返回值:链表头节点。
\end{minipage}

\begin{minipage}{1\columnwidth}
\vspace{70ex}
\end{minipage}

\pagebreak

\begin{minipage}{1\columnwidth}
\subsection{设计一个算法将字符串对调,不能借用其他存储空间。(10分)}
函数原型:\lstinline{bool ConverseString( char * chStr );}\lstinline{chStr}指向待转换的字符串。

返回值:转换是否成功。

举例:字符串\lstinline{chStr = "abcdefg"}转换后变为\lstinline{chStr = "gfedcba"}。
\end{minipage}
\end{document}
