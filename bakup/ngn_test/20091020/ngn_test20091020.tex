\documentclass  [11pt,twocolumn]{article}
\title{NGN C Test}
\author{}
\date{}

\usepackage[a3paper,landscape]{geometry}
\geometry{left=3em}
\geometry{columnsep=15em}
\geometry{right=15em}

\usepackage{fontspec,xunicode,xltxtra}
% \setmainfont[Mapping=tex-text]{Times New Roman}
\setmainfont[Mapping=tex-text]{Arial}
\setsansfont[Mapping=tex-text]{Arial}
% \setmonofont[Mapping=tex-text]{Courier New}
\setmonofont[Mapping=tex-text]{Times New Roman}

\usepackage{xeCJK}
% \setCJKmainfont[ItalicFont={Adobe Kaiti Std}]{Adobe Song Std}
% \setCJKmainfont[ItalicFont={Adobe Kaiti Std}]{Adobe Kaiti Std}
\setCJKmainfont[ItalicFont={Adobe Kaiti Std}]{Adobe Heiti Std}
\setCJKsansfont{Adobe Heiti Std}
% \setCJKsansfont{Microsoft YaHei}
\setCJKmonofont{Adobe Heiti Std}
\punctstyle{banjiao}

\usepackage{calc}
\usepackage[]{geometry}
% \geometry{paperwidth=221mm,paperheight=148.5mm}
% \geometry{paperwidth=9.309in,paperheight=6.982in}
\geometry{paperwidth=7.2cm,paperheight=10.8cm}
% \geometry{twocolumn}
\geometry{left=5mm,right=5mm}
\geometry{top=5mm,bottom=5mm,foot=5mm}
% \geometry{columnsep=10mm}
\setlength{\emergencystretch}{3em}


\usepackage{indentfirst}

%生成PDF的链接
\usepackage{hyperref}
\hypersetup{
    % bookmarks=true,         % show bookmarks bar?
    bookmarksopen=true,
    pdfpagemode=UseNone,    % options: UseNode, UseThumbs, UseOutlines, FullScreen
    pdfstartview=FitB,
    pdfborder=1,
    pdfhighlight=/P,
    pdfauthor={wuxch},
    unicode=true,           % non-Latin characters in Acrobat’s bookmarks
    colorlinks,             % false: boxed links; true: colored links
    linkcolor=blue,         % color of internal links
    citecolor=blue,        % color of links to bibliography
    filecolor=magenta,      % color of file links
    urlcolor=cyan           % color of external links
}
\makeindex

\usepackage[dvips,dvipsnames,svgnames]{xcolor}
\definecolor{light-gray}{gray}{0.95}

\usepackage{graphicx}
\usepackage{wrapfig}
\usepackage{picinpar}

\renewcommand\contentsname{目录}
\renewcommand\listfigurename{插图}
\renewcommand\listtablename{表格}
\renewcommand\indexname{索引}
\renewcommand\figurename{图}
\renewcommand\tablename{表}

\usepackage{caption}
\renewcommand{\captionfont}{\scriptsize \sffamily}
\setlength{\abovecaptionskip}{0pt}
\setlength{\belowcaptionskip}{0pt}

\graphicspath{{fig/}}

\usepackage{fancyhdr}

% \usepackage{lastpage}
% \cfoot{\thepage\ of \pageref{LastPage}}

% 嵌入的代码显示
% \usepackage{listings}
% \lstset{language=C++, breaklines, extendedchars=false}
% \lstset{basicstyle=\ttfamily,
%         frame=single,
%         keywordstyle=\color{blue},
%         commentstyle=\color{SeaGreen},
%         stringstyle=\ttfamily,
%         showstringspaces=false,
%         tabsize=4,
%         backgroundcolor=\color{light-gray}}

\usepackage[sf]{titlesec}
\titleformat{\section}{\normalsize\sffamily\bf\color{blue}}{\textsection~\thesection}{.1em}{}
\titleformat{\subsection}{\normalsize\sffamily}{\thesubsection}{.1em}{}
\titlespacing*{\section}{0pt}{1ex}{1ex}
\titlespacing*{\subsection}{0pt}{0.2ex}{0.2ex}

\usepackage{fancyhdr}
\usepackage{lastpage}
\fancyhf{}
\lhead{}
\rhead{}
\chead{\scriptsize{\textsf{蜗居}}}
\cfoot{\scriptsize{\textsf{第 \thepage ~页,共 \pageref*{LastPage} 页}}}


% \usepackage{enumitem}
% \setitemize{label=$\bullet$,leftmargin=3em,noitemsep,topsep=0pt,parsep=0pt}
% \setenumerate{leftmargin=3em,noitemsep,topsep=0pt,parsep=0pt}

% \setlength{\parskip}{1.5ex plus 0.5ex minus 0.2ex}
\setlength{\parskip}{2.0ex plus 0.5ex minus 0.2ex}

% \setlength{\parindent}{5ex}
\setlength{\parindent}{0ex}

% \usepackage{setspace}
\linespread{1.25}

% 英文的破折号--不明显,使用自己画的线。
\newcommand{\myrule}{\hspace{0.5em}\rule[3pt]{1.6em}{0.3mm}\hspace{0.5em}}

\rhead{\scriptsize{\textsf{NGN软件开发一部C语言内部测验 2009 10}}}

\begin{document}
% \begin{center}
\begin{minipage}{0.9\linewidth}
\blk{部门}\blk{科室}\blk{姓名}\blk{工号}
\end{minipage}
% \end{center}

% \begin{flushright}
% {\small{\textit{考试时长:3小时,总分:100分}}}
% \end{flushright}

\section{单选题 (每题3分,共计36分)}
\subsection{有以下程序}
\begin{lstlisting}
main() 
{ 
    int a,b,d=25; 
    a=d/10%9;b=a&&(-1); 
    printf("%d,%d\n",a,b); 
} 
\end{lstlisting}
程序运行后的输出结果是\myblank{3em}

\myoptions{6,1}{2,1}{6,0}{2,0}

\subsection{有以下程序}
\begin{lstlisting}
main() 
{ 
    char a[7]="a0\0a0\0"; int i,j; 
    i=sizeof(a);
    j=strlen(a); 
    printf("%d %d\n",i,j); 
} 
\end{lstlisting}
程序运行后的输出结果是\myblank{3em}

\myoptions{2 2}{7 6}{7 2}{6 2}

\subsection{有以下程序}
\begin{lstlisting}
main() 
{ 
    int a[3][3],*p,i; 
    p=&a[0][0]; 
    for(i=0;i<9;i++)p[i]=i; 
    for(i=0;i<3;i++)printf("%d",a[1][i]); 
}
\end{lstlisting}
程序运行后的输出结果是\myblank{3em}

\myoptions{0 1 2}{1 2 3}{2 3 4}{3 4 5}

\subsection{有以下程序}
\begin{lstlisting}
#define N 20 
fun(int a[],int n,int m) 
{
    int i,j; 
    for(i=m;i>=n;i--)a[i+1]=a[i]; 
} 
main() 
{ 
    int i,a[N]={1,2,3,4,5,6,7,8,9,10}; 
    fun(a,2,9); 
    for(i=0;i<5;i++)printf("%d",a[i]); 
} 
\end{lstlisting}
程序运行后的输出结果是\myblank{3em}

\myoptions{10234}{12344}{12334}{12234}

\subsection{有以下程序}
\begin{lstlisting}
main() 
{ 
    int a[3][2]={0},(*ptr)[2],i,j; 
    for(i=0;i<2;i++) 
    {
        ptr=a+i;scanf("%d",ptr);ptr++;
    } 
    for(i=0;i<3;i++) 
    {
        for(j=0;j<2;j++)
        {
            printf("%2d",a[i][j]); 
        }
        printf("\n"); 
    }
} 
\end{lstlisting}
若运行时输入:1 2 3 回车,则输出结果是\myblank{3em}

\myoptions{产生错误信息}{1 0\newline 2 0\newline 0 0}{1 2\newline 3 0\newline 0 0}{1 0\newline 2 0\newline 3 0}

\subsection{有以下程序}
\begin{lstlisting}
prt(int *m,int n)
{
    int i;
    for(i=0;i<n;i++) *m++ = *m+1;
}
main() 
{ 
    int a[]={1,2,3,4,5},i;
    prt(a,5);
    for(i=0;i<5;i++)
        printf("%d,",a[i]);
} 
\end{lstlisting}
程序运行后的输出结果是\myblank{3em}

\myoptions{1,2,3,4,5,}{2,3,4,5,6,}{3,4,5,6,7,}{2,3,4,5,1,}

\subsection{有以下程序}
\begin{lstlisting}
#define P 3 
void F(int x)
{
    return(P*x*x);
} 
main() 
{
    printf("%d\n",F(3+5));
} 
\end{lstlisting}
程序运行后的输出结果是\myblank{3em}

\myoptions{192}{29}{25}{编译出错}

\subsection{有以下程序}
\begin{lstlisting}
main() 
{
    int c=35;
    printf("%d\n",c&c);
} 
\end{lstlisting}
程序运行后的输出结果是\myblank{3em}

\myoptions{0}{70}{35}{1}

\subsection{有以下程序}
\begin{lstlisting}
main() 
{
    int a=1,b; 
    for(b=1;b<=10;b++) 
    {
        if(a>=8)
        break; 
        if(a%2==1)
        {
            a+=5;
            continue;
        } 
        a-=3; 
    } 
    printf("%d\n",b); 
} 
\end{lstlisting}
程序运行后的输出结果是\myblank{3em}

\myoptions{3}{4}{5}{6}

\subsection{有以下程序}
\begin{lstlisting}
main() 
{
    char s[]="159",*p; 
    p=s; 
    printf("%c",*p++);
    printf("%c",*p++); 
} 
\end{lstlisting}
程序运行后的输出结果是\myblank{3em}

\myoptions{15}{16}{12}{59}

\subsection{有以下程序}
\begin{lstlisting}
main() 
{
    int num[4][4]={{1,2,3,4},{5,6,7,8},
                   {9,10,11,12},{13,14,15,16}};
    int i,j; 
    for(i=0;i<4;i++) 
    {
        for(j=0;j<=i;j++)printf("%4c",' '); 
            for(j=_____;j<4;j++)printf("%4d",num[i][j]); 
                printf("\n"); 
    } 
} 
\end{lstlisting}
若要按以下形式输出数组右上半三角
\begin{lstlisting}
1 2 3 4
6 7 8
11 12
16
\end{lstlisting}
则在程序下划线处应填入的是\myblank{3em}

\myoptions{i-1}{i}{i+1}{4-i}

\subsection{有以下程序}
\begin{lstlisting}
void point(char *p)
{
    p+=3;
} 
main() 
{
    char b[4]={'a','b','c','d'},*p=b; 
    point(p);
    printf("%c\n",*p); 
} 
\end{lstlisting}
程序运行后的输出结果是\myblank{3em}

\myoptions{a}{b}{c}{d}

\section{填空题 (每题3分,共计18分)}
\subsection{以下程序运行后的输出结果是\myblank{10em}}
\begin{lstlisting}
#define S(x) 4*x*x+1 
main() 
{ 
    int i=6,j=8; 
    printf("%d\n",S(i+j)); 
} 
\end{lstlisting}

\subsection{以下程序运行后的输出结果是\myblank{10em}}
\begin{lstlisting}
main() 
{ 
    int a,b,c; 
    a=10;b=20;c=(a%b<1)||(a/b>1); 
    printf("%d %d %d\n",a,b,c); 
} 
\end{lstlisting}

\subsection{以下程序运行后的输出结果是\myblank{10em}}
\begin{lstlisting}
fun(int a) 
{
    int b=0;static int c=3; 
    b++;c++; 
    return(a+b+c); 
} 
main() 
{
    int i,a=5; 
    for(i=0;i<3;i++)
        printf("%d %d ",i,fun(a)); 
    printf("\n"); 
} 
\end{lstlisting}

\subsection{以下程序运行后的输出结果是\myblank{10em}}
\begin{lstlisting}
struct NODE 
{
    int k; 
    struct NODE *link; 
}; 
main() 
{ 
    struct NODE m[5],*p=m,*q=m+4; 
    int i=0; 
    while(p!=q) 
    {
        p->k=++i;p++; 
        q->k=i++;q--; 
    } 
    q->k=i; 
    for(i=0;i<5;i++)
        printf("%d",m[i].k); 
    printf("\n"); 
} 
\end{lstlisting}

\subsection{以下程序运行后的输出结果是\myblank{10em}}
\begin{lstlisting}
#include "stdio.h"
int main()
{
    int a;
    int *p;
    p = &a;
    *p = 0x500;
    a = (int )(*(&p));
    a = (int )(&(*p));
    if(a == (int)p)
        printf("equal !\n");
    else
        printf("not equal !\n");
}
\end{lstlisting}

\subsection{以下程序的输出显示}
\begin{lstlisting}
unsigned int x = 0x123;
unsigned char *puch = (unsigned char *)&x;

printf("%x,%x,%x,%x\n", puch[0],puch[1],puch[2],puch[3]);
\end{lstlisting}

\begin{minipage}{1\columnwidth}
\vspace{2ex}
BIG-ENDIAN   情况下输出:\myblank{10em}
\end{minipage}

\begin{minipage}{1\columnwidth}
\vspace{2ex}
LITTLE-ENDIAN情况下输出:\myblank{10em}
\vspace{2ex}
\end{minipage}

\section{改错题 (共计26分)}
\subsection{\lookfordiff{5}}
\begin{lstlisting}
BOOL H248_GetPtFrmStr(H248_SDP_FORMAT  tSdpFmt,BYTE * pbVal)
{
    BYTE bFmtStr[H248_MAX_FMT_STRING_LEN] ;

    if((H248_UNUSE ==tSdpFmt.bUseFlag)||(NULL == pbVal))
    {
        return H248_BL8_FALSE;
    }
    H248_strncpy(bFmtStr,tSdpFmt.bStrContent,H248_MAX_FMT_STRING_LEN,tSdpFmt.bStrLen);
    H248_StrTurnToUpper(bFmtStr);
    if(gdwdebug == 7 )
    {
        return H248_BL8_FALSE;
    }
    if(H248_StrCmp(bFmtStr,"G721"))
    {
        *pbVal = PACK_TYPE_G721 ;
        return H248_BL8_TRUE;
    }
    else if(H248_StrCmp(bFmtStr,"G722"))
    {
        *pbVal =  PACK_TYPE_G722;
        return H248_BL8_TRUE;
    }
    return H248_BL8_FALSE;
}
void H248_StrTurnToUpper(char *str)
{
    if(str == NULL)
    {
        return;
    }

    while( (*str) != 0)
    {
        if( (*str) > 0x60 && (*str) < 0x7B)
        {
            (*str) = (*str) - 0x20;
        }
        str++;
    }
}
BOOL H248_strncpy(char *pDest, char *pSrc, DWORD dest_Len, DWORD copy_Len)
{
    if((pDest == NULL)||(pSrc == NULL))
    {
        H248_INCR_VAL(gvH248_Statis.tError.tTools.dwStrCpy, H248_COUNT_STEP);
        return FALSE;
    }
    if((copy_Len >= dest_Len)||(dest_Len <= 0)||(copy_Len <= 0))
    {
        H248_INCR_VAL(gvH248_Statis.tError.tTools.dwStrCpy, H248_COUNT_STEP);
        return FALSE;
    }
    memcpy((void *)pDest, (void *)pSrc, (size_t)copy_Len);
    *(pDest+copy_Len) = '\0';
    return TRUE;
}
\end{lstlisting}
\newpage
\subsection{\lookfordiff{5}}
\begin{lstlisting}
WORD32 CSCF_REG_PrintInfoToOMC(BYTE *pbData,WORD32 *pdwDataLen,
                               T_CSCF_MML_COMMANDG *ptMMLCommand)
{
    WORD32 j=0;
    JID    tRegJid={0};
    WORD32 dwRetCode = CMS_SUCCESS;
    WORD16 wCurStat  = 0;

    if ((pbData==NULL)||(pdwDataLen==NULL)||(ptMMLCommand==NULL))
    {
        return CSCF_MML_FAILURE;
    }
    if (CMS_SUCCESS!=CSC_XOS_GetJID(JOBTYPE_CSF_REG, 1, &tRegJid))
    {
        j+=snprintf((CHAR *)pbData,
                    128,
                    "CSC_REG OSS_GetPIDByName Error,\\
OMC COMMAND FAIL\n");
        pbData[j]='\0';
        *pdwDataLen=strlen((LPSTR)pbData);
        return CSCF_MML_FAILURE;
    }
    if( CMS_SUCCESS!=CSC_XOS_GetCurState(&wCurStat) )
    {
        /* reg 进程如果不等于 WORK/SLAVE 状态,直接返回进程未起来 */
        j+=snprintf((CHAR *)pbData,
                    128,
                    "CSC_REG CSC_XOS_GetCurState Error,\\
OMC COMMAND FAIL\n");
        pbData[j]='\0';
        *pdwDataLen=strlen((LPSTR)pbData);
        return CSCF_MML_FAILURE;
    }
    ......
    return CSCF_MML_SUCCESS;
}
\end{lstlisting}

\subsection{\lookfordiff{5}}
\begin{lstlisting}
#define SIP_FEATURES_SUPPORT_PRACK         0x00000001
#define SIP_FEATURES_SUPPORT_UPDATE        0x00000002
#define SIP_FEATURES_SUPPORT_EARLYSESSION  0x00000010
#define SIP_FEATURES_SUPPORT_PRECONDITION  0x00000020
#define SIP_FEATURES_REQUIRE_PRACK         0x00010000
#define SIP_FEATURES_REQUIRE_EARLYSESSION  0x00100000
#define SIP_FEATURES_REQUIRE_PRECONDITION  0x00200000

void C_IMSBCM_FillStimulatedFeatures(SIP_FEATURES* ptDestFeatures,
                                     SIP_FEATURES* ptSourceFeatures,
                                     BOOL blActiveState)
{
	memcpy(ptDestFeatures, ptSourceFeatures, sizeof(SIP_FEATURES));
    if (ptDestFeatures->dwFeatures & SIP_FEATURES_REQUIRE_PRECONDITION)
    {
        ptDestFeatures->dwFeatures &= !SIP_FEATURES_REQUIRE_PRECONDITION;
        ptDestFeatures->dwFeatures |= SIP_FEATURES_SUPPORT_PRECONDITION;
    }
    if (ptDestFeatures->dwFeatures & SIP_FEATURES_REQUIRE_PRACK)
    {
	    ptDestFeatures->dwFeatures &= !SIP_FEATURES_REQUIRE_PRACK;
    }
    ptDestFeatures->dwFeatures |= SIP_FEATURES_SUPPORT_PRACK;
    if (TRUE_B8 == blActiveState)
    {
        ptDestFeatures->dwFeatures = SIP_FEATURES_SUPPORT_EARLYSESSION;
    }
    ptDestFeatures->bType |= SIP_FEATURES_CREATE;
    return;
}
\end{lstlisting}

\subsection{\lookfordiff{5}}
\begin{lstlisting}
BOOL8  D_DR_AlarmSendComInfo(WORD wAlarmCode,void * pAlarmInfo,BYTE bLength)
{
    if(NULL == pAlarmInfo)
    {
        bLength = 0;
    }
	D_DR_ALOCK();  /* 加锁 */
	if(tDRAlarmPool.wCount)
	{
        /* 先发送告警池的告警 */
		if(!D_DR_SendAlarmInPool())
		{
		    /* 把最新的告警加入告警池 */
			D_DR_ForceAddAlarm2Pool(wAlarmCode,pAlarmInfo,bLength);
		    return FALSE;
	    }
    }
    if(!D_DR_BaseSendAlarm(wAlarmCode,pAlarmInfo,bLength))
    {
        D_DR_ForceAddAlarm2Pool(wAlarmCode,pAlarmInfo,bLength);
    }
    D_DR_AUNLOCK();/* 解锁 */
    return TRUE;
}
\end{lstlisting}

\subsection{\lookfordiff{6}}
\begin{lstlisting}
#define DB_SUBSYS_H248  0xff1c

DBBOOL  _func_get_all_tuplehandle_by_ipaddr(
                      _db_t_database_handle dbHandle,
                      _db_t_table_handle    tableHandle,
                      T_IPAddr              *pIPAddr,
                      LP_T_DB_HANDLE_LIST   ptHandleList)
{
    _DB_RET  bret,ret;
    T_IPAddr tSubNet = {0};
    T_IPAddr tMask = {0};
    WORD     wLength=0;
    BYTE     IpLength =0;
    DWORD    i, j;
    BYTE     bMskSize =0;
    BYTE     abMaskSizeList[DB_TMPRESULT_M] = {0};
    BYTE     bEqualFlag=0;

    if (pIPAddr == NULL || ptHandleList == NULL)
    {
        XOS_SysLog(DBS_PRNLEVEL_ERROR,
                   "[PSSDB]:Line:%d  _func_get_all_tuplehandle_by_ipaddr\\
( :%s failed \n",
                   __LINE__,
                   DB_SUBSYS_H248 );
        return FALSE;
    }
    ptHandleList->dwHandleNum = 0;
    /* 根据入参的数据库和表句柄,遍历该表 */
    bret=_db_skip_first_tuple(dbHandle,tableHandle, &htp_handle);
    while ((bret== _DB_FOUND) && (ptHandleList->dwHandleNum < DB_TMPRESULT_M) )
    {
        .......
        bEqualFlag = TRUE;
        /* ipv4 取四个字节,IPV6取16个字节,
           分别与 Mask 相与,如果等于子网 IP,
           记录下掩码的为 1 的个数,与当前的
           掩码的计数器比较,直到找到最长匹配的记录 */
        for (i=0; i< IpLength;i++)
        {
            /* IP 地址与掩码相与后不等于子网,
               将标志置为 FALSE ,不取当前记录 */
            if((pIPAddr->unIPAddrValue.
                    byIPV6Addr[i]&tMask.unIPAddrValue.byIPV6Addr[i]) !=
               (tSubNet.unIPAddrValue.
                    byIPV6Addr[i]&tMask.unIPAddrValue.byIPV6Addr[i]))
            {
                bEqualFlag = FALSE;
                break;
            }
        }
        if ( bEqualFlag == TRUE) /* IP 地址与掩码相与后等于子网的情况 */
        {
            ret=bCountOneBits(tMask.unIPAddrValue.byIPV6Addr,CSCF_IPV6ADDR_LEN,
                              &bMskSize);
            if (ret)
            {
                for (i = 0; i < ptHandleList->dwHandleNum; i++)
                {
                    if (bMskSize > abMaskSizeList[i])
                    {
                        /* 从 i 开始的数据后移一位 */
                        for (j = ptHandleList->dwHandleNum-1; j >= i ; j--)
                        {
                            abMaskSizeList[j+1]            = abMaskSizeList[j];
                            ptHandleList->atTupleHandle[j+1] = ptHandleList->atTupleHandle[j];
                        }
                        break;
                    }
                }
                /* 将新数据插入数组中 */
                abMaskSizeList[i] = bMskSize;
                ptHandleList->atTupleHandle[i] = htp_handle;
                ptHandleList->dwHandleNum++;
            }
            bMskSize =0;
        }
        bret = _db_skip_next_tuple(dbHandle,tableHandle,
                                   htp_handle,&htp_handle);
    }
    ......
}
\end{lstlisting}

\section{编程题 (共计20分)}

设计一个算法,判断一个算术表达式中的括号是否配对,算术表达式保存字符串中。在进行检测括号是否匹配的时
候,需要考虑到各种情况:

\begin{enumerate}
\item 匹配。例如:\lstinline$(())$
\item 左括号不匹配。例如:\lstinline$(()$
\item 右括号不匹配。例如:\lstinline$())$
\item 其他情况,例如:\lstinline$)($
\end{enumerate}

函数原型:\lstinline$int MatchBracket(const char *chExpr)$

参数说明:\lstinline$chExpr$表示待分析的算术表达式字符串

返回值说明:返回匹配结果,能够根据返回值区分各种不匹配情形。


\end{document}
