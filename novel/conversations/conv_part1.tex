\section{Remembering auto\_ptr}
\begin{flushright}--- A class with an auto\_ptr member: How can copying go wrong? Let us count the ways\ldots\end{flushright}

\textit{I had met Jeannine only the day before, in the crew rotation staging area now far below us. ``I'll always remember
  my first job,'' I told her after the flight attendant walked past, checking our belts.}

\textit{``What about it?''}

\textit{``The senior programmer on my project,'' I smiled, remembering. ``She was an odd duck. We called her the Guru.
  Management didn't like to assign new programmers to her team; I was the only one of four hires that year to last through
  the probation period.''}

\textit{Jeannine tilted her head and was about to ask a question when the last chime sounded and the rumbling acceleration
  took hold, cutting off conversation for a few minutes. At the end of the burn, as we left orbit, I told her the story of
  my second day at work.}

\begin{center}
- - - - -
\end{center}

We were programming in early C++. On the afternoon of my second day, tired of reading employee handbooks, I wrote a utility
class that contained a raw pointer as one of its members:

\begin{footnotesize}
\begin{lstlisting}
#include "xStruct.h" // definition of struct X

class xWrapper
{
  X* xItem;
public:
  xWrapper() : xItem(new X) { }
  ~xWrapper() { delete xItem; }
  void dump() { /* dumps xItem to cout */ }
};
\end{lstlisting}
\end{footnotesize}

Of course, the program using this class kept crashing intermittently with memory corruption, because I'd violated the Law of
the Big Three: Whenever you provide any one of a destructor, copy constructor or assignment operator, you will generally
need to provide all three\footnote{M. Cline, G. Lomow, and M. Girou. C++ FAQs, 2nd ed. (Addison-Wesley, 1999).}. ``So,''
said I to myself, said I, ``I have to handle copying and assignment. Simple\ldots auto\_ptr already has a copy constructor and
assignment operator, so I'll use that.'' (You've heard of the auto\_ptr in the original C++ library, right?)

Since the auto\_ptr automatically deletes the object it points to, all I had to do was change the type of xItem and remove
the delete statement in the destructor -- auto\_ptr would take care of the rest, right?

\begin{footnotesize}
\begin{lstlisting}
class xWrapper
{
  auto_ptr<X> xItem;
public:
  xWrapper() : xItem(new X) { }
  void dump() { /* dumps xItem to cout */ }
};
\end{lstlisting}
\end{footnotesize}

Unfortunately, my program was still crashing, this time because it was trying to dereference a null pointer. I had been
trying to puzzle out the problem for about half an hour when the Guru, thin as a rail, decided to walk by, carrying a thick
book open in one hand. She did that a lot\ldots show up at propitious times, I mean; I think it was some sort of prescient
thing. Downright spooky, actually.

``Uh, what're you reading?'' I asked her, pointing at her book to deflect attention away from my screen and hoping she would
go away.

The Guru blinked. ``The writings of Josuttis,'' she said softly, marked her page, and closed the book. ``What's that you
have there, young one?''

``I'm having problems with this wrapper class I'm writing,'' I admitted. ``I'm using an auto\_ptr member, but in my test
harness its pointer keeps getting reset to null for some reason.''

``Show me your writings,'' the Guru said. I showed her the screen. ``Ownership,'' she said immediately, after barely a
glance.

It was my turn to blink.

``Ownership, child; your problem is ownership semantics. No person can serve two masters, and no pointer can serve two
auto\_ptrs.''

Her words, although certifiably strange, made me realize my mistake. ``Okay, right,'' I nodded. ``When you copy an
auto\_ptr, the original one relinquishes ownership and gets reset to null. My xWrapper copy constructor reuses that default
behavior, so the original xWrapper object's auto\_ptr is being reset, and when I try to access it I'm dereferencing a null
pointer.''

``Correct,'' the Guru agreed. ``You have done well to reuse the true tools of the Standard, but you must take care with
them. For xWrapper, you must still manage xWrapper copying and assignment yourself.''

``But I can't implement them in terms of auto\_ptr's own versions because those won't do the ri--- Oh. I get it. I'll use
auto\_ptr's dereference operator to access the owned objects.'' I quickly wrote out the two functions:

\begin{footnotesize}
\begin{lstlisting}
xWrapper::xWrapper(const xWrapper& other)
  : xItem(new X(*other.xItem))
{ }

xWrapper\& xWrapper::operator=(const xWrapper &other)
{
  *xItem = *other.xItem;
}
\end{lstlisting}
\end{footnotesize}

``Hey, cool.'' This, I liked. ``I don't even need to check for self-assignment in the assignment operator.''

``That is correct.''

I should have stopped then and kept my mouth shut, but I wasn't that smart yet. ``auto\_ptr sure is easy to misuse. If only
it could have told me I was trying to transfer ownership when I didn't expect that to happen\ldots''

``Peace!'' interrupted the Guru. ``The fault is not with auto\_ptr in this instance. You should have said that you did not
want the auto\_ptr to be copied, had that been your desire.''

``But how? That's not possible.''

``Ah, but it is. Remember the blessings of const-correctness. The way to state that an auto\_ptr is immutable is to make it
const. Had you made the member a const auto\_ptr, the compiler would not have been able to silently perform the copy of the
xWrapper object. Alternatively, had you used something like the strict\_auto\_ptr from the Second Revisionist rendering of
Cline 30:12, the compiler would not have been able to mistakenly generate incorrect xWrapper copying and assignment. Of
course, in this case making it const would have been simpler and sufficient.''\footnote{ibid., FAQ 30.12, pages 426-8.}

She reopened her copy of Josuttis and resumed reading as she started to walk away, still talking to me absently, both she
and her voice gradually drifting away: ``A word of caution, my child\ldots auto\_ptr is a useful tool, but as you have
discovered, it is not a panacea. Read and meditate upon Josuttis chapter 4\footnote{N. Josuttis, The C++ Standard Library
  (Addison-Wesley, 1999).}. You must never instantiate a Standard container of auto\_ptr, such as vector$<$auto\_ptr$>$,
because auto\_ptr does not meet the copying and assignment requirements of the Standard. Furthermore, never attempt to use
auto\_ptr to point to an array of objects, for the auto\_ptr's destructor uses non-array delete to delete the object it
owns; for an array of objects, use a vector instead. The library\ldots''

But then she turned a corner, and was gone. It was only my second day; I wondered, not idly, if I ought to update my
r\'esum\'e while I could still pretend this job had never happened.

\wuxchcenterline

\textit{``Weird lizard,'' Jeannine opined, sipping coffee as we exited the Terran local traffic control area, still gaining
  velocity. ``So, did you leave?''}

\textit{``She was, but no. I'm not sure why,'' I added honestly. ``A few run-ins like that, and I was ready to leave during
  probation like the rest. I guess he grew on me, though. Didn't you ever work with a quirk like that?''}

\textit{``Mmm. Some. I suppose.''}

\textit{It was not the last time I would speak with Jeannine about the Guru, or about more pleasant things.}

\pagebreak
\section{Null References}
\begin{flushright}--- Can references be null? Case in point: A factory that returns a pointer to the newly created object.\end{flushright}
\textit{Aboard ship, the last thing you want to feel is wind. Jeannine and I happened to be unlucky enough to be the closest
  to the incident; straining together, we managed to wrestle the bulky door shut and seal it, isolating the breached
  compartment. As we leaned against the door, breathing deeply, the klaxons stopped sounding.}

\textit{``What in tarnation was that?'' the watch officer's voice demanded shrilly over the comlink.}

\textit{``Just a small penetration, sir,'' Jeannine responded. ``We sealed off the compartment. All clear, no casualties.''}

\textit{``You sealed off the compartment? Why didn't it seal automatically? Didn't we just refit that door?''}

\textit{Jeannine and I glanced at each other. ``Uh, sir, it seems that the maintenance crews assumed the new locking
  mechanism would be fine, because it was brand new from the factory, so they didn't check it. It was fine, only they didn't
  take off all the packing materials and it couldn't seal by itself.''}

\textit{There was a pause, then: ``All right, good work. There's a damage crew on the way.'' Then in the background, just
  before the comlink was broken, we heard the XO's voice say: ``Mr. Johnson, have the chief of engineering see me in my
  cabin. Now. Tell Reilly to--''}

\textit{I grinned at Jeannine. ``I'd give a week's pay to hear that conversation. This reminds me of a code safety problem we had once, a long time ago\ldots''}

\wuxchcenterline

I had been on my first programming job for a couple of weeks, and I'd met the other programmers on my project. One of them
was Bob. He was the antithesis of the Guru in almost every way: Bob was smug, his code was difficult for others to maintain,
and he had a tendency to violate programming rules.

I was well into my first project, and Bob had been assigned to perform a code review on some stuff I'd checked in. He came
by my desk, holding a latte, and leaned against the partition. ``Your code is crashing with an access violation inside your
helper function,'' he said. ``You better fix it before the Space Cadet hears about it.''

``The Guru really is an excellent programmer, Bob,'' I bristled. True, the Guru was odd, and I was still uneasy around her.
Every time I talked to her, I kept thinking about updating my rsum. But Bob's attitude just rubbed me the wrong way somehow.

``Yeah, whatever,'' Bob waved his cup dismissively, sloshing a little. ``Anyway, your helper function is using a null. Let's
take a look.''

``But the helper function doesn't use any pointers,'' I frowned. ``Just a reference to an xWrapper class.'' I called up the
code in my text editor. It was something like this:

\begin{footnotesize}
\begin{lstlisting}
class xWrapper
{
  /* ... */
public:
  virtual void dump() { cout << name << endl; } 
};
void helper( xWrapper& w )
{
  w.dump(); 
  // ... do other stuff with w ... 
}
\end{lstlisting}
\end{footnotesize}

Bob leaned at the monitor. ``Yeah, it's crashing at that dump() statement,'' he said, scratching his nose.

``That's not possible.''

``Sure it is, Junior.'' Bob was really starting to bug me. ``All I do is get a brand new pointer from the factory, and
dereference it when I pass it to you. If that pointer's null, 'course, you have a null reference. You should check for
that.''

``Uh, well,'' I faltered, less certain now. After all, I was just fresh out of school, and Bob had several years of
experience. ``I didn't think that was possible to see if a reference was null.''

``All you do,'' Bob said, ``is check the address of the reference. Like I said, you want to be safe, you should check for
that\ldots''

``No,'' the Guru's quiet voice startled us both. Once again, she had appeared at the right moment, and now stood behind us,
an open tome in hand. ``There is no need of such an abomination. The Standard teaches that it is not possible to have null
references.''

``But it is possible, dearie,'' Bob insisted. ``We've just shown that. That's his problem.''

I wanted to smack him. The Guru merely favored him with a chill glare. ``No. You should check for that. Never dereference a
null pointer. Change your code: If the pointer is null, do not call helper(). Show me the working code this afternoon.
Begone, you instrument of malformed programs,'' she waved him away.

Bob blinked. But she was the senior programmer on the team, so he shut up and left.

``My apprentice,'' she continued to me alone, ``one of the major reasons to use a reference instead of a pointer is to free
you from the burden of having to test to see if it refers to a valid object. The only way to create a null reference would
be to dereference a null pointer -- an act clearly forbidden by the Holy Standard, well inside the realm of Undefined
Behavior.''

``Okay, but why tell only me? Why not tell Bob?''

She shook her head sadly. ``This he knows. I cannot teach him; he is drawn to Undefined Behavior. My apprentice, beware the
path to Undefined Behavior. Once you start down that path, it will dominate your life forever. It will consume your time, as
you try to track down and fix problems.''

I considered this. ``Well,'' I asked, ``what about accidentally creating an invalid reference? Say you pass an object to a
function, and the object you pass goes out of scope before the function is finished with it. How do you guard against
that?''

She shook her head. ``You cannot. In this respect, references and pointers are very similar. Consider this parable.'' The
Guru quickly scribbled out some code on my whiteboard:

\begin{footnotesize}
\begin{lstlisting}
T* f() 
{ 
  T t; 
  return &t;  // return a dangling pointer 
}
void f1( T* t ) 
{ 
  if( t ) 
    t->doSomething(); 
}

int main() 
{ 
  T *tPtr = f(); 
  f1( tPtr );   // evil 
}
\end{lstlisting}
\end{footnotesize}


``This f() quite clearly lives in infamy,'' she said, pushing a graying lock behind one ear, ``as it returns a pointer to a
local object that oversoon goes out of scope. Thus tPtr in main() points to an object that has been destroyed. f1() does the
best it can -- it checks for a null pointer. Some platforms also provide compiler-specific functions that can test whether a
pointer points to a valid region of memory. However, such functions would still not detect the case that, inside f1(), the
variable t does not point to a valid T object. It is impossible for f1() to be certain that t is valid; therefore as the
programmer writing f1(), you must have faith. Faith that your fellow programmers will do everything in their power to
provide you with a pointer to a valid object.''

``Just as you must assume that a non-null pointer is valid, you must assume that a reference is valid. You must have faith
in your fellow programmers.''

``Even Bob?'' I asked.

She nodded with an air of sadness, then drifted quietly away down the corridor, reading her tome.

\wuxchcenterline

\textit{``Did Bob last long?'' Jeannine asked.}

\textit{``Longer than he should have.'' I stopped smiling. ``Our company also built software that got used in pacemakers.
  You know, the devices they used to put inside people's chests. One time he checked in a piece of code that didn't verify
  all its preconditions, and the code got out into production. And\ldots''}

\textit{Jeannine lifted an eyebrow.}

\textit{I nodded soberly. ``Bob finally did get another job, in retail sales. He can be glad it happened before the
  programmer liability laws.'' Then the damage control team relieved us, and we got ready to go on watch.}

\textit{It was not the last time I would speak with Jeannine about the Guru, or about more pleasant things.}

\pagebreak
\section{Genesis}
\begin{flushright}--- How we got here from there: A colorfully rendered history of the C family of languages.\end{flushright}
\textit{I burst into the compartment. ``Hey, have you heard the rumors?''}

\textit{Jeannine looked up from her work. ``Hay is for horses. What rumors?'' }

\textit{``About Ganymede,'' I explained. ``I hear they found something under the ice. No sign of recent melting, either;
  it's something old, and definitely not natural.''}

\textit{That got her attention. She blinked and stretched. ``Really? And not ours?''}

\textit{``That's the buzz. Your neurons tingling yet?''}

\textit{``Sure, sure. Wow. What else?''}

\textit{``No more information yet,'' I admitted, letting myself drop down onto the couch. ``And it's just a rumor. But, man,
  wouldn't that be exciting, if something big happened on our tour? We'll be there on station in less than two months.''}

\textit{``In the ice\ldots?'' Jeannine considered this. ``Whether it's an artifact or not, it must be ancient. Makes you think
  about all the big questions, doesn't it? Our own race's history, where we came from, where we're going.''}

\textit{``Roots and directions, right. Y'know,'' I said, ``this reminds me of something that happened back when I was on my
  first job\ldots''}

\wuxchcenterline

It was a slow day, and I was talking with Wendy. She was the girl next door\ldots literally, I mean. Her cubicle was next to
mine.

``Y'know,'' I said, ``I heard that Stroustrup chose the name C++ because it was an incremental change to the C language -- C
incremented, right?''

``Right,'' Wendy said. ``But if you think about the semantics of the name, what does it mean?''

I puzzled for a moment, then shook my head. ``I dunno.''

``Well, it uses the post-increment, so it means 'take the C language, increment it, and use the original.' `` We both
laughed.

``A common jest.'' The Guru's voice startled us both. We turned, and I tensed, but then I saw an unusual thing: A twinkle in
the Guru's blue eyes. She added: ``But how well do you know the history of the language you are abusing?''

I relaxed again, quickly at ease. ``How do you mean?''

The Guru closed the book in her hand -- I recognized the cover of the D\&E\footnote{B. Stroustrup, The Design and Evolution
  of C++ (Addison-Wesley, 1994) ISBN 0201543303.}. She raised her head slightly, her eyes focusing off in the distance. Then
she began to recite, in almost a sing-song way, and her thin body seemed to resonate with the chanted words:

\begin{quote}

``In the beginning, computer languages were in chaos, and high-level computer languages were void, and without form. And from
his desk at Bell Labs, Richard Martin did look upon the computer languages, and saw that a high-level language was needed.
And he went forth, and developed BCPL, and he saw that it was good.

``And Ken Thompson did look upon BCPL, and saw that it was good, but it had to be smaller to run on a PDP-7. And Thompson
went forth, and did produce a new language, and he did call it B, which is a contraction of BCPL\footnote{see \url{http://cm.bell-labs.com/cm/cs/who/dmr/chist.html} for alternate etymologies of ``B''.}. And he saw that it was
good.

``But B was without types. And Dennis Ritchie saw that it was without types, and he too went forth, and extended the B
language. Thus Ritchie added structs, and added types to the language, and he called it C, for C is the next letter after B,
both in the alphabet and in 'BCPL.' And Ritchie saw that it was almost good, but he was not satisfied, and he continued to
toil in the sweat of his brow and to refine the language. And in 1978 did Brian Kernighan and Dennis Ritchie proclaim 'The C
Programming Language\footnote{B. Kernighan and D. Ritchie, The C Programming Language, 2nd edition (Prentice Hall, 1998) ISBN 0131103709.}.' And there was much rejoicing among the multitudes, and they saw that C was good. 'Yea, verily,
this language is good,' proclaimed the multitudes.

``And it came to pass that the multitudes began to use C. And new features were added, but not by all compiler vendors. And
the multitudes were dismayed, and cried out. 'Yea, verily, we need a Standard C,' they did cry. And ANSI did hear their cry.
And in 1989 did ANSI say, 'Behold, I bring you good tidings of great joy, which shall be unto all programmers. For unto you
this day is adopted a Standard for C, which is X3.159-1989.' And then did ISO take up the cry, and did proclaim ISO/IEC
9899-1990. And there was much rejoicing among the multitudes.

``Now it also came to pass, before the C Standard was adopted, that Bjarne Stroustrup did also labor to perfect the C
language. And Stroustrup labored to add Classes to C, and function argument type checking, and other features possessed of
niftiness. And so he went forth, and in 1980 did he proclaim 'C With Classes.' And there was much rejoicing and excitement
among the multitudes.

``But Stroustrup was not satisfied. And he made further changes to the language, and he did call it C++, which is to say, the
C language incremented. And so he went forth, and in 1986 proclaimed 'The C++ Programming Language,' and there was much
rejoicing among the multitudes\footnote{B. Stroustrup, The C++ Programming Language, Special Edition (Addison-Wesley, 2000) ISBN 0201700735.}.

``But as with all living things, the C++ language evolved. Templates, exception handling, and other features were added to
the language. And there was rejoicing among the multitudes.

``But also did the multitudes complain. For in those days, did various compiler vendors implement different solutions to
templates and exceptions, and other features. Even did some compiler vendors refuse to implement them. And so the ISO went
forth, and in 1998 -- which is to say, the sixth year of Clinton, while Chr��tien was prime minister of all Canada with the
possible exception of Qu��bec, and Lewinsky was celebrity of choice in the media, for Simpson was no more and there was no
news that year -- in the ninth month, on the first day of the month, did ISO proclaim, 'Behold, I bring you good tidings of
great joy, which shall be unto all programmers. For unto you this day is adopted a Standard for C++, which is ISO/IEC
14882:1998(E).' And ANSI did take up the cry, and proclaim the very same Standard, on the seventh month, on the
twenty-seventh day of the month, which was technically before the day on which ISO did proclaim it because that's just how
these things work out sometimes. And there was much rejoicing among the multitudes. 'Yea, verily, now maybe we shall get
some work done,' the multitudes did proclaim.

``Now it came to pass, during those days, that Patrick Naughton did work for Sun Microsystems. And Naughton was sore tried,
and sought to leave Sun. But the Company counter-offered, and did say unto him, 'Thou shalt have a team of developers at thy
disposal, if thou wilt but give us something cool.' And so was born the team known as Green.

``And Green team did seek solitude, and went into exile in the wilderness. And they did seek an Object Oriented Language to
be used in embedded devices, and they did start to modify C++. But C++ was too cumbersome for their needs, and so did they
create a new language based on C++, and they called it Oak, for James Gosling did look out his window and see an oak tree.
And the team looked upon Oak, and saw that it was good.

``Now in those days it also came to pass that the National Center for Supercomputing Applications did beget Mosaic, and thus
did they bring forth the World Wide Web. And there was much rejoicing among the multitudes. And then did Bill Joy seek to
give away the source code to Oak, and to make Oak useable in web browsers. And Sun looked upon the idea, and saw that it was
good. But the trademark Oak was already taken, and therefore did Sun call the new language Java. And so it came to pass that
Sun did proclaim the Java Programming Language. And there was much rejoicing among the multitudes. 'Yea, verily, now we have
another truly platform-independent language,' they did proclaim, 'we think, and it does really cool stuff.' "

\end{quote}

The Guru finished her recitation, and focused on me. ``And that, my young apprentice, is the story of the C family of
languages.'' She pushed a gray lock behind her ear, stood still for a moment in a respectful silence, then bowed her head
and silently walked away, reopening her D\&E.

I sat there, stunned, and turned to Wendy.

``Hey, don't look at me,'' Wendy shrugged. ``You'll get used to her. She really is the best programmer I've ever worked
with.''

I barely heard her. I was trying to remember where I'd put the floppy disk with my r\'esum\'e.

\textit{``What did you spend more time at -- doing work, or thinking about how to get away from her?'' Jeannine asked,
  smiling facetiously.}

\textit{``Yes,'' I smirked back. It was not the last time I would speak with Jeannine about the Guru, or about more pleasant
  things.}
\pagebreak

\section{So Who's the Portable Coder?}
\begin{flushright}--Writing portable code. Case in point: Scope of variables declared in a for-init-statement.\end{flushright}
\textit{``Hi, buster.'' Jeannine came over and sat down across from me in the mess. ``Hear any more rumors?''}

\textit{I glanced at the time just to make sure, but I wasn't due to go on for another half hour. ``Hi, busted. No, not yet.
  They're keeping a lid on it, whatever it is. You're off watch early?''}

\textit{``Brzenkowski couldn't sleep, so he relieved me. I owe him one.''}

\textit{``Thoughtful of him,'' I agreed around another mouthful of lunch. ``I really like him. He always does good work,
  makes sure it won't affect other people, understands how he fits into the big picture. I wish we could keep working with
  him after planetfall next month. Everyone should be that professional.''}

\textit{``Most people are,'' Jeannine smiled. ``You're no slouch yourself.''}

\textit{``Actually,'' I chuckled, ``it reminds me of Bob. You know, I've told you about him before, the other weird from my
  first job. Man, sometimes he'd do things that'd just make you fry a lobe. It got to the point where if someone did
  something dumb we'd say he 'pulled a Bob.' ''}

\textit{I proceeded to explain\ldots}

\wuxchcenterline

A voice beside me said: ``Hey, Junior, you broke the build.''

I sighed silently and closed my eyes. I'd been on the job nearly two months and already decided I hated the Guru's calling
me ``my apprentice'' less than anything this rotten programmer, Bob, might call me -- especially ``Junior.'' I counted to
ten, then answered without turning: ``What are you talking about, Bob?''

``Your latest code checkin,'' Bob responded amiably, and sipped at his latt\'e. ``My compiler barfs on it. It complains that
an iterator is already defined. You broke the build, man. Better fix it before Her Holiness finds out.'' I winced; as
inexperienced as I was, I knew that breaking the build was A Bad Thing. I could just imagine the Guru's reaction: Remember,
my young apprentice, the second commandment of team membership: Thou shalt not break the build.

``Okay, fine,'' I sighed again, audibly this time. ``What part of the code are you talking about?''

Bob leaned over, hit a few keys, and pulled up a function that looked something like this:

\begin{footnotesize}
\begin{lstlisting}
// OBJMAP is a typedef for a standard container
void f( OBJMAP &theMap )
{
    for( OBJMAP::iterator iter = theMap.begin();
         iter != theMap.end();
         ++iter )
    {
        // do something
    }
    // other code in here
    for( OBJMAP::iterator iter = theMap.begin();
         iter != theMap.end();
         ++iter )
    {
        // do something else
    }
}
\end{lstlisting}
\end{footnotesize}

``What's wrong with that?'' I asked. ``I checked it out in Stroustrup, and the iterator should go out of scope at the end of
each loop \footnote{Bjarne Stroustrup. The C++ Programming Language, 3rd Edition (Addison-Wesley, 1997).}.'' ``Yeah, well,
not with my compiler.'' Bob replied, picking at his teeth. ``It complains that iter is already defined. Easy fix,
though\ldots just get rid of the second declaration of iter. In fact, that was how my code originally read when I checked it
in. Then you went and broke the build.'' 

Then I understood. I had to be diplomatic because I was still on probation, though.
``Bob,'' I said reasonably, ``this has to compile on all the platforms we're targeting. I put the second declaration in
because my compiler complained about the second iterator being undefined, and that's how it's supposed to work, according to
Stroustrup.'' Bob seemed a little put off. ``I don't care about your newfangled compiler. I've been using mine for years and
it's never let me down. Your code doesn't work.'' ``Bob. The code's right. It compiles fine for me.'' ``So you got a buggy
compiler. Not my fault, I'm just doing my work, I don't care about other compilers,'' Bob shrugged again, more defensively.


A quiet voice behind us said: ``I sense much fear in you.'' I jumped a little; Bob jumped more, sloshing his latt\'e. The
Guru was far too good at gliding up silently behind us, a tome in one hand as she always had. The Guru shook her greying
head slowly, frowning. ``Fear leads to anger,'' she continued. ``Aw, cut it out. I'm not afraid of other compilers!'' Bob
said hotly, sucking the spilled latt\'e from his wrist. ``I don't care about them at all.'' ``Anger leads to hate. Hate��
leads to suffering.'' ``I'm suffering, all right,'' I muttered under my breath. ``I'm just trying to write portable code.
---All right, Bob,'' I capitulated, anxious to be rid of him, ``I'll fix it so it works. Okay?'' ``Like I said, Junior, just
get rid of the second declaration.'' He glared once more, then turned and left. Wendy gophered up from the next cubicle,
watching him go. ``So what's Bab complaining about this time?'' she asked us. While the Guru stood nearby, having resumed
reading in her tome, I showed Wendy the code and explained what Bob wanted to do. ``Typical,'' Wendy snorted. ``Give him a
choice between implementing a portable solution, and a solution geared towards his compiler, he'll unerringly choose the
latter. His code is not portable, just like you said. Y'know, in this case, you don't really need to declare a new iterator
for each loop anyway. Just declare it once, before the first loop, and it'll work on both compilers.'' She made some quick
changes to the code

\begin{footnotesize}
\begin{lstlisting}
// OBJMAP is a typedef for a standard container
void f(OBJMAP &theMap)
{
    OBJMAP::iterator iter;
    for( iter = theMap.begin();
         iter != theMap.end();
         ++iter ) {
        // do something
    }
    // other code in here
    for( iter = theMap.begin();
         iter != theMap.end();
         ++iter ) {
    // do something else
    }
}

\end{lstlisting}
\end{footnotesize}

The Guru looked up from her reading and glanced at the solution. ``Very wise, daughter.'' Wendy raised an eyebrow, but only
a little; she was used to it. ``And, as you implied, that is not the only solution. Another elegant one might be to avoid
the iterator declarations entirely, and perhaps improve readability, by using for\_each and expression templates, where
possible \footnote{See the feature article on expression templates in the May 2000 issue of C++ Report.}. My apprentice,''
the Guru continued to me, ``while it is important to know, understand, and obey the Holy Standard, it is also important to
know your tools, even those that are Deviant. Bob is using a compiler that does not obey the scope of a variable declared in
a for-init-statement. He cares not about portable code, as long as his own compiles.''

``Hmm\ldots'' I said, rubbing my chin. ``Writing portable code is harder than I thought. So, we have to wait for a new
release of Bob's compiler before we can take advantage of the code I originally had?''

The Guru looked off in the distance for a moment. ``I am not a prophet, I cannot see the future. However, this I do foresee:
Portability issues will never truly disappear. Compilers will not all become Standards-conforming at the same rate. Some
vendors may even leave outdated behavior in their compilers, in order not to break legacy code.

``Also, some compilers will introduce new features to extend the Standard. In and of itself, this is good, else would the
language become stale, inflexible, and eventually die stagnant. Programs destined for only one platform may benefit from the
new features. Using those non-Standard features in portable code, however, would be disastrous -- and some novitiates, such
as yourself, may inadvertently use them, without being aware of their non-portability. Do not come to love one compiler so
much as to fear others. Fear is the path to the unportable.''

She paused a moment, then turned and walked off. As she left, we could hear her soft voice saying, ``I wish I could remember
the name of the wise one who said, 'The difference between theory and practice is greater in practice than it is in theory
\ldots' ''

\wuxchcenterline

\textit{``Bob just wasn't a big-picture fellow,'' I finished.}

\textit{``Nobody had better pull a Bob on Ganymede,'' Jeannine agreed. ``It's not quite that forgiving a place, just yet.
  Hey, you'd better go. Anderson's waiting.''}

\textit{I looked at the time and stood up. ``Duty calls. What are you and Laura doing after third watch?'' I asked, since I
  was discovering that there were, of course, more pleasant things to talk about than Bob, or even the Guru \ldots}

\pagebreak
\section{Virtually Yours}
\begin{flushright}--An application of the Template Method pattern, and a discussion of when to use public vs. protected vs.
  private virtual functions.\end{flushright}

\textit{I gave the wrench another twist, but it was useless. I could have tried pounding the wrench with the hammer to try
  to force it over, but that didn't seem like a good idea. ``It won't budge,'' I said. ``There's almost enough room for the
  new module, but what's in there already is all just wedged in too tight.''}

\textit{``Kee-ristopher,'' Jeannine complained, watching because there was only room for one of us to work on the machine.
  ``It can't be that hard to extend the hardware.''}

\textit{``It sure can be when it's not put together right,'' I sighed. ``Whoever slapped this together must've figured that
  as long as it worked it was good enough. Everything's been put together too tightly coupled. We can extend it, all right,
  but first we'll have to take it all the way apart and rebuild it.''}

\textit{``Welcome to Europa, the pull-your-own-weight capital of the system. Back home we'd have techs doing this for us.''}

\textit{I mopped my brow. ``You know,'' I said thoughtfully, ``this reminds me\ldots'' Jeannine smiled; that was the signal
  for a caffeine break, and I began to tell her another story from my first job.}

\wuxchcenterline

I was working on some code that Wendy, the programmer in the next cubicle, had designed. She had created an abstract base
class for the project, and my job was to derive a concrete class from it.

As usual when studying a new class, I began with the public interface:

\begin{footnotesize}
\begin{lstlisting}
class Mountie
{
public:
    void read( std::istream & );
    void write( std::ostream & ) const;
    virtual ~Mountie();
}
\end{lstlisting}
\end{footnotesize}

No surprises here �� a virtual destructor, implying that this is intended to be used as a base class. Non-virtual read and
write members, though �� I wanted to look into that one. Then I moved on to the protected interface:

\begin{footnotesize}
\begin{lstlisting}
protected:
    virtual void do_read( std::istream & );
    virtual void do_write( std::ostream & ) const;
\end{lstlisting}
\end{footnotesize}

After a moment, I realized that Wendy was using the Template Method pattern: the public, non-virtual functions clearly would
invoke the protected virtual functions \footnote{E. Gamma, R. Helm, R. Johnson, and J. Vlissides. Design Patterns: Elements
  of Reusable Object-Oriented Software (Addison-Wesley, 1995).}. Pleased with my own quick understanding, I was feeling
pretty smug by this point �� ready to handle anything that would be thrown at me. Until I looked at the private interface:

\begin{footnotesize}
\begin{lstlisting}
private:
    virtual std::string classID() const = 0;
\end{lstlisting}
\end{footnotesize}

I stopped cold. A pure virtual, private function? How could it possibly work? I gophered up. ``Um, Wendy,'' I said, ``your
\textbf{Mountie} class can't work. It's got a private virtual function.'' ``Have you tried it?'' Wendy asked, without
looking up from her keyboard. ``Uh, well, no,'' I admitted. ``But my derived class can't possibly override the
\textbf{classID} function.'' ``So certain are you?'' The Guru's quiet voice startled us both. ``Always with you it cannot be
done. Have you learned nothing from me in these months you have been my apprentice?'' Despite the calm quiet of the Guru's
voice, I was taken aback by the force of her words. ``My child,'' she continued, ``you have forgotten that access privilege
and virtualness are independent of each other. Determining whether the function is to be statically or dynamically bound is
the last step in resolving a function call. You need to read and meditate upon the Holy Standard, chapter 3.4 verse 1, and
chapter 5.2.2 verse 1 \footnote{ISO/IEC 14882:1998, ``Programming Languages �� C++,'' clauses 3.4 and 5.2.2.}.'' 

I decided it was time to demonstrate my brilliance. I opened my mouth\ldots ``Oh, yeah. Uh, right.'' So much for brilliance.
I decided to take another tack �� distraction. ``But I still don't understand why it's private. It just doesn't make sense to
me.'' ``Meditate upon this parable, my apprentice. You are creating a class, and after reflecting on the design, you decide
the class requires a member function that should not be called by any other classes, not even its derived classes. What do
you do?'' ``You make it private, of course,'' I replied. The Guru looked at me, her eyebrows raised expectantly. My brain
kicked into overdrive trying to figure out how to link together private and virtual functions. Wendy spoke up. ``Have you
examined the implementation of the \textbf{Mountie} class, especially the \textbf{write} function?'' I quickly turned to my
keyboard, glad to briefly escape the Guru's unrelenting gaze. I found the function fairly quickly:

\begin{footnotesize}
\begin{lstlisting}
void Mountie::write(std::ostream &Dudley) const
{
    Dudley << classID() << std::endl;
    do_write(Dudley);
}
\end{lstlisting}
\end{footnotesize}

I could tell Wendy had spent way too much time watching cartoons as a child. ``I see now,'' I said. ``\textbf{classID} is an
implementation detail, used to signal the concrete type of the class that is being saved. Derived classes must override the
function, but since it is an implementation detail, it's kept private.'' ``Good, my child,'' the Guru agreed. ``Now explain
to me why \textbf{do\_write} and \textbf{do\_read} are not private.'' That stopped me for a moment. But then the penny
dropped: ``Because the derived class has to call its parent class's implementation of those functions so that the parent
class has a chance to read and write its data.'' ``Very good, my apprentice,'' the Guru was almost beaming. ``But why not
make them public?'' ``Because,'' I was really warming up to this now, ``they must be called in a controlled fashion,
especially the \textbf{do\_write} function. The object type has to be written to the stream first so that when the object is
read in, the factory knows which type of object to instantiate and then load from the stream. Allowing public access to the
functions would cause chaos.'' ``Very wise, my apprentice.'' The Guru paused a moment. ``As with spoken languages, there is
more to knowing the C++ language than just knowing the grammar and the rules. You must know the idioms.'' ``Yes, Coplien's
book was the next one on my�� \footnote{Coplien. Advanced C++ Programming Styles and Idioms (Addison-Wesley, 1992).}'' The
Guru held up her hand. ``Peace, child. I do not refer to the prophet Coplien. I refer to the linguistic sense of the word
idiom �� the implied meaning behind certain constructs. You know, for example, that a virtual destructor tells you `I am
meant to be used as a polymorphic base class, you may create from me little ones as you need' whereas a non-virtual
destructor tells you `I am not meant to be used as a polymorphic base class, do not derive from me for that reason, I pray.'
``In a similar manner, the access privilege given a virtual function conveys an idiomatic meaning to the initiated. A
protected virtual member tells you 'my little ones should �� or perhaps even must �� invoke my implementation of this
function.' A private virtual function tells the initiated 'my little ones may or may not override me as they choose, however
they may not invoke my implementation.' `` I nodded as I absorbed this. ``What about public virtual functions, though?''
``Avoid them where possible, and prefer to use the Template Method. Consider this parable.'' She picked up the dry erase
marker, and began writing in her delicate script:

\begin{footnotesize}
\begin{lstlisting}
class HardToExtend
{
public:
    virtual void f();
};
void HardToExtend::f()
{
   // Perform a specific action
}
\end{lstlisting}
\end{footnotesize}

``Consider that you have released this class, and your requirements have changed,'' she continued. ``In the second release
of this class, you discover you need to implement the Template Method on the function \textbf{f()}. It will be nigh on
impossible to accomplish. Can you see why?'' ``Uh, um. Yes, well\ldots I think\ldots No, I can't.'' ``There are two possible
ways to convert this class to the Template Method. The first way would be to move the implementation code of \textbf{f()}
into a new function and make \textbf{f()} non-virtual, like this:

\begin{footnotesize}
\begin{lstlisting}
class HardToExtend
{
// possibly protected
    virtual void do_f();
public:
    void f();
};
void HardToExtend::f()
{
    // pre-processing
    do_f();
    // post-processing
}
void HardToExtend::do_f()
{
    // Perform a specific action
}
\end{lstlisting}
\end{footnotesize}

  ``However, \textbf{HardToExtend}'s little ones will expect to override \textbf{f()}, not \textbf{do\_f()}. You must now
  change all classes derived from \textbf{HardToExtend}. If you miss just one class, that class will attempt to override a
  non-virtual function. This could introduce, in the words of the prophet Meyers, 'schizophrenic behavior' into your class
  hierarchy \footnote{S.Meyers. Effective C++: 50 Specific Ways to Improve Your Programs and Design, 2nd edition
    (Addison-Wesley, 1998); Item 37: "Never redefine an inherited non-virtual function."}. ``The other solution is to
  introduce a new, non-virtual function and move f() to the private section, like this:''

\begin{footnotesize}
\begin{lstlisting}
class HardToExtend
{
// possibly protected
    virtual void f();
public:
    void call_f();
};
\end{lstlisting}
\end{footnotesize}

``This will cause users of the class no end of headaches. All the clients of \textbf{HardToExtend} will attempt to invoke
\textbf{f()}, not \textbf{call\_f()}. Those clients will no longer compile. Furthermore, derived classes will most likely
leave \textbf{f()} in the public interface, and clients that use those derived classes directly, rather than using a pointer
or reference to \textbf{HardToExtend}, will still have direct access to the function you want to protect.

``Virtual functions should be treated very much like data members �� make them private, until design needs indicate a less
restricted approach is indicated. It is much easier to promote them to a more accessible level, than it is to demote them to
a more private level.''

\wuxchcenterline{}

\textit{``When did all this happen?'' Jeannine asked.}

\textit{``Just before the turn of the millennium �� the real turn, that is. It was at the end of 2000, I think. We were just
  about to celebrate New Year's Day, and we were really looking forward to the special year of 2001, because of all the
  literature, you know\ldots''}

\textit{Oddly, it was shortly after that conversation that we heard about the discovery of the obelisk under the surface of
  Europa.}

\pagebreak

%%% Local Variables:
%%% TeX-master: "conversations"
%%% End:
