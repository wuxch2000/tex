\documentclass[11pt,a4paper,onecolumn]{article}
\usepackage{fontspec,xunicode,xltxtra}
% \setmainfont[Mapping=tex-text]{Times New Roman}
\setmainfont[Mapping=tex-text]{Arial}
\setsansfont[Mapping=tex-text]{Arial}
% \setmonofont[Mapping=tex-text]{Courier New}
\setmonofont[Mapping=tex-text]{Times New Roman}

\usepackage{xeCJK}
% \setCJKmainfont[ItalicFont={Adobe Kaiti Std}]{Adobe Song Std}
% \setCJKmainfont[ItalicFont={Adobe Kaiti Std}]{Adobe Kaiti Std}
\setCJKmainfont[ItalicFont={Adobe Kaiti Std}]{Adobe Heiti Std}
\setCJKsansfont{Adobe Heiti Std}
% \setCJKsansfont{Microsoft YaHei}
\setCJKmonofont{Adobe Heiti Std}
\punctstyle{banjiao}

\usepackage{calc}
\usepackage[]{geometry}
% \geometry{paperwidth=221mm,paperheight=148.5mm}
% \geometry{paperwidth=9.309in,paperheight=6.982in}
\geometry{paperwidth=7.2cm,paperheight=10.8cm}
% \geometry{twocolumn}
\geometry{left=5mm,right=5mm}
\geometry{top=5mm,bottom=5mm,foot=5mm}
% \geometry{columnsep=10mm}
\setlength{\emergencystretch}{3em}


\usepackage{indentfirst}

%生成PDF的链接
\usepackage{hyperref}
\hypersetup{
    % bookmarks=true,         % show bookmarks bar?
    bookmarksopen=true,
    pdfpagemode=UseNone,    % options: UseNode, UseThumbs, UseOutlines, FullScreen
    pdfstartview=FitB,
    pdfborder=1,
    pdfhighlight=/P,
    pdfauthor={wuxch},
    unicode=true,           % non-Latin characters in Acrobat’s bookmarks
    colorlinks,             % false: boxed links; true: colored links
    linkcolor=blue,         % color of internal links
    citecolor=blue,        % color of links to bibliography
    filecolor=magenta,      % color of file links
    urlcolor=cyan           % color of external links
}
\makeindex

\usepackage[dvips,dvipsnames,svgnames]{xcolor}
\definecolor{light-gray}{gray}{0.95}

\usepackage{graphicx}
\usepackage{wrapfig}
\usepackage{picinpar}

\renewcommand\contentsname{目录}
\renewcommand\listfigurename{插图}
\renewcommand\listtablename{表格}
\renewcommand\indexname{索引}
\renewcommand\figurename{图}
\renewcommand\tablename{表}

\usepackage{caption}
\renewcommand{\captionfont}{\scriptsize \sffamily}
\setlength{\abovecaptionskip}{0pt}
\setlength{\belowcaptionskip}{0pt}

\graphicspath{{fig/}}

\usepackage{fancyhdr}

% \usepackage{lastpage}
% \cfoot{\thepage\ of \pageref{LastPage}}

% 嵌入的代码显示
% \usepackage{listings}
% \lstset{language=C++, breaklines, extendedchars=false}
% \lstset{basicstyle=\ttfamily,
%         frame=single,
%         keywordstyle=\color{blue},
%         commentstyle=\color{SeaGreen},
%         stringstyle=\ttfamily,
%         showstringspaces=false,
%         tabsize=4,
%         backgroundcolor=\color{light-gray}}

\usepackage[sf]{titlesec}
\titleformat{\section}{\normalsize\sffamily\bf\color{blue}}{\textsection~\thesection}{.1em}{}
\titleformat{\subsection}{\normalsize\sffamily}{\thesubsection}{.1em}{}
\titlespacing*{\section}{0pt}{1ex}{1ex}
\titlespacing*{\subsection}{0pt}{0.2ex}{0.2ex}

\usepackage{fancyhdr}
\usepackage{lastpage}
\fancyhf{}
\lhead{}
\rhead{}
\chead{\scriptsize{\textsf{蜗居}}}
\cfoot{\scriptsize{\textsf{第 \thepage ~页,共 \pageref*{LastPage} 页}}}


% \usepackage{enumitem}
% \setitemize{label=$\bullet$,leftmargin=3em,noitemsep,topsep=0pt,parsep=0pt}
% \setenumerate{leftmargin=3em,noitemsep,topsep=0pt,parsep=0pt}

% \setlength{\parskip}{1.5ex plus 0.5ex minus 0.2ex}
\setlength{\parskip}{2.0ex plus 0.5ex minus 0.2ex}

% \setlength{\parindent}{5ex}
\setlength{\parindent}{0ex}

% \usepackage{setspace}
\linespread{1.25}

% 英文的破折号--不明显,使用自己画的线。
\newcommand{\myrule}{\hspace{0.5em}\rule[3pt]{1.6em}{0.3mm}\hspace{0.5em}}

\begin{document}
\setcounter{section}{0}
\pagestyle{fancy}

\chead{\scriptsize{\textsf{暧昧}}}

\begin{center}
\makebox[0.2\textwidth][s]{\Large \textsf{暧昧}}
\end{center}

\section[\thesection]{} 
七厅八处的例会每周一次,时间在周一下午。徐佩蓉到八处第一天就赶上了。例会人不齐,副处长老陈和副处调聂
于川出差未归。老陈倒无所谓,没看见聂于川,她心中多少有些怅惘,连处长老冯传达文件也听不进去。传达过文
件,老冯又通知说三处副处长老周儿子结婚,邀请八处集体出席。众人一片叹息。老冯笑道真不巧,徐科长是第一
天来处里,就得凑份子了。大家都笑起来。处里的杂务内勤是科员小李。他去年刚毕业,年纪尚轻,吃亏不够,还
没学会说话前用脑子过一遍,就有些没心没肺道,冯处,还是处里收齐了,一并送去?

此言一出,众人都很生气,心里怪他多嘴。处里这个传统很不好。结婚随礼,应该听凭自愿,红包里究竟多少谁也
不知。一旦处里统一收,就等于公开了,谁多谁少就有了比较。老冯正在上党校,处里事情又多,实在无心管这
些,就拿了500元交给小李,说就按老规矩办吧。完了又特意补充说,老孙你辛苦点,小徐刚来,多带带她。

老孙今年五十四,副处调吊了七年,虽说对提拔的渴望从未消弭,但希望毕竟越来越渺茫。不料原副处长老何一
死,机会重现,宛如一声春雷唤醒了冬眠。副处调和副处长虽说都是副处级,但一个是非领导职务,一个是领导职
务,就像伪钞和真币,看上去一样,却经不起揉捏。何况在机关,人人眼里都有验钞机,真假一看便知。既然知道
真假,态度自然不一;既然态度不一,难免有所区别。即便别人不把区别表露出来,当事人岂能没一点察觉?察觉得
多了,蓄在心里如同洪水。老孙想,省里还能有个泄洪区,自己虽有洪却无处可泄,岂不悲哉。不过老冯今天要他
关照徐佩蓉,证明老同志还是有用处的,多少是个心理安慰。回到办公室,老孙给小李400块,说这是我的份子钱。

老韩拿勺子使劲搅着中药,不以为然道,自我要求这么高,看来是要有好消息了啊。

老韩没能嫁个好老公,却生了个好儿子,在中央某部委当秘书。她临近退休,无欲无求,又在更年期,看谁都像看
昆虫,恨不能一脚踩过去,用力拧几拧。日子一长,大家都习惯了她见谁灭谁。老孙也不跟她计较,笑着说,就是
真提拔也该了,工龄都三十年了,赶上小徐的年龄了。

徐佩蓉正在打文件,闻言不由笑道,孙处,那您得多关照啊。

老孙坦然一笑,弹了弹烟灰,好像在表示关照容易得很,只要他想。老韩继续不以为然,对徐佩蓉说小聂也是副处
调,出差了,就坐在你对桌——你也是省大的,认识他吗?徐佩蓉笑着说,同校但不同届。老韩问得直接,她说得巧妙。
不同届不代表没见过,不认识不等于不熟悉。聂于川读研期间年轻气盛,办诗社搞辩论,一时风头出尽。徐佩蓉当
时还是情窦初开,暗恋过他几年。聂于川毕业后再未见过,不想在七厅重又聚首。她离婚也三年了,谁知道这是不
是上天安排呢? 她微笑着把文件打印出来,送给老孙审阅。老韩故意叹息说,小聂人不错,可惜了。

因何“可惜”,徐佩蓉并没追问。这让办公室里的其他人感觉很遗憾。其实故事还有下文的,既然她不问,他们也
不好主动说。删节版总不如完整版好看,而删掉的东西,往往都很暧昧。原来聂夫人不在得并不光彩,是跟单位的
一个司机一起死在车里。这倒不出奇,出奇的是两人都没穿衣服。一肚子话不得泄洪,三人都有些不爽,于是集体
失语。徐佩蓉觉得莫名其妙,只好陪着沉默。一直熬到下午下班,四人先后起身离开。老孙走得最晚。出门之际,
他碰见五处的老安。五处管人事教育,老安跟他同年提的副处调,现在已是副处实职到手。老孙拉他进屋,说知道
老弟爱喝茶,这次回老家特意带了盒特级品。老安当然是连声道谢。老孙趁机道,我们八处新来的小徐\myrule{}

老安脸色一凛,习惯性地看看门口,低声说,她可有来头,钟厅长亲自安排的。

老孙手抖了起来。糟糕,下午徐佩蓉让他多关照,他竟信以为真了。看样子,还他妈的指不定谁关照谁呢。老孙心
里发慌,下意识地摸烟。老安继续说,究竟是什么背景,我也不清楚。反正最近几年厅里进的人,数她跟钟厅长关
系最近。老孙狠狠抽了两口烟,苦笑说谢谢老弟,我明白了。送走老安,他后悔莫及。其实抽屉里还有两盒茶叶,
不过给老安的最贵。今天他看见徐佩蓉也喝茶,早知道留给她了。三百多一盒呢。给老安好茶叶有屁用,提拔又不
是他说了算。

第二天徐佩蓉上班,对面的桌子还是空空荡荡。她想了想,公事公办道,韩老师,陈处他们出差几天?有个文件厅办
催得紧。老韩正在看报上的健康讲座,头也不抬,不耐烦地说,不知道。老孙马上说,今天就回,小徐,厅办虚张
声势腆了,别着急。小李也赧颜道,徐科长,这事该我做的,您就别操心了。昨天我忙昏头了,怎么能让您打文件
呢?老孙心中鄙夷地冷笑,臭小子,肯定也知道消息了,变得这么快!

昨天下午,小李跟厅办小朱一道骑车回家,东拉西扯聊到了徐佩蓉,聊毕,小李后悔得两腿发木。回到家,心惊胆
战地跟女朋友汇报,又被骂得体无完肤。骂过,女朋友忍痛拿出盒东西,让他找机会送给徐佩蓉,好歹弥补一下今
天的怠慢。小李认出那是她姨妈从美国捎来的羊胎素,贵得很,她一直舍不得用,就感动地说谢谢老婆。女朋友嘤
嘤道,你什么时候改一改呢?你看小朱,跟你一年进的七厅,人家都是副主任科员了。小李自卑至极,不敢再言语。
当晚,他主动以身为报,竟然绵软不举,更平添了一层焦灼。

八处有三间办公室,老冯一间,老何死后老陈独占一间,其余人挤在大的一间。现在徐佩蓉已成大办公室里的晴雨
表,除了老韩,老孙、小李都下意识地勘察她的表情。徐佩蓉心中满满当当,没意识到下班了,呆坐着不动。老孙、
小李见她不走,也不便下班。老韩则无所忌惮,没等到点就溜了。于是徐佩蓉上网看新闻,老孙装模作样读报,小
李埋头发信息,三人谁都不提下班的事。又磨蹭了一阵,门却开了。聂于川提着行李和电脑包进来,诧异地看着大
家,说早下班了,怎么都还在?

老孙站起,把报纸塞进公文包,说有篇评论写得好,看得忘了时间了,下班下班。小李如蒙大赦,赶紧走人,只是
遗憾没能把羊胎素送出去。聂于川见二人走了,把东西放好,仿佛这才发现办公室里多了一个会呼吸的生物,惊讶
道,你就是小徐?

徐佩蓉笑吟吟站起道,是啊师兄,好久不见了。

师兄?聂于川一愣,你哪一级的?

比师兄低几届,我上本科,你读研一。我大三,你毕业。

聂于川恍然道,好,好。厅里又多了个校友。钟厅长也是咱们校友,你知道吧?

徐佩蓉当然不能说,我太知道了,我就是她安排进的八处,于是笑而不答。聂于川为难说,本该请师妹吃顿饭的,
可我今天刚回来,孩子又发烧了,改天好不好?她失落得厉害,但还是笑着说,师兄别见外,机会多的是。他抱歉地
一笑,居然真的转身走了。徐佩蓉再也无心上网,长长地一声叹息。

其实徐佩蓉那点底细,聂于川早就知道了。故意不说,是因为他有想法。这次跟老陈一起出差,没少聊到她。老陈
最近要提拔了,去厅属研究院做书记,正处级。因为要离开,信息就可 以共享,至少能留个人情在。聂于川使劲回
想,终于想起主编校刊的时候,好像真有一个姓徐的小师妹投过散文,附了封暧昧的信。时过境迁,当年的小师妹
竟跟钟厅长对上号了,幸亏钟厅长也是女的,不然还真有些暧昧。老陈鼓励他跟徐佩蓉拉关系,搞一搞曲线救国。
又说当今有四大铁,一起扛过枪,一起同过窗,一起嫖过娼,一起分过赃。小聂你跟她毕竟是同窗,跟她搞好关
系,钟厅长那里有利无弊。你看老何不在了,我也要走了,处里少了个副处长,你比老孙强多了,努把力,争取赶
上这次厅里大提拔。聂于川叹气说,同窗又不是同床,再说了,同过床的还信不得呢。老陈知道他又想起往事,摇
头不说话了。回到省城,两人在火车站分手。聂于川没回家,先去了厅里,见办公室里灯火通明,便暗暗替自己的
决定喝彩。而徐佩蓉见到他时的态度,更加坚定了他的信心。至于扭头就走,那更是精心安排的神来之笔。聂于川
不是当年的聂于川了,现在,他是个高手。想到这里,他的脸上露出了一丝笑容。很内敛,很暧昧。

聂于川原本对暧昧并不在行,也不在意。真正开始练兵还是妻子出事之后。几番试探,出手、交战和整编,他已然
修炼成了暧昧高手。大凡高手都会有底气,他自然也有。处里老冯做了多年正处,在厅里人气正高,距离厅党组咫
尺之遥。老何已死,老陈即将升迁外放,只有老孙能构成威胁。相比之下,老孙资历老,经验多;他年纪轻,能力
强。但是这跟提拔与否关系不大。厅长看好老孙,他就是嘴上没毛办事不牢;看好他,老孙就是年纪太大不堪重用。
如今天上掉下个徐妹妹,跟钟厅长交情莫逆,又曾追求过他,还是离了婚的,内因具备外因有利,只要运作得当,
还愁副处长被老孙抢走?还愁赶不上大提拔的末班车?就算都不提拔,副处长空置,他今年才36岁,以时间换空间,
积小胜为大胜,熬也把老孙熬退休了。数风流人物,还看今朝:当然,这是有前提的。就像一列火车,时刻表已
定,仅需沿着轨道走下去,早晚会到站\myrule{}只要不出轨。如今妻子已飘居云端,出轨的基础不复存在。至于玩
玩暧昧,并不能和出轨画等号,不但不能画等号,还可以得到意外收获。以前不懂,恪守什么兔子不吃窝边草。一
经解放了思想,才发觉“不吃”完全是对低智商说的。既然都是草,为何不能吃?难道将窝边之外的草吃尽,只剩下
窝前窝后郁郁葱葱的一堆,就不会被猎人发现了?所以关键是要知道怎么吃。到处都吃一点,自己也饱了,大地依旧
绿草如茵,小窝才越发安全。

聂于川第一个暧昧的对象也是个离婚女人,叫苏一文,比他大了四岁。两人是在工作组认识的。她在六厅工作,已
离婚好几年,独自带着女儿。第一次见面是工作组成立聚餐。酒过三巡,带队领导安排工作,说小聂你负责写简
报,小苏你就负责喝酒。大家都笑起来。苏一文说领导真幽默,弄反了吧?领导笑道我不打无准备之仗,早咨询过了。
你在六厅是有名的“只会喝撑,从不喝蒙”,根本不知道什么是醉。苏一文爽朗地大笑。聂于川自觉酒量尚可,暗
中还对她蛮不服气;等到了驻地,与接待方几次拼酒下来,方知领导法力无边,慧眼如炬。很快到了收官阶段。工
作组人心涣散,都在苦等省里总结大会的通知。一个周末,聂于川安排父母带威威旅游去了,就没回去。晚饭时,
他意外发现苏一文也在。她解释说女儿去了前夫那里,回去也是一个人,索性省了车马劳顿之苦。饭后是散步。两
人散到一家电影院门前。苏一文突然说,敢不敢请我看场电影?

聂于川都记不得上次看电影是什么时候了。好像是个节日,厅工会组织看的,他写的观后感还得了一等奖,发了条
蚕丝被。落座之后,他无意中碰到她的手背,宛如蚕丝般的顺滑。看完一场,苏一文又要看,于是接二连三,直到
子夜过后。他一晚上都恍恍惚惚。三十多年中被灌输的各种理念、信条、规范在心里人仰马翻,尸横遍野,再无片
刻安然。她倒是平平静静,不时无声地笑笑,明明是笑给他看,却故意不去瞧他。现在回味起来,她真是一身的高
手风范。回到酒店,两人并排走,到了他房间门口,两人停下脚步。苏一文忽然拉住他的手,步履稳健地走向自己
的房间。整个夜晚,聂于川都感觉钻进了一条柔密滑腻的蚕丝被里,四处都是密不透风的暧昧。

就在不久前,省里办了一个驻村成果展,各厅局都要派人捧场。会展中心里一时人头攒动。聂于川混迹其间,碰见
了不少熟人,亲热地打招呼,仿佛彼此声音越大关系就越近。徐佩蓉恭维道,聂处人缘真好。他笑了起来,说都是
以前喝酒喝出来的。她就垂下头,边摇边笑。两人走到六厅展区,迎头看见一幅大照片。苏一文头戴草帽,和一群
脑满肠肥的人站在一起,旁边一个黑瘦的农民笑得花团锦簇。底下一行小字注解:全省驻村先进工作者、我厅干部
苏一文深入田间地头。几年不见,苏一文却看不出变化。好像时光专门去抢别人的容颜,却对她手下留情。徐佩蓉
见他看得专注,过来小声说,也是熟人?聂于川惊讶于她能掐会算,就点点头,轻描淡写说以前一起下过工作组。她
看过注解,揶揄道你们组里出干部啊,混到正处级了。他随口道那下回再有机会,把你也推荐去。她低声说,你
去,我就去;你不去,我也不去。

聂于川很自然地岔开话题,说你不是有个表弟在某某县吗?苏一文在那儿挂过职,回头我请她多关照关照。徐佩蓉似
乎对他的捉摸不定早已习惯,便轻叹一声,不再说话了。

回到家,聂于川翻出当年的通讯录,找到了苏一文的电话,打过去。不等他说话,她在那边爽朗地笑道聂处啊,怎
么想起老姐姐了?他一窘,忙说恭喜老姐姐提拔啊。苏一文哧哧地笑,毫不客气地说,虚伪!去年的事了,今天炒什
么冷饭。两人聊了几句,他趁机把徐佩蓉表弟的事说了。苏一文说不就是想吃财政饭吗,应该问题不大。她停顿了
一下,又柔声说,弟弟你的电话我一直存着,手机换了好几个,也一直存着,没删。聂于川沉默了几秒钟,说谢谢
老姐姐,回头见面好好聊。放下电话,他笑了笑,自言自语说,高手啊。只是不知她现在结婚没有,是不是还在跟
谁暧昧着。在暧昧上,男人的杀伤力大而短,女人的杀伤面广而长,自成一派各有千秋。苏一文只是聂于川暧昧的
开始,但就像初恋,总归难以忘怀。以前是靠一场电影,一瞥眼神,一次牵手;现在无非变成了几句对白,一个电
话。而人也总是要成长的。当年的毛头小伙,如今已渐入佳境,开始平静地享受暧昧带来的一切。回忆过去的自
己,就像隔着毛玻璃,面目变得含混,神态变得陌生,想回也回不去了。因为时过境迁,此消彼长,已是两个世界。

老韩终于找了个机会,向徐佩蓉讲了聂夫人的事迹。徐佩蓉何许人也,当然知道这些典故,却也不便明说。老韩提
到聂于川这两年一心带孩子,没考虑再婚,也没听说跟谁有暧昧关系。这倒是个意外收获,让徐佩蓉就好感倍增。
她离婚,他丧偶,或许原本就没什么差别。都成了单身,而且配偶都有不忠的经历。前夫虽然花心,但并不想离
婚,前公婆至今对她仍很好,是她执意要离的。究己本心,可能还是根本就没爱过。前夫出轨无非是个再好不过的
由头。老韩见徐佩蓉若有所思,就问她老公在哪里工作。徐佩蓉淡淡一笑,说我前夫出国了。老韩的嘴巴保持着
O形,看得见舌头和齿间黑色的中药渣滓,像灭蝇灯上星星点点的苍蝇尸体。徐佩蓉当然明白只要老韩知道了,就等
于全厅都知道了。“非典”是需要空气传染的,绯闻当然也需要。老韩就是再好不过的空气。尽管她并无恶意,只
是纯粹出于传播更多信息之目的。徐佩蓉并不在乎闲话,和前夫结婚又离婚,闲话还少吗?

老孙得知徐佩蓉离过婚,下意识摸烟,内心又慌乱起来。岁月不饶人哪,只恨不能年轻个十岁、二十岁,那时他还
是有两分姿色的,起码比聂于川强些。更可气的,是徐佩蓉当着大家的面问聂于川今晚有没有时间,说自己初来乍
到,想跟厅里的校友们搞个聚会,人已经通知过了。他想也没想就说好。钟厅长的履历出身全厅路人皆知,以徐佩
蓉和她的关系,今晚的聚会肯定要出席。聂于川又和徐佩蓉一个部门,近水楼台先得月自然少不了。老孙犹豫再
三,还是找了个合适的机会,咬咬牙把两盒茶叶拿出来,送给徐佩蓉。徐佩蓉一连声道谢,可他怎么看都觉得虚伪。

老孙并不是无的放矢。眼下就有个天赐良机。钟厅长半月前从北京回来,在部里要了个大项目,定为七厅一号工
程,光是一期投资就要十几个亿。按厅里惯例,这样的项目要成立筹委会,参与的不是领导就是骨干。八处是业务
核心部门,自然不会缺席。如果老何没死,或者是老陈不走,都不会轮到老孙或聂于川。偏偏时势造英雄,老何也
死了,老陈也要走了,谁能进筹委会,和副处长就更近一步。老孙自我安慰说老子辛辛苦苦三十年工龄,在八处也
十几年了,写的文件汗牛充栋,进筹委会当然够格。尽管如此,他还是跑到党校,请老冯出来吃了顿饭,巧妙地说
起了一号工程,又巧妙地点明自己经验丰富,能力足够,热情饱满。为了进一步打动老冯,他还提起了当年一同当
科员,一同进八处的老话。老冯也有些感慨万千。到了最后,两人借酒劲竞相奋勇埋单,弄得都挺激动。可晚饭过
后,两人冷静下来,又都觉得刚才更像是一场表演,根本没涉及什么深层次的东西。老孙骑车回家,发现老冯真是
狡猾,自己除了赔上一顿饭钱,一句瓷实话都没得到;而老冯开车回家,也发现老孙真是可笑,快退休的老朽了,
还想跟年轻人争,他要是进了筹委会,年龄最大的都比他小,成何体统嘛。何况钟厅长在昨天一次会上,公开提到
八处有个小聂,在厅里内部刊物上发了篇论文,很有想法。“很”字还加了重音。

那篇论文是徐佩蓉授意聂于川写的。说授意可能有些刻薄,但事实如此。聂于川聪明得很,熬了一个通宵写成,看
似和一号工程无关,实则等于是工程的实施方案雏形。他拿着论文,请厅办主编内刊的老裴喝了顿酒,轻松搞定。
有了这篇钟厅长大加赞扬的文章,聂于川顺理成章地进了筹委会。老孙气得在办公室摔了个茶杯,老韩多情得很,
以为他是针对她,撒泼大闹一回,请了半月病假。又过两天,厅里研究这次提拔人员名单,钟厅长提议加上聂于川。
会上也有人说到老孙。钟厅长避而不谈,转而问管组织人事的厅长老任,小聂副处调的年限够吗?回答是够了。学历
符合吗?回答是符合。八处有副处长空缺吗?回答是有。有更合适的人选吗?老任一笑摇头,算是回答。钟厅长就不再
问。提老孙的人更不好意思再说。一个月后,提拔的文件下来,里面赫然有聂于川的名字。

既然是提拔了,视野为之一展,空间为之一开,聂于川就得搬离大办公室,到老陈那里去。老陈这次也提拔了正处
级,但设计院的老书记还有三个月才退休,不便立刻就去抢班夺权。老陈笑道,这是天意啊,咱哥儿俩能再混上仨
月。两人都大笑。笑毕,老陈有些忘形,邀功说,老哥我的主意不错吧,跟小徐搞好关系,有利无弊。聂于川心里
恨得发紧,却不便多讲。徐佩蓉来找他,他一副公事公办的模样,倒显得老陈一肚子小人之心。徐佩蓉也惊讶于他
的突然疏远,难过之后,明白了这是他在撇清。毕竟谁都看不起吃软饭的男人。何况厅里已有议论。不管怎样,徐
佩蓉发现自己真的爱上他了。聂于川比前夫强太多,有能力,有风度,懂珍惜,会生活,简直浑身都是优点。更要
命的是,他也是单身,正需要一个女人。

然而聂于川似乎总也不开窍。在电梯里遇见,他会一如既往地向身旁的人介绍,这是我们处新来的徐科长,研究生。
而徐佩蓉也就温婉一笑。没有其他人的时候也有,但很少。聂于川提拔后不久,一次上班迟到,他不无狼狈地冲进
电梯,蓦地发现只有徐佩蓉一个人,正错愕地看着他。两人互相点头微笑,竟一时无语。还是她先说,师兄,送孩
子了?

聂于川忙说,是啊,你骑车来的?

徐佩蓉倒不说话了,只是点点头,又垂下头去,似乎在忍着笑。聂于川心中一动,不明所以。进而一想,有些生气。
他觉得她有点骄傲了。不错,你有背景,也帮过忙,但沾沾自喜是要不得的,表现出来更是不妥。这关系到厅里人
的看法,也关系到暧昧主动权的得失。两者都不能含糊。于是他也不说话,盯着闪烁的楼层号码。到了办公室,还
没落座就被老冯叫了去。老冯一见他张口便大笑起来,说小聂你也太随意了,好歹也是副处长,牙上有个补丁都不
知道。聂于川顿时大窘,背过身一抹,可不是一小块韭菜?老冯笑毕,叹气说家里没个女人就是不行,你看你,都成
什么样了。

回到办公室,聂于川真想给徐佩蓉打个电话,或者发个信息,却左思右想不知道怎么措辞,只好作罢。一个上午浑
浑噩噩地过去。徐佩蓉垂头忍笑的样子不住地在他脑海里忽隐忽现。像是看着墙外的人荡秋千,一会儿出现在这
头,一会儿出现在那头。看来他有点错怪她了。其实自己提拔后,厅里的确有人怀疑他曲径通幽,幽自然是钟厅
长,这径就是徐佩蓉。为了撇清,他处处小心翼翼,跟她保持着距离。他当然能推测出她的想法,只是十年不见,
他对她了解甚少,她的真实意图一时不敢确认,也就拿不定主意。一段时间观察下来,徐佩蓉大约是真心的。其实
跟一个有背景的单身女人暧昧一下,似乎也无伤大雅。反正仅是暧昧,发现不妙溜之大吉就是。正巧到了午餐时
间,厅后勤公司的小姑娘送来两个盒饭,聂于川心里一动,说你忙别的处吧,这个就交给我。

聂于川后来向徐佩蓉承认,他的确是别有用心的。厅家属院有两处,近的近在咫尺,远的也有班车,在厅里吃午餐
的要么是加班,要么是单身,要么是占公家便宜。盒饭有两个,处里七个人,老冯在党校,老陈、老韩、老孙家里
都有学生,得回去做饭。符合条件的只有他、小李和徐佩蓉。不过小李新近谈了恋爱,中午也要抓紧时间缠绵
\myrule{}他一边想,一边推开门\myrule{}果然只有她。

徐佩蓉站起来,笑吟吟说,聂处师兄亲自送饭,怎么担当得起啊。

聂于川把盒饭放在桌上,叹气说亡羊补牢啊,我得赶紧讨好。不然下次还是不提醒我,害我出丑。还师兄呢。

徐佩蓉一下子笑得花枝招展,随即垂下头。那一瞬间聂于川更加确认了自己的决定。他有 些委屈地叹气,上前打开
盒饭,惊讶说后勤公司不过日子了?居然还有鸡腿。徐佩蓉好容易敛住笑,闻言又笑起来。她把盒饭推给他,说想吃
就拿走,我不喜欢油腻的。聂于川连连摇头,说吃人嘴短,我是受害者,想一个鸡腿就打发我啊。

徐佩蓉看着他的眼睛。她是鼓起勇气这样做的。来七厅这么久,两人还是第一次四目相对。聂于川不傻。拜十几年
机关生活所赐,他的眼神不但会克隆,还能变异,各种神采招之即来,来之能战,战之能胜。比如徐佩蓉看他的时
候,他就知道她渴望看到什么。一个受过高等教育的女人,一个三十露头、离婚久旷的女人,一个可能对爱还有些
许幻望的女人,难道不想心猿意马?即使真没有,也是因为没遇到让她心猿意马的男人。聂于川发挥得很好。他的目
光充斥着深邃、平静,不妨再加些骤然而至的冰冷。这样的鸡尾酒式目光杀伤力很大。他相信,徐佩蓉一定会中招
的。

回到办公室,他信心满满地打开盒饭,看着那个一模一样的鸡腿。油汪汪的皮下带着片脂肪,略带酱色的腿肉,突
兀亮泽的关节\myrule{}多好的一个猎物啊。聂于川想,生活多无趣啊,工作多无聊啊,有一个暧昧的对象多好。好
就好在一切都在自己掌控之中,好就好在可以不主动,不主动就可以不拒绝,不拒绝就可以不负责。他吃得精神抖
擞。爱情足以让女人容光焕发,而对于男人,则是暧昧。

门开了。徐佩蓉托着盒饭进来,在他桌边坐下,将一次性筷子倒过来,用筷子屁股把鸡腿拨给他。她笑着说,我还
没动呢!师兄放心,我知道一个鸡腿打发不了师兄。回头我补请,好不好?

整个动作流程里;聂于川内心愉快外表惊讶地看着她,直到她说完。他有些尴尬地看了看面前的猎物,又看了看身
边的猎物,说,太客气了,其实你帮我不少忙,该我请你的。他的表情又意外又诚实。徐佩蓉一笑道,那算什么,
师兄你才客气呢。聂于川觉得既然环境许可条件成熟,工作力度就不妨稍大一些。他说干脆你也在这儿吃吧,一个
人吃也无聊。

这顿午饭进行得很融洽。徐佩蓉毕竟是初来乍到,一些厅里行政和业务部门还不熟悉。聂于川就一一讲解,语气很
柔和,态度很认真,不时讲两个笑话来调和。比如他说她前途远大,模样绝不像是恭维,也不必恭维。徐佩蓉愕然
看着他,问为什么。他一本正经道,因为你是无知少女。她越发讶异。他解释说,无,是无党派人士;知,是知识
分子;少,是少数民族;女,当然就是女干部了。小徐你看看,最容易提拔的四大要素,你一人就占了仨,能不是
前途远大吗?

吃过饭,聂于川给她削苹果。长长的苹果皮打着卷,垂到桌面。徐佩蓉不由自主地说,想不到师兄削苹果的水平还
这么高。他笑而不答,递给她苹果,开始削第二个。削完了却不吃,仍旧递给她。她忙推辞。聂于川说,我老家盛
产苹果,小时候吃伤了。她说原来如此,看来这刀功也是有渊源的。他却说哪里有渊源,小时候吃苹果谁削皮?我不
爱吃,可我儿子喜欢,这点技术也是这两年才练成的。为了儿子嘛。

聂于川很清楚,忠诚家庭溺爱子女的男人,很容易获得异性的好感。尤其是婚姻失败的异性。徐佩蓉的目光一刹那
温散开来。她像是自语,也像是感慨,说结婚好几年也没要孩子,离婚了就什么都没有了。

火力侦察的效果很好。聂于川没有答话,站起来端着两个空饭盒,说纸巾在那儿,我出去一下。等他扔掉垃圾,抽
了根烟回来,徐佩蓉自然已经离开了。桌子上的茶杯里有新沏的茶,水汽袅袅婷婷。他靠在椅子上,又点了支烟。
两种类型的烟雾在房间里交错,苏烟和铁观音的气息在周遭弥漫。这就开始了?聂于川微微笑起来。

\section[\thesection]{}

聂于川搬走后,老孙的状态一落千丈,老何死后疯狂滋长的野心化为乌有。这也难怪,一个快五十五岁的老吊,直
接提拔正处级的希望如同海市蜃楼,虽就在眼前,却是一片破碎的虚空。老孙彻底明白了老韩快乐的源头。有容乃
大,无欲则刚。老子什么都不图了,谁能奈我何?总不会把副处调再撤了吧?从此通知开会,心情好就去,心情不好
就不去;即便是去了,也不再唯唯诺诺,想什么说什么,管你爱听不爱听;分管的一摊事,统统推给徐佩蓉,放着
年轻人不用,还让老骥跑长途?老韩再说什么怪话,老孙毫不客气地反击,有时候两人争得面红耳赤。老韩气得三天
两头请病假,老孙泄洪成功,才懒得管她。他想,老子一心向善,厅里偏偏不许,那老子就当回恶人吧。

老孙撂了挑子,徐佩蓉的压力骤增。她刚来不久,遇见难题只有找聂于川。而这段时间,聂于川也很忙。筹委会不
好进,进去了就得好好把握,好好表现。筹委会主任是钟厅长,而她还要主持全厅工作,不能事必躬亲。几个副厅
长都想替她分担一些。常务副厅长老任、副厅长老钱最积极。筹委会需要撰写的材料浩若烟海,有些是老任分管,
有些是老钱负责,周游在两位厅长之间,聂于川不得不越发小心翼翼。老任管八处,是他的顶头上司,伺候的日子
不短了,还算得心应手。而他跟老钱接触不多,不够了解,便多揣了一份谨慎,一有动静就跑去汇报。次数一多,
老钱皱眉说小聂,别动不动就来请示,这点事用得着吗?聂于川就笑,笑过之后,该请示还是照请示,也没看出老钱
有多讨厌。一次汇报完毕,老钱满意道,对,这事就得这么办,小聂干得不错。聂于川当然说,这是领导指挥得好。
老钱说,这么优秀的同志,提拔得太晚了,就这党组里还有人不同意呢!我当时就堵回去了,有能力就要破格嘛。你
今年也快四十了吧?聂于川赶紧说,三十六了。

按说也不算太晚。可你也知道,在七厅这种单位,有时候错过一次提拔,不知得等多少年才有机会。像老孙,今年
五十五了,也就错过了两次。只两次,十年就过去了。

感谢领导关怀。

老钱挥挥手,说我还是那句话,该提的就要提。不但你,小徐我看也不错。我向钟厅长建议过,一个女同志,又是
研究生,不妨也破格。

聂于川老老实实道,小徐最近很积极,工作压力也大。

是老孙撂挑子了吧?老钱喷出一笑。唉,不说他了。小徐跟你是校友,业务上你得多帮帮她。你和小徐有什么需要我
这边协调的,尽管说。我喜欢跟年轻人打交道,显得自己也不那么老。言毕,老钱哈哈大笑。聂于川赔笑站起,告
辞。回到办公室,他对徐佩蓉简直肃然起敬。老钱是厅里巨头,连他也向徐佩蓉示好。看来徐佩蓉的底细掌握得还
不准确,她肯定不仅仅跟钟厅长关系近,不然老钱也不会如此说话。难道是关系源自省里?不过他看过她的档案,她
父母都是普通的高校教师,那么就是亲戚?师生?他自失地一笑,反正是裙带关系,至于是裙是带,没有追究的必要。
转而一想,老钱为何点破他和她是校友?又为何通过他来转达关心?对于徐佩蓉的主动,他一直是态度混沌,既不拒
绝也不接受,难道这也被别人看出了?看来,不管他愿不愿意,他与徐佩蓉都成了暧昧的代名词。他想,无论如何,
这次暧昧都有必要开展下去了。何况已经有了精彩的伏笔,不利用太可惜。

如果说第一次共进午餐是意外,第二次是试探,第三次是心照不宣,第四次就成了习惯。而习惯一旦形成,就很难
改变了。只要聂于川中午没有应酬,处里又没第三人在场,徐佩蓉总会端着盒饭到他那里去。老钱谈话的第二天,
赶上厅工会有活动,徐佩蓉代表处里参加,回到办公室已经快一点了。盒饭安安静静地躺在桌子上。她看着盒饭,
心里有些慌乱,想了半天还是打了电话。她有些不自然地说,谢谢师兄,吃了吗?说过之后,她简直可以听到自己心
跳的声音。他在那边笑起来,说没有啊,你进入跳棋决赛啦?

都一点了,你在加班?

他片刻没说话,继而低声说,等你呢。

她想不到他会突然这么直爽。他明明是在暗示她什么,可又无法深究,只怕说穿了彼此都不自在。她就只好马上
说,那我这就过去。

聂于川果然在噼噼啪啪地打字,一堆文件小山似的摊在手边,封面的发文签上密密麻麻全是批示。徐佩蓉莫名地有
些失落。她坐下笑道,以为真在等我呢,原来还是在加班。他说不能这么讲,应该说真是在等你,顺便加加班。两
人一起笑了。徐佩蓉垂头要打开盒饭。聂于川拦住她,说,先别打开,猜一猜今天是什么菜色,猜错的请对方打球。
她抗议说这太不公平了,你肯定早就看过。聂于川肃容说,我保证没有看过。

两人看着对方,房间里一时很安静。徐佩蓉凑到盒饭前,闻了闻,抢着说这太简单了,一定是鸡腿,谁不知道后勤
公司都成养鸡场了。聂于川笑着摇头,说不对,是西葫芦,不信打开看看?她微笑着朝他努努嘴。聂于川就打开,得
意地看着她。徐佩蓉难以置信地打开自己那份,里面却是两个鸡腿。她马上明白了,脸颊顿时润红,说你耍赖,明
明是你换过了。

你代表处里参加比赛,责任重大啊。多吃一个鸡腿,保证你决赛得第一名。

徐佩蓉忍住笑,说我参加的是跳棋,又不是跳远,再说了,这有联系吗?

小鸡蹦得多快啊。

这几句对话奠定了愉快的气氛。吃完饭,聂于川扔了空饭盒回来,脚步放得很轻。因为他注意到徐佩蓉没有回去。
透过门缝罅隙,他看见徐佩蓉站在饮水机旁,手里端着他的杯子,却不去接水沏茶,像是在等什么。聂于川只觉一
股暖洋洋的气流涌遍四肢百骸,所过之处温畅通泰。中午时分,大楼里静悄悄的。他见四周无人,便蹑手蹑脚地退
后了几步,加重步伐走到门口,推门进去。她果然已经在给他接水,动作很自然。聂于川惊讶说,不用麻烦,我自
己来就行了。

客气什么。你是师兄,又是领导,为领导服务而已。

聂于川接过杯子,说了声谢谢,顺手放在桌上。杯子把上有些滑腻。不知是她手心的汗,还是淡淡的护手霜。他打
开一份文件,浏览、翻动,像是在找着什么。徐佩蓉应该在看着他。那种傻傻的目光,傻子都能感受得到,可是他
却偏偏不去回应。她待了一阵,看见一份杂志,捞到救命稻草似的拿起,说师兄没落下专业啊,还有小说杂志,借
我看看吧。他这才抬头看着她,微笑说没问题,尽管看。随即又翻着文件。徐佩蓉握着杂志刚刚坐下,猛地发现再
也找不到继续待下去的借口,勉强一笑说那您忙吧,就起身离去。他的冷淡突如其来,没有丝毫征兆,但他也明
白,这必不可少。顺水推舟谁都会,平地波澜才是高手。徐佩蓉出门之际,聂于川瞥见她黑色牛仔裤下精致而轻颤
的臀部,转过头来,啜了口滚烫的茶。暧昧其实并不是一捅就破的窗户纸,而是贴了一层纸的窗玻璃,想一捅就破
那就错了。但也不能让想来捅的人一蹶不振,再不复来。那就不叫暧昧,而是欺骗。何况徐佩蓉还有比较暧昧的背
景?像他这样的高手并不想骗人,却也不愿被人一览无余。好在一切才刚刚开始,而且都在自己的掌控之间。

由于午饭吃得晚,午睡也不现实了,聂于川索性真的打起了文件。文件是老任要的,说实话真还挺急。钟厅长对筹
委会的前期工作很不满,会上点名说了老任几句。老任有些委屈,又有些得意。你又没明确究竟是谁负责,怎么板
子打到我头上了?反过来一想,这等于是将此事交给自己,既然变相排除了老钱,就不能说不是好消息。老任精神一
振,一上班就召集筹委会的几个骨干开小会。既然是小会,老钱就不必参加了。小会上,老任让八处赶紧拿个考察
方案出来,他再报给钟厅长。聂于川明白,说是八处起草,老冯贵为处长,肯定不会亲自动笔,老陈忙着落实去设
计院,所谓方案,说到底还是落在自己头上。果然,老任一边给大家发烟,一边对他说,怎么样小聂,有难度吗?

这么大的材料,冯处又在学习,我一个人怕是能力有限。最好是陈处跟我一起搞。

老陈说走就走,帮不上你。老任弹了弹烟灰,笑道你们八处我还不知道,老冯就指望你了。对了,不是还有个研究
生小徐吗?

老冯忙说,小徐才来不久,业务上还不熟悉。

老任皱眉道,都好几个月了,还叫不久?业务也该熟悉了吧?你和小聂得多帮帮她嘛。她是钟厅长挖来的人才,重点
人物,得好好培养。钟厅长说过,我分管的处室,有几个人才,小聂和小徐都不错。

下楼的时候,老冯的脸色有些难看。聂于川赶紧敬烟,点火。到了办公室,老冯坐下叹道,这个小徐!几位厅长都夸
她,真喜欢,直接调秘书科算了,搁在八处成尊佛了,不敬她不行,不用她也不行。接着就不说话,抽烟。一开
始,聂于川心里也不痛快。大凡下属,都渴望受重用,更渴望受专用。本来这个狗屁方案并不难,他一人足以搞
定,为何加上个徐佩蓉?难道是不信任?可看见老任忧心忡忡的样子,他又平衡了。人家徐佩蓉原本就有背景,老任
也好,老钱也好,各个场合都表示看好她。老冯都被噎了几句,自己一个刚提拔的小副处长,吃什么醋?

聂于川斟酌着说,我看老任这次明确负责筹委会,也有小徐的因素。钟厅长说八处有人才,他又分管八处,协调起
来方便。

也不尽然。老冯苦笑,小徐再有能力,也是个小科长而已。别的不说了,方案要紧!

那小徐呢?

你就受点委屈,方案你写,上报时就说是你们俩搞的。靠,真他妈的麻烦。

聂于川有点啼笑皆非,搞不懂这是他占徐佩蓉的便宜,还是徐佩蓉占他的便宜。回到办公室,老陈照旧不在。最近
他有些神出鬼没。这也难怪。老陈正处级是有了,但还未上任;既然没上任,就像美味珍馐送到桌上,却找不着筷
子,只能干巴巴看着,而且随时会有变数。聂于川无心管他如何找筷子,把一号工程的文件摆满一桌,琳琅满目。
其实考察方案并不难,有的是以往的模板,加工组合一下就行。既然心里有了主意,就不急于动笔。写那么快干什
么?送上去的时间越早,领导修改的空间就越多,劳动量就越大。现在不急,可以先上上网,看看新闻,有场球赛错
过了,正好补一补。但两支烟过后,一场球没完,聂于川蓦地惊醒。还是太幼稚。哪个厅长不是宦海风云里过来
的,手下秘书笔杆这点伎俩焉能不知。糊弄领导虽是常事,却不可常为,领导在意的事更是糊弄不得。就像找领导
签字报销,发票里头有没有水分,有多少水分,领导心里清楚得跟明镜似的,却问也不问就签了。不是领导好糊
弄,而是领导对你睁一只眼闭一只眼。你要是不识好歹,非逼领导把另一只眼也睁开,吃亏的还是自己。在七厅摸
爬滚打这么多年,聂于川总结出一个真理,该应付的事绝不能认真,该认真的事绝不能应付。看似简单,实则玄奥。
瓢泼大雨没听说能淹死人,一口小井却能要命。

这边一身冷汗未消,那边老冯就来电话,问开始动笔没有。聂于川惊讶于他有透视眼,忙撒谎说已经开始了。老冯
欣慰说,好好干,老陈已经正式办了调动手续,今晚请八处全体聚餐,算是告别。聂于川说这么快!老陈瞒得真严,
我跟他对桌,愣是藏得滴水不漏。老冯笑道这事怨不得老陈,他不像你年轻,岁数大了机会有限,不敢出岔子。不
一会儿老陈到了,笑呵呵地通知他今晚七点,天鹅阁聚餐。他照例是一番祝贺。筷子到手,尘埃落定,老陈实在是
开心,兴奋得直搓手,又出去通知其他人。聂于川一边翻文件,整思路,一边酸酸地想,老陈也不容易,虽说是个
事业单位,毕竟也算一方诸侯,有了自己的独立王国。他正胡思乱想着,老陈皱眉回来,说八处真是得整顿了。大
白天的,老孙那边居然一个人都没有!

聂于川笑着解释,老韩生病请假,老孙和小李的主业是打乒乓球,小徐去工会参加跳棋比赛了,哪里还会有人?你有
事先去忙,我负责通知。

老陈和他各自点烟,抽着。老陈有一搭没一搭地道,小徐最近挺忙的。

嗯,老孙成了甩手掌柜了。聂于川生怕他又说什么拉近关系有利无弊的话,想把话题转移到老孙身上。可老陈偏不
上当,说小徐是你的贵人呀,她一来,你就提了副处长。开玩笑开玩笑,不过她的关系挺硬的。

聂于川认真道,你知道她的关系?

老陈看看他,说别瞒我了,你能不知道?你肯定知道。

聂于川说,我是真不知道。

老陈看了看门,低声道,小徐上面有人。

这句很暧昧,也实在是废话。聂于川还是一脸恍然大悟,哦,原来如此。

老陈说,她在八处,对八处是好事,对你也是好事。你想想,领导们自然喜欢看到她出成绩,她的成绩还不都是你
的成绩?老冯进了党组,就是你管着她嘛。

再别说谁管谁了。他苦笑说,唉,原以为来了个干活儿的,这下谁还敢使唤她。

可不是嘛。老陈诚恳地说,我这一走,八处可就只忙你一个人了。聂于川心里说,你就是不走,又干过多少?嘴上却
道,老兄当一把手去了,可别忘了弟兄们。

老陈连连摆手,幅度很大,像是溺水的人。他低声说,老弟放心,设计院虽说是事业单位,也算是一级组织吧。处
里再有不好消化的账,给我送去。

老陈这话水分不多,真诚的比例很大。八处负责全省的行业管理,设计院也在管理范畴之列。行政审批就像皮筋。
熟人来了,抽根烟讲个笑话的工夫就办了手续;遇见生瓜蛋子,那就是“十五个工作日”。更倒霉的,随便挑他个
毛病,打回重做,再送来还是“十五个工作日”。要不是靠着这点行政资源,设计院那么臃肿腐朽的老单位,早混
不下去了。聂于川见老陈说了交心的话,也肃然说,老兄见外了吧?从今往后,设计院的事八处上门服务。

老陈哈哈大笑着站起来,叮嘱说,晚上天鹅阁,七点,别忘了!

聂于川送走老陈,又点了支烟。徐佩蓉果然是有背景,而且来头还真不小。这些虽然都料到了,但一经确认,仍是
有些意外。由于暧昧的程度还不到,他没有问过她离婚的缘起,不过无论如何,看来这次有益无害的暧昧都有必要
进行下去。他摁灭烟蒂,看看表已经十点多了,便不敢再偷懒,扎扎实实地写了起来,直到后勤公司的人来送饭。
老冯又打电话来,说中午有饭局,问他能不能过去。聂于川为难说恐怕去不了,写方案呢。老冯马上说那就回头再
说吧,一切以方案为重,等方案通过了请你吃活海参。聂于川说话的时候,正在把自己盒饭里的鸡腿拨给徐佩蓉那
一份,忙忍住了笑,说谢谢领导关怀。

那两个鸡腿在中午的暧昧里成了重要道具。暧昧已过,下午上班,聂于川就挨个打电话通知晚上的饭局。大家莫不
响应。连老孙也爽快地说要去的要去的,好好沾一沾喜气,看我这个老吊什么时候也能进步进步。倒是徐佩蓉的话
很简短,只是说知道了,一定去。看来中午那点突如其来的冷淡,留下了一些后遗症。聂于川当然有准备,就说,
没事的话你过来一下,有点专业问题咨询咨询。

老陈一走,办公室里就剩下他一个人。空间一大,暧昧的难度系数就小了,不用因有别人在场而费力斟酌语言。他
见徐佩蓉进来,忙站起,把位子让给她。他说请你看一下,你是专业出身,别让我闹出白脖话来。徐佩蓉显然没想
到这个。她稍稍迟疑一下,欠身坐在他的椅子上。聂于川坐在一旁,看着她身子前倾,翘臀不安地挪动,握鼠标的
手指也在微微颤着。他刚刚站起,椅子上应该还有他的体温。他想,她坐在上面当然要心旌荡漾的。徐佩蓉看了几
行,情不自禁说聂处就是聂处,写得真好哎。

又不是写小说,哪里谈得上好。聂于川苦笑,指不定领导那里怎么修改呢。再说了,今后别叫我聂处,还是叫师兄
吧。

叫聂处,是同事关系;叫师兄,是同学关系。显然后者更适合暧昧的气氛。徐佩蓉刚想说话,聂于川突然探出手
来,伸到她前胸下缘和抽屉间的缝隙里。那个缝隙很小,小得像少女初吻前微张的双唇。尽管聂于川的手努力在回
避,却还是触到了她的衣服。准确地说,是触到了她的胸部。徐佩蓉本能地朝后退了退。聂于川将半个身子斜过
去,几乎碰到了她的大腿,这次她是退无可退了。徐佩蓉第一次离他这么近,甚至可以看得清他后脑上的根根头发。
她的呼吸明显地屏住了。聂于川顺利地拉开抽屉,拿出里面的一个苹果。

不能让你白忙活,大苹果伺候。

聂于川朝她晃了晃手里的苹果,脸上多了份不常见的调皮。似乎刚才那白驹过隙的一触根本不存在。即便真的存
在,他的脸上、眼睛里也是一尘不染,让她不由得为自己的多情而羞愧。她嘴角旁边绯红嫣然,说,师兄太客气了。

其实聂于川这一切举动根本谈不上客气,而是高手才有的收放自如。他熟练地削苹果,递给她,心里说,看你能抵
御到什么时候。其实徐佩蓉早已丢盔卸甲,像是手里干干净净的苹果,再无一丝可以遮掩的东西。她小小地咬了口
苹果,顶端有股夹着果香的淡淡的烟草味。她问他,我第一次参加处里聚餐,大家喝得厉害吗?

老陈是三两的功夫。聂于川擦手,将纸巾团成一团。不过今天他做东,自然要超水平发挥。老冯半斤八两的酒量,
控制得也好。小李呢,前三杯挺唬人,接下去就露馅了。老韩肯定会说她不能喝。

那你呢?

酒量不行,酒风还可以。

老孙呢?

聂于川皱眉,像是组织语言。徐佩蓉笑着说,难道师兄也总结不出来?

不是总结不出来。聂于川一笑,只是词儿不太文雅。

你就说好了,我会过滤的。再说,我不信师兄还能有多不文雅。

老孙属于\myrule{}有酒必喝,逢喝必醉,简单地说,是有酒瘾没酒量。

徐佩蓉笑起来,这词儿没什么不文雅啊。对了聂处,老韩好几天没来上班了,说是生病,用不用去看看? 不用。老
韩就是这样,一年里有一多半都请病假。

她挺敢说话的。一现在是老孙更敢说吧?

徐佩蓉一愣,笑起来。聂于川岔开了话题,说快看文件吧,大苹果不能白吃啊。她撇嘴说,真小气,三句话不离本
行。聂于川明白,她已经完全从中午的后遗症里解脱了出来,对他突如其来的主动惊喜异常。可惜,暧昧高手的主
动都是伏笔,而与之呼应的,难免是突如其来的冷淡,就像今天中午。

天鹅阁离厅里不远不近,是省城极有名的一个饭店。没到下班时间,小李就在办公室里嚷嚷,说早知道晚上有大
餐,中午饭都省了。徐佩蓉愉快地微笑,没有回应。老孙大口抽烟,说那算什么,等俺老孙也提拔了,请你吃国宴。
由于气氛和谐,老韩也没讽刺他痴心妄想。

据老韩说,她是特意从医院过来的。这一点无人有心去考证。六点钟下班,大家有说有笑地下楼。一个中年人在大
门口候着,见了聂于川忙上前,说聂处,我们陈书记让我等着处里领导们,车在停车场。聂于川认出他是设计院的
办公室关主任,就笑着给大家介绍。于是众人啧啧赞叹,说老陈太客气,就两步路还派车接。又说老陈不忘本,当
了书记还惦记着大家。到了停车场,一辆考斯特冲他们眨眼。关主任请大家上车。车上一个位置前有桌板,一看就
是给领导安排的。老冯从党校直接去饭店,无可争议的人不在,事情就微妙起来。微妙面前,众人都心照不宣地微
笑。聂于川看也不看宝座,大步走到后边。老孙最后一个上来,见众人都望着他笑,便一屁股坐下,扭头说大闹天
宫三十年,一夜回到花果山,今天俺老孙也坐坐玉帝如来的位置,大家没意见吧?大家一起笑,纷纷说他坐得好,坐
得正确。老孙又扭头看着聂于川,说聂大处长也没意见吧?聂于川笑着摇摇手。徐佩蓉对老孙很鄙夷,但见聂于川开
朗地在笑,又想起了中午的那次碰触,脸上不由得一阵发紧。她忙垂下头,让那一抹绯红藏进脸颊的阴影里。

到了饭店,走近包间,老陈在门口迎接,俨然一派东道主的风采。老陈说老冯打电话了,路上堵车,得迟到一会儿。
大家情绪高涨,大声酝酿着罚老冯酒。主位空下来给老冯,其余人等各找各座。聂于川和老陈自然分坐主位两旁。
老孙挨着老陈坐下,说是要沾沾老陈的喜气,又说老韩坐俺老孙身边,顺带也沾一点。老韩不客气说,我不要二手
的,你要是好心就让我挨着老陈坐。大家都笑起来。老韩也笑了,还是坐在老孙旁边。徐佩蓉松了口气,大大方方
地坐在聂于川一侧。她小声对他说,今天挺热闹的。

热闹还在后边呢。聂于川也是小声。徐佩蓉会意一笑。他明白,这样的窃窃私语正是她需要的。老孙对他们大声
说,小聂、小徐注意点,今天是老陈大喜的日子,咱们不能开小会。老韩说,眼红什么,咱俩也说说悄悄话?老陈立
刻鼓掌,赞成。包间里笑语喧哗,气氛烘托得很好。关主任来回伺候,像是个幽灵。需要他的时候必然就在眼前,
不需要的时候根本看不见他。不一会儿老冯来了,大家站起迎接。老冯见还未点菜,连连责怪老陈太见外。老陈把
菜单送到老冯手里,说点菜这么大的事,还是老领导亲自来。老冯也不客气,合上菜单,说不用一个个点了,就你
们招牌四宝吧,四个菜。除了老陈,大家都面面相觑。连聂于川都不明所以。小李沉不住气,说四个菜够不够啊?服
务员捂嘴笑。老冯说小子,等会儿你就知道了。

不多时,有厨师推车进来。原来这四宝还是现场做的。服务员鱼贯而入,流水似的上着菜,很快摆满一桌。大家这
才明白说是四个菜,其实是四个主菜,其余都是奉送。小李见连甲鱼龙虾乌鸡都奉送上来了,有点瞠目结舌,问服
务员四个主菜是什么。服务员笑着介绍,我们的招牌四宝,是东星斑、鳄鱼血米饭、穿山甲熬老参、秘制河豚。大
家一时静默无语。老陈说,用什么酒水,老领导也指示一下。老冯说,菜不便宜,酒水就简单点,52度水井坊吧。
老陈颇有底气地说,先拿三瓶。

这顿饭吃得宾主尽欢。大家干过前三杯集体酒,开始轮流过圈敬酒。不多时两瓶酒一滴不剩。老孙知道这酒也不便
宜,喝到的机会不多,一滴都舍不得洒。一开始信誓旦旦“不喝不喝”的老韩也被灌了几杯。劝酒劝到徐佩蓉这
里,聂于川以师兄身份替她挡了几次,遭到一致抨击。众人没见过徐佩蓉的酒量,却多少知道她的背景,敬酒也敬
得坦率真诚,仿佛喝在她嘴里的酒,最后都进了钟厅长的肚子。老冯控制得很好,微笑着看着大家你来我往。水井
坊开到第四瓶,老孙已经撑不住了,小李扶着他到洗手间。老冯见状对老陈说,今天差不多了,收工吧。老陈朝门
口招招手,关主任鬼魅一般地浮现在他身边,低头欠腰听了几句,转身飘走。大家又等了等,老孙这才回来,吐得
面如土色。老冯看了眼老陈,老陈会意地站起,举杯说,八处是我的根据地,设计院全靠八处支持。反正各位不来
视察,我定期上访就是了。大家纷纷笑起来,随着他一饮而尽。

席终人散,徐佩蓉回到家,洗过澡,躺在床上。本来满腹心绪和酒精掺杂一处,满身周游;现在酒精蒸发殆尽,剩
下心绪无处寄托,只觉阵阵头痛欲裂。聂于川分明看见她也喝了不少,为什么连个关心的电话都没有。难道在他的
心里,她真的一点位置都没有吗?她拿着电话反复摩挲,像燧人氏钻木取火,居然把电话弄得烫手。打,还是不打?
要不然,发个信息?

电话善解人意地响了。竟是他。到家了吗?

睡不着。

聂于川笑起来。答非所问啊,不过,你的酒量不会这么小吧?

心里有事,喝一点都能醉了。你没有跟老陈一起活动活动?

无非是洗洗澡打扑克,本人向来没兴趣。你倒对这一套挺熟悉啊。

我又不是刚毕业的大学生,别太小瞧人了。刚才路上,老孙一直唠叨,说不知道你和老陈去哪儿了。

他见怪不怪地笑了一声。老孙就是这样,生怕我和老陈单独活动,不带他玩。他顿了顿,又说你快休息吧,明天还
要上班。

徐佩蓉握着电话,忽然有种想哭的情绪。这种情绪瞬间燃烧起来,烧得她沉默下去,像是一截灰烬。聂于川笑着
问,怎么不说话了?

她幽幽说,我在想,如果没有你这个电话,我该有多难过。

他显然没有想到她会这么直接。这个\myrule{}我很荣幸。不过,我觉得你真的有点多了。别瞎想了,快睡吧。

徐佩蓉还想说,我怎么睡得着呢?却无论如何也说不出来,仿佛刚才那句话已经耗尽了她所有精力、所有勇气。她知
道像聂于川那样的男人,一般都会等着别人先挂电话。他对谁都那么客气,对谁都那么彬彬有礼,看不出态度,辨
不清喜恶。以至于在他面前,她根本感觉不到自己和老孙、和门卫有什么区别。好久,她才艰难道,好的,你也早
点睡。说完就挂了电话,然后垂下头埋在膝间,哭湿了睡裤。

聂于川检查了儿子的作业,又漫不经心地陪父母亲看了一会儿电视。洗漱之后,他决定再给她发个信息。他知道她
肯定没睡。刚才那个电话貌似关心,实则极富侵略性,应该是把她弄得心神俱疲。其实暧昧的发展也要讲科学,讲
可持续性。不能一味让主动的一方感觉没有 回报,有时候回报也是必需的。对她而言,由于背景特殊,回报不能太
吝啬,要具体问题具体分析。她躺在床上,认真斟酌一番,写了个信息:很后悔没有送你回家,已经很晚了,希望
你能睡个好觉。等显示对方已收到,他便关掉了手机。他想,即便是她幕后推手实力雄厚,即便是她一味主动需要
回报,眼下也仅此而已。足够了。

\section[\thesection]{}

考察方案一层层报上去,再一级级批下来,按常理该是一周以后。但此事重大,只过去两天,钟厅长“同意”的批
示就到了。老任组织开了个协调会,决定派两拨人出去,八处老冯和厅办老文各带一队,一南一北。老冯在党校刚
好有个去港澳新马考察的安排,两下里档期冲突,虽当面应承了老任,终归有些不舍。这事聂于川听他提过,知道
他左右为难,就出主意说,反正去港澳新马也就十天,也得在广州出境。领导您先带队去广东,而后跟着党校的团
出去,我们几个按部就班地去上海、江苏。您回了国直接飞南京,咱们会合后一道回来。您看这样行不行?

言毕,聂于川谦卑地看着他,暗暗给自己喝彩。这是个一招致命的主意。果然,老冯微笑着扔过来一支烟,自己也
点上,慢悠悠地说,那你可就得多操心了。

聂于川笑,说给领导分忧,这是我的责任嘛。

那你看处里谁去呢?

他想了想,说任厅长定了三个名额。您一个,我一个,剩下一个,您看着定。

让小徐去吧。老冯吐了一个浓浓的烟圈。这件事涉及专业,她能帮上忙。老任说要重视她,这不就是重视?

他早知道会是她,却为难道,您肯定不一直跟着,我跟她不太,不太方便吧。

老冯笑道有什么不方便?你还能弄条绯闻出来?她可是钟厅长的人,给你个胆子你也不敢吧。聂于川站起,笑着点头
出去。走到门口,老冯叫住他,欣赏地说好小子,比我当年强,好好干吧。

回到办公室,他对刚才的表现很满意,对老冯的话很生气。凭什么不敢动徐佩蓉?就算钟厅长是她妈,还能奈我何?
是她主动进攻,又不是老子率先勾引。何况她离婚,老子丧偶,你情我愿的事情,管天管地,还能管老子的生殖器?徐
佩蓉再有背景,也是个离婚茬子,总还要再婚。帮助她解决婚姻问题,是老子放弃了多少黄花闺女后慨然献身,她
身后的背景高兴还来不及呢。你老冯是处长,遇到难题不也是一筹莫展。幸亏帮你安排得周密,带队出发带队返
回,鬼才知道你中间都去了哪里。要不是老子,你就梦里去港澳新马吧。聂于川越想越气,恨不能立刻去隔壁把徐
佩蓉就地拿下,再四处炫耀一番。他气鼓鼓地等到快下班了,打电话给徐佩蓉,说小徐你过来一下,有事找你。

在他的精心引领下,徐佩蓉最近的状态很好。买了新衣服,做了新发型,估计是确定要发起攻势了。30岁的女人
了,又经历过婚姻,太知道该如何去吸引男人的注意。徐佩蓉进来时,他注意到她换了新行头。一件瘦紧的牛仔
裤,裤脚塞进灰色靴子里,一件灰白色大毛衣罩住臀部,却显得曲线更加风致了。一切都很自然。眼下这年头,越
自然的东西就越刻意。像聂于川这样的高手,当然不会对任何刻意视而不见,况且他本就有心。他笑着说,小徐今
天真漂亮。

暧昧高手的话都会留下很多切口。比如她可能会说,难道以前不漂亮吗?也可能会说,聂处肯夸奖,真不容易哦。
还可能会说,一个月工资没了,聂处扶扶贫,管几顿饭吧。还有可能\myrule{}或者干脆什么也不说,只是垂头在笑。
徐佩蓉似乎没想到他今天会这么慷慨,一时有些不适应。等回味过来,她笑着坐下,说聂处真会夸人,有什么指示?

他把批示递过去。她翻了翻,不解说,这跟我有什么关系?

他哭笑不得,点题道,老冯的意思是你和我,跟他一起去。

我和你?她的眼睛瞬间睁得很大,随即又黯淡,自语说还有冯处。那一瞬间,聂于川决定再大方一点,把惊喜提前给
她。一男一女,三十多岁,偏巧都是单身,偏巧男的打算暧昧,女的已经进攻。这样的氛围,这样的心态,单独出
差十来天,难免会发生一些有趣的事。他需要给她一定的时间和空间去准备,各方面的准备。暧昧没有准备,就像
演唱会已经开始,粉丝已然尖叫,而歌手却找不到话筒。效果大打折扣。

听了解释,徐佩蓉果然兴奋地脱口而出。那就是说,实际上还是你和我?

这倒有点出乎聂于川的预料。好歹也是30岁的女人了,还有那么优越的背景,应该不至于如此。在异性的示好前这
么不矜持,这么没城府,确实有点不太正常。她也仿佛看出了他的疑惑,不由自主地垂头下去,钩着毛衣一角坐下。
那,什么时候出发?

周末,今天名单报上去,厅办会订票的。

徐佩蓉赧颜抬头,看了他一眼,又垂下去。这么急啊,我得赶紧准备一下。干好几天呢。

那就去吧,下午不用来了。

她走了。聂于川看得出她有多欢喜。那样的欢喜好像只有少女才有的。她大概相信总有一天他会爱她,因为她认为
他没有理由不爱。她很出色,很努力,而他身边也正好缺一个女人。她为了他做的一切都心甘情愿,做的每一分钟
都甜蜜不已。即便受到冷遇,她也总能从以前的点点滴滴中找到坚持的理由。然而她还是错了。聂于川并不缺女人。
一个省直厅局36岁的单身副处长,想要找个老婆并不难。难的是找到之后,就不便再随意暧昧了。然而不能随意暧
昧当然很不理想,但如果换来一点额外回报做补偿,也还不错。徐佩蓉正好能给他补偿,即便她不能,她的背景也
能,这也是他决意暧昧的最大缘起。他有些庆幸,辛苦没有白费,彬彬有礼地拒绝了那么多暧昧,总算等到了。

出发前夜,威威睡了,聂于川陪父母看电视。父亲问同行的都有谁。听到“徐佩蓉”三个字,父亲意味深长地看他
一眼,说,女的吧?他笑着点头。又聊了几句,父亲突然问他打算什么时候再婚。他想了想,半是玩笑半认真地说,
等做到副厅级吧,起码也要正处到手。父亲没说话。母亲不满道,没听说还有这个条件的,副厅级?老家一个市几百
万人,副市长才几个?他笑嘻嘻说,你儿子现在是副处长了,相当于副县长。一个县也有几十万人吧?

父亲开口了,说,儿子说得不错,在他这个年纪,这个位置,结婚要慎重。

聂于川马上说,可不是嘛。

你听我说完。父亲打断他的话。不过你总要结婚的。你别看我一辈子只是个正科级,但我经历的多了。市里也好,
厅局也好,其实都是一回事。你现在不急,是因为你还年轻,又是副处长。看起来拥有很多,可是有多少是你能够
放弃的呢?没有,一点都没有。

母亲只对他的婚事感兴趣,一见跑了题,立刻长长地打了个呵欠,说你们爷儿俩扯淡吧,我睡了。明天还要做饭呢。
她把桌上的瓜子皮苹果核拢到一起,搓进手心,捧着离去。父亲看着母亲的背影,递给他一支烟。抽吧,咱爷儿俩
说说话。

聂于川接过烟,点上。父亲才是一个完整的官场缩影。“文革”老大专生,中学教师出身,靠一支笔杀人官场。有
呼风唤雨,有堕入尘埃,有众星捧月,也有大势已去。自己现在享受的一切,父亲都经历过。而父亲痛入肺腑的往
事,似乎离自己很远,又有可能明天就会碰上。在这个不可理喻的世界里,什么事情是注定会发生的,什么事情是
注定不会发生的,谁都不知道。父亲抽烟时喜欢深吸一口,存在口腔,缓缓吐出,又忽然吸进去。一团浓雾刚在嘴
边蔓延成形,却转眼不见。聂于川看着父亲一吞一吐,把玩着青色的烟气,不由笑道,您老就说吧。

我这一辈子,基本上是功不成名不就。但我也有安慰。老婆不离不弃,儿子出人头地,孙子学习努力。我都六十多
的人了,还想什么呢?我一直担心的是你。好赖也在机关混了一辈子,你现在的花花肠子我太清楚了。想学西门庆,
玩儿上几年,勾搭几个,最后再找个过日子的。对不对?

聂于川看着父亲。他有些无耻地笑,不说话。“过日子的”对他而言,要求太低了。他决定不向父亲提起徐佩蓉的
背景。虽说是父亲,大可以无所顾虑;但爷儿俩都是男人,吃软饭毕竟不太光明正大。就算还没吃到嘴里,男人一
旦有了这个想法,也难免让人瞧不起,即便是父亲。

父亲继续说,我是个官场的失败者。可有时候,真理并不是胜利者总结出来的,他们只顾享受胜利果实了。就拿你
的状态来说吧。你的底气,是因为你是个副处长,领导又赏识,还可能再提拔,觉得自己还算是个人物,挑挑拣拣
也无可厚非。对不对?

聂于川还是笑。

其实呢,你这底气是不错的,也该有。但你也想想看,你这底气,多少是牢牢握在自己手里呢?你是输不起的。你吸
引女人,是因为你穿着这身官衣。可官衣是党给的,是组织给的,总之不是你的,什么时候要回去也由不得你。你
玩儿的东西是炸药包,太有摧毁性了,只适合一无所有的人玩儿。你呢?一个不小心,副处长就没了。副处长没了,
你就一切都没了。

聂于川说,那我是该恋爱呢,还是该谈恋爱?

男人的一生,肯定不会只有一个女人。父亲看了看卧室,坦然说,我也不例外,你也不例外。当然,女人也有很多
种,但这不是今天的话题。你年纪不小了,官也比我大,我没法告诉你该怎么样。我只是想提醒你,要小心翼翼。
记住,你输不起。如果每次跟女人周旋都牢记这个,起码不会摔跟头。

聂于川摇摇头,那我也太被动了吧。

父亲哧哧笑了。他站起,亲昵地拍了拍儿子的头,像是回到了三十年前。父亲说,想不被动,当然也好办。你现在
不是副处长吗?等你当了处长、厅长,就不用这么小心翼翼了。

聂于川睡得很晚。父亲的话一直折磨着他,拨动着他的心弦。思绪不定之际,他给老陈打了个电话,说要去广州出
差,用不用给老陈岳父家捎东西。老陈笑着感谢,说我岳父岳母来看闺女,眼下就在我家,不用麻烦了。聂于川也
笑,趁机说你看老冯安排的,要我跟徐佩蓉一块儿去,这才是麻烦呢。老陈那边敛住了笑,沉默片刻,认真道老弟
啊,你得好好把握自己。小徐的背景,我知道得不多,一句话也说不清楚。小徐人是不错,不过最好再观察观察。
你又不是没见过女人,对吧?我说话有点直了,老弟别介意。

老陈话里有话,可能他真的知道,但不便说,或不想说;也可能他真的不知道,所以无从说起。聂于川知道自己和
他的关系,也只能说到这一步了。通知徐佩蓉出差时,他几乎已经确认要在期间跟她再进一步。可是现在,他冷静
了。父亲也好,老陈也好,自己的戏言也好,其实都是一个意思。他现在只有副处长这一身衣服,虽然比老孙、老
韩的赤身裸体强些,但还不安全。渴望已久的正处长什么时候才能降临呢?如果不断升迁下去,衣服就多了;衣服一
多,即便脱去一层,也不会有一丝不挂的尴尬。他一边抽着烟,一边想。真到了那个时候,父母养老,儿子上学就
都不是问题,连玩玩暧昧也更有底气了。他忽然发现,自己对提拔升迁的焦灼从未如此具体,如此真诚。

接待方很热情,只是酒量不行。老冯象征性地带队考察一天,就跟着党校同学直奔香港而去。考察组只剩下聂于川
和徐佩蓉。晚上到了酒店,进了电梯,她骄傲地说,聂处,小女子没给你丢人吧?

聂于川点评说,跟他们比喝酒,起步太低了吧。徐佩蓉刚才喝了不少,把接待方吓得目瞪口呆,没人敢提出跟聂于
川碰杯。他见她撅起了嘴,笑道,不过还是值得表扬。我们的小徐不但业务好,交际、应酬,各方面都很优秀。她
不接腔,反问道,刚才下车的时候,那个小焦小声跟你嘀咕什么?

两人出了电梯,聂于川微笑不答,点上一支烟。电梯口对面是沙发,茶几上摆着烟灰缸和糖果、瓜子。他坐下,叼
着烟摇头。你怎么凡事都这么好奇?今天的考察记录做好了吗?老冯可不像我、这么好说话。

在这儿谈工作,太不严肃了吧?

那还能在哪儿,总不能包个会议室。

聂于川当然知道她想去他的房间,正等着他的邀请。如果没有父亲和老陈那席话,说不定他还真就答应了。至少也
是一起喝喝茶,聊聊天。但是现在,他决定不这样做。虽说她的背景尚不明朗,但毕竟的确是有。对这样的女人,
要比寻常对手更谨慎,更小心。与其一呼一应,倒不如欲擒故纵。徐佩蓉不说话了。她把他的如履薄冰看成了有意
回避,而此时的回避其实就是紧逼。他的步步紧逼让她很难堪。他连这点主动都吝啬。再泼辣的女人,也不至于在
独处的第一晚就投怀送抱吧?她难以想象他到底在犹豫什么。

聂于川拿起一块糖,剥去一半的皮,捏着底部递给她。她赌气不接。他叹道,好吧好吧,真拿你没办法。小焦问
我,晚上用不用安排。我当然拒绝。怎么样,满意了?

徐佩蓉的脸瞬间通红。她拿过糖,狠狠地嚼着,说这也太离谱了,我难道不是女的?话音刚落,他就忍不住大笑起来。
她这才明白话里有语病,脸色更红。她结结巴巴地纠正,我的意思是,有女同事跟着还这么明目张胆,太不像话。

聂于川站起来,好了好了,快回去吧,明天还要考察。徐佩蓉垂头起身,乖乖地跟在他后边。进了房间,他打开电
视,把声音调得很大。她就在隔壁,一会儿肯定要过来的。他简单地冲洗了一下,没有穿宾馆的浴袍,套上带来的
睡衣,仔细系上每一个扣子。一切停当,他靠在床头,给家里打电话。正跟父亲说闲话,有人敲门。他微笑继续,
并不响应。他有意说了很长时间。刚放了电话,铃声就响了。徐佩蓉有些不满道,敲门没人理,手机又关机,电话
还一直占线,聂处的日程安排得挺满嘛。

聂于川帮她直奔主题。有事吗?

我的电脑坏了,今天的记录没法整理,怎么办呀?徐佩蓉的语气有些撒娇了。这是个好现象。他想,起码懂得迂回进
攻了。

那今天就不用整理了。他想笑。不然,我把电脑给你送过去吧。

她马上说,哪里敢劳领导大驾,我过去拿好了。

他笑着放下电话,起身去开了门。刚开门她就到了。显然是精心梳洗过,香氛幽幽,也没有穿浴袍,而是一身很合
体,稍显身材又不过于性感的家居服。他一侧身,她就钻了进去。这是她第一次进入他生活起居的地方,因而显得
很好奇,不住地左顾右盼。聂于川笑道,这里没别人,你瞎看什么。

徐佩蓉撇嘴说,谁知道有没有呢?谁知道 现在没有,一会儿有没有呢?

别说胡话。他板下脸,指着电脑。就在那儿呢,你拿去吧。

这么放心我拿走,里面就没什么秘密?

等你往里面输入一些秘密,不就有了。

我真的输入了,你也未必找得到。

喂,你怎么开机了?

我呢,还是在领导这里打,万一有问题可以及时请示嘛。

总算到了你来我往的地步。聂于川想,她终于进步了,不再是欲言又止。她不乏主动,但主动也要用在刀刃上,要
懂得营造过招的气氛。暧昧中的过招是很重要的。陌生人可以借此熟悉,老熟人可以增进好感,继而做出最后的判
断。聂于川心里很愉快。随你吧,跑了一天,你不累我倒累了。

徐佩蓉歪着头看过来,说那好办,你睡吧,不影响我。

他一跃而起。那我还是别睡了。

她哧哧地笑起来,啪啪地按键。没几下,她又歪头看着他,问电脑里有没有歌。他让她自己找,又说我这里只有老
歌,你们年轻人的歌我听不懂。她摆弄一阵,居然真找到了,惊讶说王菲、邓丽君,居然还有郭兰英!他说这就是所
谓代沟了。其实他36岁,她31岁,代沟的说法无非是提醒她年龄差距并不显著。她果然摇头感慨,师兄还年轻着呢。
他笑了笑,刷牙去了。等他出来,王菲谜一样的嗓音正在房间里到处弥漫,偏巧就是那首《暧昧》:

\begin{center}
{\itshape 你的衣裳今天我在穿

没留住你却依然温暖

徘徊在似苦又甜之间

望不穿这暧昧的眼
}
\end{center}

聂于川吃惊地站住了。这回是真的吃惊。如果说是巧合,那这简直是天意;如果说是刻意,难道她也成了高手?幸好
这电光火石的一愣并没被她看到。他平静了一下,走到她身边,说,怎么样了?

快好了。徐佩蓉说,你就知道问这个。

她故意又问了几个问题,好让他不便离开。她的家居服并不暴露。但他居高临下,如果用心,倘若有意,一点点春
光还是难免看到的。徐佩蓉从面前的镜子里悄悄打量着他。可惜,他压根就没看她,脸色也有些生硬,声音却柔软
下来。好了,别闹了,弄完了就回去吧。她刚想说什么,他又补充道,好好休息,今天刚开始,出差还长着呢。

出差的每一天,聂于川都要给老冯发信息汇报工作。有时候一写就是半天。徐佩蓉笑他发得慢,他索性把手机给
她,让她代劳。她的表情分明在说,她当然愿意代这个劳,而且简直是求之不得。于是聂于川口述,她飞快地按键。
其实她见过他发信息,并不是这样慢,好像是有意如此。但这又如何?她巴不得多一些这样的小伎俩,好证明他也渴
望有一些事情发生。信息写完,聂于川又看一遍,笑着点头。她就说,那你得请我吃饭。

晚饭的时候,两人婉言谢绝接待方的好意,说是想自己走走。接待方会意地不再坚持。徐佩蓉像是想起了什么,急
匆匆跑回房间。聂于川猜测她一定是换衣服去了。果然,她再出来的时候,已经不是白天的打扮,一身休闲装。聂
于川说,你这样穿戴,倒显得我一本正经了。徐佩蓉快乐地看着他,说那你也去换。他摇头说,本人只知道此行的
目的是考察,又不是逛街,没带。她越发快乐,说那更好办,咱们买衣服去。

出门就有商场,霓虹灯闪烁,像是在招手。徐佩蓉视而不见,直接拦了辆出租车,说去某某商场。聂于川也不去点
破,微笑着靠在坐椅上。和省城远隔千里,又没有老孙、老韩有意无意的敏锐目光,对于暧昧而言,这里简直是天
堂。他打定主意,今晚就让她发挥,看能到什么水平。进了商场,徐佩蓉的手自然地搭在他臂弯。聂于川悚然一
撤,她只抓住了他的袖子。徐佩蓉是有来头的人,可她好像把这些统统放开,积蓄了莫大的勇气才伸出手来。他有
些心软了。就在这一软的刹那,她的手又来了。可能由于孤注一掷的决绝,她竟然捏到他的皮肉。聂于川情不自禁
地叫了一声。两人一愣之后,都笑了起来。

他看着前方,侧头小声说,你就不怕别人看到?你可是女的。

这里又不是省城,谁认识我们?徐佩蓉也小声说。再说了,你是单身,我也是,就算真的,真的那个,起码不违反党
纪国法吧。

听上去挺悲壮的,悲壮之余还有些悲凉。聂于川不再拒绝。两人手臂相挽,一边走一边私语。远远看上去好像真的
在“那个”了。他一副无可奈何的样子,故意总也不看她,手臂僵硬,保持着最初的姿势。这点简单的幸福,对她
而言已是沉重如山。作为高手,他当然知道这一点,所以也不必更多地给予。他有的是她想要的幸福,只是现在还
不是给予的时候,至少不能立即给她全部。一次性给予就像一次性筷子,用过也就没用了。她拉着他进进出出,走
走停停,终于留步。徐佩蓉神气地拿起一件,说,你试试这件。店员夸张地赞叹起来,太太的眼光真好,先生穿上
一定好靓仔的。

聂于川无奈地走进试衣间。他本能地先看钱包。该死。身上只有一千多块,本以为随便吃点什么绰绰有余,就没回
房间去取。卡上自己的钱也没多少。而且真要是连付现金带刷卡,身为男人的面子何在?他气得一拳打在墙上。父亲
的话是对的。小小的副处长,连在女人面前充一充潇洒、玩一玩暧昧都如此困难。而就是这个副处长,他也是战战
兢兢地坚忍了多少年,付出了多少代价才得到的啊。他跟她暧昧,最大的诱惑是她的幕后,而最大的障碍也是。在
世俗生活面前,他的前途、未来、能力、品格全是狗屎,只能估算而无法折现。眼前这个猫戏鼠、鼠戏猫的游戏,
本就不平等,多亏他是高手,懂得把握,善于经营,才保持了相对平均的态势,才不至于让她太有优越感。一旦底
气全消真相大白,她发现他不过也是只猥琐的小蚂蚁,有求于己受制于人,还能暧昧下去?他咬牙切齿地抱着新衣
服,坐在椅子上,暧昧的念头荡然无存。醒目的价格标签不无嘲弄地看着他,提醒一切尚未成功,同志仍须努力。
我本善良,标签也本善良。只是自己的标签上,价位还很可怜。

门外,徐佩蓉小声问,好了吗?

聂于川匆匆把衣服拆开,抖了抖,推门出去。徐佩蓉一脸诧异。他耸耸肩,有点太小了。她释然说,北方人身材要
高一些,怪我没想到。聂于川摇头说,算了吧,我觉得\myrule{}

怎么能算了呢?徐佩蓉皱眉。我已经付过钱了。

聂于川恨不得把衣服团成一团,塞住她微微撅起的嘴。他勉为其难地二进宫,换上新衣。说实话,她的眼光还是不
错的。可惜此刻的他无心欣赏。离开之际,店员躬身说,先生太太走好。徐佩蓉使劲点头,用力挽住他的臂弯。人
流喧嚣中,聂于川突然感到一阵恐怖。这么下去肯定要出事的。如果是别的女人,他还可以控制,但像徐佩蓉,他
实在不敢确保安全生产无事故。他的准备还不充分,她的攻势太过迅猛,一味腻在暧昧里,到头来吃亏的还是他。
这可不是高手的作风。

她的头凑近了他的肩膀,轻轻靠上去。像春风中两枝柳条搭在一起,也像小猫睡觉时前爪遮住眼睛。她的表情一定
很陶醉。他却感觉前后左右全是摄像头,一五一十地录下她和他,变成光盘,出现在老冯、老韩、老孙、小李办公
桌上,出现在某个网站里。他顿时一个激灵,下意识快走一步,她的手和头都落空了。他有些尴尬地回身。她已经
垂下头,额前发丝遮住了眼,看不到表情。她仿佛弄丢了心爱玩具的乖孩子,不知哪里寻找,不懂怎样耍赖,又不
敢放声痛哭。聂于川走近,看着她,说对不起,我觉得\myrule{}太快了。

徐佩蓉并不抬头。如果你真的这么想,我可以等。可你总要告诉我,你究竟对我怎么想的,你究竟会让我等多久啊。

四周都是来来往往的人。他俩像是剪刀,把平整的人流裁成两列。聂于川怎么能对她说,等我当上处长,当上厅
长,再跟你好?他只能缓缓地摇头,说我不是木头人。你对我的态度我都能感受得到。徐佩蓉终于抬起头。她的脸上
全是泪,而声音却固执得像砖头。你还是没有对我说,你会不会爱我,会让我等多久。

我只能说,就像这个。聂于川耐心地看着她,指了指旁边的一个招牌。她看过去,那里写着“一切皆有可能”。这
样的幽默恰到好处。既不拒绝也不接受,又留下了充足的空间给以后。徐佩蓉轻轻一笑,长长地叹息、摇头。聂于
川松了一口气,用掌心抚住她的肩角,微微用力,转过她的身子。两人朝大门走去,再也不讲一句话。

离开广州前一天晚上,徐佩蓉在告别宴上拼命地喝,开了白酒、红酒、啤酒的酒戒三种全会,喝得接待方五体投
地,也把自己喝得酩酊大醉,吐了好几次。到了最后,她连吐的力气都没有了。聂于川扶她回房间,她像个祭品,
软在床上四肢舒展,脸庞光泽闪耀。他褪去她带着秽物的衣服,只剩白色的内衣。她浑身都是汗,他也是。她的身
体在灯光下,到处亮晶晶的,毛茸茸的。他在床边坐下,指尖轻轻触及她的皮肤。如果她是装醉,肯定会有战栗。
但是没有。她平静地躺着,毫无反应,任凭男人的指尖游走,听任男人的任何举动。他的头里霍霍地响着,像是火
车在山洞中叫嚣,也像是钻头在石壁上跳跃,所过之处碎屑横飞。他还在试探,试探是因为不放心,不放心是因为
顾虑太多。坐在她身边,他感觉到了自己的坚硬,又柔软,又坚硬。他远不是正人君子,他做惯了小人和孙子才做
的事。可是偏偏眼前唾手可得的占有,他却难以担当。他甚至想,她为什么不是苏一文呢?为什么要有背景呢?他现
在不是不想玩,而是玩不起。如果他和她实力持平,背景相等,他就会毫不犹豫地放纵本能。这些他都没有。不仅
没有,还可能因此失去既得的全部。所以即便是男人的本能,他也不得不扼杀掉。这是另一种本能,无关道德,无
关修养,仅出于恐惧。他最后看了她一眼,拉起被子,代自己压在她的身上。合上房门,站在走廊里,他感觉硬邦
邦的地板上波涛滚滚,他就仿佛是巨大风浪上的一艘小舢板。走也走不动,站也站不住。想伸手扶墙,没想到那里
也是汹涌澎湃。他踉踉跄跄地走,不无悲哀地想,这都是因为他现在是个不上不下的副处长。级别高一些,就有了
底气;或者低一些,就没了顾虑。可惜他底气尚无,顾虑甚多,于是连做一回男人也成了奢求。

离开广州,到了上海,继而是南京。老冯在马来西亚打来电话,说后天回国。两人只得多逗留两天。这段日子每到
夜晚,徐佩蓉都要以各种理由到聂于川的房间,要么打文件,要么聊天。对于那晚的事,两人心有默契地都不提起。
离开前的晚上,两人一直聊到十二点多。他打了个呵欠,嘴里却说,茶凉了吧?我再烧点水。徐佩蓉莞尔道,你明明
是暗示我该结束了,老奸巨猾。这就是所谓领导艺术吧。

我不是领导。聂于川摇头。老冯才是领导。

我不是指官位。我指的是我的心。在那里,你是领导。

聂于川笑起来。夸张了吧?明天老冯就到了,我劝你还是早点休息。让他意外的是,徐佩蓉并不再说什么,顺从地站
起,笑笑就离开了。这倒让他有些看不透。如果是不再恋战,她又何必夜夜来聊天?如果是不死心,又怎会说走就
走?聂于川抽了两支烟,思绪跟烟雾似的飘忽不定。他来到大落地窗前,拉开窗帘。远处昏黑的一片依稀就是玄武湖。
他重又点上烟,深吸一口,拿起电话。

怎么会给我打电话?没拨错吧。

我也不知道。你不想听,挂掉就是了。

我想听。你说吧。

说些什么好呢?聂于川踌躇了。暧昧与真话并不兼容。他当然不能说,我有些想你了,我不想失去你,但我也不敢现
在就得到你,所以我们只能暧昧。他听到她的呼吸声,仿佛月光下玄武湖上一波波荡来荡去的涟漪。宛如两人就在
湖畔,而她就在身边。不知静默了多久,他终于说,你那里看得见玄武湖吗?

当然,我就在窗前。她笑了笑。你也是吧。

是啊。不但有玄武湖,还有月亮。

徐佩蓉还在笑。聂处越来越像个诗人了。

诗人有的,我没有。诗人没有的,我也没有。我怎么会像诗人?只是个普普通通的男人。

男人哪。徐佩蓉叹口气。动情容易,守情难。动心容易,专心难。而我们的聂处呢,看不出动情,也不像动心。守
情和专心就更谈不上了。

那我算什么人呢?

她不回答,却说你见过盖大楼吗?设计、施工、监理、验收,很辛苦的。我就像在盖楼。我做了很多准备,很投入很
仔细地去盖。而你呢,就像是来拆楼的。

聂于川马上警惕起来。这才几天。徐佩蓉的成长太快了。她的话若即若离,点破又不说破,看透并不讲透,说得轻
松留下沉重,这都是高手才有的作风。他换了个姿势,认真地斟酌着。世间万物好像突然销声匿迹,只有他和隔壁
的她。她无非想让他承认,他却不肯,因为承认背后就是承诺,承诺背后就是承担。而对承担,他觉得还无能为力。
他现在不想让她离太近,但也不想把她推太远,就在目光所至触手可及的地方最好。困难之际可以帮帮忙,疲惫了
可以解解乏,繁忙时又可以不挂念,冷落她还可以不担心。这多好啊。

两人一时无语。静谧的沉默中,聂于川终于顿悟,继而彻底弄明白了自己的处境。他的徘徊和痛苦并非来自暧昧,
而是源于自己。徐佩蓉有光环笼罩,人人侧目敬畏,在她的光环照及自身时,看似遥不可及的副处长居然到手。他
是受益者,所以无法也不忍断然拒绝她。但也正是她的光环太过耀眼,让人看不清,深怕投鼠忌器,也怕得到之后
守不住,故而自卑,故而不敢爽快接受。这就是他一直以来进退维谷的原因了。

聂于川慢慢说,我想知道,你是为什么离婚的。他还是忍不住去问。他太想探究她的光环了。他的问题很突然,徐
佩蓉愣了一下,说这很重要吗?

不方便就算了。当我没问。

其实也没什么。他总在外边乱搞,我受不了,就离了。不过,他的家人对我不错。她苦笑说,他父亲跟钟厅长很
熟,我调到七厅也是……

她的话戛然而止。原来如此。他屏住呼吸,又长长地出了口气。徐佩蓉的声音稍微有些沙哑,也有些激动说,这都
不重要,关键是你知道我爱你,可你爱我吗?

她刚才的话还没完,她想说什么?聂于川还在衡量着。他忽然感到很悲哀,很倦怠。明明可以两情欢悦的,但限于各
种说不清道不明的缘由,他不能够去爱,又不忍放弃,唯有尴尬地暧昧着。他只好深沉地兜圈子,说我们都吃过婚
姻的苦。悲欢离合,阴晴圆缺,有太多的事情是我们根本无法左右的。比起这些,我们是多么渺小。可我们偏要在
这里说爱,说不爱,说不顾一切,好像天地都是我们掌握似的。

我明白了。徐佩蓉的声音有些气恼。你的意思是,我们左右不了什么,所以不提结婚,但可以恋爱;不谈爱情,但
可以暧昧。

婚姻让我很辛苦,爱情也如此。如果威威的妈妈没有死,我到现在可能还不知道她做了对不起我的事。就算我和你
在一起了,就一定会幸福吗?至少我现在还不敢确认。他说了一半真心话。他是真的不敢确认,只能把一切矛头转向
曾受的伤害。

所以你想慢慢确认,想慢慢来。来什么?暧昧吗?她沉默了一会儿,大概相信了他的托词,说你要知道,我不是那样
的女人。如果是,我根本不会离婚的。

我知道,一切都顺其自然。好不好?他说了这句,她再也没有回答。很久了,他简直以为她已经睡着了,然而那边终
于挂了电话。硬冷的塑料撞击声落在他心里。他可以猜到徐佩蓉是多么难过,但错不在他。如果她只是个寻常的离
婚女子,他就再无犹豫。他会马上到她的房间去。平心而论,他是爱她的,两个人也本可相爱。但这又如何?他只能
暧昧,只能等待,只能在无法估量的日子里去决定接受或者拒绝。这一切都不由他们,不是相爱就能结合。如同提
拔不由自己,不是有能力有水平就能升迁。抽掉最后一支烟,他想,每个人对暧昧的理解都不同,他认为暧昧就是
暧昧,而她认为暧昧是婚姻的前奏。在这个问题上,他是游戏规则的制定者,她却不是。这就是她痛苦的源头。她
打算退却了吗?他有些遗憾。其实这也没什么,他安慰自己,只当是一段暧昧结束了吧。结束了也就结束了。暧昧本
身就是生活的副产品,给平淡的日子添一抹色调而已。

第二天见面的时候,两人的神态和平常一样。昨晚的彻夜对话像是根本没有发生过,顶多仅仅是两人做了同一个梦。
在梦里说的云遮雾罩的话,再怎样也是不切实际。下午,老冯匆匆飞到南京。他连机场都没出,马上带着聂于川和
徐佩蓉飞回省城。老冯的急切不无道理,厅里出事了。确切地说,是老任出事了。

\section[\thesection]{}

聂于川回到六厅,老任已经消失了两天。有人说是双规,有人说是逮捕,有人说是接受调查。总之人不见了,但事
情还未盖棺。在悬而未决与尘埃落定之间,许多人成了倒树猢狲,惶惶不可终日。老冯和聂于川就是如此。老冯在
厅里待了半天,见事情千头万绪,便借口党校课程紧溜之乎也,躲清静去了。聂于川没课可上,无处能躲,考察总
结也尚未完成,只有老老实实蹲在办公室里。他敲着键盘,心中全是旁骛,浑身布满杂念。就算总结写好了,该交
给谁呢?此事是老任分管,按理说该交给他。此情此景怕是不好办。不过老任确实是命悬一线,但谁知这线是棉纱还
是钢丝绳?

聂于川提拔得顺利,虽然有徐佩蓉帮衬,有钟厅长赏识,不过他的直接领导是老冯,老冯的直接领导是老任,说来
说去逃不过老任的影子。何况老任几次越级直接给他安排工作,厅里人都看在眼里,难免有想法。本来,一个研究
生毕业、五尺高的男人,被人呼来唤去形如家狗,就是可悲;甘为五斗米摧眉折腰献媚领导,自觉地化家狗为走
狗,那更是可鄙;如果刚努力当上走狗,主人却没了,重新沦为野狗,可谓双料的可耻,踢一脚还脏了鞋。以往在
办公室里坐着,不时会有人进来,笑着叫声聂处,吸几支烟,喝两口茶,聊聊工作,说说天气。老任出事之后,这
里摇身一变,成了野鬼唱歌的乱坟岗,大白天都无人问津。给人打电话,明明是说公事,也被淡淡几句应付了。聂
于川有些生气,老子脸上又没写“任”字,犯得着吗?生气之后是不安。万一传闻属实,该如何应对?反戈一击并不
难,别人的目光再鄙夷也无妨,关键是重新归属的落脚点不易找到。不安之后,当然是难过。没想到父亲曾经的痛
楚阴魂不散,不请自来。一切都乱套了。他也想过请徐佩蓉帮忙。但这次出差,她是怀了多大的希望去的,归来时
却一无所获。她恨他还来不及,这两天明知他的窘况,连句关心的话都没有。他陡然后悔起来。应该在广州把她拿
下的。钟厅长自不待言,老钱也屡次表示看好她,拿下她,就像是穿上了防弹衣,厅里就是天翻地覆,也可以不惧
了。可惜自己前怕狼后怕虎,居然拒绝了她。简直是大傻。他好容易平静了一些,有人敲门进来。他惊诧地进出一
丝笑,说是小徐啊,有事吗?

徐佩蓉在他桌边坐下。有些事情,我还是想跟你说一下。

聂于川飞快地揣测她的来意。是嘲讽?是可怜?还是来挽救?难道她还爱着自己吗?他勉强笑了笑,你说吧。一个处
的,又是老校友,别见外了。

徐佩蓉微笑。我就说嘛,你穿这件衣服很好的。她的声音有些凄然。

聂于川摸了一支烟,点上,笑起来。他的笑容沉重得仿佛秤砣,在脸上挂都挂不住,掉在桌面,发出铿然的声响。
徐佩蓉显然是听见了,叹口气,说师兄,我想告诉你的是,老任就快回来了。

聂于川强忍住没说话,狠狠抽了口烟。徐佩蓉见他不吱声,解释说,我前夫回国了,他有个朋友知道一些。我和他
昨天见的面。

听起来不像是假的。可这也太巧了吧。聂于川弹了弹烟灰。他说,没事就好。她垂下头低声道,是啊,没事就好。
他看着她,犹豫半天,还是说你能肯定吗?

当然。她的头垂得更低。他跟人聊的时候,我听见了。不会错的。

聂于川这才放心。他知道她能说这些话已是不易。不过,怎么又冒出来个回国的前夫?还见面了?他安慰自己没必要
吃醋,徐佩蓉又不是自己老婆;又忍不住罪恶地想,其实就算他们不只是见面,而是上了床,做了爱,也是老一套
了,又不是陌生人。想到这里,他遽然发现自己还是在吃醋,他真的爱上她了。他颤声道,别说了。谢谢你。徐佩
蓉缓缓摇着头,并未抬起。他继续说,我早发现了,你跟别的女人不一样。

她一下子昂起头,有些不满,有些委屈,有些恼怒。她说,我不喜欢你拿我跟别的女人比较。

有比较才有鉴别嘛。聂于川笑道,就像你送我衣服,不挑挑拣拣怎么选得出合体的。

更不像话了。徐佩蓉虽这么说,脸上却有了笑意。连挑挑拣拣都出来了,女人真的就是衣服吗?

你的不同之处,是你总爱垂头。

垂头丧气而已。她笑起来。你就这点发现啊。

每次见你这样,我都有些难过。我忍不住想,是什么让你不舒服,让你为难,让你想逃避。他递过一张纸巾,示意
她擦擦眼泪。她乖乖地照做,说,你放心,我不会再见他了。我以前的婆婆病了,他说一时到不了,要我去帮忙照
顾一下。谁知他又过来了,还带了一堆朋友。

你不要再这么说了。聂于川还是说出了心里话。不过,你能不能答应我,以后别再跟他见面。好不好?

徐佩蓉的眼泪又出来了,擦都擦不及。她欢喜地点着头,哽咽着说不出话。你这样肯定没法再回办公室了。他又递
给她纸巾,叹息道,这样吧,你今天就别上班了,回家好好静一静。徐佩蓉为难说,我也不想让老孙、老韩看见这
副模样,可包还在办公室啊。聂于川不假思索道,那你去某某路的某某饭店,开个房间,我办完了事去找你。她的
眼睛顿时睁得好大,情不自禁说今天我\myrule{}他不容她说下去,把钱包递给她,简短地命令:听我的,去吧。

她走了。聂于川在办公室里来回踱步。徐佩蓉瞬间被抛到脑后。老任居然还能全身而退,可见其资本雄厚法力无边。
钟厅长想搞好工作,少一个有实力的对手固然可喜,但多一个能办事的搭档也算不错。徐佩蓉的信息很及时。大海
航行靠舵手,舵手要靠指南针。现在徐佩蓉就是他的指南针。谢天谢地谢人,他知道该怎么做了。

从钟厅长办公室出来,聂于川自觉两脚生风,心旷神恰。他再不流连,直奔宾馆。可举手敲门之际,他又犹豫了。
他很清楚进去后会发生什么。作为离婚少妇,她长相不错,身材尚可,有经验,懂配合,算得上是个尤物。刚才在
钟厅长那里,他嘴里在汇报,眼前却总是浮现出一个男人压在徐佩蓉身上的画面。他们在不停地翻滚,不停地呻
唤,男人兴高采烈,女人心满意足。那个男人的脸时隐时现,时而是他,时而是一个陌生的面孔。徐佩蓉显然爱的
是他,不是那个男人。但躺在她床上、享受她肉体的倒是后者。他在钟厅长办公室里竟然坚硬了起来。按理说他已
经过了冲动的年纪。但是,他又实在找不出继续克制冲动的理由。他已经克制太久了。即便要顺其自然,也该发展
到这一步了。他的手指终于按在门上,那声动静又短又轻,像是一枚树叶伏落于地。可就是这个瞬间,门开了。徐
佩蓉泪流满面地站在门口。她说,我一直在看着你,我知道你一定会敲门的。他不再说话,拦腰抱起她,直挺挺地
走进房间。她倒在床上娇喘,他粗鲁地剥去她的衣服,随手扔在床边。一切都很顺利,很自然。她很快衣不遮体了。
她慌乱地叫着不要,不要。聂于川压了上去。最后一个关口,徐佩蓉猛地拦住他的手,死死护住了下身。他的双眼
血红血红,凶狠地盯着她。她喃喃地说,对不起,今天不行。

为什么?聂于川野兽般低低地吼着。

她眼角飘着泪,羞惭万分道,来那个了。不信你看。

他掰开她的手,难以置信地看去。果然如此。他张大嘴,只是不知该放声大笑还是放声大叫。多可笑的事啊,简直
像某种行为艺术。难得有适合的铺垫,适合的情调,适合的环境;难得他已决定接受,她也执意付出。可老天偏偏
不许,大笔一挥,统统抹杀掉了。错过今天,什么时候才有如此天衣无缝的机会呢? 然而生活就是这样,一切都是
这样。人太脆弱了,再精心的安排也敌不过一个小小的意外。在冥冥的主宰面前,他和她唯有俯首帖耳的份儿。

老任回来之后,一举收复了所有失地,老钱处于战略防御态势。老冯结束了党校学习,不久就荣升党组成员。但是
也不够完美。他没能当上副厅长,只是助理巡视员。当然这都是大家的揣测,大可一笑置之,并不能当真。无论如
何,老冯一走,聂于川就顺理成章地主持了八处的工作。而且钟厅长对他暗示过,八处是核心部门,处长一职不会
空悬太久,只要时机成熟,他就是七厅最年轻的正处长。一开始他还觉得这太突然,但想到徐佩蓉和钟厅长的关
系,又觉得这很正常。徐佩蓉当然有她接近钟厅长的渠道,她既然能在关键时刻拉他第一把,就会有第二把,第三
把。他没有去问她,她也没有邀功。暧昧的人彼此付出,根本没有道理可言。

他虽说还是副处长,毕竟是在主持工作。老陈作为八处出去的老同志,送来一辆车作为祝贺,说是借给处里便于开
展业务。车在设计院名下,各种支出自然由陈书记负责。处里开会,不再一人之下四人之上,也可以发号施令了。
然而聂于川还算年轻,还要奋斗,还有空间。副处长和正处长,仿佛一低一高两个台阶。主持工作好比穿上了高跟
鞋,虽然位置不变,高度却有了。不过高跟鞋穿着并不舒服,走起路一摇一晃,仍不如脚踏实地的感觉好。要想实
实在在地上一个台阶,就要低调。低调是门学问,内涵很多,外延颇广。比如用车方便了,就得多想想处里的同志。
小李和女朋友避孕失败,不得不结婚,聂于川就安排车辆接他的准岳父岳母来省城。在暧昧上更要低调。况且徐佩
蓉也主动提醒他,要注意形象。什么是形象?机关男人的好形象,无非是有人缘,有能力,作风正派。大概女人对不
正派的事都很敏感,徐佩蓉也不例外。她对他的人气和水平并不担心,而他正派与否,说到底还是取决于她。

那天之后,聂于川对暧昧有了新的升华,再没有跟徐佩蓉有过什么亲密接触。两人的暧昧纯洁得宛如空气,而空气
是不可或缺又无处不在的。他想,高手也需要不断进步,也需要发展,总是停留在原地,早晚会被超越。在他心
里,如果说徐佩蓉以前是对手,现在则是伙伴。和对手是你死我活,与伙伴是共同进步。何况她的成长也很快。她
已经默认了聂于川若即若离的态度。熏陶日久,徐佩蓉误以为他是精神恋爱的信徒,为了不被瞧不起,她也努力成
为高雅的柏拉图一党。显然她是错的。高中生都知道客观规律有其普遍性和特殊性,聂于川对她精神恋爱,不代表
对别人也是。和久违的苏一文通电话一个多月后,他果断地策划了一次饭局,理由是她帮忙让徐佩蓉表弟吃上了财
政饭。本来要带徐佩蓉去的,偏巧她不舒服,就未能成行。这就省去了他和苏一文之间的一切繁文缛节。两人默契
地直奔主题。云收雨住之后,苏一文细细地帮他擦拭,还是熟女懂得体贴。聂于川想,按说徐佩蓉也不小了,就不
如苏一文懂。

苏一文慧眼如刀,见他闭目不言语,笑道怎么,想你的小朋友了?给我说说她。聂于川一笑,只说她姓徐,是同事,
离过婚,30岁了。徐佩蓉的背景他没说,因为苏一文也是高手,他唯恐她笑他吃软饭。

好好培养培养,是个老婆的苗子。对了,你准备什么时候再婚?

再等几年吧。你不是也闲着。

我快结婚了,也就是今明两年吧。

聂于川好奇心大起,追问新郎是谁。苏一文平平淡淡地说,是三厅的老厅长,年龄到站退居二线,不是人大就是政
协,老婆去年不在了。聂于川谄笑说恭喜老姐姐梅开二度花正艳,春风又绿江南岸。苏一文笑着打了他一下,说他
可能管七厅这个口,需要帮忙别客气。聂于川一愣,这倒是个意外收获。他自然不会客气,对老新郎,对苏一文,
都不会客气。

时候不早了,聂于川准备告辞。苏一文忽然道,别对你的小朋友太苛刻了。你奔四的人了,也别嫌弃人家离过婚,
差不多就娶了人家吧。聂于川一边穿衣服,一边笑道老姐姐挺会关心群众的。苏一文叹口气,说你就是没正形。女
人是等不起的,过了三十岁,比二十多岁更娇嫩,说话间就要枯萎。这个年纪的女人,想要不靠一纸婚书而抓住一
个男人,尤其是你这样的男人,太难太难了。小徐她不傻,她知道的。

聂于川的动作停下来。他沉默了一会儿,说老姐姐,你觉得她适合我?

我最适合你,可你要我吗?苏一文笑起来。聂于川赔着苦笑。苏一文说,你我这个年纪再结婚,不过是各取所需而
已,没什么适合不适合的。小徐需要的,只是两个人在一起。你需要的,是一个能孝顺老人,会教育孩子,出得厅
堂人得厨房的女人。既然给予对方的都不困难,何苦这么拖着?你别忘了,女人的青春最不易留,你把人家青春的尾
巴都耽误了,小心遭报应。

苏一文最后一句话让他很震撼。她是个饱经风霜的女人,与自己并无利害冲突,而且有过肌肤之亲,她的忠告应该
没有歹意。他开车回家,一路上都在沉思。思绪像催租的悍吏,叫嚣乎东西,隳突乎南北。到老家属院门口,他停
下车,点上烟,静静地抽着,心烦意乱地抽着。或许苏一文说得不错,他再暧昧下去,的确要遭报应的。徐佩蓉够
不错了,拥有背景却毫无优越感,甘受招之即来挥之即去,冷落也行,暧昧亦可,还能主动提醒他注意分寸,别做
傻事。一次聂于川生病在家,徐佩蓉借口来送文件,实际上是看望。父亲得知她就是耳熟能详的徐佩蓉,非要留她
吃晚饭。徐佩蓉大显身手,做了一桌子菜。腾腾热气,浓浓饭香,父亲、母亲和威威都吃得神清气爽。母亲甚至当
面要求他送她回家,全然不理他还在咳嗽。回家路上,徐佩蓉一直挂着微笑,一点城府和掩饰都没有了,眼角还有
些许泪花。从此一到放假,父亲母亲就让他请小徐来家里做客。而她每次都不忘给威威买玩具买衣服,给老人带补
品带礼物。几回下来,居然讨足了一家老小的欢心。想到这里,聂于川不由得笑了。他把烟头扔出去,随手拧大了
电台的音量,靠在椅背上。

到底是不是走出这一步呢?他还是有些犹豫。他毕竟只是个主持工作的副处长,离处长的目标还剩一步之遥。如果提
了正处之后再结婚,就完美了。而且七厅有个不成文的规矩,夫妻双方不能在同一单位,真要是结婚了,徐佩蓉怎
么安置?无论在何处落脚,她自然都无怨无悔,可为了今后的生活,总不能安排得太差吧?厅里既有成规,打破了难
免惹人非议,也背离了低调的原则……

电台忽然发出一阵粉丝的尖叫,暂时中断了他漫无边际的思路。周杰伦跟着唱了起来:

\begin{center}
{\itshape 该不该搁下重重的壳

寻找到底哪里有蓝天

随着轻轻的风轻轻地飘

历经的伤都不感觉疼

我要一步一步往上爬

等待阳光静静看着它的脸
}
\end{center}

聂于川怀疑这首歌是不是专门唱给他的。该不该搁下重重的壳。太形象了。我要一步一步往上爬。太贴切了。此情
此景,此曲此歌,仿佛脚气病人背着人使劲抠着脚指缝,又解痒又自在,舒爽无比。原来重重的壳与往上爬并不矛
盾,而且彼此依存,互为因果。聂于川想,看来自己又要进步了,不但暧昧上要进步,工作上也要进步。

苏一文的婚期很快就到了。时间是元旦。选择在公历新年伊始办喜事,越发显得一对新人大公无私。婚宴并不夸
张,只邀请了信得过的人,总共不过五六桌酒席。聂于川有幸被邀,自然受宠若惊,因为在场的除了新娘,似乎只
有他还是处级干部。老新郎挨桌敬酒的时候,苏一文特意给他介绍聂于川,说这是我的好朋友小聂,在七厅八处工
作。人很年轻,已经主持工作了。老新郎笑笑,说你们钟厅长是我小妹妹,你既然是小苏的好朋友,以后常来家里
坐坐。聂于川听见这话,喝死在当场的心都有了。苏一文揶揄地笑,似乎看穿了他的心思。毕竟四十岁的人了,她
没有穿得大红大紫,简简单单的一身水红色中式夹袄,腰身收得很好,中年女人的风致显露无遗。聂于川遗憾地
想,可惜结婚了,今后只能远观而不可亵玩焉。

婚礼是在周六,宴请已毕,聂于川还要回厅里加班。关于那个大项目的报告几经修改,又请省政府的几位大秘把了
关,估计最后完善一下就可以上报了。聂于川折腾了一个下午,终于大功告成。这份报告前后历时四个多月,要说
贡献,他算是居功至伟,不过至伟也就至伟,万不可自傲。还是得低调。省里一旦批下来,厅里自然会论功行赏。
老任、老钱、老冯都跟他说过,项目上马后,他就是管委会里管基建的副主任,好几个亿的大工程,基建是重中之
重,这不正是领导关怀吗?有付出未必就有回报,但不付出肯定没有。聂于川握着厚厚的一沓文件,像握着自己的后
半生,澎湃的心潮急于找人分享。电话刚一接通,徐佩蓉就说,你猜我在哪里?他快活地说猜不到。她笑着说,我领
着威威逛商场呢。聂于川心里一暖,说你们玩儿吧,我得再加会儿班。晚上一起吃饭。

徐佩蓉的成熟让聂于川刮目相看。他已经做好了提拔正处就结婚的准备,而她却久已不提什么爱不爱、结婚不结婚
之类幼稚的话题了。好像她默认了两人暧昧的状态。这么长时间了,他那点态度和底线,她了解得很清楚,反倒放
心。他不马上挑明,她就不去强迫;他不急于结婚,她也听之任之。他要暧昧就随他,只要他不跟别的女人暧昧就
好。她和他同一部门,办公室一墙之隔,他每天在干什么,应酬时都有谁,应酬后去了哪儿,她都能洞若观火
\myrule{}只要她想。即便没有具体的承诺,缺乏婚姻的保障,她也有信心把他牢牢地拴在身边。经过漫长的磨砺,
进出无数个关口,徐佩蓉也算是高手了,这都是他逼出来的。日子一久,厅里人都看得出他和她的关系。其实在她
还是新手时,热情不懂遮掩,出招大开大合,大家就有所觉察,私下里也有过非议。好在徐佩蓉她来头特殊,他行
事低调,两人又都是独身,郎情妾意的事情谁也不好说什么,只是觉得她有点过于奔放,不太合纲常。发展到今
天,大家已不再关心他们是不是在相好,而是揣测他们什么时候结婚。道理很简单,聂于川不是同性恋,也不是柳
下惠,肯定早已得手。既然睡都睡了,人家条件也不错,为何吞吞吐吐不肯结婚?难道是玩弄?这就牵涉到道德和作
风问题了。如此一来舆论风头陡然劲转,倒是聂于川势成骑虎,仿佛拼酒时不得不含了一大口,吐又不便吐,咽又
咽不下。得民心者得天下,民心得了,天下就得了,区区一个老男人,还怕得不到?徐佩蓉当然明白这些,就越发有
信心。她也满心希望他能够再上一层楼,双喜临门的事情谁不憧憬呢?

春节过后,省里的批复正式下来。七厅上下群情欢动。接着就是学习批示,领会精神,组织动员,统一思想,常规
流程过后,管委会正式成立。聂于川不负众望地兼了副主任。基建伊始,他忙得不亦乐乎。徐佩蓉当仁不让,舍我
其谁,自觉做好后勤。以前和聂于川父母打电话,她都要躲到楼梯间去。现在不必了,在办公室里就可以。老孙长
叹几次后,也就懒得再去感慨,就是摔茶杯又有屁用?还是打打乒乓球,锻炼锻炼身体更实际一些。徐佩蓉没有孩
子,出于母性,对威威很上心。跟老韩议论的话题也从做头发、买衣服、购物,转为孩子健康、学习等等。一次办
公室里没人,老韩忍不住问她什么时候结婚。徐佩蓉既不否认又不承认,只是摇头笑笑说还早呢,又不是没结过,
跟多稀罕的东西似的。老韩笑个不停。妙就妙在两人并没说起男方是谁,老韩没问,她也不说。因为老韩觉得无须
问,她也认为不必说。反正都知道就是聂于川。

到了五一,基建已经初具规模,省里下来视察,带队的正是苏一文的丈夫。这种场合,厅长们自然是全程陪同的。
老新郎对聂于川还有印象,有意当着众人问了他几句,聂于川的回答也很到位。看得出,厅长们对他的表现很满意。
厅里已经在研究八处的正处长人选了,老新郎在这个时候出现得再好不过。又过了十几天,老任把他叫了去。老任
主管人事,进门之际,聂于川幸福得两脚发软。应该是代表组织谈话了。谈话之后,就是考核,然后是公示。公示
结束,正处就到手了。正处到手,就该结婚了吧?

老任倒是四平八稳,问了问最近的工作,表扬了一番。聂于川的态度谦虚而低调。老任并没马上进入主题,话锋一
转,说你是不是认识苏一文?六厅的。

认识,还挺熟的。以前一起下过工作组。

老任点点头。苏一文的丈夫,就是前些天来视察的领导,专门跟我提到了你。让我对你多关照。

聂于川不敢多说话,只是欠了欠身子。热血汹涌流遍周身上下。

老任说,你和你们处里的小徐,关系怎么样?

聂于川不知道该怎么回答。斟酌了几秒钟,他说,挺好的。

小徐以前的爱人回国了,你大概知道吧?当然,小徐对他有意见,不然也不会离婚。事情都要向前看,现在他提出来
复婚,小徐却不同意。我跟他是朋友,他就托我做做工作。我想这种事情,我不太好出面。你是小徐的领导,也是
朋友,所以我希望你能帮助我做做她的工作。劝和不劝分嘛,能破镜重圆,也是功德。

聂于川盯着脚尖,他想说,操你妈。

厅里对八处的工作很重视,八处是重要部门,正处长也不能老空着。你主持工作这么长时间,也该动一动了。小聂
你前程远大啊。

接下来的话,聂于川统统听不见了,只看见老任嘴唇一张一合,时笑时静,像极了打盹的河蚌。出了门,他连路都
走不稳,重心时而倒向这一边,时而倒向另一边。好容易回到办公室,他拼命抽了几支烟,定下神来,给苏一文打
电话。他现在也只有打给她了。

苏一文默默地听后,说弟弟你别着急,有什么想法也别表达出来,老姐姐帮你打听打听。对了,你告诉我小徐的名
字,什么时候离婚的。

聂于川看着电话,像看着生死簿,眼神寸步不离。一个多小时过去,他抽烟抽得嘴都麻了。电话刚响,他就闪电般
地拿起,却一句话也说不出。苏一文略带指责地说,你早跟我讲就好了。这种事,你跟我还隐瞒什么?

聂于川哆哆嗦嗦地点烟,怎么点也点不着。他无论如何都想不到,徐佩蓉的前夫如此有背景,这就是她暧昧不明的
一切。一开始,钟厅长们的确是打算让他接处长,可他和徐佩蓉正暧昧着,而她和前夫一家的关系,谁都吃不准,
也可以说是暧昧。有两种暧昧已是复杂,偏偏老任这次出事,她前夫马不停蹄地回国,一番运筹之后,成功将他捞
上岸来。老任深知她前夫对徐佩蓉旧情难舍,虽已离婚,却似乎不愿她再跟别人好。出于知恩图报,老任先是找到
徐佩蓉,婉转地建议她跟前夫见面,交往,重新了解,说不定还能复婚。徐佩蓉当然是一口拒绝,也当然不会告诉
聂于川老任的好意。老任见徐佩蓉无动于衷,索性直接找聂于川摊牌。

苏一文说,你打算怎么办?

聂于川只知道沉默。苏一文不追问,也没挂电话,就那么静静地等着。事情其实很简单。老任在他这里得不到答
复,自然会去找钟厅长。钟厅长也无法核实真伪\myrule{}这种暧昧的事,找谁核实去?于是局面马上明朗了,那就
是他断然做不得正处长。投鼠忌器,每个人都会考量考量,何况是厅长们,何况是提拔。

聂于川终于说,我不要正处长了,我要结婚。

苏一文笑了笑,说我知道你肯定会这样,我替小徐谢谢你。你也别太灰心,我给我老公说说,看能不能帮忙挽回一
点。

谢谢老姐姐。我知道了,我会泰然处之的。

话虽然这么说,放下电话,聂于川还是掉了眼泪。他一边擦,一边去把门反锁上。不料泪水越擦越多,越擦越密。
他实在是真的难过。不知是太看重这个正处长,还是即将到手又蓦然失去的落差,抑或是一番辛苦,八处的工作有
目共睹,到头来居然成全了别人,这让他一时难以承受。在他的概念里,正处长一到手,就和徐佩蓉结婚,再不暧
昧了。可现在所有遽然已是空想。整整一个下午,他坐在办公室里,谁的电话都不接,谁来敲门也不开,就那么坐
着,像个得道的高僧。他随便挑了篇新闻,一字一句打了起来。新闻很快打完,就全部删除,再打一遍。不知打了
几个回合,他的脑子才慢慢恢复正常。他把新闻打印出来,团成一团朝天空扔去。纸团落下,砸倒了桌上的相框。
那是项目开工时管委会的合影,钟厅长、老任、老钱、老冯都在,他也在。大家一团和气,都戴着橘红色的安全
帽,像一盏盏欣欣向荣的火苗,映得一张张笑脸如火如荼。那个时候,他是多快乐,多骄傲,多飘飘欲仙。不过几
个月后,一切已恍若隔世。错过了这次提拔,虽说不至于万劫不复,至少是个惨痛挫折。好像跋涉万里终于找到了
心爱的女子,却看见她正欢天喜地地跟人洞房花烛,还得笑着送上祝福。那份失落,那样不堪,那么不值得。

敲门声又起,徐佩蓉小声说着,聂处,聂处\myrule{}于川,你在吗?

聂于川长叹一声,站起,开门。徐佩蓉进来,诧异地看着他。抽了这么多烟?你怎么了?都下班了,一个下午都没见
你出来。

他没说话,冷冷地反锁了门。她还在说,威威奶奶的中药快没了,我给她买了一些,记得带上……

聂于川突然粗暴地抓住她,朝办公桌那儿推。徐佩蓉惊愕地看着他,傻住了。他一直沉默,手上的力度丝毫不弱。
他把她推倒在办公桌上,翻起她的裙子。没有任何前奏,没有一点铺垫,他和她都毫无准备,就进入了。徐佩蓉死
死地咬着自己的手指,泪流肆意,她一时猜不透他何以如此,但一声不吭,也不反抗,只是默默地承受着。他的动
作很剧烈,撞击力把整个桌子都撼动了。文件、报纸、笔筒、烟灰缸,桌面上所有的东西都随着战栗起来。他的目
光落在合影上。钟厅长、老任、老钱、老冯,一个个都在笑,开始笑得一本正经,后来都绷不住了,捧腹大笑,前
仰后合,全然不像一群厅局级干部。他们不约而同地从合影里走出来,围着聂于川和徐佩蓉,吸着烟,在热烈地讨
论什么,对他的动作评头论足,声音很大,笑语喧哗,好像还有人鼓掌。照片上只剩下他一个人肃穆地站着,身边
空空荡荡,橘红色的安全帽扔了一地,好像四处都在燃烧。聂于川闭上眼,不敢去看火堆里的自己。他还在撞击着。
这是两人的第一次。然而他们都疑惑是不是第一次。在以往暧昧的日子里,在两人的幻觉中,已经不知这样多少次
了。他们有过太多的机会,比现在好得多,有情调,有气氛,有准备。可她太主动,他太精明,两人都在得失之间
一步步精心算计着,试探着,退缩着。如今不再暧昧,忽然变成真的,难免有些恍惚。周遭猛地安静下来,不知是
厅长们都走了,还是都又回到了合影照片里。他抖着双腿,觉得地板也在抖动,整栋大楼都在抖动,整个城市全在
抖动。大地上所有的建筑物高高地颠起,又落下,再颠起。就在最高的一次起伏的顶点,一切归于平静。他抱起徐
佩蓉,把脸深深地埋进她怀里,无声地痛哭。

\section[\thesection]{}

周一下午,八处开例会。处长老孙传达完文件,又说,厅办处长老文的儿子结婚,大家都是处里老人了,还是照老
规矩吧。老韩乜斜他一眼,说,以前你可是最讨厌集体凑份子。老孙 笑起来,说俺老孙不是当上处长了嘛。小聂、
小徐的手续办完了?

聂于川笑了笑,说正在办,你看她还凑吗?

老孙想了想,说还是算了吧。想想也有意思,去年给三处的老周凑份子,她刚来八处,今年就走了。老韩说,那是
好事!小徐不去设计院,和小聂怎么结婚?也不知道谁定的这么个破规矩,真不是东西。大家都笑起来。

回到办公室,聂于川马上给徐佩蓉打电话,汇报了凑份子的事。他对徐佩蓉日渐依赖,好多事情都先向她讨主意。
徐佩蓉笑了笑,说那还给咱省了几百块呢,好事。他又给苏一文打电话,说接到了省委党校处长班的入学通知,特
意向她和老新郎表示感谢。苏一文客气一番,说经历些挫折不是坏事。他说,老姐姐为我做得太多了。不是姐夫帮
忙,怎么会提拔老孙?如果派了个年轻的处长来,我还有出头之日吗?

苏一文笑着说,其实还是钟厅长关心你,老孙再有几年也就退了,慢慢等吧。停了一下,她又说,情场得意官场失
意,看来你和小徐的好事近了。真的,我很羡慕小徐。不是谁都能像你这样。我没看错你。聂于川挂了电话,微微
一笑。不过是再等几年而已。他还年轻。

苏一文的判断不错。老任摊牌那天晚上,聂于川没回家,去了徐佩蓉家里。可能是下午的交欢过于突如其来,当晚
的缠绵就显得从容不迫。赤诚相见后,他发现原来她也是熟女,她知道的并不比苏一文少。第二天,两人一起上
班,虽不便牵手,但彼此眉宇间的牵挂却难以敷衍。下班后等班车,大家聚在一起闲聊。老韩更年期仍然未过,目
光依旧敏锐,发现了人群里的徐佩蓉,马上问道,小徐你搬到老家属院了?许多目光或善意或火辣地扎过来。徐佩蓉
臊得无地自容。聂于川笑着解围,今天威威过生日,非要他徐阿姨也去。这句话暧昧到了顶点。大家不约而同地
“哦”了一声,像是领导结束讲话后全场起立鼓掌。那天还真是威威生日。晚饭后,他奉一家老小之命送她回家。
走到老家属院门口,她忽地停住,一步也走不动了。聂于川从后边拉住她的手,笑道,你怎么了?

你是有意的。她垂下头,说我终于知道你为什么不开车,非要坐班车了。

你是说这个啊。聂于川握着她的手,两人细步走着,手再没分开。这么长时间了,都快一年了,不能总是暧昧呀。
你说,我们什么时候结婚?

她还是垂着头,眼泪扑簌簌掉下来。聂于川握紧她,说你看是等新处长来了办,还是现在就办?她扑哧笑出来,说真
好笑,这有联系吗?

晚上九点多的省城,路面还是熙熙攘攘。一辆公交车驶过,灯光晃得他们不约而同地放慢脚步。她忽然说,若不是
你受了委屈,肯说这些话吗?他打了个寒战,好久才说,我父亲跟我说过,我现在所有的一切都可能随时失去,果然
应验了。所以我想抓住一个不容易失去的。想来想去,身边只有你。

她又垂下头,说我有些害怕,如果没发生这件事,如果是你提拔,你是不是还打算跟我暧昧下去?

你还记不记得我跟你说的话?悲欢离合,阴晴圆缺,这个世界上有太多的事情我们无法左右\myrule{}

徐佩蓉气恼道,这个时候了,你还说不能左右!还想暧昧吗?

不是的。聂于川笑起来。既然无法左右,那我们就接受好了。不过,我记得你早就说过你爱我啊,现在变了吗?她笑
着不回答,只是使劲地掐了掐他的手心。

此后不久,老任找老孙谈话,宣布了组织的决定。其实老任对这个决定也不满,他有自己推荐的人选,却被钟厅长
否决,力荐老孙。老任开始想不通,后来也明白了。他生生放倒了聂于川,已是胜利;再推荐人,自然不会通过。
得陇望蜀,也仅仅是望而已。老孙听了决定,有些好笑,诚恳道,任厅长,我也是老同志了,这么开玩笑不妥吧。

老任正想将成人之美的义举归到自己身上,听见这话气得一笑,准备好的全忘了,正色说,老孙,我是拿这种事开
玩笑的人吗?你做副处调这么长时间,有能力,有资历,有水平,比谁差?早该提了;要有自信嘛。

老孙嘟囔着,八处一直是小聂主持工作。

八处是七厅的八处,是组织的八处!你是组织任命的处长,有什么好顾虑的?文件下来,我亲自去八处宣布,你就好
好准备一下,对将来的工作要有个整体的想法……

没等老任说完,老孙两只老眼已经蓄满泪水,需要泄洪了。他缓缓站起,喜不自胜地说,是真的?是真的,真是真的。
老任瞠目道,老孙,你说什么? 老孙摘掉眼镜擦泪,边擦边说,真是真的。然后连连鞠躬,说谢谢任厅长,谢谢任
厅长。老任呆呆地看着他出去,好半天冒出一句,怎么会提他!

老孙一路小哭,走廊里、电梯里遇见同事,不分男女就说,真是真的,你知道吗,真是真的。弄得大家莫名其妙,
以为他精神错乱。回到办公室,小哭已成号啕。当时只有老韩在,而她趁没其他人,正按着报上讲的乳房保健操给
自己做保健。老孙蓦地闯进来,蹲在地上,泪雨缤纷。老韩羞愤怒目道,不会敲门吗?老孙不理她,号啕继续,仿佛
清白之躯刚刚惨遭蹂躏。此事经老韩之口传遍全厅,成为美谈。奇怪的是钟厅长对此微微笑过,不置一词。此后又
过不久,一个聚会上,钟厅长意外遇见了徐佩蓉的前婆婆。前婆婆对前儿媳赞不绝口,说离婚是自己儿子不争气;
虽然离了,但小徐跟自己亲闺女一般,还要钟厅长帮忙找个好归宿。钟厅长后悔不迭,对老任谎报军情愤懑不已,
但也晚了。文件已下,正处已提,老孙虽无才能,但也无过错,哭都哭过了,人也丢过了,不好再弄下来。好在不
得不下赌注时已有所铺垫,老孙过几年就退了,不至于将聂于川的前途彻底赌进去。厅里又提拔徐佩蓉为副处,到
设计院当副院长。聂于川酸酸地开玩笑,说我的正处长没了,你倒是升了,多好笑的事。徐佩蓉不答理他,她有的
是事情去忙。一边调动工作,一边还要看房子、搞装修。验收那天,聂于川有事来不了,她一个人去了。许多人都
认为是聂于川遭遇沉重打击,这才万念俱灰,赌气结婚。她虽不这样想,但其实何尝不是如此?正处长的意外落空成
全了她,腰斩了暧昧。如果没有此事,两人都不知道还要暧昧多长时间,将会耗去多少岁月。他和她都在为自己打
算,只不过她一心要嫁给他,他一心要暧昧下去。可能老天就为成全她,让他唾手可得的正处长毁于一旦。徐佩蓉
不经意间成了暧昧高手。她的一步步运筹帷幄,一点点精心算计,费尽如许周折,却全不如一次可笑的官场变局。
她并未意识到这是上天的眷顾。她只觉得一切都顺理成章。于是,她指了指墙上的挂件,断然说这个不对,这里将
来要挂结婚照的。

工头不满地哼了一声,招呼着工人上来。徐佩蓉不去管他们,兀自看着墙上并不存在的结婚照,幸福而暧昧地笑了
起来。

\end{document}

