\documentclass[11pt]{article}
\title{山楂树之恋}
\author{艾米}

\usepackage{config}

\linespread{1.3}

\begin{document}
% \thispagestyle{empty}
\pagenumbering{Roman}
\cfoot{\textsf{\thepage}}
\tableofcontents

\clearpage
\setcounter{page}{1}
\pagenumbering{arabic}
\pagestyle{fancy}

\section{序:一本独特的书(林达)}

这是一本很独特的书。这是刚开始工作不久的一个中国留学生参与美国政治活动的实录,作者作为一个外国
人,因为他生活在这个社区中,在美国人眼中,他自然获得了参与政治活动的权利。早在晚清,中国的学者已经注
意到美国的政治制度。美国制度文本,在将近一百年前就被译成了中文。对这些早年的中国学者来说,研究动力既
来自他们改革国家的需要,他们要探究美国强大的原因;研究动力也来自于美国政治制度的内在逻辑和理性,对学
者们智力探索的吸引。最近二十年,适逢中国再度面临改革,中美交流的规模也今非昔比,因此,有了很多谈论美
国政治制度的书籍。但是,却很少有大陆来美的第一代移民,深入美国社会``草根'',去亲自``运作''美国政治。
谈的人多,运作的人少。这本书,是我们看到的第一本。

我们看作者和美国人在一起``运作''的草根政治,免不了会比照我们中国人的``搞''政治。这本书给我们描绘的美
国搞政治的情景,和我们的习惯很不一样,有些可以说是匪夷所思。作者因自己的政见,在美国加入了一个小党---自
由党。他们自己贴钱贴时间,耗神耗精力去开会、讨论、游行、集会、插标语、发传单,挨家挨户去地说服动员。
而他们的这个党,小得甚至都上不了州里的选票,即使上了,得票率也微乎其微,八辈子也难有当选``执政''的可
能。那么,他们到底是想要什么呢?这正是我们往往难以看懂的地方。

这要从中美社会对政党的不同观念说起。

政党作为一种光明正大的结社结伙形式,是西方议会政治的产物,是议会政治中相同观点的表达形式。同一观
点者就是同一政党,目的是把自己的观点更有力地表达出来。等到这种东西传入中国,要翻译成汉语的时候,却找
不到一个对应的东西,只能取形式相近者,名之为``党'',即``会党''之党。古语说,党,犹亲也。结党就是分个
亲疏。我们中国古代说到党,就是朋党、乡党、会党。给人留下的印象,是一群声气相投、利害相顾的人。民众之
结党,相当于刘、关、张桃园结义,结拜把子兄弟。所以《论语》说君子群而不党,然而却总有党,而且结党必营
私,故有``党祸''一说。

等到20世纪初,中国人变革图强,走向共和,也需要政党的时候,不幸的是先有党后有议会,这先于议会产生
的是革命党。革命党有明确的功利目的性,``政权问题是革命的首要问题'',革命围绕着权力而结党。这样的党,
必然脱胎于古代的``会党'',讲究的是对组织的忠诚。进来前有高门槛,要考验,进来后要生死相许,讲究忠诚。
对党的理念认同上升到信仰的高度,改变看法就是叛逆行为,开除更是莫大的羞辱。党员和非党员有着本质区别。
从政是往上走的。

这本书描述的西方政党的基层活动,和我们习惯上的革命政党相差很大。

西方一般的政党,是围绕着怎样表达自己的政治观点而组织活动的政党,特别是本书作者参加的自由党这样的
小党,他们从一开始就知道,这个政党和执政权力没有关系,永远也不会因此而上台当个执政的``官''。可是他们
觉得这个社会,这个国家,或者他们生活的社区有了一些他们担忧的问题,他们想让大家知道,想用更大的嗓门来
说出自己的担忧。公民的责任让这些素不相识的人走到一起。他们是积极的,他们付出,却从一开始就不指望有
``权力''的回报。这种党,没有什么门槛,没有什么纪律,没有什么党的建设,谈不上开除出党。然而却正是这样
一些热衷政治活动的人,组成了美国社会城乡社区基层政治的基本形式,他们就是美国的草根政治,在那里,``从
政''是在草根层进行的,是要争取民众的认同,是必然往下走的。

这种政党的存在,这种以``反对''姿态的小党的存在,是一个制度健康的标志。而有一批像本书作者这样,热
衷于草根层政治活动,熟悉并恪守游戏规则的人,就是一个社会趋于政治健康的保障。其实,任何社会里,反对派
都是有的。如果一个制度不给反对党一丝生存空间,则反对意见必转入地下,铁马金戈的寒嗖嗖风声就出现了。美
国政治制度干脆彻底开放反对党的空间,两百年来反对党就成为社会开明变革的动力,反对之处,一派详和。

这本书其实是分为两部分的。一部分是以地方自治、社区建设为目标的草根政治,而另一部分,也许有相当于
一半甚至更多的内容,是记录作者投入美国大选的助选阵营。虽然,表面看来,作者仍然是在参与基层的政治活
动,可是这两部分内容是完全不同的。作者自己也意识到,

``我比较遗憾的是,本来我计划花一半精力帮助恰克或者吉姆来竞选,因为正如美国人所常说:``所有政治都
是地方政治'',我觉得众议员选举其实比总统选举更有意义,也更能体现民主政治的本质。但后来克里阵营需要投
入的时间太多,自由党人也没有大规模竞选的计划,我因此失去了一次更近距离体验草根政治的机会。''

在这样的助选阵营中,不论是替哪一边助选,看上去你站到了一个更高的位置,实际上,你的视野却可能变得
狭窄了。因为在这样的地方,天然地聚集着持强烈政治倾向的人,相互感染。作者是一个有独立思考能力的人,在
事后却也看到,自己``也曾多次陷于这些非理性情绪而不自知''。因此,作者有时不由自主地把他在这个圈子里看
到的景象,替代了整体的图景,也在影响他的评论。

比如说,作者感受周围助选圈子的言论,看着两党大会都开成造神大会,因此认为``我对美国政治不敢恭维的
第二点,是他们对候选人个人品格的包装,有时甚至超出了对政策的讨论。''其实这次大选之后,几乎没有评论家
认为,这是大众对候选人个人好恶的结果。绝大部分美国民众,在地方选举中,有选能人的意味,会更多追随个人
的政绩政见,而不是单纯追随政党归属。而在大选中,大多数投票方向相对稳定的民众,民众的选票基本上是跟随
政党的。也就是说,某政党所代表的政策、理念、价值观,是他们更在乎的东西。他们对对方候选人的不满,往往
是建立在这个基础上的。因此,他们对自己赞同的政党的候选人是谁,就并不那么在乎。

至于两党大会的煽情,不同制度的区别在于,非民主的地方是一头煽,而且从小孩子的教育做起,煽你没商量。
在这样的情况下,民众其实是经不起煽的,尤其在只有一个救星、一个希望的时候,而且是从小就被煽起的话。而
在美国,首先,学校的教育是中立的。学校不容许办成小党校。其次,媒体要是打算办得有人缘,也必须中立。所
以,一个电视台播放了民主党的大会之后,必然也要播放共和党的大会。也就是说,煽情的是党派大会,而不是媒
体。作为媒体,把两党的煽情一视同仁地播出之后,媒体本身起的作用就不是煽情,而是灭火了。普通民众一看,
一个政党候选人代表着``美国希望''在冒出来,劲头也在被煽起来,可是,马上就看到了另一个政党如法泡制,也
在推出另一个``美国救星''。民众再愚蠢,也会如同被浇上一瓢冷水,明白了这只是一种宣传把戏。真正的选择,
还是必须在撇开这些热闹和喧哗之后,认真考察两边的具体政策取向,看看哪个真正符合自己的利益,然后再决定
投票方向。

所以,民主不是不煽情。民主制度下要竞选,也煽得很让人起鸡皮疙瘩。可是,那是抵消式的煽情。独一煽
情,会越煽越大,民众更可能被盲目调动,而对立煽情,会引出比较和思考。而竞选中的政党煽情的重要目标之
一,是需要始终维持一群铁杆助手,竞选小圈子就在其内。在那里风云跌荡,兴奋莫常。可是,在今天的美国,虽
然接连两次出现票数极为接近的情况,甚至2000年大选结果产生争议,可是,绝大多数民众没有走上街头。他们正
常地工作、生活,很个人化地考虑了他们的投票,然后去投票站投票、等着结果依法出来、并且接受这个结果。而
没有作者感受到的那么多极端对峙、亢奋情绪和热闹。这是因为美国的民主已经成熟。他们眼中的美国政治和竞
选,会有很多看法和作者是不同的。如果仅仅呈现最五光十色的一面,它的成熟度就没有被表现出来。

因此,作者深入助选中心圈内,身在此山中,而且是在某一党派的山中,必定``有得也有失''。作者对此有非
常清醒的认识,在一开始就强调,``只不过希望为读者的'兼听则明'多提供一个可听的渠道。''有一点遗憾的是,
国内的读者,取得另一面资讯的渠道,可能要少得多。

这本书是其他书所不能替代的,它是第一手的资料,有着非常详尽的记录和生动描述。它让我们看到,以反对
派的形式推动美国的革新,是美国的一个政治常态,让我们看到它的宽容度在哪里。这里既有极右的新纳粹党,也
有极左的黑人党;公平竞争,和平相处。而更多的是并不极端的政党,也就是不同意见的反对派,他们的实质是建
设性的。作者深谙此理,他在最后提到,``我自己的计划是,希望将来可以多参与些当地的政治。政治其实不仅是
投票和竞选,更重要的是和社区的互动。''大家积极参与的、建设性而不是争斗性的政党活动和政治活动,才是美
国草根政治的活力所在,才是美国的活力所在。美国人把这种左右远近像大树小树一样群党林立的景象,叫做``政
治风景''。这片风景已经有了两百多年历史,读了本书,我们更相信,这样一片风景,是光明的。

\begin{flushright}
2003
\end{flushright}

\clearpage
\section{我和美国自由党}

\subsection{关于我}

我是一个普通的中国留学生,来到美国已经五年,目前在宾夕法尼亚州一家电脑公司工作,住在宾州蒙哥马利
郡普王市(King of Prussia)。这是个小城,从费城往西北方向开半小时的车就可以到达,再往前走,就是美国独立
战争时的革命圣地``福吉谷''(Valley Forge)。1777年冬,费城陷落后,华盛顿将军率军在此整训,度过了独立战
争里最艰难的时间。如今,这里已辟为美国国家历史公园,绿草如茵,风光秀美,每天都有很多人在这里跑步、骑
自行车,我和朋友们也常在周末去那里烧烤、爬山。

普王市地处城市和乡村的交界处,从这里往南走,越来越繁华,最后就到达了费城;往北走,越来越幽静,山
路渐多,民风也渐淳朴。普王市又有一个号称美国东部最大的购物中心,附近最主要的高速公路都在这里会合,是
个重要的交通枢纽。因此,这里既有城市的热闹、商业的繁荣,又有乡村的宁静、环境的优美。居民以白人为主,
政治上是共和党占主导地位。

我并不是美国公民,没有选举权和被选举权,但我对布什政府的内外政策都相当不满,就决定去为反对布什的
连任做些义工。由于美国是所谓的两党政治,朝野力量基本都在共和党和民主党这两大党手中,有能力打碎共和党
布什的总统连任计划的,只有民主党。作为一个中国人,我必须承认,民主党的左派色彩对我有天然的吸引力,但
我对民主党的很多主张也相当不以为然,尤其对``大政府''主义更是深恶痛绝,所以我对于帮助民主党也没有什么
积极性。

然后,我自然想到网络法宝Google,搜索``总统选举'',结果找到一个网站,列出了2000年总统大选所有参选
党派的得票率。除了共和党、民主党这两大党,绿党、社会主义党等我早听说过的党派外,我还找到了一个自由党
(Libertarian Party)。

顺着那个网站的链接,我来到了自由党的主页(http://www.lp.org),原来它是美国第三大党。主页上有一个有
趣的小测试,是用来测试人们的政治倾向的(附于文后)。我其实早就知道自己的政治面貌,是个极端自由主义者兼
彻底怀疑论者,在做完那个测试后,果然,我落在了``自由党人''这个范围内----终于有一个可以基本代表我的主
张的政党了。

从美国自由党的主页,我连到了宾夕法尼亚自由党的主页,在那里下载了加入该党的表格。还好,表格异常简
单,只要姓名和联系方法。其他如年龄、性别、种族等等一律不用,更不容说社会安全号码(相当于中国的身份证号
码)。

党费有好几档。如果只是想做个合作党员(Associate member),是免费的,只要把Email地址填上就行了;如果
想成为正式的州自由党党员,一年需交15美元。我选择了成为正式党员。

然后就是美国这个商业国家必不可少的折扣特价了:本来,要加入全国自由党要25美元,但本州自由党的正式
党员只要再交10美元,就可以成为全国的自由党党员,定期收到美国自由党的期刊和各种信息,相当于打了60\%的
折。我也没能抵挡住折扣的诱惑,选择了正式加入全国自由党。这样,我还是交了25美元,却同时成为了全国和宾
州的自由党正式党员,也算是买一送一吧。

\subsection{自由党}

填表的日子是7月14日,因为是攻占巴士底狱的纪念日,所以我到现在还记得。但直到9月回音才来,不知道是
因为天下的官僚主义都是一样的,还是因为宾州自由党的经费和人手缺得真有这么厉害。在收到正式的回音之前,
我从蒙哥马利郡自由党的网页上看到,他们将在8月7日晚开会。我不想再等下去,就直接去参加了那次会议。

开会地点是在《费城问讯报》(Philadelphia Inquirer)的报社大楼。《费城问讯报》是费城地区最有影响的一
家报纸,在蒙郡的分报社离我家非常近,只有不到10分钟的车程。

我提前5分钟到达了会议室,只有一个40岁左右的人在那里,看上去很精明能干,风度翩翩,说话幽默风趣。我
们互相介绍了自己。他叫克恩$\cdot$克若恰科(Ken Krawchuk),后来我才知道,他是全宾州的自由党主席,因为
家住在蒙哥马利郡的阿宾屯(Abington),所以总来参加蒙郡自由党的会议。

克恩得知我是个不远万里,来到美国,把美国人民的自由主义事业当做自己的事业的国际主义战士后,对我大
加欢迎。他向我介绍说,蒙郡自由党有3000个党员,准确的说,2995个注册党员。我笑着纠正说:``现在有2996个
了。''

不过,对这个数字我还是比较吃惊的,我本来以为全蒙郡有300个党员就算很多了。当然,3000个党员也只能让
一个第三党感到自豪,对于两大党来说,这点人只是毛毛细雨:蒙郡注册在案的民主党党员有18万5千人,共和党有
25万6千人。

克恩所住的阿宾屯是个自由党的重镇,有三百多个党员,并且今年可能会有一个叫葛锐格(Greg)的自由党人出
来竞选该镇的镇委员(township commissioner)。他已经登记为候选人,可是由于自由党的经费和人手有限,不能给
他提供强大的支持,而他本人也比较忙,所以他正在犹豫着要撤选。明天就是撤选的最后期限,克恩说:

``我们决定,只要他今天来了,我们就把他五花大绑起来塞进衣橱,到后天再放出来。''

大概是葛锐格识破了克恩的诡计,这天居然没有来。后来他果然撤选了,因此今年的选举里,整个蒙郡都没有
自由党的候选人。这样我就不能帮他们助选了,让我多少有些遗憾。

其实克恩自己的风度、口才都非常好,如果去竞选的话,应该可以赢得不少支持,但他去年已经作为自由党的
候选人去竞选宾夕法尼亚州州长,辞了工作全天候奔忙,花去太多精力和金钱。自由党是小党,全州主席也没有工
资,因此今年他找了份咨询公司的职业,先稳定一下后方。他告诉我,很多人听了他的演讲后,对他说:``你应该
去竞选总统!''我说:``是的,我也觉得你应该去。''克恩说,他正在考虑参加2008年的美国总统选举,我觉得他有
这个实力。

陆续来参加会议的其他人,有蒙郡自由党的主席吉姆$\cdot$巴伯(Jim Babb)、副主席拉瑞$\cdot$古拉特
(Larry Goulart)、政治干事查尔斯$\cdot$弗涅尔(Charles Fournier, Political Action Coordinator)、恰克
$\cdot$莫尔顿(Chuck Moulton)、特殊协调员杰夫(Jeff, Outreach Coordinator),加上我总共不过七个人,我
算是刚加入自由党就进入党支部核心了。

吉姆是个三十多岁的大高个,留着齐腰的长发,在脑后绑成一捆长长的马尾巴,穿着T恤、短裤、凉鞋,确实是
我们这一桌人里看上去最自由散漫的,怪不得是他当上了本郡自由党的主席。他主持了这次会议,大家主要讨论日
后的活动安排。

蒙郡自由党和《费城问讯报》有协议,每月的第一个星期四、第二个星期一、第三个星期二、第四个星期三可
以免费使用这个会议室到晚上10点,因此他们固定在每个月的第一个星期四开月务会议,第三个星期二开一个自由
论坛(Liberty Forum),邀请人来演讲,其余两次用来给自由党演讲俱乐部(Libertarian Toastmasters)活动。今天
便首先讨论两周后的自由论坛的安排,谁当接待人,谁带饮料,谁带点心,由于经费有限,这些都是没法报销的,
所以大家并不踊跃。

第二项议程是刚被任命的新党员干事(New Member Coordinator)恰克介绍他吸引新党员的办法:一是直接寄信
给感兴趣的人们,二是加强在学校的活动。大家都同意他的建议,但这两项活动都是要花钱的,对于经费怎么使
用,大家的争议很大,讨论了很长时间。

很快就到了10点,我们撤到了附近的一家酒吧里喝酒聊天。在那里,大家就谈些更轻松的话题。我问清楚了每
个人的职业,原来除了恰克还在上大学外,都是普通的中产阶级,而整个自由党里,有一半的人是工程师。

弄清楚这些人里面并没有资本家后,我就放心了。中学时模模糊糊地听说要反对资产阶级自由化,那时候我觉
悟不高,政治没学好,害得我至今也不明白什么样的自由化就是资产阶级的,什么样的自由化就是其他阶级的。好
在现在所谓的中产阶级,只是叫得好听,其实还就是工人,那么我加入了以中产阶级为主体的自由党,不是资产阶
级自由化,而是工人阶级自由化,算是进步青年,无反动堕落之虞。

\subsection{自由之州}

这个月的自由论坛主讲者是来自新泽西的开尔文$\cdot$普拉特(Calvin Pratt)。他是来为``自由之州''行动
(Free State Project)做广告的,演讲题目是``自由之州行动:自由党人的乌托邦,还是第三党的无奈之举?''。

在美国这个两党政治的国家里,任何第三党几乎都不可能对政治产生重大的影响,包括自由党这个第三大党在
内。比如2000年的美国总统选举,布什得了50,456,002张选票(47.87\%),戈尔得了50,999,897张(48.38\%),而自
由党的候选人布郎尼(Browne)只得了384,431张(0.36\%),根本没有胜选的希望。

因此,自由党陷入了一个困境:选民觉得他们的主张太激进,不可能付诸实践,因此从不投票给他们;可如果
自由党从来不能赢得选举,那他们又怎么可能有机会实践主张?由此形成了一个恶性循环,给自由党的问政之路打了
个死结。

两年前,两个耶鲁大学政治科学系的学生发起了``自由之州''行动,他们想发动两万名自由党人搬到一个州
去,从而显著地改变该州的政治面貌,寻找机会将自由党的政治理想付诸现实。``自由之州''行动于2001年9月1日
正式开始实施,为期五年。阶段性的目标是:在报名人数满5000人后,就开始投票决定到底要搬到哪个州去。如果
到了2006年9月1日还不能征集满2万人,则宣告``自由之州''行动失败。

这个行动的关键在于,美国的州是相当独立的,虽然区区一州不能影响到全国的政策,但在州内有很大的自主
权。教育、州税等等当然是州的管辖范围,枪支管理、婚姻政策(比如同性恋婚姻)等等也都是各州自己决定的。这
两位学生大概也读过《毛泽东选集》,知道集中优势兵力的道理,因为革命本钱太小,和两大党打阵地战肯定是没
有出路的,只能打游击战。

2万人说多不多,可是,对于自由党这样的小党来说,要在全国范围内找到2万个``我是一块砖,任凭党来搬''
的积极分子,愿意放弃目前的定居处,搬到另外一个还不知道是哪里的地方,谈何容易!2万人说少又很少,对于民
主党和共和党这两大党来说,他们在一个州里号召出2万张选票轻而易举。

还有一个问题是,革命火种来之不易,必须谨慎使用。他们研究了全国的形势,决定只能选择那些人口少于
150万的州,或者那些每两年的选举里花费不超过2000万美元的州。另外也要考虑当地的政治环境,因为这些自由党
人不是过去打仗的,而是要去融入当地的社区,成为当地的一员,建设一个自由主义新天堂,所以只能到群众基础
良好的州去建立根据地。最后决定将在以下十个州里选择:阿拉斯加州、爱德华州、南达科他州、北达科他州、佛
蒙特州、怀俄明州、缅因州、内华达州、特拉华州、新罕布什尔州。

这个行动的主意不错,但是否真的可行?还好,行动的第一步目标看来已经达到:到这个月初,报名参加的就已
经超过了5000人,可以开始就这十个州来投票选择目的地了。可目的地选定之后,能在剩下的三年内再召集到1.5万
人来参加吗?即使能够动员出两万人到那个州去,自由党就能够赢得选举吗?开尔文也知道人们的重重疑虑,不无自
嘲地把自己的演讲题目定为``自由之州行动:自由党人的乌托邦,还是第三党的无奈之举?''。

开尔文介绍完``自由之州''行动后,大家开始提问。我也问了个问题:``自由之州行动要在5年内征集到2万
人,那每年就得平均征集到4000人,可现在快两年过去了,才征集到5000人,你们是不是预计在大家投票选出一个
州后,会有更多的人参与呢?''

开尔文的回答是肯定的。他认为,在开始阶段,由于目的地没有选定,可能只有非常积极的自由党人才会报
名,党指哪就打到哪,而大多数人都在观望。比如说宾州的人,去特拉华州没什么问题,但要去阿拉斯加就要考虑
一下了。而在大家投票选出一个州后,应该会有更多的人加入进来。

也有人随即指出,也可能存在相反的情况,有人本来打算要参与,结果投票结果出来后,他们觉得搬到那个州
太难,反而放弃了。开尔文说:``这是必须要付出的代价。''他表示,不管最后的投票结果在哪里,他都会尽快搬
到那个州去。然后他会等到三年后,如果行动成功了,有2万人搬过来,他们就开始(对于像他那样的先行者可能算
``继续'')参与到当地的政治里去。他们认为,占有人口4\%以上的坚定主张者就可以赢得选举,2万积极参与政治的
人,在一个人口不到150万人的州里占上风是相当容易的。如果行动最终失败了,那么到时候他们再决定是否继续在
那里住下去。

投票结果出来,是在10月,这是自由论坛之后两个月的事了。最后,大家选定了新罕布什尔州----美国东北角
上的一个小州,与加拿大接壤。宾州的自由党人当然都有些失望。如果是附近的特拉华州,估计我们蒙郡就会有很
多人搬过去。克恩还说,如果是特拉华州,他可以在那里竞选州长,因为他在特拉华州也有很多支持者。那就不再
是第三党的空想,而是确实有可能选上的。可惜,新罕布什尔州离这里实在太远了。祝他们好运吧。

\subsection{升官发财}

随后几个月里,蒙郡自由党的活动也都类似,主要是准备自由讲坛、讨论日后计划。随着选举日逐渐临近,我
们的主要精力放到反对``开阔地带''计划上去了(详见后面《``开阔地带''计划》)。

在十月份的月务会议上,吉姆告诉大家,蒙郡自由党的会计要搬离蒙郡,所以辞职了,现在我们需要选举出一
个新的会计来。其实这个会计就基本上没干过活,吉姆说他已经一年没有看见她了。在我去的第一次会议里,杰夫
就建议我来做会计,被我拒绝了,因为我知道自己是个非常散漫之人,生平两大绝技就是账单忘了付被罚款,以及
把重要文件丢东拉西。让我做会计,恐怕账会做得比安然公司还神鬼莫测。

但这次我看来是在劫难逃了,我是出席会议者中唯一一个没有职务的,这个壮丁是被拉定了。吉姆见我还在犹
豫,就说:``不用担心,你知道现在的会计都做些什么吗?''

我说:``做什么呢?''

吉姆大笑着说:``她什么也没有做!你会不会`什么也不做'呢?''

这我总还是会的,于是我就接受了提名,然后全票通过,成为了蒙郡自由党的会计。随着会计这一职务而来
的,还有筹款事项。但考虑到我完全是个新手,而且对美国多如牛毛的筹款法律一窍不通,所以这件事暂时不用我
做,而仍然由大家商量。

蒙郡自由党的经费是非常紧张的,现在总共只有一千多块钱。就在那天晚上,为了一项89美元的支出是否可以
报销,还差点吵起来了。那是上个月,一位明年总统选举的自由党候选人加里$\cdot$诺兰(Gary Nolan)来参加自
由论坛,吉姆自己掏腰包为他定了旅馆。大家都同意这笔钱可以报销,但都建议以后不要订那么贵的旅馆了。吉姆
表示,这已经是非常便宜的了,不管怎么说,那是我们请来的客人,而且是个总统候选人啊。最后,大家决定各自
去寻找价廉物美的旅馆,以备日后之用。

我虽然当上了会计,但没想到我们的经费拮据至此,所以虽然升官,发财是肯定没有的,动脑筋怎么让蒙郡自
由党发财才是正事。

\subsection{世界上最小的政治测试}

本测试共有两组问题,每组列出五个自由党人的信条,如果你同意,就给自己加一分,如果反对,就给自己减
一分(分数可以是负数),如果拿不准,或者不赞成也不反对,则不加分也不减分。把最后得分代入后面的图中,即
可得到自己的政治倾向。

第一组、个人事务:

1、军役应当出于自愿。(而不是征兵制)

2、政府不应当控制广播、电视、媒体或者互联网。

3、关于成年人之间自愿的性行为的法规应当被废止。

4、关于毒品的法律害大于益,应当被废止。

5、人们应当可以自由地穿越国界,选择他们生活和工作的地方。

第二组、经济事务

1、公司和农场的运行中不应当有政府补助。

2、自由贸易比关税更好。

3、关于最低工资的法律造成了失业,应当被废止。

4、用付服务费来代替付税。

5、所有的外国资助只能来自私人(我不清楚这里的外国资助是指美国资助外国,还是外国资助美国)。

将自己的得分代入下面的图,其中横轴代表个人事务,竖轴代表经济事务:

比如我,在所有的问题中,除了第二组第三个问题拿不准外,都答了同意。于是我的得分是:个人事务5分,经
济事务4分,落在一区内。测试的设计者是这样定义各区的:

一区:自由党人。自由党人愿意在个人事务和经济事务上都由自己决定。他们相信政府的唯一目的是保护人民
免于强迫和暴力。他们看重个人责任,容忍经济和社会的多元化。

二区:左翼自由派。左翼自由派倾向在于个人事务由自己决定,而经济事务则由一些中央机构来决定。他们要
求政府帮助弱势群体以达到公平。左派容忍社会多元化,追求经济平等。

三区:极权主义者。极权主义者要求政府通过精确的中央计划来推动社会和个人进步。他们经常怀疑个人为自
己作主的可行性。左翼极权主义者又叫社会主义者,法西斯主义者则属于右翼极权主义者。

四区:保守派。右翼保守派倾向于在经济事务上由自己决定,但在个人事务上有一定的普遍标准。他们要求政
府捍卫人们的道德素质不受威胁。

五区:中间派。中间派认为政府应当有选择性地介入个人和经济事务,着重解决方案的现实性。他们对各种新
生事物都比较能接受。许多中间派认为政府应当注意不要让自由过度。

\section{``开阔地带''计划}

``开阔地带(Open Space)''是目前美国很多地方正在开展的计划,主要用来防止人类社会开发环境过度,保护
我们周围已经所剩不多的开阔地带,比如草地、树木、水流和其他自然资源,资金一般由政府提供。宾夕法尼亚州
和邻近的新泽西州都已经开展这个计划多年,由于当地居民环保意识良好,得到了广泛的支持。

在我所住的宾州蒙哥马利郡,政府早在1993年就拨出1亿美元,开展了一个为期10年的计划,来标识和保护郡内
正在消失的开阔地带。这个计划到了2003年就结束了,政府又提出了一个更为雄心勃勃的``绿地--绿镇(Green
Fields - Green Towns)''计划,准备在下一个10年内投入1亿5000万美元,更有效地保护开阔地带。由于郡政府拿
不出这笔钱,他们将向银行贷款,同时计划在今后33年内,增收5\%的房地产税来还债。增税在美国是个非常敏感的
行为,郡政府为了避免日后被指责,不敢擅自决策,便把这个计划交由选民投票,让选民自己来决定是否愿意多交
一点钱换取更多的绿地。

在民风保守的蒙郡,大多数居民具有浓厚的环保意识,愿意支持这个计划。共和党和民主党这两大党也一反平
时相互攻击的习惯,联手号召党员投票支持。唯一的反对声音则来自自由党。

难道自由党人都没有环境保护的观念吗?或者是他们都穷得揭不开锅了,宁愿竭泽而渔,也要先富起来再说?下
面是一篇我投给当地报纸的文章,您看完后也许就能明白自由党反对这个计划的原因所在了:

蒙哥马利郡的选民们将在11月4日决定是否批准``绿地--绿镇''计划。这个计划将贷款1亿5000万美元,来保护
开阔地带、水流和自然资源。是啊,绿地、净水、自然资源,难道这些不值得我们多交点税去保护吗?然而,这里还
有些人们所不知道的事实。

首先,蒙郡的纳税人的最终付出将远比计划所称的1亿5000万美元多,利息一项据估计就将有1亿美元之巨。也
就是说,这个计划事实上将耗资2亿5000万美元,其中40\%会进入那些金融资本家的腰包。当然,羊毛出在羊身上,
这些钱都是纳税人来出的。

其次,这笔钱是用来付给那些土地拥有者,让他们不再在自己的土地上构建任何建筑。但是,土地仍然属于他
们。这笔钱只是用来买下他们在这些土地上兴建土木的权力。---谁说这世上没有免费的午餐?只不过这午餐只提供
给那些富有的开阔地带拥有者而已。``绿地--绿镇''的支持者们总在告诉人们,开阔地带属于所有的人。他们说得
没错,只是还有一点区别没讲:我们唯一拥有的是付税来保护开阔地带的权力,而它们的主人才有收钱来保护它们
的权力。

再者,蒙郡早已有一个已经运行了10年的``开阔地带''计划了,它平均1年只耗资680万美元。耐人寻味的是,
这个计划得到了1亿美元的拨款,可是只花掉了6800万美元---为什么我们还要再把计划的预算提到1亿5000万美元
呢?据8月8日的《蒙郡新闻》文章《各镇要求支持``开阔地带''》(作者比尔$\cdot$斯班勒)报导,``开阔地带''
计划主管桑德拉$\cdot$德丝匹奥说:``如果新计划不被批准,原有的计划将继续,不过预算小得多而已。''那
么,为什么我们不继续原有的已经被证明是行之有效的计划呢?这将减轻纳税人的负担。

许多人都已经认识到这些事实。据《先驱新闻》记者丹$\cdot$凯利在10月29日的报导《东诺瑞屯镇行政官员
将不支持``开阔地带''计划》:``(东诺瑞屯镇)行政委员会一致拒绝批准为期10年的耗资1亿5000万美元的``开阔地
带''计划。''委员会主席刘易斯$\cdot$马克奎斯指出:``对于我们镇来说,交通阻塞是更大的问题,钱应当被用
到修建马路上。''这篇报导还引述说,镇政府寄给居民一份民意调查,问他们是否愿意付税来保护开阔地带,回应
是``压倒性地反对''。

那么,还有谁会投票支持这个计划呢?它将让纳税人在今后的33年内背上额外的5\%房地产税,而其中的40\%会
被金融家们拿走,律师也要染指一大份,最后剩下的不过是给富人的社会救济。

在10月的自由党月务会议上,政治干事查尔斯要求每个人都写一篇文章给报社,去宣传反对``开阔地带''计划。
我也答应了,不过写英文文章对我来说不是件容易事,结果拖到10月底才写了出来。葛锐格帮我润色了英文后,我
在10月31日把它用Email寄给了报纸的编辑。可能是距投票日(11月4日)太近了,他们不再有版面留给``开阔地带''
计划,没有登出来。

当然,自由党的行动不会仅仅停留在纸上。除了给报纸写信外,查尔斯还组织了几次散发传单的活动,我也参
加了一次。

那是在11月1日,即投票日前的那个周末。在发传单前,查尔斯带我们来到当地的一家公立图书馆,查最近的报
纸。因为法律规定,这种投票问题,必须在投票日前15~20天内在当地报纸登出,如果我们在报纸的``法律通知''
部分没有找到他们的通告的话,那我们就去告他们违法,兵不血刃、不战而胜了。孙子曰:``不战而屈人之兵,善
之善者也。''看来查尔斯也精通兵法。

可惜,我们很快就找到了关于``开阔地带''计划的投票问题的通知说明。葛锐格说:``看来某些人还没有忘记
自己的工作。''但查尔斯一番仔细推敲后,认为还是有可以吹毛求疵的地方,于是就把它复印下来,准备回去继续
研究找茬。

查尔斯是法国人,生得颇有法国小生的俊朗。他3年前来到美国,立刻就加入了自由党,但他也还没有成为美国
公民,和我一样,是一名``不远万里,来到美国,把美国人民的自由主义事业当做自己的事业''的国际主义战士。
不过,他一来美国就加入自由党,比之我来了5年之后才摸索寻找到组织,觉悟不消说是高得多了。他的英语口音很
重,一开始我根本听不懂,后来慢慢地也就习惯了。他作为党的政治干事,精力充沛,头脑又灵活,点子很多,最
难得的是韧性十足,有股从不服输的精神,可谓是蒙郡自由党的发动机,不愧是来自于伏尔泰、卢梭的国度。

从图书馆出来后,我们分为两拨,我和恰克、葛锐格以及一个叫乔的人一起,去附近的沃尔玛超市散发传单,
查尔斯则和另一人去另外一家店。

沃尔玛的生意总是非常好的,尤其在周六的下午,进出口处更是人流不断。我们满怀欣喜地来到门口,正要开
始散发传单,却发现沃尔玛的自动门上贴了一张告示,不准做广告。我们商量了一下,恰克比较犹豫,建议我们再
换个地方,葛锐格却说:``没关系,我们又不是发广告,我们是政治活动。''我说:``我们先开始发吧,实在不
行,他们要赶了,我们再走,不赶我们就继续发。''

大家都同意了。后来我才想起,这个讨论其实很有意思,在遇到问题的时候,美国人的第一反应是抠字眼,而
我这个中国人首先想的则是应对方法;所以美国人律师多,绕着法律条文做文章;中国人则是习惯于上有政策,下
有对策,我承认你的权威,但趁你不注意先钻钻空子再说。上纲上线地分析完后,我自己也禁不住莞尔一笑。

恰克又提议说,传单应该只给出来的人,不给进去的人,一来防止传单被沃尔玛的工作人员注意到,二来进去
购物的人,谁会认真地看传单呢?肯定是把传单随手往购物车里一扔就算了。大家都认为有理,采纳了他的意见。沃
尔玛超市有两个门,我和恰克一边,葛锐格和乔一边,各把守住一个门,开始发传单。

我站在靠里的地方,基本上是出来一个人,就来声``嗨!'',然后把传单递过去。大部分人都会接了,并说声谢
谢。不要的就不伸手接,或者说``不,谢谢'',或者一个``不''字就打发了我们。我看到恰克在外面无事可做,就
建议他和我各站一边,因为有些人看见我就绕开走了,如果恰克站在另一边的话,那他们就疏而不漏、无处可逃了。
恰克却说:``如果他们绕开走,就说明他们不感兴趣,我们硬塞也没有用。''

恰克是附近维勒诺瓦(Villanova)大学的法律系学生。他从小就对政治感兴趣,在没有获得投票权之前,就积极
游说其他人的投票选举。他18岁以后,每一次都把票投给自由党,是个坚定的自由党人。自由党组织的每次活动,
他都一次不拉地参加,是蒙郡自由党的重要力量。法律规定25岁以后才能参选议员,恰克明年正好满25岁,他准备
立即竞选众议员,真是锐气可嘉,不放过任何机会。但是他的形象仍然是学生样的不修边幅,而且我觉得他有时候
还可以更主动大胆些。

在无人进出的时候,我和恰克就顺便聊了一下他明年的竞选计划。到了下午4点半,我们与其他人重新碰头。查
尔斯他们俩没有去那家店铺,改在路口发传单。查尔斯一手举着自由党的大牌子,一手发传单,身手敏捷地穿行在
车辆当中,左右开弓,宛然一现代佐罗。但是这个路口并没有安全岛,他们与车争道,很不安全,后来被警察赶走
了。我开玩笑说,这一定是哪个民主党或共和党人打电话举报的。

查尔斯他们俩先回家了。我们四个干劲仍然很足,决定继续在店铺门口发传单。我说,现在看来,一个门口放
两个人,实在多余,不如葛锐格和乔留在沃尔玛,我和恰克去另外的店铺看看。大家都同意了。

可是,我们很快就发现,我去的那家店明文写着:此处不许做广告、散发文件。恰克去的地方则生意不是太
好,门口鞍马冷落,半天也发不出一张传单去。最后我们只好再回到沃尔玛去和葛锐格他们会合,变成了我和葛锐
格一边,乔和恰克在另一边发传单。

葛锐格,我在前面里介绍过,他曾经打算竞选自己所在的阿宾屯镇委员,后来因为时间不够而退出了。他身材
高瘦,虽然可能有40多岁了,看上去却很年轻。他总在温和地微笑着,很有亲和力,我给报社的投稿发到自由党的
邮件组后,也是他主动地帮我润色了英文。葛锐格并不是那种滔滔不绝地讲话的人,但却绝不缺乏口才,我觉得他
撤选真是太可惜了。他把传单递给人家时,不像我只是一个微笑加一声``嗨'',而是详细地对别人说:``11月4日将
有一个问题出现在选票上,是关于开阔地带\ldots ''然后指着传单上的各点理由,解释给他们听为什么应该投票反
对``开阔地带''计划。人们一般都会驻足听他讲完,这样的效果无疑比我只塞个传单好多了。

我于是也学会了这招,在发传单时说:``请投票反对增税。''人们听到``税''这个字时还是很敏感,反应明显
和刚才不一样,有些人还一边接过来一边说:``我反对任何税。''然后我再乘机向他们解释情况,宣传我们的主张。
这时我才发现一个门口站两个人还是很有必要的,因为经常会有人来问详细情况,那时就需要第二个人继续发传单
了。

当然,对我们反感的人也很多。恰克在那边就遇到了一个民主党的女士,站在沃尔玛的门口和他辩论起来,最
后当然谁也说服不了谁。我遇到的情况更有趣些,有位西班牙裔的中年女士,推着堆满了货物的购物车出来,我把
传单递过去,说:``请投票反对增税。''她向我斜看一眼,不屑一顾地说:``我不交税!''

在沃尔玛出入的,当然是穷人居多,基本没有太多的房地产商,增加5\%的房地产税对他们的影响并不大。还有
很多是少数民族,很可能是新移民,恐怕也没有投票权的。我说:``我们应该去普王市的购物中心,那里都是高档
商店,去的以中产阶级居多,可能增税对他们的影响更大。''葛锐格分析说,虽然增税的影响对他们更大,但是他
们不像穷人那样对每年增几十美元的税那么在乎。我们的传单还是在沃尔玛分发更有效果。另外,普王市地处蒙郡、
特拉华郡、切斯特郡的交界处,而且名声在外,顾客四面八方来的都有,不像这个当地的沃尔玛,顾客基本上都是
蒙郡居民,都可以投上一票的。

到了下午6点时,天已经完全黑下来了,我们也要回家了。我发出去的最后一张传单是给一个白人男子,他问我
说:``增税?什么税?''我立刻向他解释:``蒙郡计划斥资2亿5000万美元,来保护开阔地带,可是这钱有40\%是付利
息的,只有60\%会被真正用在计划上。而以前的``开阔地带''计划,1年只需1000万美元,他们还花不掉,现在却开
口要1年1500万。我们应当投票反对这个计划,继续以前的便宜计划。''

他点着头说:``那你是民主党喽?''

我说:``不,我是自由党。''

他叫了一声:``哈!自由党?你们赢不了选举的!''

这一下可把我噎住了。幸亏葛锐格立刻接过话来说:``这不是选举,这只是个选票问题。''我得到他的提示,
也跟上来说:``对,我们不是在竞选,而只是呼吁大家在这个问题上投反对票。''

他说:``哦,原来如此,那我会反对的。投票是在星期\ldots 星期四对吧?''

我们一起叫道:``是星期二!''我可真没想到美国人的参政意识淡薄到这个地步,因为每年的投票日都固定在
11月的第一个星期二,他竟然连这都不知道。

这是投票日前的最后一个周末,对我们这些都身有工作的人来说,也是最后一次集体活动发传单了。两天后就
是正式的投票日,我请了一天假,在投票点继续宣传反对``开阔地带''计划。

\section{11月8日附记}

最后的投票结果是104,367票赞成``开阔地带''计划,占77.64\%,30,057票反对,占22.36\%。对``开阔地带''计
划最终获得通过,我们早有预料,并不感到难过,反而对22.36\%的反对票很是满意,因为整个蒙郡只有0.3\%的自
由党人,而我们面对的是占压倒优势的两大党联盟。换句话说,在蒙郡,有将近四分之一的投票者在这个问题上认
同自由党的理念,而不是共和党或民主党。用吉姆的话说:``他们都是潜在的自由党人。''可惜我们没有办法查到
究竟是谁投了反对票(《美国选举日流水账》里有解释),不然正好可以给他们寄自由党的宣传资料,邀请他们参加
自由党。

其实,我们还可以做得更好。``开阔地带''计划的支持者们为了让这个提案通过,做了很多工作,早在今年上
半年就开始宣传,花费在1万美元以上。自由党是小党,不能四面出击,开始时把重点放在寻找候选人来开展竞选
上,直到葛锐格撤选后,才集中力量来反对``开阔地带''计划,那已经是9、10月间了,我们只有不到两个月的时间
来宣传。

经费上的困难就更不用说了。整个蒙郡自由党的金库里只有1000多美元,砸锅卖铁全拿出来,也只有对手的十
分之一(最希望``开阔地带''计划通过的是那些银行家,我们怎么可能和他们比钱多?)。最后,我们的经费基本上都
是靠几个成员的捐款,加上寻找到的几个免费支持,比如传单的制作、印刷等等。

虽然没钱没时间,我们还是取得了很大成果。尽管近四分之一的反对票并不都是受到自由党的影响而投,但如
果没有我们的积极活动,反对率肯定会低得多。更重要的是,由于自由党是蒙郡唯一反对``开阔地带''计划的组
织,受到了媒体的广泛注意。美国报纸一般都要强调平衡报道,对任何选举、提案、争议都希望能看到来自双方的
意见。平时,这种对决往往来自于共和党对民主党、右派对左派,自由党人少言微,很少能发上言。但这次,由于
共和党和民主党一致支持``开阔地带''计划,那么报纸要找反对派的言论,除了零散的读者投稿外,只能来自于自
由党。吉姆、克恩、查尔斯等人都或被报纸约稿,或者主动投稿,在当地媒体上畅谈自由党的理念,很好地为这个
小党做了公众广告。

我们的活动也受到了全国自由党的注意。他们专门在投票日前发了一篇文章,介绍了我们反对``开阔地带''计
划的努力。

投票日后的第三天---11月7日,吉姆给大家发来一封Email,题目就是``自由党走向世界了!''原来,当天的
《华尔街日报》第二版上有专文报道目前在美国各地开展的``开阔地带''计划。在讲完了该计划所获得的支持后,
文章的最后说:

并非所有的经济保守派都支持这个计划。比如说,蒙郡自由党声称,加上利息,投票者们几乎要付上他们所批
准的1亿5000万美元的两倍。他们建议,由自愿捐款来保护开阔地带。``为这种东西付如此巨大的抵押,低效得可
怕。''该郡自由党主席吉姆$\cdot$巴伯说。

\section{美国选举日流水账}

\subsection{体验生活}

每年11月的第一个星期二,是美国的选举日。今年不是大选年,没有总统、联邦议员、州长选举,所以从媒体
到民众,选举都不是热门的话题。电视里舞照跳、马照跑,偶尔出现了政治话题,不是伊拉克,就是民主党为明年
的总统选举内斗正酣;身边的朋友,则激动于《黑客帝国》第三集明天终于要出来了,两眼放光地进行着哲学讨论。

不过,我作为对美国选举非常好奇的国际友人,兼蒙郡自由党会计,还是特地请了一天假,去和美国选举来个
``零距离接触'',同时散发传单,反对``开阔地带''计划。当然,我不是美国公民,所以没有选举权和被选举权,
但政治精神和奥运精神大概也是相同的:``重在参与'';况且我已经洋插队5年,当年插队的知青们全面体验了贫下
中农的生活,我也得体验一回这美国的政治生活。

今年的选举日是11月4日。前一天晚上,我在网上查到了蒙郡所有的投票地点,发现共有400处左右,密密麻麻
地铺在蒙郡每一片地方上。比如我所在的上梅仁(Upper Merion)地区,面积是23.34平方英里,共有16个投票点,平
均不到1.5平方英里就有一个,星罗棋布,方便之极。

投票是早上7点开始,晚上8点结束。我在大约早上7点半时,来到了离我家最近的一个投票处:``上梅仁老人服
务中心''。

门口有两张桌子,摆着些传单、标志,左边是民主党的,右边是共和党的,倒也壁垒分明。民主党的桌旁有两
个五六十岁的中年妇女,衣服上都别了很多徽章,章上全是``民主党''的字样,或者各个民主党候选人的名字。她
们看见我过来了,立刻招呼我到桌边,让我拿些传单。我拿了之后,有一位还殷勤地一边帮我拉门,一边指着传单
说:``请你按照这张表里的人选投票。''

我只好告诉她:``我不是公民,不能投票,只是好奇,来看一看的。''

她愣了一下,立刻说:``没关系。你有什么问题吗?''

我看到远处又有些选民过来了,不好意思耽搁她的正务,就说:``没有,谢谢。''直接走进了老人服务中心的
大厅。顺着``投票''的指示,我很快找到了投票的房间。那是个教室大小的房间,靠门的两边各放着一排桌子,另
外两边陈列着四台投票机,都用巨大的深褐色布罩着,以防别人会看见投票者的选择。有人正在里边投票。

侧面的桌上是登记表,桌后坐着两个慈眉善目的老太太,见我进来,就招呼我过去登记。我又一次解释说,我
不是公民,只是想来看看。

房间里的人又愣了一下。左边的桌后坐着的一个老太太说:``那你是想看看选举制度是怎样运行的吗?''

我说:``是的。''

她说:``我们可以给你展示投票前的步骤,然后\ldots ''她有些犹豫地对一个坐在投票机旁边的老头说:``鲍
勃,我们可以让他去投票机那里去投票试试看吗?''

我连忙声明说:``我并不会真的投票,只是想看看你们是怎么投票的。''

鲍勃走了过来,点头说:``我看可以。''

老太太就开始教育我:``首先,你应当把这些传单收起来。''她是说我握在手里的民主党传单。我转头往房间
里的其他人一看,果然,她们没有任何摆在明处的宣传资料。老太太继续说:``在投票的地方,任何宣传资料都不
能出现,你看我们身上都很干净,什么徽章都没有别。''

我把传单塞进口袋,又问:``但你可以是共和党人或民主党人吧?你们并不一定要是不属于任何党派才能在这里
工作吧?''

老太太回答说:``是的,但我们不可以在这里给选民任何暗示。其他人在外面拿了传单,也只能在这个房间外
面看,进了投票处后就必须收起来,不能让别人看到。''

然后鲍勃开始给我解说投票的过程。公民必须要在此之前就已登记为投票者,然后今天再在这里登记下你的身
份,才能去投票。他带我到一台没人在用的投票机前,掀开那巨大的褐布,露出了它的真面目。

投票机正面是一块很大的塑胶板,板上印着一张表,最左边一栏是今年选举的职位,第二栏是民主党的候选人,第
三栏是共和党,然后是些独立候选人或第三党候选人。自由党今年在蒙郡没有推出任何候选人,所以就没有出现在
上面。选民在这板上按要选的名字,选票就被记录下来了。还有两个地方,是为忠实的民主党或共和党党员设计
的,只要一按,就自动选了本党所提名的所有候选人,倒也省去按钮的麻烦,但我想,这样做的人必须要非常信任
本党的委员会吧,相当于``傻瓜版''。机器的最下面还有个键盘,如果选民对所有的候选人都不满意,要选其他
人,可以自由输入名字。

开完眼界后,我问:``选民自己的名字在哪儿?''

鲍勃回答说:``早就登记了啊。''

我说:``我常看到,报纸上有统计说,百分之多少的民主党人选了谁谁谁,百分之多少的共和党人选了谁谁谁。
如果选民自己的名字、党派没有和选择一起被记下来,这个统计怎么做得出来呢?''

鲍勃这才明白我的意思,说:``那是有记者呆在投票的地方外面,看见个人出来了,就上去问,您是哪个党
的,选了谁,然后弄出来的统计。选民可以回答,也可以不回答,甚至可以回答虚假的信息。至于某人到底是投了
谁,没人知道。''

由于普遍采用了电子投票机,所以现在计票非常方便,而且也极大地减少了舞弊的可能。鲍勃解释说,到了晚
上8点后,投票结束,机器会打印出投票结果。他给我看了一下早上他们测机器时的打印结果,是一串候选人的名
字,当然每个人后面的得票都是零。最后他们把这些结果汇总上去清点,所以投票处的人对投票结果是毫无影响的。
同时,机器的内存也被保存着,以备日后清点对照的需要。

我谢过他们,离开了这个投票点。出门时才发现共和党的桌子后坐的是个十七八岁的金发小姑娘,很腼腆的样
子。我也从她那里拿了点资料,并指着民主党的桌子问她:``你们不武斗吗?''小姑娘咯咯地笑起来,连连摇头。

\subsection{友谊第一,比赛第二}

体验完生活后,我也该开始发传单了。但我觉得既然和两大党的人都打过交道了,不太好意思又回去踢场子,
而且这儿的人好像也确实不多,大概两三分钟才有一个人。于是我又去了另一个投票处,是在一个犹太教堂里。

那里离我家也很近,开车大概3分钟就到。这次,我有了在老人服务中心踩过场子的经验,拿了反对``开阔地
带''计划的传单,下车就直奔教堂而去了。那里却只有一张桌子,我走近看时,发现是共和党,有两个中年妇女在
那里。看见我过来了,近处的一个连忙递给我资料。

我扬了扬手里的传单说:``对不起,我是来给自由党发传单的。''她二话不说,立刻转身把资料递给我后面的
人。我很不高兴地想:``这人咋这么势利呢?''不料待得她打发走那人后,就回过头来跟我双倍热情地打招呼,说她
叫文迪,握手问我的姓名,并把我介绍给她的同事。

寒暄未毕,又有人过来了,我和她一起停止说话,争着发传单。我还是和上次一样,递传单的同时强调``开阔
地带''计划会增税,她们则立足于根据地,兵精粮足,给投票者推销共和党的20多个候选人。我粗粗看了一下,她
们主要关心的是竞选,对半遮半掩地挂在选票最后的``开阔地带''计划其实兴趣不大。看来不会打起来,又兼双方
公私分明,我也就放心了。

这时,我才看见对面站了一个黑人老头,左手拿着传单,右手拎着一个牌子,写着民主党,手腕上还系着一副
画,是头用星条旗背景剪出的驴子(民主党象征)剪影。我对民主党的好感还是远大于共和党的,就过去跟他打了个
招呼,聊了起来。他叫特德,和天下其他的黑人兄弟一样健谈,这一聊就没停止。

先是自我介绍阶段。我刚说完我不是公民,但对美国民主制度很感兴趣。特德立刻就说:``什么呀,从本质上
说,所有的政治制度都是一样的,都是关于权力。美国制度和你们国家的制度并没有什么不同。''

真是一听就知道,肯定是个民主党,一开口就是典型的左派的愤世嫉俗。我当然不能同意,跟他争论说,区别
还是很大的。他得知我是从中国来的后,对我的洋插队史非常感兴趣,刨根问底地问清楚了我的每一个历史疑点问
题,尤其是当他知道我是做计算机这一行后,就开始给我讲他家的计算机的一个问题,问我是否知道是怎么回事。
我问他:``你家计算机是哪一年的?''他想了一下,说:

``大概20年前吧。''

我差点没当场晕过去。20年前的计算机,那得是个什么古董?我立刻谦逊地表示我的学识非常有限,对他的问题
毫无头绪。他表示理解。

我一般不会打听别人的隐私,但特德既然如此不见外地把我的底问了个仔细,我也不必客气,开始查他的户口。

特德的历史还挺复杂,做过很多职业,但基本上都是蓝领。最近十几年在一家公交公司干活,先做司机,后来
做维修,1年前刚刚退休。他原本住在费城,几年前破产了,就搬到这里来了。

我问他:``那你也是第一次来做这个吗?以前要上班,没法来吧?''

特德说:``不是。我从19岁就开始参与政治。''

我听了敬佩不已,又问:``那你也竞选过吗?''

特德说:``没有。''我心里暗暗地猜想,可能是因为竞选是一件费钱费力的事,他负担不起吧。特德继续
说:``我21岁时,曾经有个机会,共和党让我参加他们党,保证给我一个当地领袖的位置,不过我没去。''

我觉得很奇怪,因为黑人向来是铁定的民主党支持者,共和党怎么会来拉他呢?他说:``主要是因为不是每个人
都愿意在业余时间自愿为政治工作,所以共和党觉得我挺难得的,想把我拉过去。''

到了9点多时,人数渐少。其实这里人本来就不多,比老年人服务中心还要少,四五分钟才有一个人来。其中还
有一半是明确表示不要任何传单,或者直奔共和党的桌子而去的。愿意接受传单的人里,有些是非常礼貌地听我说
完全部内容,才收好传单,进教堂投票去,虽然我也不知道他们究竟听进去了多少。还有些人,本来没有那么感兴
趣的,但听到``税''这个字,就又多听了我几句。一部分人从投票处出来时,还把传单还给我们,非常客气地说我
们也许需要它们。当然,我也忍不住``不惮以最坏的恶意''的习惯,去推测也许是他们不愿意去找垃圾桶。

我一直和特德在一起。由于特德有着鲜明的民主党旗帜,而我并没有任何自由党的标志,我不由得想,不会有
人以为我也是民主党吧?民主党其实是赞成``开阔地带''计划的,人们从我这里拿走的传单里却在鼓吹反对``开阔地
带''计划。不过,和共和党一样,民主党人更注重的是选举,而不是这个选票问题,特德也毫无嫌弃我``沾光''的
意思,反倒和我聊得愈发欢了。

9点20分,我看实在没什么人来,就回家休息了。快到12点时,我才又出来。因为我觉得,大部分人都要上班,
只能在早上、午饭、下班三个时间来投票,所以我也只在这三个时间出来。

再回到犹太教堂那边时,但见一片融融。民主党已经换班了,另一个黑人兄弟大咧咧地坐在共和党的桌子旁,
和共和党的两位女士正在聊天。他叫詹姆斯,我过去后,他马上又跟我热络了,不停地开玩笑。有些玩笑我没听
懂,文迪她们却都笑起来,我说:``英语不是我的母语,你这些笑话是什么意思呢?''

詹姆斯就问我:``你从哪里来?''我说:``我从中国来。''他说:``哦,我还以为你是从你妈妈那里来呢!''说
完,他们又一起大笑起来。

詹姆斯见我不解的样子,正要给我解释,特德又恰好回来换班了。詹姆斯忙着和他交接班,待得忙完后,才对
我说:``来,问我'你从哪里来(Where are you from)?'''

我莫名其妙,只好问他:``你从哪里来?''

詹姆斯呵呵地笑着说:``我从我妈妈那里来。''

我怔了一下,这才恍然大悟。文迪笑着说:``你要习惯詹姆斯。''看来他们相互之间很熟。她们又说:``我们
对他已经很习惯了,所以知道他开的每一个玩笑。''

詹姆斯乐呵呵地和大家告别。他到下午4点还会回来换特德。共和党的两个女士望着他远去,一起笑着感慨万分
似地说:``我们爱詹姆斯。''特德笑着说:``每个人都喜欢詹姆斯。他就是那种你永远也惹不恼的人,你对他说什
么都可以。''

经过詹姆斯这一说,我们的气氛更轻松了。早上我和特德还``廉者不饮盗泉之水'',宁可站着侃大山,也不坐
共和党的桌子。现在大家也就不再客气,三党人坐在一张桌子旁,东拉西扯。中午的风不小,刮得桌上的传单到处
乱飞,我和特德一起手忙脚乱地帮她们按住。当然,大家按住传单的主要原因不是由于传单有多宝贵,而是怕传单
落到地上弄脏了地方。

最让我惊讶的一幕是:文迪居然拿着共和党的候选人名单。问特德:``你看看,这个史迪文$\cdot$奥尼尔是
否也在你们的候选人名单里?他是竞选法官的。''

特德低头一查,说:``对,他也在我们这边。''

我奇怪地问:``这个人是共和党人还是民主党人?怎么会同时出现在两边的候选人名单上?''

文迪说:``是我们共和党的。不过,他也争取到了民主党的提名。''

``民主党怎么会提名一个共和党人?''

文迪解释说:``这很常见。如果本党在某一职位上没有候选人,对方又有一个我们能接受的候选人,那提名他
我们又何乐而不为呢?''

我把双方的推荐候选人名单对照了一下,果然又发现了另一个人,也是被共和党和民主党都提名的。这一次这
个候选人是民主党的。``每年都有很多人这样越过党派,到对方那里去争取支持。''文迪告诉我。这些人都是几乎
肯定可以当选的。

我所关心的另一个问题则是两党的经费,大概是由于职业习惯吧,毕竟我是蒙郡自由党的会计,在其位就得谋
其政么,先来调查研究。

今天的助选活动,民主党人和我一样,都是义工,自愿来帮忙的。但文迪她们共和党的人就有钱拿,一天25块
钱,当然她们本身也确实是共和党人。我初步估算了一下,蒙郡有约400个投票点,假设共和党在每个地方放一到两
个人,那就要花费1万5000美元。也就是说,共和党仅仅在这一项上的支出,就已经是蒙郡自由党所有经费的10倍还
不止,还不算印刷传单、租用桌椅等其他花费。这个对比真让我心惊肉跳。看来共和党确实比民主党有钱,至于比
其他第三党,那就是比尔$\cdot$盖茨和我的数量级差别了。

我问文迪,共和党的经费是从哪里来的。她说,主要是人们的捐款,另外,有时候她们也会挨家去募捐。

我说:``愿意捐款给你们的人应该早就捐了,再去挨家募捐有效吗?''

文迪一迭声地说:``有效有效,当然有效。我们去挨家募捐,一般人们都会给,包括民主党人都会给呢。''

``民主党人也会给?''我觉得很奇怪。民主党人同意提名共和党人,还可以说他们自己没有候选人,给对方做个
顺手人情,但捐款给共和党,这钱却会被共和党用来反对民主党,这可是资敌行为啊,可以定为``民奸''的。

文迪说:``当然,最死硬的民主党人是不会给的。但是,一般的民主党人也会捐一些。主要是他们看到你挨家
募捐,觉得你挺辛苦的;大家都认为,一个人付出了劳动,就应该得到报酬。虽然我是来自于另一个党派,但我来
到他们的家门口募捐,他们一般总会表示一下。''

这种说法对我,当然新鲜之极。

总的来说,中午的发传单是胜利的活动、团结的活动。看来即使是两党政治,也远不是我原来所想象的阶级斗
争的弦绷得很紧,而是生活为主,政治为辅。大家首先还是街坊邻居,然后才是持不同政治意见的党员。即使两党
的高层在媒体和国会里斗得不可开交,似乎是阶级斗争天天讲,对于不那么狂热的普通党员来说,还是友谊第一,
比赛第二。

\subsection{大串连}

鉴于投票处并没有太多的人来投票,我也没发出去多少张传单,下午我决定去邻近的特拉华郡,帮助那里的一
个自由党人竞选,同时也过一把有候选人可以明确支持的瘾。

那是一个叫格兰诺德(Glenolden)的小镇,离我住的地方大概是40分钟的车程。自由党的候选人叫大卫
$\cdot$约翰,竞选格镇自治议会的议员。今年这个议会共空出四个席位,来竞选的只有五个人,除了大卫外,其
余四人都是共和党人,所以自由党觉得这次有赢的希望。蒙郡的很多自由党人都过来替他助选,因为自由党是小
党,能赢得一次选举不容易,大家虽然没读过毛主席兵法,但一旦出现机会,集中兵力的道理还是懂的。

我和大卫约了见面的时间、地点,就开车过去了。由于我是第一次到格镇,找不到地方,所以误了时间不说,
连地点都没找对,误打误撞地到了另一处投票点。那里早有两个小女孩在为大卫发传单,不过她们真是小得吓人
了,都不到10岁的样子,一问,原来是大卫的侄女,让我对大卫的人手匮乏很是感叹了一阵。

好在我很快就发现自己错了。她们的母亲本来也在这里,只不过刚好出去了。而在我下午的串连里,我也发现
大卫其实人手不少。我看这里没我什么事,就拿了一些传单,到其他各个投票点串连去了。

格镇共有五个投票点,我去了四个。除了刚才去的那个我没呆多久之外,在其他三处都呆了一会,正是``各村
都有自己的高招'',助选员们都干得很出色,而且风格各不相同:

第一处是在救火厂的投票处。那里有一个叫格尔的当地自由党人,带着一个小孩在发传单。格尔身材高大健
壮,为人开朗风趣。我因为对候选人情况不了解,就问他:``当我发传单时,应该怎么为大卫宣传呢?我的意思是
说,大卫和其他候选人最大的不同是什么?''

格尔一本正经地想了一会儿,说:``大卫是最矮的一个。''

这当然是开玩笑。爱开玩笑是格尔最大的特点,有一次,来了一位30多岁的女士,格尔连忙迎了上去。但几个
回合下来,他便败下阵来,原来她是前来支援共和党的助选员。格尔蔫蔫地又发了几张传单后,忽然对我说:``你
看我可以在投票完后和她约会吗?''

我忙再将那女士打量一番。她一副职业女士的打扮,眉目清秀,淡施粉黛,微笑着站在门口,别有一番成熟女
性的风韵。正当我在给她和格尔的配对打分之时,格尔已经声明说:``哈哈,我是开玩笑的。''

``那就好,''我说,``那我就排你前面了。''

格尔在发传单时也不忘作怪。他看见有人来时,就凑上去,嘴里做出蚊子飞的声音,有时候腰还一扭一扭的,
说:``吱~吱~啊,对不起,蚊子又来了。''对方当然知道这是他在拿自己比作到处骚扰人的蚊子打趣,就在会心
一笑的同时,格尔已经把传单递了过去,指着传单上大卫的名字对他们说:``请选大卫$\cdot$约翰,他是这次选
举唯一的独立候选人,其他人都是由'上头'选的,只有大卫是自己来参选的,把他选上去可以提高自治会议的独立
性。他才代表人民的利益,而不是少数政党大佬的意见。''

原来格镇是共和党的天下,自治议会里无论怎么洗牌,翻出来都是共和党的清一色,庄家坐稳了铁打的江山。
今年选举里,有4个议员任期结束,不知道为什么,共和党决定不再支持他们连任,另外提出四个候选人来。在他们
看来,当然是觉得这四个人正好可以填进去,不料大卫横地里杀出,因此现在就是从五个人中选四个,所以他的胜
算不小。

听完格尔的介绍,我不由得庆幸,共和党太轻敌,如果他们多提出几个候选人来,那大卫岂不又危险很多。但
从另一个方面看,也可以说共和党坚信他们的优势不可动摇,对各种挑战不屑一顾。在这里,民主党连候选人都懒
得提,因为反正也没有赢的希望,何必浪费人力财力。据说,人们拿到民主党的传单,都当垃圾一样扔掉的,对自
由党的候选人倒还和善些。所以,这里的选举基本上只是走过场,一切早在此前的共和党党内会议里就已经决定好
了,相当于党内民主就是全民民主,甚至,按照格尔的说法,共和党连党内民主都没有,几个大佬就把板拍下来
了,所以,他才向选民竭力强调大卫的独立性。

第二处是我们的老朋友克恩在为大卫助选。他9点钟就到格镇这边来站岗了。他听说我早上7点半就开始活动
时,还特意解释说,他首先还是得在蒙郡当地转一转,和当地的政客们见一下面,打个招呼,然后到这边来又赶上
上班时间的堵车,所以到9点才到。克恩作为全宾州自由党的主席,当然不会把目光只局限在他所住的镇、郡,大卫
是这一带最有可能赢的自由党人,克恩自然要来倾全力挺之。

我是和大卫一起到这个投票点的,一停车,两个小孩就扑了上来,向他要``徽章'',就是个勋章大小的东西,
写着大卫$\cdot$约翰和自治议会,相当于助选标志,早上见到的共和党和民主党人就都戴满了这种东西。原来,
这两个小孩是住在附近的人家的,出来玩时,被克恩花言巧语迷惑住(克恩会捏了嗓子学卡通人物说话,维妙维肖,
简直可以去做配音演员了),成了自由党的义工。

两个小孩一黑一白,黑的9岁,白的才8岁,长得粉琢可爱。大卫的竞选口号是``诚实、开放和有限的政府'',
那白小孩连字都认不全,却拿了大卫的传单,看见有人来了就拦住,结结巴巴地背诵说:``大卫$\cdot$约翰,自
治议会,诚实、开放的政府。''黑小孩就喊:``你又把'有限'丢掉了!''两个人说话奶声奶气的,煞是可爱,逗得大
人们个个都要和他们说几句,传单自然就更不好拒绝了。大卫站在两个小孩后面,在孩子们表演完后,乘机上前和
人们握手,说:``你好!我就是大卫$\cdot$约翰,我竞选自治议会的议员。''

可惜天色渐晚,到了5点半,孩子的父母就来把他们领回家了。不过此地有克恩镇守,我们也没什么可以担心的。
他今天西装革履,打扮得衣冠楚楚地站在门口,风度翩翩,乍一看还以为是候选人在拉选票呢。他的口才与形象俱
佳,竞选的经验又丰富,应该能吸引很多投票者。

第三处是一所学校,在发传单的又是来自我们蒙郡的吉姆和恰克。

吉姆也是个非常出色的助选员。大卫自己经常就是微笑站着,不大说话,吉姆站在大门口,见人过来,就上去
打招呼说:``你认识大卫$\cdot$约翰吗?''来人一般会迟疑一下,多半都是不知道吧,大卫就上来握手,自我介
绍:``我是大卫$\cdot$约翰,我竞选自治议会议员。''

吉姆接着说:``大卫非常谦虚,从不夸耀自己。''然后开始大做广告。他主要是强调大卫的奉献精神和参与精
神,介绍大卫此前虽然不是自治议会的议员,但却出席了议会的所有会议,对本镇发展有过重大贡献;如果他当
选,将会努力使镇政府变得诚实、开放、有限。最后,仍然是落到了大卫的独立性上,强调他是真正会为人民的利
益说话的。他的口才很好,一开口就能滔滔不绝地说下去,人们也都会礼貌地听完才进去投票。

有了吉姆在,我们也就省了不少劲,只有在吉姆正纠缠别人的当儿又有人来时,我和恰克才能捞到一点生意做。
由于我们这些蒙郡的自由党人都过来支援了,使这边的自由党人的活动一片热气腾腾,和蒙郡那边的冷清简直是天
壤之别。我和大卫串连时,每一处都造成了自由党的人数大大超过了共和党人的局面,至于民主党,就干脆看不见。

有没有候选人还是很不一样的,尤其是大卫精力充沛,不停地转战于各个投票点之间,与人握手。和候选人有
了``眼见为实''的接触,投票者再看到选票上的名字时,就能把他与那个活生生的人联系起来,一般都有显著的好
效果。

不过我们也都清醒地知道,格镇是共和党的地盘,大多数人都会投共和党候选人的票。也曾有一位选民来时,
一听吉姆介绍说和大卫竞争的其他四个候选人都是共和党的,就轻蔑地说:``哼,共和党。''但大多数人还是两不
相干,听了我们的介绍,又去共和党那边拿些资料,但也不像认识共和党的助选员的样子。

\subsection{从胜利走向胜利}

我们蒙郡的自由党人早就计划好,这天晚上投完票后,在一家酒吧聚会,就叫``胜利派对''。当然,真正的胜
利,比如说``开阔地带''计划受挫,或者大卫成功当选,大家也知道可能性不大,但总不能叫``检讨派对''吧,我
们还是欢欣鼓舞地聚在酒吧里,看着电视里陆续报出选举的各个结果。

最后的结果并不出人意料。投票赞成``开阔地带''计划的,有104,367人(77.64\%),而反对票只有30,057张
(22.36\%),大概可算是毫无反抗余地的失败。但考虑到在整个蒙郡,两大党都号召选民支持这个计划,而唯一的有
组织的反对派,是只占选民人口0.3\%的自由党,这可算是自由党很大的胜利了。

我们对大卫$\cdot$约翰的竞选亦作如是观。他败选了,四个共和党候选人的得票都是他的两倍左右,但是克
恩说,基本上,选票有两类,一类是四个全选共和党人,一类是只选了一个大卫,所以这是一次大卫单挑共和党的
壮举,结果是一比二。由于格镇是个小地方,下次只要我们动员起来,再发动一百多个中间力量,比如民主党人,
就可以胜选。所以看来以后选战还要在那里展开。有人建议大卫去选市长,因为他现在已经在反对共和党的选民中
有了知名度。可是吉姆对大卫甚是同情,说他花了自己的1000多美元来参选,家里的人都花了数不清的时间,最后
还是失败,折腾不小。

\subsection{自由党筹款晚会}

蒙郡自由党的经费非常拮据,为此,我们举办了一次筹款晚会。大家投票决定,12月18日晚上在费城的尼罗河
酒吧举行。

在那里一个人的酒水消费大约是20多美元钱,捐款共有三个档次:自由之友,50美元;自由情人,100美元;自
由傻瓜,150美元。另外还有抽奖活动,筹款的一半变成奖品,另一半捐给自由党。不过如果你囊中羞涩,那也没关
系,在吃喝结束之后,你再来就是了,一样地可以参加活动。活动主要是自由党的一个总统候选人加里$\cdot$诺
兰发表演讲,以及大家的社交讨论。

照理说,我不是美国公民,不能捐款给政治组织,不过我们蒙郡自由党是注册为与政治无关的非盈利组织,所
以我也可以参加。今天晚上一共来了20多人,蒙郡自由党的活跃分子只占一半,还有从其他各郡、甚至新泽西过来
的自由党人,包括我们上次支持过的大卫$\cdot$约翰。安排给我的座位周围的人,我一个也不认识,只好硬着头
皮和众人社交,还好自由党人基本都十分健谈,很快就热络了。

坐在我对面的是一位金发女子和她老公。她是纽约的一个自由撰稿人,出过几本书,不过销量一般,都是那种
谈人生、谈社会、也涉及些政治的书。她说,她以前从来就没有听说过自由党,只是对民主党和共和党的理念都不
认同,所以一直徘徊彷徨着。后来认识了她老公,他是多年的自由党人了,以前住在宾州,还竞选过,当然最后失
败了,后来他搬到新泽西,遇到她后,在把她骗到手的同时,也轻而易举地就把她变成了自由党人。她这一说,我
心大悦:原来没有听说过自由党的人并不只有我一个。加上我对写作也很有兴趣,便越聊越投机。

坐在我左边的是一个40岁左右的男子,留着一头长发,也加入了我们。先是互相介绍,他说他也是搞计算机的。
不过再问下去,原来他不是写软硬件,而是给人家刻音乐光盘的,同时也做包装啊什么的。我一听又很高兴,立刻
声称我对音乐也很感兴趣,又聊起了音乐。他也弹吉他,不过没有自己的乐队。

正聊得高兴时,尼罗河酒吧的节目来了。一阵中东音乐声中,一位舞女扭着肚皮舞出来,在各席间穿行。大家
纷纷鼓掌。她便邀请客人同舞。另一桌的很多年青人都高兴地起来和她一起跳。她再邀请我们这一桌时,大家便拘
谨得多,只有几个人接受了邀请。

加里$\cdot$诺兰是个阿拉伯裔美国人,要出来竞选总统,虽然还没有得到全国自由党的提名(这个会要到明
年5月份才开),不过也是今晚最重要的贵宾了。他便坐在我的左侧面,大家自然撺掇舞女跳到他身边去。

加里本人是一个广播电台脱口秀主持人,因此口才是极好的。大家吃完后,便请他发言。先是他的竞选经理,
鼓动大家为他的竞选作贡献。

加里发表演讲,无非是他的竞选纲领,其要点都是典型的自由党主张,我也没有留下什么鲜明的印象。在说到
伊拉克形势时,他强调美军应当立即撤回,大家都鼓掌。他讲完后,大家提问,又讲到伊拉克形势,他说,美国人
不应当管世界的闲事,没必要为了别人的事牺牲自己子弟的生命,最后结果反而弄得局势越来越糟。

这孤立主义我却也不同意,就举手问了个问题:难道你不认为美国应当在世界上承担某种责任吗?比如,伊拉克
入侵科威特的时候。

他说:你知道伊拉克为什么会入侵科威特吗?事实上,萨达姆入侵科威特之前,跟美国人打过招呼,美国人说,
没问题,你干吧。结果,伊拉克人一入侵科威特,美国人马上就组织联军把他们赶走,乘机在中东驻扎下来,这次
又乘势把萨达姆连根拔了。他的意思就是,这一切都是美帝国主义的阴谋。如果美国人不去趟这淌混水,原本世界
多美好啊(这一句是我引申出来的)。

我对他的说法不太服气,但对他引用的内幕真相也无从反驳。他乘势就结束了演讲。

然后是抽奖。奖品包括书、日常用品、印着``2004: Bush or Liberty(布什或自由)''的T恤,一等奖是现金,
数目不是个整数,而是90多美元。我当时觉得奇怪,后来才想明白,因为抽奖是要把买奖券的钱的一半捐给自由
党,一半还给抽奖者,其他各奖品的价格都早已定好,而到底来宾会买多少奖券事先却不知道,所以最后一等奖便
是总收入减去其他奖品成本,就成了个奇怪的数字。

一等奖抽出后,是一个蒙郡自由党人得到了。有人大叫:``为何不把这笔钱就捐给蒙郡自由党算了?''抽奖的吉
姆笑道:``当然我们会非常高兴,如果乔把这笔钱捐给蒙郡自由党的话。''又有人喊:``捐给加里$\cdot$诺兰的
竞选活动!''

得奖者站在那里,一脸犹豫。我看了,心想,嗯,换了我,我也舍不得。这时有人大喊一声:``做一个自由党
人!捂紧你自己的钱包!''

大家都笑了,鼓掌。我也鼓起掌来。自由党确实在经济上就是主张钱由自己支配,而不是交给政府来统筹安排。
后来听说他捐了其中的30美元。

这次捐款晚会,我们一共收入1000多美元,去掉各项开支后,我们的纯收入是500多美元。虽然不多(听说布什
的捐款晚会,最低档的入场券也值1000美元),不过也有我们总账户的一半了。

\subsection{退休法官反对``反毒战争''}

今天是1月份的第三个星期二,我们照例在《费城问讯报》大楼举办自由论坛。本期论坛的主持人是新泽西高等
法庭的退休法官马丁$\cdot$海因斯(Martin L.~Haines),题目是《从司法角度看禁毒的失败》。

海因斯法官看上去已经70多岁的年纪了,说话很慢,时常需要看稿才能继续演讲,口齿也不是特别清楚,因此
我听得很吃力。而且我显然不是唯一一个听着听着就走神的人,坐在我旁边的一位先生就一直在打盹。不过,听了
几分钟之后,我也就慢慢适应了。

法官的演讲以他曾亲自审判过的一个案例开始,那是一个在佛罗里达州被逮住的司机,警察怀疑他藏有毒品,
搜查了他的车,结果果然找出了300多磅的大麻。海因斯认为根据法律,司机有罪,但也质疑警察的搜查是否违反了
宪法对个人权利的保护。

随后,他列举数据和事实,说明了政府的``反毒战争''的失败。``反毒战争''开始于20世纪80年代初期,结果
20年的战争打下来,美国因毒品而坐牢的犯人反而越来越多,增长了13倍(当然这也可以说是政府``反毒战争''的战
功显赫,但也说明毒品禁止没什么效果,还是有很多人从事毒品交易)。目前,美国有超过200万的犯人在服役,每
个犯人每年需要政府花至少20000美元来供养,光这笔开销就是400亿美元,比很多国家1年的国民生产总值还多,而
且,现在犯人的数目还在不断增加,各地都在扩建监狱。仅加州就需要扩建21座监狱,大约需花费5亿美元。我们知
道,加州前任州长戴维斯正是因为经济太糟而下台的,如果不建这些监狱,能把5亿美元省出来,或许不至于被罢免。

海因斯法官还指出,犯人中黑人的比例,远高于黑人在美国人口中的比例;而与毒品有关的犯人中黑人的比
例,又远高于黑人在犯人中的平均比例。这里面显然存在着深刻的问题。同样,在80年代初,女犯人中与毒品有关
的占20\%左右,如今这个比例已上升到80\%。他认为,这说明``反毒战争''存在着不平等的因素。

至于用什么方法取代``反毒战争'',法官说,时间不允许他展开讨论,但他的观点是:当初``反毒战争''的发
动太过仓促,没有经过深思熟虑和充分的讨论;而在毒品问题上,社会上的政治影响太大,以至于人们对毒品形成
了惯性思维,看法陷入了固定模式。海因斯法官并没有提出他的应对方法,只是说明,``反毒战争''是一场失败的
战争,造成了很多不必要的损失,我们应该重新讨论这场战争是否应该发动,该怎么进行,或用什么替代。

在我看来,他的观点还是比较温和的,并没有激进地立刻要求毒品合法化,而只是指出现行方法的弊端,呼吁
人们重新审视这个问题。有些自由党人提出,只要在毒品的交易、使用过程中没有涉及暴力,就应当是合法的,美
国历史上曾经实行过臭名昭著的``禁酒令'',结果酗酒没有减少,黑帮倒因此而兴起。海因斯法官也说,目前世界
上流通的毒品大概有数千亿美元之多,这些毒品的消费者中,美国人占了大头。其实,毒品本身并不值这么多钱,
是禁毒使得毒品流通的成本大增,才让贩毒成了暴利行业。如果毒品能够合法化,那么毒品将大幅度贬值,美国人
的财产也可以少向海外流失很多。

演讲后的提问非常多。有人也提出,毒品可以像酒那样合法化。海因斯法官说,这是一个办法,酒店旁边就是
毒品店,只要年龄足够,就可以限量购买,当然其中的细节还需要再讨论。但他本人的意见是由人民、政府、立法
机关再协调商量,并不认定毒品合法化就是唯一的解决方案。有人问他,您作为一个法官,认为政府和立法机关对
毒品的禁止是否违反宪法。他想了一会儿,老实地说他无法判断。

我身边一直打瞌睡的老兄这时醒了,无比英明地指出一点:80年代时,人们有``执法工业''的说法,那么现在
的形势是,有很多警察、法官、监狱人员就是靠抓毒品犯罪吃饭的,如果``反毒战争''不打了,这个行业的股票必
然要暴跌,这些人可就失业了。海因斯法官也笑了,说,毋庸置疑,执法人员和毒贩虽然互相对立,但其实都靠对
方养活着,执法人员靠抓毒贩来领工资,毒贩靠执法人员的禁止来冒险牟取暴利。也有人说,其实如果毒品合法化
了的话,这些失业人员可以被配置到教育部门去,向公众,尤其是孩子们,宣传毒品对身体的伤害和如何正确使用
毒品。这样的使用效果,比发动``反毒战争'',将几百万人投入监狱要好得多。

在去年的选举中,我曾经收到过一封信,来自于竞选蒙哥马利郡司法公职的一个民主党人。这是一个韩裔,大
概是看到我的名字像是亚裔,所以发信来寻求支持。他的主张里最主要的一条就是要严厉打击毒贩,提高对毒贩的
刑求标准。他这还是典型的民主党思路,要靠政府和社会来解决问题。在我看来,人的欲望都是洪水猛兽,靠堵是
堵不住的,必须要导。人们往往被洪水吓坏了,本能地就想靠禁欲来解决问题,结果欲望压制不住,反倒派生出其
他问题。禁酒令就是个典型的例子。道德对情欲的压抑,也造成了类似的效果。有了这么多前车之鉴,看来人们还
是没有总结出教训来。

这个人在选举中最后输给了共和党的候选人,但这和他的主张有多少关系,我就不知道了。

今晚,布什在电视演讲里也提到了毒品问题,他提出一个6亿美元的计划,来帮助额外的30万美国人戒毒。这句
话说完后,掌声一片。我注意到听众里似乎是民主党坐在左边,共和党坐在右边。布什不论讲些什么,只要一个意
思讲出来了,共和党员们马上就掌声雷动,很多人热烈鼓掌还不过瘾,还要起立致敬。民主党的人则坐在那里无动
于衷,面无表情。如果光看后来的记录稿,布什的讲话真是1分钟要被掌声打断两次,盛况空前,不过看了现场录像
才知道,那不过是共和党自己在给总统捧场。但在这个毒品问题上,我看到很多民主党人也鼓掌了,可见在毒品问
题上,两党的立场还是很接近的。

当然,我不反对戒毒计划。很多人在清醒时,痛哭流涕地要戒毒,可毒瘾一发,又哪怕卖身也要吸毒了。这种
人,在清醒时表示了要戒毒的愿望,只是由于毒瘾已深而无法靠自己的意志做到,社会自然应该施予援手。这和希
望毒品合法化并不冲突,就像市场上减肥产品、健身秘方层出不穷,可我们还是不能禁止那些高脂肪食品一样。在
荷兰,吸毒(限量)、嫖娼都是合法的,因为荷兰人相信任何成年人的自愿行为都应当是合法的,不过在道德观念根
深蒂固的美国,人们还是习惯先用道德眼光来审判问题。

\section{反毒战争}

这次自由论坛是反对``反毒战争''的系列讲座的第二部分,主讲者也来自新泽西州,叫艾德$\cdot$弗尔钦
(Ed Forchion),自称``新州草民''(NJ Weedman),自创了一个``大麻合法党''。我一看见传单上他的形象就喜欢上
了他,觉得这肯定是个非常叛逆的家伙。

艾德出生在一个基督徒的家庭,却从小就对基督教反感。他认为基督教是白人强加在黑人身上的宗教,黑人应
当信黑人自己的宗教。后来他信了一种牙买加黑人在30年代发明出来的宗教Rastafarian。

16岁时,艾德抽上了他生平的第一支大麻烟,他发现大麻对他所患有的气喘病有很好的疗效,于是,他不再服
用医生开出的``化学''药品,改用大麻来对付气喘病,效果非常好。从那时起,他就得到了个``weedman(草
民,此'草'又指大麻)''的外号。

随后,他和平常人一样参军、恋爱、结婚、生子。下面是1989年6月他随军驻扎在西德时的全家福。

在军中,他自然不能再使用大麻,由于他的气喘病实在厉害,很快就因病退役了。这家伙把自己的名字改为
``爱德华'',重去参军。这次,他偷偷地继续使用大麻,结果一直服役到1990年正式退役,气喘病也没有发。

退役后,他买了辆卡车,在美国两岸之间跑运输。1992年在亚利桑拿州,他遇到了一群致力于使大麻合法化的
人,艾德也积极参与了他们的活动。他成立了``大麻合法党'',参加过几次竞选,曾经得到过两千多票。

1997年11月,他由于被发现藏有大麻,被新泽西警方逮捕。艾德拒绝了州方的认罪求情协议(plea deal,和政
府方达成的为了避负较重的处罚而承认轻罪的协议),决定要上庭抗争。他引用马丁$\cdot$路德$\cdot$金博士
的一段话说:``要反抗不公正的法律,必须是公开的、充满爱心的,而且愿意接受任何惩罚。我认为,一个人去故
意违反那些他的良心告诉他是不公正的法律,并且愿意为此入狱,以激起人们的良心感知到不公正的存在,这样的
人正是在表达对法律最高的敬意。''

在法庭上,他被判入狱10年,但在17个月后,他被释放出狱,处于强化监督管制(Intensive Supervision
Program ,简写为ISP)之下。强化监督管制是比假释更严厉的处置,警方随时可以搜查他的身体和住处是否藏有大
麻,并不许探望他自己的孩子,不许继续从事政治活动。艾德感到自己的合法权利被极大地侵犯了,从此,他改称
自己为``宪法第一修正案活动者(1St Amendment Activist)'',认为自己有权说他任何想说的东西,政府对自己的
迫害违反了宪法第一修正案。

由于艾德仍然四处公开宣称大麻应当合法,在2002年,新泽西警方再度逮捕了他。法庭判决艾德不再适合强化
监督管制,必须继续服完他10年的刑期。这件事引起了媒体的广泛关注,人们质疑这是对言论自由的侵犯。新泽西
最大的报纸《明星记事报》(《The Star Ledger》)报道了此案,这是登在该报上的听证会时的照片:

2003年1月,联邦法官埃尔纳斯(Irenas)推翻了新泽西法庭的判决,下令把艾德释放,重新置回强化监督管制。
艾德继续从事他对``大麻合法化''和``捍卫宪法第一修正案''的活动。

以上是艾德的传奇半生。今天他演讲的题目是``种族歧视的反毒战争''。他指出:

1. 毒品使用者中有13\%是黑人,可在被因逮捕而遭毒品危害的人群中有35\%是黑人,在认罪的人中有55\%是黑
人,而在最终入狱的人中,则有高达74\%的是黑人。

2. 黑人因毒品而被判入狱的比例是白人的13倍之多。

3. 毒品使用和贩卖在少数民族和非少数民族中的比例是一样的,可少数民族被阻截、检查、逮捕、判决、入狱
的比例高于白人。

4. 在面对同一罪名是,有色人种通常会比白人判得重。

因此,艾德宣称反毒战争是一场种族战争,有色人种应该起来反对。他曾经数次去华盛顿抗议,一方面以他的
亲身经历说明政府对公民人权的侵犯,一方面宣传``反毒战争''是一场不公正的政府运动。

同时,他给我们放了他和别人合作所拍摄的三部反对``反毒战争''的电视广告片。在这些广告里,艾德站在美
国国旗前说:

在这世界上最伟大的旗帜之一前,我要告诉你,这个旗帜所代表的``自由'',正处于危险之中。正如20年代的
禁酒运动一样,``反毒战争''正在毁掉我们的自由社会,甚至连宪法第一修正案所保证的言论自由、宗教自由,都
被削弱了。政府会因为你公开宣称``大麻合法''而带走你的孩子---这就发生在我的身上!现在,是该结束``反毒战
争''的时候了,因为它实际上是一场``反个人自由战争'',而它正在美国进行着!

您听到过我们的政府说大麻很危险、容易上瘾、没有医疗价值吗?---我听到过。但我也知道一些科学上的事
实:大麻从未致死过人、帮助过很多人、并且被用作药物已有上千年之久,甚至医生们也已经公开质疑关于大麻危
险的说法,许多医生推荐使用大麻来治疗某些疾病。您相信谁呢:您的医生,还是您的政客?

您怀疑过``无毒美国''的现实性吗?很多人意识到,不可能同时存在``无毒美国(drug free America)''和``自
由美国(free America)''。我选择自由。美国对自由世界的领导,正由于``反毒战争''而遭到怀疑。我们怎么可能
领导着自由世界,而被关押在监狱里的公民却是世界上最多的?您觉得这是美国吗?让我们使自由重归美国,结束
``反毒战争''!


艾德筹集到资金,想在电视里播放这些广告,却被电视公司Comcast拒绝。艾德因此把Comcast告上了法庭,结
果被驳回。艾德决定还要继续把官司打下去。

费城也是艾德常来的一个地方。由于美国法律规定,在宗教仪式中使用大麻是合法的,艾德便和一个白人帕特
每月的第三个星期六来一次费城的自由钟,在那里进行``宗教仪式'',并公开吸食大麻。自由钟是美国独立的象
征,不过我至今也不知道艾德他们在那里进行的是何种``宗教仪式'',是``爱国教'',还是他所信奉的
Rastafarian教?

第一次去时,艾德为防万一,还特地请了个记者一起去摄像。果然,他们还没吸多久,就有两个警察过来,询
问他们在干什么。可能是由于有记者在摄像吧,警察未敢造次,只是分别询问了两人,然后给两人发了传票,却没
有写明出庭日期。于是艾德他们从此就放心大胆地每月去自由钟吸毒一次了。费城警方对他们两人现在也是见惯不
怪,由他们去了。

后来,他们看记者拍摄的录像,艾德忽然发现原来自己被警察搜身了,而帕特却没有。说到这里时,坐在台下
的帕特笑着对大家说:``我那时使用了我的'白人力量(white power)'!''

艾德对此自然十分愤慨,以种族歧视的名义把警察告上了法庭,但好像没有什么结果。

这次大概是自由论坛最吸引人的一次了。艾德一直讲到晚上10点钟,仍然滔滔不绝,听众也都意犹未尽。但由
于时间已到,才不得不停了下来。随后,我们按惯例转战到酒吧里去继续聊天。我是第一个到酒吧的,然后恰克也
来了,他兴奋地说:``这是迄今为止最好的一个演讲者!''

大家都到了之后,我问了艾德一个问题:``我常在电影里看到一些关于监狱的故事,都是些虐待啊、黑幕啊,
以你的体验,那些都是真的吗?''

他说:``以我的亲身体验,那些都是夸大。---但是,我所在的监狱都是关押毒品犯的,那些刑事、暴力的罪
犯,都不关在那里,所以我只能说,以我在的那个监狱来看,电影里的那些情节都是夸张。''

聊到快12点,我便回去了。后来上了他的网站 http://www.njweedman.com/ 去看了一下,很不错。

\subsection{报税日的参选签名活动}

每年的4月15日是美国的所谓``报税日'',去年的税表必须赶在这天之前填好寄给税务局;实在填不完的,也要
在此之前申请延期。所以,每年的这一天,邮局里都会挤满了寄税表的人。一方面,拖拉精神是普天之下各国之人
所共有的,美国人也不例外,很多人总是吃喝玩乐优先,税表能拖则拖,实在拖不下去了才填;另一方面,填税表
也确实是件非常麻烦的事,美国关于税的法规多如牛毛,一不小心填错了,要是自己吃亏也就罢了,要是政府吃了
亏---你就等着瞧吧,美国税务局在江湖上声名卓著,岂是浪得虚名之辈?

我们蒙郡自由党觉得这天是一个开展参选签名(ballot access petition)的好机会。所谓参选签名,是由于法
律规定,在上次选举中的得票不满一定数目的,就必须在下次的大选里征集到足够的签名,才能让自己候选人的名
字出现在选票上,具体到宾州,是要求两万个有投票权的公民的签名。自由党是小党,自然每次大选都逃不过这一
劫。更糟的是,签名征集到后,``敌人''(主要是共和党,他们有人有钱,又斗志昂扬,抱有``损人不利己''的精
神)一定会来挑刺,企图把一些签名说成是无效的,而那时再去重新征集签名已经来不及了,所以我们必须征集到3
万个签名,才能保证打退敌人的猖狂进犯。

这是一件费钱费力的事情。据估计,每个签名大约要花费3美元。美国确实有这种公司,专门征集签名,如果我
们把这件事``外包''给他们,至少要9万美元。宾州自由党显然花不起这个钱。美国自由党总部倒有专门用来进行参
选签名的经费,但只能用来支持落后地区,宾州是自由党比较茁壮的一个州,就不用指望从全国自由党总部拿到1分
钱了。为此,我们只好自己动手,丰衣足食,有钱的出钱,有力的出力,四处征集签名去。

我一直到昨天晚上才决定参加这个活动,随即给蒙郡自由党主席吉姆打了电话,叫他把签名表的电脑文件发给
我,我明天自己打印出来。他告诉我说,这个我没法打印,必须到他那里取。那时已是晚上10点多,我只好冒雨开
车到他家里去,拿了三张签名表。原来那是州政府发出来的正式文件,从总统候选人到参议员,各个职位都有,自
由党也找到21个积极分子,把他们的名字填了进去。这些人并不是真的要竞选这些职位,只是走个形式,使我们的
参选签名活动可以开展。当然,相关的填表、交费这些麻烦还是少不了的,所以,当初为了动员大家出来``捐''个
名字,也都费了不少劲。

我本来打算去我家旁边的普王市邮局,吉姆又给我推荐了一个大的邮局。这两个邮局都是处在蒙哥马利、切斯
特、特拉华三郡的交界处,因此吉姆给了我三张表,因为一张表上的签名都必须是来自同一个郡的。每张表上可以
有116个签名。然后我又问了他些问题:

``必须是美国公民才可以签名吗?''

``是的。''

``一个人如果已经签了其他党的,还可以签我们的吗?''

``我认为可以。''吉姆想了一会儿又笑着说:``我看不出不可以的理由。''

我心里打了个疑问。不过今天的经历证明我完全过虑了,因为这些特殊情况根本就没有发生。

中午时, 我从公司出发,12点半不到的时候到了普王市邮局。毕竟自去年选举日以来,半年没有干这种``沿街
乞讨''的事了,内心又开始惴惴。给自己打了半天气,才鼓起勇气来,拿着签名表进了邮局。

果然,邮局里排着长长的队,我便在过道里站着,看到有个人寄完税表,走出来了,便迎上去说:``您可以为
我们的参选签名吗?''那个人奇怪地看着我,说:``这是干什么的?''我正要解释,已经走来一个邮局的工作人员,
对我说:``你不可以在这里进行这些活动,这里是邮局的财产,你必须到外面,离开邮局一段距离才可以进行这些
活动。''我问她:``那我需要离多远?''她说:``你先到外面去,我马上来给你解释。''就把我赶了出来。那个刚才
被我拦住的人已经乘机走了。

糟糕的是,这个邮局有两个出口(我只骚扰出来的人,因为进去的人都在赶时间,不会有心情理我的)。我权衡
了一下,认为某个出口的客流量会更大,便守到了那个门口。很快又有人出来,是个职业妇女的打扮,大概也是乘
着午饭的时间出来寄表,我和她一说,她很爽快地便签了。

旗开得胜,我心里放松了很多。看着邮局的玻璃门,我想:有人出来时,我为他们拉门,再要求他们签名,他
们拒绝的可能性一定就会小些了;另外,有人进来时,我也为他们拉门,他们一定就会对我留下好印象。这个策略
不错,可是我又觉得不好意思,好像也太``乞讨''了。然后我想到了去年选举日在老人中心时,那个民主党的助选
员就是为人开门的,当时确实觉得她热情过分,好像必有所求,不过同时也觉得她很周到体贴。最后我决定要讲政
治,不讲脸皮,还是为别人开门吧。

不过这个善意还没有来得及开花结果,便告夭折了。我刚开了第一个门,正在给人家解释呢,那个邮局的工作
人员又钻了出来:``这里仍然是邮局的财产,你在这里会堵住路。你必须到停车场上去。''我又问:``你到底要我
离这个建筑多远呢?''她说:``在停车场的白线之外。''自然,我刚拉到的那个``顾客''也乘机溜走了。

我没有办法,只好再退到停车场。好处是这样离另一个出口倒近了,我可以往来于两个出口之间,来回奔袭。
坏处是这里车来车往,风又特别大,我需要一边腾挪于车辆之间,一边防止来之不易的胜利果实被风吹走了。

总的来说,大概有三分之一到二分之一的人愿意签名。当然,爽快程度不等。有的人几乎不听我解释,大笔一
挥就签了。有的人要打听清楚我到底在干什么,这时候我就要告诉他:我们只是希望征集到足够多的签名,这样宾
夕法尼亚人在大选时就可以多一个选择,您在这里签名,并不等于您就必须同意我们的主张,或者必须到时候为我
们投票。有人就会继续追问:你们是谁?我报上自由党的名头后(其实这个词很醒目地印在表上,不过大部分人没空
细看而已),有人就明白了,有人还会继续问:你们的主张是什么?然后我再解释:我们主张政府越小越好,税收应
当被服务费所取代;我们认为目前的报税制度荒谬透顶,我们交了很多税,政府却连收税这么简单的事情都弄不
好,还要我们每年4月像傻子一样自己填税表,那些税扔到水里还能打个水漂呢\ldots 这番说辞自然能得到很多人
的共鸣。

不过一般来说,以上的每一步也都会引起某些人的拒绝。比如,有人一听到签名,就说:``我从不参与政
治。''有人则在听到自由党的名字后立即娥眉倒竖(也许是巧合,这样的反应全来自白人中年妇女),如同被毒蛇咬
了一口似地往后一缩,立即严词拒绝,客气一些则说:``对不起,我的立场不容许签这个名。''还有人则是在听到
我的解释后,摇摇头而去,大概觉得这个主意太不现实了吧。

其实,参选签名的要点,不在于签名者同意自由党的主张,而在于签名者乐见选票上有尽可能多的选择。正如
一位年轻的男士在签名时说的:``这就是民主的意义所在。''类似的,下午还曾有另一位男士在签名时说:``我是
共和党人,但我支持你们出现在选票上。''

另外两个比较鼓舞人心的场景是:一位中年男士在听说我是自由党后,很高兴地说:``太好了,我正打算今年
投票给自由党呢。''他说,他对民主党和共和党都不满意。我当然非常欢喜,赠送了他一份自由党的宣传材料。他
签完后,和我热烈地握手道别,说以后再见。那感觉跟地下党和组织接上了头似的,觉得全世界都不在话下了。

还有就是有一位老人,柱着拐杖踯躅走出邮局,我向他从最开始一直解释到最后,他还跟我争论上了,说用服
务费代替税是不可能的,会使整个政府的收支完全失衡。我说这就是为了要约束政府的权力,不让它乱花钱。这样
的争论最后自然谁也说服不了谁,双方互说一声谢谢而别,我再去拉别人。那个人很爽快地签了名,签完后,我一
转身,却发现那个老人又站到了我的身后,说:``我想了一下,我还是应该给你签这个名。''

总的来说,这里的人都很客气,不签的人大多说没有时间,还经常附上句``谢谢''。最有礼貌的是一位带着小
孩的年轻母亲,她文静地低头走过来,我向她解释了我们的活动后,她摇头拒绝了,然后说:``但我非常感谢你愿
意给我这样的机会。''她说话很慢,每个单词的多个音节都清清楚楚地发出来。

只有一次,三个年轻人走出来时,我刚迎上去开了个口,一个人就干脆利落地说:``No!''然后就和他的伙伴们
窃笑起来,大概是觉得很酷吧。我也只好耸耸肩,自嘲地笑笑。

我自己的原则是:进去的人不拦,只拦出来的人,而且如果出来的人在打手机,我也不会去拦。美国人比较担
心隐私问题,有几个人在签名的时候都问:``我把地址留在这里,不会收到任何东西吧?''这个问题我是有把握的,
告诉他们,这个表是要交给州政府的,我们绝不会给您寄任何宣传材料(我没有告诉他们的是:本党穷得叮当响,人
又少,就是想要给您寄东西也有心无钱)。

还有一次,一位女士在看了表后,发现需要留地址,就直捷了当地说:``我不签,我不信任你们这些人。''说
实话,这是今天唯一让我吃了一惊的话,因为大家一向都说本人面相憨厚、神情纯朴。我有点喊冤似地说:``您不
相信我?''她一边钻进车,一边说:``我从不会把自己的信息给政治组织!''

快到1点半的时候,我回家吃午饭,到4点时,又回到普王市邮局。

到了5点5分,我觉得这边的事情差不多结束了,就开车去了昨天吉姆推荐的那个大邮局。那是在切斯特郡,不
过只有10分钟的车程。到那里一看,果然不愧是大邮局,排场就不一样:进出路口各有一辆亮灯的警车,站着一个
警察,一个邮局的工作人员,在那里为过往车辆收税表,免去了他们进邮局的麻烦。当然,前提是你的邮票必须已
经贴好。不然的话,还是得进邮局。他们平时开到晚上9点半,今天是特殊日子,足足开到午夜12点。我进去瞻仰了
一下,果然队伍比普王市邮局长了很多。

这个邮局倒只有一个出口,可是门口的走道有两个方向,因此我只好又选了一个站岗。这里的人平均说来,稍
微比普王市的人不礼貌一些,可能是地方一大,人就粗鲁的通病吧。比如,有位白人中年妇女在听说自由党的名头
后娥眉倒竖道:``没门!''还有一位男士,看见我走近就大声说:``别来烦我,我已经够烦的了!''---也不知道他
今年又被美国税务局榨出了多少油水。

当然,大部分人仍然是非常礼貌的。比较有意思的是一位亚裔老人(顺便说句政治不正确的话,今天所有的亚
裔,大概近10个人吧,只有一个像是美国出生的亚裔女孩签了,其他年纪大一点、明显带有故国气质的人统统不肯
签。倒不是因为她们可能没有公民身份,而是我一提出,他们就拒绝了),我向他解释完为什么自由党要选择今天来
进行活动时,他激动地说:``用服务费来取代税,那是行不通的!---难道富人和穷人要收一样的服务费吗?''我
说:``我们不是为富人谋利益的,我们只是认为,平均来说,我们付了太多的税,被政府滥用了。''他仍然很激
动,说:``这是不对的!一个百万富翁,和一个穷人,不应该交一样的服务费!''我只好放弃和他的争论,说:``您
说得对。不过我们不是要求您同意自由党,而只是想在11月投票时宾州可以多一个选择。''他说:``让你们上选票?休
想!我不签!''激动地走了。

还有一位白人老人,在我解释后,对我说:``你们这些人想干什么?你知道美国宪法是严格限制政党活动的
吗?''我说我知道,心想,若不是美国法律限制,我还搞这个参选签名干什么。他继续说:``你们想把美国弄得像欧
洲那些国家一样吗?弄上几十个小党,整天吵架你们才开心是不是?''我说不是,我们只是想出现在选票上,我们自
己也知道打不破两大党的天下的。他摇着头进了自己的车。过了一会儿,他又摇下了车窗,叫我过去。我还以为他
回心转意了,高兴地过去,他说:``你有共产党的签名表吗?你要有我就签。''我说:``很不幸我没有。我不知道原
来您是支持共产党的。''他说:``那你明年带共产党的表来,我一定签。---你们就是一定要弄出85个小党来互相
拆台,让整个政府瘫痪!''我这才明白他是在挤兑我。

好的情况当然也有。一位女士在知道了我是自由党人后,立即说:``这个名我得签。我的儿子和女儿都是登记
的自由党人。''还有一位男士,在签完名后,又问我:``如果你非要在民主党和共和党之间选一个,你选哪个?''我
回答说:``就我个人来说,我会选择民主党。但我知道也有很多的自由党人会选共和党,不过总的来说,我们对民
主党和共和党都不同意,这也是我为什么今天会在这里的原因。''他笑着说:``是的。我对这两个党也都不满
意。''我连忙说:``那么请您投票给自由党!''他说:``我会的!''

到了7点钟的时候,我一共征集到60个签名,算算也差不多了,天气渐冷,肚子渐饿,便回去了。在路上我自己
默算了一下,我大概总共花了3个小时来征集签名,平均每小时征集到20个,每一个花3分钟。这3分钟里,大概有1
分半是在等待,有半分钟是搭茬被拒,有1分钟是解释和签名的时间。效率还算不错。不过,如果一个签名要花3分
钟的话,那么3万个签名就需要9万分钟,也就是1500小时。就算整个宾州能号召出150个义工,每个义工也需平均花
上10个小时。任重而道远啊。

比较遗憾的是,有十几个人已经答应要签了,结果发现他们不住在蒙、切、特三郡,只好放弃。另外,我想,
如果自由党再有钱些,做个``反税''的标语牌,我再来邀请别人签名,一定就容易多了。

比较好玩的是,居然在邮局遇到一个熟人,是公司里的中国人。好在她上次到我家来参加派对,看到过我竖在
家里的自由党大牌子,因此还没有太惊讶,不过还是向她解释了一会儿我到底在干什么。可惜她有绿卡却是非公
民,因此没法帮我签。由此想来,其他义工的工作应该比我容易,因为光是他们周围,大概就可以征集到几十个签
名吧。

\section{蘸谈美国政治}

在感性地讲了一些关于我参加自由党的活动后,下面我要开始介绍我为克里助选的事情了。在此之前,我想先
提供一些背景知识,大体地谈谈我所理解的美国政治。

首先要强调的是,美国是个极其复杂的国家。她幅员辽阔,人口众多,虽然没有中国多,但种族构成要比中国
复杂得多:有所谓的社会主流WASP(White Anglo-Saxon Protestant,盎格鲁--撒克逊裔白人新教徒),有欧洲其他
各国来的移民,有祖上被贩来当奴隶的黑人,有从墨西哥和中南美洲持续涌入的西班牙裔,还有亚裔。这些来自世
界各地,文化传统非常不同的人们,在不同的历史时期由于不同的原因来到美国,对社会、国家、政治的看法自然
会千差百异,甚至针锋相对。我常想,有空得再去查查美国历史书,看看是不是除了南北战争之外,他们还漏记了
几次内战。

对于这么复杂的一个国家,任何企图在一篇文章里来谈这个国家的人,大概都应该被流放到阿拉斯加去。所以
我必须首先声明,本文挂一漏万,只是最最粗浅的一些个人感受,是比``浅谈''还要浅的``蘸谈''。其次,既然是
个人感受,必然会带上我的很多思想烙印,为了防止被我的观点误导,读者看完本文后,应该再去看几篇极权主义
者对美国的描述,可能才会有比较全面的结论。

美国号称``种族熔炉'',因此多元化可以说是美国最大的特点。任何一个读过王小波杂文的人,一定都会被他
开口便引的那句罗素名言``参差多姿,乃是幸福的本源''叨扰得不堪其烦。如此想来,当年他在匹兹堡度过的时
光,一定很是幸福。我初到美国时,有两件事印象很深,充分说明了什么是``多元化''。

一是没有``美国普通话''。刚到美国的中国人一般都会为口语烦恼,因为自己不说标准的美国英语,就像在中
国不会说普通话,在英国不会说BBC腔,感觉低人一等。但很快我就发现,压根就没有``标准美语''这一说,这个词
大概只存在于中国某些书的封面之上。就算是电视台的新闻播报员吧,比如纽约的电视台,那也是白人说所谓的
``纽约英语'',黑人说黑人英语,而如果是南方的嘉宾,他们就说带南方口音的英语,大家各说各的,井水不犯河
水。每个人都理直气壮地带着自己的口音。所以,我现在坦然地说我的中国英语,周旋于众多的美国英语、印度英
语、欧洲英语乃至英国英语之中。

另一件事,是我做助教时,第一次带习题课,心中很是忐忑,就去问一个美国朋友,习题课应该怎么带。她说
了一下该注意的问题,最后说:``最重要的是,做你自己(Be yourself)。''这个说法当时让我新鲜了很久。我本来
总觉得,习题课这种东西,系里应该有个惯例或者规定,一二三四五,就清清楚楚地把怎么上课都讲明白了。没想
到其实怎么带一门课,完全是带课老师自己的事。我至今很感谢她告诉我的这句``做你自己'',常引为自己跳出陈
规、我行我素的依据。

多元化的源头是对个体的尊重。随之而来的,必然就是``宽容'',因为如果人们不宽容不同意见,多元化就不
可能实现,必然就还是``万马齐喑''、思想定于一尊。在美国,纳粹党和三K党都是合法的,60年代时,纳粹党一度
被禁,美国人权组织认为他们的权利也应当受到保护,任命了一位律师替他们辩护。这个律师是犹太人。官司一直
打到最高法院,大法官们以5 : 4判纳粹党也有信仰和言论自由。同样,当年为三K党辩护的律师是个黑人,这大概
是``我不同意你的观点,但我誓死也要维护你说话的权利''的最好例子。在宗教上,美国极为自由,充满了各种千
奇百怪的教派,都充分享有各自的信仰自由。一夫多妻的摩门教、直呼上帝之名的耶和华见证人、信奉教主的大卫
教派,五花八门,应有尽有,如同农贸市场般热闹。他们只要不触犯法律,就可以自由地传教。大卫教派是由于私
藏火器,并且拒绝联邦人员调查,才被围攻的。那次围攻由于借用了军队资源,还被人权组织找了很长时间的麻烦。

然而,与在政治和宗教上的宽容相反,美国对个人生活却是不那么宽容的。比如,在美国的50个州中,49个州
禁止嫖妓,只有赌城拉斯维加斯所在的内华达州允许嫖妓。再比如最近弄得沸沸扬扬的``同性恋婚姻''事件,竟然
闹到布什扬言要修改宪法来禁止,真让人对美国的保守程度要刮目相看。这些在很多欧洲国家都不是问题,甚至同
在北美的加拿大人都对此不以为然。

这充分说明了美国社会的右派本色。从历史上看,构成美国主流社会的中西欧早期移民,来到美国的主要原因
有二:一是在旧大陆所受的宗教迫害;二是对新大陆的经济机会的向往。他们虽以新教徒为主,但教派也是五花八
门,况且他们对自己所受过的宗教迫害还心有余悸,因此在新大陆提倡信仰自由,以及由此带来的新闻自由、言论
自由。但同时,由于他们都是虔诚的教徒,所以在道德上要求甚严,对个人生活很是自律。欧洲宫廷里的那种浮华
的生活作风,和这些贫下中农是格格不入的。

在经济上,新大陆当时尚未开发充分,个人只要努力,获得成功的机会是很大的,这就是至今仍在流传的``美
国梦''。所以,他们反对国家对经济的干预,认为经济应当是自由竞争,由个人对财富的渴望来推动,而不是由国
家来规划。

这就构成了美国右派的主要立场:个人道德保守、经济自由。他们主张低税收、小政府,强调个人奋斗。由于
他们是社会的既得利益者,因此比较维护现有的社会秩序,本能的防范新兴思想。

本来,美国简直就是右派的天堂了,可惜除了这些主流社会,美国还有其他各种三教九流的人。比如,被贩卖
来当奴隶的黑人,他们在被解放后,政治上是自由了,经济上却``一穷二白''。``美国梦''仍然存在,可是他们这
些奴隶的后代怎么和奴隶主的后代竞争呢?他们祖先的劳动被别人无偿地占有,在竞争中天然处于不利的位置。白人
老爷们的后代必须为此做出补偿,虽然不能重新卖身给黑人为奴,但美国开始推行``肯定性法案''、``反歧视法''
等,要求学校、公司录取的人员和人口中的民族成分相合,使得少数民族在同等条件下其实可以被优先录取。同
时,政府又提高税收,为生活无着落者提供福利。

风起云涌的60年代,无疑是世界上所有右派的噩梦,对美国人来说,尤其如此。除了少数民族争取人权的斗争
以外,美国还上演了``女权运动''、``反战运动'',当然,还有``性解放''。这都和右派观念完全相反。在经济
上,罗斯福新政更是早从20世纪30年代起就开始了政府对经济的干预,大政府主义已经在美国占有一席之地。左派
势力在美国这个世界右派的大本营蓬勃发展起来了。它的主力是少数民族、穷人、年轻人、知识分子。左派的主要
立场就是:个人生活要自由,经济要国家干预,补偿弱势群体。

美国的两党政治就是左、右派对峙的反映。共和党代表右派,是所谓``主流社会''的代言人,他们认为:国家
提高税收来给穷人发福利,是一种对工作阶层的剥削,伤害了人们工作的积极性,最终会鼓励大家都不工作,坐在
家中等救济。国家过多地干预经济,会造成经济的僵化,使经济彻底失去活力。同时,他们的观念保守,对国家有
很高的认同感和荣誉感,把反战人士看作是国家的叛徒(美奸),对性解放深恶痛绝,对传统的家庭价值非常看重。
共和党的成员,大多是中产阶级和富人,他们代表着这个社会的既得利益者。共和党所代表的价值观,是美国的主
流,因此又被称作``保守派''。

民主党的观念和他们几乎完全相反(当然,在信仰自由等基本人权方面,全美国都是一致的)。他们深受从欧洲
大陆传来的左翼思潮的影响,认为国家应该干预经济,用计划来调节经济,以免经济盲目发展。同时,他们强调平
等,认为国家应当负起照顾弱势群体的责任,不能使贫富过度分化。在个人事务上,他们认为只要不伤害别人,个
人有权利处理自己的一切事务,不需要社会或者国家的指导---我不信上帝,或者同性恋,又干卿底事?这些观点在
传统的美国人看来简直是离经叛道,因此被称为``liberal'',即``思想开通者'',也称``自由派''或新自由主义者。

不过这个``自由派''和我所属的``自由党人(Libertarian,或古典自由主义者)''还是不一样的。自由党人处于
左右派之间,基本上是要左派的个人事务自由和右派的经济事务自由,反对左派的国家干预经济和右派的传统道德
束缚。由于自由党人在个人和经济方面同时要求自由,因此被左派和右派共同认为太激进,在美国基本上没有太大
的影响力。美国人还是觉得,要么个人自由,经济计划,要么经济自由,社会传统。一只脚是自己的,一只脚拄着
国家的拐杖,才好走路。两只脚都由着自己的性子走,只怕会走入歧途吧?因此有人半开玩笑地说:自由党是很好
的,可是美国人民还不够聪明。看来特殊国情之说,中外俱有。

在我看来,左右派并无谁对谁错之说。本来,我作为一个中国人,从小熟读``不患寡而患不均'',即使没有政
治课和语文课上所受的教诲,也觉得左派是比右派更天然正确的。不过,一方面在我成长的过程中,周围的环境也
正在经历着激烈的经济改革,以及由此引起的翻天覆地般的观念变化,另一方面自己的阅历日增,对生活的理解更
加深刻,看的书也比以前多了,后来我的思想开始脱离左派,往中间偏自由的方向走。

西方有句话说,一个人如果20岁以前不是左派,他就没有良心;如果30岁以后还是左派,他就没有头脑。我觉
得这话说出了问题的部分实质。年轻人性格反叛,要求个人自由,又富有理想和同情心,容易被``平等''、``权
利''等字眼简单地打动,是左派的骨干力量。等到他们长大以后,他们才会逐渐认识到传统道德的合理之处,以及
美丽字眼的空洞之处,就会慢慢地成为右派。人老了么,自然会保守些。

从国际上看,著名的右派国家是美国、英国,主流思想是实用主义,国民比较注重实利,不喜欢过激的思想;
著名的左派国家是法国,主流思想是理想主义,国民普遍有浓厚的浪漫情调,容易冲动。这大概也是对左派、右派
的另一个注脚。

在美国,右派的共和党可以说是社会主流共识,但左派的民主党则得到穷人和自由派人士的支持。2000年的总
统大选是美国历史上最接近的一次大选,因此也可以说是对美国两党势力划分最清晰的一次说明。从地图上可以清
楚地看出,共和党赢得了民风保守、思想传统的南部和中西部,民主党则掌握着号称自由主义大本营的东北部、以
芝加哥为中心的中部工业区,以及以加州为中心的西海岸(加州大学伯克利分校被认为是美国自由主义气氛最浓厚的
一所大学)。

美国的选举制度非常特别,并不是在普选中得到多数票的候选人就得胜,而是在各州分别计票,除了缅因和内
布拉斯加之外,每个州的胜者将得到这个州的所有的选举团票数(即该州在参众两院的总票数)。如果一个候选人在
某些州大败,在某些州小胜,那么他就可能总得票数输给对手,却仍然赢得大选。

我个人认为这种设计是不合理的。它的本意是为了防止小州的声音被大州压倒,但2000年时的情况说明,这样
的设计并没有达到目的。2000年的计票工作到最后时,布什和戈尔的票数基本相当,还有佛罗里达以及其他几个小
州的票还没有开出来。但那时大家对那几个小州的情况已根本不再关心,而把全部注意力都放到了有25个选举团票
的佛罗里达,因为那几个小州的所有选举团票都加起来也没有佛罗里达多。如果是采取普选多数票得胜的选举制
度,就不会出现这种小州无关紧要的局面。

4年过去了,美国的两党分裂没有被弥合,反而愈演愈烈。布什坚定的共和党立场为他赢得了共和党人的衷心爱
戴,也燃起了民主党人和其他自由派人士的满腔怒火。现在看来,两党的主要势力范围并没有太大的变化,南方和
中西部仍将支持布什,东北部和西海岸仍将支持克里。主战场仍将在佛罗里达,以及一些选举团票众多的边缘州,
比如俄亥俄州、宾州等。

在我这个自由党人看来,民主党和共和党各有我可支持的一面,也各有我所反对的一面。我认同民主党的个人
自由理念,但反对他们的大政府主张,在经济政策上我更倾向于共和党。两相比较起来,我的平等情结最终发挥了
作用,对共和党对弱势群体的相对冷漠比较难以接受,因此如果在两党之间一定要选一个的话,我会选民主党。

如今的形势又有了变化,白宫里的共和党人已不再是我们所熟悉的共和党人,而是所谓的``新保守主义者''。
他们一方面仍然坚持保守主义的宗教和社会价值观,比如反对同性恋婚姻,强调虔诚敬拜上帝,另一方面却背弃了
共和党的小政府理念,将政府的规模空前扩张,创造了美国历史上最高的赤字,还借反恐之名推出了《爱国者法
案》,使政府部门可以合法地侵犯公民隐私乃至人权。这一切都使得共和党本来对我的吸引之处都荡然无存。因
此,在今年的总统大选中,我将为克里的竞选奔走,为把布什赶出白宫尽一份力。虽然我并不完全认同克里的政
策,可民主本来就是要在两个魔鬼中选一个较不坏的,而不是寻找一个完美的明君。

\section{克里支持者聚会}

这是我第一次参加克里支持者的聚会,就在附近的一家酒吧。聚会原定晚上7点钟开始,我按照参加自由党活动
的经验,7点半的时候到了酒吧,结果发现那里已满是人了,分成几张桌子坐开,至少也有40人。这阵势让我吓了一
跳,不愧是大党,轻轻松松地就纠集起这么多人来。

我签了到,拿了张贴纸写了名字,贴到了胸前,便找了张桌子坐下。桌子上的人讨论得正起劲,我先向边上的
人打听了一下情况,她告诉,这张桌子上的人都是筹款组的,主要讨论筹款的事情,另外还有宣传组、媒体组、义
工组。我决定还是加入到义工组去。

义工组的桌子是最长的一张,坐了20多个人,看来大多数人还是和我一样,没啥特长,只有点傻力气可卖。义
工组的组织者是一个很年青的小伙子马修,他为我找到个座位,我挤了进去。那个地方正好是两组讨论的人的分界
处,大家都背对着我,朝着另一个方向说话。我只好先点了杯饮料,听听他们都在说什么。

坐在我左边的,是一个叫劳拉的30多岁的女士,眉清目秀。她是属于``媒体观察(media response)组''。这不
同于媒体组,媒体组是管把大家的活动在媒体上进行报道的,而这个组的主要工作是关注媒体动向,注意媒体是否
对克里有负面或失实报道,然后进行反击。劳拉问我是否对这个感兴趣。我说:``关注媒体动向,写文章进行澄
清,我都是很喜欢的。但英文读写实在让我头大,我还是去参加那个义工组算了。''她说,其实她也是一不小心坐
到这个地方,被拉进来的,她也是对做义工更感兴趣。

在劳拉左边,坐在桌子尽头的,是一个大学生模样的男孩。他自己办了个网站
(http://www.thetruthin2004.com )为克里呐喊助威。这次他还带来了一些复印出来的资料,叫``布什 VS 真
相'',用数据和事实来说明布什都干过些什么。后来我去看了一下他的网站,内容还挺多的。

坐在我对面的,是一位叫维尔的女士。我问她,是否也是第一次来这里活动。她说:``是,不过我以前参加过
几次迪恩支持者的活动。''维尔认为,迪恩才是真正代表民主党的人物,也是她全心支持的候选人,不过既然他已
经退出了初选,她就改为支持克里,反正迪恩也已经正式支持克里了。

这让我想起了另一个问题。我问大家:``你们支持克里,是因为他的主张,还是因为他反对布什?''她们几乎毫
不犹豫地一起说:``都有。''

这个回答有点出乎我的意料。长期以来,媒体给我的印象,就是支持克里的人当中,大多数人只不过是因为他
的当选可能性比较高,所有讨厌布什的人都不管与克里的主张有什么歧见,都来支持他。这就是所谓的
ABBA(Anyone But Bush Again,除了布什谁都行),我就是这样的人(我家里还有一张ABBA的CD呢,下回可以拿去播
放),所以才以一个自由党人之身,来支持民主党的候选人。

这时,斜对面的一位女士苏插话说:``这没什么,我还是登记的共和党人呢。''大家都吃了一惊。我说:``那
您一定是被布什真正地激怒了。''她笑着说:``那倒不是。我一向投票支持民主党。不过我登记为共和党人,这样
他们初选的时候,我就可以去投票反对那些我最讨厌的共和党人。比如这次,我就不投布什,去投其他人。''

大家这才舒了口气。这种情况我倒听说过,某些人登记为对方的党员,然后在初选时,故意投票给那些最不可
能当选的候选人,以帮助本党在正式的大选中取胜。反正在美国,登记为党员是政党求之不得的事情,而且又免费
(自由党这种小党才收费,不然无法维持)。我常怀疑布什在2000年的共和党总统候选人身份就是这么给内奸投出来
的。

苏以前是住在新泽西的,那里的初选是开放式的,非党员也可以去初选投票,所以这套把戏她在新泽西就已经
玩得轻车熟路了。搬到宾州后,这里只有党员才可以在本党的初选里投票,所以她只好特地去登记为共和党人,继
续进行地下斗争。她的丈夫还是登记为民主党人,这样他们两边的信息都不会错过。``另外,如果有什么事,我们
可以同时向两党的人呼吁。''当然,这样平时她就常会收到共和党的宣传资料,以及要求捐款的信件。她笑着
说:``我从来都把他们直接扔到垃圾箱。''

闲聊中,我发现维尔也是住在普王市。她又把我介绍给了琳达和戴安---两位坐在我右边的女士,她的邻居。
今晚从普王市来的就是我们四个,其中戴安是民主党的积极分子,曾经为其他候选人助选过,所以对义工活动很熟
悉。基本上,义工的职责就是去动员人们投票。据她介绍,大概只有一半的公民会去投票。如果能把大众动员起
来,就能获得压倒性的优势。

但问题是,共和党也在行动,这里本就是共和党占优势的地区,布什又非常重视宾州,上任以来已经访问过宾
州28次,在美国所有州里是最多的。``他简直就是住在宾州了。''有个人忿忿不平地说。这里的人对布什不用说都
是深恶痛绝。很多人连他的名字都不愿意提,直接说``他''。只要一开口,听到批评了,肯定就是冲着布什去的。

她们对我的自由党人和国际主义战士的身份也很感兴趣。维尔问我,为什么会成为自由党人。我说,就个人事
务而言,自由党和民主党的观点基本类似,就是反对国家干预个人生活,个人不论性别、种族、国籍、性取向,都
应当享有完全平等的权利。就经济事务而言,我认为国家对经济的干预往往适得其反。在这方面,她们显然不同意。
于是我祭出了一样法宝:我是从中国来的,成长于计划经济向市场经济转换的年代,我亲眼目睹了国家干预对经济
所造成的束缚和损坏,也亲眼目睹了个人在为自己的经济前途奋斗时所能迸发出来的巨大潜力。因此,我主张由市
场来主导经济,国家的干预越少越好。至于社会福利、医疗保险等方面,大概是由于我来自中国吧,我怎么看都怎
么觉得美国已经做得非常好了(哪位民主党人不服,请赞助我一张去瑞典的机票,也许我去参观学习后就又皈依左派
了),尤其在媒体上,更几乎是三分天下有其二,所以不觉得平等是个紧迫的话题。

8点45分左右,活动的主持人凯丽站起来说:``时间已经不早了,我想请各组的协调人来总结一下今晚的讨论结
果,然后大家可以继续在这里讨论,也可以回家。''

第一个发言的是宣传组。英文名叫propaganda team,我对这个名字有些吃惊,因为我记得以前在网上看到
过,propaganda在英文里是个不好的词,主要是用来指别有用心的、经常扭曲事实的宣传,常用于纳粹德国、前苏
联等。这个组的协调人站起来的第一句话也是这个调调:``印象即真相(perception is reality)。''顿时让我对他
们的印象(perception)很差。还好,他们的主张没有那么不择手段,主要的一条,是希望给本地的民众造成一个印
象:民主党人仍然在活动,克里可以赢得选举!因为这里是共和党人占优势的地区,为了防止一些选民习惯性地投票
给共和党人,他们要求大家广泛活动,多在公众场面出现,多在媒体发出自己的声音,多佩戴使用克里的宣传徽章、
汽车贴纸之类,使人们相信,民主党将在这一带压倒共和党。另外,他们鼓励大家多读书,并列举了一些连我都没
听说过的大书,希望大家读后互相交流,用理论来武装自己。

随后发言的就是我们义工组了。那个看上去非常年青的小伙子马修站起来说:``今年我只有16岁,但我从小就
是个忠心的民主党人。我还不能投票,但我可以做义工。4年前,布什从戈尔手里偷走了选举,今年,是该他偿还的
时候了!我们必须为此努力工作,劝说你的朋友、你的邻居出来登记,出来投票。我们可以给周围的人发Email,每
个人发给30个人,这30个人中,每人再发给30个人,4轮下来,就是8万人!不要小看你自己的力量,我们每个人都可
以做出贡献!''

他不像是在总结工作,倒像是在演讲,不但声音铿锵激昂,而且手势坚决,在说到``你''的时候还伸出右手食
指来用力地点向听众。现在正好是影片《阿拉莫》的档期,这是一部爱国影片,讲的是19世纪美国人和墨西哥人在
阿拉莫的一场血战,相当于中国的台儿庄战役吧。马修继续说道:``现在,人们都在说:要记住阿拉莫。那么,我
要说,要记住佛罗里达。4年前,布什让我们记住了佛罗里达。今年,我们不会让佛罗里达的事情再重演!''

大家纷纷鼓掌,互相说,16岁的小孩,有如此表现,真是令人惊异。有人喊:``你应该去选总统!''马修兴奋地
说:``yeah!也许在2024年,我将出来竞选美国总统!''

不过我不喜欢他的表现。也许和我的中国背景有关吧,我不欣赏过激的言论和举止。这些我已经见得多了。政
治并不崇高,只不过是老百姓的衣食住行的最佳方案而已。安上个崇高的目的,做慷慨激昂的演讲,号召民众为了
某个伟大目标而奋斗,这都让我敬而远之。这大概也是我成为自由党人的最主要的心理原因吧。我认为,政治应当
是平和的、妥协的、切实的。

劳拉也代表媒体应对组发了言,不过她们这个组很弱,只有``两个半到三个半''人,其中的``半个人''就是她
自己,因为她其实对做义工更感兴趣。她更多地是在介绍本组的工作,希望能从其他组挖上几个人过来。可惜却没
有任何反应。凯丽说,那看来我们这个组必须和其他组合并。

接下来是媒体组和筹款组的发言,很快就变成了这两个组的聊天,我没有听到什么要点。

总结完毕后,我和维尔、琳达、戴安互相留了电话和Email,简单地谈了些以后的计划,便各自回家了。

总的来说,参加这个活动的感觉和在自由党内是很不相同的。首先,他们的人可真多,让我这个习惯于小党活
动的人大开眼界,这下算见识到大党的场面了。其次,他们的人口构成和自由党完全不同,大部分是妇女,占了三
分之二以上,其中又有一半以上是老年退休妇女或家庭主妇,还在工作的好像不多。男性中也是小的小(除了马修
外,还有个18岁的小孩),老的老,像我这个年纪的不多。自由党每次开会,都是清一色的青壮年工作男性,而且大
多从事工程类或其他需要受过高等教育的工作。两党唯一相似的是,都是满眼的白人,一个黑人、西班牙裔都没
有,我在两处都是唯一的非白人,可能是因为这个地方的居民本来就是以白人为主吧。

因此,自由党开会,有一种小范围交流的感觉,所有的人我都认识,感觉比较放松,而且大都逻辑清晰、筹事
周密。这次民主党的聚会我觉得就有些闹哄哄的,可能是人太多了,而且很多人说不到要点,让我听了半天不知所
云。希望以后大家变熟了,能够好些。

\section{11月14日附记}

``你为什么会成为自由党人?''这个问题,不光是你想知道,在我参加克里竞选活动的过程中,也不断地有人问
我。我想很多读者大概也有此疑问,在这里简单地说一下我成为一个自由党人的过程,也就是如何自我思想解放的
过程。

自由党的英文是Libertarian,词根是Liberty(自由)。我当初看到这个词时,第一反应却是``解放党'',因为
我在初中最先学到的单词之一就是``解放(Liberation)''。显然,``解放''和``自由''密切相关,尤其在思想上,
一个人只有从各种束缚中解放出来后,才能达到自由。美国人有句半开玩笑的话:大学里学到的错误观念,要在毕
业后花五年才能完全消除。从出生开始,我们就被铺天盖地的主流舆论和传统思想重重包围。小时候,我相信书本
和教师说的一切。我还记得在上初中时,一次广播里放国际新闻,我听了之后为全世界社会主义运动陷入低潮而忧
心忡忡。

到了高三,我班上有个同学要出国了,拿了个同学录请大家留言,其中有一项是偶像。我前面的人填了里杰卡
尔德,我想了半天,不知道自己的偶像是谁,最后大笔一挥,填了``毛泽东''。高中毕业的那个暑假,我去一个亲
戚家做客,在他家的书橱上看到一本《马克思主义哲学》,拿起来一翻,居然就津津有味地看完了。那时我已经比
初中有进步,知道其社会科学部分的不足之处,不过自然哲学部分却仍然令我大为折服。大学一年级时,我在日记
里写下:我这辈子的奋斗目标,依次是祖国、真理、我。

到大学毕业时,祖国已经被排在真理之后了。这个转变很容易,一方面,我见闻日多,自然就看出以往所受的
爱国主义教育有愚民之嫌。我依然爱国,但不是出于课本上的那些可笑的爱国理由,而是出于对个人价值的确信。
很显然,一个人如果爱自己,也就会爱国,如此自然的情感,竟然也会被拿来大做文章,不能不让我倒胃。

另一方面,如果一个人的爱国主义不是建立狭隘的民族诉求上,而是建立在个人价值的基础上,那么一个顺理
成章的结论就是,全世界人类福祗相连,各国的国家利益,归根到底是一致的。凡欲通过对外扩张、打击别国来为
本国谋利益的,最终无不反受其害。夺取生存空间的德国、输出革命的苏联、在中东翻云覆雨的美国,皆为明证。
二战时谋刺希特勒的德国军官,比沙场捐躯的党卫军更懂得爱国的真义。祖国这个奋斗目标可以被真理所容纳---
这个真理,已不仅指世界的客观规律,而是包括社会公义的泛称。

真理观的动摇,发生在出国后。美国有很多华人基督徒,我也被拉去参加过很多次他们的聚会。甫一接触耶
稣,我便大为倾倒:不意世上竟有此等人!立刻成了个``文化基督徒'',即认同基督教的思想,却不接受其中关于神
的部分。在我看来,无神论能自圆其说,凡是现在还不能解释的,可以推给未来科学的发展。基督教也能自圆其
说,凡是我们不能理解的,可以推给上帝的高深莫测。成为一个基督徒或无神论者都需要相当大的信心。是否有神
的问题显然超过了我的智识和胆量所能达到的范围,我从一个无神论者变成了疑神论者。

我的真理观当然也跟着变化。宗教信仰是人类道德准则的基础,如果我连是否有神都无法确认,那么一切真理
也都在可怀疑之列。疑神论的立场让我不再仰视那些从前认为是不容置疑的道理,而可以从容地推测,这条道理是
如何从对神的崇拜推出来的,如果没有神的话,它是否还能成立。

历史上道德观的演变,已经非常清楚地说明,道德不是上帝颁发的诫命,也不是宋儒所称的``天理'',而不过
是人们在组成社会时所约定俗成的规则。由于它们确实有助于维系社会稳定,人们给它们编出了崇高的理由,披上
了光芒万丈的外衣,但其实它们绝非天经地义,而且在很多情况下,对个人的弊大于利。如果仔细追究的话,道德
观念其实自相矛盾、无法解释之处甚多,因为它们不是从``神''或者某个绝对理念那里严谨地推导出来的,而是人
为总结出来的,还常被强权者任意打扮,自然会漏洞百出。比如中国传统的忠孝难两全,如果不是被道德唬住了的
话,根本就不是问题。

由此,我不相信道德审判,甚至反对一切道德评价,主张只要不违反法律的,就可以做。当然,前提是你自己
的满足感。由于环境影响,很多人的满足感其实有一大部分是来自自我的道德评价,那么,你为了使自己更快乐而
遵守道德,乃至于牺牲掉一些较次要的快乐,也完全符合人的本性。

参加自由党后,自由党人在毒品问题上的立场,让我开始思考人是否有吸毒的自由。这个问题很方便地就转为
另一个更重大的问题:人是否有自杀权?

从理论上讲,人当然有自杀权。这由信仰自由可以直接推导出来。有些印度教教徒认为朝拜扎格纳特神可以免
除轮回之劫,便投身于放置神像的巨车之下,以一死求彼岸的永生。一个人怎样才快乐,只有她自己知道,也只有
她自己才有权决定,别人无权干涉,任何宗教或政府都无权把某一种人生观强加给其他人。因此,当一个人决定活
着比死去更痛苦时,我们有什么权利强留她在这个世界受苦?

这个道理极为简单明了,然而在现实中却完全行不通。基督教相信人不可以自杀,除了上帝外,谁也没有剥夺
别人或自己生命的权力。由于基督教在当今世界所处的强势地位,他们提出的人道主义也被全世界所普遍接受,甚
至连安乐死都可以拿这个冠冕堂皇的理由来反对。

对道义的稍加推导,就会陷入尴尬的境地,因为道义不是从神那里降下来的旨意,而只是层漂亮的外衣,其实
质是实力对比后总结出来的约定。不同的环境会产生不同的道义,说得都很义正词严,其实都是根据实际需要编出
来的。或者,可以让我们稍减羞愧的说法是,各种道义都有人在真诚地传教,但只有那些最符合实际需要的才会占
据主流地位。宗教自由并非来自于人们的互相尊重,而来自于各教派的势均力敌,正好这时大家手边有个``天赋人
权''的说法,就拿出来作为遮羞布了,因为``君子喻于义'',我们的让步可不是因为吃不掉对方,而是因为尊重对
方。

想通了这一点后,所有的道德规范、真理公义都不能再让我心存敬畏。仿佛《黑客帝国(the Matrix)》里的尼
奥(Neo),在别人看来确实存在的世界,在他眼中不过是虚构出来的矩阵;制约着别人行为的客观规律,他却可以随
意打破。我们每个人都可以成为尼奥。只要看穿那里其实空无一物,所有束缚只是横竖排列的虚线,我们就可以把
自己从表面的假相中解放出来,如同尼奥那样自由地飞翔在精神世界。

然而,在精神自由后也会有精神困境:尼奥无法只生存在虚拟世界,无论精神多自由的人,也必须生活在现实
世界。这其中的冲突是显而易见的。更糟糕的是,完全的自由带来的是混乱和失落,如同飞翔在天上的风筝,一旦
挣脱了系在大地上的绳子,它会暂时飞得更高,但最终却必将落下。破除了旧有的迷信后,我陷入了新的茫然,放
眼望去,皆是虚空,无觅岸处。

子曰:``思而不学则殆。''我认识到,独想狂奔可以解放自己的思想,但马上得天下,却不能马上治天下,下
面应该去看些书,从老庄到释耶,从波普、哈耶克到兰德、弗里德曼。新圣人胡适则说:``多研究些问题,少谈些
主义'',也是至理。参加自由党和克里阵营的草根政治,就让我对很多问题有了更深的看法。我希望以后我可以有
机会做更多的实事,在精神解放之后,再建立起一个精神新世界。

\section{自由党候选人见面会}

今天本来是每月一次的自由论坛,但由于大家对自由论坛的热情都不高,所以本月就暂停了。不过《费城问询
报》大楼的房间已经预约了,不能浪费,吉姆就决定今天在这里开个自由党候选人的见面会。

到会的共有5个候选人:蒙郡自由党的吉姆竞选州议员,恰克竞选联邦众议员;还有其他郡来的杰$\cdot$罗
素(Jay Russel)竞选参议员;罗斯$\cdot$戴蒙(Russ Diamond),就是上次筹款晚会时坐在我左边的长发男,以及
第一次见到的马特,都是竞选众议员。

此外,还有两个跑龙套的,一个是罗斯的朋友,还有一个就是我了。总共7个人而已。

人都到了后,吉姆问:``这里谁是最有经验的候选人?''

杰说:``应该是我吧。我已经连续10年参加选举了。''

我不由得在心里不敬地想:那看来您已经连续九年失败了。

他的经历挺复杂的,开始是自由党人,后来作为共和党的候选人参加过选举,最后又变成了宪法党
(constitutional party)的候选人,今年又回到了自由党。这10年倒也没白过,螺旋式前进吧。

他的要点,是大家要行动起来,把参选签名的工作做好,因为如果宾州自由党不能成功筹集到3万个签名,那总
统和参议员候选人都无法上选票,一切都是白搭。

对其他4位众议员和州议员候选人来说,日子稍微好过些,所需的签名数和他们所在的选区有关,大约只需要一
千多个签名就够了。但问题在于,这些签名都必须在他们的那个选区获得,因此工作量也不小。

接下来,除了马特没准备外,罗斯、恰克和吉姆开始谈他们的竞选纲领。罗斯先说,主要是:

1. 小政府、减税、消除政府赤字、要求州政府拒绝来自联邦的经费(相应地自然也就减低了自己对联邦的义务);

2. 反对爱国者法案、要求保护个人隐私、持枪合法、要求废除``无受害者犯罪''(即没有伤害到别人的犯罪,
比如吸毒);

3. 健康保险私有化、市场化(即反对政府的干预)、教育私有化、福利政策改革;

4. 自由市场、自由贸易、反对政府干预经济。

基本都是自由党的老生常谈了。但他的一个观点引起了争议:废除死刑。他说:``政府的权力是人民给的,人
民的权力是上帝给的。上帝没有赋予人剥夺他人生命的权力,那么人自然也就无法赋予政府剥夺他人生命的权力。
死刑是政府滥用了人民所赋予他们的权力。''

我不由得脱口而出:``这个说法简单而富有美感!''

罗斯得意地说:``就是这么简单!''

可是仔细一推敲,我又有问题了:``可是,上帝也没有赋予人剥夺他们自由的权力,为什么政府就可以判处某
些人坐牢呢?''

罗斯连忙解释说:``那是因为他们犯了错误,所以必须要受到惩罚。''

这下大家都看出问题了,不等我发问,马特就发难说:``这些人犯了错误,所以要受到剥夺行动自由的惩罚,
哪怕上帝没给我们这权力;那某些人犯了严重的罪刑,为什么不能受到剥夺生命的惩罚呢?''

于是众人陷入混战。我在死刑问题上是比较糊涂的,一方面觉得死刑有保留的必要,另一方面又觉得没有什么
罪是需要用剥夺生命来惩罚的。本来这样也不错,像蝙蝠一样,看到鸟就说自己是鸟,看到兽就说自己是兽。可我
中了怀疑主义的毒太深,反倒是见到鸟就说兽的道理,见到兽就说鸟的道理。可能,我反对的是斩钉截铁的道理本
身吧。反正我也一向越是大是大非问题越糊涂(小问题如商店少找了我两毛钱我一向是很清楚的),到最后只好认为
大家都有道理,还是不要争个高下,求同存异吧。

扯远了。这番混战花了不少时间。最后杰说:``候选人采取什么立场确实是个重要的问题,可如果我们的候选
人都不能出现在选票上,那还有什么意义?我们还是继续原来的议题吧。''

下面是恰克介绍他的竞选立场。他拿出一本厚厚的书来,题目是《克图在国会的立场》,是他从克图学院
(Cato Institute)的网站上下载了,打印出来自己装订成书的。克图学院是美国的自由党人思想库,恰克把这本书
的链接给了我们:http://www.cato.org/pubs/handbook/handbook108.html

最后是吉姆。他虽然只是``纸面候选人'',即只是为了帮助自由党进行参选签名而报名参加竞选,并非真的想
选上州议员,但他准备的材料倒最漂亮,是彩色打印出来的一张宣传材料。我粗粗翻译了一下:

少一些政府,多一些自由和个人责任

我为什么要来竞选?---我对哈里斯堡(州议会所在地)不断增长的腐败和失去控制的花销十分关注。现在,我们
的政客们把宪法看作是一本古经,在讨论立法时从不去考虑它。两大党都有人在投虚假的``幽灵票''。我再也不能
坐视很多宾州人越来越穷,政府的花销、贷款、税收却越来越高。作为一个自由党人,我愿意领导州议院去使宾州
恢复成我们宪法上所明言的小政府。

什么是自由党?---作为一个自由党人,我认为每个人都有权利选择自己的生活方式。我不主张用强迫手段来达
到政治或社会目的。成立于1971年的自由党是蒙郡的第三大党,也是宾州和美国的第三大党。自由党相信美国的自
由传统:个人自由和个人责任、自由市场经济、不干涉主义的外交政策、和平、自由贸易。


我的承诺---一旦当选,我将仔细阅读我将投票决定的每一个议案。当我投票批准新的法律时,我将当众指出
此法律符合宪法的哪一段。所有不能被宪法明确解释的法律,我将投票反对。我将带领州议院取消那数以千计的损
害我们的自由和繁荣的违宪法律。我要求我的共和党和民主党的对手也做出类似的承诺。

一些议题

政府的正确功能---政府的功能是保护生命和财产。

工作和经济---``我们认为,一个国家想通过税收来促进繁荣,就是像一个人站在桶里却想把桶拎起
来。''---温斯顿$\cdot$邱吉尔

政府能创造工作机会,这是个普遍流传的神话。当新工作机会出现时,政客们会很乐意地将此归功于自己,但
其实这些工作机会是由企业主和他们成功运转的生意所创造出来的。政府所能做的最好的事情就是什么都不做。一
旦我们削减了税收、企业补助、政府专卖,宾夕法尼亚就会像磁石一样吸引新的生意。我们需要让宾州的企业主们
来创造繁荣的机会。

环境---对于污染,应该负责的是污染者,而不是纳税人。这样我们才能吸引更多的无污染经济来到宾州,促
进繁荣,增加就业,同时也保护了环境。保护我们的环境应当是宾州人的一项头等大事。我将领导议会要求关闭政
府的垃圾输入和解毒工业。现在我们的政府通过广泛的``有偿污染''来赚钱,然后又让纳税人来付清洁宾州的账单。
通过保护我们自己的私有财产权,我们可以让污染者为污染负责,而不是纳税人。这样还可以吸引清洁工业到我们
州来。

立法机构的``鬼票''和假票---信不信由你,我们亲手选出来的议员们会公开地将笔套塞进投票仪,这样他们
就可以去钓鱼,投票仪则会投一整天的赞成票。有人公开承认曾代替缺席的议员投假票。这足以说明他们对我们所
付的税是什么态度。

教育---没人能对宾州的政府教育现状满意。只有把教育的责任和权力交还给父母,我们才能扭转当前危急的
教育程度下降趋势。

持枪权---宾州宪法第21条:``公民持有武器以保护他们自己和宾州的权利,毋庸置疑。''---无需再多说。

\section{第二次克里支持者聚会}

克里支持者的聚会是每月第4个星期四。在5月底,要参加第二次聚会的时候,我们自由党的参选签名活动也正
在紧锣密鼓的开展中。如果筹集不到足够的签名,自由党的候选人就无法出现在选票上,只能坐看民主党和共和党
两党相争了。

最近,我看到一些文章,分析第三党对今年的总统大选所可能带来的冲击。我们知道,4年前,绿党候选人纳德
尔吸引了大量的左翼民主党人的选票,是戈尔输掉了选举的主要原因之一。如果纳德尔没有出现在某些州的选票
上,那么这些选民应该就会投票给戈尔,而不是布什。事实上,纳德尔在佛罗里达州所得到的选票,远远超过了布
什和戈尔的得票差。如果不是纳德尔,上次选举都不用闹到最高法院去,戈尔就能轻松拿下佛罗里达,进而赢得整
个大选。

从那以后,民主党人对纳德尔就微辞不断,不料他今年越战越勇,又出来参加总统选举了。面对这位选举毒
药,民主党人虽然不满,却也无法阻止他行使自己的合法权利。共和党则暗自乐在心头,听说还偷偷地给纳德尔的
竞选捐款。

不过,如今的竞选斗争又出了新动向,分析家们发现,不光是纳德尔,其他第三党也能对总统选举产生重大影
响。这是由于今年的选举实在太势均力敌了,以至于往年弱小的砝码,在今年也能使整个天平倾斜。自由党作为一
个经济保守、个人自由的政党,对共和党中的温和派一向很有吸引力,尤其在布什这个共和党的右翼当权了4年后,
很多温和的共和党人无法认同他的很多极端政策,因而打算今年不再投票给他了。

由此我想出了一个``一箭双雕''之计,即鼓动民主党人来帮助自由党的参选签名活动,因为只要自由党的候选
人能够出现在选票上,对布什肯定是个坏消息。我起草了一封给克里支持者的信:

亲爱的克里支持者们:

我们都知道纳德尔是如何地影响了2000年的总统选举。这次他又得到了改革党的支持,正在努力试图使他的名
字出现在尽可能多的州的选票上。不过我们不用担心,因为乔治$\cdot$W$\cdot$布什正面临着一个更大的威
胁:自由党的候选人。

那些一向支持共和党的经济保守派们,在今天不再愿意把票投给他们。布什总统搞出了创纪录的财政赤字,使
联邦政府越来越大。许多经济保守派们正在转向自由党的候选人,因为他们坚信小政府和保守的经济理念。

汉弗莱(Humphrey)学院在今年2月的民意调查支持这一说法。劳伦斯$\cdot$R$\cdot$雅各布斯(Lawrence
R.~Jacobs)在4月20日的《基督教科学箴言报》发表文章《第三党威胁:不仅仅是纳德尔》指出:``民意调查显示,
在布什和克里一对一的选举中,布什可以赢得共和党的87\%的选票。然而如果是布什、克里和一个保守的第三党候
选人,共和党对布什的支持则掉到75\%。''雅各布斯得出结论说:``自由党对共和党的冲击,将比纳德尔对民主党
的还要大。''

可是,这一切要发生,首先需要自由党把他们的候选人放到选票上去。法律要求他们筹集到2.5万个签名。这正
是我们可以``助人以助己''的地方。想像一下,当布什看见自由党候选人出现在宾州的选票上时的心情!

共和党人正在给纳德尔的竞选活动捐款。他们希望他可以照4年前的葫芦依样再画一次。现在该是我们反击的时
候了。请您至少在附上的签名表上签个名,再让您的朋友和家人也签个名,然后或许填满第一页。这将帮助一个能
蚀食布什的共和党支持率的候选人出现在选票上。这对于约翰$\cdot$克里在这个``战场州''击败布什是至关重要
的。

请注意,在签名表上,地址应当是你的居住地址,而非通信地址。在一张签名表上,只有同一个郡的人可以签
名。如果您需要更多的签名表,请和我联系。我的联系方法可以在信末找到。

请把填好的签名表寄到``某某处''。所有的签名表都应当在7月6日前寄出(7月4日独立日是个绝好的筹集签名的
机会,对吧?)。

谢谢!团结起来,我们可以一起击败布什!

我把这封信打印了50份,又从自由党主席吉姆那里要来了50张签名表,再去买了一大堆信封和邮票。去参加聚
会的前一天晚上,我足足忙了两个小时,把各样东西在信封里分装好,连邮票都贴好了,还辛辛苦苦地把每封信的
回信地址也写了,这样,民主党人把签名筹集到后,直接丢进邮筒就可以了。

到了聚会后,大家先说竞选的各种正事,待到快要结束时,我才站了出来,把我的主意说了一下,大意就是信
里说的内容。结果,正是登高一呼,应者廖廖,很多人一边听一边笑着摇头。最后只推销出两份。其中一个还
说:``我们不需要这个,我们一对一就能击败布什。''大概只是见我没有其他人反应,才友情赞助了一份。

\section{8月3日附记}

我一直等到7月底,也没见这两封信回来,想来他们也没有当真吧。我想不通这些民主党人,虽然自由党的观点
和你们很不同,可敌人的敌人就是朋友,共和党很多人都在暗中支持纳德尔,怎么这些民主党人就不明白这个``远
交近攻''的道理呢?难道他们真的自大到以为可以靠自己就肯定可以击败布什吗?

两个月后,在又一次克里支持者聚会上,我却意外地发现,一个有拉丁气质的帅哥站了出来,说:``我不是民
主党人,但我这里有个可以帮助民主党候选人的方法。我正在帮助一个保守的政党进行参选签名,如果他们的候选
人能够上选票,将会吸引大量本来是共和党的选票。''接着他又解释了一下细节,还没有说完,人群中已经有人喊
了:``好吧,那还等什么,快告诉我们在哪里签!''

大家都笑了。他也就结束了讲话,找了一张桌子坐下来,拿出表来让大家签名,一半以上的人都来签了名。我
看了又是佩服又是羡慕,还以为他也是自由党的,上去和他搭话。他告诉我,他是来为宪法党筹集签名的,其实他
本人并不是宪法党的,只是来帮朋友的忙。

我这才明白,不是民主党人真的不懂``远交近攻''的道理,而是我自己上次没有说好。当然,来参加聚会的大
部分都是女士,这位年轻人的帅哥形象肯定也帮助了他不少。另外,我发现我的问题是一方面太贪,试图说服克里
支持者们帮我们去做参选签名,而不像这位帅哥那样,只是当场拿出表来,要你签一下就完了,举手之劳,大家都
会乐意去做,我那样只是把事情搞得过于复杂了。另一方面呢,我总疑心自己搞的这个伎俩不太高明,怕被别人看
扁,因此在介绍时不是很自信,反观他就大大方方地说了出来,这肯定也更容易赢得别人的好感。

好在最后我们自由党人依靠自己的努力,还是成功地做完了参选签名,并没有误什么事。我倒是从中获得了些
经验,也不错。

2004年7月---8月

\section{自由主义VS自由主义}

今晚照例开自由党的月务会议。和以前大同小异。重点仍然是在参选签名上。因为政治活动里最重要的,就是
参加选举。如果11月的选票上没有自由党的候选人,那一切都是白费劲。

4月份开会时,吉姆曾经建议说,我们拨出一些钱来,付给那些收费签名公司,让他们为我们来收集签名。我反
对这种做法,因为那些公司至少要收两美元一个签名,我们蒙郡自由党总共才1000出头的经费,也不可能拆房子卖
地全花在这上面,顶多花个五六百块钱,那么只能买到300个签名,对于整个宾州所需要的3万个签名来说,只是杯
水车薪。我们把自己搞得灯枯油尽,吐血而亡,于大事却无补,我不觉得这是聪明的做法。吉姆说:``参选签名是
我们最重要的事情。如果最后我们失败了,而我们其实却没有尽到全力,我会感到羞愧的。''我说:``我同意我们
要尽力去做这件事。可我仍然认为,钱必须要花在最有效的地方。''吉姆就没有再坚持。

5月的会我没去。在那次会上,吉姆又再次提出了这个建议,并且主动提出他可以捐出400美元,希望蒙郡的党
部可以也分配400美元。经过投票,这个方案被通过了。

今天晚上,我提交了会计报告。最近的收入不错,我们收到了州自由党发下来的会员费,有500多美元。现在我
们的总经费已有近2000美元,我不知道别人的感觉如何,反正我觉得有些飘飘然,好像已经暴富或者至少进入小康
阶段了。这其中有一部分经费,是去年年底的筹款晚会上筹集来的,当时就已经向捐款者保证了,将只用于征集参
选签名。签名表的印刷花去了300多美元,现在还剩200多。再加上吉姆和蒙郡党部各出400美元,我们已经有1000多
美元可以拿来用于征集参选签名了。

大概是由于现在已经是7月初,离参选签名的截至日期8月2日只有一个月了,大家的紧迫感越发强了。同时吉姆
又说,他也许可以找到1美元一个签名的公司,或者1.5美元一个。就算不能帮宾州自由党把参选签名弄好,至少可
以保证当地的候选人能上选票。因为正如美国人常说的,``所有的政治都是当地政治。''本地的候选人其实比总统、
参议员这些政客更重要,他们才是真正和我们的生活息息相关的。本地的自由党候选人只需要2000多个签名就可以
大功告成了,把钱花在这上面还是有可观的成果的。于是我又提出,既然我们新近有了收入,还可以再投入300美元。
我们授权吉姆去和签名公司联系,并且可以视情况而定,决定是否把那300美元投进去。

于是,本次会议开成了团结的大会,胜利的大会。会后,大家照例去泡吧。我们的话题从最近引起很大争议的
电影《华氏``9.11''》开始。我告诉大家,我从克里支持者的邮件组上看到,切斯特郡民主党党委会的一位女士,
在电影院散发投票登记表时,被警察逮捕了。吉姆有些惊讶地问:``她在哪里散发的?''我说:``电影院的停车场。''

``那是电影院的私有财产啰!''

我回答说:``那封Email说,她认为她是在公共场所发登记表,因此当警察要求她离开时,她拒绝了。最后警察
只好将她逮捕。''

吉姆呵呵地笑了起来:``真的是个民主党人!---他们总是把别人的私有财产当成大家的公共财产!''

显然,自由党人对民主党人也不报好感。只不过因为现在是共和党在台上,所以大部分攻击都是向着共和党而
去的。但想来以前克林顿当总统的时候,他们也没少说民主党的坏话。我的立场是有些介于古典自由主义
(libertarianism)和新自由主义(liberalism)之间的。主要可能是由于生为中国人,中国文化里的平均思想深入骨
髓吧,再加上小时候受的教育,潜移默化之下,总觉得``不患寡而患不均''是个道理。今晚来泡吧的只有我们4个
人,恰克坐了一会儿就走了,杰夫话又不多,于是我和吉姆开始了主义之争。重点自然就是他作为死硬的自由党
人,和我的自由派苗头之间的争论。首当其冲的是关于``平等''。

(以下I为我,J为吉姆)

I:我觉得福利制度还是有必要存在的。我们必须照顾那些弱势群体。

J:我没有说弱势群体不需要照顾。我只是觉得我们不需要通过政府来做这件事。政府是非常没有效率的。你知
道吗?为社会福利而收的那些税,只有十分之一是真正地送到穷人的手里了,绝大部分都在政府部门里消耗掉了。

I:这我相信。你知道我是坚决反对大政府,最不信任政府的。可是,如果政府不来做这件事,那些穷人怎么办
呢?

J:如果让我来决定,我将这样做:政府不再收福利税,让我把这些钱拿去投资,扩大我的生意。这样,经济会
更繁荣,我就可以雇更多的人,这才是对穷人的真正的帮助,让他们可以有自食其力的机会,而非坐等救济。

I:好吧,就算你说得有理,可那些生有残疾、或者没有工作能力的人怎么办呢?

J:我们可以让私人的慈善机构来帮助他们。

I:凭慈善机构是不可能做好福利的,因为人都是自私的,大家都更愿意把钱留给自己,谁愿意捐那么多钱给慈
善机构呢?必须要通过政府的强制手段,也就是收税,来保证穷人能得到救济。

J:那你不知道美国的慈善机构是多么的强大。在美国,每年都有几十亿美元捐给了慈善机构。这还是在大家交
了税之后。如果我们不用交那些福利税,那该又多出多少捐款出来?关键在于:交税,你是被强迫的,不管你是不是
真心愿意帮助别人,你都必须交,不然就得坐牢。这对我们是羞辱性的。而捐款是自愿的,你捐款的时候会感到心
情愉快,因为你知道有人将会被你帮助到了,这和你被迫交税的感觉是完全不同的。我们知道,人皆有恻隐之心,
我们有帮助不如自己的人的生物本能。在没有更重的税务负担的情况下,完全自愿性的给慈善机构的捐款的作用将
会胜过政府的福利制度,而我们的感觉也会很不一样。

I:可是,你不能光考虑我们的感受,也不能光看效率。你必须还为那些穷人想一想。政府的福利制度是靠得住
的,而慈善机构,谁知道他们明天会不会倒闭?谁知道他们明年能收到多少捐款?人不是光靠面包活着的,还有精神
状态。在政府的福利制度下,穷人知道明年他们仍然可以领到救济,而慈善机构不能保证这一点。我们不能光填饱
他们的肚子就完了,还应当保证他们不生活在担惊受怕中。

J:这你不用担心。红十字会存在了多少年了?他们做了多少事?他们做的事情比政府还多!每当什么灾难发生
时,比如火灾、龙卷风什么的,最先站出来帮助灾民的,总是红十字会,而人们也总是乐于帮助他们。对于这些慈
善机构,我们不用担心他们会消失,也不用担心他们会收不到捐款。帮助别人是人的本性。

I:那你对那些高福利国家,比如瑞典,怎么看?(当时其实我还有另一个问题:我同意帮助别人是人的本性,但
人最大的本性是生存。当经济不好,自保尤不及的时候,不靠政府的强制手段,光靠慈善机构,穷人真的能得到足
够的帮助吗?---下次再去拿这个问题为难他。)

J:我没有去过瑞典,不过我去过丹麦。这些北欧国家,征收高得怕人的所得税。在丹麦,你收入的75\%要交给
政府!

I:对,我也听说了。我想,这是个矛盾,那儿的人生活得很舒服,可是他们的经济远不像美国这么有活力。

J:他们倒也有些大公司,不过都在试图往国外转移。可是政府都定下了严格的法律,规定往国外转移公司要交
很高的罚款。

I:那他们不抱怨吗?

J:普通的人民并不抱怨。他们觉得这样很好,因为他们并不知道还可以有其他的选择。他们一直生活在这样的
社会中,已经视之为理所当然。

I:说实话,我觉得这样也没什么太不好的。虽然交很多的税,多得足以令人丧失进取心,但一切都有保证。反
观自由党,虽然一切道理都说得头头是道,可是我总觉得太激进了,难以实现。自由不仅仅是个人作主,还意味着
个人要为自己的决定负责。大多数人不敢对自己的决定负责,所以他们需要政府来作主。大多数人甚至不相信自己
可以作主。

J:这也是为什么很多自由党人是工程师的原因,因为他们相信逻辑。

I:也不是所有的工程师都这么想的。我曾经遇到过一个德克萨斯州来的电脑工程师,他说,自由党的理念好是
好,可美国人还不够聪明,没法实施它。也许只有新罕布什尔州的自由之州行动成功了,他们把古典自由主义
(libertarianism)实施后,我们才能知道它是否真的可行。---可即使如此,他们能改变的东西也不多,只有州权
范围内的一些制度,对联邦政府,我们仍然无可奈何。

J:已经有很多东西了,像州税,像教育。

I:最好还是自由党人自己组成个国家算了!---新罕布什尔有权从美国独立出来吗?

J:当然有。美国只是个各州的联合体,比如宾州的宪法上就写着,在某些情况下,宾州可以脱离合众国。只要
大多数人民同意,没有理由哪个州不能独立。

I:那为什么还会爆发南北战争?

J:那是一场非法的战争!现在我们看到历史书上说,林肯说,我们要解放南方的奴隶,我们要废止奴隶制。这
些都是谎言。事实上,战争和奴隶制无关。战争的唯一目的就是要把南方留在合众国内。

I:这和我的印象不符。如果是这样的话,为什么林肯如今还遭到如此的尊崇?

J:因为他赢了!历史是按照他的说法写的。

I:可是100多年都已经过去了,就算有那些谎言和借口,也应该都水落石出了。

J:人们总是读着历史书上的这一说法,也就信以为真了。

I:那么,宾州内的一个郡,有权利从宾州独立吗?

J:这我就不清楚了。我知道以前曾经发生过这样一件事:新罕布什尔的一个郡,邻近马萨诸塞州,要求脱离新
罕布什尔,加入到马萨诸塞州去。最后没有成功。

I:从理论上讲,如果州可以从合众国独立,郡当然也可以从州独立。

J:我不清楚法律上是怎么说的。

I:这也是我长期以来的一个疑问。我一直觉得南北战争是一场非法的战争。抛开奴隶制的问题不说,从法理上
讲,合众国是各州自愿组成的,他们当然也就有权自愿离去。可是,说实话,从实际效果来看,南北战争对美国是
大有好处的。

J:我不这么认为。就算出现了两个美国,又怎么样呢?也许会比现在更好。

I:首先,如果没有南北战争,出现了两个美国,那以后美国向西部扩张的时候,也不能以一个联合的国家行
动,最后,在现在的美国的版图上,就会出现四五个国家。

J:就算最后有了六个美国吧,又会怎么样呢?

I:那美国就会失去现在她的超级大国的位置,就不会有目前的全球无与匹敌的竞争力。

J:为什么呢?

I:首先是规模。其次,从历史看来,美国崛起的过程和两次世界大战密不可分。如果美国分裂了的话,她就不
能如此顺利打赢两次世界大战。

J:那又有什么关系呢?如果有六个美国,那他们也许有的想帮助英国,有的想帮助德国,有的谁也不想帮,最
后就是对欧洲不产生影响。那么,就让德国占领全欧洲吧。苏联不还是占领了整个东欧了吗,又怎么样了呢?希特勒
并不比斯大林更糟糕。

I:(这要争下去就问题太大了,所以我换了个方向)事实上,问题不在于欧洲,而在于如果美国分裂成六块,那
么你们就会形成另一个欧洲,内斗不息,就像欧洲一样。

J:现在欧洲有几十个国家,他们也没有内斗不息。他们互相之间可以自由地旅行,统一使用货币。美国只会做
得比他们更好。

I:那是现在。你想想100年前的景象。那时候欧洲的民族主义高涨,接连打了两次世界大战。也许最终分裂了
的美国也会变成像如今的欧洲一样,成为一个大的共同体,可是在那之前,你们恐怕也要打上一两个美洲大战才会
收手呢。

J:不,这是不一样的。

I:有什么不一样呢?欧洲国家林立,因此打仗;凭什么美国就会不同呢?以前的世界可是武力横行的。

J:可如果南北美国是和平地分开的,你认为他们还想再打仗吗?你想,如果当年南方退出联邦的时候,林肯潇
洒地挥一挥手,说,再见,祝你们好运!这样的事情都可以和平解决,还有什么不能通过谈判来解决呢?

I:这倒是。---那么自由党人是反对一切战争的吗?

J:是的。

I:可如果有这种情况,比如像这次伊拉克战争,假设我们有确凿的证据,知道伊拉克人民被暴君所压迫,普遍
希望能够推翻暴君,可苦于力量不足。这时联合国通过决议,决定出兵解救伊拉克人民。这样的战争你也不支持吗?

J:不支持。我认为,谁要去打仗,谁自己去打,不要用国家的名义。你希望解救伊拉克人民吗?那你捐钱吧。
如果你正好是个青年男子,那么你自己扛上枪去打萨达姆吧。

I:那是不可行的!个人组织是绝对打不过正规军的!你带杆枪去伊拉克去就可以打败他们的飞机、坦克、大炮吗?

J:你也可以买飞机、坦克、大炮啊。不是有人捐钱吗?

I:这倒是,美国的飞机、坦克、大炮都是私人公司生产的,只要你出钱,他们就会卖给你。可是,志愿者组成
的军队能打败正规军吗?

J:为什么不能?他们也将经过一切和美军类似的训练,装备同样的武器。

I:但是,如果招募不到士兵呢?大家出钱交税还可以接受,自己上前线就会犹豫很多的。

J:难道现在上前线的不是美国人吗?如果大家都不愿意承受这个牺牲去和暴君战斗,那就说明政府无权去发动
这场战争。如果人民认为这场战争必须打,那么就会出现志愿者。你要知道,志愿者的士气和军队是完全不一样
的,因为他们自愿来做这件事,而军人只是在做一份工作。

I:这种设想倒有点像美国初成立时,用民兵来代替军队一样。

J:对,我认为目前美国的问题就是离他们当初所订立的宪法越来越远。政府越来越大,而且扩大的速度也越来
越快。这些都是违反宪法的。你再看看那个《爱国者法案》!我希望他们能够回到宪法所规定的美国上来。

I:可是,你要知道,美国宪法曾经``严格''地被执行了很多年,最后人们根据形势进行调整,才到了目前的这
个地步。你所理想的政府,不是没有试过,比如这个用民兵来代替正规军,可是最后被证明不符合现实,美国又强
化了联邦政府,成立了强大的军队。这一切过程,都是不可避免的。就算你今天把当初的美国再来试一遍,最后仍
然还会变成一个结果。

J:我不这么认为。我觉得我们还可以再试一次。

I:比如持枪权吧,这是当初宪法所保证的。我也明白枪支是公民反抗政府的重要手段,并且在正常情况下,人
们将只把枪支用于自卫。可现在,在这个酒吧里,如果那些喝得醉醺醺的男人们都带着枪,你和我还能够这样安坐
着吗?我们不能总假设别人是理智的、清醒的。

J:你知道当初那些西部的人们是怎么做的?在酒吧的门口,有个``挂枪处'',所有配枪进来的人,都必须把枪
挂在那里,不然酒吧就不卖酒给他们。

I:哦,如果他们不肯解枪,那么大概也有不须挂枪的酒吧,不过,如果你去那里喝酒,那么后果就只好自负了。

J:对,你自己选择。

I:这倒有点像我的一个想法。我的看法是:将来政府迟早要消失,取而代之的将是些保安公司。这些保安公司
负责你的安全,同时开出不同的条款,比如这个公司不允许你配枪,他负责你的全部安全,那个公司允许你配枪,
但就又会有其他的很多规定。这两个公司互相竞争,我们从中选择比较适合自己的公司,和他们订立合同。这样,
你和这个保安公司是平等的合作关系,而不是像现在这样,我们简直是在被政府压迫。

J:这可真是个有趣的想法。

I:同样的,还有高福利的公司,就是你交很高的费用,但他保证你的福利,就像保险一样。那么,又有低福利
的公司,你交的费用很少,但如果你穷苦了,也没人来帮你。这些公司费用,就相当于税,不过现在我们没有选择
而已。每次和公司订合同,可以在四五年后再讨论是否续约,也就和现在4年搞一次选举差不多。关键在于,这些公
司互相竞争,为了拉客户,他们肯定会尽可能多地为我们考虑,而不像政府那样,垄断了这些服务,所以会成长为
一个庞然大物的怪胎,自给自足地扩张剥削,而不理会人民的真正需要。

J:可是,如果这些公司之间起了争端怎么办?

I:那会有专门的仲裁公司来处理。

J:如果仲裁公司的处理不合理怎么办?有没有上诉的途径?最高的一级仲裁是什么?权威来自哪里?

I:没有更高的仲裁公司了。当初你启用这家保安公司的服务时,合同上就有一条,说明了本公司如果和其他公
司起了争端,将由何家仲裁公司来处理。你如果对这家仲裁公司不满,你可以选择别的。

J:那你怎么防止仲裁公司的腐败、作弊?

I:关键在于仲裁公司之间的竞争。在未来世界,资讯充足,一个仲裁公司做出不合理的仲裁,其他公司和个人
马上就会知道,这样这家公司很快就会失去客户,最终退出市场。由于有这样的压力,他们不得不提供最好的服务。
他们必须维持自己的信用,不然就赚不到钱。长此以往,那些``坏''公司就被市场淘汰掉了。

J:嗯,很有趣的想法。

I:总之,就是将一切都商业化,让利益来驱动,让那些强加给人们的国家、政府、宗教、道德都见鬼去!

我所提到的关于未来的设想,只是个初步的构想,细节我自己也没有仔细考虑过。倒是吉姆所极力推崇的自由
党人新天堂,我回家后自己又想了一下,发现了他的一大漏洞:他要求政府从除了安全之外的所有事务走开,一切
都由个人的志愿活动来承担。从表面上看,这是极右,即主张个人完全自由,个体充分发展;但这和极左却又暗通
了。他总在假设人有愍恻之心,因此社会问题可以通过人们自觉、自愿地来解决,这和马克思假设人到了共产主义
社会后极为自觉、劳动成为人的需要、大家各取所需的设想又有什么不同呢?

我对政府深恶痛绝,但也不能不同意恐怕现阶段这仍然是个``必要的邪恶''。我们牺牲自己的自由和权利,交
给政府我们的部分权利,换取安全和其他利益。目前我们可以做的,只是警惕政府不要越过我们给它所授的权力的
范围,不要侵犯我们作为个人本身的权利。也许,将来,我们才可以开始考虑让政府和国家脱钩,让国家这个概念
消失,这样才能形成政府之间的良性竞争,最终通过竞争,提供给我们最好的服务。

\section{独立日焰火晚会的参选签名}

如我以前介绍过的,我们亲爱的恰克同志,今年刚满25岁,立刻就冲出来竞选众议员了。他需要1500个签名才
能够上选票,而为了对付敌人的挑剔,他需要收集到2000个签名以防不测。7月1号我们蒙郡自由党开会时,他才收
集到近300多个签名,离8月2日的期限已经不远,因此他决定乘着美国7月4日独立日的东风,来发起两次征集参选签
名的战役。

两次活动都是定在附近的焰火晚会上。这是美国人民所喜闻乐见的一种庆祝方式,每年的独立日庆祝活动几乎
都少不了的。恰克在他的选区内挑了两处,一处在一个叫安布勒(Ambler)的小镇,他们星期五(7月2日)晚上在当地
的中学放焰火;另一个在阿宾屯(Abington),在7月4日当天。我们公司给独立日放了两天假,从7月2日一直放到7月
5日,我便参加了恰克在安布勒的活动。其他参加的还有恰克的一个朋友菲尔,以及我们的老相识吉姆、 达仁、 乔
和葛锐格。

我们计划下午6点的时候在安布勒火车站碰头,吃恰克准备的比萨饼,7点钟时赶到安布勒中学,开始战斗,9点
钟时开始放焰火,我们就撤退到附近的一个酒吧,由恰克请大家喝酒。不过由于吉姆、 乔和葛锐格都比较忙,只能
在7点钟时和我们在安布勒中学直接会合,所以只有我、菲尔和达仁在安布勒火车站享用了恰克带来的比萨饼。菲尔
是恰克少年时起的朋友,年初刚从大学毕业,还没找到工作,所以比较有时间,常帮恰克出来跑,恰克目前筹集到
的签名就是他们俩上次一起干的。

恰克这次准备得比较充分。首先,他提供了印有他的名字的签名表,我以前从吉姆那里拿的签名表都是整个宾
州通用的签名表,目的是让在全州进行竞选的候选人上选票,比如总统、参议员的候选人。这次是专为恰克筹集签
名的,所以签名者必须来自他的选区,而且签名表上必须要印有他的名字。当然,我们也各自带了几张全州通用的
签名表,这样万一有人是从外地来的,还可以也让他们签名。不过,恰克估计,焰火晚会这种纯粹当地的活动,大
部分人应该都是来自他的选区。

另外,他把大家的手机号码都打在一张小卡片上,我们不会在焰火晚会里失散。他又打出了他所在的选区的所
有城市名,以备签名者参考。他还准备了一堆名片,我们每人拿了10张,如果有人对恰克感兴趣,我们可以把名片
给他。

到了安布勒中学,菲尔半开玩笑半感慨地说:``我们就是在这里被虐待了4年!''原来他和恰克都是从这个学校
毕业的。恰克先找到他的一个朋友,用他的地方停车。回来后他对我们说:``我的朋友说停一辆车收我4美元,但我
不知道她是开玩笑还是说真的。''达仁听了笑着说:``一定是在开玩笑呢。''我说:``我猜是因为中学的停车场今
天要收费,5美元一辆车,不然她不会这么说的。''果然,后来吉姆遇到我们,一见面就伸手要5美元的停车费。恰
克掏出了一张5美元的钞票给他。这个细节给我留下了深刻的印象,因为吉姆是个小业主,远比做学生的恰克有钱,
而且他也不是吝啬的人,曾经一出手就给付费参选签名捐了400美元。很显然,他认为今天是来帮恰克的忙,那么所
有的花费,当然应该找恰克报销。

焰火将在中学的草地上放。我们商量了一下战术问题,最后决定大家分头行动,分别卡住进场的几个进口,拉
人签名。话音未落,吉姆已经转身拦住了正在旁边走过的一对夫妇,拿着签名表说:``你们可以为我们签个名吗?我
们想帮助恰克$\cdot$莫顿上选票。''那位女士有些疑惑地问:``恰克$\cdot$莫顿是谁?''吉姆神气地一指站在
旁边的恰克说:``就是他!''

出乎我们意料的是,那位女士立刻说:``哦,我认识你!我看过你的辩论!我同意你的很多立场。''恰克赶紧
说:``谢谢。''吉姆马上乘机说:``那么请您签个名吧。''这两个人自然立刻二话不说,签上了大名。

首战得胜,我们都士气大振,立即分散去站岗。我和达仁、菲尔走了同一个方向,然后再各自分开到不同的路
口。这次我有了可以隆重推出的候选人,自然气势不同以前,见人过来后,先说一声:``晚上好!''别人一般都会回
答一声``晚上好!''然后我就说:``您可以为我的朋友恰克$\cdot$莫顿签个名吗?他将在11月竞选众议员,可首先
他必须筹集到1500个签名,才能出现在选票上。您可以为他签个名吗?''

有的人也会问一句:``这个人是谁?''我便告诉他们:``他是维拉诺瓦大学法学院的学生,今年25岁,是个好样
的年轻人。''有时候我还指指草地上的人群说:``恰克$\cdot$莫顿就在那里。我们一起来搞这个活动的,只不过
我在这边,他在那边罢了。我不能给你指出他在哪儿,但我可以肯定地告诉你,他就在这附近。''

由于这次更加师出有名,所以回应比上次积极了些。当然,也多了一种拒绝的方式:``对不起,我不了解这个
人,因此我不能签名。''这时候,我往往会抬头找一找恰克,不过从来也没有找到过。后来听吉姆说,他曾经在遇
到了这样的回答后,一个电话把恰克叫了过来,希望恰克的出现能够打消人们的顾虑。不料那一对夫妇围绕恰克的
政治立场,向他足足询问了有15分钟,最后还是决定不签。恰克后来无奈地对我说:``讨论政治是很有趣的,不过
我觉得这时间如果用来筹集签名会更有效。以后遇到这种情况,一两分钟后我就会退出了。''

今晚的另一个好处是,美国人往往把焰火晚会当作是一次家庭活动的机会,基本上都是拖家携口地出动,而一
般一对夫妇中只要有一个人签了,另一个人也不会拒绝。反正我这次特地准备了两个签名表,可以同时左右开弓,
两个人一起签,也不浪费时间。不过也有些夫妇中的其中一个好像是迫于无奈地被我拦住后,另一个人却带着孩子
继续往前走,然后站在不远处,满脸不耐烦地在那里等。这种情况下,我当然也不好意思再冲上去要求另一个人来
签名。

一个意想不到的问题却是,经常有人被我拦住时说:``我已经签过了。''看来是菲尔已经进一步冲到我的前
方,抢先埋伏了。我只好换到另一个路口,才解决了这个问题。

接下来的发展,和以前的类似活动也就大同小异,仍然是有人问也不问,大笔一挥就签了的;也有人丝毫没有
兴趣,或者说不想参与政治就直接走开了的;仍然是有人会问:``他是哪个党的?'',然后听说是自由党后立刻掉头
就走;也有听说是自由党后神色顿时缓和下来,很乐意地签名的。

到了8点左右,恰克给大家打来电话,要进入下一阶段作战,目标是草地上的人们。这时离焰火晚会开始只有一
个小时,大部分人都已经在草地上铺开毯子,坐在上面聊天玩乐,等待晚会开始了。我觉得他们坐在那儿挺自在
的,一开始不好意思去打搅,又在路口坚守了一阵子,才走到了草地的最后面,从那里开始战斗。

结果却顺利得出奇。想来是因为这些人都在无聊地等着焰火开始,反正也无事可做,所以对我都比较友善,不
像路口的人们都急着要赶到草地上来占位置。大部分人都肯签,而且这时往往会有两三家人坐在一起,只要一个人
签了,其他三五个人也通常会愿意签,而且也都喜欢和我多聊几句。

比较有意思的是有一次,两个女士坐在一起,我在她们的椅子后面的中间位置坐下---这是我的一贯战略,这
样可以同时对两个人说话---开始游说。一位女士又问了那个问题:``恰克$\cdot$莫顿是谁?''我回答说:``他
是维拉诺瓦大学法学院的学生,虽然没有我好看,不过也可算英俊了。''两位女士顿时笑得前俯后仰。我乘机
说:``那你们可以为他签个名吗?''右边的女士指着她的同伴说:``她是不能签的,她不住这里。''

我说:``那没关系。您是住在宾州的吧?''她说:``不,阿根廷。''

这可真让我吃了一惊,不过我立刻高兴地说:``啊,阿根廷,我喜欢的国家!我正打算要去学西班牙语呢。而
且,我特别喜欢博尔赫斯呢!''

她却疑惑地说:``谁?''

我想,大概是我的发音不准吧,又把博尔赫斯的名字重复了几遍。她却仍然不知道我说的是谁。我暗暗地想,
也许我说我喜欢马拉多纳或许会更容易些。这时,另一位女士说:``你叫他们签了吗?''她指着右前方坐的两位男
士,应该是她们的丈夫吧,说着她便叫他们:``你们签了这个年轻人的签名表吗?''

那两个男士有点不耐烦地说:``没有!''我便走上前去,请他们签。结果他们听完我的解释后,说:``我不知道
这个恰克$\cdot$莫顿是谁,我不能签。''我指着前面的草地说:``他就在那边筹集签名。''他们还是说:``我不
知道他的立场,所以不能签。''

我没有办法,只好道了声谢后,又回到那两位女士那里。结果她们倒也恪守妇道,见丈夫不签,便也不肯签。
唯一的进展是我把博尔赫斯的名字写了下来,和那位阿根廷女士总算接上头了。她对他的诗印象更深,而我则对他
的小说赞不绝口。

离开了她们后,我又在草地上漫游了一阵,这时候天已经比较黑了,很多人说看不清签名表,不肯签了。当晚
我最后一次的努力,是在一处朝天灯的旁边,也是两对夫妇坐在一起。我走到他们中间,对他们说:``我的朋友恰
克$\cdot$莫顿在竞选众议员,需要1500个签名,您可以为他签个名,让他11月出现在选票上吗?''

一位女士问我:``你是哪个党的?''我说:``自由党。''

她顿时高兴起来,说:``那你们是要打算把布什从白宫里赶出去啰?''

我回答说:``是的。我们也有自己的总统候选人。''心里暗暗地想,看来是遇到民主党人了。他们立即接过了
签名表,仔细地看了起来。由于天色实在太暗,已经看不太清了,那位男士干脆站起身来,走到灯光旁去看。女士
则善意地对我说:``你应该带个电筒来的。''我说:``是的。下次我们就会有经验了。''

她仍然爽快地签了字。我便劝他们的朋友也来签,不料他们是从外地来的,只能签在我准备的宾州通用签名表
上。这时,那位男士也已经看清楚了表,回到座位上签起名来。我说:``那你们是民主党人啰?''

他们一起笑着说:``我们是登记的共和党人!''

我吃惊地说:``那你们怎么会反对布什?''我想,难道又遇到上次在克里支持者聚会那样的卧底了?

他们争着说:``因为布什乱来,使美国成为国际上的笑柄!''``他完全背离了共和党的信条,胡作非为!''

这下大家找到共同的话题了。我又问他们:``那你们看了《华氏``9.11''》吗?''他们说:``没有,我们也不打
算去看。''

``为什么?你们既然反对布什,就应该去看这部电影啊。''

他们呵呵地笑着说:``哦,我们已经很反对布什了,不需要去接受再教育了!''

我连忙告诉他们:``可是这部电影很搞笑。你们就算不去看它的内容,光看看它的搞笑,也会很享受呢!''

这下轮到他们惊讶了:``哦?这样的电影也会搞笑吗?我们还以为它很严肃呢!''

我肯定地说:``很搞笑。去看吧,你们不会后悔的。说实话这电影挺片面的,但只要它足够有趣,不就应该去
看看吗?''他们都点头同意。那位女士还举起手中的零食袋,请我品尝。我见她盛情难却,便从中取了三四颗花生,
吃了起来。这时他们的朋友们也已经签完,我和他们谈了几句关于自由党的信条,便谢过他们走了。

时间已经过了9点,天完全黑了下来,焰火随时可能开始放。我给恰克打了个电话,问他是否已准备撤退。他却
显然斗志依然昂扬,回答我说,焰火还没有开始,我们还可以继续。不过我觉得人们已经看不清签名表了,便走回
当初的路口,去找其他人。

很快我找到了达仁、乔、葛锐格和吉姆。大家再去找恰克,忽然砰的一声大响,人群中欢声雷动,原来,焰火
终于开始放了。我们也抬头观看。过了一会儿,吉姆笑着说:``我们能看到焰火,这都得感谢中国人啊。''

我回答说:``是的。我们中国人常说,我们发明了火药,用来做焰火,你们西方人却把它拿去做枪炮。''

吉姆说:``说得好!''乔却不服气地说:``那是因为中国人没有找到用它做枪炮的方法。我看过一些文章,中国
人也有大炮的。如果他们能发明现代的枪炮,他们一定也会发明出来杀人的。''

这下轮到我无话可说了,只好继续看焰火。

和恰克、菲尔会合后,大家清点战果,我筹集到45个签名,外加三个宾州通用签名,大家总共筹集到400多个。
主要是恰克和菲尔比较多,都筹集到100多个。恰克弄到这么多倒不足为奇,因为他本人就是候选人,别人看见了这
个人活生生地站在面前,签的可能性比较大。菲尔也筹集到100多个,我是很佩服的,可能是因为他是这里毕业的,
算地头蛇,说起来比较和当地人谈得来吧。

然后我们聚到酒吧里,总结经验教训。大家一致认为,应当一开始就到草地上的人群中去的,那里的效率比路
口高得多,而且人们的态度也好得多。在路口,大部分人尽管没有流露出不满的意思,但都不会喜欢拦路剪径的骚
扰者,而坐在草地上无所事事地等待焰火开始时,对我们就会比较欢迎了。大家散去后,恰克往整个宾州参选签名
的邮件组上发了封信,把我们的经验和所有人分享。除了要早去草地外,他还提醒其他想去焰火晚会的人要带上电
筒。不过他的另一条建议:``当第一个人在签的时候,立刻开始要求下一个人签'',我是不大同意的。事实上我好
几次遇到这种情况,一个人在签的时候,又有其他很多人在走过。我总是站在那里等待第一个人签完,道过谢后才
去拦截其他人,如果人群已经走过,我宁愿再等待下一拨。我觉得别人好意为你签名了,你却立刻丢下他,去找其
他人,是不够礼貌的。

8月3日附记

我们筹到的签名仍然不够,吉姆雇佣了收费签名者,花了大约500多美元,最后,所有的自由党人都筹集到了足
够多的签名,可以出现在秋天的选票上。

波士顿北游缘起

两个月前,有人往我所在的``费城支持克里(Philadelphia for Kerry)''邮件组里寄了封信,向大家介绍了一
个``亚太裔进步组织(Asian Pacific American for Progress,以下简称APAP)'',希望能够吸引到更多的亚太裔参
加进来。

我看了很感兴趣,便顺着信里的介绍,来到他们的网站:http://www.apaforprogress.org/ 。原来,他们本是
亚太裔迪恩支持团体(APA for Dean),迪恩初选失败后,又和亚太裔克拉克支持团体(APA for Clark)合并,成为了
现在的``亚太裔进步组织''(APA for Progress),其核心成员都来自原来的迪恩竞选团队。早就听说迪恩竞选团队
的活动开展得最为有声有色,尤其擅长通过互联网来组织,果然名不虚传。

我加入了这个APAP,随后就开始不定期地收到他们的Email。不过我们亚裔还是比较善解人意的,大概两个星期
才有封Email,不像费城克里支持者那个邮件组,每天都恨不得有十封信之多,有时候还能彼此吵将起来。

到了6月,APAP在一封信中宣布,乘着民主党全国大会(Democratic National Convention)在波士顿召开的东
风,他们将在波士顿举办一个训练班,为对政治活动感兴趣的亚太裔提供基本技能训练。民主党全国大会是从7月
26日(星期一)开到29日(星期四),APAP的训练班则从7月24日(星期六)开始,星期二结束,历时4天。由于找到东家
赞助,学生参加只要25美元,对于我这样的工作族来说,也只不过50美元,还提供住宿、全部早饭和部分中晚餐。
我当时正好觉得自己缺乏经验,要帮助克里选举,却没个下嘴处,这个价廉物美的训练班岂不正是个好开头处哉?于
是立即报名参加了。

训练班的前两天是周末,通过一个叫民主党基层组织和网络(Democratic Grassroots Activities Institude
and Network,以下简称GAIN)的组织来进行,内容是我们和GAIN自己组织起来的几千个美国人一起进行专题讨论
会,加上周日晚上一场摇滚投票(Rock the Vote)音乐会。星期一是APAP自己的活动,晚上还可以去看一场亚裔晚会。
星期二则属于自由活动,不过APAP推荐了一个名叫``革命女性大会''(Revolutionary Women Events)的活动,这是
由于民主党全国大会正在波士顿召开,各种民主党属下的组织也乘机活动,我看这个活动有希拉里$\cdot$克林顿、
前国务卿奥尔布赖特等女性政治名人出席,也就花了10块钱登记参加了。

剩下的就是买机票,准备衣物之类。好在我刚刚在两个月前去波士顿玩过,心里比较有数。当然,反过来说,
上次刚开车去过一次,现在又买机票专程跑一趟,好像有些资源浪费。不过,我的理解是,上次去是玩,这次去是
取经,就当一次北游记吧。

出发前两天却又忽然收到克里竞选团队(Kerry Campaign)一封信,说克里本人将在下周二来到费城,召开一次
拉励(rally,台湾一般译为``造势'',不过我觉得译为``拉励''更好一些。),呼吁我们踊跃前去参加。这种活动是
免费的,我当然很想参加,可惜下周二我正在波士顿参加``革命女性大会''呢,再改机票已经来不及,只好很惋惜
地放弃了。

去波士顿的路上不太顺。按原计划,我到波士顿大约是8点,坐地铁和公共汽车到目的地里吉斯学院(Regis
College),正好还赶得上APAP在9点的简短的欢迎会。可由于暴风雨,航班被整整耽搁了4个小时,我踏上波士顿的
土地时,已是半夜12点35分。我来到机场的服务中心,询问是否还有地铁,服务人员同情地告诉我,地铁到12点30
分就停了。

``难道民主党大会期间没有特殊班次吗?''我怀着一线希望问。在我出发前,就收到过订票网站的Email,警告
说由于民主党大会会给波士顿带来巨大的客流量,很多班次和时间表都改变了,要我们提前做好准备。也许,地铁
也超时运转?

``没有。''服务人员斩钉截铁地说。不过她仍然抽出一份地铁图给我,以供参考吧。

没办法,我只好自掏腰包,花了50多美元,打的前往。由于排长队等待出租车又等了不少时间,最后,我到达
里吉斯学院时,已是凌晨1点半了。还好,组织者仍然在那里等着我,看见我终于到达,上来就给了我个热情的拥抱。
我谢过了她们,迅速地签了到,不无内疚地问:``我大概是最后一个吧?''

``不,你后面还有四个呢!''

``哦,''我笑着说,``这下我不再觉得那么有罪了!''

次日早上,我在餐厅里遇到其中一人,她告诉我,昨晚她们直坚守到4点。相比起来,我睡了四五个小时,已经
算多的了。里吉斯学院的住宿条件还是很不错的,每个房间有两张床,不过由于来的人并不多,大部分人都可以独
自占据一个房间,所以这一觉虽短,睡得却挺踏实的。

\section{有钱出钱,有力出力}

今天的日程安排是去市区的海恩斯(Hynes)会议中心,进行GAIN的训练。APAP租了一辆大巴士,装上所有的人,
开了近一个小时,才到达会议中心。下车进了大楼一看,里面豁然开朗,人流涌动,中间是两座巨大的电梯,上下
的人群络绎不绝,宽敞的过道边上,桌子一张接着一张排开,坐着很多工作人员:有些负责登记;有些负责分发材
料;有些则是某个组织的特别联络处。

我们上了二楼,忽然对面一阵骚动,一个西装革履的人在众人簇拥下正往外走,周围的人们不停地问他问题。
我旁边的很多人也纷纷拿出相机,对着他一阵猛拍,还有人奔过去想和他直接说话。有人告诉我,这个人是新罕布
什尔州的州长比尔$\cdot$理查得森(Bill Richardson),在民主党内人气颇高,被GAIN邀请来做开幕演讲的。后
来我拿到日程表才知道,原来整个训练在8点半就开始了,由于我们住得离市区太远,所以到达这里时是9点,已经
错过了这个开幕演讲。

二楼的人比一楼更多,这一层是训练的主会场,正对着电梯的是大会堂,十几个会议室顺着走廊在两侧排开。
开幕演讲刚结束,到处都是匆匆奔往会议室的人,以及站在一旁帮忙的工作人员,绝大部分都是年轻人,青春活
泼,人声鼎沸,间或着打闹嘻笑,让我顿时生出一种身处盛会的感觉。

由于参加训练的有3000多人,除了大会堂外,每个会议室只能容纳100多人,因此组织者随机地把我们分发到
20多个房间里去。各个房间的主讲者虽然不同,但主题却都是事先一致定好的,因此去哪间也都一样。

我们签完到,又领到一大堆材料,再赶到应去的房间时,主讲者已经开始了,主题是《Field 101》。101是基
础课的意思,美国大部分大学的课都是用数字编号,按难易程度往上递增,101就是第一课,最基础的那门。Field
这个词不太好翻译,我勉强把它译作``选区活动'',主要是进行实质性的活动,比如和选民接触,督促他们登记、
投票。

主讲者是个叫柯林(Collin)的小伙子,他说:``大家都知道,所有助选的人们都认为自己的工作最重要,筹款
组的人说,如果没有我们,你们哪来的钱?宣传组的人说,如果不是我们,选民怎么知道我们的候选人是谁?我和大
家一样,不喜欢这样的夸夸其谈。可是,我必须诚实地说,---选区活动确实是整个助选中最重要的。''大家都笑
了。

他接着介绍了选区活动的四个主要步骤:第一步,登记选民。登记了要去投票,并不意味着非要去投票不可,
更不意味着一定要投票给某个候选人,所以一般来说,人们对此并没有太大的抵触情绪。第二步,分析选民。我们
需要分析每一位选民的背景,从而提出最有可能打动她的议题,比如,对于老年人,可以用福利政策,对于年轻
人,则可以用宗教自由、反战等。第三步,说服工作,就是运用第二步的结果,说服选民投票给我们的候选人,这
就部分取决于个人的能力了。第四步,GOTV---投票动员(Get Out The Vote),让人们出来投票。柯林开玩笑
说:``共和党人在决定了要把票投给谁后,就会在选举日去把票投了。民主党人却有三不投:下雨天不投,没理发
不投,晚起床了不投。''所以,仅仅说服了选民是不够的,一定要保证他们确实在选举日去投票了,为此,我们可
以打电话、发Email提醒,并且应当主动提供交通条件。

选区活动人员需要挨家挨户地访问,但分析选民的基本策略还是必须的。柯林说:``背景相似的人,相对比较
容易被对方接受。比如,如果需要到黑人区去访问,我们会找个黑人去;要到高收入(upper class)区去,我们就雇
这种人家的孩子。''这时,听众都笑了,让我倒有点莫名其妙,也许是笑他们没有钱,只能雇孩子,没法让高收入
人群自己的一员来工作吧。

柯林继续举例说:``不要小看这里的学问,准备工作一定要做仔细了。我在佛罗里达州的时候,那里有很多西
班牙裔人,在我们看来,他们都是一样的。可是有一次,我们派一个西班牙裔的义工去他们的社区,却很不受欢迎。
原来,这个义工是波多黎各人,而那个社区里大部分是多米尼加人!反过来,如果你派个多米尼加义工到波多黎各社
区去,估计也得给赶出来。''

我想,可不是么,西方人也分不清中国人和日本人的区别,在他们看来,不都是亚裔么,可你要是派个日本人
到中国移民社区去,弄不好反倒激起逆反心理来,投票给你的对手了。

除了对选民的生活背景需要调查外,对他们的政治态度也应当事先弄清楚。如果一个地方太偏向共和党,那就
根本没必要去了,因为我们的时间和精力都有限,容不得我们冲进共和党人的既有阵地,淹没在他们的人民战争中。
我们还是应该去那些民主党稍占优势的社区,说服他们去投票,这才是最有效的选区活动。

柯林说:``自由派总认为,所有的人都是可以说服的。很不幸,现实不是这样。事实上,人们的投票经常会很
莫名其妙。''比如说,他自己在佛罗里达的老板,坚决反对伊拉克战争,出来参加竞选,却半途就退出了,因为虽
然说起来好像所有的人都反战,可一到选举中,只有他老板这么一个候选人正面提出反战,而且由于没有选民的支
持,被迫提前放弃。

很快,一个小时就过去了,短暂的休息之后,我们进入下一个主题:``让年轻人去投票。''我早就注意到,这
次来参加训练的绝大多数是年轻人,后来在和他们的接触中发现,他们基本上都是来自``纽约州青年民主党人''、
``宾州大学生民主党联合会''这样的组织,无怪乎都那么年轻。相比较而言,我们APAP就是什么年龄阶段的人都有
了,一下子置身于年轻人的天地中,感觉还真不错。

这次的主题由两位女士和一位男士共同主讲。一位女士类似于主持,在串场的同时,强调要寓政于乐,尤其对
于年轻人来说,活动应当开展得有趣,千万不可开成忆苦思甜大会或者战前动员、英模报告。她说,美国年轻人参
与政治,最早是社区服务形式,我想那大概是很久以前的纯真年代,如童子军之类。后来,年轻人参与政治主要是
通过各种活动---当然,其中抗议活动占绝大多数,尤以上世纪60--70年代为烈。到了近年来,年轻人对抗议活动
也失去兴趣了,一股脑儿地全hip-hop(一种音乐形式,曾译作说唱乐,港台也叫嘻哈音乐)去了,这时,年轻人在政
治中主要的能量就体现在助选了。可是,要把他们拉到助选活动中来,或者说服他们为民主党候选人投票,就必须
想出有趣的创意。我想,这个说法有道理,现在的年轻人,啥没见过,再走以前童子军的高尚路线、60年代的知青
路线,肯定都会被认为笨蛋,喻之以义、动之以情都没有用,只有诱之以趣了。

这位女士还说到做到,先后和我们做了两个游戏:一个是写下各种名字和歌曲,然后分析这都代表些什么;另
一个是在纸上画只猪,然后通过图画的样子来分析自己的性格。这两个游戏我以前都玩过,不过在座的美国年轻人
倒一个个都玩得很起劲,虽然我没有想明白这和助选有什么关系。

另一位女士只简短地讲了三点:第一、组织活动,要事先定下基调,能够吸引到别人的注意;第二、有趣,不
仅仅是要让别人觉得有趣,自己也应该享受其中的趣味,而不是仅仅作为自己的政治手段;第三、思路要开阔,比
如筹款晚会可以不叫筹款晚会,直接说要开派对,才会吸引到更多的人。

那位男士是个非洲裔,不用说是活力四射了。他本人是一所大学的民主党政治顾问,这次主要讲怎么在校园里
组织政治活动。他先点了两位听众,都是金发女孩,问她们叫什么,是从哪里来的,什么专业等等,和大家建立起
良好的互动关系,会议室的气氛顿时活跃起来。他说,大学政治活动里,最重要的是GOTV:让人们去投票。因为大
学里的年轻人中,一般自由派都是占压倒性的优势,年轻人从来都是民主党的天然支持者,所以不需要我们费太多
口舌,最需要担心的其实是年轻人对政治不感兴趣,宁愿在宿舍里睡大觉,也懒得去投票。所以,每次选举日那
天,他都会很忙,到处打电话,催促人们去投票,甚至有一次,他开了辆装喇叭的大车,在校园里招摇过市,广而
告之:今天是选举日,大家快去投票!

显然,他也是一个做事喜欢别出心裁的人。他强调说,在这个时代,我们进行政治活动需要一些非传统方式。
为此,他问听众:``你们进行过什么非传统的活动吗?''

有人举手说:``我们为布什举办过提前退休派对。''

还有人说:``我们以支持布什的名义卖各种礼品,然后把钱寄给克里。''顿时引起了哄堂大笑。我想,这个够
狠。我曾经听说,由于共和党的全国代表大会将在纽约召开,有些纽约的年轻人就报名去做志愿者,但到时候却不
出现,准备用这个办法给共和党制造些麻烦。想出这两个点子的人真可谓一时瑜亮,令我的佩服之情如滔滔江水,
连绵不绝。

他们三人讲完后,我们又休息了一阵,柯林重新上阵,继续讲选区活动的组织。他说,他已经做了六年的选区
组织者了,``这个世界上没有比选区组织更难的事情了。我经常会对天祈祷:上帝啊,你为什么要这么惩罚我?---可
是,这件事必须要有人做。''

为了强调选区活动的重要性,柯林又开始损筹款和宣传组了。他说,筹款组筹集来的钱,绝大部分都是用到电
视宣传里去了。但是,那些最重要的拿不定主意的选民,改变主意比换内衣还频繁,很少会被电视广告影响观点;
再者,由于双方阵营都大放广告赞扬自己、攻击对方,最后的结果就是互相抵消。所以,到底选举结果鹿死谁手,
还是要靠拼选区活动。人们对于电视里的东西,总会本能地半信半疑,选区活动却是派出人来进行一对一接触,一
个人活生生地站在你面前,她/他的话就可信得多,而且也有的放矢得多,不像电视广告,经常花了几百万美元,说
出来一堆假大空。

他还举了这次民主党预选的例子。迪恩一开始呼声极高,为什么会在爱荷华州和新罕布什尔州连折两阵,从此
一蹶不振呢?柯林说,这是由于迪恩看起来声势浩大,把全国各地的支持者都带到爱荷华州和新罕布什尔州去了,这
些人当然也努力工作,可是他们不是当地人,和选民接触时缺乏共同语言,难以真正地沟通;而克里的选区组织者
就很聪明,他们找来当地广受尊重的人士,请他们去为克里助选,这些人能够和选民就当地的情况,讨论当地最受
关心的议题,选民更容易被他们打动。因此克里在爱荷华州和新罕布什尔州一举翻身,拔得头筹,很顺利地便取得
了党内提名。

上午的讲座到这里就结束了,我们去餐厅吃饭。提供的当然都是美式食品,餐厅里却没有桌椅,工作人员叫我
们都去大会堂,原来那里会有讲话。我刚刚在会议室里认识了一个APAP的朋友杨蕙,便和她一起到大会堂吃饭。她
来自``纽约州青年民主党人'',和大家一起坐大巴士从纽约开来的,在香港出生,学过中文,但不会说普通话,我
也不会说粤语,因此只能用英语聊天,后来聊到一些人名,实在无法沟通,只好在餐巾纸上写下``金庸''、``鲁
迅''之类。

大概下午1点钟的时候,大家的午饭都差不多吃完了,大会堂里也已经坐满了人,一位女士上台讲话了。她来自
加州,风度气质酷似特蕾莎$\cdot$亨氏$\cdot$克里,可惜我没有记下她的名字。她讲话的主题是:``以做一
个自由派为荣'',因为近年来美国社会愈趋保守,``自由派''都快成为一个贬义词了,政治人物都避之不及。然
后,她介绍了中午的演讲者:新泽西州州务卿雷吉纳$\cdot$托马斯(Regena Thomas)。至于她的演讲,并无出众
之处,我也没留下什么印象。

演讲之后,大家正要各自作鸟兽散,组织者又宣布说,由于受雷雨天气的影响,下午的好几个主讲者无法及时
赶到,因此我们被迫跳过某些议题,而且每两个会议室的人合并到一起,以照顾那些主讲者没有来的听众,反正议
题都是一样的,也不会有影响。

我倒还是在上午的那个房间。早上由于去晚了,座位都已坐满,我只好坐在窗户边上,这次我便及早进场,占
了个靠前的座位。

下午的主题是``筹款''。虽然上午柯林刚刚奚落了``筹款''一把,但我对它还是很感兴趣的,部分原因是由于
我是蒙郡自由党的会计,我必须也学会些筹款的技巧,不然也太尸位素餐了。

首先开讲的是一位西班牙裔男士,他先通报了个好消息:目前民主党在21个``摇摆州''的开支都已经超过了共
和党,这还是历史上头一遭,因为传统上总是共和党比民主党有钱的。接着他回答了一个经常被提出的问题:为什
么有人建议克里不要在这次民主党全国大会上接受提名?

原来,美国法律规定,两大党候选人在接受本党提名,正式成为总统候选人后,就不能再接受任何捐款了。由
于民主党全国大会在7月底召开,共和党大会要到8月底才开,因此共和党会多出一个月的筹款时间。共和党的资金
本来就比民主党充裕,如今再多出一个月的宝贵时间,两党的差距就更大了。所以,在去年迪恩气势最猛的时候,
他阵营里的谋士觉得候选人资格已是囊中之物,便开始未雨绸缪,提出一条计策:迪恩在民主党全国大会上先不接
受提名,这样可以再筹上一段时间的款,然后再宣布接受提名不迟。正所谓``上有政策,下有对策'',劳动人民的
智慧是无穷的。后来迪恩在预选中败北,这个策略又被克里阵营的人接过去,炒作了一阵子,后来被否决了。大概
克里觉得这对他的光辉形像有碍吧,他已经被共和党牢牢地描了个``机会主义者''的脸谱了,再这么做就更证实了
对方的指控。

这位主讲者由于是西班牙裔,讲了些如何在少数民族中筹款的经历。他说,少数民族对筹款的看法和主流社会
是不同的。主流社会一般把捐款看成是人生的一部分,或者至少是生意的一部分,他们认为,你在这个社会里不是
孤立的,你捐款为自己创造一个更好的环境,这也是游戏的规则。但少数民族就更谨慎些,他们会退后一步,仔细
观察这个来要捐款的家伙,是否真的能对自己的生活产生正面的影响,然后再做决定。所以,在少数民族里筹款,
是要付出更多的努力的。

这个说法大获我心。我也早就观察到华裔社群对捐款的态度,比大部分美国人要冷漠得多。当然,一个很现实
的原因是:少数民族的生计一般都比主流社会中的人艰难---不过他说的文化差异也确实存在。

他也讲了一些开展筹款活动的原则,比如,不要把眼光只盯在支票上,应当把筹款活动看成是一个建立关系网
的机会,哪怕别人不捐款,只要他们来参加你的活动,就说明他们感兴趣,留得青山在,不怕没柴烧,以后再向他
们筹款也是一样的。又如,不要一开始就向别人要求大数目的捐款,可以把这个捐款分成若干个小数字,类似于分
期付款,在每次快到期时,发个友好的提醒,这样的效果会比较好。另外,他也提到,互联网是个强大的筹款工
具,迪恩的竞选就是个明例,但应该注意,给别人发Email时,要发给他个人,不要用群体发信,这会让别人感觉比
较舒服。

在他之后,是一位叫克里斯汀的女士来讲。她似乎是一位职业筹款者,上来居然就问:``你们是为克里筹款的
吗?''看来不像别人那样是民主党的支持者,而是GAIN请来做经验介绍的。

她首先要大家打消对筹款的负面印象。这倒也有理,比如我,如果要我去做选区组织之类的体力活,那肯定没
问题,可如果是要我去做筹款,要求别人捐款,那我就要踌躇了,总觉得那比较难开口。

克里斯汀让大家来说筹款的不好之处,最后总结出四大类:首先当然就是脸皮问题,看来大部分美国年轻人也
和我一样,觉得筹款比较难以开口。克里斯汀说,这需要我们不把筹款看成是个人的事,而是你为之筹款的组织的
事,这样呢,如果别人捐了款,你并不欠她人情,如果别人不捐,她也不是对着你来的。我想,她这样解释当然容
易,不过实际施行起来,我们还是很难扮演机器人或者出纳员的角色,总还是要把个人感情带进去的。

第二类困难,是难以面对别人的拒绝。牵涉到钱的问题,说起来总有些尴尬,被拒绝的可能性也比选区活动大
得多。失败的次数多了,难免会感到特别沮丧。克里斯汀说,我们要这么看筹款活动:这里没有失败,只有收获,
因为我们从空白的账号开始,慢慢地筹到了越来越多的钱,这都是收获,别人的拒绝,并没有减少我们的账号于分
毫,所以不必沮丧,要用积极的态度来看待。

大家担心的第三点是媒体。长久以来,关于筹款活动的负面报道几乎在媒体上从来没有断过,比如贪污、账目
不清。你为竞选活动募到了多少钱,媒体也不会关心,可一旦出了什么差错,他们可就马上像鲨鱼闻到血腥味一样
地扑过来啦。克里斯汀说,这些负面报道毕竟是少数情况,大家把自己的事情做好就行了,身正不怕影子斜么。

最后,克里斯汀和大家讨论了一些不好的例子,比如有些筹款人员会按百分比拿分成。可能是由于她的职业的
缘故吧,克里斯汀并没有觉得分成有多坏,只要双方都同意就行了。不过看来大部分美国年轻人和中国年轻人也差
不多,对于自己为之筹款的事业总是有着比较纯洁的信仰,所以觉得这难以接受。另一个例子是用筹来的捐款进行
商业投资,以冀得到更多的资金。这被她坚决否定了。

在开导完如何``放下包袱''后,克里斯汀便开始讲``开动机器''了。首先是预算。筹款本身也是要花钱的,比
如电话费、买卡片的费用,更不用说举办筹款晚会的花销了。这些都需要提前仔细筹划好。另外,打算筹到多少
钱,这个目标也应该早早地明确。

其次是实际操作。克里斯汀描述了筹款的四个阶段,一是所谓``种子资金'',来自于你周围最近的人们,如亲
戚、好友,他们总会愿意帮助你的,这一笔钱相当于启动资金,除提供了此后活动的花销外,还有点``开门利市''
的意思。二是以前的施主,这些人一般会乐意再次捐赠,因为他们显然是强烈支持此事项的。三是要化缘的,是我
们所知道的各种联系人、团体组织等等,这些人和我们的交情就没那么铁了,不过胜在范围大,积少总能成多的。
四是所能指望的,是``朋友的朋友'',要求别人也广为散发消息,将筹款活动进一步推广到茫茫人海中去。

然后她讲了些筹款的技巧问题。在和别人联系时,一定要把希望对方捐赠的数目说出来,不要说``您能捐多少
就捐多少'',而应当提前分析对方可能捐赠的数目,然后提出自己的要求。这里面当然也免不了讨价还价,所以开
价还是很重要的。有个听众说:``是不是估计对方能捐100美元,就先要求他捐200美元,然后再降下来?''克里斯汀
说,这也要看具体情况,因为开价太高,会把潜在的捐赠者吓跑的。

在别人捐赠、收到他们的支票后,应当立刻寄张感谢的卡片过去,不仅是表示我们的感激之情,而且也可以为
日后的合作打下良好的基础。如果条件不允许寄卡片,那么至少要打个电话,或者发个电子邮件、传真等等。

在筹款时,除了钱之外,人们有时还会进行``实物捐赠''。倒不是旧衣服、旧被面,而是指提供设备,比如电
话、上网、印刷、传真机等。另一类更隐蔽的``实物捐赠''是职业人士的时间,比如律师自愿提供法律咨询、计算
机工程师免费为你管理数据库等。

她讲完后,下午的日程也就差不多结束了,我们又回到大会堂。首先登场的是两个黑人青年,用音乐会上常用
的暖场方式,把全场的气氛都搞得热烈起来了。然后他们说起了``相声'',其实就是在轮流rap(源自早期美国舞厅
中的DJ现场表演的一种说唱乐),另一个人在旁边敲边鼓,其内容也不是搞笑,而是控诉布什政府执政下黑人的悲惨
生活。说到激动处,两人一起放声大叫:``啊---啊---我再也无法忍受了!''最后当然号召大家要把布什赶出白
宫,选克里做总统。

接着有人来介绍下一个演讲者:民主党全国主席特里$\cdot$麦可利(Terry McAuliffe)。看来他的人气超级
高,介绍者直接称他为``有史以来最棒的党主席'',他出场时,年轻人们都沸腾起来,全体起立,有节奏地高
呼:``特里!特里!''。

我以前只在电视上看到过他一面,是他驳斥副总统切尼对克里的攻击,顺便反击布什的军事服役记录,当时他
脸上青筋迸露,目露凶光,很是杀气腾腾。今天再看这位主席先生,却是神采奕奕,笑容可掬,而且风度翩翩,着
实是一个帅哥。民主党全国大会后天就要召开了,他大概是此刻全波士顿最忙的人了,能够抽空到这里来,应该是
脱稿的即席演讲,风度却极好,侃侃而谈,自信而又富有感染力。

他的演讲主题是称赞台下的年轻人,因为美国的年轻人向来有不关心政治的名声,虽然是民主党的一大铁杆,
却很少去投票,这次他看到会堂里济满3000余年轻人,自然感到大有希望。我们也被这位党主席拍得极是受用,不
断欢呼鼓掌。最后是万变不离其宗:``我们要把乔治$\cdot$布什送回他在德克萨斯的农场,我们要让约翰
$\cdot$克里成为下一届美国总统!''

演讲结束后,他甫一下台,便被记者团团围住,闪光灯不停地闪,话筒、录音器像鱼杆一样直伸到他面前。再
外面又围了一圈,是他的崇拜者们,要求和他合影。我们APAP便去提出和他合一张集体照,他听说是某个组织,很
爽快地答应了,伸臂搂住左右的人,合了一张影。

今天的日程至此全部结束了。本来APAP的流程表上说,晚饭也由GAIN提供的,但看来这显然是APAP弄错
了,3000多个年轻人都涌出大楼,奔向波士顿的大街小巷。我们商量了一下,最后决定去中国城吃饭,纠集了十几
个人,分乘几辆出租车,到饭店碰头。

晚饭间,大家自然也乘机联络了一下感情。我们这一桌8个人,竟有5个是华裔,不过只有我和另一位女士是从
中国大陆来的,另外有两位女士是从香港来的,还有一位则是美国出生的。至于现在在美国的住址,大家不是来自
纽约州就是加州,都是铁定的民主党票仓,只有我是从一个``摇摆州''来的。大家纷纷说:这回就靠你了。我当然
也不客气,回答说:``没问题。今晚我吃双份。''

饭后,我们又到一个酒吧里去坐了坐,聊到11点多,眼看地铁快没了,才赶紧回去。最后折腾到凌晨1点多才到
达住处,我给手机又设了闹钟,便赶紧睡下了。

\section{寓政于乐}

早上一觉醒来,已是9点。我大惊失色之余,忙拿来手机一看,原来昨天上课时,我把它设为会议状态,后来一
直忘了改回来,估计今天早上它在7点时狂震不已,我却依旧酣睡。

我匆匆漱洗之后,冲出宿舍楼,正好赶上由里吉斯学院开往最近的地铁站的校车。在车上我遇到了三个年轻
人,也是去海恩斯会议中心的APAP成员。我问他们:``你们也是起晚了吗?''他们正色说:``不,我们是想省钱。''因
为坐APAP去会议中心的大巴士,需交5美元,而校车是免费的,地铁我昨天坐过,是1美元。我看他们都是学生模
样,怪不得连4美元都要省,不禁佩服他们在如此经济拮据的情况下,还来参加这种活动。

到了地铁站一看,却收3美元。原来地铁从城里出来是1美元,进城却要3美元,不知是何道理,大概是想抑制进
城人潮吧。他们三人昨天就是坐地铁去的,毫无惊讶之色,买了票就进站了。我想,嗯,原来他们不是为了省钱,
而是俺的睡友,不然他们也应该再起早1个小时,以免误了早上的课。

我在地铁上和他们聊了一会儿,他们虽然看上去差不多,其实却分别是华裔(来自中国台湾)、韩裔、菲律宾
裔,来自佛罗里达、纽约和新泽西,都是大学生。我和轻车熟路的他们在一起,倒也省了不少找路的麻烦,从地铁
站钻出地面,很快就找到了海恩斯会议中心。这时已经10点了。

按照日程表,今天的第一课是从早上8点到9点,《交流101》,就算我及时起来,坐APAP的巴士也赶不上的。第
二课是从早上9点到10点半,题目是《``哦,右派''言论(The ``O'Right'' Factor)---反击右翼媒体》,这个题目
是从FOX电视台的金牌主持人欧瑞利(Bill O'Reilly)的节目``欧瑞利言论(The O'Reilly Factor)''来的。欧瑞利可
以算是全美最受欢迎的右翼电视台主持人了,因此也最遭自由派痛恨,没事就拿他来开涮。(【10月30日补记】最近
这位道貌岸然的保守派爆出骚扰女同事的性丑闻,而且还涉及极富想像力的变态性行为,真是有趣啊有趣。)

我也不知道今天的会议室分配是否和昨天不同了,仍然径直走到昨天的房间。一位女士正在讲如何与选民沟
通,她让助手扮一个选民,从听众里点出一位又高又帅的男大学生,来做现场演示。这男生刚开始讲,就被她叫
停,伸手在他背后一拍:``挺胸!''又一拍他腹部:``收腹!''

男生站好姿势,向``选民''介绍自己,并试图说服她投票给民主党,主讲者在旁边不断纠正他的用词、口气和
身体姿势,总的来说,就是要自然,不能太生硬吓着别人,也不能太绵软腻着别人。说完后,她让两人从头再走一
遍。结果男生恢复成习惯的歪斜站姿,被她又拍了两下:``挺胸!收腹!''拍完后,还惊叹了一声:``哇!你身材真好
啊!常锻炼吧!''

大家都哄堂大笑。男生有点腼腆地说:``对,我游泳、跑步\ldots ''有女生尖叫了几声,大家笑得更起劲了,
他索性也拍了拍主讲者的腹部:``你身材也不错!''

这一部分好像还是属于《交流101》,而非《反击右翼媒体》。我没来得及吃早饭,GAIN又不提供中饭,水桶到
处都是,我便溜出会议中心,到街对面店里买了几块巧克力,边吃边听。

课程结束后,大家又都来到大会堂,主办人先讲话,感谢大家的参与、所有工作人员的劳动、大会和课堂主讲
者们的演讲,以及来自包括民主党委员会等各组织的协助。接下来是一个关于如何使用互联网进行草根政治活动的
演讲,由两个人同时主讲。而说到互联网草根政治,举出来的例子当然是迪恩竞选团体。

迪恩在民主党预选前只是个名不见经传的佛蒙特州前州长,但由于后``9.11''时代美国人的爱国情绪高涨,其
他民主党总统候选人都为了迎合选民而立场向中靠拢,迪恩却依然高举民主党正统旗帜,赢得了很多民主党人的青
睐,使他在预选前的民意调查里遥遥领先。他们一改以前竞选团队看重大金主的传统,注重草根建设,用互联网把
成千上万的平民联系起来,平均每人只捐75美元,但由于人数众多,使迪恩成为预选中筹到最多款项的候选人。同
时,他们在使用互联网进行组织、发动方面也卓有成效,比如APAP的核心成员,就主要是以前迪恩阵营的。

不过,由于大部分民主党人认识到,今年的首要任务是击败布什,而不是实施``原教旨''民主党党纲,所以到
了预选的最后一刻,他们放弃了心爱的迪恩,转而选择克里,因为克里有越战经验,和有逃兵嫌疑的布什形成鲜明
对比。用他们自嘲的话说,是``约会迪恩,嫁给克里'',和我们中国人的那句``东家食、西家眠''颇有异曲同工之
妙。

今天的听众基本上都是些年轻人,他们显然更喜欢迪恩旗帜鲜明的立场,在提到迪恩和他的竞选团队时,都大
声尖叫鼓掌,看来很多人以前就是迪恩阵营的义工。毫无疑问,迪恩虽然失败了,但他对互联网的运用会大幅度改
变美国政治的运作方式,尤其是草根层次。这也不稀奇,互联网已经出现这么多年了,使用者越来越多,理应成为
影响越来越大的草根政治载体。

随后演讲的是来自加州的阿瑞安娜$\cdot$哈芬顿(Arianna Huffington)女士。她是一个专栏作家,在加州颇
受欢迎,去年加州州长罢免案中也想去搅一杯羹,结果自然不敌施瓦辛格的魅力。她是希腊裔,因此在讲话中倚希
卖希:``我们希腊人发明了民主,结果却被你们美国人糟蹋成这个样子!''台下的美国人不以为忤,反倒一起鼓掌大
笑。

她针对布什阵营总拿``9.11''来作为连任依据说:``共和党人总在提醒我们要记住``9.11''。是的,我们永远
也不会忘记``9.11''。可是,我们也应该想起``9.12''。在2001年9月12日,全国人民从震惊和悲痛中恢复过来,自
发地团结在一起,献血、捐款、追思,钱物和义工源源不断地涌向纽约。``9.11''代表了恐怖,而``9.12''代表希
望。布什希望人们把``9.11''和他联系在一起,我们则把克里和``9.12''联系在一起。现在我们要的,不是一个总
用恐怖来恐吓我们的总统,而是一个能带给我们希望的总统!''

这个比方真是打得太妙了,全场都起立热烈欢呼鼓掌。真不愧是专栏作家!

压轴演讲的是民主党预选时的总统候选人阿尔$\cdot$夏普顿(Al Sharpton)。他是来自纽约的黑人牧师,一
出场,大喇叭里就开始放一首节奏感极强的黑人歌曲``起来(Get up, Stand up)''。这首歌我在``新州草民''的网
站上也听到过,很是喜欢。夏普顿牧师的立场比较左,而且没有什么政治资历,是属于大家都早知道选不上,只是
来``重在参与'',乘机宣扬自己的观点。我在电视上看到过他的几次演讲,确实激情远过其他人,观赏性是很高
的,自然也很得这些大学生的喜爱,热烈鼓掌欢迎。

夏普顿先说,克里阵营打过招呼,在这次民主党全国大会上,大家不要攻击共和党,不要进行负面宣传,要开
成团结的大会、奋进的大会,所以他就不攻击布什了。但由于共和党最近爆炒一个说法:解放了黑奴的林肯是共和
党人,因此黑人应该投票给共和党,夏普顿说,共和党人当时还许诺给每个黑人40亩地和一头骡子,结果却从未实
现过这个承诺,``因此我们黑人决定,始终把票投给那个用驴子做代表的党!''

全场大笑鼓掌。驴子是民主党的象征,这个词ass同时又有``屁股''的意思,而俚语``kick-ass''字面上是``踢
屁股'',引申的意思则是``强有力地行动起来'',所以民主党的很多网站上都有一头在往后抛蹶子踢屁股的
驴,GAIN也干脆把它的这个训练班叫``Kick in Ass Training''。

大家的屁股踢完之后,GAIN的训练就正式结束了。晚上会有一场``投票摇滚''音乐会,下午则是自由活动。我
们APAP的几个人又一起杀奔中国城吃饭不提。由于民主党全国大会明天就开了,波士顿的街上到处都是民主党人的
标志,各种活动也此起彼伏,我们在中国城外就看到一些中国宗教信仰人士的宣传活动。

晚上APAP会在我们住的里吉斯学院办一次烧烤晚餐(barbeque),顺便社交联络感情,我们吃完饭后就坐地铁回
去。由于里吉斯学院的校车一小时才有一班,我们在地铁站等待,这时一个APAP成员忽然掏出一本记录簿来,开始
采访大家。他是出生在美国的台湾人,现在在加州当教师,业余为一家亚裔杂志写稿。他首先问清每个人的背景,
比如族裔、来自哪个州、来美国的时间,然后问两个问题:你为什么要来参加这个训练班?你对到现在为止的内容感
受如何?采访完后,还拿出相机来拍上一张照片,以供杂志采用。

除了我之外,唯一一位非美国公民是来自中国内地的女士,她80年代末考入北京大学,后来辗转来到美国。她
说:``我来这里,是因为我自己就是个种族歧视的受害者,后来在别人的帮助下,我把官司打赢了。但我认识到我
不能总等别人来改变现状,我必须自己也加入到这一过程中,所以我开始积极参与政治。''

其他人的回答大同小异,对目前训练的内容都还尚感满意,但我对GAIN的训练班太过流水化,全是原理、几乎
没有触及实际操作的内容很不满意,大放了一炮。当然,这也可能是苛求了,正如柯林昨天曾说的:``义工基本不
需要太多训练,只要常识即可。''本来,对于草根义工来说,恐怕只要付出时间就行了,没有什么了不起的技巧。

回到里吉斯学院后,大家各自归屋休整。很快就到了开饭时间,组织者已经把各种食物烧烤完毕,我们去享用
就行了。大家围坐在几张桌子旁,随意地聊天。最先成为话题的是今晚的``投票摇滚''音乐会。昨天夜里我们才从
组织者那里拿到票,是免费的,但组织者``建议捐款''25美元,当然大家都捐了。票印得很漂亮,最让人激动的则
是上面列出的嘉宾名单:比尔$\cdot$克林顿、希拉里$\cdot$克林顿、特里$\cdot$麦可利、阿尔$\cdot$
夏普顿,还有很多演艺界人士,比如《星球大战》的女主角娜塔丽$\cdot$波特曼(Natalia Portman,我更喜欢她
的另一部作品《Leon:the Professinal》(这个杀手不太冷))、政治搞笑节目主持人乔$\cdot$斯特瓦特(Jon
Stewart)等。

我很关心今晚的音乐演出,不过看那个名单,好像没有特别有名的乐队。不过那也没关系,能在摇滚乐的轰鸣
之下,和成千个年轻人一起共舞欢呼,只要鼓点够劲、吉他够疯,管他是哪个乐队的演出呢!

除此之外,最让我们激动的当然就是克林顿会来了。我已经听到数不清的人说,克林顿的演讲风度无与伦比,
与观众沟通的能力极为高超,我真想亲眼看看。我在电视里看民主党初选时,那几位候选人的演讲水平都不敢恭
维,当然他们的私生活可能都比克林顿强,施政能力也不见得就差,可在美国政治,演讲能力才是成功的第一要素
(布什的演讲能力貌似低下,其实只是不入知识精英的法眼,却极能打动他的基本选民)。

说到这几个民主党候选人,我忍不住抱怨说:``今年民主党预选里怎么就找不出个好的候选人来!要是再有一个
克林顿该有多好啊,我们就赢定了!爱德华兹有点像,可是他太年轻,没有经验,大家不敢把宝压在他身上。''

有人说:``迪恩不错啊。''我也忘了APAP的前身是``亚太裔支持迪恩''了,顺口说:``什么呀,迪恩太不可
靠,他的当选可能(electibility)太低,你瞧他在爱荷华州初选失败后喊的那一嗓子!''

这下可捅了马蜂窝了,三四个人同时对我说:``那是个阴谋!他们故意要整迪恩!电视台故意把麦克风的声音调
大的!''``在那么大场合下讲话,不大声喊能行吗?电视台抓住这么一声大喊,就在新闻里反复播放,给人造成一种
他狂热不冷静的印象!''``迪恩不是那么冲动的人,他是要给支持者打气!''

这些说法实在不能令人信服,我和他们争辩了几句,却难挡他们围攻,只好求同存异:``好了,好了,我们不
争这个。不管怎样,现在我们大家都支持克里了。''

``谁支持克里?''一个南亚大胡子不屑地说,``我们只是反对布什。''

我赶紧声明,我对那些初选候选人也没有看得特别顺眼的,只是反对布什,算是结束了这场内战。他们对迪恩
的爱戴看来是出于衷心。也难怪,其他民主党候选人的面目实在太模糊了,只有迪恩旗帜鲜明地代表了民主党的基
本教义。

随后我们开始聊些轻松话题。有一对才16岁的双胞胎姐妹,出生在美国,父母是中国台湾人,居然也让她们俩
来参加训练班,大家都很喜欢她们,对她们问长问短。

吃完饭后,大家登上校车,转乘地铁,浩浩荡荡地杀往波士顿的城里。票上说晚上8点半开门放人,我们为了能
够早点进去,占个好座位近距离看看克林顿,8点就到了举行音乐会的那家夜总会。可到了那里再一看,不由得大吸
一口凉气:那里早排起了长队,蔓延拖过好几个街区,怎么看都有上千人!

没有办法,我们只好站到队尾,老老实实地开始排队,后面迅速又加进了更多的人,使队伍越来越长。街的对
面就是波士顿的职业棒球队红袜队的主场,体育场内正在进行一场比赛,时而传来一阵山呼海啸般的欢呼,那一定
是红袜队得分了。这支可怜的队伍,自从80多年前把那时最伟大的球星贝比鲁斯卖给纽约扬基队、导致贝比鲁斯诅
咒该队再也得不到冠军之后,虽然实力雄厚,却真的一直再没得过冠军。今年他的调子还不错,要是能打破80多年
的诅咒,得一回世界系列赛冠军,那波士顿今年的风水就不是一般的旺了,克里没准也能乘回东风,赢得选举。
(【11月3日补记】红袜队后来倒是真的得了冠军了,可根据``运气守恒定律'',它把克里的运气也透支掉了。)

8点半到后,门口开始放人,速度却奇慢无比。我到前面去看了一下,夜总会门口站了两个穿制服的彪形大汉,
手拿金属探测器,叫人拿出身上所有的金属物,检查无武器后,还要核对驾照照片,才能放进去。大概是由于有克
林顿夫妇这样的重要人物吧,所以才这样如临大敌。不过以这样的效率,大概没三四个小时是轮不到我们的了。

好在队伍里也挺有趣,大家都是反对布什的年轻人,随便地聊成一团。有义工过来提醒大家都要提供驾照或其
他有照片的身份证件,一眼看见了我的驾照,是宾州格式,便和我认了老乡。原来他是来自``宾州大学生民主党联
合会'',大学就在离我家不太远的地方。他连忙给我发了名片,叮嘱以后再多联系。

当然也少不了抗议者。一位中年男士,身上挂了个大牌子:``约翰$\cdot$克里是个中了邪的孩子,愿主治愈
他。''两眼烧得通红、满脸义愤地在队伍旁游走,所到之处,自然都引起人们嘘声大起,或者大喊:``克里!克
里!''作为回答。

还有一队人,展示大牌子:``克里和布什是一丘之貉,纳德尔才是真正的选择!''这可真是哪壶不开提哪壶,民
主党正私下里埋怨纳德尔使布什偷走了总统宝座呢,难得他们自动送上门来,当然``若是那纳德尔来了,迎接他的
有嘘声''了。不过毕竟大家都属左派,所以还可以沟通,于是杀出很多人来,和他们辩论。

队伍虽然挪得慢,不过看着队里队外的这么多事情,也挺好玩的。我们唯一担心的,就是别等到我们终于进去
了,克林顿他们也讲过了话,早离开了。还好夜总会内一直在传出线报:克林顿还没来!其他名单上的人也一个还都
没有出现。于是我们安心等待,到了11点时,我们终于挪到了门口,经过严格检查,被放了进去。

这家夜总会其实不大,大概只能容纳上千人吧,听说今天却发出去三四千张票,所以挤得厉害。不过大部分人
都挤在主舞厅,那里有乐队在表演,但却不是摇滚乐,而是hip-hop,全场人都跟着节奏在扭动。我拎了一瓶啤酒,
也去扭了一阵,感觉还不错。

所有的人都在互相询问:``克林顿什么时候来?''却没人知道答案。乐队表演完之后,就撤了下去,全场开始有
节奏地大喊:``我们要比尔!我们要比尔!''

喊了一阵之后,有工作人员出来,回答说:``我们刚和克林顿夫妇联系过。他们正在赶来波士顿的途中,大约
在凌晨1点时可以到达。''大家便像吃了定心丸一样,又开始跟着音乐继续跳舞。我便和几个APAP的朋友找到另一个
小厅,那里有沙发可以坐下聊天,然后定时打电话跟主舞厅的朋友打听动向。

结果什么动向也没有。凌晨1点时我们又去主舞厅,大家这时已经骚动不安了。最奇怪的是,不光克林顿,其他
所有在名单上的人,除了一两个之外,也都没有出现,倒是不在名单上的迪恩,曾经来演讲过,可惜我们那时都在
门外,没有听到。

莫名地又等了近一个小时,忽然一个工作人员走上舞台,对麦克风说道:``我们刚得到克林顿夫妇的消息,由
于日程冲突,他们无法前来。''然后连个抱歉也不说,就走回后台去了。

大家自然牢骚飞起三丈多高,但也无可奈何,只好各自回家了。这时地铁、校车都早已停了,我们三四人一
组,打出租车回里吉斯学院,路上忿忿不已。我想,大概这个音乐会的组织者没啥经验,那些名人,有个影子的就
往票上写,其实却并未核实确认,最后才闹出这么大个乌龙来。

\section{亚裔社区活动}

今天早上一觉醒来,又是9点多!可是我昨天设定手机闹钟时,记得它不是会议状态的啊?拿过手机一看,原来它
没电了!

没办法,只好再度连滚带爬地洗漱、以最快的速度冲出宿舍楼。还好今天不用再去市区,而是在里吉斯学院内
进行亚裔社区活动的讲座,我在校区乱撞了一阵后,才好不容易遇到几个人,找到了APAP所借用的那栋楼。

上午的讲座分两节,每节都有好几类在同时进行,我挑了一个关于``政治游说''的讲座,冲了进去。主讲者也
是来自中国内地的留学生,到美国已经很多年了,我昨天和他在夜总会外的队伍里认识的。他对中美关系很感兴
趣,认为中国人应当组织起来,向国会施加压力,促进中美友好。今天他主持的这个讲座就是讲如何向议员和当地
政府游说的。教室里大约坐了五六个人,东亚、西亚、南亚的面孔都有,有人在谈进行当地公共服务的经验。不过
我们没谈多久就结束了。

下一节我挑了个关于``媒体攻势''的讲座。主讲者叫克拉克$\cdot$李(Clark Lee),是来自加州的民主党人。
他给我们发了名片,原来他还是某个选区的党主席,并且是加州民主党执行委员会成员、加州青年民主党人的通讯
干事。他来讲``媒体攻势'',当然是轻车熟路。

克拉克说,和媒体打交道要当心,首先要牢记两点:一是媒体不是你的朋友;二是一切都有案可查,千万不可
存任何侥幸心理。我们应当和媒体建立良好的合作关系,一旦你经常能供给媒体一些好的稿件或者消息来源,那么
以后他们也会主动来找你的。

媒体攻势的首要任务是使我们引起社会的注意,因此,如果我们有任何活动,比如会议、演讲、聚会、甚至人
事变动,都应当设法让媒体来报道,或者自己写了稿子,请媒体发表。当然,自己写稿时一定要采取中立客观报道
的立场,不能有倾向性。事实上,媒体并非高不可攀,他们也在整天满世界找稿件,所以对于送上门的报道或者事
件总是很欢迎的。

其次,要留心媒体上关于我们的报道,应当有一支``媒体观察(media response)组'',主要是关注媒体上对我
们的攻击、报道错误或者丑闻等。美国媒体都讲究平衡报道,一般来说,他们都会乐意给双方同等的说话权利,因
此,在遇到恶意攻击时,一定要坚决回击,澄清事实、阐明我们的立场。

如果有能力的话,我们也可以花钱在报纸上做广告。对很多亚洲语言报纸来说,可能这是最有效的方法,因为
他们大都是小报,报道能力有限,生存也较艰难。

除了报纸外,克拉克还谈到广播台。加州人有多家用亚洲语言广播的电台,这在我们宾州是想也不敢想的奢侈。
克拉克说,应当争取被电台的谈话节目邀请,让大众听到你的声音。当然,这里面也有技巧:这类节目往往会有和
听众的沟通,你要先安排好几个``托儿''打电话进来,提几个``托儿''问题,帮你进一步地阐明自己,也减少了敌
人捣乱的机会。大家听了都哈哈大笑。

讲座结束后,我们回到餐厅吃中饭。正是说曹操,曹操就到,我坐的饭桌上来了两个记者,一个是《世界日报》
的,另一个来自波士顿的当地中文报纸。我正要把刚听到的东西现炒现卖,她们倒先来采访我了,问题和昨天的大
同小异:你从哪个州来?为什么来参加这个会议?等等。我也没什么可以发挥的空间,只好老老实实地回答了她们的
问题。

采访完后,大家就随便聊了起来,原来她们都是从中国内地来的,说起来籍贯和我也很近。我顺便发了几句关
于昨天的音乐会的牢骚,《世界日报》的那位记者说:``昨天我在波士顿街头,看见一座旅馆前面挤满了女孩子,
我想,肯定是哪个大歌星或者大影星住在那里。一问别人,才知道要来的是比尔$\cdot$克林顿!这些女孩子都希
望能得到他的一个签名!''

由于是民主党全国大会期间,各地的民主党政客云集,于是APAP饭后的节目就是请一些亚裔政客来演讲,其中
有第一位华裔联邦众议员吴振伟。大家的讲话重点都在鼓励亚裔参与政治上,因为选举即将来临,他们也急需我们
的支持和帮助。

演讲完后是自由提问时间,我第一个举手,问了个我一直想弄清楚的问题:``我来自宾州费城地区,一直想在
费城的中国人社区做些草根工作。我想很多中国人很关心台湾问题,那么他们可能会问我:克里和布什的台湾政策
有什么区别?请问吴议员,我该怎么回答?''

吴振伟回答说:``这个细节我也不清楚,你可以去问克里阵营的工作人员,比如费城民主党部,或者上他的网
站,应该能找到这方面的解释。''

会后休息片刻,我们继续进行亚裔社区活动的讲座。不过,主讲者涉及的细节并不多,大部分是在讲述我们参
与政治的必要性、为什么应当支持民主党。有一位韩裔主讲者说了个很有趣的细节:韩裔美国人的政治捐款积极性
特别低,因为他们想不通,我投票给你,为什么你还要向我要钱?在韩国,都是政客付钱给选民投票给他,哪有倒过
来的道理?

讲座都结束后,日程表上是进行``地区沟通'',即和来自本地区的人联络感情、商讨未来的计划。我是算在新
泽西区的,因为来自宾州的就我一个。不过我从组织者那里要到了APAP费城地区负责人尼娜(Nina Ahmad)的电话号
码,打算回费城后就和她联系,加入到热火朝天的亚裔社区草根政治中去。

晚上在城里有一场亚裔晚会,我们APAP的成员也都得到了一张免费的票。我们吃过晚饭后,便往城里进发。不
过民主党全国大会已经开始了,城里封了好几条路,我们只好步行了十几个街区过去。我当然无所谓,那一对台湾
双胞胎姐妹还穿着高跟鞋,可就不方便了,拉在后面。

我陪她们慢慢走,顺便聊起了昨天的音乐会,她们比我们的怨气还大:``我们排到门口,出示了身份证后,夜
总会说,你们未满18岁,不准进入。我们只好早早地就回来了。''我说:``那其实是好事,总比我们被骗进去,坐
在那里苦等了3个小时好吧。''

她们说:``可是我们都交了25美元啊!早告诉我们18岁以下不让进,我们也不会浪费这笔钱。''她们俩还在上高
中,父母的管教好像比较严,一方面鼓励她们来参加这种活动,另一方面却在财政上控制得很紧。

我半开玩笑说:``你们毕业后,也别念书了,回国去演艺界发展算了。''她们听了一个不作声,另一个回答
说:``对,大家都这么说。''看她们的神情,好像她们自己也挺向往的,只是父母不允许。

晚会在波士顿科学博物馆进行,门口大厅有个巨大的电视屏幕,正在全程报道民主党全国大会。我们来得比较
早,晚会还没有开始进场,我们就站在大厅里,先后看到了阿尔$\cdot$戈尔、吉米$\cdot$卡特的演讲。

晚会的主办者是民主党全国大会的亚太裔委员会主办的,每个来宾都发了一本印刷很精美的介绍材料,第一页
是主办者的欢迎辞,随后四页分别是克里、爱德华兹、议会亚太裔会议主席本田(Mike Honda)、特里$\cdot$麦可
利给亚太裔支持者的信,第五页上是参议院民主党领袖、众议院少数党领袖、波士顿市长、吴振伟的四封信。接着
是对民主党全国大会日程的介绍,把四位主讲者的照片都印了出来。

后一半才是晚会的内容,包括表演者和演讲者的名单,大部分我都不认识,只有一位主持人蔡明昊(Ming
Tsai),看了眼熟。后来他出来前别人介绍他时,我才想起来,他是美国的明星厨师,常在电视上看到的,还曾被
《娱乐周刊》(Entertainment Weekly)评为美国最性感的人之一。

不过大部分讲话都了无新意,基本上都是这几天听得耳朵起茧的内容。只有一个``广岛乐队''的表演还不错,
这支乐队的成员黄、白、黑皆有,乐器也是日本大鼓和萨克斯风齐奏,中国古筝与西洋吉他共响,既有东方音乐的
神韵,又有西方音乐的丰富,甚是好听。

博物馆的地方并不大,主讲台前只有一小块空地,不过我来得比较早,而且对这个又比较感兴趣,占了正中的
位置,坐在地上听。他们正在演奏时,忽然一阵骚动,大家都纷纷往外走。原来电视里克林顿开始演讲了,我一
想,他的演讲错过了下回还可以看录像,这么精彩的现场乐队表演错过了,下次可就不知道了,因此没有走。

克林顿讲完话后不久,特里$\cdot$麦可利忽然站到我们这边来,发表演讲。我对他的精力很是佩服,民主党
全国大会刚开完,就继续串场来了。当然又有一大群人围住他要合影签名不提。

麦可利走后,我对其他演讲也不感兴趣,就在博物馆里逛,顺便和人聊天社交。晚会开到1点多时还没有结束的
意思,我们几个人看看也没有什么值得听的,就自个打的回里吉斯学院了。

\section{革命女性大会}

今天不再有APAP的训练,我们去市里参加革命女性大会。早上大家收拾完房间,把钥匙扔到楼下的盒子里,就
告别了住了四晚的里吉斯学院。

我们一行共有五人:杨蕙、中国台湾的双胞胎姐妹、来自南卡的汤姆和我。我们坐校车、地铁来到市中心时,
已是中午,下午的会要开到五六点,大家就先找了个饭店吃饭。在饭间,我们又聊起了政治。

这次的话题是工作外包(outsourcing)。很显然,我的自由党人立场马上陷入了她们的围攻。事实上,对丢失工
作机会的炒作,是克里阵营至关重要的一招,反对工作外包也是民主党人的基本立场。我当然不能同意这违背自由
市场规律的举动,我说:``你们不能光看到工作机会的丢失,你们也要看到丢失的是什么样的工作机会。制造业被
转移到海外去了,美国人才能轻装上阵,继续领先推动IT业。这是有利于美国经济的,使得美国人及时告别旧的夕
阳工业,总保持世界范围内的火车头地位。''

她们说:``可是,你不能光看经济的整体,你要为那些丢掉工作的人想一想。他们上了年纪,无法再学新的东
西,旧的工作丢掉了,新的工作找不到,怎么办?''

我说:``IT业和其他国家正在纷纷涌入美国的金融业,能创造比制造业多得多的财富。一个中国制造的布娃娃
值多少钱?一张微软的光盘值多少钱?华尔街的一小时服务费值多少钱?这些人赚的钱不会扔到水里去,必然是消费返
回社会,那么社会就必然又创造出新的工作机会,像饭店、车行。自由经济才能带来更多的繁荣,反对工作外包只
会使美国公司在世界上失去竞争力,最后大家共同贫穷。''

她们反驳说:``我们并非要求完全阻止工作外包,只是主张应该恰当保护美国工人,把某些工作机会留给美国
人自己。''

我几乎叫了起来:``为什么呢?为什么要强留工作机会给那些不合格的人呢?难道一个中国打工妹,只因为生在
中国,就不能和美国工人竞争吗?''

那一对双胞胎姐妹虽然小,说起道理来却也一套一套的:``可是政府有保护本国人的义务。''我常在自由党人
的Email里看到对政府的批评,一听到``政府''两个字,想也不想就顺口说:``政府不能解决问题,政府本身才是最
大的问题。''

她们马上给我盖棺定论:``那你是共和党。''我辩解说我是自由党,又解释了半天自由党和共和党的区别。

吃完饭后,我们继续往会址走。就在女士们快要叫苦时,我向一个警察问路,他一指街对面:``看见那辆大巴
士了吗?那就是开往革命女性大会的会址的。''我们谢过了他,坐上巴士,松了一口气。

这次大会共有4000多个参加者,大厅里早排起了长队。每个人都发一个包,包里除了这次大会的材料外,还有
一本女性主义杂志、一小袋化妆品、一小筒巧克力、一件T恤。希拉里$\cdot$克林顿、奥尔布赖特等人的压轴演
讲在一楼,首先是在二楼各个会议室的讲座,共有11个:

* 做一个女性参议员的真正感受

* 做一个女性众议员的真正感受

* 州长观点:我们掌握着问题关键

* 女性与政治媒体

* 内部参考:民意调查、政治咨询和竞选管理

* 女性、工会和她们的政治影响

* 发出我们的声音:让女性投票

* 国会山的新面孔:竞选议员的女性

* 青年行动:年轻人如何参与政治

* 入门:如何开展当地政治

* 女性与公共政策:2004年热点问题

杨蕙对中美关系、国际政策很感兴趣,我就和她去听了``女性与公共政策:2004年热点问题''讲座。这个讲座
共有四位主讲者:乔治城大学公共政策学院院长朱迪$\cdot$费德尔(Judy Feder)、美国健康保险组织的艾莲
$\cdot$古隆贝克(Ellen Golombek)、加州联邦众议员简$\cdot$哈尔曼(Jane Harman)、克洛迪亚$\cdot$肯
尼迪将军,其中前两位主讲健康保险,后两位讲外交政策,都是今年的热点话题。

讲座结束后,我们连忙赶往一楼大厅,不过已经晚了,靠近讲台处坐满了人,我们只好坐在后面。大厅边上摆
了很多桌子,都是形形色色的妇女组织,给过往行人分发材料、吸引新会员。大会里人来人往,热闹非凡。最有意
思的是三位装扮成自由女神的姑娘,浑身都漆成铜绿色,手举金黄色的火炬,在大厅里游走展览。我也和她们合了
张影。

介绍演讲者的是迪恩,他一出场,顿时全场起立鼓掌。演讲者中最引人注目的是前国务卿奥尔布赖特、众议院
民主党领袖南希$\cdot$佩罗西(Nancy Pelosi)和希拉里$\cdot$克林顿。在介绍希拉里时,迪恩开玩笑
说:``下一个演讲者不用介绍\ldots ''

大会组织者又向大家介绍了在讲台右侧就坐的波士顿女性政客,然后开始轮流演讲。杨蕙她们需要赶回纽约的
巴士,因此听了一半就走了。我的飞机是晚上9点,便留了下来,听演讲之余,也饶有兴趣地观察希拉里的举动。

这革命女性大会都不是外人,演讲者们也就有话直说,至少有两人提到,希拉里将成为美国的第一个女总统。
每当她们这么说时,希拉里便将手一摆,一副``别开玩笑了''的神情,笑着别过脸去。其他时候她都认真谛听的样
子,但我注意到,凡是别人赞扬克里时,她会和大家一起鼓掌,演讲者攻击布什时,她却微笑不动,虽然听众爆发
出来的掌声或许更响亮,只有一次例外,是在演讲者提到布什的政策危害了妇女权益时。

终于轮到她讲话时,好多听众都走到台前给她拍照。还有人站在过道,以正在演讲的希拉里为背景,让别人给
她们``合影''留念。

希拉里的演讲很不错,风度和表达上佳,内容则紧扣民主党的两大法宝:妇女儿童权益保护和健康保险,顺便
也批评了布什一下。我这几天听到的其他演讲者,简直都把布什当会场春药使,不论何时何地,只要抛出一句``把
布什赶出白宫'',立马全场欢声雷动。相比之下,希拉里的批评算是不太严厉的。当然最后万变不离其宗,她也不
忘记表明立场,声言相信如果克里当选总统,一切都会好转。这种正面赞扬的话,她怎么说都可以,听众也报以掌
声。

演讲结束后,大家都涌往台前,我也向前走,希望有机会和希拉里握手合影。结果她们一溜烟地就下台从旁边
过道里走了,一个歌手带领乐队蹦上台来,开始演唱,我也就正好停在台前听。

有人忽然传过一叠小长卡片来,我拿过来一看,上面印了个避孕套的图画,套上写着:``请使用它''。我转头
看看,四周基本上都是小姑娘,便把卡片往裤袋里一塞,面不改色地把剩下的传给别人。

到了6点,我就离开去飞机场了。结果今晚又赶上暴风雨,等了3个小时才上了飞机,到家已经凌晨1点了。我在
网上发了个贴,套用那个著名的``万事达卡''广告,算是这次波士顿北游记的总结:注册费:50美元飞机票:170美
元APAP的T恤:20美元和几千个年轻人一起上课、拉励、派对:无价!

\section{自由党人通信集}

这个题目是抄自《联邦党人文集》和《亚当斯与杰斐逊通信集》,起得虽然吓人,其实不过是讲讲我们蒙郡自
由党人邮件组上的一些``通信''。非是小子僭妄,敢比诸先贤,实是这个名字取来方便耳。

和大部分互联网时代的组织一样,我们蒙郡自由党人也建有一个自己的邮件组,大家在上面讨论事务,发布消
息,征求帮助等等。这给大家带来了很大的便利,几乎所有的活动都是通过Email组织起来的。我们还专门有一位志
愿者约翰$\cdot$海利(John Haley),平时负责收集材料和文献,发在邮件组上供大家参考。出于美国人幽默的天
性,自然笑话也少不了。我且在这里发几个。

一 1984\ldots 呃,我是说2005

``9.11''发生以后,乘着反恐的东风,国会迅速通过了一个《爱国者法案》,授权政府、情报部门(如联邦调查
局(FBI)和中央情报局(CIA))可以监视合法居民的通信、收集公民的各种记录资料,甚至允许有关机构越过固定程
序,搜查、逮捕居民。此案一出,群情大哗,有人认为,现在美国处在战争状态,我们必须让出部分权利,来获取
更多的安全,但民权组织则批评说,《爱国者法案》侵犯了公民的隐私权,违反了宪法所保证的公民权利。

自由党以个人自由立党,自然也不遗余力地反对这个法案。约翰在我们的邮件组上转来一个段子《1984\ldots
呃,我是说2005》,来暗示如果我们任由《爱国者法案》发展下去,美国将会成为乔治$\cdot$奥威尔的经典作品
《1984》里所描写的那个被无所不在的``老大哥''所控制下的国家:

2005年订比萨饼记

电话售货员: 感谢您光顾必胜客。您可以告诉我您的全国身份号码吗?

顾客: 嗨,我想订个比萨饼。

电话售货员: 先生,您可以先告诉我您的身份号码吗?

顾客: 我的全国身份号码,哦,等一下,呃,是6102049998-45-54610。

电话售货员: 谢谢,西汉先生。您住在米道兰大街1742号(1742 Meadowland Drive),电话号码是 494-2366,
您在林肯保险公司(Lincoln Insurance)工作,办公室的电话号码是745-2302,手机号码是266-2566。您是从哪个电
话打过来的?

顾客: 啊?我是在家里。你从哪儿弄到这些信息的?

电话售货员: 我们和整个系统相连,先生。

顾客: (叹气) 哦,好吧。我要订两个你们的全肉比萨。

电话售货员: 这可不是个好主意,先生。

顾客: 你什么意思?

电话售货员: 先生,您的医疗记录显示您有高血压和高血脂,您的全国健康保险提供商不会允许如此不健康的
选择。

顾客: 我靠。那你有啥可推荐的?

电话售货员: 您可以试试我们的低脂豆制比萨,我肯定您会喜欢它。

顾客: 你怎么知道我会喜欢这个东西?

电话售货员: 先生,您上个星期从当地的图书馆里借了《豆制品美食食谱》这本书,而且您的超市购买记录里
有四分之一磅豆腐,所以我做出这个建议。

顾客: 好吧,好吧,给我两个全家级的比萨饼。

电话售货员: 对您和您太太以及四个孩子来说,那应该够了。你需要付49.99元。

顾客: 我给你我的信用卡号码。

电话售货员: 对不起,先生,我想您必须付现金。您的信用卡已经刷爆了。

顾客: 你们把比萨饼送到之前,我会到ATM(自动柜员机)那里取些现金。

电话售货员: 先生,那也不行的。您的银行帐户已经透支了。

顾客: 你别管了,把比萨饼送来好了,我会准备好钱的。你们大概要多久?

电话售货员: 我们现在比较忙,先生,大概要45分钟。如果您着急的话,您可以在取现金的同时来把它取了,
不过在一辆摩托车上带着比萨饼,也许会有些尴尬。

顾客: 你他妈的!怎么知道我是骑摩托车的?

电话售货员: 记录显示,您无法偿付您的汽车贷款,因此您的车被取回了。可您的哈雷(Harley)摩托车却已经
付清了。

顾客: $@$\#\%/\$$@$\&?\#!

电话售货员: 我建议您注意您的语言,先生。您在2004年7月已经因为骂警察而被定过一次罪了。

顾客: (说不出话来了)

电话售货员: 还有其他什么吗,先生?

顾客: 哦,我有一张折扣券,可以免费得到两升可乐。

电话售货员: 对不起,先生,我们的广告里说了,这个折扣不能给糖尿病患者。

二 一封对布什的血泪控诉信

自由党是第三党,因此对共和党、民主党都没啥好感,只不过由于现在共和党在台上,所以对他们的批评更多
些。不过当有挖苦民主党人的好笑话出现时,自由党人当然也不会忘记拿出来跟大家分享。

我们知道,民主党人一直在批评布什政府让美国经济变糟,很多人失去了工作,失去了健康保险,甚至失去了
退休金。至于伊拉克战争,那就更是民主党攻击的一个好靶子。大家对他们的抱怨听多了,网上便出现下面这封信:

亲爱的先生:

我是一个老年公民。在克林顿政府期间,我有一个非常好的工作,我经常度假,而且有我的度假屋。布什总统
上台后,我的生活就越来越差了。我丢了工作。我的两个儿子都在那可怕的伊拉克战争中死去。我没了健康保险。
事实上,我几乎失去了一切,最后无家可归。

更糟糕的是,当局发现我住得像个老鼠一样之后,非但不帮助我,反而逮捕我。我愿意为布什总统的败选做任
何事;我愿意为支持克里参议员做任何事,以确保明年还是民主党人占据白宫。布什必须离任。

我想您以及您的朋友们也许想知道一个老年公民是如何看布什政府的。谢谢您花时间看这封信。

您真诚的萨达姆$\cdot$侯赛因三 主义之争笑话总是最受欢迎的,不过有时候也免不了会发生问题之争,甚
至主义之争。今年二月间,约翰转来了一首叫《非法诗歌》(Illegal Poem)的诗。他说:``我不知道这首诗的作者
是谁,但我觉得诗写得不错。''这首诗显然是在用搞笑的口吻,我把它试译如下:非法诗歌俺飞越重洋穷又苦,坐
上车去见雇主;那个好人对俺说,你得去找福利部。

福利部:您不用为这走一趟,我们会把钱寄到您府上;福利支票让您直奔小康,医疗保险让您身体倍棒。

渐渐俺的存款也不少,感谢你啊,美国傻冒;赶紧写信给国内亲友,这里人傻钱多,快来捞一票。

他们戴着头巾,成群而来,俺用福利支票买下豪宅;济济一堂俺们住在一块,福利越拿越多,日子越过越自在。

一下子来了14个家庭,邻居们吓得胆战心惊;最后白人都乖乖搬走,俺正好买下他们的房子草坪。

``再多来几个老外,俺有房子租赁'',在后院里俺支起帐篷;甭管是谁,尽管都来,来了的都能拿到福利一份。

事情正如俺们的计划,社区很快就成了俺们的天下;俺们的爱好就是生儿育女,福利负责他们的吃喝拉撒。

不怕没牙医,也不怕没药片,万事皆免费,不怕有帐单;美国人真是傻啊,花上那么多钱,建起个福利系统,
帮俺们梦圆。

这美国真他奶奶的是个好地头,好到那白人压根就不配享受;他们要是不喜欢俺们,尽管另去高就,巴基斯坦
正缺人要修地球!

诗末还附了一行:``把这首诗寄给你所认识的每一个美国纳税人。''

我当然完全理解,这首诗的攻击目标是福利系统。我作为自由党人,也对福利系统一向有所不满,可是这首诗
拿外国移民作靶子,令我很不舒服。我往邮件组写了封信,简单地表明了我的看法:我们自由党人相信,人们应当
有自由选择居住地的权利,我觉得这首诗的内容是和这个信条相悖的。这首诗的作者大概只是想攻击福利系统,可
是用移民来代表福利系统的受益者是不公平的。事实上,大部分非法移民都在从事着其他居民不愿意干的工作。他
们并不只是在美国领福利和生孩子。

很快,我们的主席吉姆同志就在邮件组回信了:我完全同意老摇的看法。我肯定约翰的本意是拿这首诗来攻击
社会福利系统。我妈妈就曾经把这首诗寄给我过。在我和她讨论之后,她同意说,这首诗不大让人舒服,而且是在
宣传一种错误的成见。

第二天,约翰也回了一封长信,信的开头说:首先,如果这首诗冒犯了任何人,请接受我的道歉。我并没有想
冒犯谁\ldots 任何认识我的人都知道,我不是个种族主义者或者顽固分子。

他表示,他转发这首诗是为了表达对社会福利体制的不满。这些都是自由党人的老生常谈了。同时他承认,他
同意我的看法,``贫穷的移民只是接受政府福利的人们中的一小部分\ldots 这首诗和自由党关于人们有权选择居住
地的信条相悖''。然后,他也指出:大部分移民是为着自由和机会而来到这里,但如果我们认为没有本地居民或移
民在利用我们的福利系统,我们也就太天真了。有些人就是不想工作,有些人如果能够白拿白吃,就比什么都开心。
有些福利接受者已经把它职业化成一门学问了。据报道,很多人会移居到新的城市、郡、州去,因为那里提供的福
利更好。常识告诉我们,在世界范围内一定也有人是出于同样的原因到美国来的。

这就是我和很多自由党人在移民问题上有不同意见的地方。只要这些政府福利系统还在那里,我们就必须严格
限制移民。正如我以前给别人信中所说:我们正在出口工作、进口人口。这个情形无法再维持下去。如果美国有真
正的自由市场,那么这里会有大量的工作,我们的经济将能够轻松地容纳新来的劳力。可是我们并没有自由市场。
我们的经济被政府所控制着,再加上一个实际上鼓励人们不劳而获的政府福利系统,我们很容易就能看出这个国家
正在陷入大麻烦。

自由党的朋友们,我们必须重新考虑本党的移民政策。在这个问题上,我们正在远离大部分美国人的想法,其
程度超过了反毒战争、联邦个人所得税、教育私有化、健康保险、退休计划(社会保险)等问题。由于``9.11''的发
生,大部分美国人需要更安全的国界。由于经济衰退,很多美国人担心移民会抢走他们的工作。保护我们的国界是
政府的一个合法功能。我们自由党人最好能提出一个关于移民的可行方案(一个可能性是让所有的移民签一份合同,
保证他们将永远不接受政府救济。我打赌他们中的大部分将会迅速成为自由党人,为缩小政府规模而奋斗。),不然
的话,仅仅这个问题就可能使美国人无法接受我们的方案。


最后,我再次向那些可能被这首诗冒犯了的人道歉。

我个人是喜欢看长信的,因为那更能完整地体现作者的意思,如果写得好的话,看好文章总是一种享受。不过
我一般不在邮件组发长信,因为我确信大家都不指望在Email里看到什么长篇大论,比如约翰这封信,恐怕真正看完
的不超过十分之一。于是我又回了封短信:

约翰,你不用道歉。我们都认识你很久了。我感谢你长期以来的工作,你为我们提供了很多好文章。

你可以从我的信中看出,我知道那首诗的作者是想攻击福利系统,而不是移民。我在福利系统上的看法和你一
样。对移民问题,我保留我的不同意见。在我看来,改革福利系统会比限制移民更好,因为:1、福利系统是问题根
源;2、非法移民是被向往更好生活的人性本能所驱使的;3、在非法移民的两个原因中,对福利系统的改革更容
易,也更公平。

再次感谢你为我们所做的工作。

信发出了,后来我才意识到自己并没有真正回答约翰的问题。他的论点是:``只要这些政府福利系统还在那
里,我们就必须严格限制移民。''也就是说,在默认福利系统无法改变的情况下,我们的务实方案是什么?我的回答
却是,先推翻三座大山,自由主义天堂就实现了,可谓驴头不对马嘴。

正当我犹豫是否要再回一封信加以说明的时候,吉姆也回了一封长信,写得很精彩:

开放国界?

我反对行业补助,但我不反对行业;

我反对外国捐金,但我不反对外国;

我反对福利,但我不反对穷人;

我反对全国教育协会(NEA),但我不反对艺术家。

我反对纳税人出钱养活移民,但我绝不会阻止一个平和的人进入这个国家。我不知道如果没有他们,我们的社
区会变成什么样。没有哪一个街区里没有移民在做生意。他们并不是来攫取税款的,他们是来纳税的---数目不菲
的税。他们不是来抢工作的,他们是来创造工作的---数量可观的工作(他们也构成了蒙郡自由党活动分子的一个重
要部分)。所以,上帝啊,请送来更多的移民吧!

即使同为自由党人,我们对开放国界的观点也常大相径庭,但是,当我用``无强迫原则(the non-initiation
of force principle )''来分析这个问题时,我只能有一个回答,那就是:一个人持抢站在国界或机场上,不让个
人自由地移动,是在进行强迫。

工会长期以来一直在推动反移民法案、进行保护主义宣传,他们不愿意和其他低成本工人竞争。媒体完全认同
这个调调,政府也乐得把他们的过错都推到移民头上。在加州,人们经常抱怨监狱里和公立学校里大量的移民,这
确实是个问题。我们需要先停止``反毒战争''和公立学校,那时,我们就会看到还有什么问题留下来了。

自由党人应该提醒人们,所有自愿的关系都是可以接受的。如果我们考虑一下镇、郡、州之间的边界时,我们
就会发现开放国界根本就不是什么复杂问题。国界凭什么就和其他边界有区别?我并不会害怕马里兰州的侵略甚于加
拿大的侵略(OK,新泽西人还是让我有些害怕的)。

正如约翰所提到的,在目前的形势下,开放国界仍然存在很多问题。我们需要在丢弃国界的同时,丢弃一些其
他东西,比如干涉主义的外交政策、贸易关卡、反毒战争,以及那些社会主义化的项目。一旦外国人不再恨我们,
开放国界会变得容易得多。现在我必须承认,这个仇恨还是很大的,但就在不久前,情形还不是这样的---自己查
一下,你就会明白。把事情复归原轨不会在一夜之间发生,自由党人现在已经习惯于现状。想像一下,如果大政府
的沉重负担被去掉后,将会有多少资本(包括人力资源和金融资本)流入这个国家。那时,企业主和富有创造性的人
们所面临的机会之多将是难以置信的。

我支持开放国界,因为这不仅符合道义,而且非常现实可行。其他任何方案都是在承认:那些政客有高于我们
的道德权力来选择谁能够进入这个国家。桥梁才能有益生意,壁垒不能。

听说过那些用一辆59年老别克车从古巴来的家庭吗?上帝啊,这就是美国所需要的天才!可你们所交的税却被用
来击沉这样一件艺术品,然后把他们送回给卡斯特罗。

``自由未来基金会''的这个网址:http://www.fff.org/immigrationproject/index.asp ,有一些自由党人对
移民问题的很好的看法。

我希望其他人也可以就这个论题发表看法。

随后约翰又回了一封长信,限于篇幅,我就不再引了。他的观点是,自由党人的理念确实很好,可是我们必须
回答美国人民的现实质疑,比如:我们怎么才能知道进入美国的人是平和的、有创造力的呢?投票人会对我们的移民
政策怎么反应?尤其在这个反恐战争正进行得如火如荼之秋。他说:``我们仍然坚持原则地认为,人们有权选择居住
地,但是我们需要做出如下解释:由于我们政府的干涉主义政策在全世界范围内造出这么多敌人,并且由于我们政
府的福利系统会吸引那些想不劳而获的人们,政府觉得有必要严格限制平和的人们的流动。除非我们解决了福利系
统的问题,并且停止在全世界内继续制造敌人,我认为我们仍然需要对试图进入我国的人们进行审查。''


这场主义之争至此就基本结束了,如我所预料的,在互联网上谁也不可能说服谁,重要的是发出了自己的声
音,说明了自己的观点。

我完全理解约翰的让美国人投自由党一票的盼望,不过,我必须老实地承认,我对所谓``让美国人接受我们的
方案''并不太感兴趣,因为我对美国的普通民众在近期内接受自由党观点是不抱希望的,与其改变本党的宗旨来迎
合他们,不如专心致志地发出真正的自由党人的声音,让公众听到了这一种新的想法,也就是我们的胜利了。时间
长了,他们自然会作出自己的抉择。``务实''当然是好的,可是如果真的要务实,我们全体加入民主党或共和党,
曲线救国岂不是事半功倍?

\section{纽约示威记}

今年的共和党全国大会(RNC),别出心裁地选在自由派横行的纽约。此选择一出,我的第一反应是:奇怪,一向
道貌岸然的共和党要开会,不去思想保守的中西部,不去民风淳朴的南方,到这个罪恶堕落的大都市来干什么?难道
不知道纽约别的没有,多的是满脑子大逆不道、与主流社会格格不入的嬉皮愤青吗?后来,民主党在波士顿开全国大
会的时候,我听人解释说,这是因为共和党全国大会的日期(8月30日到9月2日)接近``9.11''周年,所以特地选在纽
约,让人们用脊髓就能回忆起当时的情形,把布什和``反恐总统''的美号联系起来,从而帮助他连任。

这个策略也许可算高明,可是其中的心机实在令人不敢恭维。当初,我听到民主党阵营有人提出克里暂不接受
提名,待一个月后捞足了钱,再正式接受提名的主意,也曾有此感觉,还好克里没有这么做。相比之下,共和党的
这一招可让人更难受,对``9.11''``资源''的这种利用,实在是走了下乘。

我从波士顿回来后,一次,偶然地在网上看到一个叫和平正义联合会的组织(United for Peace and
Justice),号召大家在8月29日,也就是共和党全国大会召开的前一天,去纽约示威,以示``纽约不欢迎共和党全国
大会''之意。这个组织,顾名思义,是个左倾反战组织,提出来的口号也就是``向布什计划说不!''。看来``说不''运
动现在全球流行,革命形势一片大好,不是小好。他们的准备工作做得很细,连中文传单都有,号称将是纽约历史
上最大的一次游行示威。

这个提议正中我的下怀,反正是星期天,不去何待?到了周末,我便起了个早,杀奔纽约而去。当然,顺便美美
地在中国城吃了一顿早茶,再坐地铁,来到市里的联合广场。

从地铁站里一钻出来,地面上的热烈气氛就扑面而来:联合广场早已人山人海,各路英雄都打出旗号,招摇过
市。果然,绝大部分火力都是冲着布什和伊拉克战争去的。像这个标语,故意把Bush的S写作卐,显然是指布什是法
西斯。另一个标语则是反对征兵制的。

我正要过街,加入到广大群众的行列中去,一低头,却见布什``横尸''地上,而且还被别人写了咒语,镇压在
身上,以防他僵尸还魂,起来继续祟害人间。

这董卓级待遇的制作需要成本,还有部分群众因陋就简,随便写上两句话,往地上一摆,也是个造型:

我正看得起劲,忽然一阵鼓起,一队年轻人挥舞着大幅红旗,欢呼雀跃地奔了过来。为首的是个黑人小伙子,
大旗上是个红五角星,上面有个举枪的人影,看样子不像朝鲜人民军,也不像苏联红军,应该是位光荣的游击队员。
这队人一路跑过,红旗漫卷,战鼓振奋,煞是威风。跑到我面前时,他们正好停了下来,我赶紧向他们要了张传单。
那传单也是全赤红色的,上面标明了,他们是``共产主义革命青年军''。

读者可不要说我编造,就在广场旁边不远处,我还看到一家``革命书店'',可见美国的革命形势,确实是一片
大好,不是小好。

我在广场四处看热闹,只见各种标语,琳琅满目,有搞笑型的,比如下面这个``我仍然知道你上次选举干了什
么'',就是出自美国一个很有名的恐怖片《我知道你上个夏天干了什么》及其续集《我仍然知道你上个夏天干了什
么》。

当然也有愤怒型的,比如这位97岁的老太太。

街上不时有大队人马经过。美国人么,自然都少不了乐队。

在广场的一角,搭了个舞台,有乐队在表演。舞台下挂着大幅标语:``我对布什计划说不!''

侧面则挂着另一幅标语:``对世界宣战?不要以我们的名义!''

很快,游行时间到了,人潮开始涌往既定的路线。这次来参加游行的组织有好几十个,大多组织精良,有兵有
粮有乐队,声势浩荡。像我这样的散兵游勇,就只有四处串联,哪里热闹便往哪里钻了。很多人都举着标语牌,我
并没有准备,好在我很快就发现地上有人放了一堆``弹劾布什''的标语牌,便捡起来一张,顿时觉得走在游行队伍
中,也神气多了。

这次的游行既然是和平正义联合会组织的,``反战''当然就是最主要的口号。``要做爱,不要战争''这个最经
典的反战标语,我也在路上一再看到。

当然这个版本的``不要战争,要做爱''我更喜欢。图中的驴子是民主党的象征,大象则代表共和党,其中含
义,不言而喻。

与此有异曲同工之妙的是下面这张照片里,左边的一个标语牌:``不掉炸弹,掉裤子。''另一张标语牌上则写
着:``不要占领外国,要性解放''。还有一个人更有趣,拿了满满一筐香蕉,向游行的人们分发。至于香蕉代表什
么,想来大家都知道。

另一个反对扔炸弹的标语牌说:``送去资本主义,而非炸弹。''

有一个游行队伍搞抬棺大游行,虽然那些棺材都是纸做的,但触目之下,却也惊心。

在如潮的棺材林中,他们颇费心机地设计了一个大纸牌,上面是布什的头像,从他的口中源源不断地吐出写着
``恐惧''、``谎言''、``仇恨''的纸条。

有人则质疑布什政府发动战争的动机,她们问道:``一加仑(汽油)多少条命?''意思是布什政府发动战争的目的
是掠夺石油资源。


还有人拉着一个游动宣传栏,上面详细控诉着布什发动的战争所带来的后果,并直呼布什为``世界一号屠夫''。

也有人把布什当作恐怖分子来通缉。

把``Terrorism(恐怖主义)''一词开头的T去掉,就成了Errorism,英语里并没有这个词,从字面上分析,是
``错误政策''的意思。``结束恐怖主义''是布什政府的口头禅,这位示威者则提出``结束错误政策:把布什选下去!''

最惊心动魄的标语牌是这两个。一个打着美军虐待伊拉克囚犯的巨幅照片,下面写道:``得到民主啦?''背景用
着褪色的美国国旗,讽刺美国政府号称是为了解放伊拉克人民而去,结果却在大牢里干出和萨达姆一样的勾当。

另一幅就更让人触目惊心了,照片里是一个在战争中失去手臂的伊拉克儿童,上面的文字是:``W(布什的绰
号),谢谢你毁掉了伊拉克的危险武器。''而``武器''一词,在英语中,也指``手臂''。

这样的场合,当然也少不了和平鸽。

有一位女士则独具匠心地把自己打扮成死神模样,在她的衣服上、伞上却挂满了千纸鹤。

有一些示威者显然是激进的反美帝国主义者,他们把伊拉克战争和整个反美的第三世界都联系了起来,比如这
个标语牌上,将美国对伊拉克的占领和以色列对巴勒斯坦的占领相提并论,要求美国结束对以色列的支持。

还有一张标语牌要求五角大楼和中央情报局让海地前总统阿里斯蒂德回国,显然,他们相信海地发生的政变是
由美国暗中操纵的。

这张照片里的布什,不但佩戴了个纳粹徽章,如希特勒般地举起右手,而且还有个超长的鼻子,那是取材于童
话人物匹诺曹,那个可爱的一说谎鼻子就会变长的木偶。当然,拿着这个充气模型的人不是夸布什如匹诺曹般可
爱,而是说他撒谎。

还有人则举张布什的照片,在他鼻子的部位直接插根红棍子,也是说他撒谎的意思。正好旁边还有人举了张
``布什撒谎,谁人命丧?''的标语牌,补充得丝丝入扣。

除了关心第三世界的反美斗争外,资本主义国家本身也有人站出来反对布什,比如这个赤裸着上身的猛男,大
概是个瑞士人,脖间系着瑞士国旗。众所周知,瑞士是个中立国家,在两次世界大战中都置身度外,这位猛男打出
的标语便是:``连瑞士人都恨布什!''

这张标语牌则试图用地图诠释我们中国的一句老话:``得道者多助,失道者寡助。''可惜的是,其实支持布什
的地方远不是他的地图里所标出的那么少。

这次示威游行在很大程度就是一次对布什进行人身攻击的狂欢节。劳动人民的智慧是无穷的,美国人民的幽默
感也是无穷的。他们的标语牌都是五花八门、充满创意,比如这张:``德克萨斯州的某镇怀念它的那个白痴!''(布
什来自德克萨斯州。)

这张标语牌里的``chickenhawk''是个生造出来的词,``chicken''在美语里是常用来比喻怯懦,大意是
说:``把那个假装鹰派的胆小鬼送回猪场。''

说猪场,猪还立刻就到了。这张照片里的猪乍一看也还蛮可爱,还是睫毛长长的、粉红色的猪。

等它走过来,我们才看见它身上的标签:``GOP'',意思是``老大党'',即共和党的别称。

也有人拿着``Buck Fush''的牌子。从字面上看,这两个词没啥意义,不过,其实大家都明白它的真实意味。

好在美国别的没有,却从来不缺猛男,上面这位绅士不好意思说出口的话,自有别人大大咧咧地举着招摇过市。

反战人士大多认为布什政府故意说谎,引导美国进入了一场不必要的战争,除了给他安长鼻子外,他们还有其
他方法来暗示布什说谎。像下面这张照片上,布什浓妆艳抹,旁边写道:``涂抹得好!''

这位老先生则用箭头来标明布什和真相背道而驰。

这张宣传画是套用的那张人们非常熟悉的``你的国家需要你'',那是号召人们为国效力的。原版里的那个绅士
在这里变成了布什,文字也相应地变成了``我需要你'',后面又接上两行:``相信我的谎言。''

这个气球上写着``不要世界范围的愚弄''。这句话是化用了``世界范围的优势''这个词组,改变``优
势''(Dominance)一词里的两个字母,就巧妙地把它拼成了``愚弄''(Dumbinance)。

甚至连电话亭里都贴上了传单。

前来友情客串的还有一只小狗,它身上的``衣服''写着``把我从布什手中解放出来!''以及``2004年,布什舔一
舔。''

至于化装成布什的人,也有好几个。那个面具自然做得特别夸张。

还有更敬业的,除了化装成布什外,还拿了个充气的地球一抛一抛地玩,意思是:布什在玩弄整个世界。

再往下就有些少儿不宜了。有位男士把一个充气娃娃顶在头上,娃娃的背面写着``布什强奸美国'',正面则是
``布什真臭''。

最好玩的则是,我正走在人群中时,忽然后面有人说:``对不起,请给布什先生和迪克先生让个路。''我转头
一看,哗!果然是一个``迪克''和一个``布什'',摇摇摆摆地走了过来,边走还边和大家招手。``布什(Bush)''在英
文中是``灌木''的意思,那个绿色的植物就是指他。迪克(Dick)是指副总统迪克$\cdot$切尼,至于``迪克''一词
在英语中是什么意思,大家看了照片,自然也就明白了。


迪克$\cdot$切尼号称是有史以来权力最大的副总统,很多人认为他才是布什政府的真正政策制订者,因此人
们也``恨布及切'',对布什的愤怒,很多都发到他的头上。他曾经担任首席执行官(CEO)的哈里伯顿(Halliburton)
公司未经竞争就得到了重建伊拉克的巨额订单,切尼作为公司的前首席执行官也着实大捞了一把,这对切尼的名声
是个很大的污点。美国的家政女皇玛撒$\cdot$斯图尔特(Martha Stewart)由于在股票买卖上说谎,最近被判入
狱,有人代为打抱不平,举出标语牌说:``玛撒的牢房应当属于切尼!''看得我灵感大发,下次要有机会,我也要举
个标语牌:``窃钩者诛,窃国者侯''。

今天的太阳非常好,虽然是在曼哈顿的高楼大厦之间,也晒得人们满头大汗。我的策略是:每走一个街区,就
出来在背荫处乘一会儿凉,正好可以等这一批队伍走过去,加入下一批新的队伍---因为今天来游行的人实在太多
了,搞笑的创意也层出不穷,如果总跟着一群人走,未免有遗珠之憾。

好在每一个新团队的到来都是很好辨认的,他们往往都打有自己组织的大旗或者横幅,衣服也基本是统一的色
样,尤其是他们的团体不管人多人少,一套鼓乐是必然少不了的,只要听到远远的一阵鼓声喧嚣而来,就知道又有
一个新的团队赶上来了。随着行军的鼓声,团队里的人都踩着鼓点,摇摆着且舞且行,并跟着鼓声一齐喊口号。

比如这个团队:她们跳了一阵,就喊:``不要克里,也不要布什,给我们一个选择!(No Kerry, No Bush,
Give us a Choice, Give us a Choice!)''看来这是一个年轻人的激进组织,民主党和共和党一起反的主儿。果
然,她们很快就喊出了她们真正的声音:``一、二、三、四,布什克里去打仗!五、六、七、八,我们每个州都需要
纳德尔!(One two three four, Bush and Kerry go to war! Five six seven eight, We need Nader state to
state!)''很是合辙押韵。

纳德尔是美国著名的工人权利活动家,是比民主党还要左的左派,迹近于社会主义者了,因此很投合叛逆期的
年轻人心理。在2000年的大选中,如果不是纳德尔在好几州抢走了极左翼的选票,戈尔就已经赢得总统宝座了。为
此,民主党对他极有微辞,上次推他为总统候选人的绿党,这次也不再找他了。不过,纳德尔自己却越战越勇,今
年又作为独立候选人来参选,并且也确实在很多州都登上了选票。他的极左立场深受激进的年轻人欢迎,因此在今
天的游行队伍里也显得声势浩大。他们打出的标语牌自然是克里布什一起反,只有纳德尔才是正确的方向,如下面
这张照片里的``布什+克里=战争;纳德尔+加米欧(纳德尔的竞选伙伴)=和平'',以及用克里和布什的头像点缀的
问题``选择在哪里?''。

对于``选择''的问题,他们自然也早有答案。一阵鼓声之后,只听见他们齐声喊道:``我们需要纳德尔来发出
我们的声音,共和党和民主党都是一丘之貉!(We need Nader to give us our voice, Republicans and
Democrats there's no choice!)''

纳德尔的看家绝技是工人权利,当然他的追随者们也不会忘记在口号里表现出来:``钱要用在工作和教育,而
不是战争和占领!(Money for jobs and education, not for war and occupation!)''

这些琅琅上口的口号,非为他们所独有,所有的组织都有一套准备好的口号,还印在传单上,随着队伍的行进
到处分发,让其他散兵游勇也跟着一起喊。不过,纳德尔毕竟太激进了,虽然这些年轻人们激情十足,但大部分人
要么只是反布什,要么是支持克里的,跟他们一起喊的不多,举着克里的牌子的人倒不少。

总的来说,游行队伍里什么样的政治观点的人都有,甚至还有共和党人,但有一点是相同的,就是大家都反对
战争。比如下面这个举着``一个反战的共和党人''的老先生。

还有这一位:``内布拉斯加共和党人反对布什!''

共和党的一大旗帜是``爱国主义'',以至于任何反战的人,他们都给扣个不爱国的帽子。有人在标语牌中
说:``我爱我的国家,可我憎恨它的战争!''

我在游行中看到的一个很有趣的场景就是:一位穿着白色吊带上装的女士弯着身子,让一位男士在她的衣服后
背写上:``我是一个爱国者!''而我今天最欣赏的一个标语牌则是这句美国国父、也是我最佩服的美国总统托马斯
$\cdot$杰斐逊的名言:``异议是爱国的最高形式。''

这个观点很是符合我这样的自由党人的胃口,类似的还有这张焦距没有调好的照片上的标语:``政府即暴力!''

我也非常同意这个标语牌上的话:``一个愿意牺牲自由来换取安全的国家,必将同时失去、也不配得到自由和
安全!''

这句话当然是冲着所谓的《爱国者法案》来的,该法案乘着``9.11''的东风,在国会通过,授权``有关部门''
可以合法地侵犯公民的隐私权甚至人身权利,是美国各种人权组织所极力反对的一个靶子。下面这张标语牌也是冲
着它来的:``危险,人权法案!''

在这样的场合里自然也少不了其他要求权利的组织,比如同性恋,他们对布什是恨之入骨,因为布什总是摆出
一副虔诚的基督徒模样,引用《圣经》来反对同性恋,甚至曾经试图修改宪法来禁止同性恋婚姻。他们当然不会放
过任何反对布什的游行机会。


还有要求堕胎权利的,这也是布什的死敌。

堕胎在美国是个极为敏感的话题,双方都可以摆出无数理由来,就不用我再来评论了。这位女士对堕胎却有个
极为独特的视角,她的标语牌上的画面背景是一群看上去冷酷无情的中年白人男子,配着的文字说:``77\%的反堕
胎组织者是男人,他们100\%不会怀孕。''

最有意思的是这位女士,身上的女权口号T恤居然还有中文。

参加游行的还有一些环境保护主义者,他们挥舞着一面地球大旗。布什政府为了保护大企业的利益,退出了
《京都议定书》,这使环境保护组织极为不满。

美国媒体当然也不会放过这个绝好的机会,在路边经常可以看到他们抓住一个游行者,进行采访。

我正随着人群往前移动,忽然前面的人们都站住了,对着街边的一座建筑大发嘘声。再仔细一看,那座大楼上
挂了个横幅``欢迎共和党全国大会'',显然是个共和党代表下榻的旅馆。在大楼的阳台上还架着一个摄像机,往下
拍着游行的盛况。大家走到这里时都停了下来,对着他们大喊:``布什下台(No more Bush)!布什下台!''

不过,无论人们怎么群情激愤,阳台上那几个戴着墨镜的人,还是依然故我,不是脸无表情地看着人群,就是
自得其乐地在聊天。还好,没有哪个特别义愤的青年会冲进旅馆,打砸抢这些胆敢昧着良心和布什一伙做生意的卖
国商人。事实上,由于共和党全国大会的来者众多,又得到全国媒体的高度关注,众商家都是不退反进,纷纷抢占
这个大好商机。比如共和党开会的麦迪逊广场花园的对面,就有这么一个百威啤酒的广告,自称是``老大党''的选
择。

其实,纽约作为全美国的自由派大本营,显然大部分人都不喜欢布什,那一小撮唯利是图的坏分子,并不会影
响我们对纽约人民的总体判断。我们在游行中,可以不断地看到来自纽约人民的支援,他们在看热闹之余,晒太阳
的百忙之中,也有个别积极分子打出了自己的标语,比如这个挂在公寓楼上的巨大横幅:``拯救美国,打败布什!''

还有些同志就干脆走人身攻击的极端了:``我的灌木(指布什)闻上去像大便。''

一边走一边看热闹,慢慢地也就走到了这次游行的目标:麦迪逊广场花园,共和党全国大会的主会场。到了此
处,看到会场外面挂的大幅``欢迎'',以及``www.gopconvention.com''的大会网址广告,大家都像公牛见到了那块
红布似的,正是仇人相见,分外眼红,人们的情绪陡然亢奋起来,一起对着会场大喊:``布什下台! 布什下台!''

这次口号持续的时间特别长,不像在其他地方,喊了一阵子口号大家就懒了,我们在斗牛场里逛了一个下午,
等的不就是这一刻吗?当然不能轻易放过。人群在会场前站定,大喊了一阵``布什下台''之后,又开始喊:``无耻
(Shame)!无耻!''先是一起喊,后来站在我前面的一位女士蹦了出来,由她领喊,别人跟上,这样一唱一和,更有节
奏感。

如此喊了一会儿,群众又换了一个花样,一群人依旧喊:``无耻!无耻!''另外一半改喊:``你们不受欢迎!(You
are not welcome!)''这是针对会场前的``欢迎''横幅来的。横幅下站了不少人,有的是警察,剩下的看样子都是共
和党的工作人员。大家一边喊,一边伸出手来用力地指向他们,那架势,简直赶上偶像歌星的演唱会了。

喊口号还不过瘾,大家见那帮人无动于衷(在我们之前大概已经有好几万人在这里示威过了,他们有啥要动于衷
的也早该动完了),又改来嘘他们。随着嘘声,人们还举起双手,大拇指向下地挥动,意思是鄙视。

在一片嘘声中,我旁边的一位猛男开始大吼:``布什撒谎,万人命丧!布什撒谎,万人命丧!''他反复吼了几声
后,周围的人便也跟在后面一起喊,又成了新的一轮口号攻势。

这时,后面的人群已经赶上来了,他们在会场前站住,先来一轮``无耻!无耻!''算是炮火侦察。我们也跟在他
们后面喊了几句后,便继续往前移动,把地方腾给他们。在往前走时,那位刚才领喊的女士又开始喊:``谁的国
家?''大家答:``我们的国家!''一问一答,她喊得激情四溢,大家也答得坚决果断。再听后面的人群,口号也已经
改为``布什下台''了。

在会场前站着一些警察,不过我们和他们并没有发生任何冲突。大家只顾喊口号,对他们如同视而不见,而只
要我们不干什么出格的事,他们也不会管我们,因此双方什么接触也没有。

其实,在整个游行地段,到处都可以看到警察。不过,``和平正义联合会''既然有个``和平''的宗旨,当然是
极力主张要平和地游行的。本来,当初他们还计划在中央公园游行的,被市政府以``保护草坪''的理由否决了。由
于纽约市长布隆姆伯格(Bloomberg)是个共和党人,大家都怀疑这是故意刁难。有激进分子就声称将不顾禁令,仍要
去中央公园示威,结果``和平正义联合会''的组织者出来说,希望这是一次无暴力的游行,大家要遵守警方的指挥。
后来,从电视新闻看到,有一小部分人仍然去中央公园示威了,但由于人数很少,大概一百人都不到的样子,好像
也没有发生什么冲突。大多数人还是听从组织者的号召,在曼哈顿的市区里平和地游行。


我今天从中国城出发,一路都是向警察问路才到的游行地点,其实我的行头,一看就是来参加游行的,他们却
也没有对我另眼看顾,仍然把我指到了正确的地方。除此之外,我就几乎感觉不到警察的存在了,只有一次在乘凉
的时候,一男一女两个警察站在我旁边,那时我又热又渴,就问他们:``这游行什么时候结束?''女警察无奈地
说:``我也不知道,''她叹了口气,``我觉得一个小时前就该结束了。''看来他们也疲于差事。

离开麦迪逊广场花园后,整个游行就基本上结束了。这时,街上的一个福克斯(FOX)电视台的巨幅广告又让人们
发现了新的目标,因为福克斯在美国的各大电视台中是比较偏右的一个,今天前来游行的基本上都是左翼,当然不
会喜欢它,于是给它来了一阵``福克斯电视台,马屁精!(FOX sucks!)'',算是回光返照,功德圆满了。

其实,不仅福克斯,就连我觉得已经算偏左的有线新闻网(CNN),也有人不满意。他们打出一个标语牌,从美国
广播公司(ABC)、哥伦比亚广播公司(CBS)到有线新闻网、国家广播公司(NBC),统统是马屁精,而最最爱拍马屁的,
自然还是福克斯。

过了这个广告牌,再转一个弯,人群就越走越少,慢慢地稀疏起来,游行也接近结束了。在路边,可以看到刚
才敲锣打鼓的一个团队在休息,地上也是一片狼藉。

这时,街边却又出现了一群所谓``反示威者'',即来和示威者们叫板的。有人举着``支持布什总统、信耶稣''
的牌子,人们到了他们跟前,自然都会嘘他们一下,不过他们也见惯了,仍然高举着牌子,顽强地站在那里。可是
我总疑心他们是反讽,因为他们的标语牌上还印着各种武器的式样,我从来没见过任何标语牌以武器为正面符号的。
但我问了一下别人,他们都说:``不是开玩笑,是认真的'',而且起劲地嘘这些人。

另外有一帮人就确实是在搞笑了,而且这笑也搞得确实高明。由于今天是一次反战游行,大多数人是左翼甚至
极左翼,自认为是社会主流的右翼自然不满,挖空心思来嘲笑他们。比如这个标语牌,针对左翼的反战呼声,上面
写道:``战争不是答案'',中间印着左派青年奉为偶像的切$\cdot$格瓦拉的大幅头像,下面接着说:``除非你是
社会主义游击队。''这个讽刺真是绝妙,可惜我没看见那些来参加反战示威的``共产主义革命青年军''看到时会是
什么反应。

左派经常以``公平''为号召,把持其他政治主张的人不分青红皂白地都称为``极右翼'',下面右边这张标语牌
就搬了那张著名的签署独立宣言的油画,称他们是``右翼极端分子'',而油画里的人物包括托马斯$\cdot$杰斐逊、
本杰明$\cdot$富兰克林、约翰$\cdot$亚当斯以及其他美国国父们。另外两张标语牌分别写道:``极权主义才
害死了区区一亿人,我们应当再给它一次机会!'',``投票给绿党,这样我们就同样贫穷了!''看看左下角那个人笑
得多么开心,我们就可以知道他们对自己的讽刺是多么得意了。

这帮人办了个网站,叫``共产主义者支持克里(communistsforkerry)'',当然,他们不是真的共产主义者,更
不是真的支持克里,而是讽刺克里是个共产主义者。幽默的力量是无穷的,我最喜欢的标语牌是这个,将克里和马、
恩、列三大革命导师并列。我后来到他们的网站去看过,果然有趣得很,一切都是模仿苏联的惯用宣传模式,标志
便是那个四大革命导师头像,还有一面红旗,上面是俄文写的``Kerry''。

示威者走到这里,或者大放嘘声,或者跟他们争论,乃至对喊口号。我却觉得这些人着实有趣,正要上去搭
话,两个示威组织者却一路清场过来,对大家说:``我们不用和他们浪费时间,这些人不值得我们注意。大家往前
走吧!''我只好继续往前走。

不过,仍然有很多人留在路边,看后续部队的表演,我走了一阵之后,便也留在街边观看。在最后压阵的好像
是``和平正义联合会''自己,他们打出了数不清的``说不''的旗帜。

还有人抬出上帝来反战,因为布什总是以虔诚的基督徒自居,平时说话做事都好像是秉承上帝的旨意一样,甚
至敢冒``政治不正确''之大不韪,把他发动的战争叫``十字军''。这张标语牌说:``上帝是爱,而非战争'',说得
确实很好。战争本身的是非我们可以讨论,不过给战争赋予宗教意义,那又回到中世纪的黑暗年代了。

还有一位猛男,戴着拳击手套,光着上身,一副拳击手打扮,拉着一个布什的模型,走在游行队伍中。当然,
他不时会回身给布什几拳,以慰观众。

在游行队伍终于全部走完后,街上却摇摇摆摆地走来几个长衫飘飘的人,他们手持乐器,哼唱着一种奇怪的音
乐,听上去像是南亚音乐,给人以很和谐舒弛的感觉。

我很好奇地上去问他们:``你们是哪个组织?''

有个人给我说了一大通长长的名词,我也没有听明白,不过抓住了一个关键词:``Meditation(修炼)'',大概
是练瑜伽的吧。我更奇怪了:``那你们和政治有什么关系呢?''

那个人笑着说:``没有关系。我们来参加这个游行,走在最后面,只是想告诉人们,生活中不止有政治。除了
政治以外,还有其他重要的东西。''


是的,现在游行已经结束,大家正在作鸟兽散,我也该回到生活中去了。

不过,政治却似乎无处不在。我在找地铁站时,经过一条小街,街边有一家店正在从货车里卸货,一个戴着小
圆帽、穿着长袍的人忽然叫住了我:``你是去游行的吗?''

我说:``是的。''

他问我:``那你是支持布什,还是反对布什?''

我想,那还用问吗,``当然反对。''

``为什么?''

``这可就说来话长了,''我说,``我最反对的,是布什的泛道德主义和过度的宗教虔诚,好像他代表着耶稣一
样,好像他是正义的化身,反对他的都是邪恶轴心似的\ldots ''

那人似乎没听懂我在说什么,或者是见我说不到要害,急于把我引领到正确的辩论轨道上来:``你是因为反对
战争吗?''

我有些犹豫,因为我在伊拉克战争上的立场是很矛盾的,一方面我从良心上不可能反对一场可以除掉人民无力
推翻的独裁者的战争,另一方面布什把美国引入战争的借口实在让我倒胃,处理战后局势的方法更是黑幕共粗暴一
色,丑闻与血腥齐飞。好在不等我开口讲述这个复杂的问题,他已经主动出击了:``伊拉克战争难道不对吗?''

我说:``我也曾经支持过战争,但现在看来\ldots ''

他又打断了我的话:``如果没有战争,萨达姆还在台上,那你抬头看看天空吧,三天两头就会有飞机来撞大楼!''

我说:``那倒不对。事实证明,萨达姆和基地组织没有联系。''

``怎么没有?''他越说越激动了,``这些国家都应该打一打!美国还应该进攻伊朗、叙利亚、约旦\ldots ''

我再也忍不住了,问他:``你是犹太人吧?''

他说:``当然是!''

我无意再和他争执下去了,就说:``对不起,这个话题太长了,我不想再讨论了。''便走为上,转身回家了。

2004年9月---10月

\section{克里到阿伦镇拉励}

共和党的全国大会开完后,布什的支持率果然往上窜了一窜,在他原本领先的州拉大了领先幅度,在克里领先
的州缩短了差距,在几个摇摆州里,则开始反超克里。宾州是最关键的摇摆州之一,盖洛普的民意调查显示,布什
的支持率由此前的稍微落后,已转为1\%领先克里。

克里也恁般命苦,好不容易这次独立候选人纳德尔很可能被排除在宾州的选票之外(因为纳德尔没有能够筹集到
足够的有效参选签名),他刚可以松口气,布什阵营又祭出了他的镇山之宝---``快艇老兵寻求真相''组织。这个组
织由一些越战老兵组成,最近出了一本新书,称克里在越战期间的经历并不是那么英雄,并发布电视广告,攻击克
里在战场上的立场。虽然这个组织里的老兵并没有谁亲眼目睹过克里的作战经历,可他们硬是把故事说得有鼻子有
眼,任凭曾经和克里真正并肩作战过的老兵怎么出来辟谣,克里的支持率还是止不住地下滑了。

宾州对布什和克里都是极为重要的,可以说,谁如果赢得了宾州,就赢得了选票的一半。布什自上任以来,除
了他心爱的德克萨斯牧场外,来的最多的州就是宾州了。克里自然不能坐视布什来收揽宾州人心,他也多次来过宾
州。除了预选期间外,光在我们这附近的拉励,一个多月来就开了三次。

第一次是在7月27日,那时民主党全国大会已经开了两天,他在费城进行接受提名前的最后一次拉励,会后就直
接奔波士顿去。当时我正在波士顿参加亚裔组织培训,那天晚上才能回到费城,只好眼睁睁地擦肩而过了。

第二次是在7月30日,即民主党全国大会刚开完的次日,他也不休息,立即开始了一次竞选旅行,起点站就是我
们这里,中午在北边的斯卡鲁屯,下午在西边的哈里斯堡,离我住的地方都是约两小时车程。这一次的阵容很强
大,克里夫妇和爱德华兹夫妇四人都出动了,我当然计划要去参加。可是,我却犯了个难以置信的错误:我居然把
日期记成了下一周,可能是我潜意识地以为他刚来过这里,总要再隔一个星期再来,结果是错过了那样一次绝好的
一网打尽的机会。

不过我并不担心,因为宾州对克里来说太重要了,他一定还会再来的。果然,共和党开完全国大会后不久,我
就收到一封Email,是由克里的官方网站发出的,通知大家:克里将在9月10日到阿伦镇来拉励。阿伦镇也在我住的
地方北边,只有1小时车程,这次我不会再弄错,虽然那天是星期五,不过请半天假也是值得的。按照惯例,Email
里都包含有一个链接,我到链接所指的网站登了记,也就是写下了自己的名字、地址和联系方法,然后打印出了一
张票。这是免费的。在登记时,有个选项:``您愿意为这次大会做义工吗?''我也选了``是''。

到了星期五,并没有人来找我去做义工,看来他们人手足够。拉励下午3点半开门,于是我便在2点半出发了。
这次又流年不利,我走错了高速公路的一个出口,绕回来时,已经过了半个多小时,然后又遇上路上事故堵车,等
到达拉励场所时,已是4点半。一路上就看见人群不断地往会场的方向走去,大多数人,看上去是附近的居民,好像
只是来看热闹的,也有相当一部分人身穿支持民主党和克里的T恤,大概是从外地赶来的支持者;我还看到有两个女
孩子,拿着``布什---切尼2004''的大牌子招摇过市。

入口处早排起了长队,绕着会场转了一个弯,我才找到队伍的尽头。开门都已经一个小时了,居然还有这么多
人没有入场,看来今天来参加拉励的人还真不少。就在离我不远处,我又看到了那两个拿``布什---切尼2004''大
牌子的小姑娘,另外有个小伙子和她们在一起,三个人穿的T恤上也印有``布什---切尼2004''。我听到有人在嘀
咕:``她们来干什么?''显然,这些人是布什支持者,来挑今天的场子的。不过,大家虽然不满,和她们也只是一起
排队,相安无事。只有一个义工在经过队伍时,停留下来和她们交谈了几句,好像是询问她们为什么要支持布什。
这种问题,自然谁也无法说服谁。

过了不久,又有一帮穿着``布什---切尼2004''T恤的人经过这里,把她们叫走了。望着她们的背影,有人
说:``她们也太年轻了,还是高中生吧。''意思是不明白为什么她们这么小就也来``重在掺乎''了。但这群人显然
不会消失,过了大约10分钟,我们又看见她们大队人马从队伍的开始处一路游行过来,举着``布什---切尼2004''
的大牌子,一边走一边整齐地喊:``再干4年!再干4年!''

我们当然报以嘘声。我忽然想起了迈克尔$\cdot$摩尔在共和党全国大会上的回答:``再干两月!''意思是两
个月后的选举中,布什就要被选下台了,于是合着她们的节奏,喊了回去:``再干两月!再干两月!''

很快,大家就都跟着我喊了起来:``再干两月!再干两月!''我们的人数远远多于她们,一下子就把她们的声音
压住了。她们见势不好,便掉转花枪,不知道从哪里变出一双双拖鞋来,一边拍,一边喊:``墙头草!墙头
草!(flip-flop!flip-flop!)''这是共和党给克里度身定做的一个外号,指他的投票记录不能坚持立场,总在投机。
这下该怎么回答,我一时想不出来,看看队伍中的其他人,也都只能摇头或者不理。显然,这些布什支持者们是有
备而来,在口号战上我们防不胜防,我待要向摩哥电话请示回击策略时,她们已经走过去了。


宾州总的来说,是一个比较保守的州,民主党的支持者都集中在城镇,主要是费城和匹兹堡,乡村地区则是共
和党的天下。阿伦镇算是民主党和共和党平分秋色,所以共和党的势力也相当强大,对于来拉励的克里,报以一点
示威,也是正常不过的事。比起纽约人对布什的示威来,这些只是毛毛雨啦。

我倒是觉得这些示威活动挺好玩的,至少排遣了排队过程中的无聊。这天下午太阳挺好,路边又没有树荫,晒
得很厉害。我到旁边的商店去买水,结果发现都卖光了,只买到了可乐和冰绿茶。

``大家都有票吧?有红票或者蓝票的吗?''一个女义工从队伍旁走过,摇动着手里红色和蓝色的票问大家。很多
人都拿出自己的票来给她看,不过都是些白色的票,也包括像我这样从网上下载下来自己打印的票。她解释
说:``这行队伍是给拿白票的人们排的,如果你有红票或者蓝票,你就可以到另外的地方去排队。''

自然,红票或蓝票的队伍要短得多。这引起了大家的不满。她苦笑着说:``我自己就是工会的,我也痛恨这票
里居然还分等级制度,可是我们也没有办法,必须这么组织,请大家谅解。''

这时已经5点了,我们听见会场内传来一阵阵欢呼声,都不由得暗自嘀咕:是不是等到我们终于排完队进了场,
克里的演讲也完了,我们唯一能做的,就是目送他坐上巴士离开(如果谁带枪来的话,大概会忍不住开枪为他送行
的)。那位女义工赶紧安慰大家说:``不要着急,我们刚接到克里他们的电话,5点半之前他是不会开始演讲的。''

等到我们终于绕过街角,排到会场的大门前,顿时豁然开朗,队伍的移动速度陡然变快了。这是一个类似于体
育场的大会场,正中搭起了一个舞台,背面挂着一副巨大的美国国旗,对面是已经几乎坐满了的观众席。在观众席
和舞台之间有二十多米的空地,用栏杆隔开。这时我才看到所谓红票和蓝票的队伍,果然比白票短了许多。他们需
要通过一道安全检查的关卡,然后进入那个用栏杆隔开的区域,也就是可以直接站在台下听克里演讲。我们倒不需
要安全检查,可以直接奔到观众席上,但离舞台就相当远了。

待我找到个座位坐定,再打量四周,才发现在大舞台的左侧,还有一个小讲台。那个讲台上早架起了麦克风、
摄像机,后侧还站着几排支持者做背景,看样子那才是演讲者们讲话的地方。可是,大舞台才是对着观众席正中
的,这个小讲台就太偏左了,我们这些后进来的人基本都坐在观众席的右侧,离小讲台不足一两百米,上面的人头
都看不清。我只好暗自希望:最好演讲者还是到这个大舞台来。

再看周围的人群,倒是熙熙攘攘,各色人等俱有。绝大多数是白人,也有少数黑人,我甚至还看到一两张亚洲
面孔。大家的穿着都很随意,看起来大部分是民主党的支持者,说说笑笑,聊天打电话。很多人都带着孩子,会场
内飘着很多红白蓝三色的气球,有些孩子手里也攥着一个,整个气氛都很热闹轻松。

到了5点半时,人群一阵骚动,大家都站了起来,往舞台的左侧看。只见一队西装革履的人走上了那个小讲台,
向大家挥手致意。虽然我离那里很远,不过草草一瞥之下,也认出了里面没有克里。原来,这些都是先来暖场的民
主党人,大小各级名人都有,其中最引人注目的是众议员乔$\cdot$霍福尔,他正在竞选宾州参议员。

这几位民主党人开始依次讲话。坐在观众席右侧的人都积极反应,不时鼓掌或者鼓噪,却苦了我们这些离小讲
台很远的人,不但看不清楚人影,那几个大喇叭也摆得离我们很远,使我们连他们的讲话都听不清楚。台上仍然讲
得起劲,台下已经开始大喊:``我们听不见!我们听不见!''

可看来双方确实离得太远了,我们听不见他们,他们也听不见我们。那几个演讲者继续演讲,右侧的听众继续
鼓掌,我们也只好安静下来,听天由命,努力捕捉喇叭里的词句,能听懂多少算多少了。

好容易他们讲完了,组织者为了防止冷场,组织大家造人浪,又往台下扔印有``克里---爱德华兹2004''的
T恤,有时候是手扔,都被近处的人得到了,有时候则用超级粗大的橡皮筋往外射,倒能飞很远,不过离我们这边还
是差了十万八千里。

扔完T恤后不久,忽然小讲台那边传来一阵欢呼,大家又纷纷站起,再往那边看时:有一个人小跑着上了讲台,
高瘦身材,灰白头发,正是约翰$\cdot$克里。他上台后,立即脱下了西装,穿着衬衫上阵,挥手向全场致意。场
内发出山呼海啸般的``克里!克里!'',一时气氛热烈至顶点。

宾州的民主党籍州长埃德$\cdot$伦德尔先来铺场讲话,自然是把布什大大地攻击了一顿,把克里大大地赞扬
了一番。克里自己讲了大约45分钟的话,主题仍然是攻击布什。当然布什可以攻击的地方也确实不少。克里先从健
康保险开始,这是民主党人的老生常谈,我看到今天来参加聚会的老人不少,这个开头肯定满足他们的口味。

布什的全名是乔治$\cdot$W$\cdot$布什,他的父亲老布什全名则是乔治$\cdot$H$\cdot$W$\cdot$
布什,两人只有中间名不同,所以人们常用``W''来称呼小布什。在共和党大会上,就有妇女举着``W for Women''
的牌子,意思是``W帮助妇女''。克里今天也拿W来做文章,声称``W意味着错误的选择,错误的方向,和对美国的错
误的领导'',因为``错误''(wrong)一词在英文中也以W开头。


克里随后又攻击了布什的攻击型武器政策。在1994年,美国通过了一项禁止攻击型武器的禁令,为期十年,克
林顿签署生效了这项禁令。十年弹指一挥间,到了下周一,这项禁令就过期了,布什没有推动更新它,而是坐视它
过期,因为共和党相信人民有配枪的权利。但由于美国此前发生过多起校园枪击案,所以美国父母普遍担心孩子在
一个枪支泛滥的社会里的人身安全。今天来聚会的母亲很多,看来克里这个话题也选择得不错。

当然,要攻击布什,克里不会漏掉伊拉克。最近,美军在伊拉克的死亡人数已经过千,而伊拉克的局势目前仍
像泥潭一样,遥遥不见解决之期。除了人员伤亡之外,巨额的军费支出也使布什政府创造出有史以来最高的财政赤
字,并使布什在竞选时许诺的教育计划成了空中画饼。克里的每段讲话的结尾,都是``W意味着错误的选择,错误的
方向,和对美国的错误的领导'',将布什攻击得体无完肤。

由于我坐得离他太远,虽然能勉强听清楚他在说什么,但却看不清楚他的样子,模糊的印象是觉得他的演讲水
平并没有显著提高。对布什的攻击是很猛烈了,但他自己的对策并没有清楚地提出来,这不能帮助他吸引到中间选
民。

演讲结束后,克里走到台下,和众多的支持者们握手---但那自然是拿蓝票和红票的人的特区。大部分人都开
始离场回家,我逆流而上,走到蓝票区的入口,希望等里面的人走出后,可以进去一下。守在那里的除了警察外,
还有一个身材高大,穿黑色西装、戴墨镜的壮男,一看就是做安全工作的。我问他:``呆会儿我们可以进去吗?''他
说:``除非等到克里先生离开了。''我想:也是,这里面的人都是经过安全检查的,自然不能让我们这些人随随便
便的进去。没有办法,我只好也顺着人流回去了。

这次拉励给我的感觉不太好,组织有些混乱,比如把讲台放到遥远的右侧,喇叭也没有设置好,使左边的观众
既看不清楚,也听不清楚。至于这红白蓝三色票的设计,就更让人不舒服了,虽然大家也都能理解。在排队的时
候,我和旁边的一位老年人聊过一阵,对目前克里在宾州乃至全国的支持率都落后于布什感到担心,我说:``看来
共和党这个大会开得还不错,把布什的支持率给挺上去了。''他说:``是的,共和党明显组织得比民主党好。你看
他们的攻击做得多到位。''他是多年的民主党人了,他告诉我说,民主党的组织一向就不如共和党。确实,不论是
开这个拉励,还是全国大会;不论是为本党候选人树立形象,还是攻击政敌,好像共和党都比民主党做得更好。我
想,这大概是由于共和党的主要力量是所谓``社会主流'',多是专业人才,做事精益求精,肯干实事;而民主党稍
微有些偏于理想化,旗下多有激进人士,做事就有些毛糙、不够稳重吧。

\section{爱德华兹的街区派对}

在见到克里才一个星期后,9月17日我又收到封Email,邀请大家去参加爱德华兹在星期天的街区派对。所谓街
区派对,就是大家走出家门,聚在街头某处开派对。这次派对的地点在凤凰镇(Phoenixville),我在那里住过一年
多,对地形很熟悉。那是个小镇,以前宾州的炼铁业发达的时候,是一个小小的工业基地,但随着炼铁业彻底退出
了这里,这里就日益破败。镇政府为了维持收支,还收居民1\%的收入税,地方又比较闭塞,交通不便,我后来就搬
离了。不过它的风景是挺漂亮的,有乡村气息,商店也很多,是比较典型的郊区地带。

吸取了上次克里拉励的教训,这次我从网上打出了票后,又跑到附近的克里郊区总部,要到一张``蓝票''。登
记时我仍然表示愿意做义工,结果在星期六就接到他们的电话,希望我能提前一小时到那里,帮忙疏导人群。我答
应了。

到了星期天,大概是生性疏懒的原因,我起来时已经10点了,赶到凤凰镇时是11点,比预定时间晚了半小时。
不过那里人并不多,有二十几个人在排队等待,旁边的义工却足有十几个人,都闲闲地坐在那里聊天。我找到了负
责人,跟他说我是来做义工的。他便把大家都聚集起来,说要分配任务,可是说了半天,也没什么任务可分配。主
要就是三件事:布置派对会场,这有专业人士在做,而且已经基本做好了;维持秩序,但现在来的人太少,根本不
需要维持;还有就是让来的人都登记在册,留下Email地址和电话,这样以后可以联系,可是他们准备的笔记板太
少,只有四五个,大部分人还是无事可干。

我无所事事地站在那里,和别人有一搭没一搭地聊天。这时又来了一位示威者,举着``布什---切尼2004''的
牌子,引起大家一阵哄笑。有人说:``他得到批准了吗?我们可以打电话叫警察!''负责人说:``没必要。我们可以
叫警察,但我们不会这样做的。''那个人得以继续隔街示威,我想:怎么今天只来了一个?可远没有阿伦镇的共和党
示威者有组织啊。

接着出现的是两个穿白大褂的人,举着一个巨大的横幅,我草草扫了一眼,只见几个关键词是:``投票起诉约
翰$\cdot$爱德华兹''。我吓了一跳,以为又是来示威的,人群也纷纷涌了过去,和那两个人说话。再仔细地读一
遍那条横幅,原来是说:``如果你喜欢高价药费,投票支持那些起诉参议员约翰$\cdot$爱德华兹的医生。''原来
是变着法儿夸爱德华兹参议员为民谋福利的。既然是自己人,他们就穿过了横栏,在会场四周游行了一圈。

我看看自己实在没什么事做,便干脆加入人群中去排队了。这时队伍里已经有一百多人了,没办法,我只好老
老实实地呆在最后。有义工在散发气球,小孩子们大多人手拿了一个。街旁也已经摆开了几个小游戏,有父母陪同
孩子在玩。食品摊儿当然也必不可少,热狗、汉堡、爆米花、棉花糖的味道混在一起,飘在空中。初秋中午的天气
不冷不热,这旁边就是个小公园,绿树成荫,我们站在树荫下排队,也不觉得热。前方有个当地的小乐队在演奏,
歌声轻松悠闲,鼓点不紧不慢。这确实是次派对,在室外举行的街区派对。

时间一到,门口就开始放人,却没有任何安全检查。我走到跟前时,有人指示:白票往右转,其他票往前走。
我直走到``特权区'',这里离讲台很近,不过栏杆前都已经站满人了,中间的人很多,我便走到右边,那里离乐队
近,人也少些,我站到了第三排的样子。

小乐队表演结束后,凤凰镇的镇长登上讲台,欢迎大家。然后是一位当地民主党的官员,很风趣地要大家深吸
一口气:``你们是否能闻到兴奋的味道?''又要大家伸展手臂摇晃:``你们是否能感觉到空气要变天了(意指白宫要
易主)?''最后让我们跺跺脚:``你们是否感觉到了,宾州的大地将支持约翰$\cdot$克里?!''引起大家热烈鼓掌欢
呼。

他随后开玩笑说:``下面我开始谈政治。我的一个朋友告诉我说,政治其实和开车是一样的,如果你要后退,
那就挂R档(倒车档),但如果你想要前进,就必须换到D档(驾驶档)!''听众又大笑鼓掌,因为R是共和党的首字母,
而D是民主党的首字母。

两位当地民主党候选人也不失时机地前来拉票。一位是竞选众议员的路易丝$\cdot$墨菲,一位是竞选参议员
的乔$\cdot$霍福尔。他们的讲话仿佛就是从克里那儿克隆过来的,只不过把攻击对象由布什变成了本州共和党的
候选人,或者列数对手如何与布什的主张完全一致,反正在民主党支持者那里,``W''就是错误的意思,只要一提布
什,大家的血压都立即升高。

最后,霍福尔介绍出爱德华兹,全场沸腾起来,爱德华兹从街边一户人家的房子里步出,一路握手过来,登上
讲台。他首先拥抱了墨菲和霍福尔,然后展开他那著名的阳光笑容,举起他那著名的大拇指,答谢台下的欢呼。

待欢呼声逐渐平息,墨菲和霍福尔也下了台,爱德华兹开始演讲。他首先说:``让我们向佛罗里达被台风袭击
的人民表达我们的慰问。''我离他很近,很遗憾地没在他脸上看见任何慰问的表情,而且他在说话的同时拿手摸
脸,虽然这可能只是无意之举,却显得他这句话没有多少诚意,只是过场文章而已。随后,他半刻停顿也没有,立
即又开始感谢大家,更让我觉得他那句慰问纯粹是多余,徒显虚伪。


爱德华兹的演讲内容本身无甚新意,都是听过的。唯一的新料是,切尼最近公开声称,如果克里当选,美国很
可能会再遭受一次``9.11''。这种指责显然太过分了些,属于诉诸恐惧的竞选手段,其实,布什最近在民调上的形
势不错,切尼本不用如此口不择言。爱德华兹就以这句话为例,证明共和党已经堕落到什么地步,并点出原因:布
什阵营的执政记录一团糟,以至于他们无法进行正面的竞选,只能寄希望于靠肮脏的攻击来取胜。

爱德华兹半开玩笑地说:``事实上布什总统在共和党大会上的讲话里,有一句话我是同意的。''他举起右手食
指:``我必须强调:就这么一句话---'人们会通过我的执政记录来评价我。'---是的,人们将通过他的执政记录
来评价他!''

在他演讲的过程中,忽然有两个年轻人,举起两张纸,并齐声大喊纸上的内容:``布什撒谎,谁人命丧---省
下军费,帮助非洲的艾滋疫情!''两人都戴着墨镜,有一人还留着大胡子,一派嬉皮作风。由于这次派对不能带标语
牌,所以他们只能带写着大字的纸来。爱德华兹停了下来,面带微笑地听他们喊完,然后对众人说:``他们说得不
是很好吗?''

爱德华兹的风度还是很不错的,他的南方口音,在我这个听惯了新泽西、纽约、宾州英语的耳朵里,尤其好听。
听说他以前的形象保持得很好,一直不肯开口攻击别人,现在当了副总统候选人,本职工作就是攻击,他当然也免
不了,不过还没有像切尼那样赤裸裸的,所以大家对他的印象仍然不错。

演讲一结束,大喇叭里立即开始放音乐,他跳下台来,和前排的支持者逐个握手。一时间,手臂如林般往他伸
去,他双手并用,一面笑容满面地说:``谢谢!''一面批发握手。我站在第三排,合影是没有指望的,握手却没问题。
他走到我们这里时,我把手伸过去,虽然够不到栏杆,但他仍然把手探进来握了一下。我试着喊了一声:``约翰,
可以跟你合个影吗?''但音乐声太响,他显然没听见。

我左前方一位女士在握住他的手后,不知道在跟他说些什么,他侧耳倾听,不住点头,好像是在听取她的故事
或者建议。人们仍然围在栏杆这边,我也站在原地,爱德华兹一边听那位女士说话,一面眼观六方,看见我站在后
面,以为我没跟他握过手,又伸手过来握了一次。

那位女士最后要求和爱德华兹合影,他当然同意了。她的身材不高,两人被栏杆分开着,必须都向中间侧身才
能合到影,这样的姿势,就象依偎在他怀中一样。

爱德华兹巡游一周后,握手、谈话已毕,就又回到了他出来时的那座房子。看来那是个民主党人的房子,临时
被借用了。人群开始逐渐散去,还有些人不肯走,仍然靠在栏杆那里等他再出来。

乔$\cdot$霍福尔又出来了,跟大家握手谈话。他的人气显然不及爱德华兹远甚,所到之处并无人群追捧,他
反而要主动和别人说话。

过了大约15分钟后,大部分人都已经离去,只有十几个人,仍然倚着栏杆等待。我扫了一眼,好像大部分都是
中年妇女,爱德华兹号称``师奶杀手'',果然名不虚传。

我觉得爱德华兹大概是不会再出来了,不过却又想到个裸奔的主意。我走到警戒线前,问安全人员:``爱德华
兹先生呆会儿还会出来吗?''

他说:``我不知道,这取决于他的日程安排。不过我想这可能性非常小。''

我说:``我来自于一个叫APAP的组织,我们今晚会在全国范围内举行家庭派对,大概会有上千人参加。我可以
进去请爱德华兹对我们亚裔支持者们说几句话吗?只要1分钟就可以了。''

他说:``不可以。爱德华兹先生正在和他的助手们开会。''

我说:``那你能够把我的这个请求带给他吗?''

他仍然一脸冷酷地拒绝了,因为那不属于他的工作。不过他看我态度还算诚挚,又说:``这样吧,如果有他的
助手出来了,我会指给你,你自己和他们说去吧。''我赶紧道谢。

很快,一个西装革履的年轻人从房子里走出,那个安全人员是面对人群的,我忙指给他看:``那个人是爱德华
兹的助手吗?''他看了一眼,说:``是的。''我正准备走过去,他阻止了我:``你不能越过这条线。''他自己走过
去,把那个人叫了过来。

我和这个人说明了情况,他说:``爱德华兹先生的日程非常紧。我理解你的心情,我自己也是来自于基层组
织,不过我想爱德华兹先生不会有空。''

没有办法,我只好谢过他们,回家去了。晚上的家庭派对,我也是主办者之一,必须做些准备。

\section{APAP的家庭派对}

我从波士顿回来后,就和APAP在费城的协调者尼娜联系上了。不过,由于APAP才成立不久,成员和能量都很
小,我本来以为一旦找到了组织,就有无穷多的活儿可以干,结果尼娜她们也只是刚起步,总共APAP在费城地区的
成员就几个人。

我们要办的第一件事是家庭派对。早在波士顿的时候,APAP组织者就提醒大家,将在9月19日举办全国范围内的
家庭派对,邀请亚裔候选人,通过会议电话来和大家交流。我住在公寓里,当然办不了这样的派对,就去帮尼娜。

上周六,我到尼娜家去,商量家庭派对的细节。她家住在费城北郊,房子很大,有古典风格。到了那条街上
后,我一眼就认出了她家,因为她家的草坪上除了插着``克里---爱德华兹''、``乔$\cdot$霍福尔''之外,还保
留着``霍华德$\cdot$迪恩''的牌子。

尼娜的丈夫阿桑(Ahsan)来迎接我。在她家后院站台上,尼娜准备了一些印度点心,桌上还有饮料、开心果,当
然也有纸笔。狗被关在后院里,以防它扑上桌来拨拉食物。她母亲躺在躺椅上享受秋日下午,小女儿在桌旁玩游
戏,大女儿则不安定地跑进跑出。

尼娜是印度人,在宾州大学做学者,眉目秀美,是个典型的印度美人。阿桑是位房地产经纪商,怪不得她家的
房子这么漂亮。她的朋友冬尼娅已经到了,大家相互介绍认识了后,就继续讨论派对事宜。尼娜已经打出一张邀请
函草稿,上面除了派对信息如时间、地点、方式外,还列出了将拨进电话会议的候选人名单。我看了一下,发现很
多人都在APAP的讲座上见到过,比如吴振伟。

派对的建议捐款额是30美元,我觉得太多,可能会吓着很多本来想来的人,建议改为25美元,数目差不了多
少,心理感觉却好多了。我的私心是,尼娜和冬尼娅都号称能带不少朋友来,我到现在却不知道会有哪个朋友感兴
趣,如果再要他们捐一笔钱,恐怕就真的一个也不来了。

我们分派了任务,主要是要准备食物,然后又讨论了些细节,比如要准备好各个候选人的介绍之类。尼娜由于
是主办者,还要把电话连线和麦克风预先都调好。

我回去后将尼娜传给我的邀请传单草稿在计算机里重新编辑了,修改了部分地方,比如尼娜对派对的介绍居然
还是``支持迪恩''。对七位亚裔候选人的介绍是从APAP网站抄的:

和亚裔领袖一起参加电话会议(晚上8:30):

众议员鲍比$\cdot$斯科特(Bobby Scott)(民主党---弗吉尼亚):第一位来自弗吉尼亚的黑人和菲律宾混血
裔众议员;

众议员吴振伟(民主党---俄勒岗):第一位华裔众议员;

斯科特$\cdot$川崎(Scott Kawasaki)(民主党---阿拉斯加):将成为阿拉斯加的第一位日裔州议员;

斯瓦提$\cdot$丹德卡(Swati Dandekar)(民主党---爱荷华):第一位南亚裔女性州议员、爱荷华州克里竞选
团队共同主席;

乔伊斯$\cdot$陈(Joyce Chen)(绿党---康州):将成为康州第一位华裔州议员,现为纽黑文市议员;

约翰$\cdot$蒋(John Chiang)(民主党---加州):将成为加州第一位华裔财政部长,现为加州平等委员会成
员;

乔$\cdot$车(Jun Choi)(民主党---新泽西):将成为东海岸第一位韩裔州级公务员,竞选新泽西州议会。

这些候选人是清一色的左派,大部分是民主党,还有一位左到绿党去了。这当然不算意外,因为APAP叫``亚太
裔进步组织'',我到美国时间长了,渐渐也就琢磨出来了:``进步''就是左派的代名词。

我把这个传单Email给了我所有的当地亚裔朋友,邀请他们前去参加,当然,建议捐款数目已经改为25美元了。
可惜,只有一个朋友感兴趣,但到了今天下午,他又打电话给我,由于临时有事,他去不了了。我只好单身一人前
往。

派对在晚上7点半开始,电话会议开始于8点半,这是为了照顾西海岸的人们,那是他们的5点半。我作为主办者
之一,当然要提前到,7点就到了尼娜家。我在当地的一家中国食品``香港超市''买了一些中国点心,比如萨其马、
大白兔奶糖、饼干之类的,尼娜家早准备好一堆食物,大部分是印度风味,在餐桌上摆开。

客人陆续到来,有七八位是尼娜和阿桑的朋友,大都是印度人,也有白人,还有一位日本女士,是尼娜的好朋
友,特地从曼哈顿驱车来参加这个派对的。冬尼娅也带了一位朋友来。冬尼娅是宾大的学生,并不是亚裔,而是非
洲裔,但和尼娜的关系特别好。尼娜对我说:``她就是20岁时的我!''冬尼娅搂过她说:``我们是双胞胎!''尼娜的
大女儿在旁边大叫:``还有我!我是12岁的你!''

人来得差不多了,我们就开饭了,边吃边随便聊天,内容有政治也有日常生活。到了8点半,尼娜招呼大家到客
厅,开始听电话会议。七位候选人轮流打电话进来,先简短地介绍了自己,然后由APAP成员提问。问题早在派对开
始前,就发布在APAP的网站了。尼娜作为主办者之一,也提了个问题,是给黑人与亚裔混血的鲍比$\cdot$斯科特
的:

``你的血统使你可以在非洲裔和亚裔社区之间搭起桥梁,这两个社区之间曾有过不愉快的关系。你在过去和将
来为此采取什么行动?''


我看了她的问题后,曾经吃了一惊,顺口说:``我不认为亚裔和非洲裔社区间有问题啊,我们都是少数民族,
应该团结起来对付种族歧视才对。''尼娜诚实地说:``我不知道华裔社区的情况,但我知道在印度裔里,很多人看
不起黑人,在和黑人打交道时发生过摩擦。你说得对,我们应该团结起来,但现实并非如此,所以我才要问他这个
问题。''

我想起中国人社区里其实也普遍存在着歧视黑人、南亚人的同时,又惧怕白人的心理,就不作声了。

无巧不成书,其他候选人都顺利拨号进来,做了个自我介绍,感谢大家的支持,并回答了问题,可斯科特议员
却出了技术故障,无法加入电话会议。主持人仍然让尼娜问了这个问题。当尼娜首先介绍自己时:``我来自宾州费
城\ldots ''我们乘机尖声怪叫,让全国都听到来自费城的声音。

电话会议结束后,我们便自己来讨论这个问题,因为在场的有黑人,有亚裔,有白人,甚至还有一位西班牙裔。
大家都承认这个问题在某些人之间确实存在,但不过很快都认为这也不是大问题,最主要的还是如何打败共和党那
帮白人老爷。

在座的有一位黑人姑娘,来自霍福尔竞选团队,向我们介绍了霍福尔竞选参议员的情况:形势很不妙,在民意
调查里霍福尔落后共和党候选人阿伦$\cdot$斯柏克特十几个百分点。她问大家:你们认为霍福尔还有什么没做到
的地方?

冬尼娅快人快语地说:``霍福尔应该更主动些。比如像前一阵的波特黎各人游行,他只发了封信祝贺---他应
该到场和大家一起游行!你想那可以激发出多少票来!我听说他曾经访问过穆斯林教堂,那么他应该保持这种作风!''

我对霍福尔的竞选基本一无所知,因此也提不出什么建议来。尼娜兴奋地问大家:``你们看到霍福尔的最新广
告没有?看到我没有?我就站在他后面的人群中!''我倒是收到过尼娜的Email,说霍福尔阵营需要亚裔面孔出现在他
的广告里,但我因为忙,就没去。

说到广告,冬尼娅恨恨地说:``我们不能再进行正面竞选了!要走负面!进行负面竞选!不管怎样,先拍几个广
告,把斯柏克特搞臭再说!''

我倒是听说过,斯柏克特被``公民反政府浪费组织(The Citizens Against Government Waste)''授予``2004年
度猪官(Porker of the Year)''称号。``porker''一词本来是``小猪''的意思,不过由于``pork''在美语中除了
``猪肉''外,还指政府奖金,所以美国人就把浪费纳税人金钱的官员戏称``猪官'',其中尤具创造性、娱乐性、大
无畏精神者,就授于``本年度猪官''这一``光荣''称号。不知斯柏克特到底干过什么丰功伟绩,竟能夺得此``大
奖''。

大家讨论完后,尼娜将我们引到一面贴纸栏前,上面是七位候选人的介绍,以及他们的竞选团队名称,以便人
们写支票。未满18岁或者外国公民是不能捐款给政治组织的,于是我对尼娜说:``我就不捐了。''

尼娜说:``你忘带支票本了吗?没关系,下次再捐就是了。''

我说:``不,我还未满18岁。''

尼娜大吃一惊,上下打量着我。我笑着说:``开个玩笑。其实是我还不是美国公民。''

尼娜也笑了,说:``没关系。你已经做出很多贡献了。''

大家捐完钱后,阿桑和另一个人各自抱出吉他来,我们开始唱歌。不过,可能是由于大家都是移民出身吧,竟
然没几首歌是所有人都会唱的,最后我们主要唱了六七十年代的歌,像《Blow in the Wind》、《Yesterday》。

闹到快晚上11点,我们明天还要上班,有些人还是从外州开车来的,就各自散去了。

9月23日附记

鲍比$\cdot$斯科特发来Email,回答了尼娜的问题:

首先,谢谢你的问题,尼娜。我为不能直接在星期天晚上回答你的问题而道歉。

我是国会黑人会议、亚太裔会议、西班牙裔会议成员。这些会议一起合作得很好,并且有很多议题相近,因此
形成了一个三族会议(Tri-Caucus)。虽然在三族会议中仍然有文化差异,我们仍然一起努力确保一个强健的公共教
育系统、捍卫人权法律(尤其是反歧视法)、保证我们的社区有价格合理的健康保险。

我们还要使人们在处理有关健康保险和公共健康法律的问题时充分尊重文化差异。三族会议的成员和我们的社
区都必须继续努力推进我们社区的建设和立法,这是保证我们梦想成真的最好办法。

展示秀

克里和爱德华兹这么频繁地来宾州,布什当然也不会坐视不理。9月22日,他老人家御驾亲征,到我们普王市来
了。克里和爱德华兹每次来的时候,共和党都会有人来抗议,现在布什来了,民主党当然也要有所动作。

不过,布什并不是来开拉励或者街区派对的---那些``与民同乐''的活动,有大量民众参加,抗议也比较容
易---他这次是到普王市的``福吉谷会议中心''来出席一次会议的,直来直去,并不和群众见面,因此等闲群众难
以接近。克里在费城郊区的总部最后决定,在附近的交通路口举办一次``展示秀(Visibility)'',即向民众展示克
里阵营的标语牌,以示普王市是反对布什、宾州将属于克里。这种``展示秀''以前也办过,其要点在于,我们这里
一直被认为是共和党的地盘,现在我们高调出秀,让周围的人们知道克里支持者的存在,并增强给克里投票的信心。

展示秀地点选在202号公路与加尔夫路的十字路口,那里正是著名的普王购物中心的进口处,附近的几条高速公
路都在这里交汇,因此交通甚是繁忙,车流极多。而且,去``福吉谷会议中心''的汽车绝大多数都会经过这里,既
然我们无法靠近会议中心,那么,在这个十字路口进行展示秀确实就是最好的选择了。

这个地方离我住的地方开车只要5分钟,我当然要去。展示秀开始于上午11时15分,我不慌不忙地拖到11点10分
才出发,结果一上路就发现加尔夫路已经被警察封住了,我只好拐弯上了高速,在高速公路上绕了一大圈才到达目
的地。在路上,我注意到高速公路下到``福吉谷会议中心''的出口也被封了,天上还盘旋着一架直升飞机,看来布
什的保卫工作做得确实严密。

这个十字路口由于交通繁忙,铺得很大,除了路口的四角都有水泥台阶之外,在202号路的两个方向中间,也筑
有水泥台阶。我到那里时,已有两组人扶着巨大的``克里---爱德华兹:一个更强大的美国''标语牌,站在加尔夫
路的右侧了:一组在202号路的中间,一组在对面,标语牌都正对着穿过202号路顺着加尔夫路往北开的车,那也正
是去``福吉谷会议中心''的方向。

我随身带了一个``克里---爱德华兹''的小标语牌,站在了202号路的南侧,将标语牌举在胸前,向过往车流展
示。不过这个标语牌太小了,我觉得不大过瘾,好在马上又有一位女士走了过来,拉着一个巨大的标语牌,我便帮
她扶住另一边,站在水泥台阶上,倒也威风凛凛。

不久又先后来了两组人,占据了对面的两个水泥台阶,这样,除了一处被围墙挡住路角之外,所有的``要塞''
都被我们占领了。

过往的车辆看到我们的标语牌,喇叭声不断。不过喇叭不会说话,他们也许是在表示支持,也许是反对。我是
一概把他们都当作支持,只要听到喇叭声就向他们挥手。有时候我也可以看到车里的人向我们招手,或者竖起大拇
指来向我们致意,这时大家就或摇动小标语牌、或是挥手怪叫,倒也热闹。

当然反对的人也不少。有一次红灯亮起,一辆车在我旁边徐徐停下。他的车窗是开着的,驾驶员是个白人,坐
在副驾驶座的则是一个黑人。这个黑人靠我们比较近,车还没停稳,就拿手指着那个白人对我们笑着说:``你们别
听他的,他其实很爱约翰$\cdot$克里!''

我立刻明白了,这个白人一定是支持布什的。果然,他把车停住后,就对我们说:``你们在这里干什么?克里是
个墙头草!''

我说:``他并不是墙头草,他以前的某些投票记录是基于布什给出的错误信息。他被布什误导了!''

他嘿嘿一笑,大概是觉得我孺子不可教也,转过头去,再不跟我说话。那个黑人仍然呵呵地笑着,说:``你放
心,他会投票给克里的!''

不过这个人还算是客气的,有些人对着我们把大拇指向下指,还有人则干脆竖起中指。对此我要么不理,要么
是更大幅度地摇动小标语牌作为回答。

我后来和组织这次展示秀的杰娜丽聊了一会儿,她说:``我发现,那些对我们不友好的,大多是开小车的白人
男子。我不知道这说明了什么。''我说:``是的,我也发现了,要是开卡车的司机,几乎都会对我们挥手致意。''

总的来说,友好的回应比不友好的要多得多,不过,那些友好的表示,也许部分是出于看热闹的心理,而不喜
欢克里的很多人,出于礼貌也没有对我们做出任何表示。至少民意调查显示出,克里和布什在这里是平分天下的。

我们秀了1个小时之后,也就收摊了。在回去的路上,我经过了``福吉谷会议中心'',那里果然警察成群,戒备
森严,我也就没有拿着标语牌冲下去找麻烦。后来杰娜丽又发出一封Email,计划明天再来,看来她感觉效果还不
错,不过我第二天要上班,所以就没有参加了。

\section{克里来宾大}

9月24日,克里要在宾夕法尼亚大学进行一次拉励。我从克里在费城郊区的总部拿到蓝票,起了个大早(早上7点
半啊,对于我来说是超级大早了),开车赶到宾大,穿过他们不大不小的校区,赶到会场时,正好9点15分。开门时
间是10点半,我之所以来这么早,是希望能够站到最前排,仔细看一下活生生的克里是什么样子,演讲风度如何,
以及举手投足间的气质和人品。当然,如果能找个机会和克里合张影,就更好了。

会场外面停着一辆大卡车,上面有``克里---爱德华兹2004''的大幅标语,看来是专门为克里阵营布置会场用
的。再看会场内外,都已挂起了各色标语,天空飘着红白蓝三色气球。会场在一个露天青草地上,平时大概是宾大
学生聚会的场所,会场的西北角现在搭起了一个台子,上面的主题标语是``为宾州家庭而战''。草坪的中央则是媒
体的搭台,十几架摄像机一字排开,正对着讲台。空地上则竖着很多栏杆,将草地弯弯曲曲地分割成好几块,想来
是给不同的票划分的区域。场外站着很多警察,设了两个入口,都有那种金属探测门,警察手里还拿着金属探测
棒,严阵以待。

门还没有开,我就先在队伍里排队。这时队伍并不长,我排在很靠前的地方,看来地利不错。旁边草地上也坐
着一群义工,好像无事可做。附近有很多人在兜售克里阵营的徽章、T恤,内容五花八门,无奇不有:从各式各样地
支持克里和爱德华兹的,爱戴特瑞莎$\cdot$克里的,到攻击布什、切尼的,以及反战的,甚至还有拿克林顿的头
像来做广告的。有的人只是在那里摆摊,有的就直接拿货物沿着队伍贩卖,一个徽章5美元,一件T恤15美元,也不
知道他们是真的支持克里,还是只不过以此赚钱。

还没到10点半,门就提前开了。我顺着队伍往前走,一位义工向大家说:``白色的票请往左走,其他票往右
走。''右边就是那两个安检门,左边则一马平川,直接进到草地上去了,但肯定是离讲台非常远的了。大部分人都
往左边走,我心中暗喜,走到安检门前,交出摄像机包,里面除了摄像机外,还有一瓶水、两袋饼干。警察
说:``里面不允许带食物。''我没有办法,只好把它们扔了,算是丢车保帅,过了这一关。

顺着写有``蓝票''的大牌子的指示,我找到了``蓝票区''。那里几乎没人,我很容易地就占到了最佳位置,但
却一点也高兴不起来,因为我到了这里才发现:原来有一道栏杆将蓝票区和讲台分开,中间的区域有几个人正在进
入,看他们手中的票,乃是红色。看来这种拉励还真是等级森严,我这次又差了一级。

我站的地方正好邻近媒体席,在靠近我这边的尽头,有三个长着东方人面孔的记者。离演讲开始时间还早,他
们暂时没有什么可拍摄的,眼看着进入场地的人越来越多,他们就走了下来,开始随机采访这些支持者们。旁边的
朋友告诉我,他们是从韩国来的,``韩国人非常关注目前的美国大选,因为这可能会关系到美国的朝鲜政策,这对
韩国人来说是非常重要的。''

我说:``看来他们对美国大选的关心非同一般,这不过是克里的一个拉励而已,又不是他的外交政策发布会,
他们都赶过来拍摄了。''我们随即讨论了一会儿美国的朝鲜政策。我对这个问题也比较关心,因为我有一个私心:
要处理朝鲜问题必然要借助于中国的协调,这有助于发展中美之间的互信,也会促进中国进一步溶入国际大家庭。

我本想找他们聊聊,但他们却正在找别人聊,想到这是他们的工作,我也就没有去打扰他们。本来还想等他们
回来,不料我的朋友们却发现红票区现在不查票了,只要通过了安检的人都可以进去,就拉我一起进去。我看那边
已经站了很多人,即使进去也不可能占到前排的了,但好歹也近了很多,就跟他们一起过去了。

到了红票区,朋友左钻右窜,居然钻到最前面去了。我跟在她后面,也到了前排,再往前一看,不由得倒吸了
一口冷气,正是``道高一尺,魔高一丈'',饶是我恁般钻营,前面又有一道栏杆拦住,在主讲台和我们之间又隔出
一片小区域出来。再看里面的人,拿的票却是黄色!

至此我不禁感叹:小小一个拉励,居然也有如此之多的等级。好在这里虽然没法和克里握手合影,但至少离讲
台已经非常近了,比起上次在阿伦镇,已经好太多了。

这时,人已经进得差不多了,台上开始暖场表演。十来个宾大学生鱼贯走上了舞台,开始唱歌。唱完歌后,终
于轮到政客露面。来的还是那几个老熟人,即费城地区的民主党名人,比如乔$\cdot$霍福尔之类。比较引人注目
的是拳王伯纳特$\cdot$霍普金斯,他是费城的骄傲,世界拳击理事会(WBC)、世界拳击协会(WBA)、国际拳击联合
会(IBF)三大组织的中量级冠军。台下的人群多是宾大的学生,最崇拜的便是体育明星,霍普金斯一出场,便引起大
家的一阵欢呼。他的讲话富有黑人式的激情,也和大部分黑人一样是民主党的坚定支持者。出乎我意料的是:他的
讲话相当有条理,仪态和表达都很出色,改变了我对拳击运动员的某些偏见。

在他之后讲话的黑人是一位律师,一头及腰的长发,打成一缕缕辫子系在脑后,无论是黑人还是律师当中,这
副模样都是极罕见的。他首先自豪地报出:``我是宾大毕业的!''然后就从法律的角度开始大肆批评布什,比如伊拉
克战争违宪、《爱国者法案》违宪,而最引起听众共鸣的,则是他大胆地谈论到堕胎问题(虽然没有明确点
出):``妇女的身体属于她自己,政府无权说三道四!''此言一出,顿时引起台下一片如雷的叫好声。真是年轻人啊。


最后压轴的是费城的黑人市长斯爵特,照样也是激情四溢。不过我对他不太感冒,因为他去年在市长选举时闹
出窃听器和涉嫌贪污的丑闻,虽然最后连任成功,但形象已大损。

他们都讲完话后,场内气氛开始冷了下来。有人为了活跃气氛,往台下扔印着``克里---爱德华兹2004''的T恤。
不过他们没有那种超级橡皮筋,因此扔不远,我前面又站了一个大高个儿,所以一点抓到T恤的希望也没有。

好在克里没有让我们久等,才不过10分钟的样子,人群中忽然爆发出山呼海啸般的欢呼声,``克里!克里''的叫
声有节奏地在全场响起。我知道肯定是克里来了,可往前一看,所有人都把手中的标语牌举了起来,只见手臂如林、
标语如云,好不壮观。好容易等到大家兴奋完毕,放下手来,我才看见克里已经站在台上了,西装已经脱去,正在
挽袖子,一面向四周挥手致意。

照例,克里不会自动讲话,需要别人介绍他。一位女士走到台前,开始发言。她自称是一个共和党人,以前从
来没有投过民主党人的票,但这次她要支持克里,因为布什的战争政策完全错误。有六位女士在她身后站成一排,
她逐个介绍了她们,都是美军现役官兵或老兵的妻子或母亲,她们都参加了一个叫``母亲任务(Moms with a
mission)的组织。至于她自己,有两个儿子,一个在``9.11''后报名参了军,被送到伊拉克,好容易回来后,另一
个儿子又很快被征召去了,现在仍然在伊拉克的某个地方。她认为布什的战争决策太轻率,她说:``美国需要一位
曾经亲历过战争的总统,他知道战争意味着什么,这样他才会在发动战争时慎重考虑。''这当然是指克里,因为克
里参加过越战,而布什却有逃服兵役的嫌疑。

她讲完,却没有介绍克里,而是介绍了宾州州长埃德$\cdot$伦德尔。由于她自己是共和党人,她还顺便开玩
笑说:``他以前是我最喜欢的民主党人,不过现在他只能排第二了。''伦德尔的演讲很不错,有激情,又幽默风趣。
他直称:``让我们欢迎克里参议员来到全美国最棒的大学!''这句话自然说得台下的宾大学生甚是舒服。不过我们都
知道,克里是耶鲁毕业的,伦德尔便接着说:``克里参议员不是这所学校毕业的,我也不是这所学校毕业的,可
是,我每年都要花上大把银子,为的是让某人能在这儿毕业,所以它最好是美国最棒的大学!''此话引得台上台下一
片笑声,大概他的某位子女正在宾大读书吧。

其实伦德尔本人也在宾大授课,教的是政治学。他还问了一句:``这里有上我课的同学吗?''台下只举起廖廖几
只手。我心中暗想:``他是不是要说,你们都得A了!''他却说道:``你们知道吗?法律规定,学生可以在自己的学校
所在地登记注册投票。所以,如果你是来自纽约,那么你应该在宾州登记;如果你是来自康州,你应该在宾州登
记;如果你是来自加州,你也应该在宾州登记。''

他的意思是,纽约、康州、加州等地将铁定被克里赢得,因此学生们在那里的一票没有什么大用,还不如投到
宾州这个摇摆州来,就可能是决定性的了。我旁边的人咕哝了一声:``那么德克萨斯呢?''因为德克萨斯将铁定属于
布什。伦德尔好像在台上听见了似的,笑嘻嘻地继续说:``如果你是来自德克萨斯,那你也应该在宾州登记!''众人
都大笑。

伦德尔基本上没有谈论太多政治本身,但他的这番话却也许是今天所有演讲者中最实用的,风度也很轻松自
然,看来他能从共和党人手中抢回州长宝座,确实不是白给的。

伦德尔之后演讲的是一位参议员,由他最后向观众推出克里。终于轮到正主儿了,听众们又是一阵欢呼,将标
语牌举得漫天都是。就在我旁边,有几个人举着一个大牌子,上面用超大字体写着``沃顿MBA:克里,你被雇佣
了!''。 沃顿就是宾大的商学院,在全美的商学院里长期排名第一,带着这个牌子来的是个女士,身上的T恤还印着
``沃顿商界妇女''。不过我觉得这个标语牌不够带劲,应该模仿NBC的``学徒秀''节目,写``沃顿MBA:布什,你被
解雇了!''就更有趣了。

克里显然早就在台上看到这个牌子了,他一上台就为大家读了一遍:``沃顿的MBA都是很聪明的,他们今天告诉
大家:克里,你被雇佣了!''全场都大笑鼓掌,我前面的那人由于个子高,被选来举牌子,这时更是又跳又摇,恨不
得让所有人都看见这块牌子。克里在后来的演讲也多次提到宾大、沃顿,也算是入乡随俗吧。

克里以他几天前在佛罗里达州奥兰多讲的一个笑话开头。奥兰多是著名的迪斯尼乐园的所在地,有四个迪斯尼
主题公园,号称``幻想世界''。克里说:``我和布什的区别是:我经过幻想世界,而他住在幻想世界!''揭开了他攻
击布什的序幕。

首当其冲的是伊拉克战争,这正好可以和站在他身后的那几位女士联系上。他许诺说:``如果我成为总统,我
将只在我们必须参战时进入战争,而不是在我们想参战的时候!''随后他指责布什的战争政策太过于草率,根本没有
想好如何善后,就心急火燎地入侵占领了一个外国。如果他上台,将把美军``在安全的前提下,以最快的速度撤
回。''

他的讲话总的来说还是那些老生常谈,把布什从头到脚都批评了一遍,但说到他自己的对策时,却乏善可陈。
当然这也不能完全怪他,伊拉克目前的这个烂摊子,换了谁也提不出个妥善可行的好计划。相比之下,布什硬着头
皮喊``坚持'',倒更简单明了,而且在道义上很合选民的胃口,虽然在我看来这纯粹是他打肿了脸充胖子,拿美军
士兵的生命和纳税人的大把银子来为自己做政治冒险。


工作外包、企业福利、社会保险、健康保险乃至干细胞研究,克里也都拿出来说了一番。这些都是典型的民主
党理念,但也是我和民主党分歧最大的地方。我身后站着一位女士,手里举着个``克里---爱德华兹2004''的牌
子,听到高兴处,不时高声叫道:``是!''``对!''``说得好!'';而我就只好保持沉默了,除了干细胞研究之外。克
里强调说:``我相信科学。''这是针对布什``相信上帝''来的,显然,宾大那些年轻的大学生们更拥护一个相信科
学的总统。

演讲内容虽然不太有新意,不过我总算得到了一个难得的近距离观察克里的机会。在别人的冗长介绍过程中,
他在台上站了大约半个小时,被太阳晒得够呛。不过他的仪态始终保持得很好,那位女士开始讲话时,他注意到麦
克风的高度没有调节好,还主动上去帮她调低,这绅士风度赢得台下一片掌声。台上太热,义工给他扔水瓶,连扔
了两瓶,他左手接住一瓶,右手又抄住一瓶,身手确实敏捷。伯纳德$\cdot$霍普金斯在他讲话前赠送给他一双拳
击手套,大概意思是要他狠狠地揍布什一顿,我一度以为他会戴上它们向大家挥舞一番,还好他没有。

不过克里在台上时,显然没有一直认真地听别人的讲话,有时就只顾自己走神了。有一次,别人讲到他,还转
身向他伸手致意,他却毫无反应,仍然呆呆地望着前方天空,不知道在琢磨什么。

最后,克里引用了两个故事来结束他的演讲。一个是两个不到10岁的小孩子,整个暑假里都在街上摆了个桌
子,自制小饰物出卖,最后为克里募到了600多美元。他说:``他们可以做的,你们也可以做到,我们都可以做
到!''另一个故事是当他在中部竞选时,有一个妇女眼含泪水地对他说:``约翰,你简直就是在讲我的故事。我每天
都必须对我的女儿讲几百遍No。买玩具,No!买衣服,No!买礼物,No!''克里说:``我们要改变!我们要让成千上万
个这样的母亲能对孩子说Yes!我们要对选择说Yes!对和平说Yes!对健康保险说Yes!对社会保险说Yes!对科学说
Yes!''

演讲一结束,大喇叭里立刻放出超级响的音乐,克里也按照惯例跳下台来,和站在最前排的支持者们握手。人
群立刻分为两股,大部分人往后走,四散回家,一小部分人却涌上前去,想和他握手。我也跟在人群里往前走,到
了最前面时,早已挤得水泄不通,后面的人还在不断地往前挤。我已经很久没有这么挤过了,这些年轻人简直拿出
了追逐摇滚歌星的劲头来追逐克里。有人在旁边问:``你和他握到手没有?''一个人回答:``没有。不过我碰到他的
手指头了。''我在人群中随波逐流了几分钟,眼看着没有希望挤到最前面去,便自行撤退了。

走出不远,在下一个街头,却迎头看见一群学生,手持标语牌,都是支持布什的。这时已是下午1点半,我的饼
干被迫扔掉后,到现在颗粒未进,当然没空理会他们,赶紧走回到车里,开车去吃饭要紧。

\section{美国纳粹党拉励}

我们知道,纳粹党在美国是合法的,他们注册的名字是``国家社会主义运动''。这听上去似乎和纳粹毫无关
系,然而,``纳粹(Nazi)''本就是德语国家社会主义(Nationalsozialismus)的缩写,而德国纳粹党的全名也是``国
家社会主义工人党''。极权主义和种族主义,正是纳粹的主要特征,美国纳粹党人也不惮自称``纳粹''。

以前我在电视上看到过关于美国纳粹党的一些活动,没想到他们居然会选择今天在``福吉谷国家历史公园''举
行拉励。还是在一周前,我参加民主党副总统候选人约翰$\cdot$爱德华兹在凤凰镇的街区派对时,有人发给我一
张传单,上面写道:``反对种族主义,结束仇恨,让我们一起向纳粹和三K党示威!''地点便是福吉谷国家历史公园。

福吉谷是美国的革命圣地,1777年冬,费城陷落,华盛顿率领败兵残将在这里修整,其间冻死、开小差的士兵
不计其数,是整个独立战争里最艰难的时光。但同时华盛顿也利用这段时间重新训练了军队,过冬之后,又杀出谷
来重新和英军较量,最终赢得了独立战争的胜利。因此,美国政府把这里划为国家历史公园,是个进行爱国主义教
育的好场所。

但是,如我们所熟知的,爱国主义很容易滑为狭隘民族主义。美国纳粹党也强调爱国主义,但他们只爱白人的
美国,主张消灭犹太人、黑人、西班牙裔以及亚裔,建立纯洁的雅利安人国家。如今,全世界的纳粹运动陷入低谷
已经60年了,他们选择福吉谷来拉励,除了彰扬爱国精神外,当然也带有以独立军当年的艰难处境来自励的意思。

像纳粹党这样的世界珍稀动物,跑到我们家门口来展出,如果不去参观一下,也实在太浪费了。我本来打算参
加反对种族主义的示威,但因为近来事情太多,抽不出身来,只好今天下午跑过去看一下热闹。好在公园离我家极
近,开车不过3分钟而已。

还没到公园,就发现四处早已站满了警察,都戴着头盔,手持警棍,如临大敌地布置在公园四处。天空还盘旋
着一架直升飞机,也不知道是巡逻用的还是真的架着机关枪。

我往前又开车走了没多远,发现对面走过来一群人,都是全身黑衣,以巾蒙面,打着各种旗号,怒气冲冲、群
情激愤的样子。

初看过去,都是白人,我也来不及看他们的旗号上写着什么,赶紧取出摄像机来照相。他们早看见我了,一起
举臂指向我,大拇指向下,想必口中还在不干不净地骂着。车开近了,有一个毛头小伙子向我竖起中指,我心下大
怒:``敢歧视我是亚裔?!''当即竖起中指还击。

这下他们更愤怒了,有一个小姑娘竟然冲过来,想抢我的摄像机,当然没有得逞。还有一个人飞起一脚,向我
的车踢来,被我一拐方向盘,躲了过去。我虽然早知道这帮纳粹党人是激进的种族主义者,但还是没有想到他们竟
然敢如此公然撒野。我把车绕着公园开了半圈,找到停车场停下,拿好相机,又回头去找他们了,心下预先制定好
战略:``如果他们骂我'中国佬',我就骂他们'白猪';如果他们叫我'滚回中国去',我就叫他们'滚回欧洲去';如
果他们动手打人,那我只好扯呼,叫警察来收拾丫的。''

待我大踏步回到当地,再仔细看他们的旗号时,却是些``反对种族主义''、``结束仇恨''之类,原来是自己人。
他们故意打扮得如此怪样,以抗议纳粹党和三K党,因为三K党人都是一身白衣,头罩尖顶白帽,只在面罩上戳两个
洞给眼睛。再看他们对待公园里开过来的每一辆车,都是大放嘘声,好像只要今天出现在这个公园里,就一定是纳
粹分子一样。其实这个公园是附近人家健身、游玩常来的地方,这些年轻人如此激进义愤,也就是动机和纳粹不
同,行为方式和纳粹倒有些神似。

不过他们的示威也不是没有幽默,有个人装扮成公鸡,举着一个牌子:``纳粹使白肉成了坏词。''``白肉''本
指鸡肉、鸭肉等家禽肉,在这里被用来讽刺纳粹分子的``白人至上主义''。

既然分清了敌我,我便过去和他们聊了几句。他们中有人还记得我,我说:``你们有些人把我当成了敌人,其
实我是你们的朋友。''有一个中年人忙说:``我为我们中间某些人的行为向你道歉。''我说:``没关系。''心里却
不明白:我一看就是个亚洲人,他们怎么会把我当成纳粹分子呢?大概是因为我在车里,又戴了墨镜,外面看不清楚
的缘故。

这时,警察却开始赶人了,原来纳粹党的大会就要开始了,警察要求这些人只能在公园外示威。他们只好成群
结队地往外走,我找警察去问纳粹党的会址在哪里,警察却不肯告诉我。我只好再往公园里走,正好有两位女士站
在路边,我问她们,她们也不知道,倒反问我了一句:``你是记者吗?''我说:``我不是,我就住在这儿旁边,好奇
过来看看。''

我们正说着话,路上走过来五六个人,当先一位女士,身上穿着大红的吊带装,胸前是个白色的圆圈,画着黑
色的卐字---这件衣服其实就是纳粹党旗。再看他旁边的一位男士,赤裸着上身,露出胸前的黑卐字刺青,就是我
们常在电影里看到的纳粹军官的那种黑卐字。他怀里抱着个小孩,看样子是他和那位女士的孩子。还有两个人手拿
旗子,都没有展开,但可以看出来一面是美国国旗,另一面则是卐字旗,因为我看到黑色的条纹了。


他们说着话,从我旁边走过。我一直盯着他们看,他们却毫无反应,自顾着走过去了,我还以为他们看见非白
人,就立刻要口出秽语呢。除了那几样标志之外,他们的长相、打扮和普通的美国人毫无不同,并没有半点暴力倾
向的样子。

我既然问不到他们开会的地方,就干脆回家歇了一会儿,忙了一阵子其他事,才又重新出来。这次车再开到公
园前,才发现公园已经被封了,汽车开不进去。我只好绕到其他地方停了车,再步行过去。这次我找到了一条路,
只见两边都用栏杆挡住,路口还有一个告示:``进入此区域后,你的行动将在电子设备的监视之下,并且你可能会
被搜查身体。''

我知道,这下可找对地方了,顺着这条小路往前走了大约5分钟,果然来到一个安检门前。通过安全检查后,前
面是一个舒缓的小山谷,有栏杆挡住。很多人站在这里,有的在对着谷底喊口号,有的则在看热闹。

谷底有好几面平行竖起的高墙,大概是纪念当年的大陆军士兵的。墙前站了一排纳粹分子,大多穿着整齐的军
装制服,中间有个讲台,上面铺着鲜红的卐字旗,有人正在上面讲话。有趣的是,他不是面对纳粹分子,而是面对
山坡,喇叭也是对着外面。看来他的宣讲对象不是自己人,而是外界,尤其是山坡上那些反对者们。

在他的右侧,有人挥舞着卐字大旗,还有人拿着美国国旗。一面大横幅展开在他们的队伍中,上面印着卐字,
写着``国家社会主义运动'',然后还给出了他们的地址和联系电话。

一些媒体在下面采访拍摄,至于我们这些人,则是过不去的,除了有栏杆拦着之外,还有一排警察在警戒。我
注意看了一下那些示威者们,果然都没有上次看见的那些奇装异服,着装也不统一,看来警察确实是担心那些人的
暴力倾向,因此把他们隔在公园之外。

谷底的讲话我没有留神听,只听到他们不时冒出几句口号,然后那些纳粹分子就一起举臂喊道:``白人至上!''可
惜他们的人数实在太少,不仅没有示威者多,也没有看热闹的人多,甚至没有警察多。

纳粹思想在美国的兴起,起于30年代以来美国的德国移民。二战之后,纳粹思想成为千夫所指,美国纳粹党基
本消声匿迹。20世纪60年代,人权运动兴起,在黑人争取民权的同时,白人至上的声音也随之激荡,纳粹党乘势重
起。他们一度被地方政府禁止过,后来美国人权组织认为他们的权利也应当受到保护,任命了一位律师替他们辩护。
这个律师是犹太人,官司打到最高法院,大法官们以5 ∶ 4判纳粹党也有信仰和言论自由,从而使纳粹党在美国获
得合法地位。

我以前在电视里看到过对纳粹党的介绍,活动仪式与德国人搞的如出一辙,都是高举旗帜,全副武装,盛大游
行。他们举的是美国国旗,但如果不细看旗帜,很容易以为是德国纳粹,因为他们也佩戴卐字。主席台上也供个巨
大的人像,不过不是希特勒,而是戎装的乔治$\cdot$华盛顿。不过我看到的这些都是30年代他们刚兴起时的好时
光,现在的纳粹已经基本在美国没有什么影响了,只有一些极端的白人至上主义者才会成为信徒。他们还对《圣经》
作出了新解释:雅利安人是上帝的选民,希特勒是先知,第三帝国就是天堂的一个样板,现在的美国是新耶路撒
冷,我们要在这里建设纳粹新天堂。

每次纳粹党的年会,都有大批反种族歧视者尤其是犹太人前来示威。今天,他们挑选的这个日子恰巧是犹太教
最重要的节日``赎罪节'',大概是为了乘着犹太人都呆在家里过节,来开个安心的会,不料光是来示威的白人,人
数就超过了他们。

我耐心地等到他们的讲话结束,想在他们回去的时候,再仔细看看纳粹分子的模样。不料他们开完会后,并不
从我们这边走,而是登上几辆巴士,从山谷后面走小道开走了。

我只好跟着人流,一起从原路回去。刚走到公园门口,忽然迎面走来一队防暴警察,我想:``看来大会开完
了,他们也该下班了。''可再仔细一看,不对啊,他们中的有些人看起来可实在不像是防暴警察啊,而且着装也不
整齐,个个面露凶光、脸带杀气,急匆匆地往会场的方向行军。正在好奇间,忽听得他们爆发出一声口号:``黑人
至上!黑人至上!''于是才恍然大悟,原来也是来示威的。

我旁边正好站了两位黑人妇女,她们疑惑地说:``纳粹的会都结束了,他们还来干什么?''我搭话说:``他们好
像来迟了。''另一个女士大笑起来:``他们来迟了!就像其他所有黑人一样,他们又迟到了!''我也笑了起来,开玩
笑说:``说不定他们没迟到,他们故意算准了时间的。那边一个人看见纳粹的会议开完了,就赶紧打手机通知他
们:纳粹已经走了,你们可以来了!''两位黑人女士一起哈哈大笑。

后来我才想起,他们也许是因为装束太暴力化,遂被警察给予和前面那些黑衣蒙面白人一样的待遇:只能在公
园外示威,不能进公园,直到现在纳粹的拉励已经结束了,才准许进场。我们嘲笑他们迟到,说不定倒是冤枉他们
了。

这队示威者们的服装确实威风,全身上下都是黑色,脚蹬军靴,有人还戴着防暴头盔,军服上有个标志,是非
洲地图,显示他们以自己的非裔血统为傲。大家看见半路杀出这么个程咬金,都纷纷猜测他们要干什么。他们也不
理会众人,仍旧急行军向前。我小时候看过一些打仗的电影,听说两条腿是跑得过四个轮子的,因此不免猜测他们
也许能追上纳粹分子,跟他们以暴对暴,看看黑人力量(black power)和白人力量(white power)到底哪个更厉害,
岂不有趣?连忙跟在他们后面看热闹去。其他也有很多人跟了过来,甚至已经往外开的新闻采访车也开了回来。


很快,他们就走到安检门了,不过他们并没有冲上去和警察搏斗,而是笔直地站成纵队,走过去让警察搜查。
媒体这时已经赶到,纷纷拍照的拍照,采访的采访。有一个白人示威者手里拿着个``要爱,不要仇恨''的标语牌,
和队伍里的最后一个人慢条斯里地说:``我们的目标是一样的。但是,我们不应当诉诸暴力,应当用和平的手段,
用爱来对待仇恨\ldots ''

显然,这些黑人没有带任何凶器,陆续都通过了检查。随着整齐的号令声,他们进军到山坡草地上,栏杆在
前,他们无法再往前走。两个首脑模样的人商量了一阵子,忽然大呼口号:``黑人至上!''其余人都跟着对谷底大喊。
如是者三,首脑发一声令:``向右转!''便收队回朝了。

他们如此收场,倒令众多想看热闹的人有些失望,只好随着他们往回走。走到半路上,那架在天上盘旋的直升
飞机缓缓地降落在路旁的草坪上,看来,今天的大戏终于结束了。

晚上的新闻里,也播了纳粹党的这次拉励大会,不过各电视台都只给了大概1分钟,其中还有一半时间是给了示
威者。我这才知道,原来除了那些现场示威者,还有人在大约1里外的地方举办了一个反种族主义的拉励,在那里有
音乐表演,还有被伊拉克人斩首的美国人质的父亲的讲话。他说:``那段日子是非常艰难的。但我明白了一个道
理:对暴力我们不能还报以仇恨。我的心里已无仇恨。''

对纳粹分子的采访又是另外一副景象。在电视画面上,有个很可爱的小女孩,庄重地举起右臂,行纳粹军礼。
她的父亲接受采访说:``我们不是个非暴力的组织。当时机来临时,我们会毫不犹豫地诉诸暴力。''

不过对此我倒不担心,他所期待的时机显然永远也不会到来。纳粹在德国的兴起是有其深刻背景的,德国本身
就有普鲁士的浓厚军国主义传统,人们很容易认同极权思想,在经历上千年的分裂、德意志帝国短暂的光荣、巴黎
和会所遭受的巨大屈辱之后,德国人很自然地选择了极端民族主义。这些促使纳粹诞生壮大的土壤,除了``核心国
家''的类似地位外,今日的美国都没有。美国人当然也对他们的国家异常自豪,但他们并没有由外辱刺激出来的、
由深埋心底的自卑感改头换面爆发出来的``合群的自大'',相反,在美国盛行的是自由主义传统,人们崇尚的是个
人的自大。因此,纳粹思想注定在美国无法壮大。

\section{教堂里的多党辩论}

美国自由党在5月份开全国大会,确定了迈克尔$\cdot$班纳瑞克(Michael Badnarik)为总统候选人。我们蒙
郡自由党主席吉姆$\cdot$巴伯发起了一个叫``宾州支持班纳瑞克''的组织,作为宾州的自由党总统竞选活动协调
者。宾州是自由党较为活跃的一个州,班纳瑞克当然也要来此进行竞选活动。他在宾州停留约一个星期,吉姆为他
安排了一份很紧凑的日程表。今晚,他将在附近的一个教堂参加多党辩论。

这次辩论是一个叫``天主教和平与正义团(Catholic Peace and Justice Groups)''的组织主办的,地点在费城
北郊的一座天主教教堂,邀请了所有党派的所有候选人参加。当然,布什和克里不可能来,其他大部分候选人也来
不了,只派来了代表。蒙郡自由党人达仁和他们联系上,保证所有的自由党候选人都将亲自出席。

我在几天前发Email问达仁,是否有什么需要帮忙的。他回信说:``我们需要一个人打扮成公鸡,站在教堂外散
发自由党的传单。你可以来做吗?''---这当然是开玩笑。他说,主办者都已经组织好了,我们不需要帮忙。

我在晚上6点三刻到达教堂,会场内早摆好了椅子,大概能坐两百多人。左侧靠墙一排长桌上放着饮料和蛋糕,
我正好没来得及吃晚饭,取了一些蛋糕吃了,好像是人们自家做的,比店里平常买到的蛋糕要有味道得多。

辩论还有15分钟开始,大家一边吃东西,一边聊天。

班纳瑞克还没有来,我加入了其他自由党人的圈子。有人身上佩戴着与众不同的``班纳瑞克''徽章,得意地
说:``别人戴的都是班纳瑞克获得自由党提名之后制作的徽章,我这个是在5月份自由党大会之前的,我在那个时候
就支持班纳瑞克了。''原来,在自由党大会上,班纳瑞克原本并没有被人看好,以前参加过蒙郡自由党活动的加里
$\cdot$诺兰和一位来自好莱坞的亚隆$\cdot$罗索才是总统候选人的大热门,最后班纳瑞克奇迹般地在代表投
票中胜出。这个人在大会前就``慧眼识英雄'',也难怪他得意。

我和一位自由党人罗伯特坐在一起,他给我讲了他成为自由党人的过程:他出生在费城,父亲是死硬的民主党
人,激进的自由派。他却从小就讨厌左派,长大后成了一个坚定的共和党人。20世纪90年代,共和党看到互联网有
资讯泛滥的危险,提出要控制互联网,他觉得荒谬绝伦,因为他本人是个计算机工程师,他知道这在技术上不可能
达到;更重要的是,他认为互联网是对自由的促进,控制互联网就是在扼杀自由。于是,他断然退出共和党,加入
自由党。

辩论快开始时,班纳瑞克赶到了。首先进行的是总统候选人辩论。主持人宣读了辩论规则:每个人有1分钟的开
场演讲、2分钟的回答问题时间和1分钟的总结陈词。先进行的是总统候选人辩论,主办者邀请了共和党、民主党、
绿党、自由党和独立候选人纳德尔(纳德尔正在就他的参选签名上诉,目前尚不知道他是否会出现在选票上,主办者
为了确保选民能听到所有的声音,也邀请他了),所有各方都答应了要来参加,但民主党的代表最终没有出现。

在开场演讲中,共和党、绿党都说了自己是代表某党的总统候选人,班纳瑞克却气势轩昂地说:``我就是自由
党的总统候选人。''听众都``哦''了一声,鼓掌欢迎这位亲自莅临的总统候选人。

班纳瑞克向大家介绍了自由党:``很多人不了解自由党,以为自由党就是自由派的党。我们不是自由派,也不
是保守派,我们是关于自由的政党。人天生就热爱自由,喜欢自己做主。如果你不为自己做主,政府就会为你做主。
自由党就是要求限制政府权力、为我们争取自由的党。''

随后,主持人开始提问。第一个问题就带有浓重的宗教色彩:``请说明你在干细胞研究上的立场,尤其是联邦
政府是否应当拨款支持干细胞研究。''

共和党、绿党、班纳瑞克、纳德尔代表轮流发言。共和党人自然捍卫了布什政府目前的政策,绿党和纳德尔阵
营则偏左,觉得干细胞研究可以进行。班纳瑞克说:``干细胞研究有两种,一是成年人自愿捐赠的干细胞,那是个
人自由,我们无权干涉;第二种是由胚胎分离出来的干细胞,显然,我们无法得知胚胎是否同意自己的细胞被用于
研究,因此,自由党人在胚胎干细胞研究上不持任何立场。但是,我们相信,无论干细胞是否可以被用于科研,联
邦政府都不应该插足。科学研究应当由私人机构进行,与政府无关。人们以前总认为基础科学研究必须由政府组
织,私人机构无力负担,但即使是像航天技术这样复杂的领域,最近也有机构出于商业目的,造出了足以和美国航
空和宇宙航行局(NASA)相比的航天飞机。联邦政府不应该拨款支持任何科研。''

第二个问题是伊拉克战争。这下局势一边倒,除了布什的代表外,其余三人都反战。天主教会本来就反对战
争,这次活动的主办者又叫``天主教和平与正义'',当然来的更是反战者,于是绿党、班纳瑞克、纳德尔的代表都
得到了掌声。

按照辩论的规定,观众不能提问,唯一能发出的声音就是鼓掌以示同意,尖叫或者倒彩是绝对不允许的。今晚
由于有班纳瑞克,因此吸引了很多自由党人前来,无论他说什么,只要话音一落,吉姆立刻大力鼓掌,然后其他自
由党人也跟上,声势完全压过了其他候选人代表。不知情的人,还会以为班纳瑞克最能得到听众的青睐呢。


不过,也有几次,一些明显不是自由党人的听众也为班纳瑞克鼓掌。我本来以为既然这次活动是天主教组织主
办的,那么大部分人应该都是道德保守派,绿党、自由党、纳德尔在道德上都是自由派,恐怕讨不了好去。现在看
来这些听众都能把各个不同的议题分开,听到合意的就鼓掌,不合意的就沉默,没有把政党爱憎两极化。

后来有人告诉我,在美国,凡名字里带个``正义''的组织,跑不了是左派,``和平''也带点左倾色彩,那么这
其实是一个左倾天主教徒组织。无怪乎在后来问到的健康保险、环境问题上,绿党和纳德尔代表者也得到了一些掌
声。

克里的代表者没有能来,让我比较遗憾,不然看他当场和布什的代表者明争暗斗,一定有趣得很。纳德尔本来
和绿党在立场上就几乎没有什么区别,他四年前还曾代表绿党竞选总统,现在虽然分开,但也大同小异,都是左得
厉害。共和党和自由党在经济问题上都是右派,民主党的缺席,使我们今晚没能听到温和左派的声音。

另外,克里本人是天主教徒,美国两百多年的历史上,只出了肯尼迪一个天主教徒总统,本来克里很可以借此
吸引天主教徒的选票的,但今年由于宗教道德问题被爆炒,克里被贴上``自由派''标签,因此天主教并不支持他。
我想,如果克里的代表者来的话,大概会巧妙地搞起这点香火之情,来争取天主教徒的支持的。

问题都问完后,各人总结陈词。共和党代表者将布什这四年来的执政成绩夸了一番,重点突出了在``9.11''之
后,布什对反恐战争卓有成效的领导。他没有明言攻击克里,但几次使用``一贯''、``稳定''、``坚强''这样的字
眼来形容布什,暗示克里是个对外软弱的墙头草。

绿党和纳德尔代表者在表达立场之余,也把各自的候选人夸奖了一番,班纳瑞克不好意思自夸,但他在总结陈
词打了个绝好的比方:``政府就像火,没有它我们就会冻死,可是我们必须把它始终限制在壁炉里,不能让火蔓延
出来,烧毁我们的房屋。这个壁炉,就是宪法。宪法规定了政府的权力,凡不在规定之内的,政府就不可以做。如
果我们不限制政府,我们就将失去我们赖以生存的自由。''

总统候选人辩论进行了一个小时,然后暂时休场10分钟。自由党人在墙角的一张桌子上摆了很多资料,包括班
纳瑞克阵营的宣传、自由党人的一般性介绍、自由党的入党卡、徽章和车尾贴纸。我去看了一下,却发现除了其他
候选人也在那儿摆资料外,教堂自己也放了些传单,我拿了几张,最上面的那张上写着:``天主教徒投票指南''。

在这个``天主教和平与正义''的网站上,我看到过一篇文章,是一位意大利红衣主教写的,措辞极为严厉地指
出:堕胎和安乐死,是天主教徒投票时要考虑的头等大事,排在反战和死刑问题之前。除非你确信,某个候选人可
能会导致核战争,不然就应当以是否反对堕胎和安乐死来投票。这篇文章其实说穿了就是要支持共和党,因为民主
党大多持反战和废除死刑的立场,但共和党却反对堕胎和安乐死。这位红衣主教怕大家立场不坚定,再来注射预防
针。

我在桌旁遇到吉姆。我笑着说:``你掌鼓得不错!''他先是有点惊诧加不好意思地说:``很吓人吗?''但随后就
又神气起来:``那当然,我是个摇滚乐手!''

吉姆的正职是IT小业主,业余为一个摇滚乐队打鼓,不久前还曾到丹麦去为女王演出过。他的理论是,一个人
能做好摇滚,就可以做好任何事,比如鼓动投票(Rock the Vote)。

这时他对我说:``要不,下面你来带头鼓掌好了。''我说:``没问题。别忘了,我也是个老摇(rocker)!''

不过我虽然夸下海口,其实却还是不太好意思带头猛烈鼓掌。好在吉姆仍然保持他的一贯勇猛风格,在接下来
的联邦参议员和众议员候选人辩论中,还是自由党人话音一落,便把手掌当鼓,敲起摇滚节奏来。

自由党的参议员候选人是一位女士贝茨$\cdot$萨默丝(Betsy Summers),我在自由党候选人论坛上见过她,
是位爽朗可亲的单身母亲。她主动向我介绍自己,我正好在此前看过她的网站http://www.voteforbetsy.com,很让
我意外的是上面居然还有中文版,就跟她提起了。她立刻很认真地问我:``你觉得那个怎么样?''

说实话,我当初看的时候,差点笑死。那显然是用翻译软件从英文硬翻过来的,软件不能识别她的名Betsy(其
实是伊丽莎白的昵称),却能识别她的姓,于是通篇都称她为``Betsy夏天'',其余像把``贝茨的传记
(biography)''翻成``贝茨的生物''之类更遍地都是了。

我和她说完后,她立刻承认了,那确实是程序翻译的。我告诉她:``不用担心,全宾州只有我这么一个华人自
由党!''她立刻放声大笑,腰都笑弯了。

今天她的对手是民主党参议员候选人乔$\cdot$霍福尔的代表。共和党和宪法党的候选人说了要派代表来,却
缺席了。我本来还以为,布什的代表者也将代表参议员候选人呢。看来,他们的工作做得挺细的,并不是党部派一
个人来就完了,而是各阵营有各阵营的算盘。

不过他们的立场基本相似。乔$\cdot$霍福尔的代表说出来的话和克里也差不多,贝茨的主张也是我们极为熟
悉的。主持人问的问题和刚才总统候选人辩论完全一样,所以新意不多。


由于只有两个候选人,参议员辩论半小时就结束了,随即进行联邦众议员候选人辩论。这次共和党候选人压根
就没有来,因此又是民主党和自由党打擂台。

自由党候选人是我们的老朋友大卫$\cdot$约翰,去年选举日我曾去帮他助选。今年年初,克恩$\cdot$克
若恰科因为私务繁忙,辞去了宾州自由党主席一职,接任的就是大卫。我们当然照例给他热烈鼓掌。

辩论结束后,我去找班纳瑞克,称赞他表现不错。本来,他以总统候选人之身,与一群候选人代表在一起,就
显得鹤立鸡群,至于表达能力、演讲风度,就更完全盖住了全场。

有一个人马上问我:``你觉得他表现最好的是哪一部分?''我一愣,待要仔细回想,他却也不等我回答,自己又
滔滔不绝地开讲了。原来他是班纳瑞克的演讲顾问。

班纳瑞克自己的话倒不多,微笑着站在那里,听大家聊天。他的班子当然不能和布什、克里比,总共就三个
人:自己、司机和演讲顾问。今晚他只有这么一项活动,因此仍然神采奕奕,司机笑着说:``看班纳瑞克先生的样
子,还能再参加一个筹款晚会。''

于是他们就开始商量去酒吧聊天。我在网上查过他的行程,前几天在宾州西部时都是住旅店,到我们这里却住
进了``巴伯庄园''---其实就是吉姆的家,``庄园''一词是开玩笑,他的家也就是美国常见的那种大房子。

\section{``侃弗死'': 亚裔公寓户访}

``逐户游说(canvass)'',指由义工敲开居民的门,游说他们投票给自己的候选人。这是``选区活动(field)''
的主要形式之一,甚至可以称得上是草根政治的核心内容。现在选举进入最后一个月的冲刺阶段,阵势已经摆开,
下面就该冲锋陷阵式的逐户游说,双方短兵相接、刺刀见红了。

我不知道中文里``canvass''这个词译作什么,要我说,就叫``侃弗死''好了,不过我们暂时还先把它称为``户
访''吧。克里阵营自然一直在组织此类活动,最近发出Email,号召大家去参加户访,号称是有史以来最大的一次,
将敲开100万户门。我本来也打算参加,周四却忽然收到一个叫诺亚的人的Email,说他在费城为``前进网
站''(MoveOn.org)做户访时,遇到了很多不说英文的中国人,问我是否可以在这个周末过去帮忙。

我想,在郊区也不缺我一个,还是到城里去更能发挥我的作用吧,于是就答应了他。不过,据我所知,城里的
很多中国人,如果不说英文的话,基本也不说普通话,而讲粤语。我邀请另一个讲粤语的朋友同去,他没有回信,
我只好自己去了。

MoveOn.org是一个以反布什为己任的网站,曾经悬赏征集最佳的反布什广告,还筹款打算在电视台播放。选举
开始以来,他们更是兴高采烈地高唱参战,网站的首席执行官更成为了克里竞选阵营的网络宣传总干事。发誓要推
翻布什的金融大鳄索罗斯,一口气给它投资了数百万美元,使得他们可以频频重拳出击,包括户访。诺亚事先给我
寄来了MoveOn.org准备的户访说辞:

您好!我叫\_\_\_\_,请问\_\_\_\_在吗?

$[$如果你找的人在,继续;否则问他们什么时候会在,然后感谢与你交谈的人。$]$

我就住在\_\_\_\_$[$邻近地区$]$,正在为选举做义工。请问您决定了要投票给谁吗?

1、$[$克里$]$

太好了!我是MoveOn PAC(Political Action Committee,政治活动委员会)的义工,力图击败布什,让克里当选。
为了能够动员投票,我们正在整理一份克里支持者的名单,您能给我您的电话号码和Email地址吗?(递过笔记本)

可以:非常感谢!

不肯:OK,最后一个问题。当您决定选总统时,哪项事务是最重要的?(笔记本上列出了经济和工作机会、教育、
养老金、反恐、伊拉克战争、健康保险等议题。)

好,谢谢您的时间,\_\_\_\_$[$名字$]$,谢谢你的支持。请记得在11月2日投票。这次选举非常势均力敌,但
如果我们能够都去投票,我们就一定能打败布什。这里是一张MoveOn PAC关于这次大选的事实对照表。如果您愿意
帮助我们在这一带做义工,我们很高兴欢迎您来参加我们下次的会议,并且开始参与活动。给我打电话或者去我们
的网站就可以了,这些信息都在那个事实对照表上。

2、$[$尚未决定 $]$

好的,那么下面的这些立场哪个跟您最接近?(递过立场表)

当您决定选总统时,哪项事务是最重要的?

好,谢谢您的时间,\_\_\_\_$[$名字$]$。我是MoveOn PAC的义工,我们相信布什在伊拉克、经济和其他问题
上都走向了错误的方向,因此我们力图使克里当选。我想给您这张事实对照表,并且希望您将决定在11月2日投票支
持克里。

3、$[$纳德尔$]$

您可以说一下您是倾向支持纳德尔还是坚决支持纳德尔?

倾向支持纳德尔:好的,谢谢您的时间,\_\_\_\_$[$名字$]$。我是MoveOn PAC的义工,我们相信布什在伊拉
克、经济和其他问题上都走向了错误的方向。打败布什的最好办法是投票给克里。我想给您这张事实对照表,并且
希望您将考虑在11月2日投票支持克里。

坚决支持纳德尔:OK,谢谢您参加我们的调查,祝您愉快!

4、$[$布什$]$

OK,谢谢您参加我们的调查,祝您愉快!

我在星期六中午到了诺亚的住处。他住在费城,我们就各种问题讨论了一会儿,主要是我对布什、克里的某些
政策并不熟悉,需要预先准备好别人提问时的回答。他给了我一摞表格,上面都是附近一座公寓楼里居民的信息,
有的还有电话号码。大部分人他都访问过了,但是由于遇到一些不说英文的亚洲人,他就直接跳过那些名字看上去
像是亚洲人的住户。

他告诉我说,这上面都是以前投票支持过民主党,但是并不总是去投票的人,其中有个住六楼的简女士是个铁
杆民主党,把联系方法都留下来了。我松了一口气,说:``那就好办多了,不过那些中立选民难道就没有人去游说
了吗?''诺亚说:``我也不知道。我被分配到这些人。其他人大概有别人去游说吧。''

我们步行了5分钟,就到了那栋公寓楼。诺亚把所有的表格都收进了他的书包,说:``可不能让保安看见这个了。
我们就说去找六楼的简,不然他们不让进的。''我奇怪地问他:``你上次怎么进去的?''他说:``上次正好没保安,
我就溜进去了。''

走进大楼,今天有保安在值班,诺亚走上前说:``我们来找六楼的简。''保安并不怀疑,说:``那你们需要先
登记。''我们在一张表格上填了名字,便坐电梯上了六楼。在电梯里,诺亚又掏出表格,交给了我。


按照名单,先找到第一家。我按了下门铃,一位华裔妇女开了门,我笑容可掬地问她:``您好!您讲中文吗?''
她有些茫然地看着我,我连忙用粤语说:``广东话?''她便点头,说了一串我听不懂的话。

我以前有两年时间颇听过一阵粤语歌,知道一点点粤语和普通话之间互换的规律,不过长久不用,早就忘光
了,只会说这个``广东话''和``多谢''。我试着用广东腔问:``国语?''她竟然也听懂了,回以摇头。大家尴尬地各
说各的话,一会儿(我试着说了几个广东词,像``投票'',结果不要说她听不懂,我也不知道我在说什么),我只好
说声``多谢'',狼狈撤出了。

出师不捷,诺亚倒不气馁,按着名单就要去下一家。我却长了个心眼,沿着名单查了一下,决定先去走廊尽头
的一家人。走到门前,这家人的门虚掩着,我正要按门铃,诺亚提醒我说:``你按的时候要当心,不要把别人的门
乘势推开了。别人开门后,你也不要走进去,站在门口跟他说话就行了。''

这次出来的也是位女士,我照例问她:``您说中文吗?国语?''她说:``是。''

我这下如释重负,也想不起来MoveOn准备的那个说辞,直接就问她:``您打算今年11月2日去投票吗?''

她说:``会啊---你们是?''

我连忙解释说:``我们是来自MoveOn.org的义工,是来做调查的。我想问一下,您打算投票给谁呢?布什还是克
里?''

她说:``哦,是这样的,我投不了票,11月的时候我在中国大陆呢。''

我又问:``那如果您可以投票的话,您想投给谁呢?''

她毫不犹豫地回答说:``克里。''

这下我兴奋起来了,更神奇的是,诺亚在旁边就像听懂了我们的话似的,从书包里掏出一摞纸来,对我
说:``问她是否需要缺席投票申请表。''

我问了之后,她说:``我不需要。我跟朋友说了,他可以帮我处理投票。''

我有些奇怪,又说:``这里就是缺席投票表,你填了就可以投票了,不用找其他人了。''

她说:``填了这个就可以了吗?''我却也不太有把握,又问诺亚,诺亚解释说:``这只是缺席投票申请表。你填
了这个之后,我们帮你把这个交上去,然后政府会把缺席投票表给你寄过来,你就可以缺席投票了。''

我这才知道我搞错了,连忙给她解释。她大概也觉得麻烦,说:``算了,我还是让朋友帮忙处理。我都托好人
了。''

我说:``那谢谢您支持克里。''便和诺亚离开了,一路上给他解释说,中文姓名的英文拼法,中国内地和香港
是不一样的,所以从名字上往往就可以推测这个人是来自中国内地,还是香港。不过这门学问就太高深了,诺亚听
来听去也没明白。

显然,这栋楼里的中国人大部分来自香港,采取中国内地的拼法是少数。不过既然开张过一回生意了,下面我
按图索骥,一个个来就是了。

幸运的是,下一个为我们开门的老先生虽然很明显是香港人,但也说国语。我介绍了来意后,他热情地邀请我
进去,我推辞不过,只好和诺亚进去了。他让我们在沙发上坐下,自己进里屋拿出一张卡片来,问我们:``凭这个
可以投票吗?''

我看了一眼,上面写着``投票证。''诺亚看过后说:``可以,凭这个就可以了,如果您上次就是在这儿投的
话。''我给他翻译过去,那个老先生说:``是的,我上次刚刚在这里投过票。不过,前几天我在华盛顿街看见中华
妇女联合会在办选民登记,我就又登记了一次,有关系吗?''

诺亚说:``没关系。只要您以前在同一地点投过票,您就不需要再登记。不过多登记一次也没事。''

我问老先生:``那您打算投谁的票呢?''

他说:``我还没想好。''

我终于开始正职工作了:``我们亚裔应该把票投给民主党的候选人。您知道,民主党更关心弱势群体,共和党
则更注重主流社会。为什么黑人总是投民主党的票呢?就是因为他们知道民主党才会帮助他们,像平权法案哪、反种
族歧视啊,都是民主党在搞。我们亚裔在美国是少数,也应该支持民主党。还有像健康保险啊、养老金啊这些东
西,也是民主党做得多。还有现在的布什,在国际关系上飞扬跋扈,像牛仔一样乱搞,到处捅篓子,这也会影响中
美关系,对香港的稳定繁荣也不好\ldots ''

我一边说一边观察那老先生的反应,看他似乎没被说动,又没什么新辞了,只好把话转轱辘又说了一遍。诺亚
在一旁看见我一开口就停不下来了,不知道是怎么回事,乘我短暂停息的间歇,问我:``你在说什么呢?''我说:``我
在劝说他呢。''

好在功夫不负有心人,在我的车轱辘转第三遍的时候,老先生终于点了点头说:``我现在基本拿定主意了。''
我心中暗喜:``那您打算\ldots ''他说:``我要投票给民主党的克里。''

总算大功告成了,我谢过他,提醒他一定不要忘了去投票,热情地告诉他,我们在投票前的那个周末还会再
来,如果需要开车接送去投票尽管和我们联系等等。最后,宾主皆欢喜地说再见。

诺亚看他再跟着我也没有什么事可做,就决定自己单干去了。临走前,他敲开了简老太太的家门,因为他们事
先约好她在家里等待,万一保安不让他进来的话可以让她出面。诺亚谢过了简,介绍我和她认识,解释了一下今天
的进展。简非常热情,问长问短,还说她的邻居们虽然不说英语,但是人都很好。


诺亚走后,我又继续行动。有好几家没人,不过随后又敲开了一家说国语的,一位老先生也邀请我进去坐,我
没再进去,站在门口向他解释了来意。他很干脆地说:``我投票!我一向都投票给民主党!''他的话非常直率:``民
主党是穷人的党,共和党是富人的党,我是穷人,不敢高攀共和党。''

虽然他支持民主党,不过他也担心,这次大选恐怕还是共和党会获胜。他问我:``我不太明白,上次明明不是
戈尔胜了吗,怎么最后是布什当了总统呢?''我给他解释了一下选举团制度,但是一时半会儿也说不清楚。最后我告
诉他:``现在宾州的选举情况非常接近,每一张票都可能改变最后结果。宾州的结果对最后谁当总统是至关重要
的,所以我们才需要您去投票。只要您去投票,大家都去投票,我们就可以把布什赶下台去。''

老先生说:``我一定会去投票的。''他摇了摇头,``按理说,美国对我们真不错,我们有社会救济领取,有安
全保障,我不该说美国的坏话。可是,这届政府实在干得不怎么样,我自己知道,这几年的日子是一年比一年差。''

我再次激励他去投票,谢过了他,就去下一家了。

这一家人又是只说广东话,我和女主人``交谈''了几句,无法继续,正要走开,她却极聪明地拿出一张纸来,
让我写下我的话。我写道:``您投票吗?''

她看了看,边点头边说话。我听不懂她在说什么,只好又写:``布希?柯瑞?''---这是我在繁体字报纸上看来
的对``布什''和``克里''的翻译。她肯定地指向``柯瑞''。我继续写道:``我们会再回来,有会说广东话的朋
友。''当然,用我所能想到的繁体。

她点了点头。我谢过了她,正要离开,楼道里又走过来一位亚裔女士。她们俩用广东话聊起天来,刚才和我交
谈的女士还指着我给她看,我连忙把那张纸也给她,她在第一位女士的解释下,也点头表示会投票,并指向``柯
瑞''。

话犹未了,一位亚裔男士走了过来,和她们打招呼,似乎还说了几句话。我试图和他交谈,但``国语''和``广
东话''都没有用,英语也不行,最后他嘴里冒出三个字:``柬埔寨''。这下我彻底放弃了。

有了这张纸后,我的工作就容易多了,再遇到说广东话的人,就把纸递过去,指着上面的问题比划。不过可能
因为是星期六下午吧,大部分人都不在家。很快,我就结束了这栋楼的工作。

在来到楼道口等电梯时,我发现他们在每一层都有一个布告栏,上面发布着各种信息,其中有一张纸上是关于
选民登记的。由于这里是个老人公寓,大部分人行动不便,因此有人专门到楼下大厅里来,帮助他们来登记成为选
民。这张纸旁边还贴着一个中文版,把一切都翻译得清清楚楚。

这个布告栏上的中英文传单几乎各占一半。大部分通知都有中文版,比如去赌场的汽车之类,都是手写的中
文,看来是本楼的义工翻译的。只有一个音乐会的通知纯是英文。纯中文的是一张本楼华人联谊会的财政报告,那
可真是五花八门,收入主要是某人惠赠和某机关拨款两类,支出则有寿星彩礼、某酒楼开张大吉之类,煞是有趣。

下楼后,我和在另一处户访的诺亚又合兵一处,这时他的一个朋友也来帮忙,我们三人一起前往第三栋公寓楼。
那个地方离得颇远,我们在市区里步行了15分钟才到。进去一看,几个保安端坐在桌子后,想混进去是不可能的,
诺亚只好上前说道:``你好!我们是来自MoveOn.org的义工,请问我们可以进去做些调查吗?''

保安理所当然地拒绝了他。

出门后,我说:``其实我们可以采取这种策略:这张表上既然都是以前投过民主党的票,那么很可能也有像简
那样的热心支持者,我们下次可以先打电话给他们,肯定能找到这样的一个人,那么我们就可以进去说,我们是来
找某某的。现在弄得像推销员似的,保安当然不让进。''

诺亚说:``对,我也打算下次采取这样的策略。''

于是我们就又走了回去。时当初秋,风和日丽,户访之余,漫步费城,亦一快也。

\section{两大党的总统辩论}

自从共和党开完全国大会,再加上游艇老兵的攻击,克里的支持率便大幅下跌,现在已全面落后于布什了,在
几个关键的摇摆州,也由以前的微弱优势领先,转为微弱落后。最近,《华氏``9.11''》的导演迈克尔$\cdot$摩
尔又发出了一封Email,专门来给大家打气。

他直接了当地说:``民意调查是错的。''原因有三:``第一,他们是调查那些'可能的投票者','可能'是指在
过去的选举中持续投票的人,那么就不包括这次将第一次投票的年轻人,和很多以前不投票但这次决定要投票的人。
第二,通过住宅电话进行的民意调查不找那些主要使用手机的人,而这又意味着他们不调查年轻人。最后,正如调
查者约翰$\cdot$左格比(John Zogby)在上个星期所揭示的,大部分民意调查都太倾向于共和党。你要是相信那些
民意调查,那可就太蠢了。''

他狠狠地嘲笑了那些悲观者一番,甚至承认说:``这就是为什么我们私下里敬仰共和党人:他们冷酷无情,从
不放弃。''摩尔随后一段话说得极有趣:``如果我再听到谁跟我说克里是个差劲的候选人、毫无胜机\ldots 他妈
的,他当然是个差劲的候选人---他是个民主党人!这个可怜的政党,居然把上次赢得的选举又输掉了!你还能指望
什么,布鲁斯$\cdot$斯普里斯汀(Bruce Springsteen,坚决支持克里的超级歌星)来竞选?布鲁斯会是个超级总
统,但他那样的人不会来竞选---你我也不会。克里这样的人才会去竞选。是的,当然我们谁出马都会比克里干得
好,当然我们会把那些骗人的游艇老兵砸个稀巴烂,可我们没来竞选总统---克里来了。那么就不要再抱怨了,我
们唯有从实际条件出发。''

摩尔来信的要点就是给大家树立必胜的信心,他在信的最后说:``醒来吧!我们才是大多数!一半以上的美国人
是赞同多元化、要求加强环境保护、反对攻击性武器的---而且54\%的人相信伊拉克战争是错的。你都不用去说服
他们---你只要去给他们一线希望,并开车带他们去投票。你能做到吗?你会去做吗?''

最近就有个机会摆在克里面前:总统候选人辩论,而且不带其他第三党或独立候选人一起玩,就是他和布什面
对面直接交锋。第一次是在9月30 日,在佛罗里达的迈阿密大学举行,主持人是来自公共电视网(PBS)的吉姆
$\cdot$莱勒,他轮流向两位候选人发问,他们有2分钟的回答时间,另一方有90秒的回复时间,如果需要,还可
以给双方再加各30秒的辩论。观众无法提问,也不能鼓掌或发出嘘声,只能保持绝对缄默。所有的问题都由吉姆自
己撰写,并且不曾泄漏给任何人。候选人只能回答问题,不能向对方提问。

辩论开始时,布什和克里分别从舞台的两侧步出,握手致意。不出我所料的是,布什首先松开手,回到自己的
讲台,克里还在那里意犹未尽地向大家挥手。因为布什比克里矮一头,所以他显然不愿意和克里站在一起太多时间。
听说事先双方签了个厚达24页的协议书,详细规定了辩论里的各方细节,包括讲台的高度,及相互距离。布什一方
希望两个讲台离得远些,以免布什看起来不够高大,并且在辩论中两人不得随便走动。

不过说实话,他们在辩论中说的话我都早已听到过多次了,双方无非一再强调重复而已。不久,辩论就由剧情
片转变为效果片,我觉得看他们俩说话,比听他们在说什么要更有趣。由于克里比布什高出一截,而讲台是一样高
的,使得布什站在那里,只露出胸和头,必须以肘支桌,看上去很矮小,克里则可以将手自然地摆在讲台上,露出
整个上身,形象要顺眼得多。

有些电视台将屏幕分为两半,同时播发两人画面。布什在克里发言时,总做出一副无辜的样子,表情十分可
爱,加上他的身材看上去比较小,远看便像个小孩子一样。克里的身材高瘦,在布什发言时总是面带微笑地做笔
记,看上去像在上课的大学生。更有趣的是,由于两人的身高有差别,每当镜头由克里单独发言切换到两人共存
时,摄像机就会调整高度,使两人看上去肩并肩一样高,那时,克里的讲台就会低下去一大块。

辩论结束后,一堆政治评论员马上如雨后春笋般地在电视屏幕上冒了出来,开始左分析、右分析,其中某些人
是有鲜明的党派立场的,自然一边倒地为自己的候选人叫好,另外一些中立的评论员则意见各殊,但基本上都认
为,双方表现都不错,欲知后事如何,还听下回分解。电视台乘机做广告,提醒大家后面还有一场副总统候选人辩
论,两场总统候选人辩论,欲知后事如何,还看下回电视。

我则另外还有事情可做:我早就收到克里阵营的Email,要求大家在辩论结束之后,上到各个媒体的网站参加投
票,务必要使克里的支持率超过布什。他们的口号倒也堂皇:``不要让共和党又偷走一场胜利!''看来他们也向共和
党学到了一点,就是不管自己有没有胜,都要做出胜的样子来。用我第一次参加克里支持者聚会时宣传组的话说就
是:``印象即事实''。他们办展示秀,在十字路口展示克里的巨幅招牌,也正是同一个意思:让选民知道,克里有
大量的支持者,即将赢得选举。

以前我总想不通,美国IT泡沫破裂之后,那么多首席执行官(CEO)都靠什么吃饭去了,现在我明白了:他们大部
分都投身政治来了,敢情选总统和做网站是一个道理---``炒''。老子曰:``治大国若烹小鲜。''这个版权还是我
们中国人的。


Email里早就给出了各大媒体的链接,我上去一看,有几家已经开通了投票,让观众说出心目中的辩论胜者,结
果自然是克里压倒性获胜。可惜这样的结果实在没有什么可信度。

第二天早上,我打开电子信箱,哇塞,民主党乘胜追击,又发了一封Email,不仅仅要求我们去网站投票,还要
我们给当地的电视台、广播台打电话,甚至还要给报纸写信。

随后在10月5日举办了唯一一场副总统候选人辩论。我对这场的兴趣,其实还超过总统辩论,因为双方一是老谋
深算的官僚,一是形象阳光的诉讼律师,比他们的老板---话都说不好的德州牛仔和话说得别人听不懂的老政客,
肯定有趣得多。

果然,在辩论里双方的表现都极为精彩。套句老话说,如果话语是子弹,那切尼早已被打成了筛子,而爱德华
兹也身中百弹,光荣身亡。这次辩论的规则和上次总统辩论完全一样,即主持人提问、2分钟的回答时间、90秒的回
复时间及可能的各30秒的再回复。辩论前一半时间是讨论反恐和战争,后一半时间讨论国内政策。两人的立场当然
和各自的总统候选人完全一致,但火力之猛,即使全球的恐怖分子都站在他们面前,也能被他们枪毙三遍了,只可
惜他们这火力都给了美国人自己。

上次民主党人事先组织大家在总统辩论后去各大媒体网站投票,结果造成克里压倒性胜利的假象,共和党人这
次也学乖了,也发出Email来(我也参加了他们的邮件组,知己知彼么),要求支持者去投票。但是互联网使用者中,
以支持克里者为多,所以这次的网站投票,仍然是爱德华兹胜,只是优势不像上次那么大而已。

两天后举行的第二次总统候选人辩论,形式变成了观众提问。观众都是中立方选出的中立选民,每人给两个候
选人各准备一个问题,由主持人来挑选。候选人的回答时间和上次是一样的,即2分钟回答时间、90秒回复时间及可
能的各30秒再回复。由于问题是观众来提的,因此论题不限,角度也不限。

事实证明,劳动人民的智慧是无穷的,集体的力量也是无穷的,这些观众从四面八方的立场提出的问题,如果
不比名牌主持人苦心研究出的问题更好,至少更``新鲜''。比如有人要求克里直视摄像机,``用简单和明确的语
言'',承诺在他的一个任期内,不给年收入低于20万美元的家庭加税;有人要布什举三个意识到自己错误的例子,
并讲一讲是如何纠正它们的;还有人引用家人在国外旅游的经验,表示担心外国人对美国人的恶感。这些问题大都
有鲜明的个人色彩,不像主持人提出的问题,学究气太重。我一直耳闻,知识分子倾向于支持克里,大老粗倾向于
支持布什,以前总是对这种现象不以为然,现在我发现,平民百姓的语言就是更有亲和力,无怪乎抛开政策不谈,
就个人来说,美国民众更喜欢布什。

辩论后的分析和采访中,仍然和以前一样,双方都各自宣称获胜,中立评论者则大多说双方表现都不错---虽
然在我看来,其实是双方表现都很糟糕。大家都是一团和气,揭丑的重任最后自然落在了``周六夜直播(Saturday
Night Live,以下简称SNL,一个搞笑娱乐的电视节目)''身上。

SNL的本季度开场秀,就是模仿总统辩论,有人摹仿``节目主持人''说:``我们的规则是,每个人都有两分钟时
间回答问题,然后布什总统将把所有问题都引到``9.11''上去,而克里参议员将提醒您他在越南得到过的勋章。''

在对第二次总统辩论的模仿里,``布什''总是上窜下跳,挤眉弄眼,屁股坐不住似的,而``克里''仍然手势不
断,尤其是在说到``大规模杀伤武器''时,就用手画箱子状,说到``核武器''时,就用手画导弹状。至于内容,对
布什的嘲笑除了拿他的``internets''(布什在辩论中把互联网说成了``互联网们'')口误做文章外,还让布什扮演者
大义凛然地说:``你口口声声说自己支持部队,你怎么会不投票支持增加军费案?你如果真的支持部队,怎么会不支
持自己国家的总统?你如果真的支持部队,就根本不应该出现在这场辩论中!''

这个说法算是击中布什阵营的要害了。道理是显然的,支持部队就一定要投票支持增加军费案吗?难道无论多么
荒谬的提案,只要是要增加军费、提升前方将士装备,就必须支持吗?那以后这种提案都直接通过算了,扣上这么大
一顶帽子,还有谁敢反对呢?托马斯$\cdot$杰斐逊有句名言说:``异议是爱国的最高形式。''布什阵营的拿手好
戏,就是把问题黑白化,非善即恶,反对我们的做法,就是反对我们的立场,就是反对自由,就该打下十八层地狱。
建议联邦调查局(FBI)去卡尔$\cdot$罗夫(布什的智囊、竞选的总设计师)家里搜查,我担保能搜出一本《梁效文
集》来。

对克里,嘲笑则集中在他反复说``我有个计划'',却从来说不出计划的细节上。克里扮演者无论在回答什么问
题时,总不忘记说几遍``我有个计划''。

在对副总统候选人辩论的模仿中,搞笑的重点是切尼女儿的同性恋。英文中的``straight''一词,既指``坦
率'',又有``异性恋''的意思,``爱德华兹''便说:``现在我们看到了,不仅总统、副总统对美国人民不
straight,连副总统的女儿对美国人民也不straight!''


总的来说,SNL的老巢虽然是在自由派横行的纽约,但并没有好克里恶布什,而是看谁身上有笑料就拿谁开心,
对双方基本上一视同仁,都作虚伪观。搞笑要紧,政治靠后,这个立场我很喜欢,应了那句话:``开口便笑,笑天
下可笑之人。''可是,当他们的观众,也得``大肚能容,容世上难容之事。''他们的笑话经常让观众``啊''出来,
尺度让观众都受不了,不过我是反正没有``政治正确''观念的,还是比较欣赏SNL的胡搞。

\section{远交近攻}

作为一个为克里助选的自由党人,我一直觉得,克里阵营和自由党其实可以合作,因为自由党主要是吸引保守
派的选票,这无疑会削弱布什的得票率。如果自由党的竞选活动取得较大成功的话,很可能布什就会因此输掉整个
选举。

这种挖墙角的事情在选举中经常发生,比如1992年的总统大选里,独立候选人佩罗就吸引了很多保守派的选
票,是老布什输给克林顿的主要原因之一。更有名的例子则是四年前,绿党总统候选人纳德尔,以比民主党更左的
姿态出现,抢走了大量的左翼选票,尤其是在后来决定了总统归属的佛罗里达州,他得到了9万多票,而布什和戈尔
的差距不过是区区500票。其实,不用说佛罗里达这样的大州,就算是只有四个选举人票的新罕布什尔州,只要纳德
尔不参选,戈尔就能从布什的手中夺得此州,那也足以改变整个选举结果。

我在5月27日的《第二次克里支持者聚会》里,就提到过我想动员民主党人来帮助自由党人,结果不太如意。绝
大部分人,包括媒体,对第三党候选人的注意力主要都集中在纳德尔身上。不过所有的人都预测,他绝不会再得到
四年前那么高的选票,因为大部分左翼选民自己也知道布什连任的后果。

我在见到自由党总统候选人班纳瑞克的时候,就问他是否和克里阵营联系过,也许民主党会乐意帮助他从保守
派那里拉选票。他说他们试过了,民主党人没有回音。

在上周的自由党月会上,我又提出了这个建议,即在宾州范围内寻求民主党人的帮助,大家都认为是个好主意。
我随后从自由党的主页上找到了一些资料,主要是媒体的分析和报道,即预测班纳瑞克将成为今年的纳德尔,极大
地帮助克里。我把这些报道都打印下来,在上周六做完费城里的``侃弗死''后,就开车来到郊区的克里总部,准备
再次``侃弗死'',说服他们帮助自由党人。

那天杰娜丽不在,我和一个小伙子聊了起来。很显然,他们不会直接公开地与我们合作,这个小伙子也明显地
不太相信自由党能够起到任何作用。他在听我介绍完来意后的第一句话就是:``我们一般认为,第三党候选人是会
损害到民主党的\ldots ''在我费了半天口舌解释清楚了自由党的立党宗旨后,他才勉强同意自由党参选损害的会是
布什。

一个很重要的事实是,在今年的参议员初选中,共和党寻求第五次连任的老资格参议员阿伦$\cdot$斯柏克
特,是个花起钱来堪比民主党的``中间派'',他的对手帕特$\cdot$图米,则是个传统意义上的共和党人。初选刚
开始时,图米遥遥领先,后来布什公开支持斯柏克特,甚至为他助选,最后斯柏克特才以51\%对49\%的选票险胜。
这正说明很多共和党人还是更认同传统的共和党观念,即``小政府'',而不是布什他们的``新保守派''。

最后,这个小伙子同意把我留下的资料转给杰娜丽。三天后,我又回到那里,这次找到了杰娜丽,但是她显然
没看那些资料。我重新把事情给她解释了一遍,她显得很感兴趣,不过,她不可能给我们任何财政、人手上的支持。
我看她至少不反对,就告诉她我会给``费城克里支持者''的邮件组发信,动员大家给自由党捐款,我们会专门给共
和党人寄信,来动摇布什的根基。

我把和杰娜丽她们所侃的内容写成了一封Email,发了出去:

亲爱的克里支持者们:

拉尔夫$\cdot$纳德尔仍然在努力要使自己出现在宾夕法尼亚的选票上, 但是我们不需要为此烦恼,因为乔
治$\cdot$W$\cdot$布什正面对一次更大的威胁: 自由党候选人迈克尔$\cdot$班纳瑞克。

事实是,今年有许多共和党人由于布什的财政计划,不愿意再投票给他。想想看,即使在布什公开支持阿伦
$\cdot$斯柏克特并且为他助选之后,帕特$\cdot$图米仍然在初选里得到49\%的选票!共和党人没有我们想像的
那么团结!

一个保守党候选人对于布什,正如4 年前的纳德尔对于戈尔。斯柏克特知道这一点,他在电视辩论里一遍又一
遍地说:``给克莱默(宪法党参议员候选人)的一票,就是给乔$\cdot$霍福尔(民主党参议员候选人)的一票。''霍
福尔知道这一点,克莱默也承认了他的参选签名有相当一部分是来自霍福尔的义工。

共和党人一直在玩这种游戏。他们正在暗地里资助纳德尔,并且帮助纳德尔进行参选签名。我们必须以其人之
道,还治其人之身。自由党总统候选人迈克尔$\cdot$班纳瑞克从保守派那里吸引到很多票数,民意调查显示,他
在新墨西哥州能获得5\%的选票,在内华达则有3\%。他能使这两个摇摆州摆向克里。

他的竞选阵营已经意识到了经济保守派是他的基本选民。``宾州支持班纳瑞克''计划给图米支持者寄信,但是
没有足够的资金。我鼓励你们捐款,这里是链接:http://tinyurl.com/4cap9,或者写一张支票给``宾州支持班纳
瑞克''。

克里无法承受失去宾夕法尼亚州。这里太重要,我们必须尝试所有可能给我们带来胜利的方法。当共和党人仍
然在计划他们的``十月惊讶行动'',并且为纳德尔提供资金时,我们能通过投资我们自己的秘密武器来让他们大吃
一惊!一个保守党对宾夕法尼亚州的每个登记在册的共和党人的邮件攻势,肯定把共和党人对布什的支持率刮掉好几
个百分点。也许只有1~2\%,但是在如此难解难分的一次选举,这还不够吗?


请也把这电子邮件转发给其他克里的支持者,并且告诉别人这个主意。现在只剩19天了,不过找到一个``敌人
的敌人''仍然不算太晚,可是我们真的必须赶快了!谢谢!

这封信会引起的结果无非三种:一是有回信说这是个好主意,踊跃捐款;二是有回信说这没啥用,甚至怀疑我
的动机;三是没反应。前两种结果都不错,有正面回应最好,即使是反对意见,我也可以进一步解释。不过,和我
事先的担心一样,这封信发出后便如石沉大海,毫无反应。

随后我发现信里的那个链接其实是为班纳瑞克的电视广告捐款的卡片,我想建议吉姆改一下(他是``宾州支持班
纳瑞克''的主席),就给他打了个电话。他告诉我,只要捐赠者在卡片上注明,捐款专用于邮件攻势就可以了,并给
我的Email一些其他信息。我据此给``费城克里支持者''又发了一封信,用网络论坛的的黑话来说,就是把自己的帖
子给顶了上去:

下面的链接会显示是电视广告。我给``宾州支持班纳瑞克''的主席打了电话, 他告诉我,捐赠者可以在支票或
卡片上注明,你希望捐款被用于邮件攻势。

这是班纳瑞克针对共和党人的一个电台广告,可供您参考: http://badnarik.
org/Multimedia/BadnarikConservative\_national.mp3

他也告诉我说,一封信只花37美分邮费,因为打印是免费的。这样的话,哪怕收到信的共和党人中,只有3\%的
人改变主意( 图米得到了49\%的选票!) ,您每捐赠10美元,就能去掉给布什的一票!

现在只剩两周时间了,我们必须赶快行动,并且尝试一切可能的方法!请把这封Email也转发给其他决心要打败
布什的人,和/或已经给克里捐赠了2000 美元的人。谢谢!

随后我也加入了``匹兹堡支持克里''的邮件组,给他们寄了一份。

可是看来结果并不理想,我后来打电话给吉姆,他告诉总共只收到一笔匿名的捐款:20美元。看来,这一次
``远交近攻''又失败了。

\section{香港超市散发传单记}

作为亚裔,我最想做的其实是在中国城帮助克里竞选,就像上次帮诺亚在亚裔集中的公寓楼户访一样。不过,
和克里阵营的联系一直不太顺利,他们总说中国城已经组织得很好了,我们无需再去重复,只要在其他地区发发传
单即可。两周前,我加入了一位叫林(Lien)的义工的队伍,帮助她在费城南部的一家``远东超市''前发传单。最近
她又给我发来Email,邀请我去费城东北郊的香港超市发传单,尤其是由于她早上有事,不能参加,上午9点半到中
午1点这个时间段严重缺乏人手,我当然就义不容辞地顶上。

和我一起发传单的是林的先生明,他带来了全副武装:桌、椅、传单、大幅标语牌,甚至还有缺席投票申请表。
另外他们的两个孩子,也坐在婴儿车里,放在汽车外,大概是因为夫妇两人都外出了,没人照顾吧。我吃了一惊,
因为当时的天气挺冷的,我说:``你不如把他们放回车里,外面太冷了。你在车里照顾他们,这儿我一个人发传单
就够了。''

他说:``没关系,他们也不喜欢总呆在车里。---你觉得桌子摆在哪儿好?''

我说:``出口吧。人家进去的时候都要买东西,谁会愿意看传单呢?''他同意了。我们把桌椅设置好,将大幅标
语牌贴到墙上,徽章、贴纸、车尾贴纸都在桌上排开。还有一个大筒,我往里一看,都是选民登记的宣传资料,印
刷得极为精美,各种亚洲语言都有,可惜今天用不上了。我粗粗地将传单看了一下,有彩色印刷的``克里-爱德华
兹''标准传单,有最近《费城问讯报》正式支持克里的文章,有专门解释为什么亚裔应当选克里的传单,还有一套
攻击布什的健康保险计划的小文件夹。我各样都拿了一些,便站到出口处,进入战备状态了。

没想到香港超市一大早就有这么多顾客,出口处人流不绝,大概每5秒钟就有一个人推着购物车出来,或单身、
或夫妇、或一大家子。我看菜下料,人过来时就先说声:``早上好!''一般人们都会回一声:``早上好!''这时我就
递过一张彩色的``克里---爱德华兹'',如果是亚裔,就再给一张亚裔传单,说:``请投票给约翰$\cdot$克里,
他是我们亚裔的选择!''人们接过传单后,有的直接仍进购物车,有的则开始看起来。如果有机会,我还会再多说几
句,由于时间有限,当然只能采取最直接的方法:``布什是富人的总统,我们亚裔是少数民族,应当投票给民主
党!''另外,如果是老年人,我会分发那份健康保险宣传品,加上一句:``布什的健康保险计划完全失败,我们必须
阻止他!''

对于非亚裔,我就给那份《费城问讯报》的文章,至于说辞,则无需多费心,因为这家超市的顾客,除了亚裔
外,主要就是黑人兄弟姐妹。只要我说句:``请投票给约翰$\cdot$克里。''大部分黑人就会回答一句:``我会
的!''或者``哦,那当然!''这时我只需对他们说:``请别忘了去投票,11月2日,两个星期后的星期二!''还有些经
过香港超市的黑人,在看见我们的大牌子后,会对我们大喊:``约翰$\cdot$克里!''我便跟他们一起喊,并提醒
他们别忘了去投票。

亚裔当然也有人会反应这么积极,不过是少数。我想,可能一来是因为亚裔本来就不像黑人那么外向,大部分
人都只是拿过传单后自己看;二来是亚裔的政治立场分歧较大,尤其是越南裔,和其他亚裔多半支持克里不同,越
南裔倾向于支持布什。我和明讨论过这个问题,他说:``越南裔大部分是难民身份来到美国的,对共产主义特别痛
恨,共和党的对外政策更强硬,因此越南人更喜欢共和党。''

我对这种看法很惊讶:``难道他们指望布什入侵越南吗?''我想说:和平演变比遏制对抗更有效,可是不知道英
语怎么说,只好说:``共和党的对外政策强硬,不代表就对越南好。事情不是非黑即白那么简单的,美国人吃强硬
的苦头还没有吃够吗?''

``道理是这样的,''明说,``可是很多人就是不明白。''

我想,也是,很多人支持布什,本来就是因为布什够简单,一切都一清二楚,非敌即友、邪恶轴心、指哪打
哪,多过瘾。我今天确实见到几个人很干脆地说:``我会投票给布什!''或者``谢谢,我不喜欢克里。''

还有很多人不是公民,没有选举权,有人直接告诉我说:``我不是公民。''把传单还给我,还有人则半开玩笑
地说:``我也想啊!''有位女士还说:``我的心在他(指克里)这边,我总希望布什能够下台。不过我无法投票。''我
说:``那您可以告诉您的朋友投票给克里。''

明比我有经验,他迎上去说:``那您可以考虑做点义工吗?''把她引到桌子旁。他一直守在那里,给感兴趣的人
散发徽章、贴纸、车尾贴纸,另一项任务就是用越南语咨询顾问。这里的亚裔中,以越南裔最多,其次是说广东话
的人,说普通话的主要是附近大学的学生和原来工作在中国内地的新移民。有些人看到我们的``克里---爱德华
兹''大牌子,就会主动过来问问题,还有些人想知道更多的信息,他们往往是说越南话的,正好投中明的罗网。

这样我们俩分工明确,我负责散发传单,是广种;他负责特别辅导,是精耕。不过他的活儿不太多,加上天气
较凉,他怕两个孩子受不了,就推着他们进超市去暖和暖和。这时如果有人过来要徽章等东西,我就过去招呼。


可能我也太积极了,传单很快就几乎散发殆尽了。好在大概11点多的时候,希文带人来视察了,他是负责组织
这件事的,问我进行得怎么样。我说,一切都好,就是传单快发完了。他们马上又给我拿来了一叠新传单,和明带
来的都不一样,不过也是专门针对亚裔宣传的。他们还给我贴上了新的大幅标语牌,上面写着:``亚太裔支持克
里---爱德华兹''。我一看,乖乖,他们的设备果然先进,连贴标语牌的胶带都是专用的,上面也印着``克里---
爱德华兹''。同样的,他们还提供了很多徽章,上面也是``亚太裔支持克里---爱德华兹'',看来他们为亚裔社区
做了很多工作。

他们也给了我一些选民登记表,我说:``这恐怕用不上了,已经过期了。''希文说:``今年已经过期了,不是
还可以为明年登记嘛。''我一听有理,就留下了。其实今天一早有位越南裔被明说动了,却还没有登记,明问我还
来得及吗,我说来不及了,只好遗憾地把他送走,如果当时早听到 希文的这个建议就好了。

他们走后,我在明的文件夹里又发现了很多新的传单,主要是黑白版的``克里---爱德华兹''标准传单。虽然
现在粮草充足了,我也不敢再像刚开始那样大手大脚了,现在我对亚裔只发一张亚裔传单,对非亚裔则只给《费城
问讯报》的文章,那个发完后,就给黑白版的``克里---爱德华兹''。

支持布什的人仍然不少。有几个白人在经过我们桌子时,明劝说他们来拿点传单,一位男士说:``我才不会支
持一个墙头草呢!''便走过去了,我冲着他背后喊:``布什是个撒谎者,这可比墙头草危险多了!''

还有一次,我把传单递给一位单身空手出来的白人男士,照例说:``请投票给约翰$\cdot$克里!''他接也不
接,说:``没门!我爱布什!''走过去几步,又想起来什么,回身说:``你们为什么要为克里干活?他杀过很多越南
人!''当时他已经走到停车场了,我只好对他大喊:``第一,我不是越南人;第二,克里后来反对越战!''

后来我想,更好的回答是:``没有布什杀的伊拉克人多!''可惜,后来也没派上用场,因为再也没人拿这个理由
来谴责我。

当然,大部分人没有这么激烈的反应。有位亚裔女士拒绝了我的传单后,在把购物袋装上车时,还很不解地问
我:``你干这个,有什么好处?''她可能是在质疑这样做的效果吧,我却立刻理解成问我个人有什么好处了。这可正
挠到我的痒处了,我回答说:``我不会从中得到任何好处。没有人付钱让我来干这个,我们都是义工。布什阵营的
人则不同。上次我在城里干这个,有些人也在为布什发传单,你知道他们拿多少钱?每小时11美元!可是这些人只为
钱工作,他们根本不知道布什和克里有什么区别。我和他们聊过天,我问他们为什么支持布什,他们根本答不上来。
他们其实对布什一无所知,而我们都是因为了解克里,所以才来做义工的。''

她便问我:``那你对克里了解多少?''

我这才发现刚才话说过了,因为我对克里这个人也有不少不满之处,只好绕个弯子说:``当然我对他这个人不
太了解,因为从没有和他说过话。但是,我了解他的政策!布什是个右翼极端分子,他只顾自己那帮人的利益和观
点,使整个美国更加分裂\ldots ''

我还要滔滔不绝地再说下去,她却已经把东西装完,上车拜拜了。

更有意思的是一对黑人夫妇,丈夫推车在前,我把传单递给他,被他拒绝了:``我不要,我支持布什。''结果
妻子在后面说:``我要,我支持克里。''我连忙把传单给她,并对她丈夫说:``我们应该把票投给克里,因为布什
是为富人的,民主党则更关心我们少数民族,所以我们应该支持克里。''妻子说:``对,就是这样。''那位丈夫却
直摇头。后来妻子从购物车里捡起个纸团,半嗔怪地打在他脸上。我想他们大概最终会各投各的票吧。

我比较高兴的是:居然招到了一位义工。那是一位来自中国内地的女士,她在接到传单后就说:``我会投票给
他的,另外,我可以拿些徽章之类的东西吗?''我立即给她拿了些,然后按照惯例问她:``那您愿意做些义工吗?''
她说:``可以啊。可是现在还有什么东西需要做呢?登记选民不是已经结束了吗?''

显然,她是比较关心政治的,对这些活动还都比较了解。我说:``登记选民已经结束了,不过我们还有其他很
多事情要做,像户访啊、今天这样的发传单啊,还有在选举日那天的投票动员啊,都需要人。''不过我毕竟从来没
有参与组织过这些活动,所以她再往下问时,我就回答不出了,只好给她留下电话号码和Email,叫她跟我再联系。

她看到我的姓名拼法明显是汉语拼音,就开始跟我说中文。她说她明天会去中文学校,希望能够拿些传单去分
发。这下我可为难了,因为我自己都不够,只好留下了她的联系方法,等我弄到更多的传单后再给她。好在每周去
中文学校的都是完全相同的一拨人,错过一星期也没关系。

最让我难忘的谈话则是来自一位亚裔老太太,我递给她传单时,她接过了,可我说什么她显然没听懂。她用普
通话说:``你说中文吗?''我当然很高兴,立即用中文对她说:``这是关于选总统的。我希望您能投票支持克里。''


她点头说:``我会投票给他的。现在的这个总统啊\ldots ''她摇了摇头,压低了声音,``打仗。我不喜欢。''我
说:``我也是反战的。''她说:``打仗不好。我从越南来,我知道。我的丈夫兄弟儿女,全死了,都是打仗打死的!''

她显然还了解克里和布什的服役记录:``现在的这个总统没有打过仗,他不知道。我这辈子看见过,我知道的。
越南打了几十年的仗啊,我的家人都死光了。''

我问她:``您是南越还是北越呢?''

她说:``南越,不过后来也被北越占领了。我先生是平民,炸弹乱飞,炸死了。儿子和女儿,当兵死掉了。''

我听说过越南全民皆兵,不过没想到妇女也要上战场,又问她:``女儿也要当兵?''她说:``是啊,女儿也去,
做后勤。都死了,打仗打死了。''她告诉我她是25年前来美国的,那就是1979年,越南已经先后和日本、法国、美
国、柬埔寨、中国交兵40年了,我当然没有追问详情。

她后来又没头没尾地说:``打不赢的\ldots ''我问她:``哪里?您是说越南,还是伊拉克?''她说:``都打不赢
的。我见过,我知道。美国打不赢的。现在的这个总统啊,太'有深'\ldots ''

我没听明白:``有深?''她解释说:``就是不是自己的东西,还到处抢哪。''我想,可能她是说``野心''吧。

这位老太太很谨慎,在说现任总统的坏话时,声音立刻会低下来,还会往周围看。我说没关系,她摇头
说:``要当心,''右手做了捏的动作,``会抓的。''

到了12点多时,林来了。她替换我发传单,我去吃饭,然后再回来替换她,她和明带着孩子去吃饭。大约1点半
时,希文又带着四五个人来,都是从华盛顿、马里兰那边支援过来的州际主义战士。这下我们人多势众,弹药也得
到了补充,希文送了一半人到旁边的另一家亚洲超市,其他人和林留在这边。我和明就先回去了。

\section{媒体攻略}

媒体观察组(Media Response),是我在第一次参加克里支持者聚会时就听说了的,后来在波士顿,又在亚裔的
座谈会上听克拉克$\cdot$李(Clark Lee)讲过。由于我自己比较喜欢写东西,因此一回到费城,我就想着手建立
一支东亚裔媒体应对组,专门在东亚语言的报纸上为克里助选。费城的东亚裔,主要是华人和越南人,也有一些韩
国人,其他族裔就不太多了。可是,我找来找去,包括去APAP的家庭派对,愣是一个感兴趣的越南裔或韩裔都没有
找到。

在9月初,倒有一个``美国亚太裔投票组织(APIA VOTE)''来找我。她们是一个总部在加州的中立亚裔组织,并
不特别支持哪个党派,只想致力于促进整个亚裔去投票参与政治。她们要制作一些广告,在当地的电台上播放,因
此四处招募能说亚洲语言的义工,APAP的莫尼卡来询问我们,我觉得这很有趣,就报名来帮忙。

9月10日早上,在我们事先约好的时间,她们从加州打电话过来。广告词她们当时才刚刚翻译好,用Email寄给
我:

我们是父亲,是儿子,是兄弟,无论是结了婚的,还是单身的,也无论是年青的,还是年老的,更重要的是,
我们都是来自亚太地区的美国公民,我们重视美国文化,重视美国的家庭生活和经济发展,只要我们来投票,今年
的选举一定会全然不同的。请在今天去我们的网站登记,网址是apiavote.org,并在11月2日投票。您的选票就是您
的声音, 请发出您的声音吧!以上信息是由美国亚太裔投票组织 (APIA VOTE) 向您播报的。

我在电话里给她们念,她们在那边录。我在念的时候必须每句之间停顿一下,或者用英文计数,因为到时候进
行录制合成的是美国人,必须让他们知道我在念的中文是哪段里的哪句。

大概她们对我录的效果还挺满意的吧,过了几天,她们又发Email来,直接找到我,请我帮她们再录三个广告。
我当然也是满口答应。这次的词是这样的:

一、上届总统选举时,可以参加投票的亚太裔美国妇女人数是250万,但是只有110万人投了票。这就是说,在
2000年的选举中,有140万的亚太裔美国妇女没有去登记投票。.如果她们投了票,她们或许会决定选举的结果。要
知道,2000年的选举差额只有537票。在11月2日,我们都有机会来选出心目中的候选人,他们保证我们的医疗保险
计划,为我们提供高质量的教育,给我们自由的生活, 给我们家庭团聚的机会,他们照顾移民,保护家庭暴力的受
害者,反对非法交易,确保公平收入,让我们有一份可以丰衣足食的工作,保护我们的公民权。您和每一位您所认
识的亚太裔美国妇女一定要去登记,并在11月2日投票。投票选举将决定我们的未来,请在今天就去我们的网站登
记, 网址是www.apiavote.org,并在11月2日投票。以上信息是由全国亚太裔美国妇女论坛向您播报的。

二、上届总统选举时,可以参加投票的亚太裔美国妇女人数是250万, 但是只有110万人投了票。这就是说,在
2000年的选举中,有140万的亚太裔美国妇女没有投票。如果他们投了票,他们或许能决定选举的结果。今年, 可
以参加选举的亚太裔美国妇女有300多万人,但是,并不是每一个人都去登记投票的。这届的选举涉及许多关系重大
的问题,比如说:平等的就业机会和商业发展机会、外交政策、公众安全以及公民权利等等。如果我们不投票,那
么作为亚太裔的美国妇女,没有人能听到我们的心声,也没有人会来重视我们所关心的重大问题。让我们共同努
力,使得您和您所认识的每一个亚太裔美国妇女都能去进行登记,并在11月2日投票选举。以上信息是由亚太裔美国
妇女领导机构向您播报的。请在9月24~25日参加我们在加州三藩市举行的第四届全国APAWLI精英论坛。更多的信息
请去www.apawli.org网站查询。

三、上届的选举我没有投票,因为我觉得自己的选票无足轻重。但是今年,有那么多意义重大的问题值得关
注:收入平等、教育问题、环境保护、健康状况、公民权利、社会安全、战争以及经济发展。我也要发出我的声
音,我一定要投票。我鼓励每一位妇女都应该去进行登记,去了解那些对您来说意义重大的问题,并在11月2日投票。
在上届的总统选举中,5900万妇女投了票。但是,有3800万妇女,虽然有投票权却没有投票。也就是说,有2100万
的单身妇女和1700万的已婚妇女,如果她们也投票选举,那么她们可以决定2000年的选举结果。今年的选举差额也
可能会非常小,一票之差也许会决定胜负。请一定让所有的妇女都参与投票选举。我们的选票就是我们所发出的声
音。我们的选票举足轻重。请在今天就去www.apiavote.org网站或是www.shevotes.org网站登记,并在11月2日投票。
以上信息是由www.shevotes.org网站向您播报的。

这三个广告都是针对亚裔女性的,我觉得她们也许该找个说中文的女士来录更好,不过我当然不介意让自己的
声音回响在西海岸。

加州的亚裔人口众多,所以有专门针对亚裔的电台,这在费城是想也不敢想的。我如果想做点媒体应对,就只
能打报纸的主意。费城的亚裔社区里,除了像《世界日报》这种全美发行的大报外,还有两家只在费城发行的小
报:《远东时报》和《中华周报》,都是周刊。《中华周报》就是一般的中文报纸,《远东时报》则很有趣,上面
同时有中、越、柬三种文字,而且报道内容各不相同,分别针对三国族裔。


我第一次拿到《远东时报》时很失望,因为上面的中文版全是些风花雪月、奇闻怪谈,几乎没有对新闻的关
注,而越、柬版我虽然看不懂,但却看到所配的照片是布什,那么肯定是和美国政治有关的了。

那时我已经认识林了,便写了一份英文稿子,给林寄去,请她叫明翻成越南文,准备向《远东时报》投稿,至
于《中华周报》,则打算另外写一篇。可是和《远东时报》的联系很不顺,他们报纸上的电话号码打过去永远没人
接,《中华周报》的编辑则说可以考虑发表我的文章,要我先寄过去给他们看看。我便把那篇文章翻成中文寄了过
去:

华裔选民们,请积极投票!

还有两个星期,美国就要举行四年一度的总统大选了。这次选举的重要性已无需我再多说,它将决定已经站到
了十字路口前的美国将走向何方。对于我们宾州华人来说,这也是一次难得的机会,让外界可以听到我们的声音,
因为宾州是这次选举中最重要的几个摇摆州之一,宾州的归属,将在很大程度上决定谁将入主白宫。共和党和民主
党都在宾州投入了大量的人力财力,我们有幸居住在宾州,一票相当于其他很多州的好几票,就更应当珍惜自己的
权利,发出我们的声音。

我们犹然记得,四年前,布什在竞选中说他是一个``富有同情心的保守派'',在就职演说中,他又保证会是一
个``联合者'',再后来,他称自己为``战时总统''。这些名词诚然非常好听,可是,我们不能光看一个人说了什
么,更应该看他做了什么。

这位``富有同情心''的总统提出了``不让任何孩子落后''的教育计划,可是却没有拨足款项支持,因为政府的
钱都被拿去做退税了,其中的大部分进了少数百万富翁的腰包。布什确实``富有同情心'',可是他的同情心是给了
那些最富有的阶层,而非孩子或中产阶级。

这位``保守派''上任时,继承了克林顿政府留下的巨额盈利,可他迅速地把它变成了历史上最高记录的财政赤
字。是什么样的``保守派'',会如此挥霍美国人民的未来、会使联邦政府膨胀到史无前例的地步?

这位``战时总统''执意发动了伊拉克战争。此前,全世界都在同情美国人民所遭受的恐怖袭击,自愿地加入反
恐联盟,现在,他们却不再把美国看作是自由世界的领袖,而是对世界和平的威胁。任何肯睁开眼睛正视伊拉克形
势的人都知道,布什的这场战争,不是消灭恐怖分子的战争,而是制造更多恐怖分子的战争。

这位``联合者''经历了美国历史上最有争议的一场选举,输掉了全民普选,靠最高法院的法官才当上了总统。
他不顾美国已经开始分裂的事实,不采取中间立场去和大多数人沟通,反而更加变本加厉地推行他的极右翼原教旨
主义政策。他在同性恋婚姻上的立场,甚至使得他的副总统都无法跟他联合!

而这只不过是他的第一个任期。想像一下,如果布什连任的话,下四年我们会怎么度过:贫富差距进一步扩大、
财政赤字继续升高、世界各国的恶感、严重分裂的国家。我们的选票无比珍贵,我们必须做出正确的选择!

这篇文章寄过去后,一直没有回音。《中华周报》每周四出报,离大选前只有一期的机会了。我又给他们打电
话,结果一个编辑说没版面了,再打过去,另一个编辑却说会录用。我也没有办法,只好期望他们录用吧。

在本周末新出的《远东时报》上,却出现了一整版的关于``不让一个孩子掉队''的广告,虽然它表面上不偏不
倚地只是向家长介绍这个计划,却以正面赞扬的语调将它与布什联系了起来,其实是个软性助选广告。我对布什的
这个``不让一个孩子掉队''计划一向是很不感冒的,当下立刻给林和希文等人发信,提醒他们,共和党正在利用这
个话题巧妙地做广告,我们必须回击。我的建议是,也在同一家报纸上发广告,谈谈布什在教育上的失败之处。

我发Email给《远东时报》,询问他们的广告价格。他们很快回信说,半版是100美元,全版是200美元。我想我
们大概做个半版广告就够了。这时,林和希文那边也陆续传来了回音,他们把信转给了克里阵营,想寻求经费支
持,结果克里阵营回答说,我们已经在很多中文报纸上做过广告了,应该足以覆盖费城的亚裔社区。于是这次媒体
应对行动也就不了了之了。

\section{投票动员义工会}

离选举只有十天了,一方面大家仍然卯足了劲地在户访、打电话,另一方面,如何组织好投票也提上了日程。
克里的费城郊区总部发出Email,邀请大家来参加一个投票动员义工会,我也报了名。

这是在星期六的早上,我在九点半准时赶到克里阵营的分部,停车场上就看见很多车尾贴着``克里---爱德华
兹''的贴纸,有人还贴着支持其他民主党候选人的。从停车场到大楼,到处都是三三两两的人群,陆续地在往楼里
走,在进口处竟然排上了队,让我觉得这不是政治活动,而像是看电影。

门外有位义工在散发表格,我也拿到一份,上面首先是姓名、地址、联系方法,然后问我们是否可以在选举日
及前面那三天里做义工。这几天我是打算去亚裔社区,只有投票日前的那一天可以在郊区干活,因此只选了一天。

表格上列了四种投票动员的活儿:打电话提醒选民去投票、户访、开车带人去投票、在投票处展示本党招牌和
散发本党传单。在投票日之前的义工也有四种,除了电话攻势、户访、展示秀,还有一个头衔叫door hangers,我
不明白是什么意思,问了分发表格的义工,她也不知道,旁边有个人回答说,那是负责把传单挂到人家门上的人。
我也不管这些具体是什么,大笔一挥,全选了,她们爱把我放哪里我就去哪里吧。

交回表格,我已经随人流进入了楼内,那里有人在收表格,并分发一份投票动员的资料,上面写着:``11月2日
胜利十诀'',依次为(原件次序如下):

10、给报纸写信,告诉大家你支持克里和民主党候选人的原因;

9、打电台的脱口秀电话,让别人听到你的声音;

8、提醒10个人去投票;

7、为义工提供食物,或者为外州义工提供住处,以支持当地的``宾州胜利04''(这是民主党宾州竞选活动的名
字);

6、如果你将在投票日在其他选区做义工,现在就进行提前缺席投票,因为到时候你会忙到来不及投票的;

5、参加电话攻势,从今天开始劝说选民;

4、本周为克里和民主党候选人进行逐户敲门;

3、招募到5个朋友和你一起来填满我们所有的义工岗位;

2、下个周末,我们将努力提醒人们去投票,请来帮助我们打电话、户访;

1、在投票日做义工,以帮助克里和民主党候选人获胜!

后面两页则基本上是关于投票日的注意事项、常见问题。

旁边的会议室里摆了几张圆桌,椅子早已坐满,人们熙熙攘攘地站着。桌上摆着早点,墙边的一张桌子上提供
咖啡,大部分人手里都拿着一杯咖啡,今天早上的天气也确实有点冷。

杰娜丽站到一个箱子上,主持了这次会议。她显然被今天来的人数所鼓舞了:``一般来说,像这样的星期六早
上的聚会,来的只有你妈妈、你自己,其他都是想来吃免费早点的人---可是看看今天来了多少人!''

人群中发出怪叫,大家都一起大笑鼓掌。而且今天不仅人多,而且还有电视台的记者在现场录像,杰娜丽就站
在摄像机和采音器下面说话,看来媒体对这次会议也很关注。说不定今晚的新闻画面里还会有我呢。

杰娜丽先介绍了费城郊区四郡的协调者。这位女士来自华盛顿州的西雅图,她打电话给克里的竞选阵营,说想
到全国最重要的地方去,对方回答说:``我给你订了一张明天去费城的机票。''她说:``大家都知道,费城郊区是
全宾州最重要的地方,宾州又是全国最重要的州之一,赢了费城郊区就能赢得整个宾州,赢了宾州就能赢得选举。
我们现在就正站在这场选举中最重要的土地上!''

她接着又说:``我为很多场选举工作过,相信我,我们将会赢!''大家都兴奋地鼓掌。她说:``就在昨天的电话
会议上,我还在怪杰娜丽,只有11个人报了名,我还以为今天会是大家坐在圆桌旁,做个人的交流呢!---结果我们
却来了这么多人!''

有人问她:``最新的《纽约时报》报道说,共和党开出100美元的工资,派人在投票日那天来宾州活动,我们如
何应对?''她说:``这个我们早就知道了。他们将会来进行投票骚扰、压制选民,所以我们那天才需要更多的义
工。''随后她具体讲了几点,其实,共和党的``投票骚扰''不过是干和民主党义工一样的活儿,但由于立场不同,
她毫不客气地便把``骚扰''这个词赠了过去。

共和党同时还组织了很多律师,有鉴于四年前的教训,民主党也针锋相对,已经动员了数百名律师,将在投票
日巡视于各个投票点之间,一旦发生法律问题上的纠纷,随叫随到。看来民主党这四年下来,也是憋得怨气冲天,
这回要誓死血战了。

杰娜丽随后介绍了一位克里阵营的参谋,也是克里的私人好友。他只是简单地提醒大家,不要忘了最重要的
事:你自己的投票!如果投票日太忙的话,你可以提前投缺席票。

在这些讲话过程中,仍然不断有新人来到,义工过一会儿就得叫一嗓子,让大家往里再走一点,给外面的人腾
出点空间来。

杰娜丽解释了这次会议的目的,是收集所有愿意在投票日帮忙的义工的资料,预先统筹策划,尤其是按地域把
义工派到最近最需要的地方去,以免到了投票日当天,上千义工涌了进来,把总部弄得手忙脚乱。至于投票日的
活,在周三晚上会有一次培训,感兴趣的人可以去,不过我想那大概也就是常识性的细节问题,就不打算去了。


在讲话过程中,她一再强调她们非常需要义工的帮忙,不仅是在投票日,在这最后一个星期里,也需要大量人
手来做电话攻势、户访、展示秀。到了投票日那天,当然更是多多益善了,哪怕手持招牌站在投票点外,也能对选
民起到暗示作用:``有这么多人支持克里,那么恐怕他确实值得我投一票吧。''

她还提醒人们,星期一在费城会有克里的一次拉励,克林顿也将到场助威,这将是克林顿手术后第一次为克里
公开助阵,因此引起了媒体的广泛关注。她还提到,届时将有一位特别的音乐嘉宾,不过现在还不能透露。大家纷
纷猜测,第一反应都是布鲁斯。``布鲁斯$\cdot$斯普里斯汀?''杰娜丽笑着说,``不,不是。''她把这个关子卖
到底了。

由于外州的义工会在10点半大举赶到,杰娜丽看看差不多了,让大家自由提了一些问题之后,就结束了这次会
议,下面就是户访。我已经报名参加明天的户访,今天就不去了。

我在开会时曾经看到林进来,现在再去找她时,已经找不到了,却意外地看到了一个朋友张桦。我和她聊了一
会儿,她打算今天去做点义工,不过拿不定主意是该去亚裔社区,还是参加郊区的户访。我连忙和林联系,她却已
经开车直奔香港超市去了,只好把张桦的电话号码给她,让她们以后自己联系,张桦便去参加杰娜丽她们组织的户
访了。

晚上,她给我打了个电话,大略地谈了谈户访的情况,顺便抱怨一下民主党的组织混乱。我说:``是啊,我早
就发现民主党这帮人的组织很不得力了。''这可能跟民主党总部的人都很年轻有关系,今天杰娜丽还说,她们总部
里竟然没有一个人满25岁!

我留意了一下,今天到会的只有两个黑人兄弟,亚裔面孔倒有四个,除了我、林、张桦之外,还有一位也是中
国人,我和他互换了电话号码,打算以后再联系。

\section{郊区户访记}

我参加过好几次克里支持者的聚会,又在支持克里的网站上报过名,联系资料自然早就在当地的竞选总部掌握
之中。竞选活动中的电话攻势,有一个主要事项就是和各个潜在的义工联系,让他们出来帮忙。这样的电话,我当
然已经收到很多次了,每次都告诉他们,我主要在亚裔社区帮忙,户访、发传单、选民登记,早已忙得不亦乐乎了。
不过他们仍然有人锲而不舍地过一阵子就打个电话过来,在这个星期三我又收到一个,问我周末可不可以去当司
机,就是外地的义工来户访时,由我把他们送到具体地点去。我终于不好意思再推辞,就答应了,还决定顺便就连
户访也一起做了。

正好纽约的杨蕙也想到宾州来做户访,她是坐长途汽车来的,我开车到费城的中国城接了她,再掉头来到郊区
的克里竞选办公室。已经晚了15分钟,我们在停车场上看见有大队的年轻人正在准备要出发。进了大楼,又看见会
议室里,义工排起了小队,气氛很是热烈。

有人过来逐个讲解,我和杨蕙是一组,拿到了地图、户访提示稿、劝说要点,以及厚厚一沓的分发材料。在门
外还有一堆``克里---爱德华兹''的大标语牌子,旁边写着:``每组拿十个''。我们拿了十个,放进车的后箱,便
出发了。

这时已经快12点了,预计户访会进行到下午4点钟,因此我在半路上看到个加油站时,便拐进去买点水和食物。
我当时胸前贴了个``克里---爱德华兹''的贴纸,刚把东西放到柜台上,还没掏出钱包来呢,就听见售货员猛问一
声:``你要投票给克里?为什么?''

我一下子没反应过来,本能地说:``因为布什太糟糕了。''``他糟糕,可是克里更差劲!他撒谎、投机,还娶了
个亿万富婆,他根本不关心普通人的想法!''

我顺口说:``娶个亿万富婆也有错吗?我也想娶个亿万富婆呢!''

他说:``他那么有钱,才不会真的关心我们小人物呢,不要听他说得好听!''

我说:``算了吧,有钱也有罪?''

他见人身攻击无效,只好换了个方向:``克里在反恐战争上让人无法信任,他太软弱了。他投票批准伊拉克战
争,现在又反对。我们需要布什来打这场战争。''

这个话题我熟悉,因此可以反击了:``强硬并不代表就正确,你必须打一场聪明的战争,不能光靠强硬,头脑
发热,徒然多增牺牲。克里在反恐战争上是可靠的,如果他在``9.11''时是总统,他也一定会去打阿富汗。''

这时旁边的一个人也来帮腔了:``克里是个墙头草,不可靠,他现在给了很多承诺,可你看吧,等他当选总统
后,他马上又会改变主意,承诺全都不见了!''

我知道再这么争论下去,三天三夜也不会有结果,就结了账往外走:``对不起,我下面还有活儿要干,没法跟
你们讨论了。不过谢谢你们!''

我走出门去,那个顾客却又跟了出来,和我继续讨论。我先承认了他对克里的某些指责是有道理的,但坚持布
什更差。他问我:``那你最反对布什的什么?''

我说:``布什分裂了这个国家!他是个极端分子!他经历了美国历史上最有争议的一次选举,在就职演说中保证
会做个联合者,而非分裂者。可其实呢?他的政策一个比一个右,完全不顾自由派的那一半。我尊重他个人的信仰,
可反对他利用政府资源来把自己的极端立场强加到别人头上,像修宪禁止同性恋婚姻,还有那个《爱国者法案》,
是对人权的侵犯!''

他居然点了点头,说:``是的,其实我是个二十多年的人权活动者,在这点上我同意你。不过这次我要投票给
布什。''

我们在门口舌战正酣,售货员在里面看得心痒,又冲了出来,说:``克里无法保护我们的安全。事实上,后来
基地组织的人说,如果早知道``9.11''会导致阿富汗战争,把塔利班赶下台,把他们撵得满山钻洞,他们就不会策
划``9.11''了。正是因为克林顿政府对恐怖袭击的反应太软弱,才鼓励了他们!''

我只好向车里的杨蕙招手,把她叫了出来,对他们说:``你们看,她是来自纽约的,可是她今天也到这里来做
义工。为什么遭受了恐怖袭击的纽约人并不支持布什呢?因为她们知道,布什的战争计划是错的,并不能有效地阻止
恐怖分子!''

这下双方人数相当,又一场混战,不过我们看看时间耗不起,反正也不可能说服他们的,便及早抽身退出了。

随即我们开了大约10分钟的车,来到地图所标的地方。我们所需负责的共有五个街区,不过那个地图标得太简
略,其中有条黑线,我怎么也找不到实际所对应的路,走到那里一看,只是一片草地,问了当地人,他们也不知道
有这条路。最后我才醒悟过来:原来这条线所代表的是选区间的界线,根本就不是路,克里总部的义工忘了跟我们
说明了。

他们准备的户访说辞也是过时的---是用来进行说服对方的,而我们今天来的地方据说是个``很民主党化''的
地方,只要提醒大家去投票就行了。这个任务不难,没有说辞模板我们也能行。由于杨蕙以前做过户访,而我并没
有在郊区户访的经验,因此我建议,我们一起行动,我先说,她再补充。

我们敲开的第一户人家,来开门的是位老年女士。我说:``您好!我们是来自克里阵营的义工,想问您是否会在
11月2日去投克里一票?''


她说:``当然会去!''

``太好了!''我一面递过去一套克里的宣传资料,一面又问她,``那您在投票日需要帮忙吗?比如我们来开车送
您过去\ldots ''

她说:``不用了,我自己可以过去。''杨蕙又问:``那您需要插在庭院里的大牌子吗?''

她说:``好啊。''我连忙跑回车里,拿出四个大牌子来,送了一个给她,其余三个让杨蕙先拿着,省得以后常
常要跑回车里。我们谢过了老太太,老太太也谢过了我们。

下一家来应门的是一位黑人老头,他也爽快地说,会去投克里一票。我们照例给了他宣传资料,至于庭院标语
牌,他说不用了,``反正这里是民主党的,放那个也没啥大用。''

他说得没错,就在下一家,草地上早已插上了大牌子,虽然不是``克里---爱德华兹'',而是另一位民主党候
选人,不过我们也知道这家人是不用我们操心的了,肯定会在选举日去投克里一票,就直接跳过了这家。

第四家人则接受了我们送的牌子,杨蕙顺便就帮他把牌子插在草地上了。我们走了之后没多远,忽然看见主人
又出来了,从地上捡起一块石头,把那个牌子敲了敲,使它插得更深了。我们忙说:``谢谢!''

下一家就更有意思了。来开门的是位老太太,门一开,我就听见他们家电视里传来的比赛声,那是费城的橄榄
球队``老鹰队''。我照例问过问题后,老太太说:``我会的!不但我要投克里,我丈夫也要投克里!''

她丈夫这时正坐在客厅里看电视,问来的是谁。老太太向房内说了几句话,然后转过头来说:``我丈夫说你们
正在做一件大好事,并且感谢你们的劳动!''她然后又说:``其实他是共和党人,我是民主党人。不过我们这次都投
票给克里!''

我们其余的经验也类似,人们的反应都非常友好,几乎全都表示会去投克里一票。如果家里没人,我们就在门
上留一套宣传资料,希望可以借此提醒他们去投票。

可是,等我们转到下一条街时,情形就完全变了。第一户人家门口就停着一辆贴着``布什---切尼''的车,我
想了想,还是上去敲门了,不过没人应。随后连续好几家,也要么是没人,要么很坦率地说:``我会投布什!''

有一户人家的门口堆满了沙发之类的家具,以至于我们无法去敲门。我本想在他门口留下资料就算了,后来却
透过窗户看见这家里有人,好像正在粉刷室内墙壁,于是拼命挥手,想引起他的注意。过了好久,里面终于出来个
人开了门,干净利落地拒绝了我们,说会投给布什。离开这家后,杨蕙说,其实那个人早就看见我们了,我们``克
里---爱德华兹''的大牌子实在太明显,所以他故意不开门,后来看我总不走,没办法才出来打发我们。

这条街上还有户人家屋顶插着面我不认识的旗帜,杨蕙告诉我,那是美国内战期间的南方旗,建议我试都不用
去试了,免得人家端出一杆枪来。我们俩在街上面面相觑:不是说这是个``很民主党化''的区吗?

我们有点怀疑也许是自己走错地方了,于是走到街的尽头,那儿有位男士正在家门口修什么机器,我们上前打
听街名,结果证实我们没有走错。我们便就势问他:``您会在11月2日投克里一票吗?''

他说:``不,我不投票。''

我们都惊讶地问:``为什么?''

``因为我这一票反正也不会发挥作用。''

杨蕙连忙说:``那可不对。每一票都有作用,我们每个人都去投票,就能发挥作用。''

他一副无所谓的样子:``那又怎么样呢?政治根本不掌握在我们手中,哪个党上台都一样,对我没有区别,我过
好我的生活就行了。''

我们又劝说了他一会儿,试图说服他支持克里,结果他仍然坚持说不会去投票,我们只好谢过他,回过头来继
续走这条街。

真没想到,美国不光南北在政治上鸿沟那么明确,连这小小的一个小镇,换一条街,政治风向就马上转弯。我
们在这条街上仍然遇到千篇一律的``我支持布什'',其中还有位救火员,我告诉他:``您要知道,全国救火员协会
可是支持克里的。''

他说:``我知道,不过我们这里的救火员全都支持布什,因为他给我们做了好事。他给我们增加了拨款,使我
们买得起新车,不用再去发动别人捐款了。''

我又和他争论了一会儿,自然没有结果,只好撤退了。

就在这条街快走完时,我们终于遇到了一家克里支持者,那是一家西班牙裔,很高兴地接过了我们的资料,说
肯定会去投票支持克里。然后风向就彻底地转了,连续几家都是支持克里的,我们乘机大派送那``克里---爱德华
兹''的大牌子,让他们插在地上,或挂在门外,也算是我们对那些布什支持者还以颜色吧。

不过,这条街只是个特殊情况,随后我们在整个下午,所遇到的克里支持者都远远超过了布什支持者。其实,
就算是布什支持者,他们对我们的态度也都很客气,虽然拒绝了我们,都还会说句``谢谢''。杨蕙说,上次她们在
其他地方户访,有些共和党人就直接当面摔门的。再往西,听说干脆被共和党人追得满街跑都有的。

于是,我们同意这个地区还是``很民主党化''的,虽然有部分反对分子聚集在一条街上,但他们显然都早已学
会了和民主党人和平相处的方法。


这个地方其实不是典型的郊区,人口比较密集,几乎没有独户房,多是三四户住在一起,算有点破败的小镇。
在几乎是最脏的房子那里,我们遇到了一位男士,穿得很邋遢,身上也不太干净。我们以为他肯定会支持克里,结
果他却说,克里和布什都是一丘之貉。他打着手势说:``我是属于下层阶级。布什上台,对我大概有这么一点点好
处,换了克里,顶多就是再多这么一点点,又有什么意义?''

杨蕙说:``布什的政策专门造福富人,克里才对穷人有利,从福利政策、健康保险到工作外包,克里都是保护
穷人的。''

这位男士却觉得谁能带来的好处都有限,不值得他去投票。我们又和他讨论了一会儿,最后才终于勉强把他变
成``倾向克里''了。我私下里奇怪了一阵子,这还是我在美国遇到的第一个自认是下层阶级却不支持民主党的人。

到下午3点左右,我们便把整个区都走完了,大约敲开了四十家门,很多人家没有门铃,门又厚重,我们的手关
节都敲红了。大部分人都是支持克里的,十个大牌子也早早就送完了。另外大概有一半家里没人,我们就留下了克
里的资料,也算``温暖送到户''了。

\section{克林顿费城助拳记}

在民主党内,人气最高的显然是前总统克林顿。但由于他曾有性丑闻缠身,戈尔在四年前的选举中没有利用
他,而且还找了个清教徒利伯曼做副总统候选人,以示和克林顿划清界限。这被认为是戈尔失利的主要原因之一。

到了今年,克里自然要大力借助于克林顿,不过他流年不利,克林顿突发心脏病,做了搭桥手术,一个多月没
见大动静。另外很微妙的一点是,人们普遍预测,希拉里$\cdot$克林顿想在2008年选总统,如果克里这次当选,
她显然就没有机会了,因此很多人觉得其实克林顿夫妇并不希望克里当选。好在克林顿及时病倒,别人也难以猜测。

现在选举已经进入最后一周,是死是活,在此一举。克林顿已经休养了一个多月,便正式出马,帮助克里进行
最后冲刺,在各个摇摆州为克里拉励,对外拉选票,对内平谰言。

不出大家的意外,第一站选在费城,除了因为宾州是最重要的摇摆州之一外,还因为克林顿在黑人中的人气极
高,如今布什通过黑人教会,已经在黑人里面发展出不少支持者,克里和黑人的沟通则总似乎不太有效,因此需要
克林顿来到这个黑人聚集的城市为克里拉票。

我这次仍然是提前一个小时到,结果发现这回的声势和以前的克里拉励可大不相同,入口处早已排得人山人海。

队伍旁还有人在发贴纸,除了常见的几种外,居然连``亚太裔支持克里''都有,看见亚裔面孔就发一张,看来
今天民主党人准备得挺细的。

这次希文帮我搞到了红票,到10点后,有义工招呼红票的人先出来。在放我们进去之前,他们将安全检查事项
先说了一遍,基本上就是机场的规格了,金属物品都要拿出来,还好没要脱鞋解皮带。

他们还叮嘱说,不许带饮料、食物,以及其他任何可以扔到台上去的东西。我吸取上次拉励时饥渴交迫的教
训,今天又带了一纸盒饮料和一小袋饼干,放在外套的口袋里,也没有拿出来。走过安检门后,一个警察拿着金属
探测器在我身外游走,将饮料和饼干都缴获了。我当时感觉就像老大娘藏在炕底的粮食被日本鬼子发现了一样,不
过他掂量了几下后,又还给我了。

我走到红票区后,发现这里确实和舞台很近,不过舞台正面却不让人进,一个义工说,那是``绿票区'',给当
地政客或者VIP的。没办法,我只好站在侧面了。一开始还好,后来电视台为了拍摄,打开了另一侧的强光灯直照过
来,照得我眼睛直闪,连舞台上的人都看不清了。

我站定之后,便开始四处打量。今天来的人可真不少,号称10万余人。拉励是在肯尼迪(JFK)广场举行的,不过
美国的大多数所谓广场,其实只是一片小方地而已,只容得下媒体席和绿票区,其他人延伸到好几个街区之外,应
该是如杰娜丽前天所说,历史上最大的一次总统选举拉励了。

由于这是克林顿复出后的第一战,也引起了媒体的广泛关注,我在现场看到了NBC的硬球节目(Hardball)的主持
人克里斯$\cdot$马修斯(Chris Matthews,国内出版有他著的《硬球》一书)。

舞台就搭在街道上,对面便是人家居住的公寓楼。不知道是早有安排,还是正好那里住着民主党人,有户人家
的阳台上垂下大幅标语:``为我们而战'',下面还标着克里阵营的网站。

在远处一栋高楼上,也有人挂起了``克里---爱德华兹''的巨幅标语。

不过显然不是所有住户都支持克里的,有人偏偏唱反调,在自家的阳台上挂了个标语:``民主党人支持布什。''

在舞台的背面,有几个义工坐在那里,她们将会站在演讲者的背后,作为支持者的象征。我旁边有位黑人老太
太不干了,她叫住一个组织者:``为什么我不能坐到那里去?''组织者回答说:``她们都是义工,为竞选做了很多
事。''老太太不忿地说:``我也做了很多事!分发传单、登记选民、打电话\ldots ''

组织者无奈地说:``谁站在那里,都早已安排好了。''老太太说:``我觉得累,下面还要站好几个小时,我顶
不住。''那人便走开了,大概是找人商量去了,过了一会儿,他又回来,把她带到台上去了。不久,又一个老太太
也步履蹒跚地上了台,两人坐在那里休息。

人群继续涌入,舞台上的人也越站越多。有个可爱的小女孩,扎着两条大辫子,穿着支持克里的衣服,手里还
拿着个大纸牌:``宾州孩子爱克林顿和克里总统。''

还有一群中学生,坐在舞台下的一侧,正好在我对面。不知道为什么,好像女生比男生多很多,兴奋地在那里
讲话,大概是因为要见到克林顿和克里了吧。

义工开始给大家分发标语牌,除了常见的几种外,还有一个很有意思的:``还有8天:重新开始''。不过我小农
习气不改,觉得印了这么多只能用一天的标语挺浪费的。

时间差不多后,各色人等便开始上舞台暖场了。打头阵的还是那些当地政客们,尤其是正在竞选的民主党人,
比如乔$\cdot$霍福尔,这种场合当然不会缺了他。连我上次在爱德华兹的街区派对上看到的路易斯$\cdot$墨
菲---一个远在郊区的候选人---也赶来了,看来她们都看好克林顿的人气,想来借点光。

政客讲完话后,我所期盼许久的音乐嘉宾终于登场。可惜,不是布鲁斯,不是REM乐队,也不是邦$\cdot$乔
维(Bon Jovi),而是费城当地的一位黑人女歌手帕蒂$\cdot$拉贝里(Patti LaBelle)。更可惜的是,今天她献上
的不是自己的歌,而是一曲美国国歌,只不过用R\&B的风格来演绎了。键盘响起时,大家还没搞清楚她要唱什么,
等到歌词一出口,台上的政客都明白了,赶紧右手抚胸,一本正经地肃立。


台下的听众有些也跟着一起唱,有些却好像纯从音乐的角度在欣赏,当帕蒂的声音在高音部连玩了几个花后,
大家便尖叫不绝了。

帕蒂在费城的人气挺高,她唱完之后,跟台上的政客们开始握手言欢,我旁边有几个人叫住台上的人,请她转
过身来,给她们拍照。她很合作地摆了几个姿势,我也抢到了一个镜头。

此后又是照例扔T恤。捱了一段时间之后,人群终于开始兴奋起来,宾州州长伦德尔、费城市长斯爵特走上台来。
他们的发言主要集中在对克林顿的赞扬上,甚至说:``第四十二任总统克林顿,是有史以来对费城帮助最大的总统。
不过,我们希望第四十四任总统克里可以超过他!''

随后,克里和克林顿并肩走上通台,向大家挥手示意。人群立刻沸腾起来,尖叫共欢呼一声,手臂与标语齐
飞,天空中甚至飘洒起彩色纸屑。两人如同超级模特一般,在通台上走了一圈,向人们微笑挥手,才上了舞台。那
个拿着``宾州孩子爱克林顿和克里总统''的小女孩当然不失机会地请他们在牌子上签名,两人都很高兴地签了。

先讲话的是克林顿,台下的鼓掌欢呼经久不息,我身后有位女士连声尖叫:``我们爱你!''克林顿等到掌声稍微
下去后,笑着说:``如果这不能使我心情更好,我不知道还有什么能了。''大家都大笑鼓掌。

克林顿看上去精神好像仍然不大好,讲话很短,声音不太有力,而且经常需要看讲稿。他的讲话本身也没有太
多新意,很快就进入了对布什的攻击,历数了布什执政以来宾州和费城所丢失的工作机会、财政赤字、禁枪法案等
等之后,还俏皮地说:``这些都与以前形成了对比,那时总统是另一个人---我。''听众又一起欢笑鼓掌。

不过他的结束语很有力:``如果在竞选中,有一个人总在恐吓你,另一个人则让你思考;一个人诉诸恐惧,另
一个人则诉诸希望,你最好投票给那个让你思考和希望的人。''

克林顿讲完话后,克里上前和他拥抱,克林顿和伦德尔等人站到一排,克里开始演讲。我身后的那位女士
说:``克里看上去很棒!''旁边一位男士回答说:``仪表很好!''

克里先引用了克林顿的一个小幽默:``我问克林顿:'你和布什有什么相同的地方?'他想了一会儿后说:'再过
八天,我们就都是前总统了!'''

克里在演讲中大量引用克林顿的政绩,时不时就来一句``我会像克林顿总统一样,做什么什么事,而不像布什
总统那样\ldots ''好在对于费城的中下收入阶层来说,克林顿确实比布什好得多,所以他有话可说。

听众对克里的支持也很高,经常有节奏地高喊:``克里、克里!''或者``布什必须下台!''我注意看了一下台
上,伦德尔和斯爵特也在和大家一起喊,克林顿却从来没有喊过,不知道是不是因为心脏乏力。有好几次,也是大
家都鼓掌,克林顿却仍然只是微笑着站在那里。

只有一次,当克里谈到健康保险计划时,克林顿带头鼓掌,本来大家可能都还没有鼓掌的意思呢。我不免有点
小人之心地猜测,这是否和希拉里$\cdot$克林顿的著名健康保险计划(Hillary Care)有关呢。

克里演讲结束后,又和克林顿再次拥抱,高音喇叭里响起他的竞选主题歌,漫天又洒起了彩色纸屑,一时间气
氛热烈,宛若胜利已到。

两人并肩向台下的听众挥手示意。

随后,克里从左侧下台,和听众握手,克林顿则直接从右侧,也就是我们这一侧,通往出口。大家立刻尖叫欢
呼不绝,``我爱你''``我们爱你''此起彼伏,克林顿连连向台下致意。

不过他一直站在通台上,人们想要和他合影是不可能的,只能和他遥相握手。我也跟他握了一下手,感觉他的
手有些凉,可能真的没有完全恢复健康。

我旁边有个小伙子很聪明地带了一叠杂志来,赶紧递过去请他签名。克林顿接过杂志,退到通台中央,耐心地
给所有的杂志都签了名。

对面的中学生们以双倍于我们的热情在呼唤克林顿。克林顿走过去时,她们顿时爆发出一阵尖叫声,手臂如林
般地伸向他。克林顿和她们一一握手后,意犹未尽,又走到台下,深入人群,给她们签名去了。

那边克里绕了一个圈子,也快过来了。不过他就要走到我们这里时,却又上了通台,只是俯下身来和人们握
手,交谈、合影就不可能了。我和他也握了手。

他走过去后,我再伸手,他显然也不会记得谁已经握过了,只要看见是手就握,于是又和我握了一次。他的手
很有力,我要用劲才能把手抽出来。

靠得这么近看时,我发现他不像电视里那么老,长相甚至可以说不在克林顿之下。我以前和美国人聊天时,有
位女士就说,克里不难看啊,他是那种``传统美男子''型呢。可惜,在这个电视时代,光长得好看没有用,还得上
镜啊。

拉励至此结束了,我顺着人流往外走。自然,照例街头又有布什的支持者在抗议。这回他们的攻击重点,一是
堕胎,二是克里反越战,因此甚至把克里的全名John F Kerry换作了简$\cdot$芳达$\cdot$克里(Jane Fonda
Kerry),因为简$\cdot$芳达在很多美国人看来是叛徒。

这天晚上,忽然有个朋友告诉我,在美国有线新闻网的新闻上看见你了,还是个大特写!可惜我自己没有看到,
不过我确实注意到,当时有数不清的摄像机就设在我对面,那么就为克里贡献一下我的肖像权吧。


我本来是觉得这个主意很好的,后来有高人指出,投票是个人权利,个人应当以哪个候选人最能代表自己的利
益为原则来投票,以民族团体的名义来号召个人抛弃小我原则,形成集体力量,实为国家主义思想之残余。当时我
看了觉得醍醐灌顶,虽然仍然认为80~20有道理,但已经不再像以前那样无条件支持了。

丁教授对80~20的批评则纯粹从现实出发,以黑人为例,说他们每次都投票给民主党,因此他们的利益实际上
被两大党所忽视。亚裔也应当声音多元化,才能吸引别人来给出诱饵。

我觉得他的说法并不太能让人信服,因为80~20至少从理论上说,是存在着每次都投给不同的候选人的可能性
的,而且我很怀疑如果这次80~20支持了布什,丁教授是否还会这么说。

时间已经很晚了,丁教授说完后就匆匆和我握手走了。平心而论,他的演讲风度还是很好的,如果不看观点,
我认为他事实上赢了今晚的辩论。

\section{中国城MoveOn新官上任}

自从大选开始以来,我就一直想在中国城做点什么,但是和克里阵营的联系不太畅通,总没有机会。上次我帮
助MoveOn PAC的诺亚在费城的一座老人公寓户访了那些华裔选民之后,就顺便请他帮我注意一下,如果其他MoveOn
的义工也遇到了不说英文的华裔,可以来找我。结果他把我的情况跟MoveOn的组织者说了,一位叫索尼亚的女士在
本周三跟我联系,邀请我来担任中国城选区的MoveOn领导。

我有些犹豫,因为我已经报名参加郊区的一些活动了,如果成为MoveOn的选区领导,显然就无法再在郊区参加
任何活动,而我对郊区活动还是很感兴趣的;再者,我多次为克里阵营做义工,而MoveOn是个527组织(指根据美国
《国内收入条例》第527款设立的为政治活动筹集资金的免税组织),应当持非党派立场,不知道这里面是否会有法
律问题。

索尼亚咨询了她们的顾问后,告诉我这在法律上没有问题,在中国城的户访也确实很吸引我,于是我就答应了。

索尼亚给我把账号设置好,我登陆上MoveOn的网站,那里有我的选区信息、工作指导等等。我下载了选民列
表,结果居然有2242人!吓得我给所有认识的克里阵营亚裔义工发Email,邀请他们参加在中国城的户访,因为按照
经验,一个人一天能户访到50个人,就算很不错了,那么2000多人,至少也需要20个义工干上两天。不过好像大家
也都忙于自己的义工活儿,没有人响应。

虽然升官成了领导,可我手下一个人也没有。好在索尼亚给了我南费城高中一位林老师的联系方法,她在教一
个中英文双语班,班上有一些中国来的学生,想做些社区工作,林老师就推荐她们来做MoveOn的义工了。共有七个
人,五个女生:林潮妹、卢钦、阮裕珊、范洁萍、欧阳灵芳,两个男生:黄勇和杨子昌。最让我高兴的是,她们大
多会说方言,尤其是范洁萍和杨子昌会说广东话,我想应该可以帮上大忙。

我也找到了杨蕙和张桦,不过她们一个要从纽约赶来,11点才能到费城,一个已经答应了林去香港超市发传
单,下午才能到中国城来。

早上我们先去参加MoveOn办的培训,地点在南费城靠特拉华河的哥伦比亚大道上。那些高中生都住在南费城,
因此她们在华盛顿街上的东方超市集合,然后步行过去。我本来还想开车去接她们,但由于她们人太多,我的车也
装不下,只好约定在MoveOn总部碰头。

第二天的培训倒很容易,基本上就是一些常识。他们也顺便介绍了MoveOn的历史,原来是始于1998年,克林顿
被弹劾期间,一帮人发起网络签名,呼吁中止这种无聊的行动,关键词就是``我们应当move on(往前走)'',有两万
多人签名。说到这里,主持人还问大家:在座的有谁参加那次签名了?结果居然还真的有四五个人举手了。他们由此
发现网络可以对政治发生影响,便注册了MoveOn.org,到后来便成了网络反布什的总司令部。

培训结束后,大家分组讨论。我带着组员来到索尼亚的那个组,向她问了个问题:``我们如果遇到支持克里的
居民,会问他们是否可以给我们做义工,如果他们问义工在选举日要做些什么,我怎么回答?''

索尼亚说:``基本上有两件事:一是在投票处外,让选民签到,然后第二件事就是其他人可以给还没有投票的
人打电话。''

我问道:``如果人们不愿意告诉我们姓名怎么办?''

索尼亚说:``那就算了呗。''

然后我又请她帮我找个人开车送人,因为我的车上最多只能带四个人,还有三个人需要别人送。她给我找到了
一个人,不过需要我等一会儿。我乘机向高中生们解释了今天要做什么。MoveOn给每个人都分发了资料,他们没有
那么多笔记板可以提供,就用硬纸板代替,上面用橡皮筋箍住一叠资料,除了欢迎辞、费城地图(今天大部分义工都
是来自外州,主要是纽约州,也有从加州来的)、义工表(用以记下愿意帮我们做义工的人的信息),当然还有克里-
布什对照表,用以说服未决定的选民。

我又到车里拿来一叠选民表,分发给她们,让她们对今天的工作有个直观的印象。在那上面,每个选民都有拒
绝、不在家、不说英语(对我们来说就是不说英语及中文)、废弃、搬离、错误地址/号码、留言、交谈等选项,在交
谈后面,又有支持克里、倾向克里、未决定、倾向布什、支持布什、倾向纳德尔、支持纳德尔这七个选项。

我告诉她们:``如果有人还未决定,你们可以花上两三分钟跟他们交谈,但不用太纠缠,因为我们需要户访的
人太多,花不起这时间。至于倾向布什或者纳德尔的人,说声谢谢就可以走了,绝对不要和他们争论。如果是愿意
投票给克里的人,我们要问他们什么时候去投票,是否可以留下一个电话号码,这样到时候我们可以提醒他们,以
及他们是否需要开车接送、看小孩之类的帮助,最好还可以问一下他们是否愿意做义工。这些选项都在表上列着,
你们一个个地问就是了。''

由于她们还是高中生,我不想让她们分开行动,因此把她们分为两人一组,原则是:男女生搭配,不同的方言
搭配,有手机的和没有手机的搭配。不过这些孩子们显然有她们自己的主张,不等我开始分组,她们自己就配好对
了,基本上就是要好的女生愿意在一组,剩下两个男生没人要。我一再强调安全第一,女孩子们都说没问题,最
后,我只好听了她们的,杨子昌跟我走,黄勇和欧阳灵芳一组,其余两组就都是女生了。


我们等了几乎半个小时,那个司机还没有来。杨蕙都已经到了费城了,在汽车站等我们。最后我终于等不下去
了,决定分两趟把她们拉过去。结果我一上路,没几分钟,索尼亚又给我打来电话,说另外一个人把剩下的人也送
来了。

选民表上每页有10个人,我本来想按顺序给每组分配20页,可是在车上的时候,那四个孩子就自己翻起表来
了,挑走了那些看上去都是中国人名字的部分。我只好让她们都按照整数来拿,并且把所拿的页数都记下来,这样
还不至于太杂乱无章。

中国城的投票站在救火场,我们就在那里集合,这样如果别人问起在哪里投票,她们可以有个直观的印象。今
天正好有个乐队在表演。

大家集合后,孩子们就要出发了,我再次叮嘱:``你们一定要注意安全第一,看上去有危险的地带,就不要去
了。不一定要把表上的名字都访问到,一两个人不访问,也没有什么大不了的,万一你们出了什么事,那才是无法
挽回的。绝对不可以进到别人的家里去,不要逼别人回答问题,不要争论或者惹人恼火。总之,你们要记住:你们
自己比谁当总统更重要,可千万别为了克里而冒险。如果有任何情况,就给我打电话。''

她们都答应了,便四散奔向自己的地盘。我由于和诺亚约好了,还要再去那个广东人聚集的公寓,便开车带杨
子昌和杨蕙去那边了。诺亚把上回我们遇到的铁杆民主党人简老太太的房间号给了我们,我们在公寓外呼她的房间
后,便顺利地进了公寓。

诺亚把他的名单给了我们,其中亚洲人的名字都已经用彩笔勾出来了。为了节省时间,我们三人基本上是敲开
一家门,然后杨蕙先上去说,我带着子昌到下一家,因为子昌是第一次做户访,不太有经验。正好我们敲开的第一
家是说国语的,我便先按照程序问了他一番问题,这家人也是要投克里的,因此一切都很顺利。随后再遇到只会说
广东话的人,子昌就足够应付了。

这栋大楼的中国人其实组织得不错,在那个楼道口的布告栏上,我又发现了一个通知,来自于费城老人中心,
通知大家他们将会有巴士接送选民去投票处,有9点半和10点两班,在那里停留半小时后回来。中文版就贴在英文通
知的旁边。还有一个英文版的``如何投票''指南,上面有人手写了中文翻译,插在每行之间。后来碰到选民有问题
时,我们就会告诉他们那天将有巴士接送,可以来看这边的通知。

在这栋广东人居多的公寓里,我基本上就是拿着名单,帮忙敲门而已,然后杨蕙和子昌两人上阵,哇啦哇啦地
说什么我也听不懂,只能在另一人做完后赶紧告诉她们下一站。

好在我广东话虽然不懂,汉字还是识的。有一户说广东话的人家,还未决定投票给谁,丈夫拿出一张中文报纸
来,好像是《明报》或者《世界日报》之类的大报,上面有一张布什、克里的立场对照表,他显然仔细研究过,在
各栏都分别打了勾,我瞥了一眼,发现是给布什打的勾更多。杨蕙仍然在试图说服他,我借助报纸,大体能猜出两
人在说什么,就在旁边支招。不过我很快就发现,我是在帮倒忙,反而分了杨蕙的心,于是干脆闭嘴了。最后,杨
蕙终于成功地把他们这两票由倾向布什变为倾向克里。

我也不是一次上场的机会都没有,杨蕙在敲开一家门后,忽然叫我:``国语!国语!''我连忙跑过去,那边有一
位老太太,我照例问她是否会投票,她说会,我再问她:``您会投给克里吗?''她露出茫然的神色,显然不知道我在
说谁,我试遍了英文的发音,以及``克里''、``凯瑞''、``柯瑞''等各种译名,她一概不知。

我只好问她:``那您会投票给布什吗?''这下她听懂了,马上摇头说:``不会。''我说:``您投票给民主党对
吧?''她说:``是。是一号吧?''

我说:``不,是2号,11月2号,星期二。''她说:``不,我是说他们排位的顺序,民主党的人是一号吧?''这我
可就不知道了,她说:``我又不知道他们叫什么,我不懂英文,你得帮我写下来,免得我到时候投错了。''

她拿过来一张纸,我便写下:``共和党 Republican Bush,民主党Democrat Kerry。''还要再标上布什坏克里
好之类的,她说:``这就够了,我应该能认出来了。''

我说:``您别担心。投票处可能会提供语言协助,如果您还有问题的话,就打我的电话好了,我们会派人来帮
助您的。''说着便把我的电话号码也留在了那张纸上。

她说:``没关系,我们楼下到时候会有巴士来接人去投票,这么多人去,没问题的。''看来她是知道那个费城
老人中心的通知的。

这位女士在我们来时正准备出门,问答既毕,她便锁上门,往电梯走去,一边对我说:``你们去二楼吗?那里有
很多人在打麻将。我也是去打麻将的。好几十个人呢,都是中国人。''

我问她:``她们都支持民主党吗?''她连连点头:``都支持。你们也去吧,能找到很多人呢。''我有点心动,但
一想到中国城那边还有很多事情,只好放弃了,对她说:``我们今天只管五楼到十一楼,下面有其他人去户访,我
们就不去了。您帮我对她们说一说吧,让大家11月2日都去投票,投民主党。''她答应了。


到下午2点,我们就把诺亚名单上的亚裔住户都访问完了。我赶紧带着他们回到中国城,因为子昌和黄勇需要去
一家饭店,他们想在那里打工,经理约他们在2点半面试。由于大家都还没有吃中饭,我们便决定顺便在那家饭店把
饭吃了。

这么多女生一起吃饭,自然是唧唧喳喳成一片。她们争先恐后地跟我说早上的户访经验,有好的也有坏的。有
一组人抱怨她们表上的地址都找不到,还有一组抱怨她们表上的都是外国人(这可是她们自己挑的!),要求换新的地
址列表。我当然都答应了。

我比较关心的是,他们有没有遇到任何危险情况?有没有人对他们态度很恶劣?还好,都没有。

这时张桦也来了。她本来是和林她们一起在香港超市发传单的,但对在中国城户访也很感兴趣,就决定在两边
各做半天,正好希文开车在各处巡视,就把她带到城里来了。

吃完饭后,我们便重新分配任务。小姑娘们自然又来挑了新的名单,退回来的旧名单,张桦、杨蕙形成一组去
户访,因为她们的英语都很好;子昌加入了两个都不会说广东话的女生组,我则拿了剩下的名单,试图去招募新队
员。因为我们今天在中国城看到有很多克里阵营的人,摆着桌子在街头宣传,到处发传单。我找到他们,问是否有
人愿意来帮忙,因为中国城里来往的大多是游客和顾客,居民其实不多,公民就更少,况且,``三天后就选举了,
再发传单已经没有意义,我们应该集中精力做投票动员才对。''

当然,我也必须提醒他们,MoveOn是527非党派组织,克里阵营的正式员工不能参与,但如果是义工,那就没有
关系了,可以以个人身份来做MoveOn的义工。

可惜,他们都对此不感兴趣,只有一个人给了我他的名片,说让我晚上给他打电话,也许他可以帮我找到一些
人。

我只好自己出发户访去了。我选择了偏北和偏东的区域。中国城坐落在费城市中心的东北部,``中国城选区''
是我方便的叫法,其实这个选区内除了中国城外,北部是比较荒凉破败的街区,东部则是费城最主要的大街之一的
宽街(Broad Street),住的大部分是非亚裔。

我先沿着中国城以北的一条街走,按名单挨个敲门,可能是因为时间不巧,星期六下午,所有人家都没人应
门,还有一部分地址就根本是错的,根本不存在。

顺着这条街快走到宽街时,有座巨大的楼房,看上去像是公寓楼,我对了一下地址,选民表上有一百多人住在
这里。我心中暗喜:终于找到一个``富矿''了,下面的问题就是怎么混进去。

我绕着这栋大楼走了半圈,却没找到常见的那种进口,也看不到保安。它只有一个小门,是锁着的。我只好翻
出选民表,找到一个留有电话号码的,给他打电话。那边有人接了,却不是名单上的这个人,我道过歉后,又
说:``您是住在这个地址的吗?''

他说是,我连忙解释说:``我是来自MoveOn的义工,想问一下您打算在星期二投票给克里吗?''

他不肯回答,我又问:``那您可以为我开门,让我进来和里面的住户谈话吗?''

他想了一会儿,说需要请示别人,让我等一会儿。我在楼下等了许久,却也不见他回话,只好把电话掐了,再
找一个人重打。还没有拨号呢,一个黑人从楼里推门出来,问我:``你要找谁?''

我把情形向他解释了一遍。他摇摇头说:``你不能进入这里。你知道这儿是什么吗?''

我傻乎乎地说:``是个公寓楼啊。''

他说:``不,这是个矫正中心(correctional center)。''我没反应过来这是什么意思,他就继续解释说:``这
里的人都是犯了法、认了罪,但罪还不至于进监狱的人;或者监狱里的犯人,表现还不错,可以提前释放了,就先
在这里过渡一下,学习一些技能,以重新回到社会。''

我这才明白过来,敢情这里是个教化所,但仍然硬着头皮想试试,就说:``那我能够进去和他们谈谈吗?他们既
然出现在这张表上,就说明他们的公民权并没有被剥夺,仍然可以投票。''他说:``你不能进入这栋楼。凡要进来
和里面的人面谈的,都需要提前预约,犯人愿意,我们批准,才能进来。再说了,这里的人都是临时住户,都早就
离开了,你按照这个表恐怕一个也找不到。''

不过,他最后仍然给我支了个招:``你也可以等到星期一,等负责人来上班了,跟他通个电话,看能不能让他
把所有犯人召集起来,你来给他们讲话,鼓动他们去投票。''说着还给我留下了电话号码,就回楼内了。

我谢过了他,心里还真有点跃跃欲试,毕竟对这么多犯人讲话,这种事可不是轻易能碰上的。不过想到星期一
的计划早就安排满了,都是在郊区,恐怕不会有空开车到城里来,而且成功的可能性实在太小,也就算了。

然后,我拐上了宽街,这是一条大街,两旁的建筑物多是商家,公寓楼极少,单户人家则根本没有,凡是地址
留在这条街上的,不是无效就是办公室,都不可能进入,还有些地址,找上去一看,却是美军征兵处之类的联邦机
构。公寓楼里照例都有保安坐班,不让我进。我试图打电话找里面的住户,也总找不到人,除了有一处告诉我号码
错了,我向他解释后,却只换来他一声``滚开!''


我怀疑这人是布什的支持者。这时有位路人经过,见我站在街头往楼里望,就好意地告诉我:``这是座监狱,
你进不去的。''

我没有办法,天慢慢地黑下来了,我们说好下午6点集合收队的,这时已经5点40分了,我也差不多走到了这个
选区的最边缘,抬头就可以看到市中心了,赶紧往回走。今天下午我走了两个多小时,竟然一个选民都没有接触
到,实在是我户访史上的一大奇观。

大家仍然在救火场集合。我先送杨蕙回到汽车站,然后分两次把孩子们送回家,因为她们住在南费城,比较
远,地方也比较偏。明天她们还会来,我不想让她们弄得太累,因为她们星期一还要上课,就让她们下午1点钟到中
国城,5点钟结束,其中有些交通不便的,我在12点45分去南费城接。

张桦则是自己回家的,晚上她给我打电话,聊起今天的经历。原来她们也走到很多奇怪的地方,比如妇女救世
军,还有妓女聚集的地方。MoveOn的这份选民表真让人哭笑不得,不知道他们怎么弄出来的。好在除了我之外,其
他人都颇有所获,尤其是那些高中生,在亚裔社区内做得很不错。

晚上回到家后,我想:这领导还真不好当啊,除了要组织之外,还得负责接送、负责安全,最后冲锋陷阵的事
也一样不能少干,还全是别人捡剩下来的。

索尼亚打电话来询问我的进展,我说大概找到50个支持者的样子,意思有点惭愧。她却很高兴地称赞不已,说
有些人做了一个月了,到现在不过才找到60个支持者,我能够一天就干出这个成绩,已经很了不起了。

她又询问我是否需要更多的义工来帮忙,我当然来者不拒、多多益善。我也给白天遇到的克里阵营的人打了电
话,他说明天再帮我找义工。

晚上的最后一件大事就是整理名单。现在我已经知道,凡是那种一个地址上有上百人,却不注明公寓号的,都
肯定是无效地址,不是监狱就是其他组织。我把这类地址都从名单里剔了出来,最后一数,竟有近千人---也就是
说,我拿到的名单上有近一半地址是错的。

\section{继续轰炸中国城}

昨晚美国从夏时制改回正常时间,今天下午5点就相当于昨天的下午6点,那时天就会比较黑了,于是今天早上
一起来,我就赶紧手忙脚乱地给那些高中生打电话,通知她们12点集合,下午4点结束,即实际时间不变。

索尼亚没有食言,又给我派来了两个义工,汤姆来自纽约,鲍勃则来自华盛顿。我们在中国城的救火站集合,
我事先已经挑好了一份看上去全是美国人的名单,对他们解释说:``这些孩子们会说亚洲语言,因此我让她们主攻
亚裔社区,非亚裔的选民,就靠你们啦。''他们当然没有任何意见,恐怕还很高兴不用去不说英语的社区,便拿了
名单上路了。

今天有一个高中生没能来,我正好把来的六个人分为三组,照例任由她们挑了名单。大家都出发后,剩下一堆
支离破碎的名单给我,我整理了一下,挑了一条比较大的街,自己也去户访了。

我虽然搬到费城地区3年多了,中国城也来过不知道多少次,但如此深入的盘桓,今天还是第一次。街面的那些
建筑物,以前我只注意到他们底楼的饭店、铺面,这次才上楼深入到住户的世界,看到他们所住的环境,感受到别
人的日常生活。

不过这份名单实在太气人了,又有大约三分之二的地址不对。偶尔地址对了的,人都不在家。我走了大约1小
时,才接触到第一个选民,是位白人姑娘。她爽快地说,会投票给克里,甚至可以考虑做义工。

我在中国城总算开张了。但随后半个小时里,我又陷入了楼层的迷宫。中国城的房子都有些老,上楼后往往深
悠悠的,灯光昏暗,房门紧闭,要么找不到名单上的房号,要么没有人在。

在这条街快要出这个选区时,有一栋大公寓楼,但照例无法进入。我掏出手机想打电话,却碰巧有个老美推着
一大堆行李出来,我帮他拉住了门,他谢过我后,我就乘机溜进去了。

按照名单,有三位支持者住在三楼,我乘电梯来到三楼,到各个房间一看,却找不到门牌号码,只见极厚实的
金属门一字排开,却不知道哪间是301,哪间是305。楼道里静悄悄的,一个人也没有,墙壁上雪白一片,没有任何
标志。我心想:坏了,难道这次溜进监狱里来了?

我忐忑不安地坐电梯上五楼,想再试一下。从五楼的电梯出来时,有位男士正要进电梯,我看他不像犯人的样
子,赶紧问他:``这栋楼的房间号码是怎么排的?''

他说:``最左边是1号,然后向右依次排开。''

我谢过了他,找到名单上的房间。我开始敲自己要找的第一家,这里的房门很厚,要用力敲才听得见,差不多
是捶门了。

主人看上去是个大学生的模样,我说明来意后,他马上就笑了:``我也是MoveOn的成员。''我也乐了,赶紧和
他握手。他当然会投克里的票。

这下我不用再疑神疑鬼,很顺利地就走完了整栋公寓楼,除了有一位先生明确表示要投布什的票外,其余选民
都表示支持克里。

这时将近下午3点,有一组高中生给我打电话,原来她们已经把自己的份内活干完了,正好有两个女孩子在3点
半就要先走,我便把另一组也打电话叫了过来,让她们先走。剩下的两个男孩子组成一组,把没走到的亚裔社区户
访完。

我自己也抓紧这最后的一小时,走到旁边的一条街上,那里有个大公寓楼。我等了一会儿,乘有人搬家进出的
时候,溜了进去。

在这栋楼里我所获颇丰,几乎所有的人都在家,除了一个看上去是中国名字的住户。奇怪的是,坐落在中国城
之中,我所接触到的所有人却都是美国人,而且还清一色地是白人女孩,可能是因为附近有大学的缘故吧。她们都
一边倒地支持克里,很多人都留下了电话号码,让我在投票日可以打电话过去提醒她们,有人还同意做义工。

从这栋楼出来后,时间也差不多了,我沿街一路敲了几家门,都没有人。到了下午4点,我们又会合在救火站那
里。

汤姆和鲍勃干得很不错,把名单上的美国人几乎都走到了。不过他们去的也大多是公寓楼,有的楼保安看着不
让进。我曾经建议过他们打电话让人来开门,他们却觉得那样做不好,最后都是在楼底下,通过对讲的喇叭和住户
联系,倒也不错。而且这样一来,不在家的选民,他们也留下提醒投票的留言,比敲门不遇,只好离开强。

今天结束得比较早,孩子们不用送,自己就可以回家了。我谢过了他们,也赶紧回家准备输入户访结果。这是
MoveOn选区领导必须做的一件事,以整理出在投票日使用的克里支持者名单来。我们的任务,就是确保所有能找出
来的克里支持者都去投票,至于其他人,就不用花力气去说服教育了。

MoveOn的网站大概使用的是我所见过的最糟糕的服务器,我每次一按``提交''键,长长的等待之后,就是一个
``找不到服务器''。后来我意识到输入所有结果,比如无效地址、布什支持者、不在家之类已经不可能来得及了,
就只输入克里支持者,一次只输入十个,还是屡屡失败。

我想,大概是因为这两天大家都出去户访,所以今晚输入结果的人特别多吧,我反正后面三天都请了假,明天
白天输入也许会容易些。

剩下的时间里,我开始做电话攻势(phone banking)。林给了我一份选民名单,共有50人,大部分是亚裔,她自
己不会说亚洲语言,就让我来打。


我依次打过去,如果是外国人就说英语,如果看上去是亚洲名字就问他们是否说中文,结果除了少数韩裔和越
南裔外,大部分人果然是说中文的。我在电话里提醒他们,后天就是投票日,可别忘了去投克里一票啊。当然,有
些人的资料里标着是共和党,我就不打了,巴不得他们忘了投票呢。

有很多人对投票显然是稀里糊涂的,不知道需要带什么,也不知道该去哪里投。这个林早有准备,她给我一个
网址:http://www.mypollingplace.com/,我坐在计算机前打电话,凡是不知道自己投票地点的,上网一查就查到
了,再在电话里告诉他们,也算是我做了一点实事。

2004年11月

\section{最后冲刺}

今天早上一起来,马上又扑到计算机前,继续与MoveOn网站的持久战。结果仍然和昨天一样:一按``更新''键
就``此网页不存在''。我打电话给MoveOn的技术支持,也没有人接,只能留言。

我只好去找索尼亚,问她如果实在输不进,该怎么办。她说:``那你只好手工输入到一个Excel(一种表格型财
务软件)文件格式中,自己编辑打印。''我当然不愿意,她又出了个主意:让我每次不要全选所有名字,而是一条街
一条街地输入,这样我不会一下子弄出个两千多人的表格来,而只有一百多人,应该就不会那么容易死机了。

这一招终于奏效了,我吭哧吭哧地把前两天的结果都输了进去,当然只包括克里支持者的列表,其他错误地址
之类我就实在没有精力输入了。

下午我的计划是参加克里阵营的``挂门''活动,并帮助久违了的自由党人恰克去插他的竞选牌子。我先赶到克
里的郊区总部,只见一路上插满了各阵营的牌子,除了布什和克里阵营外,竞选联邦议员和州议员的牌子也比比皆
是。我在路上还曾看到一个很逗的牌子:``恐怖分子支持克里'',当然肯定是布什阵营的人插的。

只见那里热闹非凡,人来人往,一队队的义工正在停车场上准备出发。总部前搭起了个帐篷,里面摆着一张桌
子,上面放着好几堆资料。门口站着一位义工,见人过来就打招呼,我告诉他,我是来参加``挂门''活动的,他便
把我引到一个人群中。

一位女士已经在那里讲话了,基本上是在说``挂门''活动的注意事项。她首先通报了最新的民调情况,克里在
几个关键的摇摆州都在领先,甚至连共和党自己搞的民调,克里也领先了。她还拿现场这热气腾腾的场面来鼓励大
家,我们都鼓掌叫好,以示信心。

``挂门''就是把传单挂到别人的门上,提醒他们去投票。需要去的地方有三种:最优先的是民主党占优势的地
区,就如我上次在郊区户访的地方一样,沿街挨家挨户挂过去就行了;第二优先的是双方平分秋色的地区;最后则
是共和党占优势的地区了。克里阵营为这两类地区都提供了支持者名单。

那位义工告诉大家,今天我们有很多选民要找,为了节省时间,把传单挂在别人家门上就可以了,不要敲门,
即使和住户遇上,也不要花太多时间谈话。凡是在我们名单上的人,都是克里支持者,我们已无需再说服他们,一
个提醒投票的传单足矣。

她还叮嘱我们,不要怕路远、路绕,尤其是去第二、三类地区的义工,所有在名单上的人家都务必要送到。
``大家不要忘了四年前的佛罗里达,500票就足以改变整个选举结果。哪怕你到这户一看,已经有布什的传单挂在那
里了,你也不用怕,依然要把我们的传单挂过去,因为这很可能是丈夫支持布什,妻子支持克里,我们要把信息传
达给所有的支持者,让她们知道我们需要她们的行动。''

传单是一个狭长的硬卡片,上方有个圆洞,正好可以套入门把。正面的画面是克里和克林顿在上周费城的拉励
上,反面是克里和一群黑人在一起亲密拥抱,然后是``星期二投票''的大号文字。我想这个传单大概是费城总部做
的,使用了克林顿和黑人形象,应该是想激励黑人出来投票,但我们这里的住户多是白人,郊区总部居然改都不
改,直接就用,我觉得有些失职。

我由于自己开车,所以可以带一些人一起去,尤其是外州义工。组织者却告诉我,今天有很多司机,所以暂时
没有人需要我带。考虑到我呆会儿还要去帮恰克插牌子,可能时间比较紧,我就干脆一个人去算了。她们特意给我
找了个较少的任务。

在高速公路上开了20分钟后,我到达了目的地。这个社区看上去很不错,比我上次户访的地方要好,当时是下
午3点,周围一片静悄悄的,太阳已偏西,映出一片树叶的金黄色。

我一边走家窜户挂传单,一边看景色,倒也不错。有些人家门前插着``克里---爱德华兹''大牌子的,那肯定
是铁杆群众,明天一定会去投票的,我就跳过去了。至于有``布什---切尼''牌子的,只要他们在我的名单上,我
也硬着头皮上去挂一份传单,还好没有遇到主人冲出来破口大骂。

昨天是万圣节,又叫``鬼节'',大家都用各种稀奇古怪的饰物,故意把房子外面布置得阴森森的。现在是白天
了,再看已没有昨晚的气氛,不过仍然很好玩。

由于大部分住宅都没有人在家,所以我就一路闷头挂传单,偶尔遇到的人,都是老头子老太太在家。他们对我
都很和蔼热情,当我说明来意后,都说:``我们明天当然会去投克里的票。谢谢你的劳动。''

我分配到的地区颇大,横竖共有十几条街,每条街上少的只有一户人家要挂传单,多的则有七八户。不知不觉
地就到了下午4点,我才送到三分之一。孩子们已经放学了,他们在街上或骑自行车追逐,或玩球类游戏,我经常走
到一户人家门前,也不用上去,问问街头的儿童:``你们谁住这里?''往往就会有个小孩子站出来,那么我就把传单
给他们,叮嘱他们交给父母。

有意思的是,有时候孩子们看这个传单好玩,居然争相向我要。我只好每人分发一份,并叮嘱他们,要交给他
们父母看。他们都答应了,然后把传单挂在自行车车把上,招摇过市。


住户也都逐渐下班回来了。由于我的名单上都是克里支持者,所以即使遇到有人在家,他们的回应也都很正面。
还有位女士说:``克里已经得到我这一票了。''大概是已经提前投过票了。

最好玩的则是一位坐在家门口休息的男士,远远地看见我拿着传单往对面人家门上挂,就问我:``你在发什么
传单?是餐馆吗?''---看来他对中国人的第一反应就是开餐馆的,我连忙高声回答:``不是。这是提醒大家投票
的。''

总的来说,克里阵营的情报工作做得不错,所有的地址都是有效的,比我前两天在中国城的奔命好多了。一个
可能原因是:郊区的地址本来就不像城里那么混乱,至少没有公寓和监狱,一座房子一户人,当然好找。

我忙到下午6点多才完,天已经完全黑下来了,下面我还要把资料送回到克里总部,因为他们明天还要用这同一
份表来进行最后的选民动员。我知道已经不可能再有时间为恰克插牌子了,只好给他打了电话,改为明天帮他站
岗,在投票站外发传单。

一切都忙完后,回到家已经晚上8点。我一面继续最后的MoveOn输入工作,一面赶紧给各路联系人打电话,因为
我明天上午要先帮郊区的自由党人助选,然后才能到城里去,因此需要找到人帮我看中国城的投票站。另外我很久
以前报名参加了一个叫亚裔投票权益保护组织的活动,就是在投票处调查亚裔的投票情况,并帮助解决亚裔所可能
遇到的问题,我需要在下午3点到6点的时间段做这件事。好在明天那群高中生会来,我可以让她们帮我。

林及时给我打了个电话,介绍了一个华盛顿来的义工仙特尔(Shantel),可以在明天早上7点就到中国城,一直
干到下午。我当然大喜过望,正好张桦也要去做亚裔投票权益保护组织的义工,她的时间段是下午,因此可以在早
上帮我。我把克里支持者列表的文件从网上发给了她,让她打印出来,明天和仙特尔会合。

终于一切都安排好了,就看明天了。

\section{最长的一天}

一 郊区自由党

按照计划,今天第一件事是帮吉姆站岗。我在大约一月前跟他约好,在一个投票站见面。那是一个小学,离我
家10分钟车程,我在7点准时到达。投票站外孤零零地站着一个义工,身上带着个``克里---爱德华兹''的徽章,我
跟他打了个招呼,走进投票站。

投票站里早已排起了20多人的长队,等待投票。我看了一眼,里面的阵势基本上和我去年所看到的是一样的,
一张桌子后坐了三四个人,摆开了三个投票机,前面都罩着蓝色的厚布,这样外面的人不会看到里面的投票。

我走到队伍的最后,发现那里的布告栏上贴着关于如何投票的说明,还有一张选票的模板,让选民可以在进入
投票间之前,就基本上明白了自己要怎么做。

顺着走廊,尽头是另一个出口。开门出去后,外面豁然开朗,草坪上密密麻麻上插满了标语牌,从总统、参议
员、众议员到州议员,什么牌子都有,其中一张``布什---切尼''的巨型牌子,特别引人注目,选举的气氛一下子
扑面而来。原来这边才是进学校的正口。

有三个义工站在那里,一个是中年男士,一个是老太太,还有一位中年女士站在``MoveOn PAC''的牌子旁,我
便先上前和她打招呼。我做了自我介绍:我也是MoveOn的义工,负责中国城的选区,但现在是帮我的自由党朋友站
岗来了。这位女士叫怀德拉(Wyndra),她说她以前一直是注册的民主党人,后来因为某些民主党人的作为让她很不
满意,她就转为共和党人,两年前又正式转为独立选民,现在则为MoveOn做义工。

我们很自然地开始谈论两党制。怀德拉说:``我觉得目前的两党制很不好,好人绝无出头的希望。现在钱权勾
结太厉害,你如果有可能当上总统,马上就会有那些大财团、大富翁来找你,给你捐钱,让你当了总统后给他们好
处。''我说:``我同意,没有比我们自由党人更痛恨两党制的了。我见过自由党的总统候选人,我保证他是比布什
和克里都好一百倍的人。可是,就算是三党制、四党制吧,只要你能够当上总统,这些事情仍然会发生,和几党制
没有关系。这就是政治,整个系统都已腐败,只有已经腐败的人才能上去。''

其实我不喜欢两党制,主要是因为其他较小的声音很难被别人听到,而不在于怀德拉说的人品。我说:``我经
常会惊异于人们在选举中花了那么多力气去攻击对手的人品,而非政策。我们应该知道,政客就是那么回事,重要
的是他们是否能带给我们好处。''

怀德拉说:``是的。我觉得我们应该更注重议题,而非人品或者党派。我从不会只把选票投给某个党,而要注
意看具体候选人的主张。''

另外两位义工,不出我所料,中年男士是共和党义工,穿得西装革履,头发胡须都经过精心修饰;民主党的义
工老太太则是一身休闲的穿着,慈眉善目的。不过我们四个并无隔阂,吉姆还没有来,我无事可做,怀德拉说另一
个会带来名单的义工还没有来,所以也只是等待,那两位义工也都比较温和,并不硬往人们的手里塞传单,于是大
家站在一起随便聊天。

他们都是本地的居民,尤其是怀德拉,父亲就是这所小学毕业的,她自己和孩子也毕业于此,现在是她的孙子
在这里上学。进口处人来人往,她不停地和人打招呼,好像有一半人她都认识。我觉得这种做本地的义工真是不
错,不像我最近总在陌生的地方跑,基本上是硬着头皮上,心里其实没底。

那两个义工之所以不硬发传单,是因为他们不知道人们到底是来投票的,还是送孩子上学的。我问道:``难道
在这里上学的人不都在这个选区吗?''她们都说不是,怀德拉还指着对面一条街说:``我认识就住在那里的一家人,
他们却要到另一个选区去投票!''

就我所看到的,到这儿来的人绝大多数是白人,有两三个亚洲面孔,黑人一个也没有。怀德拉告诉我,这里的
共和党人和民主党人比例大约是2 ∶ 1,我说:``听说最近民主党人正在增多。'' 怀德拉说:``是的,因为现在不
断地有人从城里搬到郊区来,另外由于有些新技术公司在这里,所以新移民越来越多。''

他们几个然后又谈起了当地人不满意把投票站放在小学,主要是父母担心孩子们的安全问题,因为今天会有很
多成年人到来。

怀德拉又到投票站里去转了一圈,回来告诉大家,在第一个小时内,已经有105人投了票:``这是个从来没有过
的纪录!''

我们正聊天时,又来了一位女士,脖子上挂着一个牌子:``投票法律顾问队(民主党)'',原来就是传说中的被
派到宾州来的民主党律师。她叫琳达,很精明能干的样子,跟我们寒暄之后,首先声明:``我并不只为民主党工作。
我来的目的是帮助任何有投票问题的人,不管她的政治取向是什么,我都要保证她能投出自己的一票。''

不过她并没有什么事可做。一来,我看这里的居民都常积极参与政治,对投票很熟悉;二来,这里的选举委员
会主席很公正严格,他们几个都认识这个人,对他的人品都一致肯定,认为他不会故意刁难任何选民,也不会给任
何人钻漏洞的机会。

于是她也加入了我们的聊天。她说:``我来自北卡罗来纳州,我们那里静悄悄的,一点选举的动静都没有。当
我来到宾州,哇,一股选举的气氛立即扑面而来,简直是文化震荡。''


怀德拉说:``是啊,我每天都要收到五六个电话,提醒我去投票,或者劝说我投谁的票。''

民主党的义工说:``我上次去户访,敲开人家的门,他们说,你们怎么又来了,今天你们的人已经来过三四次
了。''

这时一阵风刮起,将琳达带来的放在民主党架子上的``投票法律顾问队''的牌子刮落地上。她问大家:``谁有
胶带纸?''结果大家都没有。我想起我车里好像有,就回到车里去拿。

穿过投票处,我从另一个门出来,却发现早上遇到那个义工还站在那里。我想:``这倒不错,他们民主党来两
个人,一人守一边。''就问他:``你是克里阵营来的吗?''

他说:``不,我是MoveOn的。''我说:``哦,其实我也为MoveOn做义工---那另一边的那个MoveOn的人跟你是
一起的吗?''

他惊讶地说:``哪一边?''原来他一直在等怀德拉,把表格交给她。他作为义工,不能进入投票处,所以一直不
知道那边还有个门,于是两个人站在不同的出口处,相互苦等了一个小时。

正好我看到他的脚下有一盘胶带,就让他带着所有的东西,跟我进去。当然,我提醒他把身上的那个``克
里---爱德华兹''的徽章拿下来了。

然后他们两个人见了面,自嘲了一番。琳达借了他的胶带把牌子粘好。我看看时间,已经8点半了,还不见吉姆
的踪影,早上给他的电话留了言,他也没打回来。我想,算了,不如过去帮助恰克吧。于是和大家告别,回到停车
场,一边给恰克打了个电话,他也不在,我只好留了言。

这时吉姆却又忽然回电话了,原来他刚收拾好要出发,马上就赶过来。我只好折回,过了一会儿他到了,先拿
出``自由党''和``班纳瑞克''两个大牌子,插在投票处的进口,然后拿出他自己的传单来,看到个人过来,就往他
手里塞了一份:``请投我一票!''

他这样让候选人亲临选举站拉票,一般效果都是比较好的,因此我建议,我没必要在这里和他一起站岗,不如
他给我一部分传单,我到其他地方去发,最好是既能帮助他,又能帮助恰克的选区。吉姆说,他们俩的选区不重
合,我又不认识其他的投票站,不如让我留在这里,他走。

这期间的人已经不太多了,大概是因为很多人都是在送孩子来上学的同时进行投票,现在又都上班去了。仅有
的几个人,我给他们发传单,他们都不要,不知道是因为讨厌这一套,还是因为他们干脆就不是来投票的。

9点时,恰克来了电话,原来他也是还没出门。我早上在等吉姆时,和怀德拉她们开玩笑说:``我的朋友太自由
党了,所以还没有到。''看来还真没有说错。他给我指明了方向,我开车过去和他会合,然后他带我去了他选区的
一个投票点。

那也是所小学,恰克昨天就已经过来插了四个标语牌,都有他的名字在上面,加上他的主要竞选口号。我最喜
欢其中的``5\%统一收入税'',就叫他站在那里,让我拍了个照片。

其他三个是``大麻合法化''、``允许同性恋婚姻''和``结束伊拉克战争'',和他的民主党与共和党的对手的牌
子正好插在一起。

这里的进口处有台阶,上面早已站满了人,有六个义工在那里发传单,我看实在挤不下了,就在台阶下站岗。
还没开始发呢,台上的一个人就跑过来问我:``你是哪里来的?''

我说:``我是自由党人,来帮我的朋友的。''这时我看见他胸前别了个``克里---爱德华兹''的徽章,就
问:``那你是克里阵营的了?其实我也在克里阵营做义工,并且是MoveOn在中国城的选区领导。''

他说:``对,我是克里阵营的,不过我却不同意他的全部观点。我其实主张小政府。''

我拍着他的肩膀说:``那你是个自由党人!---不过我和你一样,觉得今年更重要的是打败布什,所以为克里做
义工---可你以后应该投票给自由党人!''

他笑了笑,说:``我会的。''就跑回台上去了。

我站了一会儿岗,发现没什么人来,就走到投票站里面去看看,结果发现学校对我们这些人防备森严,只能在
一定的区域内走动,绝对不让我们靠近孩子们的教室。

出来后我也站到了台上,这里共有两个克里阵营的义工,两个MoveOn的义工,共和党的人只有两个,而且还主
要是来推销他们的一个州议员的,对布什的连任好像不是太上心。

我和MoveOn的义工聊了一会儿,他们送给我一些MoveOn的徽章和绶带。另一个克里阵营的义工则过来很高兴地
和我聊中国,原来他的太太是香港人,他还给我看他们的全家福照片。

不过这时候的选民已经来得不多了,而且大部分都对传单不感兴趣,可能因为今年的选情太紧张了,他们都早
已拿定主意了吧。到了十点,我便离开了。临走前给他们拍了张照片,这时只有四个义工站在那里,不用说,那位
衣冠楚楚地如同小马哥一般的,是共和党人,而另外三位衣着随便的则是克里阵营和MoveOn的义工了。

二 南征北战

虽然我早上在郊区,不过和城里的联系一直没断过。首先是给 仙特尔和张桦打电话,了解那边的情况,然后给
索尼亚汇报。我很快就发现自己忘了给张桦一份按字母顺序排列的克里支持者的表格,好在索尼亚帮我打印了,给
我又派来了一个义工南(Nan)带过来,这样她们三个人可以一个看投票点,两个去户访。我需要定时给索尼亚汇报,
所以必须每过一会儿就给她们打个电话,确保一切都很顺利。


帮恰克站完岗后,我终于可以去城里了,由于路上堵车,我在11点半才到了中国城。张桦已经在帮那个亚裔投
票权益保护组织干活了,我找到仙特尔和南,大大感谢了她们一番。她们俩都是从早上7点开始战斗的,我想现在也
该累了,就让她们先去吃午饭。杨蕙也从纽约赶到了,我和她先看着这里。

杨蕙会说广东话,今天可帮上大忙了。中国城的投票站是救火站,就在中国城最繁荣的一条大街上,杨蕙站在
街边,见人就用广东话问:``您投票了吗?''结果还真给她拉进了几个本来只是路过的人。

我则拿着我们的克里支持者表格,见到投票站有人出来,就去问他们的姓名,然后在表上把他们勾掉。不过我
们的表上只有一百多人,大部分选民都不在上面,所以进展甚微。

仙特尔和南吃完饭后,我正打算也去吃饭,杨蕙却领来了三位老人。原来他们是来投票的,拿出选民登记卡一
看,投票点却不在这里,在他们家附近。他们自己倒是开车从家里到中国城来的,但一来语言不通,没有能找到那
些地方的把握,二来他们从没有投过票,需要人帮忙。我责无旁贷,立即开车带他们去了。

这三位老人,一位姓张,一家姓杨,张先生说广东话和一点点英语,但能听懂国语,杨先生和杨太太说广东话、
福州话和国语,于是我们各种语言全上阵,在车里混聊起来。

据杨先生说,他们今天只是和往常一样,到中国城来喝茶。喝完茶后,大家一商量,也没啥事,去投个票看看
吧,于是就跑到救火站来,然后才知道原来自己不是在这里投票。他说:``华人应该投票。我们这些人到美国这么
多年,像张先生,到美国60年啦,这还是他第一次投票。我们都不知道这个东西啊,也从来没人告诉我们应该去投
票。''

这个问题有现成答案,我说:``是啊,如果我们不投票,政客们就不会关注我们的利益。---那你们这次为什
么要登记为选民呢?是在超市门口被人家拉住的吗?''

杨先生说:``不是啊。我们自己忽然就收到了这个选民登记卡,于是我们今天就带着出来了。''

我觉得很奇怪:``不会啊,你如果自己不登记,永远也不会有人来帮你登记的,因为需要你自己的签名的。是
不是你们自己登记的时候,别人没说清楚,你们以为是其他的活动,就签名登记了?或者,你们的家人替你们登记
的?''

杨先生说:``不会。如果我们自己登记,我们会记得的。我们家人也没帮我登记啊,像张先生,他家里就他一
个人,谁帮他登记?''

我也无法解释,心里暗想:早就听说美国选举里有很多黑幕,也许有人看准了这些老华人从不投票,便偷偷地
拿他们的名字登了记,然后到投票快结束时,去冒名投票?

我们先到张先生的投票处。那个地方比较特殊,竟然设在某家人的住处,因此不像一般的小学、教堂这么好找。
我们在那里转了一圈,半路上忽然遇到一个中国人,张先生大声叫我停车,下车和他寒暄兼问路。不过这个人也不
认识路,我赶紧问他:``您投票了吗?''他说:``我不能投票。''

好容易找到地方,那里果然比其他投票处冷清多了,门前几乎没有任何牌子,有三个人站在门口聊天,其中一
人是坐在轮椅上,好像全是义工。张先生认识其中一人,上去和他大声打招呼。

我们走进投票处,那是在人家住宅的楼梯间,分外狭窄低矮,摆开了两台投票机,一张桌子。我上前解释
说:``我不是来投票的,但我的朋友不太通英文,我可以来帮他吗?''工作人员说:``当然可以。''

张先生拿出了他的选民登记卡,他们在一本厚厚的记名册上翻找到他的名字,记下了一个数字,即他是当天第
几位选民,然后请他签上名,就可以投票了。我再次询问后,被允许进入投票机的幕布后,帮助张先生投票。

这里的选票机和我去年在普王市看到的是一样的,都是一张矩阵表,如同早上在蒙郡的那张,横排是各个职
位,从总统、参议员直到州议员,竖排是各党的候选人,依次是民主党、共和党、自由党、绿党、宪法党,最右侧
是自己写入的候选人。比如,第一行第一格是克里和爱德华兹,第二格是布什和切尼,如果你要投纳德尔,他不在
选票上,就只好自己来写入。

要选某个候选人,只需按他名字下的按钮,这一格的红灯就会亮,表明你选了他,不然的话,这一行的职位格
下的红灯仍然亮着,提醒你还有些职位的候选人你没有选。每一列的党派名字下,也有个按钮,一按之后就自动选
了该党的所有候选人,倒也省事。

除了各候选人外,最下方还有个公投的市政问题:``费城市应当借款92,195,000美元来从事于:交通、街道和
卫生、市政建筑、公园、健身和博物馆、经济和社区发展吗?''

我向张先生解读了选票后,他首先选了克里,其他就不知道该如何选了,就叫我帮他选。我说:``这投票权是
您的,我可不能代您做出选择。''他用广东腔很重的国语说:``可是我也不知道怎么选啊。你知道的,你帮我选好
了。''我说:``这是您在投票,不是我投。我的选择要是和您不同,不是损害了您的权利吗?''他说:``没关系,反
正我也不认识这些人,也不知道该选谁。''


我推辞不过,心里也难挡投上一票的诱惑,就帮他选了。右下方有个绿色的投票键,我请他按了一下,投票间
里的灯光顿时灭了,这个票算是投上了。

我们谢过了工作人员,工作人员也谢过了我们来投票,并感谢了我的帮忙。随后我们开车去杨先生的投票处。

那是个小学,坐落在个小镇上,可能由于是人口聚集处吧,门口的义工也特别多,怕没有十来个,牌子插了一
地,看见我们过来,立刻殷勤地上来递传单。

我带杨先生夫妇登了记,正要去投票,却见工作人员把他们俩的登记卡撕掉了。我吓了一跳,怕里面有什么猫
腻,忙问她们是怎么回事。她们回答说:``将会有新的登记卡寄过来。''

这里的投票机和那边的是一样的,杨先生也是选了克里之后就不知道该怎么办了,又要我帮忙。我刚才在车上
反省了一下,觉得我的观点跟他们肯定不同,不能滥用人家的信任,浪费他们的选票,因此这次坚持不肯。他后来
就让我全选了民主党,至于那个选票问题,我念完之后,他说:``选Yes!''

杨太太的票就好投得多了,我正要向她解释整个选票,她很干脆地说:``你帮我选和我先生的一样就行了。''
于是三个键一按就投完了。

投完票后,我问他们,是要我送他们回家,还是回中国城。他们说要回中国城,因为他们的车还停在那里。

回到中国城时,已是下午2点半,我从早上6点起来,到现在8个小时颗粒未进,早已饿得肚皮贴后背,赶紧去吃
饭了。

三 中国城,救火站

吃完饭后,仙特尔和南已经走了。我和杨蕙两人继续工作。不过,由于让选民签到的活儿基本上没有什么效
果,我们打算直接去做户访。在此之前,我先到投票站里看了一下。

现在是下午3点多,人并不多,大概只有五六个选民在排队,听说早上的队伍曾经排出救火站,蜿蜒转过了街角。
张桦仍然在做着亚裔投票保护,基本上就是每当有亚裔投完票后,上前请他们帮忙填个表,调查选民的族裔、何时
成为美国公民、何时登记为选民、是否首次投票、英语程度、是否需要翻译、党派、总统和参议员都投了谁、最关
心议题、最关心移民议题、平时新闻来源及语种、教育程度、年纪、性别等,当然都是匿名的。

最重要的一个问题,是问他们是否在投票中遇到任何问题。调查表上列出了各种可能:名字不在投票名册上、
必须投临时票、工作人员阻挠投票、工作人员能力不足、无语言协助、被带到错误的投票处、翻译文字太小、机器
坏了,或者其他问题,并要请投票者再填一个投票问题报告表。

他们是无党派组织,所以身上没有任何党派标记,可以呆在投票站里头,等选民投完票后便请他们参与调查,
目的是保护亚裔选民的投票权益,并做一个关于亚裔选民的调查。

这时我又遇到一个华裔选民,不通英语,于是我和他一起排队到投票桌前,那里坐着四位工作人员,两位是亚
裔,两位是白人。他拿到了自己的号码,正好是800,给他开号码的是一位白人女士,却也知道这个数字吉祥,开玩
笑说:``你的运气不错!''

我则惊讶于今天来投票的人数之多,这时候才下午4点,5点钟之后大家都下班了,可能会再掀一次投票高潮,
那么今天的票数是肯定要超过一千了。工作人员也说:``今天投票的人特别多,往年从来没有这么高的投票率的!''

我陪这位先生走进投票间,他却是很有主见的,直接说:``全选民主党!''至于那个选票问题,他也选了``是''。

然后我才注意到投票机旁有一位亚裔女士,想来她也是可以提供语言协助的,不过我越俎代庖了。

这时,那群高中生陆续都到了。我给她们布置任务:``我们有两件事,主要是确保我们名单上的克里支持者都
出来投票了,那么一群人可以去户访敲门,一群人可以去打电话提醒。现在天慢慢地要黑了,让你们去户访我不太
放心,你们就专门打电话吧。''

结果她们都说没带手机,因为学校里不让带。我自己的手机快没电了,再说我一个人的手机也不够她们打的。
我只好给她们找了另一件事:亚裔投票保护,去顶替我的岗位。不过她们这么多人都去干这个,显然太多了,好在
有几个人又觉得还是在外面发传单、摇旗号更有趣,于是三个人在里面,四个人在外面,分配得正好。

索尼亚也打电话来,询问这边的情况,主要是来了多少选民、有多少是我们名单上的、我们接触了多少选民、
现在有多少义工、是否需要新的义工。我告诉她,现在我们只有两个义工,需要有手机的义工来帮我们打电话。她
说马上给我派两个人来。

乘着新的义工还没来,我便和杨蕙户访去了。结果出师不利,所到之处皆是公寓,我们进不去,只能在外面打
电话,结果是一打一个准。我们觉得再走下去也徒然无益,便回到救火站。

这时救火站外可热闹了,足有七八个克里的义工在这里活动,有一位义工举着个``为克里鸣笛(Honk for
Kerry)''的牌子向过往车辆晃动。费城是民主党的地盘,大部分人都是亲民主党的,自然笛声不断,有些老外还一
边鸣一边向我们挥手或大叫。每一次鸣笛都引起我们的尖叫回报,气氛极是热烈。


我们也加入到她们的行列中,这倒不是在助选,而是在玩乐了,大家借此机会,在此起彼伏的汽车喇叭声中,
大喊大叫大跳。那几个站在外面的高中生也乐不可支地参与进来。

新的义工很快就到了,她们先拿出两张卡片,上面写着她们的名字、电话号码等,交给了我,算是报到。我把
克里支持者的名单给她们,上面的一百多人中,已经有三十多人被确认投过票了。我请她们给上面的外国人名字打
电话,然后我同时借杨蕙的手机,给看上去是亚裔的名字打,争取在六点之前,打完一轮,然后进行最后的户访,
去找那些没有留电话号码的人家。

结果,我打过去的几乎一半的人是说广东话,然后我又要叫杨蕙来帮忙。总的来说,有大约三分之一的人号码
错误或者联系不上,其余的大部分都说已经投票了,只有几个人说即将来投票。

我的电话还没有打完,那两位义工已经把外国人都轰炸过一遍了,于是我把结果汇总了一下,给她们一份依街
道排列的克里支持者名单,让她们到所有还没有投票的人家去敲门,我自己留下来继续打电话。

天色已经完全黑了下来,新的义工又涌了进来,大部分是克里支持者,也有几个布什阵营的人。其实今天早上
就有一两个布什支持者在这里,有一个还戴着``老兵支持布什''的徽章,在投票站外也贴了很多布什阵营的宣传资
料。

这里的气氛并不像郊区那么友好共存,双方都忍不住想辩论。不过我们的人数远远超过他们,因此他们后来只
好放弃了,专门找那几个高中生说道理去了。

小道消息也在满天飞,我们这里没有电视看,也没有网可上,不过仍然有人打电话四处探听消息。不久就听
说,宾州的出口民调克里领先10个百分点,大家一片欢喜。又有人说,布什在新泽西州领先20个百分点,于是大家
又一片惊疑。我是压根不信的,连声说:``这是假的,假的!''旁边的人也都说:``对,是假的!''

高中生们到了六点就走了,我连忙和她们合了张影。

有一个小孩子穿了个大纸箱,上面贴着``学生支持克里''的标语,还写着:``克里:礼物;布什:捣蛋。''前
天是美国的万圣节,孩子们都会打扮得奇形怪状的,到人家敲门要糖,咒语便是``不给礼物就捣蛋''。这个孩子却
给大家分发克里的传单,嘴里还念叨着:``这是礼物,不是捣蛋。''

地上还有不知道谁写的``克里''两个字,大概是谁向好奇的老外显示克里的中文写法吧。

我依然在打电话。随着时间越来越晚,接电话的人几乎都清一色地说已经投过票了。有户人家,一接电话,我
刚说了个``你好'',他马上说:``我们已经投过票了。''不过总算没有立即挂电话,让我还来得及谢过了他。

那两位义工也回来了,她们在户访时进入了一栋公寓,在敲门时却被安全人员发现,很粗暴地被赶出来了。

我把电话名单和她们的户访结果对照了一下,发现剩下的人也不多了。这时已经快到晚上8点,投票站即将关闭。
我找到一户人家,就住在这条街上,三个街区之外,决定再做最后一次尝试。结果跑到这家门口,却发现地址不
对,找不到这个人,只好折回来。

按照规定,凡是在晚上8点之前进入投票站的人,都可以投票,不论这个队伍有多长。我到投票站里一看,却已
经没有人在投票了,大家都顺利收摊。亚裔投票保护组织的负责人一边收拾一边说:``最新消息,选举人票,克里
以77比66领先!''

我说:``我还以为你是非党派的呢!''

他笑着说:``8点钟以前是。现在投票结束了,我又变回支持克里了!''

我们克里阵营和MoveOn的义工也开始撤退,大家握手感谢。

四 胜利派对

劳累了一天,我心里却轻松起来:``终于结束了!''从五月份到现在,多少忙碌,终于结束了。

按美国人的性格,大家当然不会就这么做鸟兽散,必然是要聚到一起开派对。MoveOn的派对就在旁边,不过我
问了周围的人,都是要去民主党的派对,地点就在克里在费城的总部旁边,于是我也去参加那个派对。

开车过去,一路上仍然到处可见选举的痕迹,尤其是到了派对附近,满街都是克里支持者,电视采访车也停了
好几辆。有人举着``为克里鸣笛''的牌子,我当然给他狠命地鸣了几下。

派对在一家酒店里,也不需要什么证件,人群络绎不绝地进去。

走到二楼,派对却分为三处,我先走到主会场,发现那里太挤,就另外选了个僻静的房间,拿了些食物,买了
瓶水,找了个沙发,坐下来慢慢吃。也实在太累了,我眼睛都疼了。到处都是的电视却不体谅我,仍然在不停地报
导各州的消息,害得我不时地要看一看。

宾州的消息最让人兴奋了,一开始报出来是克里72\%对布什27\%,引得大家狂叫不已。虽然我们都知道,这肯
定是由于费城或者匹兹堡的投票站最先出来了结果,因此严重偏向克里,但仍然忍不住要高兴。托克里的福,乔
$\cdot$霍福尔在参议员选战中也遥遥领先阿伦$\cdot$斯柏克特。

吃饱喝足后,我基本恢复了元气,便又回到主会场。那里挂着三个巨大的屏幕,其中一个在放美国有线新闻
网,一个在放国家广播公司,还有一个是竞选网站的更新。


在会场一侧,有个乐队在表演。

媒体当然更不会缺席。

会场中间,很多大学生模样的年轻人席地而坐,听着各种更新的消息,只要有任何利好消息,她们就一阵阵地
尖叫鼓掌,甚至连克里在马萨诸塞州、纽约州这样的铁杆州领先都要让她们尖叫一番。随着郊区票数的点入,克里
在宾州的领先幅度一路下滑,不过仍然保持着相当的优势,这当然也让她们激动雀跃。

虽然票还没有完全开出来,电视台已经开始预测各州的胜者,像马萨诸塞州、纽约,自然是归克里,德克萨斯
则毫无疑问地属于布什。由于4年前电视台报错选情,误导了选民投票,所以今年他们格外谨慎。我去看的时候,布
什正领先60多票。

好在我们很快就被打入了一针强心剂:电视台宣布克里赢得宾州!一时间,全场沸腾,所有的人都站了起来,尖
叫鼓掌,持久不息,因为这意味着我们在场的宾州克里支持者的辛苦终于有了收获。场内气氛极为热烈,大家都一
片喜气洋洋,好像已经赢得最终胜利。

紧接着,加州的投票即将结束。当电视上开始倒计时西部各州的投票结果时,大家都高声跟在后面喊:``四、
三、二、一、耶!''然后欢声尖叫。电视台连统计都懒得做,立即把拥有55票之多的加州算入克里囊中,这时克里和
布什的票数几乎已经相同,难怪大家都那么兴奋。

这些铁票的涌入,虽然也是进展,不过毫无意外之处。关键还是在各摇摆州。宾州我们已经获胜,比较让人担
心的是另外两个关键的摇摆州:佛罗里达和俄亥俄,克里在已经清点的投票里总是以51\%对48\%的样子稍微落后。
我们只有耐心等待,期盼奇迹出现。

结果天不遂人愿,佛罗里达州的计票结束,布什获胜。这极大地冲淡了我们方才的喜悦,现在双方又站回到同
一起跑线,就看谁能赢得俄亥俄了。

我本想在这里一直等下去,直等到克里最终获得胜利,和这些年轻人一起共度那久盼的时刻。可是我停车的地
方到11点45分就关了,我只好在11点半的时候走了。

自由党人今天也照例会聚在当地的一个酒吧开胜利派对,我便开车过去。一路上收听广播台,消息好像都不太
妙,共和党在参议院和众议院的选举中都是得大于失,进一步扩大了他们在两院的多数地位。尤其是宾州竞选参议
员的民主党人乔$\cdot$霍福尔,也输给了寻求连任的共和党人阿伦$\cdot$斯柏克特。

待我开到自由党人聚会的酒吧时,发现它已经关了。我只好转道回家。

回到家后的第一件事当然就是打开电视机。这时的选情已经很明显,克里和布什都已经得到超过250票,谁赢得
了有20票的俄亥俄州,就能达到270票的多数,当选总统。俄亥俄的计票几近结束,布什一直稍微领先,国家广播公
司和福克斯已经干脆把俄亥俄划给布什,并因此宣布布什胜选了,有线新闻网还在顽抗,不过看样子已经没什么希
望了。

最后克里阵营祭出一招:要求清点临时选票,希望能够扳回,但那需要大约十天的时间。这时已经是凌晨两点
多,我20个小时没合眼,既已无望等到结果,只好睡觉去了。

这时我明白基本大势已去,睡前给杨蕙发了封Email,感谢她多次从纽约奔来帮忙。她本来曾经计划今天去俄亥
俄州的,最后还是决定来宾州。我在信中开玩笑说:现在你知道你有多重要了。

\section{选后余波}

下午2点,克里阵营正式承认败选。俄亥俄州的临时选票还没有清点完,但是克里阵营已经看出他们不可能赢得
这关键的20票了,不如就此认输,还不失体面。1小时后,布什发表胜利演讲。至此,大选正式结束,俄亥俄州和其
他两个州的点票仍在继续,但已无关大局。布什连任成功。

说实话,我一开始不能相信克里失败了,因为这几个月来,我看到民主党人非常勤奋地工作,对选民地毯式轰
炸般地户访,再加上布什的极端立场,激起了空前的反布大联盟,应该最大程度地动员起投票来了。至于民意调查
里布什稍微领先,我一直比较相信迈克尔$\cdot$摩尔在他的信里所说的,民意调查没有计入新选民和大部分年轻
人,因此结果严重偏布什;所谓布什稍微领先,其实是布什稍微落后。昨天我跑了好几个投票处,城里固然是克里
的义工完全压倒布什,至关重要的郊区,也是克里阵营的义工占绝对优势。所以昨天晚上投票结束时,我已经比较
相信克里会胜了。---事实上,在宾州克里确实胜了,要怪只能怪我没去俄亥俄或者佛罗里达看一眼。

我当时甚至开始不无同情地设想克里下面这四年的总统生涯。在我看来,他如果当选,靠的是选民对布什的厌
恶,而非对他的支持。大家只是把他当成赶走布什的一个工具,他一旦当选,选民就要过河拆桥,找他的麻烦了。

从我个人的立场来看,如果克里当选,我也会反对他的那些经济政策。迈克尔$\cdot$摩尔在民主党全国大会
期间说:``如果克里当选,他的就职演说一结束,我的摄像机就对准他。''这个说法深获我心。在美国,专门有一
批人,并没有确定的政治立场,但总投票反对在职者连任,以防止一党独大。对于我来说,政府是人民的``敌
人'',谁上台就反对谁,也是再正常不过。

我本来想,如果克里当选,共和党往右急转的势头也许可以被抑制住,在2008年推出一个温和派来竞选总统,
比如前纽约市长朱利安尼。那我甚至会去做共和党的义工,因为共和党温和派的立场是最接近自由党人的。现在看
来,共和党从极端立场尝到甜头,更可能会在四年后又推出一个右翼候选人。

至于民主党,我希望他们能够找一个中间派来竞选总统。目前希拉里$\cdot$克林顿的呼声很高,可是她太左
了,在正常情况下,恐怕连党内初选都过不了。当然,有比尔$\cdot$克林顿在,一切还很难说。

自由党人的选举结果也出来了,班纳瑞克在全国范围内一共赢得0.3\%的选票,略少于纳德尔的0.4\%,但是比
其他第三党人候选人的得票加起来都多。班纳瑞克的得票率比我估计的要低,主要可能是因为今年的选举太激烈
了,很多本来会投给第三党的人,比如我们公司的雷蒙,这次都觉得自己的选票奇货可居,最后投给了两大党。

吉姆总共得到了11\%的选票。他的选区横跨蒙哥马利和切斯特两个郡,吉姆的助选活动都集中在蒙郡,因此在
蒙郡的几个投票站,他的得票率有15\%之多(如果他起得再早些,大概还会更多)。对于一个自由党候选人来说,这
是很了不起的成绩了。当然,这里面最重要的原因是,民主党在这个选区没有推出候选人。在共和党候选人稳胜的
情况下,有一部分人不喜欢共和党的人会投票给自由党候选人,共和党内的温和派也可能投票给他。

恰克的运气没有那么好,两大党都有候选人,他的生存空间就小多了,最后只得到了1.3\%的选票。

我比较遗憾的是,本来我计划花一半精力帮助恰克或者吉姆来竞选,因为正如美国人所常说:``所有政治都是
地方政治'',我觉得众议员选举其实比总统选举更有意义,也更能体现民主政治的本质。但后来克里阵营需要投入
的时间太多,自由党人也没有大规模竞选的计划,我因此失去了一次更近距离体验草根政治的机会。

11月13日附记

选举已经结束,所谓愿赌服输,下面我们必须面对现实。胜负乃兵家常事,大选四年一次,小选年年都有,我
对政治得失本来就看得比较开,沮丧开始于投票日晚上克里在俄亥俄州落后之时,到第二天晚上也就基本结束了。

APAP的尼娜在3日晚上给我打来了电话,聊了一下各自的经历和感受,勉励我继续战斗,还说要办个派对,大家
讨论一下以后的计划。随后我也给合作过的人发Email,感谢他们的帮助,不管怎么样,至少我们赢得了宾州,这使
我们稍可自慰。我还给南费城高中的林老师打了电话,感谢她的学生们,她们的帮忙是我这次助选活动中非常愉快
的一部分。

Email信箱里也充满了这种感伤的告别和道谢。克里发出一封正式的感谢信,题目是``一份真诚的感谢''。除了
感谢之外,他也把调子往上拔了拔:不要丧失信念,你们所做的一切都将有回报,终有一天,我们将改变世界。他
随后还将担任四年的参议员,在败选演讲中,他保证将继续为人民而战。

郊区总部的杰娜丽,来自马萨诸塞州,在启程回家前,发信感谢大家。MoveOn也发Email感谢所有的义工,诺亚
专门给为他户访过的人发了感谢信,索尼亚则给我打来电话,我们互相感谢了对方的辛勤劳动。还有一些不认识的
人,在邮件组里发信,感谢所有的参与者。


大部分人都已经接受了失败的事实,不过几乎都没有忘记继续激励,因为选举还会一年一年地办下去,我们还
有卷土重来的机会。我很喜欢美国人的一句话:所有的政治都是当地政治。我们虽然在总统选举中倾注了大量的精
力,但其实当地的小候选人,才是与我们的生活更息息相关的。

在宾州,民主党干得还是很不错的,总统选举可以算胜,参议员选举输掉了,州的高级官员选举2胜1负,联邦
众议员的选举5胜6负,但在州参议员选举中,则是11胜6负,州众议员79胜13负。当然,这里面可能有很多选区没有
民主党人参选,因此没被计入,但如此压倒性的胜利,足以让宾州民主党人自豪了。

很多Email开始谈论将来的计划,比如明年的选举、后年的联邦议员选举,号召大家继续团结在一起,四年后重
新翻身。英文中``竞选活动(campaign)''一词,本义就是``战役'',现在大家也正如输掉了战役的老兵,一面撤退
休整,放假回家,一面互相激励:我们曾英勇战斗,战争还没有结束,我们最终必将获胜。

平时我一天大概收到七、八封Email,选举结果出来后,一下子激增到一天20多封。除了前面说的感谢、激励之
外,还有一大部分是对选举结果的怀疑和对共和党人的嘲笑。最近也确实流言满天飞,诸如俄亥俄州某选区布什的
得票数是该区注册选民总数的好几倍,佛罗里达州某郡的投票机往反方向计数,俄亥俄州的投票机制造者Diebold公
司有可疑的共和党背景,更成为流言风暴的中心。

有些人对克里的过早认输很不满,他们成立了一个``我们不认输''组织,一方面继续关注这次选举的最后计票
工作,一方面也是维系住血脉,大家继续团结在一起为其他议员战斗的意思。

MoveOn也不甘心如此失败,他们起草了一份``调查投票''的请愿书,呼吁大家去签名。

还有很多Email是嘲笑布什支持者。最流行的话题就是智商了。有一个表格显示,平均智商较高的州,基本都支
持克里,布什赢得的大多是平均智商较低的州。这个表格后来被证明是伪造的。有人则不知道从哪里收集来一份全
美笨蛋笑话集,然后在每位笨蛋后面加上一句:``可是她会去投票。''还有人说:``克里支持者想选一个智商比他
们高的总统,布什支持者也是。''

我是不大喜欢这种论调的。我们必须看到,布什支持者中除了所谓的价值观捍卫者外,还有大量的小政府主义
信奉者。这些人基本住在郊区和农村,因为城市的政府太强大。他们相信勤劳致富,因此反对政府拿他们的钱去救
济那些好逸恶劳者。他们都有一定的专业技能,是美国经济的构成基础。至少我是不敢说他们笨的。

虽然受过高等教育的人基本上都倾向于支持克里,但我们必须承认,克里支持者中的一大部分,是经济状况不
太好的人群,因此主张通过政府来重新分配财富。我同情他们的遭遇,但他们中有很多人,并没有自食其力的想
法,这是我所不能认同的。

我们知道,共和党是小政府主义者和道德主义者的大杂烩,民主党则是大政府主义者、知识分子以及社会异端
的一锅炖。以民主党中的一部分来代表整个民主党,单独拿共和党中的一部分出来攻击,这一招我认识,咱们老祖
宗两千年前就用过,叫田忌赛马。

有一个表格则列出了各州的富裕程度,显示较富的州倾向于投票给克里,较穷的州则被布什赢得。这份表格是
真的,也颇有点令人大惑不解:那些红州(支持布什的州),都主张小政府,其实却从联邦政府那里得到大量的经济
援助;蓝州(支持克里的州)对联邦政府的贡献大于获利,却主张大政府。难道民主党人真的比共和党人更大公无私?我
想这大概再次说明,美国政治极为复杂,任何简单化都会导致谬误。

蓝州囊括了美国经济中最重要的几部分:新英格兰(即包括马萨诸塞州、纽约州等地的东北部)、西海岸、五大
湖区,红州在经济上远远无法相提并论。因此有人干脆主张蓝州从美国独立出去,并入左倾的加拿大好了,让你们
红州自己实行小政府去吧,看看你们是会变得更富还是更穷。

有人还专门制作了地图,把蓝州和加拿大合并为``加拿大合众国'',支持布什的中西部和南部,则叫``耶稣
国''。

``耶稣国''这个名字,自然是由于不满布什阵营对宗教情绪的利用。有人甚至在邮件组上说,他有个重大发
现:``在我看来,对很多人来说,投票比去教堂或者遵循真正的基督徒生活方式容易。 有些人整天看庸俗电视节
目,一年只在复活节和圣诞节去两次教堂。这些人相信(感谢主流媒体!),投给乔治$\cdot$W$\cdot$布什的一
票就是投给上帝的一票,上帝会喜悦他们的行为。他们投票给布什,便积到了足够的神圣点数,再加上他们每年去
两次教堂,一个'基督教原教旨主义者(因为他们投票给布什)'的标签就是他们去天堂的通行证。投票很容易,思考
却难得多。''

这是典型的自由派对宗教虔诚人士的嘲笑。确实,在这次选举中,共和党巧妙而且有效地利用各种非理性情
绪,比如恐惧心理、宗教情结、反智主义等等。不过我对这种嘲笑是不以为然的。我觉得选民有权以任何标准来投
票,我反对布什利用政府资源来推行他的原教旨主张,但我们虽然不同意选民把信仰带入政治,却必须尊重他们这
么做的权利。


类似这样的发泄情绪的信件还有不少。不过这类信在数量上虽然很多,发信人却只有那么几个,只是由于他们
比较激愤,所以声音显得很大,大多数人已经沉默地回到生活的正轨了。

\section{大戏观后感}

大戏终于落幕。我虽然也在戏中跑过几回腿,不过都只是龙套中的龙套,没有太入戏,大概可以基本上以一个
观众的身份,来说说我的观后感。

桃李不言,下自成蹊,关于美国政治的好处,我想我没有必要在这里再多讲。有个冷战时期的笑话说:一个美
国人和一个苏联人在一起聊天,美国人说:``我敢冲进白宫,拍着美国总统的桌子对他说,你是个大混蛋!''苏联人
说:``那有什么了不起,我也敢冲进克里姆林宫,拍着苏联总书记的桌子对他说,美国总统是个大混蛋!''以此看
来,在下迁居美国有年,如果却不能对美国有所批评,未免对不起开国诸贤保护言论自由的苦心。

美国政治让我吃惊的第一个地方,是过度的爱国主义。我在观看民主党全国大会时,演讲者们对爱国情感的一
再利用就着实让我惊讶不已,随后的共和党全国大会则有过之而无不及,他们毫不犹豫地使用``全世界最伟大的国
家''来称美国,别人批评美国经济就是``悲观'',反对越战就是美奸,甚至反对增加军费的提案就是不爱护子弟兵。

美国人确实有很多可以为自己国家骄傲的理由,但问题在于,很多美国人的爱国主义,不仅仅是我们所认为的
``爱祖国、爱同胞''而已,他们相信自己的国家是全世界最好的一个,是指引人类文明前进的明灯,是拯救世界的
警察,是审判世界的法官。很多人的心里,正如一部喜剧片里的总统候选人所说的:``愿上帝保佑美国,而非其他
各国!''

在我看来,爱国主义如同天花,每个国家都应该出一次,不出的话则难以抵抗别国的传染,但出完之后,也就
应该对它有免疫力了。爱国主义让欧洲的青年们兴高采烈地投入了世界大战,大战之后他们才醒悟过来,开始抛下
国家之间的成见,相互联合起来共同发展。美国的发展历程实在太顺,他们现在如同1913年的德国那般自信,如同
1839年的中国那般自大。由于没有经历过优势文明的入侵,美国人缺乏作为一个民族所应当拥有的谦虚。比如流行
的好莱坞影片里,凡是入侵地球的外星人,一律都是军事原始社会的文明。他们不相信别人会有比自己更优越的地
方,也不认为外面的世界还有任何值得关心之处。

这种``天朝心态''的爱国主义,在很大程度上已经不是自爱爱人、自尊尊人,而是傲慢自大了。它又让美国经
受不住任何挑战的刺激,一个``9.11'',立即激发出几乎歇斯底里的爱国情绪,任何逆耳之言都被方便地划为非友
即敌。更糟糕的是,政客们显然深谙利用爱国主义这面大旗之道,知识分子还可以对过头的爱国主义提出批评,政
客却只会推波助澜,调子唱得比普通民众还要高,因为这是他们可以用来向选民兜售自己的最有效的工具。

我觉得这是民主制度的一个弊端,候选人为了能当选,专门捡选民喜欢听的说。其他各种言论都有可能引发不
同利益集团的争议,爱国主义却可以让全体选民都皆大欢喜。在人民普遍比较理智的国家里,这还不会导致严重后
果,但当大部分美国人都自豪于他们的天朝大国时,以爱国主义为诉求,对外很容易变成霸权主义,对内很容易搞
出个《爱国者法案》来,以国家安全的名义侵犯公民权利。

我对美国政治不敢恭维的第二点,是他们对候选人个人品格的包装,有时甚至超出了对政策的讨论。民主党和
共和党的全国大会,都基本上开成了造神大会。两大党在这次选举中的攻击,也有相当一部分火力集中在对手的人
品上。共和党在这方面做得尤其成功。

我想这可能是由于各候选人的政纲都比较复杂,一般选民没空也没兴趣去仔细了解,因此除了几条``减税''、
``反战''式的口号之外,很多兴趣就集中到候选人身上来了。所以,在这次大选中,我很喜欢那些草根政治活动,
但高层的人和事,我则觉得更像一场大戏。双方卯了劲演出,争取自己的``票''房超过对手,全国大会开得像堂
会,总统和副总统候选人辩论的娱乐效果也不在相声之下,电视里满是正面赞扬和负面攻击的广告,我觉得这与其
叫选举,不如叫选秀。

我并非反对政治的庸俗化、商业化,我认为这些其实有助于使政治更加世俗化。我反对的是政客们在做一件世
俗的事时,却安上一个崇高正义的理由,而全社会都对此不以为怪。当政府这个世俗机构要选举管理人员时,人们
却仿佛在选举教堂的牧师。这是中世纪政教不分的心理残余。我认为我们应当让属于耶稣的归耶稣,属于恺撒的归
恺撒,政府绝不可限制人民的信仰自由,人民也绝不能把政府看作我们的牧羊人。

在我看来,理想的政府应当是完全世俗化的政府,理想的选民也应当是完全世俗化的选民,对候选人的考量基
本上是出于利益计算,而不应当被各种假大空的言辞所迷惑。

当然,选民手中的这一票属于她自己,她有权以任何标准来决定如何投票,包括候选人是否言辞中听、是否英
俊、是否人品伟大。当选民认为人品是压倒一切的价值时,她用投票给看上去最诚实的候选人的方式,来捍卫这一
价值观;或者如果她宁愿用选票来做戏票,看电视上斗猴,我也只能尊重她所作出的任何选择。

同时,我必须承认,我在参加草根政治活动,也曾多次陷于这种非理性情绪而不自知。假如我手头有这个选
票,恐怕也会经常把理性扔到一边,凭感觉来投票,先过把瘾再说。这大概是人性使然吧,不能过于苛责,我想美
国人民的政治智慧也绝对远在我之上。


我对美国政治最不满的一点,还是民主制度对个体的压迫。这种压迫体现在两方面。对于政治参与者来说,为
了赢得选举,他们不可能单打独斗,必须组成联盟,而党派一旦形成,人就必须放弃部分自我。比如亚利桑那州参
议员麦克恩,必须在共和党全国大会上发言,违心地吹捧他曾经公开面斥为``无耻''的布什。民主党的总统候选人
克里,为了赢得支持而说违心话,结果弄巧成拙,选票没有拉到,反倒落了个墙头草的名声。

我们这些草根层次的义工也会遇到这种冲突。比如我在为克里助选时,散发的传单上几乎毫无例外都有``反对
工作外包''这一条,这违背我的自由市场信条,我只能跳过去不和选民说。就算是在我们自由党内,我仍然不同意
很多观点,比如完全不干涉的外交政策。还好我这次基本没有为自由党助选,不然伊拉克问题肯定会被经常问起,
我肯定也无法面不改色地说:``美国没有必要承担任何国际义务,体面地撤军就行了。''

人和人的观点总是不一样的,人们为了能更好地实现自己的愿望而组成政党,这本来无可厚非,也不失为一种
明智的博弈手段。但任何组织,一旦形成,就会压迫成员的个性。组织的严密程度和个人的参与程度不同,其压迫
程度也不同。西方政党是相对松散的组织,对一般的党员几乎没有任何约束力,但积极参与的义工或者高层操作的
领袖,就必须放弃个人的部分主张甚至利益。

我比较喜欢佛门的一句话:``佛由自做,教由魔主''。《水浒传》里个性鲜明的英雄,上了梁山后就成了冲锋
陷阵的符号;灵隐寺里道貌岸然的老僧,佛性又何尝能及游荡酒肆的济公之万一?诚然,集体行动是我们在社会上所
必须采取的策略,但在我看来,``恶''已经潜伏其中。

从另一方面说,无论你是否积极参与政治,一旦你在选举中处于少数,就必须服从多数的意见。民主的本质就
是多数将自己的意见加诸全体,包括持反对意见的少数。它毫无疑问是我们目前所能找到的最好的制度,在法律保
护了少数派的基本权利不容侵犯后,已经比我们所试验过其他各种制度都要公平合理。然而,政府应当是属于所有
成员的公器,在一场选举后,49\%的人的意见得不到声张,要听从另外51\%的人对100\%的政府资源的安排,这显然
是对少数的一种压迫。

比如在这次选举后,很多克里支持者无法接受失败,甚至有人为此自杀,对布什支持者的嘲笑、对共和党作弊
的怀疑,不绝于耳。相比于共和党人的志得意满,民主党人的满腔怒火更让人心惊肉跳。目前,美国民众间的分裂
不仅未见愈合的迹象,反而有扩大的趋势。

在我看来,民主(democracy)当然胜过独裁(autocracy),但又比不上自主。我们的权利如同一个圈,外侧是公
共事务,我们必须组织公共机构来解决,并通过民主方式来管理这个公共机构;圈的内侧是个人事务,由我们自己
做主。公共事务的圈子越大,我们自己的圈子就越小。政府管的事情越少,我们才越能为自己做选择。

在大政府体制下,政府深深地介入我们的生活,掌控着我们的收入、规定着我们的权利、甚至还可以强制我们
上战场。我们无法为自己做出选择,而必须听从政府的安排,这时政府的政策走向当然会引起我们的极大关注。选
举是我们能唯一做出选择的机会,可无论选举的结果如何,总有人会感到自己被胜者压迫,所以这次选举后会有那
么多人反应过激。政治成为人们关注或参与的热门,并不是社会之福,而是给社会敲响的警钟。

所以我主张小政府。我相信个人有为自己做出选择的能力,也相信无数个人从自己利益出发所形成的合力,要
远比少数官僚或专家所安排的政策能带来更大的幸福。政府保护好我们的安全就行了,其余事情让我们自己决定。

比小政府更好的是互相竞争的政府。目前政府垄断了公共事务,任何垄断必然带来劣质服务和贪污腐败,我们
能够选择管理人员,却永远也不可能杜绝政治黑幕和钱权交易。只有当我们可以自由选择政府时,自主才真正地上
了一台阶。也许可以用保安公司来代替政府,使政客变为推销员,选民变为顾客,用商业合同来代替选举,让``政
府''在市场上竞争,而不是像现在这样,一个人只能选择一个国家,一个国家只能有一个政府。

最后,我必须强调,美国政治复杂多元,我所批评的其实只是其中一部分,与我立场相似的声音在美国舆论中
也很多。或许我的这些不满,只是公报私仇,因为这三条的得利者都是布什阵营。至于民主制度,它如同科举制
度,无论我们如何批评它,它肯定都比野蛮愚昧的门阀世袭制度好得多。我现在的言论,有些类似于``何不食肉
糜'',很多读者可能更关心如何填饱肚子,米饭和面包到底哪个更适合我们的肠胃,对肉糜还觉得太遥远。不过我
虽然自己也饿着肚子,无论米饭还是面包我都无限渴望,但我还是要说一句,其实米饭和面包都没有肉糜好吃。

\section{后记}

这本书记载了我参加美国草根政治活动的一些经历。在写这本书时,我所考虑的首要问题,是如何最大程度地
记录真实。本来,我对美国草根政治的观察就如同盲人摸象一般,绝不敢夸口面面俱到,只不过希望为读者的``兼
听则明''多提供一个可听的渠道;如果我连真实都不能最大程度地保证,那这本书就毫无意义了。

我并不缺少记录真实的决心,只是担心自己行文的习惯。在我看来,书写有两种方式:文言文和白话文。所谓
``文言'',就是写文章用的语言,所谓``白话'',就是平时说话用的语言。与``五四''时期的文言文与白话文之争
类似,今天我们也可以看到两种文章:用``规范的书面语言''写的,和用口头语言写的。这就是新的文言文和新的
白话文。我们从小学习写作时,就是使用的新文言文,在报纸杂志上、甚至在电视新闻里看到的,也是这新文言文。

我对文言文本身并没有意见,相反,我热爱司马迁。在我看来,书写和口语本来就应该是不同的,书面语言应
该比口语精练优美、张弛有节,对于汉语这门特殊的语言来说,最好还能富有形象上的美感。典型的例子就是诗。
无论诗如何地脱离口语,我们都只能把她供奉在语言神殿的中央。然而,在很多情况下,使用文言文无疑会给人一
种``不说人话''的感觉。新文言文在初生时也曾朝气蓬勃,但它随后承载了太多其他东西,在短短的几十年里,就
迅速僵化成新的八股文。这使它更加面目可憎。

文言文通行于公共场合,白话文繁衍于私人空间,这本书是关于公共事务的政治题材,本应使用新文言文来写。
但新文言文是一种斩钉截铁的文体,它的口气毋庸置疑,它的立场褒贬分明,这使我无法客观地叙述。新文言文又
是一种基本僵化了的语言,我在这本书里要写的是自己的亲身经历,我希望能够在书中和读者对坐着聊天,而不是
让读者在心里暗骂我装腔作势。

幸好,在我们周围,除了无处不在的新文言文外,我总还能找到其他语言来源。首先是我的家乡话,这是一样
已经存在发展了几千年的语言,虽然不可避免地已经加入了许多新词汇,但那基本上是些``专有词'',语言本身的
质地并没有遭到污染。其次是古典小说,尤其是《水浒传》那无与伦比的白描,使我从此不习惯感情过于强烈的文
字。然后是港台小说,主要是金庸的武侠小说,让我看到了书面语言的其他可能。

从1995年开始,我接触到了一种全新的书写方式:网络语言。这是一种完全没有束缚、完全口语化的语言。更
重要的是,在网络上,``话语霸权''被打破,能够发表的不再只有文言文,无数和我一样的人在互联网上撒开了脚
丫狂奔。正是在网络上,我第一次读到了王小波,也第一次听说了``语言狂欢''这个词。我必须感谢这些语言来
源,让我可以使用彻底的白话文来写这本书。

除了语言问题外,要最大程度地记录真实,还有一个叙述的问题。显然,我无法使用任何叙述技巧,也不能采
取任何预定立场。我只能做一个摄像机,把自己所看到听到的,尽量如实地录下来,转告给读者。我不能做一把剪
刀,把所拍摄到的材料,按自己的意图重新剪辑;也不能做电脑特技,为了达到``美''的效果,就把画面随意加工
或再创造。

所以,我采用几近于流水帐的方式来写这本书,宁可使内容显得有些乏味,也不愿把素材精心组织成具有强大
感染力的故事,因为我相信,对材料的取舍侧重,已是对事实的伤害。平铺直叙虽然单调,却是传递真实的最好办
法,比如余华在《许三观卖血记》里,采用一种纯白描乃至沙化的语言风格,让读者完全失去了《在细雨中呼喊》
的阅读快感,但那种不动声色的叙述却更直指人心。

我的另一个原则是,尽量使用中性的词汇来叙述,不论是在讲我支持的一方,还是我反对的一方。我相信纪实
作品没有为读者做结论的权力,我所能做的只是陈述事实,结论还是要读者自己来下。当然,我不可避免地会经常
提到自己的观点,那我总会加上``我认为''之类的按语,强调这只是我的个人观点,并不一定就正确,更不敢把它
们塞给读者。

也许会有一些读者不太习惯,因为从我的叙述,抵达一些现成的结论,有时只有一步之遥。对此,我愿意引用
王小波在《黄金时代》后记里的一段话:``我知道,有很多理智健全、能够辨别善恶的人需要读小说,本书就是为
他们而写。至于浑浑噩噩、善恶不明的人需要读点什么,我还没有考虑过。不管怎么说,我认为咱们国家里前一类
读者够多了,可以有一种正经文学了;若说我们国家的全体成年人均处于天真未凿、善恶莫辨的状态,需要无时无
刻的说教,这是我绝不敢相信的。''

于是,您打开本书所看到的,就是用大白话式的语言、平淡无奇的口气,所记下的一本关于美国草根政治的流
水账。对于本书的乏味,我非常抱歉。不过,我所看到的美国草根政治也原本就是如此:质朴、自然、零乱、有时
还自相矛盾。草根政治是人们争取自己利益的活动,它既不浪漫,也不丑恶,而是我们生活中平常的一部分。相对
于天下大势、高层分析这类主题,草根政治确实不够宏大明亮、激动人心,但也应当得到中国读者的关注。


好在我尚可自慰的是,我比较喜欢开玩笑,包括把文言文里那些早有特定含义的说法拿出来,往美国国情上套。
这大概可以给这本书稍微增加一点趣味。从另一方面说,我也确实喜欢我参与的这些活动,用美国人的话说就是
``have fun(好玩)'',因此总是会带着``好玩''的心态来记载;我由衷希望读者在看完后,也能有此同感。

在我联系出版社屡屡受挫的时候,世纪沙龙网站的veron版主把我推荐给图书策划人莫之许。我和老莫以及他工
作室的编辑舞雩合作得非常愉快。我要感谢veron、老莫、舞雩,以及其他为这本书的出版付出劳动的人们。

我还要感谢林达先生慷慨挥笔为本书写序。他的《近距离看美国》系列对我影响很大,能有这样一位令我尊敬
的作者写序,是我的荣幸。本书主要是采用个人视角,林达先生的序文则进行宏观的分析,是对本书很好的补充。

\end{document}