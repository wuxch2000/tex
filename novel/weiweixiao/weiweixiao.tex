\documentclass[11pt,a4paper,onecolumn]{article}
\ProvidesPackage{config}

\usepackage{fontspec,xunicode,xltxtra}

\setmainfont[Mapping=tex-text,Ligatures=Common]{Adobe Garamond Pro}
\setsansfont[Mapping=tex-text,Numbers=Uppercase]{Myriad Pro}
\setmonofont[Mapping=tex-text]{Courier New}

\usepackage{xeCJK}
% \setCJKmainfont[ItalicFont={Adobe Kaiti Std}]{Adobe Song Std}
\setCJKmainfont[ItalicFont={Adobe Kaiti Std}]{Adobe Heiti Std}
% \setCJKmainfont[ItalicFont={Adobe Kaiti Std}]{Adobe Kaiti Std}
\setCJKsansfont{Adobe Heiti Std}
% \setCJKsansfont{Microsoft YaHei}
\setCJKmonofont{Adobe Heiti Std}
\punctstyle{banjiao}

\usepackage[table]{xcolor}

%生成PDF的链接
\usepackage{hyperref}
\hypersetup{
    % bookmarks=true,         % show bookmarks bar?
    bookmarksopen=true,
    pdfpagemode=UseNone,    % options: UseNode, UseThumbs, UseOutlines, FullScreen
    pdfstartview=FitB,
    pdfborder=1,
    pdfhighlight=/P,
    pdfauthor={wuxch},
    unicode=true,           % non-Latin characters in Acrobat’s bookmarks
    colorlinks,             % false: boxed links; true: colored links
    linkcolor=blue,         % color of internal links
    citecolor=blue,        % color of links to bibliography
    filecolor=magenta,      % color of file links
    urlcolor=cyan           % color of external links
}



% 表格
\usepackage{booktabs}

\usepackage{caption}
\usepackage{fancyhdr}
\usepackage{graphicx}


\usepackage{geometry}
\geometry{a4paper}
\geometry{hmargin=.1cm}
\geometry{vmargin=1cm}


\title{``微微笑''的记录}
\author{}
\date{}
\begin{document}

\rhead{\scriptsize{\textsf{``微微笑''的记录}}} 
\pagestyle{fancy}

\textit{看到那么多的朋友``不辞劳苦''将自己的经历和经验写下来告诉大家,同时又看到还有许许多多的后来者
  迫切地希望看到更多的信息,终于决定克服懒惰,将自己这两年的一些心得也记录下来,和大家分享,但愿能给
  有需要的朋友一点小小的帮助,同时也为自己做一个小小的总结。}

\textit{粗略想了几个题目,可能最后成文时会有些出入。另外还想声明的一点是,写此文的目的最主要还是想给
  有需要的朋友提供一点点参考,写一些我们的过往经历,无所谓好坏对错。既不想出名也不为别的,所以对于善
  意的建议表示欢迎,与写作初衷无关的提问恕不回答。}

\section{准备长登}

打算长登的决心下定后,为了让孩子提前适应环境,我们决定不等到国内的小学放假,就于2005年的6月中旬(孩子
当时差几个月满8岁)来到了加拿大,因此女儿在国内的学历应为2年级``肄业''(未参加期末考试)。

先说说女儿毛豆在国内的情况。

她是属于语言能力比较强的那种类型,说话很早,也很清晰。举个例子,幼儿园小班刚入园老师就发现她虽然平时
话不多,但却挺会说,没事时爱逗她说话。一日无事,一老师问她,``老师漂不漂亮?'',答曰``漂亮'',这当然说
明不了问题了,关键在于后面又补充了一句``象白雪公主一样漂亮'',老师闻言大喜。另一老师心有不甘,也问同
样问题,回答同样也是``漂亮'',这次补充的是``象孙悟空一样漂亮'',此老师的表情需要我们自己去想象。

当时她在园期间因盛行反对``幼教小学化'',所以幼儿园寓教于乐,没什么课业负担。我们在家正好又有时间便教
她认了好多字,所以在上小学前,她已经可以独立阅读。出国前已经达到可以阅读全文字大部头的格林,安徒生的
全译本以及冰心什么的。出国后这两年的中文教育以后再叙。

她在国内上学期间成绩也还行(小学嘛,成绩说明不了什么问题),临行前参加了全市的数学邀请赛拿了个一等奖,
还玩着参加了一个普通话大赛,也是两名一等奖之一。这么小的年纪,这些奖说白了没有什么含金量,只是带孩子
练练胆量,长长见识,增加点自信心而已。

在国内时曾在一个热心的大学教授办的``国学''班里背诵着学了半部《论语》(不知将来治天下能否用得上),当时
还挺有兴趣的,不过现在基本忘了;学了一年的短式网球(儿童拍,硬海绵球,小场地),正拍的基本功比较扎实;
小时候学了几年儿童舞,兴趣不大,停了;钢琴因为重视不够,学了几年,眼见周围几个关系近的同学都比自己弹
得好(有后来去了上音附小的,有两个妈妈本身就是教钢琴的等等),把个信心弄没了,加之练琴枯燥不情愿,出国
后,说啥也不学钢琴了,说要改小提琴,后来又是小提钢琴一起学这也留到后面再说;决定出国后临行前我们娘倆
分别找教练学游泳,我以三十几岁的高龄,基本学会了蛙泳但不敢进深水,孩子怕水,也只学会了蛙泳的基本姿
势,后来发现这边的游泳都是从自由泳开始,于是06年暑假回国狂补了一阵子,也随后专题介绍吧。

至於最关键的英语,在国内时并没有补习过(怕出力不讨好),只是随学校里学习,后来据她回忆,说国内学的在这
儿基本用不上。我看过她的英文课本,比我们上学时已经是进步了不知多少倍了,可算是灵活多变且感觉上还算与
国际接轨,真正走出来才发现还是差距很大。

决定常住前,老公于4月一人再次短登,想做些铺垫。在一次闲逛时发现路边好多人排队,他好奇心起,问过后知道
是在买房子。嗅觉敏感的他立码决定应该马上买房,我们只是通过电话沟通就买了房,这在后来看是作出了一个非
常伟大的决定,一是房价飞涨,二是避免了因为租房搬家而带来的种种麻烦,算是比较幸运的。我们6月长登先住了
2天宾馆就搬到了自己的新家,当时卖30几万的townhouse,2年后时至今日仅政府估价就涨了10多万。

\section{暑假生活,为开学积极做准备}

即来之,则安之。温哥华的夏季可以说是一年当中最黄金的季节,天气不热(实际上是晚间有时会觉得冷,车上必备
一件长袖衣服),天儿又长,往往到晚上10点多天才完全黑。当时恰巧老公的两个好朋友都待业,轮流陪我们安家置
业,所以在我们登陆后买车前的20多天里,多亏了他们,我们甚至竟然没有过一次走路买菜的经历,也没坐过公交
车。

这边的学校一般6月底之前放暑假,到9月的第一个星期二(Labour Day后)开学。我们备好了文件,先去辖区的小学
报到后,通知我们去教育局做ESL的测试,测试结果当然是1级。我个人觉得这边的测试和国内的考试不同,应该照
自己平时的水平去答题,让考试结果真实地反应考试人的水平,千万别为了面上好看提前准备硬给自己拔高,因为
这个评定结果将会直接影响下一步的学习,高于自己的水平或低于自己的水平都没什么好处,有时反而有害呢。

之所以提前来,还有一个想法是去学校要一份``暑假作业''在家温习温习,到了之后才发现哪儿有什么暑假作业,
这想法也太有中国特色了。作业要不成,我们也不能闲着,先是分开给她报了2周的社区夏令营,全天的
9:00--3:00,05年的价格好像是每周84刀。我的感觉,那些``小老师''(每班3-5个)可能是在作义工的中学生,每
班学生有二三十个,每天的活动有室内也有室外,还有远足,游泳一类的。参加这样的活动主要目的是让她感受一
下氛围,了解一些风土人情,以及当地小朋友间如何沟通交流。反正全是玩,不存在功课的压力,即便什么也听不
懂,也无所谓。事实上刚开始几天几乎就是什么也听不懂。记得头一天我去接她,老师在通知明天去远足,应该带
什么东西,我事先已在门口的小黑板上看到了写给家长的通知,所以心中早已有数。再看看女儿,正坐在小凳子上
美滋滋地享受着老师刚发给她的冰棍呢,很明显老师的话进不了她的耳朵,不是她不在乎,而是听不懂,不过看样
子还算放松,没有太多的焦虑紧张,我悬着的心总算放下了一半。在我看来,这,就是成功的一半了!

后来还有一次,我去接她时,她正在playground上玩,老师大声喊她的英文名字,叫了好几遍,她一点反应没有。
人,应该对自己的名字极其敏感才对,可是叫了八九年的名字一下子换了,是需要适应好长一阵子的。看着这一
幕,我有一点点心酸,没跟她说。

但是孩子的进步也的确是超出我们的想象,2次夏令营下来,到最后,她不光能基本明白老师的指令,老师布置第二
天该带什么,她也能听个八九不离十了。

不过这之后一次和朋友聊天谈起她的前后进步,先举例``叫名''这件事情时,她不让我说,并难过地哭了。过后我
忙向她解释妈妈是为了对比着表扬她的进步,绝没有一丁点在外人面前笑话她的意思。由此可见,当时孩子虽然口
上不说,心里其实也是蛮着急的。

我们还找了一个私人小补习社,上了四堂课就停了,效果不好。为了听不懂老师布置作业的事情也哭过一次。

后来还报了一个教育局办的补习班,考虑到她的承受能力,我们选择了个最低级的班,叫``ABC'',实际上是为当地
孩子上小学一年级做准备的。我们抱着让孩子熟悉课堂,听明白老师指令的目的。内容虽浅,但孩子挺开心的,把
26个字母画出来,做了整整一个大本子,并积累一些以26个字母开头的单词。同学都比她小很多,所以她总是最先
完成。我们最关心的不是她学到了什么,而是上课能不能听明白老师的要求。

还有一个很重要的就是泡图书馆。第一,去图书馆听讲故事或参加各种活动(无论主馆分馆,只要时间排得开都去,
关键信息要畅通),无论给多大小孩讲的我们都参加。唱歌跳舞做游戏,她实在不好意思跟那么小的小孩一起做我也
不强求,只要她肯去认真地听,就可以。第二,从ESL的一级开始借书,通常10几20几本,天天大声朗读3--5本(根
据心情情绪),她读我也读。1级的刚开始时她几乎本本都有生词呢。可是一旦坚持了,提高飞速,2级到3级,之后
就开始阅读picture book,再后来就是easy chapter book(当然这是很久以后的事了)。第三,音像图书,最好是既
有书又有碟的那种,看听结合着搞,效果非常好。可以借故事的,歌的,我们曾去温哥华图书馆的儿童图书馆借过
一套很好的儿童ESL教程,既有学生用书还有教师用书及配套练习,一套好几个级别,我们好像坚持学到4级。里面
有对话有唱歌,有几首她爱唱的歌现在还记忆犹新呢。那套教材的名字记不清了,好像叫``Let's Talk''。第四,
非常幸运的是,开学后,我们报名参加了一个Reading Buddy的活动,每周一次45分钟,由中学志愿者给1--3年级的
小同学读书听。

夏季各社区的免费活动特别多,或大或小几乎每周都有,带孩子多参加,家长也要以积极勇敢的态度去迎接新生
活,面对新挑战,既增加学说英语的机会,又能多了解一些当地的文化,还能让自己的生活丰富多彩,更让孩子感
受到家长的心态,一举数得。

还有个小建议,如果孩子来适应了一段时间后,千万不要以孩子语言还没过关为借口拒绝给孩子报名参加那些特长
学习班。其实那些班大多是社区办的,收费不贵,学起来很轻松,老师大多很有耐心,听不懂也可以做示范,真正
是语言技能双提高。

看电视更是必修课之一,无论什么节目,只要爱看就行,哪怕是低幼节目。前几个月她每天坚持看一档幼儿园级的
节目Dora,等到自己的``水平''高了,马上就喜新厌旧了。这里还有个好处,大多电视上的动画,在图书馆有一个
``喜爱的朋友''专柜,都能借到配套的图画书。

这期间去美国10余年的小叔子来这里看望了我们,我们一起去Whistler小住了几日。他在的日子,天天都是学习了
解的过程,包括逛街购物都有很多问题要问。

准备工作差不多了,万事俱备,就等开学了。

\section{开学篇及第一学年之一}

待要开写时才发现这个题目有点大,还是以英语学习为主试着挂一漏万吧。

终于等到开学的那一天了(开学没多久教师工会就闹罢工可真把我们急坏了,这里就不多说了)!

第一天,新转来的同学不分年级单独在一个教室由家长陪同,开了个小会,大致给介绍了学校的情况。我们的学校
是个在内街的250人左右的小学校(跟国内一个年级的学生人数差不多),很安静。后来发现小学校的好处是,如果校
长负责任,学校就会象一个大家庭,师生之间,家长之间,家长和学校之间会有良好的互动;更有条件多组织一些
field trip,孩子展示自己的机会也较多等。

第二天把这些新同学分别编到适龄的班级,加拿大是按自然年来分级的,凡是1997年出生的2005年就是3年级。他们
每年都会根据学生老师的人数重新分班,所以这只是暂时的,到周五会通知学生到底在那个班。好多学校为了节省
资源,通常采用混班制,即不同年级的学生编在一个班。有兴趣的家长可以参考教育局的相关网站,查阅这种分班
的好处。

虽然不同年级的学生被分在一个班,但他们很多课是打乱班级来上的,象法语,数学等都是相同年级不同班级的学
生一起上课。这里老师是按教室来分配的,即一个教室1--2个老师,教室也是他们的办公室,该上哪个老师的课,
就去哪个教室,老师不动,学生动,有一点点象国内读大学的时候。按统一规定低年级每班不能超过21人(3年级以
下),高年级(4--7年级)每班不超过29人。毛豆的学校加学前班共11个班(division)。

加拿大地大啊,多数学校是一层楼,但占地面积很大,每个教室都是两个门,一个冲走廊,一个对操场(一旦发生危
险十几秒钟就能逃生,更不可能发生踩踏)。通常小学的操场是块很大的绿地(后来发现人口是加10倍的美国学校操
场的确没这么大)。每个学校还有个室内的体育馆,体育课,开会,演出,展览都在这里举行。开学第一天,毛豆下
课活动时差点迷了路,真够她紧张的。

这里,我们还遇到了几件幸运的事。

一是毛豆被分到了一个50岁左右的西人老师的班级,这个老师是建校元老,十分有经验,尤其在处理孩子间的小纠
纷以及道德方面的培养上很见功力。老师童心未泯,万圣节还把自己化装成米老鼠。

二是这个学校的新移民尤其是讲普通话的新移民较少,华人中讲粤语的居多,无形中给孩子断了在学校讲母语的后
路,从进校那天起到现在,她几乎没在学校讲过中文,这为她口语的快速提高提供了非常有利的外在条件。反观我
们一个在西区的朋友,因学校讲国语的较多,虽和我们前后脚登陆,一段时间后两个孩子的口语相差极大(当然他们
的是男孩,可能开窍晚一些)。

三是因我们住的是新小区,自然会有很多新邻居,巧的是隔几个门的家里也搬来一个小女孩正好和毛豆分在了一个
班。

这个女孩,在毛豆加国学习的第一年还是起了非常关键的作用的,这里边也有不少欢笑和眼泪。首先,让她们两个
人在开学的第一天,都交到了新朋友,彼此都不孤单。这个女孩子家是广东人,会讲普通话,6岁来加,已经不需要
读ESL了(通常4级以后就不需要单独拉出来pull out上ESL课了),也应是语言能力强的那种类型,她妈妈说她的发音
很标准。

女孩每天要练至少3小时的钢琴,爸爸妈妈也在家给她补习一些中文数学(后来知道是为了回流,但据说补习的效果
很不好毕竟没环境也没基础),所以孩子们回到家时反倒很少有时间在一起玩。在学校刚开学那一段时间,毛豆对她
相当依赖,实在听不懂了,就问问她,包括上课老师布置任务。后来我听说老师给她们调换了座位,经我观察思
考,想是她的家长找老师所致,这些没所谓,我们不能要求别人什么。别人的孩子是没有义务帮助我们的,我也从
未对她的家长和女孩开过这个口,但只要有机会见面便一再对她

的帮助表示感谢。

毛豆属于特善良的那种类型,任何时候也不爱伤害别人,也没什么心计和嫉妒心。几个月后的一天,她跟我说起这
个女孩时,很伤心地哭了(加上前面暑假那两次,因为不适应毛豆哭的就这几次,应该不算多)。原来是这个女孩不
止一次地向他人说毛豆的``坏话''(具体内容不清楚,可能是什么也不懂之类的吧),令她非常不舒服。开始我以为
毛豆多心,但问过后发现并非如此。其一,一次她和一个黄头发的女孩再说类似的话时,那个黄发女孩(转过年刚开
学时尽管不在一个班了,毛豆还是作为为数不多的几个人之一获邀去黄发女孩家参加了生日party)很有正义感地当
场大声告诉她``我不认为毛豆是那样子的'',可见不是好话;其二,有一次女孩因故没来上课,另外一个同学和毛
豆说,``我挺喜欢她不来的,你呢?''了解清楚后,我们决定不能视而不见,于是马上约了女孩的父母来家里,很坦
诚地谈了我们的看法,而且一再告诉他们回去后千万不要过于严厉地批评孩子。在我们大人谈话时,两个孩子在楼
上玩,仍是好朋友。虽说他们父母当时并没为此事道歉,还说他们的孩子其实很单纯,但之后类似的事情没有再发
生。

女孩仗着语言优势处处都想占毛豆上风,脾气还挺大,课间活动时必须按她的意思来。毛豆好脾气,仅仅回来说几
句,倒也不怎么在乎,毕竟她也不愿意失去朋友。我这做家长的心里感觉不是太舒服的,在没想出什么好的解决办
法时也不便说什么。事有凑巧,2006年1月份,在诞下小弟弟后(这期间我经常帮助他们接送女孩上下学),她们举家
在3月份回流了。毛豆颇难过了一阵子(我心里倒是暗暗松了口气),不过马上就找到了新朋友,那时她的语言在玩的
时候已经没太多障碍了。

现在回想起来,刚开始时,毛豆从没拒绝过去学校,也没感觉到她有很多的压力,每天都高高兴兴地背着只装个饭
盒的大书包去上学,真有点象朋友说的,``你愁什么,孩子来了光等着enjoy就行了''(这可能是大多数年龄小的孩
子刚来时的心态,打算来常住的家长们大可不必满面愁云)。我们也真的从心里感激那个女孩,让毛豆很快适应了学
校的新生活。

经验分享:

经验一,家长在这一阶段不能心急,更不能上火,要相信孩子解决问题和面对困难的能力。

经验二,在家或课外补习要有度,不能给孩子太大压力,要多鼓励,让他们尝到进步的喜悦。

经验三,家长一定一定要细心,留意孩子情绪的点滴变化,并对症下药,心理疏导比学业辅导要重要得多得多。

经验四,如果有条件,在开始阶段尽量选择一个家长工作,另一家长全职。这样的好处多得实在写不过来,如果家
长这时只顾着

生计或自己的提高而忽略了孩子,那可是得不偿失,跟我们大多数有孩子的家庭来加拿大的根本目的背道而驰了就。

经验五,课外兴趣班不要一上来就全面开花,应该循序渐进,根据孩子的适应情况慢慢展开。

\section{第一学年之二}


孩子年龄小来的好处显而易见,没有顾虑敢于开口,管它什么时态人称对不对,一律说之。很快,从几个单词地往
外蹦,以别人基本明白自己的意思就行的阶段就过度到整个句子的阶段了。记得06年春假我们去美国Disneyland,
当时为了避开高峰,我们选择先开车去西雅图,再乘飞机。结果在西雅图机场附近存车时,人家问了我们好几遍取
车还是存车,我们两个大人都没弄明白他们那又快又短又听不清楚的句子啥意思,还是毛豆在旁帮我们解了围。真
是不服不行。


虽然一个年级就那么几个人,但英文数学等科目还是根据程度分了好几个等级来授课。毛豆的数学因为国内的底子
肯定在最高级,而且优势很快显现出来并被老师同学注意到。有一次测验,班里只有一个满分,老师说出她的名字
后,马上一个同学接话道``我早就知道是她'',类似的事情让毛豆在异国他乡的第一年就慢慢开始恢复了自信。


这边除了发达的社区图书馆,各学校都有学校图书馆,每个班还有班图书馆。每年都有个叫scholastic的组织在学
校搞book fair,每月也有些订书活动,均按一定的购书比例给学校或班级图书馆赠书。在国内大都习惯买书看,国
外的书则比较贵,图书馆又这么发达自然就以借书为主了。但借书往往有期限,06年暑假回国时,毛豆的班主任就
借给了她10几本书,连借条都没打。按这边的规矩,这些书都是老师用她们的自主经费补充的,有点``半私有''的
性质(个人理解)。


毛豆低年级的ESL课老师也是个颇具经验的当地老教师,人非常nice,她和班主任在毛豆第一年的英语学习上给予了
无数无私的帮助,包括当毛豆的听说读提高后,我提出自己的水平难以在家给她辅导写作时,老师义不容辞地答应
帮她批改,这可是额外的工作啊。在此提醒家长,和老师的沟通非常非常重要。


经过一段时间后,毛豆的英文课被老师从初级班转到了中级班,这个跨越可是真够大的,初级班以单词生字造句为
主,中级班已经在读小说了。上了一两节课,毛豆就喊难,有点吃不消了,想打退堂鼓。这可不行,我决定找老师
谈谈。开始老师误会了我的意思(可能我表达得不够清晰),以为我想让毛豆退回初级,一再告诉我回到初级对她就
是耽误。我告诉老师,我希望老师在班上能多给她一些关注,并希望老师能告诉我在家里如何能帮助她。老师告诉
她别急,慢慢来,有生字不要紧,理解每章的大意就可以了,这个要求显然比较容易达到。


记得转班的当天放学,毛豆告诉我要去书店买本书,原来学校的书不让往家拿(看样子这边不怎么鼓励学生有太多的
课业负担),她觉得自己学起来有困难,想买一本放家里学。她已经记下了书名和作者名,小家伙还蛮有心的嘛。这
个想法马上得到了我的支持。但书店没有,经店员联网查看后发现downtown的书店有卖。我们马上又去了图书馆,
书也被借走了。不过第二天我还是在温哥华的一家图书分馆如愿借到了书。


厚厚的一本字书(chapter book),在此之前我们虽然一直坚持阅读,但大都以picture book为主啊。于是我便提前
查好生字,然后和她一起读。后来叔叔得知此事,也在美国借来此书,忙里偷闲通过MSN帮助她。我发现,别看他去
美国10多年了,因为没孩子,尽管他的经济,技术类词汇量比老美还大,但小孩词汇他可并不十分拿手。有时候,
毛豆问,这个什么意思时,他也要查一下呢。


就这样,``初升中''的问题圆满地解决了。我们也通过此事教育毛豆,以后遇到困难不要轻易往后退,而要想尽办
法去努力争取。

孩子的进步再次让我们感到欣慰,她因为有了预习,居然开始大胆地在班上举手发言了。更因为有时她提前知道了
结局,令实习老师故意不叫她,这种有默契的不叫也让毛豆有了些小小的自豪。

就这样我们的另一个顾虑也开始慢慢消除了,孩子上课能够认真听讲,已经开始发言,而且慢慢恢复了自信。我们
在家里所做的一切其实就是帮助她能快一点跟上,让孩子少一些挫败感。


这本书之后,我们的参与也慢慢淡出,几乎都是她自己的事情了。家长在开始阶段的帮助和鼓励很重要,不要让孩
子感到孤立无援,请和孩子一起共度难关,你会因此发现,你收获的不仅是浓浓的亲情,更有不知不觉中英语的提
高呢。英语上实在帮不了多少忙的家长也不要气馁,可以多带孩子和别的小朋友玩玩,尽可能多和当地的小朋友接
触,帮助孩子多交新朋友。


3年级开始有了一些可选择的家庭作业,以英语和数学为主。老师专门写了家长信,说这些作业可做可不做,可多做
可少做,可难可易,每天做或每周做几张都行。毛豆数学专挑难的,英语专挑简单的(开始时也需要我稍微辅导一
下)。老师有个大盒子,里面盛满了过塑的题卡,写着不同的编号,通常题的背面有答案。每做完一张都在本子前面
都做上记号,做完10张批一次,做完50张有个小奖励,100张有大奖励。毛豆贪心啊,于是成了那学期结束时班上3
个做完百张的学生之一,从老师的百宝箱里挑了自己心仪的玩具。


这里每年3个学期,9月开学到圣诞节是第一学期,放假2周,之后到3月放春假一周是第二学期,再到6月底是第三学
期。小学期间没什么期中期末考试一类的,基本看平时的表现及测验以及上课回答问题参与讨论的表现给出评语。
令我感慨不已的是每当6月份临近学年结束时,当国内的孩子们正为准备期末考试而没白没黑地整天在题海里泡着的
时候,这边的孩子们却在蓝天白云绿草地上无忧无虑地享受着大自然的恩赐,天天不是看电影就是音乐会,郊游游
泳一类的。老师说了,整整一学年下来,孩子们很努力也很辛苦,他们值得这样去享受一下。可跟国内的孩子比起
来,我怎么觉得他们每天都在玩呢?!


加拿大因是移民国家,有相关政策会根据新移民的英语程度给予一定时间的免费帮助,大人小孩都有。通常孩子给
5年时间免费上到5级。按我周围过来人的经验,只要家里能帮上的,通常用2年左右就基本可以过语言关。家里帮不
了的孩子如果自己还算爱学的,每年升1级也基本够用。可是周围也见过不少孩子,5年后仍达不到要求的,也只好
听之任之了。


第一学年结束后,ESL老师通知她,开学后由1级直接升入3级,班里只有她1个人是这样。老师对她的评价是
amazing一类的,老外们总是这么爱夸张的。记得有一次我看花样滑冰比赛,中国的运动员摔倒后仍强忍剧痛完成动
作,外电用amazing一类的词评价原本是非常普通的惯用词,可国内一翻译感觉就完全变了味道,成了非常了不起的
事情了。


另外想告诉大家的是我的英语水平也就是指着在国内上大学本科非英语专业的那点底子,工作后除了考职称基本没
接触过,本人也没参加过雅思培训(毛豆爸爸考的)。但我曾经在若干年前,跟着电视很认真系统地学习了《走遍美
国》,所以现在听力是听说读写中相对最好的一项,相信很多的爸爸妈妈英语水平都比我高。我个人认为,孩子小
学阶段来,象我这样水平的家长只要用点心是能给孩子帮上忙的。安顿好孩子之余,我也去参加了英语学习,现在
已读完成人高中的English 11,后文专叙。

小插曲:毛豆的中西合璧式生日party


来加2年期间,毛豆参加过不少回同学朋友的生日party,从开始的不了解,新鲜,兴奋,到最后自己也想办一个。

为此她也磨过我几次,我都没答应,一是她出生在8月,每年我们都要回国探亲,国内也没这个氛围和条件呀;二是
我一直要求毛豆生活上要向低标准看齐,凡是没有十分必要的要求,通常不会痛快答应。一直以来,我坚持不带毛
豆逛商场(超市买食品除外,那是毛豆爸拉拢女儿的拿手鐗),尤其是买衣服,宁可不合适我再跑一趟调换。原因是
我不想女儿过小就开始思考穿衣打扮一类的事情而分心。这一点上毛豆至今都很``配合'',从不挑衣服,买什么穿
什么,除了感觉不舒服的再好看也坚决不穿,其他是给啥穿啥。更有甚者,我让她试穿时,她都不去镜子前臭美一
下。所以没有顾虑的毛豆妈为了满足自己的虚荣心,可愿意给女儿买漂亮衣服了。

今年6月回国前的一次购物,无意中逛到生日party用品处时,她再次提出此事,并说这是个比较特别的生日,因为
她已经进入两位数(10岁)了。我略一思考,觉得理由还算充分,便马上答应了她的请求,把个孩子乐得呀。于是我
们买了一堆party用品,很多东西国内没得买。真是party尚遥遥,心情却早已乐陶陶。

回国后为了选址,我转了不少地方,怎么也没找到可心的适合10岁左右孩子玩的室内场所。生日一天天临近了,外
地的好朋友也为此赶来了,无奈中我只好很有中国特色地把地点定在了饭店(也是经过考察,房间比较大,带独立卫
生间,吃过饭后可以玩得开)。

接下来按加拿大的程序就进入发请柬的环节,这个在国外的生日party是再普遍不过的,国内的小朋友和家长可都没
见过。小孩子们立码觉得备受重视,以前见过的都是大人才能这样被隆重地对待啊,而且大都是参加婚宴。共邀请
了8位小朋友,全部如约前来,他们都觉得新鲜,包括家长们。

我和毛豆还有她表哥提前用几个气球简单装点了一下气氛,然后我帮他们点好菜,简单宣布了一下纪律和注意事
项,就到楼下和我的同学(其中几个受邀孩子的妈妈)用餐了,把时间空间完全交给了孩子们(当然有服务员在旁关
照)。

孩子们最小的8岁,最大的12岁,3男6女,有毛豆过去的同学,还有我的和毛豆爸老同学的孩子等,除了毛豆认识所
有的人,他们相互之间并不完全熟悉。但是,事情进行得很顺利,孩子们边吃边聊,很快就熟络了。偶尔妈妈代表
也会进去抻一头,了解一下情况,但很快就被冠以``不受欢迎''给婉拒出来。

饭吃得很快。爱吃的基本底朝天,不爱吃的基本未动过,怕饿着这帮兴奋的蠢蠢欲动要下饭桌的的孩子,让每人又
吃了一个水煎包。饭后,我和另一个妈妈加入服务行列。

这之后,他们就开始玩了,有三两个一起下棋的,顶气球的,传着打气球的,后来撒了气又改为玩水球的,击鼓传
花吃薯片等等。还参照加拿大这边的风俗,玩碎了一个spongebob的pinata(打碎一个装着小礼物和糖果的空盒子,
抢到东西归自己)。玩的过程中,规则大都是临时制定,即兴修改,显然不是什么传统项目,全是现场发挥。有时全
体参加,有时多人参与,有时三三两两,偶有争执,但没有冲突,很快就达成一致了,根本不需要裁判。可怜啊,
这帮独生子女们,有了伴玩得这个开心啊!

我控制着时间,中间让他们一起分享了哈根达斯的冰激凌蛋糕,孩子们吃前用双语唱了生日歌。

时间过得很快,家长们按约定的时间纷纷来接孩子,毛豆将准备好的goody bag(内装小糖果和小礼物)分给每一个小
伙伴,这也是个国内没有的习惯,于是大家happy!孩子们眼尖得很,早就对盛着小礼物的那个大袋子惦记上了,活
动还没结束时毛豆也急着想马上分给大家。我提醒她,没看婚礼都是宾客离开时才发喜糖吗,她才作罢,哈哈!

第二天,我电话回访了家长们,孩子们的反应异口同声``玩得太高兴了!''毛豆亦然,10岁喽!


从这次开始,我有了个新主意(看孔庆东博客来的灵感),既然题目是分享,就该人人有份参与,不能总是听我一家
之言,而应百花齐放。

本次分享话题:

1,你为孩子开过party吗,有什么特别之处可以分享?

2,你如何看待为孩子开party?同意或不同意请阐述理由。

随便聊,各抒己见,仁者智者请随便,1个2个题目也随意,不过坚决反对彼此不同意见者互相攻击,谢谢分享!

\section{第二学年----四年级很关键}


四年级在这边是个分水岭,学习的内容和要求都与低年级有较大的不同。report card也开始有了ABC的评定等级,
不象低年级只是达标与否。


先说毛豆直升到ESL的3级后,刚开始也是嚷嚷难,不过这次很快就适应了,没有再说退出的话。而且很快每次测验
成绩都是数一数二的(另一个和她不相上下的孩子是个香港人,从小在南非上的英语学校)。高年级的ESL课时比低年
级的多,每周四次pull out,他们上此课的时候,其他同学在班上上英语课,只不过大家读的书难易程度不同。


这边的作业少,孩子们的课外阅读量相对比国内大。这时毛豆的英语水平已经提高的很快。当有时ESL老师不在时,
她在自己班听那些同学英文课读的那些书时,感觉有兴趣的也借来自己在家读了,不过她在家就只能做到泛读,很
多生字囫囵吞枣过去了,我发现大人阅读遇到生词时的感觉也和小孩不太一样,大人更容易被卡住,搞不明白就读
不下去似的。


这时去图书馆更多是她自己挑选喜欢的书,我挑的有很多她不怎么爱读了。我抽空也翻翻她爱读的那些书,了解她
到底读了些什么,做到心中有数,以便加以引导。有时候,我自己学习累了,读读她的书也是很好的放松休息呢。


我喜欢选一些获奖的作品让她读,不知为什么多数时候她不太感冒。记得有一本叫《爱德华的奇妙之旅》,我在当
当网上看到中文译本的评价很好,便借来原版,我和她都看了,我觉得挺好,她看完后却说还行。 不过偶尔也有些
获奖书是她喜欢的,我还是想不太明白,但也没怎么逼她,好像并不是因为难易程度不同。她就那么自己天天读
着,我则苦于一直还没想出一个让她在家阅读时,能再多积累一些词汇的有效办法,得是她能接受的,显然和我们
大人的方法是不同的(不知各位有什么高见可以分享一下)。


曾经看到过一个关于阅读难度的``巴掌理论'',觉得蛮有道理的。说拿一本书让孩子读,每遇到一个生字就伸出一
个手指,如果读完一页,生字没超过一个巴掌(5个手指)便算合适,否则就是太难了。


最后一学期,业余时间曾给她报了个教育局办的4-5年级的写作辅导班,无非是阅读写文章增加点词汇量,后来看效
果也比较一般,她在里面算是好的,所以感觉提高不大。写作很重要,以后还要另想办法加强。


四年级的科学课比较难,学天气,学人体结构,那些词看得我都直眼晕。毛豆也遇到了挺大的困难,第一学期成绩
很不理想。我们后来做了分析总结,调整了学习方法,笨鸟先飞,终于在第三学期拿到了A。A虽然并不能代表什
么,在当地很多同学看来也颇不费力气,可是我知道孩子拿这个A是挺不容易的,自己是付出了不少努力的,她自己
也非常开心。希望这个A能让她继续努力,不轻言放弃。


在这里插一句,Science World关于人体结构今年设了专项展,有要上四年级的家长可多带孩子去看看。我大致了解
了一些人,这边虽没什么课本,但还是有教学大纲的,各学校学习的内容大差不差,基本相同。这些科学场馆只要
孩子喜欢还是办年卡划算,基本3张门票钱就等于1张年卡。


新知识的学习中我还发现了另一个问题,当一个新词汇出现时,如果这个词是第一次接触,她从一开始起就只记住
了英文说法,并不知道中文如何表述。记得有一次问她当天学了什么,她说了一个词儿octagon,我不知道什么意
思,问她,她说``我也不知道什么意思,要不然给你画一画吧。''这还是3年级刚来不久的事呢,当时我就有点晕,
难道我们倆以后说话经常要通过画画不成?为了与孩子始终保持良好的沟通,做家长的也要坚持不懈地学习学习再学
习啊!


作为一个ESL生,她的成绩单上,英文科是没有成绩的,只用个*号代替。第三学期结束的时候,其他的科目已经基
本是A偶尔个别是B(体育)了,小毛豆自己也很想早日成为全A生呢。我们并没有给孩子施加这方面的压力,只是在旁
关注着她的成绩和不足,孩子挺不容易的,这毕竟不是自己的地儿啊。其实第一学期之所以没拿很多A,通过评语
看,她最主要还是因为语言能力尚待提高,另一方面是上课发言不够积极,参与得太少(说白了还是语言存在障碍,
毛豆以前在国内可是很愿意回答问题的)。


有了第一年的铺垫,第二年过得比较轻松。学期结束后,老师通知她,她的听说读写基本不是Level-4就是
Level-5,下学期可以不用单独出来上ESL课了,可以和同班同学一起在班上上英文课了,不过对他们这样的学生在
上课时仍会有ESL老师在班上给予一些特别的关注和帮助。这次还是就她一个人升级,好像是根据学期末教育局统一
组织的测试成绩定的,她也没弄得很明白。这边虽没什么大考,但小测验也蛮多,基本是学完一样考一考,考好考
坏也没什么大不了的,通常要拿给家长签字。2年,毛豆基本跨越了语言关。这一年,有些新转来的同学,以为她是
当地出生的小孩;还有些人因从没听她讲过中文,竟然以为她是韩国或日本人。她自己,也渐渐地感觉到自己和土
生土长的同学间的差距越来越小了,也更有信心了。我们,也终于可以轻轻地舒一口气了。


我还喜忧参半地发现,她除了和我继续讲中文外(一半是我的要求,一半是她听我的发音感到不舒服),和小朋友则
更喜欢用英文交流。记得有一次她还说,放学回到家可真累,脑子里一会中文一会儿英文。原来是因为我总是问她
学校里的情况,她需要把当时的英文情境转换成中文讲给我听,翻来覆去的的确挺累。


BC省每年5月份要对小学4年级和7年级的学生(低级别的ESL学生不参加)进行考测,叫FSA,主要考阅读写作和数学几
样,成绩第二年才出来,菲沙机构就根据这个整个排名出来供大家参考。毛豆的学校一不小心放了颗卫星,06年的
排名一鸣惊人。很多家长来之前对排名很迷信,我们刚来时也一样。慢慢才发现,这个排名没有太多的参考价值,
多少能反应一些问题,但偶然性极大,除了私立学校排名稳定,公立学校的变化很大。


如果家长想更多更好地了解孩子的在校情况,积极参加学校的义工也是个好办法。义工的种类很多,图书馆,Food
Day,各种活动的司机及活动助理,运动会等等。既能和老师建立联系,也可以和其他家长多沟通交流。有些家长担
心自己的语言不够好,不敢去做义工,其实这是不对的,开始时不妨尝试和语言稍好一点的家长结伴,明白套路后
就会发现,并不象你想象得那么难。做不做义工是个态度问题,与能力无关,顺便也可改变一下大陆华人不怎么爱
做义工的形象。



本次分享话题:

1,你经常做义工吗,除了学校,还有哪些地方?

2,你有帮助孩子提高写作水平的方法可以分享吗?

3,你的孩子经常阅读哪些书,可以分享一下他们的最爱吗?

\section{小学教育之体育篇----体育成绩,田径,运动会,游泳及其他(1)}


毛豆的个子不高,可能属于晚长的一类(但愿如此,爸妈都是北方人,中等个),在国内就一直坐``第一排''。我们
为了让她快点长个儿,从小就特注意体育锻炼。什么滑板车,跳绳,踢球,溜旱冰,拍球什么的,能玩的了的很多
从幼儿园开始就让她尽量早去接触。以至于她小时候,我们小区的一些人竟然认为我是个小学体育老师呢。还有冬
练三九夏练三伏坚持打了一年的网球,目的也无非如此。


虽然她现在个子还是不高,但因为一直重视锻炼,身体素质还是不错的。从小到大,毛豆生病只是吃吃药养几天就
好了(当然干过妇幼保健的姥姥功不可没),从没打过针(预防针除外),更没挂过吊瓶。


三年级那年,因为成绩单没有ABC等级评定,我们只是知道她大冬天的也被要求穿着短裤去GYM里上体育课,放学后
外面批上羽绒服光着两条腿放学(真跟电影上演的老外似的),并不知道她到底都学了些什么。问时,只是说玩了这
个那个的,挺好的。而在KM club的长跑活动中她也是跑进了100km的。


四年级第一学期成绩单一发下来,体育是C,我们有点傻眼了。赶紧找问题,先是从评语中找,发现是``参与性''不
够,再问毛豆,都学了些什么,她是怎么表现的,基本就心中有数了。问题是这样的,他们现阶段的体育课内容就
是初步介绍各种群体运动的入门知识,并做一些初步的练习,不求精但面较广,涉及冰球(陆上曲棍球),篮球,排
球,足球,棒球等运动。据毛豆讲,很多运动难免会有身体的碰撞接触,这时她往往就选择避开躲在一边。


为此事,我也特意去咨询了体育老师,其实是校长兼的。他说的和毛豆讲的基本一致。找到问题了,下一步就是如
何解决了。首先就是要作通她的思想工作,让她意识到问题,并自己制定目标,家长经常提醒督促。由于那些运动
不是我们华人的强项,又是集体项目,她兴趣也不高,平时也没多少机会练,暂时只能如此了。


3年级时,我就发现4--7年级每年春季有多校(一般是五六个学校一起)联合的田径运动会(track and field),当时
看那些孩子训练时,感觉比国内的学生差很多。乱糟糟的什么姿势都有,一点也不规范。哪个学生没跳进沙坑,老
师不是批评你``你怎么跳的!''而是说``噢,宝贝(甜心),你真可爱!''听了这话,我真挺晕的。


果然,开春后,天暖了,他们就开始训练了。跑的,跳的,投的,分别由不同的老师专人负责,男女生分开,按年
级逐项练习,在运动会之前按计划展开,哪天该哪个班练哪项,都制定出详细的时间表,贴在校门口的墙上。


记得有一天放学回家,毛豆告诉我,``妈妈,我今天长跑(4年级的长跑指400米)跑了个第一''(谁都爱报喜不报忧
啊,我不也尽写些我们做的比较好的一些事情给大家看吗)。乍一听,多唬人哪,谁眼前不浮现出个飞人小女侠的形
象啊!就凭她?自个儿家的孩子咱能不了解吗,这不一问就明白了,整个四年级女生一共11人,跑了两回,她一次第
1,一次第3,总成绩第2。后来我把这事儿当笑话讲给国内的朋友听,也按这个叙事顺序,开始都把他们镇住了,哈
哈,想想就想乐。最关键是你们如果来看看就明白了,这里的孩子大多数都不把这个当回事,完全是吊儿郎当地当
玩,一点没有国内那种拼抢争先的精神。毛豆干事比较认真,稍一用力就捡着了。


不知这个会不会和体育成绩挂钩(咱也不好意思向老师问得太直白,显得怪较劲的与这儿的懒散文化不符),管他有
用没用,权当个突破口吧。这之后捷报频传,毛豆多项成绩进入前4,可以代表学校去参加联合运动会。成绩出来后
运动会之前,全校张榜公布。


毛豆是班上项目多的人(无论成绩好坏,每个人至少能参加一项比赛)之一,有4项,她当然开心了,以前在国内参加
学校运动会,选拔比这可严多了,``第一排''的毛豆永远是做一整天的观众,老老实实坐在看台上,还不能随便乱
动(也怪不得她上学后的理想是当一名老师,原因之一就是老师可以走来走去)。来到加拿大,她居然成种子选手
了,而且当学生,她也可以走来走去了,无论是上课(得有前提)还是开运动会。


我们虽然清楚她的实力,知道她不会有太大的作为,但也还是抽时间陪她去minoru练习练习,重在参与,也要有良
好的精神面貌嘛。校长因知道我关心毛豆的体育成绩,运动会前夕在参加Richmond Heritage Fair(毛豆因作品入选
代表学校参加该活动,我是义工司机)时碰到我,还特意向我说了她的表现。嗯,估计成绩不能给C了至少,我心中
窃喜,感觉自己真的挺功利的。


毛豆在联合运动会的表现确实和我们估计的一样,乏善可陈,但我们平时的训练让她在赛场上表现得比较顽强也算
没白忙活,凡事不能要求太高嘛。(待续)



请参与分享:您家上小学的学童在体育锻炼上有可以分享的经验吗?

\section{小学教育之体育篇----游泳,运动会,田径,体育成绩及其他(2)}


学校里每年也都会举办一次sport's day,有点象趣味运动会,都是些托球跑,扔球,跨障碍什么的。全校的学生打
乱班级年级分为红黄蓝绿四个队,各队都是从小到大每个年级的学生都有。各队按颜色着装,画脸,喷头发都可
以,开始前在GYM里誓师,当场决出精神面貌分。之后拉到操场上展开较量,10几分钟一个项目,每次两队对决,按
胜负分别记成绩。中间也有加餐,还有午饭。下午是最激动人心也是得分最高的最终决定胜负的拔河比赛。


这个活动,不常见到国内那种你拼我抢,你争我夺,更多见到的是大手拉小手,因为每个组都是有大有小从学前班
到7年级都有,而很多项目,那些小孩子都必须在大孩子的帮助下才能完成。从没见到过哪个同学为了求快而责备那
些慢吞吞的小弟弟小妹妹的现象发生。


每年3,4月份,他们学校1--4年级还有一次为期两周的游泳课。3年级刚来时,因为经验不足,毛豆只能无奈地混迹
于1级班。这里我先简单介绍一下游泳方面的情况。这里游泳的大众教育是和红十字会绑在一起的,原先是12级现在
改为10级(如果3级和5级不是1次过还包括3A和5A两个中间级),循序渐进地介绍5种泳姿(有一种象躺着的蛙泳一样的
叫elementary back stroke,在国内没见过),潜水,跳水,以及救生常识等,当然都是些普及知识。关于10级后上
什么学加什么分之类的说法我不清楚也没去研究过(有哪位知道并愿意分享的吗)。


大家也许知道国外是从自由泳入门,国内则是蛙泳。毛豆出国前只基本学会了蛙泳,自由泳一点不会,英雄无用武
之地,所以3年级那次课只能从1级班开始,从零开始学习自由泳,仰泳的分解动作。


我始终认为,孩子只有成绩好才能有助于培养他们的自信心,无论什么科目都一样。我们的语言出于劣势,但总得
帮孩子找些强项出来吧,总不能语言差,什么都差吧。好在到4年级上游泳课还有的是时间,我们的目标是争取让她
在最后一次的学校游泳课上,能进入高一点的级别。


06年暑假回国期间,我们为她请了一对一的游泳教练(相对便宜啊),还是从她比较熟悉的蛙泳开始,真正把这种泳
姿完全固定住了。由于夏季教练较忙,训练时间有限,她的自由泳刚开始我们的假期就结束了。


温哥华的冬季漫长多雨,为了给毛豆创造锻炼机会,我们特意办了张半年的游泳卡,有空就去练(每周3-5次)。初步
打算是,先这么练着,合适的时候再报个班,争取在游泳课开课前进行测试。后来事情的发展是,通过这种密集练
习,毛豆在彻底掌握了蛙泳,完全熟悉了水性之后,参加了个2级班(她在加只有1级毕业证,所以只能报2级)继续巩
固自由泳和仰泳的基本动作,然后自己摸索着学会了自由泳和仰泳。为了保险我们还为她找了个老师单独辅导了几
节课,然后在开课前去做了个测试(免费),如愿以偿,升入5级。


定级后第二天,毛豆的游泳课开课了。因人数有限,5--8级被编在一个组,是最高级别的组(9岁10岁的孩子,有意
无意中是会有一些的peer pressure存在的,我相信这会在无形中提高孩子的自信的)。但因为是级别组中最低的,
毛豆又嘟囔了好几天觉得难,中间还因不舒服缺了2堂课。但到泳课快结束时,她已经感觉自己十有八九能过。果
然,一次过!其实虽然不同级别编在一个组,但要求不同,她有个8级的好朋友这次就没通过。


高兴之余,我心中还偷偷地想,不知道这个能不能和体育挂点钩啊?


不管怎样,学期结束时这学期的体育成绩最终总算变成了B,真是功夫不负有心人哪。那些天生体育就是A的人,知
道我们拿个B要费多少力吗?


这个暑假的游泳任务是巩固自由泳和仰泳(自学的毕竟太三脚猫)并学习蝶泳,目标已基本完成。不知再次测试结果
怎样,希望她能用少的时间精力早日通过10级。


毛豆也插空一级级学了溜冰,还没完成。因为叔叔喜欢滑雪,也算高手,在06年的2月春节和12月圣诞节两次来看望
我们,大家一起去Whistler。通过学习和后来的练习,现在,毛豆可以滑部分蓝道,毛豆爸爸滑部分黑道,毛豆妈
也被迫被赶鸭子上架,在不摔一跤的基础上勉强学会滑简单绿道,不思进取啊。毛豆妈不爱吃苦,学游泳也几乎没
呛过一口水,所以到现在只会一种泳姿。比比孩子也蛮惭愧的,以前总得硬撑着给她做榜样,好在她已经懂事了,
告诉她不要象妈妈这样的不刻苦我就可以偷偷懒了。


一点温馨小提示:来之前,体育方面能多学就多学,技不压人哪。国内的教练费便宜,训练手段强硬,只要是练过
两下子的出来很快就能大显身手。来了看看人家老外的体魄体型,就知道我们的人种还是有差距的。无论男孩女孩
都一样,这里虽没有``德智体''的提法,但全面发展还是挺重要的。个人感觉,至少应学学游泳,篮球,网球(这里
场地多,而且免费)。除了为健身,对日后的社交也有帮助,不然将来和朋友交往起来,啥也不会挺没面子的吧我想。
如果有点单排轮滑的基础,来之后学溜冰滑雪都能容易一点。



看来是毛豆妈对形势估计不足,想是家园里的新移友居多,我的分享话题选的不太合适,还是听我一人侃的居多,
真遗憾听不到更多人的经验。


\section{小学教育之音乐篇\myrule 很想听听大家的意见}


毛豆在国内学过好长一段时间的钢琴,但没考过级。兴趣从开始的比较浓,发展到后来的不怎么爱练,练琴经常哭。
当时我们周围曾流行这样一种说法,说谁要把你得罪了,你就劝他去装修房子(这活准让他扒层皮);谁要彻底把你
得罪了,你就劝他孩子去学钢琴。因为我当时工作经常需要出差,这事就没盯得很紧,结果有一搭没一搭干出来的
事效果就可想而知了。看到身边学琴的孩子大多弹得比自己好,孩子的自信也没了,来到加拿大后,坚决提出不学
钢琴了,要学就学小提琴。


毛豆爸爸开始不同意,后来在毛豆妈的帮腔下,终于同意了。05年9月份,很偶然地也比较幸运地找到了离家很近且
水平比较高要求比较严的老师。在老师的要求和帮助下,我们买了前面学生倒下来的二手手工琴,拉了一段时间耳
朵有些长进后,再对比听其他的,音质音色的确不一样。我这门外汉觉得,老师的建议是非常有道理的,从开始就
要用好琴,这种影响是潜移默化的。


严师出高徒,毛豆虽然在艺术上没什么天赋,但她仗着钢琴的基础和比较认真的态度,在老师的调教下,07年1月的
2级考试中,居然得了91分的高分(first honour with distinction),评语也写得夸张得离谱,这个成绩在全国排
名的4\%以内。


曾经在琴行和一个来自北京的小伙子聊天,他16岁来加,问他两边的琴童有什么不同,他说因为教育不同,国内的
孩子太注重技巧,这边则比较注重综合素质的培养。这个不同好像和其他所有的教育都有相似之处,都是国内的基
本功比较扎实,但知识结构不全面。


不知道国内的小提琴考级怎样,这边要求拉曲子时必须带钢琴伴奏。还好,老师的女儿可以配合我们,这个从小受
父母熏陶的可爱女孩,从小自觉每天练4小时琴,精通大提琴和钢琴,却并没有步父母后尘,而是选择了其他的专
业,且工作干得非常出色,因曾参加过一届选美,在当地的华人社区里也算是名人,遇到当地的重大外事活动(比如
国内的领导人访温一类的)她经常担任双语主持。她也是和毛豆同样的年纪出国,现在可以讲北京普通话,发音很标
准,但只是几句话还可以,明显用英语表达比中文流利。由于父母选择不让孩子学汉语,所以她不认中国字。有一
次她和毛豆谈到学音乐的好处时说,她发现周围凡是学音乐的人学习以及做其他事情都很专心专注,有几个因种种
原因学过但后来放弃了的,长大后都很后悔没能坚持下来。毛豆听得很认真,来自她喜欢的大姐姐的寥寥数语有时
比妈妈的唠唠叨叨还管用呢。有趣的是,当时大姐姐正讲着汉语,当她想说这一段话时,马上就转到了英语,并告
诉我,``对不起,这个有点难,请允许我用英语说''。毛豆长大后中文该不会是这样吧?!


从刚开始的每天半小时到现在的每天一小时,毛豆练琴已经很自觉,从不用妈妈督促。老师要求很严,每首曲子都
要拉得满意才能过。因怕她的动作变形,也怕温度湿度不同影响琴音,在低阶时暑假这两次回国都没让她带琴练琴。
如此一来,我们的进度与其他孩子比起来,就更显得比较慢了。她现在仍在3,4级的曲子上游走着。


这边的音乐考级还要考听力,所以弦乐离不开键盘乐器的支持。因为来加后看到会弹钢琴的孩子实在是太多了,而
且还发现普通老师会弹琴的也不少,犹豫着,毛豆提出想重拾钢琴作第二乐器(当然也是在毛豆妈的旁敲侧击之下,
因为既然练耳要用到钢琴,买回来闲着不如用一用)。


要求一经提出,我们很长一段时间内只是论证其可行性,种种困难都事先和她进行了沟通。最后的讨论结果
是,``我要学''。说实话,我也替她捏把汗呢。于是,今年的4月开始,钢琴也开始了,现在每天至少40分钟,可真
够她忙的。好在我们没什么其他的业余兴趣班,这里的功课又不太紧,暂时时间还能排得开。


在这儿有个问题我非常非常想听听大家的意见,关于学琴的进度,这个问题毛豆妈和毛豆爸是有些分歧的。


很多华人的孩子,小小年纪就考出7级,8级,甚至10级来,这种现象,在我们身边不能不说比较普遍。可是,按毛
豆老师的说法,孩子太小,对音乐的理解是达不到那么高的,不应该提前那么多。更有的学生,只拉考级的曲子,
然后一级级过,考完后说终于可以结束了,再也不用练琴了,这和我们让孩子学琴的初衷背道而驰吧。还有的孩
子,由于基础不牢,做成了夹生饭,考到一定的级别就怎么也上不去了。可是如果从另一个角度分析,孩子小,才
有更多的时间练琴,大了,功课紧了,哪有那么多时间练琴呢?


从老师的教学进度看,毛豆到10年级(相当于国内的高一)时也不一定考完10级啊,毛豆爸爸有点抻不住了,很想换
个老师试一下。毛豆妈妈希望毛豆能把音乐作为终生的爱好,希望她把基本功练扎实了,而且跟现在的老师学习,
每次考级都绝对有把握,一点不紧张也不痛苦,分数肯定也低不了,更重要的是老师严谨的作风已经给了孩子潜移
默化的影响。虽说这边也有很多老师是打着几年考出几级的牌子来招生的,可是毛豆妈一直在观望,希望找到两全
其美的办法。我们到底该作什么样的选择才好呢?家有琴童的家长们,分享一下您家孩子课余时间的分配好吗,毛豆
妈妈很想多听听大家的经验和建议。

毛豆妈妈,很感谢你的文章,我也是有两个小男孩的妈妈,你的经验给我很多启发.同时,我也是一个钢琴老师,所以,
就你的问题,说说我的一点粗浅意见.


首先,我觉得你做的很正确,孩子现在从厌倦学钢琴到enjoy小提琴然后又恢复学钢琴,这中间有老师的严格要求,也
有你们父母的正确选择.我想,我们作父母的,要孩子接触音乐,无非是为了陶冶性情,增加修养,所以,培养孩子热爱
音乐本身比考级不知道要重要多少倍.毛豆现在老师的意见是十分中肯正确的.我教琴的时候,会遇到很多家长,希望
通过考级来检验孩子的水平,这无可厚非,但当考级成为学习目的,或者加快进度的目的时,就会背离我们要孩子学琴
的初衷.我们都知道,考级就是那些曲目,当孩子用所有的时间,就练习这些曲目时,进度当然会快些,但是,这牺牲掉
了孩子享受音乐,享受成功,享受成长的机会.


孩子学琴,程度快慢都是不同的,每个人需要不一样的时间,只要在一个正常的进度范围内,我强烈反对为了考级轻
易换老师.每个老师的个性/方法/教学理念/对孩子的了解都是不同的,如果毛豆已经从现在的老师这里学会了自主
练琴,并喜欢音乐,这是很难得的.艺术教育对老师的要求很高,一个好的老师会给孩子带来终身的受益.以我自身为
例,我的钢琴启蒙老师是我到目前为止经历过的众多钢琴老师中最优秀的一位(她是已经退休的中央院附中校长),她
给我的影响是终生的。举个小例子,我的老师上课的时候,严谨认真,虽然是在她家里,也决不允许任何事情打
断,我现在给学生上课的时候,也会很自然地效仿她,任何急事都不能打断教学。可以说,如果童年时代我接触音
乐的时候,没有遇到她,我以后也不一定走上专业道路。现在想起来,在我的启蒙老师以后的几位老师,虽然也是
各有特点,但如果当时不换老师,应该在童年时代对音乐的全面修养和潜力的挖掘会有很大的不同。这体会在换了
老师后的第一节课就有了。孩子对某种课程的喜爱,往往是在对老师的认同之后才产生的,回想我们的童年,我想
我们大家都有体会。


总之,一个称职的懂得学生心理的敬业的老师,会在这一学科的学习中甚至遥远的未来给孩子深远的影响。

\section{在家进行的中文和数学教育}


毛豆妈妈现在暂时不工作,因为发现在加拿大,孩子的教育更需要家长的参与。单看他们的上学时间吧,就得有个
人靠上,早9晚3,每月还有一天教师进修日(Pro D Day)学生全天不上课。于是毛豆妈在家便有了一份新职业,充当
毛豆的中文和数学老师(刚开始还当英语老师来着,后来因学生水平提高我便待业做part time了)。我们买来国内的
教材,基本按国内的进度在家开始了中文和数学的课外学习。


先说进行得比较顺利的数学教育吧。我一直告诉毛豆数学和艺术一样是无国界的,一定要学好,毛豆爸更有个理
论,认为数学好的人才叫聪明人。可能是小学期间的数学不太难,目前看,进展还算令人满意的。毛豆仗着国内的
底子,加上我们自己开的小灶,数学仍然在班上有绝对优势(这里学的真是太肤浅了)。


这边的数学教育内容比较浅,涉及的面其实比较广,不过很多问题只是浅尝辄止,并不做很多练习加以巩固,所以
不少概念学生们理解得并不深刻就过去了。计算能力,尤其是口算心算更是一点要求都没有。1年级的加减法都是靠
数数来解决的。以前看过个帖子,提醒华人家长一定要教孩子用中文背``小九九'',因为做题省时间。比如``二三
得六'',中文干脆利落张几下嘴便完事;``two times three equals six'',英文拖泥带水舌头得绕好几圈才行。不
知有没道理,仅供参考。


有段时间看新闻,说不少家长也在为这里的数学教育发愁呢。还有一则报道说,很多学生进入大学后,也感觉数学
比较吃力,很多大学对多数新生的数学水平也不满意,可见不光小学如此,中学的数学教育也很一般。提醒各位应
该对此引起足够的重视。相反很多较高年级才从国内来的孩子,本来并不是什么好学生,来了后发现数学竟成了自
己的强项。

就这样``摸爬滚打''着,每年暑假回国,还继续给她报名参加奥数班的学习(主要是锻炼思维和解决问题的方法)。
可是我们做梦也没想到的是,06年回去的第一堂课,毛豆竟然最大的障碍是用中文上课感到不适应,看来孩子的英
文进步和中文退步速度之快都超乎我们的想象。今年虽然遇到抄题也比较慢,但因为上次的教训有了思想准备,已
经没上次那么明显了。只是说第一节课发现周围的同学回答问题竟然都说汉语,一时间感觉怪怪的。


说起中文教育来,问题就大多了。首先毛豆越来越不愿意写汉字,认字读课文还说得过去,中文课外阅读也非常喜
欢,可就是不爱写字,更加不爱动笔写文章。现在回头看看她1年级写的日记,现在竟然也就那个水平,有的还不如
那时呢。毛豆妈想尽一切办法刺激她的积极性,威逼利诱,糖衣炮弹,使了不少招术,本上不爱写,就写白板,写
黑板,写大字报,再不就我们倆一起比赛组词,可是目前收效不大,时好时坏。毛豆妈心有不甘,但也不能太强
求,进退两难。虽说女孩子将来不一定让她再回国寻求更大的发展空间,但多一条后路总是稳妥点。学习任何语言
也离不开听说读写啊,可千万别象毛豆爸说的,我们这代人30好几了跑出来成了``文盲'',等孩子长大以后回国也
变成了``文盲''。最近在想,要不,让她只是学会打字就行了?诚恳地问一句,大家有什么经验可以分享吗?我目前
的想法是,中文至少要争取教她到小学毕业,力争只做个``半文盲''。


我还问过不少在这边上中文学校的孩子,无一例外,都不能进行中文阅读,就认识那么几个字,看来靠人不如靠
己,环境的作用真大啊!


曾在学校的一次义工活动时和一个台湾妈妈交流过,她说开始时也是自己在家教中文,后来发现把两个人的关系搞
得好差,索性放弃了还是送出去学。我们也的确会经常为此产生矛盾。


回国毫无疑问是个挺好的学习中文的机会,发现她回来后读课文明显流利了许多,在加时很多课文读起来磕磕绊绊
的,自然越读越没心情。

这次回来还买了不少中文书,把毛豆喜欢的向有兴趣的家长推荐,大都可以在当当网上找到评价。

《窗边的小豆豆》:强烈推荐给家有令人头疼的顽童的家长,亲子共读,换个角度看孩子。

《昆虫记》(韩译本,10册):法布尔原著,以讲故事的方式讲述科普知识,字少画多。

《哈佛家训》:有点类似心灵鸡汤,已经出了2本了,温馨感人励志。

《小淘气尼古拉的绝版故事》:一个调皮的小男孩的故事。

还有杨红缨的马小跳,笑猫,哭猫日记,郑渊洁的皮皮鲁,鲁西西等,以前买了很多,这次把新出的补齐。

另外有一套韩国的《我的第一本科学漫画书》(4本)她也十分喜欢,可以看很多遍,男孩女孩都适合。


毛豆的中文学习目前给我的最大安慰就是她十分喜欢阅读,先这么坚持着吧。不识汉字,将来她就会丧失和先贤
``对话''的机会,少一些人生乐趣。当时我在陪她背《论语》时就强烈地感觉到,中国古代的文化真是灿烂的奇
葩,不能去亲近真是人生一大憾事。


庆幸自己的题目是分享,不是经验介绍,越写到后面越发现自己的经验越来越少,困惑越来越多,还是希望能多听
听大家的经验,呼吁有更多的爸爸妈妈把你们的经验写下来与家园的朋友们一起分享,快点行动吧,哪怕是一点一
滴也好啊!

\section{大人的英语学习}

毛豆妈妈的英语学习路是这样过来的:ELSA\myrule ESL\myrule (这中间省略了ALP)\myrule 成人高中。不知是否
准确,我是这样来比喻的,ELSA是幼儿园, ESL是小学,ALP是初中,然后English10以上是高中。

毛豆妈妈自打大学毕业后,除了考职称,就没怎么碰过英语。可是就这水平,据过来人讲,如果不伪装一下,也很
难享受政府的免费ELSA课程。这个也因人而异,身边也有个别西部院校毕业过来的男生,不用装就是1级的。


果不其然,去中心测试时,咱也怕太给中国的本科生丢脸,人家说什么我就跟人家聊什么,半程时觉得表现得有点
过,在后来的听力时就保留了起来,但还是给定了个听说4级,当时只有3级以下才能上免费课,于是之后的读写我
就干脆就没太敢答卷,好容易混进了革命队伍。


当时因为是刚来,不了解诚实的品质在这边是多么的重要,身上难免还带了些为了一己私利而弄虚作假的作风,在
此表示一下内心的忏悔。后来知道有的同学为了能上这个课,很坦诚地告诉测试人员实情,说自己英语虽然有些基
础,但工作后不曾用过,希望能得到一个机会学习学习。此后,我再遇到类似的事情都是采取了诚实的态度,能得
到想要的机会最好,得不到也不去强求,再不愿意去作假。


混进3级班后,我选择了全天上课。那段时间还真挺赶,早上起来床褥都来不及整理。比较幸运的是碰到了个好老
师,地地道道的加拿大人,她年届50,个子近1米7,大大的眼睛,蛮漂亮的,人很善良,常会为一些小事感动得热
泪盈眶。她天天换衣服配饰,虽然并不贵,但颜色款式总是搭配得当。据说她为了环保的原因,基本买二手衣服。
最关键的是,这样的老师不会误人子弟,任何我们想要问的主题,她都可以讲得头头是道。后来才知道,这样的老
师,基本靠运气才能在ELSA里碰到。


在那三个月里,我最大的收获,就是初来乍到就完全掌握了加拿大的行政区划,我们在课堂上以各种方式学习背
记,很快就搞清楚了各个省和区以及各自的首府,地名熟识之后听18台的天气预报很有用。后来发现,人家毛豆的
social study也学这个,而且学起来比我们要快也不费力气,我不得不感叹自己和孩子已经开始有了差距。


因为要照顾多数人,课程进行得比较慢,但我们的内容倒是五脏俱全的,经常会有presentation,field
trip,potluck,其他的更是逢节必过,着实了解了不少加拿大的风土人情以及少许的地理历史知识。虽然没一点压
力,但因为学的东西不够多,也算是浪费了一些时间。


说到feild trip,让我想起了一个笑话。有一次老师带我们乘公车(我的第一次公车体验)去downtown,在Canada
Place里面参观的时候,路过一间叫``ball room''的大房间,同学们想当然地从字面上理解认为是个打球的地方。
那一时期毛豆课堂上正在学经典童话,我受过熏陶,知道``ball''还可以当舞会讲。你说这么简单一个词儿,人家
土生土长的小孩子没一个不知道,我们这帮三四十岁的人还有多少这样的词不明白啊。所以来之后才发现,我们成
年人,任你怎么学,你永远都是``老外'',这是无法改变的现实。这一点也是国内的人永远不会理解的,单从这一
点来说,如果想完全融入,孩子还是小一点来比较好。


别小看这ELSA-3,除了从ELSA-2升上来的水平稍差一些,只要是直接进来的,大多都有几下子。本科生不少,当然
都是30好几以上的,年轻的不会来这儿混日子,个别20几岁的估计都是在国内不好好学习的。还有的居然是考雅思
出来的,但听力口语实在太差,也给打发到这儿了,后来一接触就发现,他们的词汇量的确大。


ELSA毕业的人想继续学习的,有的去读VCC,有的去了昆特兰,有的去了BCIT,还有的象我,去读成人高中。上昆特
兰(语言班除预备级外有3级,要求很严,老师负责,进步蛮快)据说作业多,花钱也相对多,我暂时没计划给自己太
大的压力。


结业考试,我的成绩都在6分以上(3分结业,人人做得到),可能代表ELSA-6,不太清楚,反正这个结业证书是一丁
点用处都没有的。后来听说拨款增加,ELSA可以免费读到5级,据了解,因为这个,结业考的成绩反而有点用了,太
低的和太高的都不允许继续。不过,也是听说,4级5级不是正规上课,而是辅导你如何找工作。


结业之后我去Richmond的教育局做了测试,准备下一阶段的学习。列市和温市都有成人高中课程,但两边的形式还
是有些不一样的。据说,温市测试后基本就可以选课开始上了,好像还有English 9,每门20刀,不及格的可以重
学,所以总是人满为患。周围不少人在southhill上过课,感兴趣的不妨自己查一
查,http://southhill.vsb.bc.ca/。


而列市的情况是,多数象我这样水平比较低的,测试后往往要先去读一个收费的ESL(注意和政府免费的ELSA是不同
的),水平高一点的可以直接读ALP,更高的也可直接读English 10。


ESL也由低到高分很多的等级,advanced为最高级。这个课程是收费的,通常每周2次课,每期要200刀左右,日程安
排和价格教育局网站上能查到$[$url$]$http://www.sd38.bc.ca/。ESL毕业的学生要再进行测试,方可进入下一阶
段ALP课程的学习。如果水平不够,不读ESL是没有资格读ALP的。


ALP分四级(少数水平不够的也有个预备级),天天上课,每天2个半小时,政府补贴学费,自己掏书本费,每期百多
元左右。这个课程主要以阅读写作为主,口语较少。大多数ESL毕业的学生再次测试后能进入ALP-1(象我ELSA一起毕
业的同学在读过ESL高级班之后基本进入ALP-1),少数进入ALP-3(包括曾在新西兰留学的人士)。可也有象我这样的
人,没什么水平却一不留神直接混进了English 10。进去之后才发现,虽然省了时间,但学起来还是比较吃力的,
和其他同学的差距不小。毛豆妈调整了心态,不去和人家比,只和自己比。想着只要自己慢慢地积跬步,纵然行不
了千里,毕竟也人在旅途。


English10以上的课程也是政府补贴学费,没有书本费,但开始要交百多元的deposit,学期结束后有大部分的退
费,自己也就掏二三十刀。虽说和温市的高中费用相差不大,但因为门槛比较高,比较繁琐,加上知道的人不多,
名气不如温市的成高响亮,经常会招不齐人。好处是据两边都学过的人比较,老师更负责任。


English10开始就可以选一些课程读diploma了,象不少朋友感兴趣的中文教师资格证书就必须是English10以上才能
读。也可以选择再自修一些其他学分,直接拿一个高中文凭。


很多超龄(国内高中毕业)的移民子女(19岁以上的居民没有当地的高中文凭者),也在上这个课程,如果12年级的省
考分数过关,据说申请大学可以免托福。


毛豆妈现在已经读完English 11,上学期间,我们要学诗歌,写日记,编报纸,读小说,莎士比亚更是必不可少。
上学期我们还排演了莎士比亚的名剧《马克白》,那份投入和认真可真是一种全新的经历和感受。

不少人对这个题目感兴趣,但毛豆妈觉得自己的英语学习经历没有什么代表性,也不是什么最佳选择,只是在做全
职妈妈的同时给自己找点事情做,所以仅供参考。哪位朋友如果有更好的经历,不妨在此和大家一同分享。

我是这样想的,除了打发时间,首先,我们虽然暂时不工作,但也不能就这样被淘汰了,以后无论继续留下还是回
流,工作与否,多学点英语都不吃亏;二来,也应当给孩子树立个榜样,身教胜于言传,妈妈可以不优秀,但不能
不进步;三呢,在这里英语不好,是很难和孩子保持良好沟通的;最后一点,我认为搞好``自身建设''(英语是其中
一项)的女人,才能立于不败。


关于``大人英语学习''的一点补充


知道毛豆妈为什么觉得成人高中读起来不轻松吗,首先是因为毛豆妈在测试时选了一个简单(三选一)的作文题(写作
成绩直接影响定级),导致测试结果严重高于我的实际水平(尤其让人难以容忍的是写作恰恰是我最薄弱的一项),要
不然以我的水平绝不可能连蹦这么多级。


再看看都是些什么人在读吧。我了解的人中,一些是来了10多年的;一些是考过雅思的;一些是香港人,以前工作
处处用到英语的;有的是在英国待过几年的;还有的竟然是台湾某私立中学的英语老师;当然更多的是ALP读了4级
一路升上来的。有土耳其人,俄罗斯人,韩国人,马来西亚华裔,伊朗人,等等,他们的优势是很难找到说母语的
人,所以口语比我们进步快。


唉,你说我这水平跟他们比,一时半会儿哪赶得上嘛!以前习惯了作``鸡头'',感情这``凤尾''的滋味不是很好受呢。
不过,学期结束时,挣扎着好歹咱也能拿个B,更让我很阿Q地聊以自慰的是,我的进步幅度比他们大,嘿嘿!


今年上半年,《Vancouver Sun》搞促销,我试着订了仨月,唉,可真把我给累坏了!压力太大了,都有点喘不过气
的感觉(毛豆妈也是财迷,不忍心花钱买废纸)。天天捧着本字典,边查边读边抄录,经常是熬到半夜刚把当天厚厚
的一沓解决掉,第二天清晨,新的一沓已经在门口静悄悄地等着我了。


至于好处,我就不多说了,大家自己发挥想象力好了,实在想不出,订份试一试就知道了。


读成人高中还有个好处忘了说,可以知道我们的孩子将来高中英语学什么,提前多一些了解。


另外一个很深的感触是,莎士比亚在西方的地位远高出我们的想象,其影响可以说已经渗透到每一个受过教育者的
血液里。我发现《Friends》里有一集,Ross被朋友调侃指导着追求Rachel,用到``woo''her,这个简单得只有3个
字母的词儿现代英语不常见到吧,《Romeo and Juliet》里面就有,我不知是否是出处。在此小小地卖弄一下,不
知对错,让大家见笑了。不过,我坚信不疑,这样的例子在生活中比比皆是。学一个算一个,不学永远不知道。


还有毛豆妈要在这里做个自我批评,怎么一回到国内就光知道吃喝玩乐了呢?刚回来那几天还坚持上《Vancouver
Sun》或CBC News的网站浏览浏览,后来就\ldots\ldots 唉,还是不刻苦啊!真是惭愧呀!

\section{朋友及他乡的生活}


毛豆来到加拿大后的一些学习上的经历已经和大家分享得差不多了。最后趁这机会来说说我们的一些生活点滴吧。


来到陌生的国度,生活展开了全新的一页。两年下来,我们的生活由新鲜到慢慢稳定,逐渐适应。没经历什么风
雨,日子过得简单宁静。这样的生活对于我这种不喜热闹应酬的人还是很容易适应的。喜欢热闹的XDJM们想来长登
可要做好充足的思想准备啊,这里可不象国内那么热火朝天,也没有什么灯红酒绿的。


我有个作家朋友把现在的中国比作一个沸腾的工地,感觉还是相当贴切的。不过加拿大尤其是温哥华也过于安逸
了,以至于毛豆的评价是``加拿大已经睡着了,我感觉她老在睡觉'',对比着她同时还说``中国现在要挣钱,所以
到处盖高楼。''这是她来加半年时的感受,品味起来还真有那么点意思。


还有一点,我的做法不知大家是否认同。移民加拿大是我们成年人自主作出的选择,没有人强迫。来了,觉得能住
下的就安安稳稳地住下,觉得适应不了的就打道回府,天大地大何处都可以是我家。何必象某些帖子那样整天价为
了讨论到底是加国好还是中国好而大打口水仗呢?所以每次只要一坐上飞机,无论回中国还是加拿大,我都要象倒时
差那样调整自己的心态。我希望无论时间长短,在哪边就过好哪边的生活。回中国,我就假想自己从没出去过,不
去批评什么脏乱差,不去评价是非曲直,乐得自己高兴,他人愉快。回加拿大,我就让自己放慢脚步调低嗓门(外交
部最近也提醒国人出国旅游要小声说话呢),开开心心地去适应那份宁静和自给自足。凡是遇到讨论加国好还是中国
好的同胞,我一律是只听不答不接茬,因为没得说,他们说得都对也都不完全对。您同意我的观点吗?


不知是不是我存在什么偏见,来了以后发现,我们和港台的同胞确实存在着文化方面差异,总觉得不能达到心灵的
沟通,所以来了以后,关系比较近的朋友都是大陆背景的。我们比较幸运的是,有好几个在国内就是毛豆爸爸的极
要好的朋友也在这里,这些就不多说了。


我们的townhouse是新建的小区,几十户人家以华人为主,可能讲粤语的比例多一点,洋人也就一两户。在国内住惯
了老死不相往来的单元房,几年下来不知对面邻居为何人的,想必是我们多数人的写照。没想到的是,来到加拿
大,我居然找回了小时候住姥姥家时的那种邻里之间的感觉,互相照应,共同分享。现在我们和隔壁,已经亲如一
家。


有些老移民为了提高自己孩子的国语水平会希望他们的孩子和新来的小朋友多讲国语,但他们的子女往往不愿意,
所以刚来时交一些背景相似的朋友,孩子感觉能舒服一些。时间久了,英语好了,他们能够见人讲人话,见鬼讲鬼
话的时候,交友面就广了。这时,我主要留意孩子的朋友是否有上进心,有没有不良习惯。如果有共同爱好,相同
的行为习惯,他们自然关系就近了。


除了那些天生交友能力强的孩子,大多数孩子在小的时候交朋友是需要大人的帮助和引导的,尤其是在人生地不熟
语言又不通的异国他乡,在此提醒新移民家长,要象关心孩子的学习一样去关心孩子的交友。为了帮毛豆多交一些
互相能谈得来的同龄朋友,尤其是比她稍微大一点的女孩子(毛豆现在的玩伴以比她小的居多,我和毛豆爸都觉得她
有点太晚熟,特别小儿科),我也在不停地留意。所以只要碰到合适的,我就主动一点,多付出一点,接接送送的能
做的尽量抢着做,多给别人制造些方便。有一次我就是一个人带了三个孩子(当然都是年龄大一点的,小的不敢一人
带)去了Science World,他们都是独生子女,几个人凑一起玩得特别开心。要是这其中能碰到关心孩子教育的家
长,不光孩子开心,大人之间能沟通的话题也就多了,更有利于孩子的日后往来。


朋友在我们的生活中真是太重要了,大人孩子都是如此。意识到这一点,为避免等毛豆完全熟悉了加拿大,就不爱
回国的后顾之忧,我还注意每次回国期间都让她多和同龄的朋友交往,希望以后回国不光有亲戚(这在孩子的成长中
不可避免的吸引力会越来越小),还有她熟悉的朋友,以增加她回国的热情。


我始终认为,如果有条件,让孩子多回国转转,是件挺好的事。我也会经常提醒她中国目前还是个发展中国家,不
让她带有色眼镜偏激地看待一些现象,但愿她随着成长能比较客观地建立自己的人生观和价值观。


我们的容貌是无法改变的,香蕉人的芯儿再白,在老外眼中皮儿都是黄的,永远是``中国人''。不管别人怎么样,
我不希望孩子忘记自己的根。话又说回来,加拿大的多元文化是有目共睹的,只温哥华就有母语70余种,毛豆一个
200来人的学校,在和平纪念日那天,就有16个孩子手捧蜡烛,用他们的母语说出了``和平''。个人认为,``融入''和
``根''并不是水火不容的。

\section{分享之结束篇}

这边的很多活动,都是以亲近大自然为主题的,象我们刚来时办了钓鱼证(大人的每年20几刀,小孩的免费但不能单
独办)经常去白石镇钓螃蟹,一笼下去总有个10几20只左右(个头通常不符合标准,随后就放生了);去看三文鱼回
流,几百万条红鲑不吃不喝出生入死地回到它们的出生地产卵,然后在河床静静地死去,完成它们的生命轮回,也
怪不得小镇称此项参观活动为``向三文鱼致敬'';也放生过三文鱼苗,5月份,在小溪边,志愿者们用车载来鱼苗,
大人孩子们用小桶将它们放入水中,它们便会奋力经河流游向遥远的太平洋,在那里长大成鱼。类似的活动,每年
在各网站上都会有相关的信息,这里不再赘述。


来加之后还发现,这边的很多业余活动大都是以家庭为单位的,包括请客吃饭,社区活动,尤其是周末假期,更是
到处能见到拖儿带女尽享天伦之乐的情形,体验之后我们的家庭观念也前所未有地得到加强。感谢加拿大,让我们
学会了知足,学会了感恩,知道了生命中什么才是我们最应该珍惜的,这一点也是现阶段浮躁的国内被大家已经忽
视了很久的。


另一个感触是,以前国内多数人认为老外们不管孩子,不象中国的父母这样对孩子如此上心,来了发现这也是另一
个误会。首先,这边的小孩业余生活非常丰富,各种特长也比国内的孩子普及,家长不付出怎么可能;还有,我们
说国外孩子学习没压力,没负担,其实是表现形式不同,真正抓教育的西人家庭(虽然这样的家庭不是多数,但至少
他们才是代表了西方的主流),其付出绝不亚于我们,甚至是有过之而无不及。记得美国曾经有份调查显示,白人女
性受过高等教育者的平均收入低于其他族裔,乍看的确匪夷所思。而深入调查后的结论是,白人男性受过高等教育
者的平均收入明显高于其他族裔,为了教育好下一代,很多女性放弃工作回归家庭所致。


而在《世界是平的》一书中,比尔盖茨在被该书的作者问到美国的教育优势\myrule 强调创造力而不是死记硬背的
学习方法时,表现出的态度是全然的蔑视。他认为``创造力是通过接受测验激发出来的,我从来没有遇到过不会使
用乘法的软件开发者'',``你必须首先要记住原有的知识,才能在其基础上拓展到更广阔的领域''。``学海无涯苦
作舟''的祖训一点没错,不吃苦就能学得好根本就是天方夜谭。刚在一个网站看到讨论这边中学IB班的帖子,看到
了中学阶段学习的紧张和压力。如果学习可以没有压力,为什么那么多美国的顶级名校会有``裸奔节''呢?


诚然,中国式的教育存在着方方面面的大小问题,但真的就一无是处,应该完全放弃吗?我不知道,也不想在此和大
家讨论这个问题。我只是知道我们中国家长在孩子学习上的付出和投入已经越来越得到西方主流社会的认可,而越
来越多的西方教育界人士也开始反思他们在教育上的欠缺。


尽管我们的知识水平能力都很有限,孩子也普普通通,但我们不想放弃努力,努力学习和快乐生活不是互相矛盾的
吧,我想。



有时候,望着不知何时悄悄爬上眼角的皱纹,看着不再苗条已经接近中年人的体态,对逝去的青春难免会有丝丝怅
惘,但每当看到毛豆那吹弹得破掐得出水凝脂般的肌肤,开始有点婷婷玉立雏形的小体型,也会有更多的欣慰涌上
心头,我家有女初长成!都说女儿是爸爸的前世情人,我也心甘情愿地任这个小女生肆无忌惮地和我一起分享着毛豆
爸爸的爱。


曾经给我的作家朋友(正是在她的鼓励下,不会写东西的我开始尝试去记录一些自己的感受,这才会有现在与大家的
``分享'')写了封邮件,最后以此和大家共勉。



蓦然发现,我现在好像是生活在很久很久以前。


天很蓝很蓝,飘着的白云是一朵一朵的;天不蓝的时候一定是阴天,同样能看得很远,灰色的天空颜色有深有浅,
只因云层的厚度不同,不小心在某个地方也会露出一点点的蓝。有人说中国现在的富人可以极尽奢侈,但他们唯一
做不到的就是看不到蓝天,必须和穷人一样呼吸污浊的空气。


我也像以前的妇女一样,做专职的家庭主妇,买菜做饭洗衣打扫卫生接送孩子上下学,活得极其简单。不像现今光
鲜靓丽的OL们时常泡吧、shopping、小酌,那么小资。说到这儿,让我想起,去台湾时,那里的女人脸上的表情给
我留下极深极深的印象,那是一种在大陆女人脸上找不到的表情,那么安宁、平和,无论美丑,因此显得格外干
净,让人感觉特别舒服。你试着找找你身边的人,有这样表情的女人多不多。


我还重新拿起了课本,上学、写作业,努力做一个好学生。


我还找到了从前的邻里关系,有事招呼,互相照应,做了好吃的大家送点分享。


我还学会了精打细算,买东西前会留意一下价格标签,买衣服的热情也不像从前。


更令人意想不到的是,在结婚十年的时候,发现自己的婚姻居然还挺美好的,挺让人留恋的,不再仅仅是在外人的
眼中。


你说,我是不是挺像回到从前?


在大家的关注下,我的``分享''坚持到今天并打算告一段落,感谢这么多人一路的参与,鼓励和支持,毛豆妈妈代
表全家在此祝各位移友们在加拿大的新生活顺利,开心!希望我的分享能给您带来一些小小的帮助,也希望在家园能
看到更多的分享。朋友们,下次``分享''再会吧!

\end{document}
