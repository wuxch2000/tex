\usepackage{fontspec,xunicode,xltxtra}
\setmainfont[Mapping=tex-text]{Times New Roman}
\setsansfont[Mapping=tex-text]{Arial}
% \setmonofont[Mapping=tex-text]{Courier New}
\setmonofont[Mapping=tex-text]{Times New Roman}

\usepackage[CJKnumber]{xeCJK}
\setCJKmainfont[ItalicFont={Adobe Kaiti Std}]{Adobe Song Std}
% \setCJKmainfont[ItalicFont={Adobe Kaiti Std}]{Adobe Heiti Std}
\setCJKsansfont{Adobe Heiti Std}
% \setCJKsansfont{Microsoft YaHei}
\setCJKmonofont{Adobe Heiti Std}
\punctstyle{banjiao}

% \usepackage{indentfirst}

\usepackage[dvipdfmx]{hyperref}  % options: dvipdfmx, pdftex 
\hypersetup{
    % bookmarks=true,         % show bookmarks bar?
    % bookmarksopen=true,
    pdfpagemode=UseNone,    % options: UseNone, UseThumbs, UseOutlines, FullScreen
    pdfstartview=FitH,      % options: FitH, FitV
    pdfborder=1,
    pdfhighlight=/P,
    pdfauthor={wuxch},
    unicode=false,          % xetex should set to false
    colorlinks,             % false: boxed links; true: colored links
    linkcolor=blue,         % color of internal links
    citecolor=green,        % color of links to bibliography
    filecolor=magenta,      % color of file links
    urlcolor=cyan           % color of external links
}
\makeindex

\usepackage[dvips,dvipsnames,svgnames]{xcolor}
\definecolor{light-gray}{gray}{0.95}

\usepackage{graphicx}
\usepackage{wrapfig}
\usepackage{picinpar}

\renewcommand\contentsname{目录}
\renewcommand\listfigurename{插图}
\renewcommand\listtablename{表格}
\renewcommand\indexname{索引}
\renewcommand\figurename{图}
\renewcommand\tablename{表}

\usepackage{caption}
\renewcommand{\captionfont}{\scriptsize \sffamily}
\setlength{\abovecaptionskip}{0pt}
\setlength{\belowcaptionskip}{0pt}

\graphicspath{{fig/}}


% % 嵌入的代码显示
% \usepackage{listings}
% \lstset{language=C++, breaklines, extendedchars=false}
% \lstset{basicstyle=\ttfamily,
%         frame=single,
%         keywordstyle=\color{blue},
%         commentstyle=\color{SeaGreen},
%         stringstyle=\ttfamily,
%         showstringspaces=false,
%         tabsize=4,
%         backgroundcolor=\color{light-gray}}

\usepackage{titletoc}
\usepackage[sf]{titlesec}
\dottedcontents{section}[1.5em]{}{2.3em}{1pc}
\titleformat{\section}{\Large\sffamily\bf\color{blue}}{\textsection~\thesection}{.1em}{}

% \titleformat{\section}{\centering\Large\sffamily\bf\color{blue}}{第~\CJKnumber{\thesection}~章}{.1em}{}
% % \titleformat{\section}{\normalsize\sffamily\bf\color{blue}}{\textsection}{.1em}{}
% % \titleformat{\subsection}{\normalsize\sffamily}{\thesubsection}{.1em}{}
% \titlespacing*{\section}{0pt}{1ex}{1ex}
% % \titlespacing*{\subsection}{0pt}{0.2ex}{0.2ex}
% % \titleformat*{\section}{\sffamily\bfseries\Large}

\usepackage{fancyhdr}
\usepackage{lastpage}
\fancyhf{}
\lhead{}
\cfoot{\scriptsize{\textsf{第 \thepage ~页,共 \pageref*{LastPage} 页}}}


\usepackage{enumitem}
\setitemize{label=$\bullet$,leftmargin=3em,noitemsep,topsep=0pt,parsep=0pt}
\setenumerate{leftmargin=3em,noitemsep,topsep=0pt,parsep=0pt}

% \setlength{\parskip}{1.5ex plus 0.5ex minus 0.2ex}
\setlength{\parskip}{2.0ex plus 0.5ex minus 0.2ex}

% \setlength{\parindent}{5ex}
\setlength{\parindent}{0ex}

% \usepackage{setspace}
\linespread{1.25}

% 英文的破折号--不明显,使用自己画的线。
\newcommand{\myrule}{\hspace{0.5em}\rule[3pt]{1.6em}{0.3mm}\hspace{0.5em}}


%%% Local Variables: 
%%% mode: latex
%%% TeX-master: "newcop"
%%% End: 
