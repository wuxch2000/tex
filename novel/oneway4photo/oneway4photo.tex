\documentclass[12pt,a4paper,onecolumn]{article}
\title{TITLE}
\author{}
\date{}

\ProvidesPackage{config}

\usepackage{fontspec,xunicode,xltxtra}

\setmainfont[Mapping=tex-text,Ligatures=Common]{Adobe Garamond Pro}
\setsansfont[Mapping=tex-text,Numbers=Uppercase]{Myriad Pro}
\setmonofont[Mapping=tex-text]{Courier New}

\usepackage{xeCJK}
% \setCJKmainfont[ItalicFont={Adobe Kaiti Std}]{Adobe Song Std}
\setCJKmainfont[ItalicFont={Adobe Kaiti Std}]{Adobe Heiti Std}
% \setCJKmainfont[ItalicFont={Adobe Kaiti Std}]{Adobe Kaiti Std}
\setCJKsansfont{Adobe Heiti Std}
% \setCJKsansfont{Microsoft YaHei}
\setCJKmonofont{Adobe Heiti Std}
\punctstyle{banjiao}

\usepackage[table]{xcolor}

%生成PDF的链接
\usepackage{hyperref}
\hypersetup{
    % bookmarks=true,         % show bookmarks bar?
    bookmarksopen=true,
    pdfpagemode=UseNone,    % options: UseNode, UseThumbs, UseOutlines, FullScreen
    pdfstartview=FitB,
    pdfborder=1,
    pdfhighlight=/P,
    pdfauthor={wuxch},
    unicode=true,           % non-Latin characters in Acrobat’s bookmarks
    colorlinks,             % false: boxed links; true: colored links
    linkcolor=blue,         % color of internal links
    citecolor=blue,        % color of links to bibliography
    filecolor=magenta,      % color of file links
    urlcolor=cyan           % color of external links
}



% 表格
\usepackage{booktabs}

\usepackage{caption}
\usepackage{fancyhdr}
\usepackage{graphicx}


\usepackage{geometry}
\geometry{a4paper}
\geometry{hmargin=.1cm}
\geometry{vmargin=1cm}


\begin{document}

\pagestyle{empty}
\begin{center}
\includegraphics{f1.jpg}
\newline
\newline
\end{center}

\begin{center}
\textsf{\Huge{漫漫摄影发烧路} }
\end{center}

\pagebreak
\pagestyle{fancy}


这几年在这条路上摸爬滚打过来,又看到周围无数的影友也在重复着和我一样的道路,未免还会碰到我原来的烦恼
和痛苦。所以斗胆将我和我熟知的影友的发烧历程总结出来。希望要上路的影友能得到些借鉴。您如果已经是“过来
人”了,而且文中所述和您的经历类似,请会心一笑,表明咱们产生了“共鸣”。


\begin{wrapfigure}{l}{0.5\textwidth}
\vspace{-2ex}
\includegraphics[width=0.5\textwidth]{f2.jpg}
\caption{有时候迷上摄影,仅仅是因为一张《国家地理杂志》的照片}
\vspace{-2ex}
\end{wrapfigure}


另外要说明的非常重要的一点是,这里所说的影友或发烧友指的是和我一样的广大业余爱好者。咱们共同的特
点是:第一,完全是自费发烧(花钱买罪受?);第二,经济并不十分宽裕。判断依据是您最近时常想念的机身或镜头
的售价是您月收入的二至二十四倍。而且倍数越大,说明您发烧的度数越高。正是这两个特点,注定了我们广大的
影友发的发烧路无比的曲折。如果您用的是公家的顶级器材,或者觉着开始就买“顶级”机身加一套专业镜头并不是
很大负担的话,本篇文章肯定就不适用于您了。


大多数人在正式发烧之前都有一定的拍照经验,或用过给家庭留影的“傻瓜”相机,或使用过家里的老式相机,或者
借用过朋友的单反相机,甚至你本来就拥有一架带标准镜头的单反相机,但是从来没有想过后来会如此的痴迷。

\begin{wrapfigure}{r}{0.5\textwidth}
\vspace{-2ex}
\includegraphics[width=0.5\textwidth]{f3.jpg}
\caption{买了单反相机,就要开始这漫漫发烧路了}
\vspace{-2ex}
\end{wrapfigure}


驱动你踏上这条路的动力是多种多样的,或看到了一次摄影展,或看到一本摄影画册,或仅仅看到了朋友拍到的一
张风景,你好像一下就被感染了,就像得了流感一样。你开始想我是否也能“制造出”这样的相片。碰巧你刚从一个
旅游胜地归来。看着自己用“傻瓜”相机拍的惨不忍睹的照片,你暗下决心,一定要好好钻研一下“摄影”。


大凡关于摄影的指南都是这么说的。只用过135“傻瓜”相机的你觉着以前拍的照片不行的原因归结于没有135单反机。
买单反相机之前你要经过一番考察。这“考察”使你慢慢推开了相机世界的大门,各种品牌型号林林总总,你肯定会
感慨于这个世界的博大和你囊中的羞涩。

\begin{figure}
\begin{center}
\vspace{-2ex}
\includegraphics{f4.jpg}
\includegraphics{f5.jpg}
\caption{这时你发现名家的镜头要么比你焦距长,要么比你广角广}
\vspace{-2ex}
\end{center}
\end{figure}

经过了一番痛苦的抉择,你终于下定决心,把你能承受的最贵的一台国产或进口的带标准镜头或标准变焦镜头的或
手动对焦/曝光或自动对焦/曝光的单反相机请了回家(***本文尽量没有提到相机的品牌,但“品牌”有可能是在以后
使你苦恼的重要因素之一***)。你在不知不觉之中已把双脚踏上了这条发烧“不归路”。


在你用你新买的单反机拍了几卷胶卷以后,发现要出“作品”远不是有单反机就行那么简单。回过头来看名家的作
品,好像用的镜头要么是比你镜头广的广角镜头,要么是比你镜头焦距长的的长焦镜头。你恍然大误,原来如此。
这是你发烧历程中又一次飞跃(发烧温度的攀升)---你开始关注镜头了,虽然你现在看到的只是镜头的焦距这一项。

\begin{wrapfigure}{r}{0.5\textwidth}
\vspace{-2ex}
\includegraphics[width=0.5\textwidth]{f6.jpg}
\caption{大变焦比的镜头对新手来说似乎是个一劳永逸的选择}
\vspace{-2ex}
\end{wrapfigure}

你开始挖空心思要“凑”齐所有“常用”的焦距段。不幸的是当你在开始买相机时并没有考虑到这一点,所以刚刚“放完
血”的你更加苦恼于资金的不足。可配齐广角,中焦,长焦的迫切愿望是那么的强烈以至使你盯上了“便宜”的小口径
变焦头。这些变焦头使你用有限的资金完成你的计划。你甚至为找到了这些大变焦比的镜头感到洋洋得意。


果然,你在镜头上的投资没有白费,你的照片开始有别于你周围朋友拍的留念照了。虽然这可能仅仅是由于你镜头
广角端给人的视觉冲击或长焦端较浅的景深给你的照片带来的“与众不同”的感觉。你周围人对你的“作品”的认同和
夸赞无疑又给你增添了继续前进的动力。

再添一个机身的想法此时又滋生出来了。你说服自己的冠冕堂皇的理由是:

\begin{enumerate}
    \item 频繁更换镜头太麻烦
    \item 可同时使用不同的胶卷
    \item 有一个备用的机身以防万一
\end{enumerate}


你试图用这三个理由说服自己但又不愿意深究这三个理由是否成立。因为你心底深处真正的想法是要买一个更好的
机身(时下叫机身升级)。可能在你买第一个机身时怎么也想不到会烧到现在这个程度或者当时确实财政吃紧,总
之,你认为只有再买一个更高级的机身才能配齐你的“摄影系统”,而如果只有当前这一个机身的话肯定会妨碍你
出“作品”。


经过长时间的紧衣缩食,你终于又实现了你的这个愿望。有着两个机身的你终于有了点“专业”的自我感觉,特别是
跟影友们一起去“采风”的时候。可是新鲜感一过,你才发现:

\begin{enumerate}
    \item 同时带两个机身要远远比换镜头麻烦
    \item 你两个机身很少装不同的胶卷
    \item 好像只有胸前两个机身偶尔互相碰撞会让你感到心疼以外从来不用担心机身会有什么故障
\end{enumerate}

\begin{wrapfigure}{l}{0.3\textwidth}
\vspace{-2ex}
\includegraphics[width=0.3\textwidth]{f7.jpg}
\caption{专业摄影师都有2台或2台以上机身,不过人家也有摄影助理的}
\vspace{-2ex}
\end{wrapfigure}


慢慢地,你又开始只用一个机身了,当然是更高级的那个。原先的机身在摄影包的底层找到了自己的归宿。你原本
以为在器材上的投资已经结束了,可不幸的是这只是你花钱高潮的序曲。

随着你摄影水平的提高你开始有一些片子要放大一下。原本在五寸、七寸照片上看着很清楚的风景、人物,放大到
十二寸左右就难以满足你越来越挑剔的眼光了。你现在也知道了“镜头素质”是怎么回事,越来越注意自己拍的片子
色彩还原好不好、有没有暗角、逆光时眩光严不严重……你总是拿着五倍后来又换成十倍的放大镜看底片。你开始意
识到刚开始买镜头时只考虑焦距范围可能是最不成熟的行为。虽然你的片子不清晰很有可能仅仅是因为手震或聚焦
不实;你的片子色彩还原不好是因为扩片小姐刚跟男朋友分手心情太差,但你还是越来越不满足你那些“低素质”的变
焦镜头了。

\begin{figure}
\centering
\vspace{-2ex}
\begin{minipage}[t]{0.4\textwidth}
    \centering
    \includegraphics[width=1\textwidth]{f8.jpg}
    \caption{你开始越来越不满足你那些“低素质”的变焦镜头了}
\end{minipage}
\hspace{2em}
\begin{minipage}[t]{0.4\textwidth}
    \centering
    \includegraphics[width=1\textwidth]{f9.jpg}
    \caption{看着柜台中的恒定光圈大口径专业变焦镜头,你都快崩溃了}
\end{minipage}
\vspace{-2ex}
\end{figure}

这次你确实比原来成熟了,学会了在买镜头之前先找齐所有的资料,反复对比各项参数。还是由于资金的原因,你
不得不考虑性能价格比。虽然开始你仍然觉着没有必要买恒定光圈大口径的变焦镜头去追求那大一至两级的光圈,
但以拥有很多镜头资料的你后来慢慢发现恒定光圈大口径的的“专业”变焦镜头和那些“业余”镜头的差别不仅仅是大
一至两级的光圈,而是镜头光学素质(最起码是在镜头的数据上的)和机械做工的不可比性。这一点在镜头的重量和
价格上充分体现了出来。换句话说你觉着现在只有这种镜头才能满足你用十倍放大镜看底片这么高级别的需要。

“要买恒定光圈大口径专业变焦镜头”的想法刚冒出来的时候着实下了你一跳,因为这将意味着你要用两倍以上于你
花在机身上的资金投入去买镜头!这远远超出了你最初机身和镜头投入比例保持1:1的心理底线。况且这钱不是一下
能赚的来的……

\begin{wrapfigure}{l}{0.3\textwidth}
\vspace{-2ex}
\includegraphics[width=0.3\textwidth]{f10.jpg}
\caption{“严肃的摄影师都用定焦镜头”也是圈内流行的一种说法}
\vspace{-2ex}
\end{wrapfigure}

然而你现在“中毒”已经太深,你开始每天晚上睡觉都梦到那一两只镜头。经常在陪家人逛街的时候跑到摄影器材商
店,“趴在柜台上痴痴地看货架上的‘梦中情人’直到口水流出来”(一位坛友哼哼牛?的妙语)。你终于无法承受这一番
痛苦折磨,痛下决心,砸锅卖铁也要买下这镜头。你首先想到的是把你原来的镜头出手。可到了二手器材店你才知
道你悉心爱护下的足有九五成新的镜头店主最多能给你五成的价钱。你终于明白了“鸡肋”的确切含义。由于你太需
要钱了,虽然你心里痛骂着“奸商”但还是把镜头留给了他……

\begin{wrapfigure}{r}{0.3\textwidth}
\vspace{-2ex}
\includegraphics[width=0.3\textwidth]{f11.jpg}
\caption{现在你终于明白了为什么专业摄影人会用120相机}
\vspace{-2ex}
\end{wrapfigure}

当你终于把“梦中情人”拿到手上以后自然是无比的欣喜,虽然欣喜中带有一丝心痛。试拍报纸和砖墙的结果表明
这“高素质”的镜头果然不负众望,这使你心理得到了极大的安慰。可在以后的实拍中你越来越发现在80\%到90\%的
条件下这贵了几倍的镜头和你原来的“低素质”镜头并没有明显的差别。毕竟平时拍摄风景而且光照好时小光圈用的
多,聚焦不实的时候用什么镜头都不清楚,扩片小姐的心情仍然起伏不定……特别是当你扛着“梦中情人”翻过一个山
坡累的气喘吁吁时,你甚至开始怀念起你原来那套已属于别人的轻便的镜头来了。


你原来一直以为(特别是经过各种形式的广告的引导)变焦镜头是技术进步的结果,而且其素质已经赶超了原来的定
焦镜头。孰不知现在所有镜头厂家设计镜头时用的还是数十年前甚至上百年前的光学公式,而变焦镜头由于其镜片
组数多虽几经努力还不能做到像定焦镜头那样好。

再对比一下定焦镜头的价格,就算买几只镜头也要低于你那支大光圈的变焦镜头。你现在脑子里完全充斥着你的镜
头这个可能是唯一的但现在是你最介意的缺点。你越想越觉得你好像又犯了个错误。


定焦镜头,你真的需要了。虽然主要是在心理上…

中幅相机一般指的是120相机。在“国产经典双(镜头)反(光)相机”网站的首页有这样一段话:“你可能已拥有了一整
套相当不错的135相机,这是你花了很长时间建立的系统,因为你买每个镜头前都要考察一番。你又买了坚实的脚架
和云台。你使用遮光罩,快门线,反光镜锁……以保证像质。 你对你放大照片的结果非常满意。这时该发生的事情终
于发生了--你看到一张放大到n×n (n>24) 寸的 一张作品,你知道你没戏了。 It's big, it's sharp, it's
beautiful - 这就是中幅!

虽然你一年也没有几张要放大到24寸以上的照片,但这段话对发烧至此的你极具煽动性。你原来只知道120相机只有
专业人员和影楼才会用,现在你终于明白了为什么他们会用120相机。虽然120相机系统单比镜头的分辨率等参数赶
不上135相机的镜头,但其占据大底的优势。就连国字号的摄影杂志也发表了(在放大到一定尺寸以上时)国产的双
反120相机比世界上最顶级的德国某牌号135相机怎么怎么样的言论。

\begin{wrapfigure}{l}{0.3\textwidth}
\vspace{-2ex}
\includegraphics[width=0.3\textwidth]{f12.jpg}
\caption{领导和同事每天都用异样的眼神看着你包里的这套单反相机……}
\vspace{-2ex}
\end{wrapfigure}


虽然由于120相机的产量小,不具备135相机的规模效益但你还是可以有多种选择。从一二百块的旧国产双反,到动
辄上十万的欧洲顶级,你都可以量力而行,做出适合你的选择。而他们之间的差别从“像质”角度讲要小
于120和135两种系统之间的差别。

可当你真的投资买了中幅相机以后,你会发现比起135相机来讲120相机太麻烦了,还要负担比135高的多的胶卷和冲
扩费用。这就是为“追求像质”而付出的代价。

然而对“追求像质”是无止境的,比120底片还要大的多的是4×5, 比4×5底片大的多的 是8×10……这什么时候算到个头
啊?

虽然这条路上的每一步都伴随着痛苦,但走到现在你才理解什么是真正的痛苦:拥有这么多摄影器材的你每天还要
为生计忙碌,每天上班下班,好容易赶个节假日还要打理家务,串亲访友。能出门旅游的机会一年也没几次,即使
能出门了又很少是专为拍片。除了测试镜头外光剩空按快门了。真有英雄无用武之地的感觉。

终于你受一些大师的启发,找到了一个解决办法:何不抓拍你周围的人和事呢?只要出家门都能拍,没准哪天能赶上
个“决定性的瞬间”。好多大师不都是这么起家的吗?


说干就干!你打量了你那些摄影器材,大摄影包是不能带的;硕大的变焦头更不能带。看来只能用轻便的小广角或标
头了……

\begin{wrapfigure}{r}{0.3\textwidth}
\vspace{-2ex}
\includegraphics[width=0.3\textwidth]{f13.jpg}
\caption{永恒的大师布列松,成名全靠徕卡旁轴相机}
\vspace{-2ex}
\end{wrapfigure}

不幸的是你这种“创作”的状态没坚持几天就夭折了,因为领导和同事每天都用异样的眼神看着你包里的这套单反机。
虽然你只用了你最小体积的镜头,但你的领导和同事还是具有“单反相机就是专业相机”的认识水平的。不用开口你
就能从眼神里看出他们心理的疑惑“莫非这小子有跳槽的想法……?”


更让你觉得沮丧的是当你在上班途中好不容易碰到个“闪光点”,你的单反机总是引起被摄者的注意。落到底片上的
大多是别人对你镜头怒目而视的脸。

你终于明白为什么大师们都用旁轴相机了。



符合你条件的旁轴相机并不难找。可先如今大部分旁轴相机都是程序曝光的。你想要的有光圈优先曝光的旁轴机大
都价格不菲,随然现在的程序曝光功能的机子是从只有光圈优先曝光功能的机子进化来的。

无奈,你只有选择一台程序快门,有电动卷片,装有定焦或变焦镜头的旁轴相机。虽然别人还是称之为“傻瓜
机”,但你管它叫有自动功能的“旁轴相机”……

三脚架,你可能买到第三个才知道前两个的钱完全是应该省下的。黑白放大机,到时候你就会受人煽动,试图控
制“摄影”的全过程。幻灯机,反转片……

\begin{figure}
\vspace{-2ex}
\begin{minipage}[t]{0.5\linewidth}
\centering
\includegraphics[height=0.5\textwidth]{f14.jpg}
\caption{野外拍摄、夜景、风景等题材都需要一副结实的三脚架}
\end{minipage}%
\begin{minipage}[t]{0.5\linewidth}
\centering
\includegraphics[height=0.5\textwidth]{f15.jpg}
\caption{想玩黑白?想自己控制摄影全过程?那么你需要一个暗房}
\end{minipage}

\begin{minipage}[t]{0.5\linewidth}
\centering
\includegraphics[height=0.5\textwidth]{f16.jpg}
\caption{想玩幻灯片?你最后会被不知不觉地淹没在反转照片中}
\end{minipage}%
\begin{minipage}[t]{0.5\linewidth}
\centering
\includegraphics[height=0.5\textwidth]{f17.jpg}
\caption{烧到后来,你可能会需要一个自己的摄影棚}
\end{minipage}
\vspace{-2ex}
\end{figure}

静下心来想一想,算一算,你在摄影器材上的花费可能是你们家除了住房以外最大的花费了。虽然你在尽量克制你
的预算,可结果还会或多或少影响到了你家人的生活。你或许应该反省一下自从你有了这爱好以后,对家人的关爱
是否少了许多?或许你现在仍然无怨无悔,那祝愿你能更加理智些,认清器材对摄影的作用,一步一个脚印地走下
去。

你可能要对于正要上路的或没有陷的太深的影友建议别买变焦头焦一步就照3个定焦计划,Xitek也大声疾呼“便宜买
死人”,可他们没人能听的进去,就像当初的你一样。可能这发烧发的就是这个过程。

是谁说来着:“让一个男人破产的最简单的办法是,给他一架相机。” 

艺术家都是不容易赚到钱的。摄影艺术家陈冠希老师早年连专业相机都没有,用手机就拍出了那么牛的照片。作品
中简陋而温馨的环境下人物表情彷徨、迷离,眼神里流露出对现实生活的不惑而又执着的探寻幸福的真正意义,颇
有荒木经惟(Nobuyoshi Araki)摄影的神韵。纪实主义朴素风格拍摄的图片正是弱势群体艰辛生活的真实写照,记
录下了不朽的历史瞬间。

   
柏芝制服系列可以看作是由纪实主义手法向时尚表现主义转型的试验性作品,虽然布光略显不足,但是我们不难看
出其作品深受西方早期时尚摄影鼻祖——盖.伯丁(GUY BOURDIN)的影响。勤奋的冠希拍了诸多惊艳的照片,却不能
靠这个发财只能屈就于商业环境下作艺人谋生。这是时代的悲哀。

著名战地摄影大师卡帕曾经说过:“如果你拍的不够好,那是你离的不够近”。陈冠希永远手持手机战斗在床榻前。
   
冠希老师的作品的影响力,将持续久远.中国是摄影大国,很多人玩着器材,追着器材,如今,这种错误的观点,被冠希老
师的作品深深的教育了一道:好的作品,与器材无关!

\pagebreak

\renewcommand\listfigurename{附:本文插图}
\listoffigures

\end{document}
