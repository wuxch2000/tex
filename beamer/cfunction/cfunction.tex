\documentclass[hyperref={bookmarks=ture}xcolor=pdflatex,svgnames,table,compress]{beamer}

\usepackage{config}

\title{自定义封装函数的问题}
\author{}
\date{}

\begin{document}
\begin{frame}
\titlepage
\end{frame}

\section{基本情况}
\subsection{}
\begin{frame}
主要问题:
\begin{itemize}
    \item 函数参数定义问题
    \item 函数实现问题
\end{itemize}

\textit{以下举例说明}
\end{frame}

\section{函数参数定义问题}
\subsection{}
\begin{frame}[containsverbatim]{参数顺序不一致}
对于\lstinline{memcpy}:

\begin{itemize}
    \item SIP模块使用第{\color{red}3}个参数作为实际使用的复制大小:
\begin{lstlisting}[basicstyle=\ttfamily\scriptsize]
WORD32 P_SIPCA_memcpy(OUT VOID *pvDst,IN const VOID *pvSrc,'\hilight{IN WORD32 dwNeedCopyLen}',IN WORD32 dwDstMaxSize);
\end{lstlisting}
    \item MGCP/H248模块使用第{\color{red}4}个参数作为实际使用的复制大小:
\begin{lstlisting}[basicstyle=\ttfamily\scriptsize]
LPSTR PMGCP_MEMCPY(LPSTR pbDstBuffer,LPSTR pbSrcBuffer,WORD32 dwDestLen,'\hilight{WORD32 dwSrcLen}');
LPSTR PH248_MEMCPY(LPSTR pbDstBuffer,LPSTR pbSrcBuffer,WORD32 dwDestLen,'\hilight{WORD32 dwSrcLen}');
\end{lstlisting}
\end{itemize}
\comment{不一致的定义容易引发混淆}
\end{frame}


\subsection{}
\begin{frame}[containsverbatim]{参数定义与标准函数不一致}
对于\lstinline{memcpy}:

\begin{itemize}
    \item 使用\lstinline{WORD32}作为返回值,而不是缺省的目的指针地址:
\begin{lstlisting}
'\hilight{WORD32}' P_SIPCA_memcpy(OUT VOID *pvDst,IN const VOID *pvSrc,IN WORD32 dwNeedCopyLen,IN WORD32 dwDstMaxSize);
\end{lstlisting}
    \item 标准函数定义:
\begin{lstlisting}
'\hilight{void*}' memcpy (void* dstpp,const void* srcpp,size_t len);
\end{lstlisting}
\end{itemize}
\comment{如果使用返回值,容易出现错误}
\end{frame}

\subsection{}
\begin{frame}[containsverbatim]{参数定义与标准函数不一致}
对于\lstinline{memcpy}:

\begin{itemize}
    \item 使用\lstinline{LPSTR}作为参数类型,而不是缺省的void*:
\begin{lstlisting}
'\hilight{LPSTR}' PMGCP_MEMCPY('\hilight{LPSTR}' pbDstBuffer,'\hilight{LPSTR}' pbSrcBuffer,WORD32 dwDestLen,WORD32 dwSrcLen});
\end{lstlisting}
    \item 标准函数定义:
\begin{lstlisting}
'\hilight{void*}' memcpy ('\hilight{void*}' dstpp,const '\hilight{void*}' srcpp,size_t len);
\end{lstlisting}
\end{itemize}
\comment{这样会导致PCLINT对指针大小的判断失效}
\end{frame}

\subsection{}
\begin{frame}[containsverbatim]{参数定义与标准函数不一致}
对于\lstinline{strcpy}:

\begin{itemize}
    \item 封装函数:
\begin{lstlisting}[basicstyle=\ttfamily\tiny]
'\hilight{VOID}'  PH248_STRCPY(LPSTR strDes, LPSTR strS);
'\hilight{WORD16}' P_SIPCA_strcpy(OUT BYTE *pbDst, IN const BYTE *pbSrc, IN '\hilight{WORD16}' wMaxDstLen);
'\hilight{LPSTR}' PMGCP_STRCPY(LPSTR pbDstBuffer, LPSTR pbSrcBuffer, '\hilight{WORD32}' dwDestLen);
'\hilight{WORD16}' P_SIPCA_strncpy(OUT BYTE *pbDst, IN const BYTE *pbSrc, IN '\hilight{WORD16}' wNeedCopyLen, IN '\hilight{WORD16}' wMaxDstLen);
\end{lstlisting}
    \item 标准函数定义:
\begin{lstlisting}
char *strncpy (char *dest, const char*src,size_t n);
char *strcpy (char *dest, const char*src);
\end{lstlisting}
\end{itemize}
\comment{应该返回什么?到底用WORD16还是用WORD32。}
\end{frame}

\section{函数实现问题}
\subsection{}
\begin{frame}[containsverbatim]{封装不彻底}
对于\lstinline{strcpy}:

\begin{itemize}
    \item 封装函数:
\begin{lstlisting}
VOID  PH248_STRCPY(LPSTR strDes, LPSTR strS)
{
    if(strDes != NULL)
    {
        strcpy(strDes,strS);
    }
    return;
}
\end{lstlisting}
\end{itemize}
\comment{仅仅判断第一个指针? 另:此函数的注释是memcpy的注释}
\end{frame}

\subsection{}
\begin{frame}[containsverbatim]{内部实现欠妥}
对于\lstinline{memcpy}:

\begin{itemize}
    \item 封装函数:
\begin{lstlisting}[basicstyle=\ttfamily\scriptsize]
void* PH248_MEMCPY(void* pbDstBuffer, void* pbSrcBuffer, WORD32 dwDestLen, WORD32 dwSrcLen)
{
     ......
     if(dwDestLen < dwSrcLen)
     {
         /** chengliang added for agcf */
         '\hilight{memcpy(pbDstBuffer, pbSrcBuffer, dwDestLen);}'
         '\hilight{return NULL;}'
     }
     memcpy(pbDstBuffer, pbSrcBuffer, dwSrcLen);
     return pbDstBuffer;
}
\end{lstlisting}
\end{itemize}
\comment{返回null,表示函数执行失败,此时仍然执行复制了。\lstinline{PH248_MEMSET}一样。}
\end{frame}

\subsection{}
\begin{frame}[containsverbatim]{内部实现欠妥}
对于\lstinline{strncpy}:

\begin{itemize}
    \item 封装函数:
\begin{lstlisting}[basicstyle=\ttfamily\tiny,escapechar=@]
LPSTR PMGCP_STRCPY(LPSTR pbDstBuffer, LPSTR pbSrcBuffer, WORD32 dwDestLen)
{
    WORD32 dwSrcLen = 0;
    if(NULL == pbDstBuffer || NULL == pbSrcBuffer||0 == dwDestLen)
    {
        return NULL;
    }
    @\hilight{dwSrcLen = strlen(pbSrcBuffer);}@
    if(dwDestLen <= dwSrcLen)
    {
        strncpy(pbDstBuffer,pbSrcBuffer,dwDestLen);
        *(pbDstBuffer+dwDestLen -1) = '\0';
    }
    else
    {
         strncpy(pbDstBuffer, pbSrcBuffer,dwSrcLen);
         *(pbDstBuffer+dwSrcLen ) = '\0';
    }
    return pbDstBuffer;
}
\end{lstlisting}
\end{itemize}
\comment{且不说strlen额外进行了一次遍历,如果pbSrcBuffer没有结束符,strlen不安全}
\end{frame}

\subsection{}
\begin{frame}[containsverbatim]{无意义封装}
对于\lstinline{strlen}:

\begin{itemize}
    \item 封装函数:
\begin{lstlisting}
WORD16 P_SIPCA_strlen(IN const BYTE *pbString)
{
    WORD16 wRet;
    wRet = strlen((char*)pbString);
    return wRet;
}
\end{lstlisting}
\end{itemize}
\comment{仅仅是多一次调用}
\end{frame}

\section{后续建议}
\subsection{}
\begin{frame}
后续建议:
\begin{itemize}
    \item 参数和返回值定义请按照C标准库定义,可以增加参数,但是不要改变原有的参数定义
    \item 函数实现请参考标准库的源代码,或者是gnu libc的源代码。这些基本函数会被多次调用,有必要仔细
      推敲
    \item 不需要每个模块都定义一套实现,应该部门内部共享共用
    \item 函数名称改变需要修改pclint检查配置,否则会使pclint检查失效
\end{itemize}
\end{frame}


\end{document}
