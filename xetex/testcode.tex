\documentclass[12pt,a4paper,onecolumn]{article}
\title{TITLE}
\author{Wu_Xiaochun}
\date{}

\usepackage{fontspec,xunicode,xltxtra}
\usepackage{xeCJK}
\setmainfont[Mapping=tex-text]{Georgia}
\setsansfont[Mapping=tex-text]{Myriad Pro}
\setmonofont[Mapping=tex-text]{Courier New}
\setCJKmainfont{宋体}
\setCJKmonofont{黑体}

\usepackage[dvips,dvipsnames,svgnames]{xcolor}
\definecolor{light-gray}{gray}{0.95}

% 嵌入的代码显示
\usepackage{listings}
\lstset{language=C, breaklines, extendedchars=false}
\lstset{basicstyle=\ttfamily,
        % frame=single,
        frame=trBL,
        keywordstyle=\color{blue},
        commentstyle=\color{SeaGreen},
        stringstyle=\ttfamily,
        showstringspaces=false,
        backgroundcolor=\color{light-gray}}
\lstset{emphstyle={\color{red}}}


\begin{document}

\begin{footnotesize}
\lstset{emph={hdb_plat}}
\begin{lstlisting}
APP_RET create_r_rack( _db_t_database_handle hdb_plat )
{
    ret= _db_create_table(hdb_plat,
                        "R_RACK",
                        &hdb_r_rack); /* 创建关系表 */
    ......;
    return APP_SUCCESS;
}
\end{lstlisting}
\end{footnotesize}

\begin{footnotesize}
\begin{lstlisting}
typedef struct tagJID
{
    WORD32  dwJno;          /* JOB 号 */
    WORD32  dwDevId;        /* 设备ID */
    WORD16  wModule;        /* MP编号 */
    WORD16  wUnit;          /* 单板号 */
    BYTE    ucSUnit;        /* 单板上的处理器号 */
    BYTE    ucSubSystem;    /* 子系统号 */
    BYTE    ucRouteType;    /* 路由类型,区别左右板位或者主备 */
    BYTE    ucPad;          /* 填充字节 */
}JID;
\end{lstlisting}
\end{footnotesize}

\begin{footnotesize}
\begin{lstlisting}
APP_RET   load_sig_table_from_sync( )
{
    CHAR tableName[][_DB_TABLE_NAME_MAX]={ "r_employee", "r_office",...);
    WORD TableNum=2;
    _db_t_database_handle hdb_plat;
    if (_DB_SUCCESS != _db_load_data_from_sync(dbName, tableName, TableNum ))
    {
        return APP_ERROR;
    }
    return  APP_SUCCESS;
}
\end{lstlisting}
\end{footnotesize}
\pagebreak
\begin{footnotesize}
\begin{lstlisting}
 /*数据区单元编号占用18个bit,从1开始,最大为262143*/
#define DA_INDEX_MASK     (WORD32)0x0003ffff
 /*按低位0bit编号,实例号占用18bit到24bit*/
#define DA_INSTANCE_BIT   (BYTE )18
#define DA_INSTANCE_MASK   (WORD32)0x01fc0000    
 /*按低位0bit编号,数据区类型占用25bit到31bit*/
#define DA_TYPE_BIT       (BYTE )25
#define DA_TYPE_MASK             (WORD32)0xfe000000
/*根据CallID获取数据区单元编号*/
#define DA_GETINDEX(CallID)      (WORD32)((CallID) & DA_INDEX_MASK)
/*根据CallID获取数据区类型*/
#define DA_GETTYPE(CallID)    (WORD16)( ((CallID) & DA_TYPE_MASK) >> DA_TYPE_BIT )
/*根据CallID获取数据区所属实例号*/
#define DA_GETINSTANCE(CallID) (WORD16)( ((CallID) & DA_INSTANCE_MASK) >> \DA_INSTANCE_BIT )
/*根据数据区类型,数据区所属实例号,数据区单元编号合成CallID*/
#define DA_GETCALLID(DA_TYPE, DA_INSTANCE, DA_INDEX)         \
    ( ((DA_TYPE) << DA_TYPE_BIT) | \
    ((DA_INSTANCE) << DA_INSTANCE_BIT) | (DA_INDEX) )
\end{lstlisting}
\end{footnotesize}

\begin{footnotesize}
\begin{lstlisting}
typedef struct{
    DWORD pageNo;    /* 对象所在页面     */
    WORD  objectID;     /* 页面内部对象编号 */
    WORD  reserved;    /* 保留填充字节 */
}_PACKED_1_ _db_t_object_handle;
\end{lstlisting}
\end{footnotesize}

\pagebreak
\begin{footnotesize}
\begin{lstlisting}
#define USE_RET_CODE /* 说明需要使用返回结果标志 */
#define USE_DM_XXX /* 说明需要调用的接口 */
#include "dbaccess.h"" /* 引入数据库头文件”dbaccess.h” */

_UseDBDM_XXXDemo() /* 接口调用的函数 */
{
    /* 入参的申明 */    
    DM_XXX_REQ tReq;
    /* 出参的申明(对于异步调用、异步调用无回应可以不需要) */
    DM_XXX_ACK tAck; 

    /* 入参tReq的初始化 */
    
    /* 调用接口(对于异步调用,调用者进程需要对响应消息进行相应的处理) */
    dbAccess( DM_XXX, (LPSTR)&tReq,(LPSTR)&tAck ); 

    /* 出参tAck的处理(对于异步调用、异步调用无回应可以不需要) */

}
\end{lstlisting}
\end{footnotesize}

\end{document}
