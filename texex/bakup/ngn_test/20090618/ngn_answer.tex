\documentclass  [11pt,onecolumn,a4paper]{article}
\title{NGN C Test}
\author{}
\date{}

\usepackage[a4paper]{geometry}
\geometry{left=3em}
% \geometry{columnsep=15em}
\geometry{right=3em}

\usepackage{fontspec,xunicode,xltxtra}
% \setmainfont[Mapping=tex-text]{Times New Roman}
\setmainfont[Mapping=tex-text]{Arial}
\setsansfont[Mapping=tex-text]{Arial}
% \setmonofont[Mapping=tex-text]{Courier New}
\setmonofont[Mapping=tex-text]{Times New Roman}

\usepackage{xeCJK}
% \setCJKmainfont[ItalicFont={Adobe Kaiti Std}]{Adobe Song Std}
% \setCJKmainfont[ItalicFont={Adobe Kaiti Std}]{Adobe Kaiti Std}
\setCJKmainfont[ItalicFont={Adobe Kaiti Std}]{Adobe Heiti Std}
\setCJKsansfont{Adobe Heiti Std}
% \setCJKsansfont{Microsoft YaHei}
\setCJKmonofont{Adobe Heiti Std}
\punctstyle{banjiao}

\usepackage{calc}
\usepackage[]{geometry}
% \geometry{paperwidth=221mm,paperheight=148.5mm}
% \geometry{paperwidth=9.309in,paperheight=6.982in}
\geometry{paperwidth=7.2cm,paperheight=10.8cm}
% \geometry{twocolumn}
\geometry{left=5mm,right=5mm}
\geometry{top=5mm,bottom=5mm,foot=5mm}
% \geometry{columnsep=10mm}
\setlength{\emergencystretch}{3em}


\usepackage{indentfirst}

%生成PDF的链接
\usepackage{hyperref}
\hypersetup{
    % bookmarks=true,         % show bookmarks bar?
    bookmarksopen=true,
    pdfpagemode=UseNone,    % options: UseNode, UseThumbs, UseOutlines, FullScreen
    pdfstartview=FitB,
    pdfborder=1,
    pdfhighlight=/P,
    pdfauthor={wuxch},
    unicode=true,           % non-Latin characters in Acrobat’s bookmarks
    colorlinks,             % false: boxed links; true: colored links
    linkcolor=blue,         % color of internal links
    citecolor=blue,        % color of links to bibliography
    filecolor=magenta,      % color of file links
    urlcolor=cyan           % color of external links
}
\makeindex

\usepackage[dvips,dvipsnames,svgnames]{xcolor}
\definecolor{light-gray}{gray}{0.95}

\usepackage{graphicx}
\usepackage{wrapfig}
\usepackage{picinpar}

\renewcommand\contentsname{目录}
\renewcommand\listfigurename{插图}
\renewcommand\listtablename{表格}
\renewcommand\indexname{索引}
\renewcommand\figurename{图}
\renewcommand\tablename{表}

\usepackage{caption}
\renewcommand{\captionfont}{\scriptsize \sffamily}
\setlength{\abovecaptionskip}{0pt}
\setlength{\belowcaptionskip}{0pt}

\graphicspath{{fig/}}

\usepackage{fancyhdr}

% \usepackage{lastpage}
% \cfoot{\thepage\ of \pageref{LastPage}}

% 嵌入的代码显示
% \usepackage{listings}
% \lstset{language=C++, breaklines, extendedchars=false}
% \lstset{basicstyle=\ttfamily,
%         frame=single,
%         keywordstyle=\color{blue},
%         commentstyle=\color{SeaGreen},
%         stringstyle=\ttfamily,
%         showstringspaces=false,
%         tabsize=4,
%         backgroundcolor=\color{light-gray}}

\usepackage[sf]{titlesec}
\titleformat{\section}{\normalsize\sffamily\bf\color{blue}}{\textsection~\thesection}{.1em}{}
\titleformat{\subsection}{\normalsize\sffamily}{\thesubsection}{.1em}{}
\titlespacing*{\section}{0pt}{1ex}{1ex}
\titlespacing*{\subsection}{0pt}{0.2ex}{0.2ex}

\usepackage{fancyhdr}
\usepackage{lastpage}
\fancyhf{}
\lhead{}
\rhead{}
\chead{\scriptsize{\textsf{蜗居}}}
\cfoot{\scriptsize{\textsf{第 \thepage ~页,共 \pageref*{LastPage} 页}}}


% \usepackage{enumitem}
% \setitemize{label=$\bullet$,leftmargin=3em,noitemsep,topsep=0pt,parsep=0pt}
% \setenumerate{leftmargin=3em,noitemsep,topsep=0pt,parsep=0pt}

% \setlength{\parskip}{1.5ex plus 0.5ex minus 0.2ex}
\setlength{\parskip}{2.0ex plus 0.5ex minus 0.2ex}

% \setlength{\parindent}{5ex}
\setlength{\parindent}{0ex}

% \usepackage{setspace}
\linespread{1.25}

% 英文的破折号--不明显,使用自己画的线。
\newcommand{\myrule}{\hspace{0.5em}\rule[3pt]{1.6em}{0.3mm}\hspace{0.5em}}


\begin{document}
\begin{center}
\begin{minipage}{0.8\linewidth}
\blk{部门}\blk{科室}\blk{姓名}\blk{工号}
\end{minipage}
\end{center}

\begin{flushright}
{\small{\textit{考试时长:3小时,总分:100分}}}
\end{flushright}

\normalsize{\sffamily{试题背景说明:}}

\begin{enumerate}
    \item 基本数据类型定义:
\begin{lstlisting}
typedef  char   *               LPSTR;
typedef  unsigned  char   **    LPLPSTR;
typedef  signed    char         CHAR;
typedef  unsigned  char         BOOL8;
typedef  unsigned  char         BYTE;
typedef  unsigned  short        WORD;
typedef  unsigned  long         DWORD;
typedef  unsigned  short        WORD;
typedef  unsigned  long         DWORD;
\end{lstlisting}
    \item 数据接口调用函数原型说明:
\begin{lstlisting}
void dbCall( WORD wEvent, LPSTR pReq, LPSTR ptAck );
\end{lstlisting}
其中:\lstinline{wEvent}表示事件号,\lstinline{pReq}表示入参结构体指针,\lstinline{ptAck}表示出参结构
体指针。

入参结构体的定义一般如下:
\begin{lstlisting}
typedef struct 
{
    BYTE      bMsgType;  /*消息类型:同步调用或异步调用,调用者必须填写该参数*/
    ......
}D_XXX_REQ, * LPD_XXXX_REQ;
typedef struct
{
    WORD     wRetCode;     /*接口调用结果-返回成功或者失败码,接口必须返回该参数的,供调用者使用。 */
    ......
}D_XXX_ACK,* LPD_XXX_ACK;
\end{lstlisting}
\end{enumerate}

\section{改错题(共80分,每道题目至少1处错误或者隐患)}
\subsection{请找出下面代码中的隐患或者错误,说明故障原因并改正。(5分)}
\begin{lstlisting}[escapeinside=`']
typedef  CHAR  *  PCHAR;  
#define  MAX_TABLE_NUN  512;
BYTE  _CreateTable( LPSTR dbName, DWORD dbModule, DWORD dbCapacity, ... )
{
    PCHAR       param;
    va_list        ap;
    `BYTE        dmNum = 0;     循环变量DWORD 2分'
    if (dbName == NULL)
    {
        return 0;
    }
    va_start( ap, dbCapacity );
    while ( TRUE && dmNum < MAX_TABLE_NUN )
    {
        param = va_arg( ap, PCHAR );
        if ( !param[0] ) `------ if ( param == NULL||  !param[0] ) 3分'
        {
            break;
        }
        dmNum ++;
    }
    va_end( ap ); 
    return  1;
}
\end{lstlisting}
\subsection{请找出下面代码中的隐患或者错误,说明故障原因并改正。(5分)}
\begin{lstlisting}[escapeinside=`']
WORD PlanCtx_PlayTone_Body( BYTE ucAwakeReason, T_PlayToneBody *ptPlayToneBody, BYTE *pbCurStep,  BYTE *pbCache)
{
    WORD wResult             = PLAN_EXCEPTION;
    BYTE   bNeedSuspend  = FALSE;
    BYTE   ucSuspendStep = PLAN_STEP_1;

    if((ptPlayToneBody == NULL) || (pbCurStep == NULL)` || (pbCache == NULL )) ------ 合法性检查缺少对pbCache的检查 1分'
    {
        return wResult;
    }
    wResult = PLAN_EXCEPTION;    
    while(!bNeedSuspend)
    {
        switch(*pbCurStep)
        {
            case PLAN_STEP_1:  /* step 1 */
            {
                if(PLANSERV_OK != PlayToneStep1(ucAwakeReason, ptPlayToneBody, &wResult))
                {
                    wResult      = PLAN_EXCEPTION;
                    bNeedSuspend = TRUE;
                }
                else if( bNeedSuspend )
                {
                    *pbCurStep   = PLAN_STEP_2;
                    wResult      = PLAN_SUSPEND;
                    bNeedSuspend = TRUE;
                }
                else
                {
                    *pbCurStep = PLAN_STEP_2;
                }
                break;
            }    
            case PLAN_STEP_2:  /* step 2 */
            {
                if(PLANSERV_OK != PlayToneStep2(ucAwakeReason, ptPlayToneBody,&ucSuspendStep, &wResult))
                {
                   `wResult = PLAN\_EXCEPTION;  ------ 死循环  4分,return;'
                }
                else if (bNeedSuspend)
                {
                    *pbCurStep   = ucSuspendStep;
                    wResult      = PLAN_SUSPEND;
                    bNeedSuspend = TRUE;                
                }
                else
                {
                    *pbCurStep = PLAN_STEP_3;
                }
                break;
            }    
            case PLAN_STEP_3:  /* step 3 */
            {
                if(PLANSERV_OK != PlayToneStep3(ucAwakeReason, ptPlayToneBody,&wResult))
                {
                    return PLAN_EXCEPTION;
                }
                else
                {
                    return PLAN_FINISH_OK;
                }
                break;
            }    
            default:
            {
                return PLAN_EXCEPTION;
                break;
            }
        }    
    }
    return wResult;
}
\end{lstlisting}
\subsection{请找出下面代码中的隐患或者错误,说明故障原因并改正。(8分)}
\begin{lstlisting}[escapeinside=`']
#define HAVE_SEND_DEASSIGNCALL     (WORD)0x00040000 
`------ \#define HAVE\_SEND\_DEASSIGNCALL     (DWORD)0x00040000  2分'
extern DWORD brs_recvmsg (INT32 iSockfd, CHAR *pcMsg, INT32 iFlags);

DWORD BRS_RecvMsg_wrap(SOCK_MSG * ptSocketIoMsg, WORD * pwDataLen)
{
    BRS_RECV_MSG  tBrsSocRecvMsg;   

    if((ptSocketIoMsg == NULL) || (pwDataLen == NULL))
    {
        return 0;
    } 
    memset(tBrsSocRecvMsg,0,sizeof(tBrsSocRecvMsg));
    *pwDataLen = brs_recvmsg(ptSocketIoMsg->s, (char *)&tBrsSocRecvMsg, 0);
    `*pwDataLen = brs\_recvmsg(ptSocketIoMsg-$>$s, (char *)\&tBrsSocRecvMsg, 0);  类型不匹配,pwDtaLen类型修改为DWORD * 3分'
    if ( 0 != (*pwDataLen & 0x80000000) )
    {
         /*将对应的标记位清0*/
        ptSocketIoMsg->dwScmFlag =  ptSocketIoMsg->dwScmFlag & !HAVE_SEND_DEASSIGNCALL;  
        `ptSocketIoMsg-$>$dwScmFlag =  ptSocketIoMsg-$>$dwScmFlag \& ~HAVE\_SEND\_DEASSIGNCALL;  1分'
        printf("\nIpfun:Receive Data from brs_recvmsg error: %ld in gIpFunReadSocketData! In %s(%s)",*pwDataLen, __FILE_,__LINE__);
        `printf("\verb|\n|Ipfun:Receive Data from brs\_recvmsg error: \%ld in gIpFunReadSocketData! In \%s(\%lu)",*pwDataLen, \_\_FILE\_,\_\_LINE\_\_); 1分'
        return 1;
    }
    return 0;
}
\end{lstlisting}
\subsection{请找出下面代码中的隐患或者错误,说明故障原因并改正。(8分)}
\begin{lstlisting}
#define  URI_LTH    128
typedef  BYTE       SZ_PUI[URI_LTH];
/* 定义数据表的关键字合成函数 */
#define R_PUI_PKEY(DB, PKEY, PUI)  \
     _db_merge_index_key((DB), hidx_r_pui, PKEY, sizeof(PUI), (PUI))  `------ 3分   ,括号1分,sizeof(PUI) 2分'
#define  R_EMERGENCY_PKEY(DB,PKEY,EMGURI)  \         `------ 3分  ,括号1分,sizeof(URI_LTH) 2分'
     _db_merge_index_key((DB),hidx_r_emergency,PKEY,\
                                 sizeof(URI_LTH),(BYTE *)&(EMGURI))
DBBOOL _dbGetPuiTupleInfo(SZ_PUI   tPui  , _db_t_tuple_handle *ptPuiHandle)
{
    TUPLEKEY  tTupleKey;
    _DB_RET   ret;
    _db_t_tuple_handle tImplsetHandle;
    if( NULL ==ptPuiHandle)
    {
        return FALSE;
    }

    `------ 缺少对tPui的合法性检查  2分'

    /* 定位表R_PUI */
    R_PUI_PKEY(hdb_cscf_proc, tTupleKey, tPui);/*合成关键字*/
    /*根据关键字定位记录*/    
    ret = _db_locate_tuple(hdb_cscf_proc, hidx_r_pui, ptPuiHandle,tTupleKey);
    if( _DB_FOUND != ret )
    {
        return FALSE;
    }
    return TRUE;
}
DBBOOL  _func_judgeisemergency( CHAR  *pchEmgUri, BYTE *pbIsEmg )
{
    _DB_RET                  ret;
    TUPLE_KEY_APP            tuplekey;
    _db_t_tuple_handle       hEmgTuple;
    if( NULL == pchEmgUri  || NULL == pbIsEmg)
    {
        return FALSE;  
    }   
    *pbIsEmg = 0;   
    R_EMERGENCY_PKEY( hdb_pss_config, tuplekey, pchEmgUri ); /*合成关键字*/
    /*根据关键字定位记录*/
    ret = _db_locate_tuple( hdb_pss_config, hidx_r_emergency, 
                                          &hEmgTuple, tuplekey); 
    if (_DB_FOUND == ret)
    {
        *pbIsEmg = 1;
        return TRUE;
    }
    return FALSE;  
}
\end{lstlisting}
\subsection{请找出下面代码中的隐患或者错误,说明故障原因并改正。(7分)}
\begin{lstlisting}[escapeinside=`']
typedef struct
{
    ......
} P_RECV_INFO_T,* P_RECV_INFO_PT;

WORD P_GetIdleRecvBuf(P_RECV_INFO_PT  ptIdleBuf)
{
    DWORD dwI = 0;
    time_t tNowTime;
    time(&tNowTime); /* 获取当前时钟 */
    ptIdleBuf = NULL;
    for(dwI = 0; dwI < UDT_MAX_RECV_BUF_NUM; dwI++)
    {
        /* 当前缓冲区为空闲 或者 已经超时 */
        if( (UDT_BUFFER_IDLE == g_DBS_UDT_Global.tRecvInf[dwI].bStatus) ||
            ((time_t)(g_DBS_UDT_Global.tRecvInf[dwI].tRecvTime + g_DBS_UDT_Global.tRecvInf[dwI].dwValidPeriod) <= tNowTime))
        {
            ptIdleBuf = &(g_DBS_UDT_Global.tRecvInf[dwI]);
            return UDT_SUCCESS;
        }
    }
    return UDT_FAILURE;    
}

`指出值传递

WORD D\_UDT\_GetIdleRecvBuf(D\_UDT\_RECV\_INFO\_PT  * ptIdleBuf)   4分

*ptIdleBuf = \&(g\_DBS\_UDT\_Global.tRecvInf$[$dwI$]$);'

WORD P_SyncDataReqProc(char *  pchBufferAddr)
{
    P_SYNC_MSG_PT  ptReqMsg = NULL;
    P_RECV_INFO_PT ptBuf = NULL; 
    WORD wRet  = UDT_SUCCESS;
    
    `ptReqMsg = (D\_UDT\_SYNC\_MSG\_PT)pchBufferAddr;   合法性检查1分'
    
    ptReqMsg = (P_SYNC_MSG_PT)pchBufferAddr;
    if (ptReqMsg->tMsgHead.bPackageNo == 1)
    {
        /* 清空接收缓冲区中default pui的历史记录  */
        P_CleanBufByDef(ptReqMsg->tMsgHead.dwDefPuiId); 
        wRet = P_GetIdleRecvBuf(ptBuf); /* 获取空闲buf */
        if (UDT_SUCCESS != wRet )
        {
            return UDT_FAILURE;
        }

`       if (UDT\_SUCCESS != wRet || ptBuf == NULL)  

------  因依赖于getidelrevbuf的函数说明,有的指针空则不可能返回成功,因此ptbuf==null不作为扣分项。'

        ptBuf->bStatus      = UDT_BUFFER_USED;
        ptBuf->dwUcVersion	= ptReqMsg->tMsgHead.dwUcVersion;
        ptBuf->dwDefPuiId   = ptReqMsg->tMsgHead.dwDefPuiId;
        ptBuf->dwMsgLenSum  = ptReqMsg->tMsgHead.dwMsgLenSum;
        ptBuf->dwCurPos     = ptReqMsg->tMsgHead.dwLength;
        memcpy(ptBuf->pchMsgBuf,ptReqMsg->bMsgBuf,
               ptReqMsg->tMsgHead.dwLength);  
`        if ptReqMsg-$>$tMsgHead.dwLength $<$= sizeof(ptBuf-$>$pchMsgBuf)  memcpy合法性保护 2分'
    }
    return UDT_SUCCESS;
}
\end{lstlisting}
\subsection{请找出下面代码中的隐患或者错误,说明故障原因并改正。(8分)}
\begin{lstlisting}[escapeinside=`']
void  smM2SProc(DB_AREA_STAT * pDbList)
{
    T_SM_DATA_AREA *ptSmInst = NULL;
    WORD wRet = PDF_ERROR;
    `BYTE  i = 0;  ------ 循环变量类型定义,2分'

    if (pDbList == NULL)
    {
        return;
    }
    /*遍历所有实例数据区*/	
    ptSmInst = (T_SM_DATA_AREA *)smGetNextUsedUB(T_UB_SM_INST, NULL);	
    while(NULL != ptSmInst)
    {
        if(SM_DATA_NOT_USED == ptSmInst->bUsedFlag)
        {
            continue;
            `------ 死循环2分;增加  ptSmInst = (T\_SM\_DATA\_AREA *)smGetNextUsedUB(T\_UB\_SM\_INST, (BYTE *)ptSmInst);/*找下一个实例数据区*/'
        }	 
        if(SM_TRANS_ACTIVE == ptSmInst->tTransInfo[i].bActive &&
           i <= MAX_UB_INST_NUM)   
` ------ 合法性检查错误,调整顺序!,另外不能等于。 4分,合法性检查、顺序各自一半分数。

if(i $<$ MAX\_UB\_INST\_NUM  \&\& SM\_TRANS\_ACTIVE == ptSmInst-$>$tTransInfo$[$i$]$.bActive ) '
        {            
            pDbList-> dwIndex [i] = ptSmInst->dwIndex;
            i++; 
            wRet = CMSResBackup(g_dwSMInstResHandle,
                            ptSmInst->dwIndex, emCmsSyncOpRelease);
            if(PDF_SUCCESS != wRet)
            {
                return;
            }
        }
        /*找下一个实例数据区*/
        ptSmInst = (T_SM_DATA_AREA *)smGetNextUsedUB(
                                    T_UB_SM_INST, (BYTE *)ptSmInst);
    }
}
\end{lstlisting}
\subsection{请找出下面代码中的隐患或者错误,说明故障原因并改正。(5分)}
\begin{lstlisting}[escapeinside=`']
typedef struct
{  
	D_TRANSFER  tCtrlPuiInfo [MAX_NUM];
     ......
} P_CTRL_INFO_T;
P_CTRL_INFO_T       g_DBS_UDT_CtrlInfo;

WORD P_CTRL_SyncEndReqProc(D_TRANSFER_END_REQ_PT ptReq)
{
    WORD wResult        = UDT_SUCCESS;
    DWORD dwCtrlCardIdx = 0;
    DWORD i             = 0;
    DWORD dwPos         = 0; 
    D_TRANSFER_END_APPEND_PT ptTransEndAppd = NULL;
   
    if (ptReq == NULL )
    {
        return UDT_FAILURE;
    }
    /* 所有迁出板尚未迁移完毕,防止迁移结束后处理重复的迁移结束消息 */
    if (g_DBS_UDT_CtrlInfo.bAllOutEnd != 1)
    {       
        /* 判断是否有附加消息,只有指定用户迁移有附加消息 */
        if (ptReq->dwAppendNum != 0)
        {
            g_DBS_UDT_CtrlInfo.dwCtrlPuiNum  = ptReq->dwAppendNum;
            memset(g_DBS_UDT_CtrlInfo.tCtrlPuiInfo,0,sizeof(g_DBS_UDT_CtrlInfo.tCtrlPuiInfo));
            dwPos = sizeof (D_TRANSFER_END_REQ_T);
            for (i = 0; i< ptReq->dwAppendNum; i++)
            `for (i = 0; i$<$ ptReq-$>$dwAppendNum \&\& i $<$MAX\_NUM; i++)   2分'
            {
                ptTransEndAppd = (D_TRANSFER_END_APPEND_PT)(ptReq + dwPos); 
                `ptTransEndAppd = (D\_TRANSFER\_END\_APPEND\_PT)((char *)ptReq + dwPos); 3分'
                dwPos = dwPos + sizeof(D_TRANSFER_END_APPEND_T); `跟上面一句的逻辑是有矛盾的,修改下面一个也可以。'
                g_DBS_UDT_CtrlInfo.tCtrlPuiInfo[i].dwPuiId      = ptTransEndAppd->dwPuiId;
                g_DBS_UDT_CtrlInfo.tCtrlPuiInfo[i].bTransferTag = ptTransEndAppd->bTransferTag;
                g_DBS_UDT_CtrlInfo.tCtrlPuiInfo[i].bCallTag     = ptTransEndAppd->bCallTag;
            }
        }       
        /* 置迁移结束标志 */
        g_DBS_UDT_CtrlInfo.tCtrlCardInfo[dwCtrlCardIdx].bEndTag = 1;   
    }      
    return UDT_SUCCESS;
}
\end{lstlisting}
\subsection{请找出下面代码中的隐患或者错误,说明故障原因并改正。(10分)}
\begin{lstlisting}[escapeinside=`']
#define MAXCAP  380
#define DB_SDP_LTH    (WORD)1024
#define SDP_CHAR_MAX   ( WORD )255

typedef struct 
{
    WORD	  wRetCode;
	WORD  	  wEventnum;							 	
    CHAR      chLocalSDP[SDP_CHAR_MAX]; 
    ......
} DB_GETTERMPROPERTY_ACK,*LP_ DB_GETTERMPROPERTY_ACK;
typedef struct 
{
    BYTE     bMsgType;          
    BYTE     bAction;             
    WORD     wSDPLen;            
    SDP_HEAD tSDPHead;   		
    CHAR     chSDP[DB_SDP_LTH]; 
} DB_SDPCOMPRESS_REQ,* LP_DB_SDPCOMPRESS_REQ;
typedef struct
{
    WORD		wRetCode;			
    WORD		wSDPLen;			
    SDP_HEAD tSDPHead;			
    CHAR		chSDP[DB_SDP_LTH];						
} DB_SDPCOMPRESS_ACK,* LP_DB_SDPCOMPRESS_ACK;

BYTE C_BCM_StrLen(BYTE* pbNumber,BYTE bMaxNumberLen)
{
	BYTE bNumLen = 0;
	if (NULL == pbNumber)
	{
        bNumLen = 0;
	}
	else
	{
		bNumLen = strlen((CHAR *) pbNumber);
        if (bNumLen > bMaxNumberLen)
        {
            bNumLen = bMaxNumberLen;
        }
    }
	return  bNumLen;
}
void C_BCM_DbGetTermProperty(CCB *ptccb,WORD * pwMediaScriptLen, BYTE *pbMediaScript, WORD wProperId,BOOL8 bUseDefMedia)
{
	DB_GETTERMPROPERTY_REQ tGetTermPropertyReq;
	DB_GETTERMPROPERTY_ACK tGetTermPropertyAck;
	DB_SDPCOMPRESS_REQ     tSdpCompressReq;
	DB_SDPCOMPRESS_ACK     tSdpCompressAck;
	WORD wRes = 0;
	char chMedia[] = {
		0x00,0x00,0x00,0x00,0x00,0x00,0x00,0x00,0x76,0x3D,0x30,
        0x0D,0x0A,0x63,0x3D,0x49,0x4E,0x20,0x49,0x50,0x34,0x20,
        0x31,0x30,0x2E,0x31,0x30,0x2E,0x31,0x30,0x2E,0x31,0x30,
        0x0D,0x0A,0x6D,0x3D,0x61,0x75,0x64,0x69,0x6F,0x20,0x31,
        0x30,0x30,0x30,0x20,0x52,0x54,0x50,0x2F,0x41,0x56,0x50,
        0x20,0x30,0x20,0x38,0x20,0x31,0x38,0x0D,0x0A,0x61,0x3D,
        0x70,0x74,0x69,0x6D,0x65,0x3A,0x32,0x30,0x0D,0x0A};

	if((NULL == ptccb)||(NULL == pbMediaScript)||
       (NULL == pwMediaScriptLen))
	{
		return wRes;
	}
        `memset(pbMediaScript, 0, MAXCAPS); -----初始化2分或者在C\_BCM\_O\_CreateCallTest 中初始化!'
	if(wProperId)/*如果为0则直接考虑取默认的*/
	{
              `memset(\&tGetTermPropertyReq, 0, sizeof(DB\_GETTERMPROPERTY\_REQ));  初始化1分'
		tGetTermPropertyReq.bMsgType = MSGTYPE_SYNC;
        	tGetTermPropertyReq.wProperId = wProperId;
		dbCall(EV_PD_GetTermProperty, (LPSTR) & tGetTermPropertyReq,
               (LPSTR) & tGetTermPropertyAck);
		if(tGetTermPropertyAck.wRetCode == RC_OK)
		{
                    `memset(\&tSdpCompressReq, 0, sizeof(DB\_SDPCOMPRESS\_REQ)); //初始化1分'
			tSdpCompressReq.bMsgType = MSGTYPE_CALL;
			tSdpCompressReq.bAction = DB_SDP_COMPRESS;
			tSdpCompressReq.wSDPLen =
                min(strlen(tGetTermPropertyAck.chLocalSDP),
                    MAXCAPS); /* SDP信息长度 */
                    `tSdpCompressReq.wSDPLen = min(strlen(tGetTermPropertyAck.chLocalSDP), DB\_SDP\_LTH);   1分'

			memcpy(tSdpCompressReq.chSDP ,
                   tGetTermPropertyAck.chLocalSDP,
                   tSdpCompressReq.wSDPLen);
			dbCall(EV_PD_SDPCOMPRESS, (LPSTR) & tSdpCompressReq,
                   (LPSTR) & tSdpCompressAck);
			if(tSdpCompressAck.wRetCode == RC_OK)
			{
				wRes = tSdpCompressAck.wSDPLen;
                            `if(wRes > MAXCAPS)  ---保护2分

				{

					wRes = MAXCAPS;

				}'
				memcpy(pbMediaScript, tSdpCompressAck.chSDP,wRes);
			}
			else
			{
                wRes = C_BCM_StrLen(tGetTermPropertyAck.chLocalSDP,
                                    (BYTE)MAXCAPS); /* SDP信息长度 */
`\verb|----- wRes = min(strlen(tGetTermPropertyAck.chLocalSDP), MAXCAPS); /* SDP信息长度 */ 3分|
\verb|或者将函数|
\verb|BYTE C_BCM_StrLen(BYTE* pbNumber,BYTE bMaxNumberLen) |
\verb|修改为:|
\verb|WORD C_BCM_StrLen(BYTE* pbNumber, WORD bMaxNumberLen)|'

		memcpy(pbMediaScript ,
                       tGetTermPropertyAck.chLocalSDP,
                       wRes);
			}
		}
	}
	if((wRes == 0) && bUseDefMedia)
	{
		wRes = min(sizeof(chMedia), MAXCAPS);
		memcpy(pbMediaScript, chMedia, wRes);
	}
	* pwMediaScriptLen = wRes;
}
void C_BCM_O_CreateCallTest(CCB* ptccb, WORD wProperId)
{
`\verb|    WORD wLctLgth ;|
\verb|    BYTE bMediaScript[MAXCAP];|'
    
    C_BCM_DbGetTermProperty(ptccb,&wLctLgth, bMediaScript, wProperId,
                            TRUE);
    ......
}
\end{lstlisting}
\pagebreak
\subsection{请找出下面代码中的隐患或者错误,说明故障原因并改正。(8分)}
\begin{lstlisting}[escapeinside=`']
void ProxyCSM_ST_WORK_takeEvent(ProxyCSM* const me, short  id ,otProxy_Para pParams )
{
    if (me == NULL || pParams == NULL)
    {
        return ;
    }
    switch (id)
    {
        case UP_EV_PPLAT_DNS_RESPONSE:
        {
            DWORD  dwL3CallID = pParams ->tDNSResp.dwAppID; 
            DWORD  dwIndex = 0;
            PP_PROXY_DATA*  ptDataArea;   
            do
            {   
                if( P_FindDataAreaEx( me->tDBFaced,dwL3CallID,
                    (BYTE**)&ptDataArea ) )/*申请主数据区*/
                {
                    if( P_FindHash( me->tDBFaced,
                      dwL3CallID,&dwIndex) )/*申请HASH数据区*/
                    {
                        FROM  tFrom;
                        FROM  tTo;
`\verb|if( dwIndex >= PP_PROXY_INSTANCE_NUM)  合法性检查4分|
\verb|          {|
\verb|              DBFaced_P_DelHash( me->tDBFaced,dwL3CallID );|
\verb|              DBFaced_P_DelDataArea( me->tDBFaced,dwL3CallID );|
\verb|              break;|
\verb|          }|'
                        if(!P_PP_DecodeFromTo( &tFrom,&tTo,
                                               ptDataArea->abDialogState ))
`\verb|if(!P_PP_DecodeFromTo( &tFrom,&tTo,ptDataArea->abDialogState )) ptDataArea的合法性检查 2分|'
                        {
                            ProxyCSM_SetDBFaceTraceByURL( me, &tFrom, &tTo );
                        }                             
                        /* send event to proxy */ 
                        if( pParams ->tDNSResp.dwErrCode == 0 )
                        {
                            Proxy_EV_DNS_OK( &me->itsProxy[dwIndex],
                                             &pParams ->tDNSResp);
                        }
                        else
                        {
                          Proxy_EV_DNS_FAIL(&me->itsProxy[dwIndex],
                                            &pParams ->tDNSResp);
                        }

                        Proxy_EV_RESET_DNS_TIMER(&me->itsProxy[dwIndex]);
                    }
                    else
                    {
`\verb|             DBFaced_P_DelDataArea( me->tDBFaced,dwL3CallID );  忘记释放数据区 2分。|'
                        return;
                    }
                }
                else
                {
                  psyslog(LOG_WARNING,"PP ERROR: cannot find data area when receive UP_EV_PPLAT_DNS_RESPONSE\n" );
                }
            }while(0);                 
            break;
        };
        default:
            break;
    }
}
\end{lstlisting}
\subsection{请找出下面代码中的隐患或者错误,说明故障原因并改正。(16分)}
\begin{lstlisting}[escapeinside=`']
int Cmd_Show_AccessList(MSG_COMM_OAM *pRevMsg, MSG_COMM_OAM *pRtnMsg)
{
    OPR_DATA *pRtnOprData = (OPR_DATA *) pRtnMsg-> Data;    
    `----合法性检查1分'
    OPR_DATA *pRevOprData = (OPR_DATA *)pRevMsgStruct->Data;
    `----合法性检查1分'
    UINT8    *cycleNum;/*回显循环次数*/

    DWORD   aclNo       = 0;
    DWORD   ruleCount   = 0;
    int loc = END_ACL_LIST;
    WORD strLen         = 0;
    BYTE paraNo         = 0;  

    CHAR singleRule[256];
    ACL_STANDARD_T *pAclStd = NULL;
    DWORD filterAclNo       = 0;	
    DWORD cookie            = 0;

    memset(singleRule,0,sizeof(singleRule));
    loc = GetFirstAclItemLoc(aclNo);

    while (loc != END_ACL_LIST)
    {
        pAclStd = (ACL_STANDARD_T *) GetAclItem(aclNo,loc);
        if (pAclStd->rule_id == 0xff)
        `\verb|if (pAclStd->rule_id == 0xff)  ―――空指针检查2分|'
            break;
        loc = GetNextAclItemLoc(aclNo,loc);
    }

    /*每次打包最多 ACL_ONCE_SHOW_RULE_MAX 条rule ,
     rule 最大不超过200 byte ,保证每次打包数据不超过 1300 个字符左右。*/

    while ((loc != END_ACL_LIST) && (ruleCount <= ACL_ONCE_SHOW_RULE_MAX))
    `\verb|while ((loc != END_ACL_LIST) && (ruleCount < ACL_ONCE_SHOW_RULE_MAX))――合法性检查错误1分|'
    {
        /*打包一条rule*/
        Packet_AclRule(ACL_TYPE_STD,aclNo,loc,singleRule);
        pRtnOprData->ParaNo = paraNo++;
		pRtnOprData->Type = DATA_TYPE_STRING;
		strLen = strlen(singleRule);
		pRtnOprData->Len = strLen + 1;
		strcpy(pRtnOprData->Data,singleRule);
              `\verb|strcpy(pRtnOprData->Data,singleRule); ―――合法性保护,替换为strncpy函数。2分|'
		memset(pRtnOprData->Data + strLen,'\0',1);
		pRtnOprData = (OPR_DATA *) (pRtnOprData->Data + strLen + 1);
             `\verb|pRtnOprData = (OPR_DATA *) (pRtnOprData->Data + strLen + 1);―――移动错误,应该是(char *) (pRtnOprData->Data )+ strLen + 1) 2分|'
		
        loc = GetNextAclItemLoc(aclNo,loc);
        ruleCount++;
        memset(singleRule,0,sizeof(singleRule));
    }
    *(UINT8 *) cycleNum = ruleCount;/*保存循环次数*/ 
`\verb|*(UINT8 *) cycleNum = ruleCount;/*保存循环次数*/   cycleNum定义为 UINT8 *cycleNum ,其值为随机数,如果为空指针,则宕机。2分|'
    if (loc == END_ACL_LIST)
    {
        /*获取显示完的acl 的下一个要显示的acl*/
        aclIndex = GetFirstIPv6AclNo(&cookie);
        while(aclIndex != aclNo)
        {
            aclIndex = GetNextIPv6AclNo(&cookie);
        }
`\verb|while(aclIndex != aclNo)----死循环保护 3分|'
`\verb|aclIndex = GetNextIPv6AclNo(&cookie);|'
		aclNo = GetNextIPv6AclNo(&cookie);
		if(aclNo != 0)
        {
            aclType = GetAclType(NULL, aclNo);
            loc = GetFirstAclItemLoc(aclNo);
            if(aclType == IPV6_STANDARD_ACL)
            {
                pIPv6AclStd = (IPV6_STD_ACL_T*)GetAclItem(aclNo,loc);
                SU32_TO_HOST((DWORD *)(pRtnMsg->LastData+sizeof(DWORD)),pIPv6AclStd->rule_id);
                `\verb|SU32_TO_HOST((DWORD *)(pRtnMsg->LastData+sizeof(DWORD)), pIPv6AclStd->rule_id);合法性检查 1分|'
            }
            pRtnMsg->ReturnCode	= SUCC_CMD_NOT_FINISHED;
            pRtnMsg->DataLen	=
                (int)((char*)pRtnOprData - (char*)pRtnMsg)-10;
            `\verb|pRtnMsg->DataLen    = (int)((char*)pRtnOprData - (char*)pRtnMsg); 合法性检查2分|'
            pRtnMsg->Number	    = paraNo;
            return SUCC_AND_NOPARA;
        }
    }
}
\end{lstlisting}

\pagebreak

\begin{minipage}{0.9\textwidth}
\section{编程题(共20分)}
\subsection{使用标准C语言实现下列功能:设pHeadA和pHeadB分别是两个带表头结点的非递减有序单链表的表头指
  针, 试设计一个算法, 将这两个有序链表合并成一个非递增有序的单链表。要求结果链表仍使用原来两个链表的
  存储空间, 不另外占用其它的存储空间。表中允许有重复的数据。(10分)}
节点类型定义如下:
\begin{lstlisting}
typedef struct node
{
    int data;
    struct node *next;
}NODE,*PNODE;
\end{lstlisting}
函数原型:
\begin{lstlisting}
void  MergeList (PNODE pHeadA, PNODE pHeadB);
\end{lstlisting}
\end{minipage}

答案:
\begin{lstlisting}
void  MergeList (PNODE pHeadA, PNODE pHeadB)
{
    NODE *pa, *pb, *q, *p;
    pa = pHeadA->next ;
    pb = pHeadB->next;
    pHeadA→next = NULL;	//结果链表初始化

    while ( pa != NULL && pb != NULL ) 
    {  /*两链表都未结束时*/	
        if ( pa→data <= pb→data )
        {
            q = pa; 
            pa = pa→next; /*从pa链中摘下*/
        }			
        else
        { 
            q = pb; /*pb链中摘下*/
            pb = pb→next; 
        }			
        q→next = pHeadA→next;  pHeadA→next = q; /*链入结果链的链头*/
    }p = ( pa != NULL ) ? pa : pb;				/*处理未完链的剩余部分*/
    while ( p != NULL )
    {			
        q = p; 
        p = p→next; 
        q→next = pHeadA→next;  pHeadA→next = q;
    }
}
\end{lstlisting}

\pagebreak
\begin{minipage}{0.9\textwidth}
\vspace{40ex}
\subsection{使用标准C语言实现下列功能:有两个双向循环链表A,B,知道其头指针为:pHeadA, pHeadB,请写一
  函数将两个链表中data值相同的结点删除,并保证双向循环链表的完整性。(10分)}

\begin{minipage}{0.9\textwidth}
\vspace{1ex}
\small{\textit{例如:A链表中节点的data依次为:1,2,2,3,4,B链表中节点的data依次为:4,3,3,1,8。}}

\small{\textit{则,函数运算后的结果:}}

\small{\textit{A链表中节点的data依次为:2,2,B链表中节点的data依次为:8。}}
\vspace{2ex}
\end{minipage}

双向循环链表结点定义为:
\begin{lstlisting}
typedef struct node
{
    int data;
    struct node *front,*next;
}DBLNODE ,* PDBLNODE;
\end{lstlisting}
函数原型:
\begin{lstlisting}
void  DeleteXNode (PDBLNODE pHeadA, PDBLNODE pHeadB); 
\end{lstlisting}
\end{minipage}

答案:

\begin{lstlisting}
/* 删除一个节点 */
void  DeleteOneNode (PDBLNODE pNode)
{
    if (pNode == NULL)
    {
        return;
    }
    PDBLNODE q = pNode->front ;
    PDBLNODE r = pNode->next ; 
    if (q != NULL)
    {
        q->next = r ;
    }  
    if (r!= NULL)
    {
        r->front = q;
    }
    free(pNode);    
}
/*删除一个链表中所有重复值的节点*/
BOOL  DeleteSameValueNode (int data, PDBLNODE pHead)
{
    BOOL bRet = FALSE;
    PDBLNODE q,p;

    if (pHead == NULL)
    {
        return FALSE;
    }

    p= pHead->next;

    while (p && p->next != pHead)
    {
        q= p->next;
          
        if (p->data == data)
        {
            DeleteOneNode(p);
            bRet = TRUE; 
        }
        p=q;
    }
    return bRet; 
}
/* 主函数 */
void  DeleteXNode (PDBLNODE pHeadA, PDBLNODE pHeadB)
{
    PDBLNODE q,p;
    if (pHeadA == NULL || pHeadB == NULL)
    {
        return;
    }
    p= pHeadA->next;

    while (p && p->next != pHeadA)
    {
        q= p->next;
        if (DeleteSameValueNode(p->data,pHeadB))
        {
            DeleteOneNode(p);    
        }     
        p =q;
    }
}
\end{lstlisting}
\end{document}
