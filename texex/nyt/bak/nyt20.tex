\documentclass[12pt]{article}
\title{Digest of The New York Times}
\author{The New York Times}

\usepackage{config}

\makeindex
\begin{document}
\date{}
\thispagestyle{empty}

\begin{figure}
\includegraphics*[width=0.7\textwidth]{The_New_York_Times_logo.png}
\vspace{-15ex}
\end{figure}
% \renewcommand\contentsname{\textsf{Digest of The New York Times}}
\renewcommand\contentsname{}
{\footnotesize\textsf{\tableofcontents}}

\clearpage
\setcounter{page}{1}

\section{Inside the Fog of War: Reports From the Ground in Afghanistan}

\textit{This article was written and reported by C. J. Chivers, Carlotta Gall, Andrew W. Lehren,
  Mark Mazzetti, Jane Perlez, and Eric Schmitt, with contributions from Jacob Harris and Alan
  McLean.}

\lettrine{A}{}\mycalendar{Jul.'10}{26} six-year archive of classified military documents made public
on Sunday offers an unvarnished, ground-level picture of the war in Afghanistan that is in many
respects more grim than the official portrayal.

The secret documents, released on the Internet by an organization called WikiLeaks, are a daily
diary of an American-led force often starved for resources and attention as it struggled against an
insurgency that grew larger, better coordinated and more deadly each year.

The New York Times, the British newspaper The Guardian and the German magazine Der Spiegel were
given access to the voluminous records several weeks ago on the condition that they not report on
the material before Sunday.

The documents -- some 92,000 reports spanning parts of two administrations from January 2004 through
December 2009 -- illustrate in mosaic detail why, after the United States has spent almost \$300
billion on the war in Afghanistan, the Taliban are stronger than at any time since 2001.

As the new American commander in Afghanistan, Gen.~David H.~Petraeus, tries to reverse the lagging
war effort, the documents sketch a war hamstrung by an Afghan government, police force and army of
questionable loyalty and competence, and by a Pakistani military that appears at best uncooperative
and at worst to work from the shadows as an unspoken ally of the very insurgent forces the
American-led coalition is trying to defeat.

The material comes to light as Congress and the public grow increasingly skeptical of the deepening
involvement in Afghanistan and its chances for success as next year's deadline to begin withdrawing
troops looms.

The archive is a vivid reminder that the Afghan conflict until recently was a second-class war, with
money, troops and attention lavished on Iraq while soldiers and Marines lamented that the Afghans
they were training were not being paid.

The reports -- usually spare summaries but sometimes detailed narratives -- shed light on some
elements of the war that have been largely hidden from the public eye:

$\bullet$ The Taliban have used portable heat-seeking missiles against allied aircraft, a fact
that has not been publicly disclosed by the military. This type of weapon helped the Afghan
mujahedeen defeat the Soviet occupation in the 1980s.

$\bullet$ Secret commando units like Task Force 373 -- a classified group of Army and Navy special
operatives -- work from a ``capture/kill list'' of about 70 top insurgent commanders. These
missions, which have been stepped up under the Obama administration, claim notable successes, but
have sometimes gone wrong, killing civilians and stoking Afghan resentment.

$\bullet$ The military employs more and more drone aircraft to survey the battlefield and strike
targets in Afghanistan, although their performance is less impressive than officially portrayed.
Some crash or collide, forcing American troops to undertake risky retrieval missions before the
Taliban can claim the drone's weaponry.

$\bullet$ The Central Intelligence Agency has expanded paramilitary operations inside Afghanistan.
The units launch ambushes, order airstrikes and conduct night raids. From 2001 to 2008, the C.I.A.
paid the budget of Afghanistan's spy agency and ran it as a virtual subsidiary.

Over all, the documents do not contradict official accounts of the war. But in some cases the
documents show that the American military made misleading public statements -- attributing the
downing of a helicopter to conventional weapons instead of heat-seeking missiles or giving Afghans
credit for missions carried out by Special Operations commandos.

White House officials vigorously denied that the Obama administration had presented a misleading
portrait of the war in Afghanistan.

``On Dec.~1, 2009, President Obama announced a new strategy with a substantial increase in resources
for Afghanistan, and increased focus on Al Qaeda and Taliban safe-havens in Pakistan, precisely
because of the grave situation that had developed over several years,'' said Gen.~James L.~Jones,
White House national security adviser, in a statement released Sunday.

``We know that serious challenges lie ahead, but if Afghanistan is permitted to slide backwards, we
will again face a threat from violent extremist groups like Al Qaeda who will have more space to
plot and train,'' the statement said.

General Jones also condemned the decision by WikiLeaks to make the documents public, saying that
``the disclosure of classified information by individuals and organizations which could put the
lives of Americans and our partners at risk, and threaten our national security.''

``WikiLeaks made no effort to contact us about these documents – the United States government
learned from news organizations that these documents would be posted,'' General Jones said.

The archive is clearly an incomplete record of the war. It is missing many references to seminal
events and does not include more highly classified information. The documents also do not cover
events in 2010, when the influx of more troops into Afghanistan began and a new counterinsurgency
strategy took hold.

They suggest that the military's internal assessments of the prospects for winning over the Afghan
public, especially in the early days, were often optimistic, even na\"ive.

There are fleeting -- even taunting -- reminders of how the war began in the occasional references
to the elusive Osama bin Laden. In some reports he is said to be attending meetings in Quetta,
Pakistan. His money man is said to be flying from Iran to North Korea to buy weapons. Mr.~bin Laden
has supposedly ordered a suicide attack against the Afghan president, Hamid Karzai. These reports
all seem secondhand at best.

The reports portray a resilient, canny insurgency that has bled American forces through a war of
small cuts. The insurgents set the war's pace, usually fighting on ground of their own choosing and
then slipping away.

Sabotage and trickery have been weapons every bit as potent as small arms, mortars or suicide
bombers. So has Taliban intimidation of Afghan officials and civilians -- applied with pinpoint
pressure through threats, charm, violence, money, religious fervor and populist appeals.

\textbf{FEB. 19, 2008 | ZABUL PROVINCE Intelligence Summary: Officer Threatened}

An Afghan National Army brigade commander working in southern Afghanistan received a phone call from
a Taliban mullah named Ezat, one brief report said. ``Mullah Ezat told the ANA CDR to surrender and
offered him \$100,000(US) to quit working for the Afghan Army,'' the report said. ``Ezat also stated
that he knows where the ANA CDR is from and knows his family.'' Read the Document »

\textbf{MAY 9, 2009 | KUNAR PROVINCE Intelligence Summary: Taliban Recruiter}

A Taliban commander, Mullah Juma Khan, delivered a eulogy at the funeral of a slain insurgent. He
played on the crowd's emotions, according to the report: ``Juma cried while telling the people an
unnamed woman and her baby were killed while the woman was nursing the baby.'' Finally he made his
pitch: ``Juma then told the people they needed to be angry at CF $[$Coalition Force$]$ and ANSF
$[$Afghan National Security Forces$]$ for causing this tragedy'' and ``invited everyone who wants to
fight to join the fighters who traveled with him.'' Read the Document »

The insurgents use a network of spies, double agents, collaborators and informers -- anything to
undercut coalition forces and the effort to build a credible and effective Afghan government capable
of delivering security and services.

The reports repeatedly describe instances when the insurgents have been seen wearing government
uniforms, and other times when they have roamed the country or appeared for battle in the very Ford
Ranger pickup trucks that the United States had provided the Afghan Army and police force.

\textbf{NOV. 20, 2006 | KABUL Incident Report: Insurgent Subterfuge}

After capturing four pickup trucks from the Afghan National Army, the Taliban took them to Kabul to
be used in suicide bombings. ``They intend to use the pick-up trucks to target ANA compounds, ISAF
and GOA convoys, as well as ranking GOA and ISAF officials,'' said a report, referring to coalition
forces and the government of Afghanistan. ``The four trucks were also accompanied by an unknown
quantity of ANA uniforms to facilitate carrying out the attacks.'' Read the Document »

The Taliban's use of heat-seeking missiles has not been publicly disclosed -- indeed, the military
has issued statements that these internal records contradict.

In the form known as a Stinger, such weapons were provided to a previous generation of Afghan
insurgents by the United States, and helped drive out the Soviets. The reports suggest that the
Taliban's use of these missiles has been neither common nor especially effective; usually the
missiles missed.

\textbf{MAY 30, 2007 | HELMAND PROVINCE Incident Report: Downed Helicopter}

An American CH-47 transport helicopter was struck by what witnesses described as a portable
heat-seeking surface-to-air missile after taking off from a landing zone.

The helicopter, the initial report said, ``was engaged and struck with a Missile \ldots shortly
after crossing over the Helmand River. The missile struck the aircraft in the left engine. The
impact of the missile projected the aft end of the aircraft up as it burst into flames followed
immediately by a nose dive into the crash site with no survivors.''

The crash killed seven soldiers: five Americans, a Briton and a Canadian.

Multiple witnesses saw a smoke trail behind the missile as it rushed toward the helicopter. The
smoke trail was an important indicator. Rocket-propelled grenades do not leave them. Heat-seeking
missiles do. The crew of other helicopters reported the downing as a surface-to-air missile strike.
But that was not what a NATO spokesman told Reuters.

``Clearly, there were enemy fighters in the area,'' said the spokesman, Maj. John Thomas. ``It's not
impossible for small-arms fire to bring down a helicopter.''

The reports paint a disheartening picture of the Afghan police and soldiers at the center of the
American exit strategy.

The Pentagon is spending billions to train the Afghan forces to secure the country. But the police
have proved to be an especially risky investment and are often described as distrusted, even
loathed, by Afghan civilians. The reports recount episodes of police brutality, corruption petty and
large, extortion and kidnapping. Some police officers defect to the Taliban. Others are accused of
collaborating with insurgents, arms smugglers and highway bandits. Afghan police officers defect
with trucks or weapons, items captured during successful ambushes or raids.

\textbf{MARCH 10, 2008 | PAKTIA PROVINCE Investigation Report: Extortion by the Police}

This report captured the circular and frustrating effort by an American investigator to stop Afghan
police officers at a checkpoint from extorting payments from motorists. After a line of drivers
described how they were pressed to pay bribes, the American investigator and the local police
detained the accused checkpoint police officers.

``While waiting,'' the investigator wrote, ``I asked the seven patrolmen we detained to sit and
relax while we sorted through a problem without ever mentioning why they were being detained. Three
of the patrolmen responded by saying that they had only taken money from the truck drivers to buy
fuel for their generator.''

Two days later when the American followed up, he was told by police officers that the case had been
dropped because the witness reports had all been lost. Read the Document »

One report documented the detention of a military base worker trying to leave the base with GPS
units hidden under his clothes and taped to his leg. Another described the case of a police chief in
Zurmat, in Paktia Province, who was accused of falsely reporting that his officers had been in a
firefight so he could receive thousands of rounds of new ammunition, which he sold in a bazaar.

Coalition trainers report that episodes of cruelty by the Afghan police undermine the effort to
build a credible security force to take over when the allies leave.

\textbf{OCT. 11, 2009 | BALKH PROVINCE Incident Report: Brutal Police Chief}

This report began with an account of Afghan soldiers and police officers harassing and beating local
civilians for refusing to cooperate in a search. It then related the story of a district police
commander who forced himself on a 16-year-old girl. When a civilian complained, the report
continued, ``The district commander ordered his bodyguard to open fire on the AC $[$Afghan
civilian$]$. The bodyguard refused, at which time the district commander shot $[$the bodyguard$]$ in
front of the AC.''

Rivalries and friction between the largest Afghan security services -- the police and the army --
are evident in a number of reports. Sometimes the tensions erupted in outright clashes, as was
recorded in the following report from last December that was described as an ``enemy action.'' The
``enemy'' in this case was the Afghan National Security Force.

\textbf{DEC. 4, 2009 | ORUZGAN PROVINCE Incident Report: Police and Army Rivalry}

A car accident turned deadly when an argument broke out between the police and the Afghan National
Army. ``The argument escalated and ANA \& ANP started to shoot at each other,'' a report said.

An Afghan soldier and three Afghan police officers were wounded in the shootout. One civilian was
killed and six others were wounded by gunfire. Read the Document »

One sign of the weakness of the police is that in places they have been replaced by tribal warlords
who are charged -- informally but surely -- with providing the security the government cannot. Often
the warlords operate above the law.

\textbf{NOV. 22, 2009 | KANDAHAR PROVINCE Incident Report: Illegal Checkpoint}

A private security convoy, ferrying fuel from Kandahar to Oruzgan, was stopped by what was thought
to be 100 insurgents armed with assault rifles and PK machine guns, a report said.

It turned out the convoy had been halted by ``the local Chief of Police,'' who was ``demanding
\$2000-\$3000 per truck'' as a kind of toll. The chief, said the report, from NATO headquarters in
Southern Afghanistan, ``states he needs the money to run his operation.''

The chief was not actually a police chief. He was Matiullah Khan, a warlord and an American-backed
ally of President Karzai who was arguably Oruzgan's most powerful man. He had a contract, the
Ministry of Interior said, to protect the road so NATO's supply convoys could drive on it, but he
had apparently decided to extort money from the convoys himself.

Late in the day, Mr.~Matiullah, after many interventions, changed his mind. The report said that
friendly forces ``report that the COMPASS convoy is moving again and did not pay the fee required.''

The documents show how the best intentions of Americans to help rebuild Afghanistan through
provincial reconstruction teams ran up against a bewildering array of problems -- from corruption to
cultural misunderstandings -- as they tried to win over the public by helping repair dams and
bridges, build schools and train local authorities.

A series of reports from 2005 to 2008 chart the frustrations of one of the first such teams,
assigned to Gardez, in Paktia Province.

\textbf{NOV. 28, 2006 | PAKTIA PROVINCE Civil Affairs Report: Orphanage Opens}

An American civil affairs officer could barely contain her enthusiasm as she spoke at a
ribbon-cutting ceremony for a new orphanage, built with money from the American military.

The officer said a friend had given her a leather jacket to present to ``someone special,'' the
report noted. She chose the orphanage's director. ``The commander stated that she could think of no
one more deserving then someone who cared for orphans,'' it said.

The civil affairs team handed out blankets, coats, scarves and toys. The governor even gave money
from his own pocket. ``All speeches were very positive,'' the report concluded. Read the Document »

\textbf{DEC. 20, 2006 | PAKTIA PROVINCE Civil Affairs Report: Not Many Orphans}

The team dropped by to check on the orphanage. ``We found very few orphans living there and could
not find most of the HA $[$humanitarian assistance$]$ we had given them,'' the report noted.

The team raised the issue with the governor of Paktia, who said he was also concerned and suspected
that the money he had donated had not reached the children. He visited the orphanage himself. Only
30 children were there; the director had claimed to have 102. Read the Document »

\textbf{OCT. 16, 2007 | PAKTIA PROVINCE Civil Affairs Report: An Empty Orphanage}

Nearly a year after the opening of the orphanage, the Americans returned for a visit. ``There are
currently no orphans at the facility due to the Holiday. (Note: orphans are defined as having no
father, but may still have mother and a family structure that will have them home for holidays.)''
Read the Document »

\textbf{FEB. 25, 2007 | PAKTIA PROVINCE District Report: Lack of Resources}

As the Taliban insurgency strengthened, the lack of a government presence in the more remote
districts -- and the government's inability to provide security or resources even to its own
officials -- is evident in the reports.

An official from Dand Wa Patan, a small sliver of a district along the border with Pakistan, so
urgently wanted to talk to the members of the American team that he traveled three and a half hours
by taxi -- he had no car -- to meet them.

``He explained that the enemy had changed their tactics in the area and were no longer fighting from
the mountains, no longer sending rockets toward his compound and other areas,'' the report noted.
``He stated that the enemy focus was on direct action and that his family was a primary target.''

Ten days earlier the Taliban crept up to the wall of his family compound and blew up one of the
security towers, the report said. His son lost his legs in the explosion.

He pleaded for more police officers, weapons and ammunition. He also wanted a car so he could drive
around the district he was supposed to oversee.

But the Americans' situation was not much better. For months the reports show how a third -- or even
a half -- of the team's vehicles were out of service, awaiting spare parts.

\textbf{NOV. 15, 2006 | PAKTIA PROVINCE Civil Affairs Report: Local Corruption}

For a while the civil affairs team worked closely with the provincial governor, described as ``very
charismatic.'' Yet both he and the team are hampered by corrupt, negligent and antagonistic
officials.

The provincial chief of police is described in one report as ``the axel of corruption.''

``He makes every effort to openly and blatantly take money from the ANP troopers and the officers,''
one sympathetic officer told the Americans.

Other officers are more clever. One forged rosters, to collect pay for imaginary police officers. A
second set up illegal checkpoints to collects tolls around Gardez. Still another stole food and
uniforms, leaving his soldiers underfed and ill equipped for the winter.

The governor, meanwhile, was all but trapped. Such animosity developed between him and a senior
security official that the governor could not leave his office for weeks at a time, fearing for his
life. Finally, the corrupt officials were replaced. But it took months.

\textbf{SEPT. 24, 2007 | PAKTIA PROVINCE Civil Affairs Report: The Cost of Corruption}

Their meetings with Afghan district officials gave the American civil affairs officers unique
insights into local opinions. Sometimes, the Afghan officials were brutally honest in their
assessments.

In one case, provincial council officials visited the Americans at their base in Gardez to report
threats -- the Taliban had tossed a grenade into their office compound and were prowling the hills.
Then the officials began a tirade.

``The people of Afghanistan keep loosing their trust in the government because of the high amount of
corrupted government officials,'' the report quoted them as saying. ``The general view of the
Afghans is that the current government is worst than the Taliban.''

``The corrupted government officials are a new concept brought to Afghanistan by the AMERICANS,''
the oldest member of the group told the civil affairs team.

In conclusion, the civil affairs officer who wrote the report warned, ``The people will support the
Anti-Coalition forces and the security condition will degenerate.'' He recommended a public
information program to educate Afghans about democracy. Read the Document »

The reports also evoke the rivalries and tensions that swirl within the presidential palace between
President Karzai's circle and the warlords.

\textbf{OCT. 16, 2006 | KABUL Intelligence Summary: Political Intrigue}

In a short but heated meeting at the presidential palace, the Kabul police chief, Brig. Gen.~Mir
Amanullah Gozar, angrily refuted accusations made publicly by Jamil Karzai that he was corrupt and
lacked professional experience. The report of the meeting identified Jamil Karzai as the president's
brother; he is in fact a cousin.

General Gozar ``said that if Jamil were not the president's Brother he would kidnap, torture, and
kill him,'' the report said. He added that he was aware of plans by the American-led coalition to
remove him from his post.

He threatened the president, saying that if he were replaced he would reveal ``allegations about
Karzai having been a drug trader and supporter of the Pakistan-led insurgency in Afghanistan,''
presumably a reference to Mr.~Karzai's former links with the Taliban.

Incident by incident, the reports resemble a police blotter of the myriad ways Afghan civilians were
killed -- not just in airstrikes but in ones and twos -- in shootings on the roads or in the
villages, in misunderstandings or in a cross-fire, or in chaotic moments when Afghan drivers
ventured too close to convoys and checkpoints.

The dead, the reports repeatedly indicate, were not suicide bombers or insurgents, and many of the
cases were not reported to the public at the time. The toll of the war -- reflected in mounting
civilian casualties -- left the Americans seeking cooperation and support from an Afghan population
that grew steadily more exhausted, resentful, fearful and alienated.

From the war's outset, airstrikes that killed civilians in large numbers seized international
attention, including the aerial bombardment of a convoy on its way to attend President Karzai's
inauguration in 2001. An airstrike in Azizabad, in western Afghanistan, killed as many as 92 people
in August 2008. In May 2009, another strike killed 147 Afghan civilians.

\textbf{SEPT. 3, 2009 | KUNDUZ PROVINCE Incident Report: Mistaken Airstrike}

This report, filed about the activities of a Joint Terminal Attack Controller team, which is
responsible for communication from the ground and guiding pilots during surveillance missions and
airstrikes, offers a glimpse into one of the bloodiest mistakes in 2009.

It began with a report from the police command saying that ``2X FUEL TRUCKS WERE STOLEN BY UNK
NUMBER OF INS'' and that the insurgents planned to cross the Kunduz River with their prizes. It was
nighttime, and the river crossing was not illuminated. Soon, the report noted, the ``JTAC OBSERVED
KDZ RIVER AND REPORTED THAT IT DISCOVERED THE TRUCKS AS WELL AS UP TO 70 INS'' at ``THE FORD ON THE
RIVER. THE TRUCKS WERE STUCK IN THE MUD.'' How the JTAC team was observing the trucks was not clear,
but many aircraft have infrared video cameras that can send a live feed to a computer monitor on the
ground.

According to the report, a German commander of the provincial reconstruction team ``LINKED UP WITH
JTAC AND, AFTER ENSURING THAT NO CIVILIANS WERE IN THE VICINITY,'' he ``AUTHORIZED AN AIRSTRIKE.''
An F-15 then dropped two 500-pound guided bombs. The initial report said that ``56X INS KIA
$[$insurgents killed in action$]$ (CONFIRMED) AND 14X INS FLEEING IN NE DIRECTION. THE 2X FUEL
TRUCKS WERE ALSO DESTROYED.''

The initial report was wrong. The trucks had been abandoned, and a crowd of civilians milled around
them, removing fuel. How the commander and the JTAC had ensured ``that no civilians were in the
area,'' as the report said, was not explained.

The first sign of the mistake documented in the initial report appeared the next day, when another
report said that at ``0900 hrs International Media reported that US airstrike had killed 60
civilians in Kunduz. The media are reporting that Taliban did steal the trucks and had invited
civilians in the area to take fuel.'' Read the Document »

The reports show that the smaller incidents were just as insidious and alienating, turning Afghans
who had once welcomed Americans as liberators against the war.

\textbf{MARCH 5, 2007 | GHAZNI PROVINCE Incident Report: Checkpoint Danger}

Afghan police officers shot a local driver who tried to speed through their checkpoint on a country
road in Ghazni Province south of Kabul. The police had set up a temporary checkpoint on the highway
just outside the main town in the district of Ab Band.

``A car approached the check point at a high rate of speed,'' the report said. All the police
officers fled the checkpoint except one. As the car passed the checkpoint it knocked down the lone
policeman. He fired at the vehicle, apparently thinking that it was a suicide car bomber.

``The driver of the vehicle was killed,'' the report said. ``No IED $[$improved explosive device$]$
was found and vehicle was destroyed.''

The police officer was detained in the provincial capital, Ghazni, and questioned. He was then
released. The American mentoring the police concluded in his assessment that the policeman's use of
force was appropriate. Rather than acknowledging the public hostility such episodes often engender,
the report found a benefit: it suggested that the shooting would make Afghans take greater care at
checkpoints in the future.

``Effects on the populace clearly identify the importance of stopping at checkpoints,'' the report
concluded. Read the Document »

\textbf{MARCH 21, 2007 | PAKTIKA PROVINCE Incident Report: A Deaf Man Is Shot}

Members of a C.I.A. paramilitary unit moved into the village of Malekshay in Paktika Province close
to the border with Pakistan when they saw an Afghan running away at the sight of their convoy, one
report recounted. Members of the unit shot him in the ankle, and medics treated him at the scene.
The unit had followed military procedure -- first shouting at the man, then firing warning shots and
only after that shooting to wound, the report said.

Yet elders in the village told the unit that the man, Shum Khan, was deaf and mute and that he had
fled from the convoy out of nervousness. Mr.~Khan was ``unable to hear the warnings or warning
shots. Ran out of fear and confusion,'' the report concludes. The unit handed over supplies in
compensation. Read the Document »

The reports reveal several instances of allied forces accidentally firing on one another or on
Afghan forces in the fog of war, often with tragic consequences.

\textbf{APRIL 6, 2006 | HELMAND PROVINCE Incident Report: Friendly Fire}

A British Army convoy driving at night in southern Afghanistan suddenly came under small-arms fire.
One of the British trucks rolled over. The British troops split into two groups, pulled back from
the clash and called in airstrikes from American A-10 attack planes. After several confusing
minutes, commanders realized that the Afghan police had attacked the British troops, mistaking them
for Taliban fighters. One Afghan police officer was killed and 12 others were wounded.

The shifting tactics of the Americans can be seen as well in the reports, as the war strategy veered
from freely using force to trying to minimize civilian casualties. But as the documents make clear,
each approach has its frustrations for the American effort.

Strict new rules of engagement, imposed in 2009, minimized the use of airstrikes after some had
killed civilians and turned Afghans against the war. But the rules also prompted anger from American
troops and their families. The troops felt that their lives were not sufficiently valued because
they had to justify every request for air or artillery support, making it easier for the Taliban to
fight.

\textbf{OCT. 1, 2008 | KUNAR PROVINCE Incident Report: Barrage}

In the days when field commanders had a freer hand, an infantry company commander observed an Afghan
with a two-way radio who was monitoring the company's activities. Warning of ``IMMINENT THREAT,''
the commander said he would ``destroy'' the man and his equipment -- in other words, kill him. A
short while later, a 155-millimeter artillery piece at a forward operating base in the nearby Pech
Valley began firing high-explosive rounds -- 24 in all.

\textbf{NOV. 13, 2009 | HELMAND PROVINCE Incident Report: Escalation of Force}

As the rules tightened, the reports picked up a tone that at times seemed lawyerly. Many make
reference, even in pitched fights, to troops using weapons in accordance with ``ROE Card A'' --
which guides actions of self-defense rather than attacks or offensive acts. This report described an
Apache helicopter firing warning shots after coming under fire. Its reaction was described as ``an
escalation of force.''

The helicopter pilots reported that insurgents ``engaged with SAF $[$surface-to-air fire$]$''and
that ``INTEL suggested they were going to be fired upon again during their extraction.''

The helicopters ``fired 40x 30mm warning shots to deter any further engagement.''

The report included the information that now is common to incident reports in which Western forces
fire. ``The terrain was considered rurally open and there were no CIV PID IVO $[$civilians
positively identified in the vicinity of $]$ the target within reasonable certainty. There was no
damage to infrastructure. BDA $[$battle damage assessment$]$ recording conducted by AH-64 Gun Tape.
No follow up required. The next higher command was consulted. The enemy engaged presented, in the
opinion of the ground forces, an imminent threat. Engagement is under ROE Card A.~Higher HQ have
been informed.'' Read the Document »

The reports show in previously unknown detail the omnipresence of drones in Afghanistan, the Air
Force's missile-toting Predators and Reapers that hunt militants. The military's use of drones in
Afghanistan has rapidly expanded in the past few years; the United States Air Force now flies about
20 Predator and Reaper aircraft a day -- nearly twice as many as a year ago -- over vast stretches
of hostile Afghan territory. Allies like Britain and Germany fly their own fleets.

The incident reports chronicle the wide variety of missions these aircraft carry out: taking
photographs, scooping up electronic transmissions, relaying images of running battles to field
headquarters, attacking militants with bombs and missiles. And they also reveal the extent that
armed drones are being used to support American Special Operations missions.

Documents in the Afghan archive capture the strange nature of the drone war in Afghanistan:
missile-firing robots killing shovel-wielding insurgents, a remote-controlled war against a low-tech
but resilient insurgency.

\textbf{DEC. 9, 2008 | KANDAHAR PROVINCE Incident Report: Predator Attack}

Early one winter evening in southern Afghanistan, an Air Force Predator drone spotted a group of
insurgents suspected of planting roadside bombs along a roadway less than two miles from Forward
Operating Base Hutal, an American outpost.

Unlike the drones the C.I.A. operated covertly across the border in Pakistan, this aircraft was one
of nearly a dozen military drones patrolling vast stretches of hostile Afghan territory on any given
day.

Within minutes after identifying the militants, the Predator unleashed a Hellfire missile, all but
evaporating one of the figures digging in the dark.

When ground troops reached the crater caused by the missile, costing \$60,000, all that was left was
a shovel and a crowbar. Read the Document »

\textbf{SEPT. 13, 2009 | BADAKHSHAN PROVINCE Incident Report: A Lost Drone}

Flying over southern Afghanistan on a combat mission, one of the Air Force's premier armed drones, a
Reaper, went rogue.

Equipped with advanced radar and sophisticated cameras, as well as Hellfire missiles and 500-pound
bombs, the Reaper had lost its satellite link to a pilot who was remotely steering the drone from a
base in the United States.

Again and again, the pilot struggled to regain control of the drone. Again and again, no response.
The reports reveal that the military in Afghanistan lost many of the tiny five-pound surveillance
drones with names like Raven and Desert Hawk that troops tossed out like model airplanes to peer
around the next hill. But they had never before lost one of the Reapers, with its 66-foot wingspan.

As a last resort, commanders ordered an Air Force F-15E Strike Eagle fighter jet to shoot down the
\$13 million aircraft before it soared unguided into neighboring Tajikistan.

Ground controllers picked an unpopulated area over northern Afghanistan and the jet fired a
Sidewinder missile, destroying the Reaper's turbo-prop engine. Suddenly, the satellite link was
restored, but it was too late to salvage the flight. At 5:30 a.m., controllers steered it into a
remote mountainside for a final fiery landing. Read the Document »

As the Afghanistan war took priority under the Obama administration, more Special Operations forces
were shifted from Iraq to conduct secret missions. The C.I.A.'s own paramilitary operations inside
Afghanistan grew in tandem -- as did the agency's close collaboration with Afghanistan's own spy
agency.

Usually, such teams conducted night operations aimed at top Taliban commanders and militants on the
``capture/kill'' list. While individual commandos have displayed great courage, the missions can end
in calamity as well as success. The expanding special operations have stoked particular resentment
among Afghans -- for their lack of coordination with local forces, the civilian casualties they
frequently inflicted and the lack of accountability.

\textbf{JUNE 17, 2007 | PAKTIKA PROVINCE INCIDENT REPORT: Botched Night Raid}

Shortly after five American rockets destroyed a compound in Paktika Province, helicopter-borne
commandos from Task Force 373 -- a classified Special Operations unit of Army Delta Force operatives
and members of the Navy Seals -- arrived to finish the job.

The mission was to capture or kill Abu Laith al-Libi, a top commander for Al Qaeda, who was believed
to be hiding at the scene of the strike.

But Mr.~Libi was not there. Instead, the Special Operations troops found a group of men suspected of
being militants and their children. Seven of the children had been killed by the rocket attack.

Some of the men tried to flee the Americans, and six were quickly killed by encircling helicopters.
After the rest were taken as detainees, the commandos found one child still alive in the rubble, and
performed CPR for 20 minutes.

Word of the attack spread a wave of anger across the region, forcing the local governor to meet with
village elders to defuse the situation.

American military officials drew up a list of ``talking points'' for the governor, pointing out that
the target had been a senior Qaeda commander, that there had been no indications that women and
children would be present and that a nearby mosque had not been damaged.

After the meeting, the governor reported that local residents were in shock, but that he had
``pressed the Talking Points.'' He even ``added a few of his own that followed in line with our
current story.''

The attack was caused by the ``presence of hoodlums,'' the governor told the people. It was a
tragedy that children had been killed, he said, but ``it could have been prevented had the people
exposed the presence of insurgents in the area.''

He promised that the families would be compensated for their loss.

Mr.~Libi was killed the following year by a C.I.A. drone strike. Read the Document »

\textbf{APRIL 6, 2008 | NURISTAN PROVINCE Incident Report: A Raging Firefight}

As they scrambled up the rocks toward a cluster of mud compounds perched high over the remote Shok
Valley, a small group of American Green Berets and Afghan troops, known as Task Force Bushmaster,
were confronted with a hail of gunfire from inside the insurgent stronghold.

They were there to capture senior members of the Hezb-e-Islami Gulbuddin militant group, part of a
mission that the military had dubbed Operation Commando Wrath.

But what they soon discovered on that remote, snowy hilltop was that they were vastly outnumbered by
a militant force of hundreds of fighters. Reinforcements were hours away.

A firefight raged for nearly seven hours, with sniper fire pinning down the Green Berets on a
60-foot rock ledge for much of that time.

Casualties mounted. By midmorning, nearly half of the Americans were wounded, but the militants
directed their gunfire on the arriving medevac helicopters, preventing them from landing.

``TF Bushmaster reports they are combat ineffective and request reinforcement at this time.''

For a time, radio contact was lost.

Air Force jets arrived at the scene and began pummeling the compounds with 2,000-pound bombs, but
the militants continued to advance down the mountain toward the pinned-down group.

The task force reported that there were `` 50-100 insurgents moving to reinforce against Bushmaster
elements from the SW.''

Carrying wounded Americans shot in the pelvis, arm and legs -- as well as two dead Afghans -- the
group made its way down toward the valley floor. Eventually, the helicopters were able to arrive to
evacuate the dead and wounded.

Ten members of the Green Berets would receive Silver Stars for their actions during the battle, the
highest number given to Special Forces soldiers for a single battle since the Vietnam War. By Army
estimates, 150 to 200 militants were killed in the battle. Read the Document »

\textbf{MARCH 8, 2008 | BAGRAM AIR BASE Meeting Report: A Plea for Help}

Toward the end of a long meeting with top American military commanders, during which he delivered a
briefing about the security situation in eastern Afghanistan, corruption in the government and
Pakistan's fecklessness in hunting down militants, Afghanistan's top spy laid out his problem.

Amrullah Saleh, then director of the National Directorate of Security, told the Americans that the
C.I.A. would no longer be handling his spy service's budget. For years, the C.I.A. had essentially
run the N.D.S. as a subsidiary, but by 2009 the Afghan government was preparing to take charge of
the agency's budget.

Mr.~Saleh estimated that with the C.I.A. no longer bankrolling the Afghan spies, he could be facing
a budget cut of 30 percent.

So he made a request. With the budget squeeze coming, Mr.~Saleh asked the Americans for any AK-47s
and ammunition they could spare.

\section{In Ruling on iPhones, Apple Loses a Bit of Its Grip}

\lettrine{A}{pple}\mycalendar{Jul.'10}{27} likes to maintain tight control over what programs can
appear on the iPhone -- a task that just became a little bit harder.

The Library of Congress, which has the power to define exceptions to an important copyright law,
said on Monday that it was legal to bypass a phone's controls on what software it will run to get
``lawfully obtained'' programs to work.

The Electronic Frontier Foundation, a nonprofit digital rights group, had asked for that exception
to the Digital Millennium Copyright Act to allow the so-called jailbreaking of iPhones and other
devices.

``This is a really important victory for iPhone owners,'' said Corynne McSherry, a senior staff
lawyer with the foundation. ``People who want to tinker with their phones and move outside of the
Applesphere now have the ability to legally do that.''

The issue has been a topic of debate between Apple, which says it has the right to control the
software on its devices, and technically adept users who want to customize their phones as they see
fit. In a legal filing last year with the United States Copyright Office, which is part of the
Library of Congress, Apple argued that altered phones infringed on its copyrights because they used
modified versions of Apple's operating system.

Apple also said that altering the phones encouraged the pirating of applications, exposed iPhones to
security risks and taxed the company's customer support staff.

``Apple's goal has always been to ensure that our customers have a great experience with their
iPhone, and we know that jailbreaking can severely degrade the experience,'' said Natalie Kerris, a
spokeswoman for Apple. ``As we've said before, the vast majority of customers do not jailbreak their
iPhones, as this can violate the warranty and can cause the iPhone to become unstable and not work
reliably.''

But iPhone hobbyists say they simply want to have free range to use certain features and programs on
their phones that Apple has limited or failed to offer.

For example, one popular unapproved application lets users sync their music and video clips with
their computer over Wi-Fi, without using a cable. Another enables tethering, or the ability to share
the iPhone's Internet connection with a computer, something for which iPhone owners are supposed to
pay AT\&T an extra \$20 a month.

An underground network of forums that walk iPhone owners through the jailbreaking process have
flourished online, as have storefronts that sell the unapproved applications.

Mario Ciabarra, who operates a software store called Rock Your Phone, said the jailbreaking decision
was ``extremely exciting'' for application developers.

``There's been some negative connotations with the jailbreak community,'' he said. ``That legitimacy
will go a long way in terms of bolstering our business and the apps business.''

Mr.~Ciabarra said he thought that the Library of Congress's decision could increase the appeal of
the phone and attract new users who have been turned off by what many perceive as Apple's
heavy-handed approach to the phone and its application store.

``The independent community makes the phone so much more appealing to users,'' he said. ``There's a
whole world of possibilities for what you can do with your phone.''

Apple had not actively pursued any specific sites or developers that assist people in jailbreaking.
But its penchant for exerting control over its products has kept many people on edge, said Mark
Janke, who runs a jailbreaking forum called Hack That Phone.

``I was really concerned for a while,'' Mr.~Janke said. ``People were worried about running
jailbreaking Web sites or being prosecuted for bringing their jailbroken phones into the Apple
Store.''

The ruling, he said, was ``a big win'' for jailbreaking fans.

``It gives us the ability to do the things we've wanted to do -- should be able to do -- with our
phones,'' he said.

The last few months have been something of a challenge for Apple, known for maintaining a kind of
hermetic seal around itself and its products. Recent troubles like the leak of a secret phone
prototype and the antenna problems reported by some owners of the iPhone 4 have chipped away at that
image somewhat.

``They still maintain an iron fist over everything they can. That won't change,'' said Mukul
Krishna, an analyst at Frost \& Sullivan who follows trends in digital media. He added, ``There are
certain things now beyond their control.''

Even with the slip-ups, Apple is still keeping its customers captivated, said Shaw Wu, an analyst at
Kaufman Brothers. ``Apple has gotten more scrutiny lately, but it hasn't shown up in the sales
figures yet,'' he said. ``People are buying their products in droves, and I'm not sure a government
ruling changes that.''

In addition to this decision, the Library of Congress also granted an exception to artists who remix
copy-protected video content for noncommercial work, and renewed its approval for cellphone owners
to ``unlock'' their phones or lift controls that restrict use to one wireless carrier.

\section{Even With All Its Profits, Microsoft Has a Popularity Problem}

\lettrine{M}{easured}\mycalendar{Jul.'10}{27} by profits, Microsoft trounces Apple and Google. In
the most recent three months, Microsoft earned \$4.52 billion, versus Apple's \$3.25 billion and
Google's \$1.8 billion. But, dear investors, where is the love for this beaten-down company?

Frank X.~Shaw, Microsoft's vice president for corporate communications, recently tried a new tack to
win respect. In a blog post on June 25 titled ``Microsoft by the Numbers,'' he compared Microsoft's
record in various business categories with that of competitors.

Unfortunately, by trying to argue that Microsoft is doing well in all areas, including those
dominated by Apple and Google, Mr.~Shaw fails to show Microsoft at its best. Lost from view is what
arguably is Microsoft's very best story -- its transformation into a powerhouse supplier of the
specialized software that meets the complex needs of large corporations, what the trade calls
selling to ``the enterprise.''

Microsoft's enterprise software business alone is approaching the size of Oracle. But despite that
astounding growth, Microsoft must accept that, fair or not, victories on the enterprise side draw
about as much attention as being the No.~1 wholesale seller of plumbing supplies. Microsoft won't
receive the adoring attention that its chief rival draws with products like the iPad.

In a conversation earlier this month, Mr.~Shaw explained what prompted him to write his post. ``I
noticed some pretty critical conversations going on in the technosphere among the technorati,'' he
said. ``There's a gap between that conversation -- 'the company is not doing well, period' -- and
what the company is actually doing.''

In the blog, he writes, ``With Windows 7, Office 2010, Bing, Xbox 360, Kinect, Windows Phone 7, in
our cloud platform, and many other products, services and happy customers, 2010 is shaping up as a
huge year for us.''

By encompassing just about every product category under the sun -- and then calling out Apple and
Google, of all targets -- Mr.~Shaw draws attention to Microsoft's weak spots.

Bing, its search engine, attracted 21.4 million new users in one year, Mr.~Shaw says. Very well, but
he does not mention the following: in 2007, the company's online services group lost \$604 million;
in 2008, \$1.2 billion; and in 2009, the year of Bing's introduction, \$2.25 billion.

Mr.~Shaw also points out that in its 2000 fiscal year, Microsoft's revenue was \$23 billion, and
that it grew to \$58.4 billion by 2009. He does not, however, go on to compare this growth with that
of Apple and Google, whom he had just called upon to illustrate another point. But let's call Apple
back to the stage: from 2000 to 2009, when Microsoft's revenue grew 153 percent, Apple's grew 436
percent. (Google's number, beginning from a tiny base in 2000, is too large for use as a fair
comparison.)

Perhaps the most important numbers that Mr.~Shaw did not include -- the numbers that go a long way
toward explaining the we-don't-get-no-respect tone in his post -- reflect the judgments of
investors. Microsoft continues to suffer through its very own lost decade. On Thursday, Microsoft
reported its strong quarterly earnings. But at the close on Friday, Microsoft's stock was 55 percent
below its price at the beginning of January 2000.

Apple also reported its quarterly results last week: the most Macs sold, ever; more than three
million iPads sold -- and its stock is now at \$260, or 829 percent above its January 2000 price.

``Tech investors pay for growth,'' says Sarah Friar, an analyst at Goldman Sachs, who believes that
those investors do not appreciate the durability of Microsoft's cash cows, Windows and Office. She
has many positive things to say about Microsoft's ability to innovate, pointing in particular to the
robust sales of server and database software, which are now almost equal in size to Windows revenue.

Ms.~Friar views Microsoft as a company that primarily sells to the enterprise. By contrast, Apple
and Google are primarily selling to consumers. ``How many companies are good at being an enterprise
company and a consumer company at the same time?'' she asks.

BRENDAN BARNICLE, a software analyst at Pacific Crest Securities, offers one explanation why
Microsoft stubbornly believes it can sell to consumers as well as to corporate customers: Microsoft
was able to do so in its early years, when its operating system software and then its Office
productivity suite were bought by individuals as well as companies.

``Microsoft is used to having it all,'' he says.

Mr.~Barnicle is bullish on Microsoft stock, partly because he thinks the company hasn't received
credit for its almost-half-full glass: the 40 percent of its business that is not Windows or Office.
He, too, praised its enterprise software business, formally labeled ``Server and Tools,'' as ``an
incredible business,'' accounting, he said, for about 24 percent of the company's revenue and with
an operating margin of 40 percent.

Mr.~Shaw points to the 150 million Windows 7 licenses sold in the eight months after its release
last year. It's an impressive figure; Macs, iPhones and iPads have a long way to go to catch up. But
those Windows 7 sales include pent-up corporate demand for anything-but-Windows-Vista. So that
figure doesn't give investors what they want most: portents of entirely new growth.

\section{Facebook Is to Power Company as \ldots}

\lettrine{I}{t}\mycalendar{Jul.'10}{27} was a typically vexing week for Facebook. On the one hand,
the social-networking service signed up its 500 millionth active user. On the other hand, it was
found to be one of the least popular private-sector companies in the United States by the American
Customer Satisfaction Index. Apparently, Americans were more satisfied filing their taxes online
than they were posting updates on their Facebook page.

It is a continuing contradiction: Facebook is widely criticized for shifting its terms of service
and for disclosing private information -- and yet millions of people start accounts each month.

Analysts always grasp for analogies to explain Facebook's tortured relationship with its users.
Facebook has been called the sterile suburbs to the gritty urban Internet; it is a ``walled garden''
in the organic messiness of the Web; it is Russia under Vladimir Putin; it is (and this one stings
in tech circles) today's AOL.

But perhaps the most telling metaphor compares Facebook to the other companies lurking at the bottom
of the American Customer Satisfaction Index: cable companies, wireless telephone service providers.
Utilities. Here are services everyone uses, no matter how much people dislike the companies that
provide them.

Danah Boyd, a social media researcher at Microsoft and a fellow at Harvard University's Berkman
Center for Internet and Society, argues that Facebook fits that mold.

On her blog in May, she posted:

``I hate all of the utilities of my life. Venomous hatred. And because they're monopolies, they feel
no need to make me appreciate them. Cuz they know that I'm not going to give up water, power,
sewage, or the Internet out of spite. Nor will most people give up Facebook, regardless of how much
they grow to hate them.''

Dr.~Boyd argues that even Facebook sees itself this way, describing itself as ``a social utility
that helps people communicate more efficiently with their friends, family and coworkers.'' The
company's goal, Dr.~Boyd says, is to become part of the infrastructure of the Internet.

Utilities are generally subject to added levels government regulation because of their fundamental
importance and tendency toward monopoly, so calling a company a utility comes with an uncomfortable
implication.

Mark Zuckerberg, Facebook's chief executive, was asked last week whether the company should be
regulated, since he likened the company to an electric utility. He rejected the comparison.

``We're here to build something useful,'' he said. ``Something that's cool can fade. But something
that's useful won't. That's what I meant by utility.''

Is Facebook important enough to be considered a utility? And is there no real alternative?

Facebook is clearly not as vital as the electric grid. But many technologies now seen as essential
were once dismissed as mere frivolity.

``At one time we could have had this conversation about the phone, and it would have sounded just as
crazy,'' said Christian Sandvig, an associate professor of communications at the University of
Illinois at Urbana-Champaign.

There is also debate about how firm a grip Facebook has on the market.

Some experts argue that Facebook has grown so large that it is nearly impossible for a viable
alternative to emerge, despite the presence of other social networks. Facebook gains data from each
additional participant, growing more useful as it expands. This makes it progressively harder for
users to leave, an economic principle known as lock-in.

Giving users greater control over their own data would help offset this tendency, said Dr.~Sandvig.
Users could then take their personal information and social connections to competing sites, much in
the way that cellphone users switch carriers without changing their phone numbers. Companies would
be encouraged to offer alternatives -- stronger privacy settings or ad-free sites with paid
subscriptions.

But it is unclear how this would work. Even if social networking companies agreed to common
standards that allowed a free flow of data -- or were forced to accept such standards -- there is no
clear line between a person's own data and the data of the people in her networks. What right does
someone have to her friends' data?

There could also be unintended consequences. Treating a company like a utility, for instance, can
help to lock in its dominance and discourage innovation.

With this in mind, Cindy Cohn, the legal director of the Electronic Frontier Foundation, says that
no one has developed the vocabulary needed to address social networking.

``I worry that we'll end up with solutions that are familiar but not correct if we start from the
wrong metaphor,'' she said. ``And I'm not sure there is a good metaphor for Facebook.''

\section{BlackBerry's Era May Be Ending}

\lettrine{R}{esearch}\mycalendar{Jul.'10}{27} In Motion's future looks bleak. R.I.M., which makes
the popular BlackBerry devices, has had a traditional stronghold in sales to American companies. But
that has been cracked open.

Apple said last week that more than 80 percent of the Fortune 100 companies were testing or
deploying its iPhone. Meanwhile, phones using Google's Android operating system appear to be making
inroads, too. That spells trouble for R.I.M.

Corporations and consumers used to be happy with handsets that served up e-mail reliably, promptly
and securely. R.I.M.'s products do this very well. Now, though, Android and Apple handsets
adequately handle e-mail, while also doing much more. For instance, iPhone users can download about
30 times as many apps as are available to BlackBerry users, and the process is more user-friendly.

R.I.M.'s share of the American smartphone market reflects this shift. It fell to 41 percent in the
first quarter from 55 percent in the previous year, according to Gartner. The combined share held by
iPhones and Android handsets rose to 49 percent, from 23 percent over the same period.

True, R.I.M.'s sales in overseas markets are increasing, enabling it to hold its share of the global
smartphone market more or less steady. But handset trends increasingly originate in the United
States because of the growing importance of smartphone software.

Moreover, R.I.M.'s overseas push has been accompanied by a sharp fall in the average price of its
handsets. While phones are priced differently overseas, the suspicion is the company is losing
pricing power. Nokia's troubles, and the decline in its stock price, show that increasing volume
doesn't matter if prices fall too fast.

The BlackBerry maker's shares may appear cheap. They trade at 10 times estimated earnings for this
fiscal year. Yet Apple and Google's dominance in apps means they are becoming de facto standards in
the smartphone market. Technology companies that lose such wars often suffer shockingly fast profit
declines.

R.I.M. still has a shot. This fall, it is due to introduce a new operating system, which might help
it fight back. But time is running short.

\section{China Pushes to End Public Shaming}

\lettrine{T}{he}\mycalendar{Jul.'10}{28} Chinese government has called for an end to the public
shaming of criminal suspects, a time-honored cudgel of Chinese law enforcement but one that has
increasingly rattled the public.

According to the state-run media, the Ministry of Public Security has ordered the police to stop
parading suspects in public and has called on local departments to enforce laws in a ``rational,
calm and civilized manner.''

The new regulations are thought to be a response to the public outcry over a recent spate of ``shame
parades,'' in which those suspected of being prostitutes are shackled and forced to walk in public.

Last October, the police in Henan Province took to the Internet, posting photographs of women
suspected of being prostitutes. Other cities have been publishing the names and addresses of
convicted sex workers and those of their clients. The most widely circulated images, taken this
month in the southern city of Dongguan, included young women roped together and paraded barefoot
through crowded city streets.

The police later said they were not punishing the women, but only seeking their help in the pursuit
of an investigation.

The public response, at least on the Internet, has tended toward outrage, with many postings
expressing sympathy for the women. ``Why aren't corrupt officials dragged through the streets?''
read one posting. ``These women are only trying to feed themselves.''

But much of the anger has been directed at the police, who are a focus of growing public mistrust.
Although corruption among the police is rife in China, the disdain has been further heightened by a
series of widely publicized episodes involving the torture of detainees, suspects who mysteriously
died in custody and innocent people jailed on trumped-up evidence.

One man spent 10 years in prison for murder after the police extracted his confession -- only to be
freed when his supposed victim turned out to be alive.

Mao Shoulong, a professor of public policy at People's University in Beijing, said the new
regulations were necessary to rein in the worst impulses of the police.

``There are more modern tools for law enforcement,'' he said. ``Besides, if these kinds of tactics
are allowed, the police will get used to dealing with problems outside of the law.''

The most recent wave of prostitution arrests involving thousands of suspects is part of a
seven-month ``strike hard'' campaign aimed at gambling, drug use and violent crime. As part of the
increased law enforcement efforts, judicial authorities have been encouraged to mete out swifter,
and harsher, punishment. It is the fourth such campaign since 1983.

Public shaming of the accused and the condemned has been a long tradition in China -- one that the
Communist Party embraced with zeal during episodes of class struggle and anticrime crusades.
Although public executions have been discontinued, provincial cities still hold mass sentencing
rallies, during which convicts wearing confessional placards are driven though the streets in open
trucks.

The practice has also taken hold in some Chinese neighborhoods of New York, with some supermarket
owners threatening to post photographs of shoplifters and call the police unless the suspects hand
over cash, sometimes demanding hundreds of dollars. The legality of the practice, however, remains
in question.

It is unclear whether the directive against the humiliation of suspects will have the desired
effect. Similar rules and regulations have been passed down through the years, beginning in 1988,
when the Supreme People's Court ordered prosecutors and the police to protect the identities of the
accused. In 2007, the country's top judicial and law enforcement bodies issued a similar notice that
forbade the parading of convicts.

Even if such directives must be issued repeatedly, Joshua Rosenzweig of the Dui Hua Foundation, a
human rights group, said he was somewhat encouraged that the government recognized the need to
abolish such practices.

``Repetition can increase pressure and help force change, but ultimately it will take a great deal
of political will to implement these kinds of changes,'' he said.

\section{Yahoo Japan Teams With Google on Search}

\lettrine{Y}{ahoo}\mycalendar{Jul.'10}{28} Japan will use Google technology to power its Internet
search engine and search advertising platform, the Japanese company announced Tuesday, diverging
from a nascent alliance between the U.S.~Internet portal Yahoo and Microsoft.

The deal puts Yahoo Japan, partly owned by the U.S.~company, on a sharply different path from that
of its American cousin, Yahoo, which is planning to use Microsoft's Bing search technology by the
end of this year under an agreement announced in 2009.

The partnership between Yahoo Japan and Google would create a powerhouse that combines Google's
search technology with Yahoo Japan's popular content and services. Financial terms of the deal were
not disclosed.

Yahoo holds 34.8 percent of Yahoo Japan, while Softbank, the Japanese cellphone and Web giant, owns
38.6 percent.

Google's share of Web searches in Japan has been growing, but it still trails Yahoo Japan, the
market leader. According to Net Ratings, a research firm, Yahoo Japan grabs 53.2 percent of Web
queries in the country, followed by Google at 37.3 percent. Microsoft's MSN and Bing searches
garnered just 2.6 percent.

Globally, though, Google dominates -- the company had 85 percent of Internet queries, compared with
6.2 percent for Yahoo brand searches in June, according to the analysis arm of Net Applications, a
Web services company.

Yahoo Japan's adoption of Google's search technology would mean that about 90 percent of Web queries
in Japan would be powered by the company. Yahoo Japan used Google technology for its search engine
from 2001 to 2004, but then switched back to Yahoo's system.

Masahiro Inoue, chief executive of Yahoo Japan, said that after a year of careful analysis, the
company had concluded that it would ultimately benefit more from Google's search engine than from
Microsoft's because of Google's record in Japanese-language queries. Google has had a presence in
Japan since 2001.

``At the present time, we feel there are quite a few areas where Microsoft is not yet ready,''
Mr.~Inoue said Tuesday at a news conference in Tokyo. ``Google is one step ahead in
Japanese-language services.''

Mr.~Inoue said Yahoo Japan would pay to use Google's search technology, while Google would receive
up-to-the-minute content updates from Yahoo Japan, something he said would help Google search
results stay timely. He did not offer financial details.

Still, by building its content on top of Google search results, Yahoo Japan will continue to compete
with Google, Mr.~Inoue said. The chief executive raised eyebrows when he told Nikkei Business
Magazine in January that Google services, like Street View and Book Search, were ``nothing
impressive.''

``We're switching engines to Google, but what we build on top of that will be exclusive to Yahoo,''
Mr.~Inoue said Tuesday. ``That will continue to give Yahoo the edge.''

Yahoo Japan's deal seems to mirror Yahoo's overall goals, at least: to focus on the company's
strengths as a creator of Web content and as a marketer and leader in online display advertising,
while tapping outside Web search technology.

The dominance of Yahoo Japan has been the product of its universe of content, spanning everything
from online shopping and auction sites to pages devoted to news and stock market data. Unlike Google
Japan's simple landing page, the Yahoo Japan portal contains a large array of icons and links.

Microsoft's Bing search engine, which tries to put search results in better context than its rivals
do, has won favorable reviews. But as has happened elsewhere in the world, it has failed to gain
much of a foothold in Japan.

Microsoft and Google have been vying for alliances with Yahoo for years.

In the spring of 2008, Microsoft made a \$47.5 billion hostile offer to buy Yahoo after on-and-off
talks about a merger had led nowhere. Google sought to torpedo the merger and began talks with Yahoo
on an advertising partnership. They reached an agreement, but it was abandoned amid antitrust
concerns in the United States. Microsoft ultimately walked away from its offer for Yahoo, and in
June 2009 the two sides agreed on the search partnership.

Akira Kajikawa, chief financial officer of Yahoo Japan, said that Yahoo backed the Google deal
announced Tuesday and that there would be no changes to Yahoo Japan's relationship with Yahoo. Jerry
Yang, Yahoo's co-founder and former chief executive, will keep his seat on Yahoo Japan's board,
Mr.~Kajikawa said.

In a blog post, Daniel Alegre, Google's vice president for the Asia-Pacific region and Japan,
described the deal as one that would bring benefits to both sides. Mr.~Alegre also said the
partnership highlighted a lesser-known part of Google's business: the licensing of Google technology
to rival search engines.

Yahoo Japan will ``be able to customize Google search to provide its own search service, one that
fits its own users,'' Mr.~Alegre said.

It is unclear what Yahoo Japan's move might mean for the Yahoo brand in other countries. Softbank
also owns shares in Alibaba Group, which operates the Yahoo portal in China.

Google was engaged in a lengthy confrontation with Chinese regulators this year over censorship
issues and alleged hacking. The government recently renewed Google's license to operate its
www.google.cn Web site, although the company is now largely pointing users in mainland China to its
site in Hong Kong, which is not censored.

\section{As China's Economy Grows, Pollution Worsens Despite New Efforts to Control It}

\lettrine{C}{hina}\mycalendar{Jul.'10}{29}, the world's most prodigious emitter of greenhouse gas,
continues to suffer the downsides of unbridled economic growth despite a raft of new environmental
initiatives.

The quality of air in Chinese cities is increasingly tainted by coal-burning power plants, grit from
construction sites and exhaust from millions of new cars squeezing onto crowded roads, according to
a government study issued this week. Other newly released figures show a jump in industrial
accidents and an epidemic of pollution in waterways.

The report's most unexpected findings pointed to an increase in inhalable particulates in cities
like Beijing, where officials have struggled to improve air quality by shutting down noxious
factories and tightening auto emission standards. Despite such efforts, including an ambitious
program aimed at reducing the use of coal for home heating, the average concentration of
particulates in the capital's air violated the World Health Organization's standards more than 80
percent of the time during the last quarter of 2008.

``China is still facing a grave situation in fighting pollution,'' Tao Detian, a spokesman for the
Ministry of Environmental Protection, told the official China Daily newspaper.

The ministry said the number of accidents fouling the air and water doubled during the first half of
2010, with an average of 10 each month. The report also found that more than a quarter of the
country's rivers, lakes and streams were too contaminated to be used as drinking water. Acid rain,
it added, has become a problem in nearly 200 of the 440 cities it monitored.

In recent days, the state media have provided a grim sampling of China's environmental woes,
including a pipeline explosion that dumped thousands of gallons of oil into the Yellow Sea, reports
of a copper mine whose toxic effluent killed tons of fish in Fujian Province, and revelations that
dozens of children were poisoned by lead from illegal gold production in Yunnan Province.

Two weeks ago, the state media reported on thousands of residents in the Guangxi Zhuang Autonomous
Region who clashed with police as they protested against unregulated emissions from an aluminum
plant.

Ma Jun, director of the Institute of Public and Environmental Affairs in Beijing, said many of the
government's efforts to curtail pollution had been offset by the number of construction projects
that spit dust into the air and the surge in private car ownership.

In Beijing, driving restrictions that removed a fifth of private cars from roads each weekday have
been offset by 250,000 new cars that hit the city streets in the first four months of 2010.

Many of the most polluting industries were forced to relocate far from the capital before the 2008
Summer Olympics, but the wind often carries their emissions hundreds of miles back.

``We're at a stage of unprecedented industrialization, but there have to be better ways to handle
the problem,'' said Mr.~Ma, whose organization has a registry of environmental scofflaws.
``Sometimes it's painful to look at the data.''

A particularly hot summer has added to Beijing's high pollution levels.

Even if they are fond of griping about bad air, Beijing residents have learned to take it all in
stride. Looking wilted amid the heat and haze Wednesday, Wang Dong, 34, a livery-cab driver, said he
tried to counteract the smog by eating more vegetables and drinking more water. Annie Chen, 26, a
sales clerk, revealed a tactic she had learned on television: apply an extra layer of makeup to
protect skin from contaminated air.

Then there was Zhang Hedan, 46, a street vendor who was fanning his flushed face with a piece of
paper. ``Maybe it will blow away the dust,'' he said hopefully. He added: ``Well, maybe that's not
so effective, but at least I feel better psychologically.''

\section{In Price War, New Kindle Sells for \$139}

\lettrine{A}{mazon}\mycalendar{Jul.'10}{29}.com will introduce two new versions of the Kindle
e-reader on Thursday, one for \$139, the lowest price yet for the device.

Amazon is hoping to convince even casual readers that they need a digital reading device. By firing
another shot in an e-reader price war leading up to the year-end holiday shopping season, the
e-commerce giant turned consumer electronics manufacturer is also signaling it intends to do battle
with Apple and its iPad as well as the other makers of e-readers like Sony and Barnes \& Noble.

Unlike previous Kindles, the \$139 ``Kindle Wi-Fi'' will connect to the Internet using only Wi-Fi
instead of a cellphone network as other Kindles do. Amazon is also introducing a model to replace
the Kindle 2, which it will sell for the same price as that model, \$189. Both new Kindles are
smaller and lighter, with higher contrast screens and crisper text.

``The hardware business for us has been so successful that we're going to continue,'' Jeffrey
P.~Bezos, Amazon's chief executive, said in an interview at the company's headquarters. ``I predict
there will be a 10th-generation and a 20th-generation Kindle. We're well-situated to be experts in
purpose-built reading devices.''

When Amazon introduced the Kindle in 2007, Mr.~Bezos described it as a must-have for frequent
travelers and people who read ``two, three, four books at the same time.'' Now, Amazon hopes that at
\$10 less than the least expensive reading devices from Barnes \& Noble and Sony, the new Kindle has
broken the psychological price barrier for even occasional readers or a family wanting multiple
Kindles.

``At \$139, if you're going to read by the pool, some people might spend more than that on a
swimsuit and sunglasses,'' Mr.~Bezos said.

Some analysts are predicting that e-readers could become this year's hot holiday gift. James
L.~McQuivey, a principal analyst specializing in consumer electronics at Forrester Research, said a
price war could for the first time reduce at least the price of one e-reader to under \$100, often
the tipping point for impulse gadget purchases.

Amazon has slashed the price of the Kindle at a speed that is unusual, even for electronic gadgets.
By last year, the price of the device was to \$259, down from its starting price of \$399 in late
2007. In June, hours after Barnes \& Noble dropped the price of its Nook e-reader to \$199, Amazon
dropped the price of the Kindle to \$189. The Kindle DX, which has a larger, 9.7-inch screen, is
\$379.

With Amazon's latest announcement, it is again waging a price war. Barnes \& Noble offers a Wi-Fi
version of the Nook for \$149 and Sony offers the Reader Pocket Edition, which does not have Wi-Fi,
for \$150.

Of course, price is just one factor people consider before making a purchase. The quality of the
product, adequate inventory and appealing marketing are just as important, said Eric T.~Anderson, a
professor of marketing at Northwestern University's Kellogg School of Management.

But as the e-reader marketplace has grown crowded ``there are lots of substitutes out there so the
only way they can create demand is by lowering the price,'' he said.

Still, the iPad's \$499-and-up price tag has not stifled demand for that device. Though the iPad
does much more than display books, customers often choose between the two, and are willing to pay
much more for the iPad because it is an Apple product, said Dale D.~Achabal, executive director of
the Retail Management Institute at Santa Clara University. ``The price point Apple can go to is
quite a bit higher than the price point other firms have to go to that don't have the same ease of
use, design and functionality,'' he said.

Apple says it has sold 3.3 million iPads since introducing it in April. Amazon does not release
Kindle sales figures, but says that sales tripled in the month after its last price cut.

Two of the most compelling aspects of the iPad -- a color display and touch screen -- are elements
that some customers have been yearning for on the Kindle. Keep waiting, Mr.~Bezos said.

``There will never be a Kindle with a touch screen that inhibits reading. It has to be done in a
different way. It can't be a me-too touch screen,'' he said. Earlier this year, Amazon bought
Touchco, a start-up specializing in touch-screen technology, but current touch-screen technology
adds reflections and glare and makes it hard to shift one's hands while reading for long periods of
time, he said. Color is also ``not ready for prime time,'' Mr.~Bezos said.

The new Kindles, which will ship Aug.~27, have the same six-inch reading area as earlier Kindles but
weigh about 15 percent less and are 21 percent smaller. The Kindles have twice the storage, up to
3,500 books.

\section{Judge Blocks Disputed Parts of Immigration Law in Arizona}

\lettrine{A}{}\mycalendar{Jul.'10}{29} federal judge on Wednesday blocked the most controversial
parts of Arizona's immigration enforcement law from going into effect, a ruling that at least
temporarily squashed a state policy that had inflamed the national debate over immigration.

Judge Susan Bolton of Federal District Court issued a preliminary injunction against sections of the
law, scheduled to take effect on Thursday, that called for police officers to check a person's
immigration status while enforcing other laws and required immigrants to prove that they were
authorized to be in the country or risk state charges. She issued the injunction in response to a
legal challenge brought against the law by the Obama administration.

A spokesman for Gov. Jan Brewer, a Republican who signed the law and has campaigned on it for
election to a full term, said Wednesday that the governor would appeal the injunction on Thursday
and ask for a speedy review. Legal experts predicted that the case could end up before the Supreme
Court.

The law, designed to seek and deport illegal immigrants in a state that is the principal gateway for
illegal border crossers, had provoked intense debate from coast to coast, drawing support in several
polls but generating boycotts of the state by major civil rights groups and several cities and
towns.

It renewed calls for an overhaul of federal immigration law and led to repeated rebukes of it from
President Obama and Attorney General Eric H.~Holder Jr., who maintained that immigration policy is
under the purview of the federal government, not individual states. The Mexican government, joined
by seven other Latin American nations, supported one of the lawsuits against the law; the attorneys
general of several states backed Arizona.

The ruling came four days before 1,200 National Guard members were scheduled to report to the
Southwest border to assist federal and local law enforcement agencies there, part of the Obama
administration's response to growing anxiety over the border and immigration that has fed support
for the law.

Judge Bolton, appointed to the bench by President Bill Clinton, did allow some, less debated
provisions of the law to go into effect, including one that bans cities from refusing to cooperate
with federal immigration agents.

But she largely sided with arguments in a lawsuit by the Obama administration that the law, rather
than closely hewing to existing federal statutes, as its supporters have claimed, interferes with
longstanding federal authority over immigration and could lead to harassment of citizens and legal
immigrants.

``Preserving the status quo through a preliminary injunction is less harmful than allowing state
laws that are likely pre-empted by federal law to be enforced,'' she said.

``There is a substantial likelihood that officers will wrongfully arrest legal resident aliens,''
she wrote. ``By enforcing this statute, Arizona would impose,'' she said, citing a previous Supreme
Court case, a `` 'distinct, unusual and extraordinary' burden on legal resident aliens that only the
federal government has the authority to impose.''

The judge's decision was not her final word on the case. In granting the injunction, she simply
indicated that the Justice Department was likely, but not certain, to prevail on those points at a
later trial in federal court. She made no ruling on the six other suits that also challenged the
law.

Her ruling, issued as demonstrators both for and against the law gathered here, and after hearings
in three of the seven lawsuits against the it, seemed more likely to add another log to the fire
than settle matters.

``This fight is far from over,'' said Ms.~Brewer, whose lawyers had argued that Congress granted
states the power to enforce immigration law particularly when, in their view, the federal government
fell short. ``In fact,'' she added, ``it is just the beginning, and at the end of what is certain to
be a long legal struggle, Arizona will prevail in its right to protect our citizens.''

State Senator Russell Pearce, a Republican and chief sponsor of the law, said in a statement that he
was confident that the sections blocked by Judge Bolton would survive on appeal, noting the state's
previous victories in court on other statutes designed to give it a larger role in immigration
enforcement. ``The courts have made it clear states have the inherent power to enforce the laws of
this country,'' he said.

But Gabriel Chin, a professor at the University of Arizona School of Law who has studied the law,
called the ruling ``a nearly complete victory for the position of the United States.''

He noted that she ruled in the federal government's favor on most of the points it challenged.

Aside from stopping the requirement that the police initiate immigration checks, the judge also
blocked provisions that allowed the police to hold anyone arrested for any crime until his
immigration status was determined.

``Requiring Arizona law enforcement officials and agencies to determine the immigration status of
every person who is arrested burdens lawfully present aliens because their liberty will be
restricted while their status is checked,'' she wrote.

She also said Arizona could not make it a state crime for noncitizens to be in the state without
proper documents, nor could it allow the police to conduct arrests without warrants if officers
believed the offense would result in their deportation. She said there was a ``substantial
likelihood'' of wrongful arrests.

The parts of the law she did allow were not challenged by the Justice Department, but do figure in
some of the other lawsuits filed. They include forbidding ``sanctuary city'' policies by allowing
residents to sue the local authorities if they adopt policies restricting cooperation with the
federal government in immigration enforcement.

She also let stand a provision aimed at day laborers, who are mostly Latin American immigrants, by
making it a crime to stop a vehicle in traffic or block traffic to hire someone off the street. But
she blocked a provision that barred illegal immigrants from soliciting work in public places.

The law, adopted in April, coincided with economic anxiety and followed a number of high-profile
crimes attributed to illegal immigrants and smuggling. It has become an issue in Congressional and
local campaigns across the country.

Terry Goddard, the Arizona attorney general who opposed the law and is a possible Democratic
opponent to Ms.~Brewer, was quick to condemn her for signing it. ``Jan Brewer played politics with
immigration, and she lost,'' he said in a statement.

But Republican candidates, including Senator John McCain of Arizona, who is seeking re-election,
criticized the Obama administration for bringing suit.

``Instead of wasting taxpayer resources filing a lawsuit against Arizona and complaining that the
law would be burdensome,'' Mr.~McCain said in a joint statement with Senator Jon Kyl, Republican of
Arizona, ``the Obama administration should have focused its efforts on working with Congress to
provide the necessary resources to support the state in its efforts to act where the federal
government has failed to take responsibility.''

\section{From Joy on the Fields of England to Joy Among the Thoroughbreds}

\lettrine{E}{very}\mycalendar{Jul.'10}{29} athlete knows the feeling, the fear of the void that
comes as early as midlife. You are 30-something, you have put your body on the line since childhood,
but the responses no longer meet the demands, and any amount of money you might have stored up will
not replace the buzz of competing with the best.

``If you don't have anything to get up for in the mornings, you might as well be dead,'' was how
Mick Channon summed up the end of a soccer career that stretched from 1965 to 1987. ``The tears, the
highs, the lows -- you miss it all.''

But it is still high summer in Channon's ultracompetitive life.

At 61, he channels the urge to be a winner that once drove him to score goals for England into
making others stretch every nerve and sinew. But Channon is not another soccer manager. Nothing as
simple, or as beholden to others, as that.

He trains racehorses. This week, last week, next October in Longchamp for the Prix de l'Arc de
Triomphe, December in Hong Kong, Dubai, Australia, Canada \ldots he is up with the dawn, out on the
gallops, watching, probing, plotting how to get the thoroughbreds in his care first past the post.

Channon has turned a lifelong hobby into an obsession. He saddles 200 horses at his stables and
averages more than 100 winners a season, but he has an affinity with one particular animal.

Youmzain is an Arabian stallion that Channon insists is the most genuine horse he has ever been
associated with. Others at his stables in England's West Country say Youmzain is a lovable rogue who
would, literally, bite the hand that feeds him.

When a BBC racing commentator last Saturday suggested that the horse was getting too long in the
tooth to be considered a Classic winner, Channon defended Youmzain by saying, ``He doesn't think
he's 7 years old, I can tell you that.''

``It's a question of whether the body can do what the brain tells it, but he's one horse that never
lets us down,'' he said.

Youmzain ran third that day. He was behind Harbinger, which won the \$1.6 million prize money in the
King George VI and Queen Elizabeth Stakes in a style the experts rated among the most impressive
gallops ever seen at Ascot racetrack.

And Youmzain keeps on coming up against inspired horses on the world's big race days. Three times,
he has finished runner-up at the Arc de Triomphe, one of the most prestigious races in the world.

Many a French horse lover is waiting to invest in Youmzain's offspring the day he is retired to
stud. But Channon, who bred his first racehorse while he was still a lean and coltish striker for
Southampton and England, is convinced that the horse has a winning Arc de Triomphe performance in
him.

``Longchamp in the autumn,'' Channon said on his stable's Web site, ``is a long way from Ascot,
Epsom or the Curragh in midsummer and Youmzain is an incredibly popular horse in France for three
very good reasons. We'll go straight there now for what could be his final public appearance.''

Channon's laughter, his countryman's lilt, his thirst for life appear as pronounced today as they
were when he was a player. His gait, of course, has slowed -- not just because he is no longer the
tiptop athlete with the black mane, but because two years ago he survived a devastating car crash.

It killed the driver, his longtime friend Tim Corby, who also was a horse-breeding agent. Channon
broke his ribs, punctured a lung and had to have a shattered jaw wired shut.

His best therapy was getting back to the stables at West Ilsley, west of London, where Brigadier
Gerard, one of the most famous racehorses, once galloped over the land, which is the highest point
in the county.

Who would have thought that Mick Channon, a gifted player who by his own admission was far from
enthusiastic about practicing, would end up training and breeding horses? Who would have foreseen
that the lad who once dodged a club tour to Dubai because it clashed with a racing festival would
one day be the master of all the disciplines that make up one of the world's top trainers?

``There's nothing complicated about horses,'' he said at the start of his second career. ``A lot of
it is common sense and working hard at the basics -- horse care, feeding and general discipline
throughout your stable.'' He described the job then as little different from managing players, but
added, ``Except the horses don't answer back.''

``What pleases me most is that at the end of the day, the buck stops with me,'' Channon said. ``In
soccer, a manager still has to be responsible to directors.''

Channon was once offered the job of head coach under the club manager at Wolverhampton Wanderers,
but in his typical forthright manner, he told the board it could not appoint a coach without first
appointing a manager.

He chose the horses instead. He started with a small string of 10 untried animals, owned by his
fellow players Kevin Keegan, Alan Ball and later the Manchester United Manager Alex Ferguson, and
with support from businessmen he met along the way and the celebrity photographer Koo Stark.

Channon as a player was once described by a more experienced Southampton teammate, Jimmy Gabriel, as
being someone to whom the most difficult things came easy, so much so that he lacked concentration
at the simple things.

But his disarming honesty, his competitive spirit, his willingness to work hardest when the victory
seemed most elusive, shone through. So did his knack of scoring vital goals and celebrating with his
right arm rotating like a windmill.

In retrospect, Channon's first career did strike those who enjoyed it to be a product of instinctive
skill and a huge desire to confront issues and win the unwinnable contests. He stayed, for example,
many more years with his first club, Southampton, than many might have expected, given that he
racked up 46 games and scored 21 goals for the national team, some of them while Southampton was
only a second-division club. The club was closest to his hometown, and his stables today are closer
still.

But the task he set himself, and the sense of fun he exudes in doing it, is nothing like the
uncomplicated thing he says it is. He knows, from the arthritis that set into his joints despite a
long career without major injury or surgery, that ability without discipline is worthless.

He knows, too, that a top trainer is lost without a team dedicated to the athletes -- in this case
equine -- in its care.

Channon enjoys, most of all, the challenge of humans' trying to connect to the equine mind.
Something to get up for in the mornings.

\section{The Case for \$320,000 Kindergarten Teachers}

\lettrine{H}{ow}\mycalendar{Jul.'10}{29} much do your kindergarten teacher and classmates affect the
rest of your life?

Economists have generally thought that the answer was not much. Great teachers and early childhood
programs can have a big short-term effect. But the impact tends to fade. By junior high and high
school, children who had excellent early schooling do little better on tests than similar children
who did not -- which raises the demoralizing question of how much of a difference schools and
teachers can make.

There has always been one major caveat, however, to the research on the fade-out effect. It was
based mainly on test scores, not on a broader set of measures, like a child's health or eventual
earnings. As Raj Chetty, a Harvard economist, says: ``We don't really care about test scores. We
care about adult outcomes.''

Early this year, Mr.~Chetty and five other researchers set out to fill this void. They examined the
life paths of almost 12,000 children who had been part of a well-known education experiment in
Tennessee in the 1980s. The children are now about 30, well started on their adult lives.

On Tuesday, Mr.~Chetty presented the findings -- not yet peer-reviewed -- at an academic conference
in Cambridge, Mass. They're fairly explosive.

Just as in other studies, the Tennessee experiment found that some teachers were able to help
students learn vastly more than other teachers. And just as in other studies, the effect largely
disappeared by junior high, based on test scores. Yet when Mr.~Chetty and his colleagues took
another look at the students in adulthood, they discovered that the legacy of kindergarten had
re-emerged.

Students who had learned much more in kindergarten were more likely to go to college than students
with otherwise similar backgrounds. Students who learned more were also less likely to become single
parents. As adults, they were more likely to be saving for retirement. Perhaps most striking, they
were earning more.

All else equal, they were making about an extra \$100 a year at age 27 for every percentile they had
moved up the test-score distribution over the course of kindergarten. A student who went from
average to the 60th percentile -- a typical jump for a 5-year-old with a good teacher -- could
expect to make about \$1,000 more a year at age 27 than a student who remained at the average. Over
time, the effect seems to grow, too.

The economists don't pretend to know the exact causes. But it's not hard to come up with plausible
guesses. Good early education can impart skills that last a lifetime -- patience, discipline,
manners, perseverance. The tests that 5-year-olds take may pick up these skills, even if later
multiple-choice tests do not.

Now happens to be a particularly good time for a study like this. With the economy still terribly
weak, many people are understandably unsure about the value of education. They see that even college
graduates have lost their jobs in the recession.

Barely a week seems to go by without a newspaper or television station running a report suggesting
that education is overrated. These stories quote liberal groups, like the Economic Policy Institute,
that argue that an education can't protect workers in today's global economy. Or they quote
conservatives, like Charles Murray and Ramesh Ponnuru, who suggest that people who haven't graduated
from college aren't smart enough to do so.

But the anti-education case usually relies on a combination of anecdotes and selective facts. In
truth, the gap between the pay of college graduates and everyone else grew to a record last year,
according to the Labor Department, and unemployment has risen far more for the less educated.

This is not simply because smart people -- people who would do well no matter what -- tend to
graduate from college. Education itself can make a difference. A long line of economic research, by
Julie Berry Cullen, James Heckman, Philip Oreopoulos and many others, has found as much. The study
by Mr.~Chetty and his colleagues is the latest piece of evidence.

$\bullet$

The crucial problem the study had to solve was the old causation-correlation problem. Are children
who do well on kindergarten tests destined to do better in life, based on who they are? Or are their
teacher and classmates changing them?

The Tennessee experiment, known as Project Star, offered a chance to answer these questions because
it randomly assigned students to a kindergarten class. As a result, the classes had fairly similar
socioeconomic mixes of students and could be expected to perform similarly on the tests given at the
end of kindergarten.

Yet they didn't. Some classes did far better than others. The differences were too big to be
explained by randomness. (Similarly, when the researchers looked at entering and exiting test scores
in first, second and third grades, they found that some classes made much more progress than
others.)

Class size -- which was the impetus of Project Star -- evidently played some role. Classes with 13
to 17 students did better than classes with 22 to 25. Peers also seem to matter. In classes with a
somewhat higher average socioeconomic status, all the students tended to do a little better.

But neither of these factors came close to explaining the variation in class performance. So another
cause seemed to be the explanation: teachers.

Some are highly effective. Some are not. And the differences can affect students for years to come.

When I asked Douglas Staiger, a Dartmouth economist who studies education, what he thought of the
new paper, he called it fascinating and potentially important. ``The worry has been that education
didn't translate into earnings,'' Mr.~Staiger said. ``But this is telling us that it does and that
the fade-out effect is misleading in some sense.''

Mr.~Chetty and his colleagues -- one of whom, Emmanuel Saez, recently won the prize for the top
research economist under the age of 40 -- estimate that a standout kindergarten teacher is worth
about \$320,000 a year. That's the present value of the additional money that a full class of
students can expect to earn over their careers. This estimate doesn't take into account social
gains, like better health and less crime.

Obviously, great kindergarten teachers are not going to start making \$320,000 anytime soon. Still,
school administrators can do more than they're doing.

They can pay their best teachers more, as Pittsburgh soon will, and give them the support they
deserve. Administrators can fire more of their worst teachers, as Michelle Rhee, the Washington
schools chancellor, did last week. Schools can also make sure standardized tests are measuring real
student skills and teacher quality, as teachers' unions have urged.

Given today's budget pressures, finding the money for any new programs will be difficult. But that's
all the more reason to focus our scarce resources on investments whose benefits won't simply fade
away.

\section{Job Subsidies Also Provide Help to Private Sector}

\lettrine{S}{tates}\mycalendar{Jul.'10}{29} are putting hundreds of thousands of people directly
into jobs through programs reminiscent of the more ambitious work projects of the Great Depression.

But the new efforts have a twist: While the wages are being paid by the government, most of the
participants are working for private companies.

The opportunity to simultaneously benefit struggling workers and small businesses has helped these
job subsidies gain support from liberals and conservatives. Congress is now considering whether to
extend the subsidy, which would expire in September, for an additional year. A House vote is
expected on Thursday or Friday.

Despite questions about whether the programs displace existing workers, many economists have argued
that direct job creation programs are a more cost-effective way to put some of the nation's 14.6
million unemployed back to work than indirect alternatives like tax credits and construction
projects.

The average duration of unemployment continues to break records, after all, and studies have shown
that the longer people are out of work, the less employable they become.

``I never, ever, ever thought I'd end up in an art gallery,'' said Tremaine Edwards, 35, a former
computer technician who had been unemployed for two years before he was hired in May by Gallery
Guichard, a private gallery in Chicago. Mr.~Edwards now earns \$10 an hour, financed by the
government, through the Put Illinois to Work program, to maintain the company's Web site, curate
exhibits and run gallery events.

He has also become the gallery's star salesman, selling five paintings during the most recent
gallery opening despite no background in fine arts or sales.

``I feel like if I knew I could have done this 15 years ago, I would have,'' he said, grateful for
the opportunity to escape cubicle life. ``As long as I keep selling like this, I think I'll be fine,
no matter what happens with Put Illinois to Work.''

Proponents of these national job subsidies, initially financed with \$5 billion of stimulus money,
say it is better to pay people for working in real jobs than to pay them jobless benefits for
staying idle.

Placing workers in the private sector is also more promising than giving them make-work government
jobs, they say, because market forces can be harnessed to figure out where people like Mr.~Edwards
should invest their skills for the long run.

Others contend that training and financing will accomplish little if businesses are unwilling to
hire on their own. They argue that government policies should instead be encouraging business growth
robust enough to create jobs independently.

The effectiveness of these programs will not be clear for many months, if ever. As the stimulus
money dries up, employers will decide whether to keep the workers at their own cost or cast them
back into the unemployment pool. Moreover, some economists fear that people hired with government
subsidies may simply be displacing other workers, rather than adding to total employment, no matter
how earnestly the programs are policed.

``There's always a concern that the employer or somebody else who hires them would have simply hired
someone else,'' says David Card, an economist at the University of California, Berkeley.

About 247,000 workers will have been placed in these subsidized jobs by the end of September,
according to the Center on Budget and Policy Priorities, a research organization. The jobs cover
everything from assembly-line work to white-collar positions like business development, and
typically pay \$8 to \$15 an hour, according to LaDonna Pavetti, a director at the center. There are
exceptions: San Francisco, for example, pays up to \$74,000 in annual salary, which employers can
also supplement with additional pay.

So far just over a billion dollars has been approved to create subsidized employment programs in 36
states and the District of Columbia, according to the Department of Health and Human Services. The
biggest year-round program is run by Illinois, which has put 22,000 workers in subsidized jobs (and
5,000 in subsidized summer youth jobs) and has 30,000 people on its waiting list.

Most states pay 100 percent of workers' wages up to a certain point. To qualify for the subsidy,
workers must have a low household income. They must also have minor children, or be under age 21
themselves. Employers seem to hear about the programs largely through word of mouth, and some states
actively help match eligible workers with companies.

These eligibility restrictions are part of the Temporary Assistance for Needy Families program,
created in the welfare reform legislation of 1996, which is now being used to channel the job
subsidy money to states.

States, cities and local companies have been among the biggest advocates for extending the program
beyond September. Small businesses benefit in particular since having an additional worker can make
a bigger difference to a company with a small staff. Getting a worker at no cost can also free up
cash for other types of investments.

``We have been saying to small businesses, 'This, finally, is the bailout program designed for you,'
'' says James Whelly, deputy director of work force development at the San Francisco Human Services
Agency.

Gallery Guichard, which sells paintings and sculptures of the African diaspora on Chicago's South
Side, has hired five employees, who are all college educated or are enrolled in college, through the
Illinois program. While only one had any experience in fine arts, the gallery owners gush about the
group's value to the company.

``But not for Put Illinois to Work, we would not have been able to get this injection of young new
talent and start to expand again,'' said Andr\'e Guichard, one of the owners.

Gallery revenues fell by 40 to 60 percent last year, Mr.~Guichard said, as collectors stopped buying
its works (generally, at least \$1,000 a painting) and stalled on payment plans. As a result, the
gallery began cutting its hours from seven days a week to six days a week to, begrudgingly, by
appointment only.

More worrisome, it had fewer exhibit openings and other events, which Mr.~Guichard says are the
source of most sales.

Since bringing in the new workers, he says, the gallery has had more major events than ever,
including back-to-back openings in July.

The business has been expanding in other ways, too. Mr.~Guichard's wife and business partner,
Frances, is starting a sales program tailored to corporate buyers. And the couple is traveling to
South Africa this week to hunt for undiscovered artists, something they have never done outside of
the United States. This is possible, they say, not only because they have saved enough money on
labor costs to afford the trip, but also because they now have sufficient staff to run the gallery
in their absence.

Assessing the success of these subsidies, and the sustainability of the jobs they create, is
complicated.

Mr.~Guichard said he expected to keep three of the five new workers after the program ends. But the
month before it began using the program, the gallery had four employees on its payroll (in addition
to a few who worked hours as needed); none of these workers are still there. Three left on their
own, said Mr.~Guichard, in part because they were frustrated after their hours were cut and their
income fluctuated.

But one -- Mr.~Guichard's cousin Juan Rodriguez -- was laid off.

Mr.~Guichard said he wanted to keep Mr.~Rodriguez, a 24-year-old, precocious curator and a ``hard
worker,'' but decided not to because Mr.~Rodriguez did not qualify for Put Illinois to Work.

Instead, Mr.~Guichard hired Mr.~Rodriguez's younger brother, Patrick, whom the gallery now can
employ free.

The older brother is pondering a move to Atlanta, where he hopes to find more job opportunities.
Because the program has helped his younger brother, who had been out of work a year, he tries not to
harbor a grudge.

``I think it's basically a good program, but it needs to be somewhat less restrictive basically,''
Juan Rodriguez said. ``Still, the whole idea is to put people to work, but it's a situation where
it's actually put me out of work instead.''

\section{Immigrant Maids Flee Lives of Abuse in Kuwait}

\lettrine{W}{ith}\mycalendar{Aug.'10}{02} nowhere else to go, dozens of Nepalese maids who fled from
their employers now sleep on the floor in the lobby of their embassy here, next to the visitors'
chairs.

In the Philippines Embassy, more than 200 women are packed in a sweltering room, where they sleep on
their luggage and pass the time singing along to Filipino crooners on television. So many runaways
are sheltering in the Indonesian Embassy that some have left a packed basement and taken over a
prayer room.

And in the coming weeks, when Ramadan starts, the number of maids seeking protection is expected to
grow, perhaps by the hundreds, straining the capacity of the improvised shelters, embassy officials
say. With Kuwaiti families staying up into the early hours of the morning, some maids say they cook
more, work longer hours and sleep less.

Rosflor Armada, who is staying in the Philippines Embassy, said that last year during Ramadan, she
cooked all day for the evening meal and was allowed to sleep only about two hours a night.

``They said, `You will work. You will work.' '' She said that she left after her employers demanded
that she wash the windows at 3 a.m.

The existence of the shelters reflects a hard reality here: With few legal protections against
employers who choose not to pay servants, who push them too hard, or who abuse them, sometimes there
is nothing left to do but run. The laws that do exist tend to err on the side of protecting
employers, who often pay more than \$2,000 upfront to hire the maids from the agencies that bring
the women here.

The problems in Kuwait, including a lack of legal protection, are hardly unusual or even regional;
this summer, New York became the first state to grant workplace rights to domestic employees in an
effort to prevent sexual harassment and other abuses. But human rights groups say the potential for
mistreatment is acute in several countries in the Middle East, especially those with large numbers
of migrant workers who rely on a sponsorship system that makes employers responsible for the welfare
of their workers.

That system is particularly entrenched in Kuwait, where oil riches allow many families to have
several servants, human rights advocates say. And conditions for some workers here are bad enough
that the United States Department of State in a 2010 report singled out Kuwait, along with 12 other
countries, for failing to do enough to prevent human trafficking.

The report noted that migrants enter Kuwait voluntarily, but ``upon arrival some are subjected to
conditions of forced labor by their sponsors and labor agents, including through such practices as
nonpayment of wages, threats, physical or sexual abuse, and restrictions on movement, such as the
withholding of passports.''

The informal shelters here are open secrets and touchy subjects.

Embassy officials are loath to talk about them and generally do not allow visitors, citing concerns
about the privacy of the women and a reluctance to antagonize Kuwaiti officials, whose cooperation
they need in order to repatriate many of the women. The government runs a shelter for about 50
women, but few domestic workers know about the place, according to their advocates.

Kuwaiti officials say that an overwhelming majority of the country's approximately 650,000 domestic
workers are treated well and are considered part of the families that employ them. Some bristle at
the notion that Ramadan is more taxing.

Mohammed al-Kandari, under secretary in the Ministry of Social Affairs and Labor, said many maids
received extra money from their employers during Ramadan.

``They get benefits. Their expenses and food is paid for, and they don't spend anything,'' he said.
``They send their salaries to their families. Some work here for 15, 20, 25 years.''

But even many of those who are not abused can lead lonely, Spartan lives with little time off. Some
employers forbid the women to socialize with friends, and the women themselves are often loath to
spend much money in their free time so they can save cash for the families they left behind in their
home countries.

The perils faced by many domestic workers were brought into sharp focus in recent weeks, when the
local news media reported that a Sri Lankan maid who fled to her embassy said she had been
imprisoned by her Kuwaiti employers, without pay, for 13 years. The Sri Lankan ambassador, Sarath
Dissanayake, refused a request to interview the woman and said hers was an isolated case.

Also last month, the news media reported that a Filipino maid was allegedly tortured and killed by
her employers, who the media said ran over her body with a car in the desert in order to make her
death look like an accident.

Human rights advocates say the problem of abuse persists because it is rarely punished. Domestic
workers are told to report offenses to the police, but the advocates say some employers quickly file
countercharges, accusing the maids of such offenses as stealing.

Lawmakers have been discussing new provisions to protect the workers, including a law that would
require employers to deposit salaries directly in bank accounts, but they have yet to act.

Talk of building a large shelter has circulated for years.

For now, the women rely on their embassies for shelter, along with some Kuwaitis and expatriates who
risk prosecution to house them.

In 2009, embassies in Kuwait received more than 10,000 complaints from domestic workers about unpaid
wages, long working hours and physical, sexual and psychological abuse, according to Priyanka
Motaparthy of Human Rights Watch, who wrote an as yet unpublished report on the conditions of
domestic workers in Kuwait.

Workers who flee harsh work conditions face the risk they will be charged with immigration
violations and imprisoned, or face prolonged detention or deportation, Ms.~Motaparthy said.

Alida Ali, a 22-year-old from the Philippines, described a different kind of punishment. She begged
her agency to move her from an abusive family, and when her employers found out, she said, they
threw her out of a third-floor window, breaking her back.

Ms.~Ali recently had a metal rod removed from her spine. She has been in the shelter in the
Philippines Embassy -- which considers her story credible -- for 10 months while lawyers pursued a
case against the employers. She lost the case, and now she just wants to go home.

Bibi Nasser al-Sabah, who runs an organization that advocates for domestic workers, said it would
take more than awareness campaigns to change the behavior of employers and agencies.

``This does not work,'' said Ms.~al-Sabah, who is a granddaughter of Kuwait's emir. ``People will
not change. It has to be imposed, through proper laws and strict rules -- by actions taken by the
government.''

\section{In India, Using Facebook to Catch Scofflaw Drivers}

\lettrine{T}{his}\mycalendar{Aug.'10}{02} city is famous for its snarled traffic and infamous for
its unruly drivers -- aggressive rule-breakers who barrel through red lights, ignore crosswalks and
veer into bicycle or bus lanes to find open routes.

Now, the city's overburdened traffic police officers have enlisted an unexpected weapon in the fight
against dangerous driving: Facebook.

The traffic police started a Facebook page two months ago, and almost immediately residents became
digital informants, posting photos of their fellow drivers violating traffic laws. As of Sunday more
than 17,000 people had become fans of the page and posted almost 3,000 photographs and dozens of
videos.

The online rap sheet was impressive. There are photos of people on motorcycles without helmets, cars
stopped in crosswalks, drivers on cellphones, drivers in the middle of illegal turns and improperly
parked vehicles.

Using the pictures, the Delhi Traffic Police have issued 665 tickets, using the license plate
numbers shown in the photos to track vehicle owners, said the city's joint commissioner of traffic,
Satyendra Garg.

Despite some concerns about privacy, and the authenticity of the photos, the public's response has
been overwhelmingly positive, he said.

Mr.~Garg said the Facebook page never told people to take pictures of lawless drivers. ``We wanted a
forum where people could express their views and suggest changes,'' he said Friday.

With just 5,000 traffic officers in this city of 12 million people, the social networking site is
filling a useful role, he said. ``Traffic police can't be present everywhere, but rules are always
being broken,'' Mr.~Garg said. ``If people want to report it, we welcome it. A violation is a
violation.''

Mr.~Garg acknowledged that it was possible photos could be manipulated to incriminate someone who
was not actually breaking the law. But, he said, drivers can contest the tickets if they think they
were wrongly issued. The police advise residents not to let personal animosity influence their
photo-taking, and not to do anything to compromise their own security, like antagonizing
law-breakers while snapping photos.

Some city residents have applauded the effort. ``This is a good use of police resources,'' said
Vijyant Jain, a 27-year-old manager with Orange Business Services, who drives a minivan. He posted
an alert on the Facebook page on Friday about a traffic obstruction.

``Up until now, any driver about to break traffic laws, including me, used to look around,''
Mr.~Jain said, to see if there was an officer nearby before doing so. Now, drivers will be much more
vigilant, he said, because ``it is not only traffic cops they need to worry about.''

Critics say these methods could set a dangerous precedent. Relying on people to turn in their
neighbors online is ``Orwellian,'' said Gaurav Mishra, chief executive of 2020 Social, a social
business consultancy based here.

``When you start using the Internet as way for the government to keep tabs on its citizens, I start
getting really worried, because you don't know where it will end,'' he said. The popularity of the
page shows that the ability to publicly humiliate wrongdoers ``taps into a very basic primal part of
who we are as human beings,'' Mr.~Mishra said, and it is not a pleasant one.

While the Facebook page reaches thousands of people, the vast majority of residents here are not
connected to it. Just one in four people in urban India has Internet access, and Internet users tend
to be the wealthiest. Facebook said in July that users from India passed the 12 million mark.

The authorities have embraced the Facebook informants in part because the dangers of driving in
India are ever-present. India has more traffic fatalities than any country in the world, and the
number of new, untrained drivers has skyrocketed in recent years as the Indian middle class grows.
The system of roads and the police are ill-equipped to handle the crush.

Nowhere is the problem more pronounced than in this traffic-choked city, which must contend with an
additional four million more people in the metro area on top of its own population. From the
beginning of the year until July 15, the police stopped 247,973 drivers who ran through city traffic
signals. At the beginning of 2010, there were 6.5 million motor vehicles registered in the city, and
road experts here estimate that it is adding about 1,000 motor vehicles each day.

The Delhi Traffic Police now have a dedicated team of four officers who monitor the Facebook page
around the clock, Mr.~Garg said. In addition to examining potential violations, they also post
information about closed roads and traffic jams, respond to tips about traffic snarls and answer
questions.

Almost 50 of the tickets issued based on photos on the site were given to police officers who were
breaking traffic rules, Mr.~Garg added.

Social networking services are playing a growing role in court cases and law enforcement, but the
Delhi Traffic Police's use of Facebook appears to be unique.

Dozens of police departments in the United States have Facebook pages, which are often used to keep
the public informed of changes in laws, warn them of dangers and solicit participants in
fund-raisers.

Some departments use Facebook to connect with residents and show the human side of the force. The
Houston Police Department, for example, has more than 16,600 followers, in part because of posts
about the ducks that join its cadets for roll call in the mornings, and photos of recent burglary
arrests taken through night-vision goggles.

On rare occasions, American police departments ask Facebook users to become involved in law
enforcement. The police in Baker, La., for instance, posted a photo on Facebook of a truck involved
in a theft, asking for tips. It was unclear whether the post had led to any arrest, but one user did
comment that the truck looked like one owned by a friend's brother.

In New Delhi, Mr.~Garg acknowledges that there are complications to issuing tickets based on
Facebook posts. People might use the site to settle scores, for example. But, he said, the response
has been positive so far, and he does not want to discourage anyone from posting photos.

He also had some practical advice for Delhi's would-be citizen traffic officers. ``We advise while
you are driving not to take a photo'' of a fellow driver who is breaking traffic laws, Mr.~Garg
said. Using a cellphone camera while driving ``in itself is a violation.''

\section{State-Owned Groups Fuel China's Real Estate Boom}

\lettrine{T}{he}\mycalendar{Aug.'10}{02} Anhui Salt Industry Corporation is a state-owned company
that has 11,000 employees, access to government salt mines and a Communist Party boss.

Now it has swaggered into a new line of business: real estate.

The company is developing a complex of luxury high-rises here called Platinum Bay on a parcel it
acquired last year by outbidding two other developers to win a local government land auction.

Anhui Salt is hardly alone among big state-owned companies. The China Railway Group is developing
residential complexes in Beijing after winning the auction for a huge piece of land there.

Likewise, the China Ordnance Group, a state-led military manufacturer best known for amphibious
assault weapons, paid \$260 million for Beijing property where it plans to build luxury residences
and retail outlets.

And in one of China's biggest land deals yet, the state-run shipbuilder Sino Ocean paid \$1.3
billion last December and March to buy two giant tracts from Beijing's municipal government to
develop residential communities.

All around the nation, giant state-owned oil, chemical, military, telecom and highway groups are
bidding up prices on sprawling plots of land for big real estate projects unrelated to their core
businesses.

``These are the ones that have the money to buy the land,'' says Prof.~Deng Yongheng at the National
University in Singapore. ``Because in China, it's the government that controls the money supply and
the spending.''

By driving up property prices, the state-owned companies, which are ultimately controlled by the
national government, are working at cross-purposes with the central government's effort to keep
China's real estate boom from becoming a debt-driven speculative bubble -- like the one that
devastated Western financial markets when it burst two years ago.

Land records show that 82 percent of land auctions in Beijing this year have been won by big
state-owned companies outbidding private developers -- up from 59 percent in 2008.

A recent study published by the National Bureau of Economic Research in Cambridge, Mass., found that
land prices in Beijing had jumped by about 750 percent since 2003, and that half of that gain came
in the last two years. Housing prices have also skyrocketed, doubling in many cities over the last
few years.

The report pegged a big part of the increase to state-owned enterprises that have ``paid 27 percent
more than other bidders for an otherwise equivalent piece of land.''

Critics say the central government in Beijing unwittingly propelled the land frenzy by pushing a
huge \$586 billion economic stimulus package last year and encouraging state-owned banks to lend
more aggressively.

And as the prices of new apartments soar -- in Shanghai, for instance, they often exceed \$200,000,
while the average disposable income is about \$4,000 a year -- the trend also threatens to undermine
the central government's goal of affordable housing for the rising middle class.

In some cases, local governments -- which earned over \$230 billion from land auctions in 2009 --
are also being accused of demolishing old neighborhoods and unfairly compensating residents. In a
recent poll conducted by China Youth Daily, a state-run newspaper, more than 80 percent of the
respondents said local governments were a ``major driving force'' behind the skyrocketing property
prices.

All of this is happening to the chagrin of private developers that dominated China's property market
for more than a decade but are now feeling squeezed out of a game that favors developers with
state-backed financing.

``It's a little like a son who borrows money from his mother,'' says Yang Shaofeng, head of the
Conworld Real Estate Agency in Beijing.

Last year, state banks made a record \$1.4 trillion in loans, nearly twice as much as the year
before. Analysts say they believe much of that money was diverted into the property market through
off-balance-sheet maneuvers, leading to the record land bids and soaring property prices. That
belief is adding to concerns that some of China's biggest state-owned banks may be sitting on
enormous unreported debt.

Beijing is now struggling to rein in credit without slowing the nation's roaring economy. And
regulators are trying to stop state banks from using clever maneuvers to secretly lend money to
overly aggressive state-owned developers.

Beijing also wants to restrain state companies that have little or no expertise in real estate. Last
March, the State Assets Supervision and Administration Commission -- one of the national
government's most powerful bodies -- ordered 78 state-owned companies to shed their real estate
divisions.

But analysts say the government will have difficulty stopping hundreds of state-owned companies and
their various subsidiaries from participating in what has become one of the country's hottest
industries. Experts say that more than 90 of the 125 state-owned companies directly under Beijing's
control still have property divisions. And local and provincial governments control many additional
developers.

The national government is grappling with a complex set of incentives that drive state-run companies
to speculate in the property market with the aid of local governments.

Rosealea Yao, an analyst at Dragonomics, a research consultant in Beijing, says a growing number of
municipalities have formed local investment vehicles that borrow heavily from state-owned banks to
pay to relocate residents and build infrastructure around big plots of land they intend to sell at
auction. (In China, local governments cannot directly borrow from banks or issue bonds for real
estate development.)

Those off-balance-sheet debts are essentially bets on rising land prices, she says, which could
become big liabilities if land prices were to decline sharply or the auction market were to dry up.

``This is why local governments are so enthusiastic about infrastructure,'' Ms.~Yao says. ``They
borrow to build something that raises the value of the land they want to sell at auction.''

Here in Wuhu, a sleepy industrial town about 70 miles west of Nanjing, Anhui Salt is breaking ground
on its high-rise project in the center of town -- next to a hotel operated by Anhui Conch Holdings.

The land was put up for auction in May 2009, and there were just three bidders -- another of which
was also a state-owned company. Anhui Salt, which also boasts of operating a steel trading arm, a
financing vehicle and even two Honda dealerships, says it is eager to expand beyond industrial
products and table salt.

``Platinum Bay is Anhui Salt Industry's first luxury project and targets the very rich, the very
elite class of Wuhu,'' said Su Chuanbo, marketing manager.

Asked why Anhui Salt wants to be a developer, Mr.~Su said the central government had encouraged
state companies to be more profitable, and that real estate was incredibly lucrative.

And so the government, he added, is actually behind its push into real estate.

``Even though many central government-controlled state companies are banned from the real estate
sector,'' Mr.~Su said, ``local state-owned companies like Anhui Salt can still develop its projects
within reasonable bounds. The situation is the same all over China.''

\section{Bing and Google in a Race for Search Features}

\lettrine{E}{dwin}\mycalendar{Aug.'10}{02} Perello discovered that Bing, the Microsoft search
engine, could find addresses in his rural Indiana town when Google could not. Laura Michelson, an
administrative assistant in San Francisco, was lured by Bing's flight fare tracker. Paul Callan, a
photography buff in Chicago, fell for Bing's vivid background images.

Like most Americans, they still use Google as their main search tool. But more often, they find
themselves navigating to Microsoft's year-old Bing for certain tasks, and sometimes they stay a
while.

``I was a Google user before, but the more I used Bing the more I liked it,'' Mr.~Callan said.
``It's more like muscle memory takes me to Google.''

Bing still handles a small slice of Web searches in the United States, 12.7 percent in June,
compared with Google's 62.6 percent, as measured by comScore, the Web analytics firm. But Bing's
share has been growing, as has Yahoo's, while Google's has been shrinking.

And while no one argues that Google's dominance is in immediate jeopardy, Google is watching
Microsoft closely, mimicking some of Bing's innovations -- like its travel search engine, its
ability to tie more tools to social networking sites and its image search -- or buying start-ups to
help it do so in the future.

Google has even taken on some of Bing's distinctive look, like giving people the option of a
Bing-like colorful background, and the placement of navigation tools on the left-hand side of the
page.

The result is a renaissance in search, resulting in more sophisticated tools for consumers who want
richer answers to complex questions than the standard litany of blue links.

The competition is a remarkable and surprising twist: Microsoft, knocked around for so long as a
bumbling laggard, has given the innovative upstart Google a kick in the pants. As the search engines
introduce feature after competing feature, some analysts say they have set off an arms race, with
the companies poised to spend whatever it takes to win the second phase of Web search.

``There is a cold war going on,'' said Sandeep Aggarwal, senior Internet and software analyst at
Caris \& Company, who watches both companies. ``Clearly, you can see how Bing's competition is
forcing Google to try and catch up in some places.''

Google officials agree there is more competition, but say they are not simply reacting to the
younger search engine.

Google's new features have not been in response to Bing, said Marissa Mayer, the company's vice
president for search products and user experience. ``A lot of these things have been in the works
for a long time,'' she said. ``Left-hand navigation we worked on for almost two years. We wanted to
make sure we had it exactly right.''

Microsoft's gains are far from staggering. Its share of searches has grown to 12.7 percent, from 8
percent, since Bing was introduced in May 2009, and Yahoo, which has a search deal with Microsoft,
still handles a larger share of searches than Bing. And in the newest search frontier, mobile
devices, Google has even more market share than on the Web at large.

Still, Bing's gains have impressed analysts, who have watched Google fend off repeated assaults on
its lucrative search and ad business, which accounts for some 95 percent of its revenue.

Building a more comprehensive, faster and more accurate search engine than Google is a daunting
challenge, and a long list of big companies and start-ups have failed in their attempts. Microsoft
endured plenty of ribbing as it spent years building and then scrapping search systems meant to help
it compete against Google. But it kept experimenting until it found a way.

Microsoft has spent billions of dollars building the computing centers needed to power search and
advertising systems and acquiring start-ups with niche expertise. In addition, it has thrown money
at consumers, through cash-back programs on purchases, and at partners willing to promote Bing ahead
of Google. Over the last year, Microsoft's online services division lost \$2.36 billion on revenue
of \$2.2 billion.

With Bing, Microsoft has tried to attract people like Mr.~Callan by excelling at answering
frequently asked questions, like those related to travel, health, shopping, entertainment and local
businesses. For example, Bing has flight search and prediction tools that reveal price fluctuations
for certain routes, and advises customers whether to buy or wait. Bing Health uses data from sources
like the Mayo Clinic and Healthwise.

The hope is that ``somebody would come back just for that and then, down the line, they would do
other types of searches, too,'' said Danny Sullivan, a longtime industry analyst and editor in chief
of the blog Search Engine Land.

People do not always want to click on links and dig through pages to hunt out information, so when
Bing started in May 2009, it pulled relevant information and stuck it on the top and left-hand side
of the results pages. Search ``Angelina Jolie,'' for instance, and see a slide show and a list of
her movies on top and related links on the side.

``We said, `Let's change the entire way we lay out pages,' '' said Yusuf Mehdi, a senior vice
president for Microsoft's online audiences business. ``We will not be shackled by blue links.''

Google, meanwhile, has quietly introduced its own new features that have in several instances looked
a lot like Bing's.

For example, in May, it too added the left-hand navigation tools -- though Ms.~Mayer of Google
pointed out that many of the tools had already been available, just not easily visible from the
search page.

``Certainly there's been increased competition in the space,'' Ms.~Mayer said of Bing. ``When
there's more competition, everyone's search gets better, that serves the users a lot better.''

Bing's travel tool uses technology from Farecast, which Microsoft bought in early 2008. In July,
Google announced plans to acquire ITA Software for \$700 million; ITA makes the same comparison
shopping software for flights that Bing's Farecast uses.

Then there is the look of the main search pages for each site. Microsoft has argued that the vivid
images ever-present behind the Bing search box have helped its appeal; young people and women have
shown a particular fondness for Bing. In June, Google offered people the option to have a colorful
background image like the Golden Gate Bridge on its main search page rather than the stark, white
page that helped make Google famous.

Google has also played catch-up to Microsoft in offering ways to search for and digest more images
in one go, and has trailed in adding some tie-ins to social networking sites.

``Google's new innovations have come at a slower pace,'' Mr.~Aggarwal said. ``There was no one
challenging Google until Microsoft decided it was a business they would not give up.''

Still, Mr.~Sullivan and other analysts also say Google has been making many significant but subtle
behind-the-scenes changes that make it better at responding to obscure and complex queries. Google
made 500 tweaks to its secret search algorithm last year and introduced personalized search, which
customizes results based on what users frequently click on.

Google executives often chide Microsoft that it overengineers software like Office and bombards
people with needless features. But now Google has swapped its clean, simple approach to search in
favor of a feature war with Microsoft.

``Google seems to do things because Bing has done something,'' Mr.~Sullivan said. ``It's a kind of
knee-jerk thing -- we have to do this product now because we don't want people to think we're
weak.''

\section{Technology Races to Meet Tide of Data}

\lettrine{A}{s}\mycalendar{Aug.'10}{02} smartphones and video streaming double the crunch of data
carried over mobile networks each year, big advances in optical transport, data filtering and
networking are keeping the Internet from collapsing under its own weight.

The gains, often unheralded, have allowed mobile network operators to continue selling unlimited
broadband packages to customers who, despite their often voracious data habits, have been spared
from sharing the cost of the strains being put on the system.

Take for example the quick fix made in northern Spain in June 2008, when a 915-kilometer, or
570-mile, stretch of high-speed cable from Madrid to Marseille began to overload, threatening
service to hundreds of thousands of Europeans.

The shutdown was averted in part by a 30-year-old Spanish line technician, Jesús Bergaz, who, with
conventional tools and four optical line cards roughly the size of a pizza box, expanded the
network's capacity tenfold in just 27 hours.

The repair job was the digital equivalent of converting a two-lane country road into a superhighway
in a few hours. And Mr.~Bergaz's remedy is one of several advances taking place every day that are
keeping operators and the Internet ahead of the crush, experts said.

``The doomsayers assume technology doesn't advance, but it does,'' said Jimmy Yu, an optical
transport equipment analyst at Dell'Oro, a research company in Redwood City, California. ``If you
look at history, when something is threatening, technology moves in to compensate.''

The volume of data on the world's mobile networks is doubling each year, according to Cisco Systems,
the U.S.~maker of routers and networking equipment. By 2014, it estimates, the monthly data flow
will increase about sixteenfold, to 3.6 billion gigabytes from 220.1 million.

So far, progress in many aspects of network technology, from wringing more out of long-distance
fiber optic transport to reducing the bottlenecks that can occur at the critical links between base
stations and main networks, along with eliminating some of the digital clutter produced by
cellphones, are staving off a meltdown.

``There might be a lapse between consumer demand and an operator's ability to meet it,'' said Hans
Vestberg, the chief executive of Ericsson, the largest maker of telecom networking equipment, which
is spending \texteuro3 billion, or \$3.9 billion, this year on research and development. ``But we
believe the technology coming onto the market now will go a long way toward addressing this issue.
Technological development will be crucial.''

This year, 22 operators, including TeliaSonera in Scandinavia, Verizon Wireless in the United
States, and NTT DoCoMo in Japan, are expected to start using networks with Long Term Evolution
technology, which can transmit at up to 40 megabits a second, compared with 3 to 6 megabits for
conventional networks.

But amid the economic downturn, operators' spending on L.T.E. networks has been slow, and the
mounting data traffic has required equipment makers to come up with quicker fixes.

In Spain, the technology installed by Mr.~Bergaz, who works for Interoute, a long-distance operator
based in London, was a tiny, complex optical transport system made by Infinera, a company in
Sunnyvale, California, that is a pioneer in optical miniaturization.

Infinera's technology compresses 50 optical components like lasers, modulators and diodes onto a
so-called photonic integrated circuit, or PIC, chip the size of an infant's fingernail. The chips
are the heart of a system that can push 10 times as much data through a fiber optic cable at speeds
of up to 100 gigabits a second.

Mr.~Bergaz inserted four Infinera line cards along the route of an Interoute customer, an operator
the company declined to identify, between Barcelona and Madrid. A second technician in France added
four cards in Marseilles. The installation turned a fiber optic cable the thickness of a human hair
into a high-speed digital fire hose.

Dave Welch, Infinera's chief strategy officer, said new generations of Infinera's technology would
deliver further quantum leaps in transmission speed. In 2012, it plans to start selling circuitry
that would compress more than 200 optical components onto a single chip capable of delivering data
at speeds up to 500 gigabits a second.

``As Internet bandwidth grows, it is essential that technology drive down the cost-per-bit,''
Mr.~Welch said. ``We see the capacity of our chips doubling every three years.''

Typically, however, the skyrocketing levels of Internet data are causing bottlenecks not along
long-haul cables but in dense urban areas, overloading the ubiquitous router switches that sort and
send voice, data and Internet traffic to its destination.

Many older networks were built when voice calls were the only forms of data being sent and are still
based on a technology called time division multiplexing, or T.D.M., which divides the available
bandwidth into 64-kilobit slices, with one slice for each voice call. But in an age of data-heavy
activities like video conferencing, giving that amount of space to a phone call, which typically
uses just 8 kilobits of data, results in a significant portion of unused capacity.

So operators are progressively converting their grids to computer-based Internet Protocol networks,
enabling them to send data in digital packets that can transmit real-time conversations while still
filling the microseconds of ``dead air'' with e-mail messages, video or other data.

A refined example of this approach came from Alcatel-Lucent, the French equipment maker, which in
2009 developed a router the size of two pizza boxes that could slice and dice voice, data and
Internet traffic, sending it on its most efficient path at up to 40 gigabits a second.

Sales of the company's I.P. service routers rose 30 percent in the second quarter of this year.

Many operators with large numbers of smartphone users, like AT\&T Wireless, the exclusive
U.S.~seller of Apple's bandwidth-hungry iPhone, choose Alcatel-Lucent's service routers to improve
the critical link between cell base-stations and its core network.

``If we didn't have this technology, we would be constrained,'' said Michael Howard, a co-founder
and principal analyst at Infonetics Research in Campbell, California. ``There would be no YouTube,
etc. The carriers would all have to limit traffic.''

But the solution is not only a matter of smarter lines and switches.

Huawei, the fast-growing Chinese network equipment maker, has developed software that lets operators
screen out 75 percent of the digital pulses, or ``heartbeats,'' that are continually sent by
smartphones to maintain a contact with a base station. The developers found that even when the
frequency of the pulses was reduced, the phone would remain in meaningful contact with the network,
while allowing operators to carry more calls and data.

\section{In Koreas, Floods Carry Land Mines}

\lettrine{D}{ozens}\mycalendar{Aug.'10}{02} of North Korean land mines loosened by heavy rains have
washed ashore on South Korean riverbanks and beaches near the border, presenting a lethal new threat
to residents already wary of surprises and nefarious motives from the North.

A man in the South Korean border town of Yeoncheon, northeast of Seoul, was killed Saturday when one
of two land mines he had picked up from a stream exploded, the Defense Ministry said. A friend was
seriously injured and hospitalized.

The scare came amid heightened vigilance against North Korea, following the March sinking of a South
Korean warship in border waters that was widely thought to be caused by a North Korean torpedo
attack. On Sunday, South Korea sent a message urging North Korea to prevent its land mines from
washing downstream to the South, the Defense Ministry said in a statement.

The authorities also distributed pamphlets, which carried photos of the North Korean mines, warning
people living near the border not to touch objects that look like the land mines.

In towns and islands downstream from North Korea, officials using megaphones urged villagers and
vacationers to stay off the streams and beaches.

Soldiers with minesweepers were searching river beds where the floods have retreated. Since Friday,
they have found 35 land mines. The mines, built in wooden boxes, were designed to explode when
pressed or opened.

``The mines were apparently swept down from North Korea after torrential rains,'' said an official
from the Office of the Joint Chiefs of Staff who spoke on the condition of anonymity, citing his
office's policy. He said that the safety pins of some recovered mines were not removed, indicating
that they had been in storage when they were swept away.

Heavy rains were reported north of the border in recent weeks. North Korea, where most mountains are
denuded, is vulnerable to landslides and floods. Bodies of North Koreans are occasionally found
downstream in the South after heavy rains.

The two-and-a-half-mile-wide Demilitarized Zone, which divides the two Koreas, and the land
immediately adjacent to it on either side are heavily seeded with land mines to guard against
infiltration by soldiers from the North or South. In South Korean villages near the front line, it
is not unusual to find farmers who have lost arms or legs by stepping on the mines.

But few South Korean civilians have been reported killed by North Korean land mines in the
post-Korean War years.

After the sinking of the ship, South Koreans remain wary of the North, which has a history of
unpredictability. In September last year, it astonished South Koreans by releasing water from one of
its dams without notice, causing a flash flood across the border. Six South Koreans were drowned
downstream.

North Korea recently discharged water from dams north of a river flowing to South Korea, but it
notified the South in advance, giving it time to evacuate vacationers and others downstream.

\section{Slower Pace of Growth in China}

\lettrine{C}{hina}\mycalendar{Aug.'10}{02}'s manufacturing contracted for the first time in 16
months in July, as the government clamped down on property speculation and closed energy-intensive
and polluting factories.

A Purchasing Managers' Index released on Monday by HSBC Holdings and Markit Economics fell to 49.4
from 50.4 in June. Soci\'et\'e G\'en\'erale cautioned last week that seasonal distortions raised the
risk that the July P.M.I. readings could fall below 50.

Meanwhile, a separate manufacturing index released on Sunday by the government statistics bureau and
the China Federation of Logistics and Purchasing slid to 51.2 from 52.1 in June, the lowest level in
17 months.

A deeper Chinese slowdown could weaken a global recovery already constrained by the debt burdens and
unemployment of advanced economies. While growth is cooling, China's full-year expansion may be as
much as 9.5 percent, up from 9.1 percent in 2009, a State Council researcher, Zhang Liqun, said.

The P.M.I. was the lowest since China's manufacturing stopped contracting in March 2009 and was less
than the median forecast of 51.4 in a Bloomberg News survey of 15 economists.

But some analysts said that the report was not all bad news.

The production subindex remained unchanged at 55.3 percent, which suggests that sequential
manufacturing activity growth in July did not change, Goldman Sachs said in a research note. In
addition, other categories like new orders index and new export orders index rose, the Goldman
report said.

At Morgan Stanley, the economist Wang Qing said the slowdown seemed concentrated in heavy industry,
partly reflecting a government campaign to close inefficient businesses to meet energy-saving goals.
This ``does not necessarily reflect weakening in the underlying economic fundamentals,'' he said.

The index released by the logistics federation and the Beijing-based National Bureau of Statistics
covers more than 730 companies in 20 industries, including energy, metallurgy, textiles, automobiles
and electronics.

Government measures this year, including trimming credit growth from last year's record \$1.4
trillion and discouraging multiple-home purchases, have reduced the risk of overheating.

\section{Facebook and Twitter's New Rival}

\lettrine{B}{y}\mycalendar{Aug.'10}{02} now, plenty of traditional media companies have hopped on
the social media bandwagon, pumping out news updates on Facebook and Twitter.

But do those companies have the time and resources to work yet another Web outlet into their daily
routine?

Mark Coatney certainly hopes so. Mr.~Coatney, a 43-year-old journalist, is the latest hire at
Tumblr, a fast-growing blogging service based in New York that says it has 6.6 million users.

Until last month, Mr.~Coatney was a senior editor at Newsweek, where as a side project he headed up
the magazine's social efforts on Twitter and Facebook. Last year he decided to add Tumblr to his
repertoire.

``I saw it as an opportunity to talk to our audience in a new way,'' he said. On Twitter, he said,
``the main feedback comes mostly from retweeting,'' or retransmitting an interesting message. On
Tumblr, ``the tone is a lot more conversational.''

Mr.~Coatney quickly cultivated a following on Tumblr for his thought-provoking, quick-witted posts.
Often they included commentary that was funny and bordering on acerbic -- something he was able to
get away with largely because ``no one at Newsweek really knew what I was doing,'' he said.

The credibility he established among Tumblr users, and the fact that Newsweek was one of the first
big publishers to sign on, cemented Tumblr's decision to hire him, company executives said.

Over the last few months, other media outlets have caught wind of Tumblr, which is free to use. The
newest recruits include The Atlantic, Rolling Stone, BlackBook Media Corporation, National Public
Radio, The Paris Review, The Huffington Post, Life magazine and The New York Times.

But many of those outlets have done little more than set up a placeholder page. In his new job as a
``media evangelist,'' Mr.~Coatney's role, and in some ways his challenge, is to help them figure out
what to do next.

Mr.~Coatney describes Tumblr as ``a space in between Twitter and Facebook.'' The site allows users
to upload images, videos, audio clips and quotes to their pages, in addition to bursts of text.

As on Twitter, users can follow other users, whose posts appear in a chronological stream on a
central home page known as the dashboard. Users can indicate that they like an item by clicking on a
red heart next to it or ``reblogging'' it.

One of the big differences between Tumblr and Twitter is that Tumblr does not display how many
followers a user has, said David Karp, Tumblr's 24-year-old founder and chief executive.

``Who is following you isn't that important,'' he said. ``It's not about getting to the
10,000-follower count. It's less about broadcasting to an audience and more about communicating with
a community.''

Moreover, he said, the site was designed with creative expression in mind.

``People are creating identities and personalities that Facebook and Twitter are not designed to
allow you to do,'' he said.

Since Tumblr is currying favor among a young crowd, it could prove valuable for traditional
companies and media outlets that are trying to build a relationship with that audience. And those
companies are no doubt aiming to win points by being early adopters of a site that is on the rise.

Tumblr is still dwarfed by Facebook and Twitter, which each have hundreds of millions of users and
can be significant sources of traffic for online publishers.

Mr.~Coatney estimated that posting links and notes to the Newsweek Twitter feed and Facebook page
sent roughly 200,000 to 300,000 readers to Newsweek's Web site each day. By comparison, Tumblr sent
closer to 1,000.

But Tumblr is growing quickly. It says it is adding 25,000 new accounts daily, and each month it
serves up 1.5 billion page views.

Items posted on Tumblr can also ripple out to far-flung corners of the Web.

When The New Yorker posted the Escher-inspired oil-spill-themed cover for its July 5 issue on its
Tumblr page, it drew many links from other sites.

Alexa Cassanos, director of public relations for The New Yorker, which began using the service in
late May, said the cover resonated in unlikely places, like the news aggregator Reddit.

Ms.~Cassanos said Tumblr afforded The New Yorker an opportunity to showcase some material that might
otherwise get lost online.

``We can highlight graphic content like photo essays or slide shows to an audience that may not read
the magazine,'' she said. ``You just couldn't do that, visually, on Twitter or Facebook.''

Unlike Twitter, where it is not uncommon for publishers to simply set up accounts that automatically
publish links to their articles and blog posts, Tumblr requires publishers to add more commentary
and interaction if they want to win favor with its community.

Mr.~Coatney acknowledged that this might not be an easy sell, particularly when the payoff was not
immediately obvious.

``It's a huge leap of faith for many of them,'' he said. ``Monetizing that relationship is still a
difficult hurdle because you may not be getting new readers at that particular moment, even if you
are engaging with them.''

For publishers, services like Tumblr reflect a broader shift in their relationship with their
audience, said James E.~Katz, a professor of communications at Rutgers University.

``Going back 20 years, publications like Rolling Stone didn't interact with readers except for
letters to the editor,'' Mr.~Katz said. ``One of the realizations that cultural leaders and
publishers have had is that there is a lot of expertise, wisdom and ideas in their readership.''

The ability to respond online turns readers into co-creators, he said, which can give them a sense
of ownership.

``That is an extremely valuable commodity for publishers these days, even if it does not yet
translate to revenue,'' Mr.~Katz said.

For Tumblr, which is fleshing out its business model and recently raised a \$5 million round of
venture financing from Spark Capital and Union Square Ventures, the interest from media outlets is
something of a feather in its cap.

``There is certainly some validation in it,'' said John Maloney, president of Tumblr. ``They've
decided that this is the next social media platform they want to adopt, and that certainly can
translate into a catalyst for us.''

\section{Plagiarism Lines Blur for Students in Digital Age}

\lettrine{A}{t}\mycalendar{Aug.'10}{02} Rhode Island College, a freshman copied and pasted from a
Web site's frequently asked questions page about homelessness -- and did not think he needed to
credit a source in his assignment because the page did not include author information.

At DePaul University, the tip-off to one student's copying was the purple shade of several
paragraphs he had lifted from the Web; when confronted by a writing tutor his professor had sent him
to, he was not defensive -- he just wanted to know how to change purple text to black.

And at the University of Maryland, a student reprimanded for copying from Wikipedia in a paper on
the Great Depression said he thought its entries -- unsigned and collectively written -- did not
need to be credited since they counted, essentially, as common knowledge.

Professors used to deal with plagiarism by admonishing students to give credit to others and to
follow the style guide for citations, and pretty much left it at that.

But these cases -- typical ones, according to writing tutors and officials responsible for
discipline at the three schools who described the plagiarism -- suggest that many students simply do
not grasp that using words they did not write is a serious misdeed.

It is a disconnect that is growing in the Internet age as concepts of intellectual property,
copyright and originality are under assault in the unbridled exchange of online information, say
educators who study plagiarism.

Digital technology makes copying and pasting easy, of course. But that is the least of it. The
Internet may also be redefining how students -- who came of age with music file-sharing, Wikipedia
and Web-linking -- understand the concept of authorship and the singularity of any text or image.

``Now we have a whole generation of students who've grown up with information that just seems to be
hanging out there in cyberspace and doesn't seem to have an author,'' said Teresa Fishman, director
of the Center for Academic Integrity at Clemson University. ``It's possible to believe this
information is just out there for anyone to take.''

Professors who have studied plagiarism do not try to excuse it -- many are champions of academic
honesty on their campuses -- but rather try to understand why it is so widespread.

In surveys from 2006 to 2010 by Donald L.~McCabe, a co-founder of the Center for Academic Integrity
and a business professor at Rutgers University, about 40 percent of 14,000 undergraduates admitted
to copying a few sentences in written assignments.

Perhaps more significant, the number who believed that copying from the Web constitutes ``serious
cheating'' is declining -- to 29 percent on average in recent surveys from 34 percent earlier in the
decade.

Sarah Brookover, a senior at the Rutgers campus in Camden, N.J., said many of her classmates
blithely cut and paste without attribution.

``This generation has always existed in a world where media and intellectual property don't have the
same gravity,'' said Ms.~Brookover, who at 31 is older than most undergraduates. ``When you're
sitting at your computer, it's the same machine you've downloaded music with, possibly illegally,
the same machine you streamed videos for free that showed on HBO last night.''

Ms.~Brookover, who works at the campus library, has pondered the differences between researching in
the stacks and online. ``Because you're not walking into a library, you're not physically holding
the article, which takes you closer to `this doesn't belong to me,' '' she said. Online,
``everything can belong to you really easily.''

A University of Notre Dame anthropologist, Susan D.~Blum, disturbed by the high rates of reported
plagiarism, set out to understand how students view authorship and the written word, or ``texts'' in
Ms.~Blum's academic language.

She conducted her ethnographic research among 234 Notre Dame undergraduates. ``Today's students
stand at the crossroads of a new way of conceiving texts and the people who create them and who
quote them,'' she wrote last year in the book ``My Word!: Plagiarism and College Culture,''
published by Cornell University Press.

Ms.~Blum argued that student writing exhibits some of the same qualities of pastiche that drive
other creative endeavors today -- TV shows that constantly reference other shows or rap music that
samples from earlier songs.

In an interview, she said the idea of an author whose singular effort creates an original work is
rooted in Enlightenment ideas of the individual. It is buttressed by the Western concept of
intellectual property rights as secured by copyright law. But both traditions are being challenged.

``Our notion of authorship and originality was born, it flourished, and it may be waning,'' Ms.~Blum
said.

She contends that undergraduates are less interested in cultivating a unique and authentic identity
-- as their 1960s counterparts were -- than in trying on many different personas, which the Web
enables with social networking.

``If you are not so worried about presenting yourself as absolutely unique, then it's O.K. if you
say other people's words, it's O.K. if you say things you don't believe, it's O.K. if you write
papers you couldn't care less about because they accomplish the task, which is turning something in
and getting a grade,'' Ms.~Blum said, voicing student attitudes. ``And it's O.K. if you put words
out there without getting any credit.''

The notion that there might be a new model young person, who freely borrows from the vortex of
information to mash up a new creative work, fueled a brief brouhaha earlier this year with Helene
Hegemann, a German teenager whose best-selling novel about Berlin club life turned out to include
passages lifted from others.

Instead of offering an abject apology, Ms.~Hegemann insisted, ``There's no such thing as originality
anyway, just authenticity.'' A few critics rose to her defense, and the book remained a finalist for
a fiction prize (but did not win).

That theory does not wash with Sarah Wilensky, a senior at Indiana University, who said that
relaxing plagiarism standards ``does not foster creativity, it fosters laziness.''

``You're not coming up with new ideas if you're grabbing and mixing and matching,'' said
Ms.~Wilensky, who took aim at Ms.~Hegemann in a column in her student newspaper headlined
``Generation Plagiarism.''

``It may be increasingly accepted, but there are still plenty of creative people -- authors and
artists and scholars -- who are doing original work,'' Ms.~Wilensky said in an interview. ``It's
kind of an insult that that ideal is gone, and now we're left only to make collages of the work of
previous generations.''

In the view of Ms.~Wilensky, whose writing skills earned her the role of informal editor of other
students' papers in her freshman dorm, plagiarism has nothing to do with trendy academic theories.

The main reason it occurs, she said, is because students leave high school unprepared for the
intellectual rigors of college writing.

``If you're taught how to closely read sources and synthesize them into your own original argument
in middle and high school, you're not going to be tempted to plagiarize in college, and you
certainly won't do so unknowingly,'' she said.

At the University of California, Davis, of the 196 plagiarism cases referred to the disciplinary
office last year, a majority did not involve students ignorant of the need to credit the writing of
others.

Many times, said Donald J.~Dudley, who oversees the discipline office on the campus of 32,000, it
was students who intentionally copied -- knowing it was wrong -- who were ``unwilling to engage the
writing process.''

``Writing is difficult, and doing it well takes time and practice,'' he said.

And then there was a case that had nothing to do with a younger generation's evolving view of
authorship. A student accused of plagiarism came to Mr.~Dudley's office with her parents, and the
father admitted that he was the one responsible for the plagiarism. The wife assured Mr.~Dudley that
it would not happen again.

\section{In Restive Chinese Area, Cameras Keep a Watchful Eye}

\lettrine{F}{or}\mycalendar{Aug.'10}{03} a street whose name suggests throwing off shackles, South
Liberation Road doesn't look like a very free place these days.

At the intersection with Shanxi Lane, a busy crossing in this northwest China metropolis, 11
surveillance cameras eye the bustle from a metal boom projecting over one corner. Still more cameras
stare down from the other three corners -- 39 in all, still-photo and high-resolution video.

``The whole city is under surveillance,'' said one nearby shopkeeper who, like many here, refused to
give his name. Asked why, he replied sourly, ``It's not my business.''

But it is no secret. Roughly a year ago, Urumqi's ethnic Han and Uighur populations took part in the
worst ethnic rioting in modern Chinese history, killing at least 197 people. The riots caught the
Communist Party and the local government unaware.

Now at least 47,000 cameras scan Urumqi to ensure there are no more surprises. By year's end, the
state news media says, there will be 60,000.

Video surveillance is hardly uncommon in the West. But nowhere else is it growing as explosively as
in China, where seven million cameras already watch streets, hotel lobbies, businesses and even
mosques and monasteries -- and where experts predict an additional 15 million cameras will sprout by
2014.

Much of the proliferation is driven by the same rationales as in Western nations: police forces
stretched thin, rising crime, mushrooming traffic jams and the bureaucratic overkill that attends
any mention of terrorism.

But China also has another overriding concern -- controlling social order and monitoring dissent.
And some human rights advocates say they fear that the melding of ever improving digital
technologies and the absence of legal restraints on surveillance raise the specter of genuinely
Orwellian control over society.

Video software can already spot a chosen automobile in a stream of traffic by reading license
plates, and cameras have improved so greatly that some can even take clear pictures of people inside
autos. Facial-recognition software is in its infancy, but already, China requires Internet cafe
users to be photographed, so that computers can identify them no matter which cafe they patronize,
and what identification they present.

``This is not a self-contained system of video surveillance, but one part in a much larger
architecture of surveillance that includes Internet monitoring and censorship, telecommunications
and law enforcement databases,'' Nicholas Bequelin, a Hong Kong-based researcher for Human Rights
Watch, wrote in an e-mail exchange. ``Privacy safeguards are simply nonexistent in China, making the
state entirely free to mobilize this architecture of surveillance for political ends.''

It is unclear what share of China's growing camera population is government-controlled. The Ministry
of State Security reported one year ago that police had installed 2.75 million cameras nationwide,
most in urban public spaces, and had asked local police forces to place more in rural areas.

IMS Research, a company based in Britain that tracks China's surveillance industry, estimates that
30 percent of new camera installations have purely governmental uses, from police surveillance to
cameras in libraries or prisons. Cameras on roads and in airports, subways and other modes of
transport are the second most common use.

But that underestimates the extent of state surveillance. The video cameras in China's Internet
cafes are required to be linked to government security offices. Guangdong Province, in southeast
China, last year ordered hotels, guesthouses, hospitals and places of entertainment to install
cameras in all main rooms and reception areas, joining museums and galleries, schools, newspapers
and television stations on a growing list.

In Guangdong, adjoining Hong Kong, security officials are just wrapping up a reported \$1.8 billion
installation of one million video cameras covering major cities like Guangzhou and Shenzhen. Beijing
was expected to have 470,000 cameras by the end of 2009, says the Beijing Security and Protection
Industry Association. Chongqing, a sprawling South China city, will add 200,000 cameras by 2012 to
the 300,000 it now has.

China's string of ``coming-out parties,'' from the 2008 Olympics to this year's Shanghai Expo and
the Guangzhou Asian Games, have all been preceded by security clampdowns that included extensive
installations of surveillance cameras.

Officials say the cameras leverage the latest technology to battle crime and terrorism. Guangdong
provincial officials told Chinese news services last year that their new cameras had deterred more
than 18,000 street crimes even before the one million cameras had been fully deployed. In Kunming,
in south-central China, crime dropped 10 percent after the police installed new cameras, the city's
deputy police chief told a security forum last spring.

That said -- and some Western skeptics dispute claims of the cameras' crime-fighting success --
China's video surveillance clearly has a darker side.

After ethnic rioting in Tibet in 2008 and Urumqi in 2009, security authorities installed live
cameras both inside and on the grounds of monasteries and mosques, and hoteliers were ordered to
place high-quality cameras and scanners in their buildings. Deploying video cameras for 24-hour
monitoring of dissidents and troublemakers, such as citizens seeking to bring grievances to
authorities, is now standard procedure.

Most recently, Mr.~Bequelin said, the Beijing writer Yu Jie and environmental activist Wu Lihong
have come under constant video watch after coming under official scrutiny.

The longer-term concern, he said, is that video surveillance will become a pervasive tool for
controlling not only China's comparative handful of dissidents, but the masses of people who
ordinarily would not run afoul of the state.

In Urumqi, Communist Party and security officials initially agreed to a reporter's request for an
interview about cameras there, then demurred, explaining that cameras were a well-known anticrime
tool and that there was nothing new to say. Still, recent reports in the Chinese news media, which
was given broad access to security officials to report on the surveillance system, hint at the
cameras' potential.

Urumqi's taxi fleet has had live video cameras for two years. Officials said they had since posted
cameras on the city's 3,400 buses and in 200 bus stations, 200 major stores and markets, 270 schools
and along 4,400 roads -- and would continue to mount new cameras until the entire city is blanketed.

In the city's Tianshan district, a Uighur neighborhood racked by riots a year ago, a report on the
Chinese Internet portal NetEase described 20 staff members at the local Public Security Bureau
scanning the monitors. ``One showed the picture inside of a Line 50 bus; the other showed the
picture in front of a major supermarket on Qinnian Road,'' the report said. ``As the monitoring
camera rotated 360 degrees, every corner in front of the supermarket was in clear panoramic view.''

Which was a comforting sight, the report assured its readers. The purpose of the surveillance, it
stated, ``is to ensure the safety of the public places, and to provide good public service for all
people of different ethnicities.''

When asked, Han Chinese in the city generally saw the cameras as a good thing. ``I think the whole
thing was probably triggered by the incident last July,`` said 42-year-old Xie Gang, a wholesaler,
referring to last year's ethnic riots. ``But the significance of the cameras is not to crack down on
rioters, but to prevent crimes. If something happens, the message will get to the authorities right
away.``

Ethnic Uighurs had a markedly different take. ``Oh, the security is very, very good here,`` one man
who refused to give even his first name said, with evident sarcasm, when asked whether the cameras
deterred crime. ``You can see the police patrolling everywhere.``

\section{For E-Data, Tug Grows Over Privacy vs. Security}

\lettrine{T}{he}\mycalendar{Aug.'10}{03} threat by the United Arab Emirates to shut down mobile
services on BlackBerrys like e-mail and text messaging underscores a growing tension between
communications companies and governments over how to balance privacy with national security.

While communications companies want to be able to ensure that their customers' messages are shielded
from prying eyes, governments increasingly insist on gaining access to electronic messages to track
down criminals or uncover terrorist plots.

On Monday, Research In Motion, or R.I.M., the Canadian company that makes the BlackBerry line of
smartphones, sought to reassure its customers that its services remained secure a day after the
U.A.E. said it would ban many BlackBerry services because of national security concerns.

Internet security experts say the demands by the United Arab Emirates for certain access to
communications flowing across the BlackBerry network echo requests of other governments around the
world. Many countries have laws and regulations requiring telecommunications providers to grant
government agencies access to their systems for court-sanctioned intercepts.

The demands also come as other governments, including India, Saudi Arabia, Kuwait and Bahrain, are
reportedly considering new requirements on services like BlackBerry to ensure they can monitor
electronic messages.

``These requirements for access to communications exist on a significant scale worldwide,'' said
Anthony M.~Rutkowski, chief executive of Netmagic Associates, a consulting company specializing in
technical and regulatory issues related to online security.

At the same time, electronic communications providers are increasingly offering security measures
like encryption. For instance, after a cyberattack that originated in China and that targeted
Google's servers and Gmail, the company began encrypting by default all e-mail messages while in
transit.

As a growing volume of communications content is encrypted, governments are demanding other
information like whom customers communicate with and when, said Mr.~Rutkowski. Such information can
be useful to help gather intelligence.

Security experts say that BlackBerry's service, which uses its own network to transmit e-mail and
instant messages, may make access to such data more difficult, especially for countries in which the
company has no servers controlling that network. The experts also say that is why Research In Motion
has had frequent confrontations with governments. Other services, like Skype, have also raised
concerns in some countries.

Research In Motion issued a statement on Monday that did not directly address the company's conflict
with the United Arab Emirates or its relationship with other countries, citing the ``confidential
nature'' of its discussions with certain governments. The company said it balanced competing
demands. ``R.I.M. respects both the regulatory requirements of government and the security and
privacy needs of corporations and consumers,'' the company said in statement. In an open letter to
customers, Research In Motion, which operates in more than 175 countries, also said that its
security system was designed to ensure that no one, not even the company, could gain unauthorized
access to customers' data.

Security experts said that it was not clear what kind of access requirements the United Arab
Emirates had requested from Research In Motion and whether those requirements were more onerous than
those mandated by other governments.

``There is a lot going on that we are not seeing,'' said Bruce Schneier, chief security technology
officer for BT, the giant telecommunications provider based in Britain. ``We don't know what R.I.M.
does for other countries.''

Experts also say that the United Arab Emirates, a major business center in the Middle East, may be
focusing on BlackBerry's service, rather than Gmail or other encrypted services, because it is being
offered by local telecommunications carriers and has grown increasingly popular there.

The BlackBerry service is a frequent target because of ``its convenience, its widespread use and the
fact that it runs on its own network,'' said Marc Rotenberg, executive director of the Electronic
Privacy Information Center, a privacy advocacy group based in Washington. ``The United Arab Emirates
is not in much of a position to tell Google not to encrypt e-mail.''

Many analysts agreed that the Emirati government appeared more interested in getting some
concessions from Research In Motion than in actually shutting off access to BlackBerry data
services. The government said the telephone service would not be affected.

``Saying that the restrictions will not kick in until October is a form of saber-rattling,'' said
Jonathan Zittrain, a professor at the Harvard Law School and co-founder of the Berkman Center for
Internet and Society. ``The government is saying that as a way to get negotiations going with
R.I.M., not to ease the pain of executives who fear they may have their service cut off.''

Still, some businesspeople in Dubai seemed to be digesting the news, and waiting to see whether an
agreement could be worked out between R.I.M. and the Emirati government before the October deadline.
``People are taking a wait-and-see attitude,'' said Blair Look, the managing director of asset
management at al Mal Capital in Dubai.

\section{In Speech on Iraq, Obama Reaffirms Drawdown}

\lettrine{P}{resident}\mycalendar{Aug.'10}{03} Obama on Monday opened a monthlong drive to mark the
end of the combat mission in Iraq and, by extension, to blunt growing public frustration with the
war in Afghanistan by arguing that he can also bring that conflict to a conclusion.

The series of events, starting with a speech here to a veterans' group, puts the president in the
thick of a volatile national security debate at a critical moment for both wars as he draws down
troops from one theater and sends more to the other. While seeking to shore up domestic support, he
is also defining the limits of his ambitions in both Iraq and Afghanistan.

Mr.~Obama vowed to complete his plan to withdraw designated combat forces from Iraq by the end of
August ``as promised and on schedule,'' even though a political impasse has left Baghdad without the
permanent government that his strategy originally envisioned. At the same time, he vowed to destroy
Al Qaeda in Afghanistan while sticking to ``clear and achievable'' goals rather than aspiring to
build a fully functioning democracy.

The president's renewed public focus on the wars comes after many months in which his domestic
agenda was at the center of the national conversation. But the White House calculated that the
drawdown in Iraq and the change in mission there this month provided an opportunity to take credit
for fulfilling one of Mr.~Obama's central campaign promises even as war fatigue takes its toll.

``As a candidate for president, I pledged to bring the war in Iraq to a responsible end,'' Mr.~Obama
told a convention of the Disabled American Veterans here. ``Shortly after taking office, I announced
our new strategy for Iraq and for a transition to full Iraqi responsibility. And I made it clear
that by Aug.~31, 2010, America's combat mission in Iraq would end. And that is exactly what we are
doing, as promised and on schedule.''

The drawdown will bring the American force in Iraq to 50,000 troops by Aug.~31, down from 144,000
when Mr.~Obama took office. The remaining ``advise and assist'' brigades will officially focus on
supporting and training Iraqi security forces, protecting American personnel and facilities, and
mounting counterterrorism operations.

The mission's name will change from Operation Iraqi Freedom to Operation New Dawn, and the 50,000
transitional troops will leave by the end of 2011, according to an agreement negotiated by President
George W.~Bush and reaffirmed by Mr.~Obama. And addressing the concerns of veterans, Mr.~Obama vowed
that ``your country is going to take care of you when you come home.''

While not the end of the conflict -- at least nine people were killed on Monday in attacks around
Iraq -- the transition this month represents a significant milestone after seven years of war that
toppled a brutal dictator, touched off waves of sectarian strife and claimed the lives of more than
4,400 American soldiers and tens of thousands of Iraqis.

In his speech here, Mr.~Obama hailed the improved security in Iraq without mentioning that he had
opposed the 2007 troop buildup ordered by Mr.~Bush, which along with a strategy change, is credited
by many with turning the war around. Mr.~Obama has now assigned the architect of that plan,
Gen.~David H.~Petraeus, to do the same in Afghanistan.

Republicans were happy to remind Mr.~Obama of his opposition to the Iraq buildup, circulating his
quotations from the time. ``It's worth remembering that prior to the full deployment of this force,
some Democrats were already declaring the surge the president is referring to today a complete
failure,'' said Senator Mitch McConnell of Kentucky, the Republican leader.

Nonetheless, Mr.~Obama has adopted Iraq as a relative success story, and aides said he and other
administration officials would hold several events in August to honor returning soldiers and promote
the drawdown. The notion that Iraq would be the political selling point while the ``good war'' in
Afghanistan is now the sour note underscores how much has changed since Mr.~Obama began his
presidential campaign.

Christopher Gelpi, a political science professor at Duke University, said Mr.~Obama's challenge is
convincing Americans that Afghanistan is a worthy cause even if Iraq was not. ``You may argue this
is a good war, but they don't have any information about it,'' he said. ``But they do know about the
Iraq war and they're using that as a lens to interpret Afghanistan. This creates a big problem for
Obama because his core constituents view Iraq negatively.''

The conundrum is that his buildup in Afghanistan is supported more strongly by Republicans than by
his own party. In the House, 102 Democrats voted against a war spending measure last week, 70 more
than a year ago. As a former national security official, who requested anonymity to avoid offending
the White House, put it: ``The people who love him don't support him on Afghanistan, and the people
who support him on Afghanistan hate him.''

Moreover, the skepticism on Afghanistan comes at a time when Mr.~Obama is weakened politically. His
standing in Georgia is low enough, for example, that former Gov.~Roy Barnes, running to reclaim his
old office, chose to skip a Democratic fund-raiser starring the president after his speech to the
veterans.

The White House used the occasion to argue that Mr.~Obama is broadly reducing the American military
presence abroad. A White House fact sheet noted that even with the buildup in Afghanistan, the
drawdown in Iraq means the total number of uniformed Americans in the two countries will drop to
146,000 by September, down from 177,000 when he took office.

The president also argued the importance of succeeding in Afghanistan, reminding Americans that it
was the home of Al Qaeda when it plotted the attacks of Sept.~11, 2001. ``If Afghanistan were to be
engulfed by an even wider insurgency, Al Qaeda and its terrorist affiliates would have even more
space to plan their next attack,'' he said. ``And as president of the United States, I refuse to let
that happen.''

But in making his goal the destruction of Al Qaeda, which American intelligence believes has only
about 100 members in Afghanistan, Mr.~Obama underscored the limits of his commitment.

And he made clear it was not open-ended: ``It's important that the American people know that we are
making progress,'' he said, ``and we are focused on goals that are clear and achievable.''

\section{Chinese Carmaker Geely Completes Acquisition of Volvo From Ford}

\lettrine{T}{he}\mycalendar{Aug.'10}{03} Chinese carmaker Zhejiang Geely Holding Group completed the
acquisition of Volvo from Ford Motor on Monday and named a chief executive to lead the company.

Li Shufu, chairman of Geely Holding, said the completion of the deal -- the first time a Chinese
carmaker had acquired 100 percent of a foreign rival -- was a ``historic day.'' He pledged that
Volvo, based in Sweden, would push to expand market share around the world while keeping to the
characteristics, like a focus on safety, that have defined it.

Geely named Stefan Jacoby, the former head of Volkswagen of America, to be president and chief
executive of Volvo. Mr.~Jacoby will join the board of Volvo in August and conduct a strategic review
to determine the scope and strategy of the company.

Volvo will remain based in Sweden and keep factories in Belgium.

The Chinese carmaker paid \$1.3 billion cash for Volvo on Monday and issued a \$200 million note
payable to Ford to complete the acquisition. Further payments are expected after an audit and final
adjustments in Volvo's value, Ford said in a statement.

In March, when the deal was announced, the price tag was put at \$1.8 billion. The altered terms
reflect adjustments after due diligence in things like retirement funds and working capital, Geely
said.

The authorities in China, one of 12 countries where approval was required, gave their support to the
acquisition last week.

Alan R.~Mulally, chief executive of Ford, said that Volvo had ``returned to profits after a
successful restructuring.''

Ford said it would provide Volvo with parts, as well as engineering and information technology
support, for a transitional period.

The sale ended an era in which Ford, the only American carmaker to avoid bankruptcy last year,
sought to diversify with premium brands.

Ford bought Volvo for \$6 billion in 1999 as part of a decade-long global push in which it also
acquired Aston Martin, Jaguar and Land Rover. Of those international luxury brands, Volvo is the
last to be sold, as Ford now focuses on its so-called core North American and European businesses.
Ford, while emerging relatively healthy from the crisis during which General Motors and Chrysler
filed for bankruptcy protection, is still struggling under about \$27 billion in debt, and can use
the cash from the Volvo sale even if it has been sold at a steep loss.

Geely stock rose 5.9 percent Monday in Hong Kong. Ford shares gained 39 cents, or 3 percent, to
close at \$13.16 in New York.

\section{Some Directors Say 3-D Is One Dimension Too Many}

\lettrine{A}{}\mycalendar{Aug.'10}{03} joke making the rounds online involves a pair of red and
green glasses and some blurry letters that say, ``If you can't make it good, make it 3-D.''

The fans of flat film have a motto. But do they have a movement?

While Hollywood rushes dozens of 3-D movies to the screen -- nearly 60 are planned in the next two
years, including ``Saw VII'' and ``Mars Needs Moms!'' -- a rebellion among some filmmakers and
viewers has been complicating the industry's jump into the third dimension.

It's hard to measure the audience resistance -- online complaints don't mean much when crowds are
paying the premium 3-D prices. But filmmakers are another matter, and their attitudes may tell
whether Hollywood's 3-D leap is about to hit a wall.

Several influential directors took surprisingly public potshots at the 3-D boom during the recent
Comic-Con International pop culture convention in San Diego.

``When you put the glasses on, everything gets dim,'' said J.~J.~Abrams, whose two-dimensional
``Star Trek'' earned \$385 million at the worldwide box office for Paramount Pictures last year.

Joss Whedon, who was onstage with Mr.~Abrams, said that as a viewer, ``I'm totally into it. I love
it.'' But Mr.~Whedon then said he flatly opposed a plan by Metro-Goldwyn-Mayer to convert ``The
Cabin in the Woods,'' a horror film he produced but that has not yet been released, into 3-D. ``What
we're hoping to do,'' Mr.~Whedon said, ``is to be the only horror movie coming out that is not in
3-D.''

A spokesman for MGM declined to discuss ``The Cabin in the Woods.'' But one person who was briefed
on the situation -- and spoke on the condition of anonymity because the studio was in the middle of
a difficult financial restructuring -- said conversion remained an option.

Meanwhile, a spokesman for Marvel Entertainment said that studio had not decided on two or three
dimensions for ``Avengers,'' a superhero film Mr.~Whedon is directing.

With the enormous 3-D success of ``Avatar,'' directed by James Cameron, followed in short order by
``Alice in Wonderland,'' by Tim Burton, film marketing and distribution executives have been
clamoring for more digitally equipped theaters to keep 3-D movies from crowding one another off the
screen.

By year's end, there will be more than 5,000 digital screens in the United States, or 12.5 percent
of the roughly 40,000 total, easing a traffic jam that has caused 3-D hits like ``Clash of the
Titans,'' from Warner Brothers, to bump into ``How to Train Your Dragon,'' from DreamWorks
Animation, to the disadvantage of both.

Tickets for 3-D films carry a \$3 to \$5 premium, and industry executives roughly estimate that 3-D
pictures average an extra 20 percent at the box office. Home sales for 3-D hits like ``Avatar'' and
``Monsters vs. Aliens'' have been strong, showing they can more than hold their own when not in 3-D.

A 3-D movie can be somewhat more costly than a 2-D equivalent because it may require more elaborate
cameras and shooting techniques or an additional process in the already lengthy postproduction
period for effects-heavy films. But the added costs are a blip when weighed against higher ticket
sales.

Behind the scenes, however, filmmakers have begun to resist production executives eager for 3-D
sales. For reasons both aesthetic and practical, some directors often do not want to convert a film
to 3-D or go to the trouble and expense of shooting with 3-D cameras, which are still relatively
untested on big movies with complex stunts and locations.

Filmmakers like Mr.~Whedon and Mr.~Abrams argue that 3-D technology does little to enhance a
cinematic story, while adding a lot of bother. ``It hasn't changed anything, except it's going to
make it harder to shoot,'' Mr.~Whedon said at Comic-Con.

In much the same spirit, Christopher Nolan recently warded off suggestions that his film
``Inception,'' from Warner -- still No.~1 at the box office -- might be converted to 3-D.

On the other hand, Michael Bay, who is shooting ``Transformers 3,'' appears to have agreed that his
film will be at least partly in 3-D after insisting for months that the technology was not quite
ready for his brand of action.

``We've always said it's all about balance,'' said Greg Foster, the president and chairman of Imax
Filmed Entertainment, which has long counseled that some films are better in 2-D, even on giant Imax
screens. ``The world is catching up to that approach.''

A willingness to shoot in 3-D could persuade studio committees to approve an expensive film. But the
disdain of some filmmakers for 3-D -- at least in connection with their current projects -- was on
full display in San Diego.

Jon Favreau, speaking at Comic-Con about his coming ``Cowboys \& Aliens'' for DreamWorks and
Universal, said the idea of doing the movie in 3-D had come up, but he was not interested.
Contemporary 3-D requires a digital camera, and ``Westerns should only be shot on film,''
Mr.~Favreau said. He added: ``Use the money you save to see it twice.''

Stacey Snider, the DreamWorks chief executive, said Mr.~Favreau and the studios involved had
mutually agreed that 3-D was not right for the film. But, she added, a discussion about 3-D was
inevitable.

``It's na\"ive to think we wouldn't be having it on any movie that has effects, action or scale,''
Ms.~Snider said.

Earlier at Comic-Con, Edgar Wright, the director of ``Scott Pilgrim vs. the World,'' an
action-filled comic-book extravaganza from Universal, similarly said that his film would arrive in
two dimensions, at regular prices.

(People briefed on Universal's approach to the film said 3-D had been considered very briefly. It
was rejected, however, partly to avoid straining what promises to be a young audience with high
ticket prices, partly because the already busy look of the movie might have become overwhelming in
3-D.)

The crowds cheered, as they had in an earlier Comic-Con briefing by Chris Pirrotta and other staff
members of the fan site TheOneRing.net, who assured 300 listeners that a pair of planned ``Hobbit''
films will not be in 3-D, based on the site's extensive reporting.

``Out of 450 people surveyed, 450 don't want 3D for 'The Hobbit,' '' a later post on the Web site
said.

But in Hollywood, an executive briefed on the matter -- who spoke on the condition of anonymity
because of the delicate negotiations surrounding a plan to have Peter Jackson direct the ``Hobbit''
films -- said the dimensional status of the movie remained unresolved.

Asked by phone recently whether die-hard fans would tolerate a 3-D Middle Earth, Mr.~Pirrotta said,
``I do believe so, as long as there was the standard version as well.''

In his own family, he said, the funny glasses can be a deal-breaker.

``My wife can't stand 3-D.''

\section{Arts Playground Sprouts in China}

\lettrine{H}{ong}\mycalendar{Aug.'10}{04} Kong has always looked down on Guangzhou as its poor
mainland cousin. But while the affluent former British colony has stalled for years over plans for a
massive cultural district, Guangzhou has gone ahead and built one.

This southern Chinese city surrounded by factory towns opened its new Guangdong Museum and Guangzhou
Opera this spring. On tap are a public library and a children's art center.

The government has not put a price tag on the entire project, though media reports have estimated
that the four venues will cost 3.4 billion renminbi, about \$500 million. Guangzhou hopes to unveil
the complex by November, when it plays host to the Asian Games.

That is the plan. As is usually the case in China, the hardware was built first and the software is
still on its way.

Months after the museum's opening in May, workers are drilling and hammering amid piles of dirt and
rubble to prepare the rest of the complex. The opera house and the museum are open for business --
two beautiful architectural models rising from a junkyard. But the transport hub, taxi stands and
pedestrian walkways have not been completed, causing crowd and traffic problems, particularly when
the opera lets out in bad weather.

Rocco Yim, the Hong Kong architect who designed the museum, reported to cost 900 million renminbi,
stood at its entrance and pointed past the construction site to the spaceship-like opera house
designed by the London-based architect Zaha Hadid for an estimated 1.4 billion renminbi. ``The two
will be connected by a wide pedestrian avenue,'' Mr.~Yim said, ``so people can walk right from the
opera to the museum through open green space. Here will be a large slope where people can lie down
in the grass. Roadside pollution will be cut down by diverting vehicular traffic underground.''

The museum is an enormous cube made of gray and red puzzle pieces that light up with a scarlet glow
at night. ``I wanted to create the feeling of a lacquered Chinese jewelry box,'' Mr.~Yim said, ``an
exquisite container holding valuables inside.''

Natural light floods the museum through its jigsaw-shaped holes and skylights. A walkway and a
cube-shaped gallery float above the lobby. Spaces are divided not by walls but by translucent
screens, adding to the airiness.

There is no stand-out, priceless treasure in the Guangdong Museum's collection -- certainly nothing
comparable with the Palace Museum in Taipei, say. But there is much southern Chinese folk art, like
Chiuchow wood carvings, calligraphy and ink paintings, and the natural history section is definitely
child-friendly. Mr.~Yim said his favorite room is the vast atrium where life-sized models of whales
and dolphins are suspended from the ceiling, flooded in blue light. From there you can look straight
down to the dinosaur fossils displayed on the floor below.

The opera house -- all silvery twists and curves -- is the aesthetic opposite of the squarish
museum. Its latticework skin covers two structures: a large hall for operas and a concert hall for
recitals.

Liu Xiaolu, a Guangzhou Opera spokesman, said: ``In a short period of time it has changed the
cultural scene here, which was relatively limited until recently. Before it was just Beijing and
Shanghai. Major international productions -- whether it was opera or pop music -- would pass right
over us and go straight to Hong Kong. We just didn't have the venues. We didn't even have a stage
large enough to fit all the swans in Swan Lake. Now it's Guangzhou's turn.''

In its first two months, the house put on three fully staged operas, all of which were well
attended. Mr.~Liu noted that they had a good number of visitors from Hong Kong for the opening show,
Puccini's ``Turandot.''

Whenever an expensive project is built with state money, questions are raised about its relevance.
Lianhe Zaobao, a Chinese-language newspaper in Singapore, asked in an editorial whether top ticket
prices for ``Turandot,'' at 2,880 renminbi, were appropriate in a city where the average monthly
salary is 3,942 renminbi.

Arguably, ``Turandot'' was an exception, as it was the venue's opening gala and was conducted by
Lorin Maazel. Plus, many of the tickets went to officials, organizers and other V.I.P.s.

But even for the ``Mulan'' opera -- a domestic production that has been on tour for several years --
the best seats cost 1,200 renminbi.

The Guangzhou Opera countered that it has offered a range of discounted tickets for students and the
disadvantaged. In an upcoming Canadian production of ``Alice in Wonderland,'' for instance, a
donation from a corporate sponsor allowed seats for two of the four shows to be set aside for
disadvantaged residents. ``This is definitely a public facility,'' Mr.~Liu said.

In Chinese, the Guangzhou Opera's name actually says nothing about opera -- it is probably better
translated as the Guangzhou Center for Performing Arts. Its roster of future events includes modern
dance, multimedia shows, pop acts and children's programming like ``Sesame Street Live.'' The
spokesman said the house also is hoping to stage Yue Opera, or Cantonese Opera, with troupes from
Hong Kong or Macao.

In terms of balancing artistic ambition with public sentiment, the opera house got it right with
``Mulan,'' which was about 80 percent full. It was the operatic version of the Chinese costume
melodramas so loved by television audiences. It pulled at every populist heartstring, from the
plucky woman warrior in a bright silk robe to the backdrops of peony branches and a red sunset over
the Great Wall.

The composition for chorus and full orchestra -- complete with a conductor in tails highlighted by a
spotlight on stage -- is Western. But there was a definite Chinese influence to the singing style
and the volume of the percussion.

Or maybe the drums were there to drown out the crowd's babbling, of telephones ringing, of children
playing in the aisles and of people trying to sneak into better seats. A review of ``Turandot'' in
the Financial Times in May made note of the myriad distractions, like flash photography and the
static of the security guards' walkie-talkies.

At the Guangdong Museum, meanwhile, Wang Xiaoying, the director of education and promotion,
estimated that the venue was getting 7,000 to 8,000 visitors a day.

When construction is finished, people will be able to enter from the ground-floor entrance that is
linked to the grassy area and the walkway to the opera. For now, they are herded into a waiting area
ringed with metal barriers.

Still, on a sweltering Sunday afternoon, the line stretched down the street. Liu Jin, a Guangzhou
resident, said he had been waiting 20 minutes to get in. ``Of course it's worth it to see,'' he
said. ``It's free to the public. Plus, every big city has a big museum and now we do, too.''

\section{Iraqis Are Less Certain Than Obama About End to the War}

\lettrine{T}{he}\mycalendar{Aug.'10}{04} morning after President Obama spoke of bringing the war in
Iraq to ``a responsible end,'' insurgents planted their black flag on Tuesday at a checkpoint they
overran by killing the five policemen who staffed it. It was the second time in a week.

The rest of the day, the police blotter looked like this: Three mortars crashed in Baghdad
neighborhoods, where five roadside bombs were detonated and two cars were booby-trapped. Two other
mortars fell in the Green Zone, still the citadel of power in a barricaded capital and still a
target of insurgents who seem bent on proving they were never defeated.

By dusk, a car bomb tore through Kut, an eastern town long spared strife.

``Nothing unusual,'' said Murtadha Mohammed, a 20-year-old baker, as he shoveled rolls into bags a
short walk from one of the bombs. ``We've been raised on this.''

The word ``disconnect'' never quite captures the gulf in perceptions between two countries whose
fate remains reluctantly intertwined, however exhausted each seems of the other. Moments have come
and gone: transitional governments, declarations of sovereignty, the signing of agreements.
Mr.~Obama's announcement Monday was another.

On Tuesday, Qahtan Sweid greeted it with the cynicism that colors virtually any pronouncement the
United States makes here, itself a somewhat intangible but pervasive legacy of seven years of
invasion, occupation, war and, now, something harder to define.

``The Americans aren't leaving,'' Mr.~Sweid insisted, whatever Mr.~Obama had promised. ``For one
million years, they won't leave. Even if the world was turned upside down, they still wouldn't
withdraw.''

From the first days after the fall of Baghdad on April 9, 2003, America and Iraq seemed divided by
more than language; they never shared the same vocabulary. Perhaps they never could, defined as
occupier and occupied, where promises of aid and assistance often had the inflection of
condescension. These days, though, they do not even seem to try to listen to each other -- too tired
to hear the other, too chastised by experience to offer the benefit of doubt.

In a speech that was admittedly modest, Mr.~Obama declared Monday that violence continued to be at
the lowest it had been in years. Iraq is indeed a safer country than it was 2006 and 2007, when
carnage threatened to shred the very fabric of its traumatized society. But security, still elusive
here, is an absolute; you either feel safe or you do not.

The toll Tuesday -- 26 dead in 8 attacks -- was not spectacular for Iraq, where hundreds of people
still die each month. But it came amid growing fears that insurgents are regrouping in Baghdad,
Diyala, Falluja and elsewhere, eager to capitalize on the prospect of American troops leaving and
the dysfunction of a political class that has yet to agree on an Iraqi government, nearly five
months after the election.

In an attack in the Mansour neighborhood of Baghdad, insurgents in at least two cars assaulted a
checkpoint at dawn with pistols fitted with silencers, killing five policemen, then planted their
flag before fleeing. In the car bombing in Kut, the death toll rose to 20 by nightfall.

``Wherever the Americans go, the situation is going to stay the same as it was,'' said Abdel-Karim
Abdel-Jabbar, a 51-year-old resident of the Sunni neighborhood of Adhamiya, where insurgents overran
another checkpoint last week, burning the bodies of their victims and planting the same black
banner. ``If anything, it's going to deteriorate.

``The peace Obama's talking about is the peace of the Green Zone,'' he added.

Across town, in Sadr City, a sprawling district once a battlefield between American troops and
followers of Moktada al-Sadr, a populist Shiite cleric, puddles of sewage gathered near a bomb's
debris.

Mr.~Sweid nodded.

``If it's in your hands, then you can go ahead and be scared,'' he said over the drone of a
generator. ``If it's in God's hands, then you have no right to fear.''

In his speech on Monday, Mr.~Obama called the Aug.~31 deadline for the military to bring the number
of troops down to 50,000 the closing of a chapter.

To an American audience, it might resonate that way. Less so to Iraqis. Unlike last year, Iraqi
officials, mired in disputes often more personal than political, are not trumpeting the withdrawal
as an assertion of an Iraqi authority. Neither Mr.~Sweid nor Mr.~Abdel-Jabbar knew about the August
deadline. The same went for several others interviewed Tuesday.

``I don't know exactly when the withdrawal is supposed to happen,'' said Abdel-Hamid Majid, a
52-year-old engineer. ``All I know is it's not far away.''

Saud al-Saadi, an eloquent and informed teacher in Sadr City, was aware. But, he said, he had heard
such pronouncements before, declarations of turning points in America's experience here that seemed
to hew to the logic of American politics. The American occupation was declared over before the 2004
presidential election. The two countries signed strategic agreements weeks before the Bush
administration ended.

``But until now, to tell you the truth, we haven't grasped our sovereignty,'' Mr.~Saadi said.
``There are still American troops here, they still raid houses, we don't have a government that
makes its own decisions and the American ambassador still interferes.''

Mr.~Saadi was neither angry nor disillusioned. And in his matter-of-fact appraisal, there was a hint
of common ground between a teacher and a president. Mr.~Obama did not trumpet democracy or victory.
There was no reference to a mission accomplished. In a sober appraisal, he acknowledged that there
would be more American sacrifice here.

Mr.~Saadi was no less modest.

Interests, he called it. And the United States, he said, would try to secure its own.

``America is not a charity organization,'' he said. ``It's not a humanitarian group. There are words
and there is reality, and actions don't always match those words.''

\section{BlackBerry Maker Resists Governments' Pressure}

\lettrine{A}{}\mycalendar{Aug.'10}{04} top executive of Research In Motion, the Canadian company
that makes BlackBerry smartphones, said on Tuesday that his company would not give in to pressure
from foreign governments to provide access to its customers' messages.

That pressure increased on Tuesday as Saudi Arabia ordered local cellphone providers to halt
BlackBerry service because it did not meet the country's regulatory requirements.

Mike Lazaridis, founder and co-chief executive of R.I.M, said in an interview that allowing
governments to monitor messages shuttling across the BlackBerry network could endanger the company's
relationships with its customers, which include major companies and law enforcement agencies.

``We're not going to compromise that,'' Mr.~Lazaridis said. ``That's what's made BlackBerry the
No.~1 solution worldwide.''

Several governments have cited national security concerns in demanding that R.I.M. open up its
system, which protects customer messages with strong encryption.

The Saudi decision follows a similar move by the United Arab Emirates, which announced on Sunday
that it would block BlackBerry e-mail and text-messaging services beginning in October.

The suspension in Saudi Arabia is to take effect this month, according to the state-owned Saudi
Press Agency.

Mr.~Lazaridis denied reports that the company had already granted special concessions to the
governments of countries like India and China, which have large numbers of BlackBerry owners.

``That's absolutely ridiculous and patently false,'' he said.

Mr.~Lazaridis said the encryption that was causing alarm among foreign governments was used for many
other purposes, including e-commerce transactions, teleconferencing and electronic money transfers.

``If you were to ban strong encryption, you would shut down corporations, business, commerce,
banking and the Internet,'' he said. ``Effectively, you'd shut it all down. That's not likely going
to happen.''

Mr.~Lazaridis expressed sympathy for the concerns of the Persian Gulf nations. ``I am very
empathetic to their concerns and what they go through,'' Mr.~Lazaridis said. ``But every country
goes through these things. We have to be prepared for the ramifications of the decisions we make.''

R.I.M. issued a statement Tuesday that was intended to reassure customers, saying that ``customers
of the BlackBerry enterprise solution can maintain confidence in the integrity of the security
architecture without fear of compromise.''

Jonathan Zittrain, a professor of law and computer science at Harvard and co-founder of the Berkman
Center for Internet and Society, said the statement appeared to address only the products that the
company sold to corporate customers, not those it sells directly to consumers.

Corporate customers tend to be of less concern to governments, he said, because criminals or
terrorists are less likely to engage in illegal activities from corporate e-mail systems, and
because governments can go directly to those corporations to obtain information about their
employees.

``This doesn't put the main question to rest,'' Professor Zittrain said. ``It doesn't explain under
what circumstances would the average BlackBerry user have his communications exposed.''

A spokeswoman for R.I.M. said the company would not elaborate on its statement.

Mr.~Lazaridis spoke after a press conference in Manhattan at which executives from AT\&T and R.I.M.
introduced the BlackBerry Torch 9800, the company's first phone with both a touch screen and a
slide-out keyboard.

The Torch, which costs \$199 with a two-year data plan, will be sold exclusively for AT\&T's network
beginning Aug.~12. It has a 5-megapixel camera with a flash and runs a new version of R.I.M.'s
mobile operating system called BlackBerry 6.

Don Lindsay, vice president for user experience at R.I.M., pointed out the phone's new software
features, which include a redesigned home screen, improved support for multimedia and applications
and a better Web browser.

``It's not about bringing something new to BlackBerry but improving what we do best,'' he said.

Research In Motion has a lot riding on the release of the Torch. The company has been losing market
share and mindshare to Apple and Google as more users clamor for the iPhone and smartphones powered
by Android, Google's mobile operating system. For R.I.M., this competition has increased the
importance of markets in Europe, Asia and the Middle East.

A report released on Monday by Nielsen said sales of R.I.M. devices to new subscribers in the United
States were slowing, and that 29 percent of BlackBerry users had considered switching to the iPhone.

Another report from the research firm Canalys said that in the second quarter, Android sales were up
nearly 900 percent from a year ago, claiming 34 percent of the market in the United States. By
comparison, Research In Motion had 32 percent, and Apple staked out 21.7 percent of the market. A
year ago, R.I.M.'s share was 45 percent.

\section{For Congress, a New Vigilance in Policing Ethics}

\lettrine{I}{n}\mycalendar{Aug.'10}{04} the bazaar that is Capitol Hill, there is nothing surprising
about lawmakers' doing favors for campaign donors or intervening with federal agencies on behalf of
constituents or friends.

So why are Representatives Charles B.~Rangel, a New York Democrat, and Maxine Waters, a California
Democrat, facing the rare spectacle of public ethics trials for actions their defenders say are just
business as usual in Congress?

The charges reflect, in part, a heightened sensitivity in Washington to indiscretions by members of
Congress. The House ethics committee, which has brought the charges, has come under fire for failing
to hold lawmakers accountable in previous investigations.

Both cases also involve personal causes -- for Ms.~Waters, the financial investments of her husband,
and for Mr.~Rangel, an education center set up in his name in New York. With their integrity under
attack after widespread news reports, Mr.~Rangel and Ms.~Waters are fighting the charges instead of
simply accepting a modest punishment.

As a result, Washington has suddenly become fixated on ethics issues, including the continuing
investigation of Senator John Ensign, a Republican from Nevada, who has been accused of improperly
intervening with federal regulators at the request of a former aide, whose wife had an affair with
the senator.

``This wave of activity will remind members and staff that this is an era of more vigilance and
scrutiny and they need to be much more careful about what they do,'' said Abbe D.~Lowell, a
Washington defense lawyer who has handled a number of ethics inquiries. ``The public's low esteem
for Congress and the appearance of inappropriate conduct in general have to be confronted and dealt
with.''

Ms.~Waters on Monday became the second lawmaker in two weeks to face formal ethics charges, as the
House Committee on Standards of Official Conduct announced it was creating a new panel to hold a
trial to determine whether she took improper steps on behalf of a bank in which her husband owns
stock.

Her case illustrates the conflict between lawmakers and the ethics committee over whether her
actions represent a widely accepted norm or an egregious violation of the ethics rules.

She is accused of improperly calling Henry M.~Paulson Jr., then the Treasury secretary, to set up a
meeting in late 2008 on behalf of minority bankers pushing for a federal bailout. That group
included executives from the bank her husband invested in.

``I simply will not be forced to admit to something I did not do,'' Ms.~Waters said in a statement
on Monday. ``The case against me has no merit,'' she added.

The accusations against all three lawmakers hang in large part on the question of whether the
actions they took came in the course of their normal professional duties or constituted personal
favors that may have been influenced, or at least appeared to have been influenced, by money or
other factors.

In their defense, Mr.~Rangel, Ms.~Waters and Mr.~Ensign rely on similar arguments: The people and
industries they helped were constituents they had always helped in the past, regardless of any other
political, personal or financial ties.

Mr.~Ensign contended, for instance, that he had always been a supporter of a Nevada airline,
Allegiant Air, and an electric utility, NV Energy. Investigators are examining whether he may have
tried to hush up the affair by inappropriately helping his mistress's husband lobby federal agencies
on behalf of those companies.

Mr.~Rangel's lawyers said his charitable work raising money with City College of New York to start
an educational center named after him there reflected his long commitment to the college, four
blocks from where he grew up.

Mr.~Rangel's lawyers, in a 32-page rebuttal to the charges, said a number of other prominent
lawmakers including Senator Mitch McConnell, and former Senators Trent Lott, Jesse Helms and Robert
C.~Byrd raised money for their own favorite university causes while they were in office.

Mr.~McConnell of Kentucky, the Senate minority leader, has helped raise money from corporate donors,
including RJR Nabisco, Toyota and military defense contractors, for a center named after him at the
University of Louisville, Mr.~Rangel's lawyers point out.

``We provide these examples not as part of an ``everyone does it'' defense, but rather to
demonstrate that these activities have never been regarded as creating an improper benefit to a
member,'' Mr.~Rangel's lawyers wrote.

The House ethics panel that investigated his case disagreed.

The committee said that not only did he appeal for contributions from companies like Verizon, New
York Life Insurance and American International Group, which all had major legislative matters before
his committee, but he also made those appeals on official House stationery, with the help of his
House aides.

And there was an unusually close overlap, the committee contended, between appeals for donations and
his intervention on legislative matters, citing in particular a meeting Mr.~Rangel held in 2007 at a
New York hotel with an executive from an oil drilling company at which he made a bid for a donation
and also discussed a tax break the company was seeking.

The executive, Eugene Isenberg, and his company ended up making a \$1 million contribution to the
educational center, and Mr.~Rangel helped the company secure a tax break worth an estimated \$500
million.

``Reasonable persons could construe contributions to the Rangel Center by persons with interest
before the Ways and Means Committee as influencing the performance of respondent's government
duties,'' said the complaint against Mr.~Rangel, who was the committee's chairman.

Mr.~Rangel's lawyers acknowledge that he probably should have checked in advance with the House
ethics staff about the City College arrangement, but contended that ``the congressman did not abuse
his official position or enrich himself financially.''

Robert K.~Kelner, an ethics lawyer in Washington who is involved in Senator Ensign's case, called
the charges against Mr.~Rangel and Ms.~Waters perhaps overblown.

``This is the committee's way of showing that it is alive and well, notwithstanding the criticism
directed at it,'' Mr.~Kelner said. ``Here's an opportunity where we can bare our teeth.''

Whatever the explanation, lawmakers in Washington are noticing the crackdown, their lawyers said.

``There is definitely a heightened concern and sensitivity about political and charitable
contributions and the timing in relation to official actions,'' said Kenneth A.~Gross, an ethics
defense lawyer. ``It's a big issue out there right now, and these are very rough waters to
navigate.''

\section{4 Believed Dead in China School Attack}

\lettrine{T}{hree}\mycalendar{Aug.'10}{06} children and at least one teacher were reported on
Wednesday to have died in an afternoon knife attack a day earlier at a kindergarten in eastern
China, the sixth in a string of school assaults this year that has stunned the nation and sent
government officials scrambling to suppress public outrage.

The latest attack, in the city of Zibo in Shandong Province, also was reported to have injured 11
other people, including two seriously wounded children, according to accounts of the attack on
Chinese Web sites. Police officers were said to be uncertain of the number of attackers, but one,
described as a 27-year-old man, was reported to have turned himself in.

Details of the incident were sparse and sometimes conflicting. But postings on one blog stated that
the kindergarten, said to be in one of Zibo's most affluent communities, was limited to the children
of local government officials. That was later confirmed by Li Heping, a noted lawyer and defender of
human-rights activists in Beijing, who said in a telephone interview that he had been in touch with
a friend in Zibo.

As occurred after other recent knifings, the government swiftly slapped a news blackout on the case,
blocking all accounts of the attacks on the Internet and banning all photographs of the bloodshed.

Since March 23, when a man fatally stabbed eight children outside a Fujian Province elementary
school, at least 18 children -- all of kindergarten or primary-school age -- and 5 adults have died
in the bizarre series of attacks. Including preliminary reports from Tuesday's assault, at least 66
other children have been wounded, including 5 who were clubbed with a hammer by a man on April 30.

Most of the assaults have occurred along China's urbanized east coast, where wealth disparities are
most visible and social pressures presumably at their peak. The government convened a panel of 22
experts in April to investigate the attacks, and schools across China have installed surveillance
cameras and stepped up security.

A subtext to some of the attacks appears to have been resentment of the rich or powerful. The first
attacker confessed that he acted in fury after being rejected by his girlfriend's wealthy parents,
and targeted a primary school that was attended mostly by children of the well-off.

But analysts have yet to find a coherent theme to the assaults, aside from speculation by
criminologists and sociologists that some are so-called copycat attacks and that they highlight the
lack of adequate mental-health care in a nation where psychiatrists are rare and mental hospitals
are often warehouses for the sick.

State-run media reacted to the first attack last March with anger, grief and a series of articles
that sought to explain what could motivate such a horrific crime. But as the attacks continued,
censors clamped down on news coverage and played up state efforts to combat violence and social
upheaval.

The assault on Tuesday was not reported until Wednesday morning Beijing time, and most traces of
news reports had been expunged from the Internet by mid-afternoon. Various officials in Zibo, a city
of about 670,000 some 230 miles south of Beijing, declined to comment or said they did not know
about the attack when reached by telephone.

Reports of the Zibo attacks offered varying tolls of the dead, the injured and the number of
assailants. The attack apparently took place about 4 p.m. Tuesday when, as security guards at were
on a break, up to three men entered the suburban Boshan district experimental kindergarten and
stabbed two teachers, then turned on the children.

Internet accounts said the school's deputy director died, and some reports said that as many as 20
people had been slashed before the attacker or attackers fled. Two teachers who sought to shield the
children from the assault were reported to be among those who were seriously wounded.

\section{Portrait of Pain Ignites Debate Over Afghan War}

\lettrine{S}{he}\mycalendar{Aug.'10}{06} cannot read or write and had never heard of Time magazine
until a visitor brought her a copy of this week's issue, the one with the cover picture of her face,
the face with no nose.

On Wednesday, the young woman, Bibi Aisha, left Kabul for a long-planned trip to the United States
for reconstructive surgery. Earlier in the day, as she prepared to leave the women's shelter at a
secret location here that has been her refuge for the past 10 months, the 18-year-old was unaware of
the controversy surrounding the publication of that image.

``I don't know if it will help other women or not,'' she said, her hand going instinctively to cover
the hole in the middle of her face, as it does whenever strangers look directly at her. ``I just
want to get my nose back.''

Reaction to the Time cover has become something of an Internet litmus test about attitudes toward
the war, and what America's responsibility is in Afghanistan. Critics of the American presence in
Afghanistan call it ``emotional blackmail'' and even ``war porn,'' while those who fear the
consequences of abandoning Afghanistan see it as a powerful appeal to conscience.

The debate was fueled in part by the language that Time chose to accompany the photograph: ``What
Happens if We Leave Afghanistan,'' pointedly without a question mark.

``That is exactly what will happen,'' said Manizha Naderi, referring to Aisha and cases like hers.
An Afghan-American whose group, Women for Afghan Women, runs the shelter where Aisha stayed,
Ms.~Naderi said, ``People need to see this and know what the cost will be to abandon this country.''

As Ms.~Naderi would be the first to concede, however, things are already bad enough for women in
Afghanistan without a return to a government run by the Taliban. Noorin TV in Kabul has been running
what it has called an investigative series suggesting that the shelters, all operated by independent
charities, are just fronts for prostitution. The series has offered no evidence, and the station
never sent anyone to visit the principal shelters.

President Hamid Karzai, once seen as a champion of women's causes until he failed to deliver on
promises to appoint many women to cabinet posts, convened a commission to investigate complaints
against women's shelters. A report is expected soon. The panel's chairman is a conservative mullah,
Nematullah Shahrani, who has publicly bandied about the prostitution claim.

Even in the absence of a government run by the Taliban, Afghan women suffer from religious
extremism, although they have enjoyed a great deal of progress. Thousands of girls' schools have
opened since the fall of the Taliban, and women are active in the Parliament and the aid community,
where an estimated half a billion dollars in international assistance is now destined for
gender-equality programs.

``Feminists have long argued that invoking the condition of women to justify occupation is a cynical
ploy,'' wrote Priyamvada Gopal in The Guardian, a liberal British newspaper, on Wednesday, ``and the
Time cover already stands accused of it.''

BagNews, a left-leaning Web site about the politics of imagery in the media, saw the matter in
conspiratorial terms. ``Isn't this title applying emotional blackmail and exploiting gender politics
to pitch for the status quo -- a continued U.S.~military involvement?'' wrote Michael Shaw.

Richard Stengel, Time's managing editor, said he thought not. ``The image is a window into the
reality of what is happening -- and what can happen -- in a war that affects and involves all of
us,'' he wrote in a statement on Time's Web site.

Bibi Aisha (bibi is an honorific; Aisha asked that her family name be withheld) makes an apt symbol
of the excesses of the Taliban, and of Pashtun tribal society in remote parts of Afghanistan more
generally. Her face, aside from the disfigurement, is as beautiful as that of the Afghan refugee
girl whose cover photograph in National Geographic in 1985 became an iconic image of the country's
plight.

At age 12, Aisha and her younger sister were given to the family of a Taliban fighter in Oruzgan
Province under a tribal custom for settling disputes, known as ``baad.'' Aisha's uncle had killed a
relative of the groom to be, and according to the custom, to settle the blood debt her father gave
the two girls to the victim's family.

Once Aisha reached puberty, she was married to the Taliban fighter, but since he was in hiding most
of the time, she and her sister were housed with the in-laws' livestock and used as slaves,
frequently beaten as punishment for their uncle's crime.

Aisha fled the abuse, but her husband tracked her down in Kandahar a year ago, took her back to
Oruzgan, and on a lonely mountainside cut off her nose and both ears and left her bleeding. She said
she still did not remember how she managed to walk away to find help.

In Pashtun culture, a husband who has been shamed by his wife is said to have lost his nose,
Ms.~Naderi explained; from the husband's point of view, he would have been punishing Aisha in kind.

American aid workers in Oruzgan Province took Aisha to the Women for Afghan Women shelter in Kabul,
where at first she was too traumatized even to speak. They introduced her to a psychologist and
gradually she recovered, learning to do handicrafts, but showing little interest in school work.
Even before the Time cover appeared, the organization had found a benefactor in the Grossman Burn
Foundation in Calabasas, Calif., which agreed to underwrite eight months of reconstructive surgery.

Once during Aisha's stay at the shelter, her father visited to try to persuade her to return home to
her family, but she refused to do so. ``I am still angry that they did this to me,'' she said.
Still, she had hoped to call home to tell them that she was going to America, she said, but there
was no cellphone coverage in the Taliban-dominated area she comes from.

Aisha dressed up for her last day in Kabul in a bright pink pantsuit with tassels and beads, and she
hugged the other girls and women in the shelter. She said she was happy and excited to be going to
America.

She changed into a more somber black gown for the trip to the airport and arranged her veil just
below her eyes. Ms.~Naderi lifted her 14-month-old baby onto her shoulder and they headed for the
plane, Aisha holding her shirt as they walked. It is not, Ms.~Naderi pointed out, a happy ending.

``Her 10-year-old sister is still there and we have no idea how she is,'' she said. ``They're
probably taking all of their anger out on her now, or even demanding another girl from her family to
replace Aisha.''

\section{Kagan Joins Supreme Court After 63-37 Vote in Senate}

\lettrine{T}{he}\mycalendar{Aug.'10}{06} Senate confirmed Elena Kagan to a seat on the Supreme Court
on Thursday, giving President Obama his second appointment to the court in a year and a victory over
Republicans who sharply challenged her credentials and record.

Ms.~Kagan, who is set to be sworn in Saturday as the newest member of the court, was approved by a
vote of 63 to 37 after hearings and floor debate that showcased the competing views of Democrats and
Republicans about the court but exposed no significant stumbling blocks to her confirmation.

In welcoming the Senate action, Mr.~Obama said he expected that Ms.~Kagan would be a strong addition
to the court because she ``understands that the law isn't just an abstraction or an intellectual
exercise.''

``She knows that the Supreme Court's decisions shape not just the character of our democracy, but
the circumstances of our daily lives,'' the president said.

Ms.~Kagan, the former dean of the Harvard Law School, a legal adviser in the Clinton administration
and solicitor general in the Obama White House, becomes the fourth woman to serve on the court. She
will join two other women currently serving, including Justice Sonia Sotomayor, who was confirmed
almost exactly a year ago, and Justice Ruth Bader Ginsburg. She will be the only justice on the
court not to have served previously as a judge.

At age 50, the New York native could have a long tenure, but her confirmation is not seen as
immediately altering the current closely divided ideological makeup of the court, which is often
split 5 to 4 on major decisions. She succeeds Justice John Paul Stevens, the leader of the court's
liberal bloc, who is retiring.

``Her qualifications, intelligence, temperament and judgment will make her a worthy successor to
Justice John Paul Stevens,'' said Senator Patrick J.~Leahy, Democrat of Vermont and chairman of the
Judiciary Committee.

The court she is joining has grown more assertive in placing a conservative stamp on decisions under
Chief Justice John G.~Roberts Jr., and is likely to confront an array of divisive issues in coming
years, like same-sex marriage, immigration and the federal government's role in health care.

Among the cases she is expected to sit in on when the new term starts in October are two major First
Amendment clashes: one involving California's attempts to limit the sale of violent video games to
minors, the other on the free speech rights of protesters at military funerals.

Because of her role as solicitor general in the Obama administration, Ms.~Kagan has already
identified 11 cases on the docket for the next term in which she would disqualify herself because
she had worked on them for the White House. One concerns the privacy rights of scientists and
engineers at the Jet Propulsion Laboratory who object to federal background checks.

In the final vote, 5 Republicans joined 56 Democrats and 2 independents in supporting the
nomination; 36 Republicans and one Democrat, Senator Ben Nelson of Nebraska, opposed her. In a sign
of the import of the moment, senators formally recorded their votes from their desks.

The partisan divide over the nomination illustrated the increasing political polarization of fights
over Supreme Court nominees, who in years past were backed by both parties in the absence of some
disqualifying factor. Ms.~Kagan received fewer Republican votes than Justice Sotomayor, who was
supported by nine Republicans in her 68-to-31 confirmation on Aug.~6, 2009. Democrats balked at
Samuel A.~Alito Jr., nominated by President George W.~Bush, with only four endorsing him in a
58-to-42 vote in January 2006.

Most Senate Republicans challenged Ms.~Kagan's nomination until the end, asserting that she lacked
sufficient experience and had unfairly stigmatized the military by supporting a bar on recruiters at
Harvard Law over the military's policy against allowing gay men and lesbians to serve openly. They
said her record in both Democratic administrations and her strong ties to Mr.~Obama suggested that
she would try to imprint her own political values and those of the president on court decisions.

``Whether it's small-claims court or the Supreme Court, Americans expect politics to end at the
courtroom door,'' said Senator Mitch McConnell of Kentucky, the Republican leader. ``Nothing in
Elena Kagan's record suggests that her politics will stop there.''

Republicans said the need to interpret the Constitution strictly was, in their view, reaffirmed by
this week's federal court ruling against California's voter-imposed ban on same-sex marriage, a case
considered likely to eventually reach the Supreme Court.

Senator Jeff Sessions of Alabama, the senior Republican on the Judiciary Committee, warned that the
American public would ``not forgive the Senate if we further expose our Constitution to revision and
rewrite by judicial fiat to advance what President Obama says is a broader vision of what America
should be.''

But Democrats described the new justice as a brilliant legal scholar who would broaden the outlook
of the court.

``When it opens this fall, three women -- a full third of the bench -- will preside together for the
first time,'' Senator Harry Reid, the Nevada Democrat and majority leader, said. ``That's really
progress.''

Mr.~Obama called Ms.~Kagan's confirmation ``a sign of progress that I relish not just as a father
who wants limitless possibilities for my two daughters, but as an American proud that our Supreme
Court will be more inclusive, more representative and more reflective of us as a people than ever
before.''

Ms.~Kagan has never been a judge and her previous courtroom experience was limited -- she argued her
first case before the Supreme Court last year -- leading some Republicans to cite her lack of time
on the bench as a chief factor in their opposition. They included Senator Scott Brown, a
Massachusetts Republican, who announced Thursday that he would oppose the nomination of the woman he
introduced at her confirmation hearings.

``When it comes to the Supreme Court, experience matters,'' he said in a statement.

Democrats dismissed that argument, with Senator Christopher J.~Dodd of Connecticut noting that more
than one-third of the 111 Americans who have served on the court were not previously judges,
including former Chief Justice William H.~Rehnquist, whose tenure was highly regarded by many
Republicans.

``I would therefore submit to my colleagues that there are other important measures of the quality
of a Supreme Court nominee besides the depth of his or her experience on the bench,'' Mr.~Dodd said.

\section{Google Will Sell Brand Names as Keywords in Europe}

\lettrine{T}{he}\mycalendar{Aug.'10}{06} Internet giant Google said on Wednesday that it would
change its search policy for most of Europe to allow advertisers to buy and use as keywords terms
that have been trademarked by others.

Previously, brand owners could file a trademark complaint with Google to prevent third-party ads
from being returned alongside the results of a search of a trademarked name, like Louis Vuitton or
Prada.

The decision will be effective Sept.~14 and extends to the rest of Europe changes that were made in
Britain and Ireland in 2008. In the United States and Canada, Google has been using the policy since
2004.

Google's move stems from a decision by the European Court of Justice in March. The court broadly
ruled that Google had respected trademark law by allowing advertisers to bid for keywords
corresponding to third-party trademarks.

Brand owners, led by the French luxury goods company LVMH Mo\"et Hennessy Louis Vuitton, had argued
that only they or authorized sites should be able to buy and use such trademarked terms in searches,
so as to protect their brand value. They now face the prospect of having the ads of third-parties
offering their products being displayed in search results.

Trademark owners who feel that third-party ads confuse users as to the origin of the goods and
services will still be able to file complaints with Google, and the search company said it would
take down the ads if it agreed that they were confusing.

Dominic Batchelor, a corporate partner at the legal firm Ashurst based in London, said the decision
``will come as a blow to rights holders'' in the battle ``over policing content in a challenging
online environment.''

``The onus will be on rights holders to monitor and assert their rights,'' he said. ``How readily
Google responds to complaints about infringing use remains to be seen.''

Ben Novick, Google's communications manager for the region, stressed that the company would continue
banning ads for counterfeit goods.

``We work with multiple brands to identify counterfeit goods,'' he said.

Google's clients are now being informed of the changes.

Legal experts have said that the European ruling stopped short of the definitive precedent that
Google and brand owners alike had sought. Instead, it contained caveats that could result in a new
flurry of lawsuits over the sale of ``sponsored links'' generated by Google searches.

Google argues that selling brand names as ad keywords to multiple bidders helps consumers because it
allows them to find product reviews, sellers of second-hand goods and other information. It stressed
that the new policy would not extend to the actual display text in search findings.

``Users will benefit from seeing more relevant ads following a search on Google,'' Mr.~Novick said.

After the introduction of the changes in Britain and Ireland, there was disquiet among some
advertisers, he said, followed by ``a period of calm within a few weeks.''

The changes will benefit the company's hugely successful AdWords service, through which advertisers
bid for keywords; their ads are ranked based on the price paid and other variables. The system works
on a pay-per-click basis.

The company would not speculate how much additional revenue the decision might bring.

AdWords is the bedrock of the company's success, accounting for approximately 95 percent of revenue.

Last month, Google said its second-quarter revenue rose to \$6.82 billion, from \$5.52 billion in
the equivalent quarter a year ago. Net profit rose to \$1.84 billion, from \$1.48 billion.

\section{Google and Verizon Near Deal on Web Pay Tiers}

\lettrine{G}{oogle}\mycalendar{Aug.'10}{06} and Verizon, two leading players in Internet service and
content, are nearing an agreement that could allow Verizon to speed some online content to Internet
users more quickly if the content's creators are willing to pay for the privilege.

The charges could be paid by companies, like YouTube, owned by Google, for example, to Verizon, one
of the nation's leading Internet service providers, to ensure that its content received priority as
it made its way to consumers. The agreement could eventually lead to higher charges for Internet
users.

Such an agreement could overthrow a once-sacred tenet of Internet policy known as net neutrality, in
which no form of content is favored over another. In its place, consumers could soon see a new,
tiered system, which, like cable television, imposes higher costs for premium levels of service.

Any agreement between Verizon and Google could also upend the efforts of the Federal Communications
Commission to assert its authority over broadband service, which was severely restricted by a
federal appeals court decision in April.

People close to the negotiations who were not authorized to speak publicly about them said an
agreement could be reached as soon as next week. If completed, Google, whose Android operating
system powers many Verizon wireless phones, would agree not to challenge Verizon's ability to manage
its broadband Internet network as it pleased.

Since the court decision, involving Comcast, in April, the F.C.C. has been trying to find a way to
regulate broadband delivery, and that effort has been the subject of a series of private meetings at
the agency's headquarters in recent weeks. At the meetings, officials from the nation's biggest
Internet service and content providers, including Google and Verizon, have tried to reach a
consensus on how broadband Internet service should be regulated in light of the decision. Those
meetings continued this week, apart from the talks between Google and Verizon.

The court decision said the F.C.C. lacked the authority to require that an Internet service provider
refrain from blocking or slowing down some content or applications, or giving favor to others. The
F.C.C. has since sought another way in which to enforce the concept of net neutrality. But its
proposals have been greeted with much objection in Congress and among Internet service providers,
cable companies and some Internet content producers.

A spokesman for Verizon said that the company was still engaged in the larger talks to reach a
consensus at the F.C.C. and declined to comment on other negotiations. A spokeswoman for Google also
declined to comment. While a deal between Google and Verizon would affect only those two companies,
it could sway the opinions of lawmakers, many of whom have questioned the wisdom of the F.C.C.'s
plans to oversee broadband service.

At issue for consumers is how the companies that provide the pipeline to the Internet will
ultimately direct traffic on their system, and how quickly consumers are able to gain access to
certain Web content. Consumers could also see continually rising bills for Internet service, much as
they have for cable television.

The prospect of a Google-Verizon agreement infuriates many consumer advocates, who feel that it
would concentrate in a few corporations control of what to date has been a free and open Internet
system in which consumers decide which companies are successful.

``The point of a network neutrality rule is to prevent big companies from dividing the Internet
between them,'' said Gigi B.~Sohn, president and a founder of Public Knowledge, a consumer advocacy
group. ``The fate of the Internet is too large a matter to be decided by negotiations involving two
companies, even companies as big as Verizon and Google.''

It is not clear that the Google-Verizon talks will result in a deal, or that any agreement would
extend beyond those companies. David M.~Fish, a spokesman for Verizon, acknowledged the talks,
saying, ``We've been working with Google for 10 months to reach an agreement on broadband policy.''

But, Mr.~Fish added, ``We are currently engaged in and committed to the negotiation process led by
the F.C.C. We are optimistic this process will reach a consensus that can maintain an open Internet,
and the investment and innovation required to sustain it.''

The F.C.C. process he referred to is what is jokingly called at the agency headquarters ``the secret
meeting.'' At least nine times in the last seven weeks -- including Wednesday, with another meeting
scheduled for Thursday -- a group that includes Google, Verizon, AT\&T, Skype, cable system
operators and a group called the Open Internet Coalition has met with top F.C.C. officials to
discuss net neutrality and the agency's legal basis for regulating Internet service.

Cable and telephone companies want free rein to sell specialized services like ``paid
prioritization,'' which would speed some content to users more quickly for a fee. Wireless
companies, meanwhile, want no restrictions on wireless broadband, which they see as a different
technology than Internet service over wires.

Many content providers -- like Amazon, eBay and Skype -- prefer no favoritism on the Internet or
they want to be sure that if a pay system exists, all content providers have the opportunity to pay
for faster service.

The F.C.C., meanwhile, favors a level playing field, but it cannot impose one as long as its
authority over broadband is in legal doubt. It has proposed a solution that would reclassify
broadband Internet service under the Communications Act from its current designation as an
``information service,'' a lightly regulated designation, to a ``telecommunications service,'' a
category that, like telephone service, is subject to stricter regulation.

The F.C.C. has said that it does not want to impose strict regulation on Internet service and rates,
but seeks only the authority to enforce broadband privacy and guarantee equal access. It also wants
to use federal money to subsidize broadband service for rural areas.

While the F.C.C. is gathering public comment on its reclassification proposal, it has convened the
private talks, which are overseen by Edward Lazarus, the chief of staff to Julius Genachowski, the
F.C.C.'s chairman.

The talks have produced some common ground among the participants on smaller matters. But one
participant, who spoke on the condition of anonymity because the group members agreed not to discuss
their deliberations publicly, said there had been little movement ``on the few big issues that are
the most important.''

Frustrated with that lack of progress in the last two months, direct talks between Google and
Verizon have accelerated, according to people close to the discussions who were not authorized to
comment publicly.

Google and Verizon have their own interests at stake in negotiating separately. The Android
operating system from Google is used on many Verizon phones, including the Droid, a competitor to
the iPhone from Apple.

Consumer groups have objected to the private meetings, saying that too many stakeholders are being
left out of discussions over the future of the Internet.

Mr.~Lazarus said the meetings ``are part of our efforts to identify the best way forward in the wake
of the Comcast case to preserve the openness and vibrancy of the Internet.''

\section{Turkey and Israel Do a Brisk Business}

\lettrine{I}{sraeli}\mycalendar{Aug.'10}{06} business executives here like to point out that most of
the angry Turks who protested Israel's deadly raid on a Turkish-led flotilla to Gaza this past
spring do not know that their cellphones, personal computers and plasma televisions were made using
parts and technology from Tel Aviv.

For Menashe Carmon, chairman of the Israel Turkey Business Council, such ignorance is a blessing for
Israelis and Turks.

``Turks would find it very hard to boycott Israeli goods because you won't find any in Turkish
supermarkets,'' Mr.~Carmon said. ``But most of the software Turks use in everything from cell phones
to medical equipment is made in Israel. So unless Turks want to stop using their computers,
boycotting Israel would mean punishing themselves.''

After the raid, in which nine Turkish citizens were killed on May 31, Turkey demanded an apology
that it has yet to receive. It barred Israeli military planes from Turkish airspace, while its
Islamist-inspired prime minister said the world now perceived the Nazi swastika and the Star of
David together, according to the Hurriyet Daily News, a Turkish newspaper critical of the
government.

Israelis, meanwhile, stung by the raw contempt of their former ally in the region, vowed to keep
away from Turkey.

But when it comes to the real economy, business pragmatism is trumping political tensions. ``No
Israeli companies are leaving Turkey,'' said Mr.~Carmon, an Israeli entrepreneur who was raised in
Istanbul. ``It is business as usual and if anything, investment is growing.''

In the short term, the flotilla raid has produced some inevitable economic fallout. The widespread
cancellations of holiday bookings by Israelis will cost Turkey some \$400 million, analysts say.
Turkey, meanwhile, said it would scrutinize all military cooperation, potentially depriving Israeli
companies of billions of dollars in lucrative contracts.

Yet Israeli companies selling everything from computer software to water irrigation systems in
Turkey insist that they have not been affected by recent events. In part, that is because they
operate mostly in joint ventures with Turkish companies, making their Israeli identities invisible.
It is a sign of the times that not a single Israeli company doing business here was willing to be
quoted by name for fear that they or their Turkish customers could be hounded.

Bilateral trade between the two countries officially amounted to about \$3 billion last year. But
Israeli and Turkish business leaders say the economic ties are actually much larger.

The extensive business connections are largely camouflaged, they say, because many Israeli
businesses use their Turkish partner companies to sell to the Arab world while Turkish companies use
their Israeli partners as a gateway to American markets.

Even on the defense front, Turkish officials say that close cooperation between Israel and a Turkish
military at odds with the Islamist government in Ankara is continuing behind the scenes. Israeli
officials may be resigned to losing some immediate Turkish government contracts, but they remain
confident that pragmatic interests will win out over ideological differences.

``While the politicians are trying to profit from the conflict, the army has remained remarkably
quiet,'' said Mehmet Altan, a leading Turkish columnist. ``Both Israel and the Turkish military
establishment want a secular Turkey, so they are fighting for the same thing.''

Within weeks of the flotilla raid, a Turkish military delegation arrived in Israel to learn how to
operate the same pilotless aircraft often used by Israel to hunt Palestinian militants in the Gaza
Strip. The \$190 million deal for the drones was not canceled, even as the Israeli instructors in
Turkey were called home after the raid.

Doron Abrahami, consul for economic affairs at the Israeli Consulate in Istanbul, noted that before
the flotilla clash, Israel's military industry had teamed up with a Turkish partner to help
modernize a fleet of 170 Turkish tanks in a project valued at \$700 million. He said the Israeli and
Turkish partners were now shopping around their expertise to other countries.

``Business is business,'' he said, showing off an invitation dated July 15, co-signed by economic
agencies in Turkey and Israel just weeks after the Israeli raid, inviting Israeli and Turkish
companies to bid for a jointly financed research and development project, one of more than 20 such
efforts he said were under way.

In 1949, Turkey was one of the first countries to recognize Israel shortly after the country
declared its existence in 1948. The two have forged strong military and trade ties, but diplomatic
and political relations have deteriorated in recent years, as alarm has grown in the United States
and Europe that Turkey is turning its back on the West and courting Israel's enemies like Iran.

In January 2009, the Turkish prime minister, Recep Tayyip Erdogan, stormed out of the World Economic
Forum in Davos, after clashing with the Israeli president, Shimon Peres. In January of this year,
Israel apologized after its deputy foreign minister insulted the Turkish ambassador by forcing him
to sit on a lowered sofa.

Yet for all of the recent episodes of mutual recrimination, Turkish and Israeli business people
remain close.

Necat Yuksel is export manager at Naksan Plastik, a large Turkish plastic packaging producer in
Gaziantep, in Turkey's southeast, that imported some \$40 million worth of plastic chemicals from
Israel last year. He said sales from Israel showed no signs of abating, even as the recent clash
with Israel had exerted a damaging psychological effect on both countries.

His Israeli customers are now wary of travelling to Turkey, he said, and his best Israeli client now
refers to him as ``Erdogan,'' after Turkey's prime minister. Yet not a single contract had been
canceled. Nor has his company shelved its plans to establish a factory in Israel.

He proudly cited many advantages to doing business with Israel, including geographic proximity and a
shared mentality. ``All the problems are between the politicians,'' Mr.~Yuksel said. ``Israelis,
hot-tempered and stubborn, are just like us Turks.''

Mr.~Yuksel, who has been visiting Israel for more than a decade, argued that Israeli executives were
far more influenced by recent political events than Turks. ``For us it comes down to profits,'' he
said. ``For the Israelis, it's emotional.''

Yet most Turks are adamant that Israel needs Turkey far more than Turkey needs Israel. Sinan Ulgen,
a leading economist in Istanbul, argued that Israel had far more to lose than Turkey from severed
ties. Sales to Israel made up about 1.5 percent of Turkey's total exports of \$102 billion last
year, making it Turkey's 17th biggest market, according to the State Statistics Agency in Ankara.
Israel exported some \$1.04 billion to Turkey last year, making Turkey its eighth largest export
market.

At the political level, Mr.~Ulgen noted that when ties were strong, Turkey provided an isolated and
tiny Israel with a large Muslim ally in a perilous region.

But Rifat Bali, a Turkish Jew who had written widely on Turkish-Israeli relations, countered that
bad relations with Israel were riskier for Turkey by undermining its relations with the United
States. They also stifled Turkey's aspirations to be a regional power by depriving Turkey of the
ability to play a mediating role. He said Israel was one of the only countries willing to sell arms
to Turkey with no strings attached.

``Both Turkey and Israel,'' Mr.~Bali said, ``need each other far more than either is willing to
admit.''

\section{As Indonesians Go to Mecca, Many Eyes Follow Their Money}

\lettrine{A}{s}\mycalendar{Aug.'10}{06} the nation with the world's largest number of Muslims,
Indonesia every year sends the most pilgrims to Mecca by far. About one out of 10 believers who
performed the hajj last year were Indonesian.

Some 1.2 million of the faithful are now on a government waiting list to go to Mecca, filling this
country's annual quota through the next six years. But if the rapidly lengthening list is a
testament to Indonesia's growing devotion, it has also become a source of one of its perennial
problems: corruption.

Government officials and politicians misuse the money deposited by those on the waiting list -- now
totaling nearly \$2.4 billion -- according to government investigators and anti-corruption groups.
With friendly travel agents and business allies, officials exploit the myriad requirements of the
state-run hajj to fatten their own pockets, watchdog groups say. Corruption, they say, has
contributed to consistent complaints about cramped accommodations for pilgrims in Saudi Arabia and
catering services that stop delivering food midway through the trip.

The national Parliament and officials at the Ministry of Religious Affairs recently settled on the
price of this year's hajj after unusually protracted negotiations and accusations, widely reported
in the news media here, that some lawmakers and bureaucrats had agreed to share \$2.8 million in
bribes from the ministry. The annual negotiations are used by veteran bureaucrats and lawmakers to
hammer out personal deals, according to anti-corruption groups and the news media, which have
labeled them the ``hajj mafia.''

``We can't prove the existence of the hajj mafia yet,'' said Muhammad Baghowi, a lawmaker who was
elected last year and sits on a parliamentary commission that oversees religious affairs. ``But
given all the indications, you can really sense it.''

Parliamentary leaders and ministry officials have denied the bribery accusations. Abdul Ghafur
Djawahir, a high-ranking official at the ministry's hajj division, said anti-corruption groups had
misinterpreted the ministry's procedures and handling of the deposit money. He said they had also
wrongly evaluated the costs of flights to Saudi Arabia and unfairly compared Indonesia's hajj
management with that of Malaysia, where pilgrims are reported to pay less and get better service.

``That's what, in the end, forms the public's opinion that there is huge corruption here,''
Mr.~Djawahir said, adding that there was ``no hajj mafia'' and that the ministry was ``completely
clean.''

Ministry officials and lawmakers pointed out that the price for this year's hajj, which is scheduled
for mid-November, had been lowered by \$80 to \$3,342, compared with last year. But anti-corruption
groups argue that without graft and mismanagement the cost would be several hundred dollars lower.

Despite the convictions in 2006 of ministry officials, including a former minister, for misusing
hajj funds and bribing state auditors to validate the ministry's accounts, anti-corruption advocates
say that little has changed.

According to Indonesian Corruption Watch, in the deal-making between the ministry and Parliament,
lawmakers win hefty allowances on hajj trips for themselves and their relatives, and travel agencies
and other businesses with political ties are handed contracts for catering or transportation. In
return, lawmakers do not question the ministry's handling of the \$2.4 billion in deposits,
especially the accrued interest.

``What the money is used for, we never know,'' said Ade Irawan, a researcher at Indonesian
Corruption Watch, the country's leading private anti-corruption organization. ``That's the people's
money, public money, the pilgrims' money.''

The Indonesian Pilgrims Rabithah, a private organization that has long pressed for reform of the
hajj management, said the ministry and lawmakers negotiated away from public forums to keep their
deals hidden.

``There is never any public accountability,'' said Ade Marfuddin, the organization's chairman,
adding, ``No one knows who gets what except them.''

In a recent report, the Corruption Eradication Commission, the government's main anti-corruption
agency, identified 48 practices in hajj management that could lead to corruption. Mochammad Jasin, a
deputy chairman of the commission, said the commission would wait to see whether the ministry
carried out suggested reforms before considering a full-fledged investigation into possible
wrongdoing.

According to quotas established by Saudi authorities, 211,000 Indonesians will be allowed to go to
Mecca this year. About 17,000 of them will go on private tours costing several times the state-run
package of \$3,342 -- a sum that often entails a lifetime of savings and the sale of property or
livestock.

Unable to afford the state-run hajj, Arif Supardi, 53, entered Saudi Arabia on a business visa
shortly before the hajj a couple of years ago. (The Saudi government estimated that 30 percent of
the 2.5 million pilgrims last year went to Mecca without valid permits.) He said he managed to
complete his pilgrimage for \$2,000 by becoming what he and others called ``hajj backpackers.''

``There were many from Indonesia, mostly because of the cost,'' he said.

Prospective pilgrims must now deposit \$2,500 to register for the hajj, effectively lending the
ministry that amount until their turn to go on the hajj comes up six years later. According to the
religion ministry, between 15,000 and 20,000 people register every month. Interest in performing the
hajj, a pilgrimage that is an obligation for any physically and financially able Muslim adult, has
risen in the past decade as Indonesians have grown wealthier and increasingly given Islam an
important place in their lives.

But Ian Imron, 38, who owned a travel agency offering private hajj tours from 1988 to 2006, said the
growing interest also led to an overemphasis on the business side of the hajj. Travel agencies with
political ties and large capital have mushroomed. When he ran into financial difficulties in 2006,
Mr.~Imron took it as a sign to quit the business.

``Maybe at the beginning, it was really about religion,'' Mr.~Imron said. ``But then it became more
about business.''

In a wealthy neighborhood in southern Jakarta, Al Amin Universal travel agency boasts that it has
taken prominent politicians on the hajj on private tours. Employees at the agency said its owners --
the family of Melani Suharli, the deputy speaker of the People's Consultative Assembly, a
legislative branch -- were unavailable to talk.

Despite the widely reported poor service on the state-run trips, most pilgrims do not complain as
ministry officials warn them that airing grievances will mar their religious experience,
anti-corruption groups said.

Achmad Fachin, 50, who sold his family car to go to Mecca with his wife, said he did not complain
during their hajj but has grown angry about the corruption.

``But, in the end, let them be,'' he said. ``They'll have to take responsibility for whatever they
do. We were performing our religious duty and paid the fees with sincerity.''

\section{BlackBerry Security Stance Sows Anxiety}

\lettrine{T}{he}\mycalendar{Aug.'10}{09} 2008 terrorist attacks in Mumbai heightened concerns in
India over the government's inability to eavesdrop on encrypted communications. In the United Arab
Emirates, similar concerns escalated this year after a Palestinian operative was killed in a hotel
in Dubai, possibly by a team from the Mossad, the Israeli intelligence agency.

In both countries, those concerns have crystalized into a battle with Research In Motion, the
Canadian maker of BlackBerry smartphones, over whether and how their governments can gain access to
messages that flow over the BlackBerry network. And the dispute has put a spotlight on the
challenges faced by many governments in monitoring communications services with global reach.

Last week the Emirates threatened to block BlackBerry's e-mail and instant messaging services in
that country unless R.I.M. created back doors to allow officials to eavesdrop on the company's
customers. Saudi Arabia has made a similar threat, and news reports over the weekend suggested that
a deal had been made, but it was unclear what any deal might involve. Lebanon has also raised
concerns. Indian officials have been negotiating with R.I.M. over access to BlackBerry messages for
a couple of weeks.

Although it is unclear precisely what these countries are asking for, one demand is for the same
kind of access to BlackBerry's encrypted services that they think the company already gives
authorities in the United States and other industrialized democracies.

``I don't think the concerns raised by India are out of the ordinary,'' Sachin Pilot, the country's
junior minister for communications and information technology, said in a phone interview. ``Most
countries in the Western world have raised the issues and to the best of my information -- and I am
willing to be corrected -- their concerns have been addressed.''

R.I.M. officials flatly denied last week that the company had cut deals with certain countries to
grant authorities special access to the BlackBerry system. They also said R.I.M. would not
compromise the security of its system.

At the same time, R.I.M. says it complies with regulatory requirements around the world.

But the company, which is generally known for its secrecy, has declined to provide details on its
discussions with governments or to explain how it complies with laws around the world that require
communications companies to grant government agencies access to their systems for lawful intercepts.
This has kept alive suspicions in some foreign capitals and among computer security experts in the
United States that R.I.M. has made concessions to some countries.

``There are all kinds of rumors that various deals have been struck around the world, including in
the United States, but we don't know what those deals are,'' said Leslie Harris, chief executive of
the Center for Democracy and Technology, which is based in Washington, and a board member of the
Global Network Initiative, a coalition of companies and nonprofit groups that seeks to protect
privacy and free expression on the Internet.

Speaking privately, several United States law enforcement and security officials would not say
whether the government has a way to decrypt BlackBerry messages, explaining that they were reluctant
to divulge whether any particular service posed difficulties. But there has been little public sign
of law enforcement frustration with BlackBerry encryption.

The officials said that law-enforcement agencies in the United States had an advantage over their
counterparts overseas because many of the most popular e-mail services -- Gmail, Hotmail and Yahoo
-- are based here, and so are subject to court orders. That means the government can often see
messages in unencrypted forms, even if sent from a BlackBerry. In addition, officials said that in
emergencies, when lives might be in danger, they sometimes sought the voluntary assistance of
companies, including those outside the United States.

David Szuchman, the chief of the cybercrime and identity theft bureau at the Manhattan district
attorney's office, said investigators occasionally obtained from R.I.M. logs of Internet protocol
addresses associated with e-mails sent through the BlackBerry system, but they were unable to obtain
the contents of the e-mails from the company.

Security experts say that BlackBerry's system, while secure, is susceptible to eavesdropping. Some
even say that its design allows companies, and potentially governments, to monitor messages, though
not necessarily with R.I.M.'s cooperation.

Most experts say R.I.M.'s service for consumers is not at issue, because the messages are largely
unencrypted and easy for governments to intercept.

R.I.M. sells its most secure service to corporations. The corporate service encrypts e-mails and
messages and largely routes communication through R.I.M.'s private network. As a selling point,
R.I.M. has provided corporations with tools allowing them to monitor and archive everything that
happens on all BlackBerrys they own through BlackBerry Enterprise Servers, which act as a gateway to
a company's e-mail systems.

Security experts say governments wanting to eavesdrop on some people could easily demand access from
those companies.

``R.I.M. could be technically correct that they are not giving up anything,'' said Lee Tien, a
senior staff lawyer at the Electronic Frontier Foundation, a San Francisco group that promotes civil
liberties online. ``But their systems are not necessarily more secure because there are other places
for authorities to go to.''

When China first allowed BlackBerry service in the last few years, sales were restricted to
hand-held devices linked to enterprise servers within the country. Many security experts say Chinese
security agencies have direct access to all data stored on those servers, which are often owned by
government-controlled corporations.

China Mobile and China Telecom, the two mobile operators that sell BlackBerry devices in China, did
not respond to requests for interviews. But technology analysts in China said R.I.M. would have to
comply with the government's strict regulations over communication and perhaps give the government
some access to its encrypted data.

Whether Chinese regulators have such access or not is unclear, but a customer service representative
at China Mobile suggested that government officials could intercept communications through a
BlackBerry Enterprise Server operated by the carrier. ``If someone, like government departments,
want to monitor the message, they can do it,'' said the representative, who would not give his name.

In India, Mr.~Pilot, the communications minister, said the government would like to have the ability
to monitor communications if officials suspected that a crime was being committed. ``Time is of the
essence,'' he said. ``Even five minutes can make a big difference.''

He said that R.I.M. executives had met with government officials but that the company had not
formally responded to the government's requests. He declined to say what the government might do if
it were not satisfied by the company's response.

But legal experts said the authorities appeared to be seeking more wide-ranging access to BlackBerry
messages and codes than the company has made available to other countries.

New Delhi has powerful legal tools, including a recently changed Indian law that gives the
government the power to intercept any ``computer communication'' without court order to carry out
criminal investigations. Previously, officials had to seek an order from a quasi-judicial body, the
Controller of Certifying Authorities.

The amended law was passed by Parliament in December 2008, a month after raids by Pakistani
terrorists in Mumbai left more than 163 people dead. Lawmakers seem to have been motivated, in part,
by the fact that the attackers used Internet-based phone services to communicate with their handlers
in Pakistan.

The government's negotiations with R.I.M. could become the first significant test of its new
expanded authority.

The tensions between R.I.M. and various governments echo a debate that took place in the United
States in the 1990s, when the government sought to ban the use of strong encryption systems. In the
face of strong opposition from technology companies and privacy advocates, the government gave up.

But the debate has continued, as new technologies, like Skype or widely used encrypted Web e-mail
systems like Gmail, raise new challenges. In 1994, Congress passed a law requiring
telecommunications companies to design their systems so that authorities could legally eavesdrop on
them, but the law does not cover some new Internet communications technologies.

``It is a battle over the last 20 years as to how much we want communications to be secure and how
much we want government'' to be able to monitor illegal activity, said Mark Rasch, who for nine
years headed the computer crimes division at the Department of Justice. ``It is a balance we strike
with every new technology in every new country.''

\section{Landslides Kill 127 in China}

\lettrine{A}{}\mycalendar{Aug.'10}{09} landslide buried and flooded hundreds of homes over the
weekend in a remote mountainous region of Gansu Province. Officials said Sunday evening that 127
bodies had been recovered and that nearly 2,000 people were missing.

Prime Minister Wen Jiabao, who frequently flies to major disasters, visited a town near the
hardest-hit city, Zhouqu, in far southern Gansu, to supervise relief operations.

Rescuers told Xinhua, the official news agency, that earthmovers could not reach the scene and that
workers were digging with hands and shovels to reach people who were trapped.

Torrential rains began to fall around 10 p.m. Saturday, the head of Zhouqu County, who goes by the
single name of Diemujiangteng, told Xinhua.

``Then there were landslides, and many people were trapped,'' he said. ``Now sludge has become the
biggest hindrance to our operations. It's too thick to walk or drive through.''

The state-run broadcaster CCTV placed the death toll at 127, and Xinhua said nearly 2,000 were
missing. But it was uncertain how many people were hit by the slide or had fled.

Grainy video from the town showed streets covered in thick mud and debris.

``There's water everywhere,'' Liu Yiping, an official in the Zhouqu government, told CCTV by
telephone. ``It's flooded everything. It's just too horrible to witness. It's so awful.''

The government said it had sent 2,400 soldiers from Lanzhou, the capital of Gansu, about 170 miles
to the north, to help with the rescue. About 1,000 firefighters and other workers were sent from
nearby towns.

The slide appeared to be the worst in a string of weather-related events that have struck China over
a summer marked by torrential rain and high temperatures.

The government said Wednesday that floods had hit 28 provinces this year, killing 1,072 people and
leaving 619 missing.

The waters have wrecked 1.1 million homes and caused an estimated \$31 billion in damage.

A mudslide on July 27, caused by heavy rain, buried at least 21 people in Sichuan Province.

While floods have mainly afflicted the north, much of southern China has suffered the worst drought
in decades, forcing the government to import and burn vast stocks of coal to supplement dwindling
hydroelectric power.

The landslide on Sunday struck a rural area on the Sichuan border, which is about one-third ethnic
Tibetan and has many farmers and herders.

Officials said the slide hit about midnight in Zhouqu, a city of about 40,000 bounded by steep
mountains and bisected by the Bailong River, which runs lengthwise through the narrow valley.

The debris blocked the river, creating a lake 30 feet deep that flooded half of the town by 1 a.m.
and sent residents fleeing to upper floors of buildings. Some of the missing people were believed to
have been swept away by floodwaters.

Xinhua said the mud flow, up to six feet thick, inundated buildings and major roads in an area about
3 miles long and 500 yards wide. More than 300 homes in an adjacent village, Yueyuan, were also
covered by the slide, rescuers said.

By noon Sunday, 680 people had been rescued and 45,000 evacuated from the slide area. An additional
76 people were reported to have been injured.

By midafternoon, workers had cleared debris from roads to the town, Xinhua said, but floodwaters in
the city center were still impeding rescuers.

\section{India Swoons Over Its Chess Champ, and Even the Game}

\lettrine{T}{he}\mycalendar{Aug.'10}{09} girls gathered in a school auditorium here on a recent
Saturday were beaming with pride and nervous with anticipation. They would soon have a chance to
meet the star of their dreams: Viswanathan Anand.

``I want to be the next Vishy,'' declared Chetna Karnani, 16, referring to Mr.~Anand by his
nickname. ``I practice four hours every day.''

Mr.~Anand is no Bollywood heartthrob or pop singer. The idol the girls were swooning over was an
unassuming, bespectacled, 40-year-old world chess champion.

Mr.~Anand, who has held the world title for three years, appears to have earned the fame that India
usually reserves for movie stars, cricket players and politicians. The girls had come to school on a
Saturday with the hope of playing a game with him.

When he arrived with a retinue of four bodyguards to protect him from getting mobbed, the
star-struck students sheepishly sought his autograph and peppered him with questions about his last
title match, against the Bulgarian Veselin Topalov.

Historians say chess has roots in the ancient Indian games of chaturanga and shatranj, which were
widely played here at one time. But chess has never taken hold in modern India. Mr.~Anand is the
first Indian ever to win the championship.

But Mr.~Anand's success -- he was the world junior champion at age 17 and held his first world title
at age 31 -- has created a groundswell of enthusiasm for the game. Amit Varma, a popular Indian
blogger, equated his impact here with the following Bobby Fischer created for chess in the United
States when he defeated the Russian grandmaster Boris Spassky in 1972. ``As I write these words, the
day after his win, the newspapers and TV channels are full of him,'' Mr.~Varma wrote after
Mr.~Anand's recent defeat of Mr.~Topalov in a column for Yahoo India. ``Chess, amazingly enough,
might just be on its way to becoming a spectator sport in India.''

Mr.~Anand has used his fame to promote the game in India, sponsoring a nationwide network of chess
clubs like the one at Ms.~Karnani's high school, Sadhu Vasvani International School for Girls in
southern Delhi. Officials estimate the clubs, which are administered by an Indian education company,
NIIT, have signed up 850,000 students. Mr.~Anand and NIIT, which sells technology and curriculum to
schools, said they hoped to reach 5 million to 10 million students in the next five years.

Unlike the organized network of chess academies that the Soviet Union created to dominate the game,
the NIIT-run clubs are relatively informal and are designed as an extracurricular activity.

The goal, Mr.~Anand said, is not to produce other Indian grandmasters or champions, though he
welcomes that as a potential side effect. Rather, he wants to get young people interested in chess
as a tool to improve their ability to focus, analyze and reason. ``We are very happy to produce
chess champions,'' he said. ``But we want to create mind champions.''

As part of the program, Mr.~Anand travels around the country to meet students, play chess with them
and answer their questions. He also attends an annual nationwide competition among schools with NIIT
chess clubs, and tutors students who reach the regional finals and the national finals before and
after their games.

On the Saturday he visited the Sadhu Vasvani school, he arrived with his bodyguards and his wife and
business manager, Aruna, at 10 a.m. He gave a two-hour chess tutorial, answering questions and then
simultaneously playing matches with 30 students. To compensate those who were not lucky enough to be
picked to play with him, he posed for photos with them, smiling awkwardly.

In the afternoon, he was on a panel discussing whether and how much chess helps children improve
their mental faculties. In the evening, he was celebrated at a dinner hosted by the chief minister
of New Delhi State.

Mr.~Anand, who learned to play chess from his mother when he was 6, honed his skills at an early age
by playing at a chess club in the southern city of Chennai, where the game has long had a
stronghold.

Driving from the school to the panel discussion, Mr.~Anand said he was surprised that the students
that morning had been able to ask and answer detailed questions about the moves that led to his
victory over Mr.~Topalov in May. ``Before you had to go to a chess club to get that level of
interaction,'' he said. ``Now, you are getting it in the school system.''

\section{Boss's Stumble May Also Trip Hewlett-Packard}

\lettrine{T}{he}\mycalendar{Aug.'10}{09} events were billed as C.E.O. executive summit meetings,
exclusive gatherings, often lasting several days, where Hewlett-Packard officials wooed top
customers. When Mark V.~Hurd, H.P.'s chief, appeared at them, he sometimes relied on Jodie Fisher, a
50-year-old former reality television contestant turned H.P. marketing consultant, who would
introduce him to customers and keep him company.

Mr.~Hurd's relationship with Ms.~Fisher, which led to his ouster last week, has put an unsavory end
to one of the great executive runs in recent American business history. And it has stunted a long
search by H.P.'s employees for stability and pride at the patriarch of Silicon Valley companies.

On Sunday, Ms.~Fisher, who had accused Mr.~Hurd, 53, of sexual harassment, disclosed her identity in
a statement from her lawyer and said that she had never had a sexual relationship with Mr.~Hurd. ``I
was surprised and saddened that Mark lost his job over this,'' she said. ``That was never my
intention.''

Mr.~Hurd, who is married, has settled the matter with Ms.~Fisher for an undisclosed sum.

H.P.'s top executives on Sunday said they would no longer discuss Mr.~Hurd's situation and vowed to
find a new chief executive to keep the business running smoothly. ``We are not going to slow down
one bit,'' said Cathie Lesjak, the chief financial officer and interim chief executive.

But turning the page on the scandal will not be easy. While Ms.~Lesjak maintained that investors
remained confident in the company, H.P.'s share price tumbled 10 percent on Friday as word of
Mr.~Hurd's departure rippled through Wall Street.

Analysts had come to view Mr.~Hurd as a stabilizing presence who galvanized the formerly chaotic
company, and as the glue that held a complex organization with more than 300,000 employees together.

They saw him as having a knack for finding new areas where H.P. could lower costs, and for
maintaining order among the top executives. And Mr.~Hurd led the company's charge past I.B.M. as the
top seller of technology.

In a nod to Mr.~Hurd's influence on the company, Ms.~Lesjak said, ``Disciplined execution has become
part of H.P.'s DNA.''

The new leader of H.P. will have to maintain fiscal discipline while also coming up with a second
act that stimulates new growth for the company.

Over the past 10 years, H.P.'s employees have endured a series of unsettling events. ``If I was an
H.P. employee, I would feel betrayed,'' said Michael S.~Malone, a historian who wrote ``Bill and
Dave,'' a book about the company's founders, William Hewlett and David Packard. ``It's almost like
there is a stigma attached to this company where you have people at the top who have done you in
again and again.''

When Carly Fiorina, Mr.~Hurd's flashy predecessor, began the contentious acquisition of the PC maker
Compaq Computer, some veterans argued it would put an end to H.P.'s culture. Ms.~Fiorina went on to
preside over years of inconsistent financial performance and large layoffs.

Mr.~Hurd took over the chief executive job in 2005. H.P.'s board hoped he could put a bit of the
boring back into the company by delivering steady results and staying out of the limelight.

In his days playing tennis at Baylor University, Mr.~Hurd had long hair, wore shorts everywhere and
puffed on cigarettes during breaks. He went on to build a rather different persona as the
suit-wearing master of metrics, glasses at the tip of his nose, a memorizer of business minutiae.

``Everyone hoped for this resurrection of Bill and Dave,'' Mr.~Malone said. ``They didn't get that,
but they got the next best thing -- a guy who was low key and straightforward.''

But Mr.~Hurd soon was embroiled in controversy, as he inherited a corporate espionage debacle in
which H.P. was found to have spied on reporters, its board members and employees. Still, Mr.~Hurd
emerged from this episode with a promotion, adding the chairman title, previously held by Patricia
C.~Dunn.

Part of Mr.~Hurd's role in recent years has included showing up at the C.E.O. summit meetings where
customers would sometimes spend a weekend at a resort in the United States, Europe or Asia with
their families.

People close to Mr.~Hurd portrayed him as abhorring business lunches and dinners, preferring instead
to do business on his own. One said that while Mr.~Hurd did not have a romantic relationship with
Ms.~Fisher, he did enjoy her company at meals as he wound down from long days.

Gloria Allred, Ms.~Fisher's lawyer, said that her client, ``is a single mom focused on raising her
young son.'' She said Ms.~Fisher had a degree in political science and had worked in commercial real
estate. Ms.~Fisher appeared in ``Age of Love'' on NBC.

An executive who attended a number of these events said Mr.~Hurd would often give a talk at the
beginning and make the occasional appearance at a party.

H.P. has said that Mr.~Hurd approved paying Ms.~Fisher \$1,000 to \$10,000 per event and then dined
with her afterward. After the company stopped holding the events, Ms.~Fisher presented H.P. with a
sexual harassment claim against Mr.~Hurd, although a company investigation did not result in charges
of misconduct.

After H.P.'s board learned of the sexual harassment charges, it began the investigation and hired
the consulting firm APCO to evaluate the damage such a revelation could cause if made public,
according to people with direct knowledge of the situation.

A debate subsequently erupted over whether Mr.~Hurd needed to disclose the charges against him
publicly, according to a person close to Mr.~Hurd. APCO told the board that the company would most
likely endure a devastating public relations hit if Mr.~Hurd stayed on as chief executive, according
to that person.

Last week, H.P.'s executives said that Mr.~Hurd had falsified expense reports and had broken the
trust of the board by trying to hide his relationship with Ms.~Fisher. They cited the fiscal
malfeasance as the primary reason for his ouster, since sexual harassment was not supported by the
investigation.

The person close to Mr.~Hurd said the expense reports were not discussed in meetings leading up to
Mr.~Hurd's resignation, with the focus instead on whether the company needed to disclose the
situation.

According to his peers, Mr.~Hurd has set a high bar for any successor. ``He has done such a
brilliant job at H.P.,'' wrote Lawrence J.~Ellison, the chief executive of H.P.'s partner and rival
Oracle, in an e-mail. ``He will be very hard to replace.''

\section{In Crackdown on Energy Use, China to Shut 2,000 Factories}

\lettrine{E}{arlier}\mycalendar{Aug.'10}{10} this summer, Prime Minister Wen Jiabao of China
promised to use an ``iron hand'' to improve his country's energy efficiency, and a growing number of
businesses are now discovering that it feels like a fist.

The Ministry of Industry and Information Technology quietly published a list late Sunday of 2,087
steel mills, cement works and other energy-intensive factories required to close by Sept.~30.

Energy analysts described it as a significant step toward the country's energy-efficiency goals, but
not enough by itself to achieve them.

Over the years, provincial and municipal officials have sometimes tried to block Beijing's attempts
to close aging factories in their jurisdictions.

These officials have particularly sought to protect older steel mills and other heavy industrial
operations that frequently have thousands of employees and have sometimes provided workers with
housing, athletic facilities and other benefits since the 1950s or 1960s.

To prevent such local obstruction this time, the ministry said in a statement on its Web site that
the factories on its list would be barred from obtaining bank loans, export credits, business
licenses and land. The ministry even warned that their electricity would be shut off, if necessary.

The goal of the factory closings is ``to enhance the structure of production, heighten the standard
of technical capability and international competitiveness and realize a transformation of industry
from being big to being strong,'' the ministry said.

The announcement was the latest in a series of Chinese moves to increase energy efficiency. The
National Development and Reform Commission, which is the government's most powerful economic
planning agency, announced last Friday that it had forced 22 provinces to halt their practice of
providing electricity at discounted prices to energy-hungry industries like aluminum production.

The current Chinese five-year plan calls for using 20 percent less energy this year for each unit of
economic output than in 2005. But surging production by heavy industry since last winter has put in
question China's ability to meet the target.

The success or failure of China's energy-efficiency campaign is being watched closely not just by
economists, who cite the campaign as one reason that growth of the Chinese economy has slowed down a
little this summer, but also by climate scientists.

China's energy consumption rose so sharply last winter that it produced the biggest surge ever of
greenhouse gases by a single country. Power plants burned more coal to generate enough electricity
to meet demand.

As China has become increasingly dependent on imported oil and coal, its national security
establishment has become more visibly involved in energy policy and energy security, including
efforts to improve energy efficiency.

Efficiency improved 14.4 percent in the first four years of the current plan, only to deteriorate by
3.6 percent in the first quarter of this year, according to official statistics. Mr.~Wen responded
by convening a special meeting of the cabinet in May to address the situation.

Energy efficiency was only 0.09 percent worse in the first half of this year than in the same period
in 2009, according to statistics released last week.

Energy analysts said those statistics indicated improvement in efficiency in the second quarter that
nearly offset the deterioration in the first quarter, although the government has not released
separate figures for the second quarter.

Zhou Xizhou, an associate director for IHS Cambridge Energy Research Associates in Beijing, said
that the ministry's new list of factory closings was a strong measure to improve efficiency. But he
added that China's goal of achieving a 20 percent improvement by the end of this year compared with
2005 was ``still a tall order for the rest of the year.''

The ministry said in its statement that the factories to be closed would include 762 that make
cement, 279 that produce paper, 175 that manufacture steel and 84 that process leather.

The factories were chosen after discussions with provincial and municipal officials to identify
industrial operations with outdated, inefficient technology, the ministry said.

The ministry did not provide figures for the percentage of capacity to be closed in each industrial
sector. The ministry also did not say how many employees would be affected.

Closing factories is more palatable now than in the past because a labor shortage in many cities has
made it easier for workers, particularly young ones, to find other jobs.

The list of steel mills to be closed appeared to emphasize smaller, older mills producing fairly
low-end grades of steel.

Edward Meng, the chief financial officer of China Gerui Advanced Materials, a steel-processing
company in central China's Henan Province, said that the closing of such mills was consistent with
the government's broader goals of consolidating the steel sector and pushing steel makers into the
production of more sophisticated kinds of steel.

The International Energy Agency in Paris announced last month that China surpassed the United States
last year as the world's largest consumer of energy.

China passed the United States as the world's largest emitter of greenhouse gases in 2006. That
milestone came earlier because of China's heavy reliance on coal, an especially dirty fossil fuel in
terms of emission of gases contributing to global climate change.

In addition to the energy-efficiency objective in the current five-year plan, a plan announced by
President Hu Jintao late last year called for China to reduce its carbon emissions per unit of
economic output by 40 to 45 percent by 2020, compared with 2005 levels. Carbon emissions are a
measurement of a country's man-made emissions of greenhouse gases like carbon dioxide.

Even if China meets its energy-efficiency goal this year and its carbon goal by 2020, its total
carbon emissions are still on track to rise steeply in the next decade, according to forecasts by
the International Energy Agency.

That is because of factors including rapid growth in the Chinese economy, growing car ownership and
rising ownership of household appliances.


\section{China Seizes On a Dark Chapter for Tibet}

\lettrine{T}{he}\mycalendar{Aug.'10}{10} white fortress loomed above the fields, a crumbling but
still imposing redoubt perched on a rock mound above a plane of golden rapeseed shimmering in the
morning light.

A battle here in 1904 changed the course of Tibetan history. A British expedition led by Sir Francis
E.~Younghusband, the imperial adventurer, seized the fort and marched to Lhasa, the capital,
becoming the first Western force to pry open Tibet and wrest commercial concessions from its senior
lamas.

The bloody invasion made the Manchu rulers of the Qing court in Beijing realize that they had to
bring Tibet under their control rather than continue to treat it as a vassal state.

So, in 1910, well after the British had departed, 2,000 Chinese soldiers occupied Lhasa. That ended
in 1913, after the disintegration of the Qing dynasty, ushering in a period of de facto independence
that many Tibetans cite as the modern basis for a sovereign Tibet.

The Chinese Communists seized Tibet again in 1951, perhaps influenced by the Qing emperor's earlier
decision to invade the mountain kingdom.

These days, Gyantse resembles other towns in central Tibet. Its dusty roads are lined with shops and
restaurants run by ethnic Han migrants, whom many Tibetans see as the most recent wave of invaders.

But Chinese officials prefer to direct the world's attention away from that and to the brutal events
at Gyantse in 1904, which conveniently fit into their master narrative for Tibetan and Chinese
history.

The Chinese government insists Tibet is an ``inalienable'' part of China, and it has appropriated
the 1904 invasion as another chapter in the long history of imperialist efforts to dismantle China
-- what the Communist education system calls the ``100 years of humiliation.''

In that Communist narrative of Gyantse, the Tibetans are a stand-in for the Chinese who were
victimized by foreign powers during the Qing dynasty.

``The local people resisted the British there,'' said Dechu, a Tibetan woman from the foreign
affairs office in Lhasa who accompanied foreign journalists on a recent official tour of Tibet.
``They put up a great resistance, so it's called the City of Heroes.''

In the late 1990s, when Britain was handing Hong Kong back to China, the Chinese government started
a propaganda campaign to highlight that theme.

A melodramatic movie about the 1904 British invasion, ``Red River Valley,'' was released in 1997. It
was a hit, and Chinese still rave about it. It was also required viewing for officials in Tibet and
for many schoolchildren.

``I've also seen a musical, two plays, another feature film and a novella on the same topic, all
from that time,'' Robert Barnett, a Tibet scholar at Columbia University, said of the late 1990s. He
said that he had not seen any reference in Tibetan literature to Gyantse as the City of Heroes
before then.

In 2004, the centenary of the British invasion, officials staged activities to commemorate it,
including a musical, ``The Bloodbath in the Red River Valley.''

Then there is the museum in the fort. A sign in English once identified it as ``the Memorial Hall of
Anti-British.'' In 1999, it displayed ``shoddy relief sculptures of battle scenes, with
unintelligible captions,'' according to Patrick French, a historian who described his visit there in
his book ``Tibet, Tibet.''

So what did happen in Gyantse in 1904?

The Younghusband expedition was sent by Lord Curzon, the viceroy of India, to force the 13th Dalai
Lama to agree to commercial concessions. Tibet had also begun to figure prominently in what was
known as the Great Game, where the British and Russian empires vied for influence in Central Asia.

British officials had heard of a Russian presence in the court of the Dalai Lama and wanted to learn
the truth. That meant getting officers to Lhasa, which had never been done before.

Colonel Younghusband was teamed with Brig. Gen.~J.~R. L.~Macdonald to lead a force from Sikkim, in
British India, across the Jelap Pass into Tibet. They crossed the border on Dec.~12, 1903, with more
than 1,000 soldiers, 2 Maxim guns and 4 artillery pieces, according to ``Trespassers on the Roof of
the World,'' a history of Western efforts to open Tibet, by Peter Hopkirk. Behind them, in the snow,
trailed 10,000 coolies, 7,000 mules, 4,000 yaks and 6 camels.

Outside the village of Guru, they encountered an encampment of 1,500 Tibetan troops. Hostilities
broke out. The British troops, which included Sikhs and Gurkhas, opened fire. In four minutes, 700
poorly armed Tibetans lay dead or dying.

Later, at Red Idol Gorge, a narrow defile just 20 miles from Gyantse, the British slaughtered
another 200 Tibetans.

The Tibetans made their final stand at the fort at Gyantse, called a dzong, or jong, in Tibetan.
After they missed a deadline to surrender on July 5, the British attacked from the southeast corner
of the fort.

A thin line of officers and soldiers clambered up the sheer rock face. ``The steepness was so great
that a man who slipped almost necessarily carried away the man below him also,'' wrote Perceval
Landon of The Times of London.

The Tibetans rained down ammunition and stones. But one lieutenant and an Indian soldier made it
through the breach, followed by others. The Tibetans fled, shimmying down two ropes.

``The surrender of the jong was to have a crushing effect on Tibetan morale,'' Mr.~Hopkirk wrote.
``There was an ancient superstition that if ever the great fortress were to fall into the hands of
an invader, then further resistance would be pointless.''

The British reached Lhasa soon afterward. Two months later, the evening before leaving Lhasa for
good, Colonel Younghusband rode out to a mountain and gazed down at the ancient city, where he
experienced a curious epiphany that inspired him to end all acts of bloodshed and found a religious
movement, the World Congress of Faiths.

``This exhilaration of the moment grew and grew till it thrilled through me with overpowering
intensity,'' he wrote in a memoir, ``India and Tibet.'' ``Never again could I think evil, or ever
again be at enmity with any man. All nature and all humanity were bathed in a rosy glowing radiancy;
and life for the future seemed naught but buoyancy and light.''

\section{Pentagon Plans Steps to Reduce Budget and Jobs}

\lettrine{D}{efense}\mycalendar{Aug.'10}{10} Secretary Robert M.~Gates said Monday that he would
close a military command, restrict the use of outside contractors and reduce the number of generals
and admirals across the armed forces as part of a broad effort to rein in Pentagon spending.

Mr.~Gates did not place a dollar figure on the total savings from the cutbacks, some of which are
likely to be challenged by members of Congress intent on retaining jobs in their states and
districts. But they appear to be Mr.~Gates's most concrete proposals to cut current spending as he
tries to fend off calls from many Democrats for even deeper budget reductions, and they reflect his
strategy of first trying to squeeze money out of the vast Pentagon bureaucracy.

While large headquarters have been combined and realigned over the years, Pentagon officials could
not recall a time when a major command was shut down and vanished off the books, even though some
jobs will probably be added elsewhere to carry on essential parts of the mission.

The White House, which is under intense political pressure to address the rapid increase in the
national debt, quickly stepped in to back Mr.~Gates, saying his plan would free money that could be
better spent on war fighting.

``The funds saved will help us sustain the current force structure and make needed investments in
modernization in a fiscally responsible way,'' President Obama said in a statement.

The potential savings Mr.~Gates outlined are likely to be relatively modest in the context of a
total Pentagon budget, including war fighting costs, projected to top \$700 billion next year. The
most significant step -- in symbol and in substance -- was his plan to close the military's Joint
Forces Command in Norfolk, Va.

The command includes about 2,800 military and civilian positions supported by 3,000 contractors at
an annual cost of \$240 million. Its responsibilities, which include managing the allocation of
global forces and running programs to press the armed services to work together on the battlefield,
will be reassigned, mostly to personnel working under the chairman of the Joint Chiefs of Staff at
the Pentagon.

Mr.~Gates also called for a 10 percent annual reduction in spending on contractors who provide
support services to the military, including money for intelligence-related contracts, and he placed
a freeze on the number of workers in the office of the secretary of defense, other Pentagon
supervisory agencies and the headquarters of the military's combat commands.

And he went after one of the military's most sacrosanct personnel structures, placing a cap on the
number of generals, admirals and senior civilian positions.

Beyond that immediate freeze, the Defense Department will move to cut at least 50 general and
admiral posts and 150 senior civilian positions over the next two years. There are now just under
1,000 general and flag officers, a growth of more than 100 since the attacks of Sept.~11, 2001.

For months, Mr.~Gates has been arguing that if Congress and the public allow the Pentagon budget to
grow by 1 percent a year, he can find 2 percent or 3 percent in savings within the department's
bureaucracy to reinvest in the military -- and that will be sufficient to meet long-term national
security needs.

``Our country is still fighting two wars, confronts ongoing terrorist threats around the globe, and
faces other major powers investing heavily in their military,'' Mr.~Gates said Monday. ``It is
important that we not repeat the mistakes of the past, where tough economic times or the winding
down of a military campaign leads to steep and unwise reductions in defense.''

But members of Congress tend to fight to protect jobs and spending in their districts, and some of
the proposed cuts -- in particular eliminating the Joint Forces Command -- are certain to earn
strong opposition.

Just minutes after Mr.~Gates announced his initiatives at a Pentagon news conference, Senator Jim
Webb, a Virginia Democrat who is on the Armed Services Committee, released a statement vowing to
``carefully examine the justifications for this decision as well as its implications for the greater
Norfolk community.''

Representative Ike Skelton, the Missouri Democrat who is chairman of the House Armed Services
Committee, said the proposals ``appear to efficiently find savings within the defense budget without
taking away resources from our war fighters.''

But the ranking Republican on the committee, Representative Howard P.~McKeon of California, said, he
was unconvinced ``that these savings will be reinvested into America's defense requirements and not
harvested by Congressional Democrats for new domestic spending and entitlement programs.''

Assessing his prospects for convincing Congress not to use its power over budgets to block these
efforts, Mr.~Gates said, ``Hard is not impossible.''

Two other Defense Department agencies will also be closed, with their functions eliminated or
reassigned. They are the office of the assistant secretary of defense for networks and information
integration, with 200 employees, and the Business Transformation Agency, with 350 people.

Pentagon spending has averaged a growth rate of 7 percent a year over the last decade, adjusted for
inflation (or nearly 12 percent a year without adjusting), including the costs of the wars.
Mr.~Obama has asked Congress for an increase in total spending next year of 2.2 percent, to \$708
billion -- 6.1 percent higher than the peak in the Bush administration.

Mr.~Gates bemoaned that ``this department is awash in taskings for reports and studies,'' and he
ordered a freeze on the number of such internal assessments. For those that remain and others
ordered by Congress, the Pentagon will publish the cost of preparation in each document.

Mr.~Gates has already canceled or trimmed several dozen weapons programs, with long-term savings,
based on projections of what the programs would have cost, predicted at \$330 billion.

Mr.~Gates has ordered the armed services and the Pentagon's agencies to find \$100 billion in
spending cuts and efficiencies over the next five years: \$7 billion for 2012, growing to \$37
billion annually by 2016.

Mr.~Gates was asked at the end of his briefing what would happen to Gen.~Ray Odierno, who is ending
a command tour in Iraq to take over the Joint Forces Command, now slated to be closed.

``I've told Ray that his assignment at JFCOM is essentially the same as his assignment in Iraq, and
that is to work himself out of a job,'' Mr.~Gates said. ``And then I'll find a new and better one
for him.''

\section{News Corp.~Sells Stakes in TV Units in China}

\lettrine{T}{he}\mycalendar{Aug.'10}{10} News Corporation said Monday that it would sell a
controlling stake in three Chinese television stations and a movie library to a private equity fund
that was formed with the backing of the Chinese government.

The sale of the units is the latest indication that the News Corporation, which is controlled by
Rupert Murdoch, is pulling back from China after years of frustration over government restrictions
that prevent foreign media organizations from freely operating in a market dominated by state-run
companies.

Disney, Time Warner, Viacom and other global media giants have also struggled to break into China's
tightly controlled television and media market and have largely been forced to distribute content
through government-run media outlets.

The News Corporation, which owns 20th Century Fox Film, the Fox network, HarperCollins and The Wall
Street Journal, said in a statement on Monday that it would sell a majority stake in the properties
to China Media Capital, a fund set up partly by the Shanghai Media Group, one of the country's
biggest state-owned media companies.

China Media Capital will take a majority stake in several of News Corporation's Star China
properties -- Xing Kong, Xing Kong International, Channel V and the Fortune Star Chinese movie
library -- and form a joint venture with the News Corporation.

According to a person with knowledge of the deal, the majority stake in the three TV properties and
the movie library was valued at close to \$150 million, a small amount for the News Corporation,
which has annual revenue of about \$32 billion.

The decision to give up control of the music and entertainment channels and the movie library, which
has about 750 Chinese language titles, came a year after the company revamped its television
operations in Asia.

Analysts said the News Corporation had been retrenching in China for some years. In 2006, for
instance, the company sold a 19.9 percent share in Phoenix Satellite TV to China Mobile, the
state-owned telecom giant, significantly reducing Star TV's stake in the venture.

Speaking of the Monday announcement, Vivek Couto, a director of Media Partners Asia, said, ``It was
a cold hard decision.'' He continued, ``They decided they want to get value for the business and the
only way to do that is to sell to a local partner.''

The joint venture is the first deal for China Media Capital, a \$732 million fund that was formed in
2009 with investments from the Shanghai Media Group and the China Development Bank.

In a statement released Monday, James Murdoch, chairman and chief executive of News Corporation
Europe and Asia, said: ``The agreement with C.M.C. recognizes the value we have created in Star
China and enables us to continue to grow it for the future.''

Analysts say the Shanghai Media Group, which was the driving force in forming the China Media
Capital fund, could help strengthen the News Corporation properties and later help China Media
Capital spin them off in a public stock offering.

Jack Gao, a vice president at News Corporation and the chief executive of Star China, will become
chief executive of the joint venture company.

\section{Skype, Going Public, Hopes for \$100 Million From First Offering}

\lettrine{S}{kype}\mycalendar{Aug.'10}{10}, the popular Internet telephone service, said Monday that
it hoped to raise \$100 million in an initial public offering of stock.

The announcement -- which came a few days after news of the initial offering by another Internet
company, Demand Media -- could help revive the market for technology offerings. Analysts have
speculated that Skype's sale, planned for the Nasdaq market, could be the biggest initial offering
in the technology sector since Google went public in 2004 and raised \$1.67 billion.

Skype, which offers free or low-cost voice and video calls over the Internet, has grown rapidly
since it was founded in 2003 by two entrepreneurs, Niklas Zennstrom and Janus Friis.

In a regulatory filing, the company, which is based in Luxembourg, said it had an average of 124
million users a month worldwide. In the first six months of this year, those users made 95 billion
minutes worth of calls, the company said. But only 8.1 million Skype users are paying customers, who
are charged for calls when they are made to conventional landlines or mobile phone numbers, rather
than to other Skype accounts.

``They are a company that has, in terms of the product, executed extremely well for a number of
years,'' said Ian Fogg, an analyst at Forrester Research. ``The challenge for Skype in the future is
how to develop new business models for video and other new technologies.''

Video calling already represents 40 percent of total calling time on Skype, and analysts say this
could grow further as the proliferation of smartphones enables more people to place such calls while
on the go.

Skype, which is mostly used by private individuals, has also been trying to move into the corporate
market, where potential customers have had security concerns.

The changes of ownership have been a distraction. Skype was acquired by eBay for \$2.6 billion in
2005. But eBay struggled to integrate Skype, and last year it sold a controlling stake to an
investor group led by the private equity firm Silver Lake, in a deal that valued Skype at \$2.7
billion.

Mr.~Zennstrom and Mr.~Friis have regained a 14 percent stake in Skype, acquired when they settled a
lawsuit against eBay and the investor group.

Unlike some technology companies that turn to the market to raise money, Skype is profitable,
reporting net income of \$13 million, on revenue of \$406 million, in the first six months.

The offering will be underwritten by Goldman Sachs, JPMorgan Chase and Morgan Stanley.

\section{Slain Aid Workers Were Bound by Their Sacrifice}

\lettrine{T}{heir}\mycalendar{Aug.'10}{10} devotion was perhaps most evident in what they gave up to
carry out their mission: Dr.~Thomas L.~Grams, 51, left a thriving dental practice; Dr.~Karen Woo,
36, walked away from a surgeon's salary; Cheryl Beckett, 32, had no time for courtship or marriage.

Most of all, the 10 medical workers massacred in northern Afghanistan last week -- six Americans,
one German, one Briton and two Afghans -- sacrificed their own safety, in a calculated gamble that
weighed the risk against the distribution of eyeglasses and toothbrushes, pain relief and prenatal
care to remote villages they reached on foot.

Dirk Frans, the executive director of the International Assistance Mission, which organized the
trip, said he had raised safety concerns about whether it was too large a group with too many
foreigners in it. But, he said, the team leader, Tom Little, with 35 years' experience in
Afghanistan, had been frustrated on previous trips because he had too few hands to treat so many,
Mr.~Frans said.

Even Charles Beckett, the father of one of the victims, defended his daughter's colleagues. ``These
are brilliant people,'' he said. ``It's not like they're na\"ive and uneducated and have some
fantasy about going on trips to help some people in dangerous areas.''

Aid groups vowed Monday to continue their work despite the attack, which one organization called
``the worst crime targeting the humanitarian community that has ever taken place in Afghanistan.''
Abed Ayoub, for instance, the chief executive officer of Islamic Relief USA, said that ``currently,
there are no immediate plans to decrease our work or staff size in the country.''

Still, the attacks fueled fears that the security situation in Afghanistan was weakening and that
the longstanding custom of allowing safe passage to aid workers was breaking down. The Taliban and
another insurgent group claimed responsibility for the attack, with the Taliban accusing the workers
of spying and trying to spread Christianity.

Violence involving foreign workers rose in 2007 and 2008, then dropped off in the past year as
conditions worsened, especially in rural areas, and aid groups became more cautious. In the first
half of 2010, 17 aid workers, foreign and Afghan, were killed, along with 19 abducted, but attacks
on nongovernmental organizations dropped 35 percent compared with the same period in 2009.

The group that was attacked was returning from a three-week mission in Nuristan that included two
veteran aid workers, Mr.~Little, 61, an optometrist and the group's leader, and Dan Terry, 64, both
of whom arrived in Afghanistan in the 1970s. Mr.~Little and his wife, Libby, raised three daughters
there.

Mr.~Little had encountered Taliban fighters on many occasions at his eye camps and other rural
missions, and told friends that he always carried a bottle of soothing saline solution in case
fighters demanded treatment for eye problems so he could relieve their discomfort, at least
temporarily.

Though many of the victims were Christians and worked for Christian organizations, friends and
family of the victims denied the Taliban's charges that they had been spies or proselytizers. ``They
try to be the hands and the feet of Jesus,'' Mr.~Beckett said, ``not the mouth of Jesus.''

Several of the workers had traveled the world on aid missions. Dr.~Grams had trekked to villages
halfway up Mount Everest, carrying dental equipment by yak, and in Afghanistan had learned to
negotiate the etiquette of the burka so he could work on the diseased teeth of women who may never
have seen a dentist.

In 2007 he gave up a thriving practice in Durango, Colo., to treat patients for whom an encounter
with a dentist meant a life-changing release from pain, said Laurie Mathews, founder of Global
Dental Relief, the Denver-based organization for which Dr.~Grams did most of his work.

Dr.~Woo, a Briton, had a similarly adventurous spirit. At 16, she trained as a contemporary dancer
and then worked as a wing-walker for a flying circus, performing stunts while strapped to the upper
wing of a biplane, dressed in a scarlet jumpsuit.

At 22, she entered medical school and eventually volunteered for missions in South Africa,
Australia, Papua New Guinea and Trinidad and Tobago. Two years ago, after visiting a friend in
Kabul, she quit her \$150,000-a-year job to move there. There, she kept pet tortoises and found time
for a fashion show to raise money for charity. She was just weeks from her wedding.

But it was her medical work that anchored her life. Her fianc\'e's mother told The Sunday Times of
London that Dr.~Woo resolved to do more to promote women's rights in Afghanistan after she treated a
14-year-old girl who had been burned after refusing to marry an older man.

In March, Dr.~Woo wrote on her blog of ``burning with frustration'' after two days conducting
medicals on Afghan men. ``They say that expat women are treated like a third race -- neither male
nor female in their eyes -- and I am getting it strongly now. I feel so very alien; in my attitude,
in my upbringing.''

Ms.~Beckett, 32, had also traveled the world, often on church-sponsored mission trips, before moving
to Afghanistan six years ago. There she worked at women's clinics, planted vegetable gardens and
tried to establish a park on the east side of Kabul, denuded after years of war. She even wrote to
employees at a national park where once was an intern, asking for advice on graduate programs in
forestry.

``As you can assume, the people I'm working with want something that can help them survive,
something they can use,'' she wrote. She was invited on the expedition to Nuristan primarily as an
interpreter because she spoke fluent Dari, Mr.~Beckett said.

Daniela Beyer, 35, was also the daughter of a minister and also spoke Dari. Pierre Grosse, chairman
of the church council in the German community of Wittgensdorf where she was a member, recalled her
as a deeply religious woman who translated textbooks into Afghan languages. ``She was a little shy,
not someone who would make a big deal about herself,'' Mr.~Grosse said.

Glen D.~Lapp, 40, of Lancaster, Pa., was a nurse who ran an eye-care program and wrote home of
``trying to be a little bit of Christ in this part of the world.'' Brian Carderelli, 25, of
Harrisonburg, Va., was an Eagle Scout and videographer who had been working in Afghanistan since
September. He posted his photographs and videos online.

The medical trek had been strenuous, said J.~D. Patton, an elder at Mr.~Carderelli's church. ``It
was just a very difficult experience, climbing these mountains, going over rain-swollen rivers,
things like that,'' Mr.~Patton said.

One of two Afghans killed, Ahmed Jawed, 24, a cook, had been excitedly considering what to do with
the \$20 a day in overtime he would earn on the trip. Mr.~Jawed was the main breadwinner for his
wife, three children and extended family, and was known in his neighborhood for the collection of
500 audiotapes he would break out for weddings or parties. The second Afghan victim, Mahram Ali, 51,
supported two disabled sons on his salary of \$150 a month.

Mr.~Jawed's brother Abdul Bagin said of the killers: ``They were infidels; not human, not Muslims.
They killed my brother without any judgment, without any trial, without talking to him.''

Mr.~Bagin saw the body in the morgue in Kabul and said there was a single bullet wound, which
forensic personnel told him was fired at close range, through the heart.


% \clearpage
% \renewcommand\listfigurename{\textit{Table of Figures}}
% {\footnotesize\textit{\listoffigures}}

\end{document}
