\documentclass[12pt]{article}
\title{Digest of The New York Times}
\author{The New York Times}

\usepackage{config}

% \makeindex
\begin{document}
\date{}
\thispagestyle{empty}

\begin{figure}
\includegraphics*[width=0.3\textwidth]{The_New_York_Times_logo.png}
\vspace{-20ex}
\end{figure}
% \renewcommand\contentsname{\textsf{Digest of The New York Times}}
\renewcommand\contentsname{}
{\footnotesize\textsf{\tableofcontents}}

\clearpage
\setcounter{page}{1}

% \begin{multicols}{2}

\pagebreak
\section{Yale Plans to Create a College in Singapore}

\lettrine{Y}{ale}\mycalendar{Sept.'10}{14} University announced on Monday that it was planning to
create a liberal arts college in Singapore that would be financed entirely by the government there
and could, in time, establish a new model for higher education in Asia.

While Yale has many international programs, it has not put its name on an overseas project the way
it envisions doing at the National University of Singapore. The new institution, to be called
Yale-N.U.S.~College, would seek to import Yale's signature residential college concept -- in which
students live, study and take classes in an intimate setting -- as well as a curriculum that
encourages critical thinking and inquiry in the humanities and sciences.

But the diplomas would lack the cachet of a full Yale degree; they would be issued by the National
University of Singapore. By contrast, New York University's ambitious new college in Abu Dhabi, also
underwritten by the government there, awards N.Y.U. diplomas.

Still, Yale would be largely responsible for hiring 100 professors to teach about 1,000 students at
the college, which is scheduled to open in 2013. If Yale decides to move forward with a full
partnership with the Singapore university, it would control half of the seats on the college's
governing board and would jointly plan the curriculum and admissions strategy.

The college is envisioned as a highly elite school within an already prestigious, yet huge and
career-focused, university. Yale officials said it would draw top students from across Asia, where
liberal arts programs are rare, and attract even more qualified Asian applicants to the New Haven
campus of Yale by raising the university's profile.

``There has never been a greater need for undergraduate education that cultivates critical
inquiry,'' the president of Yale, Richard C.~Levin, said in a statement. ``In a world that is
increasingly interconnected, the qualities of mind developed through liberal education are perhaps
more indispensable than ever in preparing students to understand and appreciate differences across
cultures and boundaries, and to address problems for which there are no easy solutions.''

In a prospectus outlining the initiative for the faculty, Dr.~Levin and the provost, Peter Salovey,
pointed to Yale's early cultivation of a liberal arts ethos in the United States, with its scholars
and graduates becoming the founders or first presidents of more than 30 colleges and universities,
including Princeton and Columbia.

``By collaborating in the development of an entirely new liberal arts curriculum for an emergent
Asia,'' the prospectus said, ``Yale could influence the course of 21st-century education as
profoundly as it influenced education in the 19th century.''

As if to allay concerns about forming a relationship with a university that is 9,500 miles and
worlds away, culturally and politically, Yale's leaders pointed out that the National University of
Singapore had recently formed joint degree programs with Duke University, the Massachusetts
Institute of Technology, N.Y.U. and others.

Still, some faculty members have bristled at the notion of joining forces with a government that
reins in speech and squelches public demonstrations.

``We've had a lot of discussion groups, and a lot of people said, `Wait a minute; are we carrying
water for restrictive, repressive regimes?' '' said Haun Saussy, a professor of comparative
literature and East Asian languages and literatures, and a co-chairman of a committee that is
helping to plan the new college's curriculum. ``It's a real concern.''

The prospectus outlined assurances that Yale had received from Singapore leaders that ``the college
upholds the principles of academic freedom and open inquiry, essential core values in higher
education of the highest caliber.''

Yale leaders said that while the university had signed a nonbinding ``memorandum of understanding''
with the Singapore institution, they would seek faculty reaction to the plan, which requires a vote
by the governing board at Yale.

\pagebreak
\section{Anti-Immigrant Party Rises in Sweden}

\lettrine{J}{immie}\mycalendar{Sept.'10}{14} Akesson, 31, looks more like an up-and-coming
advertising executive than a seasoned politician. But Mr.~Akesson, the leader of the Sweden
Democrats, does not believe in a soft sell: He wants to cut immigration by 90 percent, and he thinks
that the growth of Sweden's Muslim population is the country's biggest foreign threat since World
War II.

Sweden, which is seen by many people as a guardian of liberalism and tolerance, has never elected to
Parliament a member of any party who campaigned openly against immigration. That could change in
elections on Sunday.

Opinion polls suggest that the Sweden Democrats will exceed the 4 percent threshold needed to reach
Parliament. An alliance of center-right parties appears to have a narrow overall lead, according to
the surveys, but the Sweden Democrats could hold the balance of power, something that could create a
political crisis.

That prospect has jolted a nation in which even some of Mr.~Akesson's fiercest critics now
acknowledge that too little has been done to integrate immigrants. Political analysts also say that
the rise of the populist right shows that Sweden is being buffeted by broad political currents
familiar in other European countries.

Mainstream politicians are taking this development seriously. ``These kinds of parties, they thrive
on uncertainty and political crises,'' said Finance Minister Anders Borg, a member of the governing
Moderate Party. ``They need to create turmoil and crisis, so we will push hard to their voters: Is
this really a responsible choice?''

For most of the last century, Social Democrats dominated politics here, but in 2006 the center-right
Moderates came to power under Prime Minister Fredrik Reinfeldt. This time, the parties are standing
as competing blocs: one from the center-right, led by the Moderates, and one from the center-left,
led by the Social Democratic leader Mona Sahlin, who in running for prime minister could become the
first woman to hold that job in Sweden if she is elected.

Though sidelined from much of the official campaign, the Sweden Democrats have nonetheless attracted
attention. Their biggest coup involved a blunt 30-second advertisement that showed a white pensioner
being overtaken by a group of Muslim women in burqas as they rushed toward a line for welfare
payments.

One station refused at first to broadcast the ad, before agreeing to do so with parts obscured. The
ad generated enormous publicity and made the Sweden Democrats appear to be victims of censorship.

In a televised debate on Sunday, Mr.~Reinfeldt and Ms.~Sahlin ruled out working with the Sweden
Democrats if their coalitions did not win an absolute majority.

But Mr.~Akesson, speaking before the debate, said he thought the Moderates could find themselves in
need of his party to form a governing majority. ``I think now, if you look at the polls, it is not
impossible for the right alliance to get a full majority,'' he said. ``But if they don't, they need
us to stay in government.''

Mr.~Akesson contended that his party could win as much as 8 percent of the vote. ``We are quite
confident,'' he said. ``We are underestimated in those polls. We have grown a lot since the last
elections.''

Though the party was created in 1988, it has grown slowly, recently building strongholds in southern
Sweden in cities like Malmo and Landskrona.

The populist right has been helped by structural changes in politics, analysts say. While mainstream
parties, particularly the Social Democrats, could once rely on a strong core vote, loyalties are
fading, said Jenny Madestam, a political scientist at Stockholm University.

The collectivist, egalitarian ideas that have been associated with Sweden for decades are fading.
The debate over immigration in Sweden mirrors the debate elsewhere in Europe, where economic
pressures have exacerbated tensions over the role of Islam on the continent.

``There is a general change in Swedish society,'' Ms.~Madestam said. ``Social democratic ideas are
losing their grip on Sweden, and we are getting more and more individualistic. These collectivist
ideas are not so strong.''

Ibrahim Baylan, the national secretary of the opposition Social Democrats, who came to Sweden from
Turkey when he was 10, said the recession and unemployment were largely responsible for the rise of
the populist right.

``You find a lot of people who are young, without any job or education and without any hope of
getting a job,'' Mr.~Baylan said.

But he also said that it was harder to integrate immigrants than it once was. Many immigrants are
arriving from poor nations, and some are illiterate in their own language and therefore face extra
difficulties learning Swedish, Mr.~Baylan said.

``The opportunities are still very big in this country, but we have a situation that is totally
different,'' Mr.~Baylan said. ``The people coming here are less skilled than in the 1970s and
1980s.''

At the main mosque and Islamic center in Malmo, Beyzat Becirov, who came to Sweden from Yugoslavia
more than 40 years ago, said that most Swedes were welcoming, but that perhaps 2 to 4 percent of the
population seemed to say ``that economic problems are due to the Muslims.''

He said there had been dozens of attacks on the mosque, including a serious fire in 2003. In one
office, he pointed to a window with a bullet hole. As for Mr.~Akesson's Sweden Democrats, he said
that their support was not substantial, before adding, ``But Hitler's support started small.''

\pagebreak
\section{Microsoft Changes Policy Over Russian Crackdown}

\lettrine{M}{icrosoft}\mycalendar{Sept.'10}{14} announced sweeping changes on Monday to ensure that
the authorities in Russia and elsewhere do not use crackdowns on software piracy as an excuse to
suppress advocacy or opposition groups, effectively prohibiting its lawyers from taking part in such
cases.

The company was responding to criticism that it had supported tactics to clamp down on dissent.

The security services in Russia in recent years have seized computers from dozens of outspoken
advocacy groups and opposition newspapers, all but disabling them. Law-enforcement officials claim
that they are investigating the theft of Microsoft's intellectual property, but the searches
typically happen when those groups are seeking to draw attention to a cause or an event. Allies of
the government are rarely if ever investigated for having illegal software on their computers.

The raids have turned into a potent tool to muzzle opposition voices, and private lawyers retained
by Microsoft have often bolstered the accusations, asserting that the company was a victim and
calling for criminal charges. Until Monday, the company had rebuffed pleas from Russia's leading
human-rights organizations that it refrain from involvement in these cases, saying that it was
merely complying with Russian law.

The new Microsoft policy was announced in an apologetic statement by the company's senior vice
president and general counsel, Brad Smith, issued from its headquarters in Redmond, Wash. His
statement followed an article in The New York Times on Sunday that detailed piracy cases against
prominent advocacy groups and newspapers, including one of Russia's most influential environmental
groups.

Mr.~Smith said that Microsoft would make sure that it was no longer offering legal support to
politically motivated piracy inquiries by providing a blanket software license to advocacy groups
and media outlets. They would be automatically covered by it, without having to apply.

``We want to be clear that we unequivocally abhor any attempt to leverage intellectual property
rights to stifle political advocacy or pursue improper personal gain,'' Mr.~Smith said in a post on
the company's blog. ``We are moving swiftly to seek to remove any incentive or ability to engage in
such behavior.''

Advocates and journalists who have been targets of such raids said they were pleased that Microsoft
was announcing reforms, though some added that they remained suspicious of its intentions. The
piracy cases have stirred resentment toward Microsoft in the nonprofit sector in Russia.

In his statement, Mr.~Smith appeared to acknowledge that Microsoft needed to address the damage to
its image. He said the company would set up a program to offer legal aid to nonprofit groups and
media outlets in Russia that are caught up in software inquiries. He also said the company had
retained an international law firm to investigate its operations in the country.

With the new, blanket licenses in place, any Microsoft programs on the computers of advocacy groups
would carry Microsoft's seal of approval, making it much harder for the authorities to charge those
groups with stealing the company's software, company executives said.

The licensing plan is intended to last until 2012 but could be extended, Mr.~Smith said. The policy
could have repercussions beyond Russia because the company indicated that it would apply to other
countries as well, though it did not immediately detail which ones.

Microsoft will also step up its efforts to ensure that nonprofit groups and media outlets in Russia
have access to a company program that provides Microsoft software at little or no cost. (Mr.~Smith
said that in the past year alone, the company had donated software with a market value of more than
\$390 million to over 42,000 nonprofit groups around the world.)

The article in The Times described the case of an environmental group in Siberia, Baikal
Environmental Wave, which was raided by the police in January just as it was planning protests
against a decision by Prime Minister Vladimir V.~Putin to reopen a paper factory that had long
polluted Lake Baikal.

Plainclothes officers took 12 computers from Baikal Wave and immediately charged the group with
piracy, even though its leaders said they had only licensed Microsoft software. After the raid, the
group reached out to Microsoft's Moscow office, seeking help in defending itself.

Baikal Wave asked Microsoft to confirm that its software was legal, but the company would not,
angering the environmentalists. And Microsoft's local lawyer in Siberia offered testimony to the
police in the case on the value of the software that was said to have been stolen.

Prosecutors have not yet decided whether to bring charges against Baikal Wave.

On Monday night, Jennie Sutton, who helped found Baikal Wave two decades ago, said in a telephone
interview from Irkutsk that the shift in Microsoft policy might significantly undercut the
allegations in the group's case and any future ones. ``This is a victory,'' Ms.~Sutton said. ``If
Microsoft is against the police, then it will really look as if the cases that they are bringing are
not fair and correct. And they won't have this as an excuse to try and close us down.''

Dmitri Makarov, an organizer at the Youth Human Rights Movement, said that for months, he had been
calling on Microsoft to acknowledge that the private lawyers whom it had retained across Russia had
formed unseemly ties to the police.

He said he hoped that under the company's new policy, the lawyers would never again harass the
opposition. ``This is what we have been asking for all along,'' he said.

\pagebreak
\section{After Negotiations, Israel Emerges on Twitter}

\lettrine{I}{srael}\mycalendar{Sept.'10}{14} has acquired the user name $@$israel on Twitter, the
microblogging Internet service, from the Spanish owner of a pornographic Web site, in an unusual
transaction intended to help Israel exercise more influence over its image.

The owner of the user name, Israel Mel\'endez, got it in 2007, when Twitter was in its infancy. He
struggled to use his account, however, because every posting prompted a flood of anti-Semitic or
anti-Israel comments from Twitter users, in a case of mistaken identity.

``My account was basically unused because I was getting dozens of replies every day from people who
thought the account belonged to the state of Israel,'' Mr.~Mel\'endez said.

Yigal Palmor, a spokesman for the Israeli Foreign Ministry, confirmed the purchase. He declined to
say how much Israel paid, but he said that ``it was not pro bono.''

``We thought we could put it to better use than he did,'' Mr.~Palmor said, adding that the purchase
was in line with Israel's recent efforts to expand into social media.

Mr.~Mel\'endez, who is Spanish but lives in Miami, said that an agreement had been reached last
month in a telephone negotiation that he conducted from the Israeli consulate in Miami.
Mr.~Mel\'endez described the sale price as ``adequate.''

Israel then took over his Twitter account after first closing down the original version.

The transaction was first reported this month in Público, a Spanish newspaper.

Israel has been busy lately trying to improve its global image. Among other things, it added a
military channel on YouTube, and it used the channel to defend an assault by Israeli Navy commando
troops on a Gaza-bound aid flotilla in May.

\pagebreak
\section{In Iraq, Clearer Image of U.S.~Support}

\lettrine{A}{merican}\mycalendar{Sept.'10}{14} forces provided extensive support to the Iraqi
military in a recent operation north of Baghdad, illustrating the risks that United States troops
still endure in their new advisory role there.

During two days of combat in Diyala Province, American troops were armed with mortars, machine guns
and sniper rifles. Apache and Kiowa helicopters attacked insurgents with cannon and machine-gun
fire, and F-16's dropped 500-pound bombs.

One American soldier was slightly wounded during the operation, which at times put United States
forces within the range of insurgents' hand grenades in an area thick with trenches and palm groves.

``This operation demonstrates the importance and dangers of the mission in Iraq during Operation New
Dawn,'' Col.~Malcolm B.~Frost, the commander of the Second Advise and Assist Brigade, said in an
e-mail response to a reporter's questions.

``The elements in close ground combat were about 25 U.S.~assisting and advising approximately 200
I.A. and I.P.,'' the colonel added, using the acronyms for the Iraqi Army and Iraqi police.

The Obama administration declared an end to the United States combat mission in Iraq on Sept.~1, a
step intended to underscore the Iraqis' increasing role in providing for their own security, but
which was also aimed at reassuring Americans that President Obama was keeping his promise to remove
forces from Iraq.

The American mission was renamed ``Operation New Dawn,'' and American officials stressed that the
main American role would now be to advise Iraqi troops and escort civilian American advisers.

But the six United States advisory brigades that remain retain all the weapons and forces of a
combat brigade. Their rules of engagement allow them to defend themselves if they come under attack
and to come to the aid of Iraqi forces -- which is what occurred in recent days.

The latest round of fighting began when nearly 600 Iraqi soldiers and police mounted an operation in
Diyala Province, a strategically vital area rife with sectarian tensions. A major purpose of the
operation was to uncover arms caches and detain people from Sunni insurgent groups like Al Qaeda in
Mesopotamia and Jaish al-Islami. The town of Al Hadid was one objective in a five-day campaign that
covered 21 towns and cities.

Colonel Frost's unit -- a Stryker combat brigade from the 25th Infantry Division in Hawaii that was
converted to an advisory unit -- was charged with helping the Iraqis, who began the operation with
their own forces. It turned out to be the biggest Iraqi operation that the American brigade has
supported since it arrived in July.

While searching for an arms cache near Al Hadid, Iraqi forces came under attack from hand grenades
and machine guns in an area laced with trenches. Late on Sept.~11, Iraqis asked for American help
that very evening, and the Americans responded.

All told, 49 American soldiers were on the ground in support, including two Stryker platoons and a
10-person Special Forces detachment. Apache helicopters responded with 30-millimeter cannon fire and
Kiowa helicopters fired their machine guns. F-16's dropped bombs. Iraqi helicopters and Iraqi
armored personnel carriers were also involved in the operation.

``This was a grinding, slow, close combat fight,'' Colonel Frost wrote, also referring to the Iraqi
security forces, or I.S.F. ``Over the course of two days, U.S.~forces advised I.S.F. leaders, and
through air and ground, supported them as the I.S.F. conducted several attacks against a determined
and well-armed enemy dug into a web of trenches in this thick palm grove area.''

The colonel added, ``U.S.~soldiers were right there the whole time, advising and assisting the ISF
every step of the way.''

\pagebreak
\section{For the Bad News Bulls, Adversity Is Opportunity }

\lettrine{O}{n}\mycalendar{Sept.'10}{14} jobs and the stagnating United States economy is shrouding
an immutable fact: the recovery is at hand -- you just can't see it yet.

At a time when fear-stricken hedge funds are stocking up on gold, retail investors are fleeing the
stock market and gloom-peddling economists like Nouriel Roubini are commanding the airwaves, making
a bullish case for stocks can be lonely, dispiriting work.

But therein lies the opportunity for investors like John A.~Paulson, the hedge fund executive who
made billions by betting on a housing crash, and Bill Miller, the mutual fund manager at Legg Mason
Capital Management who is best known for his 15-year streak of beating the Standard \& Poor's
500-stock index.

Mr.~Paulson has big stakes in companies like Bank of America and JPMorgan Chase as well as in the
French automaker Renault. Mr.~Miller is betting on I.B.M. and Citigroup, and says that stock in
large American companies has not been this cheap, compared with bonds, since 1951.

``The corporate earnings of both European exporters and U.S.~companies have exceeded analyst
estimates,'' Mr.~Paulson wrote in a letter to investors this summer. ``The S.\&P.~500 now trades at
only 13.8 times 2010 estimates, well below the 30-year average of 19.5.''

So far, the returns have been hard in coming.

Mr.~Paulson's Recovery fund was down 9 percent in August, and over the last three years Mr.~Miller's
main fund is down 16.5 percent, compared with a 10 percent decline in the S.\& P.~500.

For many analysts, the idea that one should ignore the chief economic indicators is absurd --
especially when some suggest that the world's largest economy may be heading into another slump.

``To steal a march on the market, one should follow the leading indicators closely,'' Albert
Edwards, a well-known bearish strategist at Soci\'et\'e G\'en\'erale, wrote in a note last week. He
is forecasting a 50 percent retreat for the S.\& P.~500. ``These are variously pointing either to a
hard landing or, at best, a decisive slowdown.''

The bad news bulls do not dispute that the relentless beat of lost jobs and sagging house sales
represents a serious economic threat.

But they argue that in many respects this avalanche of bad news has little bearing for a growing
number of corporations that are making money by relying more on technology, borrowing at rock-bottom
rates and increasing sales to galloping overseas markets in China, Brazil and India.

``Yes, the job numbers are frightening,'' said Michael Hintze, the founder of CQS Management, a
fixed-income hedge fund based in London. ``But that does not affect a company like Glaxo, which
sells a value-added product into markets like China, the U.K. and Indonesia.''

No doubt, the indicators of fear remain powerful.

They include the persistence of net short positions on the broad indexes, meaning that investors
have more money riding on a market fall than on a market rise, and, despite low interest rates, the
still relatively high cost of financing for banks, which curtails lending. A rising number of large
Wall Street banks says the likelihood of a second recession is increasing, and most significant, the
notion is still widely held that house prices in the United States will not soon recover.

Mr.~Hintze, who manages about \$6.8 billion, refers to what is commonly known in financial circles
as an ``upcrash,'' in which a market consumed with negativity finally realizes that things are not
so bad. Investors begin to climb the wall of fear, producing a sharp, and in some cases,
long-lasting rally.

What Mr.~Hintze and other like-minded investors are betting on is that in spite of the combination
of high sovereign debt and unemployment figures, global companies with worldwide brands and limited
leverage are in a position to drive a longer, more sustained stock market boom.

But for that to happen, they say, investors must break free of a tendency to either stay on the
sidelines or to continue selling whenever there is a bad report on jobs or housing.

``Apple does not depend on U.S.~housing starts, nor does BMW,'' said David F.~Marvin, chairman of
Marvin \& Palmer Associates, an institutional global equity fund based in Wilmington, Del.

Since 2007, Mr.~Marvin's assets under management have shrunk to \$4.6 billion, from \$12 billion,
and he concedes that it has been hard to persuade clients to take risks, given what has happened
since 2008.

``When you are burned that badly you get cautious,'' he said. ``But there is a real boom going on
now in emerging markets from China to Korea and India, where incomes are rising and there is demand
for the upscale products that LVMH and Mercedes produce.''

In a report on the United States economy published in July, the director of economic research for
the Milken Institute, Ross C.~DeVol, pointed out that in the second quarter, corporate investment in
equipment and software increased at an annualized rate of 24.9 percent as companies made use of
available cash and easy borrowing terms to become less reliant on human workers.

``It's a bit of a disconnect because this will retard job growth,'' Mr.~DeVol said, ``but companies
are becoming more efficient.''

Mr.~DeVol is forecasting United States unemployment of about 7.7 percent at the end of 2012, a high
figure for the country but one that he argues reflects the fact that many people who lost their jobs
in the recession will not be qualified to re-enter the work force.

And therein lies the rub. Is the rise of a global corporate elite, increasingly dependent on markets
like China, India, Turkey and Brazil, enough to generate a broad-based stock market recovery even as
house prices sink and good jobs remain increasingly difficult to find in the United States and
Europe?

Hedge funds, generally seen to be the most courageous of investors, remain for the most part on the
sidelines. Prime brokers say that many of the big funds, hurt by poor performance this year, are
largely unwilling to take on significant leverage and make a bet on stocks.

All of which leaves the bad news bulls in a distinct minority, rarely seen and not often heard --
for now at least.

``The prevailing view is definitely negative,'' said Byron R.~Wien, global strategist for the
Blackstone Group, adding that he had begun to observe a tendency for investors to take on more risk.
``But that provides an opportunity for optimists.''

Crucial Work Remains in Rewriting Bank Regulations, Officials Say By JACK EWING BASEL, Switzerland
-- Stock markets cheered new regulations announced this weekend that were intended to prevent a
recurrence of the financial crisis, but central bankers cautioned Monday that officials still must
forge agreements to limit short-term bank risk and deal with institutions considered too big too
fail.

``We have hard work to do still,'' Jean-Claude Trichet, president of the European Central Bank, said
during a news conference in Basel, where central bankers and bank regulators from 27 countries
agreed Sunday to require banks to more than triple the amount of capital they held in reserve.

``It's a work in progress on a large front,'' said Mr.~Trichet, who was chairman of the Basel group.

The group endorsed a plan to require banks to raise the amount of common equity they held,
considered the least risky form of capital, to 7 percent of assets, from 2 percent. That requirement
is the centerpiece of a host of new rules, most of which will be phased in through 2018, that are
aimed at increasing banks' ability to absorb market shocks.

But the authorities plan to develop additional rules that will apply to large, cross-border banks
that can rock financial markets when they get in trouble -- as happened when Lehman Brothers failed
in September 2008. The investment bank's collapse was instrumental in precipitating a global
financial crisis that required billions in government bailouts.

``These institutions are still too big and interconnected to fail,'' said Mario Draghi, governor of
the Bank of Italy and chairman of an international panel that is working with Mr.~Trichet's group to
determine how best to reduce risk to the financial system.

Mr.~Draghi said that regulators needed to improve their capacity to ``resolve the systemically
important institutions without creating huge market disruptions and without dipping into the
taxpayer purse.''

He said regulators needed to deal with the problem known as moral hazard, in which large
institutions are tempted to take on too much risk because their executives believe that governments
will always bail them out.

``The systemically important institutions will need enhanced supervision -- supervision which is
broader, more effective and more intrusive,'' Mr.~Draghi said. ``The stakes are way higher than with
a normal small or medium-sized bank.''

The Basel Committee on Banking Supervision, whose recommendations were endorsed Sunday by the
central bankers and regulators, is also working on new rules intended to ensure that banks always
have enough cash on hand to survive periods of market turmoil.

After Lehman's failure, lending among banks seized up. Banks like Hypo Real Estate in Munich were
unable to borrow the cash they needed for daily operations and did not have enough reserves to
survive without taxpayer bailouts. German officials said Friday that they would supply an additional
40 billion euros in government guarantees for Hypo Real Estate, bringing the total to 142 billion
euros, or \$183 billion.

Still, shares of banking companies rose Monday as investors welcomed the agreement in Basel and
expressed relief that banks would have plenty of time to adjust to the new rules. Investors may also
have simply been relieved that the agreement provided more certainty about the shape of future
regulation.

Analysts at Goldman Sachs calculated that only four of the 47 large publicly listed banks in Europe
would fall below the new reserve targets in 2012. They are ATEbank in Greece and three Italian
banks: Banco Popolare, Credito Valtellinese and Banca Monte dei Paschi di Siena.

But the analysts did not evaluate Germany's savings banks and state-controlled landesbanks, Spanish
thrift institutions and other banks that do not have publicly traded shares.

In addition, many banks may face investor pressure to raise their reserves well above the regulatory
minimums. Analysts at Credit Suisse forecast Monday that 8 percent would become the working minimum
of common equity, from an investor point of view, with 10 percent regarded as a comfortable level.

Analysts at Nomura Equity Research said they expected only a few weaker banks, like Cr\'edit
Agricole in France and Commerzbank in Germany, to take steps to raise additional capital.

Still, some banking groups continued to insist that the rules were too onerous and would throttle
lending.

``We see a danger that German banks' ability to issue credit could be significantly curtailed,''
Karl-Heinz Boos, president of the Association of German Public Sector Banks, said Monday.

Mr.~Boos said that lending to midsize businesses could suffer because the new rules would no longer
allow the landesbanks to consider nonvoting shares as a form of common equity.

Asked about complaints from the banking industry, Mr.~Draghi said, ``I think this agreement has been
welcomed by the markets.''

``By making the system more resilient,'' he said, the rules ``will support a sustained recovery.''

3-D Printing Spurs a Manufacturing Revolution By ASHLEE VANCE SAN FRANCISCO -- Businesses in the
South Park district of San Francisco generally sell either Web technology or sandwiches and
burritos. Bespoke Innovations plans to sell designer body parts.

The company is using advances in a technology known as 3-D printing to create prosthetic limb
casings wrapped in embroidered leather, shimmering metal or whatever else someone might want.

Scott Summit, a co-founder of Bespoke, and his partner, an orthopedic surgeon, are set to open a
studio this fall where they will sell the limb coverings and experiment with printing entire
customized limbs that could cost a tenth of comparable artificial limbs made using traditional
methods. And they will be dishwasher-safe, too.

``I wanted to create a leg that had a level of humanity,'' Mr.~Summit said. ``It's unfortunate that
people have had a product that's such a major part of their lives that was so underdesigned.''

A 3-D printer, which has nothing to do with paper printers, creates an object by stacking one layer
of material -- typically plastic or metal -- on top of another, much the same way a pastry chef
makes baklava with sheets of phyllo dough.

The technology has been radically transformed from its origins as a tool used by manufacturers and
designers to build prototypes.

These days it is giving rise to a string of never-before-possible businesses that are selling iPhone
cases, lamps, doorknobs, jewelry, handbags, perfume bottles, clothing and architectural models. And
while some wonder how successfully the technology will make the transition from manufacturing
applications to producing consumer goods, its use is exploding.

A California start-up is even working on building houses. Its printer, which would fit on a
tractor-trailer, would use patterns delivered by computer, squirt out layers of special concrete and
build entire walls that could be connected to form the basis of a house.

It is manufacturing with a mouse click instead of hammers, nails and, well, workers. Advocates of
the technology say that by doing away with manual labor, 3-D printing could revamp the economics of
manufacturing and revive American industry as creativity and ingenuity replace labor costs as the
main concern around a variety of goods.

``There is nothing to be gained by going overseas except for higher shipping charges,'' Mr.~Summit
said.

A wealth of design software programs, from free applications to the more sophisticated offerings of
companies including Alibre and Autodesk, allows a person to concoct a product at home, then send the
design to a company like Shapeways, which will print it and mail it back.

``We are enabling a class of ordinary people to take their ideas and turn those into physical, real
products,'' said J.~Paul Grayson, Alibre's chief executive. Mr.~Grayson said his customers had
designed parts for antique cars, yo-yos and even pieces for DNA analysis machines.

``We have a lot of individuals going from personal to commercial,'' Mr.~Grayson said.

Manufacturers and designers have used 3-D printing technology for years, experimenting on the spot
rather than sending off designs to be built elsewhere, usually in Asia, and then waiting for a model
to return. Boeing, for example, might use the technique to make and test air-duct shapes before
committing to a final design.

Depending on the type of job at hand, a typical 3-D printer can cost from \$10,000 to more than
\$100,000. Stratasys and 3D Systems are among the industry leaders. And MakerBot Industries sells a
hobbyist kit for under \$1,000.

Moving the technology beyond manufacturing does pose challenges. Customized products, for example,
may be more expensive than mass-produced ones, and take longer to make. And the concept may seem out
of place in a world trained to appreciate the merits of mass consumption.

But as 3-D printing machines have improved and fallen in cost along with the materials used to make
products, new businesses have cropped up.

Freedom of Creation, based in Amsterdam, designs and prints exotic furniture and other fixtures for
hotels and restaurants. It also makes iPhone cases for Apple, eye cream bottles for L'Oreal and
jewelry and handbags for sale on its Web site.

Various designers have turned to the company for clothing that interlaces plastic to create
form-hugging blouses, while others have requested spiky coverings for lights that look as if they
could be the offspring of a sea urchin and a lamp shade.

``The aim was always to bring this to consumers instead of keeping it a secret at NASA and big
manufacturers,'' said Janne Kyttanen, 36, who founded Freedom of Creation about 10 years ago.
``Everyone thought I was a lunatic when we started.''

His company can take risks with ``out there'' designs since it doesn't need to print an object until
it is ordered, Mr.~Kyttanen said. Ikea can worry about mass appeal.

LGM, based in Minturn, Colo., uses a 3-D printing machine to create models of buildings and resorts
for architectural firms.

``We used to take two months to build \$100,000 models,'' said Charles Overy, the founder of LGM.
``Well, that type of work is gone because developers aren't putting up that type of money anymore.''

Now, he said, he is building \$2,000 models using an architect's design and homegrown software for a
3-D printer. He can turn around a model in one night.

Next, the company plans to design and print doorknobs and other fixtures for buildings, creating
unique items. ``We are moving from handcraft to digital craft,'' Mr.~Overy said.

But Contour Crafting, based in Los Angeles, has pushed 3-D printing technology to its limits.

Based on research done by Dr.~Behrokh Khoshnevis, an engineering professor at the University of
Southern California, Contour Crafting has created a giant 3-D printing device for building houses.
The start-up company is seeking money to commercialize a machine capable of building an entire house
in one go using a machine that fits on the back of a tractor-trailer.

The 3-D printing wave has caught the attention of some of the world's biggest technology companies.
Hewlett-Packard, the largest paper-printer maker, has started reselling 3-D printing machines made
by Stratasys. And Google uses the CADspan software from LGM to help people using its SketchUp design
software turn their creations into 3-D printable objects.

At Bespoke, Mr.~Summit has built a scanning contraption to examine limbs using a camera. After the
scan, a detailed image is transmitted to a computer, and Mr.~Summit can begin sculpting his limb
art.

He uses a 3-D printer to create plastic shells that fit around the prosthetic limbs, and then wraps
the shells in any flexible material the customer desires, be it an old bomber jacket or a trusty
boot.

``We can do a midcentury modern or a Harley aesthetic if that's what someone wants,'' Mr.~Summit
said. ``If we can get to flexible wood, I am totally going to cut my own leg off.''

Mr.~Summit and his partner, Kenneth B.~Trauner, the orthopedic surgeon, have built some test models
of full legs that have sophisticated features like body symmetry, locking knees and flexing ankles.
One artistic design is metal-plated in some areas and leather-wrapped in others.

``It costs \$5,000 to \$6,000 to print one of these legs, and it has features that aren't even found
in legs that cost \$60,000 today,'' Mr.~Summit said.

``We want the people to have input and pick out their options,'' he added. ``It's about going from
the Model T to something like a Mini that has 10 million permutations.''

\pagebreak
\section{For Many, `Washroom' Seems to Be Just a Name}

\lettrine{T}{he}\mycalendar{Sept.'10}{14} next time a man hands you a hot dog after making a run to
the restroom and the concession stand at an Atlanta Braves baseball game, be careful: there is a
good chance he did not wash his hands, according to a report released Monday by a group that sends
spies into public restrooms in the name of science.

Only about two-thirds of the men observed washed their hands after using the restroom at Turner
Field -- the lowest rate for any of the locations cited in the observational study and survey on the
hand-washing habits of Americans. The study, conducted every few years, was released by the American
Society for Microbiology and the American Cleaning Institute at a microbiology conference in Boston.

Some subjects were asked about their washing habits in telephone interviews; others were watched by
undercover observers in public restrooms. Some of what the sink spectators witnessed was, well,
filthy. Consider: 20 percent of people using the restrooms at Pennsylvania Station and Grand Central
Terminal in New York did not wash their hands.

The researchers, from Harris Interactive, stood in restrooms while pretending to fix their hair or
put on makeup, said Brian Sansoni, a spokesman for the American Cleaning Institute, a trade group
for producers of cleaning products. ``After they took care of business, the observer checked whether
or not they actually washed their hands,'' Mr.~Sansoni said.

Women tended to be more responsible hand-washers than men -- and female Braves fans were no
exception: 98 percent of women observed at Turner Field exercised proper hygiene before exiting the
restroom.

The restroom observers reported that 85 percent of men and women observed at public places in
Atlanta, Chicago, New York and San Francisco washed their hands after using a public bathroom.
(Curiously, in the telephone survey, 96 percent of people said they always washed their hands after
using a public bathroom.)

The hand-washing rate dips to 89 percent for those using the facilities at home, according to the
phone survey.

People using public restrooms in Chicago and San Francisco were the most frequent hand-washers,
according to the observations, with 89 percent of adults washing before exiting.

The overall numbers are the highest reported since the study started in 1996, said Barbara Hyde, a
spokeswoman for the American Society for Microbiology. The threat of the H1N1 flu over the past year
drove home the importance of hand washing, she said.

``The message is getting out there, and I think people are responding,'' Ms.~Hyde said. ``We've
lived through a pandemic flu, and that is in part responsible for the change in behavior.''

As for sports fans, Mr.~Sansoni said, they ``might be in a rush to get back to the game.'' He noted
that the percentage of men who soaped up at Turner Field, although the lowest in the report, showed
an improvement since the last survey, in 2007.

``But if you're passing people's hot dogs down the aisle, you kind of hope that hand-washing
behavior would increase,'' Mr.~Sansoni said.

\pagebreak
\section{Chinese Remake the `Made in Italy' Fashion Label}

\lettrine{O}{ver}\mycalendar{Sept.'10}{14} the years, Italy learned the difficult lesson that it
could no longer compete with China on price. And so, its business class dreamed, Italy would sell
quality, not quantity. For centuries, this walled medieval city just outside of Florence has
produced some of the world's finest fabrics, becoming a powerhouse for ``Made in Italy'' chic.

And then, China came here.

Chinese laborers, first a few immigrants, then tens of thousands, began settling in Prato in the
late 1980s. They transformed the textile hub into a low-end garment manufacturing capital --
enriching many, stoking resentment and prompting recent crackdowns that in turn have brought cries
of bigotry and hypocrisy.

The city is now home to the largest concentration of Chinese in Europe -- some legal, many more not.
Here in the heart of Tuscany, Chinese laborers work round the clock in some 3,200 businesses making
low-end clothes, shoes and accessories, often with materials imported from China, for sale at
midprice and low-end retailers worldwide.

It is a ``Made in Italy'' problem: Enabled by Italy's weak institutions and high tolerance for
rule-bending, the Chinese have blurred the line between ``Made in China'' and ``Made in Italy,''
undermining Italy's cachet and ability to market its goods exclusively as high end.

Part of the resentment is cultural: The city's classic Italian feel is giving way to that of a
Chinatown, with signs in Italian and Chinese, and groceries that sell food imported from China.

But what seems to gall some Italians most is that the Chinese are beating them at their own game --
tax evasion and brilliant ways of navigating Italy's notoriously complex bureaucracy -- and have
created a thriving, if largely underground, new sector while many Prato businesses have gone under.
The result is a toxic combination of residual fears about immigration and the economy.

``This could be the future of Italy,'' said Edoardo Nesi, the culture commissioner of Prato
Province. ``Italy should pay attention to the risks.''

The situation has steadily grown beyond the control of state tax and immigration authorities.
According to the Bank of Italy, Chinese individuals in Prato channel an estimated \$1.5 million a
day to China, mainly earnings from the garment and textile trade. Profits of that magnitude are not
showing up in tax records, and some local officials say the Chinese prefer to repatriate their
profits rather than invest locally.

The authorities also say that Chinese and probably Italian organized crime is on the rise, involving
not only illegal fabric imports, but also human trafficking, prostitution, gambling and money
laundering.

The rest of Italy is watching closely. ``Lots of businesses from Emilia Romagna, Puglia and the
Veneto say, `We don't want to wind up like Prato,' '' said Silvia Pieraccini, the author of ``The
Chinese Siege,'' a book about the rise of the ``pronto moda'' or ``fast fashion'' economy.

Tensions have been running high since the Italian authorities stepped up raids this spring on
workshops that use illegal labor, and grew even more when Italian prosecutors arrested 24 people and
investigated 100 businesses in the Prato area in late June. The charges included money laundering,
prostitution, counterfeiting and classifying foreign-made products as ``Made in Italy.''

Yet many Chinese in Prato are offended at the idea that they have ruined the city. Instead, some
argue, they have helped rescue Prato from total economic irrelevance, another way of saying that if
the Italian state fails to innovate and modernize the economy, somebody else just might.

``If the Chinese hadn't gone to Prato, would there be pronto moda?'' asked Matteo Wong, 30, who was
born in China and raised in Prato and runs a consulting office for Chinese immigrants. ``Did the
Chinese take jobs away from Italians? If anything, they brought lots of jobs to Italians.''

In recent months, Prato has become a diplomatic point of contention. Italian officials say the
Chinese government has not done enough so far to address the issue of illegal immigrants, and they
are seeking a bilateral accord with China to identify and deport them. Some Prato residents suspect
that the flood of immigrants is part of a strategy by Beijing to exploit the Italian market, though
the Chinese government does not generally use illegal migrants to carry out its overseas development
plans.

Italian officials say Prato is expected to be on the agenda when Prime Minister Wen Jiabao of China
visits Rome in October.

\textsf{China in Italy's Backyard}

According to the Prato chamber of commerce, the number of Italian-owned textile businesses
registered in Prato has dropped in half since 2001 to just below 3,000, 200 fewer than those now
owned by Chinese, almost all in the garment sector. Once a major fabric producer and exporter, Prato
now accounts for 27 percent of Italy's fabric imports from China.

Resentment runs high. ``You take someone from Prato with two unemployed kids and when a Chinese
person drives by in a Porsche Cayenne or a Mercedes bought with money earned from illegally
exploiting immigrant workers, and this climate is risky,'' said Domenico Savi, Prato's chief of
police until June.

According to the Prato mayor's office, there are 11,500 legal Chinese immigrants, out of Prato's
total population of 187,000. But the office estimates the city has an additional 25,000 illegal
immigrants, a majority of them Chinese.

With its bureaucracy, protectionist policies and organized crime, Italy is arguably Western Europe's
least business-friendly country. Yet in Prato, the Chinese have managed to create an entirely new
economy from scratch in a matter of years.

A common technique used, often with the aid of knowledgeable Italian tax consultants and lawyers, is
to open a business, close it before the tax police can catch up, then reopen the same workspace with
a new tax code number.

``The Chinese are very clever. They're not like other immigrants, who can be pretty thick,'' said
Riccardo Marini, a textile manufacturer and the head of the Prato branch of Confindustria, the
Italian industrialists' organization.

``The difficulty,'' he added ruefully, ``is in finding a shared understanding of the rules of the
game.''

Prato's streets have slowly become more and more Chinese, as the Chinese have bought out
Italian-owned shops and apartments, often paying in cash. Public schools are increasingly filled
with Chinese pupils.

Hypocrisy abounds. ``The people in Prato are ostriches,'' said Patrizia Bardazzi, who with her
husband has run a high-end clothing shop in downtown Prato for 40 years. ``I know people who rent
space to the Chinese and then say, `I don't come into the center because there are too many
Chinese.' They rent out the space and take the money and go to Forte dei Marmi,'' she added,
referring to the Tuscan resort town.

A short walk past the city's medieval walls, past the cathedral with Filippo Lippi's Renaissance
frescoes, lies Via Pistoiese, the heart of Prato's Chinatown. Here, shop signs in Chinese and
Italian advertise wedding photography, hardware, electronics and gambling parlors.

Outside a supermarket selling foodstuffs imported from China, an electronic job board flashes a
running ticker of garment-industry jobs.

The work -- long hours at sewing machines -- takes place in back-room workshops with makeshift
sleeping quarters. The heart of the ``fast fashion'' sector is an industrial area on the outskirts
of town, Macrolotto, filled with Chinese fashion wholesalers.

Here, vans from across Europe line the parking lots as retailers buy ``Made in Italy'' clothing to
resell back home at a huge markup. By buying in relatively small quantities and taking advantage of
the fluid borders of the European Union, most manage to avoid paying import tariffs.

On a recent afternoon, a couple from Montenegro loaded racks of cotton summer dresses into boxes in
the back of their van. The wife wielded a label gun, tagging each dress ``Made in Italy.''

Just blocks away, Li Zhang, who immigrated to Italy in 1991 from Wenzhou, a city in southeastern
China known for its global network of entrepreneurs, explained how his clothing company, Luma,
produced on-demand fashion.

He showed off bolts of fabric, which he said he bought locally or in India or China. He often buys
white fabric and has it dyed and cut by other Chinese companies in Prato before giving the pieces to
subcontractors to produce the requested items -- 1,000 green skirts, in a typical example -- in a
matter of weeks, if not days.

Mr.~Zhang and hundreds of other Chinese like him are at the center of Prato's so-called gray
economy, whose businesses are partly above board in that they pay taxes, and partly underground, in
that they rely on subcontractors who often use illegal labor. (Asked if his subcontractors used
illegal labor, Mr.~Zhang laughed and said, ``You'd have to ask the subcontractors.'')

Since founding Luma in 1998, Mr.~Zhang said, he has exported clothes to 30 countries, including
China, Mexico, Venezuela, Jordan and Lebanon. He said that his biggest order was for the Italian
retailer Piazza Italia, but that he had also sold to wholesalers who said they had sold to Zara,
Mango, Top Shop and Guess, European retailers specializing in bargain chic.

The raids, he said, are hindering business, unsettling the local Chinese community to the point that
many workers had gone into hiding.

``People are afraid,'' Mr.~Zhang said. ``This was a political decision. At first, they left us too
free. Now they are tightening things too much.''

\textsf{The New Sheriff in Town}

Much of the tightening comes from Prato's new administration. In 2009, the traditionally left-wing
city elected its first right-wing mayor in the postwar era, whose winning campaign tapped into
powerful local fears of a ``Chinese invasion,'' and who seeks a broader European Union response to
Chinese immigration.

``How can China leave a mark like this in the E.U.?'' the mayor, Roberto Cenni, asked. ``Noise, bad
habits, prostitution. People can't live anymore. They're sick of it.''

An elegant man in a well-cut gray suit, Mr.~Cenni is a former president and a current shareholder of
Go-Fin, a Prato holding company that is behind several midrange Italian fashion companies. At least
one of these, Sasch, has moved much of its production to China within the last 10 years.

Powerless to reverse the broader economic currents, the mayor has instead focused on small
initiatives, including new rules that prohibit drying fish on balconies and require all Prato
shopkeepers to speak Italian. These have won him praise from some local people, but also criticism
for bigotry.

The mayor has also stepped up raids on Chinese businesses. Critics say they are little more than
media spectacles, but local Chinese have seen them as unwarranted attacks.

On a rainy recent morning, a team of police officers, tax collectors and other state officers
swooped in on two Chinese workshops in a residential and industrial area just outside Prato's
downtown.

Tucked behind apartment houses, the garage-like space was filled with rows of sewing machines, with
white fabric strewn about and lace shirts lying unfinished on the concrete floor.

The police rounded up the workers in the courtyard. A woman in plastic flip-flops carried a black
bucket filled with urine downstairs, accompanied by a young boy wearing only underwear.
``Pantaloni,'' she told the officers in broken Italian, ``Pants.'' ``O.K., let him put on pants,''
an amenable officer agreed with a shrug.

Next door, the police brought some Chinese workers in a small, windowless bedroom to be identified.
A woman in a blue T-shirt sat on the bed and sobbed uncontrollably.

The officials sorted through paperwork. ``This is the last name, right?'' one asked an interpreter.

Between the two workspaces stood a little house with hydrangeas in the yard. The Italian couple in
the doorway did not want to reveal their names.

``It's an ant colony,'' the man said. ``Who knows how many? They closed the door and covered up the
windows.''

His cautious wife tugged on his arm. ``You can't get into these discussions,'' she said, drawing him
back inside.

Soon an owner of the workspace came in from his home down the block. Paolo Bonaiuti, 73, a tall man
with white hair, blue eyes and a look of unflappability, waved his lease, showing that he rented out
the space for \$2,220 a month. To judge from their expressions, the police officers did not look as
if they believed it.

\textsf{Italy's Immigration Woes}

But crackdowns like these can only do so much. In the first half of this year, the authorities
raided 154 Chinese-owned businesses -- out of more than 3,000. To do the job, ``We'd need an army of
people,'' said Lina Iervasi, the head of the Prato Police Department's immigration office.

Earlier this year, several officers in that office were arrested on charges that they took bribes in
exchange for granting residence permits.

``We don't go on the hunt for the illegal immigrants. We're not so crazy as to do that,'' said
Mr.~Savi, the former police chief. ``But we seek a balance between norms and reality.''

That balance has been strikingly hard to find.

Many illegal Chinese immigrants arrive by bus from Russia or the Balkans, and either destroy their
passports or give them away to the organized crime groups that help bring them. Many others overstay
their tourist visas.

``Italy has a 20th-century immigration law; it tends to think of immigrants as a phenomenon linked
to work, in which people move from poor countries to rich ones,'' said Andrea Frattani, a former
social welfare commissioner in Prato's previous center-left government.

Instead, he argued, what Italy is witnessing in Prato is ``a precise strategy'' on the part of the
Chinese government to create an economic foothold in Europe.

Asked at a recent public appearance if that was the case, China's ambassador to Italy, Ding Wei,
said only that Prato had been a central issue in his portfolio since he arrived in the spring, and
that he had sent advisers to investigate.

``I've been very attentive to resolving the question of Prato, which is unique and particular,'' he
said in late July. ``It should not have an impact on the cooperation between our countries.''

Italians in Prato are feeling less sanguine. ``At 20, I was sure the world was mine,'' said
Mr.~Nesi, 45, the culture commissioner and a writer whose family sold its three-generation, high-end
textile business in 2004.

``It's hard to accept that all this happened in a short time,'' he said, bewildered. ``It makes us
feel old and without hope.''

The problems will not be resolved easily. ``There's no plan,'' said Xu Qiu Lin, a local entrepreneur
and the only Chinese member of Confindustria in Prato, echoing a widespread sentiment. ``There's no
plan; that's the problem.''

\pagebreak
\section{N. Korea Suggests Family Reunions}

\lettrine{I}{n}\mycalendar{Sept.'10}{14} a surprise gesture of reconciliation that could ease
tensions on the divided Korean Peninsula, North Korea proposed to South Korea on Saturday that they
arrange reunions of families separated by a war six decades ago.

Family reunions have usually been proposed by the South, not the North. The North's initiative
showed how the humanitarian program is used as a chip in the complex diplomacy surrounding the
peninsula.

But the South has refused to resume aid shipments or six-nation talks on ending North Korea's
nuclear weapons development before North Korea apologizes for the sinking of a South Korean warship
in March, which killed 46 sailors. The North has denied involvement in the sinking, but the South
has insisted on the North making gestures to mollify South Koreans.

The proposal came a day before Stephen W.~Bosworth, Washington's top envoy on North Korea, was to
arrive in Seoul in the first leg of a regional trip aimed at resuming the nuclear talks.

South Korea was ``favorably'' reviewing the offer for immediate discussions of family reunions so
the reunions could take place in North Korea's Diamond Mountain resort around the Sept.~22 Chuseok
fall-harvest holiday, a government statement said.

About 20,000 Koreans have been temporarily reunited since the two Koreas held their first summit
meeting in 2000. The last such reunions were a year ago, and thousands of older South Koreans wait
for a chance to meet relatives not seen for 60 years.

The scenes of teary people hugging long-lost parents and children have previously helped sway South
Korean opinion in favor of engaging the North. Meanwhile, North Korea has used the reunions to win
economic concessions from the South.

Although the new proposal is a sharp departure from the North's recent invectives against South
Korea, it remains unclear whether the South considers it enough to change a stance hardened by the
warship's sinking.

``We wonder about the North's motive behind this unexpected proposal, but whatever it might be, we
must say we welcome the offer,'' Ahn Hyoung-hwan, a governing party spokesman in Seoul, told
reporters. Opposition parties also welcomed the initiative.

China, the North's main ally and the leader of the six-nation nuclear talks, urged both Koreas to
shift from confrontation to dialogue by rejoining the talks. Although Washington stood by the South,
a crucial Asian ally, in the standoff over the ship sinking, and vowed to keep sanctions on the
North, it is also looking for ways to bring North Korea back to the talks, a central part of its
global nonproliferation efforts.

``We believe it would be critical for there to be some element of reconciliation between the North
and South for any process to move forward,'' Kurt M.~Campbell, a United States assistant secretary
of state, said Thursday.

The six-nation talks were last held in December 2008. South Korean officials have voiced a deep
skepticism of the talks. Without a concrete North Korean commitment to denuclearization, any such
talks would simply give North Korea a tool to weaken an international resolve to enforce sanctions
on the North while buying time to perfect its nuclear arsenal, they said.

After making various threats, North Korea has recently begun offering conciliatory gestures. Last
month, it freed an imprisoned American during former President Jimmy Carter's diplomatic trip to
Pyongyang. It also released a seven-man crew of a South Korean fishing boat seized a month ago. The
North's proposal for reunions came as the South was considering how much aid it should provide for
North Koreans devastated by floods.

In an interview with Russia 24-TV on Friday during a visit to Russia, President Lee Myung-bak of
South Korea offered to build a second joint industrial park in North Korea if the North changed its
policy.

\pagebreak
\section{Inflation in China Is Rising at a Fast Pace}

\lettrine{F}{or}\mycalendar{Sept.'10}{14} K.~K. Lam, a 37-year-old accountant in Guangzhou,
inflation means higher prices for pork and for vegetables like bok choy.

For Allen Dong, the sales manager for a home appliance manufacturer 700 miles to the northeast in
Ningbo, inflation means trying to persuade retailers to pay more for dehumidifiers so his company
can cover rising costs for wages and raw materials.

From street markets to corporate offices, consumers and executives alike in China are trying to cope
with rising prices. The National Bureau of Statistics announced on Saturday that consumer prices in
China were 3.5 percent higher compared with a year earlier, the largest increase in nearly two
years.

To make matters worse, inflation over the short term also seems to be accelerating. A seasonally
adjusted comparison of August prices to July prices showed that inflation was running at an
annualized pace closer to 4.8 percent.

Prices are rising in China for reasons that many Americans or Europeans might envy. The economy is
growing, stores are full and banks are lending lots of money, according to other statistics released
by the government on Saturday. Compared with August of last year, industrial production rose 13.9
percent last month, retail sales increased 18.4 percent, bank lending climbed 18.6 percent and
fixed-asset investment surged 24 percent.

All four categories rose slightly more than economists had expected, in the latest sign of the
Chinese economy's strength even as recoveries seem to be flagging elsewhere.

Separate data released on Friday by the General Administration of Customs showed that Chinese demand
for imports also remained surprisingly strong. The trade surplus narrowed to \$20 billion last
month, and would have been nearly in balance without China's \$18 billion surplus with the United
States.

But so much cash in the Chinese economy chasing a limited volume of goods is pushing up prices.
Inflation is starting to become troublesome, especially for young people entering the work force and
retirees on fixed incomes.

Young people with vocational school degrees typically earn \$200 to \$300 a month in factories near
the coast these days, and somewhat less in the Chinese interior. Rising prices have prompted many to
ask for bigger paychecks, and blue-collar incomes have increased faster than inflation.

But salaries for recent college graduates, at \$300 to \$500 a month in coastal areas, have actually
declined in the last few years, even before adjusting for inflation. A rapid expansion of
universities over the last decade has resulted in more young men and women with undergraduate
degrees than companies are ready to hire, except at lower pay.

And as in many countries, retirees are among the most vulnerable to inflation. Ms.~Lam said her own
mother lived on a pension of just \$150 a month.

Rising wages are putting pressure on companies to increase their prices. Mr.~Dong, the sales manager
at the Ningbo Deye Domestic Electrical Appliance Technology Company, said the company had to raise
wages by 10 percent a year, while raw material costs were also climbing.

``It is impossible to transfer our cost increases entirely to our customers, because if we do so,
they will all run away,'' he said. ``We are currently doing a study of our assembly line work
processes to see where we can achieve greater efficiency.''

But as the powerful growth in fixed-asset investment last month showed, Chinese companies are still
responding to rising prices by building more factories, office buildings and other equipment,
instead of cutting back.

Pan Ning, the sales manager at the Newsunda Industries Company, a manufacturer of school bags and
pocket calculators based in Shantou, said labor and raw material prices had been climbing by 5 to 10
percent. But as school years have begun around the world in the Northern Hemisphere, Newsunda has
been able to raise the prices it charges to cover the increased costs, Ms.~Pan said.

Chinese officials have said for many years that they regard 5 percent inflation as unacceptable, and
they have shown a willingness to clamp down on bank lending and investment whenever annual increases
come close to that level. They have taken some of these steps in recent months, but more recently
eased back on lending controls as some Chinese economists suggested that domestic demand might not
be as strong as the August data showed.

For now, many Chinese consumers are irked by rising prices for everyday necessities.

``I honestly don't know how young people starting out in the work world manage,'' Ms.~Lam said. They
pay nearly half their salaries for their own room in a shared apartment in a bad neighborhood, she
said, ``and if you add in food and transportation, there will be nothing extra left in your salary
to send home.''

\pagebreak
\section{The Hero of a Nation Looks Back and Ahead}

\lettrine{T}{en}\mycalendar{Sept.'10}{14} years after Cathy Freeman galvanized Australia by lighting
the Olympic caldron and winning a gold medal, she is still bringing attention to the plight of her
people.

Freeman does not appear to have changed much from the 27-year-old Aboriginal athlete who opened the
Sydney Games on Sept.~15, 2000, and won the 400-meter final 11 nights later to fulfill a nation's
expectations.

She still has a runner's physique, and still punctuates conversations with an infectious, almost
giddy laugh.

But as the name and face of the Catherine Freeman Foundation, a charity devoted to helping to
educate Aboriginal children from Palm Island in northeastern Australia, she has no problem conveying
the challenge she now confronts. ``It is a national disgrace,'' she said.

Freeman is a long way from Palm, a small group of islands off Townsville in Queensland state where
her mother was born, while sitting in the foundation's Melbourne office during an interview.

She recounts the euphoric moments of the Olympics -- and how difficult it was to love running while
hating the associated hype.

After the Olympics, she admitted to being at a loss about what to do in the future until she looked
within.

``Everyone knows I've always been proud of my indigenous roots,'' said Freeman, who drew official
rebuke but public accolades for waving both the Aboriginal and Australian flags during her victory
lap at the 1994 Commonwealth Games in Victoria, British Columbia.

When she repeated the two-flag celebration after winning Olympic gold, she was unilaterally showered
with praise.

After years of trying to dodge the fame Olympic success brought her, Freeman decided she might as
well just run with it.

``It just felt so right,'' she said. ``Instead of trying to deny my status here in Australia, I just
went with it and that's why the foundation came about -- so it doesn't feel so painful.''

On Wednesday, Australia celebrates 10 years since Sydney hosted an Olympics that many here view as
helping to redefine the country's international profile, giving it the image of a modern, successful
and enviable nation.

The bleak history of Australia's indigenous people has long been a blight on that vision.

Although Aborigines make up less than 3 percent of Australia's population of 22 million, they are
the poorest and most disadvantaged minority more than 200 years after white settlers arrived,
despite billions spent on programs to close the gap.

The Palm mission, Freeman said, was established by the government in 1918 to resettle nearly 2,000
so-called troublesome people forcibly removed from dozens of tribal groups. The community that took
root remains troubled, but is also fiercely proud.

Freeman was born on the mainland in nearby Mackay and moved around as a child with her family as her
athletic career flourished, but she retains vivid memories of her visits to Palm.

``There are some really devastating living conditions,'' she says, ticking off its socioeconomic
ills. ``Palm Island in itself is the fourth most disadvantaged community in Australia: 90 percent
unemployment; only 7 percent of kids have reached the national standards in literacy and numeracy;
3,500 residents, only 350 houses.

``There's still an aura of racism and hatred and loathing. It's hard because it does bring up the
past. Everyone's country has got a past they don't feel good about. The nice thing about who I am,
is I represent something that is possible.''

Freeman does not pretend to speak from the perspective of Palm's residents. She is just determined
to help improve their situation.

``I've never dealt with a community or let alone an individual who's been hurt so much and burnt so
badly,'' she said. ``We're talking about years and years of hurt,'' dating to when people like her
great-grandparents were ``sent there under a really sort of concentration camp regime.''

The death in 2004 of a man in police custody sparked riots on the island that left the police
station and courthouse in ruins.

An inquest found that blows from a police officer contributed to the death, but there has never been
a conviction.

``There's been one letdown after the other,'' said Freeman, whose main aim is to build trust and
bridge the gap between outsiders and the people of Palm.

``I'm doing what I think I've always been born to do in a way,'' she added. ``I'm not wondering why.
I don't overanalyze it. It's a continuum of what I've already done on the track. It fulfills. It's a
oneness more than a separate journey.''

Her foundation sponsors scholarships for seven girls to attend boarding schools. Freeman also visits
Palm every few months to reinforce the rewards-based program her foundation fosters to improve
school attendance. The highlight of those trips is playing with the children.

``I'll go for a run and they say, `There's Cathy Freeman, there's Cathy Freeman,' '' she said. Some
youngsters want to race her.

Mayor Alf Lacey of Palm Shire said Freeman's foundation has had success in improving education on
the island.

``It's really important for our community, where you don't have a lot of choices or options,'' Lacey
said in a telephone interview. ``It's about the kids, the next generation and giving them
opportunities earlier generations didn't have.''

Freeman's profile is helping on and off the island.

``Cathy's a good influence,'' Lacey said. ``She's been able to get through doors, talking it up
through the private sector and the government sector, and getting them to know it's very
important.''

Freeman was the first Australian Aboriginal athlete to win an individual Olympic gold. She was a
two-time world champion and Olympic silver medalist when she was asked to light the Olympic caldron
in 2000.

At the time, debate was raging in Australia about whether the government should offer an official
apology to thousands of Aborigines who were forcibly removed from their families under assimilation
policies that were not abandoned until 1970. A national inquiry in the 1990s laid bare decades of
abuse and trauma for many in what is known as the Stolen Generation. Australians of all backgrounds
shed tears when Freeman lighted the flame.

``A dear friend of mine says that I have a lot to do with flying that flag,'' Freeman said. ``The
indigenous culture, people are now concerned. It's just nice to have played a part.

``For me it was just a simple expression of pride. It's not a secret indigenous people were the
first land dwellers of this country. It's pure truth.''

Kevan Gosper followed Freeman's career after working with her in 1991. A former vice president of
the International Olympic Committee, he said he was concerned that the pressure of lighting the
flame and winning gold at the same Olympics would be too much. Gosper was relieved after Freeman won
the 400, and hugged her when he presented her with the gold medal.

``She's an Australian icon,'' he says. ``It made every Australian proud. She is a symbol of all
that's good about relations between indigenous Australians and other Australians.''

Freeman is only just now starting to become comfortable with reliving her Olympic experience.

``Now that I've found this purpose with the foundation,'' she said. ``I've understood I really need
to accept it and embrace it.''

\pagebreak
\section{Consumer Candidate May Avoid a Vote}

\lettrine{T}{he}\mycalendar{Sept.'10}{14} Obama administration is considering appointing the legal
scholar Elizabeth Warren to run a new consumer bureau on a temporary basis to avoid a potentially
bruising confirmation battle in the Senate, according to people who have been briefed on the search.

The Consumer Financial Protection Bureau, a centerpiece of the Dodd-Frank Wall Street regulatory
overhaul law that Mr.~Obama signed in July, was established to prevent abusive, deceptive and
fraudulent terms for mortgages, credit cards, payday loans and a vast array of other financial
products. It is to be led by a director, appointed by the president to a five-year term with the
consent of the Senate.

Two people who have been briefed on the appointment process, who spoke on condition of anonymity
because they feared reprisal, said the White House was exploring ways to have Ms.~Warren effectively
run the bureau without having to endure a confirmation battle and, potentially, the threat of a
Republican filibuster.

Mr.~Obama could name Ms.~Warren using a recess appointment, though such an appointment would last
only until the end of next year. In addition, the law appears to permit Ms.~Warren to run the
bureau's day-to-day affairs while it is nominally under the supervision of the Treasury Department,
to which Congress has delegated the powers of the bureau until it is fully established as a
freestanding agency. The bureau, which will consolidate employees and functions from a host of other
agencies, could have a budget as large as \$500 million.

On Friday, Mr.~Obama credited Ms.~Warren, a Harvard law professor, with coming up with ``the idea
for this agency,'' and he praised her as ``a dear friend'' and ``a tremendous advocate.'' He said he
had had conversations with her but was not yet ready to make an official announcement.

Amy Brundage, a White House spokeswoman, declined Monday evening to discuss the possibility of a
temporary appointment.

``Elizabeth Warren has been a stalwart voice for American consumers and families and she was the
architect of the idea that became the Consumer Financial Protection Bureau,'' Ms.~Brundage said in a
statement. ``The president will have more to say about the agency and its mission soon.''

The Dodd-Frank law gave the Treasury secretary, Timothy F.~Geithner, power to ``perform the
functions of the bureau'' until a director is confirmed. The bureau will have vast powers to write
and enforce new rules, and Treasury aides have already begun administrative work to get the bureau
running.

Under the law, Mr.~Geithner has until Sunday -- 60 days from the signing of the act -- to designate
a date for transferring to the new bureau functions currently performed by the Federal Reserve, the
Federal Trade Commission, the Federal Deposit Insurance Corporation and other agencies.

The transfer date is supposed to be anywhere between six and 18 months from July 21, when Mr.~Obama
signed the law.

But under the law, Mr.~Geithner could delay that transfer until 24 months -- or July 2012 -- if he
explained to Congress that ``orderly implementation'' of the law was ``not feasible'' within the
18-month limit.

Ms.~Warren, 61, is widely admired by consumer groups and labor unions, while banks and other
financial institutions have indicated that they would oppose her appointment. Senator Harry Reid,
Democrat of Nevada and the majority leader, picked her to lead the panel overseeing the 2008 Wall
Street bailout program.

An Oklahoma native, Ms.~Warren is an authority on bankruptcy law and contracts. She taught at law
schools in New Jersey, Michigan, Texas and Pennsylvania before joining Harvard in 1995.

Ms.~Warren has been a front-runner to lead the new bureau, though a leading Democratic senator,
Christopher J.~Dodd of Connecticut, who is the chairman of the Banking Committee, has raised doubts
about whether she could be confirmed. (Mr.~Dodd has pledged to support Ms.~Warren if she is
nominated.)

A temporary appointment would permit Ms.~Warren to shape the bureau from its inception, while
avoiding the delays that could accompany a lengthy confirmation fight.

But some Democrats also say they believe the Obama administration might benefit from taking a
prominent stance in support of Ms.~Warren, and said the White House might relish a public battle
rather than shy away from it.

\pagebreak
\section{A Thai City of Sleaze Tries to Clean Up Its Act}

\lettrine{S}{omewhere}\mycalendar{Sept.'10}{16} in the world there may be a city with a more seedy
reputation, a place more devoted to the sex industry and more notorious as a haven for criminals on
the lam. But probably not.

When dusk comes to this beach resort, a sea of pink neon bulbs casts a pale glow onto the thickly
made-up faces of thousands of women (and some men) who sit on bar stools waiting for their patrons.

If Las Vegas is Sin City, Pattaya is a bear hug from Lucifer himself.

And yet, amid the back alleys jammed with girlie bars and a beachfront peopled with what the Thais
euphemistically call ``service women,'' there are signs of change.

Indian couples, Chinese tour groups and vacationing Russian families stroll around the city. A dozen
luxury hotels cater to the weekend crowd of wealthy Thais from Bangkok who mingle with tourists at a
huge shopping mall. Pattaya has a growing number of fancy restaurants, an annual music festival and,
perhaps most improbably, regular polo tournaments.

Long derided as a city of sleaze, the city is reaching for respectability.

A two-hour drive from Bangkok, Pattaya was little more than a fishing village four decades ago when
U.S.~soldiers fighting in the Vietnam War discovered a pristine, coral-filled bay. Tens of thousands
of lonely soldiers armed with dollars sought respite from the war in a country of relative poverty,
lax law enforcement and historically tolerant attitudes toward prostitution. The result was
predictable.

Pattaya survived the departure of the G.I.'s by expanding into sex tourism. Visitors to Thailand in
the 1970s were offered brochures at the Bangkok airport showing pictures of available companions.
The booth at the airport no longer exists, but the business lives on: for at least the past decade,
men have outnumbered women as tourists in Thailand. They make up about 60 percent of foreign
visitors in Thailand compared with 52 percent in nearby, law-abiding Singapore.

In recent years the Pattaya tourist industry has sought to diversify its client base. Hotel managers
learned that, despite jokes about recession-proof industries, relying heavily on a Western male
clientele was unwise at a time when the United States and Europe were buffeted by recession.

Tourism agencies now actively seek out visitors from the rising economies of China and India.

``There's definitely been a change,'' said Shyam Anugonda, a 39-year-old lawyer from Bangalore,
India, whose first trip to Pattaya was eight years ago, when he was single.

``It was more sex oriented before,'' Mr.~Anugonda said as he shopped for Thai fabrics with his wife,
Kavitha.

This time, Mr.~Anugonda's five-day vacation included an elephant show and parasailing.

The government is encouraging the rebranding of Pattaya by developing a master plan for the city,
including a monorail to help relieve traffic-clogged streets, a redrawn waterfront and a high-speed
rail line from Bangkok. The plan is awaiting approval from the Thai cabinet.

The police, too, say they are trying to clean up the city's image.

``There are people who say Pattaya is the paradise of criminals,'' said Col.~Atiwit Kamolrat, the
head of the immigration police. ``It's now going to be impossible for them to hide here.''

His office's Transnational Crime Data Center combs through lists of wanted criminals from foreign
governments and cross-references them with hotel registration logs and visa renewal applications.
Since the beginning of the year, the office has arrested 12 foreign criminals hiding out in Pattaya,
Colonel Atiwit said.

Somchet Thinaphong, a board member charged with the city's redevelopment plan at the Designated
Areas for Sustainable Tourism Administration, a government agency, said Pattaya's face-lift would
cost 32 billion baht, or about \$1 billion. He spoke in generalities about ``sustainable
development'' and making the city more ecologically friendly.

But here in Pattaya, officials chuckle derisively at the notion that the city can be totally
sanitized. Stamping out Pattaya's sex industry is fantasy, said Niti Kongrut, the director of the
Pattaya branch of the Thai government's tourism office.

``You talk about sustainable development, how about prostitutes? They have been around for a very
long time,'' Mr.~Niti said. ``We can't close down the go-go bars. It's a free country. Besides, it
makes money.''

For decades, officials have wrestled with the question of what to do about the seedy side of the
city, Mr.~Niti said. ``Now we just ignore them and try to promote other activities.''

For visitors who have no intention of partaking in it, the sex industry has become a sort of
spectacle, a red-light district that makes its counterparts in other cities seem almost Victorian.

Olga Bidenko, 28, a tourist from Ukraine who came to Pattaya with a colleague from the marketing
company she works for, said she was entertained by Walking Street, a thoroughfare stretching a
kilometer and a half, or about a mile, blocked to motor traffic and packed with bars. Typical of the
bars is Sexy Airline, where women dressed in old-fashioned air hostess outfits call out to
prospective patrons passing by.

``We thought Amsterdam was the sex capital of the world,'' said Ms.~Bidenko, 28. ``But now that I've
been here, I think Amsterdam is a perfectly respectable city.''

\pagebreak
\section{U.S.~Adopts Tougher Stance on China}

\lettrine{T}{he}\mycalendar{Sept.'10}{16} Obama administration is moving to take a harder stance on
the Chinese government's trade and currency policies, with anger toward China rising in both
political parties ahead of midterm elections.

Treasury Secretary Timothy F.~Geithner, in separate hearings before House and Senate panels, plans
to acknowledge on Thursday that China has kept the value of its currency, the renminbi, artificially
low to help its exports and has largely failed to improve the situation as it promised to do in
June.

``We are concerned, as are many of China's trading partners, that the pace of appreciation has been
too slow and the extent of appreciation too limited,'' Mr.~Geithner plans to say, according to
excerpts of his statement released on Wednesday night by the Treasury Department.

The United States brought two cases to the World Trade Organization on Wednesday, accusing China of
improperly blocking imports of a specialty steel product and denying credit card companies access to
its markets. The move came just hours before House lawmakers demanded action on the currency issue.

The renminbi has risen about 1 percent against the dollar since Beijing promised new exchange rate
flexibility in June.

In his testimony, Mr.~Geithner is not expected to rule out declaring China a currency manipulator, a
finding that could lead to retaliatory trade measures. The administration has so far refused to take
such a step, relying instead on persuasion, though with little success.

The currency issue is increasingly likely to be a focus when leaders of the Group of 20 nations meet
in November in Seoul, South Korea. A bill with support from 143 House members from both parties
would allow the United States to impose tariffs and other penalties on countries that undervalue
their currencies.

But many economists believe that China is unlikely to yield to American pressure, and they have
called on the Obama administration to do a better job of enlisting support from the European Union
and Japan.

The office of the United States trade representative, Ron Kirk, said the timing of the new W.T.O.
cases was unrelated to the other economic tensions with China.

In one case, the United States accused China of violating world trade rules when it imposed
antidumping duties and countervailing duties on grain-oriented, flat-rolled electrical steel, which
is used to make transformers and reactors used to generate electricity. The two largest makers of
such steel are Allegheny Technologies, based in Pittsburgh, and AK Steel, based in West Chester,
Ohio.

China imposed duties as high as 65 percent in April after concluding that the American manufacturers
had sold the steel at less than fair value and had received improper subsidies from the United
States. The Americans say the charges are false.

In the other case, the trade representative's office accused China of illegally blocking American
electronic payment companies from access to its markets, through its support of a state-financed
company, China Union Pay, which has had a monopoly since 2001 over renminbi-denominated debit and
credit card payments in China.

Mr.~Kirk said his office was ``fighting for the American jobs threatened by China's actions, and
insisting on the level playing field promised in our W.T.O. agreements.''

Leaders of the Senate Finance Committee, which oversees trade, applauded the filings. Its chairman,
Senator Max Baucus, Democrat of Montana, called them ``critical steps forward in our effort to
enforce our market access rights in China.'' The panel's senior Republican, Senator Charles
E.~Grassley of Iowa, said, ``It's about time the administration decided to act.''

Mr.~Grassley added: ``The administration should go one step further and bring a case against China's
unfair currency manipulation at the W.T.O.''

On Wednesday, the House Ways and Means Committee began two days of hearings on China's currency, its
third set of hearings this year on the topic.

Its chairman, Representative Sander M.~Levin, Democrat of Michigan, said ``a multilateral approach
would be the most likely to yield the broadest results.'' Mr.~Levin also called Japan's move to
weaken the yen, that country's first intervention in the currency markets since 2004, ``a deeply
disturbing development.''

Mr.~Levin said that the International Monetary Fund had little power to enforce its rules against
currency manipulation, adding that the G-20 should take up the issue. But he warned that ``there
does not appear to be anything remotely approaching an international agreement to end predatory
exchange rate policies.''

Mr.~Levin urged the administration to bring a case before the W.T.O. arguing that China's currency
policy amounted to an illegal export subsidy. He said he thought the United States could impose
countervailing duties against China without violating its own obligations under world trade rules.

More than 140 House members have signed onto a bill sponsored by Representatives Tim Ryan, Democrat
of Ohio, and Tim Murphy, Republican of Pennsylvania, that would compel the administration to impose
such duties.

The United States-China Business Council has said it believes such a move would antagonize China
without yielding meaningful results, and the senior Republican on the committee, Representative Dave
Camp of Michigan, expressed similar skepticism at the hearing.

Manufacturers, labor unions and politicians from the Midwest have been among the most vigorous in
calling for sanctions, but there were indications on Wednesday that policy experts were increasingly
in favor of tough action.

China permitted the value of the renminbi to rise about 20 to 25 percent against the dollar from
2005 to 2008, before the government reimposed a currency peg to support its export-centered economy
after the global financial crisis.

C. Fred Bergsten, director of the Peterson Institute for International Economics, a leading research
organization here, told House lawmakers on Wednesday that a similar increase over the next two to
three years would create about 500,000 jobs. He said it would reduce China's current account surplus
by \$350 billion to \$500 billion, and the American current account deficit by \$50 billion to \$120
billion.

The United States should seek to mobilize the European Union and countries like Brazil, Russia and
India to press China to realign the renminbi, and should seek W.T.O. authorization to impose
restrictions on Chinese imports if it does not do so, Mr.~Bergsten said.

\pagebreak
\section{Europeans Like Obama but Fret Over Some Policies}

\lettrine{P}{resident}\mycalendar{Sept.'10}{16} Barack Obama's popularity remains high among
Europeans, but there are growing doubts about his handling of some foreign policy issues,
particularly Iran and Afghanistan, according to the annual Transatlantic Trends Survey published
Wednesday.

Although there was a slight decline in Mr.~Obama's approval rating, down from 83 percent in 2009 to
78 percent this year, over half of the Europeans polled said they supported the United States'
exerting strong leadership in world affairs. This is in sharp contrast to the ratings of Mr.~Obama's
predecessor, George W.~Bush, who consistently received poor scores from Europeans.

The latest survey, published by the German Marshall Fund of the United States and Compagnia di San
Paolo, an Italian foundation, was conducted in June in 11 European Union countries -- Britain,
Bulgaria, France, Germany, Italy, the Netherlands, Poland, Portugal, Romania, Slovakia and Spain --
as well as in Turkey and the United States.

The survey, a random telephone sampling of 1,000 men and women, 18 years and older, in each country,
had a sampling error margin of plus or minus three percentage points. The major topics included
attitudes toward the euro crisis, China, Turkey's foreign policy shift toward its neighbors and
Poland's waning Atlanticism.

With few exceptions, respondents in Europe said the euro has been bad for their economy. The only
countries that had majorities supporting the euro were the Netherlands (52 percent) and Slovakia (64
percent).

A plurality (46 percent) said it was the responsibility of each country's government to deal with
the economic crisis, while 39 percent said Brussels should be primarily responsible for handling it.
But most E.U. respondents (63 percent) agreed that being a member of the E.U., the world's biggest
economy, was a good thing.

Surprisingly, Germany was the only country in which a majority of participants (54 percent) said the
European Union should have primary responsibility for economic decision-making, a view not shared by
Chancellor Angela Merkel's coalition government.

On two foreign policy issues, Afghanistan and Iran, the Europeans and Americans were at odds. A slim
majority of U.S.~respondents, (51 percent, down five percentage points from 2009) felt optimistic
about stabilizing Afghanistan, compared to a quarter of Europeans. A majority of European
respondents said their countries should reduce or withdraw all troops, while a majority of Americans
supported maintaining or increasing a military presence.

As for Iran, a big majority of Americans and Europeans said they were concerned about Iran's nuclear
arms ambitions but differed on how to curb them. A majority of Europeans prefer economic incentives,
while a majority of U.S.~respondents support economic sanctions.

Another divergence was over the growing role of China, with half of Americans believing that the
United States had enough common values with China to cooperate on international problems. Almost
two-thirds of European respondents said the values were so different that cooperating on
international issues was impossible.

\pagebreak
\section{China Shifts Away From Low-Cost Factories}

\lettrine{C}{ompanies}\mycalendar{Sept.'10}{16} here in China's industrial heartland are toiling to
reinvent their businesses, fearing that the low-cost manufacturing that helped propel the nation's
economic ascent is fast becoming obsolete.

The TAL Group, which operates an immense garment-making plant in this coastal boom town, is moving
beyond piecework by helping J.~C. Penney electronically manage its inventory of dress shirts, from
factory floor to retail shelves as far away as Connecticut.

Chicony, maker of a power device used in the Xbox from Microsoft and a major supplier of computer
keyboards to Dell, is diversifying by opening department stores, with three so far around China and
seven more planned.

And after years of assembling vacuum cleaners and rechargeable toothbrushes for Philips and other
Western companies, Kwonnie Electrical Products is planning its own line of home appliances.

``We want to do more original design and build our own brand,'' Benjamin Kwok, a company founder,
said during a recent tour of a sprawling factory complex that has 3,000 workers, a huge warehouse
and labs for testing juice makers, vacuum cleaners and other appliances.

``Many customers won't be happy with the decision to compete with them,'' Mr.~Kwok said. ``But we
have no choice.''

It is too soon to know whether such makeovers will succeed. But economists consider such efforts
necessary -- and overdue.

For years, factories here in the Pearl River Delta region have served as the low-cost workshops for
global brands, turning this part of China into the nation's biggest export zone. The city of
Dongguan, about 35 miles northwest of Hong Kong, has long churned out toys, textiles, furniture and
sports shoes -- including hundreds of millions of sneakers a year for companies like Nike and
Adidas.

But now, with manufacturing costs rising and China looking to create a consumer middle class,
experts say the revamping of this region's industries could help reduce the nation's wide income gap
and encourage more balanced and sustainable economic growth.

``It is my hope that China's comparative advantage as a low-wage producer does disappear -- the
sooner the better,'' Fan Gang, an economics professor at Peking University, wrote in a recent essay,
adding that China needed to upgrade and embark on ``the next stage of development.''

Manufacturing costs have risen rapidly here in response to nagging labor shortages and worker
demands for higher wages to help offset soaring food and property prices.

Those pressures were evident a few months ago, when a series of big labor strikes in southern China
disrupted several Japanese auto factories and resulted in hefty pay raises.

There is also the looming prospect that China's currency, the renminbi, will strengthen against
other world currencies in the coming years. That would make goods produced here even more expensive
to export, and further erode what manufacturers say are already thin profit margins.

Seeking lower costs, some Pearl River Delta factories are relocating to poor inland regions of China
where wages are as much as 30 percent lower than in coastal provinces. Other factories are moving to
lower-wage countries like Bangladesh and Vietnam.

But for companies that have invested billions of dollars in factories here, simply packing up and
pulling out is not always financially feasible. That is why many owners of Dongguan factories are
experimenting with other solutions.

``We've decided that we're not going to be on the low end,'' says Roger Lee, the chief operating
officer at TAL Apparel, part of the TAL Group.

TAL, which is based in Hong Kong and says it makes one of every six dress shirts sold in the United
States, is expanding into supply-chain management for J.~C. Penney, one of its big shirt-buyers.
Through an extensive computerized system, TAL can stock and restock shirt shelves in all 1,100 of
Penney's retail stores in the United States, as demand warrants.

``Too much inventory kills retailers,'' Mr.~Lee said. ``Now, we're managing inventory in each store.
We gets sales data. We know what's in the warehouse, what's on the boat. We help reduce inventory.''

TAL is a fortunate survivor. After the global financial crisis hit, Dongguan's exports plummeted by
about 25 percent. Thousands of factories simply closed. Now -- even though exports have rebounded to
2008 levels -- there are worries that regional growth is slowing drastically.

``Since 2008, the investment environment has worsened in Dongguan,'' said Lin Jiang, a professor of
finance at Sun Yat-sen University in Guangzhou. ``A lot of companies don't see a future in Dongguan.
And they feel pressure from the government to upgrade.''

In Qingxi, an economic zone in the southeastern part of Dongguan, district government officials are
trying to help desperate factories adjust to the new realities. If many companies are reluctant to
leave, the local government is just as loath to lose the companies and their tax revenue.

The 56-square-mile Qingxi district is crowded with textile and electronics factories, mostly backed
by companies from Hong Kong and Taiwan, that produce for global brands like Burberry,
Hewlett-Packard and Sony.

The country's export boom helped Qingxi transform vast tracts of farmland into bustling factories
with noisy assembly lines. That created enormous wealth for the country and the local region. But
the labor equation is rapidly changing.

Years ago, migrant workers lined up outside factories here hoping to apply for work. As a result, 90
percent of Qingxi's 350,000 residents are migrant workers. Most of them traveled from China's poor
interior provinces to find factory jobs that today often pay about 90 cents an hour, which is the
typical wage in the Shenzhen-Dongguan area.

But a demographic shift tied to the nation's one-child policy means fewer young people are entering
the work force. And government efforts to improve conditions in the interior provinces have lifted
growth in those regions and persuaded many young workers to find jobs closer to home.

So companies here can no longer pick and choose among workers.

``We used to prefer women because they are easier to manage,'' said Frank Chen, a manager at a
Qingxi factory called Lite-On Technology, which makes Internet-access cards for Wi-Fi devices.
``Before, we wanted three females for every male. But because of the labor shortage, it's hard to
get that ratio now.''

Chicony, trying to drum up workers, has taken to sending a bus around Dongguan with a loudspeaker
blaring, ``Chicony is the best.''

Because of labor shortages and government efforts to raise the minimum wage to improve the
livelihoods of migrant workers, pay rates in the Shenzhen-Dongguan area have nearly doubled in the
last five years.

Still, factories here often have to pay middlemen and vocational schools to find migrant labor. The
Qingxi government has also tried to step in, organizing recruiting drives into the country's poorest
regions.

But longer term, district officials want to encourage innovation.

Zhu Guorong, the vice director of the Qingxi Office of Trade and Economic Cooperation, is among
those trying to remake Qingxi. Recently, he drove a sparkling blue Toyota FJ Cruiser -- a kind of
miniature Hummer -- through the city's economic zones, talking about the shift under way.

``Every company now wants to be a high-tech company, and we want to encourage them,'' Mr.~Zhu said,
as he headed for an electronics factory, where he would inquire about profitability.

The national government has preferential tax policies to encourage technology companies, and the
Qingxi district government has a research and development fund -- officials decline to say how much
money it has -- to support efforts.

One company that has already received government money for research and development is a division of
Lite-On Technology, the electronics supplier.

But even for innovators like TAL, the garment maker, success is far from guaranteed.

``The price of a shirt has gone down,'' Mr.~Lee said. ``But our costs have gone up.''

\pagebreak
\section{Competitor Sues Google Over Location Software for Smartphones}

\lettrine{S}{kyhook}\mycalendar{Sept.'10}{16} Wireless, which makes software that shows smartphone
users where they are on their phone's maps, filed a lawsuit Wednesday claiming Google had persuaded
Motorola and another phone manufacturer to break contracts with Skyhook and use Google's competing
service.

In a separate suit, also filed Wednesday, Skyhook accused Google of infringing on Skyhook's patented
methods of determining location.

The two companies are fighting for the lead in the nascent but promising business of location-based
data that uses GPS or Wi-Fi signals to locate phone users. These services not only direct people to
businesses, but collect information about where people are. That is valuable information that lets
marketers direct advertising to people where and when they are most likely to buy.

``People view it as the next frontier, the next place to get the attention of the consumer,'' said
Carl Howe, an analyst with Yankee Group, a technology research company. ``It's not big now, but we
believe it to be the next consumer battleground.''

Skyhook's interference suit against Google, filed in Massachusetts Superior Court in Suffolk County,
accuses Google of intentionally disrupting Skyhook's business relationships. It says Google has
notified cellphone makers that they need to use Google's location service as a condition of using
Google's Android smartphone operating system.

The complaint claims that Andy Rubin, Google's vice president for engineering, gave Sanjay K.~Jha,
chief executive of Motorola's mobile devices division, a ``stop ship'' order, preventing Motorola
from shipping phones with the Android operating system using the Skyhook software, called XPS.

The complaint charges that the Skyhook software had already been tested by Motorola and had
completed the Google approval process.

``It's very hard to meet compliance when Google keeps moving the goal post,'' said Ted Morgan,
Skyhook's chief executive, in a telephone interview Wednesday.

Google declined to comment. Motorola did not respond to requests for comment.

Skyhook, based in Boston, said that it had a nearly identical experience with a second company
referred to only as ``Company X'' in the complaint. The suit said that Skyhook's licensing agreement
with Company X was announced July 2, the date Skyhook announced its agreement with Samsung
Electronics. Samsung declined to comment.

In the patent suit, filed in Federal District Court in Massachusetts, Skyhook claims that Google
violated four of Skyhook's patents that gave it an advantage over competitors.

Competition to control location data is escalating because of the potential size of the market.
``Regardless of how you calculate the number, the size of the opportunity is enormous,'' said
Alistair Goodman, chief of Placecast, a San Francisco location-based mobile marketing company.

Early research also shows mobile marketing to be highly effective at reaching consumers, said
Mr.~Goodman, whose company lets people sign up for alerts that appear on their phones when they are
near the store of a client company or a site that company thinks will interest its customers.

``In aggregate, 65 percent of people in the programs made purchases, and 79 percent say the service
was valuable,'' Mr.~Goodman said. ``They didn't see it as advertising or marketing.''

The value of the data surpasses the placing of ads on phones. It also allows companies to make
inferences about a phone owner's wealth, lifestyle and shopping preferences, which is also sought by
marketers.

``We learn pretty interesting things, for instance who prefers Wal-Mart over Target, or Walgreens
over CVS, who is split, which stores they will travel to get to,'' said Thaddeus Fulford-Jones,
chief of Locately, a location analytics company.

\pagebreak
\section{Poll Suggests Opportunities for Both Parties in Midterms}

\lettrine{R}{epublicans}\mycalendar{Sept.'10}{16} are heading into the general election phase of the
midterm campaign backed by two powerful currents: the highest proportion of voters in two decades
say it is time for their own member of Congress to be replaced, and Americans are expressing
widespread dissatisfaction with President Obama's leadership.

But the latest New York Times/CBS News poll also finds that while voters rate the performance of
Democrats negatively, they view Republicans as even worse, providing a potential opening for
Democrats to make a last-ditch case for keeping their hold on power.

The poll represents a snapshot of the country's political mood as the campaign pivots from primary
contests that have revealed deep divisions among Republicans into the general election, where the
parties deliver their competing arguments to a wider audience.

The findings suggest that there are opportunities and vulnerabilities for both parties as they
proceed into the final seven weeks of the campaign.

A case for Republicans: Voters are remarkably open to change, even if they are not sure where
Republicans will lead them. Most Americans, including one-third of those in the coalition that
elected Mr.~Obama, now say he does not have a clear plan to solve the nation's problems or create
jobs. Democrats remain highly vulnerable on the economy.

A case for Democrats: They are seen as having better ideas for solving the country's problems. The
public steadfastly supports the president's proposal to let tax cuts expire for the wealthiest
Americans. And far more people still blame Wall Street and the Bush administration than blame
Mr.~Obama for the country's economic problems.

Voters have a darker view of Congressional Republicans than of Democrats, with 63 percent
disapproving of Democrats and 73 percent disapproving of Republicans. But with less than two months
remaining until Election Day, there are few signs that Democrats have made gains persuading
Americans that they should keep control of Congress.

``I really think we need to get some new blood in there,'' said Kathy Beckman, 44, an optometrist
from Lodi, Calif., who spoke in an interview after participating in the poll. ``Get them all out.''

The mood of the country is similar in many respects to the fall of 1994, when Republicans swept
control of Congress. There is an overall low Congressional approval, large numbers of Americans who
believe the country is on the wrong track and soaring discontent among voters with their own
representative.

It is that particular finding in the poll that underscores the true depths of the disgruntlement
among the public and is an ominous sign for Democrats, who have a 39-seat majority in the House and
a 10-seat majority in the Senate.

In many election cycles, voters readily acknowledge that they are dissatisfied with government or
Congress in general, but they tend to have a stronger connection toward their own representative.
That is not the case this year, with 55 percent of voters saying it is time for new leadership and
only 34 percent saying their lawmaker deserves re-election. It is a historic high for a question
asked in each midterm election year since 1990.

The economic climate is also worse this year, with 8 in 10 Americans rating the economy negatively
and 4 in 10 saying that their family's financial situation is worse than it was two years ago. In
September 1994, two months before Democrats lost their majorities in the House and Senate, more than
half of people said the condition of the national economy was good.

The economy and jobs are increasingly and overwhelmingly cited by Americans as the most important
problems facing the country, while the federal budget deficit barely registers as a topic of concern
when survey respondents were asked to volunteer their worries.

The national telephone poll was conducted Friday through Tuesday, the day that primary contests
unfolded in seven states. The survey included 990 adults, of whom 881 were registered voters. The
poll had a margin of sampling error of plus or minus three percentage points.

Voters do not perceive Republicans as having better ideas and disagree with them on the biggest
economic issue of the campaign -- whether to extend the Bush-era tax cuts for the wealthy -- a sign
the party has no real advantage on key pieces of their agenda, which makes it more necessary to run
as a generic alternative to the party in power.

The Tea Party movement, which showed its strength in Republican primaries in Delaware and New York,
has yet to be fully defined for many Americans. Nearly half of voters say they are undecided or have
not heard enough about the Tea Party to form an opinion, a sign that offers an opportunity for the
movement to define itself to many voters and help shape their views of it before Election Day.

Nearly half of all Republican voters say they have a favorable opinion of the Tea Party. But the
view of the movement among independent voters grew more negative since a Times/CBS poll was
conducted in April. Now, 30 percent of independent voters have an unfavorable view of the Tea Party,
with 18 percent holding a favorable view and more than half offering no opinion.

The president's overall job approval rating is 45 percent, with 47 percent disapproving. On the
economy, his rating is worse, with 41 percent approving and 51 percent disapproving. When asked
whether Mr.~Obama has a clear plan for solving the nation's problems, 57 percent responded that he
did not, yet twice as many give him more credit than Republicans for having a plan.

``He had a lot of plans and he's not really sticking to them like he said he would,'' Tammy Danley,
38, an independent voter from Louisburg, Kan., said in an interview after participating in the poll.
``It seems like a lot of talk and not a lot of action.''

With the Democratic majority on the line, and the outcome of the election seen as a referendum on
the president, 45 percent of voters said Mr.~Obama would not be a factor in their vote in November,
while 23 percent said their vote would be for Mr.~Obama and 25 percent said it would be against him.

The poll found that the public has an increasingly negative opinion of Sarah Palin, the former
Alaska governor and 2008 Republican vice-presidential nominee, with nearly half now holding an
unfavorable view of her. Her favorable rating is down nine percentage points since April.

Since then, she has increased her presence in the midterm election campaign, endorsing dozens of
Republican candidates across the country, most of whom were also backed by Tea Party activists.
Two-thirds of Americans think that Ms.~Palin's primary motivation is staying in the public eye,
rather than helping conservative candidates get elected.

``I think she is trying to be more in the public eye for her own benefit,'' said Kathy Allen, 51, an
unemployed worker from Idaho Falls, Idaho, who spoke in an interview after participating in the
poll, ``whether it be or financial purposes or in order to run for something again.''

\pagebreak
\section{G.O.P. Leaders Say Delaware Upset Hurts Senate Hopes}

\lettrine{T}{he}\mycalendar{Sept.'10}{16} Tea Party movement scored another victory on Tuesday,
helping to propel a dissident Republican, Christine O'Donnell, to an upset win over Representative
Michael N.~Castle in the race for the United States Senate nomination in Delaware.

Mr.~Castle, a moderate who served two terms as governor and had been reliably winning elections for
the last four decades, became the latest establishment Republican casualty. Republican leaders, who
had actively opposed Ms.~O'Donnell, said the outcome complicated the party's chances of winning
control of the Senate.

With all precincts reporting, Ms.~O'Donnell won 53 percent of the vote to Mr.~Castle's 47 percent.
The primary drew 57,000 voters, a small slice of the overall electorate.

Ms.~O'Donnell, a former abstinence counselor who had failed in previous attempts to win office in
Delaware, won the endorsement of Sarah Palin, Senator Jim DeMint of South Carolina and other leaders
of the party's conservative wing.

``A lot of people said we can't win the general election; yes we can!'' Ms.~O'Donnell said. ``It
will be hard work, but we can win if those same people who fought against me work just as hard for
me.''

The results on the last big night of primaries highlighted the extent to which the Tea Party
movement has upended the Republican Party and underscored the volatility of the electorate seven
weeks from Election Day.

In New Hampshire, another candidate with strong backing from grass-roots conservatives, Ovide
Lamontagne, was narrowly behind his main opponent, Kelly Ayotte, in the Republican primary for
Senate Wednesday morning.

``In the interest of making sure all the votes are counted,'' Mr.~Lamontagne told supporters at a
rally after midnight, ``we're going to continue to wait this out.'' In Delaware, Ms.~O'Donnell's
victory touched off a new round of recriminations among Republicans over the direction of their
party, raising the question of whether there was still room for moderates and whether the drive for
ideological purity would cost the party victories in November. The state and national Republican
Party had mounted an aggressive campaign to defeat Ms.~O'Donnell, but it fell short, with Mr.~Castle
unable to rely on independent voters who have long formed his base of support.

``The voters in the Republican primary have spoken, and I respect that decision,'' Mr.~Castle said,
addressing crestfallen supporters who gathered in Wilmington. ``I had a very nice speech prepared
here, hoping I would win this race.''

In Maryland, former Gov.~Robert L.~Ehrlich Jr.~won the Republican nomination for governor,
positioning him for a rematch with Gov.~Martin O'Malley, a Democrat who defeated him four years ago.
Mr.~Ehrlich defeated Brian Murphy, an investment executive, who was endorsed by Ms.~Palin.

In Wisconsin, Scott Walker, the Milwaukee County executive, won the Republican nomination for
governor. He defeated Mark Neumann, a former congressman, and will face Mayor Tom Barrett of
Milwaukee, a Democrat, in November.

The contests on Tuesday night were the last big cluster in a seven-month string of primaries that
will come to an end when Hawaii votes on Saturday and Louisiana holds a runoff early next month.
Seven members of Congress had already been defeated in their bids for re-election.

In Delaware, O'Donnell supporters who gathered at an Elks lodge in Dover began chanting ``Christine!
Christine!'' as returns began to trickle in and her lead steadily climbed. A little more than an
hour after the polls closed, the race was called for Ms.~O'Donnell.

In an interview, Ms.~O'Donnell said she felt confident that she would have the support of Democrats
and independents (neither group could vote in Delaware's closed Republican primary). If elected in
November, she said, she would ``work to repeal the health care bill.''

Throughout the campaign, Ms.~O'Donnell was dogged by reports -- many of them generated by members of
her own party -- that she had trouble with personal finances, had fudged her educational history and
was not fit for office. But Ms.~O'Donnell continued to rebut, repudiate and push on, with a hefty
dose of help from the Tea Party infrastructure and rank-and-file voters who were furious at
Washington

``I think she's going to make it,'' said Marie Bush, a supporter of Ms.~O'Donnell who went to her
victory rally to cheer her on. ``Too many people have been slinging mud at her, and she's a
survivor.''

Asked what the candidate might do to attract independents or even Democrats, Ms.~Bush said, ``I
think people are smart enough now to know the world we are living in is going wrong and we need
people like her to make it right.''

Republicans had been counting the Delaware seat, which was vacated by Vice President Joseph R.~Biden
Jr., as among those they believed they could use to reach a majority in the Senate. Party
strategists said on Tuesday evening that they would assess the race this week, but that they would
likely direct their money elsewhere -- a sign that they believed that Ms.~O'Donnell could not
prevail in a general election. The Democratic nominee for the seat is Chris Coons, the county
executive in New Castle County.

``There's just a lot of nutty things she's been saying that just simply don't add up,'' Karl Rove,
the Republican strategist, said in a television interview on Fox News. ``I'm for the Republican, but
I've got to tell you, we were looking at eight to nine seats in the Senate. We're now looking at
seven to eight. In my opinion, this is not a race we're going to be able to win.''

In New Hampshire, voters trickled into polling places for much of the day, with many precincts
reporting average or lighter-than-expected turnout. Slow returns delayed the outcome, and the
returns by Wednesday morning suggested the race was too close to call.

Mr.~Lamontagne, 52, is a lawyer in Manchester who has French-Canadian roots and is deeply involved
with the Catholic Church. He is a fiscal and social conservative who opposes same-sex marriage and
abortion; Democrats have consistently labeled him as ``too extreme'' for New Hampshire. Over the
course of the campaign, Mr.~Lamontagne won straw polls at Tea Party events by large margins.

He ran for governor in 1996, defeating the more moderate party favorite in the Republican primary
but losing to Jeanne Shaheen, a Democrat who was then a state senator, in the general election.
Before that, Mr.~Lamontagne served as chairman of the New Hampshire Board of Education for three
years.

While Ms.~Ayotte won Ms.~Palin's seal of approval, Mr.~Lamontagne secured two other valuable
endorsements: that of The Union Leader, a newspaper in Manchester, in late August, and that of
Mr.~DeMint days before the primary.

\pagebreak
\section{Eagles' Handling of Head Injury Draws Spotlight}

\lettrine{O}{n}\mycalendar{Sept.'10}{16} Sunday afternoon, more than 28 million people were watching
Fox's national broadcast when the Philadelphia Eagles' Stewart Bradley rose woozily, stumbled and
then collapsed onto the turf. The Fox announcers Joe Buck and Troy Aikman expressed concern and even
horror. Players waved frantically for medical assistance.

Less than four minutes later, Bradley, a linebacker, was sent back into the game.

Only at halftime was his injury diagnosed as a concussion.

The Eagles said afterward that they did not permanently remove Bradley at the time of his injury --
per new N.F.L. rules -- because their sideline exam revealed no concussion and also because no
medical person saw either the hit Bradley took or his collapse to the turf.

Considering that doctors and trainers are well represented on N.F.L. sidelines and that the league
has made concussion awareness an issue this season, the Eagles' handling of Bradley's injury raises
a stark question: If a concussion this glaring can be missed, how many go unnoticed every fall
weekend on high school and youth fields, where the consequences can be more serious, even fatal?

According to the National Athletic Trainers' Association, only 42 percent of high schools in the
United States have access to a certified athletic trainer, let alone a physician, during games or
practices. In some poorer rural communities, concussed players are taken to doctors with no
experience with head injuries. Youth leagues with players as young as 8 and 9 rarely, if ever, have
any medical personnel on hand; when a child is hurt, a parent, assuming one is present, walks out on
the field, scoops up the child and carries him or her off.

The cost of hidden head trauma among children was driven home Monday, also in Philadelphia, as a
University of Pennsylvania lineman who hanged himself in April, Owen Thomas, was found to have died
with the same progressive brain disease found in more than 20 N.F.L. players. Playing since age 9,
Thomas never had a reported concussion; his disease silently developed either through injuries he
did not report or by thousands of subconcussive blows that accumulated over time.

Research suggests that 10 percent to 50 percent of high school football players will sustain a
concussion each season, with as many as 75 percent of those injuries going unreported and unnoticed.

``Here in Rhode Island we have a state law that an athletic trainer must be at contests, but most
schools are in violation,'' Dr.~John P.~Sullivan, the University of Rhode Island's sports
psychologist, wrote in an e-mail Tuesday. ``The risk is real.''

Dawn Comstock of Nationwide Children's Hospital in Columbus, Ohio, is the nation's principal
researcher of injuries among all high school athletes, having overseen the collecting of data that
suggest about 70,000 concussions occur each year in high school football. Those that are reported,
that is.

``We have very little about what happens to high school brains during these hits,'' Comstock said.
``We have no idea at all what's happening in kids' brains while they're on the youth field or
community rec field.''

There have been improvements in the three years that concussions have received national attention.
More than a dozen states have passed laws requiring education for coaches and requiring clearance
from an appropriate medical professional before a child is allowed to return to his or her sport.
(The laws often cover only public schools, however.) At Norman High School in Oklahoma last month,
when a sophomore walked into the coach's office and asked if he could try out for the team, within
15 seconds he was handed a two-page information sheet regarding concussions that he and his parents
had to sign before he could play.

``That's new this year,'' the coach, Greg Nation, said. ``It's really changed.''

Acknowledging the league's impact on young athletes, the N.F.L. asked a skeptical Congress and
public to view its protocol changes last year as proof of its commitment to lead concussion
awareness efforts.

N.F.L. players now must be removed for the rest of the day after a concussion is diagnosed; an
independent doctor must clear the player before he can return; and a new poster warns players of
head injuries with stunningly strong language. That placard even concludes, ``Young Athletes Are
Watching.''

Yet, when the entire football world saw the Eagles put Bradley at significant safety risk by not
properly diagnosing his concussion, it only emphasized the crisis that exists in high school and
youth football, where almost no one is watching at all.

Last year, the N.F.L. requested and received praise for producing the first public-service
announcement geared toward educating young players about the dangers of concussions. This week it
has delivered a different, less scripted, message.

\pagebreak
\section{Deutsche Telekom Is Focus of Corruption Investigation}

\lettrine{G}{erman}\mycalendar{Sept.'10}{16} prosecutors on Wednesday confirmed that they had opened
an investigation to determine whether executives at Deutsche Telekom, including the current chief
executive, pressured government officials in Macedonia to keep competitors out of the market.

Jan van Rossum, the assistant prosecutor in Bonn, said his office took up its investigation at the
request of the U.S.~Justice Department and the U.S.~Securities and Exchange Commission. American
officials have been investigating Deutsche Telekom's dealings in the former Yugoslavia since 2006,
and provided documents to prosecutors in Bonn this summer, he said. Deutsche Telekom shares were
listed in the United States during the period under investigation and therefore subject to the
country's anti-corruption laws.

In August, German investigators searched the homes and offices of the chief executive of Deutsche
Telekom, Ren\'e Obermann, and of several former board members, Mr.~van Rossum said, declining to
identify the other people. The investigators took files and other documents from the offices, he
said.

``This is a probe to determine if there is a basis for further investigation and charges,'' Mr.~van
Rossum said during an interview. ``We are required by law to check out whether these allegations are
true or whether they can be dismissed.''

Deutsche Telekom said Mr.~Obermann denied wrongdoing and was cooperating fully with the
U.S.~authorities.

The inquiry covers the period from 2000 to 2005, when units of the Hungarian telecommunications
company, Magyar Telekom, made \texteuro31 million, or about \$40 million at current exchange rates,
in improper payments to consulting companies and lobbying firms in Macedonia and Montenegro,
according to Deutsche Telekom's own investigation. Magyar Telekom owns 51 percent of Makedonski
Telekom in Macedonia as well as 77 percent of Telekom Montenegro.

Deutsche Telekom owns 59.8 percent of Magyar Telekom.

At the time, both Macedonia and Montenegro were considering opening their telecommunications markets
to more competition, which would have increased the number of rivals faced by Deutsche Telekom in
the Balkans.

Mr.~Obermann became chief executive of Deutsche Telekom, in which the German government owns a 31.7
percent stake, in November 2006. At the time covered by the investigation, he was chief executive of
T-Mobile International, the company's mobile unit.

In a statement, Deutsche Telekom said German prosecutors were investigating whether Mr.~Obermann in
2005 had threatened to withhold dividend payments at Makedonski Telekom unless government officials
stopped plans to open the market to more competitors.

``The chief executive of Deutsche Telekom rejects these allegations as false,'' the company said.
``Deutsche Telekom does not tolerate corruption in any part of its of its business.''

The company, which is based in Bonn, said Mr.~Obermann had testified several times in the continuing
proceedings. Deutsche Telekom said Mr.~Obermann had been asked to appear before U.S.~investigators
as a witness, not as a suspect, in the case. ``D.T. has fully cooperated with the
U.S.~investigation,'' it said. ``Mr.~Obermann has never been a subject or target of the
investigation.''

A person close to Deutsche Telekom, who refused to be identified, said that U.S.~officials, after
four years of studying the case, were getting ready to close their investigation. But Mr.~van Rossum
said that his office had not received any such indication from U.S.~investigators.

An internal review of the situation made by the law firm White \& Case on behalf of Deutsche Telekom
concluded last December that employees at Magyar Telekom and its Macedonian and Montenegrin
subsidiaries had made \texteuro31 million in improper payments to 20 local lobbying and consulting
companies, including some incorporated in Cyprus.

The report found that unidentified employees at the companies had destroyed documentation regarding
the payments in 2006. But it did not uncover any direct payments to Macedonian government officials
or involvement by senior executives at Deutsche Telekom.

\pagebreak
\section{Twitter Revamps Its Web Site}

\lettrine{T}{witter}\mycalendar{Sept.'10}{16} unveiled a new Web site on Tuesday that it hopes will
be user friendly.

The redesigned site, which will be available to all users in the next few weeks, makes it simpler to
see information about the authors of Twitter posts, conversations among Twitter users, and the
photos and videos that posts link to.

``It's going to increase the value that people are getting out of Twitter, so in less time you can
get more information and value,'' Evan Williams, Twitter's co-founder and chief executive, said in
an interview. He had the idea for the redesign and has spent much of his time in the last few months
working with Twitter engineers on the site. He has said he was surprised that so many people use the
service -- 160 million -- given how difficult its Web site is to navigate.

That large audience is appealing to advertisers, but the unappealing Web site has not been a
welcoming place for them. Twitter, which has raised \$160 million in venture capital, has slowly
started to run ads called Promoted Tweets that people see when they search the site. Mr.~Williams
said the new site would improve ads ``because there's going to be more real estate and more
engagement.''

Twitter's new Web site could threaten the many start-ups that build apps, like TweetDeck, Brizzly
and Seesmic, to make Twitter easier to use and to provide users with more sophisticated tools.

Even though 78 percent of Twitter's unique users gain access to the service through its Web site,
the site has had some major flaws. Twitter has not been able to funnel resources into redesigning
the site until now, Mr.~Williams said, because the company has had trouble keeping up with its
growth, even struggling to keep its Web site from crashing.

If people want to learn more about the author of a post, for instance, they must go to a new page.
It has been almost impossible to follow a conversation between two Twitter users. And while a
quarter of the posts contain links, if people post a link to a photo, readers have not been able to
see the picture without going to a new site.

On the new Twitter Web site, people see two panes instead of a single timeline of posts. The
timeline stays in the left pane. In the right pane, they can see more information about posts --
like biographies of authors, photos and videos to which posts link -- and conversations that spring
from a particular post. This eliminates the need to click back and forth.

Mr.~Williams said the new site was not designed for the sake of advertisers, but the experience of
viewing ads would improve. For example, when a movie studio puts out a ``sponsored tweet'' for a
film with a link to the trailer, users will be able to see the trailer without leaving the Twitter
site.

Borrowing an idea from image searching on Google and Bing, Twitter now shows a continuous stream of
posts so people do not need to click ``more'' to view additional posts.

Twitter, which was founded in 2006, has grown so quickly in part because it opened its technology to
thousands of software developers outside the company who have built Twitter tools.

But as Twitter has matured, it has angered many of those developers by building similar tools itself
or acquiring the start-ups that built the tools, limiting opportunities for competing app
developers. Last spring, for example, it bought Atebits, which made a Twitter iPhone app called
Tweetie, and turned it into the official Twitter for iPhone app.

Other start-ups worry that if Twitter builds its own tools, they will go out of business. Indeed,
mobile Twitter users now reach Twitter through the company's own iPhone and BlackBerry apps more
than through any others, according to the company.

``We've made it pretty clear that we are going to create the best experiences we can with all our
clients,'' Mr.~Williams said. ``We made it clear to developers that it's great for everyone if we
make it as good as possible, because that will create more successful Twitter users.''

\pagebreak
\section{Russians Embrace Yoga, if They Have the Money}

\lettrine{O}{ne}\mycalendar{Sept.'10}{16} hallmark of the yuppie lifestyle adopted by Russians lucky
or talented enough to afford it is a fondness for yoga and many things Indian. For this crowd, Goa
is a popular vacation spot, and Indian clothes, furniture and food are necessary accoutrements.

This week, which has been designated Yoga Week in the country, Russians also got a guru: Sri Sri
Ravi Shankar, who went on a whirlwind tour visiting St.~Petersburg; Moscow; Kazan; Irkutsk; Sochi
and the site of a new ashram in nearby Tuapse.

The confluence of Mr.~Shankar's philosophy and Russian society turned up some incongruities. At a
seminar called ``Ethics in Business'' at the Ritz-Carlton, among Moscow's most expensive hotels,
Mr.~Shankar told a ballroom full of well-dressed people, who paid 5,000 rubles, or close to \$200, a
ticket, about Vedic philosophy and the spiritual subtext of corruption.

``Corruption begins outside the purview of belongingness,'' he said, in response to a question about
how to battle corruption, so endemic in Russia that President Dmitri Medvedev and Patriarch Kirill
of the Russian Orthodox Church regularly inveigh against it.

The only way to overcome corruption, he said, was ``to reorient people, educate them,'' adding ``the
governments, religious bodies, NGO's, business, all of them have to work together.''

Yoga, which was officially taboo in Soviet times but retained an underground following, has been
embraced by Russia's elite. In 2007, shortly before he became president, Mr.~Medvedev told Itogi
magazine that he was ``mastering yoga,'' as one activity that helps him deal with the stress of
political obligations.

That immediately led to speculation that yoga would become a national pastime, as judo has under
Vladimir Putin, a black belt in the sport who is regularly photographed displaying his mastery.

Mr.~Medvedev has not been photographed in the lotus position. But, in Moscow at least, yoga studios
have become almost as ubiquitous as coffee shops and sushi bars, and yoga is an essential part of
elite health clubs.

Mr.~Shankar's Art of Living Foundation, which was started with the vision of creating a stress-free
society, has its Russian headquarters in a Moscow business center. The organization, saying it
sought to help to alleviate stress, sent instructors to North Ossetia to work with victims of the
school hostage-taking in Beslan in 2002, and to Tskhinvali in South Ossetia after Russia's war with
Georgia in 2008, and it has also worked with the Russian military.

On Sunday, Mr.~Shankar drew about 1,000 people to Luzhniki Stadium for a meditation session. It did
not compare to the tens of thousands who came to see Bono and U2 recently at the same venue, but the
attendance was sizable for a rainy September morning, with tickets ranging from 1,500 to 3,000
rubles.

Retirees and students got a 50 percent discount, but there were also Louis Vuitton bags and Burberry
blankets among the yoga mats. Followers were offered the opportunity to join Mr.~Shankar for a river
cruise in the evening, at 5,000 rubles for a full-price ticket.

Some forms of yoga are regarded as dangerous sects by the Russian Orthodox church, which also warns
that sects hide behind good deeds, but there were no widely publicized protests over Mr.~Shankar's
tour from either the church or Muslim leaders in Kazan, capital of Tatarstan in central Russia.
Vissarion, a Siberian cult leader who was once a traffic policeman but now calls himself Jesus
Christ, was welcomed by Mr.~Shankar at his ashram near Bangalore, India in 2008, which was noted
with concern by cult watchers in Russia.

At Luzhniki Stadium, Lena Savina, a 27-year-old hairdresser, said she had changed for the better
since she took up yoga 18 months ago, following her mother's example.

``Those around me really feel and see this,'' she said. ``It helps at work, in the family, in
relations with friends. It teaches such discipline, to control emotions. We are very subject to
emotions.''

Ms.~Savina said it helped people avoid misunderstandings at work. ``You need to breathe, calm down,
and move on, so this doesn't happen.''

Of Mr.~Shankar, she said: ``It is rare for such an enlightened person to visit Russia.''

Margarita Zakarina, 47, from the city of Ufa in Bashkortostan, said the Art of Living movement saved
her after a fire destroyed her apartment and killed her husband. ``I didn't want to live,'' she
said. ``After the course, I understood that life is given once.''

Stanislav Vintslav, 49, a lawyer, took up yoga a year ago and credited it for being able to quit
smoking. He bought a discounted ticket to the Luzhniki event at the recommendation of his
instructor, but was angry that Mr.~Shankar made only a 45-minute appearance, and doubts that yoga or
deep breathing could cure Russia's larger ills. ``I don't have illusions,'' he said.

Mr.~Shankar also cut short his visit to Y Club, a new yoga and lifestyle center opened by his
followers in a basement in one of central Moscow's most fashionable areas. He raced through the
grunge-chic basement, with exposed brick walls, dim lighting, Indian music and scattered rose
petals, cringing visibly when asked by Tatiana Gevorkian, a former Russian MTV host who was the
emcee for the event, about being rated as one of the five most influential people in India, by
Forbes magazine. (Russians are obsessed with the Forbes billionaire rankings, which are filled with
Russian oligarchs.) ``Love is the greatest wealth,'' Mr.~Shankar said, before leaving because of an
allergic reaction to construction dust as final touches are added to the new center, according to
Natalia Sukhomlinova, who handles public relations for the Y Club.

\pagebreak
\section{Once Wary, Obama Relies on Petraeus}

\lettrine{W}{hen}\mycalendar{Sept.'10}{17} President Obama descended into the White House Situation
Room on Monday for his monthly update on Afghanistan and Pakistan, the new top American military
commander, Gen.~David H.~Petraeus, ticked off signs of progress.

Come December, when the president intends to assess his Afghan strategy, he will be able to claim
tangible successes, General Petraeus predicted by secure video hookup from Kabul, according to
administration officials.

The general said that the American military would have substantially enlarged the ``oil spot'' --
military jargon for secure area -- around Kabul. It will have expanded American control farther
outside of Kandahar, the Taliban heartland. And, the aides recalled, the general said the military
would have reintegrated a significant number of former Taliban fighters in the south.

``He essentially promised the president very bankable results,'' one administration official said.
(Others in the room characterized the commander's list more as objectives than promises.) Mr.~Obama
largely listened, asking a few questions, and two hours later, the White House sent an e-mail to
reporters using language that echoed the general's.

But even inside an administration that is pinning its hopes, both military and political, on the
accuracy of the general's report, there are doubters. Assessments from intelligence officials are
far more pessimistic, and Mr.~Obama regularly reviews maps that show how the Taliban have spread
into areas where they had no major presence before.

And some military officers, who support General Petraeus's counterinsurgency strategy and say he
readily acknowledges the difficulties ahead, caution that the security and governance crisis in
Afghanistan remains so volatile that any successes may not be sustainable.

How that tension plays out in coming months -- the guarded optimism of a popular general leading an
increasingly unpopular war, and the caution of a White House that prides itself on a realism that it
says President George W.~Bush and his staff lacked -- will probably define the relationship between
Mr.~Obama and his field commander. General Petraeus, who led the Iraq surge and was a favorite of
Mr.~Bush, has slowly worked himself into the good graces of a president who was once wary of him.

So far, the two men appear to be meshing well, advisers say. The men ``are actually somewhat similar
in temperament and style,'' said Benjamin Rhodes, the National Security Council's director of
strategic communications. Both are meticulous, even-keeled and matter of fact, and both like to do
their homework, studying detailed reports.

Since General Petraeus took on the commander's job in June, several aides said, the president has
struck a more deferential tone toward him than he used with Gen.~Stanley A.~McChrystal, General
Petraeus's predecessor. Often during pauses in meetings, one White House official said, Mr.~Obama
will stop and say, ``Dave, what do you think?''

Like no other figure today, General Petraeus has stepped into Gen.~Colin L.~Powell's shoes as the
face of the military to ordinary Americans, particularly as the White House extols the end of the
combat mission in Iraq, which was largely made possible by the troop surge that General Petraeus
orchestrated.

For Mr.~Obama, that may be a blessing and a curse. General Petraeus has made clear that he opposes a
rapid pullout of troops from Afghanistan beginning next July, as many of the president's Democratic
allies would like. Some in the White House, with an eye on the 2012 presidential election, fear that
the general may already be laying the foundation for keeping a large force in Afghanistan for a long
while.

Defense Secretary Robert M.~Gates said Thursday that the unresolved question was whether the
``campaign plan'' for Afghanistan was working.

``The evidence that General Petraeus is seeing so far suggests to him that it is, and both on the
civilian and the military side, not just the military side,'' Mr.~Gates told reporters. ``But he is
cautious, and I will be cautious.''

The new alliance between Mr.~Obama and General Petraeus holds risks for the general as well as the
president. In taking on Afghanistan, he is risking his reputation as perhaps the greatest general of
his generation on a war that many people think will end in a stalemate. Even if General Petraeus's
strategy is a solid one, few believe Mr.~Obama will commit the time and resources -- many years and
hundreds of billions of dollars -- needed to test the Petraeus thesis.

General Petraeus has a history of early optimistic assessments that proved largely correct; one
dates back to the Iraq surge, over which he and Mr.~Obama first butted heads. Military officials say
that during the early days of the surge, General Petraeus cited what his staff termed ``leading
indicators'' of progress, even when much of the private and public discussion of the war effort was
still negative. (During one Senate hearing with General Petraeus, then-Senator Obama accused the
Bush administration of setting ``the bar so low that modest improvement in what was a completely
chaotic situation'' was considered success.)

While General Petraeus's track record in Iraq may give added weight to his analysis on Afghanistan,
the two wars are radically different in Mr.~Obama's mind, his aides said. During meetings at the
White House, the general ``always brings up Iraq,'' one senior administration official said.

While Mr.~Obama asked General Petraeus last fall to assemble the lessons learned in the Iraq surge
that could be applied in Afghanistan, the president, by and large, ``remains focused on
Afghanistan,'' the official said.

Some officials would speak only on background about interactions they had witnessed in confidential
meetings.

In preparation for this fall's review of the strategy in Afghanistan, Mr.~Obama's first request of
General Petraeus was for new and better ways to measure success or setbacks; the general presented
them on Monday.

He started with familiar measures: how many Afghan troops have been trained and how many operations
have focused on Taliban strongholds in places like Kandahar and Helmand.

Then General Petraeus added three others: one looking at local security initiatives enacted by the
Afghan police, another at the pace of ``reintegration'' of former members of the Taliban and a third
looking at the successes of attacks by American Special Operations forces.

``These are more specific,'' said one adviser to the president. ``With McChrystal, it was `You'll
know victory when you see it.' The president has asked for a lot more visibility into what's
happening.''

Mr.~Obama gets a wider view from intelligence reports, chiefly from the C.I.A. and the Defense
Intelligence Agency, that land on his desk weekly. They assess whether President Hamid Karzai's
government is preparing to survive on its own, or whether the Taliban can successfully retreat to
their safe haven in Pakistan to prepare new attacks. Those longer-range assessments have been
significantly more pessimistic than General Petraeus's measures of battlefield progress.

Some national security experts say that the fate of General McChrystal -- now on the lecture circuit
making \$60,000 a speech -- and the fired general before him, Gen.~David D.~McKiernan, means
Mr.~Obama must make things work with General Petraeus, lest he appear unable to get along with his
commanders.

``If they have a falling out, it's not at all clear that the public would necessarily side with the
president the way they did in the McChrystal incident,'' said David Rothkopf, a former Clinton
administration official.

Added Leslie H.~Gelb, president emeritus of the Council on Foreign Relations: ``They are joined at
the hip, but the leverage lies with Petraeus. And Petraeus has made plain, publicly, that after July
2011, he doesn't think there should be a rapid pullout.''

\pagebreak
\section{Roma, on Move, Test Europe's `Open Borders'}

\lettrine{T}{his}\mycalendar{Sept.'10}{17} city is full of stark, Soviet-era housing blocks, and the
grimmest among them -- gray towers of one-room apartments with communal bathrooms and no hot water
-- are given over to the Roma population.

Roma like Maria Murariu, 62, who tends to her dying husband in a foul-smelling room no bigger than a
jail cell. She has not found work in five years.

``There is not much for us in Romania,'' she said recently, watching her husband sleep. ``And now
that we are in the European Union, we have the right to go to other countries. It is better there.''

Thousands of Romania's Roma, also known as Gypsies, have come to a similar conclusion in recent
years, heading for the relative wealth of Western Europe, and setting off a clash within the
European Union over just how open its ``open borders'' are.

A summit meeting of European leaders on Thursday degenerated into open discord over how to handle
the unwanted immigrants. President Nicolas Sarkozy of France vowed to keep dismantling immigrant
camps and angrily rejected complaints from European Commission officials that the French authorities
were illegally singling out Roma for deportation.

Migration within the 27 nations of the European Union has become a combustible issue during the
economic downturn. The union's latest expansion, which brought in the relatively poor nations of
Romania and Bulgaria in 2007, has renewed concern that the poor, traveling far from home in search
of work, will become a burden on wealthier countries. The migration of the Roma is also raising
questions about the obligations of Romania and Bulgaria to fulfill promises they made when they
joined the union. Romania, for instance, mapped out a strategy for helping the Roma, but financed
little of it. Mr.~Sarkozy has demanded that the Romanian government do more to aid the Roma at home.

Much of Western Europe has reacted with hostility to itinerant Roma, who often have little education
or practical skills. Some Roma have found marginal jobs collecting scrap iron or painting houses.
But others have signed up for welfare or drifted into begging and petty thievery, living in
unsightly camp sites.

In recent weeks, Mr.~Sarkozy has tried to revive his support on the political right by deporting
thousands of them, offering 300 euros, about \$392, to those who go home voluntarily, and bulldozing
their encampments.

The European Commission has threatened legal action against Paris over the deportation, calling it
disgraceful and illegal.

The dispute peaked at lunch Thursday between Mr.~Sarkozy and Jos\'e Manuel Barroso, the president of
the commission, the European Union's executive body.

``There was a big argument -- I could also say a scandal -- between the president of the European
Commission and the French president,'' said the Bulgarian prime minister, Boyko Borisov, according
to the Bulgarian daily Dnevnik.

Mr.~Sarkozy denied a major rift, and remained unswayed. ``We will continue to dismantle the illegal
camps, whoever is there,'' he said at a news conference. ``Europe cannot close its eyes to illegal
camps.''

Expulsions seem unlikely to offer a long-term solution. Many of the deported Roma are already
planning their return.

Privately, some Romanian officials snicker over the French action. ``They are just giving the Roma a
paid vacation,'' one official said.

Still, advocates for the Roma hope that the latest conflict will force the European Union to get
serious about helping the Roma, who are openly reviled in most Eastern and Central European
countries where they have lived in large numbers for centuries, most often under appalling
conditions.

``There is nothing to focus the minds of policy makers like an army of poor people heading your
way,'' said Bernard Rorke, the director of Roma Initiatives for the nonprofit Open Society
Foundation.

There is little reliable data on the Roma population. Originally from India, the Roma were virtual
slaves until the 19th century, working for aristocrats and in monasteries.

When democracy took hold, they were freed. But they were landless, uneducated and dark-skinned, and
they had few prospects.

Human rights activists say that Roma women are often sent to separate maternity wards. Their
children, when they attend school, are frequently steered into classes for the mentally handicapped.

In Romania, one census counted 500,000 Roma. But some advocates say the number is closer to two
million.

Those who make it out of abject poverty rarely admit their ancestry -- a factor that makes it harder
for Roma to combat the discrimination they face, advocates say.

In the years that Romania was negotiating to get into the European Union, it promised programs to
help the Roma integrate into Romanian society.

But government officials concede that few materialized. ``I think you will see the current
administration do better,'' said Ilie Dinca, the director of the Romanian National Agency for Roma.

Budget cutbacks have hurt the few successful efforts that exist. Hundreds of mediators hired to help
the Roma get their children into school and receive health benefits have been fired recently.

``What you see here these days is terrible conditions,'' said Nicolae Stoica, who runs Roma Access,
an advocacy group. ``They have no hope of getting jobs. If they get 20 euros a month from collecting
scrap metal, that's a lot. How can we tell them not to go to France and beg on the streets?''

Flortina Ghita, 21, said her family once lived in a building in the center of Constanta, Romania's
second largest city. But city officials evicted them, saying the buildings had structural damage.
The family now lives in shacks made of carpets, scraps of corrugated tin and plastic sheeting set up
not far from railroad tracks. The only source of water is a train station more than a mile away.

Mrs.~Ghita said her family had been told to fill out forms to get housing, but no one can read. Her
son, Sorim, 5, is not in school, she said, because she cannot afford the clothing, notebooks and
class fees.

Still, the Ghita family was savvy enough about Europe. Mrs.~Ghita had paperwork showing that her
mother had been to Belgium for medical care. ``Her sister lives there and she helped us,''
Mrs.~Ghita said.

Experts say the Roma population has been battered by a combination of factors. Crafts that once
sustained them, such as making brass pots and shoeing horses, are now obsolete. Recent European
regulations standardizing the sale of livestock pushed them out of one of their few remaining
businesses because they could not handle the required paperwork.

Some aspects of Gypsy culture have not helped matters, experts say. It is a clannish, strongly
patriarchal society where youngsters are pushed into early marriage and education has not been much
valued.

Not all Roma are poor, however. In the village of Barbulesti, about 40 miles northeast of Bucharest,
there are signs of success. The village is a bright cluster of mustard- and ketchup-colored houses,
with gaudy turrets and ornate gutters, many still under construction.

The village has a Roma mayor, Ion Cutitaru, 59, the only one in the country, he says. He estimates
that a third of the village's 7,000 residents have moved to Western Europe. They look for work
there, he says, but beg when they can find nothing else.

``They make do,'' he said, ``and then they come back and build their houses.''

Twenty-eight Roma residents from Barbulesti were recently expelled from France. Among them was Ionel
Costache, 30, who said he would return to France in a week or two.

``My son, who had eye problems, he got a 7,000-euro operation there that he would never have gotten
here. And when you don't have work, you can still eat with their social assistance,'' he said.
``France is a much better place than Romania.''

\pagebreak
\section{In Britain, Pope Criticizes Response to Abuse Crisis}

\lettrine{P}{ope}\mycalendar{Sept.'10}{17} Benedict XVI arrived Thursday in Scotland, offering his
strongest criticism yet of the Roman Catholic Church's handling of the sexual abuse crisis. He said
that church leaders had not been ``sufficiently vigilant'' or ``sufficiently swift and decisive'' in
cracking down on abusers.

While Benedict was received graciously by Queen Elizabeth II in Edinburgh and thousands turned out
for an open-air Mass in Glasgow, the visit was taking place under the dark shadow of the sexual
abuse scandals, which have shaken even the faithful in nearby Ireland, in his native Germany and
elsewhere in Europe.

Protests were planned by atheists and gay and human rights activists incensed by the pope's handling
of the scandals and by others opposed to the church's stance on social issues. Centuries after the
Church of England split from Rome, some Anglicans are wary of the Vatican's recent efforts to draw
traditionalists to Roman Catholicism. That the occasion for the visit is the beatification of
Cardinal John Henry Newman, England's most famous Catholic convert, has only added to their
suspicions.

Ahead of the pope's four-day visit, one of Britain's most prominent Catholic leaders spoke about the
wounds left by the church's failures in the abuse cases.

``The church has made a mess of its response to incidences of child abuse,'' said Archbishop Vincent
Nichols, the head of the Roman Catholic Church in England and Wales. ``There is nothing to be said
to excuse the crimes committed by members of the clergy against children. The damage that is done
strikes at the core of the person: in the capacity to trust another, in their capacity to love
another and -- especially in the context of the church -- in their capacity to believe in God.''

Perhaps mindful of such criticism, Benedict told reporters on his flight from Rome that the church's
``first interest is the victims'' of abuse, and that the church needed to ask, ``How can we repair,
what can we do to help them to overcome the trauma, to refind their lives?''

Responding in Italian to reporters' questions submitted in advance and relayed to him by Vatican
officials, the pope's words marked an evolution in the Vatican's response. In the heat of the crisis
last spring, top Vatican officials at first blamed the news media for stirring it up.

Critics quickly pounced on the statement, calling it evasive and out of touch. In a statement, the
United States-based group Bishopaccountability.org, which tracks abuse cases, said the pontiff's
words ``ring hollow,'' adding that he had said similar things for years with little action.

``In researching this crisis for seven years, we have not found one documented instance before 2002
of a top church official contacting civil authorities to report an allegation of sexual abuse,'' the
group said.

Benedict's visit to Britain comes as part of his sustained effort to counter a perceived loss of
religious belief in Europe and to urge a new struggle against secularism.

Benedict's is the first state visit to Britain by a pope in which he is meeting the queen and
political establishment as a fellow head of state. In 1982, John Paul II paid a pastoral visit to
Britain, but did not meet Prime Minister Margaret Thatcher, and was received by the queen privately.

The pope's first appointment on Thursday was with Elizabeth and Prince Philip at the Palace of
Holyroodhouse, a medieval castle in Edinburgh that is the queen's official residence in Scotland and
a place that figures large in the history of the schisms within Christianity that marked Britain's
evolution as a nation.

It was at Holyroodhouse that Mary Queen of Scots lived during her brief reign as the Catholic queen
of Scotland, only to be executed in 1587 by Henry VIII's daughter Queen Elizabeth I of England.
Henry had broken with Rome earlier in the 16th century, provoking centuries of anti-Catholic
passions that linger still in parts of Britain.

Benedict said he was eager to visit a society often critical of the church. ``Naturally, Great
Britain has had a history of anti-Catholicism as we all know, but also a history of great
tolerance,'' he told reporters.

In Scotland, crowds were not as tumultuous as those that had greeted John Paul -- British news
reports said many tickets for papal events during Benedict's visit remained unclaimed -- but the
mood for the pope's arrival was upbeat.

People lined the streets in Edinburgh as the papal motorcade passed, many of them waving the
Scottish flag and cheering. Inside his vehicle, with a blue-and-green Scottish tartan scarf draped
over his white papal robes, the pope smiled broadly as he made the sign of the cross.

Benedict used his visit with the queen -- the formal head of the Church of England, a church whose
relationship with Roman Catholicism remains uneasy -- to evoke what he depicted as Britain's drift
from Christianity, saying the country should ``not obscure the Christian foundation that underpins
its freedoms.''

``Even in our own lifetime, we can recall how Britain and her leaders stood against a Nazi regime
that wished to eradicate God from society and denied our common humanity to many, especially the
Jews, who were thought unfit to live,'' the pope said in English, speaking as Britons mark the 70th
anniversary of the Battle of Britain, a turning point of World War II.

He also cited the ``Nazi tyranny'' as an example of ``the sobering lessons of atheist extremism in
the 20th century,'' prompting an angry response from the British Humanist Association, one of the
country's leading atheist organizations. ``The notion that it was the atheism of the Nazis that led
to their extremist and hateful views, or that it somehow fuels intolerance in Britain today, is a
terrible libel against those who do not believe in God,'' the group said in a statement.

Last year, the Vatican upset many Anglicans when it announced a fast-track conversion to Catholicism
aimed at Anglican traditionalists uncomfortable with that church's acceptance of female priests and
openly gay bishops. (So far, it seems, few Anglicans have accepted the offer.)

On Friday, Benedict is expected to meet with the archbishop of Canterbury, Rowan Williams, and the
two are to participate in a rare ecumenical service in Westminster Abbey, where the pope is expected
to deliver the central speech of his visit.

\pagebreak
\section{Senate Panel Approves Arms Treaty With Russia}

\lettrine{P}{resident}\mycalendar{Sept.'10}{17} Obama's arms control treaty with Russia advanced to
the Senate floor with bipartisan support on Thursday, giving it a major boost toward ratification
despite the election-year polarization that has divided the parties over so many other issues.

The Senate Foreign Relations Committee voted 14 to 4 to approve the treaty known as New Start, with
three Republicans joining Democrats after negotiating an accompanying resolution addressing
conservative concerns about missile defense and modernization of the nuclear arsenal.

The vote was a rare instance in which Mr.~Obama has won more than token Republican support for a
signature initiative. But he still faces a battle to secure final approval on the Senate floor,
where under the Constitution the treaty needs a two-thirds vote, meaning at least eight Republicans.
With many Republicans still opposed, Democrats are likely to delay a floor vote until a lame-duck
session after the election on Nov.~2.

Mr.~Obama considers the treaty one of his most tangible foreign policy achievements and the
centerpiece of his effort to rebuild relations with Russia after years of tension.

Signed by Mr.~Obama and President Dmitri A.~Medvedev in Prague in April, the treaty would bar each
side from deploying more than 1,550 strategic nuclear warheads or 700 launchers starting seven years
after final ratification.

Perhaps just as significantly, it would establish a new inspection and monitoring regime to replace
the longstanding program that lapsed last year with the expiration of the first Strategic Arms
Reduction Treaty of 1991, or Start. The nine months since the end of that treaty have been the only
period since the cold war when the two nuclear powers did not have a system for exchanging
information and allowing inspectors on the ground.

``This is an historic vote that renews the bipartisan tradition that's vital in tackling the grave
threat posed by nuclear weapons,'' said Senator John Kerry, Democrat of Massachusetts and the
committee chairman. ``This bipartisan vote sends an important signal that even in the most partisan,
polarized season, ratifying this treaty is not a matter of politics. It's a national security
imperative.''

In an interview afterward, Mr.~Kerry expressed confidence that enough Republicans would join to pass
the agreement on the floor, but acknowledged that it was unlikely to receive the overwhelming vote
that past arms control treaties had received. ``I hope that we can ratify it by the end of the
year,'' he said. ``In today's world, if we get about 70 or 70-plus, it would be a very big
victory.''

Mr.~Kerry worked closely with the White House and his Republican counterpart, Senator Richard
G.~Lugar of Indiana, to shape a resolution of ratification that would bring over several
Republicans. Mr.~Lugar offered a final draft dealing with concerns of his fellow Republicans,
winning the support of Senators Bob Corker of Tennessee and Johnny Isakson of Georgia.

``Rejecting this treaty would inhibit our knowledge of Russian military capabilities, weaken our
nonproliferation diplomacy worldwide and potentially reignite expensive arms competition that would
further strain our national budget,'' Mr.~Lugar said.

But dissenting Republicans contended that the treaty would undercut national security by giving
Russia a tool to fight against American plans to build a missile defense system in Europe. They
argued that Russia has skirted the requirements of other treaties in the past and could not be
trusted, and they contended that the treaty would potentially limit new conventional missile
programs.

``Under this treaty, the U.S.~allows limits on missile defense and conventional prompt global
strike, while accepting weakened verification measures,'' Senator Jim Risch of Idaho said. ``It is
unclear what concessions were made by Russia.''

The resolution of ratification offered by Mr.~Lugar with Mr.~Kerry's support would not alter the
treaty itself, but would set out the Senate's understanding of what the pact means, a common way for
lawmakers to put their stamp on a treaty without reopening negotiations.

Among other things, it reaffirmed that the treaty imposes no limitations on missile defense beyond a
clause barring the United States from using old intercontinental missile silos or submarine
launchers for antimissile interceptors.

It likewise made clear that the treaty would not prevent the United States from developing systems
to use long-range missiles with conventional rather than nuclear warheads.

During a break in the panel's deliberations on Thursday, Mr.~Kerry, Mr.~Lugar and administration
officials huddled in a back room with one of the treaty's most outspoken Republican critics, Senator
Jim DeMint of South Carolina. They agreed to add additional language he wanted expressing the desire
to move beyond mutually assured destruction as a nuclear strategy.

In the end, Mr.~DeMint missed the final committee vote, and his office did not respond to a request
for comment. His fellow Republicans -- Mr.~Risch and Senators John Barrasso of Wyoming, Roger Wicker
of Mississippi and James M.~Inhofe of Oklahoma -- voted against the treaty.

Mr.~Lugar's resolution also laid out a commitment to modernize the nation's nuclear weapons complex,
a top Republican priority, although it is not binding. The Obama administration has proposed a
10-year plan to upgrade the nuclear complex, but Republicans led by Senator Jon Kyl of Arizona have
been pushing to lock in the commitment as much as possible and make sure enough money will be
available.

Mr.~Kyl, the Senate's second-ranking Republican, had no comment on Thursday. Mr.~Corker made clear
this week that while he was voting yes in committee, he still wanted a firmer commitment on
modernization money before supporting the treaty on the floor. That suggests several more weeks of
negotiations to find an accommodation.

\pagebreak
\section{U.S.~Steps Up Criticism of China's Practices}

\lettrine{T}{he}\mycalendar{Sept.'10}{17} Obama administration increased its criticism of China's
economic policies on Thursday, as Treasury Secretary Timothy F.~Geithner told Congress that China
had substantially undervalued its currency to gain an unfair trade advantage, tolerated theft of
foreign technology and created unreasonable barriers to American imports.

But the election year anger from lawmakers seemed to surpass even Mr.~Geithner's tougher posture.
Lawmakers expressed impatience with the administration's familiar reliance on persuasion and
negotiation, saying such tactics had yielded little.

In Beijing, a spokeswoman for the Foreign Ministry said that China would not respond to pressure and
that a revaluation of the currency, the renminbi, would do little to affect the United States trade
deficit with China. But the renminbi strengthened by 0.27 percent to end trading at 6.72 per dollar
Thursday as the government appeared to belatedly permit the greater currency flexibility it had
promised in June.

Dismay over China's currency interventions -- it buys about \$1 billion a day to maintain the
renminbi's peg to the dollar -- has been a recurring theme for years. The election-season rhetoric,
the carefully calibrated strengthening of the Chinese currency on the eve of Mr.~Geithner's
appearance, and the administration's struggle to negotiate a diplomatic line set the stage for
predictable political theater.

But now, with the United States in a stalled economic recovery and lawmakers facing a restive
electorate, the administration is clearly looking for alternative ways to bring pressure on the
Chinese.

Mr.~Geithner urged China to allow ``significant, sustained appreciation'' of its undervalued
currency and even suggested that anything less would strain the relations of the world's two largest
economies. He made it clear that President Obama would press the issue with China's leaders, giving
rise to a potentially pivotal moment in November when leaders of the Group of 20 economic powers
meet in South Korea.

Still, Mr.~Geithner's plan did not appear to mollify lawmakers.

``There is no question that the economic and trade policies of China represent clear roadblocks to
our recovery,'' Senator Christopher J.~Dodd, Democrat of Connecticut and the chairman of the Banking
Committee, told Mr.~Geithner at a hearing. ``I've listened to every administration, Democrats and
Republicans, from Ronald Reagan to the current administration, say virtually the same thing,
producing the same results. China does basically whatever it wants, while we grow weaker and they
grow stronger.''

Mr.~Dodd, who is not seeking re-election, added: ``It's clearly time for a change in strategy.''

Successive administrations have declined to formally designate China a currency manipulator -- a
finding that could initiate American retaliatory measures -- something that has frustrated
lawmakers.

``There is no question that China manipulates its currency in order to subsidize Chinese exports,''
said the top Republican on the committee, Senator Richard C.~Shelby of Alabama. ``The only question
is, Why is the administration protecting China by refusing to designate it as a currency
manipulator?''

Pointing to Mr.~Geithner, Charles E.~Schumer, Democrat of New York, said: ``I'm increasingly coming
to the view that the only person in this room who believes China is not manipulating its currency is
you.''

Mr.~Geithner avoided using any version of the word ``manipulation,'' making it clear that he thought
such a finding would only antagonize the Chinese without having much practical effect beyond
requiring American officials to engage in talks -- something the administration has already been
doing.

Partly in response to American pressure, China permitted the renminbi to rise about 20 to 25 percent
from 2005 to 2008, but then stopped the currency from strengthening as the financial crisis
threatened the strength of its export-oriented economy.

C. Randall Henning, a political scientist in the School of International Service at American
University in Washington and an authority on exchange-rate policy, said it was in China's interest
to allow the renminbi to rise in value. The currency interventions have raised the threat of
inflation and asset bubbles, hurt the poorer sectors of the country's economy and depressed domestic
consumption.

But export interests dominate policy-making within the Chinese government, and consumers have little
voice within the communist regime, so ``we are likely to see policy shift toward appreciation only
when price stability, financial stability or exports are threatened,'' Mr.~Henning said.

Business interests in the United States are sharply divided on the currency issue. Domestic
manufacturers and labor unions say that China's currency policies have eviscerated industrial
employment. But large multinational companies, particularly those with extensive production
facilities in China, benefit from a weak renminbi just as Chinese manufacturers do.

Those companies, along with Wall Street firms, are fearful -- as Mr.~Geithner acknowledged -- that
excessive pressure could lead China to retaliate against their operations in China. In the worst
case, they are worried that the two countries could get mired in a trade war.

A House bill with more than 140 sponsors would virtually compel the administration to find China to
be a manipulator and impose duties or other trade barriers in retaliation. Mr.~Geithner does not
support the proposal, which the administration says would violate United States obligations as a
member of the World Trade Organization.

Mr.~Geithner said the Treasury would ``take China's actions into account'' in preparing the
administration's next foreign exchange report, which is due to Congress on Oct.~15 but will probably
be delayed, as previous reports have been, as officials continue to talk to the Chinese.

The secretary also laid out other concerns about China's policies, including ``indigenous
innovation,'' a set of practices that American officials say result in discrimination against
foreign products and technology.

The secretary attacked what he called ``rampant'' violations of intellectual property rights and an
``unacceptable'' level of theft of technology.

He also criticized a proposal by China to require that certain products be accredited before being
sold to its government. The United States says that such requirements might violate standards of the
W.T.O., which China joined in 2001.

Mr.~Geithner pledged that the administration would be ``aggressively using the full set of trade
remedies available to us,'' including filing new cases with the W.T.O. The United States trade
representative's office filed two such cases on Wednesday.

He said the administration was ``reviewing carefully'' a complaint by the United Steelworkers union
over Chinese practices in the renewable energy sector.

\pagebreak
\section{Teaching Doctors About Nutrition and Diet}

\lettrine{W}{ithin}\mycalendar{Sept.'10}{17} days of being accepted into medical school, I started
getting asked for medical advice. Even my closest friends, who should have known better, got in on
the action.

``Should I take vitamins?''

``What do you think of this diet?''

``Is yogurt good for me or not?''

Each and every time someone posed such a query, I became immediately cognizant of one thing: the big
blank space in my brain. After all, even with medical school acceptance in hand, I was no more a
doctor than they were.

But I also soon realized that many of their questions had nothing to do with medications or
operations, or even diseases. With all the newspaper and television reports about newly discovered
carcinogens and the latest diets and miracle nutrients, what my friends and acquaintances really
wanted to know was just what they should or should not eat.

Years later, as a newly minted doctor on the wards seeing real patients, I found myself in the same
position. I was still getting a lot of questions about food and diet. And I was still hesitating
when answering. I wasn't sure I knew that much more after medical school than I did before.

One day I mentioned this uncomfortable situation to another young doctor. ``Just consult the
dietitians if you have a problem,'' she said after listening to my confession. ``They'll take care
of it.'' She paused for a moment, looked suspiciously around the nursing station, then leaned over
and whispered, ``I know we're supposed to know about nutrition and diet, but none of us really
does.''

She was right. And nearly 20 years later, she may still be.

Research has increasingly pointed to a link between the nutritional status of Americans and the
chronic diseases that plague them. Between the growing list of diet-related diseases and a
burgeoning obesity epidemic, the most important public health measure for any of us to take may well
be watching what we eat.

But few doctors are prepared to effectively spearhead or even help in those efforts. In the
mid-1980s, the National Academy of Sciences published a landmark report highlighting the lack of
adequate nutrition education in medical schools; the writers recommended a minimum of 25 hours of
nutrition instruction. Now, in a study published this month, it appears that even two and a half
decades later a vast majority of medical schools still fail to meet the minimum recommended 25 hours
of instruction.

Researchers from the University of North Carolina at Chapel Hill asked nutrition educators from more
than 100 medical schools to describe the nutrition instruction offered to their students. While the
researchers learned that almost all schools require exposure to nutrition, only about a quarter
offered the recommended 25 hours of instruction, a decrease from six years earlier, when almost 40
percent of schools met the minimum recommendations. In addition, four schools offered nutrition
optionally, and one school offered nothing at all. And while a majority of medical schools tended to
intersperse lectures on nutrition in standard, required courses, like biochemistry or physiology,
only a quarter of the schools managed to have a single course dedicated to the topic.

``Nutrition is really a core component of modern medical practice,'' said Kelly M.~Adams, the lead
author and a registered dietitian who is a research associate in the department of nutrition at the
university. ``There may be some pathologists or other kinds of doctors who don't encounter these
issues later, but many will, and they aren't getting enough instruction while in medical school.''

For the last 15 years, to help schools with their nutrition curriculum, the University of North
Carolina has offered a series of instruction modules free of charge. Initially delivered by CD-ROM
and now online, the program, Nutrition in Medicine, is an interactive multimedia series of courses
covering topics like the molecular mechanism of cancer nutrition, pediatric obesity, dietary
supplements and nutrition in the elderly.

``Physicians have enough barriers trying to provide their patients with nutritional counseling,''
Ms.~Adams said. ``Inadequate nutritional education does not need to be one of them.''

Ms.~Adams and her colleagues believe that the fully developed online curriculum helps address two
issues that frequently arise: the relative dearth of faculty in a medical school with appropriate
expertise and the lack of time in an already packed course of study.

The flexibility of the online program has already helped students at the Texas Tech School of
Medicine in Lubbock. Medical school teachers at Texas Tech, which has one of the best nutrition
education programs in the country, were finding that they had difficulty maintaining the intensity
and quality of instruction once more senior medical students began working in hospitals scattered
across the school's widely dispersed campuses. Students at a hospital that had the luxury of a
trained faculty member, for example, would be immersed in a diabetes workshop that involved
``becoming diabetic'' for a week and regularly checking blood sugar readings and self-administering
``insulin'' through a needle and syringe, while students at another hospital would be left with no
instruction at all. The online Nutrition in Medicine course allowed all the students to continue
learning about diet and counseling patients despite their disparate locations and resources.

``We didn't have to reinvent the wheel at other campuses when we already had these online courses
that are so well done,'' said Katherine Chauncey, a registered dietitian and a professor of clinical
family medicine at Texas Tech.

More recently, Ms.~Adams and her colleagues have begun working on online nutrition education
programs geared toward practicing physicians. ``Many of them are realizing that their training
wasn't adequate enough to make them feel comfortable counseling patients,'' Ms.~Adams said. Short,
focused and relatively easy to navigate, these courses are meant to help fill in those gaps in
knowledge for older doctors. Eventually, practicing physicians may even be able to earn continuing
medical education credits, a requirement of many hospitals, state licensing boards and specialty
boards.

``It's extremely difficult to get people to change their diets and their habits around food,''
Ms.~Adams said. ``Anything that improves a doctor's confidence and skill set will go a long way in
helping patients.''

Added Dr.~Chauncey: ``You can't just keep writing out script after script after script of new
medications when diet is just as important as drugs or any other treatment a patient may be using.''

\pagebreak
\section{F.D.A. Panel Urges Denial of Diet Drug}

\lettrine{A}{}\mycalendar{Sept.'10}{17} federal advisory panel on Thursday recommended against
approval of a new diet pill, the latest setback in efforts to develop treatments for the nation's
obesity epidemic.

The advisers to the Food and Drug Administration voted 9 to 5 that the potential benefits of the
drug, called lorcaserin and developed by Arena Pharmaceuticals, did not outweigh the risks.

The vote indicated how tough it might be to win approval for obesity drugs. Some committee members
said lorcaserin was not clearly unsafe. But even a mere suggestion of possible risks seemed
unacceptable to some panel advisers because the drug did not help people lose much weight.

``I really didn't have a lot of issues with the risk,'' said one panel member, Dr.~Eric I.~Felner, a
diabetes expert at Emory University. Still, he voted no. ``I just didn't see it as being that
efficacious.''

The negative vote is the second setback this year in attempts to win approval for what would be the
first new prescription weight-loss drug in more than a decade.

Citing safety concerns, the same committee voted 10 to 6 in July against approval for the drug Qnexa
from Vivus, a drug that produced a much greater weight loss among trial participants than
lorcaserin.

The F.D.A., which usually takes the advice of its committees, is expected to decide next month
whether to approve lorcaserin or Qnexa.

On Wednesday, the advisory panel split 8 to 8 on whether the diet drug Meridia, sold by Abbott
Laboratories, should be removed from the market. A study showed that it increased the risk of heart
attacks and strokes in patients with cardiovascular risk. The drug has already been removed from the
market in Europe.

Some doctors who treat obesity testified in favor of the drug at Thursday's meeting, which was held
in Adelphi, Md. They said there was an urgent need for new diet drugs, given that a third of
American adults are obese and another third overweight. There is hope that drugs that help people
lose weight would also mitigate some of the health problems linked to obesity, like diabetes, heart
disease and cancer.

Yet the F.D.A. has become safety-conscious in this area since the drugs could potentially be taken
for years by millions of people, and because of highly publicized health concerns associated with
some previously popular diet prescriptions, like Meridia and the so-called fen-phen combination that
damaged heart valves.

Lorcaserin mimics the effect of the brain chemical serotonin and has an effect on suppressing
appetite. Its mechanism of action is similar to that of fenfluramine, the part of the fen-phen
regimen that was pulled from the market.

Arena, which is based in San Diego, said its drug was developed to work in the brain and not the
heart. In its clinical trials, those who took the drug did not have a significantly higher rate of
valve problems than those who got a placebo. The F.D.A. said, however, that statistically the
company could not totally rule out an increased risk of up to 50 percent.

The biggest safety issue for the committee seemed to be the finding of tumors in rats that had
received high doses of the drug. Arena argued that those findings would not apply to people. In the
clinical trials, there was no increase in cancer rates among people who took the drug.

For the committee, even though the safety issues were not considered so bad, they were measured
against an efficacy that was not too good. The drug met only one of the two F.D.A. standards for
weight loss drugs, and it did so only by what the F.D.A. termed a ``slim margin.''

Those taking the drug lost 5.8 percent of their weight after a year, compared to 2.5 percent for
those getting a placebo. The difference, 3.3 percentage points, is below the 5 percentage point
criterion set by the F.D.A.

However, 47 percent of those taking lorcaserin lost at least 5 percent of their weight, compared to
22 percent of those getting a placebo. That just exceeded the F.D.A. standard that twice as many
people on the drug as on the placebo lose 5 percent of their weight.

Those who took the drug in the trial had modest improvements in risk measures like blood pressure,
cholesterol and blood sugar compared with those who got the placebo, but the significance seemed
unclear to the committee.

``The argument that there is an urgent need I don't think really mitigates the concern of putting a
drug that doesn't do much and may do harm on the market,'' said Dr.~Pamela S.~Douglas, professor of
research in cardiovascular diseases at Duke and a committee member.

No patients who used the drug in clinical trials testified to the committee for its approval. And
some advocacy groups for women or overweight people said the drug was not ready for approval.

Jack Lief, the chief executive of Arena, said in a statement after the meeting that the company
believed that lorcaserin ``has a positive benefit-risk profile.'' If approved, lorcaserin would be
Arena's first product, and it would be marketed by part of the Japanese drug company Eisai.

Trading in Arena's stock was halted on Thursday. But after hours, the stock price fell about 40
percent.

But shares of Orexigen Therapeutics, which has a diet pill that will be discussed by the committee
in December, rose 38 percent. Investors apparently believe that Orexigen now has the best shot of
approval. Of course, after Vivus's drug was rejected by the advisory committee in July, it was
Arena's shares that rose sharply as investors bet that it had the best shot.

\pagebreak
\section{In Search of the Grizzly (if Any Are Left)}

\lettrine{P}{ast}\mycalendar{Sept.'10}{17} the asters and aspen and subalpine fir, past the quick,
cold creeks and the huckleberry hillsides, the bear hunter stopped and cocked his tweezers.

``Here,'' said Bill Gaines, a wildlife biologist for the Okanogan-Wenatchee National Forest, ``is
the mother lode.''

Caught on a prong of barbed wire that he had strung weeks earlier in these remote mountains was a
tantalizing clue: strands of light brown bear hair.

``Oh, look at that, look at that root right there,'' he said. ``That's really good.''

It will be months before DNA tests tell the full story: did those hairs belong to a black bear, a
relatively common resident here, or were they snagged from the far more elusive grizzly? The last
confirmed sighting of a grizzly in the North Cascades was in 1996.

Now Mr.~Gaines is leading the most ambitious effort ever to document whether grizzlies still exist
here -- a century after fur trappers and ranchers killed them off by the hundreds -- at a time when
tension is high in the West over the fate of wild predators like gray wolves. While many people want
the grizzlies, an endangered species, to make a comeback here, others worry that more bears will
mean more conflict.

``Grizzlies are a threat to livestock and to humans,'' said John Stuhlmiller, the director of
government relations at the Washington State Farm Bureau. ``People might think they're neat and they
might want to go see them in the zoo, but in the wild they're not a friendly, cuddly creature.''

People whose livelihoods are not threatened by predators do not get it, Mr.~Stuhlmiller said. ``If
my 401(k) was being raided by grizzly bears, I would think differently,'' he said.

For nearly 30 years the federal government has had a program to help restore the grizzly bear
population in Idaho, Montana, Washington and Wyoming. It has made a difference in places like
Yellowstone National Park and the Continental Divide region of Montana, but not in the North
Cascades, one of six designated recovery zones. Instead, this area has been locked in a virtual
standstill as political winds shift over the preservation of large predators.

Grizzlies were named a protected species in 1975. Under protection, their population tripled in
parts of the Rockies and by 2007, they were removed from the list. But last September, a federal
judge in Montana ordered grizzlies back on, citing threats that included changes to their habitat
caused by climate change.

In the North Cascades, wildlife officials agreed 13 years ago to conduct a formal environmental
review to determine the best way to ensure recovery, including augmenting the population with bears
from elsewhere. But the money needed for the review, \$1 million to \$2 million, has never been
allocated by the perpetually strapped agency that oversees the effort, the United States Fish and
Wildlife Service.

Now experts say only a handful of grizzlies may remain in the North Cascades, likely crossing back
and forth over the border with Canada.

``If these bears are to have a future,'' said Joe Scott, the international program director for
Conservation Northwest, ``the United States and British Columbia governments must do their job --
boost Cascades bears with a small number of young animals from areas where grizzly bears are more
numerous.''

Federal wildlife officials say politics and budget limitations force difficult questions.

Chris Servheen, the grizzly bear recovery coordinator for the Fish and Wildlife Service, who has
worked on the program since its inception in 1981, said the anger among ranchers and some state
governments over wolf reintroduction, and the issue's constant churn through federal courts, had
bred mistrust in wildlife agencies that has hurt the prospects for bear recovery in some areas, at
least in the near term.

``We don't really have people jumping up and down to put grizzlies anywhere at this point, people in
the Congress that is,'' said Mr.~Servheen, who is based in Montana.

There is even disagreement over whether it matters if grizzlies roam these mountains, given the
species' relative health elsewhere and the plight of more endangered species.

``Is it so critical to the future of grizzly bears as a world species if the North Cascades fades
away?'' said Doug Zimmer, a spokesman for the Fish and Wildlife Service. ``Just asking that makes my
teeth hurt.''

Yet small steps are being taken. If the study in the North Cascades proves that grizzlies still live
in the area, advocates for recovery will probably face less political opposition. This is because
they would be augmenting the historic population, not trying to rebuild the population from scratch
when there were no bears at all.

Either way, Mr.~Gaines, who wrote his doctoral thesis on black bears, wants to know that he has
tried as hard as he can to learn what is out here, he said.

This summer and early fall, with money from a \$90,000 federal grant, Mr.~Gaines has hired horse
teams and a temporary six-member research crew to trek deep in the wilderness, far from where most
people hike. The crew has set up about 90 corrals, surrounding pungent bear bait of fish guts and
road kill with barbed wire designed to snare bear hair as animals make their way to and from the
stew. Every two weeks the crews collect bear hair and memory cards from digital infrared cameras
mounted at the corrals.

Asked whether the search so far has yielded firm evidence, he noted that black bears and grizzlies
can be surprisingly easy to confuse. He said that he would not draw conclusions until the DNA tests
come back but that the crews were searching in areas considered to be ideal grizzly habitat.

``We're looking in the right places,'' he said.

\pagebreak
\section{Charges in Manila Hostage Crisis}

\lettrine{A}{t}\mycalendar{Sept.'10}{17} least 10 people, including police officers, government
officials and journalists, should face charges in the deaths of eight Hong Kong residents who were
held hostage on a bus in downtown Manila last month, the Philippine justice secretary said Thursday.

Speaking at a news conference in Manila, Justice Secretary Leila M.~de Lima said she would recommend
criminal and administrative charges but declined to elaborate on the accusations or identify who
would be charged. Ms.~de Lima, who is overseeing a panel investigating the Aug.~23 hostage standoff,
said a report would be submitted to President Benigno S.~Aquino III on Friday.

The standoff has stirred widespread anger in Hong Kong and led to a rift between the governments of
the Philippines and China. Leaders in Hong Kong openly criticized the police's handling of the
crisis, and Mr.~Aquino himself has come under criticism by Donald Tsang, the chief executive of Hong
Kong, over his management of the standoff.

Here in the Philippines, hearings on the case conducted by the Senate have brought a number of
mistakes by the Manila police to light, including miscommunication among various units.

On Thursday, Ms.~de Lima said the hostages were killed by the hostage-taker, not by police officers,
citing the accounts of three survivors. She said they had corroborated the testimony of the bus
driver that the hostage-taker shot the victims.

Last week, Ms.~de Lima had said there was ``a big possibility'' that some of the hostages could have
been hit by ``friendly fire,'' noting that ballistic experts indicated that the hostage-taker could
not have killed all the victims.

Seven tourists from Hong Kong and their tour guide were killed after their bus was commandeered by a
former Manila police officer, Rolando Mendoza. The bus remained parked at Rizal Park throughout the
day and into the evening as the police tried to negotiate with Mr.~Mendoza. Mr.~Mendoza, who had
been dismissed by the Manila police over extortion charges, demanded his reinstatement.

As negotiations collapsed, gunfire was heard within the bus, and Mr.~Mendoza was killed by a police
sniper.

China has pushed for a thorough, speedy investigation into the case. Tens of thousands of Hong Kong
residents marched at the end of August, protesting the Philippine government's handling of the
standoff.

Mr.~Aquino, in turn, has criticized news coverage of the standoff. Television networks beamed images
of the crisis and its violent end around the world. The president has said that TV images showing
Manila police officers taking Mr.~Mendoza's brother from the scene earlier in the day, seen on a
screen inside the tourist bus, agitated the hostage taker.

The National Union of Journalists of the Philippines cautioned the government against filing
criminal charges against journalists who covered the standoff.

``While we do not deny that the lapses of certain journalists contributed to the tragic end, filing
criminal charges against reporters because of ethical or professional lapses -- in effect,
criminalizing the weaknesses of the media -- sets a dangerous precedent,'' Nestor Burgos, the
group's chairman, said in a telephone interview.

A spokesman for the Hong Kong Journalists Association said on Thursday that the group was following
the investigation and possible charges. Last month, the group sent a letter to Mr.~Aquino, urging
him not to blame the news media over the episode.

``We're deeply concerned over the case,'' said the spokesman, Hiu-yeung Chong.

Mr.~Aquino said last week that he would wait until he received the panel's report before deciding on
which actions to take. On Thursday, Ms.~de Lima, asked how high up the police chain of command the
possible charges might rise, said, ``High enough.''

``Did the president not say heads will roll?'' Ms.~de Lima was quoted on the ABS-CBN Web site as
saying. ``So we expect the president to act on the basis of our recommendations.''


\pagebreak
\section{Bonding With Fans Who Can't Get Enough}

\lettrine{T}{he}\mycalendar{Sept.'10}{21} Pittsburgh Penguins, who had a major draw in their young
center Sidney Crosby, were looking for an innovative marketing approach to spring back from recent
National Hockey League troubles.

Fan morale by 2008 had been dampened by the team's loss in the Stanley Cup finals to the Detroit Red
Wings that year and by the 2004-5 N.H.L. lockout when the season was canceled after management and
the players union could not agree on a contract.

The Penguins decided to help rebuild ties with fans via cellphone, a campaign that the team says has
resulted in a fivefold increase in sign-ups for its mobile fan club. That response prompted the team
to offer more mobile options for its coming season in its new arena.

``We did a lot of research, including focus groups, online surveys and arena surveys to see how we
could best reach fans season round,'' said Jeremy Zimmer, the team's director of new media. The
research helped the team focus on its fans who agreed to be contacted by cellphone, about one-fifth
of its 1.5 million person fan base; the team defines its base as those who have watched, attended or
listened by radio to at least one Penguins game in the last year, Mr.~Zimmer said.

The Penguins hired Vibes Media, a Chicago mobile marketing company, to help it create the Pens
Mobile Club, where fans could receive news, recaps and commercial offers -- including free tickets
from Chevrolet to the Pittsburgh auto show and discounts on Coors Light beer at local bars -- on
their mobile phones.

The result has been an increase in club members to 72,440, up from 14,000 in the 2008-9 season, the
team announced last month.

Like the Penguins, more sports teams and leagues are communicating with fans intensively by mobile
devices, largely because they are ``incredibly passionate and identify with their teams so they are
accepting of receiving a lot of information,'' said Ben Davis, a founder of San Francisco-based
Phizzle, which works with clients like the National Basketball Association's Cleveland Cavaliers,
which sends 1.5 million text messages each month -- including scores, statistics, news and other
updates -- to its mobile subscribers.

The Cavaliers' mobile alert program also delivers fan-requested content, like game schedules and
team statistics, Mr.~Davis said. The team partners with companies like the roast beef sandwich chain
Arby's for enter-to-win-via-text contests where winners -- of a free meal or a coupon -- are chosen
randomly.

Phizzle also works on mobile marketing with the National Hockey League's San Jose Sharks and the
Nashville Predators, the N.B.A.'s Philadelphia 76ers, and with Madison Square Garden -- where the
New York Knicks and the Rangers play -- this fall.

In the 2009-10 season, N.B.A. Mobile, which has 100 mobile applications, had more than 1.7 million
downloads, according to figures from N.B.A. Digital. The N.H.L. has started a mobile site,
m.nhl.com, which displays live game scores, and summaries and recaps of the scoring for completed
games.

Teams also are mindful that some two billion tickets -- for sports and other events -- are projected
to be purchased via mobile devices this year, and rise to 15 billion by 2014, according to Juniper
Research, which tracks mobile commerce and marketing.

``Mobile has a lot of tentacles, from building a database of fans, reaching commercial sponsors,
selling tickets and merchandising,'' said Michael Falato, vice president of sales for Txtstation, an
Austin, Tex., mobile marketing company that works with sports teams like the Miami Dolphins of the
National Football League.

Over the last two seasons, the Dolphins' text club more than quadrupled its initial 8,000
membership, he said. Fans can text -- but not call -- in questions and comments to the team's daily
afternoon radio show, or, for example, text in a choice for most valuable player during the football
season. And texts are tied to prizes like free tickets and other awards like upgraded stadium
seating, said Wayne Partello, the Dolphins' senior director of content and creative.

Mobile marketing raises a team's public profile and is especially valuable for collecting data on
fans, sports marketing experts say. That data can later be tied to commercial deals that benefit the
team, like signing up subscribers to cellular carriers like Verizon or AT\&T -- which several teams,
including the Penguins and the Dolphins, do.

``It's all about selling eyeballs, and mobile brings it down to the individual level, `` said
Stephen R.~McDaniel, associate professor of sports marketing and consumer psychology at the
University of Maryland. This is a win for sports organizations, he said, ``because the team is
building and maintaining fans, and, at the same time, it is enhancing its second revenue stream that
comes from sponsorships and promotions.''

No team ``has the silver bullet,'' said Jack Philbin, president of Vibes Media, which also helped
the Stanley Cup champion Chicago Blackhawks last season add 30,000 fans to their mobile database.
But, ``this is the most intimate device we can use.''

Last season, the Penguins increased its number of mobile fans by teaming up with Delta Air Lines in
a text contest in which fans could enter a random drawing to win two plane tickets from Pittsburgh
to Paris. The campaign was promoted online and in radio spots as well as at games, and in two weeks,
the team attracted 34,000 entries. Of those, more than 3,000 joined the Pens Mobile Club.

The Penguins also have iPhone and Android applications, and a BlackBerry app is in the works. The
apps offer news, archived video, game schedules, statistics and standings and the ability to follow
Twitter conversations about the team. Other teams, including the Dolphins, Baltimore Orioles, the
Chicago White Sox and the Washington Capitals, also use Twitter to give fans the latest news.

When the Penguins play this fall, those seated in the new 18,000-seat Consol Energy Center will be
able to connect, via cellphone, with the Yinzcam Mobile video system, a pilot project with nearby
Carnegie Mellon University. The system allows those in the arena to simultaneously watch the game
from six angles. They also will be able to see game statistics, roster and other information, and
view instant replay, accompanied by in-phone ads from the sponsor, Verizon -- but only in the arena.

During intermissions while the ice is being resurfaced, ticket holders will be able to send text
messages that will appear on the scoreboard, vote for the best player and receive real-time game
statistics, Mr.~Zimmer said. The Penguins system will also allow tracking of which seating section
has the highest amount of texting, and fans sitting in postgame traffic will be able to text in
comments and questions to the Penguins call-in radio show, he said.

In the Pittsburgh area, the Penguins mobile effort is focusing on more than 100,000 fans ages 21 to
24. Students often stood in line for hours to buy discounted tickets and were turned away when the
game was sold out, Mr.~Zimmer said.

So the team worked with the apparel company American Eagle Outfitters to set up the American Eagle
Student Rush Club, whose 15,500 members can be notified by cellphone texts about ticket availability
-- or absence of tickets when a home game is sold out. Each alert, which carries ``AE Student Rush''
across the top, also includes a promotional offer like the one recently that said the 4,023rd
student to reply would win four free tickets.

The Penguins are having some fun as well. Recently the team notified, by cellphone, its Rush Club
members about a ``student flush'' day. The Penguins promised the first 400 students a T-shirt and an
early look at the new arena if they showed up to help test the Consol Energy Center's sanitation
system.

One day in early June, the students converged and simultaneously flushed all the arena's toilets
``so we could see if everything worked,'' Mr.~Zimmer said with a laugh.

\pagebreak
\section{Bookseller Has Setback in Struggle Over Board}

\lettrine{B}{arnes \& Noble}\mycalendar{Sept.'10}{21} sustained a setback on Monday when a powerful
proxy advisory company endorsed directors proposed by the billionaire investor Ronald W.~Burkle over
the company's own slate, which included its chairman, Leonard S.~Riggio.

The endorsement by Institutional Shareholder Services could be crucial, coming just a week before
the annual shareholder meeting on Sept.~28. Some large institutional investors are required to vote
their shares in accordance with I.S.S.'s recommendations.

Barnes \& Noble has been battling Mr.~Burkle for nearly a year since his investment firm, Yucaipa,
began rapidly accumulating the company's shares. Barnes \& Noble instituted a poison-pill plan
intended to limit the size of Mr.~Burkle's holding; the plan was upheld by a Delaware Court of
Chancery judge last month.

But Mr.~Burkle appears to have won the latest round. In its 25-page report, Institutional
Shareholder Services supported Mr.~Burkle's contention that Barnes \& Noble's corporate governance
needed to be improved.

Yucaipa raised questions about the company's executive pay practices and some of its deal-making,
including the purchase of a college bookstore business owned by Mr.~Riggio. Investor lawsuits about
that deal are pending.

``We believe the dissidents have demonstrated a compelling case that change in the BKS board is
warranted,'' I.S.S. analysts wrote, referring to the company by its stock symbol.

The report also pointed to the slide in Barnes \& Noble's stock price as another reason for change.
Shares in the company have tumbled 28.9 percent in the last year.

An analyst at Bank of America Merrill Lynch downgraded Barnes \& Noble to underperform last week,
arguing that its digital strategy faced major challenges from wealthier rivals like Amazon.com and
Apple.

Support from Institutional Shareholder Services for the dissident slate, which includes Mr.~Burkle
and the former chief executives of Hilton Hotels and Earthlink, provides a big boost. Mr.~Burkle has
argued that independent directors would serve as a counterweight to Mr.~Riggio influence on the
Barnes \& Noble board.

``We are gratified that I.S.S. agrees with us that the challenges facing Barnes \& Noble require the
independent leadership and experience Yucaipa's three highly qualified nominees will bring to the
board,'' a Yucaipa representative said in a statement.

Barnes \& Noble said in a statement that it was disappointed by the I.S.S. recommendation, but said
that it had won the support of three smaller proxy advisers, Glass Lewis \& Company, the Egan-Jones
Ratings Company and Proxy Governance.

``While I.S.S. has a track record of supporting dissidents, we believe its analysis is flawed and
not in the best interest of our shareholders,'' the company said.

Mr.~Riggio, who fashioned his empire starting with a single Manhattan bookstore 39 years ago, argues
that the company has a promising future running bookstores while expanding in a digital marketplace
anchored by devices like its Nook e-reader. But Mr.~Burkle says the company's strategy needs change,
though he has declined to elaborate.

The recommendations of proxy advisers do not always hold sway over these investors. A majority of
Airgas shareholders, for example, voted in favor of a proposal from the company's rival, Air
Products and Chemicals, to move forward the next Airgas shareholder meeting to January. Both I.S.S.
and Glass Lewis had sided with Airgas in urging shareholders to reject that measure.

\pagebreak
\section{Panel Leans in Favor of Engineered Salmon}

\lettrine{M}{embers}\mycalendar{Sept.'10}{21} of a federal advisory committee on Monday seemed to
conclude that genetically engineered super-salmon would be safe to eat and for the environment, but
they also found gaps in the studies used to support that conclusion.

The committee met here to advise the Food and Drug Administration on whether to approve what would
be the first genetically engineered animal to enter the American food supply.

The Atlantic salmon, which would be raised on farms, contain an extra growth hormone gene that
allows them to grow to marketable size about twice as fast as conventional fish.

Committee members, who were not asked to vote on whether the fish should be approved, did not point
out anything about the fish that would seem dangerous, despite one study suggesting a possible
increase in the potential to cause allergic reactions. They said the chance the fish would escape
into the wild was low.

``They didn't see any glaring holes'' in the data, Gregory A.~Jaffe of the Center for Science in the
Public Interest, who was the consumer representative on the committee, said after the meeting ended.

Still some panel members did say the studies the F.D.A. relied on to reach its own conclusion that
the salmon would be safe were flawed, often using only a few dozen fish or even fewer.

``I do get heartburn when we're going to allow post-market surveillance to finalize our safety
evaluation,'' said one committee member, Michael D.~Apley, a pharmacology expert at Kansas State
University.

The criticisms could add to the time needed to approve the salmon. It could also provide grist for
consumer and environmental groups, many of which testified on Monday that the salmon should not be
approved.

Approval of the salmon could pave the way for other such biotech animals to enter the food supply,
like a pig developed in Canada that has more environmentally friendly manure.

The results could also influence other countries. Eric Hallerman, a fisheries expert at Virginia
Tech, told the committee that fast-growing versions had already been developed for 18 different
types of fish in various countries.

The salmon contain a growth hormone gene from the Chinook salmon and a genetic switch from the ocean
pout that turns on an antifreeze gene. That allows the salmon to make growth hormone in cold
weather, whereas salmon usually produce it only in warm weather.

Ronald L.~Stotish, the chief executive of AquaBounty Technologies, the company that developed the
salmon, told the committee that its AquAdvantage salmon would help the world meet rising demand for
seafood without further devastating natural fisheries. He said it would be economical to grow the
fish in inland tanks in the United States, saving the cost of flying in the fish from Chile or
Norway, from which the United States now gets most of its Atlantic salmon, he said.

For now, though, the company's eggs are being hatched at a company facility in Prince Edward Island,
Canada. And the fish would be grown to size in only limited quantities at a company facility in
Panama.

The company said that fish would not escape because they are grown inland in facilities with
containment mechanisms. If any did escape, it said, the rivers outside the Canadian and Panama
facilities would be too salty or warm for the fish to survive. And the fish would all be female and
almost all would be sterile, so they would not interbreed with wild salmon.

But some committee members, as well as some environmental groups, said the government's
environmental assessment should evaluate what would happen if the salmon were grown widely in many
facilities.

``The F.D.A. must consider issues related to realistic production scenarios,'' said Anna Zivian, a
senior manager at the group Ocean Conservancy.

One test showed a possible increase in the potential to cause allergic reactions that was almost
statistically significant even though only six fish were used in each group in the study.

But several committee members said the meaning of that test's results were open to question since it
was not clear what amount of increase was meaningful.

Kevin Wells, an assistant professor at the University of Missouri and a committee member, said he
doubted the fish would cause extra allergies.

``The salmon contains nothing that isn't in the human diet,'' he said.

The fish are being regulated under the process used to approve veterinary drugs. The F.D.A. held a
half-day session on Sunday to give the committee, made up mostly of veterinarians, a primer on
genetic engineering.

Approval, if it comes, is likely to take at least several months. The F.D.A. said it would prepare
an environmental assessment that would be open to comment for 30 days. If the agency decides that
there could be a significant environmental impact -- something that does not appear likely -- it
will have to do a full environmental impact statement, which could take months or years.

The F.D.A. will have a public hearing on Tuesday on whether the salmon, if approved, should be
labeled.

\pagebreak
\section{SAIC of China Is Considering a Stake in General Motors}

\lettrine{A}{}\mycalendar{Sept.'10}{21} Chinese automaker has expressed interest in buying a stake
in General Motors when it holds a public stock offering later this year, a move that could raise
concerns about foreign influence over the largest American automaker.

The Shanghai-based company, the SAIC Motor Corporation, has had a longtime partnership with G.M. in
China.

An SAIC spokeswoman, Zhu Xiangjun, said Monday that comments about a possible G.M. stake were made
last month by SAIC's chairman, Hu Maoyuan. She said his comments remained the company's position but
declined further comment.

``G.M. is our important strategic partner,'' Mr.~Hu said in August. ``We are not clear about the
details of its I.P.O. We will make the right decision when we know the details.''

G.M. is planning to hold its public stock offering in November. The offering will give the Treasury
Department its first opportunity to begin selling off the 61 percent stake in G.M. owned by American
taxpayers.

Details of the stock sale were still being worked on by G.M. and its underwriters, including the
size of the offering and the price of the shares.

Treasury officials are closely involved in determining both the size of G.M.'s offering and the
share price. The government is most interested in maximizing the value of the shares it sells
initially, and establishing a strong market for future sales of taxpayer-owned shares.

There was no comment Monday from the Treasury about the possibility of a foreign company buying a
big stake in G.M. ``We expect that potential investors will be sought across multiple geographies
with a focus on North American investors,'' the Treasury said in a filing last week.

Moreover, the government expects ``broad distribution'' of the stock. ``We expect that a large and
diverse group of institutional investors will be offered an opportunity to participate, with no
single investor or group of investors receiving a disproportionate share or unusual treatment,'' the
Treasury said.

G.M. plans to begin discussing the offering in an international stock presentation, or road show, to
potential investors, scheduled to begin in early November.

SAIC is one of China's largest automakers and recently bought half of G.M.'s India division, turning
that into a joint venture. SAIC is controlled by the Chinese government.

\pagebreak
\section{Code That Tracks Users' Browsing Prompts Lawsuits}

\lettrine{S}{andra}\mycalendar{Sept.'10}{21} Person Burns used to love browsing and shopping online.
Until she realized she was being tracked by software on her computer that she thought she had
erased.

Ms.~Person Burns, 67, a retired health care executive who lives in Jackson, Miss., said she is wary
of online shopping: ``Instead of going to Amazon, I'm going to the local bookstore.''

Ms.~Person Burns is one of a growing number of consumers who are taking legal action against
companies that track computer users' activity on the Internet. At issue is a little-known piece of
computer code placed on hard drives by the Flash program from Adobe when users watch videos on
popular Web sites like YouTube and Hulu.

The technology, so-called Flash cookies, is bringing an increasing number of federal lawsuits
against media and technology companies and growing criticism from some privacy advocates who say the
software may also allow the companies to create detailed profiles of consumers without their
knowledge.

Unlike other so-called HTML cookies, which store Web site preferences and can be managed by changing
privacy settings in a Web browser, Flash cookies are stored in a separate directory that many users
are unaware of and may not know how to control.

Ms.~Person Burns, a claimant who is to be represented by KamberLaw, said she knew cookies existed
but did not know about Flash cookies.

``I thought that in all the instructions that I followed to purge my system of cookies, I thought I
had done that, and I discovered I had not,'' she said. ``My information is now being bartered like a
product without my knowledge or understanding.''

Since July, at least five class-action lawsuits filed in California have accused media companies
like the Fox Entertainment Group and NBC Universal, and technology companies like Specific Media and
Quantcast of surreptitiously using Flash cookies. More filings are expected as early as this week.

The suits contend that the companies collected information on the Web sites that users visited and
from the videos they watched, even though the users had set their Web browser privacy settings to
reject cookies that could track them.

``What these cases are about is the right of a computer user to dictate the terms by which their
personal information is harvested and shared. This is all about user control,'' said Scott
A.~Kamber, 44, a privacy and technology lawyer with KamberLaw who is involved with some of the
cases. The suits have been filed by firms including Parisi \& Havens and the law office of Joseph
H.~Malley.

One lawsuit contends that Clearspring Technologies and media companies including the Walt Disney
Internet Group ``knowingly authorized'' the use of online tracking devices that would ``allow access
to and disclosure of Internet users' online activities as well as personal information.'' Others say
that the information was gathered to sell to online advertisers.

In August, Clearspring and Quantcast issued statements on their company blogs addressing the suits.
Clearspring clarified its use of Flash cookies and said the legal filings were ``factually
inaccurate.'' The company said it used Flash cookies, also known as Flash local storage, ``to
deliver standard Web analytics to publishers.'' The post also stated that data was collected at the
aggregate level including unique users and interaction time, but did not include personally
identifiable information.

Quantcast's blog post said that the company ``uses Flash cookies for measurement purposes only and
not for any form of targeted content delivery.''

Specific Media did not respond to requests for comment. Counsel for the media companies in the cases
declined to comment; representatives of companies that had not yet been served with the suits also
declined to comment.

Some privacy advocates said that despite the companies' claims, if enough data is collected over
time, advertisers can create detailed profiles of users including personally identifiable data like
race and age in addition to data about what Web sites a user visits. They also take issue with the
fact that Flash cookies can be used to restore HTML cookies that have been deleted from a user's
computer, circumventing a user's privacy settings.

``The core function of the cookie is to link what you do on Web site A to what you do on Web site
B,'' said Peter Eckersley, a technologist at the Electronic Frontier Foundation. ``The Flash cookie
makes it harder for people to stop that from happening.''

According to Adobe, more than 75 percent of online videos are delivered using Flash technology, with
media companies also using it to serve games and animation to users. The company says that Flash
cookies are intended to be used for basic Web functions like saving a user's volume and language
preferences or remembering where a user left off on a video game.

In a public letter to the Federal Trade Commission in January, Adobe condemned the practice of
restoring cookies after they had been deleted by a user. The company provides an online tool on its
Web site to erase Flash cookies and manage Flash player settings. At least one suit, however, claims
that the controls are not easy to reach and are not obvious to most Web users.

Chris Jay Hoofnagle, 36, one of the authors of a University of California, Berkeley, study about
Internet privacy and Flash cookies that has been used in several of the legal filings, said the
recent spate of suits pointed to a weakness in federal rules governing online privacy.

``Consumer privacy actions have largely failed,'' Mr.~Hoofnagle said. The lawsuits, he added,
``actually are moving the policy ball forward in the ways that activists are not.''

Complaints about online privacy are now migrating to mobile technology. Last week, a lawsuit was
filed by three California residents against a technology company called Ringleader Digital saying
that the company used a product called Media Stamp that ``acquired information from plaintiff's
phone and assigned a unique ID to their mobile device.''

The suit says that the information collected by the unique ID, using a technology called HTML 5,
allowed Web site operators ``to track the mobile devices' Internet activities over multiple Web
sites.''

In a statement, Bob Walczak Jr., Ringleader's chief executive, said, ``Our intent since the
inception of the company has been to build a mobile advertising platform that users can control.''
He added that Ringleader was working on ``new ways for consumers to be able to verify for themselves
that their opt-outs have taken effect.''

John Verdi, senior counsel at the Electronic Privacy Information Center, faulted the Federal Trade
Commission for not being more aggressive on privacy issues, focusing largely, instead, on
self-regulation.

``The F.T.C. has been inactive on this front and has failed to present meaningful regulation on
this,'' he said. ``There's wide evidence that online tracking is not being controlled by
self-regulation.''

Christopher Olsen, an assistant director in the division of privacy and identity protection at the
agency, said it had hosted a series of roundtable discussions about online and offline privacy
challenges from December to March and planned to issue a report in the next few months to address
those issues.

The agency is investigating several companies, but Mr.~Olson declined to comment on the specifics.

Other efforts to address online privacy are taking place at the Congressional level. In July,
Representative Bobby L.~Rush, Democrat of Illinois, introduced an online privacy bill that would,
among other things, require companies to disclose how they collect, use and maintain the personal
information on users and to make those disclosures easy for users to understand.

\pagebreak
\section{Disappointed Supporters Question Obama}

\lettrine{I}{t}\mycalendar{Sept.'10}{21} was billed as ``Investing in America,'' a live televised
conversation on the state of the economy between President Obama and American workers, students,
business people and retirees, a kind of Wall Street to Main Street reality check.

But it sounded like a therapy session for disillusioned Obama supporters.

In question after question during a one-hour session, which took place on Monday at the Newseum here
and was televised on CNBC, Mr.~Obama was confronted by people who sounded frustrated and anxious --
even as some said they supported his agenda and proclaimed themselves honored to be in his presence.

People from Main Street wanted to know if the American dream still lived for them. People on Wall
Street complained that he was treating them like a pi\~{n}ata, ``whacking us with a stick,'' in the
words of Anthony Scaramucci, a former law school classmate of Mr.~Obama's who now runs a hedge fund
and was one of the president's questioners.

``I'm exhausted of defending you, defending your administration, defending the mantle of change that
I voted for,'' said the first questioner, an African-American woman who identified herself as a
chief financial officer, a mother and a military veteran. ``I've been told that I voted for a man
who was going to change things in a meaningful way for the middle class and I'm waiting sir, I'm
waiting. I still don't feel it yet.''

A 30-year-old law school graduate told Mr.~Obama that he had hoped to pursue a career in public
service -- like the president -- but complained that he could barely pay the interest on his student
loans, let alone think of getting married or starting a family.

``I was really inspired by you and your campaign and the message you brought, and that inspiration
is dying away,'' he said, adding, ``And I really want to know, is the American dream dead for me?''

The extraordinarily personal tone of the session, coupled with more substantive policy questions
from the host, John Harwood of CNBC and The New York Times, reflects the erosion of support for
Mr.~Obama among the constituencies that sent him to the White House two years ago.

It was all the more compelling coming from such a friendly audience; one questioner, a
small-business owner in Pennsylvania, began by praising the president for turning around the auto
industry, then lamented: ``You're losing the war of sound bites. You're losing the media cycles.''

As he leads his party into what many analysts expect to be a devastating midterm election for
Democrats, the president faces overwhelming skepticism from Americans on his handling of the
economy. A recent New York Times poll found 57 percent of respondents believed the president did not
have a clear plan for fixing the nation's broken economy.

Mr.~Obama sought on Monday to address those concerns, telling his business critics that he was not
antibusiness and his middle class questioners that ``there are a whole host of things we've put in
place to make your life better.'' He cited his health care bill, a financial regulatory overhaul
measure that imposed tough requirements on credit card companies; an education bill that increased
the availability of student loans.

The president also laid down a challenge to the Tea Party movement, whose candidates have swept
aside mainstream Republicans in recent primaries in Alaska and Delaware. He said it was not enough
for Tea Party candidates to campaign on a theme of smaller government; he tried to put them in an
uncomfortable box by prodding them to offer specifics about the programs they would cut.

``The challenge for the Tea Party movement is to identify specifically: What would you do?'' the
president said. ``It's not enough to say get control of spending. I think it's important for you to
say, `I'm willing to cut veterans benefits, or I'm willing to cut Medicare or Social Security, or
I'm willing to see taxes go up.' ''

Mr.~Obama hinted that he was open to considering a payroll tax holiday to spur job growth, saying he
would be willing to ``look at any idea that's out there,'' although he went on to say that some
ideas that ``look good on paper'' are more complicated than they appear.

And he ducked a question from Mr.~Harwood about whether he would be willing to debate the House
Republican leader, John Boehner of Ohio, the way former President Bill Clinton had a debate 15 years
ago with Newt Gingrich, who was then the House speaker.

``I think it's premature to say that John Boehner's going to be the speaker of the House,''
Mr.~Obama said.

Mr.~Obama is stepping up his efforts to mobilize Democratic voters and find ways to improve the
political climate for his party heading toward Election Day. He will begin trying to build
enthusiasm among some of the voters who propelled him to victory in 2008, like college students,
while Democratic strategists are considering ways to turn the increased prominence of the Tea Party
movement to their advantage by characterizing positions taken by some Tea Party-backed Republican
candidates as extreme.

The White House denied an article in The New York Times on Monday saying that Mr.~Obama's political
advisers were considering national advertising to cast the Republican Party as having been all but
taken over by the Tea Party movement.

``The story that led The New York Times yesterday was flat out wrong,'' Dan Pfeiffer, the White
House communications director, said in an e-mail message. ``The White House has never discussed,
contemplated or weighed such an ad campaign.''

Mr.~Pfeiffer said the article ``was based on the thinnest of reeds,'' an anonymous source.

The Times stood by the report.

After his appearance on CNBC, the president flew to Philadelphia, where he appeared at two
fund-raisers for Representative Joe Sestak, the Democratic Senate candidate in Pennsylvania, and
raised \$1 million for the Democratic National Committee.

If the televised session on Monday seemed to put Mr.~Obama on the spot, he did not appear ruffled.
Rather, he seemed resigned to the frustration of his questioners.

``My goal here is not to convince you that everything is where it needs to be,'' the president said,
``but what I am saying is that we are moving in the right direction.''

\pagebreak
\section{Chinese Investors Flock to London to Buy Real Estate}

\lettrine{N}{aomi}\mycalendar{Sept.'10}{21} Minegishi, 21, a Japanese woman who lived in China for
10 years, recently took a job with the London property broker Felicity J Lord.

Ms.~Minegishi was hired not for her experience in real estate sales -- she is studying management at
a London university -- but for her language ability. She is fluent in Mandarin, an increasingly
valuable skill in London's residential real estate market.

With her help, the agency recently sold four three-bedroom apartments in a new development for
\textsterling320,000, about \$500,000, each to a different Chinese buyer and solely on the basis of
photos and floor plans. The new construction is close to the Olympic stadium, and the investors are
betting that real estate prices will rise before the Games in 2012.

Chinese clients are a dream, Ms.~Minegishi said. ``They are wealthy, they pay in cash, and they're
looking for good value.''

Chinese citizens require approval from their local authorities to invest more than the equivalent of
\$50,000 a year overseas. But many wealthy Chinese elude the restrictions with help from trust funds
and foreign bank accounts, real estate brokers say.

The London property market might have shown signs of cooling recently, but investors from mainland
China and Hong Kong are busier than ever -- bidding, for example, on luxury apartments in the
fashionable Knightsbridge district down the road from Harrods department store and on new homes near
the Canary Wharf financial district.

In some parts of London, mainland Chinese investors have already replaced those from Russia and the
Middle East as the busiest real estate buyers with deep pockets, looking for trophy assets and
pushing up prices, some brokers say.

Buyers from mainland China are a tiny portion of purchasers of high-end real estate in London,
accounting for 5 percent of all purchases by foreigners of London properties valued from
\textsterling500,000 to \textsterling1 million this year. But they are a growing presence. They
accounted for less than 1 percent of purchases in that price range last year, according to Savills,
a real estate agency.

Europeans still make up the largest portion, Savills says, although it does not break down buyers by
country.

Unlike clients from Russia and the Middle East, however, few Chinese buyers are looking for London
apartments to live in themselves. A majority of them are seeking investments in a real estate market
they perceive as more stable than their own and are planning to receive steady rental income for
years, Ms.~Minegishi said.

For wealthy Asians, fears that governments may impose more constraints on red-hot local property
markets back home have made investments abroad more attractive.

Rapid economic growth and easy credit caused real estate prices in many parts of Asia to rise
sharply late last year. In Hong Kong, for example, prices for luxury homes have jumped 45 percent
since 2009, according to Savills.

Last month, to prevent prices from overheating, Singapore raised the down payment required for many
home purchases. The step followed similar measures in mainland China and Hong Kong this year.

Prices of high-end London real estate rose 8.75 percent in 2009, compared with a drop of 2.25
percent in the southwest of Britain, Savills said.

Not surprisingly, the property industry in Britain is adapting to meet the Chinese demand. Brokers
are hiring Mandarin speakers like Ms.~Minegishi, as well as Cantonese speakers to cater to people
from Hong Kong.

Savills organized a seminar in Shanghai in July to teach 100 clients how to buy real estate in
London. A rival agency, Hamptons International, opened an office in Hong Kong with four employees
about two months ago.

Some London developers, meanwhile, are omitting the number four in new buildings because it is
considered unlucky in Chinese culture.

``Most developers in London are including China in their marketing efforts,'' said Matthew Tack, a
director at Hamptons in London. ``They'd be silly not to.''

The increase in transactions highlights a gradual shift in wealth to Asia, including mainland China.
Free of the debt levels that still haunt Western households and governments, much of Asia began to
recover rapidly from the global economic downturn last year.

And although a large majority of Asians still struggle to make ends meet, the booming growth has
catapulted many into the ranks of the wealthy and superwealthy.

In mainland China alone, the number of people with assets worth more than 10 million renminbi, or
\$1.5 million, rose 6.1 percent, to 875,000, in a year, according to the Hurun Research Institute in
Shanghai.

Then there are the fabulously wealthy, like Joseph Lau, a Hong Kong real estate billionaire who
recently spent \textsterling33 million on a six-floor mansion in Eaton Square in London, an address
he shares with the Russian oligarch Roman A.~Abramovich. Mr.~Lau's son, Lau Ming-wai, studied at the
London School of Economics and then worked for Goldman Sachs in London.

More typical, though, are Asian buyers spending \textsterling1 million or less. Because of China's
restrictions on overseas investments, most of the Chinese buyers pay cash to minimize the paper
trail. None of the London brokers interviewed for this article were willing to disclose the
identities of buyers or introduce them to a reporter.

Although Chinese are becoming more active in many overseas real estate markets, including the United
States and Continental Europe, London remains highly popular for a variety of reasons, brokers say.
Britain has almost no restrictions on whether foreigners can own real estate, and a fairly fluid
rental market, which is attractive to buyers seeking income from their properties.

Cultural issues, especially the Chinese emphasis on education, also favor the acquisition of London
addresses.

Education is generally the largest budget item in a Chinese household, and many families hope to
send their children to elite universities in Britain, which tend to admit more foreign students than
top universities in the United States, said Jeff Cao, head of the China sector for Think London, a
government-supported agency that helps attract foreign investment to the city.

The number of Chinese students at London universities rose 9 percent, to 948, last year from 867 a
year earlier, according to the Universities and Colleges Admissions Service.

For some Chinese buyers in London, Mr.~Cao said, the idea is to find apartments big enough to
provide the children with more comfortable accommodations than student dormitories and that have
spare rooms that can be rented out. Once the children graduate, their parents aim to rent out the
whole apartment.

One mainland Chinese investor who owns real estate in London and elsewhere said his London
investments were his most lucrative.

``I bought a flat for my daughter's use when she was studying in London and other flats I have
rented out or sold,'' Mr.~Lai, the owner, who declined to give his first name to protect his
privacy, wrote by e-mail.

``The U.K. traditionally has a very good legal structure with good law and order,'' Mr.~Lai wrote.
``That, together with the city's financial institutions and the British people's love for owning
their own homes, makes the property market extremely attractive.''

Real estate brokers who cater to Chinese customers say it is unclear whether and when the appetite
from China will start to ebb. But there is no indication of a pullback yet.

\pagebreak
\section{Sizing Up Consciousness by Its Bits}

\lettrine{O}{ne}\mycalendar{Sept.'10}{21} day in 2007, Dr.~Giulio Tononi lay on a hospital stretcher
as an anesthesiologist prepared him for surgery. For Dr.~Tononi, it was a moment of intellectual
exhilaration. He is a distinguished chair in consciousness science at the University of Wisconsin,
and for much of his life he has been developing a theory of consciousness. Lying in the hospital,
Dr.~Tononi finally had a chance to become his own experiment.

The anesthesiologist was preparing to give Dr.~Tononi one drug to render him unconscious, and
another one to block muscle movements. Dr.~Tononi suggested the anesthesiologist first tie a band
around his arm to keep out the muscle-blocking drug. The anesthesiologist could then ask Dr.~Tononi
to lift his finger from time to time, so they could mark the moment he lost awareness.

The anesthesiologist did not share Dr.~Tononi's excitement. ``He could not have been less
interested,'' Dr.~Tononi recalled. ``He just said, `Yes, yes, yes,' and put me to sleep. He was
thinking, `This guy must be out of his mind.' ''

Dr.~Tononi was not offended. Consciousness has long been the province of philosophers, and most
doctors steer clear of their abstract speculations. After all, debating the finer points of what it
is like to be a brain floating in a vat does not tell you how much anesthetic to give a patient.

But Dr.~Tononi's theory is, potentially, very different. He and his colleagues are translating the
poetry of our conscious experiences into the precise language of mathematics. To do so, they are
adapting information theory, a branch of science originally applied to computers and
telecommunications. If Dr.~Tononi is right, he and his colleagues may be able to build a
``consciousness meter'' that doctors can use to measure consciousness as easily as they measure
blood pressure and body temperature. Perhaps then his anesthesiologist will become interested.

``I love his ideas,'' said Christof Koch, an expert on consciousness at Caltech. ``It's the only
really promising fundamental theory of consciousness.''

Dr.~Tononi's obsession with consciousness started in his teens. He was initially interested in
ethics, but he decided that questions of personal responsibility depended on our consciousness of
our own actions. So he would have to figure out consciousness first. ``I've been stuck with this
thing for most of my life,'' he said.

Eventually he decided to study consciousness by becoming a psychiatrist. An early encounter with a
patient in a vegetative state convinced Dr.~Tononi that understanding consciousness was not just a
matter of philosophy.

``There are very practical things involved,'' Dr.~Tononi said. ``Are these patients feeling pain or
not? You look at science, and basically science is telling you nothing.''

Dr.~Tononi began developing models of the brain and became an expert on one form of altered
consciousness we all experience: sleep. In 2000, he and his colleagues found that Drosophila flies
go through cycles of sleeping and waking. By studying mutant flies, Dr.~Tononi and other researchers
have discovered genes that may be important in sleep disorders.

For Dr.~Tononi, sleep is a daily reminder of how mysterious consciousness is. Each night we lose it,
and each morning it comes back. In recent decades, neuroscientists have built models that describe
how consciousness emerges from the brain. Some researchers have proposed that consciousness is
caused by the synchronization of neurons across the brain. That harmony allows the brain to bring
together different perceptions into a single conscious experience.

Dr.~Tononi sees serious problems in these models. When people lose consciousness from epileptic
seizures, for instance, their brain waves become more synchronized. If synchronization were the key
to consciousness, you would expect the seizures to make people hyperconscious instead of
unconscious, he said.

While in medical school, Dr.~Tononi began to think of consciousness in a different way, as a
particularly rich form of information. He took his inspiration from the American engineer Claude
Shannon, who built a scientific theory of information in the mid-1900s. Mr.~Shannon measured
information in a signal by how much uncertainty it reduced. There is very little information in a
photodiode that switches on when it detects light, because it reduces only a little uncertainty. It
can distinguish between light and dark, but it cannot distinguish between different kinds of light.
It cannot tell the differences between a television screen showing a Charlie Chaplin movie or an ad
for potato chips. The question that the photodiode can answer, in other words, is about as simple as
a question can get.

Our neurons are basically fancy photodiodes, producing electric bursts in response to incoming
signals. But the conscious experiences they produce contain far more information than in a single
diode. In other words, they reduce much more uncertainty. While a photodiode can be in one of two
states, our brains can be in one of trillions of states. Not only can we tell the difference between
a Chaplin movie and a potato chip, but our brains can go into a different state from one frame of
the movie to the next.

``One out of two isn't a lot of information, but if it's one out of trillions, then there's a lot,''
Dr.~Tononi said.

Consciousness is not simply about quantity of information, he says. Simply combining a lot of
photodiodes is not enough to create human consciousness. In our brains, neurons talk to one another,
merging information into a unified whole. A grid made up of a million photodiodes in a camera can
take a picture, but the information in each diode is independent from all the others. You could cut
the grid into two pieces and they would still take the same picture.

Consciousness, Dr.~Tononi says, is nothing more than integrated information. Information theorists
measure the amount of information in a computer file or a cellphone call in bits, and Dr.~Tononi
argues that we could, in theory, measure consciousness in bits as well. When we are wide awake, our
consciousness contains more bits than when we are asleep.

For the past decade, Dr.~Tononi and his colleagues have been expanding traditional information
theory in order to analyze integrated information. It is possible, they have shown, to calculate how
much integrated information there is in a network. Dr.~Tononi has dubbed this quantity phi, and he
has studied it in simple networks made up of just a few interconnected parts. How the parts of a
network are wired together has a big effect on phi. If a network is made up of isolated parts, phi
is low, because the parts cannot share information.

But simply linking all the parts in every possible way does not raise phi much. ``It's either all
on, or all off,'' Dr.~Tononi said. In effect, the network becomes one giant photodiode.

Networks gain the highest phi possible if their parts are organized into separate clusters, which
are then joined. ``What you need are specialists who talk to each other, so they can behave as a
whole,'' Dr.~Tononi said. He does not think it is a coincidence that the brain's organization obeys
this phi-raising principle.

Dr.~Tononi argues that his Integrated Information Theory sidesteps a lot of the problems that
previous models of consciousness have faced. It neatly explains, for example, why epileptic seizures
cause unconsciousness. A seizure forces many neurons to turn on and off together. Their synchrony
reduces the number of possible states the brain can be in, lowering its phi.

Dr.~Koch considers Dr.~Tononi's theory to be still in its infancy. It is impossible, for example, to
calculate phi for the human brain because its billions of neurons and trillions of connections can
be arranged in so many ways. Dr.~Koch and Dr.~Tononi recently started a collaboration to determine
phi for a much more modest nervous system, that of a worm known as Caenorhabditis elegans. Despite
the fact that it has only 302 neurons in its entire body, Dr.~Koch and Dr.~Tononi will be able make
only a rough approximation of phi, rather than a precise calculation.

``The lifetime of the universe isn't long enough for that,'' Dr.~Koch said. ``There are immense
practical problems with the theory, but that was also true for the theory of general relativity
early on.''

Dr.~Tononi is also testing his theory in other ways. In a study published this year, he and his
colleagues placed a small magnetic coil on the heads of volunteers. The coil delivered a pulse of
magnetism lasting a tenth of a second. The burst causes neurons in a small patch of the brain to
fire, and they in turn send signals to other neurons, making them fire as well.

To track these reverberations, Dr.~Tononi and his colleagues recorded brain activity with a mesh of
scalp electrodes. They found that the brain reverberated like a ringing bell, with neurons firing in
a complex pattern across large areas of the brain for 295 milliseconds.

Then the scientists gave the subjects a sedative called midazolam and delivered another pulse. In
the anesthetized brain, the reverberations produced a much simpler response in a much smaller
region, lasting just 110 milliseconds. As the midazolam started to wear off, the pulses began to
produce richer, longer echoes.

These are the kinds of results Dr.~Tononi expected. According to his theory, a fragmented brain
loses some of its integrated information and thus some of its consciousness. Dr.~Tononi has gotten
similar results when he has delivered pulses to sleeping people -- or at least people in dream-free
stages of sleep.

In this month's issue of the journal Cognitive Neuroscience, he and his colleagues reported that
dreaming brains respond more like wakeful ones. Dr.~Tononi is now collaborating with Dr.~Steven
Laureys of the University of Li\`ege in Belgium to test his theory on people in persistent
vegetative states. Although he and his colleagues have tested only a small group of subjects, the
results are so far falling in line with previous experiments.

If Dr.~Tononi and his colleagues can get reliable results from such experiments, it will mean more
than just support for his theory. It could also lead to a new way to measure consciousness. ``That
would give us a consciousness index,'' Dr.~Laureys said.

Traditionally, doctors have measured consciousness simply by getting responses from patients. In
many cases, it comes down to questions like, ``Can you hear me?'' This approach fails with people
who are conscious but unable to respond. In recent years scientists have been developing ways of
detecting consciousness directly from the activity of the brain.

In one series of experiments, researchers put people in vegetative or minimally conscious states
into fMRI scanners and asked them to think about playing tennis. In some patients, regions of the
brain became active in a pattern that was a lot like that in healthy subjects.

Dr.~Tononi thinks these experiments identify consciousness in some patients, but they have serious
limitations. ``It's complicated to put someone in a scanner,'' he said. He also notes that thinking
about tennis for 30 seconds can demand a lot from people with brain injuries. ``If you get a
response I think it's proof that's someone's there, but if you don't get it, it's not proof of
anything,'' Dr.~Tononi said.

Measuring the integrated information in people's brains could potentially be both easier and more
reliable. An anesthesiologist, for example, could apply magnetic pulses to a patient's brain every
few seconds and instantly see whether it responded with the rich complexity of consciousness or the
meager patterns of unconsciousness.

Other researchers view Dr.~Tononi's theory with a respectful skepticism.

``It's the sort of proposal that I think people should be generating at this point: a simple and
powerful hypothesis about the relationship between brain processing and conscious experience,'' said
David Chalmers, a philosopher at Australian National University. ``As with most simple and powerful
hypotheses, reality will probably turn out to be more complicated, but we'll learn something from
the attempt. I'd say that it doesn't solve the problem of consciousness, but it's a useful starting
point.''

Dr.~Tononi acknowledged, ``The theory has to be developed a bit more before I worry about what's the
best consciousness meter you could develop.'' But once he has one, he would not limit himself to
humans. As long as people have puzzled over consciousness, they have wondered whether animals are
conscious as well. Dr.~Tononi suspects that it is not a simple yes-or-no answer. Rather, animals
will prove to have different levels of consciousness, depending on their integrated information.
Even C.~elegans might have a little consciousness.

``Unless one has a theory of what consciousness is, one will never be able to address these
difficult cases and say anything meaningful,'' Dr.~Tononi said.

\pagebreak
\section{Just Me and My Pessimism in the `Race of Truth'}

\lettrine{M}{y}\mycalendar{Sept.'10}{21} husband found the bike race, but then he refused to train
seriously for it. He was ambivalent, not sure he wanted to compete. But I signed us up anyway. I
like having goals, and we had never raced on our bikes before. I wanted to see how we'd do.

It was not what we expected -- in fact, that race was quite a shock. But it certainly taught us some
lessons. Our experience turned out to be a perfect illustration of the power of one sort of mental
strategy in racing and a perfect example of what motivates some people to stay with a sport.

The 36-kilometer (22-mile) race was a time trial: racers go off one by one at 30-second intervals
and are ranked according to their speeds. I'm used to running races where everyone starts at once
and you can be energized by the crowd and pace yourself by watching other runners. There are markers
at every mile, and often there are clocks or race volunteers calling out times so you can gauge your
effort.

Here you are all alone. No mile markers. No crowds. Just you and the lonely road. Bill, my husband,
says that is why time trials are called the ``race of truth.''

Then there were the other riders. Women, it seems, do not do this race. There were only eight of us,
compared with 73 men. But every racer, man or woman, looked like a serious athlete. Almost everyone
had aero bars -- special handlebars that look like horns and allow you to crouch into an aerodynamic
position. Many had aero helmets, which are designed to decrease air resistance, as well as aero
wheels, which reduce the drag on your bike. Almost everyone was a member of a cycling team or club.

We had none of this special equipment. We were not on a team. My heart sank -- what had we signed up
for?

We lined up according to our race numbers and set off, one by one. I was so nervous I forgot to set
my bike computer, so I had no idea how far I had gone or how far I had to go. Nor did I know my
average speed. It was like nothing I had ever done before, and I was not having fun. I kept telling
myself just to keep going. I had told my friends I was riding, and I knew they would want to know
how I did. I did not want to embarrass myself with a ridiculously slow time.

Finally, the race was over.

``I'm never doing that again,'' I said to Bill. ``I was the slowest woman.'' Well, he told me, he
felt sure he was the slowest man. We drove home, chastened. So much for thinking we were the fastest
cyclists on the road. I was glad running is my real passion. It is not fun to feel like such a
failure.

The next day the race organizers posted the results.

What a surprise. I came in sixth out of the eight women. And I beat two men and tied with one. Bill
beat six men and six women.

Of course, we were still at the bottom of the heap, but instantly our moods changed. We should do
this again, Bill said. Only we should train, get aero bars and race again in the spring. Yes, I
agreed. I'd love to see what we could do.

How weird. From despair to hope, just because we did better than we thought we did.

But actually, that's no surprise, sports psychologists say.

The way we started thinking when we saw the other cyclists is a strategy called defensive pessimism,
said John S.~Raglin, a sports psychologist at Indiana University. He explained that it consisted of
``downplaying your ability and expectations.'' That way, if you do poorly you are not crushed, and
if you do better than you expected, ``you get this payoff,'' Dr.~Raglin said.

He has done studies of track-and-field college athletes who employ the defensive pessimism strategy,
comparing them with optimists who think they'll do well. The pessimists performed just as well as
the optimists.

On the other hand, the type of anxiety we felt when we saw the other riders is also a reason many
people steer clear of competitive sports altogether -- even a reason many avoid walking into gyms,
said Ralph A.~Vernacchia, director of the Center for Performance Excellence at Western Washington
University.

``What you are looking at is a social comparison,'' he said. ``In sports, it is very visible. There
are times, they are posted on Web sites, everybody sees them.''

And many people ``never get started because they are so fearful of what can go wrong,''
Dr.~Vernacchia said. The trick is to take the first step, he added. If you are the slowest in the
race, you can train and do better next time. It helps to have a coach or a team or a support group,
though, as Dr.~Vernacchia learned for himself.

He ran his first race in 1960, when he was a freshman in high school. It was a one-mile race, on a
cinder track, and he came in dead last.

``I was so far behind that when I came off the last turn to put on my blazing kick, they had already
put up the high hurdles'' for the next event, he recalled.

But his coach and his friends encouraged him to stay with the sport, and he did. He trained hard,
became a track athlete in high school and college, and ended up as a track coach. ``It became my
career,'' Dr.~Vernacchia said.

Not everyone wants to compete, of course. Recreational athletes, says David B.~Coppel, a sports
psychologist at the University of Washington, take pleasure in a sport for its own sake and often
feel no need to train and see how well they can do.

``They don't care about their times,'' he said. ``It's not a mastery issue. They are not trying to
conquer a hill, but just trying to climb it.''

Then there are the goal-oriented people whose motivation is somewhat different. And I guess the time
trial showed where Bill and I stand.

``We have potential,'' Bill said to me.

\pagebreak
\section{Bill Clinton, in Demand, Stumps for Obama}

\lettrine{H}{e}\mycalendar{Sept.'10}{21} was against him before he was for him.

During the 2008 presidential campaign, Bill Clinton was often at angry odds with the man who
ultimately defeated his wife.

Now, in the final weeks of the 2010 midterm campaign, Mr.~Clinton is stumping hard to help his
onetime foe -- and has emerged as one of the most important defenders of President Obama's
Congressional majorities. Some candidates are asking for his help on the campaign trail, rather than
the president's. Even though Mr.~Clinton insisted on Monday that he was only ``peripherally and
fleetingly'' back in politics, he has been headlining rallies and fund-raisers across the country to
buck up the depressed party faithful.

``They shouldn't take this lying down,'' Mr.~Clinton said during a meeting with reporters and
editors of The New York Times on Monday.

Blaming Republican policies for digging the deep hole the economy is in, he said the Democrats
needed to plead with voters for more time to turn things around.

``I think we ought to say, `Look, don't go back to the shovel brigade -- give us two more years; if
you don't feel better you can throw us all out.' ''

Mr.~Clinton professes more interest in pressing humanitarian problems like clearing rubble in
earthquake-ravaged Port-au-Prince -- the stuff of his day job at his Clinton Global Initiative
charitable organization -- than monitoring turnout projections in Portsmouth, N.H.

In an interview that lasted roughly 90 minutes, Mr.~Clinton talked passionately about his plans for
the annual meeting this week of his nonprofit organization and about the need to more aggressively
develop green industries and improve educational opportunities for women around the world.

But Mr.~Clinton has not lost his fluency, and grasp of granular detail, in politics. For instance,
he readily recalled the number of voters in a recent election for county executive in Westchester
County, where he lives.

Mr.~Clinton at times spoke in the apocalyptic terms of his party's more dejected activists.

``I hope they can avoid a calamity,'' Mr.~Clinton said at one point, hastening to add, ``And I think
they can.''

``If the election is about anger and apathy colored by amnesia, we're in deep trouble,'' he said.
``If the election is about what are we going to do now and who's more likely to do that, the
president and the Congress have a real chance to come out of this fine.''

He called this year only ``partly'' a replay of 1994, the first midterm election of his presidential
tenure, when Democrats lost their majorities in Congress. ``The Republicans are not led as adroitly
as they were when Newt Gingrich had a whole plan,'' he said.

On the other hand, he said, Mr.~Gingrich, who took over as speaker of the House in 1994, was willing
to work with the White House post-election. He accused this year's Republicans of promising
gridlock.

Whatever his feelings about Mr.~Obama in 2008, Mr.~Clinton is clearly feeling the president's
strain.

``Most of the things they're saying about him they said about me, so I'm much more sympathetic to
him than most people,'' he said. ``And when you get in there, if you're an earnest policy wonk like
he is and I was, it's hard to believe there are people who really don't want you to do your job.''

In the last two weeks Mr.~Clinton has campaigned for candidates in Georgia, Nevada, Ohio and
Pennsylvania, with plans to appear in California and Massachusetts in the days ahead. He has been a
guest on ``The Daily Show With Jon Stewart,'' ``Meet the Press'' and even ``On the Record with Greta
Van Susteren'' on the Fox News Channel on Monday night.

``He's welcome anywhere in the country,'' said Gov.~Edward G.~Rendell, Democrat of Pennsylvania, who
spent a day campaigning with Mr.~Clinton around Philadelphia last week. ``He's all upside and no
downside.''

The Big Dog, as he is known among those in the tight world of Clinton associates, is actually a more
diminutive dog these days -- so slimmed down that the ``Meet the Press'' moderator, David Gregory,
asked him on Sunday, ``Are you O.K.?'' (The once voracious consumer of junk food says he lost weight
at the insistence of his daughter, Chelsea, before her wedding in July, and credited a new
heart-healthy diet.)

Mr.~Obama, of course, is also the boss of Mr.~Clinton's wife, Secretary of State Hillary Rodham
Clinton. And there is something in it for Mr.~Clinton, too. Any microphone or tape recorder is a
two-fer for him, giving him a chance to talk up his global works, as well as discuss politics. On
Monday, for example, he urged donors to give more so he can hire more staff for the daunting
rebuilding efforts in Haiti.

Mr.~Clinton said he began making appearances only because Mrs.~Clinton, in her current position, is
forbidden from politicking. ``So there was nobody besides me to honor the people that supported
her,'' he said.

It ``became something more,'' he added, when he realized how angry the electorate was, and how
people did not appreciate what he described as the Democrats' accomplishments on student loans,
jobs, health care and the economy.

On the trail, he tweaks his remarks depending on which Democrat he is speaking for -- a different
speech for a candidate who supported health care than for one who did not.

He does not find it awkward that candidates want him at their side, rather than the current occupant
of the White House.

``In '94,'' he said, ``some of them didn't want me.''

\pagebreak
\section{As Rules Shift, Donor Names Stay Secret}

\lettrine{C}{rossroads}\mycalendar{Sept.'10}{21} Grassroots Policy Strategies would certainly seem
to the casual observer to be a political organization: Karl Rove, a political adviser to President
George W.~Bush, helped raise money for it; the group is run by a cadre of experienced political
hands; it has spent millions of dollars on television commercials attacking Democrats in key Senate
races across the country.

Yet the Republican operatives who created the group earlier this year set it up as a 501(c)(4)
nonprofit corporation, so its primary purpose, by law, is not supposed to be political.

The rule of thumb, in fact, is that more than 50 percent of a 501(c)(4)'s activities cannot be
political. But that has not stopped Crossroads and a raft of other nonprofit advocacy groups like it
-- mostly on the Republican side, so far -- from becoming some of the biggest players in this year's
midterm elections, in part because of the anonymity they afford donors, prompting outcries from
campaign finance watchdogs.

The chances, however, that the flotilla of groups will draw much legal scrutiny for their campaign
activities seem slim, because the organizations, which have been growing in popularity as conduits
for large, unrestricted donations among both Republicans and Democrats since the 2006 election, fall
into something of a regulatory netherworld.

Neither the Internal Revenue Service, which has jurisdiction over nonprofits, nor the Federal
Election Commission, which regulates the financing of federal races, appears likely to examine them
closely, according to campaign finance watchdogs, lawyers who specialize in the field and current
and former federal officials.

A revamped regulatory landscape this year has elevated the attractiveness to political operatives of
groups like Crossroads and others, organized under the auspices of Section 501(c) of the tax code.
Unlike so-called 527 political organizations, which can also accept donations of unlimited size,
501(c) groups have the advantage of usually not having to disclose their donors' identity.

This is arguably more important than ever after the Supreme Court decision in the Citizens United
case earlier this year that eased restrictions on corporate spending on campaigns.

Interviews with a half-dozen campaign finance lawyers yielded an anecdotal portrait of corporate
political spending since the Citizens United decision. They agreed that most prominent, publicly
traded companies are staying on the sidelines.

But other companies, mostly privately held, and often small to medium size, are jumping in, mainly
on the Republican side. Almost all of them are doing so through 501(c) organizations, as opposed to
directly sponsoring advertisements themselves, the lawyers said.

``I can tell you from personal experience, the money's flowing,'' said Michael E.~Toner, a former
Republican F.E.C. commissioner, now in private practice at the firm Bryan Cave.

The surging popularity of the groups is making the gaps in oversight of them increasingly worrisome
among those mindful of the influence of money on politics.

``The Supreme Court has completely lifted restrictions on corporate spending on elections,'' said
Taylor Lincoln, research director of Public Citizen's Congress Watch, a watchdog group. ``And 501(c)
serves as a haven for these front groups to run electioneering ads and keep their donors completely
secret.''

Almost all of the biggest players among third-party groups, in terms of buying television time in
House and Senate races since August, have been 501(c) organizations, and their purchases have
heavily favored Republicans, according to data from Campaign Media Analysis Group, which tracks
political advertising.

They include 501(c)(4) ``social welfare'' organizations, like Crossroads, which has been the top
spender on Senate races, and Americans for Prosperity, another pro-Republican group that has been
the leader on the House side; 501(c)(5) labor unions, which have been supporting Democrats; and
501(c)(6) trade associations, like the United States Chamber of Commerce, which has been spending
heavily in support of Republicans.

Charities organized under Section 501(c)(3) are largely prohibited from political activity because
they offer their donors tax deductibility.

Campaign finance watchdogs have raised the most questions about the political activities of the
``social welfare'' organizations. The burden of monitoring such groups falls in large part on the
I.R.S. But lawyers, campaign finance watchdogs and former I.R.S. officials say the agency has had
little incentive to police the groups because the revenue-collecting potential is small, and because
its main function is not to oversee the integrity of elections.

The I.R.S. division with oversight of tax-exempt organizations ``is understaffed, underfunded and
operating under a tax system designed to collect taxes, not as a regulatory mechanism,'' said Marcus
S.~Owens, a lawyer who once led that unit and now works for Caplin \& Drysdale, a law firm popular
with liberals seeking to set up nonprofit groups.

In fact, the I.R.S. is unlikely to know that some of these groups exist until well after the
election because they are not required to seek the agency's approval until they file their first tax
forms -- more than a year after they begin activity.

``These groups are popping up like mushrooms after a rain right now, and many of them will be out of
business by late November,'' Mr.~Owens said. ``Technically, they would have until January 2012 at
the earliest to file anything with the I.R.S. It's a farce.''

A report by the Treasury Department's inspector general for tax administration this year revealed
that the I.R.S. was not even reviewing the required filings of 527 groups, which have increasingly
been supplanted by 501(c)(4) organizations.

Social welfare nonprofits are permitted to do an unlimited amount of lobbying on issues related to
their primary purpose, but there are limits on campaigning for or against specific candidates.

I.R.S. officials cautioned that what may seem like political activity to the average lay person
might not be considered as such under the agency's legal criteria.

``Federal tax law specifically distinguishes among activities to influence legislation through
lobbying, to support or oppose a specific candidate for election and to do general advocacy to
influence public opinion on issues,'' said Sarah Hall Ingram, commissioner of the I.R.S. division
that oversees nonprofits. As a result, rarely do advertisements by 501(c)(4) groups explicitly call
for the election or defeat of candidates. Instead, they typically attack their positions on issues.

Steven Law, president of Crossroads GPS, said what distinguished the group from its sister
organization, American Crossroads, which is registered with the F.E.C. as a political committee, was
that Crossroads GPS was focused over the longer term on advocating on ``a suite of issues that are
likely to see some sort of legislative response. `' American Crossroads' efforts are geared toward
results in this year's elections, Mr.~Law said.

Since August, however, Crossroads GPS has spent far more on television advertising on Senate races
than American Crossroads, which must disclose its donors.

The elections commission could, theoretically, step in and rule that groups like Crossroads GPS
should register as political committees, which would force them to disclose their donors. But that
is unlikely because of the current make-up of the commission and the regulatory environment,
campaign finance lawyers and watchdog groups said. Four out of six commissioners are needed to order
an investigation of a group. But the three Republican commissioners are inclined to give these
groups leeway.

Donald F.~McGahn, a Republican commissioner, said the current commission and the way the Republican
members, in particular, read the case law, gave such groups ``quite a bit of latitude.''

\pagebreak
\section{Short of Repeal, G.O.P. Will Chip at Health Law}

\lettrine{R}{epublicans}\mycalendar{Sept.'10}{21} are serious. Hopeful of picking up substantial
numbers of seats in the Congressional elections, they are developing plans to try to repeal or roll
back President Obama's new health care law.

This goal, though not fleshed out in a detailed legislative proposal, is much more than a campaign
slogan. That conclusion emerged from interviews with a wide range of Republican lawmakers, who said
they were determined to chip away at the law if they could not dismantle it.

House Republicans are expected to include some specifics in an election agenda they intend to issue
Thursday. Although they face tremendous political and practical hurdles to undoing a law whose
provisions are rapidly going into effect, they are already laying the groundwork for trying.

For starters, Republicans say they will try to withhold money that federal officials need to
administer and enforce the law. They know that even if they managed to pass a wholesale repeal,
Mr.~Obama would veto it.

``They'll get not one dime from us,'' the House Republican leader, John A.~Boehner of Ohio, told The
Cincinnati Enquirer recently. ``Not a dime. There is no fixing this.''

Republicans also intend to go after specific provisions. Senator Orrin G.~Hatch of Utah, a senior
Republican on the Finance Committee, has introduced a bill that would eliminate a linchpin of the
new law: a requirement for many employers to offer insurance to employees or pay a tax penalty. Many
Republicans also want to repeal the law's requirement for most Americans to obtain health insurance.

Alternatively, Republicans say, they will try to prevent aggressive enforcement of the requirements
by limiting money available to the Internal Revenue Service, which would collect the tax penalties.

Republicans say they will also try to scale back the expansion of Medicaid if states continue to
object to the costs of adding millions of people to the rolls of the program for low-income people.

In addition, Republican lawmakers may try to undo some cuts in Medicare, the program for older
Americans. Many want to restore money to Medicare's managed-care program and clip the wings of a new
agency empowered to recommend cuts in Medicare. Recommendations from the agency, the Independent
Payment Advisory Board, could go into effect automatically unless blocked by subsequent legislative
action.

Representative Michael C.~Burgess, Republican of Texas and a physician, acknowledged that repealing
the law became more difficult with each passing week, as various provisions took effect and were
woven into ``the fabric of American life.''

Michael A.~Needham, chief executive of Heritage Action for America, who is leading a campaign for
repeal, said, ``There will be technical challenges in unwinding the legislation.''

Many Republican candidates for Congress have emphasized their desire to repeal the health care law.
Their vow is an election issue, and more -- a commitment they mean to pursue, regardless of the
election results.

Efforts at repeal face several hurdles:

¶ Not even the most optimistic Republicans expect to gain the two-thirds majorities that would be
needed to overcome a veto.

¶ The law responds to a genuine need. The Census Bureau reported last week that 50.7 million people
were uninsured in 2009, an increase of 4.3 million or nearly 10 percent over the previous year.

¶ The health care law saves money, by the reckoning of the Congressional Budget Office, so
Republicans would need to find ways to achieve equivalent savings if they repealed the law. (The
budget office affirmed last month that the law would ``produce \$143 billion in net budgetary
savings'' over 10 years.)

¶ While trying to repeal the health care law, Republicans do not agree on what to replace it with.

¶ Popular and unpopular provisions of the law are intertwined and difficult to separate. People like
the idea of being able to buy insurance regardless of any pre-existing condition. They dislike the
idea of being compelled to do so. But without such a requirement, people could wait until they got
sick and then buy coverage -- a situation that has proved unworkable in states that have tried it.

Administration officials, frustrated at not getting more of a political benefit from the law, plan
to highlight consumer protections that take effect this week, six months after Mr.~Obama signed it.

In general, insurers will be required to offer coverage to children with pre-existing conditions;
will have to allow many young adults to stay on their parents' policies up to age 26; cannot impose
lifetime limits on coverage of ``essential health benefits''; and cannot charge co-payments for
recommended preventive services.

Kathleen Sebelius, the secretary of health and human services, said that repeal would mean ``taking
those benefits away.''

Moreover, she said, if Congress repeals the law, small businesses will lose tax credits that help
pay for health benefits, and officials will lose tools needed to root out fraud in Medicare and
Medicaid.

House Republicans said their agenda for the next Congress would focus on jobs, spending, national
security and adjustments in House rules and procedures, as well as health care.

Senator Lamar Alexander of Tennessee, the No.~3 Republican in the Senate, said: ``If there was a
straight bill to repeal the health care law, I would vote for it because I think it's such a
historic mistake. If that doesn't succeed, I think we'll go step by step. We can try to delay
funding of some provisions and remove some of the taxes.''

However, Mr.~Alexander added: ``The bill included several items that Republicans are for, such as
dealing with pre-existing conditions. So we would be careful to include those in any final
legislation.''

Senator Olympia J.~Snowe, Republican of Maine, said she supported efforts to make major changes in
the law because she doubted that it would work. The law, she said, could inadvertently create an
incentive for employers to discontinue coverage because the financial penalties for not offering
insurance are far less than the cost of providing it.

Mr.~Hatch is working on alternatives that would give states more discretion about how to expand
coverage. ``I would prefer to have 50 state laboratories doing it rather than the almighty federal
government,'' he said.

Without changing a word of the law, Republicans, especially if they gain control of the House or
Senate, can put Democrats on the defensive with Congressional hearings and investigations intended
to expose problems.

Republicans said they would also try to override or rewrite some of the regulations issued by the
Obama administration without a full opportunity for public comment. They could do so by attaching
provisions to spending bills for the relevant regulatory agencies, among other methods.

``Wholesale repeal is highly unlikely,'' said Gail R.~Wilensky, who ran Medicare and Medicaid from
1990 to 1992. ``But if there is a significant shift in the makeup of Congress, which appears likely,
Republicans could definitely impact the regulations.''

Senator Michael B.~Enzi, Republican of Wyoming, said he was particularly concerned about rules that
could make it difficult for employers to keep their current health plans intact. Many Republicans
also want to repeal provisions that restrict the use of health savings accounts and flexible
spending arrangements for medical expenses.

How far Republicans go will depend on how many votes they have. ``The more Republicans we have here,
the more changes we can make,'' Mr.~Alexander said.

\pagebreak
\section{Students Spell Out Messages on Immigration}

\lettrine{D}{ozens}\mycalendar{Sept.'10}{21} of college students lay down on South Beach on Sunday
afternoon, but not to sunbathe. Most were immigrants in this country illegally, and their bodies,
fully clothed, formed giant letters that spelled out a message for Floridians and one of their
senators, complete with a human exclamation point: Call LeMieux!

The students staged the surfside demonstration after Senator Harry Reid, the Democratic majority
leader, announced last week that he would add to a military spending bill an amendment that would
open a path to legal status for hundreds of thousands of illegal immigrant students. Senator George
LeMieux, a Republican, has not declared his position, and the students hoped to secure his support
for the measure, which will be put to a first test on Tuesday with a procedural vote.

Illegal immigrant students across the country have not been deterred by reports from Washington that
the measure, known to its supporters as the Dream Act, has slim chances of passing. Republicans have
denounced Mr.~Reid's move to even bring it up just six weeks before midterm elections as a ploy to
attract Latino voters during his own hard-fought re-election campaign in Nevada, and they say a
proposal on an issue as contentious as immigration should not be attached to the military
reauthorization bill.

But in Phoenix and Boston, immigrant students who want to enlist in the military under the terms of
the student bill performed drills over the weekend and on Monday in front of the offices of Senator
John McCain, Republican of Arizona, and Senator Scott P.~Brown, Republican of Massachusetts.

In Utah, students started a call-in campaign and held a sit-down vigil to draw the attention of
Senator Orrin G.~Hatch, a Republican who was one of the first sponsors of the student bill nearly a
decade ago, but has not made clear how he will vote this time. In California, immigrants wearing
caps and gowns started public fasts, scheduled rallies and unfurled banners over highways. Actions
were also reported in nine other states.

``We are literally asking people to stop their lives to support the bill,'' said Carlos Saavedra,
president of the United We Dream coalition, a national immigrant student organization.

Miguel Sanchez, 19, a Miami Dade College student who has been in the United States illegally since
he was 10, was one side of an ``A'' in the human billboard message on Miami Beach, which was filmed
by television cameras. Mr.~Sanchez, who is from Honduras, said he hung back for years, worried that
public protest could lead immigration agents to locate and deport him. But he said that if he
continued much longer without legal status, he would not be able to transfer to a larger university
to complete college.

``All of a sudden I've lost that fear,'' Mr.~Sanchez said.

The urgency for change among illegal immigrant students has made them the most outspoken flank of
the movement pushing for legislation to open a path to legal status for millions of immigrants here
illegally. Immigrants who were brought to this country unlawfully as children can generally complete
public high school without problems, but they hit a wall when they try to go to college. They cannot
receive public financial aid and in many states must pay high out-of-state tuition rates. They
cannot obtain driver's licenses, or in many cases licenses to practice a skill or profession even if
they manage to graduate.

``We just found a wall, an obstacle that we couldn't overcome,'' said Guillermo Reyes, 26, a student
at Florida Atlantic University who participated in the South Beach protest. Immigration authorities
detained Mr.~Reyes last year but deferred his deportation, so for now he is authorized to work and
attend school. But for the long term, he said, ``that is really the only hope that we have, for the
Dream Act to pass.''

While Mr.~Reid's decision to bring up the student legislation took many Republicans by surprise, it
was the result of recent discussions by Democratic leaders and White House officials with immigrant
advocates and student leaders, several participants in those talks said.

The advocates argued that after President Obama did not deliver on repeated promises to pass a
larger immigration overhaul early in his term, the Democrats had to show some action on immigration
before midterm elections that seem likely to bring gains for Republicans.

Mr.~Reid was persuaded by impatient student leaders who said they did not want the measure to wait
for a larger immigration law overhaul, Congressional aides said.

Mr.~LeMieux's offices received many phone calls in recent days, from callers both for and against
the immigration bill, said his spokesman, Ken Lundberg.

The news of the vote on the student bill brought an outpouring from academic leaders, after many
college presidents have declared their support in recent years. Michael M.~Crow, the president of
Arizona State University, and David J.~Skorton, the president of Cornell, sent a letter in June to
2,200 university and college presidents asking them to urge lawmakers to pass the bill.

The focus in Arizona this year has been on curbing illegal immigration, with a crackdown bill
enacted in April. (Important sections of that law were stayed by a federal court.) But in an
interview Monday, Mr.~Crow said his university, with 70,100 students, was supporting hundreds of
immigrants each year who lack legal status by raising private money.

``There are thousands and thousands of students who were successful in public school, who did
everything right and didn't do anything wrong on their own,'' Mr.~Crow said. ``The bill is their
pathway to innocence.''

The student bill would open a path to eventual legal residency for illegal immigrants who arrived in
the country before they were 16 years old, have been here for at least five years and have graduated
from high school. It would require them to finish two years of college or military service before
gaining legal status.

About 726,000 illegal immigrants would become immediately eligible for legal status under the bill,
according to the Migration Policy Institute, a research group in Washington.

The Defense Department has listed passage of the student bill among the goals in its formal
strategic plan for the next two years. But many Republicans argued that the student measure was not
directly related to military reauthorization. In an interview on Fox News last week, Mr.~McCain, who
has supported the student bill in the past, said that in this instance Mr.~Reid and other Democratic
leaders ``have put their political agenda ahead of the welfare of the men and women who are serving
in the military today.''

On South Beach, the students' emotions ran high on Sunday, as many laughed and some cried when they
spoke of their frustrations over trying to stay in college, and being left behind by peers who are
American citizens. But Mr.~Reyes said they were not considering defeat. ``None of us have the idea
of the Dream Act not passing even being conceived in our minds,'' he said.

\pagebreak
\section{Code That Tracks Users' Browsing Prompts Lawsuits}

\lettrine{S}{andra}\mycalendar{Sept.'10}{21} Person Burns used to love browsing and shopping online.
Until she realized she was being tracked by software on her computer that she thought she had
erased.

Ms.~Person Burns, 67, a retired health care executive who lives in Jackson, Miss., said she is wary
of online shopping: ``Instead of going to Amazon, I'm going to the local bookstore.''

Ms.~Person Burns is one of a growing number of consumers who are taking legal action against
companies that track computer users' activity on the Internet. At issue is a little-known piece of
computer code placed on hard drives by the Flash program from Adobe when users watch videos on
popular Web sites like YouTube and Hulu.

The technology, so-called Flash cookies, is bringing an increasing number of federal lawsuits
against media and technology companies and growing criticism from some privacy advocates who say the
software may also allow the companies to create detailed profiles of consumers without their
knowledge.

Unlike other so-called HTML cookies, which store Web site preferences and can be managed by changing
privacy settings in a Web browser, Flash cookies are stored in a separate directory that many users
are unaware of and may not know how to control.

Ms.~Person Burns, a claimant who is to be represented by KamberLaw, said she knew cookies existed
but did not know about Flash cookies.

``I thought that in all the instructions that I followed to purge my system of cookies, I thought I
had done that, and I discovered I had not,'' she said. ``My information is now being bartered like a
product without my knowledge or understanding.''

Since July, at least five class-action lawsuits filed in California have accused media companies
like the Fox Entertainment Group and NBC Universal, and technology companies like Specific Media and
Quantcast of surreptitiously using Flash cookies. More filings are expected as early as this week.

The suits contend that the companies collected information on the Web sites that users visited and
from the videos they watched, even though the users had set their Web browser privacy settings to
reject cookies that could track them.

``What these cases are about is the right of a computer user to dictate the terms by which their
personal information is harvested and shared. This is all about user control,'' said Scott
A.~Kamber, 44, a privacy and technology lawyer with KamberLaw who is involved with some of the
cases. The suits have been filed by firms including Parisi \& Havens and the law office of Joseph
H.~Malley.

One lawsuit contends that Clearspring Technologies and media companies including the Walt Disney
Internet Group ``knowingly authorized'' the use of online tracking devices that would ``allow access
to and disclosure of Internet users' online activities as well as personal information.'' Others say
that the information was gathered to sell to online advertisers.

In August, Clearspring and Quantcast issued statements on their company blogs addressing the suits.
Clearspring clarified its use of Flash cookies and said the legal filings were ``factually
inaccurate.'' The company said it used Flash cookies, also known as Flash local storage, ``to
deliver standard Web analytics to publishers.'' The post also stated that data was collected at the
aggregate level including unique users and interaction time, but did not include personally
identifiable information.

Quantcast's blog post said that the company ``uses Flash cookies for measurement purposes only and
not for any form of targeted content delivery.''

Specific Media did not respond to requests for comment. Counsel for the media companies in the cases
declined to comment; representatives of companies that had not yet been served with the suits also
declined to comment.

Some privacy advocates said that despite the companies' claims, if enough data is collected over
time, advertisers can create detailed profiles of users including personally identifiable data like
race and age in addition to data about what Web sites a user visits. They also take issue with the
fact that Flash cookies can be used to restore HTML cookies that have been deleted from a user's
computer, circumventing a user's privacy settings.

``The core function of the cookie is to link what you do on Web site A to what you do on Web site
B,'' said Peter Eckersley, a technologist at the Electronic Frontier Foundation. ``The Flash cookie
makes it harder for people to stop that from happening.''

According to Adobe, more than 75 percent of online videos are delivered using Flash technology, with
media companies also using it to serve games and animation to users. The company says that Flash
cookies are intended to be used for basic Web functions like saving a user's volume and language
preferences or remembering where a user left off on a video game.

In a public letter to the Federal Trade Commission in January, Adobe condemned the practice of
restoring cookies after they had been deleted by a user. The company provides an online tool on its
Web site to erase Flash cookies and manage Flash player settings. At least one suit, however, claims
that the controls are not easy to reach and are not obvious to most Web users.

Chris Jay Hoofnagle, 36, one of the authors of a University of California, Berkeley, study about
Internet privacy and Flash cookies that has been used in several of the legal filings, said the
recent spate of suits pointed to a weakness in federal rules governing online privacy.

``Consumer privacy actions have largely failed,'' Mr.~Hoofnagle said. The lawsuits, he added,
``actually are moving the policy ball forward in the ways that activists are not.''

Complaints about online privacy are now migrating to mobile technology. Last week, a lawsuit was
filed by three California residents against a technology company called Ringleader Digital saying
that the company used a product called Media Stamp that ``acquired information from plaintiff's
phone and assigned a unique ID to their mobile device.''

The suit says that the information collected by the unique ID, using a technology called HTML 5,
allowed Web site operators ``to track the mobile devices' Internet activities over multiple Web
sites.''

In a statement, Bob Walczak Jr., Ringleader's chief executive, said, ``Our intent since the
inception of the company has been to build a mobile advertising platform that users can control.''
He added that Ringleader was working on ``new ways for consumers to be able to verify for themselves
that their opt-outs have taken effect.''

John Verdi, senior counsel at the Electronic Privacy Information Center, faulted the Federal Trade
Commission for not being more aggressive on privacy issues, focusing largely, instead, on
self-regulation.

``The F.T.C. has been inactive on this front and has failed to present meaningful regulation on
this,'' he said. ``There's wide evidence that online tracking is not being controlled by
self-regulation.''

Christopher Olsen, an assistant director in the division of privacy and identity protection at the
agency, said it had hosted a series of roundtable discussions about online and offline privacy
challenges from December to March and planned to issue a report in the next few months to address
those issues.

The agency is investigating several companies, but Mr.~Olson declined to comment on the specifics.

Other efforts to address online privacy are taking place at the Congressional level. In July,
Representative Bobby L.~Rush, Democrat of Illinois, introduced an online privacy bill that would,
among other things, require companies to disclose how they collect, use and maintain the personal
information on users and to make those disclosures easy for users to understand.

\pagebreak
\section{Recession May Be Over, but Joblessness Remains}

\lettrine{T}{he}\mycalendar{Sept.'10}{21} United States economy has lost more jobs than it has added
since the recovery began over a year ago.

Yes, you read that correctly.

The downturn officially ended, and the recovery officially began, in June 2009, according to an
announcement Monday by the official arbiter of economic turning points. Since that point, total
output -- the amount of goods and services produced by the United States -- has increased, as have
many other measures of economic activity.

But nonfarm payrolls are still down 329,000 from their level at the recession's official end 15
months ago, and the slow growth in recent months means that the unemployed still have a long slog
ahead.

``We started from a deep hole,'' said James Poterba, an economics professor at M.I.T. and a member
of the National Bureau of Economic Research's Business Cycle Dating Committee, which declared the
recession's end. ``And clearly the bounce-back has not been immediate after hitting this trough.''

The declaration of the recession's end confirms what many suspected: The 2007-9 recession was not
only the longest post-World War II recession, but also the deepest, in terms of both job losses and
at least one measure of output declines.

The announcement also implies that any contraction that might lie ahead would be a separate and
distinct recession, and one that the Obama administration could not claim to have inherited. While
economists generally say such a double-dip recession seems unlikely, new monthly estimates of gross
domestic product, released by two committee members, show that output shrank in May and June, the
most recent months for which data are available. Output and other factors would have to shrink for a
longer period of time before another contraction might be declared.

Even without a full-blown double dip in the economy, the recovery thus far has been so anemic that
the job picture seems likely to stagnate, and perhaps even get worse, in the near future.

Many forecasters estimate that output needs to grow over the long run by about 2.5 percent to keep
the unemployment rate, now at 9.6 percent, constant. The economy grew at an annual rate of just 1.6
percent in the second quarter of this year, and private forecasts indicate growth will not be much
better in the third quarter. (The Business Cycle Dating Committee itself does not engage in
forecasting.)

``The amount of unemployment we've already got and the slowness of recovery lead to predictions that
we could have 9-plus percent unemployment even through the next presidential election,'' said Robert
J.~Gordon, an economics professor at Northwestern University and a committee member.

``What's really unique about this recession is the amount of unemployment in combination with the
slowness of the recovery,'' he said. ``That's just not happened before. We had a sharp recession
followed by a sharp recovery in the 1980s. And in '91 and '01 we had slow recoveries, but those
recessions were shallow recessions, so the slowness didn't matter much.''

All three of these most recent recoveries have been known as jobless recoveries, as employment
growth has significantly lagged output growth. In this recovery, the job market bottomed six months
after economic output bottomed. That is still not nearly as much of a lag as experienced after the
2001 recession, when it took the job market 19 months to turn around after output improved.

This new pattern of jobless recoveries has led to some complaints that employment should play a more
prominent role in dating business cycles and to criticism that a jobless recovery is not truly a
recovery at all. Business Cycle Dating Committee members have been reluctant to change their
criteria too drastically, though, because they want to maintain consistency in the official
chronology of contractions and expansions.

While all three recent recoveries have been weak for employment, the job market has to cover the
most ground from the latest recession.

From December 2007 to June 2009, the American economy lost more than 5 percent of its nonfarm
payroll jobs, the largest decline since World War II. And through December 2009, the month that
employment hit bottom, the nation had lost more than 6 percent of its jobs.

The unemployment rate, which comes from a different survey, peaked last October at 10.1 percent. The
postwar high was in 1982, at 10.8 percent. But the composition of the work force was very different
in the 1980s -- it was younger, and younger people tend to have higher unemployment rates -- and so
if adjusted for age, unemployment this time around actually looks much worse.

The broadest measure of unemployment, including people who are reluctantly working part time when
they wish to be working full time and those who have given up looking for work altogether, also was
at its highest level since World War II.

There is some debate, though, about whether this recession was the worst in terms of output.

Adjusted for inflation, output contracted more than in any other postwar period, according to Robert
E.~Hall, a Stanford economics professor and committee chairman.

But some economists say that a better measure would be the gap between where output is and where it
could have been if growth had been uninterrupted.

``It's definitely not as deep as 1981-82 when measured relative to the economy's potential growth
rate,'' Mr.~Gordon said.

Besides employment, nearly every indicator that the committee considers simultaneously reached a low
point in June 2009, which made that month a relatively easy selection as the official turning point,
Mr.~Gordon said. The committee previously met in April but had decided that the data were
inconclusive.

In its statement on Monday affirming the recession's end, the bureau took care to note that the
recession, by definition, meant only the period until the economy reached its low point -- not a
return to its previous vigor.

``In declaring the recession over, we're not at all saying the unemployment rate, or anything else,
has returned to normal,'' said James H.~Stock, an economics professor at Harvard and a member of the
business cycle committee.

``We clearly still have a long ways to go.''

\pagebreak
\section{H.P. Settles Lawsuit Against Hurd}

\lettrine{A}{}\mycalendar{Sept.'10}{21} fierce and public feud between Oracle and Hewlett-Packard,
two of the world's largest technology companies, has ended after all of two weeks.

On Monday, the companies announced a settlement to a dispute that centered on Oracle's hiring of
Mark V.~Hurd, the former chief executive of H.P., as a president. H.P. sued Mr.~Hurd this month,
claiming he would violate agreements to protect H.P.'s secrets by taking on such a high-level role
at Oracle. The parties declined to reveal details about the settlement but said Mr.~Hurd would
protect H.P.'s confidential information.

However, in a filing with the Securities and Exchange Commission on Monday, H.P. said it had
modified its separation agreement with Mr.~Hurd. He effectively waived about half the compensation
owed him. Mr.~Hurd agreed to give up his rights to the 330,177 performance-based restricted stock
units granted to him on Jan.~17, 2008, and to the 15,853 time-based restricted stock units granted
on Dec.~11, 2009.

Although most legal analysts said H.P. had had little chance of winning its case, the lawsuit
immediately strained the business relationship between the two companies. Oracle and H.P. have a
long history of selling technology together. About 40 percent of Oracle's business software runs on
computing systems sold by H.P., and the companies have 140,000 customers in common. After the
lawsuit was filed -- 19 hours after Oracle hired Mr.~Hurd -- Lawrence J.~Ellison, Oracle's chief
executive, warned that H.P.'s actions threatened to derail the companies' longstanding partnership.

The companies took pains on Monday to say that the business relationship was again on firm footing.
``H.P. and Oracle have been important partners for more than 20 years and are committed to working
together to provide exceptional products and service to our customers,'' Cathie A.~Lesjak, the chief
financial officer and interim chief executive at H.P., said in a statement. ``We look forward to
collaborating with Oracle in the future.''

Mr.~Ellison said in his statement, ``Oracle and H.P. will continue to build and expand a partnership
that has already lasted for over 25 years.''

``The partnership is clearly very important here,'' said David M.~Hilal, senior managing director at
FBR Capital Markets. ``It's undoubtedly an effort to kiss and make up.''

Mr.~Hilal said Mr.~Hurd would probably be prohibited from making decisions at Oracle that would
allow him to use confidential information from H.P., like its acquisition plans. The relationship
between the two companies began to fray after Mr.~Hurd resigned from H.P. last month.

In an e-mail to The New York Times, Mr.~Ellison, a close friend of Mr.~Hurd's, lambasted H.P.'s
board for the way it had handled the departure. Mr.~Hurd left H.P. after the board investigated his
relationship with a marketing contractor and found that her name had been left off expense report
items and that Mr.~Hurd had violated the company's code of conduct.

``In losing Mark Hurd, the H.P. board failed to act in the best interest of H.P.'s employees,
shareholders, customers and partners,'' Mr.~Ellison wrote.

This month, Oracle hired Mr.~Hurd to succeed Charles E.~Phillips Jr.~as a president at the company.
While the legal matter has been resolved, Oracle and H.P. will continue to have a more tense
business relationship than in the past.

Oracle's acquisition this year of Sun Microsystems thrust it into the computer hardware business,
one of H.P.'s strong suits.

At the Oracle Open World customer event here this week, Oracle executives talked at length about
their plans to conquer the hardware market. Mr.~Ellison, in particular, made an impassioned pitch on
Sunday evening, just minutes after Ann M.~Livermore, an H.P. executive vice president in charge of
enterprise computing, delivered a similar message to the audience.

Oracle executives have voiced their interest in acquiring more hardware companies, and H.P. remains
on the prowl, making some recent big-ticket purchases. H.P. has made three major acquisitions since
Mr.~Hurd left the company: 3Par, a computer storage company, for \$2.35 billion; ArcSight, a
computer security company, for \$1.5 billion; and Stratavia, a privately held database and
application automation company, for an undisclosed amount.

H.P. also remains in the hunt for a new chief executive.

\pagebreak
\section{NKorea to Hold Key Party Convention Next Week}

\lettrine{N}{orth}\mycalendar{Sept.'10}{21} Korea will hold its biggest political meeting in 30
years next week, state media said Tuesday, as observers watch for signs that the secretive regime's
aging leader will appoint his son to succeed him.

Now 68, and reportedly in poor health two years after suffering a stroke, Kim Jong Il is believed to
be setting in motion a plan to tap a son to take the Kim dynasty into a third generation by
appointing his heir to top party posts at the Workers' Party convention.

Delegates will meet Sept.~28 to elect new party leaders, the official Korean Central News Agency
said in a dispatch from Pyongyang.

The report did not explain why the meeting, initially set for ``early September,'' had been
postponed. North Korea has been struggling to cope with devastating flooding and a typhoon that
killed dozens of people and destroyed roads, railways and homes earlier this month, according to
state media.

Delegates across the country were appointed ``against the background of a high-pitched drive for
effecting a new great revolutionary surge now under way on all fronts for building a thriving nation
with the historic conference,'' the KCNA report said.

State media have been building up the rhetoric ahead of the conference, the first major Workers'
Party gathering since the landmark 1980 congress where Kim Jong Il, then 38, made his political
debut, in an appearance seen as confirmation that he would eventually succeed his father, North
Korea founder Kim Il Sung.

Kim Jong Il took over in 1994 when his father died of heart failure in what was communism's first
hereditary transfer of power.

Now, he seems to prepping his son for a similar transition. Little is known about the son widely
believed to be his father's favorite. Kim Jong Un, said to be in his late 20s and schooled in
Switzerland, has never been mentioned in state media, and there are no confirmed photos of him as an
adult.

South Korean intelligence officers believe Pyongyang has launched a propaganda campaign promoting
the son, including songs and poems praising the junior Kim. North Korean soldiers and workers
reportedly pledged allegiance to the son on his birthday in January.

North Korea's state propaganda machine has also been churning out commentaries calling for loyalty
to the Kim family, an apparent effort to set to the stage for a smooth power transition.

Delegates are expected to elect new party leaders to fill spots left vacant for years. It's not
known what party position Kim Jong Un might be granted in what would be his first official job.

Keen attention is also focused on Kim Jong Il's only sister, Kim Kyong Hui, who in the past two
years has been a frequent companion to the leader on field trips to army bases and factories. She
currently serves as the political party's department chief for light industry.

Her husband, Jang Song Thaek, has also been rising in stature. Jang was promoted in June to a vice
chairman of the powerful National Defense Commission, making him the No.~2 official to Kim Jong Il
on the regime's top state organ.

The conference is being held amid preparations for the milestone 65th anniversary of the founding of
the Workers' Party on Oct.~10, improving relations with Seoul, and attempts by diplomats from
neighboring nations to revive dormant six-nation disarmament negotiations on North Korea's nuclear
weapons program.

North Korea walked away from the talks last year in protest over U.N. Security Council condemnation
for launching a long-range rocket, widely seen as a test of its missile technology. Pyongyang
followed that act of defiance by testing a nuclear bomb weeks later.

In March, a South Korean warship went down in the waters near the Koreas' western maritime border,
and an international team of investigators blamed a North Korean torpedo for ripping apart the
Cheonan and killing 46 sailors.

But after months of tensions, there have been signs of a thaw in relations recently, with Seoul
announcing a shipment of emergency flooding aid to North Korea and the two Koreas agreeing to hold
military talks.

\pagebreak
\section{Jade From China's West Surpasses Gold in Value}

\lettrine{A}{s}\mycalendar{Sept.'10}{21} long as anyone here can remember, the muddy river that
flows through this oasis city in southern Xinjiang has yielded creamy white stones, their rough
edges polished smooth by the waters that tumble down the mountains from Tibet.

And as long as anyone can remember, those stones -- a type of semitranslucent jade -- were about as
valued as, well, a pile of river rocks.

Lohman Tohti, 30, can recall as a child heaving melon-size hunks into the sandbags that were used to
thwart rising floodwaters of the aptly named White Jade River. When Chinese buyers began arriving
here in the early 1990s and the locals got wind of the stones' potential value, his uncle made an
enviable deal: he traded a rock the girth of a well-fed hog for a skinny cow. ``Today, my uncle
would be a millionaire,'' Mr.~Tohti, now a jade dealer, said with a wince.

These days, Khotan is mad about jade, or at least the riches it has brought to a city whose previous
bout of prosperity occurred a few thousand years ago, when traders from ancient Rome and
Constantinople were making their way toward Xi'an, then the capital of the Chinese empire and the
eastern terminus of the Silk Road.

Ounce for ounce, the finest jade has become more valuable than gold, with the most prized nuggets of
``mutton fat'' jade -- so-named for its marbled white consistency -- fetching \$3,000 an ounce, a
tenfold increase from a decade ago.

The jade boom, which appears to have reached a frenzy in the past year or two, has been fueled by
the Chinese, whose new wealth and a 5,000-year affinity for the stone has turned Khotan cotton
farmers into jade tycoons.

``The love of jade is in our blood, and now that people have money, everyone wants a piece around
their neck or in their home,'' Zhang Xiankuo, a Chinese salesman, said as he opened a safe to show
off his company's most expensive carved items, among them a pair of kissing swans that retails for
\$150,000 and a contemporary rendition of a Tang dynasty beauty, her breasts impertinently exposed,
that can be purchased for \$80,000.

In a region convulsed by ethnic strife, it is notable that the manna appears to have enriched both
Khotan's native Uighurs, Turkic-speaking adherents of Islam, and the more recently arrived Han
Chinese, who are often viewed unlovingly as rapacious colonizers.

The Uighurs have largely made their fortune harvesting jade from the river and selling it to Chinese
middlemen. Because devout Muslims are proscribed from dealing in certain representational images,
the Han have come to monopolize the carving and sale of Buddhist figurines, stalking tigers and the
miniature cabbages that are popular among Chinese consumers.

``Jade has no meaning for our culture, but we are thankful to Allah that the Chinese go crazy for
it,'' said Yacen Ahmat, a Uighur who spends seven days a week working the crowds at Khotan's jade
bazaar, a frenetic marketplace dominated by prospectors trying to unload their catch on savvy
wholesalers -- or hapless tourists who often return home with overpriced rocks.

Hu Xianli, a self-professed jade fanatic from eastern Zhejiang Province, said he had been duped
countless times over the years. At best, he has grossly overpaid for mediocre specimens. At worst,
he has mistaken chemically treated rocks for mutton-fat beauties.

A retired railway engineer, he likened his relationship with jade to an overpriced college
education. ``In the early years I paid a lot of tuition, but now that I've finally graduated, I'm
not so easily fooled,'' said Mr.~Hu, 59, as a throng of overeager sellers, hands full of egg-size
stones, thrust their wares into his face.

Although archaeologists have unearthed Neolithic jade tools along the Yellow River, the Chinese
affection for the stone received a lift around 1600 B.C., when Shang dynasty royals took to sleeping
on jade pillows, signing edicts with jade chops and interring their loved ones in jade-tile frocks.
Legend suggests that only emperors were allowed to possess carved jade and that the pursuit of an
especially cherished specimen might be worth the deaths of 10,000 soldiers. It is no coincidence
that the Chinese character for king has the same root as the character for jade.

Contemporary Chinese is flecked with references to jade -- the word is used to describe beautiful
and pure women -- and many people say they believe it has medicinal and even magical powers. A chip
of jade worn around the wrist can soothe a frightened child, improve circulation or absorb bad
energy, the Chinese say. According to an age-old belief, jade provides a link between the physical
and spiritual worlds.

Some of those beliefs are bolstered by the jade's tendency to change color when worn on the body.

``Perhaps it's psychological, but as jade rubs against the skin it becomes smoother and softer, and
the wearer becomes happier, which probably improves their immunity,'' said Wang Shiqi, a Peking
University geologist who specializes in jade.

Another reason behind the spike in Khotan jade prices, at least according to traders, is that the
jade is becoming increasingly scarce. Over the past decade, bulldozers and excavators have torn
apart the banks of the White Jade River several times over. Until the practice was banned three
years ago, mining companies, some owned by local government officials, would divert the river in
their quest to find new quarry.

Even if river jade is increasingly hard to find, the promise of instant riches brings entire
families to the river, where they can be seen, heads bowed, pacing the banks. The lucky ones head
straight to the bazaar, where crowds of Uighur men in embroidered skullcaps ogle and haggle over the
latest finds.

Skeptics, however, say the rising prices have more to do with hype than scarcity. Wang Chunyun, a
jade expert at the Guangzhou Institute of Geochemistry, says a thick lode of unexploited white jade
runs through the Kunlun Mountains that skirt Xinjiang and Tibet. It can also be found across the
world, from Australia to Korea to Poland, where a lack of demand keeps it unmined. ``The rarity of
jade is a myth,'' Mr.~Wang said in a telephone interview. ``I've never said this to Chinese
businessmen because it would be too much of a psychological blow.''

Back at the market, Ai Shan Zhang, a well-to-do Uighur salesman, shook his head and smiled when it
was suggested that Khotan jade might not be as precious as diamonds. The Chinese zeal for it is so
great, he said, that he has stopped wearing it, especially when meeting with government officials
whose favor is sometimes required in the course of doing business. ``If they notice a nice piece
hanging around my neck, they ask to borrow it,'' he said. ``And once they take it, they never give
it back.''

\pagebreak
\section{Migrants' Plight Touches a Point of Swedish Pride}

\lettrine{T}{his}\mycalendar{Sept.'10}{21} land of tall fir trees and spacious lakes is famed for
its wild berries: cloudberries, blueberries, lingonberries, gooseberries, raspberries, strawberries
and more.

The problem is, no Swede wants to pick them, except, perhaps, for a couple of handfuls for dessert.

So a few decades ago, the big companies in Sweden that are among the world's largest berry
processors began importing workers for the laborious task of sweeping the rock-strewn woods and
climbing the mountains to find and harvest the berries. The practice of bringing in migrants --
Chinese, Thais, Vietnamese, Bangladeshis -- for the summer berry season seemed to work well, until
recent years.

When a cold snap this spring decimated the berry crop, it punished the industry and its workers with
a second straight poor harvest.

After last year's disaster, which sent many of the pickers home weighed down with debt rather than
profit, the Swedish authorities required the berry companies to guarantee a minimum wage of about
\$2,320 for the season. To skirt the law, however, the fruit companies in Sweden hired Asian pickers
through recruiting companies in their home countries. Since these companies were not Swedish, they
were not bound by Swedish law, and they refused to pay their workers the minimum after the crop
failed.

For Sweden, which prides itself on worker-friendly labor legislation -- and which sent 20 members of
a far-right, anti-immigrant party to Parliament in elections last weekend -- the berry pickers
quickly became the source of acute national embarrassment, with attention focused particularly on
190 Bangladeshi pickers who arrived in this modest town of pastel wooden homes earlier this year.

They had been hired in their native land by a company called Bangladesh Work Force, with the promise
of a small fortune -- the sums seemed to vary, depending on who did the promising -- if they went to
Sweden to pick berries for the summer. Many Bangladeshis put up as much as 150,000 taka, about
\$2,100, to middlemen at Bangladesh Work Force in a country where a well-paying job as a garment
worker brings in about \$42 a month, said Mahmudur Rahman, 31, a biotechnology student at Uppsala
University whose brother and brother-in-law were among workers who arrived last spring.

Mr.~Rahman said the workers were told they could earn nearly \$10,000 in two months, with even lazy
pickers pulling in more than \$5,000.

To earn the full amount, he said they were told, pickers would have to harvest 60 kilograms, about
130 pounds, of berries a day. Given the meager harvest, most of the pickers ended up bringing in as
little as 10 pounds, some only a pound or so, meaning they earned little or no money.

Some took out loans or sold property before coming to Sweden, Mr.~Rahman said. ``It was a summer job
that went wrong,'' he said bitterly.

Iqbal Akhtar, 43, left behind a wife and three children, and was told he could be a cook rather than
a berry picker, preparing meals for about 35 men. He was paid the average of what the berry pickers
received, which was virtually nothing. ``All in all, it's been very negative,'' he said. To fill
their days the men seek odd jobs in town, painting, mowing lawns or working on farms, though little
such work is available.

As summer waned, most of the pickers returned home, helped by local church groups and the Swedish
Red Cross. But several dozen remain, housed in four unfurnished homes where they sleep on mattresses
on the floor.

Arifur Rahaman, 32, left a wife and son behind in Bangladesh to seek his fortune. On a recent
afternoon, he sat drinking a Coke with three friends behind the town's train station. ``No
lingonberries, no money,'' he said, lifting empty palms skyward.

The Nordic Food Group is among Scandinavia's biggest berry processors. Bertil Qvist, an executive
there, said the requirement that companies pay a minimum wage to the berry pickers forced reliance
on middlemen in Asia, and he blamed those companies for overstating the money to be made.

``People are invited here on false grounds,'' he said. ``They think they are going to pick
cultivated berries, and people are not used to the work.''

In good years, he said, berry pickers can take home a small fortune. ``Some years,'' he added,
``there's a bad crop; this year was a bad year.''

Local people, while agreeing it was a bad year, said the workers were not warned adequately of the
possibility. ``It's like gold in the forest, they're told, quick money,'' said Lars Riberth, pastor
of the town church, which has collected the equivalent of \$9,000 to help the men.

``It's really slavery,'' he said. It was unacceptable, he added, to push the risk onto the pickers'
shoulders.

Mr.~Rahman, the student at Uppsala, said: ``Some people made a lot of money, but it's not the
workers.''

The town's mayor, Sven-Ake Draxten, 57, a man who likes his Swedish meatballs with a side of
lingonberries, noted that local companies that charge the workers for lodging, food and
transportation are thriving, while ``the pickers are the losers all the time.''

``The government must make new regulations,'' he said. ``We don't want this situation next year.''

Thongkam Persson, 57, came to Sweden from her native Thailand 30 years ago, following a Swedish
husband. While she now runs a Thai restaurant in town, when she was younger she worked five seasons
picking berries. ``It strains your back, and the berries grow on the mountains, so you're
climbing,'' she said.

``The Swedes have money, so it's just Thais, Chinese and Vietnamese who pick the berries,'' she
said. ``The Swedes go hiking in the forest.''

\pagebreak
\section{Political Earthquake Shakes Up Sweden}

\lettrine{W}{orthy}\mycalendar{Sept.'10}{21}, high-minded and often utterly predictable, Swedish
politics has rarely offered much by way of excitement. Now an electoral earthquake seems to have
changed all that.

Elections on Sunday gave an anti-immigration party its first parliamentary seats and deprived the
governing coalition of its majority, plunging the country into rare political instability.

Meanwhile the Social Democrats, architects of the modern Swedish state and one of Europe's most
successful political parties, recorded their worst performance since World War I.

Behind the upheaval lie structural changes in Swedish politics and a battle over how to preserve the
cradle-to-grave welfare system.

Though the success of the center-right suggests a long-term shift in politics, analysts say Swedes
remain deeply attached to their welfare system and want change to be gradual, not radical.

Despite failing to secure a majority, the prime minister, Fredrik Reinfeldt, is likely to become the
first center-right leader to win two consecutive terms.

He pleaded Monday for time to let the election result ``sink in'' and promised to have a government,
which could well be a minority one, in place by Oct.~4. He appeared to be courting the Greens, who
campaigned against him, and their initial response proved negative.

Mr.~Reinfeldt's options are limited, because he pledged not to work with the anti-immigration party,
Sweden Democrats, whose leader, Jimmie Akesson, described Muslim population growth as the biggest
external threat to his country since World War II.

Doing a deal with Mr.~Akesson would, in any event, not be logical for Mr.~Reinfeldt, whose success
has been built on moving to the center ground. Only in that way has he managed to tame the Social
Democrats, who for most of the last century had one of the most effective vote-winning political
machines in Western Europe. Their political philosophy forged a nation with high taxes and a
generous social safety net.

``There used to be a maxim in Swedish politics that you never won elections by offering to lower
taxes,'' said Martin Adahl, director of Fores, a center-right research institute. ``That was because
people would suspect that you were going to cut the welfare state. There has been a change, but
people still believe in the welfare system.''

This much was evident during a campaign in which even rightist populists presented themselves as
defenders of the welfare system, albeit in their case for white Swedes rather than immigrants. A
television advertisement, initially banned by one broadcaster on the ground of racism, portrayed a
stampede of Muslim women in burqas defeating a white retiree in a race for welfare payments.

The mainstream center-right parties, including Mr.~Reinfeldt's, the Moderates, acknowledged the
importance of welfare, too. The Moderate Party, which once modeled itself after Ronald Reagan and
Margaret Thatcher, is now careful to stress its centrist credentials.

Though Mr.~Reinfeldt's government curbed some benefits, most notably for those claiming to be too
sick to work, he was careful not to present himself as a radical reformer. During the campaign, Mona
Sahlin, leader of the Social Democrats, was at her most effective when health care became a central
issue.

Ursula Berge, political director of a white-collar Swedish trade union, Akademikerforbundet SSR,
said that the message from the Moderates to the electorate has been that they ``want to keep the
welfare state the way it is, just to change some small things like cheating by people who claim they
are sick but are able to work.''

While Mr.~Reinfeldt moved to the center, he was also helped by a weakening of traditional
allegiances among Swedish voters.

``There used to be a middle class that voted very solidly for the Social Democrats but there is now
a rather large part of the middle class that is outward oriented, mainly employed in the services
sector, that is not prepared to take for granted that the Social Democrats represent them anymore,''
Mr.~Adahl said.

But other factors aided the center-right, too. During testing times for the Swedish economy,
Mr.~Reinfeldt's government coped well. That allowed it to steal the clothes of the Social Democrats,
who were once regarded as the safest guardians of the economy.

For their part, the Social Democrats failed to react partly because, throughout much of the past
four years, they were ahead in opinion polls -- something that reduced their incentive to change.

``The Social Democratic Party has not renewed their politics,'' Ms.~Berge said. ``They were
comfortable with their level of support and they didn't have the courage to change policy.''

With the two main parties battling on similar terrain and with ideological dividing lines blurred,
that left an opening for the Sweden Democrats, who have capitalized on tensions raised by the loss
of industrial jobs and the rise in immigration.

``If you are an angry young man, you had to choose between a cuddly conservative and an urbane,
politically correct woman,'' Mr.~Adahl said. It was, he added, little surprise that some opted
instead for Mr.~Akesson.

But the mainstream politicians remain committed to the traditional social model, albeit a slightly
less generous one. Here in Sweden, when people come to vote, welfare remains more important than
their wealth.

Assuming he can form a government, Mr.~Reinfeldt will limit himself to reforms that underline that
there is a ``greater moral and economic difference between being on welfare and being in work,''
Mr.~Adahl said. ``It's not a revolution; it's definitely an evolution.''

\pagebreak
\section{Widespread Fraud Seen in Latest Afghan Elections}

\lettrine{E}{vidence}\mycalendar{Sept.'10}{25} is mounting that fraud in last weekend's
parliamentary election was so widespread that it could affect the results in a third of provinces,
calling into question the credibility of a vote that was an important test of the American and
Afghan effort to build a stable and legitimate government.

The complaints to provincial election commissions have so far included video clips showing ballot
stuffing; the strong-arming of election officials by candidates' agents; and even the handcuffing
and detention of election workers.

In some places, election officials themselves are alleged to have carried out the fraud; in others,
government employees did, witnesses said. One video showed election officials and a candidate's
representatives haggling over the price of votes.

Many of the complaints have come from candidates and election officials, but were supported by
Afghan and international election observers and diplomats. The fraud appeared to cut both for and
against the government of President Hamid Karzai, much of it benefiting sometimes unsavory local
power brokers.

But in the important southern province of Kandahar, where election officials threw out 76 percent of
the ballots in last year's badly tainted presidential election, candidates accused the president's
influential half brother, Ahmed Wali Karzai, of drawing up a list of winners even before the
Sept.~18 election for Parliament was carried out.

``From an overall democracy-building perspective it does not look rosy,'' said one diplomat who
asked not to be identified because he was not authorized to speak to the news media.

The widespread tampering and bare-knuckle tactics of some candidates raised serious questions about
the effort to build a credible government that can draw the support of Afghans and the Obama
administration and its NATO partners as they re-evaluate their commitment to the war.

American and international diplomats kept their distance from the tide of candidate complaints this
week, and NATO and American Embassy officials said little other than that the election was an Afghan
process and that it was the Afghans who were responsible for its outcome.

But a less than credible parliamentary election, following last year's tarnished presidential vote,
would place international forces in the increasingly awkward position of defending a government of
waning legitimacy, and diplomats acknowledged that it could undermine efforts to persuade countries
to maintain their financing and troop levels.

The Election Complaints Commission said Thursday that it had received more than 3,000 complaints
since last Saturday's election. So far they have registered case files on nearly 1,800 of those
complaints -- 58 percent of which were considered serious enough to affect the outcome of the
balloting. That may change in the course of investigations but that preliminary figure is high,
election monitors said.

The complaints are not evenly distributed and were markedly worse in 13 of Afghanistan's 34
provinces. In those 13, at least half the complaints were deemed to be high priority -- forecasting
bitter fights over the outcome.

In addition, complaints in four provinces -- Kandahar, Nuristan, Zabul and Paktika -- have yet to be
categorized, but fraud is expected to be extensive and has already been widely reported.

``That preliminary figure is bad,'' said a knowledgeable international observer.

Many analysts predicted there would be serious fraud in the unstable Pashtun belt, in the south of
the country, an important base for both the Taliban insurgents and President Karzai. But serious
complaints were also coming from provinces in the north and west.

Interviews by The New York Times in 10 provinces and discussions with election monitors elsewhere
found a resurgence of local strongmen with armed backers who coerced and threatened voters, and the
involvement of local government employees in ballot stuffing.

``In general the election has been a free-for-all, in that different power blocs were putting
forward their candidates in different places,'' said an international official who has been
following the elections.

``It's not necessarily the pro-Karzai bloc that has done so well, it's that the Parliament will be
more dependent on big power brokers,'' the official said, adding that they would be more likely to
make deals with Mr.~Karzai that did not necessarily serve the Afghan people.

Lawmakers and opposition candidates openly accused the Karzais, and in particular Ahmed Wali Karzai,
the most powerful official in Kandahar, of fixing the election for a list of favored candidates.

``Of the list of 50, it is already decided who will come'' to Parliament, said Izzatullah Wasefi, an
opposition candidate from Kandahar.

Nur ul-Haq Uloomi, a member of Parliament who won the largest vote from Kandahar in 2005, and has
since become an outspoken critic of the corruption and inefficiency of the Karzai government,
accused Ahmed Wali Karzai of manipulating the vote to deny him another term.

He said he had sent one of his campaign managers to the chairman of the Independent Election
Commission, Fazal Ahmad Manawi, in Kabul to warn of potential fraud before the election, but he was
rebuffed.

``Mr.~Manawi said: `We can do nothing about Kandahar because he is the brother of Karzai,' ''
Mr.~Uloomi recounted. ``It is a kind of preparation for fraud.''

Mr.~Manawi was too busy to take individual calls last week, his spokesman said.

In one Kandahar border district, Abdul Karim Achakzai, an independent candidate from Spinboldak,
said three groups of election workers were handcuffed and detained for the entire day of the
election by border police officers and prevented from conducting the vote in the Maruf district.

In the evening the polling papers with the results were brought to them to sign, but they refused.
They were freed the next day after promising not to complain, he said.

Mr.~Achakzai accused the provincial head of the border guards, Abdul Razziq, an ally of Ahmed Wali
Karzai, of orchestrating the detention. Mr.~Razziq, who has influence in several border districts,
was also accused of ballot-stuffing and intimidation in favor of President Karzai in the 2009
election, according to election observers.

A cellphone video from an adjoining district showed men ticking dozens of ballots in favor of
certain candidates. The video, which was recorded surreptitiously by a candidate's agent, also
captured a candidate's representatives and election officials inside a polling station haggling over
the price of votes.

``You will get as many votes as you asked, just pay 72,000 Afghanis (\$1,500),'' said the election
official, who identified himself as the head of the polling center.

In the northern province of Takhar, several witnesses described gunmen threatening election workers
and dragging voters to polling stations to vote for their candidate, Adbul Baqi. The abuse happened
in Farkhar district, according to one witness, Hassibullah, 35.

``Mr.~Baqi and his gunmen were slapping and pulling people to the ballot boxes to vote for him,'' he
said. ``He is a very cruel man.'' After that, he added, they went to the women's section of the
polling station and forced the female employees of the Independent Election Commission to put more
than 200 votes in their ballot box.

Abdul Haq, 50, another voter in Farkhar district, said that when he asked the security guards to
stop beating people, one of them attacked him with a knife. ``The candidate himself is a good man
and people do like him, but his dogs around him are not good,'' he said.

Mr.~Baqi could not be reach by phone for comment. The Independent Election Commission official for
the district, Engineer Kebir, said that the supporters of the candidate ``did make some disturbances
and violent acts and were threatening each other.'' But, he insisted, ``They did not disrupt the
election process.''

\pagebreak
\section{Games Official Angers India With Hygiene Comment}

\lettrine{H}{ad}\mycalendar{Sept.'10}{25} the statement come from a non-Indian, especially a
Westerner, it probably would have been angrily repudiated as an affront to Indian dignity. But the
offending words came from a top Indian official trying to deflect criticism for the bureaucratic
failings and lax preparations threatening the coming Commonwealth Games.

The issue was reports of unsanitary conditions inside the athletes' village, a facility promoted by
Indian organizers as world class. Officials of the New Zealand team, arriving early, had been
horrified at dirt-caked bathrooms and soiled rooms. The explanation offered by Lalit Bhanot, the
second-ranking official on the organizing committee? Indians and Westerners have different standards
of hygiene.

``These rooms are clean to both you and us,'' Mr.~Bhanot told Indian reporters this week. Foreigners
``want certain standards in hygiene and cleanliness which may differ from our perception,'' he said.

India had hoped the Commonwealth Games, a quadrennial athletic competition among nations of the
former British Empire, would serve as a public relations vehicle to show off the economic progress
that has made the country a rising power. Instead, the world is witnessing an ugly spectacle of
bureaucratic dysfunction that only confirms the image of governmental ineffectiveness that Indian
leaders hoped to dispel.

Indians acknowledge that sanitation is woefully lacking in many parts of the country. But
Mr.~Bhanot's suggestion that their government cannot, even under the glare of a global spotlight,
deliver a high standard of hygiene in an expensive new facility fueled broad public indignation that
rippled through television talk shows and Internet message boards.

``It is unbelievable that a person holding such a responsible position can make such a statement,''
said J.~Anand, vice president of a New Delhi travel agency. ``Hygiene is hygiene, whether it is in
India or anywhere else. I feel embarrassed by that statement.''

The dirty bathroom controversy is just the latest problem to plague the games. New Delhi has
experienced record monsoon rains, causing periodic flooding in low-lying areas and amplifying the
seasonal outbreak of mosquito-borne dengue fever. Missed deadlines have officials racing to finish
work, and a pedestrian bridge under construction collapsed this week, injuring 27 people. Already,
as many as nine athletes have dropped out of the Games.

Now the photographic leitmotif of the games is filthy bathrooms. Snapshots taken of an apartment in
the athletes' village with dirt-caked bathrooms and toilets, a mattress stained with dog paw prints
and a sink smeared with the spittle of chewing tobacco have ricocheted across the Internet. On
Friday, two leading Indian newspapers ran some of the photos on their front pages.

The dirty conditions have prompted several teams to delay their arrival into New Delhi, with only
eight days before the Oct.~3 opening ceremony. If the situation has been embarrassing, it mostly
reflected the abysmal management that has plagued the Games; some laborers had used the rooms during
construction, and housekeepers had failed to clean up on time.

Anyone living in India is inevitably confronted by squalor, whether the slums and shantytowns that
exist in most cities or the beggars, often children, who tap on car windows for change. Few Indians
would argue that poverty is not a paramount national concern, and many domestic critics of the games
argued that the country was still too poor to spend so much money on what is effectively a
government prestige event.

Mr.~Bhanot's comments hit a raw nerve because many middle class Indians make a distinction between
public and private standards. If public bathrooms in government buildings are usually dirty, private
homes are usually immaculate. Most people pay close attention to their appearance and cleanliness,
even as public roads are usually potholed and public buildings are often not well maintained.

``It's not that somehow people don't recognize the truth that there is a problem about public
standards of hygiene in India,'' said Pratap Bhanu Mehta, president of the Center for Policy
Research in New Delhi. ``But usually we have dealt with it by confining it to the public space. We
think private standards are very high. And he seemed to be questioning that.''

Foreigners visiting India invariably run into the chaotic, often dirty public sphere throbbing
outside their hotels, coloring their impressions of the country. Keshav R.~Murugesh, chief executive
of WNS Global Services, a Mumbai outsourcing company, said Indian companies needed to work hard to
persuade international customers that Indians could do complicated work in a timely and exacting
manner.

Mr.~Murugesh worried that clients, having seen the controversy over the athletes' village, would now
wonder: ``Is that how I'm being served?'' He jokingly added: ``I just wish they had outsourced it to
us.''

The irony of the cleanliness controversy is that the bathrooms in the athletes' village represent
the sort of luxury most Indians never experience. The walls are made with marble, and the sinks and
toilets appear equipped with expensive fixtures. For much of India, the lack of access to a toilet
and the absence of adequate sanitation are widespread problems blamed for the spread of disease as
well as the contamination of the country's rivers and other water sources.

Bindeshwar Pathak, founder of the Sulabh International Social Service Organization, has spent 40
years promoting the need to expand sanitation in India. His group has placed more than 1.2 million
household toilets around the country and operates a toilet museum in New Delhi to promote awareness
about sanitation.

Meanwhile, the effort to scrub the athletes' village moved into extra high gear on Friday. Prime
Minister Manmohan Singh ordered organizers to intensify cleaning efforts at the village immediately,
while the Indian news media reported that the government had asked several of the country's elite
private hotels to complement the cleaning effort.

Athletes trickling into the city were reportedly being put up temporarily in hotels, though
international sports officials, who have blasted the lack of preparations, sounded more optimistic
on Friday.

``Conditions at the Commonwealth Games Village are acceptable,'' said Perry Crosswhite, head of
Australia's delegation, according to The Associated Press. ``Things are getting better every time.''

For his part, Mr.~Bhanot has backpedaled, trying to argue that his comments were taken out of
context. But he is not the only local official who has offended the public.

When the hurriedly built pedestrian bridge near the main stadium collapsed during construction this
week, a senior official, Jaipal Reddy, tried to dismiss the accident as a ``minor matter,'' even
though many people thought it symbolized the risks of delaying preparations until the last minute.
Sheila Dikshit, chief minister of New Delhi, also initially described the collapse as ``minor.''

The Times of India ran photographs of the dirty bathrooms on Friday and denounced the ``criminal
unconcern'' of games officials.

``They must be made to pay,'' it blared, ``so that India's name is not dragged so willfully into the
mud ever again.''

\pagebreak
\section{Cuba Details New Policies on Budding Entrepreneurs}

\lettrine{C}{ubans}\mycalendar{Sept.'10}{25} learned on Friday the details of what they would soon
be able to do as budding entrepreneurs, including renting spaces for their businesses, hanging out a
shingle, and if things go well, hiring a few employees.

The Communist Party newspaper Granma published details of Cuba's new regulations on self-employment,
clearing a thicket of restrictions that had virtually choked off the country's minuscule private
sector.

``It's going to be a different kind of socialism,'' said Ted Henken, an expert on the Cuban private
sector at Baruch College of the City University of New York. The new policies could ``let out all of
these natural impulses to network, to contract out, to be efficient and productive.''

The Granma article was the latest step in a rollout of changes that Cuba plans to shake the nation's
economy out of its torpor.

In recent months, President Raúl Castro has said that Cuba's bloated state payroll needs to be
trimmed by as many as one million people. He warned that Cubans should no longer expect to get paid
if they do not work.

Last week, in the clearest sign that the government intended to act, the country's labor federation
announced that a half-million state workers would lose their jobs by next March and should seek work
in the private sector.

Friday's article outlined how many of them could go about doing that, listing 178 activities for
which the government will grant licenses starting in October.

Cubans will be allowed to work privately as carpenters or party clowns; they will be allowed to
repair computers or give music lessons. They can repair jewelry and carry passengers on their own
boats. Under the new rules, they can also begin to set up their own food businesses or workshops to
make shoes.

They may even be able to get loans to do it. The article highlighted that the Central Bank of Cuba
was studying how to make small-business loans available.

Many of these activities were first allowed during a brief window in the 1990s, but they never
flourished. The government stopped issuing licenses years ago, and only 144,000 Cubans are
officially self-employed.

Cuba's government has been ``zigging and zagging since 1990,'' looking for ways to revamp its
economic development strategy, Mr.~Henken said, but ``they have only been able to put out fires.''

Since taking over the presidency in 2008, Mr.~Castro has moved slowly to make changes. ``Raúl, in
economic terms, is much more of a pragmatist and much less of an ideologue'' than his brother and
predecessor as Cuba's president, Fidel Castro, Mr.~Henken said.

The Granma article said that the new policy, which also represents a sharp break because it will
allow entrepreneurs to hire employees who are not relatives, was intended ``to move away from those
conceptions which condemned self-employment almost to extinction and stigmatized'' those who began
working for themselves in the 1990s.

At the same time, new entrepreneurs will have to pay sales, employment and income taxes. In a sign
of the problems Cuba faces, Granma quoted Economy Minister Marino Murillo Jorge as saying that it
would be several years before the country could create a wholesale market to supply the new
businesses.

The government is expected to announce more changes soon, including regulations governing private
cooperatives.

``It looks as though Cuba's days of having a small, stagnant self-employment sector is over,''
Philip Peters, who follows Cuba for the Lexington Institute, a research group that promotes free
market policies, wrote on his blog about the country.

He added that ``if you take entrepreneurs and their employees, and add the yet-to-be-defined new
cooperatives,'' then ``it appears that a small and medium-sized business sector is on the horizon.''

\pagebreak
\section{A Thai Region Where Husbands Are Imported}

\lettrine{T}{he}\mycalendar{Sept.'10}{25} most dazzling creatures Nui Davis saw when she was a child
were the village girls who had found foreign husbands, visiting in their Western finery and handing
out candies to the children.

``For me, they were like a princess,'' she said. ``And I kept those pictures in my mind, and I made
a wish that one day I would like to be one of those ladies.''

Today, at the age of 30, she lives with her husband, Joseph Davis of Fresno, California, in an
air-conditioned, three-bedroom house with a driveway and basketball hoop, surrounded by flower beds
and a well-kept lawn.

``My family keeps saying, `You got it. You got your dream now,''' she said.

But unlike many other foreign husbands, Mr.~Davis, 54, did not take his wife home with him, choosing
instead to settle down in northeastern Thailand, a region known as Isaan.

He is part of an expanding population of nearly 11,000 foreign husbands in the region, drawn by the
low cost of living, slow pace of life and the exotic reputation of Thai women -- something like a
brand name for Western men seeking Asian partners. ``Thai women are a lot like women in America were
50 years ago,'' said Mr.~Davis, before they discovered their rights and became ``strong-headed and
opinionated.''

``The women now know they are equal,'' said Mr.~Davis, a retired Naval officer who has been divorced
twice, ``so the situation is not as relaxed and peaceful as it is between an American and a Thai
lady.''

It is easy to spot the foreigners' homes, with their sturdy walls and red-tiled roofs, an
archipelago of affluence among the smaller, poorer houses of their new neighbors and in-laws.

Mixed couples are common on the streets and in the markets of Udon Thani. One street where Western
men gather to eat and drink is popularly known as ``Foreign Son-in-Law Street.''

``There are villages in Isaan that are almost entirely comprising foreign houses, where the whole
village is almost entirely houses purchased by foreigners for their Thai ladies,'' said Phil Nicks,
author of ``Love Entrepreneurs: Cross-Culture Relationship Deals in Thailand.''

Isaan is one of the poorest parts of the country, the source of most low-wage workers in Bangkok and
the home of many of the women who work in the entertainment industry in the capital.

Some of the earliest Thai-American marriages were in Udon Thani, the site of a U.S.~air base in the
1960s during the Vietnam War. In the following years, most Americans left, sometimes taking a Thai
wife with them. Now the presence of American and European men is growing again. ``In the northeast
where this phenomenon is strongest, a huge majority of the women there are looking for a foreign
boyfriend or husband, and I think some of them can be quite assertive, and aggressive in their
pursuing of a foreign man,'' said Mr.~Nicks.

A clash of expectations strains many marriages, and more than half end in divorce, said Prayoon
Thavon, manager of international services at Panyavejinter Hospital in Udon Thani.

While the men -- many of them retired and living on pensions, many disappointed in their lives and
marriages at home -- may be seeking an emotional connection, the women are generally motivated by
economics, said Mr.~Prayoon, who provides counseling for mixed couples.

``For some ladies it is just money, money, money,'' he said. ``Getting married has become a business
more than love. People want to improve their social status. Sometimes these ladies spend the
husband's money, use it all, then he's cut out. There are many cases like that.''

Even though many men are retired and living on a fixed income, they are expected to help support
their wives' extended families, beginning with a dowry of several thousand dollars.

``When you get married in Thailand you are marrying the whole family, the whole village,''
Mr.~Prayoon said. ``Often the lady expects that, but the man doesn't understand.''

There seems to be less concern about differences in age, with many bridegrooms in their 50s or 60s
or even 70s.

``Age is not a factor here,'' said Mr.~Davis. ``In America if I marry a girl who is 24 years younger
than me, all you're going to get is eyes and bad talk, bad gossip. Here it's not an issue. It
happens every day.''

At the age of 63, Dennis Sorensen, a retired mathematics teacher, is 32 years older than his wife,
Pennapa, whom he met eight years ago on a beach. He spends much of his time watching U.S.~television
through a satellite hookup and cooks his own hamburgers, but he said he has done his best to keep
his wife and her family happy. He helps raise her teenage daughter from an earlier relationship as
well as their 2-year-old son.

``There's some adjustment there,'' said Mr.~Sorensen, for whom this is the first marriage, ``and
we've had issues where I run out of money and I cannot take care of everybody, and that has caused
some crises, but we've overcome everything the best that we can.''

One barrier is language, as few foreigners learn Thai. ``I can't speak English so well, but I can
live with him many years,'' Ms.~Sorensen said, speaking in Thai. ``Sometimes when he is very upset I
don't understand what he is talking about but I understand the tone and I just walk away.''

But she added in English: ``I think Dennis is good -- good for take care of my family, take care of
my daughter, take care of everything for me. Before, I don't have anything. But right now I have a
home, I have car, I not work and I only stay home and take care of my baby.''

Foreign marriage has become so common that it has lost much of its stigma here in Udon Thani, and
many girls share Ms.~Davis's dream of becoming a princess. ``It looks pretty good and they look
pretty happy,'' said Rojjana Udomsri, 30, who is married to a Thai man and has a 2-year-old son.
``They have money to spend and they can go anywhere they want.''

But she said she had her doubts.

``I don't know if they are really happy,'' she said. ``There were times I wanted to have a good life
like them, but I can't live with a person I don't love. With someone I love I can go through all the
hardships of life together.''

\pagebreak
\section{Obama Presses for Peace in Likely Sudan Partition}

\lettrine{A}{fter}\mycalendar{Sept.'10}{25} months of leaving Sudan policy on a back burner,
President Obama put the weight of his administration and his own personal esteem in Africa on the
line Friday, demanding that north and south Sudan ensure that their likely split into two nations
early next year proceed peacefully.

At a high-level Sudan meeting on the sidelines of the United Nations General Assembly, Mr.~Obama
said Washington would normalize relations should the Jan.~9 referendum for the independence of
southern Sudan be carried off calmly and the Darfur conflict be settled. Failing in either would
bring further isolation, he warned.

If Khartoum fulfills its obligations in settling the conflicts, then the United States will support
agricultural development, expand trade and investment, exchange ambassadors and eventually lift
sanctions, Mr.~Obama said.

``What happens in Sudan in the days ahead may decide whether a people who have endured too much war
move forward towards peace or slip backwards into bloodshed,'' Mr.~Obama said, establishing the
warning tone taken by all speakers at the session.

With the southern Sudan independence vote a little more than 100 days away, preparations are greatly
lagging. Voter registration that was to have been done by the end of August remains incomplete, and
many technical issues are unsettled. Almost nothing has been done to prepare for a secondary
referendum to decide the fate of Abyei, a contested area of important oil deposits.

The two sides are dragging their feet on details of the eventual divorce, given that the roughly 4.8
million eligible voters in the south, the only side voting, are likely to choose independence. With
independence scheduled six months after the vote, major issues like citizenship, borders and the
division of oil revenue have not been negotiated. At least 1.5 million southerners are believed to
be living in Khartoum, the capital, and an unknown number of northerners in the south.

United Nations officials had intended the meeting to be a small gathering of foreign ministers to
stress in the presence of senior representatives of both sides that the referendum not be delayed.

But it ballooned into something much larger after President Obama decided to attend. About 40
foreign leaders or senior officials signed up to speak. Ultimately the meeting became a highly
visible means to serve notice that the onus is on the Sudanese to carry out the last, hardest stages
of the Comprehensive Peace Agreement they signed in 2005.

All the governments participating, including major Security Council members like Russia and China,
as well as major pan-African organizations, endorsed a final communiqu\'e vowing that the referendum
would be held on time and peace respected no matter what the outcome. It also stressed the need to
support peacekeeping efforts in Darfur, the embattled western Sudanese region where violence flared
anew in recent months.

The Comprehensive Peace Agreement ended decades of war pitting the north, dominated by Arab Muslims,
against the Christian and animist south. The civil war left two million people dead and millions
more homeless.

Sudan borders on nine other African states, and many speakers worried aloud that any instability
``will not stop at its borders,'' said Sheik Hamad bin Khalifa al-Thani, the leader of Qatar, which
is trying to negotiate a peace among the Darfur militias.

The Sudanese had sought a closed meeting, but ultimately it was open, leaving Sudan's vice
president, Ali Osman Taha, and Salva Kiir, the president of the south and a vice president in the
north, to sketch their differences in diplomatic terms.

Mr.~Taha assured the gathering that the referendum would be held on time, but he criticized the
international community for supporting peace on one hand while continuing the ``demonization'' of
the north on the other. The International Criminal Court's indictment of Sudan's president, Omar
Hassan al-Bashir, on war crimes charges, economic sanctions, the lack of debt relief and Sudan's
presence on the United States list of state sponsors of terrorism were all intended to weaken the
country, he said.

Mr.~Kiir said that the Jan.~9 referendum date was sacred and that any technical delays had to be
overcome. ``Any delays risk the return to instability and violence,'' he said.

Numerous aid organizations, concerned that a lack of attention might encourage opponents of the
referendum to delay it, hailed the meeting. But some were outraged that the final communiqu\'e
welcomed a commitment by Sudan to pursue war criminals, given that Mr.~Bashir has mocked his own
indictment.

The final communiqu\'e could not really attack Mr.~Bashir while encouraging him to respect the
referendum deadline, said two senior officials involved with the negotiations, who said the
reference to Sudan's commitment to pursue war criminals was to ensure that the issue of impunity
would not be ignored.

\pagebreak
\section{Japan Retreats With Release of Chinese Boat Captain}

\lettrine{A}{}\mycalendar{Sept.'10}{25} diplomatic showdown between Japan and China that began two
weeks ago with the arrest of the captain of a Chinese trawler near disputed islands ended Friday
when Tokyo accepted Beijing's demands for his immediate release, a concession that appeared to mark
a humiliating retreat in a Pacific test of wills.

Japan freed the captain, Zhan Qixiong, 41, who left Saturday on a chartered flight sent by the
Chinese government to take him home. Mr.~Zhan had been held by the Japanese authorities since his
boat collided with Japanese patrol vessels on Sept.~7 near uninhabited islands in the East China
Sea, and Japan had insisted that he would be prosecuted.

His release handed a significant victory to Chinese leaders, who have ratcheted up the pressure on
Japan with verbal threats and economic sanctions.

``It certainly appears that Japan gave in,'' said Hiroshi Nakanishi, a professor of international
relations at Kyoto University. ``This is going to raise questions about why Japan pushed the issue
in the first place, if it couldn't follow through with meeting China's challenges.''

The climb down was the latest indicator of the shifting balance of power in Asia. China this year
surpassed Japan as the world's second largest economy and had already become Japan's biggest export
market. Japan, mired in extended political uncertainty and economic malaise, has had a succession of
weak prime ministers who have struggled to assert its interests in a region focused mainly on a
resurgent China.

China on Saturday restated its claims to the disputed islands and in a statement demanded an apology
and compensation. ``Such an act seriously infringed upon China's territorial sovereignty and
violated the human rights of Chinese citizens,'' the statement said.

At the outset, Japan had made an uncharacteristic display of political backbone by detaining the
captain, when in the past it had simply chased away Chinese vessels that approached too close to the
islands, which are claimed by both countries but administered by Japan. Apparently angered by a
rising number of incursions by Chinese fishing boats in recent years, Tokyo initially appeared
determined to demonstrate to Beijing its control of the islands, analysts and diplomats said.

Instead, the move unleashed a furious diplomatic assault from China. Beijing cut off
ministerial-level talks on issues like joint energy development, and curtailed visits to Japan by
Chinese tourists. The fact that the detention took place on Sept.~8, the anniversary of Japan's 1931
invasion of northeast China, spurred scattered street protests and calls by nationalistic Chinese
bloggers to take a firm stand against Tokyo.

In recent days, China stepped up its intimidation. Chinese customs officials appeared to block
crucial exports to Japan of rare earths, which are metals vital to Japan's auto and electronics
industries. Then on Thursday, four Japanese construction company employees were detained in the
Chinese province of Hebei.

In the end, diplomats and analysts said Japan was forced to recognize that taking the next step of
charging the captain and putting him on trial would result in a serious deterioration of ties with
China, Japan's biggest trading partner.

``At this point, Japan had only one choice,'' said a Western diplomat in Beijing, who spoke on the
usual diplomatic condition of anonymity. ``It had to charge the captain, or it would have to climb
down.''

It chose the latter. On Friday, prosecutors on the island of Ishigaki, where the captain was held,
cited diplomatic considerations in their decision to let him go, and suspended their investigation
into charges of obstructing officials on duty.

``Considering the effect on the people of our nation and on China-Japan relations, we decided that
it was not appropriate to continue the investigation,'' the prosecutors said in a statement.

Until Japan's sudden reversal on Friday, the tussle had grown to dominate both nations' diplomatic
agendas, including during the United Nations development summit meeting this week in New York.

Prime Minister Wen Jiabao of China had refused to meet on the sidelines of the meeting with Japan's
prime minister, Naoto Kan, and instead threatened additional actions if Japan did not release the
captain.

The Japanese used the summit meeting to seek American support for its position. They seemed to get
it when Secretary of State Hillary Rodham Clinton told Japan's new foreign minister, Seiji Maehara,
that America's treaty obligations to defend Japan from foreign attack would include any moves
against the islands where the Chinese captain had been arrested.

The islands, known as Senkaku in Japanese or Diaoyu in Chinese, are also claimed by Taiwan.

The fact that Japan seemed to back down after escalating the situation brought an outpouring of
criticism of Mr.~Kan, who was re-elected prime minister just two weeks ago. On Friday, members of
his own governing Democratic Party joined opposition lawmakers in condemning the decision to release
the captain.

``I'm flabbergasted that this was resolved with such a clear diplomatic defeat for Japan,'' said
Yoshimi Watanabe, leader of the opposition Your Party.

The setback appears likely to raise new concerns about the leadership of the Democrats, who took
power in a landslide election victory last year with promises to improve ties within Asia and reduce
Japan's dependence on the United States.

However, the standoff underscored how sentiment in Japan had hardened against China, even in recent
months. Ever more frequent movements by Chinese warships into Japanese waters have stirred fears
here that fast-growing China will become more aggressive in pushing its territorial claims.

However, there were also growing calls in Japan for a quick resolution to the standoff, particularly
by the business community, which has become increasingly reliant on China for trade and investment.
On Friday, the president of the Tokyo Stock Exchange, Atsushi Saito, told reporters he welcomed the
release.

``As a Japanese, I have mixed feelings about appearing so weak-kneed,'' Mr.~Saito said, ``but
realistically speaking, we had to put this problem behind us.''

In China, the captain's release appeared to be a victory for the leadership, and particularly the
prime minister, Mr.~Wen. The Communist Party is keen to show itself as defending China's territorial
claims, which enjoy strong emotional support from the Chinese people. China also views itself as
geopolitically hemmed in by Japan and other cold war-era American allies as it tries to take its
place as a regional power.

Chinese analysts agreed that Japan had appeared to fold, but said Tokyo had no choice if it wanted
to avoid a continued escalation with China.

``This was a move that Japan had to make or China would have taken further steps,'' said Wang
Xiangsui, a foreign policy analyst at the Beijing University of Aeronautics and Astronautics. ``Now
the two sides can discuss this more calmly.''

Mr.~Zhan, the trawler captain, arrived in the Chinese coastal city of Fuzhou at 4 a.m. local time,
according to the official Xinhua news agency. He was met at the airport by senior officials from the
Foreign Ministry and the Agriculture Ministry, which had chartered a plane to pick him up after his
release in Japan.

When the door of the plane opened, Mr.~Zhan was carrying flowers and immediately was greeted with
hugs by relatives waiting for him, Xinhua reported.

``Being able to return safely this time, I thank the party and government for their care,'' Mr.~Zhan
said. ``I also thank the Chinese people for their concern.''

\pagebreak
\section{Afghan Equality and Law, but With Strings Attached}

\lettrine{I}{t}\mycalendar{Sept.'10}{25} was an engaging idea.

Hundreds of children would gather on the iconic Nader Khan Hill in the capital, Kabul, on a gorgeous
Friday in September and fly kites emblazoned with slogans lauding the rule of law and equality for
women. The kites, along with copies of the Afghan Constitution and justice-themed comic books, would
be gifts of the United States, part of a \$35 million effort ``to promote the use of Afghanistan's
formal justice system.''

``The mere portrait of 500 kites soaring in the winds, against a backdrop of beautiful mountain
ranges, is enough to instill hope in even the most disheartened observer of the war-torn country,''
said a promotional release for the festival, organized by an American contractor for the United
States Agency for International Development.

What could possibly go wrong?

Almost everything but the wind.

For starters, Afghan policemen hijacked the event, stealing dozens of kites for themselves and
beating children with sticks when they crowded too close to the kite distribution tent. To be fair,
the children were a little unruly, but they were also small.

Sometimes the officers just threatened them with sticks, and other times slapped them in the face or
whacked them with water bottles. ``I told them to stop the policemen from taking the kites,'' said
Shakila Faqeeri, a communications adviser for the contractor, DPK Consulting.

But the policemen appeared to ignore her. Asked why one of his officers was loading his truck with
kites, Maj. Farouk Wardak, head of the criminal investigation division of the 16th Police District,
said, ``It's okay, he's not just a policeman, he's my bodyguard.''

The District 16 police chief, Col.~Haji Ahmad Fazli, insisted on taking over from the American
contractors the job of passing out the kites. He denied that his men were kite thieves. ``We are not
taking them,'' he said. ``We are flying them ourselves.''

At least he had not lost sight of the event's goal. ``It is so people can understand the rule of
law, and it lets the kids get together instead of wandering on the streets,'' he said.

It was not clear that the children had a much better grasp of the concept, but some did manage to
get kites and were flying them, irregularly shaped patches of color soaring to impressive heights.

Most bore messages about the importance of gender equality, but there was hardly a girl with a kite,
although plenty of girls were around. One DPK staff member pushed through the crowd to give
10-year-old Shaqila Nabi a kite; her sister Farzana, 8, had wanted one, too, but a policeman had
just swung at her with a stick and she had darted out of harm's way, and out of sight.

Shaqila raced back to her father, Gul Nabi, a horse wrangler peddling rides. He promptly took the
kite and gave it to a boy.

``He is my son and he should get the kite,'' he said.

The law and justice comic books were also a big hit. Some of the boys snatched them up and hid them
under their shirts so they could come back for more. At one point, fed-up policemen, most of whom
cannot read, just tossed piles of them in the dirt.

Mike Sheppard, the DPK project head, pronounced the event a success. ``We just gave out a thousand
kites in 20 minutes,'' he said.

But another DPK staff member, Abdul Manem Danish, stood watching the kite thievery and casual police
brutality with disdain. His job was to administer a ``kite event effectiveness survey'' at the end
to see if the festival had affected anyone's attitudes about justice.

``That's not a very good example of rule of law,'' he said. ``Maybe it is the nature of these people
that needs to be changed.''

\pagebreak
\section{China Is Said to Halt Trade in Rare-Earth Minerals With Japan}

\lettrine{A}{kihiro}\mycalendar{Sept.'10}{25} Ohata, the Japanese trade minister, said Friday that
his ministry was aware that Japanese traders were complaining of a halt from China of a crucial
category of minerals and that the government was investigating the matter.

The Chinese Commerce Ministry has denied that it has halted exports of the minerals, known as rare
earths and used in products like wind turbines and hybrid cars. And Mr.~Ohata said the Chinese
Commerce Ministry had also informed Japan that it had not issued a ban on exporting the minerals.

But industry executives said that factories in China were still not shipping to Japan after Chinese
customs agents blocked shipments earlier this week.

Eight executives, analysts and traders in the Chinese, Japanese and North American rare earths
industries said that China had suspended the shipments Tuesday in response to a diplomatic dispute
over Japan's detention of a Chinese fishing trawler captain.

Some theorized that the action might have been taken by Chinese customs agents, rather than as a
formal trade embargo imposed by Commerce Ministry regulations, to give Beijing more negotiating room
with Japan.

On Friday, the Japanese authorities said they would release the captain, but it was unclear when the
rare earth exports would resume.

Gary L.~Billingsley, executive chairman of Great Western Minerals Group, a Canadian company with
rare earth processing factories in Michigan and Britain, said China appeared to have stopped
shipping rare earths to Japan on Tuesday.

Japanese traders ``confirm that there has been a disruption in the supply of rare earths,''
Mr.~Billingsley said. Shipments loaded before Tuesday have continued to arrive at Japanese ports, he
said, adding that Great Western had not experienced any disruption because it bought supplies
directly from China.

China mines 93 percent of the world's rare earth minerals and more than 99 percent of the world's
supply of some of the most prized rare earths, which sell for several hundred dollars a pound.

A supplier for Toyota in Japan, who deals in machinery parts that require magnets containing rare
earths, said that the automaker had alerted his company Monday of a possible halt in rare earth
shipments. ``Toyota is already seeing shipments being stopped,'' said the supplier, who spoke on
condition of anonymity.

A Toyota spokesman, Paul Nolasco, had no immediate comment.

China has export quotas for rare earths, but even factories with ample quotas for further exports
had been dissuaded from making shipments, according to industry executives, analysts and two
Japanese traders.

``People are mystified why the Chinese don't acknowledge it,'' said Dudley Kingsnorth, the executive
director of Industrial Minerals Co.~of Australia, a rare earth consulting company.

An official at one of Japan's top traders in rare earths, who spoke on condition of anonymity, said
that because government offices in Beijing, including the customs headquarters, were closed
Wednesday through Friday for the Chinese autumn equinox holiday, industry players were still unsure
whether the halt in exports was the start of a longer embargo.

Trucking, ports and local customs offices continue to operate during weekends and holidays. But
Chinese rare earths operations have halted shipments to Japan for the last three days anyway in
response to strong warnings Tuesday from Beijing officials against selling to Japan, industry
executives said.

A person answering the phones at the customs headquarters in Beijing said no one would be available
for comment until Saturday.

If China continued to halt shipments, it would be extremely difficult to switch to other sources,
the Japanese trader said.

Rare earths are used in a wide variety of industrial applications, including the manufacture of
glass, batteries, catalytic converters, compact fluorescent bulbs and computer display screens.
Demand has risen in the last decade for their use in clean energy applications, like generators for
large wind turbines and lightweight electric motors for cars.

Japanese automakers in particular have been turning to rare earths for the electric motors used in
power steering in gasoline-powered cars, as well as the more powerful electric motors that help
propel gasoline-electric hybrids like the Toyota Prius.

Others in the industry said that having the customs agency halt exports of rare earths, without
calling it an export ban, carried political and legal advantages. Imposing an unannounced embargo,
they said, would have allowed China to ratchet up the pressure gradually on Japan to release the
detained boat captain.

And a halt in exports carried out through administrative measures, rather than as an act of official
policy, would be much harder for Japan to challenge at the World Trade Organization, which bans most
unilateral export restrictions. Under the trade agency's rules, countries may formally suspend
exports of natural resources only for environmental conservation.

Further confusing the rare earths issue in the past week has been uncertainty over Chinese export
quotas for rare earths, with some in the United States suggesting that shipment data showed that
exporters in China might be running out of their quotas for this year.

But experts on the Chinese export quota process said that there tended to be about a one-month lag
from the exercise of quotas until they showed up in shipment data. So large shipments last January
reflected the last-minute exercise of 2009 quotas in December and were not counted against 2010
quotas.

The 2010 quotas were not issued until Dec.~31, and exporters then typically needed several weeks to
arrange shipping.

Exports from the start of February through the end of August totaled about 23,000 tons, compared
with export quotas this year of 30,300 tons. So some experts predict that exporters in China could
run out of quotas as soon as the end of October, although that had nothing to do with suspension of
exports in the past week.

But some industry officials have suggested that China might be willing to issue the quotas for 2011
early, possibly in November, so as to allow shipments to resume quickly early in January.

\pagebreak
\section{From Young Mogul, a Gift on the Scale of Philanthropy's Elders}

\lettrine{M}{ark}\mycalendar{Sept.'10}{25} Zuckerberg, America's youngest billionaire at 26, has not
spent much money on himself. Forbes estimates his fortune at \$6.9 billion, but Mr.~Zuckerberg,
chief executive of Facebook, has yet to sell any sizable portion of his holdings in the company.

He rents an unremarkable house within walking distance of Facebook's headquarters here. He favors
jeans and T-shirts, drives an Acura and, unlike many other technology moguls, does not own a private
plane.

On Friday, Mr.~Zuckerberg announced his biggest expenditure to date: a \$100 million grant aimed at
improving public education in Newark, in partnership with Cory A.~Booker, the city's mayor, and
Chris Christie, New Jersey's governor.

Mr.~Zuckerberg's gift, which he announced during an appearance with Mr.~Booker and Mr.~Christie on
``The Oprah Winfrey Show,'' instantly propelled him to the top echelons of American philanthropy and
made him something of a hero.

But there is a competing version of Mr.~Zuckerberg's public persona, one that is on display in the
film ``The Social Network,'' a fictionalized story of Facebook's founding that paints him as a
backstabbing college student who betrayed friends and partners to assert control over Facebook.

The movie had its premiere on Friday at the New York Film Festival, just hours after Mr.~Zuckerberg
announced his philanthropic endeavor. And the timing of the gift raised questions as to whether
Mr.~Zuckerberg was simply trying to burnish his image at a difficult time.

``I don't think anybody gives \$100 million to anything if they are not thinking to some degree how
it sheds light on their beneficence,'' said David Kirkpatrick, the author of a recent book about the
company called ``The Facebook Effect.'' ``Otherwise they give anonymously.''

Mr.~Zuckerberg says he was hoping to do just that. On the Oprah show and in a later press
conference, Mr.~Zuckerberg and Mr.~Booker both said that the Facebook co-founder wanted to make his
gift anonymous. But Mr.~Booker persuaded him that the grant, which challenges New Jersey officials
to raise matching funds, would be more effective if his name was attached to it. And they said that
the timing was driven by factors out of their control.

``The movie became a complication,'' Mr.~Booker said, because of the risk that the public would view
the gift as ``an elaborate publicity stunt.''

Indeed, the announcement on the Oprah show, which showed clips of Mr.~Zuckerberg and his longtime
girlfriend Priscilla Chan in their home, was linked to a different movie, a documentary about public
education called ``Waiting for Superman'' that opened Friday and that Ms.~Winfrey has promoted for
much of the last week.

Mr.~Zuckerberg said that the \$100 million would be used to start a new foundation called Startup:
Education. The entire gift is earmarked for Newark and comes with no strings attached, giving
``flexibility to try out new things,'' he said.

Mr.~Zuckerberg's gift was praised in the philanthropy world.

``It is truly exceptional for any age group,'' said Patrick M.~Rooney, executive director of the
Center on Philanthropy at Indiana University, which tracks giving. ``Clearly when you look at most
philanthropists, significant gifts like these are made late in life or after death. For someone to
do this in their 20s is mind-boggling.''

Mr.~Rooney said Mr.~Zuckerberg's is only the third gift of \$100 million or more made this year in
the United States. Last year, there were only six donations of that size or larger, he said.

Philanthropic giving in Silicon Valley and among technology moguls is not new. A number of legendary
entrepreneurs, including Bill Hewlett and David Packard of Hewlett-Packard, Gordon Moore of Intel
and Bill Gates of Microsoft have established large charitable foundations. But their giving
typically came much later in life, after their companies and personal fortunes were well
established.

A younger generation of Internet billionaires like Pierre Omidyar and Jeff Skoll, who made their
fortunes with eBay, established foundations earlier.

``We have been seeing a very interesting phenomenon of dot-com billionaires making very generous
gifts, in many cases with a different attitude than was in the past,'' said Lester M.~Salamon,
director of the Center for Civil Society Studies at Johns Hopkins University. ``People are striking
it rich at an ever-earlier age.''

But even in those situations, those individuals established their philanthropies well after their
companies went public. Mr.~Zuckerberg's gift is unusual, in part, because Facebook is still
privately held, and there is no public market for its shares. Mr.~Zuckerberg is giving shares to the
foundation that will be sold to other private investors, a relatively new development in Silicon
Valley.

Mr.~Zuckerberg began discussing his plans to give away money with Ms.~Chan, a former teacher who is
now training to become a pediatrician, more than a year ago. The pair turned to Facebook's No.~2
executive, Sheryl Sandberg, for advice. Ms.~Sandberg, a veteran of the World Bank and the Treasury
Department, had helped to establish Google.org, Google's philanthropic arm.

Ms.~Sandberg said she immediately arranged for Mr.~Zuckerberg to meet prominent people in her
network of contacts who helped him shape his plans. They included Michael R.~Bloomberg, New York's
mayor; Joel Klein, the schools chief of New York; Wendy Kopp, the founder and president of Teach for
America; and the philanthropist Eli Broad.

Mr.~Zuckerberg also consulted with others including Mr.~Gates and Arne Duncan, the education
secretary. He firmed up the details this summer with Mr.~Booker, whom he met at a conference of
business moguls, and Mr.~Christie.

``Growing up, I was really fortunate to go to some great schools,'' Mr.~Zuckerberg said during the
press conference. He said he wanted to ensure that all children have similar opportunities.

``I really wanted to get started giving back at a young age,'' he said.

Ms.~Sandberg and Ms.~Chan, along with Mr.~Zuckerberg, will sit on the board of Startup: Education.

\pagebreak
\section{Six Technology Firms Agree to More Hiring Competition}

\lettrine{T}{he}\mycalendar{Sept.'10}{25} American job market is tough for many workers, but things
are looking even better than usual for highly paid engineers and scientists in Silicon Valley.

Six leading technology companies, including Apple, Google and Intel, reached an antitrust settlement
on Friday with the Justice Department that promises to increase the competition for sought-after
technology workers. The government had conducted a yearlong investigation into agreements among
companies not to poach employees from each other.

The investigation focused on five agreements by the companies not to make cold calls to employees
that each company had placed on a do-not-call list. Each of the pacts, according to the Justice
Department filing, involved a pair of companies: Apple and Google, Apple and Adobe, Apple and Pixar,
Google and Intel, and Google and Intuit.

The agreements to curb cold-calling of each others' workers, the Justice Department complaint said,
``diminished competition to the detriment of the affected employees who were likely deprived of
competitively important information and access to better job opportunities.''

Silicon Valley is among the most flexible and fast-moving labor markets in the world, and
job-hopping is commonplace. The movement of workers from company to company facilitates the flow of
ideas and generation of products. Agreements that inhibit that movement, antitrust officials said,
restrains not only labor but also the pace of innovation.

Laws that prohibit restraints on workers go back to the origins of antitrust law, centuries ago, in
England, when the ``restraint of trade'' literally meant ``unreasonable restraint'' on people
practicing their trades -- like bakers and blacksmiths.

``This shows that the oldest ideas in antitrust still apply to today's high-tech industries,'' said
Andrew I.~Gavil, a professor of law at Howard University. ``These people may be well paid, but this
is not an insignificant matter.''

Yet Silicon Valley companies both compete and cooperate with each other, to jointly develop new
products and services. And a company, analysts say, may be less likely to collaborate on a project
if it fears the partner company is intent on hiring away its best employees, thus undermining
innovation.

In a statement, Google portrayed its agreements against cold-calling employees at other
high-technology companies as a temporary program to maintain good relations with other companies at
a time it was hiring aggressively.

``Google grew by more than 16,000 people between 2005 and 2009 -- a fivefold increase in the size of
our company,'' Google said.

Hiring reached a peak of 40 new recruits a day in 2007, it added, when it was also building ties
with other technology companies to jointly develop services.

``While there's no evidence that our policy hindered hiring or affected wages, we abandoned our `no
cold calling' policy in late 2009 once the Justice Department raised concerns, and we are happy to
continue with this approach as part of this settlement,'' the company said.

Intuit, in a statement, said it had ``agreed to disagree'' with the Justice Department on the issue
of wrongdoing in the case. But Intuit will not ``enter into the types of broad nonsolicit agreements
that are prohibited by the settlement,'' said Laura Fennell, Intuit's senior vice president and
general counsel.

Intuit, she said, shared the Justice Department's desire ``to maintain an open, fair, competitive
market for talent.''

\pagebreak
\section{Told to Eat Its Vegetables, America Orders Fries}

\lettrine{I}{t}\mycalendar{Sept.'10}{25}'s been a busy week for vegetables.

The baby-carrot industry tried to reposition its product as junk food, starting a \$25 million
advertising campaign whose defining characteristics include heavy metal music, a phone app and a
young man in a grocery cart dodging baby-carrot bullets fired by a woman in tight jeans.

On the East Side of Manhattan, crates of heirloom vegetables with names like Lady Godiva squash were
auctioned for \$1,000 each at Sotheby's, where the wealthy are more accustomed to bidding on Warhols
and Picassos than turnips and tomatoes.

Both efforts, high and low, are aimed at the same thing: getting America to eat its vegetables.

Good luck. Despite two decades of public health initiatives, stricter government dietary guidelines,
record growth of farmers' markets and the ease of products like salad in a bag, Americans still
aren't eating enough vegetables.

This month, the Centers for Disease Control and Prevention issued a comprehensive nationwide
behavioral study of fruit and vegetable consumption. Only 26 percent of the nation's adults eat
vegetables three or more times a day, it concluded. (And no, that does not include French fries.)

These results fell far short of health objectives set by the federal government a decade ago. The
amount of vegetables Americans eat is less than half of what public health officials had hoped.
Worse, it has barely budged since 2000.

``It is disappointing,'' said Dr.~Jennifer Foltz, a pediatrician who helped compile the report. She,
like other public health officials dedicated to improving the American diet, concedes that perhaps
simply telling people to eat more vegetables isn't working.

``There is nothing you can say that will get people to eat more veggies,'' said Harry Balzer, the
chief industry analyst for the NPD Group, a market research company.

This week, the company released the 25th edition of its annual report, ``Eating Patterns in
America.'' The news there wasn't good, either. For example, only 23 percent of meals include a
vegetable, Mr.~Balzer said. (Again, fries don't count, but lettuce on a hamburger does.) The number
of dinners prepared at home that included a salad was 17 percent; in 1994, it was 22 percent.

At restaurants, salads ordered as a main course at either lunch or dinner dropped by half since
1989, to a mere 5 percent, he said.

The nation has long had a complicated relationship with vegetables. People know that vegetables can
improve health. But they're a lot of work. In refrigerators all over the country, produce often dies
a slow, limp death because life becomes too busy.

``The moment you have something fresh you have to schedule your life around using it,'' Mr.~Balzer
said.

In the wrong hands, vegetables can taste terrible. And compared with a lot of food at the
supermarket, they're a relatively expensive way to fill a belly.

``Before we want health, we want taste, we want convenience and we want low cost,'' Mr.~Balzer said.

Melissa MacBride, a busy Manhattan resident who works for a pharmaceuticals company, would eat more
vegetables if they weren't, in her words, ``a pain.''

``An apple you can just grab,'' she said. ``But what am I going to do, put a piece of kale in my
purse?''

No one really wants to admit that they don't eat vegetables. A nurse who works at the Hospital for
Special Surgery on the Upper East Side openly acknowledges that vegetables make her gag. Still, she
begged to not be publicly identified because she is in the health care field and knows that she
should set a better example.

David Bernstein, who lives in Greenpoint, Brooklyn, is sheepish about the lack of vegetables in his
diet. He waits tables at the hip M.~Wells restaurant in Long Island City, Queens, and knows his way
around the Union Square Greenmarket. But his diet consists largely of bacon, yogurt and frozen
stuffed chicken breasts.

``It's just like any other bad habit,'' he said. ``Part of it is just that vegetables are a little
intimidating. I'm not afraid of zucchinis, but I just don't know how to cook them.''

The food industry has tried to make eating vegetables easier. Sales of convenience vegetables, like
packages of cut broccoli designed to go right into the microwave, are growing. Washed, ready-to-eat
bagged salads are a \$3-billion-a-year business.

But that doesn't necessarily mean people are eating more vegetables. It just means they are shifting
their vegetable budget from one place to another, Mr.~Balzer said. An organic cucumber might replace
a conventionally grown one. A bag of lettuce replaces a head.

To be sure, vegetables are making strides in certain circles. Women, as well as people who are older
and more educated and have higher incomes, tend to eat more vegetables, said Dr.~Foltz, the
pediatrician who worked on the C.D.C. report.

The vegetable, especially when grown from heirloom seeds on small farms, is held in such high esteem
that knowing the farmer who grows the food is a form of valuable social currency. Vegetables are
becoming high art. At Sotheby's on Thursday night, the vegetable auction was part of a daylong event
called ``The Art of Farming,'' raising nearly \$250,000 to help hunger organizations, immigrant
farmers and children without access to vegetables.

But vegetables are also becoming important on the other end of the economic equation. An increasing
number of the nation's 6,000 farmers' markets allow shoppers to buy produce with food stamps. Urban
gardens are springing up in vacant lots and on rooftops. Nearly every state now has programs that
send fresh vegetables into poorer neighborhoods and school cafeterias.

The vegetable even has the first lady, Michelle Obama, on its side. She planted an organic garden on
the White House lawn and talks up vegetables as part of her ``Let's Move'' campaign against
childhood obesity.

The government keeps trying, too, to get its message across. It now recommends four and a half cups
of fruits and vegetables (that's nine servings) for people who eat 2,000 calories a day. Some public
health advocates have argued that when the guidelines are updated later this year, they should be
made even clearer. One proposal is to make Americans think about it visually, filling half the plate
or bowl with vegetables.

But clear guidance probably isn't enough. Health officials now concede that convincing a nation that
shuns vegetables means making vegetables more affordable and more available.

``We have to make the healthy choice the easy choice,'' Dr.~Foltz said. And the choices need to
become ingrained.

For another study whose results were announced this week, researchers at the University of
California, Berkeley, spent three years examining the difference between children who participated
in the Berkeley Unified School District's ``edible schoolyard'' program, in which gardening and
cooking are woven into the school day, and children who didn't.

The students who gardened ate one and half servings more of vegetables a day than those who weren't
in the program.

For students who don't have access to a school garden, perhaps the full-court press by the
baby-carrot producers will have some effect. The iPhone application, for example, is a video game
called Xtreme Xrunch Kart that starts when a player crunches a carrot (or makes a crunchlike sound)
into the phone's microphone.

But as in past attempts to revive the vegetable, none of this will necessarily be enough to change a
clear aversion to eating vegetables.

``Eating vegetables is a lot less fun than eating flavor-blasted Doritos,'' said Marcia Mogelonsly,
a senior analyst for Mintel, a global marketing firm. ``You will always have to fight that.''

\pagebreak
\section{New F.D.A.: Transparence and Flexibility}

\lettrine{D}{uring}\mycalendar{Sept.'10}{25} the Bush administration, the Food and Drug
Administration was mostly a place of black-and-white decisions. Drugs were approved for sale or they
were not, and the agency's staff was expected to publicly support those decisions.

But as Thursday's landmark decision on the controversial diabetes medicine Avandia makes clear,
things have changed under the Obama administration. Certainty, staff unanimity and even the approval
status of big-selling medicines are no longer so black and white.

Presented with what seemed to be a choice between keeping Avandia on the market or withdrawing it,
the Obama administration decided on an unusual middle path -- allowing sales, but with tight
restrictions. Even more unusually, the agency admitted that many of its top scientists disagreed,
some passionately. Competing memorandums were posted immediately on the agency's Web site.

And the agency's three top officials co-wrote a highly unusual explanation of their action in The
New England Journal of Medicine.

Some of these changes have been in the works for years, but they have accelerated under the Obama
administration, driven by increasingly sophisticated measures of drug safety and growing skepticism
about whether the F.D.A. is making the right decisions and making them appropriately.

``I think that F.D.A.'s credibility really depends on being able to explain its decisions well,''
said Dr.~Joshua Sharfstein, F.D.A.'s principal deputy commissioner. ``We can't expect people to
think that F.D.A. has decided, therefore it's the right answer.''

Some of the changes have been driven by people like Dr.~Steven Nissen, a cardiologist at the
Cleveland Clinic whose 2007 analysis of Avandia's heart risks stunned doctors, patients and
legislators, who asked why the F.D.A. had not done anything similar. When the agency revealed it had
done an almost identical analysis a year earlier and found the same result, the controversy
intensified.

``You have these third-party analysts setting the agenda for the agency in ways that never happened
before,'' said Daniel Carpenter, an F.D.A. historian at Harvard.

For the F.D.A., the Nissen analysis presented major challenges. It demonstrated that the agency no
longer had a monopoly on the information needed to make drug and device safety decisions. Data from
crucial clinical trials are increasingly being posted on public Web sites. And academics are using
sophisticated techniques to test whether popular medicines are safe.

In March, for instance, a team of academics found that a children's diarrhea vaccine contained
harmless but apparently extraneous pieces of pig virus. Blindsided, the F.D.A. had no idea what
effect the particles would have. While the agency studied the problem, the commissioner,
Dr.~Margaret Hamburg, asked its maker to stop selling -- a request she had little power to enforce.

Two months later, the agency allowed sales to continue.

Like the vaccine finding, the Nissen analysis flummoxed the agency because the science behind it was
controversial. Dr.~Nissen combined the results of many clinical trials to suggest that Avandia
substantially increased heart risks. Other studies suggested that there were higher risks.

None of these studies met the rigorous standards that the F.D.A. demands when approving new
medicines, but they were among the only information available to explore whether popular medicines
contribute to common problems like heart attacks.

The agency was torn about how to interpret the studies, a problem it rarely faced until recently.
``In the past, we would approve the drug after a couple of efficacy trials and that was it,''
Dr.~Janet Woodcock, chief of the F.D.A.'s drug center, said in an interview. ``We didn't know too
much more about the drug. It was simpler.''

Now, sophisticated analyses present the F.D.A. with a complex picture. ``It's good for public health
that we're learning more, but it creates a more complex environment in which to regulate,''
Dr.~Woodcock said.

It is an environment in which top agency officials are in some ways at sea. The agency has no
systems or standards to follow in deciding which studies deserve their attention or should lead to
changes in a drug's status. And since new tests are being created constantly, creating such a
standard would be an ever-evolving process.

Dr.~Lynn Goldman, dean of the School of Public Health and Health Services at George Washington
University, said the F.D.A. was being forced to become more comfortable with studies done in
academic rather than regulatory settings. ``They have to get used to a less controlled
environment,'' Dr.~Goldman said.

And the agency's decision to create a unique distribution program for Avandia is not one it can
repeat often or doctors and pharmacists -- who must learn a new system for each program -- will give
up.

``We have to get some standardization,'' Dr.~Woodcock said, ``or we'll burn out the system.''

\pagebreak
\section{Chinese Attitudes on Generosity Are Tested}

\lettrine{L}{ike}\mycalendar{Sept.'10}{25} most everything else in China's economy, philanthropy
here is in a boom period, fueled by phalanxes of newly minted billionaires and foundations,
encouraged by an army of professional advisers on charity and, increasingly, sanctioned by the
government.

Which makes the case of Warren Buffett and Bill Gates, who will come to Beijing next week to share
their thoughts on philanthropy with some of China's wealthiest people, all the more curious.

Mr.~Buffett and Mr.~Gates, the Rockefeller and Carnegie of this age, announced plans last month to
invite about 50 of China's superrich to discuss their concept of philanthropy, which includes
enlisting the world's wealthiest people to give away at least half their fortunes.

Things appeared to be going swimmingly until early September, when the Chinese news media quoted a
Beijing official of the Bill and Melinda Gates Foundation as saying that ``a small number of
people'' had declined to come, and that others had asked whether they would be pushed for donations.

Last week, Mr.~Gates and Mr.~Buffett issued a letter stating that they were not coming to China ``to
pressure people to give,'' but to listen. ``China's circumstances are unique, and so its approach to
philanthropy will be as well,'' they wrote.

Except for denying a report from Xinhua, the state-run news agency, that only two tycoons had
accepted the invitation, the organizers of the event have largely fallen silent.

But the Chinese are unlikely to stop talking soon. In a nation where explosive growth has opened a
yawning gap between rich and poor, reports that Chinese billionaires might stiff-arm the invitation
have spawned a sort of national Rorschach test of Chinese generosity, not to mention attitudes
toward the rich.

``Are Chinese rich scared to be charitable?'' asked The Global Times, the Communist Party's
English-language newspaper. Not at all; ``This is the Americans' conspiracy,'' wrote one of 2,000
people who posted comments on the controversy on Sina.com, a major Internet portal. Academics
grumbled about efforts to impose Western philanthropic values on Chinese tradition.

Actually, however, Chinese philanthropic tradition was being upended well before the Gates-Buffett
dinner was even conceived. In barely a decade, the Chinese economy has created at least 117
billionaires, according to a Forbes magazine ranking, and hundreds of thousands of millionaires by
the estimate of Hurun Report, a magazine based in Shanghai whose target audience is the rich. Only
the United States has more billionaires.

While China's reported philanthropic donations are now comparatively tiny -- about \$8 billion last
year, the government says, compared with \$308 billion in the United States in 2008 -- changes in
China's economic structure and in government policies make that figure almost destined to rise
quickly. And, in contrast to the past, riches are starting to flow to social and charitable causes.

``The Chinese have been very generous for a long period of time,'' Rupert Hoogewerf, who publishes
Hurun Report, said by telephone. ``The difference has been that they do it between families, and
don't publicize it. What we're seeing now is a new era of transparency.''

Translucency might be a better term. More than a few fortunes have been built on corruption, and
their owners stay in the shadows. The China Reform Foundation, an economic research group based in
Beijing, estimated last month that about \$870 billion in corrupt ``gray money'' was being hidden by
the wealthiest 10 percent of China's population.

Huang Guangyu, who built an appliance shop into a fortune valued at \$2.7 billion to \$6.3 billion,
was singled out by Hurun Report in 2007 as especially miserly. Today he is in prison, convicted of
stock fraud and insider trading.

A Global Times article this month stated that in the last decade, 17 members of an annual list of
China's 50 richest people had been convicted of economic crimes.

Ordinary Chinese, steeped in petty government corruption, are often bitterly cynical toward the
rich.

``Of course they'll decline the invitation,'' one wrote of the invited billionaires on the Sina.com
postings board. ``None of their money is clean!''

Yet a growing number of China's corporate titans are open both about their wealth and about giving
it away. The leading example is Yu Pengnian, an 88-year-old real estate baron who gave the last of
his \$1.3 billion fortune in April to a foundation he created to fund scholarships for poor Chinese
students. The latest is Chen Guangbiao, 42, a Jiangsu Province recycling-company owner who has taken
the Gates-Buffett pledge to give away his \$440 million fortune when he dies.

``Wealth is not something that comes to you when you are born,'' he said in an interview last week.
``It's like water. If you have only a cup, you keep it to yourself. If you have a barrel, you share
it with your family. And if you have a river, you share it with everyone.''

This is a new phenomenon, and not only because the money is new. China's Communist Party claims to
represent the downtrodden, and has been reluctant to turn to the private sector to address problems
of poverty and disease.

But since the outpouring of support for victims of the Sichuan earthquake in 2008, the government
now seems to be edging toward a more accommodating attitude about private philanthropy. It offers
corporations a tax break of up to 12 percent for charitable gifts, rising to 30 percent for
individuals.

This year, a reregistration drive has certified more than 1,000 nonprofit groups as able to accept
tax-deductible donations. Government officials have also said that they plan to enact the nation's
first charity law, with rules that clearly define what a charity is and how it must operate, by late
2011.

But whether revised rules on charities and nonprofit groups generally will broaden or restrict
philanthropic work is unclear, said Jia Xijin, who directs the Nongovernmental Organization Research
Center at Tsinghua University in Beijing. While the government has slowly given new leeway to some
charitable groups, especially those that provide social services, it keeps a tighter rein on groups
that advocate policy changes or raise money on their own.

The government's concern is that ``most public fund-raising organizations need some social cause,
and if you organize people,'' she said, ``that means the organization represents some social
force.''

For Mr.~Chen, the recycling magnate, the best way to encourage philanthropy by the group invited to
dine with Mr.~Buffett and Mr.~Gates is to publicize the names of people who decline to attend.

``I'll help you bash them in the media,'' he said. ``We can't be misers.''

\pagebreak
\section{China Takes a Sharper Tone in Dispute With Japan}

\lettrine{C}{hinese}\mycalendar{Sept.'10}{25} Premier Wen Jiabao ``strongly urged'' Japan to
immediately and unconditionally release from custody the captain of a Chinese trawler, threatening
further action if Japan refuses.

Mr.~Wen's comments were the first by a senior Chinese official in what is rapidly becoming the most
serious territorial dispute China has faced in a decade. The captain and crew were seized earlier
this month by Japanese naval vessels, which claimed that the fishing boat rammed them near several
uninhabited islands controlled by Japan. The boat and crew were quickly released, but the captain
faces charges of obstructing officials from performing their duty and remains in Japanese custody.

China is incensed that Japan would apply its laws to Chinese nationals and argues that the issue is
one for diplomacy, not the legal system. Known as Senkaku in Japanese or Diaoyu in Chinese, the
islands have been in dispute for decades, but Japan has mostly turned back Chinese vessels that
approach too closely.

Mr.~Wen made his comments Tuesday night to members of the Chinese-American community in New York,
where he is attending a United Nations meeting. The comments were carried Wednesday on the Web site
of the Chinese Foreign Ministry.

``This is totally illegal, unreasonable and has already caused much suffering to the family of the
captain,'' Mr.~Wen was quoted as saying. ``If Japan clings to its course, China will take further
action.''

Mr.~Wen's comments come as China as continued to ratchet up the pressure on Japan. On Tuesday, it
announced that Mr.~Wen would probably not meet his Japanese counterpart, Naoto Kan, who is also in
New York for the United Nations development conference. On Sunday, China suspended many government
contacts and other exchanges with Japan.

``Japan holds the key to solving this problem,'' the Foreign Ministry spokeswoman, Jiang Yu, said.
``The Japanese side should correctly understand the situation and return the captain immediately and
unconditionally.''

Some analysts say the issue might blow over next Wednesday when Japan must decide whether to
formally charge the captain or release him. If he is charged, the emotional issue could boil over in
China, where protests have already taken place and Internet forums are full of anti-Japanese
rhetoric.

``Japan will have to release the captain with a warning or something similar,'' said a Western
diplomat based in Beijing who spoke on condition of anonymity because of the delicacy of the
conflict. ``It's hard to imagine them actually charging and trying him.''

Sentiment in Japan, however, has hardened against China in recent years, with some calling for the
country to resist a diplomatic solution and enforce its claims by applying Japanese law.

Japan controls the islands although China draws on historical records to buttress its claim to them.
The islands have been the scene of protests for several decades, with Chinese from the mainland,
Hong Kong and Taiwan claiming that Japan seized them in the 19th century and should have returned
them after the end of World War II. Japan says the islands were not effectively controlled by anyone
and were not part of agreements at the end of the war to strip Japan of territory acquired during
its period of expansionism.

The most recent flare-up comes as China faces disputes with its neighbors to the south over control
of islands in the South China Sea. It has also objected to American military exercises in the region
and arms sales to Taiwan, which it also views as part of its territory.

\pagebreak
\section{China's Disputes in Asia Buttress Influence of U.S.}

\lettrine{F}{or}\mycalendar{Sept.'10}{25} the last several years, one big theme has dominated talk
of the future of Asia: As China rises, its neighbors are being inevitably drawn into its orbit,
currying favor with the region's new hegemonic power.

The presumed loser, of course, is the United States, whose wealth and influence are being spent on
the wars in Iraq and Afghanistan and whose economic troubles have eroded its standing in a more
dynamic Asia.

But rising frictions between China and its neighbors in recent weeks over security issues have
handed the United States an opportunity to reassert itself -- one the Obama administration has been
keen to take advantage of.

Washington is leaping into the middle of heated territorial disputes between China and Southeast
Asian nations despite stern Chinese warnings that it mind its own business. The United States is
carrying out naval exercises with South Korea in order to help Seoul rebuff threats from North Korea
even though China is denouncing those exercises, saying that they intrude on areas where the Chinese
military operates.

Meanwhile, China's increasingly tense standoff with Japan over a Chinese fishing trawler captured by
Japanese ships in disputed waters is pushing Japan back under the American security umbrella.

The arena for these struggles is shifting this week to a summit meeting of world leaders at the
United Nations. Wen Jiabao, the Chinese prime minister, has refused to meet with his Japanese
counterpart, Naoto Kan, and on Tuesday he threatened Japan with ``further action'' if it did not
unconditionally release the fishing captain.

On Friday, President Obama is expected to meet with Southeast Asian leaders and promise that the
United States is willing to help them peacefully settle South China Sea territorial disputes with
China.

``The U.S.~has been smart,'' said Carlyle A.~Thayer, a professor at the Australian Defense Force
Academy who studies security issues in Asia. ``It has done well by coming to the assistance of
countries in the region.''

``All across the board, China is seeing the atmospherics change tremendously,'' he added. ``The idea
of the China threat, thanks to its own efforts, is being revived.''

Asserting Chinese sovereignty over borderlands in contention -- everywhere from Tibet to Taiwan to
the South China Sea -- has long been the top priority for Chinese nationalists, an obsession that
overrides all other concerns. But this complicates China's attempts to present the country's rise as
a boon for the whole region and creates wedges between China and its neighbors.

Nothing underscores that better than the escalating diplomatic conflict between China and Japan over
the detention of the Chinese fishing captain, Zhan Qixiong, by the Japanese authorities, who say the
captain rammed two Japanese vessels around the Senkaku or Diaoyu Islands in the East China Sea. The
islands are administered by Japan but claimed by both Japan and China.

The current dispute may strengthen the military alliance between the United States and Japan, as did
an incident last April when a Chinese helicopter buzzed a Japanese destroyer. Such confrontations
tend to remind Japanese officials, who have suggested that they need to refocus their foreign policy
on China instead of America, that they rely on the United States to balance an unpredictable China,
analysts say.

``Japan will have no choice but to further go into America's arms, to further beef up the U.S.-Japan
alliance and its military power,'' said Huang Jing, a scholar of the Chinese military at the
National University of Singapore.

In July, Southeast Asian nations, particularly Vietnam, applauded when Secretary of State Hillary
Rodham Clinton said that the United States was willing to help mediate a solution to disputes that
those nations had with China over the South China Sea, which is rich in oil, natural gas and fish.
China insists on dealing with Southeast Asian nations one on one, but Mrs.~Clinton said the United
States supported multilateral talks. Freedom of navigation in the sea is an American national
interest, she said.

President Obama meets on Friday with leaders from the 10-member Association of Southeast Asian
Nations, or Asean. The Associated Press reported that the participants would issue a joint statement
opposing the ``use or threat of force by any claimant attempting to enforce disputed claims in the
South China Sea.'' The statement is clearly aimed at China, which has seized Vietnamese fishing
vessels in recent years and detained their crews.

On Tuesday, a Chinese Foreign Ministry spokeswoman, Jiang Yu, criticized any attempt at mediation by
the United States. ``We firmly oppose any country having nothing to do with the South China Sea
issue getting involved in the dispute,'' she said at a news conference in Beijing.

China has also been objecting to American plans to hold military exercises with South Korea in the
Yellow Sea, which China claims as its exclusive military operations zone. The United States and
South Korea want to send a stern message to North Korea over what Seoul says was the torpedoing last
March of a South Korean warship by a North Korean submarine. China's belligerence serves only to
reinforce South Korea's dependence on the American military.

American officials are increasingly concerned about the modernization of China's navy and its
long-range abilities, as well as China's growing assertiveness in the surrounding waters. In March,
a Chinese official told White House officials that the South China Sea was part of China's ``core
interest'' of sovereignty, similar to Tibet and Taiwan, an American official said in an interview at
the time. American officials also object to China's telling foreign oil companies not to work with
Vietnam on developing oil fields in the South China Sea.

Some Chinese military leaders and analysts see an American effort to contain China. Feng Zhaokui, a
Japan scholar at the Chinese Academy of Social Sciences, said in an article on Tuesday in The Global
Times, a populist newspaper, that the United States was trying to ``nurture a coalition against
China.''

In August, Rear Adm. Yang Yi wrote an editorial for The PLA Daily, published by the Chinese Army, in
which he said that on one hand, Washington ``wants China to play a role in regional security
issues.''

``On the other hand,'' he continued, ``it is engaging in an increasingly tight encirclement of China
and is constantly challenging China's core interests.''

Asian countries suspicious of Chinese intentions see Washington as a natural ally. In April, the
incident involving the Chinese helicopter and Japanese destroyer spooked many in Japan, making them
feel vulnerable at a time when Yukio Hatoyama, then the prime minister, had angered Washington with
his pledges to relocate a Marine Corps air base away from Okinawa.

His successor, Mr.~Kan, has sought to smooth out ties with Washington and has emphasized that the
alliance is the cornerstone of Japanese foreign policy.

``Insecurity about China's presence has served as a wake-up call on the importance of the
alliance,'' said Fumiaki Kubo, a professor of public policy at the University of Tokyo.

\pagebreak
\section{Chinese Leader Fields Executives' Questions}

\lettrine{W}{hen}\mycalendar{Sept.'10}{25} Bill Gates confronts the prime minister of China on the
need to honor software copyrights, it helps to have a referee -- say, a Henry Kissinger -- to
moderate the debate.

That, in fact, is what happened at the Waldorf-Astoria Hotel on Wednesday morning in Midtown
Manhattan.

In a remarkable 90-minute meeting, with Mr.~Kissinger playing M.C., Mr.~Gates and other heavyweight
executives and economists from the West engaged Prime Minister Wen Jiabao. He listened patiently,
and often volleyed back, on topics including currency and trade policies, foreign investment and
whether China needed to improve its social safety net.

Others in the circle -- literally a large ring of a few dozen chairs -- included Jamie Dimon, the
head of JPMorgan Chase; Lloyd C.~Blankfein, the Goldman Sachs chief; Joseph Stiglitz, winner of the
Nobel in economics; Kenneth I.~Chenault, chairman and chief of American Express; and PepsiCo's chief
executive, Indra K.~Nooyi.

China's top leaders rarely meet Western executives. But Mr.~Wen, in New York for a session of the
United Nations General Assembly, agreed to sit down with the group, possibly in the hope of helping
ease growing tensions between the United States and China over various issues.

The session, which Chinese officials called a dialogue with ``economic celebrities,'' was civil and
respectful. But there were some pointed words for the Chinese leader, who listened and spoke through
an interpreter.

Mr.~Chenault told Mr.~Wen that the most important asset American companies have is their brand.
Seeming to hint that Chinese companies were competing unfairly, Mr.~Chenault asked the prime
minister for his views on brand value.

Mr.~Wen responded, ``We will never usurp others' brands.''

Ms.~Nooyi of PepsiCo told Mr.~Wen that China should create incentives for companies to build
factories to high environmental standards. She also said that her company had invested billions of
dollars in China and was already his country's biggest private potato grower, and she asked whether
American companies would get equal treatment to Chinese companies there.

The prime minister replied, ``You put forward two good proposals, and the Chinese government will
accept these.''

Robert E.~Rubin, the former Treasury secretary, contended that China's huge trade surpluses with the
United States could have disastrous consequences.

``The trade imbalances are unsustainable,'' Mr.~Rubin told Mr.~Wen, urging China to restructure its
economy away from exports and toward domestic consumption. ``And this trade imbalance is creating
political problems'' in the United States.

Mr.~Wen acknowledged that global imbalances were a problem, and said Beijing was working to make
changes. But he took issue with the widely held idea that China takes the largest share of trade
benefits.

``An iPod is sold at \$299, and China in the manufacturing link will only get \$6 for it,'' he said.
The implication was clear: The bulk of the profits in producing the item accrue to Apple and others
in the supply chain.

On one of the most contentious issues -- China's currency policy -- Mr.~Wen had little to say. But
Wednesday evening, in a separate speech to a group of dignitaries involved in United States-China
relations, Mr.~Wen said more sharply that China's exchange rate was not the problem, and indicated
that China would continue to resist pressure from Washington.

``There is no basis for a drastic appreciation of the renminbi,'' he said. ``You don't know how many
Chinese companies would go bankrupt. There would be major disturbances. Only the Chinese premier has
such pressure on his shoulders. This is the reality.''

It is a topic likely to come up when he and President Obama meet in New York on Thursday on the
sidelines of the United Nations. China has repeatedly signaled that it would like to move toward a
more flexible currency, and Mr.~Wen said Wednesday that China did not ``intentionally'' seek a huge
trade surplus -- something that critics say Beijing does by keeping its currency, the renminbi,
artificially cheap.

But while China has allowed the renminbi to appreciate slightly against the dollar this year,
China's trade is booming again.

Several of the Americans at the Waldorf on Wednesday warned that the United States' sluggish economy
and high unemployment rate were inflaming protectionist sentiment in this country and could lead
Congress to impose tough trade sanctions or other measures.

Stephen Roach, a Morgan Stanley economist and a teacher at Yale, warning that American politicians
were threatening to take the ``low road,'' urged Mr.~Wen to ignore calls for China to fix its
currency and instead focus on pushing for ``pro-consumption'' policies at home. That, Mr.~Roach
said, would allow the Chinese to consume more -- and also buy more American goods.

As for Mr.~Gates, he said that he was preparing to travel to China for the Gates Foundation, his
philanthropy, but that he also wanted to press a long-running concern about counterfeiting of
American software and other intellectual property in China.

After saying that Microsoft's research lab was progressing well in China, Mr.~Gates said: ``I'll
mention one thing that is not going well, and that's related to the enforcement of intellectual
property, such as copyright. If you look at the numbers, over the last five years there hasn't been
much progress.''

Mr.~Wen took the question in stride. ``Mr.~Gates,'' he said. ``You are a business person I hold in
high regard. You also have morality running in your veins. I fully support the Gates Foundation.''

Then, Mr.~Wen -- who is called Grandpa Wen in China because of his populist approach and habit of
racing to the scene of natural disasters to comfort victims -- applied his charms on Mr.~Gates.

``I do admit these problems exist,'' Mr.~Wen said. ``We have to put in administrative measures. I
think we should have higher moral and ethical standards in this matter.''

\pagebreak
\section{Hopes Fade for Success of Commonwealth Games in India}

\lettrine{S}{kepticism}\mycalendar{Sept.'10}{25} about India's preparedness for the Commonwealth
Games deepened Tuesday after a partly constructed footbridge collapsed outside the main arena for
competition, injuring dozens.

The collapse coincided with angry words from visiting officials who described the accommodations for
athletes as uninhabitable. One visitor, the head of the New Zealand delegation, even raised the
possibility that the games might be delayed or canceled.

India's failure to complete the work for the games, which are to begin Oct.~3 and last for two
weeks, has become a major embarrassment for the country instead of a showcase for its rising
economic might. The unspoken comparison to India's rival China, which won widespread acclaim from
its preparations for the 2008 Summer Olympics, are a further source of humiliation.

Representatives of the dozens of countries participating in the Commonwealth Games, a quadrennial
competition among the nations of the former British Empire, started arriving here in recent days to
inspect facilities and conduct security checks. The athletes' village, built for the games, is not
ready, they say, and questions linger about security after an attack on tourists in Delhi on Sunday.

On Tuesday afternoon, a bridge next to Jawaharlal Nehru Stadium, the main venue, fell apart. The
footbridge collapsed into three pieces, taking several workers with it and uprooting one side of the
arch that supported it.

A police officer at the scene said that 27 people had been injured, 4 seriously.

``This will not affect the games,'' said Raj Kumar Chauhan, a Delhi minister for development, who
spoke at the scene. ``We can put the bridge up again, or make a new one.''

The accident occurred when workers were trying to pour concrete into a clip at the base of the
bridge, he said, and the clip was loosened.

Games officials had lodged formal complaints about the preparations with India's government even
before the accident. ``The condition of the residential zone has shocked the majority,'' the
Commonwealth Games Federation president, Michael Fennel, said in a statement Monday evening.
Mr.~Fennel said he had sent a letter to India's union cabinet secretary. The athletes' village is
``seriously compromised,'' he said.

``The problems are arising because deadlines for the completion of the village have been
consistently pushed out,'' Mr.~Fennell said.

The village is ``uninhabitable,'' the Commonwealth Games Federation chief executive, Mike Hooper,
told the local television channel CNN-IBN on Tuesday. ``There is dust everywhere,'' he said. ``The
flats are dirty and filthy. Toilets are unclean.''

Construction of the village, built alongside the Yamuna River on Delhi's eastern border, is severely
behind schedule. Delhi built a series of apartment towers to house about 7,000 athletes and their
families, a 2,300-seat cafeteria, and practice areas on land that was originally an empty plain.

Officials from the Ministry of Sports promised last year that the village would be ready in March
2010, but finishing touches were still being done outside buildings during a media tour last week.
And the interiors of the buildings are still not completed, some say.

Dave Currie, the head of New Zealand's Commonwealth Games team, said Tuesday in an interview with
Newstalk ZB, a New Zealand radio station, that the condition of the athletes' village was ``pretty
grim.''

Showers and toilets in the accommodations the New Zealand team was given are not working, and
post-construction cleanup has not been done, he said. ``It is certainly disappointing considering
the amount of time they have had,'' he said.

Athletes are scheduled to start arriving in Delhi on Thursday, but that date may need to be pushed
back, Mr.~Currie said, which could ultimately result in the competition being canceled. ``If the
village is not ready, the athletes cannot arrive,'' he said.

``There is a real mountain to climb'' before the village can be completed, Mr.~Currie said. It will
be a ``real challenge at this point to make it happen,'' he said.

Security at the games has also become a major concern after two tourists were shot outside the Jama
Masjid, a mosque that is one of Delhi's major attractions, on Sunday. Neither tourist was fatally
injured, and the mosque is far from the venues or the athletes' village, but the attack prompted new
fears about Delhi's ability to keep athletes and visitors safe during the games.

An e-mail sent to news outlets soon after the attack said the Indian Mujahedeen, a group the Indian
government considers a terrorist organization, would single out the games.

``Had it not happened against the almost complete disarray of the Commonwealth Games preparations,
it would not have raised much excitement,'' said Ajai Sahni, the executive director of the Institute
for Conflict Management, a group that studies terrorist activity. Athletes are worried that if
construction and planning are in disarray, security may be too, he said.

Most venues were supposed to be completed in 2007, but workers were still putting finishing touches
on many of them as well.

\pagebreak
\section{Buyers Send iPhones on a Long Relay to China}

\lettrine{T}{hey}\mycalendar{Sept.'10}{25} show up in the early-morning hours: Chinese men and
women, waiting silently and somewhat nervously outside of Apple stores in New York. On some days the
lines they form can be a block long.

These are not typical Apple fans. Instead they are participants in a complex and curious trade
driven by China's demand for Apple's fashionable gadgets -- products that are made in China in the
first place and exported, only to make the long trip back.

Participants in New York and Shanghai say the process works like this: People wait in line at an
Apple store to buy the newest iPhone for \$600, paying a premium to skip the AT\&T contract. They
then sell the phones to middlemen, usually at electronics stores in Chinatown, for about \$750.

The phones are shipped off to China, where the iPhone 4 is not yet on sale, and are distributed to
local shops and e-commerce sites, where they sell for as much as \$1,000. Once the phones have been
``unlocked'' to break their ties to AT\&T, they can be used with local carriers.

But a change to this practice is coming. On Saturday, the iPhone 4 will go on sale in China, priced
at about \$750 for the 16-gigabyte version.

Most people in China can only dream of being able to afford an expensive phone. But millions of
Chinese are developing a taste for luxury goods, and Apple products have joined Louis Vuitton bags
as totems of wealth, said Shang-Jin Wei, director of the Jerome A.~Chazen Institute of International
Business at Columbia Business School.

``These trading networks have been around for a long time,'' Professor Wei said. ``They have
recently become a lot more pervasive due to rising incomes in China -- partially as a result of
exports to the U.S.''

An Apple spokesman would not discuss the systematic iPhone purchases at Apple stores, but the
company has tried to put a stop to them -- and has found it difficult to do so.

In June, the company came under fire in New York State for refusing to sell the iPhone 4 to some
Asian customers. Andrew M.~Cuomo, the New York attorney general, opened an investigation into the
matter, and Apple apparently backed off.

Those waiting in line in New York were not eager to talk about their mission or to identify
themselves. When asked what they are doing, the stock answer was always the same: ``I'm buying the
iPhone for a friend.'' Some buy a phone, conceal it in a bag and go back into the store to buy
another. A man who was asked what the second phone was for explained: ``I have two friends.''

Apple limits purchases to two iPhones a person, and to discourage repeat visits it keeps a record of
credit cards used, though cash purchases are not tracked. Apple says the limit ``helps us ensure
that there are enough iPhones for people who are shopping for themselves or buying a gift.''

The iPhone 4 went on sale in the United States in June, but it is still in such demand that many
stores quickly sell out of new shipments.

The buy-and-export scheme is not limited to New York. There have been anecdotal reports of similar
efforts elsewhere. Kate Peters of Durham, N.C., said she was visiting an Apple store in a mall there
last month when she saw ``an Asian woman with a group of college-age Chinese men,'' perhaps eight of
them. They all bought iPhones and iPads, paying with \$100 bills, though they seemed unfamiliar with
the currency, Ms.~Peters said.

The iPhone trade appears to be widely known in New York's Chinese community. An older Chinese man
sitting on a stoop in the Sunset Park neighborhood of Brooklyn, who said he worked seven days a week
as a cook, said the opportunity was too enticing for low-income immigrants to pass up.

``Many workers make a few dollars an hour working in restaurants or factories,'' he said. ``If they
wait in line for an hour at the Apple store to buy and sell phones, they can make \$300 in a single
morning.'' For many, he said, this is equivalent to an average week's pay.

IPhone sellers in China say the phones are often brought into the country by people who hide them in
their bags or even tape them to their bodies. More organized smugglers will bring in 100 or more
iPhones a day, and some will put phones into a shipping container with other goods.

``It's all about connections and channels,'' said one seller at the Sleepless Mall, a big
electronics market in Shanghai where wholesalers distribute phones to sellers in small stalls.
``Once you have good relationships with customs and airline companies, you can ship whatever
products you like. We smuggle it both by air and by boat.''

The buyers are willing to admit that the phone's allure has a lot to do with the status it conveys.
A 26-year-old woman in Shanghai who works for a media company said she had waited several weeks to
get the iPhone 4, explaining that ``Apple is a sign of coolness.''

A college freshman at the Sleepless Mall said he had tried to order an iPhone from Hong Kong, then
bought one at the mall. ``Since there are very few people using it, it's so cool to have one,'' he
said.

But on Saturday, when the iPhone goes on sale in China, it will have ripple effects abroad. This is
already pushing down prices of smuggled phones.

Professor Wei said the legitimate sales would cut into the smugglers' profits but would not wipe
them out. He said every aspect of the pricing and availability of the iPhone in China had been
calculated to make it a highly sought product.

``Apple knows exactly how much these products are selling for on the black market in China, and the
company will price its products accordingly,'' Professor Wei said. ``Limiting the sale of the iPhone
until now in China is likely part of a bigger corporate strategy to make it a luxury product that
people will pay higher prices for.''

If that is Apple's strategy, it has had some unexpected side effects. The scrutiny that Apple store
employees have given to Asian customers in New York has led some to complain of discrimination.

Grace Meng, a New York State assemblywoman whose district in Queens includes the Flushing
neighborhood, said some of her constituents had approached her office after being told they could
not buy the iPhone 4.

``I don't deny that there is a serious concern that Apple has,'' Ms.~Meng said. ``We just want to
make sure that no one is singled out because of their ethnicity or because they don't speak perfect
English.''

Ms.~Meng forwarded the complaints to the attorney general's office, where a spokesman, John Milgrim,
said the inquiry ``remains an ongoing investigation.''


% \end{multicols}

% \clearpage
% \renewcommand\listfigurename{\textit{Table of Figures}}
% {\footnotesize\textit{\listoffigures}}

\end{document}
