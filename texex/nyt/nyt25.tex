\documentclass[12pt]{article}
\title{Digest of The New York Times}
\author{The New York Times}

\usepackage{config}

% \makeindex
\begin{document}
\date{}
% \thispagestyle{empty}

% \begin{figure}
% \includegraphics*[width=0.3\textwidth]{The_New_York_Times_logo.png}
% \vspace{-5ex}
% \end{figure}
% \renewcommand\contentsname{\textsf{Digest of The New York Times}}
% {\footnotesize\textsf{\tableofcontents}}
\pagenumbering{Roman}
\cfoot{\textsf{\thepage}}
% \renewcommand\contentsname{}

\tableofcontents

\clearpage
\setcounter{page}{1}
\pagenumbering{arabic}
\cfoot{\textbf{\textrm{{\color{Goldenrod} {\large \thepage}}}}\textsf{\scriptsize
    ~of~\pageref*{LastPage}}}

\section{I.M.F. Chief Steps Into Dispute Over China's Currency Policy}

\lettrine{T}{he}\mycalendar{Oct.'10}{08} head of the International Monetary Fund urged China on
Thursday to allow its currency to rise in value, an attempt to keep a contentious issue between the
United States and China from broadening into a bitter international dispute.

The suggestion was made by Dominique Strauss-Kahn, managing director of the I.M.F., who also
expressed worry that the international coordination of economic policies that resulted from the
global economic crisis was eroding.

``The momentum is not vanishing, but decreasing, and that is a real threat, because everybody has to
keep in mind this mantra that there is no domestic solution to a global crisis,'' he said at a news
conference as around 13,000 officials, executives and other participants gathered here for the
annual meetings of the I.M.F. and the World Bank.

``Many are talking about a currency war,'' Mr.~Strauss-Kahn acknowledged, adding: ``What we all want
is the rebalancing of the global economy, and this rebalancing cannot happen'' without ``a natural
consequence of it, which is a change in the relative value of currencies.''

Responding to calls by the United States for the I.M.F. to take a more assertive role in mediating
exchange rate tensions, Mr.~Strauss-Kahn defended the institution, saying the I.M.F. had repeatedly
stated ``that we believe that the renminbi was substantially undervalued and something has to be
done to fix this problem over time.''

Without singling out China, Jean-Claude Trichet, the president of the European Central Bank, made a
similar point in Frankfurt.

``I think that exchange rates should reflect economic fundamentals, that excess volatility and
disorderly movements in exchange rates have adverse implications for economic and financial
stability,'' he said.

Finance officials around the world appeared to take note after the United States Treasury secretary,
Timothy F.~Geithner, on Wednesday called the currency issue the ``central existential challenge''
facing the world economy, one that threatened to undermine the rebalancing that the major economic
powers agreed to undertake in response to the crisis.

Rebalancing has big implications for the world's two largest economies.

The economic theory is that China should let its currency appreciate, which would make it easier for
Chinese consumers to buy imported goods from Europe and America.

Western leaders also want China to speed its efforts to strengthen its social safety net, so that
Chinese households can save less; this would stimulate domestic demand and reduce its dependence on
exports.

Meanwhile, the United States, the world's largest debtor, should save and invest more and spend and
borrow less -- a process Mr.~Geithner says is under way.

The political reality has been far messier.

Wen Jiabao, China's prime minister, again rebuffed calls for currency appreciation this week, saying
in Brussels on Wednesday that it would cause major social disruptions by harming the export-oriented
manufacturing that has fueled his nation's growth.

On the American side, there has been rare bipartisan agreement on pressuring the Chinese; the House
last week overwhelmingly approved a bill threatening tariffs on a broad array of Chinese imports if
Beijing did not allow greater currency flexibility, as it promised in June.

Although the House bill is not expected to become law, it was intended as a signal to Beijing of
American anger over Chinese imports and a loss of jobs to Chinese factories.

``We took action because nobody else would,'' Representative Sander M.~Levin, Democrat of Michigan
and chairman of the Ways and Means Committee, who shepherded the bill through the House, said on
Thursday. ``The best way to avoid a trade war is for there to be action, and I think we hope to
stimulate that.''

Mr.~Levin said ``the I.M.F. should step up to the plate,'' or else the currency issue ``will be
acted upon unilaterally.''

The currency issue comes at a challenging time for the I.M.F., which played a big role in responding
to the European sovereign debt crisis earlier this year but is now struggling to engage big emerging
economies like China, Brazil and India.

``The fund has shown its relevance during the crisis, but it is still a question of legitimacy,''
Mr.~Strauss-Kahn said, referring to a disputed proposal for European countries to surrender some of
their seats on the I.M.F. board to give more of a voice to rapidly growing, middle-income countries.

The I.M.F. this week forecast that global economic growth will slow to 4.2 percent next year from an
estimate of 4.8 percent this year, and that the growth will be uneven.

In Asia and Latin America, ``the crisis is over,'' Mr.~Strauss-Kahn said, and even sub-Saharan
Africa, which in the past has lagged the rest of the world in recovering from downturns, is growing
at a faster rate than the United States and Europe.

``Contrary to many others, we don't believe that this recovery will end with a double dip,''
Mr.~Strauss-Kahn said.

Mr.~Strauss-Kahn said the world economy faced three other challenges beside the strains on
international cooperation. Heavily indebted countries need to put fiscal reforms into effect without
jeopardizing growth. Unemployment is likely to remain high. And while progress on new financial
regulations has occurred, he said, ``very little has been done'' to overhaul banking supervision and
crisis resolution.

But Mr.~Strauss-Kahn, a former finance minister of France, suggested that rebalancing was likely to
be one of the most protracted challenges.

``This problem is not going to be fixed in five minutes,'' he said. ``It is a long-term problem.''

The currency dispute has renewed speculation that a modern-day Plaza Accord -- the 1985 deal in
which the United States agreed to weaken the dollar's exchange rate -- might be needed.

But Mr.~Strauss-Kahn and his principal deputy, John P.~Lipsky, brushed away that speculation, saying
the more appropriate forum for resolving the dispute was the Group of 20, the steering committee for
the largest economies, whose leaders plan to meet in Seoul next month.

\section{Siemens Wins Battle of Fast Trains}

\lettrine{I}{n}\mycalendar{Oct.'10}{08} an effort to prepare for competition on cross-Channel rail
traffic, Eurostar said Thursday that it had awarded a hotly sought contract to upgrade its aging
fleet of fast trains to a Germany company, Siemens.

The announcement did not sit well in Paris, which had been backing a French champion, and officials
sharply criticized the decision.

The \textsterling700 million, or \$1.1 billion, contract will provide Eurostar -- which is majority
owned by the French state through its ownership of the national railway S.N.C.F. -- with 10 of
Siemens' sleek new Velaro e320 trains. Siemens beat out the A.G.V. trains made by Alstom, the French
industrial conglomerate. Designed by the Italian firm Pininfarina, the trains will be capable of
traveling on other networks, as Eurostar seeks to extend its own reach into Germany and the
Netherlands.

The Eurostar e320, as it is known, will carry more than 900 passengers at a top speed of 320
kilometers, or 200 miles, per hour, compared with the 750 passengers the current generation carries,
at speeds of up to 300 kilometers per hour, Eurostar said. It will also have onboard WiFi and
entertainment systems. The company hopes the new fleet will better position it for the arrival of
Deutsche Bahn, the German operator that hopes to start offering service from points in Germany to
London by the end of 2013, when Eurostar's monopoly ends.

Nicolas Petrovic, Eurostar's chief executive, said Thursday in London that the decision to award the
contract to Siemens was based on ``technical, commercial and pricing criteria.''

``It was a very competitive tender and at the end Siemens made the best overall offer,''
Mr.~Petrovic said. ``There's nothing else behind that but commercial benefit to Eurostar.''

But the news was a bad break for Alstom at a time when stagnating economies are keeping European
orders down and competition is growing in overseas markets from rising stars like China. Alstom was
bailed out by the French state in 2004, when President Nicolas Sarkozy was finance minister, and it
is seen as a national standard-bearer.

Dominique Bussereau, the secretary of state for transportation, last month went so far as to call
for the creation of ``an Airbus of the rails,'' a cooperative arrangement to end ruinous competition
between French and German champions, which he said only benefited China.

In a joint statement Thursday, Mr.~Bussereau and Jean-Louis Borloo, the minister for ecology,
energy, and sustainable development, expressed their ``stupefaction'' at Eurostar's ``failure to
take the applicable security regulations into account.''

They said they supported international competition, but considering the three fires that have
occurred in the tunnel, including a particularly damaging one in 2008, ``no deterioration in the
level of security can be considered.''

They pointed to the fact that the Siemens' trains employ ``distributed traction,'' where the
locomotive function is distributed throughout the train, instead of traditional locomotives, which
pull from the front and push from the rear.

When the tunnel safety rules were written in the 1980s, the only trains considered were Eurostar
trains made by Alstom using the traditional technology. Since then, the global trend is increasingly
toward distributed traction.

The Channel Tunnel Safety Authority, which advises the French and British regulators overseeing the
tunnel, recommended in March that this rule be relaxed to allow for distributed traction trains. The
authority also called for a reduction to 375 meters from 400 meters in the minimum train length
allowed. The length is important because emergency exits inside the tunnel are spaced 375 meters
apart. (The e320s ordered by Eurostar are 400 meters long.)

Eurotunnel, as well as a French-British intergovernmental committee, approved those recommendations
this year.

``In principle, there is nothing in this train which would be ruled out as far as it being able to
obtain authorization,'' said Richard Clifton, head of the British delegation to the Channel Tunnel
Safety Authority.

But Mr.~Borloo and Mr.~Bussereau said that the French emphasized Wednesday at a meeting of the
intergovernmental committee ``the absolute necessity for exhaustive studies'' before distributed
traction trains could be authorized.

``What is surprising about this announcement is that the Siemens train doesn't conform to the safety
standards and security rules of the tunnel,'' said Virginie Hourdin-Bremond, a spokeswoman for
Alstom.

Siemens declined to comment.

``There will just be some more discussion, but I think common sense will prevail,'' said
Mr.~Petrovic of Eurostar, in acknowledging the French concerns.

He added that he expected the contract to be completed ``in the next few weeks at maximum.''

A European Commission directive to stimulate competition in international passenger rail routes came
into force Jan.~1, but the jury is still out on whether entrenched players in passenger traffic,
including Deutsche Bahn and S.N.C.F., also known as Soci\'et\'e nationale des chemins de fer
français, will make room for upstarts. A similar 2007 directive regarding freight traffic has not
yet had the desired effect.

On Oct.~19, a Deutsche Bahn high-speed train will arrive in London after passing through the tunnel
for tests. The company hopes to show that two of its 200-meter trains tied together are as safe as a
single 400-meter train, and that distributed traction is safe in the tunnel.

Deutsche Bahn, fully owned by the German state, estimates it will carry as many as a million
passengers to London each year, from Cologne and possibly Frankfurt, once its service is running.
Currently, its high-speed trains go as far as Brussels, meaning passengers change there for the
Eurostar service.

Andreas Fuhrmann, a Deutsche Bahn spokesman, said the new service would would generate a new market
in Germany for cross-Channel rail traffic, as most people traveling to Britain today choose to fly
or take a bus.

Like Eurostar, Deutsche Bahn is planning to use the Siemens Velaro platform for its service, he
said.


\section{Cost of E.U. Rises, Even as Countries Make Cuts}

\lettrine{I}{n}\mycalendar{Oct.'10}{08} Greece, taxes are up and so is the age of retirement. In
Spain, civil servants have taken pay cuts. In Britain, spending on welfare, the military and
education could be chopped by a quarter.

Despite mounting public protests across the Continent, an austerity drive unparalleled in modern,
united Europe is building.

In Brussels, meanwhile, the bureaucracy that runs the European Union is haggling over how much to
increase next year's budget.

In 2011, the European Union will pour billions more euros into the Continent's regions for
infrastructure and other projects. Spending on justice and security is set to rise sharply, while
even purely administrative costs are expected to increase by more than 4 percent.

Supporters see the spending as an antidote to austerity, a way to keep a fragile economic recovery
alive.

Critics say it highlights the remoteness of Brussels, where pay raises are written into law,
spending priorities are decided up to seven years in advance and millions are spent on questionable
efforts to spread the message of a 27-nation bloc that often seems to have little decisive to say on
issues that matter to voters, like immigration.

National governments pay most of the bill, and some have lost patience. Vince Cable, the British
business secretary, predicts a ``backlash'' against the bloc.

``When national governments, including mine, are having to make very painful cuts in public
spending,'' he told the European Parliament, ``no one can understand why the European budget is not
being subjected to the same discipline.''

The European Commission, the E.U. executive, is seeking a 5.9 percent increase in the bloc's 2011
budget, lifting annual spending above \texteuro130 billion, or nearly \$180 billion -- about half
the annual public spending of a midsize nation like the Netherlands. National governments have tried
to limit spending to \texteuro126.5 billion, a 2.9 percent increase.

But the European Parliament, which must approve the budget, sought to add or restore a host of
measures in a mix of altruism, special interests and pork barrel politics familiar to anyone who
tracks the U.S.~Congress.

Earmarks sought include \texteuro300 million for dairy farmers, \texteuro9 million for the World
Special Olympics Summer Games in Greece, \texteuro10 million for a school fruit plan and \texteuro8
million for beekeeping. All told, the total would be about \texteuro3.5 billion more than the
governments say they can afford.

``You cannot do things in an ivory tower,'' said James Elles, a British Conservative member of the
European Parliament who has been on its budget committee since 1984. ``People back home do observe
what's going on. They have to feel we are taking responsible decisions for 2011.''

Critics point to what they say are areas of fat -- \texteuro8 million for promoting awareness of the
bloc's agricultural policy, for example. They also highlight the cost of E.U. administration, which
the European Commission would like to increase by 4.5 percent, to \texteuro8.4 billion, in 2011.

While some European countries are cutting the salaries of civil servants, the national governments
are struggling to curb the pay of E.U. officials, who often earn significantly more than their
counterparts in national capitals, while also paying income taxes at lower rates and receiving
generous benefits.

When the governments held pay increases for E.U. officials to 1.85 percent this year -- half the
amount due under the system -- they were promptly taken to court by the European Commission. (Most
observers expect the commission to win).

Indeed, overall costs of administering the European Union are rising, exacerbated by new bodies
created by the Lisbon Treaty late last year.

There is the new president of the European Council, Herman Van Rompuy of Belgium, and a new foreign
policy chief, Catherine Ashton of Britain. There is also the European Parliament, which got an
additional \texteuro9.4 million to exercise new powers.

To run his operation, Mr.~Van Rompuy got \texteuro25 million for 2010. Ms.~Ashton's new foreign
service will cost \texteuro476 million in 2011, the commission says, much of it met by transferring
existing staff, but with \texteuro34.5 million needed for new posts.

The other question is where to house the service, and whether to lease a new building at an
estimated annual cost of \texteuro10 to \texteuro15 million.

That is a tough sell when the European Union does not lack new buildings, so Ms.~Ashton is now
targeting one occupied by legal translators.

Salvador Garrigo Polledo, a Spanish center-right member of the European Parliament's budget
committee, concedes a perception problem.

``It's a very small bureaucracy,'' he said, ``but it is very visible for the European taxpayer.''

As it has for decades, the bulk of the budget -- about \texteuro110 billion this year -- will go to
regional or agricultural subsidies, seen by some as a counterweight to austerity.

``The E.U. budget is about investment in the very things we need to take care of in these difficult
economic times,'' said Goran Farm, a budget spokesman for the main center-left group in the European
Parliament. ``We must fight the austerity message from ministers -- especially the hard-line ideas
of governments in the U.K., the Czech Republic and some Scandinavian countries.''

Many economists, though, consider agriculture subsidies an added cost to European consumers and a
burden on farmers in poor countries who cannot compete on their own.

Much regional spending is mandated; infrastructure projects where contracts are already signed
cannot be canceled, said Patrizio Fiorilli, a budget spokesman for the European Commission.

Jorge Nú\~{n}ez Ferrer, an associate research fellow at the Center for European Policy Studies,
believes that the economic backdrop will concentrate minds when the bloc begins to review
longer-term priorities later this year.

``Austerity will have an impact on the budget,'' he said. ``I think there is an understanding that
you cannot just defend funding things for historical reasons or because it is politically
convenient.''

Next year, for instance, E.U. officials may have their pay cut by 0.4 percent.

But that modest sacrifice is unlikely to placate the protesting public.

Serge Colpin, 58, from Charleroi in the depressed south of Belgium, joined demonstrations last week
in Brussels. An unemployed former soldier who said he had been seeking work for eight years,
Mr.~Colpin is angry at bankers, politicians -- and people who work for the European Union.

``They earn too much, they are exempt from a lot of tax, they have cars, they have drivers,'' he
said. ``And we pay.''

\section{Obama to Veto Bill That Could Speed Foreclosures}

\lettrine{A}{s}\mycalendar{Oct.'10}{08} slipshod bookkeeping by some big mortgage lenders continued
to rattle the housing market on Thursday, another major lender indicated it would suspend sales of
foreclosed homes and White House officials said President Obama would not sign a bill that critics
suggested could facilitate foreclosure fraud.

Demands for investigations and nationwide moratoriums also grew. Representative Edolphus Towns, the
New York Democrat who is chairman of the House Committee on Oversight and Government Reform, called
on lenders to voluntarily suspend foreclosures until they completed internal investigations. On
Wednesday, Attorney General Eric H.~Holder Jr.~said that the Financial Fraud Enforcement Task Force,
which Mr.~Obama created to examine financial fraud, is looking at the growing reports of foreclosure
fraud.

The president's pocket veto -- rejecting a bill by withholding his signature while Congress is away
-- effectively kills the measure since lawmakers, who are out of town until after the Nov.~2 midterm
elections, are not in position to override his decision with a two-thirds vote of the House and
Senate.

The bill would have mandated that notarizations of mortgages and other financial documents done in
one state, including those done electronically, be recognized in other states. By the time the bill
arrived at Mr.~Obama's desk, however, it was caught in the controversy over major institutions'
acknowledgment of problems in processing documents for tens of thousands of foreclosures. Those
included suspected forgeries and notaries' failure to review the paperwork as required.

Critics, particularly consumer groups, said the measure for interstate notarizations would have made
it even easier for banks and other lenders to rush the foreclosure process. JPMorgan Chase, Bank of
America and GMAC Mortgage have stopped foreclosures in nearly half the states, pending
investigations into the process.

A fourth major lender, PNC Financial Services Group, has also suspended sales of foreclosed homes
for 30 days, according to a memo from a title insurer, Commerce Title, that works closely with the
bank.

PNC is alerting title insurance companies that it is postponing the closings effective immediately,
according to the memo. ``We have been given notice from PNC Legal that there is a moratorium going
into effect'' on residential foreclosures, the memo from Commerce Title said.

A PNC spokesman, Frederick Solomon, declined to comment beyond saying that the lender was reviewing
its mortgage servicing procedures.

PNC, which is based in Pittsburgh, became one of the country's largest lenders with the acquisition
of Ohio-based National City Corporation two years ago. National City, an aggressive lender during
the housing boom, collapsed during the financial crisis.

Given the outcry over the far-reaching foreclosure crisis, Congressional aides said lawmakers were
unlikely to take umbrage at Mr.~Obama's decision to let the notary measure expire. The White House,
in announcing the pocket veto, indicated that it could work with Congress later on some alternative.

Mr.~Obama's communications director, Dan Pfeiffer, said Thursday in a statement, ``We need to think
through the intended and unintended consequences of this bill on consumer protections, especially in
light of the recent developments with mortgage processors.''

``The authors of this bill no doubt had the best intentions in mind when trying to remove
impediments to interstate commerce,'' Mr.~Pfeiffer added. ``We will work with them and other leaders
in Congress to explore the best ways to achieve this goal.''

The measure had sponsors from both parties and was so uncontroversial in Congress that it passed
without roll-call votes -- in the House by a voice vote in April and in the Senate by unanimous
consent last week.

The president's decision against signing the measure was his first veto intended to kill a bill; he
has pocket-vetoed a measure once before, but only because it was duplicative of other legislation.
President George W.~Bush, who also came to office with his party in control of Congress, did not
veto a bill in his first term.

Mr.~Towns, the New York congressman, also asked the New York State attorney general, Andrew
M.~Cuomo, to investigate accusations of fraud by the lenders. At least a half-dozen other attorneys
general have already begun investigations.

\section{Samsung Faces Weak Outlook on Flat Screens and TVs}

\lettrine{S}{amsung}\mycalendar{Oct.'10}{08} Electronics Co, the world's largest memory chip maker,
disappointed markets with weak earnings guidance, halting its second straight quarter of record
results, due to plunging prices of flat screens and TVs.

The South Korean electronics powerhouse is braced for a tough outlook as a sputtering world economy
has hit demand for TVs and computers and exacerbated steep price falls of components such as chips
and flat screens.

``LCD (liquid crystal display) and TV performance appears to be worse than expected and the downward
pressure on earnings will only grow as chip prices are also falling and TV makers increase price
cuts,'' said Chung Young-woo, analyst at Korea Investment \& Securities.

``Usual uptick in seasonal year-end demand will be smaller this time and an earnings recovery is
unlikely until early next year.''

Samsung, the first major global technology firm to issue preliminary third-quarter estimates, has so
far performed strongly over chip rival Micron and held on to its No.~1 slot in TVs against Sony Corp
and Panasonic.

Samsung is also challenging Apple Inc with its Galaxy S smartphone model, powered by Google's
Android software.

By 0040 GMT, shares in Samsung, Asia's most valuable technology firm worth \$116 billion, dropped
2.3 percent, lagging a 0.2 percent drop in the KOSPI.

Samsung, also the world's No.2 maker of mobile phones and the No.1 maker of LCDs, has benefited from
strong demand from China and as improved corporate spending boosted sales of memory chips and flat
screens, but smartphone sales stayed weak.

cEarnings will slide further but the stock is looking attractive as the slowdown is already priced
in and Samsung will benefit most from any demand recovery, being the No.1 in many areas,c said Jung
Sang-jin, a fund manager at Dongbu Asset Management.

Jung has been increasing Samsung shares to the company's portfolio since last week.

Samsung, valued 3 times more than No.1 handset maker Nokia and its key TV rival Sony, dropped 1
percent this year to Wednesday's close, underperforming KOSPI's 13 percent rise.

On Thursday, Samsung estimated its third-quarter operating profit at a median 4.8 trillion won
(\$4.3 billion) of 4.6 and 5.0 trillion won range, lower than a consensus forecast of 5.2 trillion
won polled by Thomson Reuters I/B/E/S.

That would be down 4 percent from the previous record of 5 trillion won in the preceding quarter but
up 14 percent from the 4.2 trillion won reported a year ago. Samsung reports quarterly results in
late October.

Sales were estimated at 40 trillion won.

Profit from its chips division is set to account for nearly 70 percent of Samsung's total profit in
the third quarter.

CHIP PRICES FALL

Analysts forecast Samsung's profit will shrink to around 4 trillion won in the current quarter and
stay around that level till the second quarter of next year due to weak prices of chips and flat
screens.

Prices of dynamic random access memory (DRAM), mainly used in computers, have fallen more than 20
percent from its peak in May and may drop another 20 percent this quarter, as PC sales growth has
declined.

Research firm Gartner cut its second-half PC sales growth forecast to just over 15 percent last
month.

A wobbly global economy is also hitting sales of TVs, computers and laptops, which together account
for the majority of large-sized LCD panels, and investors are now worried demand could slow further
as China tightens its economic policy.

Most analysts expect Samsung's LCD profit margins to fall further to around break-even level in the
fourth quarter from an estimated 4 percent in the third quarter.

Its TV division faces increased competition and weak demand might force it to slash prices and hit
sales of premium products such as LED-backlit LCD models and 3D sets.

The mobile phone business, one of Samsung's weakest points due to delays in smartphone offerings, is
now the sole bright spot and firmly on a recovery path, thanks to strong sales of Galaxy S.~The
high-end model has sold more than 5 million units since its June launch.

Last month, Samsung raised 2010 smartphone sales target to 25 million units from the previous target
of around 18 million.

\section{MSNBC.com May Change Its Name}

\lettrine{N}{bc}\mycalendar{Oct.'10}{08} Universal and Microsoft, the parents of msnbc.com, are
holding high-level talks about changing its name, an unusual and potentially risky endeavor for the
third most popular news Web site in the United States.

The two parents have not yet agreed on what to call the site. But according to internal memorandums
obtained by The New York Times this week, the parents have concluded that the brand known as
msnbc.com, a strictly objective news site, is widely confused with MSNBC, the cable television
channel that has taken a strongly liberal bent in recent years.

Charlie Tillinghast, the president of msnbc.com, wrote in one of the memos, ``Both strategies are
fine, but naming them the same thing is brand insanity.'' The channel and Web site are already
separate companies.

Under the current plan, the msnbc.com Web address would become a site exclusively for the cable
channel, fulfilling the channel's desire to have an independent site to promote its TV programs. The
existing news site, called the ``blue site'' internally, would move to a new and as-yet-undetermined
Web address. There is a subsection on msnbc.com for the cable channel.

The network of Web sites under the msnbc.com umbrella are visited by almost 50 million Internet
users each month, according to the measurement firm comScore. Only two news brands, Yahoo and
CNN.com, are bigger.

Andrew Heyward, a former CBS News president and an adviser to media companies on digital strategy,
said the renaming idea had merit. ``It's incredibly important in this media cacophony for brands to
be consistent, for brands to stand for something,'' said Mr.~Heyward, who has advised NBC in the
past. ``And those two brands, each strong in their respective areas, are increasingly standing for
different things.''

Corporations change their names from time to time (Andersen Consulting became Accenture, Philip
Morris became Altria, Blackwater became Xe) but giving up a Web address as popular as msnbc.com is
highly unusual. It is akin to a business closing a bustling storefront and posting a sign that asks
customers to visit its new location.

For a Web site, at least, the new location is only a click away. ``You can quickly redirect people
who might be confused,'' Mr.~Heyward said. Nonetheless, msnbc.com risks sacrificing years of brand
loyalty by coining a new name.

NBC, which is in the process of being sold to Comcast, and Microsoft have been conducting research
about potential new names for the last few months. ``Consensus in this case is a tall order,''
Mr.~Tillinghast wrote in one of the memos.

A board meeting that had been scheduled for the end of October to talk about the change was delayed
until mid-November.

One of the new names under consideration is NBCNews.com -- something that NBC would seem to favor --
but the companies are testing entirely new names, as well, the memos show. The question seems to be:
Should they go with a trusted and recognized name like NBCNews.com or try to build a fresh new
brand?

In a statement Wednesday, Mr.~Tillinghast said, ``We have an enviable portfolio of news brands and
routinely have strategic conversations about how to maximize them.''

(The Times and msnbc.com have an agreement to share some articles and video.)

The change is being contemplated because MSNBC and msnbc.com are on somewhat divergent paths.

They were founded together in 1996 by NBC and Microsoft, with the cable television channel based in
New Jersey and the Web site based at Microsoft's headquarters in Redmond, Wash. In 2005, NBC bought
Microsoft's stake in the cable channel, but the two parents remained together for the Web site,
which is a crucial provider of content to Microsoft's MSN.com portal.

Employees at msnbc.com work closely with employees of MSNBC and NBC News. But the Web site has its
own reporters, editors, producers, photographers and advertising sales staff. And those employees
have at times felt as if they were stuck in the shadow of the cable channel.

In recent years, MSNBC's shift to the left, with hosts like Keith Olbermann and Rachel Maddow, has
further complicated the TV/Web relationship. This week, the channel introduced a splashy ad campaign
and a new tagline, ``Lean Forward'' that reinforces the opinionated nature of the programming.

The cable channel has been looking for a way to distinguish itself online; the channel's president,
Phil Griffin, briefly discussed the acquisition of The Huffington Post earlier this year, but was
rebuffed by its co-founders, as first reported by New York magazine this week.

Meanwhile, msnbc.com has remained what Mr.~Tillinghast called in Tuesday's memo an ``impartial news
product.''

He wrote that the ``Lean Forward'' announcement ``only exacerbates the brand misalignment problem''
that he had been trying to solve. He envisions a ``brand family,'' with the to-be-renamed Web site
positioned at the head of the table, joined by two existing spinoff sites, one for NBC's ``Today''
show and one devoted to breaking news alerts. But first msnbc.com's family has to agree on a new
name.

\section{Inquiry Finds Guards at U.S.~Bases Are Tied to Taliban}

\lettrine{A}{fghan}\mycalendar{Oct.'10}{08} private security forces with ties to the Taliban,
criminal networks and Iranian intelligence have been hired to guard American military bases in
Afghanistan, exposing United States soldiers to surprise attack and confounding the fight against
insurgents, according to a Senate investigation.

The Pentagon's oversight of the Afghan guards is virtually nonexistent, allowing local security
deals among American military commanders, Western contracting companies and Afghan warlords who are
closely connected to the violent insurgency, according to the report by investigators on the staff
of the Senate Armed Services Committee.

The United States military has almost no independent information on the Afghans guarding the bases,
who are employees of Afghan groups hired as subcontractors by Western firms awarded security
contracts by the Pentagon. At one large American airbase in western Afghanistan, military personnel
did not even know the names of the leaders of the Afghan groups providing base security, the
investigators found. So they used the nicknames that the contractor was using -- Mr.~White and
Mr.~Pink from ``Reservoir Dogs,'' the 1992 gangster movie by Quentin Tarantino. Mr.~Pink was later
determined to be a ``known Taliban'' figure, they reported.

In another incident, the United States military bombed a house where it was believed that a Taliban
leader was holding a meeting, only to discover later that the house was owned by an Afghan security
contractor to the American military, who was meeting with his nephew -- the Taliban leader.

Some Afghans hired by EOD Technology, which was awarded a United States Army contract to provide
security at a training center for Afghan police officers in Adraskan, near Shindand, were also
providing information to Iran, the report asserted. The Senate committee said it received
intelligence from the Defense Intelligence Agency about Afghans working for EOD, and that the
reporting found that some of them ``have been involved in activities at odds with U.S.~interests in
the region.''

The Senate Armed Services Committee adopted the report by a unanimous vote, although Republican
members issued a statement critical of the report for too narrowly focusing on case studies in
western Afghanistan.

In response to the Senate report, Defense Secretary Robert M.~Gates issued a letter saying that the
Pentagon recognized the problems and has created new task forces to help overhaul contracting
procedures in Afghanistan. ``Through the new programs we have implemented, I believe D.O.D. has
taken significant steps to benefit our forces on the ground while not providing aid to our
enemies,'' Mr.~Gates wrote.

The latest disclosures follow a series of reports, including articles in The New York Times and
testimony before a House committee, describing bribes paid by contractors to the Taliban and other
warlords to make sure supply convoys for the American military were provided safe passage.

But the Senate report goes further, spelling out the close relations between some contractors and
the forces arrayed against the Kabul government and the Americans, and saying that the proliferation
of contractors in the country is sometimes fueling the very insurgency that the military is there to
combat. It names a few of the contracting companies, and uses one base as a case study, but calls
the problems it identified pervasive.

``We must shut off the spigot of U.S.~dollars flowing into the pockets of warlords and power brokers
who act contrary to our interests,'' said Senator Carl Levin, the Michigan Democrat who is the
committee's chairman.

``There are truly some outrageous allegations here, and it's a wake-up call that we have to get a
better handle on contractors in Afghanistan and ensure that taxpayer dollars don't end up in the
hands of the enemy,'' said Richard Fontaine, a senior fellow at the Center for a New American
Security, a Washington research group.

There are more than 26,000 private security employees in Afghanistan, and 90 percent of them are
working under United States government contracts or subcontracts. Almost all are tied to the
militias of local warlords and other powerful Afghan figures outside the control of the American
military or the Afghan government, the report found.

The contracting firms are now hiring active-duty members of the Afghan military and security forces,
the investigators found, further undermining the efforts by the United States to help Afghanistan
build a stronger military that can take on the Taliban insurgency on its own.

The Senate report focuses heavily on security contracting at remote American military bases in
western Afghanistan, including the air base in Shindand, near Herat. ArmorGroup, a British-based
security firm, was hired by a contractor to the United States Air Force to provide security at
Shindand, and then ArmorGroup turned in 2007 to two warlords who had their own militias to do the
actual security work. ArmorGroup called them ``Mr.~White'' and ``Mr.~Pink,'' and few Americans knew
their real identities, although a leader of an American military team at an adjacent base had
recommended Mr.~Pink.

``The two warlords and their successors served as manpower providers for ArmorGroup for the next 18
months -- a period marked by a series of violent incidents,'' the report found.

Fights soon erupted between the forces of Mr.~White and Mr.~Pink, with Mr.~Pink finally killing
Mr.~White. Mr.~Pink then sought refuge with the Taliban. ArmorGroup then turned to Mr.~White's
brother, Mr.~White II, to run its security force, but also continued to employ Mr.~Pink's men, even
though they knew he was now working with the Taliban.

In a raid on Aug., 21, 2008, in Azizabad, Afghanistan, American forces bombed a house where a local
Taliban leader, Mullah Sadeq, was suspected of holding a meeting. It was the home of Mr.~White II;
he was killed in the raid, along with seven other men employed as security guards by ArmorGroup or
ArmorGroup Mine Action, an affiliated company with a contract with the United Nations for mine
clearing.

The Azizabad raid sparked outrage within Afghanistan. Local villagers, human rights officials and
Afghan government officials said that the attack had resulted in more than 90 civilian deaths. The
raid had a broad impact on relations between the Afghan government and the American military, and
was one of the major incidents that led to a reassessment by President Hamid Karzai of his support
for American air raids in the country.

Mr.~Karzai visited the village after the attack, and President George W.~Bush called Mr.~Karzai to
express his regret. But the report shows that the bombing raid was entangled in the interplay
between contractors and the Taliban, and occurred during a meeting between Mr.~White II and the
suspect Taliban leader, Mullah Sadeq.

Providing contracts to local militia leaders with ties to the Taliban ``gives these warlords an
independent funding source,'' observed Carl Forsberg, an analyst with the Institute for the Study of
War in Washington. ``And it gives them a feeling of impunity.''

\section{Rampant Fraud Threat to China's Brisk Ascent}

\lettrine{N}{o}\mycalendar{Oct.'10}{08} one disputes Zhang Wuben's talents as a salesman. Through
television shows, DVDs and a best-selling book, he convinced millions of people that raw eggplant
and immense quantities of mung beans could cure lupus, diabetes, depression and cancer.

For \$450, seriously ill patients could buy a 10-minute consultation and a prescription -- except
Mr.~Zhang, one of the most popular practitioners of traditional Chinese medicine, was booked through
2012.

But when the price of mung beans skyrocketed this spring, Chinese journalists began digging deeper.
They learned that contrary to his claims, Mr.~Zhang, 47, was not from a long line of doctors (his
father was a weaver). Nor did he earn a degree from Beijing Medical University (his only formal
education, it turned out, was the brief correspondence course he took after losing his job at a
textile mill).

The exposure of Mr.~Zhang's faked credentials provoked a fresh round of hand-wringing over what many
scholars and Chinese complain are the dishonest practices that permeate society, including students
who cheat on college entrance exams, scholars who promote fake or unoriginal research, and dairy
companies that sell poisoned milk to infants.

The most recent string of revelations has been bracing. After a plane crash in August killed 42
people in northeast China, officials discovered that 100 pilots who worked for the airline's parent
company had falsified their flying histories. Then there was the padded r\'esum\'e of Tang Jun, the
millionaire former head of Microsoft China and something of a national hero, who falsely claimed to
have received a doctorate from the California Institute of Technology.

Few countries are immune to high-profile frauds. Illegal doping in sports and malfeasance on Wall
Street are running scandals in the United States. But in China, fakery in one area in particular --
education and scientific research -- is pervasive enough that many here worry it could make it
harder for the country to climb the next rung on the economic ladder.

\emph{A Lack of Integrity}

China devotes significant resources to building a world-class education system and pioneering
research in competitive industries and sciences, and has had notable successes in network computing,
clean energy, and military technology. But a lack of integrity among researchers is hindering
China's potential and harming collaboration between Chinese scholars and their international
counterparts, scholars in China and abroad say.

``If we don't change our ways, we will be excluded from the global academic community,'' said Zhang
Ming, a professor of international relations at Renmin University in Beijing. ``We need to focus on
seeking truth, not serving the agenda of some bureaucrat or satisfying the desire for personal
profit.''

Pressure on scholars by administrators of state-run universities to earn journal citations -- a
measure of innovation -- has produced a deluge of plagiarized or fabricated research. In December, a
British journal that specializes in crystal formations announced that it was withdrawing more than
70 papers by Chinese authors whose research was of questionable originality or rigor.

In an editorial published earlier this year, The Lancet, the British medical journal, warned that
faked or plagiarized research posed a threat to President Hu Jintao's vow to make China a ``research
superpower'' by 2020.

``Clearly, China's government needs to take this episode as a cue to reinvigorate standards for
teaching research ethics and for the conduct of the research itself,'' the editorial said. Last
month a collection of scientific journals published by Zhejiang University in Hangzhou reignited the
firestorm by publicizing results from a 20-month experiment with software that detects plagiarism.
The software, called CrossCheck, rejected nearly a third of all submissions on suspicion that the
content was pirated from previously published research. In some cases, more than 80 percent of a
paper's content was deemed unoriginal.

The journals' editor, Zhang Yuehong, emphasized that not all the flawed papers originated in China,
although she declined to reveal the breakdown of submissions. ``Some were from South Korea, India
and Iran,'' she said.

The journals, which specialize in medicine, physics, engineering and computer science, were the
first in China to use the software. For the moment they are the only ones to do so, Ms.~Zhang said.

\emph{Plagiarism and Fakery}

Her findings are not surprising if one considers the results of a recent government study in which a
third of the 6,000 scientists at six of the nation's top institutions admitted they had engaged in
plagiarism or the outright fabrication of research data. In another study of 32,000 scientists last
summer by the China Association for Science and Technology, more than 55 percent said they knew
someone guilty of academic fraud.

Fang Shimin, a muckraking writer who has become a well-known advocate for academic integrity, said
the problem started with the state-run university system, where politically appointed bureaucrats
have little expertise in the fields they oversee. Because competition for grants, housing perks and
career advancement is so intense, officials base their decisions on the number of papers published.

``Even fake papers count because nobody actually reads them,'' said Mr.~Fang, who is more widely
known by his pen name, Fang Zhouzi, and whose Web site, New Threads, has exposed more than 900
instances of fakery, some involving university presidents and nationally lionized researchers.

When plagiarism is exposed, colleagues and school leaders often close ranks around the accused.
Mr.~Fang said this was partly because preserving relationships trumped protecting the reputation of
the institution. But the other reason, he said, is more sobering: Few academics are clean enough to
point a finger at others. One result is that plagiarizers often go unpunished, which only encourages
more of it, said Zeng Guoping, director of the Institute of Science Technology and Society at
Tsinghua University in Beijing, which helped run the survey of 6,000 academics.

He cited the case of Chen Jin, a computer scientist who was once celebrated for having invented a
sophisticated microprocessor but who, it turned out, had taken a chip made by Motorola, scratched
out its name, and claimed it as his own. After Mr.~Chen was showered with government largess and
accolades, the exposure in 2006 was an embarrassment for the scientific establishment that backed
him.

But even though Mr.~Chen lost his university post, he was never prosecuted. ``When people see the
accused still driving their flashy cars, it sends the wrong message,'' Mr.~Zeng said.

The problem is not confined to the realm of science. In fact many educators say the culture of
cheating takes root in high school, where the competition for slots in the country's best colleges
is unrelenting and high marks on standardized tests are the most important criterion for admission.
Ghost-written essays and test questions can be bought. So, too, can a ``hired gun'' test taker who
will assume the student's identity for the grueling two-day college entrance exam.

Then there are the gadgets -- wristwatches and pens embedded with tiny cameras -- that transmit
signals to collaborators on the outside who then relay back the correct answers. Even if such
products are illegal, students spent \$150 million last year on Internet essays and high-tech
subterfuge, a fivefold increase over 2007, according to a Wuhan University study, which identified
800 Web sites offering such illicit services.

Academic deceit is not limited to high school students. In July, Centenary College, a New Jersey
institution with satellite branches in China and Taiwan, shuttered its business schools in Shanghai,
Beijing and Taipei after finding rampant cheating among students. Although school administrators
declined to discuss the nature of the misconduct, it was serious enough to withhold degrees from
each of the programs' 400 students. Given a chance to receive their M.B.A.'s by taking another exam,
all but two declined, school officials said.

\emph{Nonchalant Cheating}

Ask any Chinese student about academic skullduggery and the response is startlingly nonchalant.
Arthur Lu, an engineering student who last spring graduated from Tsinghua University, considered a
plum of the country's college system, said it was common for students to swap test answers or
plagiarize essays from one another. ``Perhaps it's a cultural difference but there is nothing bad or
embarrassing about it,'' said Mr.~Lu, who started this semester on a master's degree at Stanford
University. ``It's not that students can't do the work. They just see it as a way of saving time.''

The Chinese government has vowed to address the problem. Editorials in the state-run press
frequently condemn plagiarism and last month, Liu Yandong, a powerful Politburo member who oversees
Chinese publications, vowed to close some of the 5,000 academic journals whose sole existence, many
scholars say, is to provide an outlet for doctoral students and professors eager to inflate their
publishing credentials.

Fang Shimin and another crusading journalist, Fang Xuanchang, have heard the vows and threats
before. In 2004 and again in 2006, the Ministry of Education announced antifraud campaigns but the
two bodies they established to tackle the problem have yet to mete out any punishments.

In recent years, both journalists have taken on Xiao Chuanguo, a urologist who invented a surgical
procedure aimed at restoring bladder function in children with spina bifida, a congenital
deformation of the spinal column that can lead to incontinence, and when untreated, kidney failure
and death.

In a series of investigative articles and blog postings, the two men uncovered discrepancies in
Dr.~Xiao's Web site, including claims that he had published 26 articles in English-language journals
(they could only find four) and that he had won an achievement award from the American Urological
Association (the award was for an essay he wrote).

But even more troubling, they said, were assertions that his surgery had an 85 percent success rate.
Of more than 100 patients interviewed, they said none reported having been cured of incontinence,
with nearly 40 percent saying their health had worsened after the procedure, which involved
rerouting a leg nerve to the bladder. (In early trials, doctors in the United States who have done
the surgery have found the results to be far more promising.)

Wherever the truth may have been, Dr.~Xiao was incensed. He filed a string of libel suits against
Fang Shimin and told anyone who would listen that revenge would be his.

This summer both men were brutally attacked on the street in Beijing -- Fang Xuanchang by thugs with
an iron bar and Fang Shimin by two men wielding pepper spray and a hammer.

When the police arrested Dr.~Xiao on Sept.~21, he quickly confessed to hiring the men to carry out
the attack, according to the police report. His reason, he said, was vengeance for the revelations
he blames for blocking his appointment to the prestigious Chinese Academy of Sciences.

Despite his confession, Dr.~Xiao's employer, Huazhong University of Science and Technology, appeared
unwilling to take any action against him. In the statement they released, administrators said they
were shocked by news of his arrest but said they would await the outcome of judicial procedures
before severing their ties to him.

\section{Sun Co-Founder Uses Capitalism to Help Poor}

\lettrine{V}{inod}\mycalendar{Oct.'10}{08} Khosla, the billionaire venture capitalist and co-founder
of Sun Microsystems, was already among the world's richest men when he invested a few years ago in
SKS Microfinance, a lender to poor women in India.

But the roaring success of SKS's recent initial public stock offering in Mumbai has made him richer
by about \$117 million -- money he says he plans to plow back into other ventures that aim to fight
poverty while also trying to turn a profit.

And he says he wants to challenge other rich Indians to do more to help their country's poor.

An Indian transplant to Silicon Valley, Mr.~Khosla plans to start a venture capital fund to invest
in companies that focus on the poor in India, Africa and elsewhere by providing services like
health, energy and education.

By backing businesses that provide education loans or distribute solar panels in villages, he says,
he wants to show that commercial entities can better help people in poverty than most nonprofit
charitable organizations.

``There needs to be more experiments in building sustainable businesses going after the market for
the poor,'' he said in a telephone interview from his office in Menlo Park, Calif. ``It has to be
done in a sustainable way. There is not enough money to be given away in the world to make the poor
well off.''

Mr.~Khosla's advocacy of the bootstrap powers of capitalism is part of an increasingly popular
school of thought: businesses, not governments or nonprofit groups, should lead the effort to
eradicate global poverty.

Some nonprofit experts say commercial social enterprises have significant limitations and pose
conflicts of interest. But proponents like Mr.~Khosla draw inspiration from the astounding global
growth of microfinance -- the business of giving small loans to poor entrepreneurs, of which SKS
Microfinance is a notable practitioner.

Advocates also find intellectual support for the idea from the work of business management
professors like the late C.~K. Prahalad, who have argued that large corporations can do well and do
good by aiming at people at the so-called bottom of the pyramid.

Besides Mr.~Khosla, entrepreneurs like Pierre Omidyar, a co-founder of eBay, and Stephen M.~Case, a
co-founder of America Online, have started funds with similar aims.

But Mr.~Khosla, 55, who moved to the United States from India as a graduate student in 1976, has
another motive, too. He wants to goad other rich Indians into giving away more of their wealth.

India's torrid growth over the last decade has helped enrich many here -- Forbes estimates that
India now has 69 billionaires, up from seven in 2000 -- but only a few have set up large charities,
endowments or venture capital funds.

``It surprises me that in India there is not a tradition of large-scale giving and helping to solve
social problems and set a social model,'' Mr.~Khosla said.

Mr.~Khosla is not alone in worrying about the state of Indian philanthropy. Bill Gates, the
Microsoft co-founder, who was in China last week with the billionaire investor Warren E.~Buffett,
said Thursday that he and Mr.~Buffett might go to India as part of their campaign to get the very
rich to give away half their wealth.

Charitable activities and venture capital investing have been a mainstay for some Indian business
families like the Tatas and for technology entrepreneurs like Aziz Premji of Wipro, the Bangalore
outsourcing firm. But many others have given very little.

A recent Bain \& Company study estimated that Indians give much less as a percentage of the
country's gross domestic product than Americans. Moreover, individual and corporate donations
account for just 10 percent of the charitable giving in India, compared with 75 percent in the
United States and 34 percent in Britain. The balance comes from the government and foreign
organizations.

Rich Indians ``are more into temple building and things like that,'' said Samit Ghosh, the chief
executive of Ujjivan Financial, a microlender based in Bangalore, ``rather than putting their money
into real programs, which will have real impact on poverty alleviation.''

Mr.~Khosla said his experience with microfinance had helped shape his views on the best way to
tackle poverty. He has invested in commercial microfinance lenders and has donated to nonprofit
ones, and he said that moneymaking versions had grown much faster and reached many more needy
borrowers.

He said he wanted to help create a new generation of companies like SKS, which started lending as a
commercial company in 2006. It now has 6.8 million customers and a loan portfolio of 43 billion
rupees (\$940 million).

By contrast, CashPor, a nonprofit Indian lender to which Mr.~Khosla has also given money, has
417,000 borrowers and a portfolio of 2.7 billion rupees (\$58 million) even though it started
operations in 1996.

Besides growing faster, SKS, India's largest microfinance company, has become a stock market
darling. The company floated its shares on India's stock exchanges in mid-August, and they have
risen 40 percent since then.

At current prices, Mr.~Khosla's 6 percent stake in SKS is worth about \$120 million, about 37 times
what he invested in the firm in 2006 and 2007. (Shares of SKS fell 7 percent on Monday after the
company said it had fired its chief executive, Suresh Gurumani. An SKS spokesman, Atul Takle,
declined to answer questions.)

Mr.~Khosla said it might take at least a year to set up his new venture fund. He intends to finance
it from his SKS profits and then return to the fund any profits from subsequent ventures it backs.

Mr.~Khosla has already been investing in companies that he says fit his model of profitable poverty
alleviation. One is MokshaYug Access, which sets up milk collection and chilling plants in India to
help dairy farmers. The company says it helps farmers reduce transportation costs and get higher
prices for their milk than they can with local distributors.

Philanthropy experts say commercial companies play an important role in combating poverty by
creating jobs. But they say these ``social enterprises,'' as they are sometimes known, cannot be
solely relied upon to address the many entrenched causes of poverty.

Moreover, as the fallout from the global financial crisis has made clear, the profit-maximizing
tendencies of businesses can hurt society, said Phil Buchanan, president for the Center for
Effective Philanthropy, a research organization based in Cambridge, Mass.

Nonprofits are effective because they can ``take issue with the unbridled pursuit of profit at the
expense of people's lives,'' Mr.~Buchanan said. ``I think some of that gets lost in all of the hype
around social enterprise.''

Mr.~Khosla says that he is not completely opposed to charities -- that his fund may even donate to
some nonprofit entities. But he says he is generally skeptical that nongovernmental organizations
can accomplish much because they tend to drift away from what their donors wanted them to do.

``I am relatively negative on most N.G.O.'s and their effectiveness,'' he said. ``I am not negative
on their intentions.''

\section{Drug Maker From China Pleads Guilty}

\lettrine{G}{enescience}\mycalendar{Oct.'10}{08} Pharmaceutical, a Chinese company, and its chief
executive pleaded guilty on Wednesday to federal charges of illegally distributing human growth
hormone in the United States, capping a three-year investigation.

In United States District Court in Providence, R.I., they agreed to pay \$3 million toward a ``Clean
Competition Fund'' that would support drug-free sports, and \$7.2 million in criminal forfeitures.

The company founder, Lei Jin, entered a guilty plea through a lawyer and was sentenced to five
years' probation. The company pleaded guilty to a felony.

GeneScience was implicated in 2007 during ``Operation Raw Deal,'' a crackdown on the international
trafficking of steroids and other illicit body-building drugs.

A lawyer for the company presented a \$4.5 million check Wednesday as a forfeiture of assets in the
plea agreement. The government had previously seized \$2.7 million from New York bank accounts
linked to the company's growth hormone smuggling. The lawyer, John Tarantino, said in an e-mail that
he was not authorized to comment.

Mr.~Jin did not appear in court, according to Tom Connell, a spokesman for Peter F.~Neronha, a
United States attorney.

Human-growth hormone is banned in many sports. Mr.~Neronha said in a statement, ``H.G.H., when
distributed and used unlawfully, poses a serious health threat, particularly to young people who
ignore the risks of such substances in an effort to enhance athletic performance.''

GeneScience, which identifies itself as China's most profitable biopharmaceutical company, had
distributed a growth-hormone product called Jintropin in China and around the world through the
Internet. It continues to operate in Changchun, in northern China, Mr.~Connell said.

GeneScience was founded in 1996 by Mr.~Jin, a Chinese citizen who held a Ph.D. in pharmaceutical
chemistry from the University of California, San Francisco, and had worked as a research scientist
at Genentech, one of the world's leading producers of human growth hormones.

A government complaint says the company and Mr.~Jin used e-mail aliases, offshore bank accounts and
a network of drug traffickers to illegally distribute millions of dollars worth of human growth
hormone in the United States. The distribution had not been approved by the Food and Drug
Administration.

The clean competition fund, to be administered by the Rhode Island Community Foundation, would
support antidoping in sports, drug screening and clinical research into the long-term effects of
human growth hormone, according to a court filing.

A lawyer representing GeneScience did not return phone calls or e-mail.

\section{Kim and Son Appear at North Korean Military Exercise}

\lettrine{T}{he}\mycalendar{Oct.'10}{08} North Korean leader, Kim Jong-il, watched a live-fire
military exercise with his youngest son and heir apparent in what was believed to be the first
public appearance of Kim Jong-un since he was given the rank of four-star general last week.

The North's official Korean Central News Agency reported on Tuesday that the two Kims were joined by
other members of their family and senior officials in the new North Korean leadership. The location
and date of the drill were not specified.

But it was significant, analysts in Seoul said, that the younger Mr.~Kim's first public event was a
military exercise, as the government tries to create the image of him as a capable military man. The
event also was likely to reinforce his father's guiding ``military first'' philosophy.

Kim Jong-un, who is believed to be 27 or 28, was joined at the drill by Vice Marshal Ri Yong-ho, the
chief of the army's general staff, the news agency said. Both men were appointed last week as deputy
chairmen of the party's military commission, and South Korean analysts expect Vice Marshal Ri, 67,
to tutor the inexperienced Mr.~Kim on military and political matters as Mr.~Kim prepares to
eventually succeed his father.

The news agency report said that when the elder Mr.~Kim signaled for the live-fire drill to begin,
``various ground guns showed powerful fire, while units moved in close harmony to annihilate
enemies.''

Also watching the military exercise were two other members of Kim Jong-il's ruling inner circle --
his sister, Kim Kyong-hui, who was also made a four-star general last week, and her husband, Jang
Song-taek, long rumored to be effectively in charge of the daily running of the government.

It was unclear whether the military exercises were part of the preparations for a nationwide
celebration of the 65th anniversary of the founding of the Workers' Party. The anniversary, Oct.~10,
is a major annual holiday in North Korea and usually features food giveaways by the government and a
huge military parade in the capital, Pyongyang.

The South Korean Defense Ministry also said on Tuesday that the North seemed to be preparing a large
set of war games involving army, air force and naval units. Those exercises are expected to be held
at sea, off North Korea's eastern coast.

In a separate news agency report, North Korea assailed the ``ceaseless war exercises being staged by
the U.S.~imperialists'' against the North ``in league with the South Korean war hawks.'' The United
Sates and South Korea recently concluded five days of antisubmarine drills off the South Korean
coast. The statement said that such moves would only heighten tensions between the Koreas, and that
the United States was an ``archcriminal'' that wanted to ``spark off a war on the Korean
Peninsula.''

\section{Chinese Civilian Boats Roil Disputed Waters}

\lettrine{T}{he}\mycalendar{Oct.'10}{08} diplomatic discord set off by Japan's recent detention of a
Chinese fishing trawler captain points to what foreign military officials say is a growing source of
friction along China's borders: civilian vessels plying disputed waters -- and sometimes acting as
proxies for the Chinese Navy.

The number of Chinese civilian boats operating in disputed territory and that of the run-ins they
have with foreign vessels, including warships, are on the rise, American and Asian officials say.

The boats often have no obvious military connections, and none have been discovered for the trawler
the Japanese detained. But foreign officials and analysts say there is evidence showing that they
sometimes coordinate their activities with the Chinese Navy. China's navy is seeking to expand a
maritime militia of fishing vessels and to enhance its control over civilian agencies that regulate
activities in coastal waters.

The result is an increasingly volatile situation in waters around China, especially in the contested
East and South China Seas. Foreign military officials are now wary of a wide range of Chinese
maritime vessels. American officials and a Pentagon report from 2009 warn of potential hostilities
with Chinese civilian vessels, based in part on two separate incidents last year in which American
warships had tense encounters with Chinese boats.

The Chinese Navy is determined to create a long-range global presence by modernizing its fleet. But
the use of civilian boats is part of a different goal -- to better defend and more firmly assert
sovereignty over China's coast, its territorial waters and the exclusive economic zones that extend
200 nautical miles off the coast. Using civilians is a crucial part of the doctrine that Chinese
military officials call ``people's war.''

Dennis J.~Blasko, a former military attach\'e at the United States Embassy in Beijing, said the
Chinese military articulated this in 2006 in a white paper on national defense. ``The Navy is
enhancing research into the theory of naval operations and exploring the strategy and tactics of
maritime people's war under modern conditions,'' the paper said.

In some cases, employing civilian forces ``may be less provocative and with less potential for
escalation than employing active duty PLA forces,'' Mr.~Blasko said in an e-mail.

The Chinese Navy uses civilian vessels in several ways. One is to command militias made of fishing
vessels. Another is to coordinate operations with five maritime law enforcement groups that have
some of the same functions as the United States Coast Guard, most notably the Fisheries Law
Enforcement Command, which is charged by the Agriculture Ministry with enforcing fishing bans and
operates regularly in disputed waters. Some fisheries officials now go out on boats wearing uniforms
and carrying firearms, said Bernard D.~Cole, a former officer in the United States Navy and a
professor at the National War College.

The Chinese Navy could not be reached for comment. An official at the fisheries bureau headquarters
in Beijing said that fisheries vessels ``serve the purpose of administrative law enforcement'' and
that they did not work with the Chinese military.

As for relying on fishermen, military exercises off the coast of Fujian Province and comments by
Chinese officials show that the Chinese Navy has been trying to ``more effectively organize China's
maritime militia, based on various fishing fleets,'' Mr.~Cole said. ``The maritime militia in 2010
is quite active.''

A Pentagon report last year noted that in May 2008, two Chinese warships were supplied with
ammunition and fuel at a designated spot off Zhejiang Province by fishing vessels that belonged to
the naval militia.

The latest China-Japan dispute has cast scrutiny on Chinese fishing vessels. On Sept.~8, Japanese
authorities detained Zhan Qixiong, a fishing trawler captain, and 14 crew members after the Japanese
said that the trawler had rammed two Japanese Coast Guard vessels. The Chinese and Japanese boats
encountered each other around the islands known as the Diaoyu to the Chinese and Senkaku to the
Japanese in the East China Sea, an area rich in fish and deposits of natural gas and oil. Both
nations claim the islands as their territory, but Japan administers the area. Japanese patrol boats
usually chase away Chinese vessels.

Mr.~Zhan was released on Sept.~24, but Japanese newspapers have continued to speculate on Mr.~Zhan's
background. Some call him a Chinese naval officer.

Mr.~Zhan has declined to talk to journalists. He and his employer, Wu Tianzhu, who owns 10 fishing
vessels in Mr.~Zhan's home county in Fujian Province, do not work with the Chinese military, said
Mr.~Wu's wife, who gave her name only as Ms.~Chen because of the delicacy of discussing security
matters. Mr.~Zhan has been a fisherman all his life, she said.

The day Mr.~Zhan returned to China, he said he planned to go back to the Diaoyu Islands. About three
years ago, an official document circulated in Shenhu County, where Mr.~Zhan lives, telling fishermen
not to go to the disputed waters, said an employee at a local fishing information center who
identified himself only as Mr.~Chen. But there has been no such warning in recent years, he added.

``Gradually, more and more boats went to fish there, especially when the harvest was not good enough
in other areas,'' he said. ``More boats went there last year and this year.''

Civilian boat traffic rose as China began making bolder claims to the East and South China Seas.

It is not just China that allows or encourages its fishermen to enter disputed territorial waters.
In April 2007, four Vietnamese fishing boats were detained by China in the disputed Spratly
archipelago of the South China Sea. In July 2007, a Vietnamese fishing boat sank after being rammed
by a Chinese vessel. One Vietnamese fisherman died.

Last year, two American warships were involved in prominent incidents in which the Chinese Navy
appeared to be working closely with civilian law enforcement vessels and fishing trawlers, Pentagon
officials said.

On March 4, the Victorious, an American ship, was illuminated with a spotlight by a fisheries patrol
vessel in the Yellow Sea. The next day, 12 maritime surveillance aircraft did flybys of the
Victorious. Four days later, the Impeccable, an American ship surveying off the south coast of
China, was ``harassed'' by five Chinese vessels -- four of them civilian ships, the Pentagon said.

In these encounters, Mr.~Blasko said, ``Beijing demonstrated its will to employ military and
civilian capabilities to protect what it considers its sovereignty.''

With tensions on the rise, the fisheries bureau has been eager to publicly cast itself as a
protector of China's sovereign interests. In late September, as the China-Japan feud was unfolding,
it invited a Chinese reporter from Global Times, a populist newspaper, aboard one of its vessels.
The ship was going on a regular run to the Diaoyu Islands. The reporter, Cheng Gang, wrote of a
run-in between the Chinese civilian ship and three Japanese Coast Guard ships.

The Japanese ships asked the fisheries boat to turn around. The Chinese vessel replied via
transmitter: ``We are a Chinese fisheries administration boat. The Diaoyu Islands are China's
indigenous territory, and we are carrying out official duties in Chinese territorial waters. We ask
you to leave immediately!''

\section{Migrant `Villages' Within a City Ignite Debate}

\lettrine{T}{he}\mycalendar{Oct.'10}{08} community is walled and gated, an enclosure of rows of
crowded low-rise homes and shops, where people live under the gaze of surveillance cameras and apart
from the city. The police patrol around the clock, and security guards stop unfamiliar faces to
check identification papers. In the morning, only one gate is open, through which parents head off
to work and children go to school. At night, the gate is locked, preventing street loiterers from
trespassing.

The area, Shoubaozhuang, is not one of the affluent, gated residential compounds springing up around
Beijing, but a poor village of rural migrants toiling at low-paying jobs. It was chosen, along with
15 other areas in the Daxing district of Beijing, to be walled off to outsiders, in what officials
say is an experimental effort to curb crime. The authorities say the experiment has been a success
-- the Communist Party-run People's Daily said the crime rate in the walled villages in Daxing
district dropped by 73 percent from April to July this year -- and the ``walled village'' concept is
being quickly expanded to other districts outside Beijing's center that are populated by migrant
workers.

Ultimately, the project could encompass an area of 291 square miles with a population of 3.4 million
people, more than 80 percent of them migrant workers.

Some residents welcome the walls and gates as a way of fighting crime, but critics have seized on
the ghettolike villages as a jarring sign of the barriers facing rural migrants settling in urban
areas. They say the real intent of the new measures is to keep track of the migrants, and some have
labeled the policy a form of apartheid.

This is not the first time Beijing has experimented with walling off migrant workers, though
previous attempts have been on a much smaller scale. In the weeks leading up to the 2008 Olympics,
the city similarly blocked off workers' dorms near construction sites as part of an intensive
campaign to secure and sanitize the city. The city government labels it ``community-style
management.''

But experts remain dubious about the long-term effects of life inside the walls on residents who are
already socially marginalized migrants or local poor.

``To the migrants, the policy conveys the message that the capital is not theirs, but a capital for
citizens with Beijing hukou,'' said Peng Zhenhuai, director of the Local Government Research
Institute at Peking University, referring to China's household registration, or hukou, system.

Although increasingly relaxed in recent decades to help integrate laborers into cities, the system
still effectively limits many migrants from permanently settling their families in cities by tying
access to subsidized services like public education to one's place of birth.

A newly released report from the municipal legislature indicates that Beijing's ``floating
population'' had exceeded 10 million by the end of last year, including 7.26 million migrants. As
the city's boom creates jobs in construction and service industries, some villages like
Shoubaozhuang, with less than 1,000 original inhabitants, have seen their populations explode in the
past five years with the influx of migrants. Most live in low-rise apartments or extensions to
buildings constructed by the original inhabitants.

But the population spike has also brought the problems of the city to these areas, most notably
petty crime. According to Daxing officials, 80 percent of the crime in the past five years took
place in areas dominated by migrants. In Shoubaozhuang, locals have been vexed by everything from
street fights to stolen laundry.

The Communist Party secretary for Beijing, Liu Qi, endorsed the new policy in early July, calling it
a ``positive and effective exploration in the course of Beijing's urban-rural integration'' and
proposed extending it citywide. In the Daxing district, 76 villages are set to follow the example of
the first 16 by the end of the year. The Changping district, on the northern side of the city,
announced plans in July to enclose 100 villages, 44 to be finished within the coming months.

Many experts say that instead of building new barriers, the government should concentrate on forging
ahead much faster with tax reforms, expanding low-income housing and transferring factories farther
inland, to ease the settlement of migrants into cities, satellite communities and towns closer to
their home provinces.

``The integration should come as a result of human interaction, and the government should orient its
policy to the needs of its population,'' said Lu Jiehua, a professor of sociology at Peking
University. He said the government could adopt a policy of allowing migrants in large cities access
to social services based on the number of years they have lived there.

Beijing police officials and the village committee representatives equate the enclosed villages to
the gated communities in more affluent neighborhoods.

``It offers them the same service as the high-class apartments in the city,'' Li Baoquan, a member
of a village committee in charge of public security in part of the Changping district, said in a
telephone interview. ``Of course the villagers would support it.''

And despite the heated social debate and extensive local news media coverage, villagers, burdened
with the more pressing issues of finding decent-paying jobs, affordable medical care and schooling
for their children, seem far less concerned with the walls in their midst.

A fruit vendor in Shoubaozhuang who requested anonymity for fear of reprisals, said he had had a few
less customers since the enclosure policy was instituted, but also realized the benefits of having a
security guard living next door. ``He is doing his duty,'' he said. ``I am selling my fruit. I try
not to cross paths with him.''

Gong Daocui, who arrived in Beijing from Chongqing in 2006 to live with her son, said: ``We are all
workers. What say do we have in the policies they make?''

She sells skewers of vegetables and meat cooked in a spicy soup along the main alley of Banjieta
Village, which will soon be enclosed. ``If the business can keep going, so be it,'' she said,
smiling and looking away. ``If it can't, I will pack up everything and retire to my hometown.''

\section{A Photo Exhibit, From the Dear Leader}

\lettrine{W}{ho}\mycalendar{Oct.'10}{08} knew ``Mad Men'' had hit Pyongyang?

Or so it seemed last week, when the North Korean government released photographs of a rare meeting
of the Korean Workers' Party, and delegates were seen walking purposefully through Pyongyang Station
on their way to formally designate the next ruler of the Kim dynasty. That would be Kim Jong-un, son
of the ailing Dear Leader, Kim Jong-il, and grandson of the Great Leader, Kim Il-sung.

The delegates wore suits of nearly identical, somewhat boxy cut, and carried 1960s-style briefcases
that the admen of the AMC television series might have swung on Madison Avenue. It was a flashback
for anyone lucky enough to have ever strolled the empty streets of Pyongyang, a capital in a time
machine, with its Stalin-era ceremonial buildings and limousines that could have fit the Kennedy
era.

Like the Mad Men, the attendees at the Korean Workers' Party session were trying to create brand
awareness -- and they don't have much time to do so.

Pyongyang had decades to build up mythologies about how Grandpa Kim singlehandedly ousted the
Japanese, and how Papa Kim offered benevolent ``guidance'' in building each factory. But the last
transition unfolded over 20 years. With Kim Jong-il clearly sick, this transition has to be fast,
which is why Kim Jong-un went from civilian to four-star general in an announcement on the state
radio.

It is also why the photographs last week were read so carefully here in Washington. There were hints
of what North Korea's leaders wish the world to see, and what they hope nobody will notice. ``These
are pictures about one united party, one enduring bloodline,'' said Nicholas Eberstadt of the
American Enterprise Institute, ``and an obstinate refusal to change direction.''

\section{Medical Student Distress and the Risk of Doctor Suicide}

\lettrine{S}{everal}\mycalendar{Oct.'10}{08} years ago, I learned that a physician in a town not too
far from where I was practicing had committed suicide. Neither I nor my hospital colleagues knew
him, but according to the story we heard, he was the father of young children, was respected by
doctors and patients alike and had struggled privately with mental illness since medical school.

But it was not the details of his life that haunted us; it was the details of his death. He had
locked himself in a room in the hospital, placed a large needle in his vein and injected himself
with a drug that so effectively paralyzed his muscles he was unable to breathe.

Or call for help.

For days afterward, the doctor's death came up repeatedly in conversations. We talked about the
grief his family must have been experiencing and speculated on the extent of depression and
self-loathing he must have experienced, but we dared not speak of, let alone imagine, the agony of
his final moments.

Always, we ended up asking one another the same question: How could a doctor -- who most likely knew
about what he was suffering from and about the treatments available -- never seek help?

For several decades now, studies have consistently shown that physicians have higher rates of
suicide than the general population -- 40 percent higher for male doctors and a staggering 130
percent higher for female doctors. While research has traced the beginning of this tragic difference
to the years spent in medical school, the contributing factors remain murky. Students enter medical
school with mental health profiles similar to those of their peers but end up experiencing
depression, burnout and other mental illnesses at higher rates. Despite better access to health
care, they are more likely to cope by resorting to dysfunctional behaviors like excessive drinking
and are less likely to receive the right care or even recognize that they need some kind of
intervention.

Researchers have offered several theories to explain these seemingly paradoxical findings. Some have
faulted the increasing social isolation of medical education, training and practice. Others have
pointed to the tendency for doctors to be highly critical of themselves and to blame themselves for
their own illnesses. Still others, in light of the particularly high rates of suicide among female
doctors, have suggested that workplace harassment may have a role.

Despite the many studies, theories and, more recently, student wellness programs and confidential
mental health services offered by more and more medical schools, the grim statistics for medical
students have hardly budged over the last generation. Up to a quarter of young doctors-to-be suffer
from depression, more than half may be experiencing burnout, and a just more than 10 percent may be
harboring thoughts of suicide.

These sobering numbers have remained unchanged in large part because our understanding of this issue
has been hampered by inadequate research methodologies and insufficient financial support. We
haven't had the sophisticated tools needed to analyze the causes or appropriate interventions; and
even if we did, we haven't had the money to do anything with them.

Now two groups of researchers, using innovative methods and financed by medical school programs and
departments with a keen interest in physician well-being, have published separate studies in The
Journal of the American Medical Association that go beyond incidence statistics and theoretical
considerations. Each study offers new findings about medical student distress and how the learning
environment both fosters and exacerbates it. Read together, they offer disquieting views of the
world in which tomorrow's doctors are formed.

``There's no arguing anymore over whether there's a high prevalence of distress,'' said
Dr.~Liselotte N.~Dyrbye, lead author of one of the studies and an associate professor of medicine at
the Mayo Clinic in Rochester, Minn. ``What's important now is that we hold a mirror up to ourselves
and ask why this is happening, because it is clearly not what we medical educators have intended.''

Previous studies have linked medical student distress to unprofessional behavior. But, as Dr.~Dyrbye
and her colleagues show in their research, different types of distress -- professional versus
personal -- can have very different effects on a young doctor's sense of what is right and wrong.

Surveying more than 2,500 medical students across the country, the researchers found that students
who suffered from professional distress, more commonly referred to as burnout, a constellation of
emotional exhaustion, detachment and a low sense of accomplishment, were more likely to admit to
cheating on tests, lying about the status of a patient's laboratory tests or physical exam and
espousing less altruistic views regarding their role as physicians. Conversely, students who
suffered from personal distress, defined as poor mental or physical quality of life or depression,
were not more susceptible to these unprofessional behaviors and self-centered beliefs.

``There certainly is some overlap,'' Dr.~Dyrbye said. ``But depression and burnout are two separate
entities.''

One result of erroneously conflating the two types of distress is stigmatization of mental illness.
According to the second study, conducted by researchers from the University of Michigan in Ann
Arbor, medical students who are depressed or prone to depression often believe they are viewed as
inadequate and incompetent by those around them.

``They feel this from every direction -- from other medical students, faculty members, counselors,
and even in their applications for residency training,'' said the study's lead author, Dr.~Thomas
L.~Schwenk, a professor of family medicine at the University of Michigan. While depression can cause
individuals to have negative and distorted views of their surroundings, ``the culture of medical
school makes these students also feel like they can't be vulnerable or less than perfect.''

Given that students must compete with one another throughout medical school for postgraduate
training positions, many have a difficult time admitting to any perceived weakness. For those who do
and want help, there are more obstacles: with the sense that peers, faculty members and others are
likely to judge distressed students as less competent, it is nearly impossible to find somewhere
truly safe to turn.

But this ``survival of the fittest'' mentality can affect all medical students, not just those who
are depressed or burned out. And it can affect patients by wearing away at a young doctor's sense of
empathy.

``If this is the way that students view each other,'' Dr.~Schwenk said, ``how do they view their
patients who are depressed or struggling with mental illness?''

More long-term studies are needed to test interventions and analyze the factors contributing to
student distress. ``We have to assume that starting in medical school, there's a pipeline of
experiences that leads to an increased risk of suicide,'' Dr.~Schwenk said. But without more
evidence-based interventions, even the best intentions of medical educators will continue to do
little to stem the tide of medical student distress and physician suicides.

That failure has already and will continue to come at a tremendous cost to doctors and patients. ``I
still believe that the people who are the most vulnerable are often the most empathic,'' Dr.~Dyrbye
said. ``They are the ones who get most attached and put the needs of the patient first.''

Dr.~Dyrbye continued, ``Until we know what really helps them and what works best, our learning
environment will continue to eat away at our students' empathy and altruism.''

\section{Stem Cells in Court, Scientists Fear for Careers}

\lettrine{R}{ushing}\mycalendar{Oct.'10}{08} to work at Cincinnati Children's Hospital Medical
Center one recent morning, Jason Spence, 33, grabbed a moment during breakfast to type ``stem
cells'' into Google and click for the last 24 hours of news. It is a routine he has performed daily
in the six weeks since a Federal District Court ruling put the future of his research in jeopardy.

``It's always at the front of my brain when I wake up,'' said Dr.~Spence, who has spent four years
training to turn stem cells derived from human embryos into pancreatic tissue in the hope of helping
diabetes patients. ``You have this career plan to do all of this research, and the thought that they
could just shut it off is pretty nerve-racking.''

Perhaps more than any other field of science, the study of embryonic stem cells has been subject to
ethical objections and shaped by political opinion. But only a year after the Obama administration
lifted some of the limits imposed by President George W.~Bush, a lawsuit challenging the use of
public money for the research and a conservative shift in Congress could leave the field more
sharply restricted than it has been since its inception a decade ago. At stake are about 1,300 jobs,
as well as grants from the National Institutes of Health that this year total more than \$200
million and support more than 200 projects.

The turn of events has introduced what researchers say is unprecedented uncertainty to a realm of
academic science normally governed by the laws of nature and the rules of peer review.

``We're used to people telling us, `That was a stupid idea, we're not going to fund it,' and we turn
around and think of a better one,'' said James Wells, who heads the laboratory where Dr.~Spence has
a postdoctoral position. ``But there's nothing we can do about this.''

The stem cells, which are thought to have curative potential for many diseases because they can be
turned into any kind of tissue in the human body, can be obtained only by destroying a human embryo,
which many Americans believe is the equivalent of a life.

In August, Chief Judge Royce C.~Lamberth of Federal District Court for the District of Columbia
found that the Obama administration's policy violates a law barring federal financing for ``research
in which a human embryo or embryos are destroyed, discarded or knowingly subjected to risk of injury
or death,'' and issued an injunction blocking federal money for the research.

Since then, the field's fate has appeared to shift almost weekly as the lawsuit wends its way
through the courts. Last week, the government won the right from an appeals court to continue
financing the contested research while it appeals the ruling. But there is no telling how the
appeals court will ultimately rule, and Judge Lamberth could issue a revised injunction.

Many of the nation's leading stem cell researchers do not know whether they will receive grants they
won years earlier through the standard competition, or whether new projects will even be considered.
Junior scientists like Dr.~Spence, poised to start their own laboratories, are caught in limbo.
Senior scientists like Dr.~Wells are torn between pursuing research they believe in and protecting
students from staking their job prospects on projects they may never be able to complete.

The legal roller coaster is raising stress levels and reducing productivity, researchers say.
Instead of tending to their test tubes, they find themselves guessing how each member of the Supreme
Court might vote on the case. They are also watching the midterm Congressional elections with new
interest -- and with some dismay, since many believe that new legislation will be required for their
work to continue.

Under guidelines authorized by both the Bush and Obama administrations, work that leads directly to
destroying the embryos cannot be federally financed. The government can, however, support subsequent
research on the cell lines created by that process.

Last year, two scientists filed the lawsuit, arguing that the distinction is a false one and that
the guidelines on public financing violated the Dickey-Wicker amendment, first passed in 1996 and
renewed by Congress every year since.

Moreover, they said, it siphons limited government resources from research on different types of
stem cells, which they and other scientists who share a discomfort with embryonic stem cells view as
ethically and scientifically superior. For all the hope vested in them, human embryonic stem cells
have yet to yield tangible results for patients.

In his ruling, Judge Lamberth agreed that the guidelines violated the 1996 amendment and ``threaten
the very livelihood'' of the plaintiffs.

Embryonic stem cell researchers who stand to lose their federal grants as a result argue that other
types of stem cells do not have the same properties, and that all need to be studied regardless to
determine which work best. They bristle at the intrusion of judges and politicians into decisions
usually addressed by the peer review process, in which experts in a field comment on the merit of an
idea and the best get financed.

Yet even some who believe there is a compelling scientific rationale for their research agree that
the legal basis for federal financing may be weak. ``I was astonished that Congress hadn't dealt
with this,'' said Stephen Duncan, a stem cell researcher at the Medical College of Wisconsin, who
stands to lose several million dollars in federal grants depending on the dispensation of the case.
``It's like being a little pregnant. You're either breaking the law or you're not.''

Mr.~Bush, who in 2001 limited federally financed researchers to working on roughly two dozen stem
cell lines already in existence, twice vetoed legislation that would have explicitly expressed
support for financing the contested research. No such legislation has been introduced under
President Obama, but the administration expanded the number of stem cell lines researchers could
study.

Advocates of the research now see this as a missed opportunity.

Efforts to rally Congressional support since Judge Lamberth's ruling have failed to gain momentum
among Democrats and moderate Republicans heading into the November elections.

For many, the most recent intrusion of politics into the vaunted scientific meritocracy came as a
particular shock because the Obama administration's new guidelines had only months earlier fallen
into place.

``The painful thing is that we are being stopped at a time when the velocity of this field of
research, thanks to the new administration, was finally going at maximum speed,'' said Ali
H.~Brivanlou, a professor at Rockefeller University.

Over the last few weeks, embryonic stem cell scientists have sought alternative financing from
private foundations, university administrations and state programs. But the National Institutes of
Health, which has a \$26 billion budget, is by far the source with the deepest pockets for academic
scientists.

Some researchers are weighing a switch to the private sector. Others have ordered their students to
pay no attention to the news. Others are trying to raise public awareness.

Yi Sun, 45, of the University of California, Los Angeles, has resorted to frequent meditation.

``I would be in trouble without it,'' said Dr.~Sun, whose stem cell work focuses on an autism
disorder called Rett syndrome. Born in China, Dr.~Sun said she was now renewing efforts to
collaborate with well-financed stem cell biologists there.

\section{Health Care's Uneven Road to a New Era}

\lettrine{C}{onsider}\mycalendar{Oct.'10}{08} what it would be like to have a health insurance plan
that capped annual benefits at \$2,000. For any medical care costing more than that, you would have
to pay out of pocket.

Examples of care that costs more than \$2,000 -- and often a lot more -- include virtually any
cancer treatment, any heart surgery, a year's worth of diabetes treatment and care for many broken
bones. Even a single M.R.I. exam can cost more than \$2,000. A typical hospital stay runs thousands
of dollars more.

So does this insurance plan sound like part of the solution for the country's health care system --
or part of the problem?

A \$2,000 plan happens to be one of the main plans that McDonald's offers its employees. It became
big news last week, when The Wall Street Journal reported that the company was worried the plan
would run afoul of a provision in the new health care law. In response to the provision, McDonald's
threatened to drop the coverage altogether, until the Obama administration signaled it would grant
some exemptions.

This episode was only the latest disruption that the health law seems to be causing. Also last week,
the Principal Financial Group said it was getting out of the health insurance business, while other
insurers have said they might stop offering certain types of coverage. With each new disruption come
loud claims -- some from insurance executives -- that the health overhaul is damaging American
health care.

On the surface, these claims can sound credible. But when you dig a little deeper, you often
discover the same lesson that the McDonald's case provides: the real problem was the status quo.

American families spend almost twice as much on health care -- through premiums, paycheck deductions
and out-of-pocket expenses -- as families in any other country. In exchange, we receive top-notch
specialty care in many areas. Yet on the whole, we do not get much better care than countries that
spend far less.

We don't live as long as people in Canada, Japan, most of Western Europe or even relatively poor
Jordan. Misdiagnosis is common. Medical errors occur more often than in some other countries. Unique
to the developed world, millions of people have no health insurance, and millions more, like many
fast-food workers, are underinsured.

In choosing their health reform plan, President Obama and the Democrats eschewed radical changes,
for better or worse, and instead tried to minimize the disruptions to the current system. Sometimes,
Mr.~Obama went so far as to suggest there would be no disruptions, saying that people could keep
their current plan if they liked it. But that's not quite right. It is not possible to change a
system as huge, and as hugely flawed, as ours without some disruptions.

$\bullet$

McDonald's offers its hourly workers two different health care plans, which are known as
``mini-med'' plans. In one, workers can pay about \$730 a year for benefits of up to \$2,000. In the
other, they can pay about \$1,660 a year for benefits of up to \$10,000, The Journal reported.

In a memo to federal regulators, McDonald's executives argued that their version of health insurance
``positively impacts'' the almost 30,000 workers who are covered. And that's true. A plan with a
\$2,000 or \$10,000 cap can cover some modest health problems and is better than being uninsured.

But should the litmus test for American health care really be better than nothing?

Mini-med plans force people to drain their savings accounts for dozens of common medical problems.
They also force hospitals to let some bills go unpaid, which drives up costs for everyone else.

Senator Charles Grassley, Republican of Iowa, has previously criticized AARP for marketing similarly
limited plans to its members. ``It's not better than nothing,'' Mr.~Grassley argued, ``to encourage
people to buy something described as `health security' when there's no basic protection against high
medical costs.''

Dr.~Aaron Carroll, an Indiana University pediatrics professor who studies health policy, says of
mini-med plans: ``They're great if you're healthy, because you feel like you're covered. But if you
ever need them, they're so skimpy, they provide very little.'' Gary Claxton of the Kaiser Family
Foundation adds, ``They really just shouldn't be considered health insurance.''

The plans' skimpiness is the main reason they ran into legal jeopardy. Under the new law, most plans
must spend at least 85 percent of their revenue on medical care, rather than administrative
overhead. The McDonald's plans aren't generous enough to clear the hurdle.

At the same time, it's probably unrealistic to expect McDonald's to give workers decent health
insurance. Many of those workers make less than \$20,000 a year. A typical family insurance plan
would raise their total compensation by more than half, destroying the McDonald's business model.

The workers, for their part, cannot afford to buy insurance in the so-called individual market.
Plans are even more expensive in that market, because it is dominated by people who desperately need
insurance -- which is to say, sick people.

This is where health reform comes in. It tried to solve the problem by creating what policy experts
call a three-legged stool.

First, people will be required to buy insurance, to spread costs among the sick and the healthy.
Second, insurers will be prohibited from cherry-picking only the healthiest customers, again to
spread costs. Finally, the government will give subsidies to people, like McDonald's workers, who
can't afford insurance on their own.

Germany, the Netherlands and Switzerland all use a system along these lines to cover everyone,
largely through the private sector, for less money per person than this country spends.

The recent disruptions in our health insurance market are partly a result of the fact that the
stool's three legs were not built on the same timetable. Some of the insurance regulations, like the
one on overhead costs, are starting to take effect. But the new markets for health insurance, known
as exchanges, won't be up and running until 2014. This timetable has its problems, and the Obama
administration will probably need to grant some more temporary exemptions.

In 2014, however, the choice for McDonald's workers will no longer be between a bad policy and no
policy. Through the exchanges, they will be able to buy a real health insurance plan -- one that
covers cancer, heart attacks, surgeries, M.R.I.'s and hospital stays. Dr.~Carroll notes that many
families will end up paying less than they are now paying out of pocket and will get more access to
care, too.

For insurance companies, these changes won't be quite so positive. They will no longer be able to
sell plans that devote 30 percent of revenue to salaries for their workers. They will not be allowed
to compete over which company can come up with the most ingenious ways to say no to the sick. Their
benefits and prices will become more public, thanks to the exchanges.

The health care overhaul that passed Congress is far from ideal, as I have written many times in
this space. But it does represent progress.

The fact that it is beginning to disrupt the status quo -- that some insurance policies will
eventually be eliminated and some inefficient insurers will have to leave the market altogether --
is all the proof we need.

\section{Side Effects May Include Lawsuits}

\lettrine{F}{or}\mycalendar{Oct.'10}{08} decades, antipsychotic drugs were a niche product. Today,
they're the top-selling class of pharmaceuticals in America, generating annual revenue of about
\$14.6 billion and surpassing sales of even blockbusters like heart-protective statins.

While the effectiveness of antipsychotic drugs in some patients remains a matter of great debate,
how these drugs became so ubiquitous and profitable is not. Big Pharma got behind them in the 1990s,
when they were still seen as treatments for the most serious mental illnesses, like hallucinatory
schizophrenia, and recast them for much broader uses, according to previously confidential industry
documents that have been produced in a variety of court cases.

Anointed with names like Abilify and Geodon, the drugs were given to a broad swath of patients, from
preschoolers to octogenarians. Today, more than a half-million youths take antipsychotic drugs, and
fully one-quarter of nursing-home residents have used them. Yet recent government warnings say the
drugs may be fatal to some older patients and have unknown effects on children.

The new generation of antipsychotics has also become the single biggest target of the False Claims
Act, a federal law once largely aimed at fraud among military contractors. Every major company
selling the drugs -- Bristol-Myers Squibb, Eli Lilly, Pfizer, AstraZeneca and Johnson \& Johnson --
has either settled recent government cases for hundreds of millions of dollars or is currently under
investigation for possible health care fraud.

Two of the settlements, involving charges of illegal marketing, set records last year for the
largest criminal fines ever imposed on corporations. One involved Eli Lilly's antipsychotic,
Zyprexa; the other involved a guilty plea for Pfizer's marketing of a pain pill, Bextra. In the
Bextra case, the government also charged Pfizer with illegally marketing another antipsychotic,
Geodon; Pfizer settled that part of the claim for \$301 million, without admitting any wrongdoing.

The companies all say their antipsychotics are safe and effective in treating the conditions for
which the Food and Drug Administration has approved them -- mostly, schizophrenia and bipolar mania
-- and say they adhere to tight ethical guidelines in sales practices. The drug makers also say that
there is a large population of patients who still haven't taken the drugs but could benefit from
them.

AstraZeneca, which markets Seroquel, the top-selling antipsychotic since 2005, says it developed
such drugs because they have fewer side effects than older versions.

``It's a drug that's been studied in multiple clinical trials in various indications,'' says
Dr.~Howard Hutchinson, AstraZeneca's chief medical officer. ``Getting these patients to be
functioning members of society has a tremendous benefit in terms of their overall well-being and how
they look at themselves, and to get that benefit, the patients are willing to accept some level of
side effects.''

The industry continues to market antipsychotics aggressively, leading analysts to question how drugs
approved by the Food and Drug Administration for about 1 percent of the population have become the
pharmaceutical industry's biggest sellers -- despite recent crackdowns.

Some say the answer to that question isn't complicated.

``It's the money,'' says Dr.~Jerome L.~Avorn, a Harvard medical professor and researcher. ``When
you're selling \$1 billion a year or more of a drug, it's very tempting for a company to just ignore
the traffic ticket and keep speeding.''

NEUROLEPTIC drugs -- now known as antipsychotics -- were first developed in the 1950s for use in
anesthesia and then as powerful sedatives for patients with schizophrenia and other severe psychotic
disorders, who previously might have received surgical lobotomies.

But patients often stopped taking those drugs, like Thorazine and Haldol, because they could cause a
range of involuntary body movements, tics and restlessness.

A second generation of drugs, called atypical antipsychotics, was introduced in the '90s and sold to
doctors more broadly, on the basis that they were safer than the old ones -- an assertion that
regulators and researchers are continuing to review because the newer drugs appear to cause a range
of other side effects, even if they cause fewer tics.

Contentions that the new drugs are superior have been ``greatly exaggerated,'' says Dr.~Jeffrey
A.~Lieberman, chairman of the psychiatry department at Columbia University. Such assertions, he
says, ``may have been encouraged by an overly expectant community of clinicians and patients eager
to believe in the power of new medications.''

``At the same time,'' he adds, ``the aggressive marketing of these drugs may have contributed to
this enhanced perception of their effectiveness in the absence of empirical evidence.''

Others agree. ``They sold the story they're more safe, when they aren't,'' says Robert Whitaker, a
journalist who has written two books about psychiatric medicines. ``They had to cover up the
problems. Right from the start, we got this false story.''

The drug companies say all the possible side effects are fully disclosed to the F.D.A., doctors and
patients. Side effects like drowsiness, nausea, weight gain, involuntary body movements and links to
diabetes are listed on the label. The companies say they have a generally safe record in treating a
difficult disease and are fighting lawsuits in which some patients claim harm.

The cases, both civil and criminal, against many of the world's largest drug makers have unveiled
hundreds of previously confidential documents showing that some company officials were aware they
were using questionable tactics when they marketed these powerful, expensive drugs.

Such marketing, according to analysts and court documents, included payments, gifts, meals and trips
for doctors, biased studies, ghostwritten medical journal articles, promotional conference
appearances, and payments for postgraduate medical education that encourages a pro-drug outlook
among doctors. All of these are tools that federal investigators say companies have used to
exaggerate benefits, play down risks and promote off-label uses, meaning those the F.D.A. hasn't
approved.

Lawyers suing AstraZeneca say documents they have unearthed show that the company tried to hide the
risks of diabetes and weight gain associated with the new drugs. Positive studies were hyped, the
documents show; negative ones were filed away.

According to company e-mails unsealed in civil lawsuits, AstraZeneca ``buried'' -- a manager's term
-- a 1997 study showing that users of Seroquel, then a new antipsychotic, gained 11 pounds a year,
while the company publicized a study that asserted they lost weight. Company e-mail messages also
refer to doing a ``great smoke-and-mirrors job'' on an unfavorable study.

``The larger issue is how do we face the outside world when they begin to criticize us for
suppressing data,'' John Tumas, then AstraZeneca's publications manager, wrote in a 1999 e-mail.
``We must find a way to diminish the negative findings,'' he added. ``But, in my opinion, we cannot
hide them.''

Tony Jewell, an AstraZeneca spokesman, said last week that the company had turned over all that
material to the F.D.A. as part of the approval process and updated its label over the years to show
the latest safety information.

Dr.~Stefan P.~Kruszewski, a Harvard-educated psychiatrist who once worked as a paid speaker for
several drug makers, became a government informant and now consults for plaintiffs suing drug
companies. Earlier in his career, he spoke at events for Pfizer, GlaxoSmithKline and Johnson \&
Johnson as an advocate of antipsychotics. He said one company offered him incentives of \$1,000 or
more every time he talked to an individual doctor about one of its drugs.

``When I started speaking for companies in the late 1980s and early '90s, I was allowed to say what
I thought I should say consistent with the science,'' he recalls. ``Then it got to the point where I
was no longer allowed to do that. I was given slides and told, `We'll give you a thousand dollars if
you say this for a half-hour.' And I said: `I can't say that. It isn't true.' ''

Slides for one new antipsychotic drug contended that it had no neurological side effects. ``They
made it all up,'' Dr.~Kruszewski said. ``It was never true.''

The antipsychotics found an easy route around regulations because of the leeway given to many big
drug makers.

While drug companies are prohibited from promoting drugs for conditions for which they have not been
proved safe and effective, their paid consultants, researchers and educators may do that for them
verbally and in company-sponsored studies.

``They can give a small hint, and people will take the bait,'' says Dr.~Robert Rosenheck, a
professor of psychiatry and public health at the Yale School of Medicine, who has received research
support from drug makers and federal agencies. ``Psychiatric disorders are vaguely defined enough
that you can stretch definitions,'' he says. ``So many treatments are completely ineffective, people
are willing to try anything.''

For their part, doctors are free to prescribe any approved drug for any medical condition they
choose, even if the drug hasn't been approved for that specific treatment. ``Because they're
approved, they become an alternative for doctors who can't think of what else to prescribe,'' says
Dr.~Daniel J.~Carlat, an associate professor of psychiatry at Tufts University. ``Whether they're
useful or not is unclear.''

Analysts said that given the profits that were to be made, the murkiness of mental disorders, and
holes in the regulatory regime, marketing excesses were bound to occur.

``If you have a lot of money on the table and you have clinical uncertainty over mental health
conditions, where you don't have a blood test or objective test for it, you see it's kind of a
combustible mixture,'' says Dr.~Mark Olfson, a Columbia University psychiatry professor and
researcher.

DOCUMENTS produced in recent litigation and in Congressional investigations show that some leading
academic doctors have worked closely with corporate benefactors to expand the use of antipsychotics.

The most well-known is Joseph Biederman, a Harvard medical professor and Massachusetts General
Hospital researcher. His studies, examining prevalence of bipolar psychological disorders in
children, helped expand practice standards, leading to a fortyfold increase in such diagnoses from
1994 to 2003. The increase was reported in a 2007 study by the Archives of General Psychiatry.

Between 2000 and 2007, he also got \$1.6 million in speaking and consulting fees -- some of them
undisclosed to Harvard -- from companies including makers of antipsychotic drugs prescribed for some
children who might have bipolar disorder, a Senate investigation found in 2008.

Johnson \& Johnson gave more than \$700,000 to a research center that was headed by Dr.~Biederman
from 2002 to 2005, records show, and some of its work supported the company's antipsychotic drug,
Risperdal.

Dr.~Biederman says that the money did not influence him and that some of his work supported other
drugs.

``Dr.~Biederman's research does not promote a particular diagnosis or treatment,'' his lawyer, Peter
Spivack, wrote in an e-mail on Thursday.

The increase in pediatric bipolar diagnosis, the lawyer said, ``cannot be attributed solely to
Dr.~Biederman's work.'' Treatment was expanded to help children and their families, he said.

Mr.~Spivack said Dr.~Biederman's disclosure lapses were minor and inadvertent. A Harvard spokesman
said they were still under review.

According to government investigators and plaintiffs' lawyers, many of the studies of antipsychotics
were conceived in marketing departments of pharmaceutical companies, written by ghostwriters and
then signed by prominent physicians -- giving the illusion that the doctors were undertaking their
studies independently.

Such practices continue.

``The content is preplanned,'' said one doctor who has worked as an uncredited medical writer for
antipsychotic studies. Data is used selectively and interpreted for company benefit, said the
doctor, who still works in medical writing and spoke on the condition of anonymity to preserve
future job prospects.

``Review articles and original research articles have advertising messages in them,'' the doctor
said. ``That's part of the plan.''

Such papers influence medicine in many ways, as sales representatives show them to doctors and
future research builds upon them.

ACCORDING to the Justice Department, drug companies trained sales reps to rebut valid medical
concerns about unproved uses of antipsychotics. For example, the department says, Lilly produced a
video called ``The Myth of Diabetes'' to sell Zyprexa, which became its all-time best-selling drug,
even though evidence showed that Zyprexa could cause diabetes, as well as other metabolic problems.

Lilly salespeople also promoted a ``5 at 5'' drug regimen in nursing homes -- 5 milligrams of
Zyprexa at 5 p.m. to settle down agitated older patients for the night. A Lilly spokesman declined
to say when those sales campaigns occurred. But in 2005, after a new analysis of 15 previous
studies, the F.D.A. issued a public health advisory saying the use of antipsychotics to calm older
dementia patients would increase risk of death from heart failure or pneumonia. The F.D.A. asked
drug makers to add a special warning about that on packaging.

Over the years, as psychiatrists learned more about the drugs' risks, companies promoted them more
to family doctors, pediatricians and geriatricians. Pfizer paid more than 250 child psychiatrists to
promote its antipsychotic, Geodon, at a time when it was approved only for adults, according to a
government filing with the Pfizer settlement last year.

High-prescribing doctors pocketed extra money in the form of research payments, speaking fees,
gifts, meals and junkets -- some of which the government has specifically termed illegal
``kickbacks.''

In its suit against AstraZeneca, the government produced documents showing that the company paid a
Chicago psychiatrist, Dr.~Michael Reinstein, nearly \$500,000 over a decade to do research, travel
and speak for it -- even as he led a Medicaid practice he had described to the company as one of
``the largest prescribers of Seroquel in the world.''

Dr.~Reinstein and AstraZeneca have both denied any misconduct.

In April, AstraZeneca became the fourth major drug company in three years to settle a government
investigation with a hefty payment -- in its case, \$520 million for what federal officials
described as an array of illegal promotions of antipsychotics for children, the elderly, veterans
and prisoners. Still, the payment amounted to just 2.4 percent of the \$21.6 billion AstraZeneca
made on Seroquel sales from 1997 to 2009.

LAST year, Eli Lilly and Pfizer settled investigations resulting in the largest criminal fines in
United States history. Lilly paid a \$515 million criminal fine as part of a broader, \$1.4 billion
settlement with the government. Pfizer later paid a \$1.3 billion criminal fine as part of a
broader, \$2.3 billion settlement.

The Lilly case focused entirely on its antipsychotic drug Zyprexa, while Pfizer's settlement
included \$301 million related to its antipsychotic, Geodon, along with marketing of other drugs.

In 2007, Bristol-Myers Squibb paid \$515 million to settle federal and state investigations into
marketing of its antipsychotic drug Abilify to child psychiatrists and nursing homes. Bristol-Myers
Squibb, like AstraZeneca, denied any misconduct.

Johnson \& Johnson is currently under investigation by the Justice Department, which says it paid
kickbacks to induce Omnicare, the nation's largest nursing home pharmacy, to recommend Risperdal,
government filings show. Omnicare paid \$98 million last November to settle civil charges.

J.\& J.~is fighting a government lawsuit and says in court filings that it was paying rebates -- an
argument endorsed in a filing by the industry trade group, the Pharmaceutical Research and
Manufacturers of America.

Some officials at companies say they've made systemic changes to avoid illegal marketing of
antipsychotics and other products.

``That was a blemish for us,'' John C.~Lechleiter, Eli Lilly's chief executive, said in an
interview. ``We don't ever want that to happen again. We put measures in place to assure that not
only do we have the right intentions in integrity and compliance, but we have systems in place to
support that.''

Jeffrey B.~Kindler, Pfizer's chief executive, voiced similar thoughts in an interview. ``Never
again,'' he said. ``I take this very seriously.''

Mr.~Kindler is operating under Pfizer's third corporate accountability agreement, a five-year
promise to the federal government to reform sales behavior, monitor employees and disclose
misconduct. The first was signed in 2002 for withholding rebates for Lipitor. The second, in 2004,
was for illegal marketing of the seizure drug Neurontin. The third, last year, was for illegal
marketing of the painkiller Bextra.

Pfizer officials say they inherited the first two situations with their acquisitions of two other
companies, Warner-Lambert and Parke-Davis.

``It wasn't our people,'' says Douglas Lankler, a senior vice president and chief compliance officer
at Pfizer.

Lew Morris, chief counsel for the inspector general of the Department of Health and Human Services,
says he is serious about bolstering government efforts to reform or punish drug makers for illegal
sales of antipsychotics.

``The message we want to send to the industry is it's not just the same-old, same-old,'' he said in
an interview.

He agrees that few industry employees have gone to jail for white-collar crimes, but says this may
change soon. ``We're targeting managers and executives who should have known,'' he said.

Mr.~Morris says some companies are ``too big to debar'' from government contracts, since doing so
would just hurt patients needing medicine. But he says discussions are under way about forcing one
health care company to sell off a subsidiary accused of fraud. And directors who ignore information
may face more risk of shareholder suits, he says.

Over the next year, the government is adding at least 15 prosecutors and 100 investigators to pursue
health care fraud.

The Pharmaceutical Research and Manufacturers of America, also strengthened its marketing code of
conduct two years ago, banning gifts and meals, although salespeople can still bring meals to
doctors' offices.

Some companies are also disclosing their consulting and speaking payments, as required by the
government agreements. And groups in charge of medical writing and postgraduate education have taken
steps to disclose or reduce industry influence.

But more than 1,000 False Claims Act lawsuits are still under way, most of them focused on health
care and many on lucrative antipsychotic drugs. For that reason alone, critics say they think the
industry still hasn't gone far enough to change questionable practices.

``The drug industry still rewards sales,'' says Stephen A.~Sheller, a lawyer who has represented
whistle-blowers in the Lilly and AstraZeneca cases. ``And it's still easy to market these drugs to
doctors who are rushed.''

A Guided Tour of Modern Medicine's Underbelly By ABIGAIL ZUGER, M.D. ``Adventures on the dark side
of medicine'' -- now that sounds like a lot of fun. A few juicy stories about black-market organs,
fingerprint erasure, murder and mayhem in the I.C.U. would make a welcome change from the usual
humdrum stuff of hospital and clinic, where the big events are a drug that doesn't work properly, or
a visit from a pharmaceutical salesman that screws up the entire afternoon schedule.

But no: In Dr.~Carl Elliott's survey of all that is shifty in modern medicine, those humdrum events
are exactly what make up medicine's dark side. And, indeed, Dr.~Elliott's entertaining and extremely
readable essays will have you convinced that in comparison to the shenanigans that go into the
creation of a single prescription pill, fingerprint erasure might actually be a little dull. After
all, what is more sinister than the dubious mechanics of the ordinary, the sausage factory behind
the breakfast special?

A physician who specializes in philosophy and ethics, Dr.~Elliott hails from that quiet zone of
medicine where much of the job involves thinking about, talking about and doling out medications.
Hence his primary focus is on the ever-evolving relationship between the high art of medicine and
the big business of drugs.

Some of his material has, at this point, been reviewed ad nauseam in the daily press and in books by
others, so most readers will be familiar with the bad habits of Big Pharma when it comes to subtle
data manipulation, high-pressure salesmanship and lavish gifts. Mighty and expensive are the efforts
to guide the hands that write the prescriptions. But Dr.~Elliott also spends time in places where
few other authors have ventured.

Doctors get pens and trinkets, football tickets, junkets to beach resorts. Less visible are the
large sums handed over in ``I'm going to make you a star'' projects to groom them as trusted faces
and voices in the service of some drug. Education and advertisement merge in these elaborate
ventures, as the paid professor travels the country, lecturing about disease and, incidentally, the
treatment thereof.

These ``key opinion leaders'' are bad enough, but who would ever imagine that the curricula vitae of
many academic physicians (those on a medical school faculty) are packed with journal articles
actually written by ghostwriters sponsored by pharmaceutical companies?

``Nobody expects American politicians to write their own speeches anymore,'' Dr.~Elliott reminds us,
``and nobody expects celebrities to write their own memoirs.'' Apparently doctors have now joined
the ranks of the charismatic talking heads, mouthing the words of others.

And just as ``professor'' generally describes someone who writes his or her own sentences,
``ethicist'' generally describes someone who dwells (or at least works) on an unusually high moral
plane. But Dr.~Elliott also takes a brief and very informative excursion into the world of the
medical ethicists. Once they were highly principled, underpaid gadflies, trying to sort out medical
decision making. Now they are part of a booming industry, and, speaking of industry, their ties to
the pharmaceutical industry are many and complex. Many companies now hire their own ethicists. But
who guards those guards?

Meanwhile, at the bottom of the pharmaceutical totem pole are the folks who make it all happen: the
people who volunteer to test new chemicals for safety before they are let loose on the general
public.

John le Carr\'e's book ``The Constant Gardener'' touched on some of the issues that arise when big
companies pay very poor people to test their drugs. Dr.~Elliott didn't have to head to Africa to
report this story, however: the Northeast corridor provided ample material.

In Philadelphia he found a group of professional ``guinea pigs,'' as they call themselves,
sequestered in a hospital's clinical research unit while they were testing a prospective new drug.
It was one long pajama party: ``We were just gorging ourselves at 2 a.m. on Cheez Doodles,'' one
guinea pig told him. Only one problem: Those were contraband doodles. The drug under investigation
required stringent dietary restrictions, which the subjects were systematically violating. So much
for the science of drug evaluation.

Some guinea pigs are activists -- one has founded an industry magazine, Guinea Pig Zero. Probably
more typical, unfortunately, are the subjects who spent time in a drug testing site in Miami, the
largest in the country until it was shut down for fire and safety violations. Many of them were
illegal immigrants, packed into shabby, overcrowded rooms with minimal supervision.

``Guinea pigs do not do things in exchange for money so much as they allow things to be done to
them,'' Dr.~Elliott points out. ``There are not many other jobs where this is the case.'' Yet for
all the job's built-in vulnerability, there is little monitoring of either the subjects' health or
the data's validity.

What a world, what a world, as the melting witch said in ``The Wizard of Oz.'' But there is one
small consolation: at least Dr.~Elliott didn't have to call his book ``White Coat, Black Heart.''
Now that would have been depressing. The bottom line is that much of what he describes is simply the
big business of medicine as we have allowed it to take shape. His bad actors are mostly just that:
actors caught up in a script not of their own devising. They all come home in the evening, take off
their black hats and hang up their white coats, just regular working stiffs out to make a buck.

WHITE COAT, BLACK HATAdventures on the Dark Side of Medicine. By Dr.~Carl Elliott. Beacon Press. 224
pages. \$24.95.

\section{China Central Bank's Chief Backs Gradual Rise in Currency}

\lettrine{T}{he}\mycalendar{Oct.'10}{09} head of China's central bank said Friday that he favored
letting the Chinese currency rise in value, but only gradually, as exchange-rate tensions
overshadowed an international meeting of finance ministers and central bankers.

Gathered here for the annual meetings of the International Monetary Fund, the top finance officials
of Britain, France and Japan all expressed concern about rising currency tensions -- but did not
adopt the urgent language the United States has used in prodding China to act. The head of the
I.M.F., Dominique Strauss-Kahn, said the question was not whether, but when the Chinese currency,
the renminbi, would rise in value.

In a forum sponsored by the I.M.F. and the British Broadcasting Corporation, Zhou Xiaochuan, the
governor of the People's Bank of China, said that China ``needs a market-based exchange-rate
regime'' and would ``try to move the exchange rate gradually to more equilibrium.'' But the process
had to occur ``in a gradual way,'' he said, ``rather than shock therapy.''

Mr.~Strauss-Kahn, who has been urged by American officials to play a greater role in speaking out on
the exchange-rate issue, struggled to toe a diplomatic line, encouraging China without appearing to
criticize it. ``For a long time I've repeatedly said that the renminbi is undervalued,'' he said at
the forum, with Mr.~Zhou seated to his left, ``but the solution to this is not to expect that China
will revalue its currency overnight.''

Joseph E.~Stiglitz, a Columbia University economist and a former chief economist of the World Bank,
cautioned that a currency revaluation alone would not do much to reduce the large American trade and
current account deficits with the rest of the world. Those reductions, he said, require fundamental
changes in private and public savings in the United States.

``The focus on just the exchange rate is detracting attention from much more fundamental issues,''
he said during the forum with Mr.~Zhou and Mr.~Strauss-Kahn.

Last year, the Group of 20 economic powers agreed on a framework for ``rebalancing'' the global
economy as part of their response to the global financial crisis. Under that agreement, the United
States is supposed to shift toward saving and investing more and spending and borrowing less.

Mr.~Zhou said the Chinese government was committed to doing its part by encouraging growth of its
service sector, which would lessen its dependence on the export-oriented manufacturing sector; by
improving its health care and retirement pension systems, which would allow Chinese consumers to
save less; and by spending more on rural development to lessen the economy's dependence on rapidly
urbanizing coastal areas.

``The package may be the better thing to focus on,'' rather than just the currency issue, Mr.~Zhou
said.

But many of those structural changes will take time, while Western patience on the currency issue
appears to be running thin.

In separate news briefings, officials acknowledged that the currency issue threatened to overshadow
the I.M.F. annual meetings, which have tackled other themes like how to harmonize changes in
international financial regulation and how rich, debt-burdened nations could cut back public
spending without harming the fragile recovery. But they avoided comments that might antagonize
China.

George Osborne, Britain's chancellor of the exchequer, said, ``We do need to move toward
market-oriented exchange rates that reflect fundamentals,'' and added that the I.M.F. and the Group
of 20 had to play a role in addressing the imbalances.

Christine Lagarde, the French finance minister, played down talk of a currency war, saying ``we want
to talk about peace.'' Japan's finance minister, Yoshihiko Noda, said he expected ``steady
implementation'' of China's promise in June to permit greater exchange-rate flexibility, even though
the renminbi has risen only slightly in value since then.

Japan is one of several countries, like South Korea and Brazil, that has taken recent steps to halt
or slow the appreciation of its currencies. Those currencies have risen in value, in part, because
investors in Europe and the United States, where interest rates are extremely low and growth is
sluggish, have gone overseas in search of higher yields. The result has been a surge in capital into
rapidly growing economies in Asia and Latin America.

Mr.~Noda said Friday that Japan's intervention was intended primarily to moderate volatility in
exchange rates and not as a large-scale, permanent policy.

In an interview, the financier George Soros, who has argued that movement on China's currency is
essential for avoiding a global currency conflict, said he saw merit in China's caution but noted
that international pressures were mounting.

``A 20 percent or more jump in the value of the renminbi would create widespread bankruptcies in the
export industry, and the objective of increasing consumption would be undermined by increasing
unemployment,'' Mr.~Soros said, expressing sympathy with the position of China's premier, Wen
Jiabao, who warned this week that rapid appreciation of the currency would cause major social
disruptions.

``On the other hand, 10 percent annual appreciation should be tolerable for the Chinese economy, and
it would be a tremendous relief both for the United States, and for the other developing countries
that are currently bearing the brunt of the capital flows,'' Mr.~Soros said.

``It is no longer just a conflict between the United States and China,'' he said. ``The rest of the
world -- Japan, Europe and the developing world -- are all up in arms.'' Mr.~Soros said that China
was threatened by inflation, and the United States by deflation. Rebalancing would help to reduce
both threats, he said.

The currency issue increasingly looks to be the dominant theme of the Group of 20 summit meeting
that President Obama and other world leaders will attend in November in Seoul, South Korea.

``Currency issues must be debated, must be discussed, must be resolved in pretty closed circles --
and sometimes, behind closed doors as well -- to be efficient,'' said Ms.~Lagarde, the French
finance minister.

``But that clearly has to happen,'' she said.

\section{Nobel Peace Prize Given to Jailed Chinese Dissident}


\lettrine{L}{iu}\mycalendar{Oct.'10}{09} Xiaobo, an impassioned literary critic, political essayist
and democracy advocate repeatedly jailed by the Chinese government for his activism, has won the
2010 Nobel Peace Prize in recognition of ``his long and non-violent struggle for fundamental human
rights in China.''

Mr.~Liu, 54, perhaps China's best known dissident, is serving an 11-year term on subversion charges,
in a cell 300 miles from Beijing.

He is one of three people to have received the prize while incarcerated by their own governments,
after the Burmese opposition leader Aung San Suu Kyi in 1991, and the German pacifist Carl von
Ossietzky in 1935.

By awarding the prize to Mr.~Liu, the Norwegian Nobel Committee has provided an unmistakable rebuke
to Beijing's authoritarian leaders at a time of growing intolerance for domestic dissent and a
spreading unease internationally over the muscular diplomacy that has accompanied China's economic
rise.

In a move that in retrospect appears to have been counterproductive, a senior Chinese official had
warned the Norwegian committee's secretary that giving the prize to Mr.~Liu would adversely affect
relations between the two countries.

The committee, in announcing the prize Friday, noted that China, the world's second biggest economy,
should be commended for lifting hundreds of millions of people out of poverty.

But it chastised the government for ignoring basic rights guaranteed by the Chinese Constitution and
in the international conventions to which Beijing is a party. ``In practice, these freedoms have
proved to be distinctly curtailed for China's citizens,'' committee members said, adding, ``China's
new status must entail increased responsibility.''

The Chinese Foreign Ministry reacted angrily to the news, calling it a ``desecration'' of the peace
prize and saying it would harm Norwegian-Chinese relations. The Chinese government summoned Norway's
ambassador to protest the award, a spokesman for the Norwegian Foreign Ministry told reporters.

``The Nobel Committee giving the peace prize to such a person runs completely contrary to the aims
of the prize,'' Ma Zhaoxu, a spokesman said in a statement posted on the ministry's Web site. ``Liu
Xiaobo is a criminal who has been sentenced by Chinese judicial departments for violating Chinese
law.''

Headlines about the award were nowhere to be found in the Chinese-language state media or on the
country's main Internet portals. Broadcasts about Liu Xiaobo (pronounced Liew Show Boh) on CNN,
which reach only luxury compounds and hotels in China, were blacked out throughout the evening. Many
mobile phone users reported not being able to transmit text messages containing his name in Chinese.

But on government-monitored microblogs like Sina.com, which regularly blocks searches for his name,
the news still generated nearly 6,000 comments within an hour of the announcement.

The announcement also energized international calls for Mr.~Liu's release, including one from
President Obama, who urged China to free him ``as soon as possible,'' saying that political reforms
in China had not kept pace with its economic growth.

Given that he has no access to a telephone, it was unlikely that Mr.~Liu would immediately learn of
the news, his wife, Liu Xia, said. On Friday night, dozens of foreign reporters gathered outside the
couple's building in Beijing but they were prevented from entering by the police, who posted a sign
saying the complex residents ``politely refused'' to be interviewed. His wife was also barred from
leaving her apartment.

Given his imprisonment, Mr.~Liu is not expected to accept the prize in person. The award includes a
gold medal, a diploma and the equivalent of \$1.5 million.

The prize is an enormous psychological boost for China's beleaguered reform movement and an
affirmation of the two decades Mr.~Liu has spent advocating peaceful political change in the face of
unremitting hostility from the ruling Chinese Communist Party. Blacklisted from academia and barred
from publishing in China, Mr.~Liu has been harassed and detained repeatedly since 1989, when he
stepped into the drama playing out on Tiananmen Square by staging a hunger strike and then
negotiating the peaceful retreat of student demonstrators as thousands of soldiers stood by with
rifles drawn.

``If not for the work of Liu and the others to broker a peaceful withdrawal from the square,
Tiananmen Square would have been a field of blood on June 4,'' said Gao Yu, a veteran journalist and
fellow dissident who was arrested in the hours before the tanks began moving through the city.

Mr.~Liu's most recent arrest in December 2008 came a day before a reformist manifesto he helped
shape began circulating on the Internet. The petition, Charter '08, demanded that China's rulers
guarantee civil liberties, judicial independence and the kind of political reform that would
ultimately end the Communist Party's monopoly on power.

``For all these years, Liu Xiaobo has persevered in telling the truth about China and because of
this, for the fourth time, he has lost his personal freedom,'' his wife said in an interview on
Wednesday.

An inexhaustible writer, poet and piquant social commentator, Mr.~Liu was among the first of his
generation to return to college after the Cultural Revolution of 1966 to 1976, when schools were
shuttered and intellectuals were banished to the countryside.

In a book of dialogues he published under a pseudonym with the popular writer Wang Shuo, Mr.~Liu
later described those years as a ``temporary emancipation from the education process,'' but
ultimately found them deeply disturbing for the cruelty they inspired. In one passage, he recalled
taunting an old man suspected of sympathizing with Chiang Kai-Shek, the Nationalist leader who had
been defeated by Mao Zedong's Communist rebels. The abuse, he said, brought the man to tears. ``In
that era, when people were not treated as human, we were all guilty,'' Mr.~Liu said.

After graduating from the Chinese department at Jilin University, Mr.~Liu enrolled at Beijing Normal
University, where he was first a doctoral student and then a teacher. It was in the mid-1980s that
he burst to fame for rousing lectures and incisive works of literary criticism that demanded an
honest reckoning of the historical excesses under Mao. His writings were so bracing that school
officials nearly denied him his doctoral degree.

In 1988, he left China for a series of speaking engagements in Norway, Hawaii and New York. It was
in the spring of 1989, while a visiting scholar at Columbia University, that thousands of students
began occupying Tiananmen Square, the ceremonial heart of the nation, with their calls for democracy
and an end to official corruption. Mr.~Liu later says he hesitated -- he almost turned back during a
change of planes in Tokyo -- but returned to China that May as demonstrations spread across the
country, paralyzing the leadership in Beijing.

In early June, as it became apparent the military would clear the square by force, Mr.~Liu and three
other well-known intellectuals staged a 72-hour hunger strike as a show of solidarity that he later
said was necessary to earn the students' trust as the movement lurched toward a violent end. In the
early morning hours of June 4, as the army closed in, the men pried a stolen rifle from the hands of
a distraught student and negotiated with military commissars to allow the protesters to safely exit
the square.

Over the next few days as the crackdown began in earnest and many protest organizers fled China,
Mr.~Liu was arrested and later castigated in the state press as a traitorous ``black hand'' who had
helped orchestrate what the government termed a counter-revolutionary rebellion.

After his release in 1991, Mr.~Liu was stripped of his teaching job but he continued to gather
petitions pressing for democracy, human rights and the reassessment of the government's verdict on
the Tiananmen protests. In 1995, his unbowed activism brought another arrest leading to an
eight-month detention and in 1996, he was sentenced to three years in a labor camp for a series of
essays that criticized the government and called for an end to official corruption.

In those days, Mr.~Liu bicycled across the city to the compounds where foreigners worked and lived
to fax off his writings to overseas journals.

Zhang Zuhua, a former Communist Youth League official who later played a pivotal role in drafting
Charter '08, said Mr.~Liu was a solitary advocate in the 1990s, when fear, exile and the pursuit of
self-enrichment silenced most Chinese intellectuals.

``While others were researching the same problems from a theoretical or policy standpoint, he was
actively protesting and actually doing things,'' Mr.~Zhang said.

When Mr.~Liu emerged from prison in 1999, the Internet had taken hold in China and was beginning to
transform the nature of public discourse. At first reluctant to use a computer, Mr.~Liu quickly
became a prolific commentator on overseas Web sites, later calling the Internet ``God's gift to
China.'' Over the years, he published more than 1,000 articles.

Inspired by a number of documents, including the United States Constitution and the French
Declaration of the Rights of Man and the Citizen, Charter '08 was in some ways a culmination of
Mr.~Liu's search for pragmatic ways to push for political reform in China. Although he initially
heeded his wife's pleas not to join the drafting, he later immersed himself in the three-year
effort, revising it numerous times and working to convince more than 300 people -- intellectuals,
workers and party members -- to add their names.

In its brief life on the Internet, the petition gathered some 10,000 signatures before censors
stymied its spread. In the Internet crackdown that followed, scores of blogs were shut down, the
initial 300 signatories were interrogated and Mr.~Liu was taken to an undisclosed location, and
Mr.~Liu was held virtually incommunicado for a half year before his arrest.

At a two-hour trial last December, the government cited Charter '08 and six essays he had written to
argue that Mr.~Liu had exceeded the right to free expression by ``openly slandering and inciting
others to overthrow our country's state power,'' according to the verdict. Mr.~Liu countered that he
had simply advocated a gradual and nonviolent change in governance.

In a statement he gave to the court before his sentencing on Christmas Day, he said he held no
grudge against those who sought to silence him and he even thanked his captors for treating him with
dignity.

``I firmly believe that China's political progress will never stop, and I'm full of optimistic
expectations of freedom coming to China in the future, because no force can block the human desire
for freedom,'' he said. ``China will eventually become a country of rule of law in which human
rights are supreme. I'm also looking forward to such progress being reflected in the trial of this
case, and look forward to the full court's just verdict -- one that can stand the test of history.''

\section{Nobel Prize for Dissident Is Seen as Rebuke to China}

\lettrine{F}{ew}\mycalendar{Oct.'10}{09} nations today stand as more of a challenge to the
democratic model of governance than China, where an 89-year-old Communist Party has managed to quash
political movements while creating a roaring, quasi-market economy and enforcing a veneer of social
stability.

With the United States' economy flagging and its global influence in decline, some Chinese leaders
now appear confident in asserting that freedom of speech, multiparty elections and constitutional
rights -- what some human rights advocates call universal values -- are indigenous to the West, and
that is where they should stay.

The awarding of the Nobel Peace Prize to Liu Xiaobo, 54, was a sharp rejoinder to that philosophy.
Of course, it was a Norwegian panel that gave him the prize, providing Chinese officials and their
supporters with ample ammunition to denounce the move as another attempt by the West to impose its
values on China.

But anticipating the criticism, the judges underscored the support in China for the imprisoned
Mr.~Liu's work and his plight, which they said proved that the Chinese were as hungry as anyone for
the political freedoms enjoyed in countries like the United States, India and Indonesia.

``The campaign to establish universal human rights also in China is being waged by many Chinese,
both in China itself and abroad,'' the Norwegian Nobel Committee said. ``Through the severe
punishment meted out to him, Liu has become the foremost symbol of this wide-ranging struggle for
human rights in China.''

The Dalai Lama, the exiled Tibetan spiritual leader who won the prize in 1989, highlighted the
grass-roots Chinese push for political reform in a statement praising Mr.~Liu, saying that ``future
generations of Chinese will be able to enjoy the fruits of the efforts that the current Chinese
citizens are making towards responsible governance.'' Yet the Dalai Lama stands as proof that the
struggle for rights in China is a hard one, and that winning the Nobel is no guarantee of achieving
even minimal success.

Nevertheless, the number of signatures on Charter 08, the document that Mr.~Liu co-drafted that
calls for gradually increasing constitutional rights, shows that at the very least, there is an
appetite in this country to openly discuss the kind of values that hard-line Communist Party leaders
dismiss as a new brand of Western imperialism.

The 300 initial signatures on the document snowballed to 10,000 as it spread on the Internet, even
as the government tried its best to stamp it out. Certainly many of those who signed it were
intellectuals, not exactly representative of most Chinese, but China has a rich history of political
reform led by its elites. Chinese lawyers, journalists, scholars, artists, policy advisers -- many
among them will be heartened by the Nobel Committee's decision.

``Today, many people are making efforts,'' said Wan Yanhai, the most prominent advocate for AIDS
patients in China and one of the initial signers of Charter 08; he left China temporarily for the
United States in May because of what he called police harassment. ``They're hidden, but they're
there,'' he said. ``People are organizing different resistance movements, sometimes in a peaceful
way, sometimes in a violent manner.''

Cui Weiping, a social critic who teaches at the Beijing Film Academy, said the rights struggle was
moving from a local stage to a global one. ``Like everything that happens in China today, the
democracy movement here exists in a global context,'' she said. ``So this will be a lesson to China:
it can't bottle up the democracy movement forever.''

The Internet, the vehicle that carried Charter 08 to prominence, simmered with Chinese support for
Mr.~Liu early Friday night despite extensive government filtering. Liu Xiaobo was the most common
topic on Sina.com's Weibo, a popular microblog forum. Microbloggers burned with enthusiasm for the
prize and hurled invective at the government: ``Political reform and the Nobel Prize, is this a new
start? This day has finally come,'' wrote a user named Nan Zhimo. Another user, Hei Zechuan, said,
``The first real Chinese Nobel Prize winner has emerged, but he is still in prison right now; what a
bittersweet event.''

Even before the announcement Friday afternoon, a group of supporters gathered outside the Beijing
apartment building where Liu Xiaobo's wife, Liu Xia, lives. They showed little fear of the
black-uniformed police officers surrounding them.

``I believe this award will massively open up room for political discussion in China,'' said one of
those standing outside the building, Li Yusheng, 66, a retired journalist, Charter 08 signer and
founder of a group that aims to help the poor. ``And it will exert pressure on the authorities to
change their old ways, so that they will not be able to jail people like Liu Xiaobo in the future.
They will have to change or else be driven out of power.''

But the authorities clung to their habits on Friday night, as police officers showed up at
celebratory gatherings in Beijing and Shanghai to haul people off to police stations, according to
Twitter feeds.

Some political experts here say that even China's more liberal-minded leaders have little appetite
for pushing vigorously for greater political rights, and will continue to hold back as jockeying
intensifies ahead of the 2012 leadership succession -- a time when hard-line attitudes tend to
dominate. A sharp taste of that came in March 2009, when Wu Bangguo, the head of the National
People's Congress, a rubber-stamp Parliament, made a speech in which he dismissed any move toward
Western-style democracy, mentioning it no fewer than nine times.

``We will never simply copy the system of Western countries or introduce a system of multiple
parties holding office in rotation,'' he said, adding that ``although China's state organs have
different responsibilities, they all adhere to the line, principles and policies of the party.''

Some Chinese liberals like Mr.~Wan say they see a compatriot in Prime Minister Wen Jiabao, who as
recently as August publicly extolled the virtues of political change. ``Without the guarantee of
political system reform, the successes of restructuring the economic system will be lost and the
goal of modernization cannot be realized,'' Mr.~Wen said, according to The People's Daily.

Some liberal economists like Yang Yao and Wu Jinglian have also come out strongly in support of
political restructuring, arguing that China's economy, where state-owned enterprises tied to the
Communist Party continue to dominate the largest industries, can reach maturity only with the checks
and balances that come with democracy.

The exact form of democracy is often left vague in these discussions. Liberals know that calling for
multiparty elections -- a direct challenge to the primacy of the Communist Party -- is a red line.
Mr.~Wen, whom many Chinese praise but whose actual power is dubious, shies away from mentioning
elections. Mr.~Liu and the co-writers of Charter 08 were also careful to avoid calling for any
immediate, drastic change to the Communist Party's hold on power.

``Our intention was not to threaten the party or the government,'' said Zhang Zuhua, one of the
charter's main authors. ``It was to put forth this framework of universal values, and build a
consensus within society around it, among both those within and outside the system.''

``Except the government,'' he said, ``clearly does not affirm these universal values.''

\section{Currency Rift With China Exposes Shifting Clout}

\lettrine{A}{t}\mycalendar{Oct.'10}{11} a private dinner on Friday at the Canadian Embassy, finance
officials from seven world economic powers focused on the most vexing international economic problem
facing the Obama administration.

Over seared scallops and beef tenderloin, Treasury Secretary Timothy F.~Geithner urged his
counterparts from Europe, Canada and Japan to help persuade China to let its currency, the renminbi,
rise in value -- a crucial element in redressing the trade imbalances that are threatening recovery
around the world.

But the next afternoon, the annual meetings of the International Monetary Fund ended with a tepid
statement that made only fleeting and indirect references to the simmering currency tensions.

The divergence between the mounting anxieties over Chinese policy and the cautious official response
was a striking display of the difficulty of securing international economic cooperation, two years
after the financial crisis began.

Above all, officials say, the crisis has shifted influence from the richest powers toward Asia and
Latin America, whose economies have weathered the recession much better than those of the United
States, Europe and Japan.

``We have come to the end of a model where seven advanced economies can make decisions for the world
without the emerging countries,'' said one European official involved in the weekend talks. ``Like
it or not, we simply have to accept it.''

The debate over currency valuation is pivotal. World leaders broadly agree that for the global
economy to be more stable, imbalances between creditor countries like China and Germany and debtor
countries like the United States and Britain have to be fixed. Correcting those imbalances, some
economists say, will help create jobs in the United States and reduce the threat of inflation and
asset bubbles in China.

The shifting dynamics have most noticeably affected the United States, which pushed more forcefully
than its counterparts for stronger pressure on China but has been unable to persuade them to stand
with it at the forefront of the debate.

In general, the Europeans have taken a far more conciliatory line toward China. The French finance
minister, Christine Lagarde, said on Saturday, ``It's not helpful to use bellicose statements when
it comes to currency or to trade.''

In interviews, American and European officials involved in discussions over the Chinese currency
last week outlined several reasons a unified position has been so hard to forge.

For one thing, China has moved adroitly to deflect criticism of its currency policies, by pledging
to move at a gradual pace and by pointing to other sources of global imbalances. This leaves Western
diplomats struggling to strike the right balance between forceful rhetoric and patient cajoling in
pressuring China to act.

Another factor is that the most dire part of the crisis has passed, and many countries are now more
concerned with their own national economies and no longer feel the urgency act in concert.

``We are moving from a consensual to a more confrontational period in global economic governance,''
said Thomas Kleine-Brockhoff, senior director of policy programs at the German Marshall Fund of the
United States, which promotes trans-Atlantic cooperation.

Complicating the effort is a dispute between the United States and Europe over how to change board
representation within the I.M.F. to give greater voice to the fast-growing economies that are
propelling global growth. The Americans want emerging countries, especially China, to have more
representation, and thus take on more responsibility. But Europe is reluctant to give up some of its
positions on the board.

And significantly, in the eyes of many countries, the United States has lost some of the standing it
needs to shape global policy. Not only is Wall Street viewed by many as having initiated the world
financial crisis, but also, a number of countries fear that policies by the Federal Reserve are
pushing down the dollar's value -- the same kind of currency weakening for which the Obama
administration has criticized China.

``Other countries are no longer willing to buy into the idea that the U.S.~knows best on economic
policy, while at the same time the emerging markets have become increasingly influential and
independent,'' said Kenneth S.~Rogoff of Harvard, a former chief economist at the I.M.F.

The inconclusive I.M.F. outcome means that the renminbi's exchange rate will again be a focus when
President Obama and other leaders of the Group of 20 economic powers gather next month in Seoul,
South Korea. Officials said the United States would keep up pressure on China in the weeks leading
up to that meeting.

Despite the bland language of the I.M.F. statement on Saturday, American and European officials said
the weekend meetings were not a failure.

After all, the 187-member I.M.F. is not a customary forum for decisive collective action, and
changes in national economic policies typically occur in a gradual, incremental fashion. A Treasury
official pointed out that the gatherings focused high-level attention on the currency problem, and
ended with an agreement for the I.M.F. to play a greater role in monitoring its members'
exchange-rate practices and the ``spillover effects” of each country's economic policies on the
rest of the world.

Even so, economic and political forces have made it difficult for the United States to address what
Mr.~Geithner has called the ``central existential challenge of cooperation.''

``Even though this is inherently a collective problem, there is no specific mechanism in the I.M.F.
or G-20 that looks up to the challenge,'' he said Sunday. ``Our aim is to change that by encouraging
countries to buy into a stronger set of norms and behaviors on these issues.''

In a speech on Wednesday, Mr.~Geithner in essence accused China of setting off a cycle of
``competitive nonappreciation,'' in which countries block their currencies from rising in value to
support their exporters -- as Japan, Brazil and South Korea have recently tried. Economists have
warned that this type of policy could lead to a destructive currency war.

Other officials have quietly expressed worry that the United States is itself contributing to the
currency imbalance: the Federal Reserve has adopted an expansionary monetary policy intended to
stimulate the economy, has contributed to the weakening of the dollar against other currencies.

Edwin M.~Truman, a former top official at the Fed and the Treasury Department, said that while
Europe and Japan want the renminbi to appreciate, ``they don't want the dollar to depreciate along
with it.''

Allowing the renminbi to rise would make Chinese exports more expensive and American exports
cheaper. That would assist with rebalancing: getting China to spend more and save less and the
United States to spend less and save more.

A vital part of the American strategy has been to argue that China would also benefit from letting
the renminbi rise.

Doing so, many economists say, would reduce the risk of inflation and asset bubbles, and help
reorient growth away from exports and coastal manufacturing areas and toward domestic consumer
demand and poor rural regions in need of development. China signed on last year to a G-20 platform
for ``strong, sustainable and balanced growth,'' which has become a sort of motto for global the
recovery.

But there is little agreement on how to make the motto a reality.

James D.~Wolfensohn, a former president of the World Bank, said each side had a point. ``The Chinese
have a legitimate case that they have to keep their economy going and that they're not going to let
us run their economy for them,'' he said. ``On the other hand, we have a legitimate case that China
ought to bear its share of the burden and show some leadership.''

\section{Wife Detained After Visiting Nobel Winner}

\lettrine{T}{he}\mycalendar{Oct.'10}{11} wife of this year's Nobel Peace Prize winner, Liu Xiaobo,
was allowed to meet with her husband on Sunday at the prison in northeastern China where he is
serving an 11-year sentence, but she was then escorted back to Beijing and placed under house
arrest, a human rights group said.

Prison officials had informed Mr.~Liu that he won the award -- a decision vehemently condemned by
the Chinese government -- the day before. In their hourlong visit, Mr.~Liu's wife, Liu Xia, said her
husband had told her, ``This is for the lost souls of June 4th,'' and then was moved to tears.

Hundreds died June 4, 1989, in Beijing when Chinese troops and tanks crushed pro-democracy
demonstrations in Tiananmen Square. Mr.~Liu told his wife the award commemorates the nonviolent
spirit in which those who died fought for peace, freedom and democracy, the group, Human Rights in
China, said in a statement.

In Beijing, Ms.~Liu's telephone and Internet communication has been cut off and state security
officers are not allowing her to contact friends or the media, the statement said. Nor can she leave
her house except in a police car, according to the group. Her brother's phone has also been
``interfered with,'' the statement said.

Mr.~Liu, who was active in the 1989 movement, spent the next two decades pressing for political
reform in China. A 54-year-old former literature professor, he was one of the main authors of
Charter '08, a pro-democracy manifesto that calls for expanded liberties and the end to single-party
rule in China.

Roughly 10,000 people signed the document before the government blocked its circulation on the
Internet. Based on his pro-democracy writings, Mr.~Liu was convicted last December of ``inciting
subversion of the state.''

The Chinese government has described Mr.~Liu's award as ``blasphemy'' and has imposed a blackout on
news about it. Security in some areas has been tightened, and the road to Jinzhou prison in Liaoning
Province, where Mr.~Liu is held, has been blocked.

On Friday night, the police detained 20 bloggers, lawyers and academics who gathered for a
celebratory banquet at a private room in a Beijing restaurant. By Sunday night, 10 guests had been
released, according to a prominent activist, Zhang Zuhua, another of Charter '08's main authors.
Three were given eight days in detention for disturbing the peace, and seven have been escorted out
of Beijing, Mr.~Zhang said.

The Chinese journalist Zan Aizong sent a Twitter message on Sunday saying that Chinese Internet
media outlets had been ordered to post a Xinhua News Agency article that Russian media were
attacking the Nobel Peace Prize as a ``political tool of the West.''

Analysts speculated that Chinese leaders would gather soon to define the Communist Party's position
on the Nobel award. Until then, Nicholas Bequelin, a Hong Kong researcher for Human Rights Watch,
predicted that government agencies would tread carefully.

``The statement released by the Ministry of Foreign Affairs is almost word for word what they have
said before,'' Mr.~Bequelin said. ``The treatment meted out to dissidents and lawyers is the regular
one. Everyone is sitting tight and awaiting instructions from the top.''

Several key Chinese officials are currently out of the country, including Zhou Yongkang, a member of
the Standing Committee of the Political Bureau of the Communist Party's Central Committee.

\section{Cuomo and Paladino: Shared Roots but Clashing Italian Identities}

\lettrine{S}{trategy}\mycalendar{Oct.'10}{11} sessions have been held at a restaurant called
Sinatra's. ``Sopranos''-style gold chains have shown up in campaign advertisements. Ethnic-tinged
terms, like ``goumada,'' and wisecracks about Sicilian grudges have been bandied about. And
television news crews from Italy have descended on the candidates.

In the raucous race for governor of New York this year between Andrew M.~Cuomo and Carl P.~Paladino,
an unexpected debate is mesmerizing the Italian-American community and increasingly spilling out
into public view: Is the contest shattering long-held ethnic stereotypes or reinforcing them?

The tension has recast a milestone election for the state's largest ethnic group, which has spent
decades battling for political might.

Along with feelings of pride have come moments of unease and even mortification. A Staten Island
lawmaker has scolded Mr.~Paladino, a Republican, for comments that ``degrade our shared cultural
heritage,'' comparing him with the makers of the reality television show ``Jersey Shore.'' And in an
interview, Mr.~Paladino mischievously questioned whether his opponent was really Italian.

An age difference of only a decade separates Mr.~Paladino and Mr.~Cuomo, who are both expected to
march in the city's huge Columbus Day parade down Fifth Avenue on Monday, and each traces his
lineage to southern Italy. But the two men are starkly different in how they view and express their
Italian identity.

Mr.~Cuomo, the Democrat who is the state's attorney general, prides himself on transcending the
image of the unpolished, old-country Italian, and credits his father, Mario M.~Cuomo, the former
governor of New York, for debunking many of those stereotypes.

Even as he embraces his ancestry, Andrew Cuomo is extremely sensitive about the assumptions that
surround it and the political liabilities that could attach to it: He once had his pollster question
the state's voters about their views of the television show ``The Sopranos,'' to glean insight about
how they saw Italians.

He complained after a reporter described him in an article as a ``double espresso of a politician,''
suggesting that the term amounted to an anti-Italian slur.

Behind the scenes, Mr.~Cuomo worked to avoid having the Democratic statewide ticket he leads be
excessively dominated by Italians this fall. He privately fretted that if Eric R.~Dinallo became the
nominee for attorney general and the incumbent Thomas P.~DiNapoli for comptroller, voters might find
that one Italian too many.

``People will think we're trying to open a pizza parlor,'' Mr.~Cuomo joked to a friend, according to
people familiar with the conversation. (Mr.~Dinallo was defeated in a five-way primary last month.)

In an interview, Mr.~Cuomo, 52, said he had absorbed the lessons of his father and tried to emulate
him. Mario Cuomo was a breakthrough figure in American politics who still felt the sting of bias; as
a young lawyer, he had been turned down for jobs at Manhattan's ``white shoe'' law firms.

``He is the model of decorum and civility and grace,'' Andrew Cuomo said, ``and he was on the stage
at the same time that you were watching Italian-Americans depicted in movies and television as thugs
and people who were crude.''

He recalled the hurt that his father, the state's first Italian-American governor, felt shortly
after his election in 1982, when Albany reporters performed skits for an annual satirical show that
featured gangsters and Mafia-inspired characters. ``He was offended by it, and he should be,''
Mr.~Cuomo recalled of his father, who is the son of immigrants and grew up speaking Italian at home.

The younger Mr.~Cuomo said that the polling he paid for in 2002, during his first run for governor,
showed that the stereotypes remained pervasive. ``I can tell you, it's real,'' he said, adding,
however, that he thought it was less pronounced and less negative than in the past.

By contrast, Mr.~Paladino, a Republican real estate developer from Buffalo, seems to relish his
reputation as an undiluted, street-smart, up-by-the-bootstraps Italian.

He travels to Italy up to a dozen times a year. He sometimes lapses into Italian. And he developed a
habit of greeting associates, Italian-style, with a kiss on the cheek.

Like Mario M.~Cuomo, Mr.~Paladino's father, who immigrated at age 6, endured discrimination. As an
adult, to find work he shortened his given name from Belesario to the anglicized Bill. But rather
than shrink from an ethnic style and mannerisms, his son Carl, 64, has embraced them.

One of Carl Paladino's proudest accomplishments was gaining admission to the Big Timers, a heavily
Italian social club in East Buffalo.

``He can't let go of that. He doesn't want to let go of that,'' his brother Joe said. ``He is still
connected to that neighborhood.''

Guy V.~Molinari, the former Staten Island borough president and Republican power broker, who like
many of New York's Italian-Americans is riveted by the governor's race, said, ``As Italians, they
are very much the opposite of each other.''

To Mr.~Molinari, Mr.~Paladino fits the mold of a traditional southern Italian: loud and brash, often
shooting from the hip. ``I know these type of Italians; they are in my family,'' Mr.~Molinari said.
``They all talk at once; nobody listens.''

Mr.~Cuomo, he said, is a different but equally recognizable kind of Italian: studious and reserved,
bent on obsessively thinking everything through before taking action.

What is most interesting to many watching the campaign is that the stereotypes are being stirred up
in a race between Italian-Americans, or, in the words of Stefano Albertini, a faculty member in the
department of Italian studies at New York University, ``not when there was an Italian against
somebody else, but an Italian against an Italian.''

It is not only the candidates giving this election its Italian cast. Central players in both
campaigns are Italian: Mr.~Cuomo's closest and most powerful adviser is Joseph Percoco, a pugnacious
political enforcer (a term Mr.~Cuomo finds ethnically loaded) and Mr.~Paladino's campaign manager
and spokesman is Michael R.~Caputo, who describes himself as a ``junkyard dog.''

From the start, the Paladino camp, sensing Mr.~Cuomo's sensitivity to the issue, has deliberately
injected ethnicity into the campaign. After Mr.~Paladino won the Republican primary, Mr.~Caputo
commissioned a campaign poster that, by means of an altered photo, depicted Mr.~Cuomo shirtless in
the shower, trying to wash off the muck of Albany corruption. (``Clean up Albany,'' it said. ``Start
with Cuomo.'') A sly detail was inserted: a gold chain around his neck, prompting howls of protest
from those who detected anti-Italian bias.

Mr.~Caputo scoffed at the complaints at the time, gleefully declaring to reporters, ``Carl has his
own gold chain he wears very proudly, and so do I.''

Mr.~Paladino playfully told an interviewer from Italy that perhaps Mr.~Cuomo's claim of Italian
ancestry should be viewed skeptically. ``I don't know, he might have been adopted,'' Mr.~Paladino
said impishly.

During the same interview, he showed off his mastery of Italian, such as it is. When the reporter
complimented his fluency, Mr.~Paladino begged to differ. ``It's very broken,'' he said. ``I can find
my way to the bathroom.''

Mr.~Paladino's response to Italians upset by his ethnic provocations: Stop being so touchy. He calls
himself an equal-opportunity offender: Defending his having forwarded to friends and associates
e-mails containing inflammatory images of African-Americans, he used a string of derogatory terms
for Italians.

That prompted Diane J.~Savino, a state senator from Staten Island, which is heavily Italian, to fire
off a letter to Mr.~Paladino chiding him for ``offensive'' language.

``In an environment where people still believe it is acceptable to degrade our shared cultural
heritage, whether it be mob references or the buffoonery of the `Jersey Shore,' '' she wrote, ``it
is simply unacceptable for you to lower the discourse even further, particularly in a gubernatorial
campaign.''

On the campaign trail, Mr.~Cuomo uses his Italian roots subtly, seeking to connect to voters of all
ethnic groups who feel they have not gotten a fair shake.

Speaking to black churchgoers in Brooklyn a few days ago, Mr.~Cuomo assailed Mr.~Paladino for
espousing a policy that he said would allow the police to stop people who look like immigrants. ``I
look like an immigrant,'' Mr.~Cuomo said, to warm laughter.

At times, he sprinkles in Italian phrases and speaks affectionately of his childhood in Queens,
where family dinners revolved around animated debates. ``Sundays,'' he says, ``was politics and
pasta.''

In a recent speech, he joked that his mother, Matilda, still held a grudge against then-Mayor Edward
I.~Koch for challenging her husband in the 1982 Democratic primary for governor. ``My mom is
Sicilian,'' he told the crowd. ``They never forget.''

A few weeks later, she had a new affront to simmer over. After Mr.~Paladino made unsubstantiated
claims that Mr.~Cuomo had been unfaithful in his marriage -- one Paladino adviser whispered of a
goumada, Italian for mistress -- Mrs.~Cuomo was asked about the remarks. She invoked a favorite
Italian expression.

``Passa ci sobre,'' she said. ``Pass over it. It's garbage.''


\section{New Web Code Draws Concern Over Risks to Privacy}

\lettrine{W}{orries}\mycalendar{Oct.'10}{11} over Internet privacy have spurred lawsuits, conspiracy
theories and consumer anxiety as marketers and others invent new ways to track computer users on the
Internet. But the alarmists have not seen anything yet.

In the next few years, a powerful new suite of capabilities will become available to Web developers
that could give marketers and advertisers access to many more details about computer users' online
activities. Nearly everyone who uses the Internet will face the privacy risks that come with those
capabilities, which are an integral part of the Web language that will soon power the Internet: HTML
5.

The new Web code, the fifth version of Hypertext Markup Language used to create Web pages, is
already in limited use, and it promises to usher in a new era of Internet browsing within the next
few years. It will make it easier for users to view multimedia content without downloading extra
software; check e-mail offline; or find a favorite restaurant or shop on a smartphone.

Most users will clearly welcome the additional features that come with the new Web language.

``It's going to change everything about the Internet and the way we use it today,'' said James Cox,
27, a freelance consultant and software developer at Smokeclouds, a New York City start-up company.
``It's not just HTML 5. It's the new Web.''

But others, while also enthusiastic about the changes, are more cautious.

Most Web users are familiar with so-called cookies, which make it possible, for example, to log on
to Web sites without having to retype user names and passwords, or to keep track of items placed in
virtual shopping carts before they are bought.

The new Web language and its additional features present more tracking opportunities because the
technology uses a process in which large amounts of data can be collected and stored on the user's
hard drive while online. Because of that process, advertisers and others could, experts say, see
weeks or even months of personal data. That could include a user's location, time zone, photographs,
text from blogs, shopping cart contents, e-mails and a history of the Web pages visited.

The new Web language ``gives trackers one more bucket to put tracking information into,'' said Hakon
Wium Lie, the chief technology officer at Opera, a browser company.

Or as Pam Dixon, the executive director of the World Privacy Forum in California, said: ``HTML 5
opens Pandora's box of tracking in the Internet.''

Representatives from the World Wide Web Consortium say they are taking questions about user privacy
very seriously. The organization, which oversees the specifications developers turn to for the new
Web language, will hold a two-day workshop on Internet technologies and privacy.

Ian Jacobs, head of communications at the consortium, said the development process for the new Web
language would include a public review. ``There is accountability,'' he said. ``This is not a secret
cabal for global adoption of these core standards.''

The additional capabilities provided by the new Web language are already being put to use by a
California programmer who has created what, at first glance, could be a major new threat to online
privacy.

Samy Kamkar, a California programmer best known in some circles for creating a virus called the
``Samy Worm,'' which took down MySpace.com in 2005, has created a cookie that is not easily deleted,
even by experts -- something he calls an Evercookie.

Some observers call it a ``supercookie'' because it stores information in at least 10 places on a
computer, far more than usually found. It combines traditional tracking tools with new features that
come with the new Web language.

In creating the cookie, Mr.~Kamkar has drawn comments from bloggers across the Internet whose
descriptions of it range from ``extremely persistent'' to ``horrific.''

Mr.~Kamkar, however, said he did not create it to violate anyone's privacy. He said was curious
about how advertisers tracked him on the Internet. After cataloging what he found on his computer,
he made the Evercookie to demonstrate just how thoroughly people's computers could be infiltrated by
the latest Internet technology.

``I think it's O.K. for them to say we want to provide better service,'' Mr.~Kamkar said of
advertisers who placed tracking cookies on his computer. ``However, I should also be able to opt out
because it is my computer.''

Mr.~Kamkar, whose 2005 virus circumvented browser safeguards and added more than a million
``friends'' to his MySpace page in less than 20 hours, said he had no plans to profit from the
Evercookie and did not intend to sell it to advertisers.

``That wouldn't have been difficult,'' he said. Instead, he has made the code open to anyone who
wants to examine it and says the cookie should be used ``as a litmus test for preventing tracking.''

A recent spate of class-action lawsuits have accused large media companies like the Fox
Entertainment Group and NBC Universal, and technology companies like Clearspring Technologies and
Quantcast, of violating users' privacy by tracking their online activities even after they took
steps to prevent that.

Most people control their online privacy by adjusting settings in today's most common Web browsers,
which include Internet Explorer by Microsoft, Firefox by Mozilla, Safari by Apple and Opera, which
is used mostly in Europe and Asia and on mobile devices.

Each browser has different privacy settings, but not all of them have obvious settings for removing
data created by the new Web language. Even the most proficient software engineers and developers
acknowledge that deleting that data is tricky and may require multiple steps.

``Now there are so many sources of data storage, it's very hard for browser manufacturers to handle
that,'' Mr.~Cox said.

Mr.~Kamkar and privacy experts say that makers of Web browsers should agree on one control for
eliminating all tracking capabilities at once. ``There should be simple enough controls to take care
of every single thing,'' said Ms.~Dixon, who added that some browsers automatically collected large
amounts of data unless a user told them not to.

Mr.~Lie acknowledged that such companies ``do have a lot of power.'' But he said they worry that the
privacy settings they develop could be too strict. For example, he said Opera once tried to put more
controls on certain types of cookies, but users in Russia complained that the controls prevented a
popular social networking site from working properly.

But software developers and the representatives of the World Wide Web argue that as technology
advances, consumers have to balance its speed and features against their ability to control their
privacy.

``You can do more, but you need to be aware of how your information might be used or misused,''
Mr.~Jacobs said. ``It's the human questions.''

\section{After Building an Audience, Twitter Turns to Ads}

\lettrine{T}{witter}\mycalendar{Oct.'10}{11} at last looks serious about making money.

In the last two weeks, the company has introduced several advertising plans, courted Madison Avenue
at Advertising Week, the annual industry conference, and promoted Dick Costolo, who has led
Twitter's ad program, to chief executive -- all signs that Twitter means business about business.
It's Twitter's biggest financial effort since April, when it introduced its first, much-anticipated
ad program, Promoted Tweets.

Twitter's startling growth -- it has exploded to 160 million users, from three million, in the last
two years -- is reminiscent of Google and Facebook in their early days. Those Web sites are now
must-buys for advertisers online, and the ad industry is watching Twitter closely to see if it
continues to follow that path.

``Having been in the business for as long as I have and seeing things rise, I completely have the
same vibe on Twitter as Google, Facebook and DoubleClick,'' said Curt Hecht, chief executive of
VivaKi Nerve Center, part of the digital agency Publicis Groupe. ``You can tell by the client
interest levels.''

Another telling sign of Twitter's newfound interest in pushing its advertising is that although
fewer than 20 of the company's 300 employees work on advertising, that is in contrast to one just
three months ago.

But many advertisers and executives say there are questions to be answered and experiments to be
done before Twitter becomes a must-buy, if it ever does.

``Agencies are uneducated, brands are uneducated and to a certain extent, Twitter is uneducated,''
said Ian Schafer, chief of Deep Focus, an interactive marketing agency. ``There are no best
practices. There are just hunches about what will work.''

Advertising Week was a debut for Twitter, as Mr.~Costolo shared the stage with executives from
Google and Facebook and wooed ad executives in the audience with a clear message.

``We're definitely beyond the experimentation stage,'' he told them. ``It's working.''

In an interview later, he said, ``We feel like we've cracked the code on a new form of advertising,
and we feel like we've got a hit on our hands.''

This is a sharp change for the company, which has so far been careful to say it will move slowly and
experiment a lot. Twitter started with just six advertisers and now has about 40, including
Starbucks, Ford and Microsoft. Mr.~Costolo said in the interview that it would have more than 100 by
the end of the year.

Last week, Twitter added three avenues of advertising. Promoted Accounts, which began immediately
with Xbox and HBO, allows companies to pay Twitter to suggest that people follow their free Twitter
accounts, based on shared interests. Twitter also began publishing ads on Twitter apps, starting
with HootSuite; before, ads had appeared only on Twitter's Web site. Twitter will split the ad
revenue evenly with HootSuite and the other companies that make apps.

And finally, Mr.~Costolo said that next year Twitter would offer a self-serve tool for local
businesses to buy Twitter ads, and is working on ways to deliver those ads based on location. It
will use Internet addresses, location information that users share and clues like whether someone
follows a bunch of restaurants in a particular city.

Though just a few dozen advertisers have run Promoted Tweets, and some have not worked well, over
all they have outperformed Twitter's expectations.

Advertisers pay for Promoted Tweets to appear at the top of search results. Search ``vacation,'' for
instance, and see an ad from Virgin America encouraging people to vote in Virgin's Awkward Family
Vacation Photo Contest. Advertisers bid on keywords and pay when someone clicks on a link in the ad,
replies to it or forwards it to followers. Promoted Tweets will eventually show up in Twitter
timelines, not just when people search, based on the interests of people that users follow.

Twitter also sells Promoted Trends, so advertisers can show up in the list of topics most discussed
on Twitter, for \$100,000 a day.

``It's a cheap trick but it's got a lot of eyeballs,'' said Chad Stoller, director of digital
strategy for BBDO North America.

According to Twitter, on average 5 percent of people who see Promoted Tweets are clicking on,
replying to or forwarding the ads -- much higher than the less than 1 percent of people who click on
a typical display ad.

Trends are mentioned in Twitter conversations four to seven times as often when they are promoted.

Part of the reason so many people are clicking on ads might be the initial novelty, Mr.~Hecht said.
But at least one advertiser, Coca-Cola, said its response rates had been significantly higher than 5
percent, which surprised the company because it requires the user to make two clicks -- first on the
Promoted Trend and then on the link within the Promoted Tweet.

``At the end of the day, it's a very different product'' than traditional online ads, said Michael
Donnelly, group director for worldwide interactive marketing at Coca-Cola. ``People are engaged and
looking for a specific topic, so it's relevant.''

Coca-Cola has run more than 50 ad campaigns on Twitter, including during the World Cup. Coke was a
sponsor and paid to promote the trending topic WC2010. Coke also showed messages whenever someone
searched for words like ``soccer'' and ``vuvuzela'' with links that directed fans to Coke's YouTube
page.

Twitter ads are a work in progress, Mr.~Donnelly said. Coca-Cola learned early on, for example, that
dense Twitter messages about particular plays in a soccer game were clicked on or forwarded fewer
times than short, simple ones about the World Cup in general.

Other advertising executives say clients are still wary. Twitter has already proved to be an
effective free marketing tool, so why pay for an account?

``Every one of our clients has Twitter as a part of their social media strategy, but at the moment
we're not seeing a tremendous amount of interest in the specific packages that Twitter is
offering,'' said Aaron Shapiro, a partner at Huge, the digital agency within the Interpublic Group.

JetBlue advertises on Twitter, but Morgan Johnston, JetBlue's manager of corporate communications
who operates the airline's Twitter account, said its ``primary use of Twitter is really centered on
maintaining a dialogue with customers,'' which happens in the free account.

There is a long line of companies that want to advertise, Mr.~Costolo said, but as with so many
things at Twitter, where popularity with users has outpaced the company's growth, it cannot yet
handle the demand.

In August, Twitter hired Adam Bain, former president of the Fox Audience Network, an advertising
unit of the News Corporation, as president for global revenue, and has plucked sales executives from
Google, Facebook and Yelp to run sales across the country.

``We're at this new inflection point, and it's time to move forward a lot faster,'' Mr.~Costolo
said.

\section{Putin's Party Wins in Russia's Local Elections}

\lettrine{P}{rime}\mycalendar{Oct.'10}{11} Minister Vladimir V.~Putin's ruling party triumphed
easily in regional elections here in Siberia and elsewhere in Russia on Sunday, demonstrating that
its standing has not suffered, despite several challenging months for the government.

The ruling party, United Russia, said the results were evidence of the public's faith in Mr.~Putin
and the leadership that he has installed across Russia. But other parties said the elections were
marred by fraud, and they contended that senior officials who are United Russia members illegally
used law enforcement and other government agencies to suppress the opposition.

The elections were seen as a test of the strength of Mr.~Putin's political movement in advance of
parliamentary and presidential elections over the next 18 months. Mr.~Putin, who served two terms as
president before becoming prime minister, has not said whether he will run for president again but
has indicated that he is seriously considering it. If he is not a candidate, his prot\'eg\'e,
President Dmitri A.~Medvedev, is expected to seek another term.

More than 30 million of Russia's 140 million people were eligible to take part in Sunday's regional
elections.

Preliminary results announced early on Monday seemed to offer little evidence that the country's
political dynamic has changed recently, with United Russia typically garnering roughly 45 to 70
percent -- similar to what it won in elections in other regions last March.

Opposition parties made modest gains in the March local elections, but it did not appear that they
were able to build on those on Sunday.

Russia has had several difficult months, with the Kremlin and its regional surrogates facing
criticism over the response to a heat wave and forest fires during the summer. The extreme weather
destroyed crops, pushing up prices for staples like grain and meat.

Two opposition parties -- the Communists and A Just Russia -- had identified Novosibirsk, which is
1,800 miles east of Moscow, as a place to capitalize on this discontent.

Novosibirsk is Russia's third largest city, with a relatively progressive population that some
analysts suggested might be souring on Mr.~Putin and the ruling party.

In fact, according to preliminary results, United Russia, which led Novosibirsk with roughly 45
percent of the vote, was weaker here than in other regions. The Communists received 25 percent, with
A Just Russia garnering 16 percent.

Still, the two opposition parties said that United Russia had used underhanded tactics. ``United
Russia knows its real rating, which is why it has staked everything on monstrous violations,'' said
Oleg Mikheev, who oversaw the campaign of A Just Russia in Novosibirsk. ``It can obtain the desired
results only with the help of deception and forgery.''

Local elections in Russia have regularly been faulted for malfeasance by the ruling party. After
balloting in October 2009, the opposition in the federal Parliament staged a walkout in protest over
the results.

On Sunday night, United Russia said the opposition parties were reaching for excuses to explain
their poor showing. Aleksei Bespalikov, a United Russia leader here, said it was the opposition that
was guilty of breaking the law to win votes.

``I will always remember this campaign because the crude violations by our opponents were simply
unprecedented,'' Mr.~Bespalikov said.

Interviews at polling stations in the Novosibirsk region suggested that many voters based their
decisions primarily on their views of United Russia.

Tamara Shcheglova, 55, who voted for the Communists, said the country was failing to provide the
kind of social services, especially housing, that were available in Soviet times.

``I want our children to have the same advantages that we once had,'' she said.

But Ivan Zuyenko, 80, said he supported United Russia, explaining that the country was much better
off under Mr.~Putin. ``He is an energetic and sharp man -- and most important, he's not at all
lazy,'' Mr.~Zuyenko said.

\section{Carnival Air Fills Chilean Camp as Miners' Rescue Nears}

\lettrine{C}{lowns}\mycalendar{Oct.'10}{11} dance and pass out caramels. The wives and girlfriends
of the 33 trapped miners are picking out sexy lingerie and getting their hair and nails done to
receive their men. And relatives of the miners trapped nearly half a mile underground for more than
two months have learned a new phrase -- ``motor home'' -- from their hundreds of new journalist
friends.

Such is life in Camp Hope, a moonlike outpost that has sprouted up as the temporary refuge for
family members and about 1,300 journalists, many of whom have arrived in recent days to this gold
and copper mine in northern Chile.

``It looks like a circus around here, but it's a good kind of circus,'' said Lilianett G\'omez, 30,
one of four daughters of Mario G\'omez, the oldest of the trapped miners.

After two levels of the mine collapsed on Aug.~5 and contact with the miners was lost, family
members set up camp here, hoisting tents and a statue of San Lorenzo, the patron saint of miners.
For days they barely slept as they waited and prayed for a sign that the men were alive.

On Aug.~22, their prayers were answered with the news that all 33 had managed to make it safely to
an area of about 600 square feet and that they were organizing themselves for what they knew would
be months of waiting.

The government of President Sebasti\'an Pi\~{n}era quickly moved in with more resources, setting up
temporary dwellings for family members. Cellphone and wi-fi connections followed.

Today the camp is dotted with Chilean flags, hanging laundry and posters with messages of support
for the miners. Volunteer clowns roam the dusty roads blowing plastic horns and entertaining miners'
children, who study during the week in a makeshift classroom.

Despite its isolation, there is rarely a quiet moment at the camp. Musicians perform on a small
stage in front of the cafeteria tent. A barbershop quartet sang last Wednesday afternoon. There was
a children's costume party on Sunday morning with boys dressed as Spiderman and girls as witches.

And there seems to be a constant stream of religious services, including a Mass on Sunday afternoon
filmed by some 40 journalists, where Samuel Cerna, a Catholic missionary who drove eight hours to
reach the mine, included them in his motivational sermon.

City workers from Copiap\'o, the nearest city, have provided free hot lunches and coffee for both
families and the journalists. A searing sun burns the skin during daylight hours. But around 5:30
p.m., the temperature plummets, just as the women who work in the cafeteria are handing out hot
chocolate. Soon everyone at Camp Hope is bundling up in heavy jackets and wool caps. Seventy police
officers guard the camp.

The journalists who have invaded Camp Hope -- representing more than 200 foreign media outlets and
50 Chilean ones --record seemingly every move of family members and government officials as they
wait for the moment when the men who have survived for 67 days in a sunless, humid, confined space
deep in the earth are lifted to freedom.

They have come from 33 countries on five continents, from Japan to Hungary to South Africa, a
government spokesman said Sunday.

The government will transmit the rescue of the miners, which is expected to begin on Wednesday and
to take up to two days, via a free, live satellite feed to the world.

More media outlets are covering the fate of the miners than reported on Chile's 8.8-magnitude
earthquake last Feb.~27 , one of the most potent in human history, said Tom\'as Urzúa, a government
spokesman. The quake killed some 300 people and left swaths of rubble in cities and towns across the
country.

For the miners' family members, who are simple, working-class people, what has happened at Camp Hope
has been surreal.

Ms.~G\'omez came here from Iquique, a city about 600 miles to the north, to be closer to her father,
who is 63. At first, she said, she was surprised by the number of journalists who had descended on
Camp Hope.

``I've been interviewed by people from as far away as China,'' she said.

She added: ``Over all, I think they're here to help and transmit our struggle to the world. The
conversations we've had around the fire have served to make us some new friends.''

Chilean officials have tried to play down the importance of the rescue to Chile's image and to that
of Mr.~Pi\~{n}era's government. ``I have no problem with the idea that everybody is watching this,''
Laurence Golborne, the mining minister, who has become the country's most popular minister, said
Sunday. ``We are very focused on rescuing these guys. I don't give too much attention to that.''

Still, the Chilean government, which is spending millions of dollars on the rescue effort, seems
always to be one step ahead of the growing international horde. In the past week earth-moving
vehicles have cleared new parking lots, and new cell towers have sprouted up to ensure that the
media have high-speed Internet and cellphone connections. More portable toilets have appeared. (The
BBC rented four toilets for its team of more than three dozen reporters and technicians, but has
scaled back to two.)

``We are on the surface of the moon here, it's rocks and sand,'' said Jeffrey Kofman, a Latin
America correspondent for ABC News. ``Yet the Chileans understand that in a certain way this media
event, not by design but by circumstance, is really going to paint the world's perception of this
small country.''

As drill operators neared the miners last week, Chilean officials began preparing for the first big
moments since the men were found alive. A poster went up near the entrance to the mine with the
faces of all 33 inside a large star.

On Saturday morning, operators of a portable drilling rig broke through to the miners 2,050 feet
underground. Sirens blared and a city worker rang a bell for an hour at the schoolhouse.

Now, as workers encase the top of the rescue hole with steel tubing, the miners' wives and
girlfriends are determined to look their best.

On Saturday the mother of one miner handed out sexy lingerie that she had received from an anonymous
donor to her son's wife and the wife of another miner. A social worker came to the camp early Sunday
to do manicures and style the women's hair.

Some wives have begun trying to negotiate fees for interviews in what will be a feeding frenzy once
the miners reach the surface.

Chilean officials seem unconcerned.

``We have to respect a lot the freedom that any person has in this country,'' Mr.~Golborne said
Sunday in response to a question about the fee solicitations. ``We will be there to provide
emotional and psychological support, legal support if they require it, but each person should decide
what to do with their own lives.''

\section{Regulations and Security Concerns Hinder Asia's Move to Cloud Computing}

\lettrine{T}{he}\mycalendar{Oct.'10}{11} Youth Olympic Games here in August presented organizers
with a formidable, but temporary, computing hurdle: Manage 3,600 athletes, 20,000 volunteers and
370,000 spectators for two weeks.

Rather than buy or lease the equipment necessary to run the event, organizers rented the required
computer capacity from a data center run by Singapore Telecommunications.

The Games were a showcase for cloud computing in Asia: software, data storage, networking and even
computing equipment on tap -- as much as a customer desired for only as long as needed.

``In past Olympic games, they had to buy these servers,'' said Bill Chang, an executive vice
president at Singapore Telecommunications, ``and then after the Games all this equipment would be
fire-sold away or given away.''

Mr.~Chang said that by using cloud services, customers like the organizers of the Youth Olympic
Games could save 60 percent to 80 percent of the cost of purchasing the equipment themselves.

The research firm IDC estimates that the market for cloud computing in Asia outside Japan will grow
to about \$1.3 billion this year and will continue expanding at a rate of about 40 percent a year
until 2014.

That figure is just a splash in the estimated \$68.3 billion that cloud computing will bring in
globally in 2010, according to the research firm Gartner. And for every case of avid adoption, as in
Singapore, there are other countries where acceptance is hindered by regulations, concerns about
data security and poor Internet connections.

``What will drive adoption is broadband penetration,'' said Emilio Umeoka, president of Microsoft's
Asian operations in Singapore. ``If you don't have the pipe, you can't get onto the cloud.''

For potential customers, the savings from cloud computing are enticing. Like equipment leasing
before it, cloud computing turns what once was a big-ticket capital expenditure into an operating
expense for companies -- one that can be tuned up or down, depending on business conditions. ``The
savings comes in infrastructure,'' Mr.~Umeoka said. ``Your company would have less services, less
people to manage the servers, less people to manage bugs or patches.''

What is giving Asian customers pause, though, is the same concern being voiced elsewhere: that a
company's precious financial and customer data could be lost, stolen or even rendered temporarily
inaccessible through no fault of its own.

To many, the threat posed by hackers, possible government interference and even power failures
justifies keeping their data housed on their own premises, in their own countries.

As a result, cloud computing in Asia is taking off faster among small, more cost-sensitive start-ups
than among larger companies. China's big state-owned businesses, for example, tend not to trust
third-party providers with their data.

``The propensity to outsource in China is the lowest in the region,'' said Philip Carter, a research
director at IDC in Singapore. ``It comes back time and time again in surveys to control and risk
mitigation. They want to keep control of I.T. assets.''

Adoption of cloud computing in China is also held back by regulation. Beijing, which also blocks
Facebook, prohibits companies from storing their data offshore, meaning they can use data centers
only inside China. As a result, IDC estimated, only about 4 percent of companies in China were using
cloud-based services last year, compared with 16 percent in Singapore.

But with about 40 million small and medium-size companies, the Chinese market holds enormous appeal
for the providers of cloud computing services. NEC of Japan estimates the nascent Chinese market
will grow 30 percent a year to reach around \$2.3 billion by 2012.

``The Chinese customers are more receptive to the whole way of doing business online,'' said Enwei
Xie, Microsoft's general manager for software development business in China. ``They're sort of
familiar with this whole host model.''

As is typical in China, local companies like Kingdee and Ufida so far have dominated the market,
crowding out international firms like NEC, SAP of Germany or Oracle of the United States.

Poor bandwidth access, however, is keeping China from becoming a cloud hub. That limitation also
applies to India, and followers of the industry say it is unlikely such services will be outsourced
to either country anytime soon.

And cloud computing providers are flocking to Hong Kong and Singapore. Even India's Tata
Communications is opening a data center in Singapore to offer cross-border cloud services.

IDC estimates that 24 percent of all companies in Asia will use some form of cloud computing this
year, up from 11 percent in 2009.

Technically, Singapore is not the ideal location for data centers. Land is expensive and
temperatures are high, meaning it costs more to keep servers cool. But cutting-edge
telecommunications infrastructure, a pool of technology experts and generous government incentives
more than compensate, executives said.

Microsoft, I.B.M. and Hewlett-Packard have also picked Singapore as the center of their own regional
cloud computing initiatives.

``We didn't set up a lab in Singapore because we wanted to do cloud computing,'' said Christopher
Whitney, managing director of a new lab Hewlett-Packard opened in the city-state in February, ``but
because we wanted to access its infrastructure and talent.''

\section{China's Ban on Selling Rare Earth Minerals to Japan Continues}

\lettrine{C}{hinese}\mycalendar{Oct.'10}{11} customs officials continued to prohibit all exports of
rare earth minerals to Japan over the weekend, industry officials said, but the Chinese government
showed signs of taking a more conciliatory stance toward Japan.

The last one of four employees of a Japanese construction company who were detained on Sept.~23 near
a Chinese military area returned to Japan on Sunday. All four had been taken into custody during a
dispute over a Chinese fishing boat captain whose vessel collided with two Japanese patrol boats on
one night near islands Japan controls but China claims. Japan detained the captain on Sept.~8 and
released him on Sept.~24.

Prime Minister Wen Jiabao of China told European political and business leaders Wednesday that China
had not imposed any bans on exports of industrial minerals for political purposes and that it did
not intend to stop exports in the future.

Rare earth minerals are used in the manufacture of hybrid gasoline-electric cars, computer screens,
large wind turbines and in many other applications.

Mr.~Wen spoke at a China-European Union business meeting in Brussels. Chinese officials have
consistently taken the position that they have not imposed any regulations preventing exportation of
rare earth minerals; any such regulations could be easily challenged at the World Trade
Organization.

The commerce minister, Chen Deming, suggested instead in a television interview on Sept.~26 that
Chinese entrepreneurs in the rare earth industry might have halted shipments because of their own
feelings toward Japan.

Thirty-two companies in China have export licenses for rare earth minerals, and 10 of them are
foreign. Mr.~Deming did not address why the 10 foreign companies would have strong feelings toward
Japan, or why all companies in the Chinese industry halted shipments on the same day, Sept.~21.

Throughout the halt on exports of rare earth minerals, China has allowed exports of manufactured
products that use them, like powerful magnets, and highly purified rare earth metals. Japan is the
largest importer of rare earth minerals and ores. Japanese companies use them to make a wide range
of technology products and have been reluctant to import manufactured goods from China instead.

Even before questions arose over the exports to Japan in late September, China had been tightening
caps on rare earth exports for five years. When the export halt was imposed, the quota for 2010 was
within a month and a half of being exhausted. But shipments could continue into November if customs
officials allow a resumption soon.

\section{Finding Your Way Through the Mall or the Airport, With a Cellphone Map}

\lettrine{M}{obile}\mycalendar{Oct.'10}{11} phone maps have guided people through streets and alleys
around the globe. But when those people step into a sprawling building, they can get lost.

Inside, people have to ask strangers for directions or search for a directory or wall map. A number
of start-up companies are charting the interiors of shopping malls, convention centers and airports
to keep mobile phone users from getting lost as they walk from the food court to the restroom. Some
of their maps might even be able to locate cans of sardines in a sprawling grocery store.

``It was my wife's idea -- she was six months' pregnant and she couldn't find a restroom,'' said Sam
G.~Feuer, chief executive of MindSmack, the New York company behind FastMall, one of the indoor
mapping services. ``It's the same thing for people in wheelchairs or with strollers who need an
elevator.''

Users see a floor plan of a shopping mall, for example, with stores indicated by name. Escalators,
exits, restrooms and elevators are also marked.

FastMall has a search engine to help users find stores on its maps. Enter ``Banana Republic'' and
the service places a pin on the map to show the store's location. Tap the ``take me there'' button
and the service plots a route to the destination. To find the nearest restroom, all users have to do
is shake their phone.

Most of the indoor mapping apps are free, like PointInside, FastMall and Micello, which work on the
iPhone, iPod Touch and iPad. PointInside is also available for many Android phones.

Because mobile Internet connections are sometimes difficult to make indoors, some of the services
download their maps onto users' phones when they first check in on the service. If the connection
later fails, the user still has access to the map.

The various mapping services differ in how they obtain their maps. Some get them from mall
management companies or mall developers. Others use maps that are already available online or they
copy ones posted on mall directories (sometimes by taking photographs of them or by encouraging
their users do so).

In almost all cases, the services have to customize the maps to fit a standard size and font and to
fill in any missing information.

Ankit Agarwal, chief executive of Micello, an indoor mapping service based in Sunnyvale, Calif., has
created a library of nearly 2,000 maps, most of them of American shopping malls. He said his team
could recreate a mall floor plan in a couple of hours, based on originals that they find in the
public domain.

``We never have to visit the place,'' Mr.~Agarwal said. No malls have complained, he added.

Inevitably, maps become outdated as stores close and new ones replace them. Since the mapmakers
cannot possibly keep visiting each one, they rely on users to tell them that a map needs to be
updated.

Dan Jasper, a spokesman for Mall of America in Bloomington, Minn., said his mall -- the biggest in
the United States, with 520 stores -- is working on its own mobile phone app so shoppers will have a
more reliable floor plan. He thinks it is an important tool that could increase sales and traffic.

For now, shopping malls are getting the most attention from indoor mapmakers, although many of them
hope to add casinos, stadiums, universities and hospitals -- anything big enough to get lost in and
that draws big crowds. They are also interested in outdoor destinations like theme parks, zoos and
urban shopping districts. Rodeo Drive in Beverly Hills, Calif., is already available on FastMall,
for instance.

In some cases, Micello's maps show details beyond the basic four walls. A map of the Ikea store in
East Palo Alto, Calif., features an aisle winding through the store and the locations of departments
like ``children's'' and ``closet systems.''

Aisle411, a mobile service that is set to start next month, is hoping to take the detail even
further by allowing users to find individual products inside stores. Shoppers in a grocery store can
search for ``capers,'' for instance, and then get a map to the appropriate aisle.

Nathan M.~Pettyjohn, chief executive of Aisle411, based in St.~Louis, said retailers lose a large
number of customers because shoppers cannot find what they want. The problem is compounded for
big-box retailers, whose vast stores seem built to create frustration.

Aisle411 has worked with a few retail chains and has created maps with their help. For other stores,
the company is relying on publicly available maps, some guess work and volunteers to indicate where
products are located. ``We get about 90 percent accuracy,'' Mr.~Pettyjohn said.

Given the early stages of indoor mapping, its business model is still a work in progress.
Location-based advertising and coupons are one possibility, as is charging malls to create their
floor plans. Some companies have tried licensing maps to other companies. Still others are
considering selling user data to retailers and product manufacturers.

Despite the miniboom in indoor mapping, Vikrant Gandhi, an analyst with Frost \& Sullivan, said the
niche faced challenges. Mobile marketing, the most common idea for making money, has yet to prove
itself, he said.

Google, with its dominant Google Maps, worries some in the industry. Google could crush the tiny
indoor mapmakers by creating its own competing service. But it is just as likely that it could be a
savior by buying one or more companies or licensing their data. A Google spokeswoman declined to
comment.

Mr.~Agarwal, from Micello, said he was just excited by the prospect of all that remained to be
mapped indoors. Speaking about his service last month at a mobile phone conference at the University
of California, San Francisco, he looked out the window and declared, ``I want to map every building
on this campus.''

\section{European Antritrust Deal With Microsoft Barely Affects Browser Market}

\lettrine{W}{hen}\mycalendar{Oct.'10}{11} Europe settled an antitrust case over Web browsers with
Microsoft in December 2009, it hoped to dislodge the world's biggest software maker from its
dominant position in that market by requiring it to offer rivals' products.

As part of that, Microsoft in March started sending software ballot screens to 200 million Windows
users in Europe. The screens ask users to choose a default from a list of 12 browsers including
Internet Explorer, Firefox, Google's Chrome, Opera and Apple's Safari.

Six months into the process, the initiative appears to be having only a minor influence on
consumers, prompting a renewed debate about the effectiveness of such antitrust remedies.

``I'm sure that it is increasing pressure on Microsoft, which has been losing share anyway,'' said
Aodhan Cullen, the chief executive of StatCounter, a research firm in Dublin that tracks browser
use. ``But it hasn't caused a big upheaval in rankings.''

The agreement was intended to check Microsoft's advantage of bundling its browser into Windows, a
practice that a Norwegian browser maker, Opera, challenged in a complaint to the European Commission
in December 2007. While browsers are free, they are the gateway through which companies like
Microsoft, Google, Apple and Mozilla, the owner of Firefox, earn revenue through advertising.

According to StatCounter, Microsoft's leading share of the European browser market fell to 39.8
percent in October from 44.9 percent in January. In 2009, Microsoft's share declined by 5.5
percentage points; in 2008 by 8 points.

Most of the decline has come amid gains by Google, which introduced Chrome in September 2008.
Google's share of the European market doubled this year, to 11.9 percent in October from 5.8 percent
in January.

But it is impossible, browser makers and market researchers say, to isolate the effect of the ballot
from general market trends, and there is no way to measure consumer participation in the balloting
or its direct outcome.

Debate over the effectiveness of European antitrust remedies is not new.

In 2003, the commission ordered Microsoft to sell a version of Windows in Europe without its own
media player, to minimize the company's advantage in bundling its products. But consumers largely
rejected the stripped-down version, which Microsoft sold for the same price as its full Windows
software.

Neelie Kroes, who crafted the ballot screen settlement with Microsoft in December 2009 as the E.U.
competition commissioner, is now commissioner for telecommunications and the digital economy.

Her spokesman directed questions on the ballot screen remedy to Ms.~Kroes' successor as competition
commissioner, Joaquin Almunía.

Amelia Torres, a spokeswoman for Mr.~Almunía, said recent market data suggested that competition
among browser makers was increasing, which was an indication of the ballot screen's success.

``I think one can say that the decline in Internet Explorer share can be interpreted as the ballot
screen bringing about its intended effect,'' Ms.~Torres said.

The commission never planned to dictate the market shares of competitors in the browser market,
Ms.~Torres added. She said the cyber-balloting gave Europeans another option, if they chose, to
select one of several browsers.

Brian Rakowski, director of Chrome product management at Google, said it was impossible to quantify
the ballot screen's effect, but he thought the remedy had raised awareness among Europeans about
what browser choices were available.

``We are also seeing people responding more and more to Chrome, and starting to tell their friends
about it,'' Mr.~Rakowski said.

Achim Sauerberg, an analyst at Statista, a market research firm in Hamburg, said the ballot screen
had only reinforced the status quo among European consumers, many of whom had already chosen a
browser before the commission took its action.

``People who tend to care about browsers had already made this decision,'' Mr.~Sauerberg said. ``So
this was unlikely to dramatically rearrange the market anyway. But I'm sure it is increasing
pressure on Microsoft to defend Internet Explorer.''

Microsoft declined to evaluate the effect of the settlement on its browser presence in Europe. The
company agreed to the browser settlement after fighting and abandoning a decade-long antitrust
battle with the European Commission over the bundling of its media player and the use of
confidential protocols for server software.

Microsoft, which by agreeing to the settlement avoided another round of litigation and potential
fines, is planning to release its latest browser version, Internet Explorer 9, this year.

Jesse Verstraete, a Microsoft spokesman in Brussels, said the company was complying fully with the
conditions of the commission's settlement and had distributed browser ballots to about 200 million
existing and new purchasers of Windows systems in Europe.

The company that initiated the settlement, Opera, has seen little benefit from the balloting screen,
according to StatCounter.

The Oslo-based company's market share has risen to 4.5 percent in October from 4.3 percent in
January.

When Microsoft released the first ballot screens in March, Opera said at the time that it was seeing
an increase of 800,000 downloads each month.

Pal Unanue-Zahl, an Opera spokesman, said the company had no more recent data.

``We think the ballot screen solution is definitely making the market more competitive,'' he said.

Firefox has seen its share in Europe fall slightly, to 38.8 percent in October from 39.9 percent in
January, according to StatCounter.

The market share of Apple's Safari has risen to 4.5 percent in October from 3.7 percent in January.

Ulrich Börger, an antitrust lawyer at Latham \& Watkins in Hamburg, said the market snapshot showed
that the browser wars were more competitive than ever.

``When the commission first began looking into this, the market was in a different situation,''
Mr.~Börger said. ``Now, there appears to be a very healthy competition going on.''

\section{Despite Army Efforts, Soldier Suicides Continue}

\lettrine{A}{t}\mycalendar{Oct.'10}{11} 3:30 a.m. on a Saturday in August, Specialist Armando
G.~Aguilar Jr.~found himself at the end of his short life. He was standing, drunk and weepy, in the
parking lot of a Valero station outside Waco, Tex. He had jumped out of his moving pickup. There was
a police officer talking to him in frantic tones. Specialist Aguilar held a pistol pointed at his
head.

This moment had been a long time coming, his family said. He had twice tried to commit suicide with
pills since returning from a tough tour in Iraq a year earlier, where his job was to drive an
armored vehicle to search for bombs.

Army doctors had put him on medications for depression, insomnia, nightmares and panic attacks.
Specialist Aguilar was seeing an Army therapist every week. But he had been getting worse in the
days before his death, his parents said, seeing shadowy figures that were not there, hallucinating
that he heard loud noises outside his trailer home.

``He wanted help -- he was out there asking for help,'' said his father, Armando Aguilar Sr. ``He
just snapped. He couldn't control what he was doing no more.''

Specialist Aguilar was one of 20 soldiers connected to Fort Hood who are believed to have committed
suicide this year. The Army has confirmed 14 of those, and is completing the official investigations
of six other soldiers who appear to have taken their own lives -- four of them in one week in
September. The deaths have made this the worst year at the sprawling fort since the military began
keeping track in 2003.

The spate of suicides in Texas reflects a chilling reality: nearly 20 months after the Army began
strengthening its suicide prevention program and working to remove the stigma attached to seeking
psychological counseling, the suicide rate among active service members remains high and shows
little sign of improvement. Through August, at least 125 active members of the Army had ended their
own lives, exceeding the morbid pace of last year, when there were a record 162 suicides.

``If the test for success is our numbers and our rate, then clearly we have not been successful,''
said Col.~Chris Philbrick, deputy director of a special task force established to reduce suicides.

Colonel Philbrick said that more soldiers were seeking help for psychological problems than ever
before -- it was the leading reason for hospitalization in the military last year -- but that the
number needing help had also grown at a rapid pace, a natural consequence of nine years of combat
deployments. So even though the Army now has 3,800 therapists and psychiatrists, two-thirds more
than it did three years ago, there is still a significant shortage, he said.

Advocates for veterans say the shortage of therapists means that Army doctors tend to rely more on
medication than therapy. They also say the Army screens too few soldiers for mental problems after
deployments, placing the burden on the soldier to seek help rather than on officers to actively find
the damaged psyches in their corps.

``The military still blames the soldier, saying it's financial stress or family stress, and it is
still waiting for the service member to come forward,'' said Paul Sullivan, the executive director
of Veterans for Common Sense.

In July, Gen.~Peter W.~Chiarelli, the vice chief of staff of the Army, ordered that all soldiers
returning from combat be evaluated by a mental health professional, either face to face or by video
conference.

General Chiarelli and other top commanders have argued that the roots of the rise in military
suicides are complex and that blame cannot be laid solely on repeated deployments. The majority of
soldiers who have committed suicide -- about 80 percent -- have had only one deployment or none at
all. Another factor is that after years of war, the Army is now attracting recruits already inclined
toward risky behavior and thus more prone to suicide, according to a 15-month Army review of
suicides released in July.

A close examination of some of the suicides at Fort Hood shows that they are as individual as
fingerprints. Some of the soldiers who committed suicide were receiving treatment but took their
lives anyway. Others were reluctant to seek help for fear of being labeled cowards or malingerers.

The commanders at the base have tried hard to change the never-show-weakness culture of the Army.
They have trained more than 700 noncommissioned officers and chaplains to spot suicidal soldiers and
refer them to counselors. Since April, more than 17,000 soldiers have participated in an exercise in
which actors play out scenarios involving suicidal people.

Beyond the role playing, the base's commanders have also employed a comedian who talks about the
suicide of his brother and have compelled all soldiers to watch two training films about suicide --
``Shoulder to Shoulder'' and ``I Will Never Quit on Life.'' A former commander of the base,
Lt.~Gen.~Rick Lynch, even established a holistic ``Resiliency Campus,'' where soldiers can do things
like take tai chi and yoga classes, get massages or see family counselors.

One of the those who received the training was Josh Roum, a recently retired sergeant who still
works on the base. On Sept.~25, Mr.~Roum came home after spending the night with a friend and found
his new roommate, Sgt. Timothy Ryan Rinella, sprawled in a hallway. A veteran of four deployments in
Iraq and Afghanistan, Sergeant Rinella, who was 29, had shot himself in the head with Mr.~Roum's
pistol.

Mr.~Roum said Sergeant Rinella showed none of the classic signs of being suicidal. He had talked
about plans to rebuild an old car and had bragged about his four children. ``He seemed like he had a
level head on his shoulders,'' Mr.~Roum said. ``It was like a total shocker to me.''

Sergeant Rinella had sought counseling from an Army psychologist for panic attacks, even though he
feared that a diagnosis of post-traumatic stress disorder would ruin his career, his wife, Sarah
Rinella, said via e-mail. He believed that the counselors would relay what he said in private up the
chain of command.

The deployments had been hard on the marriage, Ms.~Rinella said. Sergeant Rinella left for Iraq just
13 days after their wedding and spent more than half of their eight-year marriage abroad. He missed
the birth of his first son.

``How are you supposed to have a family life with that many deployments?'' Mr.~Roum asked.

The day Sergeant Rinella killed himself, he was facing another separation from his family, Mr.~Roum
said. A few days earlier, his wife had moved with the children to Richmond, Va., to take care of an
ailing parent.

Several of the soldiers who recently committed suicide had faced marital problems. Sgt. First Class
Eugene E.~Giger was informed by his wife, Yolanda Giger, last October that she was filing for
divorce. He was still in Iraq, doing his third tour abroad in six years.

``It did cause a chasm between the two of them,'' said Helen Giger, the sergeant's mother. ``He felt
bad because his kids were growing up without him being there.''

On June 15, two days after his 43rd birthday, Sergeant Giger hanged himself in his apartment in
Killeen. His parents said their son had always been a taciturn man and had given close relatives no
hint that he was depressed. Nor did he mention panic attacks, nightmares or other emotional
disorders.

But his former wife said in a telephone interview that she had tried to warn his superiors that he
was deeply troubled after his deployments, despite having served most of his time abroad in a
relatively safe desk job as a personnel officer. ``I made outcries to everyone, and no one
listened,'' she said, declining to say to whom she had spoken for fear of losing Army benefits.

Others did seek help. Master Sgt. Baldemar Gonzalez, 39, an airborne combat veteran in the Persian
Gulf and Iraq, began seeing a therapist at Fort Hood a year ago and received a diagnosis of
post-traumatic stress disorder. He suffered nightmares, insomnia and flashbacks, said his wife,
Christina Barrientes. He had planned to go to school for engineering on the G.I. Bill when his
enlistment ended this year.

Army psychiatrists prescribed antidepressants, sleeping pills and a tranquilizer -- a cocktail of
five drugs, she said. He started taking them in mid-March, and his personality changed. Always an
athletic, outgoing man, he became listless and quiet, sleeping much of the day and avoiding his
friends.

On Sept.~25, he dropped his daughter off at a football game at her high school, then returned home
and told his wife he was going to work on some homework in the kitchen. She found him upstairs later
in the day, dead in their bedroom closet, having apparently hanged himself.

``I blame the medication,'' Ms.~Barrientes said. ``You go and try to get help and all they do is put
you on medication.''

Specialist Aguilar, who shot himself in the Valero parking lot, also had plans to go back to school
once his enlistment was over in November. A guitarist and songwriter from a working-class family, he
had enlisted partly because the Army would pay for music school after his military service, his
family and friends said.

But by the time he came back from Iraq in July 2009, his nerves were on edge, his relatives said. He
had become severely depressed in Iraq after a good friend in his unit, Pvt. Eugene Kanakole, shot
and killed himself in a latrine. Specialist Aguilar had also lived through several terrifying
explosions -- including five in one night -- while working as a combat engineer, friends said. Army
doctors prescribed antidepressants even before he left the war zone.

Once he returned to Texas, he frequently drank too much and twice nearly overdosed on sleeping pills
and painkillers. His medical records show that Army psychiatrists diagnosed post-traumatic stress
disorder and depression, and he had been seeing a therapist every week for months. At the time of
his death, he was taking two antidepressants, a sleeping pill and a drug used to treat alcoholics.

He married a young woman from Waco early this year, but the elation of the honeymoon did not last
long. They fought often over his behavior, and they began seeing a marriage counselor within months,
friends said. The night before he died he drove to Waco to visit his in-laws. He took a gun
belonging to a close friend who was out of town.

The police say Specialist Aguilar ran into the yard of his father-in-law's house and fired shots
around 3:15 a.m., then took off in his pickup truck. His mother, Amelia Aguilar, said his wife had
told her they had been drinking and had a fight.

A local police officer spotted Specialist Aguilar's truck and followed him to the Valero station in
Hewitt, just south of Waco. He tumbled out of the moving truck, letting the vehicle roll into a
ditch, then put the gun to his head. The officer, Sgt. Robert Dillard, drew his own gun and started
to plead with Specialist Aguilar not to fire, according to the police report.

He pulled the trigger anyway. He was 26 years old.

\section{Specialists' Help at Court Can Come With a Catch}

\lettrine{H}{umberto}\mycalendar{Oct.'10}{11} Fernandez-Vargas, deported to Mexico, had run out of
options. A federal appeals court said he could not return to the United States to live with his
American wife and son. And his lawyer did not have the expertise or money to pursue the case
further.

Then the cavalry arrived. Leading lawyers from around the country, sensing that the case was one of
the rare ones that might reach the Supreme Court, called to offer free help. Mr.~Fernandez-Vargas's
immigration lawyer was delighted, and he chose a lawyer from a prominent firm here.

But there was a catch, and then a controversy. The catch was that the Washington lawyer, David
M.~Gossett, would take the case only if he could argue before the Supreme Court himself.

The controversy was that groups representing immigrants were furious, suspicious of the new lawyer's
interest in the case and fearful of a Supreme Court ruling that would curtail the rights of
immigrants nationwide.

Indeed, Mr.~Gossett faced a barrage of hostile questions from the justices, and in June 2006 the
court ruled against his client, 8 to 1. The ruling wiped out decisions in much of the nation --
notably from the federal appeals court in California -- that had favored immigrants.

Mr.~Gossett is among an increasingly influential cadre of lawyers specializing in Supreme Court
cases, attracted to the importance and intellectual challenge of the work. Many are willing to serve
without charge to draw prestige and paying clients to their firms.

Thirty years ago, 6 percent of cases accepted by the court were brought by lawyers specializing in
Supreme Court advocacy, according to data compiled by Richard J.~Lazarus, a law professor at the
Georgetown University Law Center and faculty director of its Supreme Court Institute.

In the term that ended in 2008, the number topped 50 percent for the first time. Many of the cases
involved businesses that paid large fees, but a good number were for clients who could not afford a
lawyer and have historically been represented, if at all, by public interest lawyers.

Overburdened public interest lawyers might be expected to welcome the high-powered help. But the old
guard is often wary of, if not hostile toward, the new breed of skilled and ambitious advocates,
fearing that they are more interested in the glory of a Supreme Court argument than in what is best
for their clients and the development of the law.

``There's one and only one reason they're interested,'' Barry A.~Schwartz, a criminal defense lawyer
in Denver, said of many of the dozen or so lawyers who called him after he had won an appeals court
decision on an issue that was likely to reach the Supreme Court. ``It's not because they love your
client or believe in the legal principle your case presents. They want to get the case into the
Supreme Court.''

A related development has only heightened the competition. Starting in 2004, leading law schools
created student clinics focused solely on the court, all affiliated with prominent law firms. The
result has been an intense struggle to find and press cases with Supreme Court potential.

Law firms trying to build or maintain credible Supreme Court practices must show paying clients that
they have a consistent presence in the court, and law school clinics have to provide work to the
students. Separately and in combination, firms and clinics offer their services free to poor clients
in exchange for taking a lead role in the case. That means people with legal troubles but no money
can end up with a top-tier lawyer with deep experience before the court.

The specialists respond to critics by pointing to data showing that their record in the court is
quite good. They say their ethical obligations are to the individuals they represent and not to the
agendas of interest groups.

But public interest lawyers say firms and clinics have sometimes pushed the wrong cases. ``It's not
in every client's interest to be in the Supreme Court,'' said John H.~Blume, a law professor at
Cornell and the director of its Death Penalty Project, which conducts research and allows students
to work on capital cases at all levels of the court system. ``If you're in the get-to-the-podium
business, it could compromise your judgment.''

Mr.~Gossett, a partner with the Mayer Brown law firm, said his case was not an example of skewed
incentives. ``We devoted hundreds of hours, and maybe thousands, and I continue to believe we did so
to the best of our ability,'' he said. Of the decisions the firm faced, he said, ``I think we made
every one right.''

He added that his obligations were to his client. Mr.~Fernandez-Vargas was deported after two
decades in the United States, separated from his family. ``I gave us a 30 percent chance of
winning,'' Mr.~Gossett said. ``I therefore had a client who had a 30 percent chance of being
reunited with his wife and child in the United States.''

J.~Christopher Keen, Mr.~Fernandez-Vargas's original lawyer, also dismissed the widespread criticism
from immigrant rights groups, saying they were not sufficiently alert to the human stakes in the
case.

Mr.~Keen attended the Supreme Court argument with his client's wife, Rita Fernandez, and he consoled
her after she heard cutting remarks about her husband from Justice Antonin Scalia. ``I was on the
front steps of the Supreme Court while she cried after Scalia called him `a two-time loser,' ''
Mr.~Keen said.

Still, he said, ``if you're thinking strategically, maybe it wasn't the best case.''

\emph{A Loss Leader}

As recently as 1987, Chief Justice William H.~Rehnquist remarked that the days of a few great
advocates appearing regularly before the Supreme Court were gone. ``There is no such Supreme Court
bar at the present time,'' he said.

A little more than two decades later, the fanciest firms not only represent corporations for
handsome fees but also fight to offer their services free to poor clients in exchange for a
half-hour at the Supreme Court lectern. Handling cases pro bono has turned into a loss leader and
reputation enhancer.

Lots of lawyers are members of the Supreme Court bar in a nominal way. All it takes is a check for
\$200, three years as a lawyer, being a member of a state bar and securing the sponsorship of two
lawyers who can already practice in the court. More than 250,000 lawyers have been members of the
Supreme Court bar over the years, and almost none will ever argue before the court.

Indeed, the court has said that trumpeting the mere admission to its bar as significant ``could be
misleading to the general public'' and is ``at least bad taste.''

The specialists are another matter.

Professor Lazarus developed a definition of Supreme Court expertise that is now in general use. A
Supreme Court specialist, he says, is one who has argued five cases in the court or is affiliated
with a practice whose current members have argued at least 10. He concedes ``some inexactitude at
the margins'' but says the overall trends showing the quite recent dominance of the docket by this
group are undeniable.

A pioneer in this area was Thomas C.~Goldstein, now with Akin Gump Strauss Hauer \& Feld. A decade
ago, he developed a technique to spot the cases most likely to attract the court's attention --
those in which two or more federal appeals courts had decided the same legal issue differently. Then
he offered to handle such cases free, cold-calling surprised and often delighted local lawyers. In
his first four years, he argued eight cases in the court, winning four.

Mr.~Goldstein's competitors sneered.

``If I'm going to have heart bypass surgery, I wouldn't go to the surgeon who calls me up,'' Chief
Justice John G.~Roberts Jr., who argued 39 cases before the court before joining it in 2005, told
The American Lawyer in 2000. ``I'd look for the guy who's too busy for that.''

Four years later, Chief Justice Roberts, by then an appeals court judge, said the rise of the
specialized Supreme Court bar had ``had something of a snowball effect.''

``If one side hires a Supreme Court specialist to present a case,'' he told the Supreme Court
Historical Society in 2004, ``it may cause the client on the other side to think that they ought to
consider doing that as well. This is just a variant on the old adage that one lawyer in town will
starve, but two will prosper.''

The first law school Supreme Court clinic, at Stanford, started in 2004. It has been joined by
similar clinics at Harvard, Northwestern, Pennsylvania, Texas, Virginia and Yale.

\emph{Experts Versus Amateurs}

The new specialists have not only prospered but have also compiled an impressive record. During the
last six years, they have won much more often than relative amateurs, according to statistics
compiled by Jeffrey L.~Fisher, a director of the Stanford clinic, for a coming article. (He used
essentially the same definition of experts as Professor Lazarus and considered the 292 cases in
which one of the parties was an individual who was, or could have been, represented by the clinic.
He thus excluded cases involving, say, disputes between two corporations or two states.)

As a general matter, the justices are more likely to reverse than affirm the decisions they review.
Professor Fisher explored the gaps between experts and amateurs both when they represented
individuals as petitioners -- the parties bringing the appeals, and so the likely winners -- and
when they represented individuals on the other side, known as respondents.

Experts not affiliated with the clinic who brought appeals won 67 percent of the time, compared with
46 percent for others. When representing those responding to appeals, they won 29 percent of the
time, versus 16 percent.

The Stanford clinic has generally been even more successful. In the 35 cases it has argued before
the court, it, too, won 67 percent of the time when representing petitioners but did substantially
better than expert counsel in representing respondents, winning 46 percent of the time.

Winning on the merits is not the only way to measure success. The Supreme Court agrees to hear about
one in a hundred petitions seeking review. In the Stanford clinic's first year, the court agreed to
hear its first four petitions.

``Not so long ago,'' Professor Lazarus wrote in The Georgetown Law Journal in 2008, ``only the most
accomplished private sector Supreme Court advocates could claim to have persuaded the court to grant
four petitions in an entire career.''

The clinic, which includes a dozen second- and third-year students who take no other classes for
three months and often work 60 to 80 hours a week, has handled more than 100 Supreme Court cases in
a little over six years. Its success in persuading the court to hear cases has dropped, but only to
42 percent, which is almost certainly better than any practice in the nation except for the
solicitor general's office, which represents the federal government.

Other clinics are now in the hunt, too. Indeed, the last term included the first case with lawyers
affiliated with clinics on both sides. It was an international custody dispute, and the side that
included the Stanford clinic prevailed over the one that included the clinic at the University of
Pennsylvania.

But like the pro bono lawyers specializing in the Supreme Court, the clinics have sometimes found
themselves at odds with public interest lawyers focused on the careful development of the law.
Lawyers concerned with criminal defense, capital punishment, immigration and civil liberties often
try to keep cases out of the Supreme Court, which they view as hostile.

When the court agreed to hear a case on the exclusionary rule at the urging of the Stanford clinic
in 2008, for instance, criminal defense lawyers groaned, saying nothing good would come of it. When
a 5-to-4 decision was handed down last year, it cut back significantly on the rule, which requires
the suppression of some evidence obtained through police misconduct.

In an article to be published next year in The New York University Law Review, Nancy Morawetz, a
clinical law professor at N.Y.U., explored what she calls the ``distorted incentives in Supreme
Court pro bono practice.''

``The competition for cases that may be heard by the Supreme Court on the merits,'' she wrote,
``creates a disincentive to the new Supreme Court bar to engage in full case analysis prior to
accepting a case for representation.''

There is nothing new or exceptional about tension between lawyers seeking to move the law in a
particular direction and those focused on representing a particular client.

Not everyone agreed with Thurgood Marshall's incremental litigation strategy in the battle to
desegregate public education. Many gay rights advocates opposed the lawsuit brought by David Boies
and Theodore B.~Olson challenging California's ban on same-sex marriage, fearing that it would
result in a negative decision if the issue reached the Supreme Court too soon.

Nor is there anything unusual about conflicts between the interests of lawyers and their clients.
But all of these tensions are heightened in Supreme Court practice.

``It's more pointed and more obvious,'' said Pamela S.~Karlan, who directs the Stanford clinic with
Professor Fisher, ``because there are so few opportunities and the competition is so fierce.''

\emph{A Frustrating Lesson}

Distorted incentives were in play in the case of Mr.~Fernandez-Vargas, Professor Morawetz wrote,
calling the case ``a model of what not to do.''

Mr.~Fernandez-Vargas was a Mexican citizen who re-entered the country illegally in 1982 after being
deported. When he applied for permanent resident status more than two decades later, he was arrested
and deported again.

The question in the case was whether a 1996 law denying relief to immigrants who had re-entered the
country illegally applied to him even though he had come back before the law was enacted.

In 2005, the United States Court of Appeals for the 10th Circuit, in Denver, ruled against
Mr.~Fernandez-Vargas, saying he was covered by the 1996 law. That decision was at odds with ones
from the United States Court of Appeals for the Ninth Circuit, which hears cases from much of the
West and so has a heavy immigration docket.

The Supreme Court sided with the 10th Circuit, effectively overturning the Ninth Circuit's rulings.
Only Justice John Paul Stevens dissented.

Justice Stevens's opinion was notable, said Mr.~Gossett, who had argued the case for
Mr.~Fernandez-Vargas. ``The fact that Justice Stevens's dissent tracked our argument has always
reassured me that we framed the case correctly,'' he said. ``Unfortunately, the rest of the court
disagreed not only with us but also with Justice Stevens.''

Immigration lawyers say they learned a lesson from the experience. ``Fernandez-Vargas was a turning
point,'' said Trina Realmuto of the National Immigration Project of the National Lawyers Guild.
``Immigration advocates realized we needed to be proactive about working together with the pro bono
bar.''

She added, ``Noncitizens with even more compelling cases are still living with the consequences of
the Fernandez-Vargas decision.''

Mr.~Gossett said Professor Morawetz had chosen the wrong example in focusing on his case. ``But
there is some truth,'' he said, ``to her point that the specialized Supreme Court bar is in it for
the Supreme Court work.''

\section{Republicans Poised to Make Gains in Races for Governor}

\lettrine{R}{epublicans}\mycalendar{Oct.'10}{11} are well-positioned to pick up a substantial number
of governor's seats in this year's election, with potentially far-reaching effects on issues like
the new health care law, Congressional redistricting and presidential politics.

Democrats hold 26 governorships, and Republicans 24. But the balance appears likely to shift,
perhaps markedly, with Republicans holding the upper hand in many of this year's 37 races, including
those in crucial political battlegrounds.

In Iowa, Michigan and Pennsylvania -- all states that have Democratic governors -- Republicans have
pulled away from their opponents. The Republican Party is also increasing its investment in
Democratic-leaning states like Illinois, Massachusetts and Oregon, seeing opportunities for pickups
there as well.

Democrats are in tough fights in Minnesota, Ohio and Wisconsin, hoping to block a Republican sweep
of the Midwest. And they are intensifying efforts in California, Florida and Georgia, all tossup
races in states that have Republican governors.

Much of the focus has been on Republican efforts to win control of Congress, but a wave of
Republican victories in the governor's races could have just as significant, and potentially longer
lasting, implications.

Many of the governors who are in office over the next two years will preside over the redrawing of
Congressional and legislative districts based on the 2010 census, giving them considerable influence
over the political map for the next decade. In at least 36 states, governors have a say in how
Congressional districts will be redrawn. And governors in at least 39 states have a role in shaping
state legislative maps.

Having Republicans in office in big battleground states could also complicate President Obama's
re-election campaign, which could find it that much harder to hang on to crucial swing states like
Ohio and Pennsylvania. And Republican governors would be a force in how Obama administration
initiatives in health care, education, economic stimulus and other issues are installed on the state
level, intensifying partisan policy battles.

``Governor's races are important every cycle, but this year they are arguably more important to the
future of Congress than the Congressional races themselves,'' said Gov.~Jack Markell of Delaware,
chairman of the Democratic Governors Association.

The contests have attracted several business executives, many of whom are investing tens of millions
of dollars from their own fortunes into their campaigns, which means voters are seeing dozens of
television ads each day from Republicans like Meg Whitman in California and Rick Scott in Florida.

The governor's races will also produce the next class of leaders in both parties, with the
Republican field more diverse. In South Carolina, voters could elect the first Indian-American
female governor in the country, while New Mexico could get its first Hispanic female governor.

``They want someone who isn't part of the status quo,'' said Susana Martinez, a Republican district
attorney who has campaigned for governor in New Mexico on pledges to secure the border by ending the
practice of allowing illegal immigrants to receive driver's licenses, to end a pattern of corruption
and to improve the education system.

In state after state, Republican candidates are criticizing the health care law and the stimulus
plan, even as they grapple with spending cuts, overburdened pension systems and unemployment. The
anxieties are being translated into a broader feeling from voters: a call for change not only in
Washington but also in state capitals.

The answer to that demand in several states, Democrats say, is a Democrat, not a Republican.

Alex Sink, the chief financial officer of Florida and the Democratic nominee for governor, is locked
in a fierce fight with Mr.~Scott, a Republican who has never sought public office. But even as her
opponent has tried to link her to Mr.~Obama, Ms.~Sink has reminded voters that Florida has been
governed by Republicans for the last decade.

``I am the person who is representing a change in outlook,'' Ms.~Sink said. ``But a governor can't
be a parrot of what's going on in Washington. A governor has to stick up for her state.''

Three weeks before Election Day, Democrats are bracing for deep losses. But party officials are
already pointing to races in several states, particularly California, Florida and Texas, where
Democratic nominees for governor have run stronger than candidates for the House or the Senate, a
sign, they say, that the party has not entirely collapsed.

But Republican candidates for governor are benefiting from the same climate that has put the party
in position to win control of the House and make gains in the Senate.

In Oregon, it has been almost 24 years since a Republican was governor. Chris Dudley, a Republican
who has never held office, is in a close race with John A.~Kitzhaber, a Democrat who served as
governor from 1995 to 2003.

``People are tired of the same faces,'' said Mr.~Dudley, a 6-foot-11 former member of the Portland
Trail Blazers. ``Trust in government is at a historic low. There's a lack of connection between
those in the capital and those across the state.''

Third-party candidates are having a pronounced effect on governor's races, one more sign of upheaval
and distress in the minds of voters. In at least five states -- Colorado, Maine, Massachusetts,
Minnesota and Rhode Island -- third-party candidates could be on track to win 10 percent of the vote
or more.

Eric J.~Ostermeier of the Humphrey Institute of Public Affairs at the University of Minnesota said
it was the highest number of states with prominent third-party candidates for governor since the
Great Depression, aside from the tumultuous season of 1994.

``I think there are Minnesotans and Americans who are saying, `Yes, we are angry, but now we need to
do something about it,' '' said Tom Horner, the Independence Party candidate in Minnesota, whose
support was 18 percent in a recent poll. ``There is a radical center ready to emerge.''

Mr.~Horner is in a field with Mark Dayton, a Democrat and former senator who wants to raise taxes
for the wealthiest, and Tom Emmer, a Republican who has pledged to cut the size of government and
spending in some areas to ease a projected \$6 billion state deficit in the next two years.

``In 2010, there are a lot of Minnesotans who are saying we need that kind of reform,'' said
Mr.~Horner, a former public relations executive and political analyst. ``This time we can't look to
the Democrats or the Republicans.''

In Michigan, where recession only worsened already difficult economic conditions, Rick Snyder, a
Republican and the former head of Gateway Inc., has broken away from his Democratic opponent, Virg
Bernero. Mr.~Snyder, who has promised to ``reinvent Michigan,'' is one of several first-time
nontraditional Republican political candidates.

Gov.~Jennifer M.~Granholm of Michigan, a Democrat who cannot seek re-election because of term
limits, said it was ironic that the Democratic agenda might be rejected. She suggested that
Republicans' advocacy of cuts -- to education, optional Medicaid coverage and prisons, for instance
-- would only leave people feeling less secure.

``What I think many people are suffering from is a case of acute impatience,'' Ms.~Granholm said.
``They want a quick fix. And they can't stand the uncertainty that they find themselves in today.
And they figure that any change is better than the status quo.''

\section{In Pennsylvania, Anger Fuels a Race for Senate}

\lettrine{T}{he}\mycalendar{Oct.'10}{11} voters of Pennsylvania are angry at Washington. And Pat
Toomey, the Republican nominee for United States Senate, is glad to stoke the fury.

``The policies that we are seeing coming out of Washington are preventing us from having the kind of
recovery that we could be having and should be having,'' Mr.~Toomey declared after touring the
ironworks shop of a spiral staircase manufacturer here.

His speech, variations of which he offers up at stops across the state, quickly turned into a punch
list of supposed misdeeds he hopes to address, perpetrated by the Democrats controlling Congress
and, by extension, his opponent, Representative Joe Sestak.

Bailouts: ``Taking resources from the productive companies like this one, productive workers like
you, and sending them to failing companies.''

The economic stimulus: ``Over a trillion dollars of growing government, that's what it is.''

The health care bill: ``Another huge obstacle to job creation.''

Climate change legislation: ``Designed precisely to drive energy costs through the roof.''

Some of his assertions are fair. Some are a stretch but defensible. Others are false, like
Mr.~Toomey's claim that a recent law passed by Congress to aid small businesses would have the
federal government buy a stake in local banks. The bill would channel government-backed loans
through the banks.

Details, however, are beside the point. Anger is defining this race.

``Our employees have taken a 10 percent drop in pay and we, as owners, have taken a 50 percent
cut,'' said Ron Cohen, the chief executive of the staircase company, the Iron Shop. ``I don't see
Washington doing anything similar to that.'' He added, ``All I hear are more taxes and more
regulations that make staying in business harder.''

These sorts of frustrations are lifting the candidacy of Mr.~Toomey, 48, in his race against
Mr.~Sestak, a former vice admiral in the Navy who defied the Obama White House by challenging the
five-term incumbent, Senator Arlen Specter, in the Democratic primary.

It was Mr.~Toomey who set in motion Mr.~Specter's political demise.

Mr.~Toomey, a former representative, is an unwavering fiscal conservative who supported the Bush tax
cuts of 2001 and 2003, backed using private retirement accounts in place of Social Security and
opposed the Medicare prescription drug benefit because it cost too much. He nearly ousted
Mr.~Specter in the 2004 Senate primary.

Mr.~Toomey's determination to win the nomination this year led Mr.~Specter to become a Democrat
after 44 years in the Republican Party.

Mr.~Toomey grew up in Rhode Island, attended Harvard, worked as a derivatives trader on Wall Street
and then owned, along with his brothers, two nightclubs and a sports bar in Allentown, Pa.

He served in the House from 1999 to 2004, kept his promise to retire after three terms, and then
became president of the Club for Growth, a nonprofit group favoring low taxes and limited
government.

Mr.~Toomey could hardly have hoped for a better political climate for his antitax, small government
message. For months, polls have shown him in the lead.

``These guys in Washington are creating an environment that is having a chilling effect on small
businesses and medium and big businesses, as well, and that's a big part of why we don't have the
job growth that we badly need,'' Mr.~Toomey said in his visit to the ironworks shop.

``They are creating a staggering amount of debt, debt that is eventually going to have much higher
interest rates, probably lead to very high inflation,'' he continued. ``It's a very worrisome
combination, and we have got to get on a new track.''

What that new track would be was harder to explain.

Mr.~Toomey says he favors making the Bush-era tax cuts permanent for all Americans -- which would
add \$700 billion more to the deficit over 10 years than the plan advocated by President Obama to
let the lower rates expire for the rich. But he also expresses a desire to reduce the deficit.

At the ironworks shop, Mr.~Toomey brushed aside a question from a local reporter who pointed out
that real income for American workers dropped after the Bush tax cuts, saying he did not believe the
data.

Mr.~Toomey's own prescription for the economy includes repealing laws approved by the Democrats,
blocking other Democratic goals and, of course, extending the tax cuts. ``We need to eliminate all
these threats of downside uncertainty,'' he said.

He also said that Washington should cut spending, though he offered no specifics.

In Pennsylvania, as elsewhere, any connection to Washington is a liability. Mr.~Sestak, who is still
serving in the House, refers to his opponent as ``Congressman Toomey'' at every opportunity.
Mr.~Sestak's own campaign prefers to call him ``Admiral Sestak.''

Mr.~Sestak graduated from the United States Naval Academy and, over a 31-year career, ascended to
the rank of vice admiral. He earned a doctorate in political economy from Harvard and commanded a
battle group in support of the wars in Iraq and Afghanistan.

In his campaign, Mr.~Sestak emphasizes both his military services and his independence. He is also
not backing down from his votes in favor of the major Democratic legislation. He uses his young
daughter's battle with brain cancer to help explain why he supported the health care law.

In speeches, he talks in dramatic tones about his Navy service, in particular the days after the
Sept.~11 attacks. But while skeptics typically preface their questions or criticism by saying they
admire his military service, Mr.~Sestak is clearly the underdog.

After a 40-minute speech by Mr.~Sestak to the Rotary Club in York, Webster McCormack, 85, did not
waste a moment before rendering a verdict: he will vote for Mr.~Toomey.

``They have to cut taxes,'' Mr.~McCormack said. ``You keep the government out of our pockets and we
can do a better job of running our businesses.''

During the question-and-answer session at the Rotary speech, Scott Wagner, the owner of a local
waste management company, stood up with a printout of the entire health care law and offered it to
Mr.~Sestak. ``Sir, I don't need it,'' Mr.~Sestak shot back. ``I've read every word.''

But Mr.~Wagner, who raises money for Republican candidates, was not done. ``You see this?'' he said.
``This is not good for business.''

In an interview, Mr.~Wagner expressed deep anger at the government. ``We're being regulated to
death,'' he said. But when asked for examples, he offered only I.R.S. Form 2290, which is used to
file the ``heavy use vehicle tax'' -- a tax enacted in 1982 -- and the I-9 immigration form to prove
an employee is legally allowed to work in the United States, which has been required since 1986.

Mr.~Wagner said there was no doubt about how he would vote. ``I'm supporting Pat Toomey,'' he said.
``We're on the Titanic,'' Mr.~Wagner added. ``The Titanic is going down and the music is playing
fast.''

\section{Marijuana, Once Divisive, Brings Some Families Closer}

\lettrine{T}{o}\mycalendar{Oct.'10}{11} the rites of middle-age passage, some families are adding
another: buying marijuana for aging parents.

Bryan, 46, a writer who lives in Illinois, began supplying his parents about five years ago, after
he told them about his own marijuana use. When he was growing up, he said, his parents were very
strict about illegal drugs.

``We would have grounded him,'' said his mother, who is 72.

But with age and the growing acceptance of medical marijuana, his parents were curious. His father
had a heart ailment, his mother had dizzy spells and nausea, and both were worried about Alzheimer's
disease and cancer. They looked at some research and decided marijuana was worth a try.

Bryan, who like others interviewed for this article declined to use his full name for legal reasons,
began making them brownies and ginger snaps laced with the drug. Illinois does not allow medical use
of marijuana, though 14 states and the District of Columbia do. At their age, his mother said, they
were not concerned about it leading to harder drugs, which had been one of their worries with Bryan.

``We have concerns about the law, but I would not go back to not taking the cookie and going through
what I went through,'' she said, adding that her dizzy spells and nausea had receded. ``Of course,
if they catch me, I'll have to quit taking it.''

This family's story is still a rare one. Less than 1 percent of people 65 and over said they had
smoked marijuana in the last year, according to a 2009 survey by the federal Substance Abuse and
Mental Health Services Administration. But as the generation that embraced marijuana as teenagers
passes into middle age, doctors expect to see more marijuana use by their elderly patients.

``I think use of medical marijuana in older people is going to be much greater in the future,'' said
Dan G.~Blazer, a professor of geriatric psychology at Duke University who has studied drug use and
abuse among older people.

The rate for people ages 50 to 65 who said they smoke marijuana was nearly 4 percent -- about six
times as high as the 65-and-over crowd -- suggesting that they were more likely to continue whatever
patterns of drug use they had established in their younger years. In both age groups, the rate of
marijuana abuse was very low, about 1 in 800.

Cannabinoids, the active agents in marijuana, have shown promise as pain relievers, especially for
pain arising from nerve damage, said Dr.~Seddon R.~Savage, a pain specialist and president of the
American Pain Society, a medical professionals' group.

Two cannabinoid prescription drugs are approved for use in this country, but only to treat nausea or
appetite loss. And while preliminary research suggests that cannabinoids may help in fighting cancer
and reducing spasms in people with multiple sclerosis or Parkinson's disease, the results have been
mixed.

Dr.~Savage said doctors should be concerned about older patients using marijuana. ``It's putting
people at risk of falls, impaired cognition, impaired memory, loss of motor control,'' she said.
``Beside the legal aspects, it's unsupervised use of a pretty potent drug. Under almost all
circumstances, there are alternatives that are just as effective.''

Dr.~Savage added, however, that there was a considerable range of opinions about marijuana use among
pain specialists, and that many favored it.

Older people may face special risks with marijuana, in part because of the secrecy that surrounds
illegal drug use, said Dr.~William Dale, section chief of geriatrics and palliative medicine at the
University of Chicago Medical Center, who said he would not oppose a law allowing medical marijuana
use in Illinois.

The drug raises users' heart rates and lowers their blood pressure, so doctors needed to weigh its
effects beside those of other medications that users might be taking, he said. But patients do not
always confide their illegal drug use, he said.

``It's a fine balance between being supportive of patients to gain their trust and giving them your
best recommendations,'' Dr.~Dale said. ``I wasn't taught this in medical school.''

For some families, marijuana, which was once the root of all their battles, has brought them closer
together. Instead of parental warnings and punishment, there are questions about how to light a
water pipe; instead of the Grateful Dead, there are recipes for low-sodium brownies.

But for parents, there is also the knowledge that they are putting their children at risk of arrest.

``I was very uncomfortable getting my son involved,'' said the father of Alex, 21. The father, who
is 54, started using marijuana to relieve his pain from degenerative disc disease. He soon
discovered that Alex, who lives in Minnesota a few miles away, had access to better marijuana than
he did.

Alex's father had smoked marijuana when he was younger; Alex, by contrast, had been active in
antidrug groups at his school and church. In college, he started smoking infrequently and studying
marijuana's medicinal properties.

``When he told me he was using cannabis, I think he expected it to be a bigger deal for me,'' Alex
said. ``But it opened my eyes to what he was going through.''

Before trying marijuana, Alex's father took OxyContin, a narcotic, which he said made him ``feel
like a zombie.'' He also took antidepressants to relieve the mood disorder he associated with the
OxyContin. Marijuana has helped him cut down on the painkillers, he said.

He and Alex have smoked together twice, but it is not a regular practice, both said. Yet they say
the drug has strengthened their relationship.

``We spend our bonding time making brownies,'' Alex said.

Florence, 89, an artist who lives in New York, smokes mainly for relief from her spinal stenosis --
usually one or two puffs before going to sleep, she said. She buys her pipes through an online shop
and gets her marijuana from her daughter, Loren, who is 65.

``A person brings it to me,'' said Loren, who uses marijuana recreationally. ``I'm not out on a
street corner.'' Florence said that she had told all of her doctors that she was using marijuana,
and that none had ever discouraged her or warned of interactions with her prescription drugs,
including painkillers.

``I think I've influenced my own physician on the subject,'' she said. ``She came to me and asked me
for some for another patient.''

\section{Grown-Up, but Still Irresponsible}

\lettrine{T}{hey}\mycalendar{Oct.'10}{11} have sex with friends, acquaintances and people they're
casually dating. Many have never been tested for H.I.V. or any other sexually transmitted disease,
but they rarely use condoms. Who are they?

The irresponsible scoundrels are not teenagers but 50-something singles, according to the National
Survey of Sexual Health and Behavior, one of the most comprehensive national sex studies in almost
20 years, carried out at the Center for Sexual Health Promotion at Indiana University.

It turns out that ``friends with benefits'' -- a sexual partner who is ``just a friend,'' and
neither a soulmate nor a romantic interest -- isn't just for teenagers and college students anymore,
and maybe it never was. Young adults may have given the practice a new name, but it probably started
during the '60s sexual revolution, when the middle-aged Americans of today were young themselves.

Most men over 50 do have sex with a partner. But almost 23 percent said their most recent sex was
with a ``friend'' or a ``new acquaintance.''

Among women 50 and over, that figure was more than 13 percent. Those numbers don't surprise the
experts.

`` `Friends with benefits' are uniquely suited to two groups of people -- the young, who want to
delay starting their life, and older people, who don't want to complicate it,'' said Pepper
Schwartz, a sociology professor at the University of Washington in Seattle who serves on the sexual
health advisory council of Church \& Dwight, which manufactures Trojan condoms and financed the sex
survey. ``People in the middle are building families and building a life -- they need more than a
friend, they need lifetime partners.''

For older people who are casually having sex, ``it's warm, it's nice, they care about each other,
but no one is under the illusion this is a grand love,'' Dr.~Schwartz said.

For middle-aged heterosexual women, limited expectations of a sexual relationship may be a function
of demographics: Since women outlive men, there are simply fewer older men around, said Debby
Herbenick, one of the study's authors and associate director of the Center for Sexual Health
Promotion.

Young teenagers are far more responsible than older adults about using condoms, and they are not
nearly as sexually active as many people think they are, the study found. Most past studies have
estimated that half of all adolescents are sexually active, but those figures included 18- and
19-year-olds. Among those 14 to 17 years old, the new study found, fewer than one in four had ever
had vaginal intercourse, and sexual activity increased gradually as they matured.

The vast majority of the sexually active boys -- 80 percent of those aged 14 to 17 -- indicated that
they had used condoms the last time they had sex. ``Teens have gotten a bad rap,'' said
Dr.~J.~Dennis Fortenberry, a professor of pediatrics at Indiana University School of Medicine and
one of the authors.

Perhaps they can counsel their older counterparts. Only 25 percent of those 50 and over who were
single or had a new sex partner or more than one partner in a year said they had used a condom the
last time they had sex, the study found. Almost 40 percent had never been tested for H.I.V., and a
significant number didn't know the sexual history of their partners.

Experts say many reasons could account for this behavior. Many older singles have spent much of
their adult lives in long-term committed relationships, and may think of H.I.V. and AIDS as a
concern of young people. They haven't been targeted by public health messages urging condom use, and
there's no parental figure handing out condoms along with the car keys on a Saturday night. Older
men may also worry that condoms cause erectile problems, Dr.~Herbenick said.

And they may just have gotten out of the habit, Dr.~Herbenick said. ``They may just be thinking,
`Gosh, it's been 20 years since I used a condom, I'm not going to start again.' ''

\section{Google Cars Drive Themselves, in Traffic}

\lettrine{A}{nyone}\mycalendar{Oct.'10}{11} driving the twists of Highway 1 between San Francisco
and Los Angeles recently may have glimpsed a Toyota Prius with a curious funnel-like cylinder on the
roof. Harder to notice was that the person at the wheel was not actually driving.

The car is a project of Google, which has been working in secret but in plain view on vehicles that
can drive themselves, using artificial-intelligence software that can sense anything near the car
and mimic the decisions made by a human driver.

With someone behind the wheel to take control if something goes awry and a technician in the
passenger seat to monitor the navigation system, seven test cars have driven 1,000 miles without
human intervention and more than 140,000 miles with only occasional human control. One even drove
itself down Lombard Street in San Francisco, one of the steepest and curviest streets in the nation.
The only accident, engineers said, was when one Google car was rear-ended while stopped at a traffic
light.

Autonomous cars are years from mass production, but technologists who have long dreamed of them
believe that they can transform society as profoundly as the Internet has.

Robot drivers react faster than humans, have 360-degree perception and do not get distracted, sleepy
or intoxicated, the engineers argue. They speak in terms of lives saved and injuries avoided -- more
than 37,000 people died in car accidents in the United States in 2008. The engineers say the
technology could double the capacity of roads by allowing cars to drive more safely while closer
together. Because the robot cars would eventually be less likely to crash, they could be built
lighter, reducing fuel consumption. But of course, to be truly safer, the cars must be far more
reliable than, say, today's personal computers, which crash on occasion and are frequently infected.

The Google research program using artificial intelligence to revolutionize the automobile is proof
that the company's ambitions reach beyond the search engine business. The program is also a
departure from the mainstream of innovation in Silicon Valley, which has veered toward social
networks and Hollywood-style digital media.

During a half-hour drive beginning on Google's campus 35 miles south of San Francisco last
Wednesday, a Prius equipped with a variety of sensors and following a route programmed into the GPS
navigation system nimbly accelerated in the entrance lane and merged into fast-moving traffic on
Highway 101, the freeway through Silicon Valley.

It drove at the speed limit, which it knew because the limit for every road is included in its
database, and left the freeway several exits later. The device atop the car produced a detailed map
of the environment.

The car then drove in city traffic through Mountain View, stopping for lights and stop signs, as
well as making announcements like ``approaching a crosswalk'' (to warn the human at the wheel) or
``turn ahead'' in a pleasant female voice. This same pleasant voice would, engineers said, alert the
driver if a master control system detected anything amiss with the various sensors.

The car can be programmed for different driving personalities -- from cautious, in which it is more
likely to yield to another car, to aggressive, where it is more likely to go first.

Christopher Urmson, a Carnegie Mellon University robotics scientist, was behind the wheel but not
using it. To gain control of the car he has to do one of three things: hit a red button near his
right hand, touch the brake or turn the steering wheel. He did so twice, once when a bicyclist ran a
red light and again when a car in front stopped and began to back into a parking space. But the car
seemed likely to have prevented an accident itself.

When he returned to automated ``cruise'' mode, the car gave a little ``whir'' meant to evoke going
into warp drive on ``Star Trek,'' and Dr.~Urmson was able to rest his hands by his sides or
gesticulate when talking to a passenger in the back seat. He said the cars did attract attention,
but people seem to think they are just the next generation of the Street View cars that Google uses
to take photographs and collect data for its maps.

The project is the brainchild of Sebastian Thrun, the 43-year-old director of the Stanford
Artificial Intelligence Laboratory, a Google engineer and the co-inventor of the Street View mapping
service.

In 2005, he led a team of Stanford students and faculty members in designing the Stanley robot car,
winning the second Grand Challenge of the Defense Advanced Research Projects Agency, a \$2 million
Pentagon prize for driving autonomously over 132 miles in the desert.

Besides the team of 15 engineers working on the current project, Google hired more than a dozen
people, each with a spotless driving record, to sit in the driver's seat, paying \$15 an hour or
more. Google is using six Priuses and an Audi TT in the project.

The Google researchers said the company did not yet have a clear plan to create a business from the
experiments. Dr.~Thrun is known as a passionate promoter of the potential to use robotic vehicles to
make highways safer and lower the nation's energy costs. It is a commitment shared by Larry Page,
Google's co-founder, according to several people familiar with the project.

The self-driving car initiative is an example of Google's willingness to gamble on technology that
may not pay off for years, Dr.~Thrun said. Even the most optimistic predictions put the deployment
of the technology more than eight years away.

One way Google might be able to profit is to provide information and navigation services for makers
of autonomous vehicles. Or, it might sell or give away the navigation technology itself, much as it
offers its Android smart phone system to cellphone companies.

But the advent of autonomous vehicles poses thorny legal issues, the Google researchers
acknowledged. Under current law, a human must be in control of a car at all times, but what does
that mean if the human is not really paying attention as the car crosses through, say, a school
zone, figuring that the robot is driving more safely than he would?

And in the event of an accident, who would be liable -- the person behind the wheel or the maker of
the software?

``The technology is ahead of the law in many areas,'' said Bernard Lu, senior staff counsel for the
California Department of Motor Vehicles. ``If you look at the vehicle code, there are dozens of laws
pertaining to the driver of a vehicle, and they all presume to have a human being operating the
vehicle.''

The Google researchers said they had carefully examined California's motor vehicle regulations and
determined that because a human driver can override any error, the experimental cars are legal.
Mr.~Lu agreed.

Scientists and engineers have been designing autonomous vehicles since the mid-1960s, but crucial
innovation happened in 2004 when the Pentagon's research arm began its Grand Challenge.

The first contest ended in failure, but in 2005, Dr.~Thrun's Stanford team built the car that won a
race with a rival vehicle built by a team from Carnegie Mellon University. Less than two years
later, another event proved that autonomous vehicles could drive safely in urban settings.

Advances have been so encouraging that Dr.~Thrun sounds like an evangelist when he speaks of robot
cars. There is their potential to reduce fuel consumption by eliminating heavy-footed stop-and-go
drivers and, given the reduced possibility of accidents, to ultimately build more lightweight
vehicles.

There is even the farther-off prospect of cars that do not need anyone behind the wheel. That would
allow the cars to be summoned electronically, so that people could share them. Fewer cars would then
be needed, reducing the need for parking spaces, which consume valuable land.

And, of course, the cars could save humans from themselves. ``Can we text twice as much while
driving, without the guilt?'' Dr.~Thrun said in a recent talk. ``Yes, we can, if only cars will
drive themselves.''

\section{Wider Streets for Internet Traffic}

\lettrine{O}{ur}\mycalendar{Oct.'10}{11} taste for the Internet is insatiable -- traffic is growing
so fast that its transmission systems may soon be filled to capacity. But scientists are coping,
finding ingenious ways to satisfy our deep bandwidth hunger.

Of course, they can't accelerate the speed of light as it flies down the glass fibers of central
networks carrying our Internet messages worldwide. The laws of nature limit that. Yet they can tap
other characteristics of light to pack layers of information into each optical fiber in the network,
so that far more data can flow simultaneously down those glass backbones.

Old systems used light that was either on or off -- like flashlight signals -- to send information
along the fibers in the binary language of zeros and ones. But light is an electromagnetic wave, so
it has a whole electrical field that scientists are now putting to work to add to the information on
each wavelength.

Alcatel-Lucent recently announced a system for telecommunications service providers that takes
advantage of both the polarization and phases of light to encode data. The system can more than
double the capacity of a single fiber, said James Watt, head of the company's optics division. Such
a system, for example, can transmit more than twice the number of high-definition TV channels than
can now be streamed concurrently.

The new equipment is part of a continued research drive to increase the capacity of each strand of
optical fiber, said Keren Bergman, a professor of electrical engineering at Columbia University and
head of its Lightwave Research Laboratory. ``We are stuffing more information in the same space,''
she said.

A fiber is no thicker than human hair, but can carry many wavelengths of laser light, with each
wavelength adding to the bits transmitted per second. The bit rates now attainable are in the
billions (gigabits) per second or even trillions (terabits) per second.

The need for core network improvement is pressing, said Stojan Radic, a professor of electrical
engineering at the University of California, San Diego. ``We are looking at a point soon where we
cannot satisfy demand,'' he said. ``And if we don't, it will be like going over a cliff.''

Demand is continually growing, somewhere below street level, as details of our e-mail, bank balances
and national security zip along on light waves. And consumers can't get enough video clips on
YouTube, television shows on Hulu, and movies streamed to them by Netflix that they watch on their
computers and TVs.

But that's just a fraction of the traffic. Add to it the many demands of cloud computing and
countless mobile devices and information databases, for example, and the totals become even harder
to imagine.

Next-generation systems that can handle this future traffic jam are being developed by many
companies, including Ciena in Linthicum, Md., and Infinera in Sunnyvale, Calif. Alcatel-Lucent says
it has begun to sell equipment that transmits up to 88 channels of information, each operating at
100 gigabits a second, but it has not disclosed customer names.

The new equipment from Alcatel-Lucent is expected to reduce the cost per transmitted bit of
information, compared with existing equipment, said Paul Louis Ross, a company spokesman. The
systems are based in part on the work of the researcher Gabriel Charlet, a scientist at an
Alcatel-Lucent Bell research facility in France, who last year sent data at a rate of 7.2 terabits a
second over a single fiber more than 7,000 kilometers long.

All of those added gigabits take advantage of the complex way that light can be used to encode data.
In the past, when only the intensity of light was used, the signals could transmit only one bit per
time slot, said Govind Agrawal, a professor of optics at the University of Rochester.

By contrast, in the new system from Alcatel-Lucent, two binary digits or bits can be encoded by
using four phases of light. And the polarized light can vibrate up and down or sideways. In this
way, four bits of data can be transmitted per time slot instead of one, said Andrew Chraplyvy, a
scientist and executive at the Bell Labs of Alcatel-Lucent in Crawford Hill, N.J., where fiber-optic
research originated in the 1960s.

Scientists have long known how to use polarization and phases of light to encode information, said
Dr.~Chraplyvy, a winner of the prestigious Marconi Prize for his work in communications and
information technology. ``Although we could do it, we never needed to before, because the capacities
we had were enough,'' he said. ``Now that capacity is running out.''

Dr.~Adel Saleh, a program manager at the Defense Advanced Research Projects Agency, or Darpa, in
Arlington, Va., agreed. ``The traffic requirements on the Internet double every two years,'' he
said. ``This is why we are struggling to keep up.''

\section{Wife Detained After Visiting Nobel Winner}

\lettrine{T}{he}\mycalendar{Oct.'10}{11} wife of this year's Nobel Peace Prize winner, Liu Xiaobo,
was allowed to meet with her husband on Sunday at the prison in northeastern China where he is
serving an 11-year sentence, but she was then escorted back to Beijing and placed under house
arrest, a human rights group said.

Prison officials had informed Mr.~Liu that he won the award -- a decision vehemently condemned by
the Chinese government -- the day before. In their hourlong visit, Mr.~Liu's wife, Liu Xia, said her
husband had told her, ``This is for the lost souls of June 4th,'' and then was moved to tears.

Hundreds died June 4, 1989, in Beijing when Chinese troops and tanks crushed pro-democracy
demonstrations in Tiananmen Square. Mr.~Liu told his wife the award commemorates the nonviolent
spirit in which those who died fought for peace, freedom and democracy, the group, Human Rights in
China, said in a statement.

In Beijing, Ms.~Liu's telephone and Internet communication has been cut off and state security
officers are not allowing her to contact friends or the media, the statement said. Nor can she leave
her house except in a police car, according to the group. Her brother's phone has also been
``interfered with,'' the statement said.

Mr.~Liu, who was active in the 1989 movement, spent the next two decades pressing for political
reform in China. A 54-year-old former literature professor, he was one of the main authors of
Charter '08, a pro-democracy manifesto that calls for expanded liberties and the end to single-party
rule in China.

Roughly 10,000 people signed the document before the government blocked its circulation on the
Internet. Based on his pro-democracy writings, Mr.~Liu was convicted last December of ``inciting
subversion of the state.''

The Chinese government has described Mr.~Liu's award as ``blasphemy'' and has imposed a blackout on
news about it. Security in some areas has been tightened, and the road to Jinzhou prison in Liaoning
Province, where Mr.~Liu is held, has been blocked.

On Friday night, the police detained 20 bloggers, lawyers and academics who gathered for a
celebratory banquet at a private room in a Beijing restaurant. By Sunday night, 10 guests had been
released, according to a prominent activist, Zhang Zuhua, another of Charter '08's main authors.
Three were given eight days in detention for disturbing the peace, and seven have been escorted out
of Beijing, Mr.~Zhang said.

The Chinese journalist Zan Aizong sent a Twitter message on Sunday saying that Chinese Internet
media outlets had been ordered to post a Xinhua News Agency article that Russian media were
attacking the Nobel Peace Prize as a ``political tool of the West.''

Analysts speculated that Chinese leaders would gather soon to define the Communist Party's position
on the Nobel award. Until then, Nicholas Bequelin, a Hong Kong researcher for Human Rights Watch,
predicted that government agencies would tread carefully.

``The statement released by the Ministry of Foreign Affairs is almost word for word what they have
said before,'' Mr.~Bequelin said. ``The treatment meted out to dissidents and lawyers is the regular
one. Everyone is sitting tight and awaiting instructions from the top.''

Several key Chinese officials are currently out of the country, including Zhou Yongkang, a member of
the Standing Committee of the Political Bureau of the Communist Party's Central Committee.

\section{China, Angered by Peace Prize, Blocks Celebration}

\lettrine{T}{he}\mycalendar{Oct.'10}{11} banquet organized Friday night to celebrate the news that
the jailed Chinese dissident Liu Xiaobo had won the Nobel Peace Prize was over before it began.

While the two dozen bloggers, rights lawyers and academics were arriving at the private room they
had reserved at a Beijing restaurant shortly after the Norwegian Committee's announcement, the
police rushed in and briskly led the celebrators away, according to those who were there.

By Sunday, 10 of the 20 people who had been picked up were released, said Zhang Zuhua, an activist
who is in touch with the detainees. Three were given eight-day jail terms for disturbing the peace
and seven were escorted out of Beijing, he said.

As presidents, religious figures and rights advocates around the world praised the Nobel Committee
and called on the Chinese government to release Mr.~Liu, one of China's most prominent dissidents,
the Chinese government reacted with unrestrained ire.

They called in the Norwegian ambassador in Beijing for a dressing down, placed scores of dissidents
under house arrest and angrily described the decision to honor Mr.~Liu as ``blasphemy'' and an
insult to the Chinese people.

Mr.~Liu, 54, a former literature professor who has spent the past 20 years pressing for political
reform in China, is serving an 11-year sentence for ``inciting subversion of the state,'' based on
his writings and a pro-democracy manifesto, Charter '08, that he helped to draft. The document,
which demands an end to single-party rule and calls for expanded liberties, gathered 10,000
signatures before government censors blocked its circulation on the Internet.

In an editorial on Saturday, The Global Times, a state newspaper, accused the Nobel Committee of
imposing ``Western'' values on China, showing contempt for its legal system and seeking to split the
nation by provoking social strife. ``Every Chinese can sense a deliberate maliciousness in doing
so,'' it said.

The English-language version of the same newspaper expanded on the theme, quoting an international
relations professor at Renmin University, who said that the decision to select Mr.~Liu for the prize
was intended to humiliate China. ``Such a decision will not only draw the ire of the Chinese public,
but also damage the reputation of the prize,'' said the professor, Shi Yinhong.

On Friday night, scores of journalists gathered outside Mr.~Liu's home, but the police refused to
allow his wife, Liu Xia, to come out and would not let reporters enter. The police later led her
away, promising to escort her to Jinzhou Prison, 300 miles away in Liaoning Province, to see her
husband.

She returned to Beijing on Sunday after visiting her husband, Mr.~Zhang said. Ms.~Liu could not be
reached for comment.

In an interview last week, Ms.~Liu said she had little expectation that her husband would win the
prize but said that if he did, she hoped it might prompt the authorities to release him earlier.
``As my friends have said, how can they keep a Nobel Peace Prize winner in Jinzhou Prison?'' she
said.

Few Chinese citizens seemed aware of the honor accorded Mr.~Liu, even 24 hours after its
announcement. ``Never heard of him, but we also haven't watched TV recently,'' said Yang Guwen,
dressed in a denim cowboy shirt, as he and his girlfriend walked beneath a huge television screen
that hangs over one of the capital's ritziest shopping malls.

Had he followed news reports, Mr.~Yang would not have learned that a Chinese citizen had won one of
the world's most respected prizes. Except for the Global Times editorial and a brief Foreign
Ministry condemnation posted on the Internet, Chinese newspapers and Web-based portals ignored the
news. Anyone typing the words ``Nobel Peace Prize'' or ``Liu Xiaobo'' into Google found themselves
facing a blank screen.

A veteran civil rights lawyer, Teng Biao, said he was on his way to meet a foreign journalist on
Saturday when he was stopped by national security agents at the Beijing university where he teaches.
``The officers say that the police have rigid orders from higher authorities that they must work
resolutely to thwart celebratory activities to mark this event,'' he said in a cellphone interview,
having briefly stepped away from the agents to take a call. ``They are keeping a strict eye on the
most active people, in order to reduce its impact to the smallest degree possible.''

The activists who gathered for the celebratory meal on Friday night were no strangers to police
surveillance. Earlier in the evening, they had met in a park with yellow ribbons pinned to their
shirts and clear plastic sleeves -- the kinds favored by conventioneers -- slung around their necks.
The sleeves carried two portraits of Mr.~Liu: one dark and somber, the other brightly lighted and
decidedly cheery.

When word of the Nobel Prize arrived, they turned the happy photo to face out, walked to the
restaurant and tacked a portrait of Mr.~Liu on a wall. By the time Paul Mooney, an American
freelance journalist, arrived, the police had already shown up.

After a brief scuffle, the men and women were led away, leaving Mr.~Mooney alone with a roomful of
officers and the crumpled portrait of the Nobel laureate on the floor. Out of curiosity, he asked
the young officers if they were familiar with Mr.~Liu. ``None of them even knew who he was,'' he
said.

\section{Kim Jong-il's Heir Attends Parade}

\lettrine{T}{he}\mycalendar{Oct.'10}{11} North Korean leader, Kim Jong-il, attended a massive
military parade with his youngest son and designated successor on Sunday as the ruling Communist
regime celebrated the 65th founding of its Workers' Party.

The son, Kim Jong-un, wearing a dark suit despite his recent promotion to four-star general, watched
the festivities and reviewed squads of goose-stepping troops with his 68-year-old father and other
senior politicians and generals. The event was held in Kim Il-sung Square, named for Kim Jong-un's
grandfather, the founder of the North Korean state.

Video footage from the celebration in Pyongyang, the North Korean capital, showed tens of thousands
of performers and soldiers arrayed in what was said to be the largest such event in the country's
history. For the first time, a few dozen Western media organizations, including some American
outlets, were allowed to attend the festivities and report live from the square.

The elder Mr.~Kim, who is said to be in poor health after apparently suffering a stroke in 2008, has
hurried the succession of Kim Jong-un in recent weeks. At a landmark Workers' Party meeting last
month, Kim Jong-un was made a general and received two significant positions in the party.

Other members of the Kim family and the leader's inner circle also received new posts and promotions
as the leadership hierarchy was reshuffled to provide Kim Jong-un with mentors and supporters as he
solidifies his power.

Little is known about Kim Jong-un, who is believed to be 27 or 28. He is the youngest of Mr.~Kim's
three sons -- the older brothers were uninterested or deemed incapable of leadership -- and he
attended school for a time in Bern, Switzerland. He is known to speak some English, and he likely
speaks German as well. Until last month's party meeting, very few pictures of him had been seen in
public.

The Workers' Party anniversary is typically a major national holiday in North Korea, with citizens
receiving food handouts from the government. The theme of the celebration Sunday was heavily
military, befitting Kim Jong-il's guiding philosophy of songun, or military first. Nuclear-armed
North Korea has a huge standing army, with 1.2 million soldiers, and its border with South Korea is
one of the world's most heavily militarized.

There was no immediate reaction from the South Korean government to Sunday's parade in the North,
but the conservative administration of President Lee Myung-bak has taken a hard line against North
Korea, and relations between the two countries remain strained. The South blames a North Korean
torpedo attack for the sinking of one of its naval vessels, the Cheonan, an incident in March that
killed 46 sailors. The North has denied any role in the sinking.

\section{Nobel Prize Is Seen as Rebuke to China}

\lettrine{F}{ew}\mycalendar{Oct.'10}{11} nations today stand as more of a challenge to the
democratic model of governance than China, where an 89-year-old Communist Party has managed to quash
political movements while creating a roaring, quasi-market economy and enforcing a veneer of social
stability.

With the United States' economy flagging and its global influence in decline, some Chinese leaders
now appear confident in asserting that freedom of speech, multiparty elections and constitutional
rights -- what some human rights advocates call universal values -- are indigenous to the West, and
that is where they should stay.

The awarding of the Nobel Peace Prize to Liu Xiaobo, 54, was a sharp rejoinder to that philosophy.
Of course, it was a Norwegian panel that gave him the prize, providing Chinese officials and their
supporters with ample ammunition to denounce the move as another attempt by the West to impose its
values on China.

But anticipating the criticism, the judges underscored the support in China for the imprisoned
Mr.~Liu's work and his plight, which they said proved that the Chinese were as hungry as anyone for
the political freedoms enjoyed in countries like the United States, India and Indonesia.

``The campaign to establish universal human rights also in China is being waged by many Chinese,
both in China itself and abroad,'' the Norwegian Nobel Committee said. ``Through the severe
punishment meted out to him, Liu has become the foremost symbol of this wide-ranging struggle for
human rights in China.''

The Dalai Lama, the exiled Tibetan spiritual leader who won the prize in 1989, highlighted the
grass-roots Chinese push for political reform in a statement praising Mr.~Liu, saying that ``future
generations of Chinese will be able to enjoy the fruits of the efforts that the current Chinese
citizens are making towards responsible governance.'' Yet the Dalai Lama stands as proof that the
struggle for rights in China is a hard one, and that winning the Nobel is no guarantee of achieving
even minimal success.

Nevertheless, the number of signatures on Charter 08, the document that Mr.~Liu co-drafted that
calls for gradually increasing constitutional rights, shows that at the very least, there is an
appetite in this country to openly discuss the kind of values that hard-line Communist Party leaders
dismiss as a new brand of Western imperialism.

The 300 initial signatures on the document snowballed to 10,000 as it spread on the Internet, even
as the government tried its best to stamp it out. Certainly many of those who signed it were
intellectuals, not exactly representative of most Chinese, but China has a rich history of political
reform led by its elites. Chinese lawyers, journalists, scholars, artists, policy advisers -- many
among them will be heartened by the Nobel Committee's decision.

``Today, many people are making efforts,'' said Wan Yanhai, the most prominent advocate for AIDS
patients in China and one of the initial signers of Charter 08; he left China temporarily for the
United States in May because of what he called police harassment. ``They're hidden, but they're
there,'' he said. ``People are organizing different resistance movements, sometimes in a peaceful
way, sometimes in a violent manner.''

Cui Weiping, a social critic who teaches at the Beijing Film Academy, said the rights struggle was
moving from a local stage to a global one. ``Like everything that happens in China today, the
democracy movement here exists in a global context,'' she said. ``So this will be a lesson to China:
it can't bottle up the democracy movement forever.''

The Internet, the vehicle that carried Charter 08 to prominence, simmered with Chinese support for
Mr.~Liu early Friday night despite extensive government filtering. Liu Xiaobo was the most common
topic on Sina.com's Weibo, a popular microblog forum. Microbloggers burned with enthusiasm for the
prize and hurled invective at the government: ``Political reform and the Nobel Prize, is this a new
start? This day has finally come,'' wrote a user named Nan Zhimo. Another user, Hei Zechuan, said,
``The first real Chinese Nobel Prize winner has emerged, but he is still in prison right now; what a
bittersweet event.''

Even before the announcement Friday afternoon, a group of supporters gathered outside the Beijing
apartment building where Liu Xiaobo's wife, Liu Xia, lives. They showed little fear of the
black-uniformed police officers surrounding them.

``I believe this award will massively open up room for political discussion in China,'' said one of
those standing outside the building, Li Yusheng, 66, a retired journalist, Charter 08 signer and
founder of a group that aims to help the poor. ``And it will exert pressure on the authorities to
change their old ways, so that they will not be able to jail people like Liu Xiaobo in the future.
They will have to change or else be driven out of power.''

But the authorities clung to their habits on Friday night, as police officers showed up at
celebratory gatherings in Beijing and Shanghai to haul people off to police stations, according to
Twitter feeds.

Some political experts here say that even China's more liberal-minded leaders have little appetite
for pushing vigorously for greater political rights, and will continue to hold back as jockeying
intensifies ahead of the 2012 leadership succession -- a time when hard-line attitudes tend to
dominate. A sharp taste of that came in March 2009, when Wu Bangguo, the head of the National
People's Congress, a rubber-stamp Parliament, made a speech in which he dismissed any move toward
Western-style democracy, mentioning it no fewer than nine times.

``We will never simply copy the system of Western countries or introduce a system of multiple
parties holding office in rotation,'' he said, adding that ``although China's state organs have
different responsibilities, they all adhere to the line, principles and policies of the party.''

Some Chinese liberals like Mr.~Wan say they see a compatriot in Prime Minister Wen Jiabao, who as
recently as August publicly extolled the virtues of political change. ``Without the guarantee of
political system reform, the successes of restructuring the economic system will be lost and the
goal of modernization cannot be realized,'' Mr.~Wen said, according to People's Daily.

Some liberal economists like Yang Yao and Wu Jinglian have also come out strongly in support of
political restructuring, arguing that China's economy, where state-owned enterprises tied to the
Communist Party continue to dominate the largest industries, can reach maturity only with the checks
and balances that come with democracy.

The exact form of democracy is often left vague in these discussions. Liberals know that calling for
multiparty elections -- a direct challenge to the primacy of the Communist Party -- is a red line.
Mr.~Wen, whom many Chinese praise but whose actual power is dubious, shies away from mentioning
elections. Mr.~Liu and the co-writers of Charter 08 were also careful to avoid calling for any
immediate, drastic change to the Communist Party's hold on power.

``Our intention was not to threaten the party or the government,'' said Zhang Zuhua, one of the
charter's main authors. ``It was to put forth this framework of universal values, and build a
consensus within society around it, among both those within and outside the system.''

``Except the government,'' he said, ``clearly does not affirm these universal values.''

\section{Shared Concern About China Aligns U.S.~and Vietnam}

\lettrine{A}{}\mycalendar{Oct.'10}{11} visit to Vietnam this week by the secretary of defense,
Robert M.~Gates, is the latest step in a bilateral relationship that is at its warmest since
diplomatic ties were established 15 years ago.

A steady progression of careful gestures has eroded the enmities of the Vietnam War, built a basis
of increasing trust and turned the two nations' attention, in large part, from issues of the past to
the present.

It is the second cabinet-level visit to Vietnam in four months; Secretary of State Hillary Rodham
Clinton came in July. Exchanges at this level have become almost routine.

``I would say that relations are at their highest point in 15 years,'' said Nguyen Manh Hung,
director of the Indochina Institute at George Mason University in Virginia. ``We have basically
removed the major hurdles of suspicion in military to military relations, and I would expect things
to proceed quite fast.''

Mr.~Gates was expected to meet with General Phung Quang Thanh, the Vietnamese defense minister, at a
gathering of defense chiefs from the 10-member Association of Southeast Asian Nations and partner
countries.

The main concern shared by the two nations underscores the shifts in alliances in the 35 years since
the war came to an end: Chinese claims in the South China Sea.

It is an issue with some irony. Where the United States sought during the war to contain an
expansion of Chinese communism into Vietnam, it is aligned with Vietnam today in concern over an
escalation of China's maritime claims.

China was an ally of North Vietnam in its war against the United States in the 1960s and 1970s and
is now a partner of a unified Vietnam in an uneasy relationship between Communist nations of vastly
different size.

``Vietnam worries about Chinese in the South China Sea and America worries about interference in
freedom of navigation,'' Mr.~Hung said. ``Because of this, the strategic interests of Vietnam and
the United States converge.''

Last week, Vietnam demanded the release of a fishing boat and nine crew members arrested a month ago
near disputed islands. China has said that the crew must pay a fine, and Vietnam has asserted that
the crew members have been mistreated.

In March, at least one senior China official raised the level of its territorial claim, asserting to
two senior White House officials visiting Beijing that the South China Sea was a ``core interest,''
a phrase that placed it on a par with Taiwan and Tibet, its most sensitive territorial interests.

In response, during a visit to Hanoi in July, Mrs.~Clinton hardened Washington's stance by saying
the United States has a ``national interest'' in freedom of navigation in the area.

In balancing its relations between the two major powers, Vietnam has been at pains to reassure
China, the giant on its doorstep, that it would have no alliances, military bases or military
coalitions that threaten it.

While Vietnam marked the 15th anniversary of diplomatic ties with the United States this year, it
also celebrated a much longer diplomatic relationship of 60 years with China.

Hanoi's warming toward Washington has been slowed by suspicions of American motives and commitment
to a Vietnam policy.

Hanoi understands that for Washington, relations with Vietnam have always been part of larger
international interests, analysts say, and that they could shift as those interests change.

Once again, as it was during the war, the United States stance toward Vietnam is one piece in a
broader China policy.

But step by step, the two former wartime enemies have grown steadily closer. Trade relations were
normalized in 2006. Port calls by American Navy ships have become more frequent since the first one
in 2003.

``It's a very deliberate pace that's being kept here,'' said Carlyle Thayer, an expert on Vietnam at
the Australian Defense Force Academy at the University of New South Wales in Sydney. ``Neither side
wants to be used by the other, but both want to advance the relationship.''

Mrs.~Clinton took an exuberant tone last month when she said, ``The progress between Vietnam and the
United States has been breathtaking.''

Vietnamese officials have been less effusive, but they seem to agree.

``Vietnam and the United States are enjoying an excellent period of bilateral relations,'' the
Vietnamese ambassador to the United States, Le Cong Phung, said in remarks quoted by the official
Vietnam News Agency last month.

Warming relations continue to be slowed by American concerns over human rights abuses in Vietnam and
by Hanoi's suspicion that Washington is using the issue to undermine the communist government.

The Vietnamese often use the phrases ``peaceful evolution'' and the ``color revolutions,''
expressions that refer to its view that the collapse of the Soviet Union and other European
communist governments were brought about at least partly by outside support for democracy and human
rights.

The competing concerns involving human rights renew themselves in something of a vicious circle.
Vietnam's fear of American motives leads to the arrests of dissidents it sees as connected with the
West. And those arrests in turn intensify Washington's concerns over human rights abuses.

The two nations' alignment on the issue of the South China Sea illustrates the emergence of a more
forward-looking relationship, said Kim Ninh, the country representative in Vietnam for the Asia
Foundation, which is based in California.

For the United States, the chief issue from the past continues to be a full accounting for
servicemen still missing from the war, though that concern no longer carries the power that it once
did.

For Vietnam, the chief remaining postwar issue is a demand for greater United States assistance in
addressing the effects of agent orange, a chemical defoliant that was sprayed in parts of the
country, causing widespread birth defects.

\section{U.S.~Concerned About Attitude of China's Military}

\lettrine{D}{efense}\mycalendar{Oct.'10}{12} Secretary Robert M.~Gates met his Chinese counterpart,
Liang Guanglie, in Vietnam on Monday for the first time since the two militaries suspended talks
with each other last winter, calling for the two countries to prevent ``mistrust, miscalculations
and mistakes.''

His message seems directed mainly at officers like Lt.~Cmdr. Tony Cao of the Chinese Navy.

Days before Mr.~Gates arrived in Asia, Commander Cao was aboard a frigate in the Yellow Sea,
conducting China's first war games with the Australian Navy, exercises to which, he noted pointedly,
the Americans were not invited.

Nor are they likely to be, he told Australian journalists in slightly bent English, until ``the
United States stops selling the weapons to Taiwan and stopping spying us with the air or the
surface.''

The Pentagon is worried that its increasingly tense relationship with the Chinese military owes
itself in part to the rising leaders of Commander Cao's generation, who, much more than the
country's military elders, view the United States as the enemy. Older Chinese officers remember a
time, before the Tiananmen Square protests in 1989 set relations back, when American and Chinese
forces made common cause against the Soviet Union.

Theyounger officers have known only an anti-American ideology, which casts the United States as bent
on thwarting China's rise.

``All militaries need a straw man, a perceived enemy, for solidarity,'' said Huang Jing, a scholar
of China's military and leadership at the National University of Singapore. ``And as a young officer
or soldier, you always take the strongest of straw men to maximize the effect. Chinese military men,
from the soldiers and platoon captains all the way up to the army commanders, were always taught
that America would be their enemy.''

The stakes have increased as China's armed forces, once a fairly ragtag group, have become more
capable and have taken on bigger tasks. The navy, the centerpiece of China's military expansion, has
added dozens of surface ships and submarines, and is widely reported to be building its first
aircraft carrier. Last month's Yellow Sea maneuvers with the Australian Navy are but the most recent
in a series of Chinese military excursions to places as diverse as New Zealand, Britain and Spain.

China is also reported to be building an antiship ballistic missile base in southern China's
Guangdong Province, with missiles capable of reaching the Philippines and Vietnam. The base is
regarded as an effort to enforce China's territorial claims to vast areas of the South China Sea
claimed by other nations, and to confront American aircraft carriers that now patrol the area
unmolested.

Even improved Chinese forces do not have capacity or, analysts say, the intention, to fight a more
able United States military. But their increasing range and ability, and the certainty that they
will only become stronger, have prompted China to assert itself regionally and challenge American
dominance in the Pacific.

That makes it crucial to help lower-level Chinese officers become more familiar with the Americans,
experts say, before a chance encounter blossoms into a crisis.

``The P.L.A. combines an odd combination of deep admiration for the U.S.~armed forces as a military,
but equally harbors a deep suspicion of U.S.~military deployments and intentions towards China,''
David Shambaugh, a leading expert on the Chinese military at George Washington University, said in
an e-mail exchange, referring to the People's Liberation Army.

``Unfortunately, the two militaries are locked in a classic security dilemma, whereby each side's
supposedly defensive measures are taken as aggressive action by the other, triggering similar
countermeasures in an inexorable cycle,'' he wrote. ``This is very dangerous, and unnecessary.''

From the Chinese military's view, this year has offered ample evidence of American ill will.

The Chinese effectively suspended official military relations early this year after President Obama
met with the Dalai Lama, the Tibetan religious leader, and approved a \$6.7 billion arms sale to
Taiwan, which China regards as its territory.

Since then, the Chinese military has bristled as the State Department has offered to mediate
disputes between China and its neighbors over ownership of Pacific islands and valuable seabed
mineral rights. And when the American Navy conducted war games with South Korea last month in the
Yellow Sea, less than 400 miles from Beijing, younger Chinese officers detected an encroaching
threat.

The United States ``is engaging in an increasingly tight encirclement of China and constantly
challenging China's core interests,'' Rear Adm. Yang Yi, former head of strategic studies at the
Chinese Army's National Defense University, wrote in August in the People's Liberation Army Daily,
the military newspaper. ``Washington will inevitably pay a costly price for its muddled decision.''

In truth, little in the American actions is new. Mr.~Obama's predecessors also hosted the Dalai
Lama. American arms sales to Taiwan were mandated by Congress in 1979, and have occurred regularly
since then. American warships regularly ply the waters off China's coast and practice with South
Korean ships.

But Chinese military leaders seem less inclined to tolerate such old practices now that they have
the resources and the confidence to say no.

``Why do you sell arms to Taiwan? We don't sell arms to Hawaii,'' said Col.~Liu Mingfu, a China
National Defense University professor and author of ``The China Dream,'' a nationalistic call to
succeed the United States as the world's leading power.

That official military relations are resuming despite the sharp language from Chinese Army officials
is most likely a function of international diplomacy. President Hu Jintao is scheduled to visit
Washington soon, and American experts had predicted that China would resume military ties as part of
an effort to smooth over rough spots before the state visit.

Some experts see increased contact as critical. A leading Chinese expert on international security,
Zhu Feng of Peking University, says that the Chinese military's hostility toward the United States
is not new, just more open. And that, he says, is not only the result of China's new assertiveness,
but its military's inexperience on the world stage.

``Chinese officers' international exposure remains very limited,'' Mr.~Zhu said. ``Over time, things
will improve very, very significantly. Unfortunately, right now they are less skillful.''

Greater international exposure is precisely what American officials would like to see. Americans
hope renewed cooperation will lead to more exchanges of young officers and joint exercises.

``It's time for both militaries to reconsider their tactics and strategy to boost their
friendship,'' Mr.~Zhu said. ``The P.L.A. is increasing its exposure internationally. So what sort of
new rule of law can we figure out to fit the P.L.A. to such new exposure? It's a challenge not just
for China, but also for the U.S.''

\section{Netanyahu's Moves Spark Debate on Intentions}

\lettrine{A}{n}\mycalendar{Oct.'10}{12} offer on Monday by Prime Minister Benjamin Netanyahu of
Israel to freeze West Bank Jewish settlements in exchange for Palestinian recognition of Israel as a
Jewish state -- instantly rejected by the Palestinians -- was the latest complex maneuver
engendering debate about his intentions.

The offer, made in a speech at the opening of the fall session of Parliament, was aimed either at
keeping talks with the Palestinians alive and his right-wing coalition partners in check, or at
seeking to shift the burden of failure to the Palestinians and escape blame should the talks wither
and die.

As part of a flurry of initiatives favored by Israel's right that began Sunday, Mr.~Netanyahu backed
a measure that requires non-Jewish immigrants to take a loyalty oath to Israel as a Jewish and
democratic state before they can become citizens. On Monday his government supported a bill that
would require a national referendum before any territory could be yielded in a peace deal.

``The last few days clearly are disturbing as to which direction all this is going,'' Isaac Herzog,
Israel's welfare minister, who is from the Labor Party, said in a telephone interview. ``It may all
be in preparation for the big peace step, or it may be a political maneuver to regain control of the
right.''

Mr.~Netanyahu is facing particular competition on the right from his foreign minister, Avigdor
Lieberman, who has made a loyalty oath for Arabs a central part of his political appeal to Jewish
supporters and said there will be no peace with the Palestinians for at least a generation.

But Mr.~Netanyahu also has to contend with American and other international pressure to resume a
construction freeze on West Bank Jewish settlements. Last Friday, the Arab League backed a
Palestinian vow not to return to direct American-sponsored talks without a full
settlement-construction freeze. It gave the Obama administration another month to come up with a way
to save the negotiations.

In his parliamentary speech, Mr.~Netanyahu mentioned that he was considering American proposals. He
did not specify them, but they are known to include security guarantees and military hardware in
exchange for a freeze extension of two to three months.

But Mr.~Netanyahu said recognition of Israel as a Jewish state would be enough for now.

``If the Palestinian leadership will say unequivocally to its people that it recognizes Israel as
the homeland of the Jewish people, I will be ready to convene my government and request a further
suspension of construction for a fixed period,'' he said, referring to the expired 10-month
construction moratorium.

Nabil Abu Rudeineh, a spokesman and aide close to President Mahmoud Abbas of the Palestinian
Authority, said the Palestinians had long ago recognized Israel and would not engage in defining its
character or ethnicity.

``There is no connection between settlement building and Israel's identity,'' Mr.~Abu Rudeineh said
in a telephone interview. ``A return to direct negotiations must be accompanied by a total freeze on
settlement building. That has been our clear position. The issue of Israel's Jewishness has nothing
to do with it.''

For the Palestinians, defining Israel as a Jewish state means acknowledging that Palestinian
refugees would not be permitted back to their homes in what is today Israel, a concession they are
not willing to make in advance. It also raises questions about the status of Israeli Arabs.

Earlier on Monday, Mr.~Netanyahu's government backed a bill requiring that a national referendum be
held before any territory could be yielded in a peace deal, a move seen as easing the right's
objection to a freeze extension but substantially complicating a deal later on.

The bill stipulates that any land negotiated away, including the Golan Heights and East Jerusalem,
would require approval by a parliamentary majority followed by a full popular referendum. If a
supermajority in the Parliament -- 80 of 120 members -- accepted the land deal, the referendum would
be optional.

Sponsored by a member of Mr.~Netanyahu's Likud Party, the bill passed a first reading last year but
had been blocked by the government since. Now, with the backing of the ministerial committee on
legislation, it heads to second and third readings and likely passage.

While this step could be to satisfy his right flank -- like Sunday's approval of the loyalty oath
for new citizens -- in advance of a concession to the Palestinians, it could also complicate
matters. Requiring non-Jews to vow loyalty to a Jewish state and putting territorial withdrawal to a
referendum could damage the peace process. The first would raise tensions, and the second would
submit painful, unpopular decisions to a popular test.

``He has to strengthen his coalition,'' said Efraim Inbar, a political scientist at Bar-Ilan
University, speaking of the referendum bill. ``It's just like with the loyalty oath. It allows him
to say, `No matter what I give up, I have to go to a referendum afterwards.' ''

Danny Danon, a leader of the right wing of Likud, said in a telephone interview that while he
generally did not favor referendums, he was in favor of this bill because it would make it harder to
give up land and would reduce the chances of the creation of a Palestinian state.

``Anything that adds another barrier to the prime minister seeking to give away land is a good
thing,'' Mr.~Danon said.

Both East Jerusalem and the Golan were officially annexed by Israel through parliamentary votes, so
by Israeli law they count as Israeli territory. That is not true of the West Bank, which the
Palestinians want as their future state and where Israel has settled more than 300,000 Jewish
citizens.

But Israel is expected in any peace deal to hold on to some settlement blocks and to give over to
the Palestinians parcels of its own land in exchange. Under those circumstances, the bill in
question would seem also to require approval of such an exchange by referendum.

The Palestinians watched the referendum discussion with unease, but acknowledged that they, too,
were considering such a step.

``We have these thoughts on our side as well,'' said Ghassan Katib, a spokesman for the Palestinian
Authority. ``But they are being asked to give up territory that they are controlling illegally.
Nonetheless, although we fear this will make the deal more difficult, it is their internal matter.''

Yaron Ezrahi, a political scientist at Hebrew University, said he viewed the plan of a referendum as
a kind of bulletproof vest for Mr.~Netanyahu to wear in his negotiations with the Palestinians and
with the Obama administration.

``He is trying to soften the right as he moves forward, but he also covers himself by saying
ultimately it is not up to him,'' he said.

\section{Hunting One Language, Stumbling Upon Another}

\lettrine{T}{wo}\mycalendar{Oct.'10}{12} years ago, a team of linguists plunged into the remote hill
country of northeastern India to study little-known languages, many of them unwritten and in danger
of falling out of use.

On average, every two weeks one of the world's recorded 7,000 languages becomes extinct, and the
expedition was seeking to document and help preserve the endangered ones in these isolated villages.

At a rushing mountain river, the linguists crossed on a bamboo raft and entered the tiny village of
Kichang. They expected to hear the people speaking Aka, a fairly common tongue in that district.
Instead, they heard a language, the linguists said, that sounded as different from Aka as English
does from Japanese.

After further investigation, leaders of the research announced last week the discovery of a
``hidden'' language, known locally as Koro, completely new to the world outside these rural
communities. While the number of spoken languages continues to decline, at least one new one has
been added to the inventory, though Koro too is on the brink of extinction.

``We noticed it instantly'' as a distinct and unfamiliar language, said Gregory Anderson, director
of the Living Tongues Institute for Endangered Languages in Salem, Ore.

Dr.~Anderson and K.~David Harrison, a linguist at Swarthmore College, were leaders of the
expedition, part of the Enduring Voices Project of the National Geographic Society. Another member
of the group was Ganash Murmu, a linguist at Ranchi University in India. A scientific paper will be
published by the journal Indian Linguistics.

When the three researchers reached Kichang, they went door to door asking people to speak their
native tongue -- not a strenuous undertaking in a village of only four bamboo houses set on stilts.
The people live by raising pigs and growing oranges, rice and barley. They share a subsistence
economy and a culture with others in the region who speak Aka, or Miji, another somewhat common
language.

On the veranda at one house, the linguists heard a young woman named Kachim telling her life story
in Koro. She was sold as a child bride, was unhappy in her adopted village and had to overcome
hardships before eventually making peace with her new life.

Listening, the researchers at first suspected Koro to be a dialect of Aka, but its words, syntax and
sounds were entirely different. Few words in Koro were the same as in Aka: mountain in Aka is
``phu,'' but ``nggo'' in Koro; pig in Aka is ``vo,'' but in Koro ``lele.'' The two languages share
only 9 percent of their vocabulary.

The linguists recorded Kachim's narrative in Koro, and an Indian television crew had her repeat it
in Hindi. This not only enabled the researchers to understand her story and her language, but called
attention to the cultural pressures threatening the survival of such languages, up against national
languages dominant in schools, commerce and mass media.

In ``The Last Speakers: The Quest to Save the World's Most Endangered Languages,'' published last
month by National Geographic Books, Dr.~Harrison noted that Koro speakers ``are thoroughly mixed in
with other local peoples and number perhaps no more than 800.''

Moreover, linguists are not sure how Koro has survived this long as a viable language. Dr.~Harrison
wrote: ``The Koro do not dominate a single village or even an extended family. This leads to curious
speech patterns not commonly found in a stable state elsewhere.''

By contrast, the Aka people number about 10,000 living in close relations with Koro speakers in a
district of the state of Arunachal Pradesh, where at least 120 languages are spoken. Dr.~Anderson
said the coexistence of separate languages between two integrated groups that do not acknowledge an
ethnic difference between them is highly unusual.

As Dr.~Harrison and Dr.~Anderson expanded their research, comparing Koro with several hundred
languages, they determined that it belonged to the Tibeto-Burman language family, which includes 400
tongues related to widely used Tibetan and Burmese. But Koro had never been recognized in any
surveys of the approximately 150 languages spoken in India.

The effort to identify ``hot spots of threatened languages,'' the linguists said, is critical in
making decisions to preserve and enlarge the use of such tongues, which are repositories of a
people's history and culture.

In the case of Koro speakers, Dr.~Harrison wrote in his book, ``even though they seem to be
gradually giving up their language, it remains the most powerful trait that identifies them as a
distinct people.''

\section{U.S.~Raid May Have Killed Briton Held by Taliban}

\lettrine{P}{rime}\mycalendar{Oct.'10}{12} Minister David Cameron said Monday that a British aid
worker killed in an American rescue raid in Afghanistan last week may have been killed by a grenade
detonated by a United States special forces unit -- not by a suicide bomber's vest from her Taliban
captors, as the American command in Afghanistan suggested when it confirmed her death on Saturday.

A grim-faced Mr.~Cameron appeared at a news conference at 10 Downing Street to say he had learned of
``this deeply distressing development'' on Monday from the top American and NATO commander in
Afghanistan, Gen.~David H.~Petraeus, who told him that an American-led review of the raid to rescue
the aid worker, Linda Norgrove, ``has revealed evidence to indicate that Linda may not have died at
the hands of her captors as originally believed.''

Mr.~Cameron added: ``That evidence and subsequent interviews with the personnel involved'' --
believed to have included a Navy Seals unit specializing in hostage rescues that has participated in
numerous special forces raids in Afghanistan -- ``suggest that Linda could have died as a result of
a grenade detonated by the task force during the assault. However, this is not certain and a full
U.S./U.K. investigation will now be launched.''

Maj. Sunset Belinsky, a spokeswoman at NATO headquarters in Kabul, said a review of surveillance
footage ``showed what was believed to be a member of the rescue team throwing a hand grenade in the
area near where Linda Norgrove was later found.''

On Monday American officials in Kabul and at the Central Command Headquarters in Tampa, Fla.,
announced that a senior officer in the United States Special Operations command, Maj. Gen.~Joseph
L.~Votel, had been appointed to lead an investigation into Ms.~Norgrove's death, and that he would
work ``in close cooperation'' with British officials.

Ms.~Norgrove, 36, who was from a remote area in the western isles of Scotland, was kidnapped by the
Taliban in the mountainous eastern province of Kunar on Sept.~26, together with three Afghan
companions who were later released. She was working for an American aid organization, Development
Alternatives Inc., or DAI, that carries out projects in Afghanistan under contract with the United
States Agency for International Development.

Based at DAI offices in the eastern city of Jalalabad, she had set out in an unarmored car, without
security guards, to review a project in a part of Kunar that is heavy with Taliban fighters. British
officials said that American military intelligence experts working with high-altitude surveillance
from unmanned drones had traced the kidnappers and Ms.~Norgrove to a mud-walled compound high in the
Korangal Valley, which has been the scene of bloody firefights in recent years between the Taliban
and American forces.

Mr.~Cameron appeared at pains not to sound reproving of American actions in the rescue attempt, in
which at least six Taliban fighters were also killed. After Ms.~Norgrove's death was announced on
Saturday, Mr.~Cameron emphasized that the raid had been approved by Britain's foreign secretary,
William Hague, and that he had been kept informed.

The raid recalled the operation to free a reporter for The New York Times and his interpreter from
Taliban captors in the northern province of Kunduz in September 2009. In that case, the British
prime minister at the time, Gordon Brown, approved a raid by British special forces units taken to
the hostage hideout by American helicopters, and the reporter, Stephen Farrell, was rescued
unharmed.

But the interpreter, Sultan Munadi, was killed in the raid, as were a soldier in Britain's Parachute
Regiment and two Afghan civilians.

On Monday, government officials in London disclosed that a multi-agency group led by the prime
minister that oversees high-level security operations, had met a dozen times since Ms.~Norgrove's
kidnapping to review rescue plans and to approve the final plan.

Mr.~Hague told the House of Commons that Ms.~Norgrove's captors were members of a Salafist group --
an extreme form of Islam -- and that a United States special forces team had been on 30-minute
standby to mount a rescue raid from the day Ms.~Norgrove was seized.

Beyond defending his own government's approval of the raid, Mr.~Cameron appeared eager to pre-empt
possible second-guessing from military experts and others in Britain who have been quick to condemn
American units for what the critics say is heavy-handedness in counterinsurgency operations, and
suggest that Britain's far smaller, less technology-reliant forces bring an experience and
battlefield wisdom that the Americans lack.

``I'm clear that the best chance of saving Linda's life was to go ahead, recognizing that any
operation was fraught with risk for all those involved, and success could by no means be
guaranteed,'' Mr.~Cameron said in his Monday statement, adding: ``Linda was taken and held in a part
of Afghanistan under U.S.~command. That is why this operation was carried out by U.S.~forces. From
the moment Linda was taken hostage, General Petraeus has treated her as if she was a U.S.~citizen.
He and U.S.~forces did everything in their power to bring Linda home safely.``

Mr.~Cameron said ``those on the ground and in London'' had approved the raid, in part, because they
feared Ms.~Norgrove ``was going to be passed up the terrorist chain, which would increase further
the already high risk that she would be killed.''

``We should also remember that ultimately the responsibility for Linda's death lies with those who
took her hostage,'' at a remote location high in the mountains, he said.

Mr.~Cameron said that he had spoken with Ms.~Norgrove's father, John, who had issued a video appeal
with his wife, Lorna, for the continuation of negotiations for their daughter's release --
negotiations that Mr.~Hague, the foreign secretary, said Monday had never reached ``a serious
stage.''

Senior British officials said the Norgroves were aware that a rescue attempt was a possibility but
had not been asked for their approval. Speaking from his home on the Isle of Lewis after
Mr.~Cameron's statement, Mr.~Norgrove said only, ``We might issue a statement in another day or
two.''

\section{Chinese Cancel Meeting With Norwegian Minister}

\lettrine{C}{hina}\mycalendar{Oct.'10}{12} on Monday abruptly called off a Shanghai meeting with a
Norwegian minister in retaliation for Friday's award of the Nobel Peace Prize to the imprisoned
dissident Liu Xiaobo, a Norwegian official said.

China had earlier warned Norway that giving the prize to Mr.~Liu, who is imprisoned for his
pro-democracy writings, would harm bilateral relations. A spokesman for the Norwegian Foreign
Ministry told Reuters: ``I can confirm that the Chinese have canceled the meeting as a reaction to
the Nobel Peace Prize.''

Before she left Oslo, Fisheries Minister Lisbeth Berg-Hansen told the Norwegian press that any
retaliation was unwarranted because the prize committee was independent of the Norwegian government.

``There is therefore no basis for measures against Norway if someone doesn't like the prize
winner,'' she told the Norwegian daily Dagbladet before leaving for China to commemorate 21 years of
exports of Norwegian salmon. She had already arrived and was preparing for the Wednesday meeting
when she learned it had been canceled, Norwegian authorities said.

There was no mention of the cancellation in the Chinese news media. The Chinese government has
enforced a strict news blackout and imposed stricter security measures, including a roadblock to
Jinzhou prison in Liaoning Province, where Mr.~Liu is being held.

Mr.~Liu, a 54-year-old former literature professor, was sentenced in December to 11 years in prison
for ``inciting subversion of the state.'' His conviction was based partly on Charter 08, a
pro-democracy manifesto he co-authored. The document garnered 10,000 signatures before the
government blocked its circulation on the Internet.

His wife, Liu Xia, was allowed to visit her husband for an hour on Sunday at the prison. In Beijing,
she is being kept under house arrest, according to a statement by the group Human Rights in China.

She said her husband had told her, ``This is for the lost souls of June 4,'' and then wept, the
statement said. Hundreds died when Chinese troops and tanks moved to crush pro-democracy
demonstrations in Tiananmen Square on June 4, 1989.

Ms.~Liu's telephone and Internet communication have been cut off, and state security officers are
not allowing her to contact friends or the news media, the statement by Human Rights in China said.
Nor can she leave her house except in a police car, according to the human rights group.

\section{Gates Urges Improved Military Ties With China}

\lettrine{D}{efense}\mycalendar{Oct.'10}{12} Secretary Robert M.~Gates met here with his Chinese
counterpart on Monday to directly make a case for restoring military-to-military relations broken
off by Beijing in retaliation for American weapons sales to Taiwan.

At the sidelines of a conference for Asian defense ministers, Mr.~Gates spent about half an hour
behind closed doors with Gen.~Liang Guanglie of China, and emerged to say he had explained how arms
sales to Taiwan were a decision by Washington's civilian leadership, not one made by the Pentagon or
the armed forces.

``It is fundamentally a political decision,'' Mr.~Gates said. ``Why the military relationship should
be held hostage to what is essentially a political decision seems to me curious. And I believe it
should not be.''

Restoring communications between the Chinese and American militaries is an urgent need, Mr.~Gates
said, because ``having greater clarity and understanding of each other is essential to preventing
mistrust, miscalculations and mistakes.''

Military relations between the two nations lag far behind their diplomatic and economic ties.
Exchanges between the armed forces were completely frozen by China earlier this year in retribution
for a decision by the United States to sell \$6.4 billion in arms to Taiwan, which Beijing considers
a renegade province.

``Indeed, when there are disagreements, it's all the more important to talk to one another -- more
and not less,'' Mr.~Gates said. ``I have felt for a long time that the dialogue between the two
militaries ought to be sustainable regardless of the ups and downs in the relationship.''

General Liang, as expected, extended an official invitation for Mr.~Gates to travel to Beijing, and
Mr.~Gates plans to make the visit next year.

Even so, a Chinese Defense Ministry spokesman said after the meeting that Taiwan remained the
``biggest obstacle'' to improving military relations between the two countries.

Recent Chinese bullying of its smaller neighbors has figured heavily in discussions in advance of
the session of defense ministers from the Association of Southeast Asian Nations. China has gone so
far as to threaten economic punishment as it has sought to extend its sovereignty over regional
waters.

In his public comments on Monday, Mr.~Gates again showed that he was trying to balance a desire to
restore military relations with China while restate Washington's unwavering support for its
partners, allies and the internationally recognized rights of safe passage through international
waters.

Thus there was no direct criticism of China, but instead a description of discussions about the
broader topic of maritime security. ``We have a shared interest in freedom of navigation and access
to maritime domain,'' Mr.~Gates said.

The theme was repeated as Mr.~Gates's day was spent in a sequence of high-level but low-key sessions
with counterparts not only from China but also from Japan, Vietnam and the Philippines.

The most colorful part of the day came when Mr.~Gates was introduced to an audience of military
personnel and civilian students at Vietnam National University with molar-rattling techno-pop music,
an unexpected accompaniment for the rather traditional defense secretary.

``The U.S.~and Vietnam, as well as other nations in the region, also share a common interest in
maritime security and freedom of access to the global commons,'' Mr.~Gates said in his address, in a
polite reference to the disputes by many of the region's smaller nations with China over claims to
territorial waters.

\section{Norwegian Meeting in China Canceled After Award}

\lettrine{C}{hina}\mycalendar{Oct.'10}{12} blocked European officials from meeting with the wife of
the imprisoned Nobel Peace Prize winner, cut off her phone communication and kept her under house
arrest -- acting on its fury over the award.

As China retaliated, U.N. human rights experts called on Beijing to free democracy campaigner Liu
Xiaobo from prison. Liu, a slight, 54-year-old literary critic, is in the second year of an 11-year
prison term after being convicted of inciting subversion.

He was permitted a brief, tearful meeting in prison with his wife Sunday and said dedicated the
award to the ``lost souls'' of the 1989 military crackdown on student demonstrators.

In naming him, the Norwegian-based Nobel committee honored Liu's more than two decades of advocacy
of human rights and peaceful democratic change -- from demonstrations for democracy at Beijing's
Tiananmen Square in 1989 to a manifesto for political reform that he co-authored in 2008 and which
led to his latest prison term.

On Tuesday, U.S.~officials said they were closely following the situation of Liu's wife, Liu Xia.
``We remain concerned by multiple reports that Liu Xia is being confined to her home in Beijing,''
an U.S.~Embassy spokesman, Richard Buangan, wrote in response to questions. ``Her rights should be
respected, and she should be allowed to move freely without harassment.''

The Beijing public security bureau and the foreign ministry had no immediate comment on why
authorities were apparently restricting her movements since she has not been charged with anything.
But ``soft detention'' is a common tactic used by the Chinese government to intimidate and stifle
activists and critics.

Beijing reacted angrily to Friday's announcement awarding the Nobel Peace Prize to Liu, calling him
a criminal and warning Norway's government that relations would suffer, even though the Nobel
committee is an independent organization. On Monday, it abruptly canceled a meeting that had been
scheduled for Wednesday between visiting Norwegian Fisheries Minister Lisbeth Berg-Hansen and her
Chinese counterpart.

European diplomats were prevented from visiting Liu's wife, Liu Xia, who has been living under house
arrest since Friday. Liu Xia has been told that if she wants to leave her home, she must be escorted
in a police car, the New York-based group Human Rights in China said.

She has reported that her phone communications, along with her Internet, has been cut off; both her
and her brother's mobile phones have been interfered with, HRIC said.

Simon Sharpe, the first secretary of political affairs of the EU delegation in China, said he went
to see Liu Xia at her home in Beijing to personally deliver a letter of congratulations from
European Commission President Jose Manuel Barroso.

Sharpe was accompanied by diplomats from 10 other countries, including Switzerland, Sweden, Poland,
Hungary, the Czech Republic, Belgium, Italy and Australia.

But three uniformed guards at the gate of Liu's apartment complex prevented the group from entering.

``We were told that we could only go in if we called somebody from the inside and if they came out
to meet us. But of course, we can't call Liu Xia, because it's impossible to get through to her
phone,'' Sharpe told reporters at the entrance to the compound.

The Nobel Committee has sent the official prize documents, including an invitation to the Dec.~10
ceremony, to the Chinese Embassy in Oslo, asking Chinese authorities to hand them over to Liu, said
committee secretary Geir Lundestad.

In recent days, Beijing has also stepped up its harassment of other activists, detaining several
when they tried to organize a dinner to celebrate Liu's Nobel.

Zhang Jiannan, who runs an Internet forum on political matters, told The Associated Press that he
and other activists had gone out Friday to celebrate Liu's victory. He was placed under house arrest
Saturday and warned by police not to participate in political activities.

On Monday, lawyer Pu Zhiqiang was the latest to be detained by police, according to his assistant,
who did not want to be identified. Pu had sent out a message via Twitter on Sunday that said
security officials had showed up telling him not to accept interviews with foreign media.

Meanwhile, the Dalai Lama criticized China for its response to the Nobel Peace Prize award, saying
the government ``must change,'' the Kyodo News agency reported. The Tibetan spiritual leader, who
won the prize himself in 1989, said Beijing must recognize that fostering an open society is ``the
only way to save all people of China.''

\section{For Women in France, Having It All Doesn't Mean Having Equality}

\lettrine{C}{ould}\mycalendar{Oct.'10}{12} there be anything more French than this workout?

Weeks after giving birth, French women are offered a state-paid, extended course of vaginal
gymnastics, complete with personal trainer, electric stimulation devices and computer games that
reward particularly nimble squeezing. The aim, said Agnes de Marsac, a physiotherapist who runs such
sessions: ``Making love again soon and making more babies.''

Perineal therapy is as ubiquitous in France as free nursery schools, generous family allowances, tax
deductions for each child, discounts for large families on high-speed trains, and the expectation
that after a paid, four-month maternity leave mothers are back in shape -- and back at work.

Courtesy of the state, French women seem to have it all: multiple children, a job and, often, a
figure to die for.

What they don't have is equality: France ranks 46th in the World Economic Forum's 2010 gender
equality report, trailing the United States, most of Europe, but also Kazakhstan and Jamaica.
Eighty-two percent of French women aged 25-49 work, many of them full-time, but 82 percent of
parliamentary seats are occupied by men. French women earn 26 percent less than men but spend twice
as much time on domestic tasks. They have the most babies in Europe, but are also the biggest
consumers of anti-depressants.

A recent 22-country survey by the Pew Research Center summed it up: three in four French people
believe men have a better life than women, by far the highest share in any country polled.

``French women are exhausted,'' said Val\'erie Toranian, editor-in-chief of Elle magazine in France.
``We have the right to do what men do -- as long as we also take care of the children, cook a
delicious dinner and look immaculate. We have to be superwoman.''

The birthplace of Simone de Beauvoir and Brigitte Bardot may look Scandinavian in employment
statistics, but it remains Latin in attitude. French women appear to worry about being feminine, not
feminist, and French men often display a form of gallantry predating the 1789 revolution. Indeed,
the liberation of French women can seem almost accidental -- a byproduct of a paternalist state that
takes children under its republican wings from toddler age and an obsession with natality rooted in
three devastating wars.

``At the origin, family policy wasn't about women, it was about Germany,'' said Genevi\`eve Fraisse,
author of several books on gender history. ``French mothers have conditions women elsewhere can only
dream of. But stereotypes remain very much intact.''

Or, as the philosopher Bernard-Henri L\'evy put it: ``France is an old Gallic macho country.''

France crystallizes the paradox facing many women across the developed world in the early 21st
century: They have more say over their sexuality (in France birth control and abortion are legal and
subsidized), they have overtaken men in education and are catching up in the labor market, but few
make it to the top of business or politics.

Only one of France's top companies is run by a woman: Anne Lauvergeon is chief executive of the
nuclear power giant Areva and mother of two young children.

Having those children is relatively easy in France, one reason Paris seems to teem with stylish
career women with several offspring.

At 31, Fleur Cohen has four children and works full-time as a doctor at a Left Bank hospital. As she
drops her youngest at nursery in stilettos and pencil skirt you would never guess that she gave
birth only three months ago.

Child No.~4 wasn't ``planned,'' Ms.~Cohen said, but it doesn't change all that much: Instead of
three children, she now takes four on the Metro in the morning and drops them at the public school
and subsidized hospital nursery. She joked that children are probably the best way to reduce your
tax bill. Irrespective of income, parents get a monthly allowance of \texteuro123, or about \$170,
for two children, \texteuro282 for three children and an additional \texteuro158 for every child
after that. Add to that tax deductions and other benefits, and the Cohens pretty much stopped paying
tax after baby No.~3.

Across town, Ms.~de Marsac snapped on a plastic glove, inserted two fingers between Clara Pflug's
legs and told her to think of the wings of a butterfly as she contracted her birth canal muscles.
The French state offers mothers 10 one-on-one, half-hour sessions of perineal therapy to prevent
post-pregnancy incontinence and organ descent -- and to improve sex. Ten sessions of free abdominal
exercises follow; Ms.~de Marsac promises Ms.~Pflug a ``washboard tummy.''

French women have on average two babies, compared with 1.5 in the European Union overall.

Asked by foreign delegations about ``le miracle français,'' Nadine Morano, the feisty family
minister and mother of three, says bluntly: ``We spend the most money and we offer good childcare,
it's as simple as that. Our country understood a long time ago that to reconstruct a nation you need
children.''

The 1870 defeat by a much more fertile Prussia led to first efforts to encourage childbirth. Then
came the losses of World War I.~Since 1920, when the gold Medal of the French Family -- to honor
mothers of eight or more -- was created, expenditure on pro-breeding policies has blossomed. Last
year, \texteuro97 billion, or 5.1 percent of gross domestic product -- twice the E.U. average --
went on family, childcare and maternity benefits.

Emblematic in this regard are the ``\'ecoles maternelles,'' free all-day nursery schools set up a
century after the French revolution in part, said Michelle Perrot, a historian, to stamp out the
lingering influence of the Roman Catholic Church.

La Fl\`eche houses the oldest \'ecole maternelle in France. At 8:30 a.m., parents drop off toddlers
as young as two. Classes end at 4:30 p.m. but a free municipal service offers optional childcare
until 6:30 p.m. Children are guaranteed a place in ``maternelle'' from the age of three and 99
percent of them attend.

Katy de Bresson, a single mother of two, called the enrollment of her son Arthur a
``mini-revolution.'' Free of all childcare costs, she could return to work full-time. ``I am a lot
happier and a lot more self-confident since then.''

Working mothers being the norm, Isabelle Nicolas, a nurse whose youngest son, Titouan, is in
Arthur's class and who quit work after his birth, feels pressure to return. ``I spend a lot of time
justifying myself,'' she said. ``In France you are expected to do it all.''

But ask any mother here whether school had changed the life of her husband and the answer is
``non.''

``The school is called `maternelle' for a reason,'' said principal Anne Leguen. ``In France,
children are still considered to be the responsibility of mothers.''

Forty percent of French mothers undergo a career change within a year of giving birth, compared with
6 percent of men. Both parents have the right to take time off or reduce their hours until the child
turns three -- but 97 percent of those who do are women.

Women spend on average five hours and one minute per day on childcare and domestic tasks, while men
spend two hours and seven minutes, according to the national statistics office Insee.

In Paris, Ms.~Cohen's husband is a doctor, too. But she bathes all four children, cooks and does the
Saturday shopping -- largely, she insists, by choice. ``If I didn't prepare food for my children, I
would feel less like a mother,'' she said.

At work, meanwhile, she plays down motherhood. She sneaks down to the hospital nursery to nurse her
baby son, and tries to stay longer than her male colleagues in the evenings. Otherwise, ``everyone
will just assume that I'm leaving because of my children and that I am not committed to the job.''

A majority of medical graduates in France are female. Yet all 11 department heads in her hospital
are men.

``French men have always been slow to give up power,'' said Jean-François Cop\'e, parliamentary
leader of President Nicolas Sarkozy's center-right party, who is defending a bill to oblige
companies to fill 40 percent of boardroom seats with women.

The French Republic made ``equality'' a founding principle, but women were allowed to vote for the
first time only in 1945. Since a 1998 law obliged political parties to have an equal number of men
and women candidates on their party lists, parties have tended to pay fines rather than comply.

Women leaders come under close scrutiny in what is after all the home of couture. Ms.~Morano recalls
being mocked on television for wearing the same jacket several times. Ms.~Lauvergeon likened her
outfit to ``armor.''

Four pieces of equal pay legislation have passed since 1972. But in 2009, even childless women in
their forties still earned 17 percent less than men.

``A patriarchal corporate culture,'' is the main barrier facing women in French companies, according
to Brigitte Gr\'esy, author of a 2009 report on gender equality in the workplace.

France is Latin not just in its culture of seduction, but also in its late work hours, Ms.~Gr\'esy
said. And the disproportional weight of a small number of male-dominated engineering schools in
grooming the elites has done its part in excluding women from power. Xavier Michel, president of
École Polytechnique, points out that the number of female students has risen tenfold from seven to
70 since he graduated in 1972 -- but that leaves it at just 14 percent.

Simone Veil was 18 when French women first voted and 28 when she was allowed to open her own bank
account. At 38, as health minister, she pushed through the legalization of abortion. ``A lot has
changed, but a lot hasn't,'' she says today. More comfort to her than many of the laws in recent
years is the fact that more fathers push strollers through her neighborhood.

Ms.~Fraisse, the philosopher, says more than two centuries after France got rid of the king as the
father of the nation, it needs to get rid of the father as the king of the family. ``We had one
revolution,'' she said, ``now we need another one -- in the family.''

\section{Economists Share Nobel for Studying Job Market}

\lettrine{P}{eter}\mycalendar{Oct.'10}{12} A.~Diamond, a nominee for a Federal Reserve Board
position, and two other economists were awarded the 2010 Nobel Memorial Prize in Economic Science on
Monday for their work on markets where buyers and sellers have difficulty finding each other.

The work of the winners, Professor Diamond of the Massachusetts Institute of Technology, Dale
T.~Mortensen of Northwestern University and Christopher A.~Pissarides of the London School of
Economics, is best known for its applications to the job market.

The researchers spent decades trying to understand why it takes so long for people to find jobs,
even in good economic times, and why so many people can be unemployed even when many jobs are
available.

Traditional economics, after all, would predict that wages should simply drop, helping the labor
supply to meet labor demand automatically and sweeping jobless workers into whatever positions were
immediately open.

These researchers' explanation addresses the complications that come from searching for jobs and job
candidates: it takes time for unemployed workers to be matched with the proper opening, since people
are not identical, cookie-cutter units, and neither are jobs.

While all this may seem intuitive, in the 1970s it was considered quite radical. The resulting
insights about how search costs can affect markets also helped revolutionize not only labor
economics, but fields like public finance and housing economics as well. The work is especially
relevant today, as policy makers try to understand and combat the causes of stubbornly high
unemployment in countries like the United States.

In a phone interview, Professor Diamond, 70, said that one of the implications of his work was that
more fiscal and monetary stimulus was probably necessary to speed up job growth.

``The slower it happens, the more workers lose their skills and stop searching, and so the process
goes more poorly after that,'' Professor Diamond said.

President Obama nominated Professor Diamond in April for a Fed board position, where he would serve
under his former student, Ben S.~Bernanke, the Fed chairman. But in August, under an obscure
procedural rule, the Senate sent Mr.~Diamond's nomination back to the White House before starting
its summer recess, and a senator questioned his experience.

President Obama renominated Professor Diamond for the Fed position on Sept.~13. A hearing on his
confirmation is still to come.

The work honored Monday also suggests that policies intended to help workers can have unintended
consequences. Unemployment benefits, for example, can prolong joblessness by making it less costly
to be without work.

``That's a big controversy in the U.S.~recently,'' said Robert Shimer, an economics professor at the
University of Chicago . ``Most of these models suggest that even in a depressed economy, more
generous unemployment benefits tend to raise the unemployment rate. Benefits are obviously good for
the unemployed, but there are some clear tradeoffs.''

The models help explain why European labor markets tend to be much more rigid than American ones,
where people can move from job to job relatively easily, at least in good times.

``Many European countries put restrictions on the ability of firms to hire and fire,'' said Lawrence
F.~Katz, a Harvard economist. ``If you make it harder to hire and fire, then you end up with what's
called a sclerotic labor market, with less movement between jobs and more long-term unemployment.''

Europe's struggles in the 1970s and 1980s with an underclass of chronically unemployed workers
helped inspire Professor Pissarides, 62, a Cyprus native, to study the search costs of labor markets
in the first place, he said.

Monday's announcement also played into current debates about the government's role in addressing
long-term unemployment and about whether the elevated unemployment levels today represent a ``new
normal.''

``I think the economy is very adaptive,'' Professor Diamond said in a news conference at M.I.T.
``Workers and employers will adapt to what will make the economy function. I see no reason why, once
we get fully over this, we won't go back to normal times,'' with more ``normal'' unemployment rates.

Professor Mortensen, 71, of Northwestern, said additional measures to get credit functioning more
normally, and in particular to make it easier for small businesses to get loans, were crucial to
reducing unemployment.

``From my perspective the problem right now is not the labor market,'' he said during a phone call
with reporters. ``What's happening in the labor market is a symptom of more complicated problems
with the financial market.''

The line of research begun by the three Nobel laureates is still active today. ``Search theory'' has
been applied to many other areas, like money systems and venture capital markets -- really, any
market that can be considered heterogeneous.

``Which is most markets,'' said Robert E.~Hall, a Stanford economist, ``except for maybe things like
grain.''

Justin Wolfers, a University of Pennsylvania economist, has applied the theory in his own work on
marriage and divorce, for example.

``Labor economists think about firing costs, and family economists think about divorce costs,''
Professor Wolfers said. Just as restrictions on firing an employee make fewer workers available for
new positions -- and therefore make companies skittish about making too many changes to their work
force -- low divorce rates can be self-perpetuating. With divorces rare, unhappy spouses may think
twice about getting a divorce themselves, since there would be so few eligible new mates available.

While the applications of Professor Diamond's Nobel-winning work are broad, his practical experience
was questioned this year when he was nominated for the Fed governor position.

In August, Senator Richard C.~Shelby, Republican of Alabama, said Professor Diamond did not have
enough experience for the position. The Web site CqPolitics.com quoted him as saying, ``I do not
believe that the current environment of uncertainty would benefit from policy decisions made by
board members who are learning on the job.''

It is believed that Senate Republicans will ultimately not try a filibuster to block Professor
Diamond now that he has been renominated. (His nomination was said to have been initially blocked in
retaliation for a refusal by Democrats to give a full 14-year term to Randall S.~Kroszner, who
served on the Fed board from 2006 to 2009.)

If confirmed, Professor Diamond would complete a 14-year term that expires on Feb.~1, 2014.

The Nobel in economic science is awarded by the Royal Swedish Academy of Sciences and is not one of
the original prizes created by Alfred Nobel. In addition to a medal and a diploma, the laureates
collectively will receive 10 million Swedish kronor, or about \$1.5 million.

\section{Europe May Ease Jet Carbon Fees}

\lettrine{T}{he}\mycalendar{Oct.'10}{12} transportation chief of the European Union said Monday that
airlines based in the United States could receive an exemption, at least in part, from European
carbon regulations if Washington moved to reduce greenhouse gas emissions at home.

``We are ready to negotiate and to talk about these issues and not only make declarations,'' Siim
Kallas, the European commissioner for transportation, said during a news conference. ``Adequate
measures from other countries can be taken into account.''

The European Union agreed two years ago to include in the regulations all airlines taking off from,
and landing in, the European Union starting Jan.~1, 2012.

The law is the boldest move yet by the bloc to push the rest of the world to comply with its climate
policies. It has led to widespread criticism from the airline industry, especially from carriers in
the United States.

Under the law, airlines would be charged for only about 15 percent of the cost of permits needed to
cover their emissions until the end of the decade. Still, compliance would cost the industry at
least 2.4 billion euros, or about \$3.3 billion, a year, according to the International Air
Transport Association, a trade group

Mr.~Kallas did not give any details about what the other nations could do to qualify for a partial
exemption, but he suggested that there was ``enough flexibility'' in the legislation to reach an
amicable agreement with countries like the United States.

Any exemption would probably apply only to travel to Europe from the United States and other parts
of the world. Airlines would probably still need to hold enough European pollution permits to
compensate for emissions from flights taking off from the Continent.

Environmental groups were already concerned that Mr.~Kallas had offered too many concessions in the
last two weeks during negotiations at a meeting of the International Civil Aviation Organization, a
United Nations body, to gain greater acceptance for European emissions policies.

A final resolution by the aviation group over the weekend said all countries with less than a 1
percent share of air traffic could be made exempt from so-called action plans on emissions.

Bill Hemmings, a program manager for Transport and Environment, an independent group in Brussels
that promotes environmentally sound travel, said that the European Union ``already was paying a
heavy price to weaken opposition to its plan to include international aviation in its trading
system'' by agreeing to those concessions.

Mr.~Hemmings also warned that if the union excluded inbound flights by airlines from around the
globe, including the United States, that would cut the emissions covered by the system by 40
percent, drastically eroding its effectiveness.

Whether American carriers would be able to avoid the part of the charge relating to flights taking
off from the United States was unclear.

The United States is still considering a range of measures to cut greenhouse gas emissions,
including a carbon tax and a ``cap and trade'' market similar to Europe's Emission Trading System.
None of those options is likely to be in place in the United States before 2012.

Speaking on condition of anonymity, so as not to prejudge what the United States might offer,
European Union officials said American carriers might be able to qualify for an exemption on
outgoing flights if the United States government agreed on one or more measures, like better air
traffic management to cut carbon emissions in the United States and a passenger tax at United States
airports representing a fee on such emissions.

A spokesman from the Federal Aviation Administration could not be reached Monday for comment.

The United States, Canada and Mexico initially urged the United Nations aviation body to adopt a
resolution stating that countries that wanted to establish an emission trading system covering
airlines from other countries do so only ``on the basis of mutual agreement.''

The language in the final resolution was milder, calling for ``consultations and negotiations with
other states.'' It also called on member countries to improve the fuel efficiency of airlines
through the middle of the century.

\section{Guangzhou Smoothes Out Last Details of Asian Games}

\lettrine{T}{he}\mycalendar{Oct.'10}{12} roads are freshly paved, taxi drivers are practicing
English phrases and rehearsals are in full swing at the opening ceremony venue on a tiny island in
the Pearl River.

The Asian Games will get under way in one month, bringing together more than 11,500 athletes
competing in 42 sports. The Olympics-style event is shaping up to be another well-run,
well-organized spectacular by the Chinese government, though wide-ranging preparation projects set
off grumbling among some residents who see them as an inconvenience and waste of money.

``The Asian Games will elevate Guangzhou's status in the world. It will make the city more
international. But if all things are decided by the will of the authority, while suppressing public
supervision and criticism, then it will also have a negative side,'' said Li Gongming, a former
delegate to the provincial legislative advisory body who has been an outspoken critic.

China's government has sought prestige projects like the 2008 Beijing Olympics and the Shanghai
World Expo this year as a way of bolstering its status as an emerging world power. The Guangzhou
Asian Games are ``a platform to show the tremendous achievements of China's economic and social
development,'' the organizing committee said in a statement.

It is uncommon in China for ordinary citizens to speak out for fear of retribution from the
authorities, but this issue has touched a nerve in Guangzhou. Residents here have often identified
with Hong Kong, the former British colony just two hours away by train that operates under a system
separate from the mainland and enjoys Western-style civil liberties.

``I'm not thinking about the Asian Games, I'm concerned about my house,'' said 87-year-old Tan
Yisheng, who lives in a historic part of Guangzhou slated to be demolished and converted into an
upscale shopping and entertainment district. That project is not directly related to the Asian
Games, though brick walls were hastily built around the rubble of early 1900s homes recently to hide
the mess from visitors next month.

As the clock ticks down to the 9:42 p.m. torch lighting on Nov.~12, some minor construction projects
around Guangzhou have yet to wrap up. But there is nothing like the panic in New Delhi before the
Commonwealth Games, which were so behind schedule and plagued with problems that some wondered
whether the event would happen at all.

China is going to great lengths to create a flawless event. Guangzhou has repaved its roadways,
built new subway lines and planted colorful flowers along streets and overpasses.

There is a sparkling ``Asian Games Town'' a half-hour outside the city center where high-rise
apartments will house athletes, journalists and officials. Fleets of minibuses are already parked in
neat rows, surrounded by manicured lawns dotted with picturesque stone pavilions.

More than 500,000 volunteers -- drawn from 1.5 million applicants -- will be dispatched throughout
the city. Cab drivers have been taking mandatory English lessons for months, learning phrases like
``Welcome to the 16th Asian Games!'' and ``Here is your change.''

The organizing committee did not answer a question asking how much had been spent on improvements
around the city, but it said in its statement that it was grateful to local residents for their
``understanding and support.''

``Preparation work for the Asian Games is basically completed; it has entered the final stage. There
is a smaller impact on the daily lives of Guangzhou residents,'' it said. For their troubles,
residents will get free public transportation during the games, free wireless Internet in some parts
of the city for two months and be eligible for ticket giveaways and other incentives.

So far, ticket sales have been strong, especially for events like badminton and table tennis, said
Jerry Wu, deputy director of the ticketing center, adding that a final batch of tickets would be
released for sale later this month. Tickets to the opening ceremonies, which cost as much as 6,800
yuan, or \$1,000, are sold out.

\section{Microsoft Introduces (and Pins Its Hopes on) Windows Phone 7}

\lettrine{M}{icrosoft}\mycalendar{Oct.'10}{12} on Monday formally unveiled its Windows Phone 7
mobile operating system, which it hopes will give it a greater presence in the rapidly expanding
market for smartphones.

Phones running the new software will be available on AT\&T's network starting on Nov.~8, said Steven
A.~Ballmer, Microsoft's chief executive, at a news conference in New York. Later in the year, the
phones will be available from T-Mobile, and in 2011, from Verizon and Sprint. AT\&T will initially
offer three phones, at \$200 each.

In all, the operating system will run on nine new phones, made by LG, Samsung, HTC and Dell,
Mr.~Ballmer said. Some will have keyboards, he said, while others will be exclusively touch screen.

Windows Phone 7 has been in production for two years, and is the culmination of a long effort by
Microsoft to assert itself in the smartphone market, which has grown exponentially over the last few
years. Smartphones now account for 23 percent of all mobile phones sold in the United States,
according to the research firm Nielsen. Apple's iPhone, Google's Android operating system and
Research In Motion's BlackBerry have taken the lead in new sales, while Microsoft has lagged behind.

Microsoft's earlier mobile software was based on the design and interface of its Windows desktop
operating system. Although those phones showed some early promise, their growth slowed drastically
when Apple introduced the iPhone in 2007.

Earlier this year, the company released the Kin, a series of phones aimed at younger users with
integrated social networking. But the products sold poorly and were withdrawn from stores 48 days
after their release.

Microsoft has tried to distinguish Windows Phone 7 devices from others, especially the iPhone, by
promoting features that allow users to personalize the look and feel of their phones. In his remarks
on Monday, Mr.~Ballmer stressed these features, including ones that allow updates from social
networks and messaging services to appear automatically. ``We focused in on how real people really
want to use their phones when they are on the go,'' he said.

Microsoft also plans to introduce a new app marketplace for the Windows Phone 7 platform, company
officials said in interviews after the news conference, and is working with partner developers to
build mobile applications for the new devices. For mobile application developers, it is one more
challenge. They already struggle with trying to decide which platforms to focus on, primarily the
iPhone and Android operating systems.

Consumers will be able to integrate the new phones with a number of Microsoft products, including
Zune music and video content, the Bing search engine, business products like Microsoft's OneNote
software and the Xbox gaming platform. Some developers plan to build games that can be played on
both the Xbox 360 and Windows Phone 7, Microsoft officials said.

The AT\&T phones will work with its U-verse TV service, allowing users to download television shows
to their devices, among other features.

\section{Vaccine Case to Be Heard by Justices}

\lettrine{T}{he}\mycalendar{Oct.'10}{12} safety of vaccines is at the heart of a case expected to be
heard on Tuesday by the United States Supreme Court, one that could have implications for hundreds
of lawsuits that contend there is a link between vaccines and autism.

At issue is whether a no-fault system established by Congress about 25 years ago to compensate
children and others injured by commonly used vaccines should protect manufacturers from virtually
all product liability lawsuits. The law was an effort to strike a balance between the need to
provide care for those injured by vaccines, some of them severely, and the need to protect
manufacturers from undue litigation.

Under the 1986 National Childhood Vaccine Injury Act, such claims typically proceed through an
alternative legal system known as ``vaccine court.'' Under that system, a person is compensated if
their injury is among those officially recognized as caused by a vaccine. That person, or their
parents, can choose to reject that award and sue the vaccine's manufacturer, but they then face
severe legal hurdles created by law to deter such actions.

The case before the Supreme Court is not related to autism. But the biggest effect of the court's
ruling, lawyers said, will be on hundreds of pending lawsuits that contend a link exists between
childhood vaccines and autism. Repeated scientific studies have found no such connection.

Also, in several test case rulings over the last two years, administrative judges in vaccine court
have held that autism-related cases did not qualify for compensation. During the last decade, about
5,800 of the 7,900 claims filed in vaccine court, or about 75 percent, have been autism-related,
federal data show.

Federal data shows that \$154 million was paid in fiscal 2010 to 154 claimants involved in vaccine
court proceedings. That figure was significantly higher than in preceding years and reflected
several unusually high awards, officials involved in the program said.

In the five preceding fiscal years, an average of \$68 million in compensation was paid out on an
annual basis, federal data indicates. A compensation fund is financed by an excise tax on
vaccinations.

The case to be heard on Tuesday involves an 18-year-old woman, Hannah Bruesewitz, who suffered
seizures when she was 6 months old and subsequently suffered developmental problems, her parents
say, after receiving a type of D.T.P. vaccine that is no longer sold. The D.T.P. vaccine protects
against three potentially deadly childhood diseases: diphtheria; pertussis, which is also known as
whooping cough; and tetanus.

Ms.~Bruesewitz's parents have contended in court papers that the vaccine's manufacturer, which is
now a part of Pfizer, knew at the time that their daughter was immunized that there was a safer
version of the D.T.P. vaccine but did not produce it. The company rejected that contention

Initially, Ms.~Bruesewitz's parents brought a claim on her behalf to the vaccine court, but the
severe injuries that she reportedly suffered were removed from the list of those that qualified for
compensation a month before the case was heard. An administrative judge in vaccine court
subsequently rejected her claim, so her parents filed a product liability lawsuit against Wyeth, a
Pfizer unit that had acquired the vaccine's manufacturer, Lederle Laboratories.

Lower court judges have ruled that her claims are barred by the federal Vaccine Act. As a result,
Ms.~Bruesewitz, who lives with her family in Pittsburgh and requires specialized care, has not
received any compensation, her father, Russell Bruesewitz, said in a telephone interview.

``The cost of her care is an ongoing burden,'' Mr.~Bruesewitz said.

The Supreme Court review revolves around the narrow question of whether Congress in passing the
Vaccine Act intended to bar lawsuits against vaccine manufacturers based on so-called design defect
claims. A vaccine design defect claim essentially asserts that the manufacturer should have sold a
different vaccine, which plaintiffs say would have been safer than the one used.

Those filing briefs arguing that Congress intended to permit such lawsuits include the American
Association for Justice, a plaintiffs' lawyers group, and the National Vaccine Information Center,
an advocacy group.

Those filing briefs arguing that Congress intended to bar them include the solicitor general of the
United States, the Chamber of Commerce and several professional medical groups, including the
American Academy of Pediatrics.

James M.~Beck, a lawyer in Philadelphia who defends makers of drugs and medical devices, said in a
phone interview that a ruling in favor of the Bruesewitzes would allow hundreds of lawsuits
asserting a link between vaccines and autism to go forward.

``If these cases go forward, it will make it economically unfeasible for anyone to make vaccines in
this country,'' said Mr.~Beck.

Mr.~Bruesewitz said that he and his wife were not opposed to vaccination. Instead, he said they
pressed his daughter's claim because he thought that vaccine producers needed to face the threat of
litigation to produce safer medications.

``What we want and are concerned about is to make sure that the safety of vaccines in this country
is constantly enhanced,'' Mr.~Bruesewitz said.

\section{G.O.P. Widens Targets for Picking Up House Seats}

\lettrine{R}{epublicans}\mycalendar{Oct.'10}{12} are expanding the battle for the House into
districts that Democrats had once considered relatively safe, while Democrats began a strategy of
triage on Monday to fortify candidates who they believe stand the best chance of survival.

As Republicans made new investments in at least 10 races across the country, including two
Democratic seats here in eastern Ohio, Democratic leaders took steps to pull out of some races
entirely or significantly cut their financial commitment in several districts that the party won in
the last two election cycles.

Representatives Steve Driehaus of Ohio, Suzanne M.~Kosmas of Florida and Kathy Dahlkemper of
Pennsylvania were among the Democrats who learned that they would no longer receive the same
infusion of television advertising that party leaders had promised. Party strategists conceded that
these races and several others were slipping out of reach.

With three weeks remaining to save its majority, the Democratic Congressional Campaign Committee has
increased its spending on two New York races, along with at-risk seats in Colorado, Georgia,
Illinois, Kentucky and Massachusetts, setting up a map of competitive districts that is starkly
different from when the campaign began.

The strategic decisions unfolded at a feverish pace on Monday over an unusually wide playing field
of nearly 75 Congressional districts, including here in Ohio, a main battleground in the fight for
the House and the Senate. The developments resembled pieces being moved on a giant chess board, with
Republicans trying to keep Democrats on the defensive in as many places as possible, while outside
groups provided substantial reinforcements for Republicans.

The National Republican Congressional Committee, the party's election arm in the House, can afford
to make the new investments because the U.S.~Chamber of Commerce and a host of newly formed
political organizations have come to the aid of Republican candidates who have far less money than
the Democratic incumbents.

Here in St.~Clairsville, an Appalachian town on the eastern edge of Ohio, the new investments by
Republican groups have become apparent in recent days. Television and radio advertisements are aimed
at Representatives Charlie Wilson and Zack Space, both Democrats who were elected in 2006, while new
pieces of literature tying the men to President Obama and the House speaker, Nancy Pelosi, are
arriving in the mail.

The two districts, which come together like long and jagged pieces of a puzzle, are among Ohio's
most rural and conservative. Yet even though Senator John McCain carried the region over Mr.~Obama
in the 2008 presidential race, Republican leaders had initially decided against making major
investments because they believed there were greater opportunities elsewhere in the state and
because both congressmen had strong connections to the area.

But polls taken for their Republican candidates showed steady signs of promise, party officials
said, so over the weekend the national party made an initial expenditure of \$350,000 on television
commercials in both districts. Democratic strategists believe that the spending is either designed
to be a head fake, so they are drawn into spending money on the races, or a signal to outside
groups, who are prohibited from coordinating with the party, to begin making their own forays into
the contests.

For months, Bill Johnson, the Republican challenger to Mr.~Wilson, has drawn little notice and has
struggled to raise money. But last week, things began to change.

He was invited to be the guest speaker at a weekly meeting of conservative leaders in Washington
that is organized by Grover Norquist, the president of Americans for Tax Reform. Then he appeared on
G.~Gordon Liddy's radio show, which he said helped his fund-raising efforts, as did an endorsement
from Sarah Palin.

``It is a good year to be running as a Republican,'' Mr.~Johnson said in an interview on Monday as
he drove across the sprawling Sixth District, which stretches 325 miles across 12 counties. ``People
are concerned about rising unemployment, spending and the overreaching of the federal government.''

Mr.~Johnson, a businessman and retired Air Force officer, has been largely ignored by Mr.~Wilson. He
has criticized Mr.~Wilson for declining to agree to debates. But the race gained attention over the
weekend when the Republican committee's advertisements began appearing on television, calling
Mr.~Wilson ``party line Charlie'' and highlighting his votes in favor of the economic stimulus and
health care measures.

The message was amplified in a radio advertisement playing on a country music station here, with
Mr.~Johnson saying in a chipper voice: ``On Election Day, it's time we say, `So long, Charlie!' ''

The race is springing to life here just as early voting is entering its second full week. Campaign
signs for Mr.~Johnson and Mr.~Wilson can be found in equal measure in Ohio River towns from
Bridgeport to Brilliant to Bellaire.

Mr.~Wilson, who through a spokeswoman declined an interview on Monday because he was meeting with
newspaper editorial boards in his district, has begun striking back. He argues in his own television
advertisements that he stood up to Democratic Party leaders on climate change legislation, which he
calls an ``energy tax,'' before closing with a line, ``I'm Charlie Wilson, and I'm fed up.''

The outcome of these Ohio races, along with other contests in the newly expanded Republican
battleground, will help determine whether projections of a Republican wave are realized. Democrats
dismissed the notion that Republicans were actually expanding the playing field, suggesting that
they were looking for new opportunities because efforts to knock out Democratic incumbents have
proved difficult.

Ed Good, the chairman of the Belmont County Democratic Party here in St.~Clairsville, said voters
were angry and frustrated and eager to ``shoot the messenger, if you will.'' A Tea Party rally is
scheduled for Thursday on the steps of the courthouse, the latest in a string of events that
suggests the political forces may be different for Democrats this year.

``They are going to try to pick off what they think is low-hanging fruit,'' Mr.~Good said. ``But the
only way Charlie or Zack can lose is if our party does not get out and vote.''

\section{Understanding `Ba Ba Ba' as a Key to Development}

\lettrine{A}{s}\mycalendar{Oct.'10}{12} a pediatrician, I always ask about babble. ``Is the baby
making sounds?'' I ask the parent of a 4-month-old, a 6-month-old, a 9-month-old. The answer is
rarely no. But if it is, it's important to try to find out what's going on.

If a baby isn't babbling normally, something may be interrupting what should be a critical chain:
not enough words being said to the baby, a problem preventing the baby from hearing what's said, or
from processing those words. Something wrong in the home, in the hearing or perhaps in the brain.

Babble is increasingly being understood as an essential precursor to speech, and as a key predictor
of both cognitive and social emotional development. And research is teasing apart the phonetic
components of babble, along with the interplay of neurologic, cognitive and social factors.

The first thing to know about babble is also the first thing scientists noticed: babies all over the
world babble in similar ways. During the second year of life, toddlers shape their sounds into the
words of their native tongues.

The word ``babble'' is both significant and representative -- repetitive syllables, playing around
with the same all-important consonants. (Indeed, the word seems to be derived not from the biblical
Tower of Babel, as folk wisdom has it, but from the ``ba ba'' sound babies make.)

Some of the most exciting new research, according to D.~Kimbrough Oller, a professor of audiology
and speech-language pathology at the University of Memphis, analyzes the sounds that babies make in
the first half-year of life, when they are ``squealing and growling and producing gooing sounds.''
These sounds are foundations of later language, he said, and they figure in all kinds of social
interactions and play between parents and babies -- but they do not involve formed syllables, or
anything that yet sounds like words.

``By the time you get past 6 months of age, babies begin to produce canonical babbling, well-formed
syllables,'' Professor Oller said. ``Parents don't treat those earlier sounds as words; when
canonical syllables begin to appear, parents recognize the syllables as negotiable.'' That is, when
the baby says something like ``ba ba ba,'' the parent may see it as an attempt to name something and
may propose a word in response.

Most of the time, I ask parents: ``Does he make noise? Does she sound like she's talking?'' And most
of the time, parents nod and smile, acknowledging the baby voices that have become part of the
family conversation.

But the new research suggests a more detailed line of questions: by 7 months or so, have the sounds
developed into that canonical babble, including both vowels and consonants? Babies who go on
vocalizing without many consonants, making only aaa and ooo sounds, are not practicing the sounds
that will lead to word formation, not getting the mouth muscle practice necessary for understandable
language to emerge.

``A baby hears all these things and is able to differentiate them before the baby can produce
them,'' said Carol Stoel-Gammon, an emeritus professor of speech and hearing sciences at the
University of Washington. ``To make an m, you have to close your mouth and the air has to come out
your nose. It's not in your brain somewhere -- you have to learn it.''

The consonants in babble mean the baby is practicing, shaping different sounds by learning to
maneuver the mouth and tongue, and listening to the results. ``They get there by 12 months,''
Professor Stoel-Gammon continued, ``and to me the reason they get there is because they have become
aware of the oral motor movements that differentiate between a b and an m.''

Babies have to hear real language from real people to learn these skills. Television doesn't do it,
and neither do educational videos: recent research suggests that this learning is in part shaped by
the quality and context of adult response.

To study babbling, researchers have begun to look at the social response -- at the baby and the
parent together. Michael H.~Goldstein, an assistant professor of psychology at Cornell, has done
experiments showing that babies learn better from parental stimulation -- acquiring new sounds and
new sound patterns, for example -- if parents provide that stimulation specifically in response to
the baby's babble.

``In that moment of babbling, babies seem to be primed to take in more information,'' he said.
``It's about creating a social interaction where now you can learn new things.''

A study this year by this group looked at how babies learn the names of new objects. Again, offering
the new vocabulary words specifically in response to the babies' own vocalizations meant the babies
learned the names better.

The experimenters argue that a baby's vocalizations signal a state of focused attention, a readiness
to learn language. When parents respond to babble by naming the object at hand, the argument goes,
children are more likely to learn words. So if a baby looks at an apple and says, ``Ba ba!'' it's
better to respond by naming the apple than by guessing, for example, ``Do you want your bottle?''

``We think that babies tend to emit babbles when they're in a state where they're ready to learn new
information, they're aroused, they're interested,'' Professor Goldstein said. ``When babies are
interested in something, they tend to do a furrowed brow,'' he continued; parents should understand
that babble may be ``an acoustic version of furrowing one's brow.''

Right there, in the exam room, I have that essential experimental combination, the baby and the
parent. It's an opportunity to check up on the baby's progress in forming sounds, but also an
opportunity to help parents respond to the baby's interest in learning how to name the world -- a
universal human impulse expressed in the canonical syllables of a universal human soundtrack.

\section{In Vietnam, Gates to Discuss Maritime Claims of China}

\lettrine{D}{efense}\mycalendar{Oct.'10}{12} Secretary Robert M.~Gates landed Sunday in Vietnam,
where the narrative of a past war with the United States has faded as the leadership here openly
seeks American support to counter an increasingly assertive China.

Mr.~Gates has scheduled private talks with his Vietnamese counterpart during a conference of defense
ministers from across the region, where a key issue will be how to manage China's expanded claims of
maritime rights in the South China Sea. China has backed those claims with threats of economic
retaliation against some nations in the region.

A senior Defense Department official traveling aboard Mr.~Gates's airplane to Hanoi said the defense
ministers from the Association of Southeast Asian Nations would look for common ground on the issues
of counterterrorism, peacekeeping and, with China in attendance, a response to Beijing's push for
increased sovereignty over international waters.

Mr.~Gates faces a delicate balancing act. He must reassure Asian partners and allies that the United
States will remain engaged in the region and will work for a peaceful resolution of the competing
claims over islands, undersea mineral wealth and fishing rights. But he must do so without
jeopardizing his equally important efforts to restore a healthy military-to-military dialogue with
China.

China and the United States have already sparred over China's claims in the South China Sea, with
the United States allied with Vietnam on the issue. In March, at least one senior Chinese official
raised the level of its claim, asserting to two senior White House officials visiting Beijing that
the South China Sea was a ``core interest,'' a phrase that placed it on a par with Taiwan and Tibet,
which China considers parts of its territory.

In Hanoi in July, Secretary of State Hillary Rodham Clinton hardened Washington's stance by saying
the United States had a ``national interest'' in freedom of navigation in the area.

The defense secretary's expected arguments to China are clear: Beijing's dash to become a global
economic power requires it to honor accepted standards for sharing oceans and airspace, and
harassment of ships and airplanes in international lanes off its shores will harm China's long-term
interests.

China is expected to invite Mr.~Gates to Beijing, a significant change in tone. China froze military
relations with the United States this year when the Obama administration announced \$6.4 billion in
arms sales to Taiwan.

Mr.~Gates arrived in Vietnam 15 years after normalization of relations between the two countries,
but the streets were overflowing with revelers for another celebration, the 1,000th anniversary of
the founding of Hanoi. China and Vietnam have a long history of bloody competition, one that was
buried for the years that China backed North Vietnam in pushing back American military involvement
here.

Vietnam's worries over Chinese encroachment were reflected in its recent choices for weapons
purchases. Last year, Vietnam signed deals with Russia to buy six Kilo-class diesel-powered
hunter-killer submarines for \$1.8 billion and eight Sukhoi jet fighters for another \$500 million,
according to the Congressional Research Service. Both weapons are designed for protecting
territorial waters and airspace, and the deals also illustrate Russia's support of nations trying to
curb China's power.

The United States, while seeking to improve diplomatic and military relations with Vietnam, has
offered little in the way of arms, mostly focusing its assistance on military training and officer
education. Washington has continuing human rights concerns with Vietnam, mostly about ensuring
freedom of religion here.

\section{Rescuers Lower Escape Capsule in Test at Chilean Mine}

\lettrine{C}{hilean}\mycalendar{Oct.'10}{13} officials late Tuesday began testing a capsule designed
to rescue 33 trapped miners whose two-month struggle for survival has inspired the nation and
riveted the world.

The capsule was sent, unmanned, partway down a narrow, nearly half-mile shaft that leads to the
haven of the trapped men in the collapsed mine here. The capsule was then lifted out of the shaft,
and technicians continued making final adjustments.

If workers are able to lower the capsule all the way down the shaft, they plan to raise the capsule,
still empty, before lowering a rescuer down the shaft in the next run. Eventually, the capsule is to
lift the 33 miners, one at a time, to the surface of the earth.

Each leg of the trip could take as little as 11 minutes, but each round trip could take an hour, and
that is if all goes well.

The Chilean mining minister, Laurence Golborne, appeared before cameras Tuesday evening, to say that
a test of the capsule would begin shortly, but the start of the actual rescue was about two hours
away. The capsule, he said, was still being readied, and a communication system had to be finalized.

The final preparations pushed back the expected 8 p.m. start of the operation, which has become a
made-for-television event, with the Chilean government offering a free, live satellite feed of the
rescue.

The relatives' makeshift tent city, Camp Hope, vibrated with a carnival-like atmosphere. ``The day
has finally arrived,'' said Marta Mesías, 51, the aunt of one the miners, Claudio Y\'a\~{n}ez, 34.
She said she had traveled here from the capital, Santiago, to greet Mr.~Y\'a\~{n}ez when he emerges
from under the earth. ``We're going to toast him with champagne, and feed him a bit of roasted
chicken.''

The hundreds of journalists who have been crowding into the area have been broadcasting live reports
and phoning in news feeds, as officials made hopeful and patriotic declarations to a country -- and
a world -- captivated by the resilience of men who have survived under thousands of feet of rock for
69 days. Chilean television maintained a live feed of the scene, and some towns in the country
gathered around large-screen televisions in growing excitement.

Rescuing the miners has been an obsession for President Sebasti\'an Pi\~{n}era and his government,
which has spared no expense, technological consultation or innovation -- and has been rewarded with
a national boost of popularity. Mr.~Pi\~{n}era, a billionaire businessman, arrived at the camp by
helicopter in the late afternoon.

``Today I feel incredibly emotional -- the same way that all Chileans feel,'' the Telegraph of
London quoted him as saying.

The Bolivian leader, Evo Morales, was also expected. One of the miners is a young Bolivian who has
become a hero back home.

The miners' ordeal has presented Mr.~Pi\~{n}era, one of Latin America's most conservative leaders,
and Mr.~Morales, one of its most radical, a high-profile opportunity to ease political tensions
between their countries.

Laurence Golborne, Chile's now very popular mining minister, said at a crowded, televised news
conference here this afternoon that the first rescue would occur ``toward the end of the day,'' in
the ``last quarter'' of Tuesday. Mr.~Golborne cautioned that rescuing all the men would take about
48 hours. ``We are not about declaring mission accomplished,'' he said.

``Accidents can happen,'' he also said, ``but we trained a lot and we feel that we are very well
prepared, and you will see the result of this preparation.''

The 13-foot capsule is faced in part with a green-painted metal mesh, meant to offer some give in
the twists and turns of the escape shaft. It has oxygen tanks and a hands-free phone system.

The emerging men will be shielded from the crush of news media, with only a government photographer
and Chile's state television channel permitted access, The Associated Press reported. A 30-second
transmission delay in the video feed being broadcast to the public is to control for the unexpected.

News organizations in the United States planned large-scale coverage. CNN said its news shows would
feature extensive coverage throughout the night and into Wednesday morning. MSNBC planned live
broadcasts throughout the night. Fox News had an anchor on standby during the political shows that
dominate its evening programming, with more anchors lined up through the night.

The broadcast networks all have reporters at the site of the mine as well, for their morning shows
and evening newscasts. CBS and NBC said they would update their nightly newscasts for each time zone
with the latest information about the miners.

``It's fair to say the whole world will be watching,'' the NBC correspondent Kerry Sanders wrote on
msnbc.com Monday. ``I can't predict my reaction, but like the families who have held vigil here on
the surface, I'm excited.''

He added, ``Isn't it about time we all shared some good news?''

As the vast team of rescue workers, medical personnel, technicians and mining experts prepared to
enter the final phase, the colorful scene reflected the huge scale of the operation that has
captured the attention of the world: more than 1,400 journalists, together with anxious and elated
family members of the miners, gathered to witness the rescue.

Hundreds of journalists swarmed family members for comments. Signs proclaiming ``Strength, Miners!''
dotted the camp, as red, white and blue Chilean flags waved wildly in the sunshine.

Against this backdrop, Mr.~Golborne maintained a tone both calmly optimistic and cautious. Tests on
the stability of the capsule were carried out on Monday, he said, and tests on the lifting system
still have to be conducted in the next few hours on Tuesday before the rescue can begin. ``The
capsule has not gone down all the way to the floor of the mine,'' he added.

The gold and copper mine, near the northern city of Copiap\'o, caved in on Aug.~5, and there was no
word on the fate of the miners for 17 days. But when a small bore hole reached the miners' refuge,
they sent up a message telling rescuers they were still alive.

On Saturday, a more sizeable drill finally broke through to the miners, making way for a rescue
shaft through which the miners will be raised, one by one, in the narrow escape capsule. The
American who led the successful drill effort returned to the United States on Tuesday. Brandon
Fisher, the owner of Center Rock Inc.~of Berlin, Pa., told the Pittsburgh Post-Gazette that he and
his team decided their work was done.

``Everyone wanted to be there who drilled that hole,'' he said. ``But we had to do the right thing
and back off.''

As the start of the rescue drew closer, word spread that the first miner to be raised would be
Florencio \'Avalos. La Tercera, the Chilean newspaper, quoted his mother, María Silva, as saying
that he was ``very anxious.'' The BBC reported that he is married, with two children. A foreman, he
worked at the San Jose mine for eight years.

The second man, according to some reports, will be Mario Sepúlveda, 39, who was the spokesman on
the first video taken of the men in the days after they were first found alive.

The Bolivian, Carlos Mamani, 24, may be the fourth man.

The last is now expected to be Luis Urzúa, 54, a leader of the miners during the entire ordeal.

\section{Germany, Unscathed, Is in Eye of Terrorism Scare}

\lettrine{T}{his}\mycalendar{Oct.'10}{13} wealthy port city advertises its bustling canals and
bridges and its towering 19th-century churches to draw visitors from around the world.

It is less interested in drawing more attention to Al Quds Mosque, where the Sept.~11 hijacker
Mohamed Atta prayed and which has become a destination for jihadi tourism. This summer, local
authorities closed the mosque, since renamed Taiba Mosque, altogether.

Although Germany has been spared the terrorist attacks that have hit the United States, Britain and
Spain, Hamburg -- and Germany in general -- remains a breeding ground for Islamic radicals, security
officials acknowledge. A spate of recent arrests and terrorism warnings in Europe and Afghanistan
has underscored the risk that a small number of German citizens are under the sway of terrorist
groups determined to stage new attacks, either in Germany or elsewhere in Europe.

Officials in Hamburg emphasized that the vast majority of its Muslim population -- which they put at
130,000 -- rejected violence. But a Hamburg intelligence official said there were 2,000 residents
who embrace radical ideology and another 45 who accept the ideology of Al Qaeda and global jihad.

``That's what we all experience in America and in other countries and also here, that this
phenomenon of the homegrown terrorist increases rapidly,'' said the intelligence official, who spoke
recently on the condition that he not be identified because of the secrecy of his work. ``This is an
extremism which grows right here. The recruiting, the radicalization happens right here, not in
other countries.''

Recent events have put German citizens at the center of the global terrorism scare. In July,
American forces in Afghanistan detained a German citizen, Ahmed Sidiqi, 36, said to have ties to the
men who helped plot the Sept.~11 attacks. Then just after Washington issued the terrorism alert
based on information from Mr.~Sidiqi, Pakistani officials said that several German citizens were
killed in a drone strike.

The United States military also recently issued an alert for a 23-year-old Berlin man suspected of
joining a group called the German Taliban Mujahadeen. Federal law enforcement authorities here
followed suit with an arrest warrant for the man identified as Hayrettin Burhan Sauerland amid
concerns that he might try to return here from the Pakistan-Afghanistan border region to stage an
attack, possibly against United States military personnel.

German officials said that the alerts did not add up to a specific threat, and that the intentions
and capabilities of those arrested remained unclear. They have publicly criticized the decision by
the United States to issue a travel alert for its citizens traveling to Europe, saying that such
announcements play into the hands of terrorists and scare people needlessly.

``No one should doubt that Germany is a target for terrorists, but on the other hand there are no
concrete, immediate attack plans that we are aware of,'' Interior Minister Thomas de Maizi\`ere said
on German radio.

Even so, the intelligence official in Hamburg said Germany did face the ``intense abstract danger''
of homegrown radical Islamic terrorism. And this week, the Interior Ministry also said it had
assigned an additional 200 investigators to focus on terrorism.

German officials said that hundreds of their citizens and legal residents had over the years
traveled to the lawless border region between Afghanistan and Pakistan, and that more than 100 had
returned to Germany. There has been a deluge of statistics coming from Germany's intelligence and
police operations illustrating the growing concern of radicalized young people heading off to
terrorist training camps -- and then returning home to strike.

The newspaper Der Tagesspiegel reported that federal officials identified 131 individuals as
possibly prepared to carry out violent acts. The federal security services said that over the last
two decades, about 215 citizens or legal residents of Germany received or intended to receive
paramilitary training, that 65 completed the training, and that of the total group, about 105 are in
Germany, including 15 in prison.

Those precise-sounding numbers, however, belie the fact that German officials are struggling to get
a handle on how susceptible their large and mostly peaceful population of Muslims is to the lure to
radicals in far-off places.

``These are the so-called known terrorists, the suspicious people,'' said Rolf Tophoven, the
director of the Institute for Terrorism Research and Security Policy in Essen, of the statistics.
``But the unknown people, they don't go to the mosque to pray because they are afraid to be
detected; they don't have meetings in religious institutions, maybe they come together in a private
house. That is a big concern.''

Germany first awoke to this reality in 2001, when three men who worshiped at a Hamburg mosque were
identified as leaders in carrying out the Sept.~11 plot including Mr.~Atta, who flew one of the
planes into the World Trade Center.

Memories of that case resurfaced when the American military arrested Mr.~Sidiqi, a German citizen of
Afghan heritage who once worked at the Hamburg airport. The Hamburg intelligence official that said
Mr.~Sidiqi had been under surveillance for years and that like Mr.~Atta, he had attended Al Quds
Mosque in Hamburg. The official said Mr.~Sidiqi was also a friend of Mounir el-Motassadeq, who was
convicted in 2006 for his role in aiding the Sept.~11 hijackers.

The Hamburg official said Mr.~Sidiqi was one of 11 Hamburg residents who traveled to Pakistan in
2009 in order to fight on behalf of radical Islamic groups abroad, not at home.

In Hamburg, the decision to close the mosque was made to deny radicals a symbol to rally people to
the cause. But the intelligence official and German experts on terrorism also acknowledged that
closing the mosque did not get rid of those already radicalized but instead drove them underground.

``There is some unrest in the Germany intelligence community that they have overlooked a community
of Islamic militants they didn't realize was here inside Germany, an existing group of terrorists, a
sleeping cell of terror,'' said Mr.~Tophoven, of the Institute for Terrorism Research and Security
Policy. ``They are concerned it could exist.''

\section{Judge Orders U.S.~Military to Stop `Don't Ask, Don't Tell'}

\lettrine{A}{}\mycalendar{Oct.'10}{13} federal judge on Tuesday ordered the United States military
to stop enforcing the ``don't ask, don't tell'' law that prohibits openly gay and bisexual soldiers
from military service.

Judge Virginia A.~Phillips of Federal District Court for the Central District of California wrote
that the 17-year-old policy ``infringes the fundamental rights of United States servicemembers and
prospective servicemembers'' and violates their rights of due process and freedom of speech. She
issued an injunction banning enforcement of the law and ordered the military ``to suspend and
discontinue'' immediately any investigations or proceedings to dismiss members of the armed
services.

While Judge Phillips's decision is likely to be appealed by the government, it represents a
significant new milestone for gay rights in the United States.

Two other recent decisions have overturned restrictions on gay rights at the state and federal
levels, but Tuesday's ruling, in Log Cabin Republicans v. United States of America, could have a
more sweeping impact, as it would apply to all United States service members anywhere in the world.
Christian Berle, the acting executive director of the Log Cabin Republicans, a gay organization,
applauded the judge's action, saying it would make the armed forces stronger. ``Lifting the ban on
open service will allow our armed forces to recruit the best and brightest,'' Mr.~Berle said, ``and
not have their hands tied because of an individual's sexual orientation.''

Alexander Nicholson, the named plaintiff in the lawsuit, said ``we sort of won the lottery,''
considering the breadth of the decision. Mr.~Nicholson is executive director of Servicemembers
United, an organization of gay and lesbian troops and veterans.

The government has 60 days to file an appeal. ``We're reviewing it,'' said Tracy Schmaler, a Justice
Department spokeswoman, adding that there would be no other immediate comment. The government is
expected, however, to appeal the injunction to the Court of Appeals for the Ninth Circuit to try to
keep it from taking effect pending an appeal of the overall case.

Such a move would carry risks, said Richard Socarides, who was an adviser to President Bill Clinton
on gay rights issues. ``There will be an increasingly high price to pay politically for enforcing a
law which 70 percent of the American people oppose and a core Democratic constituency abhors,'' he
said.

Critics of the ruling include Tony Perkins, the president of the Family Research Council and a
proponent of the don't ask, don't tell law, who accused Judge Phillips of ``playing politics with
our national defense.''

In a statement, Mr.~Perkins, a former Marine, said that ``once again, an activist federal judge is
using the military to advance a liberal social agenda,'' and noted that there was still ``strong
opposition'' to changing the law from military leaders.

Mr.~Perkins predicted that the decision would have wide-ranging effects in the coming elections.
``This move will only further the desire of voters to change Congress,'' he said. ``Americans are
upset and want to change Congress and the face of government because of activist judges and arrogant
politicians who will not listen to the convictions of most Americans and, as importantly, the
Constitution's limits on what the courts and Congress can and cannot do.''

The don't ask, don't tell law was originally proposed as a compromise measure to loosen military
policies regarding homosexuality. Departing from a decades-old policy of banning service by gay,
lesbian and bisexual recruits, the new law allowed service and prohibited superiors from asking
about sexual orientation. But the law also held that service members could be dismissed from the
military if they revealed their sexual orientation or engaged in homosexual acts.

Since 1993, some 12,500 gay men and lesbians have been discharged from the service when their sexual
orientation became known, because either they or others made it public.

The law has long been a point of contention, and President Obama has asked Congress to repeal it.

At an afternoon briefing on Tuesday, the White House press secretary, Robert Gibbs, said that Judge
Phillips's injunction was under review, but that ``the president will continue to work as hard as he
can to change the law that he believes is fundamentally unfair.''

The Department of Justice, however, is required to defend laws passed by Congress under most
circumstances.

In February, Defense Secretary Robert M.~Gates and Adm. Mike Mullen, the chairman of the Joint
Chiefs of Staff, asked Congress to repeal the law.

The House voted to do so in May, but last month the Senate voted not to take up the bill allowing
repeal. Advocates for repeal have pushed for that vote to be reconsidered after the midterm
elections.

Jim Manley, a spokesman for the Senate majority leader, Harry Reid, said, ``Senator Reid is
encouraged by the decision, and still hopes to be able to take the bill to the floor after the
elections in November.''

Mr.~Gates was on an official visit to Vietnam when Judge Phillips's action was announced on Tuesday.
``We have just learned of the ruling and are now studying it,'' said Geoff Morrell, the Pentagon
press secretary. ``We will be in consultation with the Department of Justice about how best to
proceed.''

Judge Phillips, who was appointed by Mr.~Clinton, declared the law unconstitutional in an opinion
issued on Sept.~9. She then sought recommendations from the parties as to what kind of legal relief
should follow.

The Log Cabin Republicans recommended a nationwide injunction. The Department of Justice recommended
narrower action.

Arguing that ``the United States is not a typical defendant, and a court must exercise caution
before entering an order that would limit the ability of the government to enforce a law duly
enacted by Congress,'' the Justice Department noted that the law had been found constitutional in
other courts.

It asked that the judge's injunction apply only to members of Log Cabin Republicans and not to the
military over all.

In the other recent cases in which federal judges have pushed back against laws that restrict gay
rights, a judge in California struck down that state's ban on same-sex marriages in August. And in
July, a federal judge in Massachusetts ruled that a law prohibiting the federal government from
recognizing same-sex marriages, the Defense of Marriage Act, was unconstitutional, opening the way
for federal benefits in such unions.

While Mr.~Obama has been critical of the Defense of Marriage Act, the Justice Department has
defended it in the federal court challenge. On Tuesday, the department filed an appeal in the case
and issued a statement that might well be echoed in coming weeks in the military case.

``As a policy matter, the president has made clear that he believes DOMA is discriminatory and
should be repealed,'' said Ms.~Schmaler, the department spokeswoman. ``The Justice Department is
defending the statute, as it traditionally does when acts of Congress are challenged.''

Advocates for gay rights said they were cheered by the direction of the three recent rulings.

Chad Griffin, the board president of the American Foundation for Equal Rights, which sponsored the
litigation against California's same-sex marriage ban, said that ``with the momentum of these three
court decisions, I think it really is the beginning of the end of state-sanctioned discrimination in
this country.''

\section{Howard Jacobson Wins Man Booker Prize for `The Finkler Question'}

\lettrine{H}{oward}\mycalendar{Oct.'10}{13} Jacobson won the Man Booker Prize, Britain's most
prestigious literary award, on Tuesday night for ``The Finkler Question,'' a comic novel about
friendship, wisdom and anti-Semitism.

Mr.~Jacobson, 68, beat out ``C,'' by Tom McCarthy, widely considered the favorite to win.

The author of 10 previous novels, Mr.~Jacobson, who was born in Manchester, England, was on the long
list for the Booker Prize twice before, for ``Who's Sorry Now?'' in 2002 and ``Kalooki Nights'' in
2007.

He accepted the award to unusually enthusiastic and sustained applause at an awards ceremony in
London.

``I'm speechless,'' he told the audience. ``Fortunately, I prepared one earlier. It's dated 1983.
That's how long the wait's been.''

The Booker is given each year to a novel by an author in Britain, Ireland or one of the Commonwealth
nations. The prize comes with a check for \textsterling50,000, or about \$80,000, and a practically
guaranteed jump in book sales and publicity. ``The Finkler Question'' was published by Bloomsbury
USA this week in the United States.

It was a small triumph for humor in fiction, an argument that Mr.~Jacobson made in a nearly
3,700-word essay in The Guardian last Saturday.

``There is a fear of comedy in the novel today -- when did you last see the word `funny' on the
jacket of a serious novel? -- that no one who loves the form should contemplate with pleasure,'' he
wrote. ``We have created a false division between laughter and thought, between comedy and
seriousness, between the exhilaration that the great novels offer when they are at their funniest,
and whatever else it is we now think we want from literature.''

The chairman of the judging panel, Andrew Motion, Britain's former poet laureate, called ``The
Finkler Question'' a ``marvelous book: very funny, of course, but also very clever, very sad and
very subtle.''

``The Finkler Question'' tells the story of Julian Treslove, an ordinary former BBC producer who
meets an old philosopher friend, Sam Finkler, and their former teacher, Libor Sevcik, for dinner one
night in London. Walking home, Mr.~Treslove is robbed, an incident that sets him on a quest for
self-discovery, wisdom and the knowledge of what it means to be Jewish.

Writing in The Guardian, Edward Docx said the novel was ``full of wit, warmth, intelligence, human
feeling and understanding.''

``It is also beautifully written with that sophisticated and near invisible skill of the authentic
writer,'' he added.

Mr.~Jacobson's selection was a reminder of the unpredictability of the Booker Prize, which is always
the subject of speculation in the weeks before it is announced. Mr.~McCarthy's book was heavily
favored, so much so that the online betting site Ladbrokes suspended betting last week after a huge
number of wagers were placed on it -- a circumstance the bookmaker called ``borderline
inexplicable.''

Rarely does the front-runner win the prize: last year's award to ``Wolf Hall'' by Hilary Mantel was
the exception, at least for recent years.

This year's Booker short list was notable for the books that were not on it. ``The Thousand Autumns
of Jacob de Zoet,'' by David Mitchell, and ``The Slap,'' by the Australian novelist Christos
Tsiolkas, both made the 13-book long list but did not make the cut.

The other titles that did make the short list were: ``In a Strange Room,'' by Damon Galgut; ``The
Long Song,'' by Andrea Levy; ``Room,'' by Emma Donoghue; and ``Parrot and Olivier in America,'' by
Peter Carey.

\section{Miners Face Difficult Adjustment}

\lettrine{T}{he}\mycalendar{Oct.'10}{13} ordinary miners are long gone, left behind in another
lifetime.

The 33 men who are expected to emerge from Chile's San Jos\'e Mine after more than two months
trapped underground are now icons, national heroes, global media stars. And that mixed status --
celebrities as well as survivors, equal parts victors and victims -- will alter the usual trajectory
of mental adjustment after trauma for many of them, experts said Tuesday.

``All the attention is likely to camouflage reactions to the trauma itself in some members of the
group,'' said Dr.~Charles Marmar, a psychiatrist at New York University's Langone Medical Center.
``It may resemble this honeymoon effect, like in the young kid who suffered some trauma in Iraq or
Afghanistan and returns as a hero, wanting to drop right back into family and community as if
nothing had changed.''

After being swept up in a natural disaster like a hurricane or earthquake, about 15 percent to 30
percent of people suffer post-traumatic reactions for months afterward, studies suggest. About 5
percent go on to suffer from mood problems, nightmares or other symptoms for a year or longer. The
same rates may very well turn up in the miners: the group lived in the shadow of near-certain death
for 17 days before rescuers on the surface made contact with them.

``The outpouring of attention may delay those reactions, but that attention is not going to last
forever,'' said John A.~Fairbank, co-director of the National Center for Child Traumatic Stress and
professor of psychiatry at Duke Medical Center. ``I suspect that a few miners will have trouble
adjusting to the new normal,'' he said, particularly in families where roles have changed since the
ordeal started.

At least two things are likely to work very much in the miners' favor, experts said. One is the
considerable upside of being a national treasure. Chilean officials have skillfully managed this
crisis and are expected to monitor the survivors closely, providing support for them, if necessary.
The government is not going to neglect the men after making a maximum effort to secure their rescue.

Another protective factor is the group itself. By all accounts, one of the miners, Mario G\'omez,
helped build a tightly organized hierarchy in which group members rationed food and work equally.

``Groups can be a positive influence or a negative one,'' said Lawrence Palinkas, a professor of
social policy and health at the University of Southern California. ``But here it appears that there
were recognized leaders, shared goals, among men who came from similar background and were able to
keep stress at a minimum.''

As for physical challenges, once at the surface, the miners will receive wrap-around sunglasses to
protect against sudden exposure to the bright sun, and then be treated with first aid at the site
before being whisked by helicopters to a nearby hospital. Many of the miners have complained of
dental pain and skin irritation that will require treatment, officials say.

Each of the 33 men will have his own tale, his own way of coping with a transformed identity and a
life story forever stuck nearly half a mile below the earth.

Yet in the weeks and months to come, the support structure of the group is likely to become
increasingly important, as the miners struggle with the demands of the media, family expectations
and a question: What next?

``Except for a few miners who find reminders of the experience too painful, this group will become
the most powerful social network these men have,'' Dr.~Marmar said. ``They're bonded for life, like
any group of cops or firefighters or war fighters who have shared a threat to life and united to
survive it.''

\section{At Mine, Plan B Works Best}

\lettrine{I}{n}\mycalendar{Oct.'10}{13} the days immediately after the discovery that 33 workers
trapped at the San Jos\'e Mine were alive, Chilean officials cautioned that it might take up to four
months to reach them with a rescue hole.

In the end, it took only about six weeks to do what mine safety experts say is a job with little
precedent: drilling a precision shaft, wide enough to accommodate a man, to a spot more than 2,000
feet -- almost half a mile -- underground.

``We have a number of mines that are this deep,'' said John E.~Urosek, chief of mine emergency
operations for the United States Mine Safety and Health Administration. ``But we've never had a
situation where we've had to rescue someone through a bore hole this deep.''

One reason Chilean officials may have thought the drilling could take far longer was that they were
not familiar with the type of drill that carved the rescue hole. Three efforts to bore through the
abrasive volcanic rock went forward simultaneously -- known as Plans A, B, and C -- but it was Plan
B that broke through to the miners first.

``To tell you the truth, I don't think anyone had a whole lot of faith in us,'' said Brandon Fisher,
president of Center Rock, a company in Berlin, Pa., that supplied the Plan B drills. ``They didn't
understand the technology.''

Mr.~Fisher and others lobbied the Chilean government to let them use the drills, known as downhole
hammers, which have air-powered bits that pound the rock as the drill rotates. The other two
drilling operations used more conventional bits that work through rotation only.

The geology of the region -- hard volcanic rock infused with other extremely hard minerals --
favored the Plan B equipment, said Maurice B.~Dusseault, a professor of engineering geology at the
University of Waterloo in Canada. ``In very hard, brittle rocks, percussion drilling is indeed an
excellent way'' to make progress, he said.

The effort was also aided by the dry conditions encountered underground in the Atacama region of
Chile, one of the driest deserts in the world. ``You can't do percussion drilling if you have water
or mud in the bore hole,'' Mr.~Dusseault said, as the liquids absorb too much of the percussive
energy.

The Plan B drillers made use of one of the small bore holes that were drilled to locate the miners
after the collapse, widening it first to 12 inches and then to 28 inches to accommodate the rescue
capsule. Using such a pilot hole ``was really our only option,'' said Mr.~Fisher, who delivered the
drills himself and stayed at the site for the entire drilling operation. ``There were so many old
workings in the mine that we couldn't risk starting a new hole. We went with a sure thing.''

Even using the pilot hole, the 12-inch drill encountered a steel roofing bolt that heavily damaged
the equipment and delayed the operation for a few days.

The use of a pilot hole also allowed the drillers to drop the rock cuttings down the existing hole,
reducing the complexity of the operation, said Frank Gabriel, a vice president of Schramm, the
Pennsylvania company that made the mobile drilling rig used by Plan B.~``Normally you'd have to
flush all those cuttings back to the surface,'' he said. ``That takes a lot more air.''

Mr.~Fisher said that now that they had seen how the technology worked, the Chileans were likely to
incorporate it in many of their mining operations to drill ventilation shafts and other holes. They
were very grateful, he said, ``thanking us for talking them into letting us come down and use this
technology.''

\section{Minutes Show Fed Leaning Toward New Stimulus}

\lettrine{T}{he}\mycalendar{Oct.'10}{13} Federal Reserve provided more evidence on Tuesday that a
critical mass of officials at the central bank favored additional actions to reinvigorate the
lagging recovery.

Most Wall Street analysts expect the Fed to decide, at its meeting in early November, to resume the
debt-buying strategy known as quantitative easing, in which the Federal Reserve would purchase
Treasury securities to make borrowing cheaper. But while that outcome is possible, it is not
certain, according to minutes of the Fed's last policy meeting, which the central bank released
Tuesday.

In essence, the debate was between those who said they believed that the Fed should act ``unless the
pace of economic recovery strengthened,'' and those who thought action was merited ``only if the
outlook worsened and the odds of deflation increased materially.''

The minutes of the Sept.~21 meeting of the Federal Open Market Committee indicated that several
officials ``consider it appropriate to take action soon,'' given persistently high unemployment and
uncomfortably low inflation.

But other officials ``saw merit in accumulating further information before reaching a decision,''
according to the minutes. The meeting lasted 5 hours and 10 minutes, longer than usual -- a sign of
the uncertainty in the economy and of the debate that has accompanied one of the hardest choices the
Fed has faced since the recession began.

The Fed lowered short-term interest rates to nearly zero in December 2008, and then bought \$1.7
trillion in mortgage-backed debt and Treasury securities in an effort to lower long-term rates, a
process that ended in March.

Now, with unemployment near 10 percent and with inflation well below the Fed's unofficial goal of
nearly 2 percent, the Fed is considering renewed intervention: creating money to buy long-term
Treasury debt. That would put additional downward pressure on long-term rates, making credit even
cheaper.

Former Fed officials interviewed on Tuesday appeared to be just as divided as the current ones.

``If you lead the horse to water and it won't drink, just keep adding water and maybe even spike
it,'' said Robert D.~McTeer, who was president of the Federal Reserve Bank of Dallas from 1991 to
2005 and is a well-known inflation ``dove,'' particularly attuned to the harm of joblessness. ``You
definitely don't want to take the water away.''

Mr.~McTeer said the markets had made too much of the prospect of additional asset purchases. The Fed
should pursue the strategy in a gradual and incremental fashion, he said, rather than making it
appear to be a significant decision.

``From the outside it might look like they're dithering,'' he said. ``Maybe they are, maybe they're
not. They haven't done a very good job at communicating.''

H. Robert Heller, a Fed governor from 1986 to 1989, had the opposite view, urging the Fed to show
restraint.

``I would do nothing,'' he said, expressing concern that the Fed might appear to be ``monetizing the
debt,'' or printing money to make it easier for the government to borrow and spend.

``If they start to monetize the federal debt, they will dig themselves a much deeper hole later
on,'' he said. ``That's what we learned from the 1970s, when the Fed undertook a very expansionary
monetary policy. It took a double recession in the early 1980s to wring inflation out of the
economy. We don't want to repeat that.''

In a speech in Denver on Tuesday, the president of the Federal Reserve Bank of Kansas City, Thomas
M.~Hoenig, said the costs of further action could outweigh the benefits and result in more
volatility. ``If we have learned anything from this crisis, as well as past crises, it is that we
must be careful not to repeat the policy patterns we have used in previous recoveries,'' Mr.~Hoenig
said in his prepared remarks.

The minutes of the September meeting showed a consensus that a new recession was unlikely, but that
growth ``could be slow for some time.'' Most officials thought the recovery would pick up gradually
next year.

But that shared assessment did not mean a unified view on whether the Fed should act.

A few Fed officials noted that recoveries set off by financial crises, like the 2008 crisis, had
historically been uneven and slow. But others worried that ``the sluggish pace of growth and
continued high levels of slack left the economy exposed to potential negative shocks.''

Some officials argued that the high unemployment rate might have structural causes: mismatches
between the jobs that are available and the skills needed to perform them, an inability of workers
to relocate because their mortgages are greater than the value of their homes, and the effects of
extended unemployment benefits.

Other officials argued against that view, saying that ``the current unemployment rate was
considerably above levels that could be explained by structural factors alone.'' Employment has
fallen across a range of industries, job vacancies are low, and the demand for goods and services is
weak, these officials pointed out.

As a potential alternative to additional debt purchases, Fed officials considered strategies for
altering inflation expectations. Such expectations, which can be critical in influencing price
movements, have remained fairly stable.

Because short-term rates are at the ``zero lower bound,'' with no room to go down any further, some
Fed officials discussed whether the central bank should try to raise inflation expectations, which
might spur economic activity in the short run. The Fed could do this by specifying a
higher-than-usual desired inflation rate, seeking a price level rather than an inflation rate or
offering a target for gross domestic product, the broadest measure of economic output.

William C.~Dudley, president of the Federal Reserve Bank of New York, recently raised the
possibility that inflation could be allowed to run above the implicit target for some time in the
future, to make up for inflation today being lower than desired. That could temporarily raise
inflation expectations and lower real interest rates.

But Frank Schorfheide, an economics professor at the University of Pennsylvania, said the strategy
was untested.

``Central banks around the world were successful in conveying that they were committed to keeping
inflation around 2 percent,'' he said. ``That seems much easier than trying to commit to keeping
interest rates a little lower than normal, at some future point in time, in order to raise inflation
expectations now.''

The committee's September meeting ended with a statement that the Fed was ``prepared to provide
additional accommodation if needed to support the economic recovery and to return inflation, over
time, to levels consistent with its mandate.'' Most analysts interpreted the statement to mean that
additional action was imminent.

Additional clarity might come on Friday when the Fed's chairman, Ben S.~Bernanke, speaks at a
conference, sponsored by the Federal Reserve Bank of Boston, on how monetary policy should be
conducted in a low-inflation environment.

\section{In the Future, Already Behind}

\lettrine{A}{}\mycalendar{Oct.'10}{13} few years ago, Silicon Valley start-ups like Solyndra,
Nanosolar and MiaSol\'e dreamed of transforming the economics of solar power by reinventing the
technology used to make solar panels and deeply cutting the cost of production.

Founded by veterans of the Valley's chip and hard-drive industries, these companies attracted
billions of dollars in venture capital investment on the hope that their advanced ``thin film''
technology would make them the Intels and Apples of the global solar industry.

But as the companies finally begin mass production -- Solyndra just flipped the switch on a \$733
million factory here last month -- they are finding that the economics of the industry have already
been transformed, by the Chinese. Chinese manufacturers, heavily subsidized by their own government
and relying on vast economies of scale, have helped send the price of conventional solar panels
plunging and grabbed market share far more quickly than anyone anticipated.

As a result, the California companies, once so confident that they could outmaneuver the
competition, are scrambling to retool their strategies and find niches in which they can thrive.

``The solar market has changed so much it's almost enough to make you want to cry,'' said Joseph
Laia, chief executive of MiaSol\'e. ``We have spent a lot more time and energy focusing on costs a
year or two before we thought we had to.''

The challenges come despite extensive public and private support for the Silicon Valley companies.
Solyndra, one of the biggest firms, has raised more than \$1 billion from investors. The federal
government provided a \$535 million loan guarantee for the company's new robot-run,
300,000-square-foot solar panel factory, known as Fab 2.

``The true engine of economic growth will always be companies like Solyndra,'' President Obama said
in May during an appearance at the then-unfinished factory. But during the year that Solyndra's
plant was under construction, competition from the Chinese helped drive the price of solar modules
down 40 percent. Solyndra rushed to start cranking out panels on Sept.~13, two months ahead of
schedule, and it has increased marketing efforts to make the case to customers that Solyndra's more
expensive panels are cost-effective when installation charges are factored in.

``It definitely puts more pressure on us to bring our costs down as quickly as possible by ramping
up volume,'' said Ben Bierman, Solyndra's executive vice president for operations and engineering.

Silicon Valley companies like Solyndra, Nanosolar and MiaSol\'e continue to receive hundreds of
millions of dollars in customer orders and some plan to expand local manufacturing. But the rapid
rise of low-cost Chinese manufacturers has made investors -- who once envisioned the region's future
as Solar Valley -- skittish about backing new capital-intensive start-ups.

``I don't see another Solyndra being done,'' said Anup Jacob, whose private equity firm, Virgin
Green Fund, has invested significantly in Solyndra.

In the third quarter of 2010, venture capital investment in solar companies plummeted to \$144
million from \$451 million in the year-ago quarter, according to the Cleantech Group, a San
Francisco research firm.

The paucity of capital and the sheer size of Chinese solar panel makers have proved particularly
problematic for companies like Solyndra and MiaSol\'e, which make photovoltaic cells using a
material called copper indium gallium selenide, or CIGS.

Unlike conventional solar cells, made from silicon wafers, CIGS cells can be deposited on glass or
flexible materials, much as ink is printed on rolls of newspaper. Though the technology is less
efficient at converting sunlight into electricity, the promise of ``thin film'' solar cells was that
they could be made cheaply. But producing CIGS cells on a mass scale has turned out to be a
formidable technological challenge, requiring the invention of specialized manufacturing equipment.

While Silicon Valley companies were working on the problem, silicon prices fell and Chinese
companies like JA Solar, Suntech and Yingli Green Energy rapidly expanded production of conventional
solar panels, supported by tens of billions of dollars in inexpensive credit from the Chinese
government as well as other subsidies like cheap land.

Arno Harris, chief executive of Recurrent Energy, a San Francisco solar developer acquired by Sharp
last month, said he chose to sign a supply deal with Yingli because the Chinese company offered low
prices, quality products and financing.

``We realized that would enable us to bid competitive power prices from projects that could also be
efficiently financed,'' Mr.~Harris said in an e-mail.

Chinese solar panel makers now supply about 40 percent of the California market, the largest in the
United States, and the bulk of the European market, according to Bloomberg New Energy Finance, a
research and consulting firm.

``We grow every year with double revenue and almost double capacity,'' said Fang Peng, the chief
executive of JA Solar, in a telephone interview from the company's Shanghai headquarters. ``At end
of the year, we will have 1.8 gigawatts of capacity and will have grown from 4,000 employees at the
beginning of this year to more than 11,000.''

By comparison, Solyndra expects to have a total production capacity of 300 megawatts by the end of
2011.

The competition from the Chinese prompted some Silicon Valley companies, like AQT Solar, to pursue
new strategies to survive.

AQT has modified off-the-shelf machines used to make computer hard drives to create CIGS cells using
a proprietary process. The Sunnyvale company, which has raised \$15 million from investors, further
cut its capital costs by manufacturing only solar cells, which it sells to other companies to
package into solar panels.

Rather than build a factory from the ground up, the company recycled a 1970s-era rental building.
``We moved in here in eight weeks, put our first 20-megawatt line up and did it for under a million
dollars. That's on Chinese time,'' said Michael Bartholomeusz, AQT's chief executive.

A mile away, another start-up, Innovalight, has abandoned solar module manufacturing altogether. The
company had developed what it calls a silicon ink, which increases a solar cell's efficiency when it
is printed on a standard silicon wafer.

After installing a 10-megawatt production line, in late 2008, Innovalight executives decided that,
rather than compete with the Chinese, they would license the patented ink technology to them and
avoid having to raise hundreds of millions of dollars to build factories of their own.

``How do you fight against enormous subsidies, low-interest loans, cheap labor and scale and a
government strategy to make you No.~1 in solar?'' said Conrad Burke, Innovalight's chief.
``Innovation will be the heart of the U.S.~strategy, and although it might not create the same
scale, we're exporting well-protected technology to China and creating well-paying jobs here.''

As part of its corporate sustainability policy, Wal-Mart Stores last month acted to bolster American
CIGS companies by signing a deal with a Silicon Valley solar installer, SolarCity, to put 15
megawatts of photovoltaic panels on its big-box stores and requiring that a significant percentage
of them come from thin-film companies like MiaSol\'e.

Even so, SolarCity's chief executive, Lyndon Rive, acknowledged that his company would also be
installing a large number of conventional solar panels for the retail giant -- nearly all of them
made in China.

\section{Japanese Firm to Buy iPhone Game Company Ngmoco}

\lettrine{D}{ena}\mycalendar{Oct.'10}{13}, a Japanese social game company, said on Tuesday that it
would acquire Ngmoco, a Silicon Valley iPhone game developer, for \$400 million -- one of the
largest deals involving an iPhone software company and another sign that Apple's products are fast
becoming the hottest mobile game devices on the market.

The acquisition is also the latest in an overseas spending spree by DeNA, which is little known
outside of Japan but aims to be a global rival to the big names in social networking and games,
including Facebook and Zynga, which makes the FarmVille games.

``The big tide in social gaming is coming, right now,'' Tomoko Namba, the founder and chief
executive of DeNA, said in an interview. ``We'd like to capture it and quickly become the world's
No.~1 mobile gaming platform.''

``We're only active in the Japanese market, and we haven't figured out how to cover the Western
market,'' she said. ``We want to enable developers to go cross-device and to go cross-border. And we
need this to happen quickly, in about the next one or two years.''

Ngmoco was founded two years ago by Neil Young, a longtime game executive who left his post at
Electronic Arts to form the company. Not long after, it received financing from the venture capital
firm Kleiner Perkins Caufield \& Byers.

The company's flagship game, Rolando, released in 2008, challenges players to navigate a smiling
round character through a colorful animated landscape.

DeNA, which runs the wildly popular Mobage Town mobile social game platform in Japan, began an
aggressive global expansion in the last year, acquiring game developers based in the United States
to add more content to its platform and translate its success overseas.

The company is racing to get a piece of the rapidly expanding market for online social games, or
games played on social networks, like FarmVille on the Facebook site. This year, the research
company Screen Digest forecast that the social games market would grow to \$1.5 billion in 2014,
from about \$640 million in 2009.

This year, Google invested in the social game company Zynga, while Disney bought Tapulous, an iPhone
application developer best known for its Tap Tap Revenge games.

Mobage Town, a platform of social games, chat rooms and virtual characters that has 20.5 million
users, has been a hit with the mobile-savvy young generation in Japan, bucking a general decline in
the Japanese console game industry in recent years. The service offers games free but requires users
to sign up and create online avatars, or virtual characters, so they can interact with one another.
DeNA then sells virtual clothes and other game accessories for the avatars.

DeNA booked sales last year of about 48 billion yen, or \$575 million. Ms.~Namba said it was on
track to double that figure this fiscal year to more than \$1 billion.

DeNA is betting that social games will increasingly be played on mobile platforms and that the
company's focus on mobile technology will broaden its user base. In May, DeNA introduced MiniNation,
a global version of its Mobage social game platform for the iPhone and iPod Touch, which lets users
play simple games, befriend other users, post messages and create communities.

DeNA plans to integrate its Mobage software with Ngmoco's own social networking platform, called
Plus+, which runs on Apple devices and smartphones that use Google's Android system.

The new global platform will allow developers to aim for both Apple and Android users and gain
access to Western and Japanese customers, said Mr.~Young of Ngmoco -- short for next generation
mobile company. Ngmoco's games have been downloaded more than 60 million times.

``Whether you're a developer in Japan working on a Mobage, or you're a developer in the West making
apps, you'll be able to work with us, and your games will be able to move across borders and move
across devices,'' Mr.~Young said.

Mobile devices will be crucial to the growth of social games, because the social relationships
encourage users to log in throughout the day, whenever they have time, said Hirokazu Hamamura,
president of Enterbrain, a Tokyo-based game research company.

``But the competition will be intense,'' he said. ``If your games can't attract a mass following,
they're not social, and there's no point.''

DeNA follows a trend of Japanese companies that have capitalized on the strong yen, which is at
15-year highs against the dollar, with acquisitions abroad.

Rakuten, Japan's biggest e-commerce company, acquired the American online retailer Buy.com in May
for \$250 million and paid a similar amount for a European Web shopping site, PriceMinister, in
June.

In the last month, DeNA said that it would invest in Astro Ape and Gameview, game developers based
in the United States. Last year DeNA bought IceBreaker, an American game studio, and acquired a 20
percent stake in Aurora Feint, a developer of mobile game platforms.

But Ngmoco, which is based in San Francisco, is DeNA's most prominent acquisition. It has grown
rapidly, partly by using venture capital to acquire the game developers Miraphonic in 2009, and
Freeverse and Stumptown Game Machine this year.

Its acquisition was approved Tuesday by DeNA's board and will close within weeks, DeNA said. Last
week, TechCrunch first reported that DeNA was considering acquiring Ngmoco.

\section{Gains in Afghan Training, but Struggles in War}

\lettrine{L}{ong}\mycalendar{Oct.'10}{13} a lagging priority, the plan to produce many more highly
trained Afghan troops is moving this fall at a rapid pace.

Two main training sites -- the Kabul Military Training Center, used principally by the Afghan Army,
and the Central Training Center, used by the police -- have become bustling bases, packed with
trainers and recruits, and there is a sense among the officers that they are producing better
soldiers than before.

The military center has been graduating 1,400 newly trained soldiers every two weeks, as the Obama
administration, eager to show progress in a slow-going war, has devoted more trainers and money to
the effort.

NATO officials hope that the clear changes in the training, both in output and atmosphere, are
grounds for a measure of optimism in a war that has frustrated those waging it and provoked
increasing opposition at home.

The ratio of instructors to students has gone from one for every 79 trainees in 2009 to one for
every 29, officers at the center say, suggesting that the new police officers and soldiers are
getting more attention than in years past. The soldiers are paid better and desert less often,
officials say.

New Afghan infantry battalions, each with roughly 800 soldiers, are regularly leaving the capital
for service in the war, sometimes making two or three battalions a month. Regional training centers
are also holding more sessions, and new bases -- including a flight school for Afghanistan's nascent
air force -- are soon to be built.

The question now is whether these new forces will allow NATO and the Afghan government to reverse
the insurgency's momentum and begin reducing the Western presence in the country.

If so, will it happen quickly enough for an American public weary of the war? Or will the promise of
these fresh troops also be undermined by desertion, poor leadership and the established pattern of
leaving much of the most dangerous work to Western hands?

Any answer must navigate a pair of remarkably different pictures of the war.

Away from the capital, in the rural areas where the insurgency rages, the Afghan military has not
performed well. In provinces where the Taliban are strongest and the fighting is most pitched, the
common view is that the Afghan Army and the police have thus far been disappointing.

At the small-unit level, Western troops and journalists have documented their corruption, drug use,
mediocre or poor fighting skills and patterns of lackluster commitment, including an unwillingness
to patrol regularly and in sizable numbers, or to stand watch in remote outposts.

At the higher levels, Western military officers often describe patronage, favoritism and an absence
of managerial acumen, rooted in part in the pervasive culture of corruption and in widespread
illiteracy. (Now, 14 percent of the combined force can read or write -- at the third-grade level.)

There is also a strong worry about Taliban infiltration into the ranks, especially among the police.

In contrast to the field, however, at the training bases the newly formed forces are clearly
improving.

Since last year, when President Obama's plan dedicated more resources to Afghan development, the
Pentagon has pursued what is in theory a simple, if expensive, approach: to recruit and field forces
quickly, and then, over time, to improve their fighting and managerial skills.

Enormous resources have been dedicated to the effort. The numbers indicate that the first step is
under way.

By June 2009, after more than seven years of war, the United States had helped Afghanistan field a
combined army and police force of about 170,000 members. Since then, the combined force has grown by
half that size again -- by more than 86,000 troops.

Today the Afghan Army has 136,000 members, and the police force about 120,000. Within a year, the
combined forces are projected to grow by 50,000.

Raw personnel numbers are only one measure. The United States is simultaneously buying and issuing
new weapons, vehicles and communications gear; building barracks, classrooms and logistics depots;
and developing a network of language labs to nudge the force toward literacy.

It is also underwriting programs intended to expand Afghan military and police capabilities,
including training bomb- and drug-detection dogs for the border police, training pilots for the
small fleet of Afghan helicopters and transport aircraft, and opening so-called branch schools to
focus on the technical development of Afghans with specialized military skills.

``Basically, there is a big change in training, the quality of the training,'' said Brig.
Gen.~Aminullah Patyani, the commander of the Kabul Military Training Center.

This month, for example, the Afghan Defense Ministry opened a school for artillerists. Some of these
programs show clear signs of progress, including scheduled flights with Afghan pilots and
helicopters to transfer wounded Afghan soldiers from Kandahar to Kabul.

``We're getting to the point where we have reliable, repeatable Afghan Air Force missions,'' said
Brig. Gen.~David W.~Allvin, who commands the NATO effort to develop an Afghan aviation capability.

Col.~John G.~Ferrari, a deputy commander of the NATO Training Mission Afghanistan, spoke of an
``inevitability factor,'' in which local security forces, in theory and if trained properly, rise in
quantity, skill and state of equipment, sharply tilting the war in the government's favor.

For that to be the case, the Pentagon must overcome a persistent problem in the Afghan security
forces: attrition. Official estimates put attrition across the force at roughly 3 percent each
month. Attrition is a powerful drain that makes growth difficult. Police officers and soldiers
simply disappear, even as replacements flow in.

For this reason, for the army to grow by 36,000 more soldiers, the government must recruit and train
83,000 Afghans, according to projections released by NATO. Similarly, for the police to reach the
hoped-for increase of 14,000, the government must train 50,000 more officers. This drives up costs
to Westerners paying the bill.

The training mission in Afghanistan also labors under a legacy of unfulfilled past promises,
inadequate training even in basic skills like marksmanship and driving military vehicles, and a
pattern of overstating how ready or skilled the forces are.

Early this year, the Pentagon and senior Afghan and American officers in Kabul insisted that the
complex operation to re-establish a government presence in Marja, a Taliban stronghold, was ``Afghan
led.''

It was not. And many Afghan units, by the accounts of many Americans present, performed poorly. Some
units openly shirked combat duty -- refusing to patrol, or sending a bare minimum of soldiers on
American patrols, sometimes only a pair of soldiers to accompany an American platoon. The remaining
Afghans stayed behind, lounging in the relative safety of outposts the Americans secured.

In the operations under way in Kandahar, reports continue to indicate that American forces are
almost always in the lead

A formal Pentagon assessment of Afghan readiness is expected in December, and under President
Obama's plan, American troops could begin a gradual drawdown next summer.

Even as these deadlines approach, many officers have spoken of managing expectations.

Brig. Gen.~Carmelo Burgio, the Italian officer who commands the police development effort, said that
NATO had made practical steps toward police competence, and that training had improved.

But developing a well-rounded police officer, much less a well-rounded force, takes many years --
perhaps much longer than America and other NATO nations have the patience for. ``We believe we are
on the right path,'' General Burgio said. ``We need time. Without time, without patience, it is
impossible.''

\section{Beijing Calls Nobel Insult To People Of China}

\lettrine{T}{he}\mycalendar{Oct.'10}{13} Chinese government continued its vilification campaign
against the awarding of the Nobel Peace Prize to a jailed dissident, Liu Xiaobo, by canceling
another meeting with Norwegian officials and denouncing the award as an affront to the Chinese
people and a ploy to try to change the country's political system.

``Some politicians from other countries are trying to use this opportunity to attack China,'' Ma
Zhaoxu, a Foreign Ministry spokesman, told reporters during a regularly scheduled news conference on
Tuesday. He added that the prize, announced Friday, ``shows disrespect for China's judicial system''
because the recipient is a convicted criminal.

Mr.~Liu, 54, a veteran pro-democracy advocate, is serving an 11-year sentence for his essays and a
manifesto he helped draft, Charter 08, that demands political reform, human rights guarantees and an
independent judicial system.

Although initially signed electronically by thousands of intellectuals, students and former
Communist Party officials, Charter 08 has since been blocked on the Internet and is largely unknown
to most Chinese.

While criticizing the Norwegian committee that made the award, the Foreign Ministry spokesman
provided no insight into whether the laureate's wife, Liu Xia, would be allowed to collect the prize
for him in Oslo.

Ms.~Liu, a photographer who generally steers clear of politics, is under a form of house arrest at
the couple's apartment in Beijing.

She spoke briefly to The Associated Press on Tuesday, saying she hoped she could pick up the award
for her husband, but indicated that could be difficult given the constraints on her.

``I am not allowed to meet the press or friends,'' Ms.~Liu told The A.P., which said she was using a
cellphone given to her by a brother, after the police shut off her old one.

``If I have to do any daily chores, like visiting my mother or buying groceries, I have to go in
their car,'' Ms.~Liu said, referring to a police car.

Ms.~Liu's new cellphone was briefly working on Tuesday but by early evening it, too, had been
disconnected.

``Her rights should be respected, and she should be allowed to move freely without harassment,''
Richard Buangan, a United States Embassy spokesman, said in an e-mail. ``We urge China to uphold its
international human rights obligations and to respect the fundamental freedoms and human rights of
all Chinese citizens.''

Although it is unlikely that Mr.~Liu will be set free soon, he appears to have received one small
windfall since the Norwegian Committee anointed him a Nobel laureate: his prison meals have improved
slightly. In recent phone conversations with friends and The A.P., Ms.~Liu said officials were
allowing him to buy individually prepared meals.

For the second time this week, Chinese officials canceled a meeting with Norway's fisheries
minister, according to the Norwegian newspaper Nordlys. The minister, Lisbeth Berg-Hansen, was to
have met China's deputy minister for food safety on Wednesday.

News of the canceled meetings was not reported by the Chinese news media, which has effectively
excised any mention of this year's Nobel Peace Prize.

As part of their efforts to reduce any impact of the prize domestically, the authorities have placed
restrictions on scores of intellectuals, academics and bloggers who have previously expressed
support for Charter 08 or Mr.~Liu. Several have been given eight-day administrative detentions,
while others have been placed under an extralegal form of house arrest that often includes the loss
of cellphone and Internet service.

Among those who have disappeared in recent days is Pu Zhiqiang, one of the country's best-known
rights lawyers. Although his phone was turned off Tuesday, a colleague in his Beijing law firm
confirmed that he had been taken away by the police.

\section{Worrying Over China and Food}

\lettrine{A}{re}\mycalendar{Oct.'10}{13} the Chinese coming?

That's the important question now being asked in Saskatchewan, a prairie province in Canada. It is
also the question of the moment on Wall Street.

Saskatchewan is home base for the Potash Corporation, the fertilizer company. If you care at all
about the future of the world's food supply, you care -- whether you know it or not -- about
Saskatchewan.

A consortium of state-backed Chinese companies and financiers may make a takeover offer for Potash
that rivals a \$38.6 billion hostile bid from BHP Billiton, and that prospect has lawmakers in
Washington, regulators in Canada and bankers on Wall Street all talking.

The politically charged subtext is this: Do we really want the Chinese to control the company that
has the largest capacity to produce fertilizer?

If that reminds you of 2005, when the China National Offshore Oil Company, or Cnooc, sought to buy
Unocal, until an outcry from Congress stopped it, you would be right.

But that outburst of protectionism was only about the nation's oil supply, and this would be about
something much more vital, food: 45 percent of Potash's production is sold to farmers in North
America. The big worry, in part, is that the Chinese could seek to redirect that supply to China,
starving other countries of a much-needed commodity.

Even for free marketers who say they believe that transactions should be able to cross borders
without political constraints, the questions being raised in Saskatchewan and elsewhere are the ones
that need to be asked.

Indeed, concern that politics may drive Chinese deal-making has grown amid recent reports that China
has banned exports of rare earth minerals to Japan. Prime Minister Wen Jiabao of China has denied
that the country has issued such a prohibition, but he acknowledged that the owners of rare earth
metals may have halted shipments because of their own feelings toward Japan.

(At the same time, however, another Chinese deal announced on Monday -- Cnooc's \$1.08 billion
investment for a third of Chesapeake Energy's oil and natural gas shale assets in Texas -- is not
expected to meet political resistance because the stake is passive.)

In the case of Potash, the Sinochem Group, China's largest fertilizer company, has been exploring a
possible bid, according to several media reports, and may win backing from funds like the China
Investment Corporation.

``It seems fairly certain that even if Sinochem puts together a financing consortium, the underlying
motivation would be to secure access to a key commodity,'' the Conference Board of Canada wrote in a
report about possible Chinese interest in Potash. ``Food security is an overriding concern in China,
arguably even more important than access to industrial materials.''

It is that kind of talk that has many analysts betting that the Chinese do not ultimately move ahead
with an offer.

``We believe that any bid from a Chinese state-owned entity would likely face significant Canadian
regulatory scrutiny,'' Glyn Lawcock, an analyst with UBS, wrote in a note to investors.

Under Canadian law the deal would have to pass muster with the government through Investment Canada,
which would need to rule that the deal was a ``net benefit'' to the country.

You might ask why BHP, the Australian commodities giant that is steadily cornering the market on a
variety of commodities, is not facing the same sort of scrutiny.

True, some questions are being raised, but the ``back up against the wall'' feeling doesn't seem to
be nearly as pronounced with BHP as it is with a Chinese state-sponsored bid.

For Saskatchewan, the deal boils down to which buyer is more likely to try to keep the price of
fertilizer high, therefore helping the tax base. Of course, that would also help keep food prices
high, which would arguably be bad for consumers all over the world, from Canada to China.

But Saskatchewan may be more concerned about local tax revenue. The Conference Board report noted:
``As a state-owned enterprise acting on behalf of consumers of potash, we assume that Sinochem has
strong incentives for lower prices and that it will not be guided by the same market discipline and
profit motive as commercial players,'' noting that ``China was one of the few countries not to cut
potash production in 2009 in response to falling demand and prices.''

In the new world of mergers and acquisitions -- one that turns China into a central actor -- the
highest bid may no longer be the ultimate criterion for accepting a deal or the test of whether the
deal is a success.

``The Chinese could justify a takeover premium as a sort of insurance premium to prevent BHP from
exercising similar market power in potash,'' the board wrote. ``Yet given the state-owned nature of
Sinochem, it becomes unclear whether this would be a corporate counterstrategy or state
counterstrategy.''

\section{In France, Labor Strikes Head for Showdown}

\lettrine{W}{eeks}\mycalendar{Oct.'10}{19} of protests over President Nicolas Sarkozy's efforts to
change France's pension system headed for a showdown on Tuesday in advance of a final parliamentary
vote on Wednesday. But with approval expected, Mr.~Sarkozy faces the possibility that winning will
not offer much political gain.

Labor unions again called for widespread work stoppages on Tuesday, adding to the disruption that
has been building since the first national protest on Sept.~7 and that has worsened with strikes at
oil refineries now in their second week and demonstrations by high-school students.

Trains have been canceled, commuters stranded and gas stations mobbed as the pumps ran dry of fuel
because of blockades of depots by oil industry workers. Violent clashes have broken out. The civil
aviation authority asked airlines to cut flights into French airports by as much as half. News
reports said a high school in Le Mans, southwest of Paris, was destroyed by arson early on Tuesday.

The capital's transport authorities said rail and subway services would be cut by as much as a half
on Tuesday, adding to the unease provoked by fuel shortages for commuters' cars.

The national railroad authority announced cancellations of around half its high-speed and normal
services on Tuesday, but said the Eurostar Paris-London link would not be affected.

At Orly airport near Paris, where half of the scheduled flights were set to be grounded on Tuesday,
travelers peered at departure boards recording a slew of cancellations and delays. At other French
airports, around one third of flights were to be affected by the strikes as the test of wills
between labor unions and the government unfolded.

Presenting himself as a champion of necessary change, Mr.~Sarkozy had proposed the measures to help
wrest France from the economic doldrums gripping many parts of Europe and to reverse years of
declining fortunes before elections in 2012. He holds a majority in both houses of Parliament. With
a Senate vote set for Wednesday and lower house approval already in hand, he believes he can bank on
success for the changes, which include increasing the minimum age of retirement by two years to 62.

But in fact, it is a high-stakes gamble magnified by the political arena in which it is played out.

``He has a clear majority in the two houses, so he has no difficulty in passing the reform,'' said
Pierre Haski, co-founder of the news Web site Rue89. ``But that does not give him legitimacy with
the public.''

At worst, Mr.~Haski said in a telephone interview, the upshot could be that Mr.~Sarkozy emerges from
the crisis as a lame-duck president for the next two years. ``It is a question of legality versus
legitimacy,'' he said.

Mr.~Sarkozy, who was aiming to be able to present himself for the next two years as a courageous
reformer in the national interest, may instead end up with the image of an elitist imposing unwanted
reforms on the poor. A widely quoted survey in Le Parisien on Monday said 71 percent of respondents
either supported or were sympathetic to the strikers. The telephone poll of 1,002 people was
conducted Oct.~15 and 16.

``The government can win it despite threats of violence in the street, despite shortages, or simply
by a vote of Parliament,'' said J\'erôme Sainte-Marie, who heads the polling institute C.S.A. ``But
these 71 percent translate into the cost of victory: It will be very high.''

He added, ``We are looking at a direct confrontation between public opinion and the president of the
republic.''

Neither does Mr.~Sarkozy have much room for retreat.

``He has gambled his prospects of victory on these reforms,'' said Mr.~Haski, the Rue89 co-founder.

The labor unions, though far from united, have exacted an enormous toll. Already, the blockades of
France's 200 fuel depots and strikes at most of its 12 refineries have left service stations starved
of fuel. Fearful that the pumps would run dry, many drivers scrambled to fill up on Monday while
they could, contributing to the pressure on supplies, particularly of the diesel fuel powering many
French cars.

The government said that only 2 percent of the country's 13,200 service stations had actually run
dry. But other estimates put the number of gas stations running out of some or all types of fuel at
10 to 15 percent.

Aad Van Bohemen, who heads the emergency policy division at the International Energy Agency, said
France had been forced to reach into its 30-day emergency supplies of fuel at depots because of the
strikes at the refineries.

``The question is how to get it to the pumps,'' he said in a telephone interview. ``Some depots are
blocked, but there's a lot of panic-buying going on among the French people.''

On Monday, half of France's high-speed train services were canceled. A spokeswoman for Air France
said that at least two long-haul flights from Paris, to Seattle and Mumbai, had taken off with
insufficient fuel, requiring refueling stops en route.

East of Paris, oil industry workers used blazing tires to prevent access to a refinery. On highways
near Lille, in the north, and Lyon, in the south, truckers and protesters snarled traffic.

In the Paris suburb of Nanterre, riot police officers fired tear gas at about 300 high school
protesters who had set fire to a car, wrecked bus stops and hurled rocks, witnesses said. The
authorities said disturbances were reported on Monday from 261 of the country's 4,300 high schools,
slightly fewer than in the unrest on Friday.

But the protesters are fighting the clock. Many analysts said that they believed high school
students would begin disappearing from the streets as school vacations that start Friday approached.

\section{Designer Labels Go Pint Size}

\lettrine{G}{ucci}\mycalendar{Oct.'10}{19}, Fendi and Stella McCartney are among the latest
designers to add luxury children's collections to their company's stable, a small step for high-end
fashion but among the first signs of growth since the recession.

Business-wise, it makes a lot of sense: Kids' garments take a beating and their owners grow out of
them fast, meaning that as long as there are children, there will always be a need for more
children's clothing. And parents with money can be persuaded to spend it on high-quality clothing
for their children.

According to the Datamonitor Group, an industry research organization, children's wear in the
European Union is outperforming the rest of the market.

Parents have ``a propensity to show how they have weathered the economic storm through their kids,''
said Marshal Cohen, chief industry analyst with the market research company NPD. Couple that with a
desire to insulate their children from economic constraints, he added, and ``the first place to
spend goes to the kids.''

So how are fashion houses creating children's clothing that still clearly represents the line?

Some are taking the ``mini me'' approach. Paul Smith offers up pint-size floral print shirts. Jean
Paul Gaultier makes tiny trench coats. Missoni has multicolored mini knitwear.

At Gucci, the creative director, Frida Giannini, said she had some concerns at the beginning because
of the ``sexy attitude and the glamour of Gucci.'' Her solution? The Gucci archives. She
incorporated images of bamboo and elements of the horse world in the clothing line, which includes
shoes, accessories and, of course, sunglasses.

For Silvia Fendi and Stella McCartney the move into children's wear is more personal. Ms.~Fendi
began a children's line in partnership with the Italian company Simonetta when she learned she was
to have a granddaughter. ``I started with bibs and strollers and then it just turned into a whole
line of clothing for kids from baby to 12 years old,'' she said.

Ms.~McCartney, who is pregnant with her fourth child, has had two successful capsule collections
with Gap and will present her signature kids' collection next month.

She said she set out to make affordable clothing that is not so precious that it never gets worn and
enjoyed.

The houses are also focusing attention on where and how their children's clothing is made.
Ms.~McCartney, known for a commitment to eco-friendly clothing, will continue that with her
children's collection. And the Gucci Kids collection will be made entirely in Italy. ``There is a
fashion element to the collection,'' Ms.~Fendi said. ``But it is important that it be clothing that
works for children and that it be made well from good materials.''

\section{U.S.~Pushes to Ease Technical Obstacles to Wiretapping}

\lettrine{L}{aw}\mycalendar{Oct.'10}{19} enforcement and counterterrorism officials, citing lapses
in compliance with surveillance orders, are pushing to overhaul a federal law that requires phone
and broadband carriers to ensure that their networks can be wiretapped, federal officials say.

The officials say tougher legislation is needed because some telecommunications companies in recent
years have begun new services and made system upgrades that create technical obstacles to
surveillance. They want to increase legal incentives and penalties aimed at pushing carriers like
Verizon, AT\&T, and Comcast to ensure that any network changes will not disrupt their ability to
conduct wiretaps.

An Obama administration task force that includes officials from the Justice and Commerce
Departments, the F.B.I. and other agencies recently began working on draft legislation to strengthen
and expand a 1994 law requiring carriers to make sure their systems can be wiretapped. There is not
yet agreement over the details, according to officials familiar with the deliberations, but they
said the administration intends to submit a package to Congress next year.

To bolster their case, security agencies are citing two previously undisclosed episodes in which
major carriers were stymied for weeks or even months when they tried to comply with court-approved
wiretap orders in criminal or terrorism investigations, the officials said.

Albert Gidari Jr., a lawyer who represents telecommunications firms, said corporations were likely
to object to increased government intervention in the design or launch of services. Such a change,
he said, could have major repercussions for industry innovation, costs and competitiveness.

``The government's answer is `don't deploy the new services -- wait until the government catches
up,' '' Mr.~Gidari said. ``But that's not how it works. Too many services develop too quickly, and
there are just too many players in this now.''

Under the 1994 law, the Communications Assistance to Law Enforcement Act, telephone and broadband
companies are supposed to design their services so that they can begin conducting surveillance of a
target immediately after being presented with a court order.

Officials from the Justice Department, the National Security Agency, the F.B.I. and other agencies
recently began working on a draft of a proposal to strengthen and expand that law. There is not yet
internal agreement over its details, according to officials familiar with the deliberations, but
they said the Obama administration intended to submit a package to Congress next year.

The disclosure that the administration is seeking ways to increase pressure on carriers already
subject to the 1994 law comes less than a month after The New York Times reported on a related
effort: a plan to bring Internet companies that enable communications -- like Gmail, Facebook,
BlackBerry and Skype -- under the law's mandates for the first time, a demand that would require
major changes to some services' technical designs and business models.

The push to expand the 1994 law is the latest example of a dilemma over how to balance Internet
freedom with security needs in an era of rapidly evolving -- and globalized -- technology. The issue
has added importance because the surveillance technologies developed by the United States to hunt
for terrorists and drug traffickers can be also used by repressive regimes to hunt for political
dissidents.

An F.B.I. spokesman said the bureau would not comment about the telecom proposal, citing the
sensitivity of internal deliberations. But last month, in response to questions about the Internet
communications services proposal, Valerie E.~Caproni, the F.B.I.'s general counsel, emphasized that
the government was seeking only to prevent its surveillance power from eroding.

Starting in late 2008 and lasting into 2009, another law enforcement official said, a ``major''
communications carrier was unable to carry out more than 100 court wiretap orders. The initial
interruptions lasted eight months, the official said, and a second lapse lasted nine days.

This year, another major carrier experienced interruptions ranging from nine days to six weeks and
was unable to comply with 14 wiretap orders. Its interception system ``works sporadically and
typically fails when the carrier makes any upgrade to its network,'' the official said.

In both cases, the F.B.I. sent engineers to help the companies fix the problems. The bureau spends
about \$20 million a year on such efforts.

The official declined to name the companies, saying it would be unwise to advertise which networks
have problems or to risk damaging the cooperative relationships they government has with them. For
similar reasons, the government has not sought to penalize carriers over wiretapping problems.

Under current law, if a carrier meets the industry-set standard for compliance -- providing the
content of a call or e-mail, along with identifying information like its recipient, time and
location -- it achieves ``safe harbor'' and cannot be fined. If the company fails to meet the
standard, it can be fined by a judge or the Federal Communication Commission.

But in practice, law enforcement officials say, neither option is ever invoked. When problems come
to light, officials are reluctant to make formal complaints against companies because their
overriding goal is to work with their technicians to fix the problem.

Once a carrier's interception capability is restored -- even if it is at taxpayer expense -- its
service is compliant again with the 1994 law, so the issue is moot.

The F.C.C. also moves slowly, officials complain, in handling disputes over the ``safe harbor''
standard. For example, in 2007 the F.B.I. asked for more than a dozen changes, like adding a mandate
to turn over more details about cellphone locations. The F.C.C. has still not acted on that
petition.

Civil liberties groups contend that the agency has been far too willing on other occasions to expand
the reach of the 1994 law.

``We think that the F.C.C. has already conceded too much to the bureau,'' said Marc Rotenberg, the
president of the Electronic Privacy Information Center. ``The F.B.I.'s ability to have such broad
reach over technical standard-setting was never anticipated in the 1994 act.''

The Obama administration is circulating several ideas for legislation that would increase the
government's leverage over carriers, officials familiar with the deliberations say.

One proposal is to increase the likelihood that a firm pays a financial penalty over wiretapping
lapses -- like imposing retroactive fines after problems are fixed, or billing companies for the
cost of government technicians that were brought in to help.

Another proposal would create an incentive for companies to show new systems to the F.B.I. before
deployment. Under the plan, an agreement with the bureau certifying that the system is acceptable
would be an alternative ``safe harbor,'' ensuring the firm could not be fined.

\section{Meeting Pakistanis, U.S.~Will Try to Fix Relations}

\lettrine{A}{s}\mycalendar{Oct.'10}{19} Pakistani civilian and military leaders arrive here this
week for high-level meetings, the Obama administration will begin trying to mend a relationship
badly damaged by the American military's tough new stance in the region.

Among the sweeteners on the table will be a multiyear security pact with Pakistan, complete with
more reliable military aid -- something the Pakistani military has long sought to complement the
five-year, \$7.5 billion package of nonmilitary aid approved by Congress last year. The
administration will also discuss how to channel money to help Pakistan rebuild after its ruinous
flood.

But the American gestures come at a time of fraying patience on the part of the Obama
administration, and they will carry a familiar warning, a senior American official said: if Pakistan
does not intensify its efforts to crack down on militants hiding out in the tribal areas of North
Waziristan, or if another terrorist plot against the United States were to emanate from Pakistani
soil, the administration would find it hard to persuade Congress or the American public to keep
supporting the country.

``Pakistan has taken aggressive action within its own borders. But clearly, this is an ongoing
threat and more needs to be done,'' the State Department spokesman, Philip J.~Crowley, said Monday.
``That will be among the issues talked about.''

The Pakistanis will come with a similarly mixed message. While Pakistan is grateful for the strong
American support after the flood, Pakistani officials said, it remains frustrated by what it
perceives as the slow pace of economic aid, the lack of access to American markets for Pakistani
goods and the administration's continued lack of sympathy for the country's confrontation with
India.

Other potentially divisive topics are likely to come up, too, including NATO's role in
reconciliation talks between President Hamid Karzai of Afghanistan and the Taliban. Pakistani
officials say they are nervous about being left out of any political settlement involving the
Taliban.

Still, in a relationship suffused by tension and flare-ups -- most recently over a NATO helicopter
gunship that accidentally killed three Pakistani soldiers and Pakistan's subsequent decision to
close a supply route into Afghanistan -- this regular meeting, known here as the strategic dialogue,
serves as a lubricant to keep both countries talking.

At this meeting, Secretary of State Hillary Rodham Clinton will formally introduce the new American
ambassador to Pakistan, Cameron Munter. Mr.~Munter, who recently served in Iraq, replaces Anne
W.~Patterson, who just wrapped up her tour of duty in Islamabad.

``No country has gotten more attention from Secretary Clinton than Pakistan,'' said Richard
C.~Holbrooke, the administration's special representative for Afghanistan and Pakistan. Pakistan's
delegation will be led by its foreign minister, Shah Mahmood Qureshi, but much of the attention will
be on another official, the military chief, Gen.~Ashfaq Parvez Kayani, who is viewed by many as the
most powerful man in Pakistan.

White House and Pentagon officials said one immediate goal of this meeting was to ease the tensions
that led Pakistan to close the border crossing at Torkham, halting NATO supplies into Afghanistan.
Officials on both sides said that acrimony from the border flare-up had already receded, soothed by
the multiple apologies that American officials made to Pakistan last week.

Last week, Adm. Mike Mullen, the chairman of the Joint Chiefs of Staff, said that General Kayani had
assured him that Pakistan's army would tackle the North Waziristan haven, but on Pakistan's
timetable. In an interview, Pakistan's ambassador to the United States, Husain Haqqani, said, ``Our
American partners understand that we have 34,000 troops in North Waziristan. Our soldiers have been
engaged in flood relief after history's worst floods. It is not a question of lack of will.''

The new security pact would have three parts: the sale of American military equipment to Pakistan, a
program to allow Pakistani military officers to study at American war colleges and counterinsurgency
assistance to Pakistani troops.

Currently, the United States spends about \$1.5 billion a year to provide this same assistance, but
it is doled out year by year. The new agreement, if endorsed by Congress, would approve a multiyear
plan assuring stability and continuity in the programs, although Congress would continue to
appropriate the financing on a yearly basis. ``This is designed to make our military and security
assistance to Pakistan predictable and to signal to them that they can count on us,'' said a senior
official.

At the last dialogue in Islamabad in July, Mrs.~Clinton presented more than \$500 million in
economic aid, including plans to renovate hospitals, upgrade hydroelectric dams, improve water
distribution and help farmers export mangoes. But the floods upended those plans, and officials said
they now planned to redirect funds to more urgent needs.

This week's meeting will also be shadowed by a new eruption of political instability in Pakistan:
the government of President Asif Ali Zardari is locked in a confrontation with the Supreme Court
over the court's demand that senior ministers be fired on corruption charges. Analysts said they
were less worried about the atmospherics than the underlying differences in perspective. The
administration's public contrition for the cross-border attack has largely resolved that issue, said
Daniel S.~Markey, senior fellow for India, Pakistan and South Asia at the Council on Foreign
Relations.

But Mr.~Markey said he saw potential friction stemming from the American openness to reconciliation
with the Taliban. With the United States facilitating rather than guiding the talks, he said, there
could be poor coordination between the Afghans, NATO and others -- all of which would rattle the
Pakistanis.

``Washington is opening the door to a range of negotiations with groups that it has discouraged
Pakistan against working with in the past,'' he said. ``This sends a mixed signal, and cannot help
but encourage hedging on Islamabad's part.''

Another potential bone of contention is one of President Obama's nuclear objectives: a global accord
to end the production of new nuclear fuel. Pakistan has led the opposition to the accord. And
without its agreement, the treaty would be basically useless.

Mr.~Qureshi blamed the United States for the situation, saying Washington signed a civilian nuclear
accord with India that discriminated against Pakistan. ``You have disturbed the nuclear balance,''
he said in a recent interview in New York, ``and we have been forced to develop a new strategy.''

\section{China Plans to Reduce Its Exports of Minerals}

\lettrine{T}{he}\mycalendar{Oct.'10}{19} Chinese government plans a further reduction, of up to 30
percent, next year in its quotas for exports of rare earth minerals, to try to conserve dwindling
reserves of the materials, the official China Daily newspaper said on Tuesday.

Plans for smaller export quotas come just four days after American trade officials announced that
they would investigate whether China is violating international trade rules with a wide range of
policies to help its clean energy industries. One of the policies under investigation involves
China's steady reductions in rare earth export quotas since 2005 and its imposition of steep taxes
on these exports.

China mines 95 percent of the world's rare earths. They are crucial for compact fluorescent bulbs,
hybrid gasoline-electric cars, large wind turbines and other clean energy technologies, as well as
for mobile phones and a wide range of military applications, like missiles.

Chao Ning, a commerce ministry official, told a conference in Beijing on Saturday that China had
sizable reserves of the lighter elements among the 17 rare earth elements, but only had 15 or 20
years' worth of reserves left of medium and heavy rare earths and needed to conserve those. Light
rare earths are used in lower-tech applications like oil refining and glass manufacturing, while
medium and heavy rare earths are used more in clean energy and military applications.

China Daily is owned and supervised by the Chinese government, and presents official views on a
range of issues. Its article attributed the planned quota reduction to an unidentified commerce
ministry official.

Bloomberg News quoted another commerce ministry official, Jiang Fan, who said at a conference in
Xiamen that she was not aware of plans for a further reduction in rare earths.

Wang Caifeng, the secretary general of the Chinese Rare Earths Industry Association, predicted at
the conference on Tuesday that domestic demand for rare earths in China would soar to 130,000 tons
in 2015, from 75,000 tons now, Bloomberg reported. The export quotas for this year total just 30,300
tons.

Commerce ministry representatives in Beijing did not answer calls for comment.

\section{Top Canadian Commander Pleads Guilty to Murders}

\lettrine{I}{n}\mycalendar{Oct.'10}{19} April, Canadians reacted with shock after a top Canadian
military commander, who frequently piloted planes for top political figures and dignitaries,
including Queen Elizabeth II, was charged with rape, murder and an extensive campaign of perverse
home break-ins.

On Monday, gasps were intermixed with tears in a courtroom here as an audience heard details that
made clear for the first time the scale and perversity of the crimes to which the military
commander, Col.~David Russell Williams, 47, pleaded guilty -- hundreds of underwear thefts, many
from young girls, that escalated to the assault and murder of two women.

The unmasking of Colonel Williams as a sexual killer has been a blow for the Canadian Armed Forces.
Until his arrest, he commanded Canada's largest air base, the logistical fulcrum for the country's
military mission in Afghanistan.

Apparently bringing his logistical and organizational skills to bear, Colonel Williams wrote
detailed accounts of his crime and meticulously photographed the evidence, compiling the data in
what prosecutors called ``a deeply nested and complex series of subfolders'' on two computer hard
drives.

Five thick catalogs of photographs, only a sampling of the evidence, sat in front of Justice Robert
F.~Scott during Monday's hearing.

Colonel Williams broke into houses primarily in the two neighborhoods where he had homes. He broke
into many of them repeatedly, nine times in one case. The break-ins were deft. Most of the victims
were unaware that their homes had been entered or that anything had been stolen until they or their
families were contacted by the police after Colonel Williams was arrested in February.

Colonel Williams came under suspicion when the body of the second murder victim was found near his
cottage in the village of Tweed, about 40 miles from the Canadian Forces Base in Trenton. After
stopping him at a roadblock, the police noticed that the unusual tread patterns on the tires of his
sport utility vehicle matched those found at the murder scene. A subsequent search of his home in
Ottawa turned up hundreds of pieces of girls' and women's underwear and his meticulous photographic
record.

It took a court clerk 40 minutes to simply read the list of 88 charges before a room that included
family members of the victims. Looking pale and with his voice barely audible, Colonel Williams
offered his plea.

Prosecutors reviewed how Colonel Williams began each break-in by photographing the victim's room and
underwear drawer. In most of the early break-ins, the photographs show the rooms of girls, including
11-year-old twins, that are decorated with stuffed animals and animal photographs.

He then photographed himself -- often sexually aroused or masturbating -- modeling their underwear.

Once back home, he photographed his total haul of the underwear, and then each individual item.
Occasionally, he added captions. One of those read, ``Merci beaucoup.''

He stole 87 pairs of underwear belonging to an Ottawa high school student in a single break-in.
Twice, he took loads of the stolen garments to the outskirts of Ottawa and burned them.

Robert Morrison, one of five prosecutors, told the court that Colonel Williams's ``peculiar
sexuality'' first led him to break into a neighbor's home to masturbate on the daughter's bed with
her underwear. That break-in, in Tweed, escalated into break-ins of 47 other homes near Tweed as
well as in Ottawa, where he spent weekends with his wife.

But the crimes escalated. He broke into the homes of two women near his air base last September,
forced them to strip and then blindfolded and photographed them. But he was not recognized by the
victims.

Last October, he broke into the home of Cpl. Marie-France Comeau, an air force flight attendant
based at Trenton who had flown with Colonel Williams. The police said she died after being beaten
and having her mouth and nose sealed with tape.

In late January, the second woman, Jessica Lloyd, 27, was reported missing. Her body was found
Feb.~8.

\section{Aviation World at Odds Over Plane Loans}

\lettrine{B}{ack}\mycalendar{Oct.'10}{19} in the mid-1980s, Boeing and Airbus avoided a trade war by
making a gentlemen's agreement not to seek government financing to sell planes in each other's home
markets. The deal symbolized an aviation world dominated by the United States and Europe.

That world is a lot different now.

No longer can Airbus and Boeing count on being the biggest manufacturers of large airplanes.
Manufacturers in Canada and Brazil are seeking to gain a foothold in this lucrative market. And not
too far in the future, China, Japan and Russia will be competing as well.

At the same time, airlines from Asia and the Middle East that did not exist three decades ago, or
were small players, are now vying for a much bigger piece of the global aviation market.

The old rules, some airlines and plane manufacturers contend, are not working anymore.

This week, negotiators from the European Union and countries including the United States, Japan and
Brazil will meet in Ottawa to grapple with the politically charged issue. Their goal is to draw up a
new agreement by the end of the year on the rules for financing the airplane business -- and
specifically, how much government assistance is permitted.

The various interests are in conflict, and it is unclear if the talks will be successful. ``The home
market restriction has become unsustainable,'' said Scott Scherer, vice president for strategic
regulatory policy at Boeing's financing unit. ``It's the big elephant in the room.''

The discussions, which are taking place under the auspices of the Organization for Economic
Cooperation and Development, are complicated by the fact that what might be in the interest of the
airlines may not necessarily work for aircraft manufacturers.

The so-called home market rule applies to only four European countries -- France, Germany, Britain
and Spain -- where Airbus planes are produced, and to the United States, where Boeing is based. As a
result, for example, Ryanair, which is based in Dublin and is one of Europe's biggest low-cost
carriers, has tapped into export-credit financing to buy most of its fleet of 200 Boeing planes. Its
London-based rival, EasyJet, cannot make a similar deal.

The loudest objections to the current rules have come from European airlines, which say that
cheaper, government-backed loans have helped fuel the growth of their rivals in Asia and the Middle
East and given them an unfair advantage.

The charge is mostly aimed at Emirates Airlines in Dubai, whose rapid expansion in recent years has
rattled airlines in Britain, France and Germany. The airline threatens to take their international
passengers away from European hubs and fly them instead through its gleaming new terminals in Dubai.

Boeing and Airbus also worry about new competition. At the moment, their biggest concern is the
emergence of a new jet from Bombardier of Canada, the C-Series. This single-aisle plane, due to
enter service in 2013, will be able to seat about 130 passengers, posing a direct challenge for the
first time to the best-selling Boeing 737 and Airbus A320.

The current rules set separate financing standards for large airplanes and regional jets with 100 or
fewer seats. But the C-Series, with its larger capacity and range, has now blurred that line, Boeing
and Airbus say. This, they argue, puts them at a competitive disadvantage for sales in the
single-aisle category, the most profitable segment of the market.

Bombardier said it would welcome a new arrangement where airlines could obtain financing, regardless
of where they were based. ``We do not support the so-called home market rule,'' Marc Meloche, the
senior director for structured finance at Bombardier Aerospace, said. ``All customers should have
access to all financing sources, based on market principles.''

Besides the Canadian factor, the other major catalyst for the talks has been the economic downturn,
which essentially shut down commercial credit markets and left official export agencies, especially
the Export-Import Bank of the United States, as outsize purveyors of financing.

In 2009, about 35 percent of Boeing and Airbus sales were financed by credit agencies, according to
the manufacturers, up from about 20 percent before the economic downturn.

In the United States, the Export-Import bank guaranteed \$8.6 billion in commercial aviation loans
in its fiscal year ending September 2009, nearly double what it typically helped finance each year
since 2002. This year, export credit agencies in the United States and Europe are expected to
guarantee more than \$15 billion in civil aviation loans, about the same as in 2009.

Export agencies, set up to help finance exports to countries with weaker credit, usually require
quicker repayment than commercial loans and impose more restrictions on the airlines. But in today's
depressed market, industry experts say, the agencies' loans are on average about three to five
percentage points lower than commercial loans. This can be a significant factor for planes whose
list price ranges from \$60 million for a single-aisle plane to about \$350 million for the largest
plane today, the Airbus A380.

The export credit agencies ``did a great job in 2009, when banks were unable to fulfill their
role,'' said Christian McCormick, chief executive of Natixis Transport Finance in Paris. ``They have
created something of an addiction by the airlines to this product that is completely out of the
market.''

Because of the home market restrictions, American and European airlines could not get government
backing for new planes. But airlines did, including Emirates Airlines, Korean Airlines and Cathay
Pacific.

European airlines were already hurting from the rapid rise of some of these carriers, particularly
Emirates and Etihad Airlines in the Persian Gulf, saying they pay lower fuel prices, airport fees
and corporate taxes. Opposition to the government loans has become the latest tactic the airlines
are now using to curtail these carriers' growth.

In a joint letter dated Oct.~11, airlines from Europe and the United States proposed limiting to 20
percent of new plane deliveries the export-credit backing that an airline can receive each year. The
letter was signed by nine European airlines, including Air France, British Airways and Lufthansa,
and the Air Transport Association, which represents airlines in the United States.

They also want export credit to be provided on less favorable terms than commercial bank loans.

The airlines may get some of what they want. A draft agreement that began circulating among
governments this month would increase the price of export agency financing to airlines, according to
one person involved in the negotiations, but not include the cap on lending.

``This is becoming a big political fight,'' said Adam Pilarski, an economist and senior vice
president at Avitas, a consultancy in Chantilly, Va. ``Incumbent airlines are beginning to feel that
if they do not do something, somebody may take their place.''

``Nobody complains about airlines from Uganda having this advantage,'' Mr.~Pilarski said.

The complaints have raised hackles in Dubai, where Emirates dismisses claims that it has received
any competitive advantage from state-subsidized loans, saying it has financed only 20 percent of its
fleet through export credits.

``We have grown without subsidy through the success of our commercially driven business model -- and
see no reason to apologize for what we have achieved,'' Tim Clark, the president of Emirates
Airline, said in a statement.

Some aircraft financiers, however, were skeptical of Emirates' claims. One European banker, who
would speak only anonymously because his bank provides loans to the airline, estimated that nearly
50 percent of the Emirates jet purchases had been subsidized by Western export credit agencies.

Emirates raised eyebrows again this summer with two more huge jet orders -- including plans to buy
an additional 32 Airbus A380 jets valued at \$11 billion at list prices, bringing its total orders
for the twin-deck superjumbo to 90. In July, Emirates placed another huge order, for 30 Boeing 777s.

\section{I.B.M. Rides Global Focus on Services to Deliver a 12\% Increase in Profit}

\lettrine{T}{he}\mycalendar{Oct.'10}{19} leading high-technology industrial companies like Intel,
General Electric and I.B.M., which have reported quarterly results in the last several days, are
seen by analysts and economists as bellwethers of the economy, because their chips, equipment and
services are used in so many industries.

But the quarterly scorecards from such giants of corporate America increasingly point to trends in
the global economy rather than at home. Large technology companies typically have most of their
sales overseas, and they are tilting more abroad to pursue the fast-growing markets in China, India
and elsewhere.

The current wave of globalization, analysts say, is different from in the past. Now companies are
also spreading work and production worldwide to improve productivity and profits.

The migration abroad of sales and operations, analysts say, also helps explain why American
corporations are in good health, while the economy sputters and unemployment remains high.

``The success of large corporations that are headquartered in the United States has less and less to
do with the success of the American economy,'' said Robert B.~Reich, a professor of public policy at
the University of California, Berkeley.

The global stance of many big corporations is contributing to a projected 35 percent jump in profits
this year among companies in the Standard \& Poor's 500.

For the last six quarters, Wall Street has underestimated the profits of S.\&.P. 500 companies by an
average of 10 percent, and Edward Yardeni, an independent economist, says analysts are struggling to
keep up. ``It's really hard for them to model into their spreadsheets the full impact of
globalization,'' Mr.~Yardeni said.

I.B.M. is at the forefront of the recent wave of globalization. In reporting its third-quarter
results on Monday, I.B.M. said it received a lift from strong growth in big emerging markets, led by
China, India, Brazil and Russia, where revenue jumped 29 percent. A broader category earmarked by
I.B.M. as growth markets, which include South Africa, Vietnam and the Czech Republic, experienced a
16 percent increase in revenue.

``The real slugger in the quarter was the growth markets,'' Mark Loughridge, I.B.M.'s chief
financial officer, said.

Last month, I.B.M. announced that it had cinched a large deal in Africa to provide the technology
behind new cellphone services. The deal, a 10-year agreement with India's largest cellphone
operator, Bharti Airtel, will be worth up to \$1.5 billion, analysts estimate.

I.B.M. reported a 12 percent increase in net income to \$3.6 billion, compared with \$3.2 billion in
the year-earlier quarter. Its earnings of \$2.82 a share were above the \$2.75 estimate of Wall
Street analysts.

Revenue was \$24.3 billion, a 3 percent rise, and slightly above analysts' forecast of \$24.1
billion.

I.B.M. raised its full-year earnings forecast to ``at least \$11.40'' a share from ``at least
\$11.25.''

Still, part of the improvement in profits came from lower taxes. And the total of services contracts
signed -- an indicator of future revenue -- was weaker than expected. I.B.M. said one big contract
signed in early October, if signed a few days earlier, would have put total signings for the quarter
at \$12.7 billion, slightly above analysts' estimates.

After reaching a 52-week high during the day, I.B.M.'s stock price dropped more than 3 percent in
after-hours trading.

The most far-reaching change at I.B.M. has been the global overhaul of its services and software
businesses, which now account for 80 percent of revenue. The transformation was partly of necessity
-- to address the competitive threat posed by low-cost Indian outsourcing companies including
Infosys, Wipro and Tata.

In services, I.B.M viewed the Indian outsourcers as a challenge unnervingly similar to the one faced
by its mainframe business in the early 1990s. In hardware, the new low-cost technology of
microprocessors, used in personal computers, disrupted the mainframe business, sending its profits
plummeting.

In the mainframe case, I.B.M.'s researchers saw the new technology and gave a warning in the 1980s.
But shifting then would have required retooling plants and cutting workers, when I.B.M. was still
comfortably profitable.

``Management waited until the tsunami was hitting before the action was taken,'' recalled Irving
Wladawsky-Berger, a former I.B.M. executive.

Under Samuel J.~Palmisano, who became chief executive in 2002, I.B.M. was determined to move earlier
in services. Indian programmers worked for a fraction of the wages of their counterparts in America
and Europe.

So the company expanded in India. In 2003, it had 9,000 workers in India. Today, it has more than
75,000, analysts estimate.

Cutting costs was a crucial step, but I.B.M. also shifted the focus of its research labs to
services. Applying science and software automation to services has increased productivity and let
I.B.M. do more sophisticated work with higher profit margins.

``In services, we got way ahead of the game,'' said Mr.~Wladawsky-Berger, who retired from I.B.M. in
2007.

The transition has not come easily. I.B.M. sets aside about \$400 million a year for severance costs
and other payments to cut about 8,000 employees annually, analysts estimate. Since 2003, I.B.M. has
cut its payroll in the United States by roughly 30,000 workers, to about 105,000.

I.B.M. has also hired thousands of workers in the United States, to tap new skills like data
scientists and analysts, even as it cuts elsewhere. The trimming these days is a steady,
evolutionary process by a company that is prospering -- not the drastic surgery of a company in deep
trouble, as I.B.M. was in the 1990s.

\section{Facebook Vows to Fix a Flaw in Data Privacy}

\lettrine{W}{hen}\mycalendar{Oct.'10}{19} you sign up for Facebook, you enter into a bargain. You
share personal information with the site, and Facebook agrees to obey your wishes when it comes to
who can see what you post.

At the same time, you agree that Facebook can use that data to decide what ads to show you.

It is a complicated deal that many people enter into without perhaps fully understanding what will
happen to their information. It also involves some trust -- which is why any hint that Facebook may
not be holding up its end of the bargain is sure to kick up plenty of controversy.

The latest challenge to that trust came on Monday, when Facebook acknowledged that some applications
on its site, including the popular game FarmVille, had improperly shared identifying information
about users, and in some cases their friends, with advertisers and Web tracking companies. The
company said it was talking to application developers about how they handled personal information,
and was looking at ways to prevent this from happening again.

Facebook's acknowledgment came in response to an article in The Wall Street Journal that said
several popular applications were passing a piece of data known as a user ID to outside companies,
in violation of Facebook's privacy policy.

Having a user ID allows someone to look up that user's name and any data posted on that person's
public profile, like a college or favorite movies, but not information that the user had set to be
visible only to friends.

Privacy advocates and technology experts were split on the significance of the issue.

``That is extremely serious,'' said Peter Eckersley, a senior staff technologist at the Electronic
Frontier Foundation, an online liberties group.

Mr.~Eckersley said advertisers could use the user IDs to link individuals with information they had
collected anonymously about them on the Web. ``Facebook, perhaps inadvertently, is leaking the magic
key to tracking you online,'' he said.

At the same time, Mr.~Eckersley said there was no evidence that anyone who had access to this data
had actually misused it.

Zynga, the maker of FarmVille and other games on Facebook that have a combined 219 million users,
did not respond to requests for comment.

Several technology pundits and bloggers minimized the issue, with some saying that credit card
companies and magazines have access to far more detailed information about customers than any
Facebook application.

Facebook also sought to play down the importance of the leak, saying the sending of user IDs
appeared to have been inadvertent. ``Press reports have exaggerated the implications of sharing'' a
user ID, Mike Vernal, a Facebook engineer, wrote on a company blog for application developers.
``Knowledge of a UID does not enable anyone to access private user information without explicit user
consent.''

In a statement, Facebook said that while it would be a challenge to do so, it planned to introduce
``new technical systems that will dramatically limit the sharing of user IDs,'' and would continue
to enforce its policies on outside applications, shutting them down when necessary. It added that
the companies that had received the user IDs said they had not made use of them.

Regardless, the problem underscores another challenge facing the company: Facebook has grown so
rapidly, in both users and in technical complexity, that it finds it increasingly difficult to
control everything that happens on its site. In addition to more than 500 million Facebook users,
there are more than one million third-party applications running on the site.

The latest information leak was made possible by a quirk in a long-established technical standard
used by Web browsers. The standard allows Web sites to record the address of the page a user clicked
on to arrive there, a bit of information known as a referrer.

Facebook has been including user IDs in these referrers for some time, and last year technology
experts pointed out that user IDs had leaked to advertisers that way. Facebook fixed that this year,
but apparently never addressed the problem when it came to referrers used by applications on its
site.

``Facebook isn't benefiting from it, and Facebook is not intentionally leaking this data,'' said
Christopher Soghoian, a privacy advocate and research fellow at the Center for Applied Cybersecurity
Research at Indiana University. ``But it is not a trivial thing to re-engineer their systems.''

This year he filed a complaint with the Federal Trade Commission, claiming Google was leaking
personal information because search terms appeared in its referrers.

The latest issue may have had particular resonance with Facebook users because the company has been
reeling from a series of privacy controversies, in part because it has been subtly pushing users to
share data more publicly.

This year, for example, many users complained when Facebook changed the way in which users expressed
preferences for certain movies or bands, essentially making it more difficult to keep that
information private.

And in May, after a series of complaints from some users and privacy advocates, the company made
wholesale changes to its privacy settings.

Mark Zuckerberg, the company's chief executive, apologized to users, saying the settings were often
too complicated for people to understand. Despite the changes, the privacy issue has continued to
dog Facebook.

``This is one more straw on the camel's back that suggests that Facebook needs to think holistically
not just about its privacy policies, but also about baking privacy into their technical design,''
said Deirdre Mulligan, a privacy expert and professor at the School of Information at the University
of California, Berkeley.

\section{Sales and Profit Surge for Apple, but Its Margins Slip}

\lettrine{S}{trong}\mycalendar{Oct.'10}{19} sales of iPads, iPhones and even Mac computers produced
record revenue and profit for Apple in its fourth quarter.

It was not enough, however, to sustain Wall Street's exuberance for the consumer electronics company
that has seemed to do everything right in analysts' eyes. The company's shares fell about 6 percent
in after-hours trading on Monday after the company announced its results.

Apple said that it sold 14.1 million iPhones in the quarter, ended Sept.~25, an increase of 91
percent from a year earlier. Consumers bought 4.2 million iPads, the tablet computer it introduced
in April. Mac sales totaled 3.9 million, up 27 percent.

But buried among quarterly results that any company would be more than happy to emulate was a
decline in gross profit margins. Investors disliked the small blemish, sending Apple's shares down.

``There's a lot of hype built up into Apple's earnings,'' said Shannon Cross, the managing director
of Cross Research. She added, ``Everything must go down at some point, I suppose. Clearly, there was
pressure on the iPhone gross margins and the iPad.''

Apple's success has helped to propel its shares up over the last year, to close on Monday in regular
trading at \$318, a high.

Otherwise, analysts remained enthusiastic about Apple, based in Cupertino, Calif. Indeed, it was a
quarter that highlighted the company's dominance in consumer electronics. The company said net
income for the quarter rose 70 percent, to \$4.31 billion, or \$4.64 a share, from \$2.53 billion,
or \$2.77 a share, a year earlier. Revenue rose 67 percent, to \$20.34 billion, from \$12.21
billion.

On average, analysts had expected Apple to report net income of \$4.06 a share on revenue of \$18.86
billion.

Apple's profit margins are the envy of the consumer electronics industry. The problem was that the
company's newest products ware not as profitable as its computers and iPod music players. Strong
sales of lower-margin products -- the iPad among them -- caused the decline, according to Apple
executives.

The company said gross margins fell to 36.9 percent, from 41.8 percent in the quarter a year ago.
Apple predicted that its margins would slump a bit more, to 36 percent, in the current quarter.
Executives repeatedly played down the importance in a conference call with analysts by saying that
they were happy with where they were.

The iPod was the only product in Apple's lineup that showed a decline, with 9.1 million sold, down
11 percent.

Steven P.~Jobs, Apple's chief executive, made a rare appearance on the regular quarterly conference
call. He said he made an exception in honor of the company's record quarter.

But he used the occasion to attack Apple's rivals, including Google, whose Android software powers a
growing number of mobile phones. Sales of mobile phones using the Android operating system surpassed
those for the iPhone in the United States in the third quarter.

Just as a variety of Windows-based PCs held Apple computers to a tiny market share for decades, a
variety of Android phones may one day make iPhones a bit player, some analysts have said. Mr.~Jobs
waved off suggestions that Google's strategy would win, casting its approach as fragmented because
of the variations of Android that he said are used by different phone carriers.

``With iPhone, every handset works the same,'' Mr.~Jobs said. He added, ``We think Android is very,
very fragmented and getting more fragmented by the day.''

Mr.~Jobs also criticized several companies that planned to challenge the iPad. He said that their
plans to sell a device with a seven-inch screen would create a poor customer experience compared
with the iPad, which has a 10-inch screen. The touch screen will be so small, Mr.~Jobs said, that
users will be unable to easily use their fingers to reach apps on the device.

Peter Oppenheimer, Apple's chief financial officer, said that a surprising number of businesses and
schools had also been buying Apple's products. IPhones provided the biggest piece of Apple's
revenue, with \$8.8 billion in sales, followed by notebook computers with \$3.2 billion. IPads and
related accessories accounted for \$2.9 billion.

International sales accounted for 57 percent of all revenue in the quarter.

\section{Justices to Hear Appeal Over Liability for Detention}

\lettrine{A}{bdullah}\mycalendar{Oct.'10}{19} al-Kidd, born in Kansas and once a star running back
at the University of Idaho, spent 16 days in federal detention in three states in 2003, sometimes
naked and sometimes shackled hand and foot.

On Monday, the Supreme Court agreed to decide whether he may sue John Ashcroft, the former attorney
general, for what Mr.~Kidd contends was an unconstitutional use of a law meant to hold ``material
witnesses.'' Mr.~Kidd says the law was used as a pretext for detaining him because he was suspected
of terrorist activities.

The material witness law is typically used to hold people who have information about crimes
committed by others when there is reason to think they would not appear at trial to give testimony.
Critics say the Bush administration radically reinterpreted the law after the Sept.~11 attacks,
using it as a tool for preventive detention.

Laws allowing the preventive detention of terrorism suspects are common in Europe. The United States
does not have such a law, but Mr.~Kidd contends that a policy set by Mr.~Ashcroft allowed federal
prosecutors to use the material witness law to the same end.

The United States Court of Appeals for the Ninth Circuit, in San Francisco, last year allowed
Mr.~Kidd's suit to proceed, rejecting Mr.~Ashcroft's claim that he was entitled to prosecutorial
immunity.

``To use a material witness statute pretextually, in order to investigate or pre-emptively detain
suspects without probable cause,'' Judge Milan D.~Smith Jr.~wrote for the majority of the divided
three-judge panel, ``is to violate the Fourth Amendment,'' which bans unreasonable searches and
seizures.

``Some confidently assert,'' Judge Smith continued, ``that the government has the power to arrest
and detain or restrict American citizens for months on end, in sometimes primitive conditions, not
because there is evidence that they have committed a crime, but merely because the government wishes
to investigate them for possible wrongdoing, or to prevent them from having contact with others in
the outside world. We find this to be repugnant to the Constitution, and a painful reminder of some
of the most ignominious chapters of our national history.''

Eight judges dissented from the full Ninth Circuit's decision not to rehear the case. They said
prosecutors' subjective intentions were irrelevant so long as they followed the letter of the
material witness law.

Wesley M.~Oliver, a legal historian at Widener University School of Law, filed a brief urging the
Supreme Court to hear the case. In an interview, he said the Ninth Circuit's reasoning would
``negate an otherwise valid material witness warrant because you think the witness is also a
criminal suspect.''

``You can now hold the innocent grandmother,'' Professor Oliver said, ``but not the guy suspected of
supporting Al Qaeda.''

Mr.~Kidd, who described himself in a 2004 interview as ``anti-bin Laden, anti-Taliban, anti-suicide
bombing, anti-terrorism,'' was never charged with a crime and was never called to testify as a
witness.

The Obama administration had urged the justices to reverse the Ninth Circuit's decision allowing his
suit to proceed. ``If permitted to stand,'' wrote Neal K.~Katyal, the acting solicitor general,
``the decision below would seriously limit the circumstances in which prosecutors could invoke the
material witness statute without fear of personal liability.''

Mr.~Kidd, represented by the American Civil Liberties Union, said the appeals court ruling had been
straightforward and correct. Mr.~Ashcroft's ``deliberate decision to authorize the pretextual arrest
of witnesses was clearly unconstitutional,'' Mr.~Kidd's lawyers told the justices.

Mr.~Kidd, who was known as Lavoni T.~Kidd when he led the Vandals of the University of Idaho in
rushing in 1995, was on his way to Saudi Arabia to work on his doctorate in Islamic studies in March
2003 when he was arrested and handcuffed at Dulles International Airport outside Washington.

In the 2004 interview, Mr.~Kidd said he did not understand why someone held as a mere witness should
be subjected to harsh treatment.

``I was made to sit in a small cell for hours and hours and hours buck naked,'' he said. ``I was
treated worse than murderers.''

Justice Elena Kagan disqualified herself from the case, Ashcroft v. al-Kidd, No.~10-98, presumably
because she had worked on it as United States solicitor general.

\section{From Obama, the Tax Cut Nobody Heard Of}

\lettrine{W}{hat}\mycalendar{Oct.'10}{19} if a president cut Americans' income taxes by \$116
billion and nobody noticed?

It is not a rhetorical question. At Pig Pickin' and Politickin', a barbecue-fed rally organized here
last week by a Republican women's club, a half-dozen guests were asked by a reporter what had
happened to their taxes since President Obama took office.

``Federal and state have both gone up,'' said Bob Paratore, 59, from nearby Charlotte, echoing the
comments of others.

After further prodding -- including a reminder that a provision of the stimulus bill had cut taxes
for 95 percent of working families by changing withholding rates -- Mr.~Paratore's memory was
jogged.

``You're right, you're right,'' he said. ``I'll be honest with you: it was so subtle that
personally, I didn't notice it.''

Few people apparently did.

In a troubling sign for Democrats as they head into the midterm elections, their signature tax cut
of the past two years, which decreased income taxes by up to \$400 a year for individuals and \$800
for married couples, has gone largely unnoticed.

In a New York Times/CBS News Poll last month, fewer than one in 10 respondents knew that the Obama
administration had lowered taxes for most Americans. Half of those polled said they thought that
their taxes had stayed the same, a third thought that their taxes had gone up, and about a tenth
said they did not know. As Thom Tillis, a Republican state representative, put it as the dinner
wound down here, ``This was the tax cut that fell in the woods -- nobody heard it.''

Actually, the tax cut was, by design, hard to notice. Faced with evidence that people were more
likely to save than spend the tax rebate checks they received during the Bush administration, the
Obama administration decided to take a different tack: it arranged for less tax money to be withheld
from people's paychecks.

They reasoned that people would be more likely to spend a small, recurring extra bit of money that
they might not even notice, and that the quicker the money was spent, the faster it would cycle
through the economy.

Economists are still measuring how stimulative the tax cut was. But the hard-to-notice part has
succeeded wildly. In a recent interview, President Obama said that structuring the tax cuts so that
a little more money showed up regularly in people's paychecks ``was the right thing to do
economically, but politically it meant that nobody knew that they were getting a tax cut.''

``And in fact what ended up happening was six months into it, or nine months into it,'' the
president said, ``people had thought we had raised their taxes instead of cutting their taxes.''

There are plenty of explanations as to why many taxpayers did not feel richer when the cuts kicked
in, giving typical families an extra \$65 a month. Some people were making less money to begin with,
as businesses cut back. Others saw their take-home pay shrink as the amounts deducted for health
insurance rose.

And taxpayers in more than 30 states saw their state taxes rise, according to the Center on Budget
and Policy Priorities.

That is what happened here in North Carolina. The Treasury Department estimated that the federal tax
cut would put \$1.7 billion back in the hands of North Carolina taxpayers this year. Last year,
though, North Carolina, facing a large budget shortfall, raised a variety of state taxes by roughly
a billion dollars.

``It was a wash,'' said Mr.~Tillis, the state representative.

The guests at the Pig Pickin' rally here could rattle off the names of the House speaker and the
Senate majority leader with ease, if with disdain, and were up on many of the political
controversies of the day. They studied the campaign fliers at their tables, and pocketed the
1.5-ounce jars of strawberry preserves with special labels urging them to vote for Judge Bill
Constangy for Superior Court (``Preserving Justice,'' the labels read).

Many volunteered that they thought the Bush tax cuts should be extended for all taxpayers, even for
the wealthy ones whom Mr.~Obama would like to exclude. But few had heard that there had also been
Obama tax cuts -- which will also expire next year unless extended, but have generated far less
public debate.

Bob Deaton, 73, who wore a ``Fair Tax'' baseball cap, was surprised to hear that there were tax cuts
in the \$787 billion stimulus bill, which was wildly unpopular with many at the rally even though
roughly a third of it was in the form of tax cuts.

``Tax cuts?'' he asked. ``Where were the tax cuts?''

Ron Julian, 50, a Huntersville town commissioner, said he thought his taxes had gone up under
Mr.~Obama. And Mr.~Paratore, a former Hearst executive, said he might have noticed the tax cuts if
his paycheck had jumped more in the weeks before he retired last year: ``I couldn't even tell you
what it was, to be honest with you.''

The Obama administration wants to extend the little-noticed tax cut next year. Jason Furman, the
deputy director of the National Economic Council, said the administration still believes that
changing the withholdings was a more effective form of stimulus than sending out rebate checks would
have been.

``In retrospect, we think that judgment was right,'' he said. ``It's harder to predict what's good
for politics. Ultimately, the best thing for politics is going to be helping the economy.''

But at least one prominent economist is questioning whether the method really was more effective.
Joel B.~Slemrod, a professor of economics at the University of Michigan, analyzed consumer surveys
after the last rebate checks were sent out in 2008 by the Bush administration, and after this tax
cut, called Making Work Pay, went into effect under the Obama administration.

After the 2008 rebates, he found that about a quarter of the households surveyed said they would use
the money primarily to increase their spending. After the Obama tax cut took effect, he said, only
13 percent said they would use the money primarily to increase their spending. The Obama
administration believes that people did spend the money, and cites analyses calling the cut one of
the more effective forms of stimulus.

Mr.~Slemrod said it was not unheard of for voters to miss tax cuts. Just a few years after a 1986
overhaul of the tax system made significant cuts to most people's taxes, he said, a survey asked
people what had happened to their taxes. ``Most people didn't answer that they went down,'' he said.

\section{Democrats Are at Odds on Relevance of Keynes}

\lettrine{A}{}\mycalendar{Oct.'10}{19} rift has emerged within the Democratic Party between liberal
economists, who generally view the 2009 stimulus package as a success and say that Keynesian
economics should remain the heart of the party's economic policy, and elected officials, who in
growing numbers have shunned affiliation with the \$787 billion effort and are expressing doubts
about the effectiveness of fiscal intervention.

For decades, Keynesian policies, which call for government spending to make up for the shortfall in
private-sector demand during an economic downturn, have been a central element of the Democratic
tool kit and a principle of the party's identity. But the unpopularity of the stimulus package
signed into law by President Obama has left many Democrats in competitive races distancing
themselves from such programs, raising questions about whether the party is beginning a more
fundamental rethinking of its approach to the economy.

The implications could extend well beyond this election cycle. To the extent that faith in deficit
spending during downturns is eroded, the Federal Reserve could face increasing pressure to deploy
monetary policy to lift the economy out of a rut -- a prospect that has unsettled officials at the
central bank. A shift among Democrats could increase pressure in Congress to rein in the growth of
government spending, slash the budget deficit and reduce the national debt -- even during a period
of weakness that traditional Keynesian theory would say requires more deficit spending.

Ambivalence about using government borrowing and spending to spur the economy is longstanding.
During the Depression, President Franklin D.~Roosevelt wavered repeatedly over the size of the New
Deal; budgetary retrenchment helped set off a second deep recession in 1937-38.

``Not until World War II, with the need for revenue so large and the unity around winning the war so
strong, was that ambivalence pushed aside,'' said Gary Gerstle, a historian at Vanderbilt
University.

A Keynesian consensus held until the 1970s -- even President Richard M.~Nixon accepted the label --
but began to be discredited during the ``stagflation'' of the latter part of that decade. The
apparent revival of Keynesian doctrine during the 2007-9 recession now appears to have been
short-lived.

``Obama had a brief opening, he probably thought it was more of an opening than it was,''
Mr.~Gerstle said. ``For him to have been able to build upon it, he would have needed more results:
not only lower unemployment but also a greater manifestation of the stimulus's impact.''

And yet, a spate of recent research from the Congressional Budget Office, Wall Street banks and
independent economists has documented that the stimulus, while imperfect, helped avert greater job
losses and a greater drop in economic output.

Moreover, top Democratic economists favor more stimulus. ``The stimulus performed more or less as
predicted,'' said Laura D'Andrea Tyson, who led the Council of Economic Advisers under President
Bill Clinton. ``The problem is the economy is worse than predicted.''

At a recent forum at the International Monetary Fund, Joseph E.~Stiglitz, who later led the same
council under Mr.~Clinton, said: ``The question is: Do we face up to the reality and do what we
clearly need to do, which is have a second round of stimulus?''

Christina D.~Romer, the chairwoman of the council until last month, favors additional action, too.
She had advocated an even larger package -- on the order of \$1.2 trillion -- than what was passed
in February 2009.

``If we take sensible actions that involve more fiscal, more monetary stimulus, we can start growing
again, we can bring that unemployment rate down to much more normal levels,'' Ms.~Romer said in the
same forum as Mr.~Stiglitz. ``If we don't, then I think we're in for a long, hard slog.''

It is difficult to find Congressional Democrats expressing such views.

Representative Chris Van Hollen of Maryland, who as chairman of the Democratic Congressional
Campaign Committee oversees efforts to help scores of House members, acknowledged that ``there are
not a lot that are arguing for more stimulus,'' but added, ``There are a lot that argue that the
recovery bill helped stabilize an economy that was in free fall.''

Mr.~Van Hollen pointed to several lawmakers who have created advertisements around stimulus
provisions, like Representatives Tom Perriello of Virginia, Steve Driehaus of Ohio and James
L.~Oberstar of Minnesota. But an ad by Mr.~Perriello, a vulnerable freshman, highlights his support
for expanding broadband, clearing park trails, preserving police officers' jobs, weatherizing homes
and promoting alternative energy -- all without mentioning the stimulus package of which those
efforts were a part. His Republican opponent has denounced the stimulus.

The design of the stimulus -- which included tax cuts, aid to state and local governments and
investments in public works -- may have contributed to the perception that it was ineffective. For
example, while Roosevelt's investments on infrastructure were highly visible, under the Obama plan,
money on public works has largely gone to mundane projects like road repair.

The administration's latest proposals -- letting businesses deduct from their taxes the full value
of new equipment purchases, making permanent a tax credit for corporate research and development,
and creating an infrastructure bank for long-term projects -- are far more modest.

``The president has this year been proposing historically bipartisan policies that would help stand
up the private sector and accelerate our recovery,'' said Austan D.~Goolsbee, who succeeded
Ms.~Romer as chairman of the council. ``I hope that at some point opposition, for the sake of
opposition, is going to lessen.''

But that seems unlikely, as long as the recovery plods along slowly. ``It would be a mistake to
attribute the distancing from Obama's stimulus entirely to political caution or opportunism,'' said
Robert S.~Weisbrot, a historian at Colby College. ``As much as those factors may be important, it is
dismaying how little evidence there is to show for it. Maybe we need even more, but surely \$800
billion should have counted for something.''

\section{From Taft to Obama, Victrola to DVD: Secrets of the Centenarians}

\lettrine{C}{entenarians}\mycalendar{Oct.'10}{19} alive today are older than the Titanic, crossword
puzzles and Mickey Mouse. They have lived through 19 presidencies and two world wars. They saw the
rise of motion pictures and radio, followed by the invention of television.

Some traveled here long ago as immigrants; others have spent their entire lives in one small town.
They have celebrated golden anniversaries and beyond, and witnessed the birth of great-grandchildren
and great-great-grandchildren.

Many of them are still happy to be driving, dancing, walking or simply breathing. But others are
clinging to their remaining abilities and wondering if they are too much of a burden to their loved
ones.

Not surprisingly, the question they hear most often is ``What is the secret to a long life?''

ESTHER TUTTLE, 99 ``I think the secret of a long life is partly genes, but I also think it's being
conscious of your body. Your body is your instrument. So I always did exercises, did a lot of yoga,
stretching exercises and walking. Eat in moderation and drink in moderation. Because everybody
smoked, I kind of had to smoke in self-defense. But moderation is a wonderful thing.''

HAZEL MILLER, 100 ``There's no secret about it, really. You just don't die, and you get to be 100.

``I have always liked to dance. But as you know, after a certain age, there are no men to dance
with. So I started line dancing, and I belong to a group called the Silver Belles, and we made
costumes, and we danced and did public service engagements.

``I don't need help with anything. I just pass the time reading, painting, going. I have a lot of
friends I visit with. I eat out two or three times a week. So the time passes. In fact, I flit
around doing a lot of things when I should be painting.

``The best part of being 100 is you live to be 100. If you can enjoy it, that is an extra good
thing.''

PHIL DAMSKY, 100 ``We go out to lunch every once in a while. So we went to a place called Hooters,
and there were eight or nine of us. Somebody told the manager that I was 100 years old. So the
manager said that they'd pay for the meal -- we didn't have to pay for the meals. Then we took
pictures with the waitresses. I thought it was very nice of them.

``My appetite was always good. You know, food is food. I like a corned beef sandwich, corned beef or
pastrami. But then I have to watch myself because some of these things have too much salt.

``I thought I was going to live forever, but there's no such thing. But enjoy every minute that
you're living. I think that's some good advice.''

ROSE KATZ, 103 ``I don't know what a cold is. I was always a healthy person -- a healthy, strong
person. I suppose that's what keeps me alive.''

\section{Chinese Promotion Puts Official on Track for Presidency}

\lettrine{X}{i}\mycalendar{Oct.'10}{19} Jinping, China's vice president, was named to an important
military position on Monday, continuing his elevation to the top echelons of China's leadership and
reconfirming that the Communist Party had selected him as the successor to President Hu Jintao.

Mr.~Xi, a provincial governor who emerged as the heir apparent in 2007 when he received a senior
rank on the Politburo Standing Committee, was named vice chairman of the Central Military
Commission, which oversees the People's Liberation Army and its branches. The post fills the last
remaining gap in Mr.~Xi's r\'esum\'e and means that he is following the succession track that Mr.~Hu
took a decade ago on his way to assuming China's top party, state and military titles.

Barring a major upset, Mr.~Xi, 57, is now on track to become Communist Party secretary when Mr.~Hu's
term ends in 2012, and president in 2013.

China's state news media announced Mr.~Xi's promotion at the conclusion of a Communist Party
planning meeting, in which about 370 Central Committee members and alternates approved a five-year
plan for governing China though 2015.

The media did not say that the promotion was part of a prearranged succession ritual for Mr.~Xi. The
internal promotion process is considered highly secret, though analysts say the leadership goes to
great lengths to avoid uncertainty and seeks to anoint successors long in advance to minimize
political infighting in the one-party state.

The planning session was watched closely for indications that members would consider a recent spate
of demands for greater individual freedoms. Instead, the party report focused on exploiting
``strategic opportunities'' to sustain the rapid economic growth that has powered China's global
ascendance.

``China aims to achieve major breakthroughs in economic restructuring and maintain stable and
relatively fast economic growth,'' said a communiqu\'e from the Communist Party's Central Committee,
according to the state-controlled Xinhua news agency.

In the vaguely worded document, the committee promised to undertake a broad transformation of the
economy in the next five years, shifting its emphasis to domestic consumption and building
high-technology industries into a driving force behind economic growth.

The communiqu\'e also committed to ``vigorous yet steady efforts'' to restructure the political
system. But one Beijing analyst of China's leadership, Russell Leigh Moses, said the document gave
scant indication that those efforts would move beyond changes in the party's internal bureaucracy
that had been the hallmark of earlier political reforms.

``They've emphasized economic reform and sidelined political reform,'' Mr.~Moses said, ``yet
again.''

The rise of Mr.~Xi has been smoothed by his connections; he is the son of Xi Zhongxun, a onetime
revolutionary guerrilla and deputy prime minister who helped Mr.~Hu rise through the ranks and
shepherded the spectacular success of Shenzhen, China's first free-market economic zone, 30 years
ago.

The younger Mr.~Xi has risen steadily up the ranks, serving as party leader in the progressive
Zhejiang Province and in Shanghai after the party leader there was dismissed for corruption.

But like Mr.~Hu before he became China's top leader, Mr.~Xi rarely speaks publicly and has not
developed a reputation that transcends his role as a senior Communist Party official.

Senior appointments in China are made by consensus among the political elite, with the elder
generation of leaders accorded a prominent say in selecting junior officials they expect to accede
to the top posts a decade or more down the road. Elders tend to favor officials in their mid- to
late 50s who have a lifetime of obedient service. Blandness -- at least to the general public -- is
not an impediment to promotion.

If tradition holds, Mr.~Xi will have less power than his titles suggest. Unlike Mao and Deng
Xiaoping, who wrested control and maintained their grip on power by excelling at political
infighting, China in recent years has managed to maintain at least the veneer of ``collective
leadership,'' in which major decisions are made only after a cross section of the elite approves
them.

While Mr.~Xi is largely unknown, his wife, Peng Liyuan, was once a celebrity folk singer. She no
longer performs and has kept a low profile since Mr.~Xi was promoted into the senior leadership.

Why Mr.~Xi emerged from relative obscurity to become heir apparent three years ago is a matter of
mystery and speculation. Some close observers of Chinese politics say the man considered to be
Mr.~Hu's personal top choice as successor, Li Keqiang, another provincial leader, failed to win the
backing of some members of China's old guard, headed by the former top leader Jiang Zemin. Mr.~Xi
was a compromise acceptable to both Mr.~Jiang and Mr.~Hu, these observers say.

Mr.~Li, who sits on the Politburo Standing Committee along with Mr.~Xi, is now expected to succeed
Prime Minister Wen Jiabao when he retires along with Mr.~Hu in 2012.

How Mr.~Xi might govern is a question mark. He is thought to favor new steps to promote
market-oriented economic reforms because he ran Zhejiang, a province known for its relatively
free-wheeling style of small-business capitalism. But his political leanings are less clear. In
2009, he was put in charge of an internal Communist Party office that has promoted a clampdown on
liberal intellectuals, the Internet and nongovernmental organizations, among others.

The annual plenum report is always sifted for evidence of subtle changes in Chinese policies and
political hierarchies, often with mixed success. Monday's document, a summary of a full report yet
to come, combined few specifics with a reiteration of previous economic goals that, on occasion,
have fallen by the wayside.

Moving the economy toward greater domestic consumption was at the heart of much of the report. While
China's booming export economy has generated trillions in foreign exchange surpluses, the share of
gross domestic product generated by domestic spending has fallen steadily to nearly 35 percent, far
lower than that of most developing nations.

Western nations, including the United States, have urged China to develop its domestic market both
to ease the global economic strains caused by its aggressive export policies and to increase
domestic demand for imports. Chinese government reports have noted that advice for years, but with
few concrete results.

For the first time, Monday's document explicitly pledges to reverse that, in part by increasing
welfare spending, holding down the prices of staples like housing, and raising the salaries of
common workers. The document also proposes to devote more money to modernize agriculture, introduce
advanced manufacturing techniques and build the nation's shrunken services industry, including
bolstering culture, entertainment, travel and sports to generate more jobs.

The new document labels culture a ``pillar'' industry, according to an analysis by Deutsche Bank's
Hong Kong office, underscoring the government's intensive effort to promote China's ``soft power''
as a leader in social and cultural ventures.

Perhaps related to the report, the government's State Council, a body akin to the White House
Cabinet, issued a document on Monday that decreed that new strategic industries -- including
biotechnology, advanced materials, energy-saving cars and technologies, and environmental protection
-- would account for nearly one in eight dollars of gross domestic product by 2015.

\section{China Escalates Fight With U.S.~on Energy Aid}

\lettrine{A}{}\mycalendar{Oct.'10}{19} dispute between China and the United States over Beijing's
subsidies to clean energy industries escalated on Sunday when a senior Chinese economic official
warned that Washington ``cannot win this trade fight.''

In an abruptly scheduled news briefing here, the official, Zhang Guobao, sharply rebuked the Obama
administration for opening an inquiry on Friday into the subsidies.

Mr.~Zhang accused American trade officials of repeatedly delaying talks over the same issues that
the White House now wanted to investigate and suggested the administration was playing election
season politics.

``I have been thinking: What do the Americans want?'' said Mr.~Zhang, the vice chairman of the
government's National Development and Reform Commission. ``Do they want fair trade? Or an earnest
dialogue? Or transparent information? I don't think they want any of this. I think more likely, the
Americans just want votes.''

With campaigning for the November midterm elections defined in large part by bleak economic and
employment statistics, candidates in both parties have increasingly blamed Chinese trade policies
for slowing the American recovery from the 2008 economic collapse.

The White House has increased criticism of China in recent weeks, even as it dispatched senior
officials to China to try to defuse trade tensions.

Mr.~Zhang's comments signaled that tensions were rising not only over trade issues but also over
longstanding claims that the Chinese have artificially depressed the value of their currency.

He is widely seen in the West as China's top policy maker on energy because he heads the country's
National Energy Administration.

It is highly unusual for an official of Mr.~Zhang's standing to convene a news conference on short
notice and denounce the American government in such blunt terms.

Should the Americans pursue the subsidy issue with the World Trade Organization, Mr.~Zhang said,
``the only ones who will be humiliated are themselves.''

Mr.~Zhang was reacting to an announcement on Friday that the United States Trade Representative's
office would investigate Chinese government support for manufacturers of wind turbines, solar energy
products, energy-efficient vehicles and technologically advanced batteries.

``The United States is committed to the international trade rules in the W.T.O. and our other trade
agreements,'' Nefeterius Akeli McPherson, a spokeswoman for the trade representative, said on
Sunday. ``We strive to ensure that our policies and programs are consistent with those rules, and
ask that other countries also do so.''

The United Steelworkers trade union filed a complaint on Sept.~9 accusing China of ignoring W.T.O.
rules prohibiting excessive subsidies of those markets. The trade representative's inquiry, in
reaction to the Steelworkers' complaint, is potentially a first step toward filing formal charges
against China with the W.T.O.

Mr.~Zhang called the Steelworkers' complaint unfounded, saying the Obama administration had proposed
subsidies totaling \$60 billion for clean energy industries, adding that the American government had
placed domestic-content provisions -- so-called Buy American clauses -- on certain clean energy
products.

In contrast, he said, ``as far as I know, Chinese subsidies coming from all sectors, including
taxes, are nominal,'' and China is a major buyer of American clean energy products.

For example, he said that while Chinese exports of wind turbines to the United States were small in
2009, Chinese manufacturers had imported crucial turbine components and ingredients like lubricants
from American manufacturers.

``What America is blaming us for is exactly what they do themselves,'' Mr.~Zhang said. ``Chinese
subsidies to new energy companies are much smaller than those of the U.S.~government. If the
U.S.~government can subsidize companies, then why can't we?''

The Steelworkers' complaint, however, echoes charges by other foreign manufacturers, who argue that
China gives hidden subsidies like free land and low-interest loans to its clean energy industries,
while substantially restricting access to its domestic market.

They also accuse China of depressing its currency so exports of clean energy products will sell
cheaply abroad, while foreign imports will seem artificially expensive. China said in the summer
that it would allow its currency, the renminbi, to appreciate. But its value has risen at a glacial
pace, and only as a result of intense pressure from its trading partners.

The economic stimulus legislation in the United States included a so-called Buy American provision
requiring that recipients of stimulus money purchase only steel and other construction materials,
like solar panels, that were manufactured in the United States or in other countries that had signed
the trade organization's side agreement requiring free trade in government procurement.

Virtually all industrialized countries have signed the government procurement agreement, which calls
for signatories to open most government contracts to bids from other signatory countries. China
promised to sign the side agreement ``as soon as possible'' when it joined the W.T.O. in 2001.

But municipal and provincial governments in China, particularly in inland provinces, have strongly
opposed opening their procurement projects to international competition. So China has not yet signed
the agreement, although some Chinese executives have begun advocating that it do so in the interest
of helping Chinese companies export more goods.

While the United States has many programs to subsidize research and development of clean energy,
China has been very successful in the last year at manufacturing and exporting clean energy
technologies inexpensively, particularly in solar panels.

World Trade Organization rules tend to be stricter against export subsidies than against domestic
subsidies, and they tend to be stricter against manufacturing subsidies than research and
development subsidies.

\section{Chinese Christians Barred From Conference}

\lettrine{M}{ore}\mycalendar{Oct.'10}{19} than 100 Chinese Christians seeking to attend an
international evangelical conference in South Africa have been barred from leaving the country, some
in the group said, because their churches are not sanctioned by the state.

Organizers say that more than 4,000 Christians from around the world will discuss faith, poverty,
the AIDS epidemic and other issues at the nine-day conference, which begins Saturday in Cape Town.
But members of the Chinese delegation said that they could get no farther than the passport control
at international airports in China before officials confiscated their documents.

``They said it is illegal to attend this conference, and they sent me home,'' said Liu Guan, 36, a
Protestant evangelical leader who tried to fly out of Capital International Airport in Beijing last
Sunday. ``The explanation was `for your own good.' ''

China's policy toward Christians is more relaxed now than a decade ago. Although only
government-sanctioned churches are considered legal, millions of Chinese -- some say tens of
millions -- worship in unregistered house churches.

While believers often complain of harassment, officials in much of China turn a blind eye to the
activities there. But Chinese house churches are one matter; global conferences are another.

The Chinese authorities said that the government intervened to prevent people from attending the
conference because Cape Town organizers failed to honor China's policy of domestic control over
religious activities. In a statement on Friday, Ma Zhaoxu, spokesman for the Ministry of Foreign
Affairs, said that instead of inviting the legal representatives of China's Christians, the
organizers ``secretly extended multiple invitations to Christians who privately set up meeting
points.''

``This action publicly challenges the principle of independent, autonomous, domestically organized
religious associations, and therefore represents a rude interference in Chinese religious affairs,''
his statement said.

Officials of the conference, the Third Lausanne Congress on World Evangelization, have protested. In
a statement, Doug Birdsall, the executive chairman, said that China's official Christian
representatives had been invited but had declined to attend. The Three-Self Patriotic Movement --
China's state-sanctioned Protestant Church -- was also involved in the process of selecting
participants, he said.

In an open letter released Friday, the Chinese delegation said China was home to tens of millions of
Christians, most of whom worshiped in unregistered churches. Pastors and elders were eager for the
chance to discuss the growth of Christianity in China and to build ties with religious leaders from
other countries at the Cape Town conference, the letter said.

The conference is the third worldwide gathering since a committee led by the evangelist Billy Graham
drew 2,700 religious leaders to Lausanne, Switzerland, in 1974. Organizers say most of the speakers
are from Africa, South America and Asia because that is where two-thirds of evangelicals live today.

Beginning in July, Chinese officials began individually contacting every Chinese citizen who had
been invited and pressuring them not to attend, church leaders said. Some had to give up their
passports, some suffered government reprisals against their churches and some were detained, the
letter said. Most were turned away at airport passport control checkpoints, according to the letter.
``This series of blocking actions violated their right of religious freedom'' spelled out in the
Chinese Constitution, it stated.

There is precedence for the government's interference. In accordance with China's policy against
foreign oversight of religion, the Chinese Patriotic Catholic Association, which officially
represents Chinese Catholics, does not recognize the authority of the pope. Ignoring that, Pope
Benedict XVI invited four Chinese bishops to attend a church conference in Rome in 2005. Government
authorities rejected the invitation.

Mr.~Liu, the Beijing evangelical leader, said a half-dozen police officers and government officials
met him and four other Christians at the Beijing airport about an hour before their Sunday flight
was scheduled to board. He said that his passport was confiscated and that he was ordered not to
speak to the foreign media. A 25-year-old Beijing education worker, who asked to be identified only
by his English first name of David in order not to call attention to his church, was sent home along
with Mr.~Liu. He said he later demanded a written explanation of why his passport was seized. The
letter he received was brief, he said. It stated that he had volunteered to give his passport to the
police.

\section{Chinese Dissidents Sign Letter Supporting Nobel Winner}

\lettrine{C}{hina}\mycalendar{Oct.'10}{19}'s spate of public agitation for political reform
continued Friday, as more than 100 Chinese intellectuals and dissidents signed and posted an online
letter asking that the Nobel Peace Prize winner Liu Xiaobo be released from prison and that
government security officers stop harassing his wife, Liu Xia. In line with Charter 08, the
pro-democracy manifesto that Mr.~Liu co-authored and which led to his imprisonment, signers of the
new letter also asked that government leaders ``make good on their oft-repeated promise to reform
the political system.''

The letter added, ``This will require it to guarantee the rights of Chinese citizens as they work to
bring about peaceful transition toward a society that will be, in fact and not just in name, a
democracy and a nation of laws.'' The letter follows another public manifesto, posted online Monday
and quickly censored, which attacked the Communist party's censorship and demanded an end to all
curbs on press freedom. Both letters refer to other statements in recent months by Premier Wen
Jiabao which appear to urge the Communist Party to give its citizens more basic freedoms and to
restructure the political system.

Meanwhile, a leading dissident and supporter or Mr.~Liu, Ding Zilin, was reported to have vanished,
last heard from in the Yangtze River delta town of Wuxi on Oct.~8, days after Mr.~Liu's prize was
announced. Ms.~Ding leads the group Tiananmen Mothers, which has fought to force the government to
recognize the hundreds of demonstrators killed by troops during the Tiananmen Square protests in
1989. Miu Xia alerted outsiders to her disappearance late Thursday, urging the public to pay
attention to her case. Local police officers were reported to have promised that they would look
into the disappearance.

\section{Once Banned but Now Pampered, Dogs Reflect China's Rise}

\lettrine{X}{iangzi}\mycalendar{Oct.'10}{25} -- Lucky, in English -- is aptly named. A trim Siberian
husky, his owner, a sports marketer named Qiu Hong, pampers him with two daily walks, a brace of
imported American toys and grooming tools, \$300 worth of monthly food and treats and his own sofa
in her high-rise apartment.

When city life becomes too blas\'e, Ms.~Qiu loads Xiangzi in the car and takes him out for a run --
on the trackless steppes of Inner Mongolia, seven hours north.

``It's a huge grassland. Very far, but very pretty,'' she said. ``He really likes to scare the sheep
and make them run all over the place.''

Metaphorically speaking, Xiangzi is not just a dog, but a social phenomenon -- and, perhaps, a
marker of how quickly this nation is hurtling through its transformation from impoverished peasant
to first-world citizen.

Twenty years ago, there were hardly any dogs in Beijing, and the few that were here stood a chance
of landing on a dinner plate. It remains possible even today to find dog-meat dishes here. But it is
far easier to find dog-treat stores, dog Web sites, dog social networks, dog swimming pools -- even,
for a time recently, a bring-your-dog cinema and a bring-your-dog bar on Beijing's downtown
nightclub row.

All that and, Beijing officials say, 900,000 dogs as well, their numbers growing 10 percent a year.
And those are the registered ones. Countless thousands of others are unlicensed.

How this came to be is, in some ways, the story of modern China as well. Centuries ago, China's
elite kept dogs as pets; the Pekingese is said to date to the 700s, when Chinese emperors made it
the palace dog -- and executed anyone who stole one.

But in the Communist era, dogs were more likely to be guards, herders or meals than companions. Both
ideological dogma and necessity during China's many lean years rendered pets a bourgeois luxury.
Indeed, after dogs first began to appear in Beijing households, the government decreed in 1983 that
they and seven other animals, including pigs and ducks, were banned from the city.

China's economic renaissance changed all that, at least in the prosperous cities. ``People used to
be focused on improving their own lives, and they weren't really acquainted with raising dogs,''
Ms.~Qiu said. ``But with the improvement in the economy, people's outlooks have changed. There's a
lot of stress in people's lives, and having a dog is a way to relieve it.''

But there are other factors in dogs' newfound popularity: Many owners also say China's one-child
policy has fanned enthusiasm for dog ownership as a way to provide companionship to only children in
young households and to fill empty nests in homes whose children have grown up.

Some say dogs have become a status symbol for upwardly mobile Beijingers. He Yan, 25, owner of two
small mixed breeds named Guoguo and Tangtang, said young Beijingers like her are dubbed gouyou, or
``dog friends.'' Dogs, she said, have become a way to display one's tastes and, not least of all, a
way to meet people with similar interests.

And for a certain class with more money than sense, owning an especially prized breed has become the
Chinese equivalent of driving a Lamborghini to the local supermarket. The pinnacle of pretension
appears to be the Tibetan mastiff, a huge and reportedly fierce breed from the Himalayan plateau
that, lore says, was organized by Genghis Khan into a 30,000-dog K-9 corps.

One woman from Xi'an, a city west of Beijing, was widely reported last year to have paid four
million renminbi -- roughly \$600,000 -- for a single dog that was escorted to its new home in a
30-Mercedes motorcade.

Mostly, though, it appears that Beijing dogs have, as in the West, become objects of affection --
even devotion -- by their owners. On a given weekend, hundreds of dog owners flock to Pet Park, a
29-acre canine spa east of Beijing that includes a dog-and-owner restaurant, a dog show ring, a dog
agility course, a dog cemetery and chapel, a dog-owner motel, an immaculate 600-bay kennel --
visitors must step in a disinfectant vat before entering -- and two bone-shaped swimming pools.

Those who board their dogs are guaranteed an hour's daily dog play, a weekly bath and a Web site
where, every Monday, they can see fresh snapshots of their pet. The park, which opened last year, is
the brainchild of a Beijing dog lover who amassed a fortune in the refrigerator business, according
to Li Zixiao, the park's sales manager.

``Everyone who brings his dog here considers his dog as a child,'' he said.

To be sure, not all Beijingers are so inclined. A Beijing Internet blog, City Dog Forbidden,
moderates a spirited debate between dog lovers and those who believe, as one wrote, that dogs ``are
seriously disturbing the normal lives of other people.''

``The birth of humans needs to be planned, but anyone can raise a dog?'' asked one incredulous post.
``The resources that you conserve from having less people, you give to dogs? This is a very serious
problem. Are you saying that people are worth less than dogs?''

Yet the doglike devotion of pet owners here seems to have softened even the hardened city government
heart. In 1994, Beijing officials relaxed their no-dog policy to ``severely restrict'' dogs; in
2003, it was changed again to allow anyone to own a dog, but to limit city dogs to no more than 35
centimeters -- a bit less than 14 inches -- in height.

The rule is widely sidestepped by dog lovers who say it is arbitrary and unfair. Daily, thousands of
large-dog owners wait until midnight, when police officers are sparse, to walk the inner-city alleys
with their beloved golden retrievers, Labradors and German shepherds. A July proposal to ease the
restrictions once more, filed with a national legislative advisory body, has drawn nearly 30,000
Internet comments, compared with a few hundred for most other proposals.

The city has even opened its own tiny dog park, with a rudimentary kennel, an agility course and a
kidney-shaped swimming pool that is as mobbed in summertime as any urban American beach.

As for stir-fried Pekingese -- well, that dog, too, may have seen its day. A formal proposal to ban
the eating of dogs has been submitted to China's semi-independent legislature, the National People's
Congress. Nothing the legislature does becomes law without a nod from higher-ups, but the proposal
has survived two rounds of public comment, which bodes well for its future.

The proposal's sponsor, a law professor named Chang Jiwen, says he is not so much a dog lover as a
China lover. ``Other developed countries have animal protection laws,'' he said in a telephone
interview. ``With China developing so quickly, and more and more people keeping pets, more people
should know how to treat animals properly.''

\section{British Kids Log On and Learn Math -- in Punjab}

\lettrine{O}{nce}\mycalendar{Oct.'10}{25} a week, year six pupils at Ashmount Primary School in
North London settle in front of their computers, put on their headsets and get ready for their math
class. A few minutes later, their teachers come online thousands of kilometers away in the Indian
state of Punjab.

Ashmount is one of three state schools in Britain that decided to outsource part of their teaching
to India via the Internet. The service -- the first of its kind in Europe -- is offered by
BrightSpark Education, a London-based company set up last year. BrightSpark employs and trains 100
teachers in India and puts them in touch with pupils in Britain through an interactive online
tutoring program.

The feedback from pupils, the schools and parents is good so far, and BrightSpark said a dozen more
schools, a charity and many more parents were interested in signing up for the lessons. The
one-on-one sessions not only cost about half of what personal tutors in Britain charge but are also
popular with pupils, who enjoy solving equations online, said Rebecca Stacey, an assistant head
teacher at Ashmount.

But the service also faces some opposition from teacher representatives who are fearful that it
could threaten their jobs at a time when the government is pushing through far-reaching spending
cuts. The 3 percent that is to be cut from the budget for educational resources by 2014 might be
small compared with cuts in other areas, like welfare and pensions, but money at schools will remain
tight.

Online learning is still controversial in Britain. Some teachers said tutors based elsewhere lacked
the cultural empathy and understanding of a pupil's social environment that could influence study
habits and performance. There is also concern about the qualifications of teachers abroad.

At the same time, many parents said they had struggled to find qualified private tutors who were
conveniently located and whose fees were affordable. With online learning, they can keep an eye on
their children's progress by listening to the lessons, and many said that being taught by someone in
India also opened the children to foreign cultures.

But Chris Keates, general secretary of NASUWT, Britain's largest teachers' union, said he was
concerned about the precedent BrightSpark was setting. ``This is wrong on so many levels,''
Mr.~Keates said. ``What next -- do without maths teachers? What about the follow-up lessons for the
pupils, and the interaction with teachers?''

Tom Hooper, the founder of BrightSpark, said teachers' unions were missing the point. ``This is
supplementary and in no way replacing teachers,'' he said. And Ms.~Stacey was quick to point out
that Ashmount was using BrightSpark's program in addition to, and not instead of, its traditional
math classes.

``For children, it's a novelty that catches their attention for longer and engages them in a
different way,'' Ms.~Stacey said. ``Eleven-year-olds aren't always enthusiastic about math classes,
so any way we can make it more fun for them is good.''

BrightSpark tutors in India are math graduates or former math teachers and go through a month of
training on the British school curriculum. Pupils in Britain log on to the service via BrightSpark's
Web site and interact with their teachers via a video phone and a so-called white board on their
computer screen, which can be written on by both parties. Lessons can be booked as long as 24 hours
in advance for any day of the week, and all sessions are recorded and can be replayed by the pupil
or the pupil's parents.

For Marie Hanson, who runs the charity Storm in South London, the online teaching tool is helpful in
keeping children away from drugs and crime. ``The kids love it because they love computers,'' said
Ms.~Hanson, ``and I love it because it helps them with their education while keeping them off the
streets.''

An earlier pilot project for four months with 30 children was successful after parents reported that
their children had improved at school, said Ms.~Hanson, who plans to seek government funding for
more sessions.

Mr.~Hooper, 31, said he had discovered there was a market for online teaching in Britain after he
quit his job as investment manager at Aberdeen Asset Management and took time off to travel. In
Panama, he met several U.S.~families who had used online learning to give their children an
education that would allow them to return to U.S.~schools without problems.

When he returned to London, Mr.~Hooper realized that there was a shortage of qualified private
tutors in Britain and that some parents spent hours driving their children to and from tutors,
sometimes paying \textsterling20, or \$31, per lesson. BrightSpark is charging \textsterling12 per
session and pupil. Tutors are being paid \textsterling7 an hour, more than double the minimum wage
in Punjab.

``There is a huge thirst for support in the U.K.,'' Mr.~Hooper said. ``That, combined with a huge
pool of skilled and available academics in India -- it doesn't take a rocket scientist to figure out
the potential.''

Mr.~Hooper is aware that offering teaching services from India in Britain could be controversial and
that there might be concerns about the quality of the teaching, foreign accents and the impersonal
nature of the Internet.

Britain -- like Europe as a whole -- is also less accustomed to outsourcing such services than is
the United States, where similar one-on-one online tutoring from India has existed for the last five
years, offered by companies like India-based TutorVista, in which the British publishing company
Pearson owns a stake. BrightSpark is also unique in selling its product to schools in addition to
single pupils.

Europe's desire to outsource services in general had been lagging behind the United States, said
Martyn Hart, chairman of Britain's National Outsourcing Association. ``There is social resistance
because outsourcing here is always coupled with unemployment,'' said Mr.~Hart.

Mr.~Hooper said he hoped BrightSpark's product would eventually make outsourced services more
popular in Britain and quash concerns among some teachers that it threatens their jobs.

But there is little doubt that online learning increases competition, at least for some in the
education sector. Lola Emetulu, a trained lawyer who now works as an office assistant, said that she
used to drive her 11-year-old son, Jesse, to his private tutor every Saturday but that ``it just
took so much out of your day.'' She recently signed up to BrightSpark and said she preferred the
flexibility.

Jesse said he preferred it, too. ``It’s better on the computer,'' he added. ``The teacher doesn't
know you that much, so he takes it easier on you.''

\section{Japan Calls on China to Resume Rare Earth Exports}

\lettrine{T}{he}\mycalendar{Oct.'10}{25} Japanese trade minister urged China on Sunday to restart
exports of crucial minerals known as rare earths that both traders and government officials say have
been blocked for the past month amid a territorial dispute between the countries.

The trade minister, Akihiro Ohata, also quoted a top Chinese official as acknowledging that customs
officials had stepped up inspections of all rare-earth shipments from China.

Industry officials said last week that China had quietly halted some shipments of the materials to
the United States and Europe, despite denials from Beijing of an official embargo.

Mr.~Ohata said the Chinese vice minister of commerce, Jiang Yaoping, who was visiting Tokyo for an
energy conservation forum, had told him that Chinese customs had strengthened checks of all
rare-earth exports, not just to Japan, as a ``countersmuggling'' measure. But Mr.~Jiang reiterated
that there was no international trade embargo, Mr.~Ohata said.

Mr.~Ohata said he had pressed the Chinese minister to normalize rare-earth shipments. Mr.~Jiang
responded that he would ``make efforts to ensure the situation will not adversely affect the
economies of Japan and China,'' Mr.~Ohata said.

Chinese custom officials have blocked shipments of rare-earth metals to Japan since Sept.~21, after
the Japanese coast guard arrested a Chinese fishing boat captain near disputed islands in the East
China Sea. Japan has since released the captain, but tensions still run high between the countries.

Anti-Japanese rallies have taken place in Chinese cities over the past two weeks, with hundreds of
protesters demanding that Japan drop its claim to the islands. Protesters attacked Japanese shops
and businesses in some cities, prompting Tokyo to demand that China do more to ensure the safety of
Japanese citizens there.

Meanwhile, the halt of shipments to other markets is certain to intensify already rising trade and
currency tensions with Western nations. Industry officials said Chinese customs officials had
imposed the broader restrictions Oct.~18, hours after a top Chinese official summoned the
international news media to denounce U.S.~trade policy.

As of Sunday afternoon, Chinese customs officials were still blocking all exports of raw rare
earths, although there have been no restrictions on the export of value-added rare earth products
like powerful magnets, computer screen components and special glass polishes, industry officials
said.

Though many of the minerals are not particularly rare, most global production of rare earths has
moved to mainland China over the last two decades because of lower costs.

China now mines 95 percent of the world's rare earth elements, which have wide commercial and
military applications and are vital to the manufacture of products like cellphones, motors for
electric vehicles, large wind turbines and guided missiles.

But to the alarm of its trading partners, China imposed increasingly tight export quotas on rare
earths in the past two years, citing growing domestic demand and environmental concerns. In July,
Beijing reduced its export quota for rare earths for the second half of the year by 72 percent.

Exporters had only six weeks' worth of quotas left when Chinese customs officials imposed an
unannounced embargo on shipments to Japan by stopping all shipments for additional inspections, a
practice that has extended in the past six days to shipments elsewhere as well.

Beijing has continued to deny that any embargo exists. Industry executives and analysts say China
could be trying to make it more difficult for other countries to bring a case against Beijing at the
World Trade Organization.

The U.S.~Congress is considering legislation to provide loan guarantees for the re-establishment of
rare-earth mining and manufacturing in the United States. But new mines are likely to take three to
five years to reach full production, according to industry executives.

Meanwhile, resource-poor Japan is racing to help develop rare-earth mines in other countries,
although they could take several years to dig, and to increase efforts to recycle the minerals from
used electronics. Mr.~Ohata said last week that Japan was seeking an agreement with Vietnam for the
joint development of rare-earth materials.

\section{App Makers Take Interest in Android}

\lettrine{T}{here}\mycalendar{Oct.'10}{25} was cold beer, hot pizza and shop talk at a recent
informal gathering of Android programmers in downtown Manhattan. Inevitably the chatter turned to
money.

One software developer, James Englert, 26, had just released his first application for Android,
Google's operating system for cellphones. When asked, he tossed out an estimate for his take from
sales of the app, a simple program that shows train schedules: ``\$1 to \$2 per day.''

The room erupted with laughter. ``That's pretty good money,'' he protested over the clamor.

The others could relate to Mr.~Englert's situation because writing Android software is not yet a
ticket to financial success. Even as Android sales surge -- Google says it is now activating around
200,000 phones a day -- the market for Android apps still seems anemic compared with that for Apple
and its thriving App Store.

Experts and developers say that is in part because the Android Market, the dominant store for
Android apps, has some clunky features that can be annoying to phone owners who are eager to make a
quick purchase. For starters, Android uses Google Checkout rather than an online payment system that
more people are familiar with, like PayPal. As a result, many Android developers make their apps
available free and rely on mobile advertisements to cover the cost.

``It's not the best impulse-buy environment,'' said Matt Hall, co-founder of a developer outfit
called Larva Labs that makes games for Android, iPhone and BlackBerry devices. ``It's hard to think
of an application that you would sit there and put your credit card information in for.''

But that tide is starting to turn as Android's popularity continues to swell and Google takes steps
to smooth out some of the wrinkles. For example, the Android Market recently began showing app
prices in a user's local currency, rather than that of the developer.

``We're still seeing the 1.0 version of the ecosystem,'' said Andy Rubin, vice president of
engineering at Google and one of the primary architects behind Android. ``We think about it every
day, how to minimize some of the friction to help the ecosystem rise with the platform.''

Mr.~Rubin said there were 270,000 developers writing software for Android, and the number of
programs available for download in the Android Market has swelled to more than 100,000, a threefold
increase since March.

Developers can feel the shift in momentum. ``I used to tell people I wrote software for Android, and
they'd look at me like I had three heads,'' said Michael Novak, who handles Android development at
Medialets, a mobile advertising software company, and helps organize the monthly New York Android
Software Developers Meetup. ``That wasn't even a year ago. Now everyone knows what it is. The
popularity has exploded.''

Perhaps the biggest point of friction for Android is the same thing that led to its success.

Because Google makes its software available free to a range of phone manufacturers, there are dozens
of different Android-compatible devices on the market, each with different screen sizes, memory
capacities, processor speeds and graphics capabilities. An app that works beautifully on, say, a
Motorola Droid might suffer from glitches on a phone made by HTC. IPhone developers, meanwhile, need
to worry about only a few devices: iPhones, iPods and iPads.

When Rovio, the Finnish software development company behind the popular iPhone game Angry Birds,
decided to release a version for Android, the company spent months testing the game on a variety of
devices to make sure it was up to par.

``It's so fragmented,'' said Peter Vesterbacka, a developer at the company. ``It's a lot more
challenging than developing for one device, like the iPhone.''

In the end, he said, it was worth the trouble. The game was downloaded more than three million times
in the first week. But the company, which charges 99 cents for the iPhone version and has made
millions of dollars that way, chose to give away the Android version and include ads. This is in
part because paid apps on the Android Market are available in only 32 countries, versus 90 for the
Apple App Store, and Rovio was concerned that people who were not able to purchase the app would
just pirate it.

But developers also say that charging for apps simply may not be the path to profit on Android.

``Google is not associated with things you pay for, and Android is an extension of that,'' said
Mr.~Hall of Larva Labs. ``You don't pay for Google apps, so it bleeds into the expectations for the
third-party apps, too.''

Google says it eventually hopes to introduce a transaction feature for Android software that will
allow purchases within apps, to help developers make more money.

Developers do say that the freedom of Android is a welcome alternative to Apple's tight control.
Android developers have more rein to tinker with the phone's native functions, like the address book
and the basic interface, something Apple has not always allowed. And Apple screens all apps before
they can reach its store, while Google imposes no such restriction, relying on Android users to flag
malicious or offensive apps.

``With Apple, you can spend months writing software only to be denied,'' Mr.~Novak said. ``The
biggest reward as a developer is getting your software out there, and quick. That makes everything
else worthwhile.''

Also unlike Apple, Google does not charge developers to download its software development kit, the
tools needed to write apps.

Developers are not abandoning iPhone for Android. Instead, they say they are slowly starting to
devote more resources to Android in the hope that those efforts will pay off.

They also note that it is a lot easier to stand out in a pool of 100,000 apps versus 300,000, the
current tally for the Apple App Store.

``Apple's App Store is getting overcrowded and saturated,'' said Eric Metois, a freelance tech
consultant who writes apps on the side for the iPhone and Android.

Mr.~Metois's first iPhone app, iChalky, featuring a dancing stick figure, has sold more than 300,000
copies on the iPhone since it was released in December 2008. His second attempt, a game called
Sparticle, was not as successful.

``I poured 500 hours into my second app on the iPhone and sold virtually no copies,'' Mr.~Metois
said. In explaining why he recently released an Android version of iChalky, he said, ``There was a
chance that on another emerging platform, iChalky would have a similar amount of success.''

Analysts say that if Google wants its mobile software to succeed, it will need to make sure that
developers do not lose patience with Android -- particularly in light of new competition, including
the slate of Windows 7 phones from Microsoft and the iPhone's inevitable expansion to other carriers
in the United States besides AT\&T.

Mr.~Rubin said he was not worried about rivals' tempering the momentum of Android because he
believed its future would stretch past the cellphone, to tablets and other devices yet to be
conceived.

``The promise of Android goes beyond one device,'' Mr.~Rubin said. ``We're going to see products
running Android that no one has ever envisioned possible.''

\section{With Kinect, Microsoft Aims for a Game Changer}

\lettrine{T}{im}\mycalendar{Oct.'10}{25} NICHOLS measures fun.

A slim, 32-year-old psychologist, he spends his days behind a one-way mirror at Microsoft's video
games research center here, watching people play the company's Xbox systems. He looks for smiles,
listens for ecstatic squawks and logs triumphant gyrations. When a game is good, it elicits all the
above and gets a ``fun score'' high enough for Microsoft to consider selling it.

And, of late, the fun quotient has skyrocketed.

The company's blend of game developers, interface whizzes and artificial-intelligence experts has
built Kinect, a \$150 add-on for the popular Xbox 360 console that hits stores next month. With its
squat, rectangular shape and three unevenly spaced eyes, this black device looks like a genetically
underserved creature from ``Star Wars.''

In fact, Kinect arrives with a healthy dose of sci-fi trappings. Microsoft has one-upped Sony and
Nintendo by eliminating game controllers and their often nightmarish bounty of buttons. Kinect peers
out into a room, locks onto people and follows their motions. Players activate it with a wave of a
hand, navigate menus with an arm swoosh and then run, jump, swing, duck, lunge, lean and dance to
direct their on-screen avatars in each game.

``I can't tell you how many times I have seen people try and do the moonwalk,'' says Mr.~Nichols, as
he recalls their first, curious encounters with their virtual mimics.

Kinect also understands voice commands. People can bark orders to change games, mute the volume or
fire up offerings, like on-demand movies and real-time chatting during TV shows that flow through
the Xbox Live entertainment service.

The mass-market introduction of Kinect -- with its almost magical gesture and voice-recognition
technology -- stands as Microsoft's most ambitious, risky and innovative move in years. Company
executives hope that Kinect will carry the Xbox beyond gamers to entire families. But on a grander
note, the technology could erase a string of Microsoft's embarrassing failures with mobile phones,
music players, tablets and even Windows from consumers' minds and provide a redemptive beat for the
company.

``For me it is a big, big deal,'' says Steven A.~Ballmer, Microsoft's chief executive. ``There's
nothing like it on the market.''

Where Apple popularized touch-screen technology, Microsoft intends to bombard the consumer market
with its gesture and voice offerings. Kinect technology is intended to start in the living room,
then creep over time throughout the home, office and garage into devices made by Microsoft and
others. People will be able to wave at their computer and tell it to start a videoconference with
Grandma or ask for a specific song on the home stereo.

``I think this is the first thing out of the consumer side of Microsoft in a long, long time where
they are in front of everyone else,'' says Joel Johnson, an editor at large at Gizmodo, the gadget
site. ``I want a Kinect in every room of my house, watching me and listening to what I am saying.
It's so sci-fi and next level that it would be amazing.''

The on-time arrival of amazing has become a rare occurrence at Microsoft, a fact not lost on
investors or Microsoft's directors.

The company continues to rely on its Windows, Office and business software franchises for the bulk
of its \$62.5 billion in annual revenue. In high-growth areas like phones and tablets, Microsoft has
long sold software but has watched Apple come out of nowhere to gobble up the most profits. With
such successes, Apple overtook Microsoft in May as the world's most valuable tech company and has
since swelled that lead to more than \$62 billion. Microsoft's board gave Mr.~Ballmer the fiscal
equivalent of a timeout by docking his bonus over the last fiscal year, pointing to lackluster
mobile technology and a dearth of innovation. And whether Kinect can revamp Microsoft's image as an
innovator remains a big question.

Critics knock Kinect games as too easy and say the gesture technology still has annoying kinks. They
also say Microsoft has had a nasty habit of gumming up its creative engines with bureaucracy.

``They often got lost in fights between all their divisions,'' Mr.~Johnson says. ``Anytime something
becomes high-profile, middle management slows it down.''

But with Xbox, Microsoft has so far done right by consumers and has barreled ahead. It has sold 42
million Xbox 360 consoles and has 25 million people signed up for Xbox Live. In September alone,
people spent a billion hours using Xbox systems.

Microsoft has long salivated over the notion of controlling the living room and becoming a major
entertainment force. Kinect may well stand as its best bet yet for turning that vision into a
reality. ``This is an incredibly amazing, wonderful first step toward making interactivity in the
living room available to everybody,'' says Mr.~Ballmer, while cautioning that Microsoft still has
``a lot of work to do.''

On a Tuesday this month, Allen Walker, 49, and his son Chris, 16, tested a Kinect car-racing game at
the research center. The test room felt clinical with its bare walls, overhead cameras and just a
television for company.

Given no instructions on how to use Kinect, the father and son reached the initial game menu on
their own in a couple of minutes -- and saw their virtual selves staring back. They waved, kicked
their legs and wiggled a bit, and their avatars followed suit. When the game started, the Walkers
tilted left and right to steer, pulled their torsos back to rev up the engine and then thrust
forward to accelerate up ramps and soar through the air.

At the end of each game, photos and videos appeared that documented their comical flailing and
elicited huge smiles from them. (The photos are likely to become prime Facebook fodder come
November.)

Several times, Mr.~Walker nudged Chris out of the way to take control of the system, thus embracing
the uncommon role of game-play adviser to his son.

Making such complex technology so easy to use bordered on the impossible three years ago, when a
small group of Microsoft employees gathered to plot Kinect's future. Plenty of companies have spent
decades refining gesture- and voice-recognition technology. Typically, however, it works best in
controlled environments. Cameras and sensors that perceive movements often need steady, abundant
light, while voice technology tends to hinge on the assumption that a microphone is near a user's
mouth.

Microsoft's engineers knew they wouldn't have the luxury of fixed settings with Kinect. They had to
build a product that could work just as well in a small Japanese living room as in a spacious
Texas-size den. And it would have to adjust for varying light conditions and the raucous commotion
of people at different distances from its sensor.

``No one had tried to solve these problems in the consumer space and put all of this together,''
says Don Mattrick, the president of Microsoft's interactive entertainment business.

For Kinect's eyes, Microsoft turned to PrimeSense, based in Tel Aviv. It links a standard Web camera
with a pair of sensors to offer depth perception. One sensor emits light near the infrared range,
giving Kinect its own light source impervious to ambient conditions. The other sensor monitors
users' distance from the device.

The eyes were nice, but if only Kinect had a brain.

Adding the smarts required Microsoft's artificial-intelligence experts and thousands of test
subjects. Microsoft found people of varying shapes and sizes and recorded how they moved by
monitoring 48 joints in their bodies. Over time, the algorithms that digest this data became better
and better, allowing the system to work with pregnant women and children in baggy clothes as well as
with average-size adults in T-shirts and shorts.

Microsoft upgrades and rewires the Kinect brain every 24 hours and can send updates to Xbox systems
via the Internet when it chooses. Kinect recognizes someone it has seen before by body shape, so
there's no need to log into the system each time a game is played. It knows your left hand from your
right and can distinguish between two players even when their paths cross.

If players have similar builds, Kinect tries to glean differences in their facial features,
haircuts, body movements and clothing color. And if identically dressed twins initially stump the
system, it will ask each to say something.

``If it can't disambiguate, we say, `Please tell us if you are A or B,' '' says Alex Kipman,
incubation director for Xbox 360. ``Then, you end up with the equivalent of a different bar code.''

On a more futuristic note, Kinect might see that you're wearing a Dallas Cowboys jersey during a
football game and switch the commentary to the voices of the Dallas announcers.

For voice commands, the device relies on four microphones in an asymmetrical configuration that
helps home in on the person giving commands and separate out the chatter of other people on the
sofa. Kinect also knows when sound comes from the TV or from a game and can block the extra noise
from interfering with the voice commands.

``If we are serious about shifting the entire computing industry to this world where the devices
understand you, then the technology needs to be robust,'' Mr.~Kipman says. ``Otherwise, it's just a
gimmick.''

Surpassing gimmick status may be Kinect's biggest hurdle.

People like Mr.~Johnson from Gizmodo note that the first batch of Kinect games differs sharply from
the war and adventure sagas that have driven Xbox sales. The weapons have been replaced by water
rafts, Ping-Pong paddles and yoga poses through games similar to those that families found with
Nintendo's Wii.

``It's not being used to its full potential in gaming yet,'' Mr.~Johnson says of Kinect. ``It's
mostly Wii-class party games and jumping around.''

Sony contemplated advancing its own gesture-recognition technology to the 3-D realm and eliminating
controllers, but it decided that gamers wouldn't find the experience satisfactory at this point.
Instead, it built Move, a wand-like controller with more sophisticated movement-tracking features
than the Wii wand.

``I totally agree that there is this magical feeling with using your hands to select something,''
says Richard Marks, a senior researcher at Sony Computer Entertainment America, who helped create
Move. ``But that feeling wears off pretty quickly, and it becomes a pretty cumbersome way to do
things.''

Sony executives tip their hat to Microsoft for trying something risky, but like some other people
who've tested Kinect, say the system seems to lag, hindering truly immersive games. Still, game
makers like Harmonix Music Systems describe Kinect as filling a void and credit Microsoft for making
something new possible.

Harmonix, which sells the Rock Band music game, will offer Dance Central, a game made for Kinect
that teaches dance routines to songs like ``Poker Face,'' from Lady Gaga, and ``Bust a Move,'' from
Young MC.

``We've been trying to find technology that would allow the player to use their whole body,'' says
Tracy Rosenthal-Newsom, a vice president at Harmonix. ``We wanted to remove the technology and
really allow people to dance.''

Harmonix hired a team of choreographers to come up with the routines, which range from simple,
rhythmic motions to acrobatic affairs that only skilled dancers can handle.

``It really is a toy, and I mean that in the best sense of the word,'' says Ted Brown, a game
designer at Buzz Monkey, which produces games for the major console makers. ``There is magic there
when you can sort of put on a skin and perform on the stage.''

The first Kinect prototype cost Microsoft \$30,000 to build, but 1,000 workers would eventually be
involved in the project. And now, hundreds of millions of dollars later, the company has a product
it can sell for \$150 a pop and still turn a profit, Mr.~Mattrick says. (People who don't have an
Xbox can pay \$300 for a package that includes the console, Kinect and a game.)

Microsoft has spent several months marketing Kinect, even setting up a speakeasy-style site in Los
Angeles where celebrities like Justin Bieber and Tony Hawk could play games, and drink and eat with
friends, after saying a password to gain entry. All told, Microsoft expects to spend ``hundreds of
millions'' to advertise the device, Mr.~Mattrick says.

For Mr.~Ballmer, Kinect is far more than a business opportunity or a pleasant diversion for
consumers. It offers a moment to prove to investors and company directors that Microsoft is capable
of an Applesque, game-changing moment under his leadership.

``I'm excited to be way out in front,'' he says, ``and want to push the pedal on that.''

\section{Turning Customers Into Creators}

\lettrine{A}{}\mycalendar{Oct.'10}{25} few young women are gathered around a conference table dotted
with bottles of colorful vitamin drinks, iPod cables and slender laptops. A whiteboard with
lime-green writing almost swallows a wall in the room.

One bites her lip; another taps lightly on the table. They're all quiet, deep in thought, as they
decide whether it would be a good idea to add a music feature to the Web site they're working on --
a new online shopping site called PlumWillow.

``The problem,'' says one of them, Sarah Murphy, ``is that there are so many genres of music that
it'd be hard to get it right what people want to listen to.''

``Right,'' Carla Larin concurs, tossing her wavy brown hair. ``But it'd be cool to have, like, a
PlumWillow station full of songs we think are cool.''

None of these girls are the company's founders, nor are they social media consultants. They aren't
at PlumWillow's office for ``take your daughter to work day,'' either.

Rather, they're part of a team of 15- and 16-year-old interns who are being tapped for their own
special brand of expertise and insight: a bird's-eye view into the life and mind of high school
teenagers, exactly the audience that PlumWillow is seeking.

``They definitely aren't shy about telling us what they like and don't like,'' says Lindsay Anvik,
director of marketing at PlumWillow, who helps oversee the internship program at its offices in
Manhattan.

The interns are also emblematic of how Web-based businesses are doing more than merely shaping their
products and services around customer preferences. The companies are corralling those customers in
the workplace and making them part of the design and marketing process, according to Susan Etlinger,
a consultant at the Altimeter Group, which researches Web technologies and advises companies on how
to use them.

Of course, search engines like Google and Bing have been racing to tailor their results to
individual users, and Facebook is constantly tweaking its algorithm to show members' updates and the
Web links that are most relevant to them.

But what's happening at PlumWillow is a sign of an even more intimate relationship between a company
and its customers.

Moving beyond ``the old-fashioned focus group and into co-creation with your demographic is
something that will happen more in the next couple of years,'' Ms.~Etlinger says. ``All business
will have to learn how to cope with a new generation of users that are used to their particular
experience of the Web.''

Because PlumWillow wants to be more than just an online shopping destination -- it's tackling the
tricky challenge of recreating the experience of a gaggle of girls going to the mall -- its success
hinges on getting all the details right, down to the pop songs that girls want to hear while hunting
for a new pair of slouchy ankle boots.

``The site needs to be authentic to their voice and how they shop,'' says Charlie Federman,
PlumWillow's chairman, whose venture capital firm also led a round of seed funding in the company.
``Adults trying to recreate that are just asking for trouble because these kids are smart and
sophisticated and know when something is phony.''

The girls were initially brought on as a ``sanity test,'' Mr.~Federman says. ``We were all excited
and talking about this great idea when we realized a wise thing to do would be to actually talk to
some teen girls.''

Once the conversation began, the dynamic changed.

``It went from us talking to them to us listening to them,'' says Scott Stone, co-founder and head
of business development at PlumWillow. ``We decided we might as well institutionalize it and make it
part of our culture.''

Two days a week, Ms.~Anvik and Tal Flanchraych, the product manager, grill the girls on all kinds of
topics, asking whether the site's comment system is too confusing, for example, or brainstorming
about prizes for the site's Halloween contest.

``Then we go back to our desks and regurgitate everything and think about how it fits into our
future planning,'' Ms.~Flanchraych says.

Nearly 20 girls have cycled through the company since early this year, PlumWillow says.

They help keep the company nimble enough to catch and fix mistakes before they are pushed out to
broader audiences, executives say.

``We watch what they click on, see what they do and how they use the site,'' says Eric David Benari,
another of the site's founders and its chief technology officer. ``It's not something we can do
virtually.''

It's nothing new for companies to gather input from audiences they serve. Gap recently reverted to
its boxy old logo after users complained about a new design. Twitter famously formalized the
shorthand of its users as the site began to balloon in popularity.

PlumWillow doesn't want to wait until it hears -- positively or negatively -- from its customer. It
wants customers in-house so it can always be ahead of the curve.

For PlumWillow, however, the trick is to find a balance between its own strategic direction and
fickle consumer feedback.

``You don't want to put a bunch of teenagers in charge of the site, but they are revealing the way
they think about it, which can be extremely useful for a start-up,'' says Josh Bernoff, an analyst
at Forrester Research. ``If you go too far in one direction, you become like a politician pandering
to its audience.''

FOR all the effort that the girls are putting into the site, what do they get in return? School
credit and ripe material for college application essays, for starters.

But there may be something more in it for them. While many of their peers may spend their
internships doing office work at various companies, the PlumWillow interns are getting a taste of
the challenges of entrepreneurship.

``I loved seeing the whole process from start to finish,'' Ms.~Larin says. ``Six months ago we were
looking at PDFs of the site; now we have the live version that we helped create. It's incredibly
cool to see.''

\section{G.O.P. Is Poised to Seize House, if Not Senate}

\lettrine{A}{}\mycalendar{Oct.'10}{25} costly and polarizing Congressional campaign heads into its
closing week with Republicans in a strong position to win the House but with Democrats maintaining a
narrow edge in the battle for the Senate, according to a race-by-race review and lawmakers and
strategists on both sides.

President Obama campaigned for a fourth consecutive day on Saturday as the Democratic Party threw
its full weight into preventing a defeat of historic proportions in an election shaped by a sour
economy, intense debate over the White House's far-reaching domestic agenda and the rise of a highly
energized grass-roots conservative movement.

But Republicans have placed enough seats into play that Democrats now seem likely to give up many of
the gains they made in the last two election cycles, leaving Washington on the brink of a
substantial shift in the balance of power.

The final nine days of the midterm election are unfolding across a wide landscape, with several
dozen House races close enough to break either way, determining whether the election produces a
Republican wave that reaches deep into the Democratic ranks. In the Senate, Democrats were bracing
to lose seats, but the crucial contests remained highly fluid as Republicans struggled to pull away
in several Democratic-leaning states.

The candidates, political parties and a torrent of outside groups made fresh strategic investments
and pumped yet another multimillion-dollar wave of television advertising into House races across
the country, hoping to press their advantages across a battleground that has expanded to nearly 100
districts.

In the House, 28 Democratic seats are either leaning Republican or all but lost to Republican
candidates, according to the latest ratings of Congressional races by The New York Times, while 40
seats held by Democrats are seen as tossups. To win a majority, Republicans need to pick up a net of
39 seats; to reach that threshold they will probably have to win at least 44 seats now held by
Democrats to offset a handful of projected Democratic victories in Republican-held districts.

In the Senate, races for Democratic-held seats in California, Colorado, Illinois, Nevada,
Pennsylvania, Washington and West Virginia are rated as tossups by The Times. Republicans seem
assured of taking Democratic seats in other states, including Arkansas and Indiana, but must win at
least five of the seven most competitive remaining races to seize a majority, and Democrats improved
their standing in at least three of those states last week.

In the final week of campaigning, Democrats are planning new investments to protect Senator Patty
Murray in Washington, while Republicans are strengthening their effort to defeat Senator Barbara
Boxer of California.

Candidates began closing arguments on Saturday, reprising divisions over Mr.~Obama's economic
stimulus bill and clashing over private investment accounts for Social Security, an extension of the
Bush-era tax cuts and a host of domestic policies.

While the outlook is grim for Democrats in the House, according to interviews with candidates,
pollsters and consultants involved in races, the field remains volatile and strong voter turnout
could save some seats. Yet even by conservative calculations, Republicans are well within reach of
winning back a majority they lost four years ago.

``There are Democratic candidates who still appear to be in the race, but our candidates are
delivering the fatal blow,'' said Representative Pete Sessions of Texas, chairman of the National
Republican Congressional Committee. ``If we look all across the country, we are seeing incumbent
Democrats in a world of hurt.''

A wave of anxiety swept across Democrats, regardless of seniority, geographic region or whether they
voted for Mr.~Obama's agenda on the hot-button issues of health care, economic stimulus or climate
change legislation.

Representative Barney Frank, Democrat of Massachusetts, gave a personal loan of \$200,000 to his
campaign to wage his toughest fight in years.

Representative John D.~Dingell, Democrat of Michigan, the longest-serving member, invited former
President Bill Clinton to his district for two stops on Sunday. Representative Gene Taylor, Democrat
of Mississippi, who often is a reliable vote for Republicans, struggled to defend a seat he has held
for two decades.

Republicans went after Mr.~Taylor with a TV ad that opens with the precise moment Mr.~Taylor
supported Representative Nancy Pelosi for speaker in 2007 to the applause of his colleagues on the
House floor. ``This is the moment Democrat Gene Taylor turned his back on us,'' the narrator said,
echoing a theme that has emerged in district after district.

As they face the certainty of losses, Democrats are in a sense victims of their own success after
winning 55 seats and expanding far into conservative territory over the last two election cycles.
Now they are trying to defy history and demographics as they struggle to hang on to the districts in
a midterm election with their party in the White House.

``We're duking it out everywhere,'' said Representative Chris Van Hollen of Maryland, the chairman
of the Democratic Congressional Campaign Committee.

Should the Democrats manage to hang on to the House, it would be considered a major political upset
at this point.

Republicans focused their efforts heavily on the Ohio River Valley, hoping to win back a trove of
districts in Indiana, Kentucky, Ohio, Pennsylvania and West Virginia. Democrats were trying to build
a firewall in the Northeast, including seats in Connecticut, New York and Pennsylvania, where a
strong performance could keep Republicans from repeating their 1994 sweep, when they captured 54
seats.

With time running out, leaders of both parties planned to spend the weekend in districts across the
country. Mr.~Obama appeared on Saturday evening in Minneapolis with Ms.~Pelosi, raising \$600,000 to
help pay for a final burst of advertising for House candidates. As she sought to rally the
Democratic crowd, she said: ``When the public knows the choice, we think that we will win -- we
know.''

Senator Mitch McConnell of Kentucky, the Republican leader, appeared in West Virginia with the top
Republican candidates, while Representative John A.~Boehner of Ohio, the likely speaker if
Republicans take the House, campaigned Saturday in Kentucky on behalf of Andy Barr, a Republican who
is trying to upend Representative Ben Chandler, a conservative Democrat.

Yet other vulnerable Democrats continued to hang tough, and their resilience led Republicans to look
elsewhere to find Democrats who had not prepared for difficult contests.

Representative Tom Perriello of Virginia, who for months has been seen by Republicans as among the
most endangered freshmen Democrats, is now in a race seen as one that could go either way.
Representative Ed Perlmutter of Colorado, whose district was aggressively pursued by Republicans,
said he had seen his re-election prospects improve in recent weeks as voters have focused more
closely on the contest, and he said he expected many of his embattled Democratic colleagues to
prevail.

``I am normally an optimistic fellow, but I am also realistic,'' Mr.~Perlmutter said in an
interview. ``I have been talking to my buddies, and they are in tough races. But they are still
right in it.''

Democrats are seeking to diminish their losses by mobilizing key voting blocs, particularly suburban
and upper-income voters who can be motivated by concerns about Republicans returning to power in
Washington and imposing a conservative, antigovernment agenda while trying to undo much of what
Democrats pushed through Congress. Students and black voters, who offered crucial support in
Mr.~Obama's 2008 campaign, are also important constituencies in several districts across the
Midwest, Northeast and South.

``I think they are going to show up far beyond what the polling indicates, and that is the secret to
winning,'' said Representative James E.~Clyburn of South Carolina, the No.~3 Democrat in the House,
who appeared for candidates in Illinois, Indiana, Nevada, Minnesota and New York, with plans to
stump in the Carolinas next week.

``What I am seeing district by district is a different result than if you are looking at the House
over all,'' said Mr.~Clyburn, the highest-ranking African-American in Congress.

Even though Republican optimism is high in the closing days, party leaders have ordered lawmakers
and candidates to avoid overconfidence.

``It's a battle to the end,'' said Representative Greg Walden of Oregon, a vice chairman of the
Republican Congressional committee. ``But only 20 months ago, Republicans were viewed like mold --
not really alive, but you couldn't kill us either. We've come back.''

\section{Pitching Movies or Filming Shows, Hollywood Is Hooked on iPads}

\lettrine{L}{ast}\mycalendar{Oct.'10}{25} month at a meeting in Hollywood, it was time to plot out
the sequel to ``Star Trek,'' last year's blockbuster reboot of the sci-fi franchise.

The attendees all brought smartphones -- gadgets far more powerful than the videophones imagined by
the ``Star Trek'' writers 40 years ago. Bob Orci brought something the writers back then could only
dream of: an iPad.

Mr.~Orci, meeting with the producers J.~J.~Abrams, Damon Lindelof and Bryan Burk, and his fellow
writer Alex Kurtzman, jump-started the discussion with an iPad slide show, showing stills from the
first film, snapshots of potential locations and a photo of a suggested actress for one of the
roles. On the woman's photo, he had used his iPad to paint on a Vulcan ear.

``When you're carrying a little TV around, you bring the power of imagery to places that you don't
normally have it,'' Mr.~Orci said in an interview.

When Apple introduced the iPad six months ago, ushering in an era of tablet computing, experts
predicted that tablets would transform the habits of groups of people like college students (who
would carry digital textbooks) and doctors (who would manage patient records). They can add
Hollywood to the list of those affected.

The iPad is the must-carry accessory on sound stages this season, visible behind the scenes of
television and film shoots and in business meetings. When Paula Abdul, the former ``American Idol''
judge, wants to preview her new dance show for prospective sponsors, she turns on her iPad and pulls
up a YouTube video. When Julie Benz, a star of ``No Ordinary Family'' on ABC, has downtime between
shoots, she plays Angry Birds, the popular physics-based puzzle game.

``It's perfect for the long hours here,'' Ms.~Benz said in her trailer on the Walt Disney Studios
lot this summer.

All this acclaim amounts to free advertising for Apple, which has rarely if ever given away its
products to A-list customers. And the use of iPads inevitably ends up inspiring story lines that
millions of people see. On the NBC show ``The Office'' this month, when one character asks what time
it is, another character pulls out his iPad, like an oversize pocket watch.

The iPad's most prominent appearance in prime time came in April, just days after it was put on
sale, on the ABC sitcom ``Modern Family,'' when one of the characters desperately wanted an iPad for
his birthday. Eyebrows were raised because Apple's chief executive, Steven P.~Jobs, is on the board
of ABC's parent, Disney. An Apple spokeswoman said the company does not pay for product placement,
and an executive producer of ``Modern Family,'' Steve Levitan, said the iPad story originated with
the show's writers.

Mr.~Levitan happens to be an avid iPad user, sometimes posting to Twitter from the tablet.

According to the Nielsen Company, Apple products have popped up about 2,438 times on television
programs through September of this year. Some of those are news programs: iPads are visible on the
set of the ``Fast Money'' talk show on CNBC, and Glenn Beck sometimes brings his to ``Fox \&
Friends,'' the Fox News morning show.

Hollywood's converts to the iPad say it can drastically reduce the amount of paper that is wasted on
script rewrites.

The actress Dana Delany, who will star in ``Body of Proof,'' a coming ABC drama, said she now has
revisions sent to her digitally. ``I think it's the greatest invention in years,'' Ms.~Delany said.
(The iPad will be a fixture in the medical examiner room on ``Body of Proof,'' which will start in
early 2011.)

Mr.~Orci's iPad has served as the daily ``call sheet'' with the day's instructions; acted as a map
in an unfamiliar location; and allowed him to keep tabs on ``Fringe'' and ``Hawaii Five-0,'' two
shows he helps produce. ``Oh, and it woke me up in the morning with its alarm,'' he said.

His one complaint is that the screen is hard to see in the sun.

There are some iPhone and iPad apps made especially for the entertainment community, like Rehearsal,
which helps actors learn their lines. John Carroll Lynch, who appeared in the A\&E series ``The
Glades'' last summer and will appear in ``Body of Proof'' next year, swears by Rehearsal, which was
created by the actor David H.~Lawrence XVII.

The app imports the script for a television episode. ``Then I run the scenes with just the others'
dialogue, and I speak when I'm supposed to be speaking,'' he said.

In his downtime in his trailer, Mr.~Lynch also plays the Madden 2011 app and watches shows he
downloads from iTunes. He said he was surprised by how much he had come to use the tablet.

``It's like we're getting to the tricorder,'' he said, referring to the scanning and recording gizmo
from the original ``Star Trek.''

\section{The Nice Guy Who Puts the Mean Into `Glee'}

\lettrine{I}{t}\mycalendar{Oct.'10}{25}'S true: Sue Sylvester is a man.

Behind most great comedic actors, the saying goes, there is a great comedic writer. Will Ferrell has
Adam McKay. Jack Lemmon had Billy Wilder. And Jane Lynch, who won an Emmy Award last month for her
portrayal of Sue Sylvester, the acid-spewing, narcissism-redefining cheerleading coach on ``Glee,''
has Ian Brennan.

It was the 32-year-old Mr.~Brennan, for instance, who wrote the classic Sue zinger: ``You think this
is hard? I'm passing a gallstone as we speak. That is hard!''

And this oh-so-subtle put-down, said to one glee club student: ``So you like show tunes. It doesn't
mean you're gay. It just means you're awful.''

One of Ms.~Lynch's favorite Sue lines -- again, written by Mr.~Brennan -- involves a threat, this
time to her main nemesis, the glee-club director Will Schuester (Matthew Morrison): ``I will go to
the animal shelter and get you a kitty cat. I will let you fall in love with that kitty cat. And
then on some dark, cold night I will steal away into your house and punch you in the face.''

Ms.~Lynch said: ``Ian is this incredibly nice, incredibly sweet guy who just happens to have a
really cruel, supremely mean sense of humor. I think it has something to do with growing up Irish
Catholic.''

When it comes to writing ``Glee,'' the hit musical comedy on Fox, the three creators of the show --
Mr.~Brennan, Ryan Murphy and Brad Falchuk -- each play a different role. Mr.~Murphy, the director of
``Eat Pray Love'' whose television work includes ``Nip/Tuck,'' picks the songs and comes up with
some of the crazier story lines, like football players dancing in formation to Beyonc\'e's ``Single
Ladies (Put a Ring on It).'' Mr.~Falchuk, a ``Nip/Tuck'' alumnus, generally handles the scenes that
jerk tears.

Mr.~Brennan -- who never had a professional writing job before -- is a one-man one-liner factory, in
particular writing the majority of the dialogue for the megalomaniacal Sue. The Web is overflowing
with lists of his quips, with no compilation complete without this one: ``I'm going to ask you to
smell your armpits. That's the smell of failure, and it's stinking up my office.''

How does he come up with this stuff?

``It honestly just kind of flows out, kind of in aria form, and I have to whittle the writing down
to something usable,'' Mr.~Brennan said while curled up on a sofa in the ``Glee'' production offices
on the Paramount Pictures lot here. ``It's just the mean things that pop into the back of your mind
that you sometimes want to say but don't. The difference is that Sue actually says it.''

Mr.~Murphy likened Mr.~Brennan to a human sponge.

``Ian is a big collector of detritus -- ideas and words and observations that he thinks might be
useful,'' he said. ``He writes them all down. I call it the magic book.''

Mr.~Brennan insisted that Coach Sylvester's attitude is not based on a real person, but conceded,
after some pressing, that the prickly Mr.~Murphy is sometimes a muse. ``Sometimes if I get stuck,
I'll think, `What would Ryan say in one of his really mean moments?' ''

It is clear that some of Mr.~Brennan's inspiration comes from himself. One of the running gags on
``Glee'' involves Sue's hatred of curly hair. (``I don't trust a man with curly hair,''Mr.~Brennan
had her say in one episode. ``I can't help picturing small birds lying sulfurous eggs in there, and
I find it disgusting.'') Mr.~Brennan has a love-hate relationship with his own long, wavy locks --
something that started after an encounter with Stephen Sondheim.

After college Mr.~Brennan was cast in a Chicago production of Mr.~Sondheim's ``Saturday Night.''
Watching a rehearsal, Mr.~Sondheim commented on Mr.~Brennan to the stage manager, ``He's one haircut
away from being attractive.''

Writing was never a dream for Mr.~Brennan. Growing up in the Chicago suburb of Mount Prospect, Ill.,
Mr.~Brennan had his heart set on acting. In junior high he got the lead in a community production.
During high school he was in competitive speech and drama, and also joined the show choir,
grudgingly, because he thought he needed to learn how to sing and dance if he was going to make it
big. ``God, those sequins were awful,'' he said, recalling the choir costumes.

After attending Loyola University Chicago he toiled as an actor in Chicago for a few years,
ultimately landing parts in Steppenwolf Theater Company and Goodman Theater productions. Then came
some bit television roles in New York.

All that time, however, Mr.~Brennan couldn't shake his show-choir experience. So, in 2005, he bought
``Screenwriting for Dummies'' and wrote a first draft of ``Glee,'' then conceived as a biting,
cynical film along the lines of ``The Virgin Suicides.'' He shopped it around and got nowhere.

Then fate struck. A friend of Mr.~Brennan's in Los Angeles was a member of the same gym as
Mr.~Murphy. The friend passed Mr.~Brennan's script to him. A year of rewrites later, ``Glee'' was
born as a television musical comedy.

About 12 million people each week now watch the series, which has spawned hit CDs, a concert tour
and a robust apparel business -- and may be turned into a Broadway show.

``The big learning curve has been figuring out how to deal with the anxiety,'' Mr.~Brennan said of
his first full-time writing job. ``You have a script deadline and this enormous operation depending
on you and -- sorry! -- you have no ideas.'' He also frets about the attention he has received,
particularly when it comes to his older sister, Sarah Brennan, who is a founder of a charter school
in a rough Chicago neighborhood. ``I feel guilty that she works so hard doing something important,
and I'm the one getting noticed,'' he said.

Mr.~Brennan remains close to his parents, who were visiting the ``Glee'' set in early October.

``I just let him be himself,'' said his father, John Brennan, a former priest. ``I can't take any
credit for his talent,''

With more than a hint of sarcasm, Charman Brennan, a middle school math teacher, said: ``What about
the brilliant mother? Feel free to leave her out.''

John Brennan added: ``Ian has always been a character. His preschool teacher told us that he was the
only one who got her jokes.''

A character indeed. Ian Brennan, who is between girlfriends at the moment, is a chatterbox with a
habit of talking with his fingers outstretched in front of him, as if he were manipulating
marionettes. He has an unusual fashion sense, piecing together vintage-shop polyester with
street-vendor jewelry in a look he calls ``70s tennis eccentric.'' (``It's almost like he wears
little costumes,'' Ms.~Lynch said.) In general Mr.~Brennan has a hard time sitting still.

``It's kind of rodentlike, isn't it?'' he said. ``Can't you picture me grabbing a nut and scrambling
up the wall?''

He stuck with the metaphor over the course of several hours on the Paramount lot, where ``Glee'' is
taped on three soundstages. On the show's choir-room set, Mr.~Brennan leaned backward against a
grand piano and startled himself by playing a chord.

``It's my tail,'' he said. ``I just finished playing a sonata with my tail.''

\section{The Latest Sherlock Hears a `Who'}

\lettrine{B}{est}\mycalendar{Oct.'10}{25} known to Americans as the reprehensible houseguest in
``Atonement'' -- is currently enjoying the privilege of playing two of British pop culture's
greatest heroes. He's being paid for just one of them, though.

As the star of ``Sherlock,'' the highly entertaining new BBC-``Masterpiece Mystery!'' production
that comes to PBS on Sunday, he is officially the latest incarnation of Sherlock Holmes, the demon
detective of Baker Street.

Anglophilic television fans, however, will sense another character lurking in the bounding
physicality and hyperverbal outbursts of this contemporary Holmes. Mr.~Cumberbatch's performance
feels like a slightly dialed-down homage to David Tennant's portrayal of the title role in the
modern ``Doctor Who.'' And it's sufficiently enjoyable that fans of that legendary science-fiction
show might wish Mr.~Cumberbatch had auditioned to play the Doctor when Mr.~Tennant left a season
ago.

This connection is not simply fanciful: ``Sherlock'' was created by Steven Moffat, now the head
writer and lead producer of ``Doctor Who,'' and by Mark Gatiss, another writer for that show. In
updating Arthur Conan Doyle's foundational detective stories, they have imported some of the
boy's-adventure, can-do spirit that informs ``Who.''

American viewers -- those who have aged into the less desirable demographics, anyway -- will also
notice the family resemblance of this Holmes to our own eccentric-genius police consultants in shows
like ``Monk'' and ``The Mentalist.'' Of course, those characters were based on Sherlock Holmes in
the first place.

Over the years the Holmes imitators (along with actual Holmes films and television series) have had
one particularly baleful effect, which is to turn the detective's observational powers into a parlor
trick. ``Sherlock'' is guilty of this, too; Holmes tells men he's just met that they've traveled
around the world twice in the last month or that they spent the night with that woman over there.
It's amusing, but the charm starts to wear off after the third or fourth time; in dramatic terms,
it's empty calories.

And Mr.~Moffat and Mr.~Gatiss have given the show some of the freneticism and the mind-bending,
confusing turns of plot that characterize ``Doctor Who.'' In the third of the three 90-minute
``Sherlock'' episodes (a second season has been commissioned), Holmes must solve five unrelated
cases before he confronts the actual villain of the piece. (At least three Conan Doyle stories were
used as sources for the plot.)

Holmes purists may find the non-Victorian pace not to their liking, but then there will be a whole
smorgasbord of things to bother them, like Holmes's nicotine patches, the self-conscious emphasis on
text messaging and the colors of cellphone cases.

In other ways the show's creators have tried, cleverly and conscientiously, to stay true to the
character's heritage. Sunday's premiere, ``A Study in Pink'' (loosely based on the first of the four
Holmes novels, ``A Study in Scarlet''), begins with Dr.~John Watson (Martin Freeman) returning home
from a British war in Afghanistan -- just as the original Watson did in the 1880s -- and looking for
a roommate. ``A Study in Scarlet'' was presented as an excerpt from Watson's memoirs; in the second
episode of ``Sherlock'' we learn that Watson has written up ``A Study in Pink'' on the blog he's
keeping as part of his therapy for post-traumatic stress.

Mr.~Freeman's deft performance as the grouchy but loyal Watson is one of the show's pleasures, along
with Rupert Graves's avuncular take on Inspector Lestrade. It should also be mentioned that
Mr.~Graves, who has been making a specialty of guest appearances on British crime dramas
(``Marple,'' ``Lewis,'' ``Wallander''), belongs with George Clooney in the pantheon of the
well-aging male.

Add Mr.~Cumberbatch, who brings to the table his piercing gray eyes and an appealingly playful
arrogance, and you have an ensemble that lives up to the verve and braininess of Mr.~Moffat and
Mr.~Gatiss's writing.

There are annoyances: the onscreen text that sometimes spells out Holmes's thoughts; the
are-they-gay jokes that now seem obligatory in every show involving male colleagues; a creeping
moralism about Holmes's shortcomings. These are elements typical of the contemporary crime drama,
which is what ``Sherlock'' resembles more than it does the Conan Doyle stories.

But it has a brio that sets it apart; the other big new British import of the season, ``Luther'' on
BBC America, looks dour by comparison. The appeal is elementary: good, unpretentious fun, something
that's in short supply around here.

\section{U.S.~Prevails in Trade Dispute With China }

\lettrine{A}{fter}\mycalendar{Oct.'10}{25} a World Trade Organization panel largely upheld tariffs
that were imposed on an array of Chinese-made steel pipes, tires and other products during the Bush
administration.

China had used a number of technical arguments in a September 2008 challenge to antidumping duties,
which are supposed to compensate for unfair pricing and countervailing duties that are used to
offset improper government subsidies. But a W.T.O. dispute settlement panel rejected most of those
arguments.

The Bush administration announced levies on \$200 million of steel pipe shipments from China, South
Korea and Mexico in July 2008, a month after imposing similar countervailing duties involving a
different kind of steel pipe. The Obama administration has defended those decisions.

``This is a significant win for American workers and businesses affected by unfairly traded
imports,'' said the United States trade representative, Ron Kirk. ``This case makes clear that the
Obama administration, including U.S.T.R. and our colleagues at the Department of Commerce, will
vigorously defend the application of our trade remedy laws.''

The duties that were upheld on Friday had been imposed on a variety of specialized goods: circular
welded pipe, certain pneumatic off-road tires, light-walled rectangular pipe and tube and laminated
woven sacks.

China's challenge revolved around many technical questions, including whether state-owned
enterprises and state-owned commercial banks could be properly considered public bodies that provide
subsidies.

The ruling comes at a time of increasing tensions over currency and trade between China and the
United States. The Obama administration has agreed to investigate a complaint brought by the United
Steelworkers over China's support for its clean energy industries, and is concerned about Chinese
efforts to block exports shipments of valuable minerals known as rare earths.

``These findings are especially important at a time when the United States is vigorously
implementing W.T.O.-consistent tools to address China's unfair trade practices and to address global
imbalances,'' said Representative Sander M.~Levin, Democrat of Michigan, one of the most outspoken
House members on China's decision to hold down the value of its currency, the renminbi. ``We should
not let the possibility of meritless allegations of W.T.O. inconsistency prevent us from standing up
for U.S.~workers and businesses.''

The W.T.O. panel was established in January 2009 and held hearings in July and November of that
year. Both China and the United States have up to 60 days to appeal the panel's ruling, which ran to
283 pages and was published on the Web site of the W.T.O., which is based in Geneva. China joined
the organization in 2001.

In a separate case, the United States International Trade Commission, an independent federal agency
that assesses whether imports unfairly damage American industry, on Friday authorized the Commerce
Department to impose both antidumping and countervailing duties on coated paper from China and
Indonesia that is used in sheet-fed presses.

The commission found that the papers, which are used to produce high-quality graphics, had been
unfairly subsidized and sold in the United States at less than market value.

Senator Sherrod Brown, an Ohio Democrat who had submitted testimony to the trade commission in the
coated-paper case, applauded the ruling.

``American producers face an inexcusable flood of dumped Chinese paper -- subsidized from 10 to 15
percent of product cost,'' he said after meeting with workers at Smart Papers, a coated-paper
manufacturer in Hamilton, Ohio.

Mr.~Brown said the decision ``shows why rigorous trade law enforcement is critical to the economic
security of our workers and viability of domestic manufacturing,'' but he also argued that China's
currency policies should be considered in future trade remedy cases.

\section{Tibetans in China Protest Proposed Curbs on Their Language}

\lettrine{T}{housands}\mycalendar{Oct.'10}{25} of Tibetan students in western China have protested
since Tuesday against proposals to curb or eliminate the use of the Tibetan language in local
schools, according to reports from Tibet advocacy groups and photographs and video of the protests
circulating on the Internet.

The protests are the largest in Tibetan areas since the March 2008 uprising that began in Lhasa and
spread across the Tibetan plateau. But unlike those protests, these have been peaceful and have
involved mostly students.

A protest against the proposed policies was also held in Beijing on Friday afternoon, drawing
hundreds of Tibetan students at a prominent university that specializes in teaching ethnic
minorities, according to witness reports and photographs.

The widespread protests over language reveal the deep resentment that many Tibetans feel over
policies formulated by the Han, China's dominant ethnic group, that Tibetans say are diluting their
culture. Many Tibetans in western China also complain of strict controls over the practice of
Tibetan Buddhism, including a ban on images of the Dalai Lama, the Tibetan spiritual leader, and
large-scale Han migration to Tibetan towns. The Han end up taking many jobs that would otherwise go
to Tibetans.

The latest resistance is over a proposal to shift school instruction fully or almost fully to
China's official language, Mandarin.

The protests this week have mostly unfolded in Tibetan towns in Qinghai Province, a vast, sparsely
populated region that is historically important as a center of Tibetan culture.

They began at a high school on Tuesday in the town of Tongren, known as Rebkong in Tibetan, and then
widened. More than 1,000 students ended up taking part, according to Free Tibet and the
International Campaign for Tibet, two advocacy groups outside China. The protesters adopted a
slogan: ``Equality of ethnicities, freedom of language.''

Photographs distributed by Free Tibet, based in London, show students in uniforms taking to the
streets in Rebkong. Some photos show the students walking by a monastery and monks joining the
rally. Rebkong is the seat of Rongwo Monastery, a 700-year-old center of scholarship that is home to
about 400 monks who regularly display photos of the Dalai Lama and openly criticize government
policies they consider overbearing. The principal of a primary school in Huangzhong County,
northwest of Tongren, confirmed by telephone that peaceful protests had taken place.

Students in Rebkong appeared to have returned to classes by Wednesday. But inspired by those
rallies, hundreds and perhaps thousands of teenagers from several schools in the Tibetan town of
Chabcha, known in Chinese as Gonghe, took to the streets on Wednesday morning. One photo shows more
than 100 students dashing through the streets; another shows policemen in white hats watching a
group of students as one raises his left fist defiantly. On Thursday, students in the town of Tawo,
or Dawu in Chinese, also protested. By 2 p.m., the police were preventing people from going out into
the streets of Tawo, according to Free Tibet.

Posts on the Internet said 400 Tibetan students held a rally on Friday on the campus of Minzu
University of China, the specialized school in Beijing. Photographs showed a large group of students
gathered on a concrete walkway lined with shrubs. Other photos showed uniformed guards milling
around some students.

The traditional mission of Minzu University is to train students from ethnic minority regions of
China who might then return to those regions and work for the government. Departments in the
university specialize in scholarship on various cultures in China, and more than 600 Tibetans study
on the campus. Telephone calls on Friday afternoon to several offices at the university went
unanswered.

The protests in Qinghai erupted over speculation that government officials plan to severely limit
the teaching of Tibetan in schools, perhaps relegating it to elective or extracurricular status. On
Sept.~30, People's Daily, the Communist Party's mouthpiece, ran an article quoting Qiang Wei, the
party secretary of Qinghai Province, as saying in a speech at a Sept.~13 education conference that
mandating Chinese language instruction was crucial.

``Officials at all levels must overcome all your worries, overcome the wrong idea that to adopt
common language education for minority students will hurt minority people's feelings or affect the
development of minority culture or affect social stability,'' he said. The article provided tinder
for the protests.

Woeser, a Tibetan blogger who lives in Beijing, circulated a cellphone text message on Friday that
said: ``In order save our mother tongue, many Tibetan students are protesting in Tibetan areas
advocating for the Tibetan language. We need your attention.''

The message also said that if ethnic Han who are Cantonese speakers can protest to defend the use of
Cantonese, then Tibetans should have the right to defend their language. That referred to protests
in July in the city of Guangzhou in which Cantonese assailed a local politician's proposal to force
prominent programs on a local television network to stop broadcasting in Cantonese and switch to
Mandarin.

\section{Lustful Opera, Censored, Befuddles Chinese}

\lettrine{A}{s}\mycalendar{Oct.'10}{27} any artist or performer in China knows, it is impossible to
predict what will set off the mercurial culture censors who have sweeping power over the content of
film, music, television and print.

On Sunday, it was the depiction of a sexually aroused, anatomically correct male donkey and
references to capital punishment that nearly derailed an ambitious interpretation of the Handel
opera ``Semele,'' the tragic tale of what happens when a lustful god, a vengeful goddess and an
impressionable young maiden are ensnared in a love triangle.

In the end, officials allowed the donkey to remain onstage, but they insisted on a number of
last-minute changes that significantly altered the production and left the audience perplexed.

The opera, directed by Zhang Huan, one of China's most boundary-pushing artists, sold out nine
performances last year in Brussels with its melding of Baroque music, Greek mythology, Chinese
cultural references and modern touches that included sumo wrestlers, flashes of nudity and rousing
audience participation.

Lady Linda Wong Davies, an opera patron whose London-based foundation brought the production here
for the annual Beijing Music Festival, said her goal was to expose Chinese audiences to
Western-style opera and to build bridges between China and the rest of the world. ``I wanted to
bring to life an 18th-century German composer's work through the eyes of a contemporary Chinese
artist,'' she said.

Even before the cast arrived from Europe, Chinese officials who saw the production in Brussels
insisted on a number of changes: they vetoed the singing of the Communist anthem ``The
Internationale'' during the finale -- too provocative, apparently -- and suggested a costume change
for the Greek chorus, whose burgundy and saffron robes too closely resembled those worn by Tibetan
monks.

Those and a few other demands -- no nudity, less violence and fewer sexually suggestive gestures --
were easy enough to meet, Mr.~Zhang said.

But after officials from the Ministry of Culture watched a dress rehearsal on Saturday, they decided
that the donkey -- two actors draped in fabric -- was revealing too much of the animal's anatomy.
More ominously, they objected to a short documentary, which was screened during the overture, that
explained how the gracefully carved frame of a 450-year-old Chinese temple had made its way onto the
stage of the Poly Theater in Beijing.

Three years ago, Mr.~Zhang bought the building and its contents from an impoverished family who had
been its occupants for two decades. While taking apart the structure, the director discovered a
diary written by the broken-hearted husband. The man, Fang Jixin, wrote about how the adulterous
behavior of his wife had driven him to alcoholism, and eventually madness. In the end, after he
murdered his wife's lover, Mr.~Fang was arrested and put to death.

``I was amazed how this tale out of contemporary China was like the Greek tragedy, and it inspired
me to do this production,'' said Mr.~Zhang.

Although he declined to discuss the censors' specific objections about the documentary, others who
worked on the opera said it was the mention of the husband's crime, and especially his punishment,
that troubled the authorities.

Their solution was not very subtle. They demanded that the Chinese-language subtitles be excised,
and they forced the projectionist to freeze the film before it could reveal the heart of the tale,
leaving the audience confused about the connection between the documentary and the opera.

``To be honest, as of 2 p.m. on Sunday, we were not sure the show would go on,'' said Mr.~Zhang.

Given her desire to foster intercultural exchange, Lady Davies, too, was not eager to criticize the
last-minute bureaucratic deus ex machina. The donkey, after all, was allowed to stay in the
production, even if officials asked that his lascivious behavior be toned down.

``I'm optimistic about the future of the arts in China, although it's definitely challenging with a
capital C,'' she said with a sigh. ``Maybe next time I ought to do a production of `Mary Poppins' or
`The Sound of Music.' ''

\section{Needing Students, Maine School Hunts in China}

\lettrine{F}{aced}\mycalendar{Oct.'10}{27} with dropping enrollment and revenue, the high school in
this remote Maine town has fixed on an unlikely source of salvation: Chinese teenagers.

Never mind that Millinocket is an hour's drive from the nearest mall or movie theater, or that it
gets an average 93 inches of snow a year. Kenneth Smith, the schools superintendent, is so certain
that Chinese students will eventually arrive by the dozen -- paying \$27,000 a year in tuition, room
and board -- that he is scouting vacant properties to convert to dormitories.

``We are going full-bore,'' Dr.~Smith said last week in his office at the school, Stearns High,
where the Chinese words for ``hello'' and ``welcome'' were displayed on the dry-erase board and a
Lonely Planet China travel guide sat on the conference table. ``You've got to move if you've got
something you believe is the right thing to do.''

On Friday, Dr.~Smith left for China, where he is spending a week pitching Stearns High to school
officials, parents and students in Beijing, Shanghai and two other cities. He has hired a consultant
to help him make connections in China, lobbied Millinocket's elected officials and business owners
to embrace the plan and even directed the school's cafeteria workers to add Chinese food to the
menu.

``We get some commodity pasta, and it makes a great lo mein,'' said Kathy Civiello, the school's
nutrition director, one of the many staff members who appeared equally excited and bemused by the
plan.

With China's emergence as an economic juggernaut, colleges, universities and private secondary
schools have tried to recruit students from China and have even opened campuses there. But
Millinocket's plan may be unprecedented among public schools, even as they scramble for new sources
of revenue.

``This is the first we've even heard of it,'' said Alexis Rice, a spokeswoman for the National
School Boards Association.

There is one hitch. Under State Department rules, foreign students can attend public high school in
the United States for only a year, a system that Dr.~Smith considers unfair, given that they can
attend private high schools for four years. He is pressing Maine's Congressional delegation to seek
a change, but in the meantime, he intends to recruit a handful of Chinese students to attend Stearns
next year.

They would come to Millinocket for a year, Dr.~Smith said, then perhaps transfer to a private school
or enroll in an American college or university.

Dr.~Smith, a native of Maine who has traveled outside New England only rarely, conceded he did not
know much about China. But from what he had heard and read in recent months, he said, two things
were clear: China had a large middle class with money to spend, and its students wanted to study
here.

``They want to learn English, and they want a college education,'' he said. ``If we can get them
into a college here, they will have achieved their major goal.''

Dr.~Smith is so certain of success that it almost feels wrong to ask: Why would Chinese parents
spend \$27,000 to send their children to Stearns High, which is housed in a 1960s building, has only
one Advanced Placement course and classroom maps so outdated they still show the Soviet Union, and
where more than half of the 200 students are poor enough to qualify for free lunch?

``Our performing arts program is one of the best anywhere,'' Dr.~Smith said. ``We have a tremendous
music department and small classes with plenty of room. In China, you're elbow to elbow.''

Fair enough. But why Millinocket, a town of 5,000 about 200 miles north of Portland, Me., that fell
on hard times after its paper mill filed for bankruptcy in 2003? Vacant storefronts pock Penobscot
Avenue, the main street, and the most popular hangout for teenagers is a supermarket parking lot.

``We're a community full of assets,'' Dr.~Smith said, pointing to Mount Katahdin, Maine's highest
peak, which looms just beyond the town, and the abundant hunting, fishing and snow sport
opportunities that the locals love. ``There's the beauty, No.~1, and the fresh air. And the roads
are good.''

Terry Given, an English teacher who was born here, was more blunt.

``I don't want to sound flip,'' Ms.~Given said, ``but why not? We won't know until we get the
opportunity to know them and give them the opportunity to know us. There's something to be said for
putting ourselves out there to see if we can be the prize that's claimed.''

At a time when shrinking budgets are forcing some public schools to require students to provide
basic supplies like paper towels and soap, looking abroad for financial help may be an act of
self-preservation. The enrollment at Stearns has fallen from about 700 students in the 1970s, when
the paper mill provided hundreds of jobs. Over all, the number of students in all of Millinocket's
schools has dropped 43 percent since 2000, to 550 from 959.

Around the country, public schools can legally charge tuition to students who do not live in the
district. Dr.~Smith is basing the \$27,000 fee on the district's average spending per student, about
\$13,000 a year. He said \$14,000 seemed reasonable for room and board based on what the state's
private schools charge for it.

Private schools, of course, have drawn students from abroad for decades, and a number of them in New
England have recruited heavily from Asia in recent years.

``All of a sudden they have 60 Chinese kids in these tiny villages in Vermont,'' said Suzanne Fox,
the consultant who is working with Dr.~Smith.

Ms.~Fox, whose company, Fox Intercultural Consulting Services, helps businesses and schools build
connections in China, said she had persuaded Dr.~Smith to start slowly.

``I've had to rein him in a little bit,'' Ms.~Fox said, adding that his new goal was to recruit
perhaps five students next fall instead of 100.

In the coming months, Ms.~Fox will make frequent trips to Millinocket to teach students, teachers
and community leaders about Chinese culture. The town is virtually all-white, though it has hosted
traditional exchange students who come for a year without paying tuition.

``We're pretty vanilla,'' Ms.~Given said. ``Because we lack diversity, those who bring it into the
community stick out like sore thumbs.''

Dr.~Smith has also sought advice from a Chinese exchange student who he said was spending the year
``with the undertaker's family'' in neighboring East Millinocket.

``I asked her what she most wanted to do while she was here,'' he said. ``She told me she wanted to
go to Florida and see Disney World, go to Boston and shop, and climb the mountain.''

That would be Mount Katahdin. The school system owns a cabin at its base, which Dr.~Smith hopes to
use for weekend retreats, where Millinocket students can get to know their Chinese classmates.
Students here seemed enthusiastic about the plan, though some, having heard that Chinese teenagers
were academically driven, feared that their class ranking would slip.

Others said the school was the town's social hub, its football games and musicals drawing crowds.
Matthew Preble, 17, said he would welcome Chinese students but wondered whether Millinocket would
feel the same.

``We're used to Stearns High School being a small hometown type of thing,'' he said. ``The fact that
suddenly we might have up to hundreds of kids from China might change that -- in a good way, but
we're also kind of scared to lose our town.''

\section{An Unlikely Pairing Bears Fruit in North Korea}

\lettrine{O}{f}\mycalendar{Oct.'10}{27} all the fantastic tales to come out of North Korea -- the
country's leader is injected with the blood of virgin girls, he made 11 holes-in-one during his very
first round of golf, each grain of rice he eats is inspected by hand for imperfections, his youngest
son and would-be successor has had cosmetic surgery to make him resemble his grandfather -- not one
of these seems as improbable as the event that took place on Monday, when a science university
founded by American evangelical scholars began its first day of classes in Pyongyang, the capital of
the secretive Communist state.

North Korea, the nuclear-armed ``hermit kingdom,'' is so poor that there is almost no supply of
concrete, bricks or window glass. People suffer shortages of rice, gasoline and even underwear. The
Internet barely exists, not to mention computers, and the economy is so moribund that most factories
barely function for lack of raw materials and electricity.

In spite of all this, classes in technical English started Monday at the Pyongyang University of
Science and Technology. A fuller curriculum in information technology, business management and
agriculture is supposed to get under way in March.

``It's amazing, and kind of a miracle,'' said Park Chan-mo, one of the founders of the school, which
was largely financed by contributions from evangelical Christian groups in the United States and
South Korea. ``Many people were skeptical, but we're all Christians. We had faith.''

The driving force behind the school was Kim Chin-kyung, an American born in Seoul who founded a
university in China in 1992. He made periodic trips from China into North Korea and in 1998 was
arrested at his hotel in the capital and thrown into prison, accused of being an agent for the
C.I.A.

The relentless interrogations went on for six weeks and almost broke him. ``I was ready to die,'' he
said in a 2001 interview, even writing out a will and bequeathing his organs for transplants and
medical study in Pyongyang.

He was finally released, he said, after convincing the authorities that ``I was not the kind of
person who would spy on them.''

In November 2000, a man appeared in his university office in China -- oddly, the same man who had
ordered his arrest for espionage in Pyongyang in 1998. But this time he had a proposal from the
North Korean government: could he duplicate his Chinese technical university in Pyongyang?

``Doing business with North Korea is not for the faint of heart,'' Mr.~Kim said on the school's Web
site, ``but the effort is ennobling and necessary.''

The first group of 160 undergraduate and master's students has been chosen by the North Korean
government, selected from its top colleges and from the political and military elite. Their tuition,
room, board and books are all free, financed by foreign donors and individual sponsors. The plans
call for an eventual student body of 2,600 and a faculty of 250, with classes in public health,
architecture, engineering and construction.

Sixteen professors from the United States and Europe arrived in Pyongyang over the weekend. For now,
no South Korean professors are allowed because of recent political tensions between the Koreas.

It seems an unlikely marriage -- the hard-line Communist state and wealthy Christian capitalists --
and it remains to be seen how well the match has been made.

North Korea, while reluctant to expose its citizens to the outside world, has been seeking foreign
investment for its decrepit educational system. For their part, the evangelicals, who have
antagonized the North by encouraging defections and assisting refugees after they cross over, are
seeking a foothold inside the churchless state.

North Korea has made a similar bargain before. The Unification Church of the Rev. Sun Myung Moon,
not only a Christian but staunchly anti-Communist, operates a car factory in Pyongyang, for
instance. But the church is allowed to make only cars, not Christians or capitalist converts.

There is no campus chapel at the new university, Dr.~Park said, and there is not one in the plans.
But neither, for now, are there any official portraits of the North Korean leader, Kim Jong-il, or
his father, the late Kim Il-sung, which hang in every school and public building in the North.

The \$35 million, 240-acre campus includes a faculty guesthouse and world-class dormitories and
classrooms, all of which are said to have running water, power and heat. The school has its own
backup generators, but with so little diesel and gasoline available in the North, fuel has to be
trucked in from neighboring China.

Classes will be taught in English, and Internet access has been promised to all students. The campus
has sirens that go off before rolling electrical blackouts, so work on computers can be saved.

``The Internet will be censored, and we can't imagine that it won't be,'' said Dr.~Park, who has
been involved in educational exchanges with the North since 2000. ``Even in South Korea things are
blocked. I'm sure North Korea has been looking at my e-mails. I keep them businesslike.''

Dr.~Park, the former president of the prestigious Pohang University of Science and Technology in
South Korea, said the university project could not have been completed without the approval of the
United States government. Officials at the school, eager not to run afoul of international sanctions
in place against the North, have even sent its curriculum to the State Department for vetting.

One request from Washington was that the name of the biotechnology course be changed for fear that
it might be seen as useful in developing biological weapons, Dr.~Park said. So the course title was
changed to ``Agriculture and Life Sciences.''

The United States government also was also ``very sensitive,'' Dr.~Park said, about young North
Korean scientists learning skills that could be used by the military or in developing nuclear
weapons. ``We can't be fooled into teaching them those kinds of things.''

Several conservative lawmakers from South Korea called for Seoul, which gave \$1 million to the
school in 2006, to cut off all support. One legislator, Yoon Sang-hyun, was angered that the North
insisted that future economics classes include lessons about juche, or Kim Il-sung's founding
philosophy of self-reliance.

Some critics also have suggested that there must have been heavy payoffs made to the North Korean
government to move the project along, but Mr.~Kim insisted that no deals had been made.

``Every brick we used, every bit of steel, every bit of equipment, we brought in from China,''
Mr.~Kim, who was in Pyongyang for the opening, said in an interview in Fortune last year. ``I have
never brought any cash into North Korea.''

``I have unlimited credit at the Bank of Heaven,'' he added.

\section{Wife of Nobel Laureate Invites Scores of Chinese Activists to Oslo}

\lettrine{I}{n}\mycalendar{Oct.'10}{27} a move to call global attention to the imprisonment of her
husband, the wife of the Nobel Peace Prize winner Liu Xiaobo has invited 143 Chinese activists,
academics and celebrities to the award ceremony in Oslo.

Her invitation list was released online with a letter saying the Chinese authorities were unlikely
to allow her or her husband to attend the event on Dec.~10. ``This prize belongs to everyone,
everyone who is Chinese and has been fearless in defending their dignity,'' the letter states.

Mr.~Liu is serving an 11-year prison term in northeastern China. He was sentenced a year after being
detained in the wake of the publication online of Charter 08, a pro-democracy manifesto he helped
draft that garnered about 10,000 signatures before the government blocked it.

His wife, Liu Xia, has been under house arrest since the prize was announced on Oct.~8, according to
relatives. She is unreachable by cellphone, and outsiders have not been able to visit her at home.

Yang Jianli, a friend of the couple who lives in Massachusetts, said he had verified that the
letter, posted on the Internet on Monday, was authentic. ``I think she wants to share the honor of
this award and at the same time, to call upon these people to shoulder responsibility for the
future,'' he said.

Ms.~Liu's invitation list includes a number of people who are unlikely to attend. Some are
dissidents who are typically kept under close surveillance. One, Zhang Zuhua, another of the main
authors of Charter 08, has been confined to his house since the Nobel announcement.

``There is a big cop outside the door, so I won't be able to go,'' he said with a laugh.

Others seemed a little befuddled about why they were invited. ``I would definitely not go myself,''
said Li Fangping, a lawyer who represented parents whose children suffered problems from
contaminated milk products. ``I feel like there are people ahead of me. I am not sure what she is
trying to do because I have not been able to contact her.''

The number of invitees underscored the symbolic nature of the letter. Sigrid Langebrekke, events
manager for the Nobel Peace Prize committee, said prizewinners typically invite no more than 30
guests.

``It could be more than that,'' she said. ``But 143 is quite too many.'' She added that guests must
pay their own costs.

Ma Zhaoxu, a spokesman for the Foreign Ministry, declined to answer questions about Ms.~Liu at
Tuesday's regular news briefing. Asked whether Ms.~Liu would be allowed to travel to Oslo, he
replied testily: ``On what grounds are you asking this question? How do you know that she intends to
go?''

When a second reporter said Ms.~Liu had been invited, Mr.~Ma answered: ``Oh, in that case, you
should ask her, not me.'' He reiterated that China considered Mr.~Liu a criminal and objected to any
attempt to interfere with the country's legal integrity.

Mr.~Yang, a pro-democracy activist who left China in 2007, said Ms.~Liu had invited celebrities like
the film director Chen Kaige, the actor Jiang Wen and the author Han Han because ``they are personal
friends or colleagues in the democracy work.''

On Monday, a group of 15 Nobel Peace Prize laureates, including former President Jimmy Carter and
the South African anti-apartheid leader Desmond Tutu, issued a letter calling on world leaders to
press for Mr.~Liu's release at next month's Group of 20 meeting in South Korea.

Mr.~Yang said: ``We need to continue the effort to try to apply pressure on the Chinese government
to return freedom to Liu Xiaobo and Liu Xia. This is more important than the award ceremony.''

\section{In California, an Encounter 2 Candidates Had Avoided}

\lettrine{W}{ith}\mycalendar{Oct.'10}{27} less than a week until Election Day, Gov.~Arnold
Schwarzenegger of California has become a chunk of kryptonite that the candidates to succeed him lob
at each other. But even as the candidates, Meg Whitman and Jerry Brown, scrambled to distance
themselves from the governor, they shared a stage with him on Tuesday for the first time this
election cycle.

The event was a panel discussion, titled ``Who We Are, Where Are We Going'' and moderated by Matt
Lauer of NBC's ``Today'' show, that was the centerpiece of an annual women's conference organized by
Maria Shriver, Mr.~Schwarzenegger's wife.

The governor entered to a swell of music after a short video that extolled his time in office.
Ms.~Whitman -- who, like Mr.~Schwarzenegger, is a Republican -- and Mr.~Brown, the Democratic
nominee, followed. After perfunctory handshakes, they flanked Mr.~Schwarzenegger around a small
table.

Mr.~Schwarzenegger has not endorsed either candidate, but his image has loomed large in recent days.
The Brown campaign introduced an advertisement featuring sound bites from Mr.~Schwarzenegger spliced
with clips of Ms.~Whitman making nearly identical remarks. In response, Ms.~Whitman said Mr.~Brown's
tax strategies and environmental ideas augured more of what the Schwarzenegger administration had
been dishing up.

The conference program also included speeches by Michelle Obama and Laura Bush, as well as talks
about ``work-life balance'' and weight loss.

When Mr.~Lauer pressed the candidates to ``pull all negative ads and replace them with positive
ads'' in the remaining days of the race, Mr.~Schwarzenegger sat back as Mr.~Brown pledged to comply
and Ms.~Whitman smiled at the cheering audience.

``I will say,'' Ms.~Whitman said after a pause, ``the things I have been called in this campaign,
it's not fair to the voters of California.''

Mr.~Lauer asked for a handshake, a verbal pledge, any clear agreement to his challenge before time
ran out on the panel. Mr.~Schwarzenegger leaned forward and assumed the role of referee. ``Maybe
it's hard to come to an agreement now, in front of an audience,'' he said. ``I can already see the
political spin masters in the back saying, `Why did you do that?' ''

The three then stood for a photograph, posing with hands stiffly at their sides for several seconds,
to a soundtrack of frantic drumming. Mr.~Schwarzenegger reached up and gripped Mr.~Brown's and
Ms.~Whitman's shoulders, pulling their heads close to his big grin.

\section{In Alaska Senate Race, Off the Ballot but on the Rise}

\lettrine{T}{he}\mycalendar{Oct.'10}{27} candidate treated like the front-runner in the Alaska
Senate race is one not actually on the ballot.

Neither Joe Miller nor Scott McAdams, for instance, was invited on stage here at the annual
convention of the Alaska Federation of Natives last week. The only candidate allowed to address the
4,000 in attendance -- and the candidate the federation eventually endorsed -- was the incumbent,
Lisa Murkowski, the Republican now running as a write-in candidate.

``You humble me, you honor me,'' Ms.~Murkowski told the crowd. ``I will fight for you as long as I
am able.''

Just weeks ago, Ms.~Murkowski's bid looked like a long shot. And it still may be -- reliable polls
in Alaska are few and far between.

But since being embarrassed in an upset by Mr.~Miller, a prot\'eg\'e of Sarah Palin's, in the
Republican primary, Ms.~Murkowski has defied conventional wisdom and her colleagues in the
Republican establishment by waging a credible race as a write-in candidate. Analysts and Alaskans
now say she could overcome the odds and logistical hurdles to win, something no senator has done
since Strom Thurmond of South Carolina in 1954. Or she could be a spoiler.

Democrats insist that their nominee, Mr.~McAdams, can pull out a victory in this heavily Republican
state if he can paint Ms.~Murkowski as too conservative, and her write-in campaign as too risky, for
Democrats who might defect to her out of fear of a victory by Mr.~Miller.

The night after the federation conference, it was Mr.~Miller and Mr.~McAdams who appeared together
for a debate at the University of Alaska Fairbanks. Ms.~Murkowski was nowhere to be found, but that
did not stop the other two from attacking her: She is too liberal. No, she is too conservative.

``Maybe we ought to debate Lisa for the rest of the night,'' Mr.~McAdams quipped at one point.
``'What do you think, Joe?'''

A few moments later, when the candidates were supposed to ask each other questions, Mr.~Miller said,
``Scott, I'm tempted just to ask questions about Lisa.''

Later, the moderator inadvertently addressed one of them as ``Scott Miller.''

Ms.~Murkowski has attended most debates, but in a year filled with unconventional races across the
country, hers is among the most unlikely. She has shed her sometimes mechanical public presence and
struck populist notes -- she even sang during a stump speech in Fairbanks last week.

``Fill in the oval, write it on the line,'' the senator sang in a shaky contralto, striving to
create an Election Day anthem out of a supporter's original tune, called ``Cinderella.''

Mr.~Miller remains the presumptive favorite, but his lead has narrowed after a string of setbacks
since his surprising primary victory.

News reports in Alaska have raised questions about everything from farm subsidies, unemployment and
government health care benefits and even a low-income fishing license that Mr.~Miller or his family
members have received. Critics say the reports have undermined his credibility when he argues
against the federal health care bill and unemployment benefits or vows to eliminate the Department
of Education and eventually privatize Social Security.

Over the weekend, after Mr.~Miller refused for weeks to answer questions about disciplinary action
taken against him when he worked as a lawyer for Fairbanks North Star Borough, a judge ordered
records of the incident released as soon as Tuesday. (The ruling also came after Mr.~Miller's
security guards handcuffed Tony Hopfinger, the editor of Alaska Dispatch, an online news site, when
he tried to ask Mr.~Miller about the matter at a campaign event.)

Mr.~Miller may still fight the judge's order, though in a debate in Anchorage on Sunday, he admitted
to being suspended from work for an ethics violation in 2008 for using government computers for
political purposes. He left the job in the summer of 2009.

Questions about transparency have followed the candidate. In addition to his reluctance to discuss
the ethics violation, he has also brushed off the handcuffing of the journalist, saying he played no
role in the incident. Mr.~Miller lives down a series of long gravel roads at the edge of Fairbanks.
Security cameras are positioned to monitor the entrance to his house, which sits out of sight.

Asked about the security cameras in a brief interview, Mr.~Miller initially asked a reporter to
identify who revealed their existence. When the reporter declined to do so, Mr.~Miller noted that he
had once been a federal magistrate judge.

``There were security issues on occasion while I was U.S.~magistrate judge,'' he said.

While Mr.~Miller worries that Ms.~Murkowski will win Republican votes, Democrats hope to cast her as
too conservative.

Alaska's other senator, Mark Begich, a Democrat, who has had several staff members join or volunteer
for the McAdams campaign, noted that Mr.~McAdams, the mayor of Sitka, might need only a third of all
votes to win. Presuming a 60 percent turnout, that is about 100,000 votes.

Underscoring both sides' concerns over the Murkowski campaign, lawyers for state Democrats and
Republicans have joined in a lawsuit accusing the State Division of Elections of improperly
providing lists of write-in candidates to all voting locations and, in at least one polling place,
in Homer, posting the lists inside voting booths.

A letter from the state elections director, Gail Fenumiai, written last week before the lawsuit was
filed, said the lists were intended to be provided only to voters who requested them, not posted
inside booths.

On Monday, after weeks of silence in the race, Ms.~Palin used her Facebook page to attack
Ms.~Murkowski for comments the senator made in a televised debate the night before in Anchorage.
Ms.~Murkowski had raised the subject of Mr.~Miller's military service and questioned whether his
conduct in the campaign was honorable.

``I find it astonishing that a sitting U.S.~senator from Alaska would challenge the honor of a
decorated combat veteran,'' Ms.~Palin wrote.

Ms.~Murkowski is having to position herself carefully.

Asked whether she would do more to win Democratic votes, the senator said in an interview that she
would not change her party. But, she said, ``I've made it very clear that when I go back to
Washington it will be because Alaskans have sent me back, not Republicans.''

She went on to name a range of constituencies she was courting, from libertarians and
environmentalists to Democrats and Republicans. A moment later, just to be safe, an aide leaned in
to clarify that she was indeed still a Republican.

\section{Judge Tells LimeWire, the File-Trading Service, to Disable Its Software}

\lettrine{A}{}\mycalendar{Oct.'10}{27} federal judge in New York issued an injunction on Tuesday
that will essentially shut down LimeWire, the big music file-sharing service that has been mired in
a four-year legal struggle with the music industry. The case has already resulted in the company and
its founder being found liable for potentially hundreds of millions of dollars in damages.

Although LimeWire, the file-sharing service that allows users to swap music that is a major
descendant of Napster, is on the verge of vanishing in its current form, the company will continue
negotiations with the major music companies about a licensing deal to offer music legally for sale
with a subscription service.

``While this is not our ideal path, we hope to work with the music industry in moving forward,'' the
company said in a statement. ``We look forward to embracing necessary changes and collaborating with
the entire music industry in the future.''

In her ruling, Judge Kimba M.~Wood of Federal District Court in Manhattan forced the company to
disable ``searching, downloading, uploading, file trading and/or file distribution functionality''
of the company's file-sharing software.

On Tuesday afternoon, visitors to LimeWire's Web site were greeted with a legal notice and the
words, ``downloading or sharing copyrighted content without authorization is illegal.'' Much of the
site was shutdown, but there was a link to a copy of the injunction.

LimeWire, founded in 2000 by Mark Gorton, a successful Wall Street trader, now appears to be headed
the way of the former Internet pirates Napster and Grokster, both of which lost legal battles
against the music industry.

In a statement Tuesday, the Recording Industry Association of America, the music industry's trade
group that had managed the suit, said: ``For the better part of the last decade, LimeWire and Gorton
have violated the law. The court has now signed an injunction that will start to unwind the massive
piracy machine that LimeWire and Gorton used to enrich themselves immensely.''

The legal fight does not end here. In May, Judge Wood ruled that the company had violated copyright
law and was liable for damages. The court is scheduled to decide early next year the amount the
company and Mr.~Gorton will be forced to pay.

``In January, the court will conduct a trial to determine the appropriate level of damages necessary
to compensate the record companies for the billions and billions of illegal downloads that occurred
through the LimeWire system,'' the recording association said in its statement.

\section{U.S.~and Europe Urged to Join Forces on Rare Earth Metals}

\lettrine{T}{he}\mycalendar{Oct.'10}{27} United States and Europe should work together with industry
to reduce dependence on China for crucial minerals because global trade bodies are ill-equipped to
solve the problem of withheld supplies, officials and executives said Tuesday at a high-level
conference.

Germany has been the most vocal country in Europe in raising concerns about the potential economic
impact of shortfalls of rare earth and other metals, and its economy minister was a featured speaker
at the conference, organized by the Federation of German Industries.

The consensus among attendees, who included representatives from industry as well as the European
Union, the World Trade Organization and the World Bank, was that trade rules were inadequate when it
came to responding to China's decision to cut back export quotas of such materials. China dominates
the mining of these metals, which are crucial for high-tech industries.

``The fact is that the existing rules do not appear to be commensurate to the problem we face,''
said Frank Hoffmeister, deputy chief of cabinet for the European trade commissioner, Karel De Gucht.

Pascal Lamy, general secretary of the World Trade Organization, said the W.T.O. had no authority to
deal with raw materials as such.

However, many broadly written international trade rules can be construed as applying to raw
materials. The United States, the European Union and Mexico brought a W.T.O. case against China last
year asserting that it had violated a ban on export restrictions by limiting exports of bauxite and
a half-dozen other industrial minerals; an initial W.T.O. decision is expected next April.

An American executive at the conference suggested that the United States and Europe work together.

``It is up to us to solve the problem,'' said Gary Litman, the U.S.~Chamber of Commerce's vice
president for Europe and Eurasia. ``The U.S.~has the resources and together with our E.U. partners
we can produce anything we want.''

German companies complained that the lack of rules gave China free rein to do what it liked with its
rare earth minerals. At the same time, flush with cash, China faces few constraints in access to
other raw materials to feed its expanding economy.

The case of iron ore was cited by Edwin Eichler, chief executive of ThyssenKrupp Steel Europe, one
of the Continent's biggest steel producers.

``Iron ore prices today have risen 100 percent compared to last April. Why? The 50-year-old
benchmark that fixed prices for one year at a time has now been changed to a quarterly basis based
on the Chinese index,'' Mr.~Eichler said. ``So we face higher prices and reduced availability.''

Other industry chiefs echoed complaints that China was hinting that if industries wanted access to
its rare earth minerals, then they should invest in China.

``Rare earth exports from China could fall even further, even by 30 percent next year,'' said Ulrich
Grillo, the chief executive of Grillo-Werke, a chemicals company. ``So what the Chinese are telling
us, directly or indirectly, is that if we want access to their rare earth material metals we should
invest in China.''

He added that Western companies like his were reluctant to do so for fear that they would lose
control of their intellectual property rights.

``China needs our know-how and machines, but we need protection for our technology,'' he said. ``We
don't want these problems to turn into China-bashing.''

The German government has asked France, which takes over the Group of 20 presidency next month, to
put raw materials at the top of its agenda. The economics minister, Rainer Brüderle, said he had
received assurances from Christine Lagarde, the French finance minister, that that would happen.

\section{Rating Agency Raises Outlook for Britain After Cuts}

\lettrine{B}{ritain}\mycalendar{Oct.'10}{27}'s deficit-cutting plans got an important vote of
confidence on Tuesday, as Standard \& Poor's said it raised its outlook for the country's debt and
affirmed its triple-A credit rating.

The decision came after the Office for National Statistics reported that the British gross domestic
product grew more rapidly in the third quarter than many economists had expected. It expanded 0.8
percent from the previous quarter as construction posted solid growth, reducing fears that budget
austerity would arrive just as the economy was threatening to fall into a double-dip recession.

The recently elected British coalition parties ``have shown a high degree of cohesion in putting the
U.K.'s public finances onto what we view to be a more sustainable footing,'' the S.\& P.~analysts,
led by Trevor Cullinan, said in a statement. They said the budget deficit -- at about 11.2 percent
of gross domestic product in 2009 -- could narrow to about 3 percent in 2014.

The Conservative-Liberal Democrat coalition government of Prime Minister David Cameron is planning
to cut almost 500,000 jobs and to curtail welfare spending to bring the public finances into line.
Officials are hoping that the private sector will be able to pick up the slack, avoiding another
economic downturn. The gross domestic product report suggested that the expansion might be strong
enough to keep the labor market growing.

Economic growth in the third quarter, at 0.8 percent, was slower than the 1.2 percent rate for the
April to June period, but was double the 0.4 percent expansion economists surveyed by Reuters and
Bloomberg expected. Construction grew 4 percent in the quarter, the Office for National Statistics
said in its preliminary estimate, and was ``widespread'' over all.

When compared with the third quarter a year ago, Britain's economy expanded 2.8 percent.

A separate report showed the services index growing 2.7 percent in August from a year earlier, well
above the approximately 0.6 percent growth economists had expected.

The surprisingly strong economic reports should reduce pressure on the Bank of England's governor,
Mervyn A.~King, to further expand a policy of quantitative easing -- essentially buying assets to
flood the market with liquidity -- a move to offset the economic effect of government budget cuts.
The central bank has set its main policy rate at 0.5 percent and bought assets valued at
\textsterling200 billion (\$317 billion) to shore up growth.

Despite the stronger-than-expected data, economists were cautious.

Howard Archer, an economist at IHS Global Insight in London, described the G.D.P. report as ``a
major upside surprise'' but said growth nonetheless ``will be markedly slower going forward,'' as
the government's fiscal tightening takes effect, credit conditions remain tight and global growth
stays weak.

The data will provide ``pleasant reading'' for Mr.~Cameron, Benjamin Williamson, an economist at the
Center for Economics and Business Research, wrote in a note.

``However, they will also add greater uncertainty to the Bank of England's monetary policy decisions
to be made next week,'' Mr.~Williamson wrote.

The data from Britain on Tuesday was echoed by reports elsewhere in Europe, showing that confidence
remained surprisingly steady despite fears that growth could stumble in the near term.

French consumer confidence improved to minus 34 in October from minus 35 in September, according to
Insee, the French statistical body.

The GfK Group, a market research firm, said German consumer sentiment was ``stable'' in October,
while surveys of economic expectations among consumers and businesses improved.

\section{Top Aide to Saddam Hussein Is Sentenced to Death}

\lettrine{T}{ariq}\mycalendar{Oct.'10}{27} Aziz, a former top aide to Saddam Hussein and his urbane
public relations representative to the world, was sentenced to death by an Iraqi court on Tuesday,
convicted of crimes against members of rival Shiite political parties.

The sentence was handed down in the latest in a series of criminal cases brought against Mr.~Aziz,
74, and other top figures in what had been Mr.~Hussein's government.

For years, Mr.~Aziz, a former foreign minister and deputy prime minister, served as the bespectacled
face of that government, a cigar-smoking emissary who sought to justify, in fluent English, Iraq's
use of chemical weapons, invasion of oil-rich Kuwait, and killings of Shiites and Kurds.

Because Mr.~Hussein rarely left Iraq out of fears for his safety, Mr.~Aziz often represented Iraq at
the United Nations and other global settings, serving as a public defender of Mr.~Hussein before the
American-led invasion of 2003.

Mr.~Aziz surrendered to American forces shortly after the invasion, aware that, for Americans, he
was among Iraq's most hunted officials and one of the best-known emblems of the Hussein era. He was
handed over to Iraqi jailers this year as part of the United States transfer of security powers to
Iraq as it withdrew its last combat troops.

Mr.~Aziz's death sentence followed convictions on charges of persecution against members of the
religious Shiite Dawa Party, whose members include Iraq's current prime minister, Nuri Kamal
al-Maliki.

It was unclear when Mr.~Aziz would be executed, if ever.

One of Mr.~Aziz's lawyers, Badea Araf Azzit, said he was considering whether to appeal. He dismissed
the sentence as a ploy aimed at distracting attention from Iraq's political stalemate and the recent
publication of a trove of American war records that described widespread prisoner abuse by Iraqi
guards and security forces.

``It is a political judgment,'' Mr.~Azzit said.

Mr.~Aziz's lawyers have long claimed that he was responsible only for Iraq's diplomatic and
political relations, and that he had no ties to the executions and purges carried out by
Mr.~Hussein's government. Mr.~Hussein was hanged in 2006, less than two months after his death
sentence was handed down.

In a telephone interview, Mr.~Aziz's son Ziad, 44, said he believed that his father was blindsided
by the news. When they spoke by telephone three days earlier, Tariq Aziz asked his son, who lives in
Jordan, to send him clothes, food and medicine, and he did not mention that sentencing was imminent.

``We don't know the next step,'' Ziad Aziz said. ``We have no chance of protecting him.''

Ziad Aziz said his father remained in poor health. In January, the American military said in a
statement that he had a blood clot in the brain. He was taken to an American military hospital north
of Baghdad for treatment.

In March 2009, Mr.~Aziz was sentenced to 15 years in prison for crimes against humanity, but he was
acquitted earlier that month on charges of ordering a 1999 crackdown against Shiite protesters after
a revered Shiite cleric was assassinated. He is also serving a seven-year prison sentence for a case
involving the forced displacement of Kurds in northern Iraq.

Death sentences were also handed down on Tuesday against other former officials in Mr.~Hussein's
government including Abed Hammoud, a former secretary to Mr.~Hussein, and former Interior Minister
Sadoun Shakir.

In other developments in Iraq on Tuesday:

¶ Gunmen in the northern city of Kirkuk, armed with hand grenades and rocket-propelled grenades,
staged an audacious evening robbery on the city's largest gold market, killing 10 people -- at least
6 of them police officers -- and wounding 15 others. It was part of a string of deadly and
coordinated robberies against Iraqi merchants.

¶ Further north, the political gamesmanship over who will control Iraq's new government continued.
Former Prime Minister Ayad Allawi traveled to Erbil to meet with the Kurdish region's president,
Massoud Barzani, whose support is likely to be critical in breaking the deadlock.

The Kurds emerged as political kingmakers after March elections failed to produce a clear winner,
precipitating a deadlock between Mr.~Maliki's Shiite coalition and the group led by Mr.~Allawi,
which narrowly won the most seats in Parliament. Mr.~Maliki, who has remained acting prime minister
during the deadlock, now appears likely to keep his job after winning the support of a bloc led by
Moktada al-Sadr, a Shiite cleric known for his anti-American bent and intimate ties to Iran.

\section{E-Mail Spam Falls After Russian Crackdown}

\lettrine{Y}{ou}\mycalendar{Oct.'10}{27} may not have noticed, but since late last month, the world
supply of Viagra ads and other e-mail spam has dropped by an estimated one-fifth. With 200 billion
spam messages in circulation each day, there is still plenty to go around.

But police officials in Russia, a major spam exporter, say they are trying to do their part to stem
the flow. On Tuesday, police officials here announced a criminal investigation of a suspected spam
kingpin, Igor A.~Gusev. They said he had probably fled the country.

Moscow police authorities said Mr.~Gusev, 31, was a central figure in the operations of SpamIt.com,
which paid spammers to promote online pharmacies, sometimes quite lewdly. SpamIt.com suddenly
stopped operating on Sept.~27. With less financial incentive to send their junk mail, spammers
curtailed their activity by an estimated 50 billion messages a day.

Why the site closed was unclear until Tuesday, when Moscow police officials met with reporters to
discuss the Gusev case. The officials' actions were a departure from Russia's usual laissez faire
approach to online crime.

They accuse Mr.~Gusev of operating a pharmacy without a license and of failing to register a
business. On Tuesday, they searched his apartment and office in Moscow, according to Lt.~Yevdokiya
F.~Utenkova, an investigator in the economic crime division of the Moscow police department.

Lieutenant Utenkova said the search of the apartment turned up seven removable hard drives, four
flash cards and three laptops. Specific, computer-crime related charges may follow after police
examine their contents, she said. The investigation began Sept.~21, six days before SpamIt.com
closed.

Mr.~Gusev's lawyer, Vadim A.~Kolosov, said in a telephone interview that his client was not the
owner of SpamIt.com and had never sent spam e-mail, but declined to respond to specific questions.

The drop-off in spam since SpamIt.com went down had been noted by companies in the United States
that monitor the Internet.

``We've seen a sustained drop in global volumes,'' Henry Stern, a senior security analyst at Cisco
Systems, said in a telephone interview from San Francisco. The company pinpointed the closure of
Mr.~Gusev's site as the cause for this easing up.

If individual computer users have not noticed changes in spam traffic, it may be because many people
have learned to use spam filters that insulate them from the junk that continuously circulates on
the Internet.

Kaspersky Lab, an antivirus company based in Moscow, said there had been a notable drop in mass
e-mail in the United States that advertised prescription drugs -- to about 41 percent of all spam at
the end of the September from 65 percent at the beginning of the month. The figures are comparable
in Western Europe, the company said. Many of the pharmaceuticals sold through Web sites promoted by
spammers are believed to be counterfeit.

Other computer security companies had reported similar reductions in prescription drug spam,
although they cautioned that spam volumes were volatile and often spring back to previous high
levels. On a typical day, spam accounts for about 90 percent of all e-mail traffic on the Internet.

Mr.~Gusev and SpamIt.com have been widely known in computer security circles, and he had lived
openly in Moscow. Spamhaus, an international nonprofit that monitors global spam, listed the
SpamIt.com organization as the world's single largest sponsor of spam.

Last year, the Russian-language version of Newsweek reported that Mr.~Gusev's sites were connected
to the same computer server farm in St.~Petersburg, Russia, called Russian Business Networks, that
was identified in a 2009 report by online security experts with NATO as a source of the attacks on
Georgia in 2008.

Mr.~Gusev filed suit against Newsweek in a Moscow court, denying links to spamming suggested in the
article. That case is still pending. In that suit, he cited phone calls from The New York Times to
his lawyer seeking comment as evidence that the article harmed his reputation.

Why, after years of ignoring spammers, Russian authorities have now acted has left online security
experts puzzled.

SpamIt.com had operated in a gray area of Russian law, cybersecurity researchers said. They said it
had paid commissions to other parties that had directed traffic to various sites operating under the
name Canadian Pharmacy, using a Russian online settlement system. Mr.~Gusev has denied in blog posts
that he promoted spam.

The spammers, meanwhile, operated entirely in the shadows, using networks of computers that had been
remotely infected with viruses, known as botnets, and turning them into relay stations for sending
e-mail from anywhere in the world.

Some American security experts have said that the spamming operation in Russia appears to have been
protected by Russian authorities -- whether for reasons of corruption, national pride or state
security.

Because most victims of online crime, and the targets of unwanted spam advertising, are in Europe
and the United States, Russian police have typically seen little incentive to prosecute online
crime, analysts say.

But recently, President Dmitri A.~Medvedev of Russia has been seeking to expand and legitimize the
domestic Russian Internet industry -- and move it away from its reputation as a playground for
hackers, pornographers and authors of darkly ingenious viruses.

In June, Mr.~Medvedev visited California to meet with Silicon Valley executives. The SpamIt.com site
closed two weeks before the reciprocal Silicon Valley trade delegation, led by Gov.~Arnold
Schwarzenegger of California, arrived in Moscow on Oct.~10.

Computer security researchers have conjectured that spamming gangs have sometimes been co-opted by
the intelligence agencies in Russia, which provide cover for the spamming activities in exchange for
the criminals' expertise or for allowing their networks of virus-infected computers to be used for
political purposes -- to crash dissident Web sites, for example, or to foster attacks on foreign
adversaries.

The Russian government has denied orchestrating computer attacks beyond its borders.

\section{Gorbachev Says Putin Obstructs Democracy}

\lettrine{M}{ikhail}\mycalendar{Oct.'10}{27} S.~Gorbachev, who once supported Prime Minister
Vladimir V.~Putin, is voicing growing frustration with Mr.~Putin's leadership, saying that he had
undermined Russia's fledgling democracy by crippling the opposition forces.

``He thinks that democracy stands in his way,'' Mr.~Gorbachev said.

``I am afraid that they have been saddled with this idea that this unmanageable country needs
authoritarianism,'' Mr.~Gorbachev said, referring to Mr.~Putin and his close ally, President Dmitri
A.~Medvedev. ``They think they cannot do without it.''

In an interview, Mr.~Gorbachev even described Mr.~Putin's governing party, United Russia, as a ``a
bad copy of the Soviet Communist Party.'' Mr.~Gorbachev said party officials were concerned entirely
with clinging to power and did not want Russians to take part in civic life.

Mr.~Gorbachev was especially disparaging of Mr.~Putin's decision in 2004, when he was president, to
eliminate elections for regional governors and the mayors of Moscow and St.~Petersburg. Those
positions are now filled by Kremlin appointees. The impact of this change was illustrated in
Mr.~Medvedev's dismissal last month of Moscow's longtime mayor, who was replaced with a Putin
loyalist.

``Democracy begins with elections,'' Mr.~Gorbachev said. ``Elections, accountability and turnover.''

Mr.~Gorbachev, the last Soviet leader, was giving interviews this month to promote a benefit concert
that his foundation is sponsoring in March in honor of his 80th birthday. The foundation runs a
research center and has raised millions of dollars for charities for children with cancer.

Mr.~Gorbachev's criticism of Mr.~Putin, while not new, appears to have grown somewhat sharper
recently, as if Mr.~Gorbachev feels that he put Russia on the path toward being a functional
democracy, only to have Mr.~Putin block its progress.

Neither Mr.~Putin nor Mr.~Medvedev has responded publicly to Mr.~Gorbachev. Asked on Tuesday about
Mr.~Gorbachev's comments, Mr.~Putin's spokesman, Dmitri S.~Peskov, seemed to choose his words
carefully. ``We do feel the deepest respect toward Mikhail Gorbachev, and we certainly respect his
point of view,'' Mr.~Peskov said. ``But that doesn't mean that we agree with it.''

Mr.~Peskov said opposition parties had failed to make gains in Russia because their leaders were
unpopular and had not developed attractive platforms. ``Neither Putin personally nor United Russia
as a political party can be held responsible for the inability of other parties to produce anything
promising for citizens of this country,'' he said.

Nursing a sore throat, Mr.~Gorbachev spoke with vigor and seemed hardly slowed by age. He met with
reporters at his foundation headquarters in Moscow, which is filled with hundreds of photographs and
other memorabilia that highlight his efforts to reform the Soviet Union.

Still, nearly two decades after the Soviet collapse, Mr.~Gorbachev occupies an awkward place in
Russian society. He is arguably more respected abroad than at home, in part because some here blame
him for ushering in the political and economic chaos of the 1990s. It is notable that the benefit
concert for his foundation will take place at the Royal Albert Hall in London, not in Moscow.

Mr.~Gorbachev, who oversaw the Soviet withdrawal from Afghanistan, offered his observations about
the current NATO mission in that country, saying that success there was impossible for an occupier.
``It would be necessary to exterminate people,'' he said, emphasizing that that was obviously not an
option.

Mr.~Gorbachev has recently dabbled in opposition politics. He is part owner of the country's leading
opposition newspaper, Novaya Gazeta, several of whose reporters have been killed or wounded in
attacks. He tried to help form a political party to compete in parliamentary elections next year,
but gave up in the face of daunting legal hurdles.

``For those who want to change the country in order to advance these processes faster, advance
democratic processes, the participation of people is needed,'' he said. ``It is necessary to have
parties. But go and try to register a party!''

Mr.~Gorbachev would not say whom he would endorse for president in 2012. Mr.~Putin, who became prime
minister in 2008 after he was barred from running for a third consecutive term as president, is
thought to be weighing a return to the presidency.

``Russia has a long way to go to usher in a new system of values, to create and provide for the
proper functioning of the institutions and mechanisms of democracy -- the institutions of civil
society,'' Mr.~Gorbachev said. ``All this is done through a major transformation in people's brains.
And this, clearly, is changing very slowly.''


\end{document}
