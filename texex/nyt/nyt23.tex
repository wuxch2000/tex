\documentclass[12pt]{article}
\title{Digest of The New York Times}
\author{The New York Times}

\usepackage{config}

% \makeindex
\begin{document}
\date{}
\thispagestyle{empty}

\begin{figure}
\includegraphics*[width=0.3\textwidth]{The_New_York_Times_logo.png}
\vspace{-20ex}
\end{figure}
% \renewcommand\contentsname{\textsf{Digest of The New York Times}}
\renewcommand\contentsname{}
{\footnotesize\textsf{\tableofcontents}}

\clearpage
\setcounter{page}{1}

% \begin{multicols}{2}

\pagebreak
\section{China Fortifies State Businesses to Fuel Growth}

\lettrine{D}{uring}\mycalendar{Aug.'10}{30} its decades of rapid growth, China thrived by allowing
once-suppressed private entrepreneurs to prosper, often at the expense of the old, inefficient state
sector of the economy.

Now, whether in the coal-rich regions of Shanxi Province, the steel mills of the northern industrial
heartland, or the airlines flying overhead, it is often China's state-run companies that are on the
march.

As the Chinese government has grown richer -- and more worried about sustaining its high-octane
growth -- it has pumped public funds into companies that it expects to upgrade the industrial base
and employ more people. The beneficiaries are state-owned interests that many analysts had assumed
would gradually wither away in the face of private-sector competition.

New data from the World Bank show that the proportion of industrial production by companies
controlled by the Chinese state edged up last year, checking a slow but seemingly inevitable
eclipse. Moreover, investment by state-controlled companies skyrocketed, driven by hundreds of
billions of dollars of government spending and state bank lending to combat the global financial
crisis.

They join a string of other signals that are fueling discussion among analysts about whether China,
which calls itself socialist but is often thought of in the West as brutally capitalist, is in fact
seeking to enhance government control over some parts of the economy.

The distinction may matter more today than it once did. China surpassed Japan to become the world's
second-largest economy this year, and its state-directed development model is enormously appealing
to poor countries. Even in the West, many admire China's ability to build a first-world
infrastructure and transform its cities into showpieces.

Once eager to learn from the United States, China's leaders during the financial crisis have
reaffirmed their faith in their own more statist approach to economic management, in which private
capitalism plays only a supporting role.

``The socialist system's advantages,'' Prime Minister Wen Jiabao said in a March address, ``enable
us to make decisions efficiently, organize effectively and concentrate resources to accomplish large
undertakings.''

\textbf{State vs. Private}

The issue of state versus private control is a slippery one in China. After decades of economic
reform, many big state-owned companies face real competition and are expected to operate profitably.
The biggest private companies often get their funding from state banks, coordinate their investments
with the government and seat their chief executives on government advisory panels.

Chinese leaders also no longer publicly emphasize sharp ideological distinctions about ownership.
But they never relaxed state control over some sectors considered strategically vital, including
finance, defense, energy, telecommunications, railways and ports.

Mr.~Wen and President Hu Jintao are also seen as less attuned to the interests of foreign investors
and China's own private sector than the earlier generation of leaders who pioneered economic
reforms. They prefer to enhance the clout and economic reach of state-backed companies at the top of
the pecking order.

``China's always had a major industrial policy. But for a space of a few years, it looked like China
was turning away from an active and interventionist industrial policy in favor of a more hands-off
approach,'' Victor Shih, a Northwestern University political scientist, said in a recent telephone
interview.

Mr.~Shih, among others, now believes that the 1980s reforms that unleashed China's private sector
and the 1990s reforms that dismantled great sections of the state-run sector are being partly
undone.

``The problem is that the reforms of the first 20 years, from 1978 to the end of the '90s, actually
did not touch on the power of the government,'' said Yao Yang, a Peking University professor who
heads the China Center for Economic Research. ``So after the other reforms were finished, you
actually find the government is expanding, because there is no check and balance on its power.''

\textbf{Divining Government's Role}

There are no comprehensive statistics to catalog the government's influence over the economy. So the
shift is partly inferred from coarse measures like the share of financing in the economy provided by
state banks, which rose sharply during the financial crisis, or the list of the 100 largest publicly
listed Chinese companies, all but one of which are majority state owned.

The statistic showing an uptick in the share of industrial production attributable to the state
sector is regarded by some analysts as a blip rather than the start of a trend. The World Bank's
senior economist in Beijing, Louis Kuijs, said the state sector's unusually rapid growth will most
likely moderate with the ending of the government's stimulus spending.

``As the growth process normalizes again, the traditional trend toward a declining SOE share will
take over again,'' he wrote in an e-mail message, using the shorthand for state-owned enterprise.
``I don't think that the senior leaders had a strategy of reversing this trend.''

But others argue that officials had always intended to create a vibrant state sector that would
tower above the private sector in important industries, even as they sold off or shut down
money-losing state enterprises that drained capital from the government budget and banking system.

Recent alarm over the expanding role of the state, said Arthur Kroeber of Dragonomics, an economic
forecasting firm based in Beijing, is mostly ``perception catching up with reality.''

In some ways, the differences in this debate are small. Everyone agrees that China runs a bifurcated
economy: at one level, a robust and competitive private sector dominates industries like
factory-assembled exports, clothing and food. And at higher levels like finance, communications,
transportation, mining and metals -- the so-called commanding heights -- the central government
claims majority ownership and a measure of management control.

Yet the two camps' view of China's future are markedly different. Those who see little evidence of
an expanding state sector generally believe that China has a decade or more of robust growth
awaiting it before its economy matures. Theirs is a Goldilocks view of state intervention -- not too
much or too little, but just enough to push a developing economy toward prosperity.

The skeptics have a darker view: they believe distortions and waste, in no small part due to
government meddling, have resulted in gross misallocation of capital and will end up pushing growth
rates down well before 2020. What drives their pessimism, the skeptics say, is that China, like
Japan a generation ago, has too much confidence in a top-down economic strategy that defies
conventional Western theory.

The skeptics also point to what they say is the growing political and financial influence of China's
state-owned giants -- 129 huge conglomerates that answer directly to the central government, and
thousands of smaller ones run by the provinces and cities.

While no public breakdown exists, most experts say the vast bulk of the 4 trillion renminbi (\$588
billion) stimulus package that China pumped out for new highways, railroads and other big projects
went to state-owned companies. Some of the largest companies used the flood of money to strengthen
their dominance in their current markets or to enter new ones.

In the last year or so, many of the 129 central government companies have moved forcefully into
China's real-estate industry, with hundreds of billions of dollars in construction projects and land
deals. State-owned steel giants have cut deals to buy out more profitable and often more efficient
private competitors. A host of government conglomerates have snapped up coal mining companies in
Shanxi Province.

``In 2009, there was a huge expansion of the government role in the corporate sector,'' Huang
Yasheng, a leading analyst of China-style capitalism at the Massachusetts Institute of Technology,
said in a telephone interview. ``They're producing yogurt. They're into real estate. Some of the
upstream state-owned enterprises are now expanding downstream, organizing themselves as vertical
units. They're just operating on a much larger scale.''

\textbf{Local Interests}

At the local level, governments set up 8,000 state-owned investment companies in 2009 alone to
channel government dollars into business and industrial ventures, Mr.~Huang said. One example
suffices: a private Chinese automaker, Zhejiang Geely Holding Group, made worldwide headlines in
March when it agreed to buy Sweden's Volvo marque from Ford. Much of the \$1.5 billion purchase
price came not from Geely's relatively modest profits, but from local governments in northeast China
and the Shanghai area.

Geely reciprocated this month, announcing that it will build its Volvo headquarters and an assembly
plant in a Shanghai industrial district.

The reasons for the state's push for greater involvement in business vary. State control of energy
supplies is crucial to China's growth, and the Shanxi coal takeovers will increase production,
guarantee fuel to some state-owned utilities and give Beijing new power to control coal prices.
State mining companies also argue that they have a superior safety record to their accident-prone
private competitors.

But in other areas the state looks more mercenary.

Take telecommunications. Upon joining the World Trade Organization, China committed itself to
opening its communications market to foreign joint ventures for local and international phone
service, e-mail, paging and other businesses. But after eight years, no licenses have been granted
-- largely, the United States says, because capital requirements, regulatory hurdles and other
barriers have made such ventures impractical. Today, basic telecommunications in China are booming,
and are virtually 100 percent state-controlled.

Take the passenger airline industry. Six years ago, the central government invited private investors
to enter the business. By 2006, eight private carriers had sprung up to challenge the three
state-controlled majors, Air China, China Southern and China Eastern.

The state airlines immediately began a price war. The state-owned monopoly that provided jet fuel
refused to service private carriers on the same generous terms given the big three. China's only
computerized reservation system -- currently one-third owned by the three state airlines -- refused
to book flights for private competitors. And when mismanagement and the 2008 economic crisis drove
the three majors into financial straits, the central government bought stock to bail them out: about
\$1 billion for China Eastern; \$430 million for China Southern; \$220 million for Air China.

One private passenger carrier that remains is Spring Airlines, a tenacious startup run by a founder
so frugal that he shares a 100-square-foot office with his chief executive and takes the subway to
business meetings.

That founder, Wang Zhenghua, survived in part by building his own computer reservation system. He
canceled a planned interview. But in Chinese news reports, he was caustic about the state subsidies
given his competitors. ``Now with the injection of 10 billion yuan'' for China Eastern and China
Southern, ``everything is in chaos,'' he told Biz Review, a Chinese magazine.

China's private entrepreneurs have a catchphrase for such maneuvers: ``guo jin, min tui,'' or ``the
state advances, the private sector retreats.''

State-owned enterprises in China have taken the best of the economy for themselves, ``leaving the
private sector drinking the soup while the state enterprises are eating the meat,'' Cai Hua, the
vice director of a chamber-of-commerce-style organization in Zhejiang Province, said in an
interview.

\textbf{First in Line}

Mr.~Cai said he believed that China needs government-run industries to compete globally and manage
the country's domestic development. But locally, he said, their advantages -- being first in line
for financing by state banks, first in line for state bailouts when they get in trouble, first in
line for the stimulus gusher -- have created a ``profound inequality'' with private competitors.

Some analysts argue that the state-owned conglomerates, built with state money and favors into
global competitors, have now become political power centers in their own right, able to fend off
even Beijing's efforts to rein them in.

Of the 129 major state enterprises, more than half the chairmen and chairwomen and more than
one-third of the chief executive officers were appointed by the central organization department of
the Communist Party. A score or more serve on the party's Central Committee, which elects the ruling
Politburo. They control not just the lifeblood of China's economy, but a corporate patronage system
that dispenses top-paying executive jobs to relatives of the party's leading lights.

China's leaders have sought occasionally in the past year to curb speculative excesses by
state-controlled businesses in real estate, lending and other areas. In May the State Council, a
top-level policy body sometimes likened to the cabinet in the United States, issued orders to give
private companies a better shot at government contracts -- for roads and bridges, finance and even
military work -- that now go almost exclusively to state-owned companies. Virtually the same rules
were issued five years ago, to little effect.

Yet it is hard to argue with success, other economists say, and China's success speaks well of its
top-down strategy. Asian powerhouses like South Korea and Japan built their modern economies with
strong state help. Many economists agree that shrewd state management can be better than market
forces in getting a developing nation on its feet.

Experts on both sides of the debate have but two questions. One is how much longer state control of
vast areas of the economy will generate that growth.

The other is whether, should that strategy stop working, China will be able to change it.

\pagebreak
\section{U.S.~Commander Fears Political Stalemate in Iraq}

\lettrine{T}{he}\mycalendar{Aug.'10}{30} departing commander of American forces in Iraq, Gen.~Ray
Odierno, said Sunday that a new Iraqi government could still be two months away and warned that a
stalemate beyond that could create demands for a new election to break the deadlock, which has
lasted since March.

While General Odierno said he believed negotiations had picked up and would prove successful, he
predicted politicians still needed ``four to six to eight weeks.''

``That's a guess,'' he said in an interview at his headquarters, whose plaster roof is still
engraved with the initials of Saddam Hussein. ``If it goes beyond 1 October, what does that mean?
Could there be a call for another election? I worry about that a little bit.''

The prospect of another election would probably throw Iraq's already turbulent politics into even
greater turmoil as the United States begins withdrawing its last 50,000 troops, scheduled to be out
by the end of 2011. While the election in March was viewed as successful, the periods before and
after included bitter disputes over disqualifications, recounts, legal challenges and score-settling
that exacerbated still smoldering sectarian tensions.

Even the suggestion of a new election underscored the ambiguity in an anxious and unsettled Iraq
these days. President Obama plans a speech from the Oval Office on Tuesday to address what the
administration describes as the end of combat operations here. But the date has largely gone
unnoticed by Iraqis, whose frustration with the political deadlock has mirrored their deepening
anger over a dysfunctional government and the shoddy delivery of the basic necessities of life.

``The longer that takes, the more frustrated they might get with the process itself,'' General
Odierno said. ``What I don't want is for them to lose faith in the system, the democratic system,
and that's the long-term risk, do they lose faith in the process.''

American officials had hoped to have a new government in place long before the Tuesday deadline, set
by the Obama administration, to bring troop levels down to 50,000. But an effort to form a coalition
between the two winning blocs -- one led by Prime Minister Nuri Kamal al-Maliki, the other by Ayad
Allawi, a former interim prime minister -- has proceeded only haltingly. Powerful groups within
Iraq, and some of the country's neighbors, remain deeply opposed to either man leading the country.

Even some Iraqi officials acknowledge there is little urgency in the talks, particularly during the
Muslim holy month of Ramadan, which began this month. The prospect of an election, however remote,
comes amid worries that insurgents might stage high-profile attacks to create havoc, that Iran has
stepped up support for militant groups and that Iraqi military leaders might be tempted to take
matters into their own hands.

``If we get the government formed, I think we're O.K.,'' General Odierno said. ``If we don't, I
don't know.''

General Odierno will leave the country Wednesday, after formally turning over command to
Lt.~Gen.~Lloyd J.~Austin III. For General Odierno, the ceremony will mark the end of more than four
years in Iraq, where he has gone from a commander criticized for heavy-handed operations in regions
that became insurgent strongholds to a figure praised, in military circles and beyond, for
overseeing the American military buildup here in 2007.

But part of his legacy may rest on the performance of Iraq's security forces in taking sole control
of the country. American officials have conceded they were dealt a blow last week by a wave of car
bombings, roadside mines and hit-and-run attacks that insurgents unleashed in at least 13 towns and
cities across Iraq.

The attacks were ``not unexpected,'' General Odierno said. ``What I would tell you surprised me a
little bit was that they were able to do it over the country with some coordination.''

In his four years here, General Odierno was often at the center of shifting American military
strategy in Iraq. He said the military learned lessons ``the hard way.''

``We all came in very na\"ive about Iraq,'' he said.

``We came in na\"ive about what the problems were in Iraq; I don't think we understood what I call
the societal devastation that occurred,'' he said, citing the Iran-Iraq war, the Persian Gulf war
and the international sanctions from 1990 to 2003 that wiped out the middle class. ``And then we
attacked to overthrow the government,'' he said.

The same went for the country's ethnic and sectarian divisions, he said: ``We just didn't understand
it.''

To advocates of the counterinsurgency strategy that General Odierno has, in part, come to symbolize,
the learning curve might highlight the military's adaptiveness. Critics of a conflict that killed an
estimated 100,000 Iraqis, perhaps far more, and more than 4,400 American soldiers might see the
acknowledgment as evidence of the war's folly.

Asked if the United States had made the country's divisions worse, General Odierno said, ``I don't
know.''

``There's all these issues that we didn't understand and that we had to work our way through,'' he
said. ``And did maybe that cause it to get worse? Maybe.''

\pagebreak
\section{Upstarts Chip Away at Power of Pakistani Elite}

\lettrine{I}{n}\mycalendar{Aug.'10}{30} Pakistan, where politics has long been a matter of pedigree,
Jamshed Dasti is a mongrel. The scrappy son of an amateur wrestler, Mr.~Dasti has clawed his way
into Pakistan's Parliament, beating the wealthy, landed families who have ruled here.

In elite circles, Mr.~Dasti is reviled as a thug, a small-time hustler with a fake college degree
who represents the worst of Pakistan today. But here, he is hailed as a hero, living proof that in
Pakistan, a poor man can get a seat at the rich men's table.

Mr.~Dasti's rise is part of a broad shift in political power in Pakistan. For generations, politics
took place in the parlors of a handful of rich families, a Westernized elite that owned large tracts
of land and sometimes even the people who worked it. But Pakistan is urbanizing fast, and powerful
forces of change are chipping away at the landed aristocracy, known in Pakistan as the feudal class.

The result is a changing political landscape more representative of Pakistani society, but far less
predictable for the United States. Mr.~Dasti, 32, speaks no English. His legislative record includes
opposition to a sexual harassment bill. He has 35 criminal cases to his name and is from the
country's conservative heartland, where dislike of America runs deep.

How this plays out is crucial to Pakistan's future. The country's fast-expanding, flood-weary
population needs local government as never before, but with political power shifting and
institutions stillborn, the state has never been less able to provide it.

``You have scarcity arising everywhere,'' said Ali Cheema, chairman of the economics department at
the Lahore University of Management and Science. ``Scarcity creates conflict. Conflict needs
mediation. But the state is unable to do it.''

In Mr.~Dasti's area, one of the hardest hit by the recent flooding, the state has all but
disappeared. Not that it was ever very present. In the British colonial era, before Pakistan became
a separate country, the state would show up a few times a month in the form of a representative from
the Raj dispensing justice.

Later, the local landowner took over. For years, feudal lords reigned supreme, serving as the
police, the judge and the political leader. Plantations had jails, and political seats were
practically owned by families.

Instead of midwifing democracy, these aristocrats obstructed it, ignoring the needs of rural
Pakistanis, half of whom are still landless and desperately poor more than 60 years after Pakistan
became a state.

But changes began to erode the aristocrats' power. Cities sprouted, with jobs in construction and
industry. Large-scale farms eclipsed old-fashioned plantations. Vast hereditary lands splintered
among generations of sons, and many aristocratic families left the country for cities, living beyond
their means off sales of their remaining lands. Mobile labor has also reduced dependence on
aristocratic families.

In Punjab, the country's most populous province, and its most economically advanced, the number of
national lawmakers from feudal families shrank to 25 percent in 2008 from 42 percent in 1970,
according to a count conducted by Mubashir Hassan, a former finance minister, and The New York
Times.

``Feudals are a dying breed,'' said S.~Akbar Zaidi, a Karachi-based fellow with the Carnegie
Foundation. ``They have no power outside the walls of their castles.''

Mr.~Dasti, a young, impulsive man with a troubled past, is much like the new Pakistan he represents.
He is one of seven siblings born to illiterate parents. Despite his claims of finishing college, he
never earned a degree, something his political opponents used against him in court this spring. One
of the 35 criminal cases against him is for murder, a charge he said was leveled by his political
opponents. Detractors accuse him of blackmailing rich people in a job at a newspaper. He said he was
writing expos\'es.

``I have more enemies than numbers of hairs in my head,'' he said, bouncing down a road in a
borrowed truck. ``They don't like my style, and I don't like theirs.''

Whatever the case, he is deeply appealing to Pakistanis, who have chosen him over feudal lords for
political seats several times. Local residents call him Rescue One-Five, a reference to an emergency
hot line number and his feverish work habits. Constituents clutching dirty plastic bags of documents
flock to his small office for help, and he scribbles out notes for them on his Parliament letterhead
like a doctor in a field hospital.

``The new faces have to work much harder because their survival depends on it,'' said Sohail
Warraich, chief political correspondent for Geo TV. ``If they lose an election, they're finished.''

He wields his lower-class background like a weapon, exhorting local residents to oppose the rich
elite and the mafias of landlords, bureaucrats and other petty power brokers who support them.

``This was not an election,'' he shouted at a sweaty crowd, referring to a race he won against an
aristocrat in May. ``This was a fight between the poor and the rich, between the public and the
powerful classes.''

Graffiti nearby said: ``Give us electricity and we'll give you a vote.''

Lineage alone is no longer a winning strategy. Ahmed Mehmoud, an aristocrat in South Punjab, lost
both Parliament seats he contested in 2008 and had to settle for a provincial assembly seat.

``The seats are no longer so safe,'' said Nusrat Javed, a journalist who is an expert on politics in
Punjab. ``You can't survive as a mere feudal anymore.''

Mr.~Mehmoud, 48, is a wealthy man of leisure, who spends more time relaxing in his house -- a pink
replica of a Rajasthani palace with a hand-carved facade -- than on his job as a lawmaker. Sometimes
he talks to his constituents, but more often he watches them go by from the window of his speedy,
white Hummer.

For years, people voted for him anyway, partly out of habit. His ancestors were considered to be
distant relatives of the Prophet Muhammad, which inspires awe and respect. But more important, his
constituents were tied to him economically. His family owned the land they worked and often their
houses. His carpet has a worn patch where generations of peasants sat in supplication.

But now, said Shama Andleep, a local voter: ``On election day, people are asking questions. People
are calculating: how much has he done for us?''

Private television stations, which exploded onto the scene eight years ago, have also had an effect.
Khusro Bakhtyar, a landowner in the area, said the women who were baking bread in his house were so
affected by the coverage of the 2007 death of former Prime Minister Benazir Bhutto that they voted
for her party, not his.

The changes have steered Pakistan into uncharted territory, and the effect for the United States is
unclear. Unlike Mr.~Mehmoud, who is unabashedly pro-American, newcomers like Mr.~Dasti are more
skeptical. Mr.~Dasti opposes the American drone program that is used to attack militants in
Pakistan, but he is not as virulently anti-American as many in his country.

The changes also leave room for Islamists. In the neighboring district of Dera Ghazi Khan, a
hard-line mullah, Hafiz Abdul Karim, came within a few thousand votes in 2008 of unseating Farooq
Leghari, a former president of Pakistan. His weapon? Efficient, Islamist campaign workers and free
water pumps.

So far, Islamists have not tapped popular frustration in a systematic way at the ballot box, and the
military, the country's oldest, strongest institution, would probably put down any broader uprising,
analysts say.

But the floods and the misery they have brought have raised the stakes.

``If you don't give the common man justice, there will be more terrorism and even bloody
revolution,'' Mr.~Dasti said. ``This is the need of the hour.''

\pagebreak
\section{Can Porsche Shine at Volkswagen?}

\lettrine{T}{he}\mycalendar{Aug.'10}{30} formal merger of Porsche and Volkswagen is not supposed to
happen until next year at the earliest, but at the factory here that produces Porsche Cayenne
S.U.V.'s, it seems to have started years ago.

The painted Cayenne bodies arrive by train from a Volkswagen factory in Slovakia, where they are
made alongside the similar VW Touareg. Likewise, the basic skin and skeleton of the new Porsche
Panamera sedan, also assembled here, begins life at a VW factory in Hanover.

Yet there are many subtle things that make the Leipzig factory -- and what it produces --
unmistakably Porsche. The plant uses fewer machines than Volkswagen's big factories and sounds
remarkably quiet, almost reverential, as workers in spotless red overalls and white T-shirts install
equipment to each buyer's specifications. Want gold plating on that shift knob? No problem.

``The customers should get their car, not just any car,'' says Joachim Lamla, chief financial
officer of Porsche Leipzig, as he shows a visitor around.

That fancy Porsches contain some plebeian Volkswagen genes has never been a secret, and it doesn't
seem to matter to the buyers sipping espresso in the elegant Porsche customer center next door to
the Leipzig factory, waiting for the keys to their new cars -- and perhaps, a high-speed spin on the
track next door.

Sharing parts is one thing, but being taken over is quite another. For most of the last 60 years,
Porsche and Volkswagen have been like a couple who never quite moved in together. Porsche supplied
engineering expertise to Volkswagen and bought components and manufacturing capacity, but it
carefully guarded its identity and independence -- a formula that delivered some of the best profit
margins in the business.

Now, with Volkswagen poised to acquire Porsche -- in a deal announced a year ago that values Porsche
at 12.4 billion euros (\$15.6 billion) -- many in the auto industry wonder whether they can make a
marriage work. The risk, they say, is that VW bureaucracy could smother Porsche's individuality, or
that plans to offer less costly models could blur Porsche's image as the ultimate in German auto
engineering.

``The whole maxim has been, `We can't let outsiders influence Porsche because we don't want lesser
minds to crush our ingenuity,' '' says Jay Greene, author of ``Design Is How It Works,'' the first
chapter of which is a paean to Porsche design genius. ``Doing that in a big corporation is going to
be a challenge.''

VOLKSWAGEN, based in Wolfsburg, is already drawing Porsche, based in Stuttgart, into a closer
embrace. Martin Winterkorn, VW's chief executive, is installing a longtime prot\'eg\'e, Matthias
Müller, as C.E.O. of Porsche. And Mr.~Winterkorn has begun musing aloud about how to expand
Porsche's lineup from four core models to six and to bolster sales by more than a third, to 150,000
cars a year.

On the drawing board is a somewhat smaller Porsche S.U.V. Mr.~Winterkorn has also floated the idea
of a car selling for less than the Boxster, which, at \$47,600, is Porsche's least-expensive
two-seater, a few hundred dollars more than the basic Cayenne.

``We can build 150,000 autos without losing any exclusivity,'' Mr.~Winterkorn said last month in an
interview with a German newspaper. (He declined to be interviewed for this article.)

Others are not so sure. ``It will certainly not be the old Porsche,'' says Ferdinand Dudenhöffer, a
professor at the University Duisburg-Essen in Germany who specializes in the auto industry.
``Porsche will enter a new era as brand No.~10 in a big company.''

Still, Mr.~Dudenhöffer and others say that Porsche needs to adapt to huge changes in the upper
reaches of the car market, and that it does not have the resources to do so alone.

These days, status-conscious buyers want to pull up to the country club in a car that looks green as
well as sexy. (Even Ferrari is building a hybrid.) And automakers in all price classes worry that
the Facebook generation is more interested in mobile phones than in fast cars; analysts say the
phenomenon has already hurt auto sales in Japan.

The industry is also moving toward battery-powered vehicles, a long-term trend that could blunt a
main selling point of conventional sports cars: their speed. Because electric motors reach maximum
power much more quickly than internal combustion engines, even a humble battery-powered car may
potentially offer Porsche-like acceleration.

Porsche, with sales of 6.6 billion euros, or \$8.3 billion, last year, could not master the design
and technical challenges of these shifts without a large, well-heeled partner. The vast resources of
Volkswagen, with sales last year of 6.3 million vehicles, and revenue of 105 billion euros (\$134
billion) will allow Porsche, which has never produced more than 100,000 cars annually, to support
the costly development of new models that consume less fuel and have fewer emissions.

The trick will be to preserve Porsche's exclusivity. ``There are opportunities to grow, but they
have to watch out they don't dilute the brand by going downmarket,'' says Gregor Matthies, a partner
in Munich for Bain \& Company, the consulting firm, who specializes in the auto industry.

Stephan Winkelmann, the C.E.O. of Lamborghini, the Italian maker of exotic sports cars, is among
those who argue that the merger can work without extinguishing the Porsche spirit.

He speaks from authority. Lamborghini, which VW acquired 12 years ago, is one of nine brands in the
Volkswagen Group -- a stable that also includes economical VW and Skoda cars, upscale Audis and
one-of-a-kind Bugattis with a price tag of well over \$1 million.

Lamborghini has sold more cars since the acquisition -- 14,000 -- than it did during all the
previous 40 years, Mr.~Winkelmann said. It even uses some VW parts, but not for more than 10 percent
of any car and never anywhere they might be visible to owners.

``The development of our company in the last 12 years stands as clear demonstration of the advantage
of being part of this group,'' Mr.~Winkelmann said by e-mail.

A BIG uncertainty facing Porsche is whether Mr.~Müller will be able to fill the void left by
Wendelin Wiedeking, the longtime Porsche C.E.O.

Mr.~Wiedeking, one of Germany's best-known managers, was an iconoclast revered within the company,
though his brash style sometimes annoyed shareholders as well as the top brass at Volkswagen.

Mr.~Müller, 57, is much more the company man, a prot\'eg\'e of Mr.~Winterkorn who has spent his
entire career at Volkswagen, primarily in the Audi unit. Mr.~Müller joined Audi as an apprentice
toolmaker in 1971 and returned full time in 1978 after completing a degree in computer science. He
led the team that produced and marketed the Audi A3 compact in the mid-1990s and eventually rose to
be coordinator of all the brand's model lines. In 2007, Mr.~Winterkorn put him in charge of
overseeing product development for the entire Volkswagen group.

At Audi, a VW unit since the 1960s, Mr.~Müller faced challenges similar to ones he will confront at
Porsche, which occupies an even more exclusive segment of the market. Mr.~Müller can take much of
the credit for Audi's success at overtaking BMW in Europe in unit sales with cars built largely from
the same array of parts as VW Golfs or Passats.

As they are with the Lamborghinis, the common parts are hidden from Audi customers' view. What
buyers perceive is more stylish design and sportier performance as well as more costly materials,
like aluminum body work. Today Audi is VW's most profitable unit by far.

But Volkswagen has not been successful with all its brands. Its Bentley unit, which makes high-end
cars that compete with Maserati and Rolls-Royce, lost 109 million euros (\$137 million) on sales of
320 million euros in the first six months of 2010. VW, which remains profitable over all, blamed
``changes in the market and product mix'' for the loss.

Equally important, Mr.~Müller enjoys the trust of Mr.~Winterkorn as well as Ferdinand Pi\"ech, the
demanding chairman of the Volkswagen supervisory board who also happens to be a scion of the Porsche
family. And Mr.~Müller seems well regarded by his peers. Mr.~Winkelmann of Lamborghini calls him
``a professional and a friend.''

Mr.~Müller will not officially take charge until October, but he has been quietly touring Porsche
offices and factories. Analysts say he is not likely to enjoy the same freedom that Mr.~Wiedeking
did.

Mr.~Müller ``is a VW guy through and through,'' says Tim Urquhart, an auto analyst at IHS Global
Insight. ``Porsche will have to fit into a wider corporate strategy. That is a big change in
culture.'' Mr.~Müller declined to be interviewed for this article.

Mr.~Urquhart and other analysts warn that Mr.~Müller will need to avoid overlap between Porsche and
Audi, which sells its own high-end sports car, the R8, starting at \$148,000, as well as the TT
line, starting at \$37,800. And at the ultra-high end of the market, Mr.~Müller will have to avoid
stepping on toes at Lamborghini.

``There is lots of crossover; they have to work out that relationship,'' Mr.~Urquhart says. ``You've
got to keep that mystique about the brand.''

Mr.~Wiedeking left last year after a byzantine power struggle for control of the company. But in the
previous 17 years, he transformed Porsche, expanding its product line beyond the core 911 to include
the lower-price Boxster, the Cayenne S.U.V. and, last year, the Panamera four-door sedan.

Porsche's legions of enthusiasts initially recoiled at the idea of Porsche family cars. But the
Cayenne, which starts at \$47,000, is now Porsche's best-selling model worldwide. It also allowed
the company to push into emerging-market countries where roads may be unsuitable for a
ground-hugging 911.

In addition, the Panamera has become Porsche's second-best-selling model in the United States, just
behind the Cayenne, less than a year after its introduction.

Mr.~Wiedeking also cut costs, introducing Japanese-style manufacturing methods and outsourcing some
Boxster production. The outsourcing allowed Porsche to continue earning double-digit margins on car
sales last year, even as sales plunged 24 percent.

His downfall was his audacious attempt, begun in 2005, to acquire Volkswagen via stock derivatives.
Mr.~Wiedeking, who declined to comment for this article, recognized that Porsche needed to cement
its ties with the larger company, which at the time was regarded as a possible takeover target.

``Wiedeking clearly understood that on a stand-alone basis, Porsche would not be able to cope with
emission requirements and the shift to emission-free vehicles,'' says Mr.~Matthies at Bain \&
Company.

The acquisition plan backfired when the financial crisis hit and banks no longer wanted to back
Mr.~Wiedeking. Instead, Volkswagen wound up rescuing Porsche from financial overreach and gaining
control.

Employees gave Mr.~Wiedeking a 15-minute standing ovation when he left the main factory in Stuttgart
in July 2009, and some veteran engineers even wept, according to several people there.

IT is safe to say that neither Porsche nor Volkswagen would exist today without the other. Their
ties go back to the `30s, when Ferdinand Porsche, who made his living as a kind of auto engineering
consultant, designed the ``people's car'' for the Nazis. First mass-produced after World War II,
with Porsche continuing to supply engineering expertise, the car that became known as the Beetle set
VW on a path to become Europe's largest carmaker. Mr.~Winterkorn's long-term goal is to surpass
Toyota as No.~1 in the world.

Meanwhile, Mr.~Porsche's son, known as Ferry, bolted a streamlined aluminum body to what was
essentially a Beetle chassis and engine to create the first mass-produced Porsche sports car, the
356, which went on sale in 1948 and marked the transition of the Porsche company from an engineering
consultant and maker of one-off racers to a full-fledged car producer. The lines of the 356 are
still visible in modern 911 models, which start at \$78,000 but can go up to \$245,000 for a
race-ready version with a top speed of 205 miles per hour.

Porsche and VW cooperated on other cars through the years, including the 914, introduced in 1970,
and the Porsche 924, which was built at an Audi factory from 1976 to 1988. Tellingly, the 914 sold
poorly in Germany, where it was branded a VW-Porsche. In the United States, where the Volkswagen tie
went unmentioned, it sold well.

The family ties have also persisted. Mr.~Pi\"ech, the Volkswagen chairman, is the grandson of the
Beetle designer Ferdinand Porsche, and was in charge of developing Porsche racecars during the
1960s. Mr.~Pi\"ech went to work for Volkswagen after Ferry Porsche banned family members from the
sports car company in the early 1970s because he felt that they were spending too much time fighting
with one another.

Last year, Mr.~Pi\"ech was one of the main characters in the battle for control of Porsche, much of
which played out behind the scenes among Porsche family shareholders at their compound in the
Austrian resort of Zell am See.

Mr.~Pi\"ech is also a formidable engineer known for his love of exotic technology, even at the
expense of short-term profit. That bodes well for Porsche's plans to make big investments in green
technology.

As part of the company's push to appear more environmentally sustainable, some of the grass near the
Leipzig factory is kept trim by a herd of oxen, which on special occasions may also turn up on the
menu at the customer center's restaurant.

When it comes to cars, as opposed to cattle, Porsche has been slower than other luxury carmakers to
go green. It introduced the first Porsche hybrid, a \$68,000 variant of the Cayenne, in June.

A survey last year by Bain of 4,000 car owners in the United States, Europe and Asia found that
premium-car buyers were more willing than those in other segments to invest in green technology. But
they also wanted their low-emission cars to look cool. Ominously for Porsche, the premium buyers
said they would be more likely to buy an electric car made by Toyota, Daimler or BMW than one made
by Porsche or VW, probably because the other companies already have hundreds of low-emission
prototypes on the roads.

Porsche said in July that it was planning commercial production of a new plug-in hybrid, the 918
Spyder, that will offer blistering acceleration but also be capable of getting 78 m.p.g. -- though
only if driven like a Volkswagen and not like a race car. The car will probably cost hundreds of
thousands of dollars.

Porsche representatives say the technology will eventually find its way into less expensive models.
The company is also experimenting with a prototype battery-powered Boxster.

But the cultlike devotion that has inspired hundreds of Porsche clubs worldwide could suffer if, as
Mr.~Dudenhöffer predicts, VW pushes the brand further into BMW and Mercedes territory.

Jim Hearnden, vice chairman of the Independent Porsche Enthusiasts Club in Britain, said, ``It's
quite feasible we could see a blurring between the top end of VW and the bottom end of Porsche.''
But Mr.~Hearnden, owner of a 1987 Porsche 944, added that ``anything that means they survive is
good.''

WORRIED devotees can still summon the ghost of Ferry Porsche on YouTube, where in an old promotional
film the scion warns of the perils of being part of a big corporation.

``Porsche must remain small and independent,'' Mr.~Porsche, who died in 1998, says as strings swell
in the background and drums pound ominously. ``No Porsche will ever be created by a committee, but
by a handful of people inside these walls who know what a Porsche is.''

\pagebreak
\section{Policy Options Dwindle as Economic Fears Grow}

\lettrine{T}{he}\mycalendar{Aug.'10}{30} American economy is once again tilting toward danger.
Despite an aggressive regimen of treatments from the conventional to the exotic -- more than \$800
billion in federal spending, and trillions of dollars worth of credit from the Federal Reserve --
fears of a second recession are growing, along with worries that the country may face several more
years of lean prospects.

On Friday, Ben Bernanke, chairman of the Fed, speaking in the measured tones of a man whose word
choices can cause billions of dollars to move, acknowledged that the economy was weaker than hoped,
while promising to consider new policies to invigorate it, should conditions worsen.

Yet even as vital signs weaken -- plunging home sales, a bleak job market and, on Friday,
confirmation that the quarterly rate of economic growth had slowed, to 1.6 percent -- a sense has
taken hold that government policy makers cannot deliver meaningful intervention. That is because
nearly any proposed curative could risk adding to the national debt -- a political nonstarter. The
situation has left American fortunes pinned to an uncertain remedy: hoping that things somehow get
better.

It increasingly seems as if the policy makers attending like physicians to the American economy are
peering into their medical kits and coming up empty, their arsenal of pharmaceuticals largely
exhausted and the few that remain deemed too experimental or laden with risky side effects. The
patient -- who started in critical care -- was showing signs of improvement in the convalescent ward
earlier this year, but has since deteriorated. The doctors cannot agree on a diagnosis, let alone
administer an antidote with confidence.

This is where the Great Recession has taken the world's largest economy, to a Great Ambiguity over
what lies ahead, and what can be done now. Economists debate the benefits of previous policy
prescriptions, but in the political realm a rare consensus has emerged: The future is now so colored
in red ink that running up the debt seems politically risky in the months before the Congressional
elections, even in the name of creating jobs and generating economic growth. The result is that
Democrats and Republicans have foresworn virtually any course that involves spending serious money.

The growing impression of a weakening economy combined with a dearth of policy options has
reinvigorated concerns that the United States risks sinking into the sort of economic stagnation
that captured Japan during its so-called Lost Decade in the 1990s. Then, as now, trouble began when
a speculative real estate frenzy ended, leaving banks awash in debts they preferred not to recognize
and hoping that bad loans would turn good (or at least be forgotten). The crisis was deepened by
indecisive policy, as the ruling party fruitlessly explored ways around a painful reckoning --
boosting exports, tinkering with accounting standards.

``There are many ways in which you can see us almost surely being in a Japan-style malaise,'' said
the Nobel-laureate economist Joseph Stiglitz, who has accused the Obama administration of
underestimating the dangers weighing on the economy. ``It's just really hard to see what will bring
us out.''

Japan's years of pain were made worse by deflation -- falling prices -- an affliction that assailed
the United States during the Great Depression and may be gathering force again. While falling prices
can be good news for people in need of cars, housing and other wares, a sustained, broad drop
discourages businesses from investing and hiring. Less work and lower wages translates into less
spending power, which reinforces a predilection against hiring and investing -- a downward spiral.

Deflation is both symptom and cause of an economy whose basic functioning has stalled. It reflects
too many goods and services in the marketplace with not enough people able to buy them.

For more than a decade, the global economy was fueled by monumental spending power underwritten by a
pair of investment booms in America -- the Internet explosion in the 1990s, then the exuberance over
real estate. As housing prices soared, homeowners borrowed against rising values, distributing their
dollars to furniture dealers in suburban malls, and furniture factories in coastal China.

But the collapse of American housing prices severed that artery of finance. Homeowners could not
borrow, and they cut spending, shrinking sales for businesses and prompting layoffs.

Early this year, some economists declared that the cycle was finally righting itself. Businesses
were restocking inventories, yielding modest job growth in factories. Hopes flowered that these new
wages would be spent in ways that led to the hiring of more workers -- a virtuous cycle.

But the hopes failed to account for how extensively spending power had dropped in the American
economy, and how uneasy people were made by every snippet of data showing that houses were not
selling, employers were not hiring, and stock prices were foundering.

Now, a new cause for concern is growing: the flat trajectory of prices, which might metastasize into
a full-blown case of deflation.

The primary way to attack deflation is to inject credit into the economy, giving reluctant consumers
the wherewithal to spend. The chief deflation fighter is the Federal Reserve, which traditionally
adjusts a benchmark overnight rate for banks that influences rates on car loans, mortgages and other
forms of credit. The Fed pulled this lever long ago, and has kept its target rate near zero since
late 2008.

The Fed has also been more creative. During the worst of the financial crisis, the Fed relieved
American banks of troubled investments, many linked to mortgages, to give the banks room to make new
loans.

This engendered the sort of debate likely to fill doctoral dissertations for generations. Most
economists praise the Fed for confronting the possibility of another depression. But the Fed added
to the nation's debts, provoking talk that it was testing global faith in the dollar.

The dramatic expansion of the national debt -- which began in the Bush administration, via hefty tax
cuts and two wars -- has ratcheted up fears that, one day, creditors like China and Japan might
demand sharply higher interest rates to finance American spending. Those rates would spread through
the economy and inflict the reverse of deflation: inflation, or rising prices, as merchants lose
faith in the sanctity of the dollar and demand more dollars in exchange for oil, electronics and
other items.

So far, the reverse has happened. As investors lose faith in real estate and stocks, they are
flooding into government savings bonds, keeping interest rates exceedingly low. Still, inflation
worries occupy the people who control money, not least the governors of the Fed. The Fed has been
seeking a graceful exit from its interventions, aiming to unload its cache of mortgage-linked
investments and -- likely in the far future -- lift interest rates.

But the recent disturbing economic news has delayed those plans. This month, the Fed said it would
take the proceeds from its mortgage-linked investments and buy Treasury bills to keep longer-term
interest rates down. The Wall Street Journal reported that this decision came amid substantial
disagreement among the Fed's governors, suggesting that future action will be constrained by fears
of inflation.

Republicans in Congress have embraced further tax cuts and less spending as the answer to the weak
economy, while accusing the administration of squandering stimulus spending on efforts that brought
little gain. Some conservative analysts liken the government's reliance on spending and credit to
imbibing another cocktail to take the edge off a hangover. In this view, the weak economy should be
welcomed for the discipline it imposes, forcing a paring back of unsustainable spending, while
building up savings that can finance investment and later feed healthy economic growth.

``The recession is the cure for the disease that affects the economy, but the politicians don't have
the stomach for it,'' says Peter Schiff, president of Euro Pacific Capital, a Connecticut-based
brokerage house. ``They're going to keep stimulating the economy until they kill it with an
overdose. The hyper-inflation that results is going to be far worse than the cure.''

Germany, which has long harbored particularly powerful fears of inflation, has done relatively well
in the current downturn without large stimulus spending, and that experience is now cited by
adherents of austerity. But it can be argued that the Germans had two advantages over Americans: A
more extensive social safety net to give consumers more money and the confidence to spend it, and a
vibrant manufacturing base to churn out more goods for export.

Most economists who are close to the policy making arena for both parties take the position that
austerity is the wrong medicine for what ails the American economy, and they dismiss warnings about
inflation as akin to focusing on the side effects of chemotherapy in the face of cancer. First, they
argue, take the medicine and stave off the lethal threat; then deal with the collateral problems.

Regardless, inflation fears persist, constraining what limited prescriptions might otherwise be
thrown at a weakening economy.

The impending elections in November -- with control of Congress hanging in the balance -- has
further narrowed the contours of political possibility

Six months ago, Alan Blinder, a former vice chairman of the Federal Reserve, and now an economist at
Princeton, dismissed the idea that America's political system would ever allow the country to sink
into a Japan-style quagmire. ``Now I'm looking at the political system turning itself into a
paralyzed beast,'' he says, adding that a lost decade now looms as ``a much bigger risk.'' Congress
and the Obama administration have ruled out further stimulus spending. The Fed appears to be running
out of powder. ``Its really powerful ammunition has been expended,'' Mr.~Blinder says.

Even after the November election, few expect a different dynamic. ``We're already in a gridlock
situation, and nothing substantive is going to change,'' says Bruce Bartlett, who was a Treasury
economist in the first Bush administration. ``Clearly, a weak economy in 2012 will be very good for
whoever the Republican presidential candidate is. It's hard to see how the Republicans lose by
blocking stimulus.''

On the other hand, if deflation emerges as a verifiable menace, many economists expect Mr.~Bernanke
-- an expert on the Great Depression -- to again champion aggressive measures, perhaps expanding the
Fed's balance sheet to buy pools of auto loans or credit card debt.

``It's very likely the Fed will bend in that direction if the economy stays soft, especially if they
are starting to see deflation,'' says Kenneth S.~Rogoff, a former chief economist at the
International Monetary Fund, and now a professor at Harvard. ``That's really starting to loom.''

On Friday, Mr.~Bernanke, whose board can operate independent of politics and the government, offered
assurance that he still had powerful therapies to use should conditions worsen. Yet he also
expressed concern about the potential side effects, underscoring a reluctance for more action.

``The issue at this stage is not whether we have the tools to help support economic activity and
guard against disinflation,'' he said. ``We do.'' Then he added: ``The issue is instead whether, at
any given juncture, the benefits of each tool, in terms of additional stimulus, outweigh the
associated costs or risks of using the tool.''

Right now, many homeowners owe the bank more than their homes are worth, prompting some to abandon
properties, adding inventory to a market choked with vacant addresses. An Obama administration
program aimed at slowing foreclosures has prolonged trouble, say some economists, by failing to
relieve borrowers of unsustainable debt burdens or making transparent the extent of losses yet to be
confronted by the financial system.

``The big question is, who's going to swallow the losses,'' says Mr.~Stiglitz. ``It should be the
banks, but they don't want to. We're likely to be in paralysis for years if they prevail.''

The Treasury sits in the middle, concerned by the continued weakness of housing, yet unwilling to
pressure banks to write down mortgage balances.

Like their Japanese counterparts a decade ago, Treasury officials worry that forcing the banks to
take losses could weaken them and risk another crisis.

By default, muddling through has emerged as the prescription of the moment.

\pagebreak
\section{Neighborly Borrowing, Over the Online Fence}

\lettrine{T}{he}\mycalendar{Aug.'10}{30} first time I unboxed my gleaming Roomba, I beamed like a
proud new parent as I placed it gently on my hardwood floor.

That evening, I watched it putter around my apartment, sweeping and inhaling dust bunnies. When it
gamely bumbled around bulky pieces of furniture, I dashed about, too, lifting the obstacles out of
its way. After the Roomba finished its chaotic dance, I put it back into its case and patted the
sweet little machine good night. The next morning, I returned it to its rightful owner.

The Roomba was mine for only 24 hours. I had rented it through a service called SnapGoods, which
allows people to lend out their surplus gadgetry and various gear for a daily fee.

SnapGoods is one of the latest start-ups that bases its business model around allowing people to
share, exchange and rent goods in a local setting. Among others are NeighborGoods and
ShareSomeSugar. Other commercial services are springing up, too, including group-buying sites like
Groupon, the peer-to-peer travel site Airbnb and Kickstarter, which allows people to invest small
sums in creative ventures.

The common thread of all these sites is that access trumps ownership; consumers are offered ways to
share goods instead of having to buy them.

Ron J.~Williams, co-founder of SnapGoods, based in New York, describes the phenomenon as the
``access economy.''

``There may always be certain products that you do need to buy,'' says Mr.~Williams. ``But there is
also a growing cultural awareness that you don't always get enjoyment out of hyperconsumption. The
notion of ownership as the barrier between you and what you need is outdated.''

The most obvious reason for all of this is financial. Recession-battered shoppers can test pricey
new devices before deciding whether to take the plunge or wait until the next upgrade. (Roombas, for
example, can retail for as much as \$600 for the newer models. I borrowed mine for a much more
palatable \$10.)

For all the promise of these new marketplaces, analysts say they aren't likely to overtake more
traditional models anytime soon.

``The holy grails of consumerism are convenience and choice,'' says Rachel Botsman, co-author of the
forthcoming book, ``What's Mine Is Yours: The Rise of Collaborative Consumption.'' ``This is not the
end of the old consumer way. But they could sit side by side. Peer-to-peer could become the default
way to share.''

There's much evidence that this is already happening. Do-it-yourself home improvers can borrow tools
for a weekend project, and hobbyist campers can rent equipment per trip, rather than splurge on
all-new gear. Travelers looking for inexpensive accommodations can spend the night in someone's
spare bedroom for a fraction of the cost of a hotel room. For people who lend their stuff, it's a
way to make extra money on possessions that are gathering dust.

``My Roomba is on track to pay for itself,'' says Luke Tucker, 31, a software engineer who rented me
his robotic vacuum cleaner through SnapGoods.

But some experts think that there may be something bigger than thriftiness at play. These services
may be gaining popularity because they reinforce a sense of community.

``It turns out to actually be a good way to meet my neighbors,'' says Mr.~Tucker, who also lists a
jigsaw, a digital camera and a wireless keyboard for rent on SnapGoods.

Charlis Floyd, a 22-year-old student, and Nema Williams, a 30-year-old comedian, who rent out their
spare bedroom in Brooklyn on Airbnb, say that while the extra income helps -- as any little bit does
these days -- they're much more interested in the revolving cast of characters they meet.

``We had a couple from England teach us how to make red curry,'' says Ms.~Floyd.

``Another guy, an artist, promised to paint a mural in our kitchen,'' adds Mr.~Williams.

Of course, that doesn't mean it always goes off without a hitch.``Sometimes people can be weird,''
says Ms.~Floyd. ``One girl drank all our milk and another person broke our toilet handle.''

Even so, Ms.~Floyd and Mr.~Williams still like being in the rental business.

``It's a win-win situation,'' says Ms.~Floyd. ``You make some extra money and make new friends.''

For entrepreneurs, there's a payoff in such commerce. Groupon, for example, says it's on track to
generate \$500 million in revenue this year; Airbnb has said it is profitable, though it does not
provide exact numbers.

Paul J.~Zak, director of the Center for Neuroeconomics Studies at Claremont Graduate University in
California, says that participating in a community like SnapGoods, Kickstarter, Groupon or Airbnb
can ease social isolation and flesh out our network of friends.

``There is an underlying notion that if I rent my things in my house, I get to meet my neighbor, and
if I'm walking the goods over, I get to meet them in person,'' he says. ``We're drawing on a desire
in a fast-paced world to still have real connections to a community.''

Mr.~Zak says he conducted a preliminary experiment indicating that posting messages on Twitter
caused the release of oxytocin, a neurotransmitter that evokes feelings of contentment and is
thought to help induce a sense of positive social bonding. He is now testing those ideas in research
on a group of 40 people.

The social interaction ``reduces stress hormones, even through the Web,'' he says. ``You're feeling
a real physiological relationship to that person, even if they are online.''

MR. ZAK says Web commerce is moving beyond transactions by individuals and companies and embracing
models that encourage social contact and interaction -- a hallmark of the already robust social
media phenomenon and a throwback to the good old days when people actually spent time socializing at
local markets.

``The Web is bringing businesses back down to the individual as the average company becomes smaller,
more niche and specialized,'' he says. ``Paradoxically, the Web is moving us back to a human-centric
business model.''

Trust is a big factor in all of this. Otherwise, how can you be sure that someone won't just rip you
off?

Marketplaces like eBay have long relied on ratings and user reviews to weed out unreliable
participants. But in addition to safeguards like preauthorizing the price of rentals through PayPal,
the latest wave of peer-to-peer systems make use of social networks like Facebook and Twitter to
engender trust.

If someone wants to rent your iPad or crash on your couch, the person's online profile leaves a
trail of digital bread crumbs that makes it harder to pull off a scam, giving potential lenders and
hosts reason to breathe easier.

``This new economy,'' says Ms.~Botsman, ``is going to be driven entirely by reputation, which is
part of a new cultural shift -- seeing how our behavior in one community affects what we can access
in another.''

\pagebreak
\section{USA Today to Remake Itself to Stress Digital Operations}

\lettrine{T}{he}\mycalendar{Aug.'10}{30} history of USA Today is full of firsts for the newspaper
business: the first general-interest national paper of its kind, the first to use color widely in
charts and photographs and once first in the number of copies printed each day.

But lately the paper has lost its grip on the national media market. Its advertising revenue has
collapsed. Its circulation has plunged -- last year it forfeited its title as the nation's most
widely circulated newspaper on weekdays to The Wall Street Journal.

Faced with this dour reality, USA Today announced on Friday the most extensive reorganization in its
28-year history. The paper will eliminate about 130 jobs, or 9 percent of its work force, and shift
its business model away from the print edition that has become ubiquitous in airports, hotels and
newsstands across the country.

The paper's focus will now be on its digital operations. It will emphasize breaking news on its Web
site, aiming to post articles within 30 minutes of a breaking news event. It will create a
stand-alone sports edition called USA Today Sports. And it will shift more of its resources toward
making content more available in digital form, an effort to win a larger share of the tablet and
mobile phone news market.

``We aren't organized to adapt to the changing audience demands on all platforms,'' the paper's
management told employees in a presentation on Thursday.

But much about the paper's plans for its future, including how it will manage a new alignment
between the business and news operations at the paper, remained unexplained.

Staff members who sat through the presentation from USA Today executives and editors said they came
away with many unanswered questions and a sense that the reorganization plan was a work in progress.

The publisher, David L.~Hunke, declined to answer questions afterward about the restructuring was
revealed late Thursday night in an article by The Associated Press. A spokesman for USA Today, Ed
Cassidy, said that the paper's announcement and the A.P. article were all that USA Today had to say
on the matter, ``and that pretty much closes the book on this for us.''

Neither the announcement nor the article said which departments would be affected by the 130 job
cuts or how the paper would address conflict-of-interest concerns that might arise from the new
editorial-business partnership, which was described in the presentation as ``a new way of doing
business that aligns sales efforts with the content we produce.''

USA Today's newly appointed executive editor for content, Susan Weiss, will, in part, serve as a
liaison between the news and business sides.

Like other print media companies that have struggled to reckon with falling circulation and
declining advertising revenue, USA Today has tried to find a way to remain viable in an era when the
kind of packaging of general news that made it a pioneer is available in abundance -- and free --
elsewhere.

Further complicating USA Today's fortunes, The Wall Street Journal has made a more aggressive push
into the general-interest newspaper market. The Journal has expanded its presence on newsstands and
in Starbucks, where it says it will be in more than 6,700 locations by the end of next month. The
Journal has also increased its circulation in hotels -- traditionally USA Today's turf -- by 15
percent, to nearly 40,000 copies daily from March 2009 to March 2010. USA Today, with nearly 400,000
copies in hotels each weekday, as of March, still dominates that business but has slipped in recent
years. More than 800,000 copies of the paper were distributed in hotels each weekday in 2007.

USA Today's Monday through Friday circulation is now 1.8 million, compared with The Journal's 2.1
million.

Some newspaper analysts and experts voiced doubts about whether a reorganization would really
position USA Today to be stronger against other media outlets that had been quicker and more
successful with their innovations.

``I don't blame them for tinkering with the model, but everybody has been trying to do that,'' said
Craig A.~Huber, an analyst with Access 342, a research firm that follows USA Today and its parent
company, Gannett. ``They have to restructure. They don't have a choice in this quickly evolving
media landscape.''

Other analysts pointed out that the type of sweeping overhaul that USA Today was trying was much
easier said than done.

``The concept of evolving into a multimedia operation appears simple in concept but is very
complicated in terms of execution,'' said James C.~Goss, an analyst at Barrington Research.
``There's no guarantee of success, but standing still would be much worse.''

The paper has experimented with ways to be more innovative about generating revenue, though not
always in ways that hew to long-accepted practices in the newspaper business. In July it wrapped its
front section in an advertisement for Jeep that obscured the entire front page. The ad stirred
outrage in the newsroom and prompted the paper's founder, Allen H.~Neuharth, to complain in a letter
to the publisher that if he were still there, ``I would have led the entire news staff walking out
in protest.''

Taking a shot at Mr.~Hunke, Mr.~Neuharth, 86, added: ``If such a stupid decision is ever made again,
I hope that will be the result. That would leave those who apparently don't understand what a
newspaper is to try to put one out without a news staff.''

What USA Today must do to thrive in the future, experts said, is recapture its innovative spirit.

``USA Today used to be the trendsetter in the business,'' said Toni Locy, a former USA Today
reporter who now teaches journalism at Washington \& Lee University in Virginia. ``Nobody had color
before USA Today. Nobody had a weather map. And USA Today has lost a step in the last five or six
years. So I think they're right to try to regain the ground they've lost.''

\pagebreak
\section{Obama Pledges Commitment to New Orleans}

\lettrine{P}{resident}\mycalendar{Aug.'10}{30} Obama on Sunday sought to assure this city, battered
by two catastrophic disasters in five years, that federal efforts to rebuild after Hurricane Katrina
would not waver even as the city struggles with the aftermath of the BP oil spill in the Gulf of
Mexico.

Speaking at Xavier University on the fifth anniversary of the hurricane that took 1,800 lives,
Mr.~Obama emphasized the resilience of New Orleans residents.

The legacy of Katrina, Mr.~Obama said, must be ``not one of neglect, but of action; not one of
indifference, but of empathy; not of abandonment, but of a community working together to meet shared
challenges.''

``There are some wounds that do not heal,'' the president acknowledged. ``There are some losses that
cannot be repaid. And for many who lived through those harrowing days five years ago, there is a
searing memory that time will not erase.''

Mr.~Obama's trip to the gulf was his second in two weeks. The visits book-ended his family's 10-day
vacation on Martha's Vineyard and served as a stark reminder of the challenges facing his presidency
as Congressional Democrats and Republicans prepare for the midterm elections in November.

The excruciatingly slow effort to plug the BP oil spill -- finally accomplished in July after 87
days -- brought the president to the gulf numerous times in recent months as he sought to avoid the
pitfalls that dogged the Bush administration's response to Hurricane Katrina.

Arne Duncan, the education secretary, announced Saturday on the White House blog a plan to award
\$1.8 billion to rebuild New Orleans schools.

And Shaun Donovan, the housing secretary, said Sunday on ``State of the Union'' on CNN that the
Obama administration had made progress in the past 18 months in returning families to their homes.
Mr.~Donovan said that 40,000 families were in trailers or on emergency housing vouchers when
Mr.~Obama took office, and that now ``98 percent of those families are in permanent housing.''

The city's unemployment rate is 8.2 percent, compared with 9.5 percent nationally, but many of the
jobs are in reconstruction. The unemployment rate dipped to 7.4 percent last year before creeping
back up as the oil spill shuttered businesses and stalled livelihoods.

In an interview on Sunday with ``NBC Nightly News,'' Mr.~Obama acknowledged the economic and other
difficulties in New Orleans and throughout the entire Gulf Coast region, though he said there had
been ``steady progress.''

He added: ``We've still got a long way to go. And part of the reason that I wanted to come down here
today, to mark the fifth anniversary, was just to send a message to the people of New Orleans, but
also the entire Gulf Coast.''

The region was ``hit pretty good over the last several years,'' he said. ``And all of America, not
just people here, not just folks in the White House, but all of America, remains concerned and
remains committed to their rebuilding.''

Most of the signs held by protesters who braved rain and wind to await Mr.~Obama's arrival at Xavier
on Sunday referred to the oil spill, which has been blamed for a drop in tourism, one of the city's
mainstays. The spill also hurt the commercial fishing industry along the Gulf Coast and damaged
fragile wetlands and wildlife sanctuaries.

But many local officials have criticized Mr.~Obama's decision to impose a moratorium on deepwater
offshore oil drilling. That moratorium is set to expire on Nov.~30.

Mr.~Obama tacitly acknowledged that politicians could no longer talk about the aftermath of
Hurricane Katrina without including the aftermath of the oil spill. Much of his speech at Xavier was
devoted to the spill and what his administration had been trying to do to limit the fallout.

``From the start, I promised you two things,'' Mr.~Obama said. ``One is that we would see to it that
the leak was stopped. And it has been. The second promise I made was that we would stick with our
efforts, and stay on BP, until the damage to the gulf and to the lives of the people in this region
was reversed. And this, too, is a promise that we will keep.''

Mr.~Obama brought his family along to New Orleans on Sunday. Shortly after Air Force One touched
down, the first family stopped for lunch at the Parkway Bakery and Tavern. Mr.~Obama mingled with
diners, and he and Mrs.~Obama shook hands and took photos.

``We're still here, and we're just going to keep on building,'' Mr.~Obama said. ``We're going to
keep on working, all right?''

As he worked the room, a voice over the loudspeaker boomed, ``Barack, pick up,'' signaling that the
family's lunch order was ready.

\pagebreak
\section{High-Deductible Plans Grow, but Not Everyone Should Get on Board}

\lettrine{F}{rustrated}\mycalendar{Aug.'10}{30} by exploding health care costs, many firms are
encouraging employees to enroll in high-deductible health insurance plans linked to tax-exempt
health savings accounts.

At first glance, it can seem like a great deal. In exchange for picking up a larger share of their
own health care costs, employees pay lower insurance premiums and are allowed to use pretax dollars
to pay out-of-pocket costs. But many consumers embracing the plans have discovered there are
pitfalls aplenty, including out-of-pocket expenses they cannot afford.

To open a health savings account, you must be enrolled in a qualified health insurance plan with a
deductible of at least \$1,200 for an individual or \$2,400 for a family.

In return for accepting the higher deductible, you are allowed to deposit pretax dollars in the
H.S.A., which are used to pay your out-of-pocket medical costs. This year, singles may contribute a
maximum of \$3,050 to an H.S.A. and families can deposit up to \$6,150.

Earnings on the account are also tax-free, and no taxes are paid on withdrawals used for qualified
medical expenses.

More than 10 million people are enrolled in high-deductible health plans linked to health savings
accounts, up from 6.1 million in 2008, according to a recent survey by America's Health Insurance
Plans, an industry trade group.

Companies like the plans because they require workers to shoulder more of their own health care
costs, and the premiums for high-deductible plans, to which firms contribute, are far lower than for
traditional policies.

If workers must pay a bigger chunk of their own costs, the argument goes, they will become wiser
consumers of health care. And since the balances in H.S.A. accounts roll over from year to year and
travel with an employee from job to job, they can help finance individuals' health care costs during
retirement.

Since Philip Derrow, owner and chief executive of Ohio Transmission Corporation in Columbus,
switched to a high-deductible H.S.A. plan for himself and 300 employees four years ago, his firm's
health insurance premium costs have increased by an average 2 percent or so a year, compared with 10
to 12 percent increases in previous years.

``This was by far the most affordable option for us, and I really believe it makes us better health
care consumers,'' Mr.~Derrow said.

Critics have been less enthusiastic about H.S.A.'s, worrying that high-deductible plans work only
for young, relatively healthy people who do not spend a lot on health care anyway. When sick people
are faced with paying high out-of-pocket costs for medical bills, they simply go without the care
they need, experts note.

What's more, shopping for the best health care prices is difficult for consumers -- sometimes
impossible, because doctors and other health care providers don't readily disclose prices.

And the new health care law may make high-deductible plans less attractive if new rules, yet to be
announced, require lower deductibles and impose other restrictions.

If you're in or are about to join an H.S.A., here are some ways to make the most of your decision.

SHOP FOR LOWER FEES Prepare to be nickeled-and-dimed. Banks and other financial institutions that
offer H.S.A.'s often charge a one-time set-up fee of \$25 to \$75, along with an annual fee in the
same range. Charges for checks, debit card and automated teller transactions can run a dollar or two
each. Of course, the usual big charges may apply for overdrafts or bounced checks.

In addition, depending on where you park your health care money, you may pay an investment or
so-called 12b-1 fee on top of the banking fees.

These charges may seem small, but over time they can add up. Shop for the financial institution with
the lowest fees, and keep as much money as you can in your account to pay medical bills.

Most employers who offer an H.S.A. option contract with a single financial institution and pick up
the administrative fees for any employees who set up their accounts there. This choice is often your
best bet. But you are not obligated to use that institution. You'll still want to compare what your
employer is offering with other big H.S.A. providers like H.S.A. Bank or Wells Fargo to make sure
the fees offered are at the low end of the spectrum.

CHOOSE SAFER INVESTMENTS Almost everyone with an H.S.A. wants to put the money in a haven. That's
why most employers provide plans with low-interest, conservative savings or money market accounts.
These days most of those accounts offer little more than a percentage point in interest. You should
expect to earn at least that much -- after all, it is better than nothing.

Other providers, like Fidelity and Wells Fargo, offer a range of investment options for H.S.A.'s,
including stocks and stock mutual funds. These aggressive strategies should probably be reserved for
people who have accumulated a fair amount in their accounts and do not plan to spend it in the short
term, said Edward Kaplan, national health practice leader at the Segal Company, a benefits
consulting firm.

For most people, there is no need to be swayed by the fancy marketing materials promoting dozens of
H.S.A. investment options when an interest-bearing savings or money market account will keep low
balances safe.

EMPLOYER INCENTIVES To get employees excited about H.S.A.'s, many firms will contribute directly to
your account. For example, two-thirds of Fidelity Consulting Group's clients contribute an average
of \$750 per employee into health savings accounts, said William Applegate, vice president of
product management at Fidelity.

In addition, many companies offer high-deductible plans that pay for annual physicals and other
wellness programs. In January, the new health care law will make it mandatory for all health plans
to pay for preventive and wellness services. That means if you are relatively healthy, you may well
be able to dodge out-of-pocket health costs for routine care, allowing your H.S.A. balance to grow
and roll over into the next year.

RECORD YOUR PAYMENTS Use your H.S.A. money only for qualified medical expenses, and always keep
receipts so you can prove your expenditures. (Check your benefits office or insurance provider for
guidelines.) Under the new health care law, any unqualified purchases made with H.S.A. money will be
subject to a 20 percent penalty starting in January, up from the current 10 percent.

Keep in mind another change also beginning in January: over-the-counter pharmaceuticals will no
longer be eligible for reimbursement from an H.S.A. unless prescribed by a doctor. If you were
accustomed to paying for, say, Claritin or Motrin with H.S.A. dollars, you'll have to recalculate.

COMPARE PRICES FOR CARE Although it can be difficult to find accurate pricing information, there are
some resources you will want to turn to if you are paying for more than routine care, like an M.R.I.
or surgery. Most insurance companies offer price information on their Web sites. And independent
sites like PriceDoc.com and HealthcareBlueBook.com offer comparisons for practitioners and common
medical procedures.

Mr.~Derrow simply calls around for the best deal. When he was suddenly paying full price for a
prescription medicine he took regularly, he said, ``You can't believe how fast I was on the phone
comparing prices at area pharmacies. I had never bothered to do that before, when I knew it was
simply a \$25 co-pay no matter where I went.''

\pagebreak
\section{Anger in Hong Kong Over Manila Siege}

\lettrine{D}{rawn}\mycalendar{Aug.'10}{30} by a mixture of anger and grief, tens of thousands of
Hong Kong residents poured into the streets on Sunday to protest how the Philippine government
handled a bus siege in Manila last Monday that ended in the shooting deaths of eight Hong Kong
residents and the dismissed police officer who had taken them hostage.

Protest organizers estimated the crowd at 80,000 people, but the police put it at 30,000. Either
figure would make it the largest protest march in memory against events overseas, although there
have been much larger protests in Hong Kong involving local politics or events in mainland China,
notably the Tiananmen Square killings in 1989.

Wearing black and white, with yellow ribbons tied around their upper arms in remembrance of the
dead, the crowd gathered in sweltering heat in Victoria Park and then marched peacefully more than a
mile to the downtown business district before dispersing quietly. A police spokeswoman said Sunday
evening that no arrests had been made.

Many marchers seemed to be fairly apolitical, soft-spoken members of the middle class who said they
had never attended a demonstration before but were offended that the Philippine government had
failed to protect the Hong Kong residents aboard the bus. The dismissed police officer, armed with
an M-16 assault rifle, had repeatedly been visible during the siege, even waving to onlookers from
the bus door, but police snipers had not tried to shoot him through most of the siege.

``Their performance is not acceptable,'' said Michael Kong, 33, a logistics manager who came with
his wife, Anna Ho, a telecommunications manager of the same age; both said they had never previously
marched for any cause.

Demonstrators demanded a full investigation into the bus siege. President Benigno S.~Aquino III of
the Philippines drew particular criticism from marchers who said he did not show adequate contrition
and remorse.

``We don't think that he has apologized to us,'' said Rachel Lam, a 23-year-old student who also
said that she had never participated in a demonstration before. ``It is very impolite.''

The bus killings have prompted some concern that Hong Kong residents might show antagonism toward
Filipinos; Hong Kong's population of seven million includes more than 100,000 live-in Filipino
domestic helpers, who come to the city on special work visas.

Domestic helpers work six-day weeks for a legal minimum of \$460 a month plus room and board, with
no eligibility for overtime pay. Their presence in homes has long made them vulnerable to abuse,
because they frequently borrow heavily to reach Hong Kong but can be sent home at any time by their
employers.

Hong Kong's Equal Opportunity Commission, a government agency, issued a statement on Wednesday in
which it urged ``all members of the community to stay calm and, in line with our good tradition of
tolerance and understanding, refrain from shifting our anger towards an innocent group, particularly
the Filipinos who are living or traveling in Hong Kong.''

But there was no sign of malice toward Filipinos at the demonstration on Sunday. ``I won't be mad at
the local Filipinos,'' said Lin Hengchoi, 49, an electrical contractor who brought his 5-year-old
son, Ken, with him.

The Hong Kong government has strongly warned its people against traveling to the Philippines in the
near future, and large numbers of Chinese tourists have also reportedly headed home from vacations
there. But there was little sign among demonstrators on Sunday that the bus killings would
fundamentally change their view of the attractiveness of the Philippines as a tourist destination
for years to come.

Mr.~Lin predicted that he and other Hong Kong residents would continue to take vacations in the
Philippines.

``I think we will still go,'' he said.

\pagebreak
\section{Beijing Opera, a Historical Treasure in Fragile Condition}

\lettrine{B}{eijing}\mycalendar{Aug.'10}{30} -- ``Watch out for that sword,'' the rehearsal director
shouted.

``I don't want anybody's head getting cut off because you don't know what you're doing.''

Lots of weapons were on stage at the Beijing Opera Academy of China here the other day. Teenage
future opera stars were armed with lances, spears, swords and daggers as they carried out an
elaborately choreographed, intricate, stylized and acrobatic fight scene, all to the clash of
cymbals, drums, wooden clappers and a substantial orchestra of Chinese string and woodwind
instruments.

Here and there in this ever more steel and glass city where old neighborhoods disappear from one
month to the next, there is a glimpse of what the previous city was like -- quiet, tree-shaded
streets with small storefronts and bicycles, a locust tree leaning over a wall that hides an old
courtyard house.

This modest and slightly shabby theater in the academy exists in a neighborhood in the southwest
part of the city that has not been entirely torn down and rebuilt yet. The academy occupies the
former site of the Beijing Dance Academy and does not seem to have been physically upgraded or
modernized. It still has dingy corridors, ancient washrooms, rusting bunk beds (six to a room), a
single fluorescent bulb hanging from the ceiling and an ancient radiator in front of the window.

And, of course, nothing could more suggest old Beijing than Beijing opera, with its masks, its
stylized movements, its atonal, strangely modern arias, its fantastically intricate scenes of
battle, and, probably most important, its audience of connoisseurs who know when to shout a throaty
``hao!'' -- good! -- after an especially well-executed movement or song.

The worry though is that, like the city's old neighborhoods, Beijing opera could fall victim to
China's rampant commercialism and modernization. If it did, it would be a bit like Italy consigning
Verdi or Donizetti to a few small halls in Milan and Rome, or to those folkloric shows for tourists
who mostly do not know much about what they are seeing.

``Objectively speaking, right now there are some difficulties,'' said Qiao Cuirong, a senior
professor at the National Academy of Chinese Theater Arts, summing up the current state of Beijing
opera. ``People are interested in money and modernity and Western things, so our own culture has
lost something.''

It would be premature to say that Beijing opera has turned into an antique relic, but clearly it is
not what it was in the late 18th to early 20th century, when it was northern China's most popular
theatrical entertainment. The big national spectacles of recent years have included the 2008 Olympic
opening ceremony, which, while drawing on China's rich tradition, did not echo the traditional
opera. There was also the lavish production of Puccini's ``Turandot,'' directed by the celebrated
filmmaker Zhang Yimou. That production was a Western import that was once banned in this country
because it was deemed insulting to China.

Beijing opera certainly was not helped by the fact that during the turmoil of the 1966-76 Cultural
Revolution, the form was deemed feudalistic and reactionary. But then again so was just about every
other art form, including Western music and modern dance, both of which have since made vigorous
recoveries.

But Beijing opera faces particular difficulties, aside from the aging and fading away of a
knowledgeable audience.

``The more you know about Beijing opera, the more you love it,'' said Liu Hua, a former performer
and now a teacher at the school. ``The problem is that it takes a lot to know it, and fewer and
fewer people have the time or the inclination.''

Also, Beijing opera is an especially demanding form, both to perform and to witness.

``It takes a very long time to study, at least 8 to 10 years just to get in the door as a
performer,'' Mr.~Qiao said. ``And the whole thing is very slow. It's not like a movie, and right now
people want things to be fast. That's why we're losing the young crowd.''

Still, there seems, perhaps paradoxically, to be no shortage of students, as all those highly
talented and professional-looking teenagers on the school stage the other day indicated. Young
people start their training at age 11, going to one of the several Beijing opera academies around
the country aimed at producing professional performers.

``Children really like it,'' Mr.~Qiao said. ``Another reason is that some parents love it, and they
want their children to learn it, even if they're not thinking about having them become
professionals.''

The early training lasts for six demanding, rigorous years. Given that Beijing opera is fading in
popularity, especially among the younger generations, it seems strange that so many young people
would want to go through it.

``It's such good training that the students can go in almost any direction even if they don't end up
in the opera,'' Ms.~Liu said.

``A lot of our students end up on television or in the movies,'' she added. ``There are a lot of
martial arts movies, and our students are all good at martial arts. Some of them become popular
singers or actors. They're not worried about their future.''

The Chinese Ministry of Culture, anxious about the form's survival, lavishly subsidizes it,
renovating theaters, commissioning new works, paying substantial salaries to the bearers of the
tradition, like Mr.~Qiao.

This year, for the first time ever, the state-run Chinese Central Television has been holding a
national Beijing opera student competition, with the finals to be televised in October. During the
preliminaries in Beijing recently, 24 contestants, each with a supporting cast of extremely
acrobatic soldiers and others, took the stage in an awesome display of skill and talent.

The emphasis was on what a nonconnoisseur might think of as the best parts -- the battle and martial
arts scenes, with performers in astonishing costumes leaping and somersaulting in midair, twirling,
jabbing, tossing and juggling an arsenal of weapons and batons, singing at the same time.

And there was the knowledgeable audience -- theater entry was free, which is perhaps itself a sign
of the form's fragile standing with the public -- shouting approval and applauding enthusiastically.
``We teachers are doing our job, and the government's Culture Ministry is supporting us,'' Mr.~Qiao
said. ``Everybody's doing their best to keep this as a cultural treasure, whether people go to see
it or not.''

\pagebreak
\section{A High-Tech Titan Plagued by Potholes}

\lettrine{C}{all}\mycalendar{Aug.'10}{30} it India's engineering paradox.

Despite this nation's rise as a technology titan with some of the world's best engineering minds,
India's full economic potential is stifled by potholed roadways, collapsing bridges, rickety
railroads and a power grid so unreliable that many modern office buildings run their own diesel
generators to make sure the lights and computers stay on.

It is not for want of money. The Indian government aims to spend \$500 billion on infrastructure by
2012 and twice that amount in the following five years.

The problem is a dearth of engineers -- or at least the civil engineers with the skill and expertise
to make sure those ambitious projects are done on time and up to specifications.

Civil engineering was once an elite occupation in India, not only during the British colonial era of
carving roads and laying train tracks, but also long after independence as part of the civil
service. These days, though, India's best and brightest know there is more money and prestige in
writing software for foreign customers than in building roadways for their nation.

And so it is that 26-year-old Vishal Mandvekar, despite his bachelor's degree in civil engineering,
now writes software code for a Japanese automaker.

Mr.~Mandvekar works in an air-conditioned building with Silicon Valley amenities here in Pune, a
boomtown about 100 miles east of Mumbai. But getting to and from work requires him to spend a vexing
hour on his motorcycle, navigating the crowded, cratered roads between home and his office a mere
nine miles away.

During the monsoon season, the many potholes ``are filled with water and you can't tell how deep
they are until you hit one,'' he said.

Fixing all that, though, will remain some other engineer's problem.

Mr.~Mandvekar earns a salary of about \$765 a month. That is more than three times what he made
during his short stint for a commercial contractor, supervising construction of lodging for a Sikh
religious group, after he earned his degree in 2006.

``It was fun doing that,'' he said of the construction job. ``My only dissatisfaction was the pay
package.''

Young Indians' preference for software over steel and concrete poses an economic conundrum for
India. Its much-envied information technology industry generates tens of thousands of relatively
well-paying jobs every year. But that lure also continues the exodus of people qualified to build
the infrastructure it desperately needs to improve living conditions for the rest of its one billion
people -- and to bolster the sort of industries that require good highways and railroads more than
high-speed Internet links to the West.

In 1990, civil engineering programs had the capacity to enroll 13,500 students, while computer
science and information technology departments could accept but 12,100. Yet by 2007, after a period
of incredible growth in India's software outsourcing business, computer science and other
information technology programs ballooned to 193,500; civil engineering climbed to only 22,700.
Often, those admitted to civil engineering programs were applicants passed over for highly
competitive computer science tracks.

There are various other reasons that India has struggled to build a modern infrastructure, including
poor planning, political meddling and outright corruption. But the shortage of civil engineers is an
important factor. In 2008, the World Bank estimated that India would need to train three times as
many civil engineers as it does now to meet its infrastructure needs.

The government has ``kick-started a massive infrastructure development program without checking on
the manpower supply,'' said Atul Bhobe, managing director of S.~N. Bhobe \& Associates, a civil
engineering design company. ``The government is willing to spend \$1 trillion,'' he said, ``but you
don't have the wherewithal to spend that kind of money.''

Sujay Kalele, an executive with Kolte-Patil, a Pune-based developer of residential and commercial
buildings, said the company's projects could be completed as much as three months faster if it could
find enough skilled engineers.

``If we need 10 good-quality civil engineers, we may get four or five,'' Mr.~Kalele said.

Beyond construction delays and potholes, experts say, the engineering shortfall poses outright
dangers. Last year, for example, an elevated span that was part of New Delhi's much-lauded metro
rail system collapsed, killing six people and injuring more than a dozen workers. A government
report partly blamed faulty design for the accident; metro officials said they would now require an
additional review of all designs by independent engineers.

Acknowledging India's chronic shortage of civil engineers and other specialists, the national
government is building 30 universities and considering letting foreign institutions set up campuses
in the country.

``India has embarked on its largest education expansion program since independence,'' the prime
minister, Manmohan Singh, said in a speech last year in Washington.

But the government may have only so much influence on what students study. And while the Indian
government runs or finances some of the country's most prestigious universities, like the Indian
Institutes of Technology, fast-growing private institutions now train more students. About
three-quarters of engineering students study at private colleges.

Moreover, many civil engineers who earn degrees in the discipline never work in the profession or --
like Mr.~Mandvekar -- leave it soon after they graduate to take better-paying jobs in information
technology, management consulting or financial services.

Industry experts say a big obstacle to attracting more civil engineers is the paltry entry-level
pay. The field was considered relatively lucrative until the 1990s, when it was eclipsed by the pay
in commercial software engineering.

Ravi Sinha, a civil engineering professor at the Indian Institute of Technology, Bombay, says
professionals in his field with five years of experience make about as much as their counterparts at
information technology companies. But those starting can make as little as half the pay of their
technology peers.

That is partly because of the lead set by government departments, where salaries for civil engineers
are often fixed according to nearly immutable civil service formulas.

And in the private sector, developers and construction companies have often been reluctant to pay
more and invest in the training of young engineers, because executives believe that new graduates do
not contribute enough to merit more money or that they will leave for other jobs anyway.

``If companies take a holistic view,'' Mr.~Sinha said, ``they have the opportunity to develop the
next generation's leaders.''

In fact, a construction boom in recent years has led to higher salaries in private industry.
Kolte-Patil now pays junior engineers \$425 a month, nearly twice the level of five years ago.

Larsen \& Toubro, a Mumbai-based engineering company that builds airports, power projects and other
infrastructure, offers Build India Scholarships for students who want to pursue a master's degree in
construction technology and management. The program produces 50 to 60 graduates a year, who are
hired by the company.

``You don't get the best quality in civil engineers, because today the first choice for students is
other branches'' of engineering, said K.~P. Raghavan, an executive vice president in L.\& T.'s
construction division. ``We are compensating with lots of training.''

\pagebreak
\section{For Obama, Steep Learning Curve as Chief in War}

\lettrine{P}{resident}\mycalendar{Aug.'10}{30} Obama rushed to the Oval Office when word arrived one
night that militants with Al Qaeda in Yemen had been located and that the military wanted to support
an attack by Yemeni forces. After a quick discussion, his counterterrorism adviser, John O.~Brennan,
told him the window to strike was closing.

``I've got two minutes here,'' Mr.~Brennan said.

``O.K.,'' the president said. ``Go with this.''

While Mr.~Obama took three sometimes maddening months to decide to send more forces to Afghanistan,
other decisions as commander in chief have come with dizzying speed, far less study and little
public attention.

He is the first president in four decades with a shooting war already raging the day he took office
-- two, in fact, plus subsidiaries -- and his education as a commander in chief with no experience
in uniform has been a steep learning curve. He has learned how to salute. He has surfed the Internet
at night to look into the toll on troops. He has faced young soldiers maimed after carrying out his
orders. And he is trying to manage a tense relationship with the military.

Along the way, he has confronted some of the biggest choices a president can make, often deferring
to military advisers yet trying to shape the decisions with his own judgments -- too much at times
for the Pentagon, too little in the view of his liberal base. His evolution from antiwar candidate
to leader of the world's most powerful military will reach a milestone on Tuesday when he delivers
an Oval Office address to formally end the combat mission in Iraq while defending his troop buildup
in Afghanistan.

A year and a half into his presidency, Mr.~Obama appears to be a reluctant warrior. Even as he draws
down troops in Iraq, he has been abundantly willing to use force to advance national interests,
tripling forces in Afghanistan, authorizing secret operations in Yemen and Somalia, and escalating
drone strikes in Pakistan. But advisers said he did not see himself as a war president in the way
his predecessor did. His speech on Tuesday is notable because he talks in public about the wars only
sporadically, determined not to let them define his presidency.

Where George W.~Bush saw the conflicts in Iraq and Afghanistan as his central mission and
opportunities to transform critical regions, Mr.~Obama sees them as ``problems that need managing,''
as one adviser put it, while he pursues his mission of transforming America. The result, according
to interviews with three dozen administration officials, military leaders and national security
experts, is an uneasy balance between a president wary of endless commitment and a military worried
he is not fully invested in the cause.

``He's got a very full plate of very big issues, and I think he does not want to create the
impression that he's so preoccupied with these two wars that he's not addressing the domestic issues
that are uppermost in people's minds,'' Defense Secretary Robert M.~Gates said in an interview.
Mr.~Obama, though, has devoted enormous time and thought to finding the right approaches, Mr.~Gates
added. ``From the first, he's been decisive and he's been willing to make big decisions,'' he said.

Senator Jack Reed, a Rhode Island Democrat who sometimes advises Mr.~Obama, said the president was
grappling with harsh reality. ``He came into office with a very sound strategic vision,'' Mr.~Reed
said, ``and what has happened in the intervening months is, as with every president, he is beginning
to understand how difficult it is to translate a strategic vision into operational reality.''

A former adviser to the president, who like others insisted on anonymity in order to discuss the
situation candidly, said that Mr.~Obama's relationship with the military was ``troubled'' and that
he ``doesn't have a handle on it.'' The relationship will be further tested by year's end when
Mr.~Obama evaluates his Afghanistan strategy in advance of his July deadline to begin pulling out.
As one administration official put it, ``His commander in chief role is about to get tested again,
and in a very dramatic way.''

\textbf{Beyond the Vietnam Debate}

Mr.~Obama was an 11-year-old in Hawaii when the last American combat troops left Vietnam, too young
to have participated in the polarizing clashes of the era or to have faced the choices the last two
presidents did about serving. ``He's really the first generation of recent presidents who didn't
live through that,'' said David Axelrod, his senior adviser. ``The whole debate on Vietnam, that was
not part of his life experience.''

Running for president of a country at war, he had plenty to learn, even basics like military
ceremonies and titles. His campaign recruited retired generals to advise him. But it still took time
to adjust when he became president. The first time he walked into a room of generals, an aide
recalled, he was surprised when they stood. ``Come on, guys, you don't have to do that,'' he said,
according to the aide.

Perhaps his most important tutor has been Mr.~Gates, the defense secretary appointed by Mr.~Bush and
the first kept on by a president of another party. They are an unlikely pair, a 49-year-old
Harvard-trained lawyer turned community activist and a 66-year-old veteran of cold war spy intrigues
and Republican administrations. But they are both known for unassuming discipline, and they bonded
through weekly meetings and shared challenges.

Mr.~Obama has relied on Mr.~Gates as his ambassador to the military and deferred to him repeatedly.
When Mr.~Gates wanted to force out Gen.~David D.~McKiernan in May 2009 as commander in Afghanistan
in favor of Gen.~Stanley A.~McChrystal, Mr.~Obama signed off. Likewise, cognizant of Bill Clinton's
ill-fated effort to end the ban on gay and lesbian soldiers, Mr.~Obama let Mr.~Gates set a slow pace
in overturning the ``don't ask, don't tell'' policy, even though it has disappointed gay rights
advocates.

Even on his signature campaign promise to pull out of Iraq, Mr.~Obama compromised in the early days
of his tenure to accommodate military concerns. Instead of the 16-month withdrawal of combat forces
he promised, he accepted a 19-month timetable, and he agreed to leave behind 50,000 for now rather
than a smaller force.

But as he grows in the job, Mr.~Obama has shown more willingness to set aside Mr.~Gates's advice.
When General McChrystal got in trouble in June for comments by him and his staff in Rolling Stone
magazine, Mr.~Gates favored reprimanding the commander. Mr.~Obama decided instead to oust him and
replace him with Gen.~David H.~Petraeus, who led the troop increase in Iraq.

``My first reaction was if McChrystal with his experience and his contacts and his knowledge were
pulled out, that could have real consequence for the war,'' Mr.~Gates said. ``It never even occurred
to me -- I kicked myself subsequently -- to move Petraeus over there. When the president raised that
with me in a private meeting, it was like a light bulb went on -- yes, that will work.''

Just as keeping Mr.~Gates provided political cover against the weak-on-defense Democratic image,
Mr.~Obama surrounded himself with uniformed officers. He kept Mr.~Bush's war coordinator,
Lt.~Gen.~Douglas E.~Lute, and tapped Gen.~James L.~Jones as national security adviser. ``Picking
General Jones was in part inoculation,'' said Bruce O.~Riedel, a senior fellow at the Brookings
Institution who led Mr.~Obama's first Afghanistan review.

But they were not always in control. General Jones has often been eclipsed by younger foreign policy
advisers with closer relationships with the president. Mr.~Obama ended up pushing out Adm. Dennis
C.~Blair as director of national intelligence, and approved the Afghan troop increase despite the
warnings of Lt.~Gen.~Karl W.~Eikenberry, his ambassador to Kabul.

Although General McChrystal was described in Rolling Stone as calling Mr.~Obama intimidated in
meeting with military commanders early in his tenure, other attendees disagreed. ``He didn't look to
me like he was one bit intimidated,'' Mr.~Riedel said. ``He did look like someone who was taking it
all in and a bit frustrated that what seemed for him to be simple questions he was getting
complicated answers to -- like how many troops do you really need?''

\textbf{Wars as a Distraction}

With the economy in tatters and health care on his agenda, Mr.~Obama was determined to keep the wars
from becoming a major distraction. When he held a videoconference on Iraq on his first full day in
office, officials recalled, he said: ``Guys, before you start, there's one thing I want to say to
you and that is I do not want to screw this up.''

But while he had given much thought to ending the war in Iraq, he had not spent as much time
contemplating Afghanistan despite a campaign promise to send more troops. When he took office, he
found an urgent request to reinforce the flagging effort. Warned by the generals that he could not
wait to study the issue, he overruled Vice President Joseph R.~Biden Jr.~and sent 21,000 more
troops. ``Both he and I frankly thought at that point we were done,'' Mr.~Gates recalled. Within
months, though, General McChrystal asked for 40,000 more troops. ``I certainly was surprised when
General McChrystal came in with the request,'' he said, ``and I think the president was as well.''

Reliant on Mr.~Gates, Mr.~Obama has made limited efforts to know his service chiefs or top
commanders, and has visited the Pentagon only once, not counting a Sept.~11 commemoration. He ended
Mr.~Bush's practice of weekly videoconferences with commanders, preferring to work through the chain
of command and wary, aides said, of being drawn into managing the wars.

So General McChrystal's request for even more reinforcements exposed the mutual mistrust,
particularly after it was leaked to the news media. The president complained he was being boxed in
while the military worried whether politics would drive the decision. At one point Denis
R.~McDonough, deputy national security adviser, pressed Adm. Mike Mullen, chairman of the Joint
Chiefs of Staff, about stopping leaks by the military, according to people informed about the
conversation. Admiral Mullen asked pointedly if that would also apply to the White House chief of
staff, Rahm Emanuel, who was skeptical of the troop increase request.

``If I had been in the White House, I would have been suspicious,'' Mr.~Gates said. ``The leak of
McChrystal's assessment was obviously very damaging in the assessment process because it put the
president on the spot.'' He added: ``My position was this is not a deliberate attempt to jam the
president. It's indiscipline.''

Last December, the president gave the military 30,000 more troops, but also a ticking clock. He
would start pulling troops out in July, on the grounds that if there was not visible progress by
then, it would mean the strategy was not working. Some saw that as a sop to his antiwar base. Others
considered it his way of reasserting control over a military that knows how to outmaneuver the White
House.

``He didn't understand or grasp the military culture,'' said Lawrence J.~Korb, a former Pentagon
official at the liberal Center for American Progress. ``He got over that particular quandary and put
them back in the box by saying, `O.K., I'm giving you 18 months.' ''

One adviser at the time said Mr.~Obama calculated that an open-ended commitment would undermine the
rest of his agenda. ``Our Afghan policy was focused as much as anything on domestic politics,'' the
adviser said. ``He would not risk losing the moderate to centrist Democrats in the middle of health
insurance reform and he viewed that legislation as the make-or-break legislation for his
administration.''

White House officials reject the linkage, but said Mr.~Obama believed that the wars should be judged
against other priorities. Preparing to announce his decision last December, he read Dwight
D.~Eisenhower's farewell address and included a line in his own speech at West Point: ``Each
proposal must be weighed in the light of a broader consideration: the need to maintain balance in
and among national programs.''

\textbf{Hungry for Information}

Mr.~Obama has made a point of seeking his own information, scribbling questions in memo margins and
scouring the Internet. At one meeting, he surprised the generals by citing a study of post-traumatic
stress disorder among soldiers serving repeat tours.

``He reads a lot,'' said General Jones, the national security adviser. ``He studies issues before he
comes to the table. That's another thing the military mind, if there is such a thing, appreciates.
When he sits down to talk about an issue, he's done his homework.''

Facing relentless and elusive foes, Mr.~Obama has turned increasingly to the sort of strikes he
authorized in Yemen and the drones in Pakistan, a form of warfare with little risk to American lives
even though critics question its wisdom, effectiveness or even morality.

But Mr.~Obama also confronts the consequences of the direct combat he has ordered. Last year, he
flew to Dover Air Force Base in Delaware to greet soldiers' coffins. During a later meeting with
advisers, Mr.~Obama expressed irritation at doubters of his commitment. ``If I didn't think this was
something worth doing,'' he said, ``one trip to Dover would be enough to cause me to bring every
soldier home. O.K.?''

In March, during his only trip to Afghanistan in office, he met a wounded soldier, maybe 19, who had
lost three limbs. ``I go into a place like this, I go to Walter Reed -- it's just hard for me to
think of anything to say,'' an emotional Mr.~Obama told advisers as he left.

The moment stuck with him. Three months later, after ousting General McChrystal, Mr.~Obama marched
into the Situation Room and cited the teenage triple amputee as he reprimanded advisers for the
infighting that had led to the general's forced resignation. ``We have a lot of kids on the ground
acting like adults and we have a lot of adults in this room acting like kids,'' he lectured.

The schisms among his team, though, are born in part out of uncertainty about his true commitment.
His reticence to talk much publicly about the wars may owe to the political costs of alienating his
base as well as the demands of other issues. Senior Pentagon and military officials said they
understood that he presided over a troubled economy, but noted that he was not losing 30 American
soldiers a month on Wall Street.

The sensitivities about calling attention to the unpopular war in Afghanistan, and particularly
America's problematic partner, played out when President Hamid Karzai visited last May. General
McChrystal and Ambassador Eikenberry wanted to take Mr.~Karzai to Fort Campbell in Kentucky to honor
troops leaving for Afghanistan, but the White House objected that it sent the wrong message, as if
Americans were fighting for Mr.~Karzai. They compromised by having Mr.~Gates go as well, but without
his Washington press corps.

``From an image point of view, he doesn't seem to embrace it, almost like you have to drag him into
doing it,'' said Peter D.~Feaver, a Bush adviser with military contacts. ``There's deep uncertainty
and perhaps doubt in the military about his commitment to see the wars through to a successful
conclusion.''

Much of the public too is confused about the president's Afghan strategy, as White House aides and
their critics acknowledge. ``There have only been a few moments when he's tried to focus the
nation's attention on Afghanistan because, quite frankly, it's competing with the other
priorities,'' said Richard Haass, president of the Council on Foreign Relations, who opposes the
strategy. ``It's probably one of the reasons public support has fallen, because they see the costs
but they don't know his thinking about it.''

If the flap over General McChrystal underscored the tensions, Mr.~Obama's response may have actually
helped ease them. ``Ironically enough, the McChrystal firing helped a lot because Obama handled it
exactly the way most senior military officers would have handled it if they had been in his shoes,''
said Stephen Biddle, a critic of Mr.~Obama's withdrawal deadline at the Council on Foreign
Relations.

Perhaps more important was his selection of General Petraeus to take over. The choice brings
Mr.~Obama full circle. As a senator, he opposed the Iraq troop increase led by General Petraeus, and
the two had a wary encounter in Baghdad when Mr.~Obama visited as a candidate in 2008. After
Mr.~Obama came to the White House, General Petraeus no longer had the regular interactions he had
with Mr.~Bush.

But Mr.~Obama came to appreciate General Petraeus's intelligence and dedication. He invited the
general to fly on Air Force One with him to West Point for his speech announcing the Afghanistan
troop increase. Six months later, after ousting General McChrystal, the president sent his personal
aide to find General Petraeus and bring him to the Oval Office for a one-on-one talk. The general
accepted the appointment without even a chance to call his wife.

``It's an extraordinary irony,'' said Mr.~Riedel, the former Obama adviser. ``He, like Bush before
him, has put all his bets down on the table on one guy -- and it's the same guy.''

\pagebreak
\section{Protests Fan Hong Kong Anger Over Manila Killings}

\lettrine{A}{}\mycalendar{Aug.'10}{30} rift between China and the Philippines deepened Saturday,
with protests and rallies planned for the weekend to demand an investigation into the hostage
standoff in Manila that left eight tourists from Hong Kong dead.

The announcement of a public rally organized by various political parties on Sunday reflected the
anger -- fed daily by new revelations of missteps -- that the Philippine police allowed the standoff
on Monday to drag on for 12 hours, throughout which the president, Benigno S.~Aquino III, avoided
phone calls from Hong Kong's chief executive.

``It is a very emotional time here,'' said Chan Kin-man, a sociology professor at the Chinese
University of Hong Kong. ``Many people in Hong Kong believe the lives of the tourists taken hostage
were not properly valued.''

Nearly 38,000 people in Hong Kong have signed condolence books for the victims, the government said.

The gunman, a 55-year-old police officer who had been dismissed after being charged with extortion,
opened fire inside a tourist bus near the end of the ordeal. Shortly after, the bus driver escaped,
screaming that everyone was dead, and the police finally moved in. The gunman, Rolando Mendoza, was
killed by a police sniper.

Ballistics tests are being done on the commandos' weapons, leaving open the question of whether the
rescuers may have shot some of the victims. Hong Kong's coroner's office has raised the possibility
of carrying out its own investigation, depending on the findings of the autopsies of the eight
victims.

Chinese fury was originally focused on the length of the standoff and the fact that the daylong
crisis played out on live television -- keeping the gunman, who was watching from a monitor inside
the bus, abreast of the actions of the police outside.

But since then, infuriating information has piled up.

A catalog of mistakes listed at a Senate hearing in Manila on Thursday included the news that the
ground commander insisted on using his own group of commandos, leaving a team with better training
and equipment from the police Special Action Force sitting idly at the scene. An offer of help from
a trained hostage negotiator was declined.

There have been suggestions that the responding police officers -- alienated themselves by a culture
of graft and favoritism -- may have sympathized with the hostage taker, who demanded a review of the
extortion case against him and his dismissal from the force.

Broadcasters have added to the outrage. On Friday, televised reports of the gunman's wake showed
family members placing the Philippine flag over his coffin. Chinese officials expressed ``strong
indignation.''

On Saturday, about 1,000 people joined the funeral procession for Mr.~Mendoza in Tanauan, south of
Manila.

After a television commentator, Anthony Yuen, suggested that Hong Kong had overreacted to the
shootings and that its chief executive, Donald Tsang, should not have tried to contact Mr.~Aquino
during the crisis, his employer, Phoenix TV in Hong Kong, suspended his program.

The episode has deeply damaged the Philippines' tourism industry, which relied heavily on visitors
from Hong Kong, and it has also wounded the new and hopeful presidency of Mr.~Aquino, the scion of
two of the country's most enduring symbols of democracy. He made erasing government corruption,
including police abuse, a campaign platform that reverberated with the public.

But the standoff has underlined those problems, said Kai-shing Wong, executive director of the Asian
Human Rights Commission, which is based here. Mr.~Wong said his group had chronicled episodes of
abuse by the police and military in the Philippines over the past 10 years, including allegations of
torture used in interrogations.

Mr.~Wong cited a recent case in which a Philippine broadcast of a video captured on a cellphone was
said to show police officers in plain clothes torturing a man at a Manila police station.

``It's not enough to probe this case,'' Mr.~Wong said. ``The Aquino government must take an in-depth
look at the law enforcement system.''

The Philippines scores low in a variety of rankings on safety and civil society. The Institute for
Economics and Peace ranks the country as 130th out of 149 on its latest Global Peace Index. The
stakes of the inquiry into the standoff are high for Mr.~Aquino, who took office in June.

\pagebreak
\section{China's Growth Leads to Problems Down the Road}

\lettrine{C}{hinese}\mycalendar{Aug.'10}{30} authorities proclaimed an end this week to an epic
traffic jam that had brought some drivers here to a dead halt for up to five seemingly endless days.
Which is heartening news, save two problems.

One is that the traffic jam has not ended. ``That's impossible,'' an officer at the Zhangjiakou
Highway Traffic Police Detachment said Friday. ``All the lanes are filled up. If you get on the
highway from Inner Mongolia to Hebei, you'll be stuck for four or five days.''

The other is that it may not end until, oh, 2012.

The Great Chinese Gridlock of 2010 -- up to 60 miles long, on a freeway linking Beijing and Inner
Mongolia's capital, Hohhot -- has earned a welter of global publicity this month on tales of drivers
marooned for days in immobile traffic lanes, and profiteering locals selling them freeze-dried
noodles at usurious prices.

``I spent five days and five nights last week without moving,'' a trucker who conceded only his last
name, Li, said during a roadside chat outside this city on Thursday. ``Apart from sleeping, you just
eat. And you can only eat the instant noodles.'' These cost about 45 cents, from a roadside hawker,
plus \$1.20, for the water needed to soften them.

The gridlock has been building for up to a year, the inevitable result of the difficulty of China's
construction crews in keeping up with China's breakneck growth.

In this case, a government decision to satisfy surging demand for electric power by tapping Inner
Mongolia's coalfields has flooded local highways with thousands of coal trucks, overwhelming police
officers' best efforts to herd them.

The government is building two new rail lines on the trucks' route, one for coal and the other for
freight, as well as a second passenger-only line to relieve congestion. But those railroads will not
open until at least 2012, and perhaps later.

And so huge traffic jams of the sort that plagued this road in August are all but guaranteed to
continue. Indeed, logistics experts here say the miracle is that more such bottlenecks do not occur.

``China probably does a better job of executing on this kind of big infrastructure than almost any
other country, anytime, anywhere,'' said John Scales, in charge of transport issues for the World
Bank's Beijing office. But even in China, where niceties like environmental impact statements are
dispensable, planning and executing huge construction projects takes years, not months.

The challenges facing Chinese builders are clear from the statistics, which by themselves are
staggering.

This nation has been on a building binge for decades -- and indeed, the highway from Beijing as it
begins its way toward Mongolia would largely be familiar to any American interstate highway driver.
In 2000, China boasted about 7,450 miles of such expressways. A decade later, it has 40,400 miles,
not much smaller than the American system, which it plans to leapfrog by 2020.

Rail construction has moved almost as quickly: 2,500 miles of new track a year, the Communications
Ministry says, along with upgrades on existing rail lines to improve trains' speed and carrying
capacity.

But the government's construction plans have not dovetailed with its equally vast energy plans.
Electricity output has more than doubled just since 2000, and coal-burning plants produce about
two-thirds of that power, compared to one-half in the United States. Shanxi Province, in Central
China, once was the main coal source for power plants, but recent production and worker-safety
problems there led the government to tap bigger coal deposits in Inner Mongolia, in China's far
north.

Therein lies a problem. Mongolian coal production has exploded -- up 37 percent to 637 million tons
last year alone, with an additional 15 percent increase expected this year. Much of the coal is
supposed to move to seaports on China's east coast, to be shipped to big cities in the south. But
pig-in-python style, even China's brand-new freeway system cannot handle the volume.

On an ordinary freeway, the 300-mile drive from Hohhot to Beijing would consume several hours. Here,
China's coal haulers say, the same trip generally requires up to three days' travel, including
weight checks and unloading coal. Recent traffic jams have pushed travel time to a week or more.

But even continuing east to Beijing -- a six-lane stretch that winds past popular Great Wall tourist
sites -- traffic jams can stall drivers for hours. On a recent evening, a passenger whiled away two
hours on a deadened stretch 60 miles from Beijing, as thousands of coal trucks idled and vendors
darted among the vehicles, selling apples and other treats.

``The more roads they build, the more congested it gets,'' one trucker, 45-year-old Wang Haihe,
volunteered. ``And then they build some more roads.''

\pagebreak
\section{Hey, Big Spender: Hollywood Isn't in the Mood}

\lettrine{J}{oel}\mycalendar{Aug.'10}{30} silver stands on the Warner Brothers lot and points to the
remnants of a house where he filmed parts of four ``Lethal Weapon'' movies. ``We blasted a toilet
out of that window,'' he says, smiling proudly. ``Over there, we drove a car straight into the
living room.''

Ah, the glory days.

Behind Mr.~Silver, the flamboyant producer of some of the biggest action hits of the last 30 years,
is the modest set for one of his current films, an R-rated comedy with no stars, almost no budget
and -- for now -- no title. Not that Mr.~Silver was ready to call the production small. ``It's a
little movie, but it's a big little movie,'' he says.

And therein lies Mr.~Silver's challenge: How does a larger-than-life, free-spending producer fit
into a movie business that has been tightening up -- and cutting some of its more grandiose
characters down to size?

In the new Hollywood, stars count for less, whether in front of the camera or behind it. Financial
firepower and technological wizardry matter more. And a generation of producers -- whose principal
assets were their industry connections and a remarkable degree of personal force -- are having to
adapt.

Mr.~Silver, 58, has been a dominant studio moviemaker for over three decades, delivering blockbuster
franchises like ``Lethal Weapon,'' ``Die Hard'' and ``The Matrix.'' The 59 movies he has produced
have generated almost \$10 billion in ticket sales, adjusting for inflation. The money he has made
for Warner alone has won him lavish treatment from the studio -- not just in compensation, but also
in perks. To make him happy, Warner once went so far as to send movie props to his Brentwood mansion
for his son's birthday party.

Warner, at least in years past, has ignored Mr.~Silver at its own peril. Six years ago, Jeff
Robinov, then a top production executive at the studio, was hospitalized after a motorcycle
accident. As he recovered, Mr.~Robinov heard that Mr.~Silver was exaggerating the severity of the
accident -- and telling people that Mr.~Robinov was unable to function.

When Mr.~Robinov asked Mr.~Silver why he was doing this, the producer said it was because the Warner
executive hadn't been returning his calls promptly.

Despite such antics, producers like Mr.~Silver used to be able to count on one studio or another to
support them in near perpetuity. So what if they fell on hard times -- as Mr.~Silver has, recently
delivering a string of flops like ``Speed Racer'' (one of the biggest money-losers in Warner's
87-year history), ``Ninja Assassin,'' ``Whiteout'' and the aptly titled ``The Losers.''

Studios no longer take such losses lightly. Bleeding from plummeting DVD sales and higher marketing
costs, they've started reducing producer deals. Warner alone has cut the number of producers it
carries by 20 percent over the last two years and has said more reductions are on the way. The
producers Warner now favors are mostly young and inexpensive or come with financial backing of their
own from outsiders, like Legendary Pictures, which teamed up with Warner to make ``The Dark
Knight.''

Warner has also been building up the production companies of directors and actors like Zack Snyder,
Ben Affleck and Todd Phillips, all of whom now challenge Mr.~Silver in a pecking order that changed
when old images of Hollywood producers -- who survived by wit, will and the occasional outrageous
moment -- began fading to black.

A particularly difficult point for both Warner and Mr.~Silver is the cost of his production deals.
In a frothier time, the lucrative arrangements struck by Mr.~Silver allowed him to get a cut of the
revenue from his films. That means he is entitled to about 8 cents of every dollar the studio takes
in for his pictures, whether they are bombs or runaway hits.

Warner is also required to distribute films from Mr.~Silver's production company Dark Castle, which
self-finances horror and other low-budget movies with \$240 million in private funding. In theory,
the deal gives Warner films from an experienced producer without risking its own production money.
In practice, the arrangement has sometimes backfired, as it did earlier this year with ``Splice,'' a
thriller about a pair of scientists who use genetic manipulation to create a monstrous child.

Mr.~Silver acquired rights to ``Splice'' at little cost. But Warner spent about \$26 million to
market the film, only to see it come up short, with just \$17 million at the domestic box office.

Against backdrops like this, Hollywood studios are nudging entrenched producers away from prized but
risky projects, if only to avoid paying them millions of dollars in participation fees while the
studio loses money.

For instance, Mr.~Silver was entrusted for years with developing ``Wonder Woman'' into a big-budget
movie. Warner recently took the superheroine away from him, to exert more control and to allow
other, less expensive producers to take a shot at it.

So even though Hollywood has always been the fabled land of comebacks and second acts -- and
Mr.~Silver recently found success with ``Sherlock Holmes'' -- the megaproducer also knows that his
head may be perilously close to the chopping block. His deal with Warner, which provides for a staff
of about 20, expires in December 2011; negotiations for a new contract haven't started.

Mr.~Robinov, now president of Warner's motion picture division, declined to comment on whether the
studio would renew Mr.~Silver's deal or simply pressure him into a more restrictive contract.

``Joel is an incredible cinephile, who is incredibly intelligent and incredibly passionate about his
job,'' says Mr.~Robinov. ``That's a lot to bring to the party.''

For his part, Mr.~Silver is playing the role of the stoic.

``Maybe I will continue with Warner and maybe I won't,'' he says over a dinner of goulash and
brussels sprouts inside his trailer. ``I hope I do.''

Still, some of his powerful friends seem worried. At the very least, they are rallying around him.

``Warner's is very fortunate to have Joel Silver,'' said Ron Meyer, president of Universal Studios.

``Let's hope he doesn't take a bullet from anybody. He's a good guy,'' says Terry Semel, Warner's
former chairman. ``Even home-run hitters have cold streaks. It's the nature of sports and it's the
nature of movies.''

BRUCE BERMAN, the chief executive of Village Roadshow Pictures, who has known Mr.~Silver since 1979
when they worked together on ``Xanadu,'' says no producer working in Hollywood better understands
the pull of mass entertainment.

``That's incredibly valuable,'' he says. Even so, Mr.~Berman allows that his pal ``is a 20th-century
man in a 21st-century world.''

Mr.~Silver, burly and bearded, has been parodied in several movies, most recently by Tom Cruise in
``Tropic Thunder,'' but he is far from the only megawatt producer under pressure or needing to
figure out a new way forward.

Brian Grazer, who operates under a deal at Universal Pictures, stumbled in the spring with ``Robin
Hood,'' which managed to squeeze about \$310 million at the global box office but cost more than
\$200 million to make -- when including higher-than-normal start-up costs and excluding tax credits
-- and more than \$100 million to market.

Scott Rudin has one of the most buzzed-about movies of the fall in ``The Social Network,'' but Walt
Disney Studios, where he is based, has made clear that its new strategy leaves little room for the
kind of highbrow films in which he specializes.

Even Jerry Bruckheimer, producer of such fare as the ``Pirates of the Caribbean'' series, is
struggling to move beyond four high-profile disappointments in a row, including ``G-Force'' and
``The Sorcerer's Apprentice,'' both of which required Disney to take quarterly write-downs.

Mr.~Silver's fortunes may be turning. Warner is bullish on his next movie, ``Unknown White Male,'' a
Dark Castle thriller starring Liam Neeson that is scheduled for release in January. And Mr.~Silver
rightly points out that ``Sherlock Holmes'' was a smash hit that cost \$80 million to make and sold
more than \$523 million of tickets globally. Mr.~Silver is starting production on a sequel.

``Sherlock Holmes,'' however, comes with an asterisk on Mr.~Silver's r\'esum\'e. The picture was
already well under way when he came on board.

Warner and the longtime producer on the project, Lionel Wigram, wanted Robert Downey Jr.~to play the
lead role. Mr.~Downey said yes -- but only if his wife, Susan Downey, helped produce it.

That created a pickle: Mrs.~Downey worked for Mr.~Silver, who told Warner that he would lend out his
executive -- but only if he was brought on the project, too. Mr.~Robinov said O.K., in part because
of Mr.~Silver's close relationship with Guy Ritchie, who directed the movie.

Mr.~Downey says he owes the resuscitation of his career to Mr.~Silver, who cast him in the 2005
thriller ``Kiss Kiss Bang Bang'' despite the actor's prison and drug record.

``Joel just kept telling me, `We've got to get a gun in your hand,' '' Mr.~Downey says. ``Joel is
one of the few relationships I care to have with a producer. Look, he's vast and voracious, and he
definitely has the ability to break into a scream about a point he would like to make. But he can
also be incredibly warm and generous.''

Questions about money -- how to get it, how to spend it, how to pay it back -- have loomed unusually
large in Mr.~Silver's Hollywood life.

``Fantasies about spending are irresistible, the best!'' Mr.~Silver was quoted as saying in the
production notes for ``Brewster's Millions,'' a 1985 fable in which Richard Pryor was supposed to
waste \$30 million in 30 days in order to inherit a much larger fortune. ``Who wouldn't want to
indulge the luxury of squandering millions and millions of dollars?''

In 1989, Mr.~Silver used the Warner jet to fly a bevy of pals and business associates to party at
Auldbrass Plantation, his South Carolina home designed by Frank Lloyd Wright. Hollywood, still in
its boom years, saw such extravagance as part of his charm.

``I'm not in this business to make art; I'm in it to make money to buy art,'' ran a much-quoted
Silver quip that the producer now regrets as a shade too glib. His art collection, however, includes
a 20-ton sculpture by Richard Serra; it is kept on the grounds of Mr.~Silver's Brentwood estate,
Casa de la Plata (Spanish for ``House of Silver'').

Expensive taste -- he also owns a Malibu house and is chauffeured around Hollywood in a Maybach
sedan -- has at times appeared to leave Mr.~Silver pressed for funds.

Bertram H.~Fields, the Hollywood überlawyer who has long worked for Mr.~Silver, says Mr.~Silver has
relied on a longstanding series of loans from Warner. He declined to describe the size of the loans
but said that Mr.~Silver had the ability to repay them.

Filings with the secretary of state of California show that Mr.~Silver's debt is secured against his
interest in various films, including, most recently, ``Splice.'' Mr.~Fields says the financial
relationship between Mr.~Silver and Warner is comfortable.

``He's their prime supplier and they do lend him money,'' he says. ``It's kind of a running account
between them.''

Mr.~Silver, who on occasion rumbles through the courts with the same animal spirit he brings to the
Warner lot, is now suing Goldman Sachs, the investment bank, and a pair of movie companies, Alliance
Films and Momentum Pictures USA, for \$35 million.

In a complaint filed in May, Mr.~Silver contended that Goldman and the film companies had breached
an obligation to pay him that sum for his interest in Dark Castle after he helped Goldman find
financing for an acquisition of Alliance.

Lawyers for Goldman and the film companies filed a response opposing all of Mr.~Silver's claims,
calling some of them ``absurd.'' A Goldman spokeswoman and a law firm representing Goldman, Alliance
and Momentum declined to comment.

EVEN brief chats with Mr.~Silver are embroidered with a cin\'easte's knowledge. During a recent
conversation in his Warner office, he recounted, beat for beat, Adolphe Menjou's long walk through a
newsroom and virtually every aspect of a paper's daily life in a memorable scene from the 1931
version of ``The Front Page.''

Inevitably, the talk also turned to the costly trappings of the office, which was built for Frank
Sinatra in 1963. It had fallen into disrepair, Mr.~Silver explains, so he persuaded Warner to allow
him to refurbish it.

Now, the office boasts accouterments like eel-skin furniture and exterior walls washed to look like
William Randolph Hearst's castle at San Simeon. (An adjoining part of the complex was once used by
Mr.~Hearst's lover, Marion Davies.) Among the trophies on the inside are photographs of Mr.~Sinatra
in his rat-pack heyday and the head of a zebra that Mr.~Silver is quick to note was bought, not
stalked.

``I hunted that with a credit card,'' he says.

Mr.~Silver grew up middle class in South Orange, N.J., the son of a public-relations-executive
father and a journalist mother. (She wrote a food column for the New Jersey section of The New York
Times.)

Mr.~Silver had a number of passions during his New Jersey years -- one of them was Frisbee. In the
fall of 1968, while in high school, he jokingly proposed adding a Frisbee game he had created with
some friends to the school curriculum. After the game spread to local colleges, Ultimate Frisbee was
born.

But, at heart, Mr.~Silver was a movie lover first, taking in ``Lawrence of Arabia'' at a Times
Square theater in Manhattan and obscure art films at the Museum of Modern Art.

Mr.~Silver recalls watching film credits scroll by on television and looking up the names of crew
members in the Los Angeles telephone book. ``That's how fascinated I was with movies,'' he recalls.

Mostly, however, he was drawn to the producers, men like David O.~Selznick. ``I was fascinated with
the lifestyle of these guys -- how they lived, how they ran the show,'' he says. ``These guys lived
like Saudi princes.''

After arriving in Hollywood in the 1970s, Mr.~Silver got his break from the producer Lawrence
A.~Gordon, who needed an assistant. Mr.~Gordon eventually employed him as an executive at his
company and later worked with him as a partner. Mr.~Silver quickly absorbed Hollywood's
rough-and-tumble ways. In 1991, he and Mr.~Gordon parted ways and still do not speak to each other.

Mr.~Silver now has over 25 films in active development, including a splashy adaptation of ``Logan's
Run,'' a 1976 movie about a futuristic society in which humans are terminated when they turn 30.
Another project involves the comic book character Sgt. Rock.

``The core of the movie business remains intact,'' says Mr.~Silver. ``And it's not descending in
scope. Studios want movies that are bigger than ever.''

PERHAPS, but studios also want small -- something that Mr.~Silver is trying to address with Project
X, that unnamed teenage comedy in production on the Warner back lot.

Wearing a pair of his limited-edition Bathing Apes sneakers, Mr.~Silver monitors a rehearsal of a
fight sequence in the film. An actress, Kirby Bliss Blanton, runs over to him and gives him a hug.

``I love you!'' she beams, before walking away.

Mr.~Silver is startled by the hug, but it barely registers because he is too focused on other
things, like a tricky scene coming up involving nudity.

``There aren't a lot of guys like me left,'' Mr.~Silver says during a break. ``But I'm a war horse.
I've been through it all. And you know something about war horses? Through the sleet, through the
snow -- they just keep going.''

\pagebreak
\section{Obama Says Iraq Combat Mission Is Over}

\lettrine{P}{resident}\mycalendar{Sept.'10}{01} Obama formally declared an end to the combat mission
in Iraq Tuesday night, telling the nation that, after seven years of war that claimed more than
4,400 American lives, it is ``time to turn the page'' toward another war, Afghanistan, and toward
pressing problems at home.

In an address from the Oval Office -- only his second as president -- Mr.~Obama reminded Americans
that, in giving responsibility for Iraqi security to the Iraqis, he was fulfilling a promise he made
while running for office. He conceded that Americans are ``understandably asking tough questions''
about Afghanistan, but urged the nation to stick with him on that war.

``We must never lose sight of what's at stake,'' Mr.~Obama said. Sounding much like his predecessor,
former President George W.~Bush, he warned, ``As we speak, al Qaeda continues to plot against us.''

But it was clear that, at a time when Americans are anxious about the economy, Mr.~Obama also wanted
to use the address to pivot toward problems at home. As he praised the courage and resolve of the
American troops, he reminded the nation of the blood and treasure that had been spilled during the
Iraq war, and said it is time for him to focus on his ``central responsibility'' as president:
restoring the economic health of the nation.

``At this moment, as we wind down the war in Iraq, we must tackle those challenges at home with as
much energy, and grit, and sense of common purpose as our men and women in uniform who have served
abroad,'' Mr.~Obama said. ``They have met every test that they faced. Now, it is our turn. Now, it
is our responsibility to honor them by coming together, all of us, and working to secure the dream
that so many generations have fought for -- the dream that a better life awaits anyone who is willing
to work for it and reach for it.''

Mr.~Obama also spoke directly to Iraqi leaders, urging them to ``move forward with a sense of
urgency'' to form a government -- a task that has eluded them since elections were held there earlier
this year. He vowed Iraq would have ``a strong partner in the United States,'' declaring, ``Our
combat mission is ending, but our commitment to Iraq's future is not.''

For Mr.~Obama, who had opposed the Iraq war from the outset, one of the central challenges of the
speech was conveying to the nation that he did not believe those killed had lost their lives in
vain.

Another was whether -- or how -- to mention former President Bush, who was the architect of the 2007
troop buildup, opposed by Mr.~Obama, that many military experts say helped stabilize Iraq enough to
make the troop withdrawal possible. Many Republicans, including Senator John McCain of Arizona, had
called on the president to publicly credit his predecessor for the surge policy.

Mr.~Obama did not. Mentioning that he had spoken by telephone with Mr.~Bush earlier in the day, he
said, ``It's well known that he and I disagreed about the war from the outset. Yet no one could
doubt President Bush's support for our troops, or his love of country and commitment to our
security.''

White House officials had said Mr.~Obama had several objectives for the speech: He wanted to praise
and thank the troops. He wanted to explain the policy in Iraq going forward. He wanted to pivot the
nation toward Afghanistan. And he wanted to make the case that he would be intensely focused on the
nation's economic woes.

But Mr.~Obama also sought to use the speech to draw some broader lessons about American power and
relationships around the world. Noting that a new push for Middle East peace will begin on Wednesday
in Washington, the president sought to make the case that the United States can, and will, lead
efforts to promote peace and democracy.

``One of the lessons of our effort in Iraq is that American influence around the world is not a
function of military force alone.'' Mr.~Obama said, adding, ``We must project a vision of the future
that is based not just on our fears, but also on our hopes -- a vision that recognizes the real
dangers that exist around the world, but also the limitless possibility of our time.''

Calling the shift away from a combat mission an ``historic moment,'' and ``a milestone,'' Mr.~Obama
acknowledged that it comes at a time of great uncertainty for many Americans.

Earlier in the day, he had said that there is a ``tough slog'' ahead in the war in Afghanistan, as
he told troops in Fort Bliss, Texas, on Tuesday.

Speaking just hours before he delivered the Oval Office address, he addressed the situation in Iraq,
where some 50,000 troops will remain until next year in a mainly advisory and training role,
Mr.~Obama warned that the American mission was not yet accomplished. Mr.~Obama told the troops that
his address was ``not going to be a victory lap; it's not going to be self-congratulatory. There's
still a lot of work.''

Mr.~Obama's address tonight was meant to convey that he has kept one of the central promises of his
campaign: withdrawing American combat troops from Iraq. But he is tiptoeing a fine line between
taking credit for the withdrawal and echoing the ``mission accomplished'' tone that President Bush
struck so famously seven years ago, and that came back to haunt Mr.~Bush in the years that Iraq fell
into further chaos.

Mr.~Obama called Mr.~Bush on Tuesday morning from Air Force One as he was en route to Fort Bliss,
White House officials said. The two spoke ``just for a few moments,'' Benjamin Rhodes, a national
security spokesman, told reporters aboard the plane, declining to give any details.

In rolling out the promises-kept theme on the Iraq withdrawal, Mr.~Obama is trying to reconcile his
record of opposition to the war, and to the troop surge ordered by President Bush which many
military officials credit for stemming violence in Iraq, with his role as a war-time commander
seeking to credit his troops with a mission accomplished.

While there were ``big debates about war and peace'' across the country, the president said, ``the
one thing we don't argue about is that we have the finest fighting force in the history of the
world.'' He got shouts of approval from the assembled Fort Bliss troops for that line. (The event
was covered by a small pool of print and television reporters.)

``The main message I have tonight and the main message I have to you is congratulations on a job
well done,'' Mr.~Obama said. ``The most pride I take in my job is being your commander in chief.''

On Afghanistan, Mr.~Obama said that he was convinced that under the command of Gen.~David
H.~Petraeus, the latest American military commander in Afghanistan, ``we have the troops who are
there in a position to start taking the fight to the terrorists.'' But he warned that there would be
heavy casualties.

In Baghdad, Vice President Joseph R.~Biden Jr.~met with Prime Minister Nouri al-Maliki and other
senior Iraqi politicians. At each meeting, Mr.~Biden previewed the main themes in President Obama's
speech and made the point that the United States wanted a long-term relationship with Iraq.

But another topic of the closed-door discussions was the Iraqis' faltering efforts to form a
government, months after elections.

After the meetings a senior administration officials declined to say if any headway had been made.

``We do come away from this day believing negotiations are extremely active and that's positive,''
he said. ``But we still need to see the Iraqis move forward and actually come to agreement and form
a government.''

As the start of the morning meeting with Mr.~Maliki, Mr.~Biden suggested that reports of increased
violence in Iraq had been exaggerated.

``Notwithstanding what the national press says about increased violence, the truth is things are
still very much different,'' Mr.~Biden said. ``Things are much safer.''

Still, Mr.~Biden traveling party got a vivid warning of the remaining threats when three alerts
sounded that the Green Zone was under possible mortar or rocket attack and they were instructed to
take cover.

In Washington, a senior intelligence official told reporters that Al Qaeda in Mesopotamia is
estimated to have just 10 percent of the strength it had during the peak of its manpower in 2006 and
2007. The official declined to provide the actual figures that the estimate was drawn from, and he
said he expected the group would have a core of fighters inside Iraq for ``a long time to come.''

The senior official, who declined to be identified because he was discussing classified intelligence
assessments, noted the sharp reduction of violence in the country and said that attacks in Iraq had
been lowered to a ``tolerable'' level.

He said that Iran's Revolutionary Guard Corps continued to provide ``equipment, training and a
refuge'' for various militant groups in the country, and that Iran's support for these groups was
almost certain to continue as the United States reduces its military presence in the country.

Earlier in the day, Defense Secretary Robert M.~Gates sounded a restrained, sober note about the
state of America's two wars in remarks to the American Legion in Milwaukee.

In Iraq, he said, the most recent elections have yet to result in a coalition government, Al Qaeda
in Mesopotamia is beaten but not gone, and sectarian tensions remain. He said the 50,000 United
States troops still in Iraq would continue to work with Iraqi security forces, who only last week
faced a flurry of coordinated insurgent attacks across the country that killed at least 51 people.

``I am not saying that all is, or necessarily will be, well in Iraq,'' Mr.~Gates, who is one of
Mr.~Obama's most influential advisers, told the legion. And he warned against ``premature victory
parades or self-congratulations.''

In Afghanistan, he said, the Taliban are ``a cruel and ruthless adversary, and are not going
quietly.'' Their leadership, he said, has ordered a campaign of intimidation against Afghan
civilians and is singling out women for brutal attacks.

``I know there is a good deal of concern and impatience about the pace of progress since the new
strategy was announced last December,'' Mr.~Gates said, referring to Mr.~Obama's decision to send to
Afghanistan 30,000 additional United States troops, who have finished arriving only this month.
Total American forces in Afghanistan now number nearly 100,000.

But in an attempt to draw a parallel between the current fragile stability of Iraq and what might be
possible in Afghanistan, Mr.~Gates said that the intensifying combat and rising casualties in
Afghanistan were in many ways reminiscent of the early months of the surge of United States forces
ordered in Iraq by President George W.~Bush in 2007, when American troops were taking the highest
losses of the war.

``Three and a half years ago very few believed the surge could take us to where we are today in
Iraq, and there were plenty of reasons for doubts,'' said Mr.~Gates, who helped make the surge
decision as Mr.~Bush's defense secretary at the time. But ``back then, this country's civilian and
military leadership chose the path we believed had the best chance of achieving our national
security objectives, as we are doing in Afghanistan today.''

He added: ``Success there is not inevitable. But with the right strategy and the willingness to see
it through, it is possible. And it is certainly worth the fight.''

Despite his cautious tone on Iraq, Mr.~Gates cited what he called dramatic security gains. He said
that violence levels this year remained at their lowest level since the beginning of the war in
2003, that American forces have not had to conduct an airstrike in six months and that Al Qaeda in
Mesopotamia had been largely cut off ``from its masters abroad.''

Mr.~Gates said the gains had been purchased ``at a terrible cost'': 4,427 American service members
killed, 34,268 Americans wounded or injured and untold losses and trauma endured by the Iraqis
themselves.

Mr.~Gates's voice seemed to choke as he then said: ``The courage of these men and women, their
determination, their sacrifice -- and that of their families -- along with the service and sacrifice
of so many others in uniform, have made this day, this transition, possible. And we must never
forget.''

In Afghanistan, Mr.~Gates promised that the United States would take a hard line against corruption
in the Afghan government. He also echoed Mr.~Obama and senior military commanders by saying that the
president's deadline for the start of withdrawals of United States forces from Afghanistan next July
would be a gradual beginning, not a massive departure.

``If the Taliban really believe that America is heading for the exits next summer in large numbers,
they'll be deeply disappointed and surprised to find us very much in the fight,'' he said.

\pagebreak
\section{Confronting Multiple Problems, Obama Faces Tough Odds}

\lettrine{P}{resident}\mycalendar{Sept.'10}{01} Obama is attempting a triple play this week that
eluded his predecessors over the past two decades: simultaneous progress on the most vexing and
violent problems in the Middle East -- Israeli-Palestinian peace, Iraq and Iran -- in hopes of
creating a virtuous cycle in a region prone to downward spirals.

History shouts that all the odds are against him. White House officials, eager to show concrete
progress on the hardest foreign policy challenges at a time when Mr.~Obama is struggling with a
variety of domestic issues, contend that that the president has changed the political climate in all
three arenas and has the best shot in years at creating positive and interlocking results.

When President Bill Clinton tried a similar strategy, he argued that a comprehensive peace between
the Israelis and Palestinians would make it easier for Arab nations to join in the ``dual
containment'' of Iran and Saddam Hussein's Iraq. It turned out that the reverse was true as well:
When one of those efforts fell apart, so did the other two.

A month before invading Iraq, President George W.~Bush argued that toppling Saddam Hussein would
create ``a dramatic and inspiring example of freedom for other nations in the region,'' leading Arab
countries ``to support the emergence of a peaceful and democratic Palestine, and state clearly they
will live in peace with Israel.'' Instead, Iraq went up in flames and hopes for peace collapsed.
Iran accelerated its drive for a nuclear capability.

Mr.~Obama's argument, which formed one subtext of his speech to the nation on Tuesday night about
the end of the American combat mission in Iraq and which will play out Wednesday and Thursday as he
gathers Israeli and Palestinian leaders for their first direct talks in two years, is more subtle
about the linkage among the issues.

``There are three big chess pieces here, and in each of those places we are now poised for
success,'' Rahm Emanuel, Mr.~Obama's chief of staff and a major voice in Middle East policy, said in
an interview on Tuesday. He argued that while the linkages are loose, ``victory begets victory, and
success will be reinforcing.''

While Mr.~Obama's thinking contains elements of the logic that drove his predecessors, there are
also some critical differences, and success or failure hinges on how significant those turn out to
be. Those differences include evidence that the United States is truly pulling out of Iraq, far
tougher sanctions on Iran and the tentative emergence of a working Palestinian government in the
West Bank.

The main problem is that success is not assured in any of the fronts in question, and the dynamic
among them is unpredictable.

``It's hard to make the case that progress in the peace process is going to resolve the political
stalemate in Iraq, or force the Iranians to reconsider their nuclear program,'' said Martin
S.~Indyk, who served as American ambassador to Israel and now is the director of foreign policy at
the Brookings Institution. ``But I think you can claim that success would help make headway in
isolating Iran, and Iran's claims to leadership in the region would be challenged. The risk -- the
one we forgot in the Clinton years -- is that failure can also diminish your credibility.''

It is in Iraq, a war Mr.~Obama campaigned to end, where he is claiming progress. While Iraq's
fractious politicians have still not agreed on a government nearly six months after an election and
insurgents have landed some punishing recent attacks, overall violence has fallen and the withdrawal
from combat missions happened a few weeks ahead of schedule. ``It is clear in Iraq a genuine
political process is under way,'' said Dennis B.~Ross, Mr.~Obama's top Middle East adviser.

Still, Mr.~Obama is loath to declare anything resembling victory, and he said Tuesday that a ``tough
slog'' remained. The question is whether the American public is willing to see more money and lives
spent there while Iraqi politicians argue.

As Ryan C.~Crocker, the former American ambassador to Iraq, wrote recently in The National Interest:
``Strategic patience is often in short supply in this country. It is not a new problem for us, and
it is not limited to Iraq.''

While 50,000 American troops remain in Iraq for now, Mr.~Obama made clear Tuesday night that he was
intent on moving on from that war, proclaiming that his primary mission now was to jump-start the
American economy and address domestic issues like energy and education.

But as the Iranians have learned in recent months, Mr.~Obama also seems persistent in finding new
ways to turn the screws, and that is another element of the strategy.

When Mr.~Obama came to office, three successive sets of international sanctions against Iran had had
little effect, and there was virtually no prospect of getting a fourth.

It took 17 months for Mr.~Obama to build the case for another round, and to orchestrate far more
damaging additional measures -- enforced by Europe, Japan, Australia and even some Arab nations --
that have cut gasoline imports into Iran, sliced access to most foreign banking, and made it
enormously difficult for shippers to obtain the insurance they need to go in and out of foreign
ports.

``We finally have leverage,'' said Mr.~Ross, noting that for the first time Iranian officials have
started calling for resumed talks with the West.

But few believe that the pain will cause Iran to give up its nuclear enrichment program. In fact,
Iran could respond by speeding it up. There is also the possibility, some believe the probability,
that Iran will seek to do whatever it can to prevent the direct talks between Israel and the
Palestinians from becoming fruitful.

Still, Mr.~Obama's advisers argue that conditions have never been better for those talks: Attacks on
Israel are down and the government of President Mahmoud Abbas of the Palestinian Authority has
brought infrastructure, policing and better living to the West Bank. Majorities in Israel and among
the Palestinians say they want a two-state solution. But many analysts are pessimistic that either
side is willing to make the sacrifices necessary to achieve it.

The big question is whether the image of America pulling out of Iraq, and of the White House
re-engaging in the peace process, will be enough to create that virtuous cycle.

``In none of these areas have we achieved success,'' Mr.~Ross said. ``But now we have the
possibility and the potential for significant progress.''

\pagebreak
\section{Strong Exports Lift Agriculture, a Bright Spot in U.S.~Economy}

\lettrine{E}{ven}\mycalendar{Sept.'10}{01} as the broader economy falters amid signs of a weakening
recovery, the nation's agriculture sector is going strong, bolstered in part by a surge in exports,
according to federal estimates of farm trade and income released on Tuesday.

The estimates confirm what economists have been saying for months: agriculture, which was generally
not hit as hard by the recession as many other segments of the economy, remains a small bright spot
going forward.

``We're just having a robust rebound in the agricultural sector and promises of more growth,'' Jason
R.~Henderson, vice president and economist at the Omaha branch of the Federal Reserve Bank of Kansas
City, said in a recent interview.

The estimates show that American farmers will ship \$107.5 billion in agricultural products abroad
in the fiscal year that ends Sept.~30. That is the second-highest amount ever, behind the record
\$115.3 billion in exports logged in 2008, when commodity prices soared as the global demand for
agricultural products was helped by fast-growing economies in the developing world.

Next year, exports are expected to total \$113 billion. In releasing the data, Tom Vilsack, the
secretary of agriculture, said exports of grains and meats were leading the rebound. He called the
new estimates ``very encouraging.''

The export growth is propelled by higher prices for many products, including wheat, whose prices
have skyrocketed as drought and punishing heat decimated crops in Russia, Ukraine and Kazakhstan.
Exports to Asia have been particularly strong, and China is forecast to surpass Mexico next year to
become the second-largest foreign buyer of American farm products. Canada is the No.~1 export
market.

Wheat exports this year are estimated at \$6 billion, about the same as last year, as much of this
year's crop had already been sold when prices started to rise. But wheat exports for the fiscal year
2011 are expected to rise to \$8 billion, because of higher prices and increased production.

Prices for other grains have risen, too, encouraging farmers.

``The better the demand, the higher the price, and it's going to put another 10, 15, possibly 20
cents in the price of a bushel of corn,'' said Bill Horan, a corn farmer in Iowa. Corn is about \$4
a bushel, which is about 50 cents higher than last year. ``It means my wife can go out and buy a new
sofa, and I can put new tires on the pickup.''

Prices have also risen significantly for cotton, meat and dairy products. Cotton exports are
expected to reach \$6 billion next year, up from \$4.8 billion this year and \$3.5 billion last
year, on the strength of a large crop here and tight worldwide supplies that have lifted prices.

Despite such increases, prices for most agricultural commodities remain well short of the record
levels of 2008. And the price paid to farmers is only a small portion of the end cost of most foods.
So economists predict that the prices consumers pay in the supermarket will rise only moderately
this year.

Other economic measures were also promising for the farm sector, which accounts for a small fraction
of the overall economy but has a strong impact on the well-being of many rural areas, and a ripple
effect for suppliers and other related industries.

Total net farm cash income for the current calendar year was estimated at \$85 billion, a 23 percent
increase from last year and well above the 10-year average of \$72 billion.

About 75 percent of farm production occurs on just 271,000 farms, or 12 percent of the total farms
in the country. Those large commercial farms were forecast to average \$220,000 in net cash income
this year, a 22 percent rise from a year ago.

When all farms are taken into account, average farm household income is expected to be \$81,670 this
year, a nearly 6 percent increase from last year.

Household income for many who live on farms comes largely from off-farm jobs and other sources, like
investments. This year, on average, 11 percent of the household income for farm families was
predicted to come from agriculture.

Income from both farm and nonfarm sources is expected to rise this year, indicating an overall
improvement in the rural economy, officials said.

Joseph Glauber, the agriculture department's chief economist, said that a strong rebound in
livestock and dairy prices had been a major factor in the farm recovery.

Dairy producers were hurt badly in the recession by high costs and low prices, which have recently
begun to recover. Cattle and hog producers also struggled with low prices, caused by overproduction.
But cattle and hog producers have managed to cut the size of their herds, pushing prices back up at
the same time that international demand recovers, Mr.~Glauber said.

``Exports are kind of driving our market,'' said Jason Anderson, who operates a cattle feedlot in
Holbrook, Neb. ``Demand is pretty good, and we've seen about a \$5 to \$7 price rally just this
month,'' he said, referring to the price per hundredweight.

Economists said that the farm sector overall was not hurt as badly in the recession because farmers
generally had better access to credit. At the same time, farms over all were not highly leveraged,
putting them in a better position to withstand the economic storm.

``The farm economy in rural America has not suffered as severely as the industrial part of the
economy and, because of the strong exports, the rural economy is recovering what it lost during the
downturn,'' said Roy Bardole, a farmer in Rippey, Iowa, and chairman of the Soybean Export Council.

\pagebreak
\section{After Years of War, Few Iraqis Have a Clear View of the Future}

\lettrine{T}{he}\mycalendar{Sept.'10}{01} invasion of Iraq, occupation and tumult that followed were
called Operation Iraqi Freedom back then. It will be named New Dawn on Wednesday.

But America's attempt to bring closure to an unpopular war has collided with a disconnect familiar
since 2003: the charts and trend lines offered by American officials never seem to capture the
intangible that has so often shaped the pivots in the war in Iraq.

Call it the mood. And the country, seemingly forever unsettled and unhappy, is having a slew of bad
days.

``Nothing's changed, nothing!'' Yusuf Sabah shouted in the voice of someone rarely listened to, as
he waited for gas in a line of cars winding down a dirt road past a barricade of barbed wire, shards
of concrete and trash turned uniformly brown. ``From the fall of Saddam until now, nothing's
changed. The opposite. We keep going backwards.''

Down the road waited Haitham Ahmed, a taxi driver. ``Frustrated, sick, worn out, pessimistic and
angry,'' he said, describing himself.

``What else should I add?''

The Iraq that American officials portray today -- safer, more peaceful, with more of the trappings
of a state -- relies on 2006 as a baseline, when the country was on the verge of a nihilistic
descent into carnage. For many here, though, the starting point is the statement President George
W.~Bush made on March 10, 2003, 10 days before the invasion, when he promised that ``the life of the
Iraqi citizen is going to dramatically improve.''

Iraq generates more electricity than it did then, but far greater demand has left many sweltering in
the heat. Water is often filthy. Iraqi security forces are omnipresent, but drivers habitually
deride them for their raggedy appearance and seeming unprofessionalism. That police checkpoints
snarl traffic does not help.

What American officials portray as their greatest accomplishment -- a nascent democracy, however
flawed -- often generates a rueful response. ``People can't live only on the air they breathe,''
said Qassem Sebti, an artist.

In a conflict often defined by unintended consequences, the March election may prove a turning point
in an unexpected way. To an unprecedented degree, people took part, regardless of sect and
ethnicity.

But nearly six months later, politicians are still deadlocked over forming a government, and the
glares at the sport-utility vehicles that ferry them and their gun-toting entourages from
air-conditioned offices to air-conditioned homes, after meetings unfailingly described as
``positive,'' have become sharper.

Disenchantment runs rife not with one faction or another, but with an entire political class that
the United States helped empower with its invasion.

``The people of Kadhimiya mourn for the government in the death of water and electricity,'' a
tongue-and-cheek banner read near a Shiite shrine in Baghdad.

The year 2003, when the Americans invaded, often echoes in 2010, as they prepare to leave. Little
feels linear here these days; the sense of the recurrent is more familiar.

Lines at fuel stations returned this month, that testament to one the greatest of Iraq's ironies: a
country with the world's third largest reserve of oil in which people must endure long waits for
gas.

``Ghamidh'' was the word heard often in those earliest years. It means obscure and ambiguous, and
then, as now, it was typically the answer to any question.

``After seven years our destiny is still unknown,'' Mr.~Sabah said, waiting in a gas line. ``When
you look to the future, you have no idea what it holds.''

Complaints over shoddy services paraphrase the same grievances of those anarchic months after Saddam
Hussein's fall. The sense of the unknown persists, as frustration mounts, Iraqi leaders bicker and
no one seems sure of American intentions, even as President Obama observes what the administration
describes as a turning point in the conflict.

``I challenge anyone to say what has happened, what's happening now and what will happen in the
future,'' Mohammed Hayawi, a bookseller whose girth matched his charm, said as sweat poured down his
jowly face on a hot summer day in 2003.

Mr.~Hayawi died in 2007, as a car bomb tore through his bookstore filled with tomes of ayatollahs,
predictions by astrologers and poems of Communist intellectuals. This week, in the same shop, still
owned by his family, Najah Hayawi reflected on his words, near a poster that denounced ``the
cowardly, wretched bombing'' that had killed his brother.

``There is no one in Iraq who has any idea -- not only about what's happened or what's happening --
but about what will happen in the future,'' he said. ``Not just me, not just Mohammed, God rest his
soul, but anyone you talk to. You won't find anyone.''

Iraqis call the overthrow of Mr.~Hussein's government the ``suqut.'' It means the fall. Seven years
later, no one has yet quite defined what replaced it, an interim as inconclusive as the invasion was
climactic. ``Theater,'' Mr.~Hayawi's brother called it, and he said the populace still had no hand
in writing a script that was in others' hands.

``The best thing is that I have no children,'' Shahla Atraqji, a 38-year-old doctor, said back in
2003, as she sipped coffee at Baghdad's Hunting Club to the strains of Lebanese pop. ``If I can't
offer my children a good life, I would never bring them into this world.''

This week, Thamer Aziz, a doctor who helps fit amputees with artificial limbs at the Medical
Rehabilitation Center, stared at Musafa Hashem, a 6-year-old boy who lost his right leg in a car
bomb in Kadhimiya in July. His father was paralyzed.

``I've believed this for a long time, and I still do,'' he said. ``I cannot get married and have a
family because I may lose them any minute, by a bomb or bullet.''

``Just like him,'' he said, gesturing toward the boy.

Even in the denouement of America's experience here, old habits die hard.

On Monday, four American Humvees drove the wrong way down a street, turrets swinging at oncoming
traffic. Cars stopped, giving them distance. The Humvees turned, plowed over a curb, dug a trench in
the muddy median, then rumbled on their way.

``See! Did you see?'' asked Mustafa Munaf, a storekeeper.

``It's the same thing,'' he said, shaking his head. ``What's changed?''

\pagebreak
\section{H.P. to Work With Hynix on New Computer Memory Chips}

\lettrine{H}{ewlett-Packard}\mycalendar{Sept.'10}{01} said Tuesday that it would commercialize a new
computer memory technology with Hynix, the South Korean chip maker.

Hynix's agreement to build computer memories using a technology H.P. scientists developed called
memristors indicates that more computer memory will be packed in even smaller devices in the second
half of this decade. The two companies said the memristors will be commercially available in about
three years.

To date, the memristor's most likely application is for dense nonvolatile memories, which is what is
used in flash memory cards for products like cameras and PCs. It is not out of the question,
however, that it might play a role in other kinds of chips, including microprocessors, in the
future.

The agreement to build the memory chips validates the work of Leon O.~Chua, a University of
California, Berkeley, electrical engineering professor. In 1971, he proposed a fourth basic circuit
element (the other three are the resistor, capacitor and inductor) and called it a memristor, or
memory resistor, as a simpler alternative to transistors. The idea languished for many years before
a team of H.P. researchers found a way to use it in 2006. Since then, memristors have attracted
industrial, academic and military interest, but have not gone beyond being laboratory curiosities.

Competing in the memory business will not be an easy battle. Memristors are still viewed as
laboratory and academic experiments by the majority of the world's leading semiconductor firms, most
of whom have settled on a competing technology known as Phase Change Memory, or P.C.M. However, H.P.
scientists said they traveled the world discussing memristors with all of the leading chip makers
before settling on their commercial development agreement with Hynix, the world's second-largest
maker of memory chips behind Samsung Electronics.

``Right now the memristor outperforms flash,'' said Stan Williams, an H.P. Labs scientist who has
led the development effort. He said the tiny switches could be turned on and off more than 100 times
as fast as flash, use a tenth of the energy and have a much greater lifespan.

The storage densities are already staggering and will become even more impressive in the future.
Next year the most advanced flash storage chips will have a capacity of roughly 64 billion bits per
square centimeter, according to the industry's annual road map. By 2014, that is expected to
increase to 170 billion bits per square inch. Rice University scientists said that memristive
storage devices could be five times as dense as the industry standard in 2014 and that the
technology was more easily adaptable to three-dimensional packaging. That would make it possible to
build even vastly denser chips.

H.P. researchers have described ways to design 1,000-layer memristor-based chips, although they
acknowledged that with current manufacturing techniques such devices would not be practical.

\pagebreak
\section{Washington Post Suspends Columnist for Twitter Hoax}

\lettrine{O}{n}\mycalendar{Sept.'10}{01} Monday morning, Mike Wise, a sports columnist at The
Washington Post, published to his Twitter account that the Pittsburgh Steelers quarterback Ben
Roethlisberger would be suspended for five games.

Now Mr.~Wise himself is suspended.

The information Mr.~Wise published about Mr.~Roethlisberger was made up -- a test, he said, of how
fast a piece of misinformation could spread online. (Mr.~Roethlisberger was suspended for six games,
not five, after he was accused of sexual assault in March, and the N.F.L. is considering whether to
reduce the suspension.) Mr.~Wise followed his initial post about the quarterback with three others
about his sourcing for the news. And by the end of the day, the paper had suspended him for a month.

On his radio program on Washington's WJFK on Tuesday morning, Mr.~Wise said, ``I could give you 10
reasons why I did this and explain what went wrong in the execution. But none of it matters today. I
made a horrendous mistake.''

Kris Coratti, a spokeswoman for The Post, said that she could not comment on personnel issues.
Before joining the paper in 2004, Mr.~Wise worked as a reporter at The New York Times for 10 years.

Sree Sreenivasan, a digital media professor at the journalism school of Columbia University, said
that because Twitter was often a source of news -- athletes posting updates about their contracts,
for example -- the posts of newsmakers and journalists were taken seriously by followers.

A journalist's reputation ``is on the line with every tweet, for better or worse,'' Mr.~Sreenivasan
said. ``People have a reasonable expectation that it's accurate or the best of what you know at the
moment.''

Within a few hours of Mr.~Wise's Twitter post on Monday, The Post's sports editor, Matthew Vita,
sent an e-mail to his staff reminding it of the paper's guidelines for using social media.

``When you use social media, remember that you are representing The Washington Post, even if you are
using your own account,'' Mr.~Vita wrote. ``This is not to be treated lightly.''

Yet within The Post, there was disagreement about whether the punishment fit the crime. Andrew
Alexander, the paper's ombudsman, wrote on his blog that Mr.~Wise was ``lucky he wasn't fired.''
Howard Kurtz, the paper's media writer, wrote in a message on Twitter that the suspension ``seems
overly harsh to me.''

But on his radio program in Washington on Tuesday, Mr.~Wise said that he agreed with the suspension.

``I'm paying the price I should for careless, dumb behavior in the multiplatform media world,'' he
said.

\pagebreak
\section{Formula to Grade Teachers' Skill Gains in Use, and Critics}

\lettrine{H}{ow}\mycalendar{Sept.'10}{01} good is one teacher compared with another?

A growing number of school districts have adopted a system called value-added modeling to answer
that question, provoking battles from Washington to Los Angeles -- with some saying it is an
effective method for increasing teacher accountability, and others arguing that it can give an
inaccurate picture of teachers' work.

The system calculates the value teachers add to their students' achievement, based on changes in
test scores from year to year and how the students perform compared with others in their grade.

People who analyze the data, making a few statistical assumptions, can produce a list ranking
teachers from best to worst.

Use of value-added modeling is exploding nationwide. Hundreds of school systems, including those in
Chicago, New York and Washington, are already using it to measure the performance of schools or
teachers. Many more are expected to join them, partly because the Obama administration has prodded
states and districts to develop more effective teacher-evaluation systems than traditional classroom
observation by administrators.

Though the value-added method is often used to help educators improve their classroom teaching, it
has also been a factor in deciding who receives bonuses, how much they are and even who gets fired.

Michelle A.~Rhee, the schools chancellor in Washington, fired about 25 teachers this summer after
they rated poorly in evaluations based in part on a value-added analysis of scores.

And 6,000 elementary school teachers in Los Angeles have found themselves under scrutiny this summer
after The Los Angeles Times published a series of articles about their performance, including a
searchable database on its Web site that rates them from least effective to most effective. The
teachers' union has protested, urging a boycott of the paper.

Education Secretary Arne Duncan weighed in to support the newspaper's work, calling it an exercise
in healthy transparency. In a speech last week, though, he qualified that support, noting that he
had never released to news media similar information on teachers when he was the Chicago schools
superintendent.

``There are real issues and competing priorities and values that we must work through together --
balancing transparency, privacy, fairness and respect for teachers,'' Mr.~Duncan said. On The Los
Angeles Times's publication of the teacher data, he added, ``I don't advocate that approach for
other districts.''

A report released this month by several education researchers warned that the value-added
methodology can be unreliable.

``If these teachers were measured in a different year, or a different model were used, the rankings
might bounce around quite a bit,'' said Edward Haertel, a Stanford professor who was a co-author of
the report. ``People are going to treat these scores as if they were reflections on the
effectiveness of the teachers without any appreciation of how unstable they are.''

Other experts disagree.

William L.~Sanders, a senior research manager for a North Carolina company, SAS, that does
value-added estimates for districts in North Carolina, Tennessee and other states, said that ``if
you use rigorous, robust methods and surround them with safeguards, you can reliably distinguish
highly effective teachers from average teachers and from ineffective teachers.''

Dr.~Sanders helped develop value-added methods to evaluate teachers in Tennessee in the 1990s. Their
use spread after the 2002 No Child Left Behind law required states to test in third to eighth grades
every year, giving school districts mountains of test data that are the raw material for value-added
analysis.

In value-added modeling, researchers use students' scores on state tests administered at the end of
third grade, for instance, to predict how they are likely to score on state tests at the end of
fourth grade.

A student whose third-grade scores were higher than 60 percent of peers statewide is predicted to
score higher than 60 percent of fourth graders a year later.

If, when actually taking the state tests at the end of fourth grade, the student scores higher than
70 percent of fourth graders, the leap in achievement represents the value the fourth-grade teacher
added.

Even critics acknowledge that the method can be more accurate for rating schools than the system now
required by federal law, which compares test scores of succeeding classes, for instance this year's
fifth graders with last year's fifth graders.

But when the method is used to evaluate individual teachers, many factors can lead to inaccuracies.
Different people crunching the numbers can get different results, said Douglas N.~Harris, an
education professor at the University of Wisconsin, Madison. For example, two analysts might rank
teachers in a district differently if one analyst took into account certain student characteristics,
like which students were eligible for free lunch, and the other did not.

Millions of students change classes or schools each year, so teachers can be evaluated on the
performance of students they have taught only briefly, after students' records were linked to them
in the fall.

In many schools, students receive instruction from multiple teachers, or from after-school tutors,
making it difficult to attribute learning gains to a specific instructor. Another problem is known
as the ceiling effect. Advanced students can score so highly one year that standardized state tests
are not sensitive enough to measure their learning gains a year later.

In Houston, a district that uses value-added methods to allocate teacher bonuses, Darilyn Krieger
said she had seen the ceiling effect as a physics teacher at Carnegie Vanguard High School.

``My kids come in at a very high level of competence,'' Ms.~Krieger said.

After she teaches them for a year, most score highly on a state science test but show little gains,
so her bonus is often small compared with those of other teachers, she said.

The Houston Chronicle reports teacher bonuses each year in a database, and readers view the size of
the bonus as an indicator of teacher effectiveness, Ms.~Krieger said.

``I have students in class ask me why I didn't earn a higher bonus,'' Ms.~Krieger said. ``I say:
`Because the system decided I wasn't doing a good enough job. But the system is flawed.' ''

This year, the federal Department of Education's own research arm warned in a study that value-added
estimates ``are subject to a considerable degree of random error.''

And last October, the Board on Testing and Assessments of the National Academies, a panel of 13
researchers led by Dr.~Haertel, wrote to Mr.~Duncan warning of ``significant concerns'' that the
Race to the Top grant competition was placing ``too much emphasis on measures of growth in student
achievement that have not yet been adequately studied for the purposes of evaluating teachers and
principals.''

``Value-added methodologies should be used only after careful consideration of their appropriateness
for the data that are available, and if used, should be subjected to rigorous evaluation,'' the
panel wrote. ``At present, the best use of VAM techniques is in closely studied pilot projects.''

Despite those warnings, the Department of Education made states with laws prohibiting linkages
between student data and teachers ineligible to compete in Race to the Top, and it designed its
scoring system to reward states that use value-added calculations in teacher evaluations.

``I'm uncomfortable with how fast a number of states are moving to develop teacher-evaluation
systems that will make important decisions about teachers based on value-added results,'' said
Robert L.~Linn, a testing expert who is an emeritus professor at the University of Colorado,
Boulder.

``They haven't taken caution into account as much as they need to,'' Professor Linn said.

\pagebreak
\section{New Job Means Lower Wages for Many}

\lettrine{A}{fter}\mycalendar{Sept.'10}{01} being out of work for more than a year, Donna Ings, 47,
finally landed a job in February as a home health aide with a company in Lexington, Mass., earning
about \$10 an hour.

Chelsea Nelson, 21, started two weeks ago as a waitress at a truck stop in Mountainburg, Ark.,
making around \$7 or \$8 an hour, depending on tips, ending a lengthy job search that took her young
family to California and back.

Both are ostensibly economic success stories, people who were able to find work in a difficult labor
market. Ms.~Ings's employer, Home Instead Senior Care, a company with franchises across the country,
has been expanding assertively. Ms.~Nelson's restaurant, Silver Bridge Truck Stop, recently reopened
and hired about 20 people last month in an area thirsty for jobs.

Both women, however, took large pay cuts from their old jobs -- Ms.~Ings worked for a wholesale
tuxedo distributor, Ms.~Nelson was a secretary. And both remain worried about how they will make
ends meet in the long run.

With the country focused on job growth and with unemployment continuing to hover above 9 percent,
comparatively little attention has been paid to the quality of the jobs being created and what that
might say about the opportunities available to workers when the recession finally settles. There are
reasons for concern, however, even in the early stages of a tentative recovery that now appears to
be barely wheezing along.

For years, long before the recession began, job growth had become increasingly polarized in this
country. High-paid occupations that require significant amounts of education and training grew
rapidly alongside low-wage, service-type jobs that do not, according to David Autor, a labor
economist at the Massachusetts Institute of Technology.

The growth of these low-wage jobs began in the 1980s, accelerated in the 1990s and began to really
take off in the 2000s. Losing out in the shuffle, Dr.~Autor said, were jobs that he described as
``middle-skill, middle-wage'' -- entry-level white-collar positions, like office and administrative
support work, and certain blue-collar jobs, like assembly line workers and machine operators.

The recession appears to have magnified that trend, Dr.~Autor wrote in a recent paper, released
jointly by the Center for American Progress, a left-leaning policy group, and the Hamilton Project,
which has a more centrist reputation. From 2007 to 2009, the paper said, there was relatively little
net change in total employment for both high-skill and low-skill occupations, while employment
plummeted in so-called middle-skill occupations.

A new analysis by the National Employment Law Project, a liberal advocacy group, takes a different
approach, identifying industries that have experienced job growth in 2010 and examining their median
wages. It is a blunter measurement because it focuses on whole industries, within which there is
often great diversity in income. Economists also cautioned that it was still too early to know
exactly which sectors would eventually lead the way in a sustained recovery.

Nevertheless, the law project analysis offers a snapshot of where the employment growth has been so
far. It found that job expansion to this point had been skewed toward industries with median wages
that are low to middling, with a disproportionate share of job growth happening in industries whose
median wages fell below \$15 an hour.

``There's a striking contrast so far between which industries have lost jobs and which ones are
growing,'' said Annette Bernhardt, policy director for the law project. ``If this kind of
bottom-heavy job creation continues, it could pose a real challenge to restoring consumer demand and
making sure working families have a way to support themselves.''

Both studies are disquieting because of the potential import for many who had once scratched out
middle-class livings and are now looking for work. A unifying theme is the stubborn march of
labor-intensive, low-paying service jobs, like the ones Ms.~Ings and Ms.~Nelson found.

There is typically a downward slide during recessions, said Till von Wachter, a Columbia University
economist, in which higher-skilled and higher-educated workers are re-employed first, often landing
jobs for which they are overqualified, squeezing out the lesser skilled and lesser educated. Indeed,
in the current downturn, the unemployment rate has climbed the most for the least-educated workers,
suggesting they have been hit the hardest.

However, while researching workers who lost their jobs in California in the 1990s, Dr.~Wachter found
that people who fell in the middle when it came to their educational background -- possessing high
school degrees or some college -- and to the skills required for their occupation tended to
experience larger and longer lasting income losses after job loss than people on both the lower and
higher end of the scale.

Ms.~Ings had worked in a variety of office and administrative roles in the wholesale tuxedo
industry. Her wages of just over \$16 an hour were enough to build a relatively comfortable life for
her and her daughter, Jillian, now 21 and in college.

``During her whole growing up, I never got child support,'' Ms.~Ings said. ``I always had to try to
find a job that paid well to help support her. That's my job, being a mother.''

When Ms.~Ings was laid off in March 2009, she dove into finding another ``corporate job.'' But she
found that nearly everyone seemed to be looking for people with at least a college degree, if not
more. She had only a high school diploma.

As a teenager, Ms.~Ings had worked in a nursing home and enjoyed it. So, after getting her certified
nursing assistant license, she applied at the Home Instead office in Lexington, which has been
steadily hiring, said Jack Cross, the franchise owner. Nationally, the company has created more than
2,400 jobs this year, and home health aides are one of the country's fastest growing occupations.

Ms.~Ings adores her job, but her finances remain taut, even though she is working 50 hours a week.
She had been without health insurance for her first few months, but soon the company will begin
deducting for it -- a further pinch on her already meager paycheck.

``I'm going to be coming home with nothing,'' she said.

In Arkansas, Ms.~Nelson has been hampered by her decision several years ago to quit college after a
semester. She has worked a variety of jobs, including a three-year stint as a secretary, earning
about \$12 an hour.

Last year, she and her husband, Kenneth, and their son, Riley, now almost 2, moved to Colton,
Calif., where they had relatives and believed the job market would be better. They moved back to
Arkansas this year, however, after struggling to find steady work.

He quickly accepted a factory job at \$8 an hour, but she got rejection after rejection trying to
find office work.

Ms.~Nelson eventually gave up and took up waitressing. The couple is living with her mother, trying
to save enough for their own place.

``I don't know, with the jobs we have, if we're ever going to be able to make it on our own,''
Ms.~Nelson said.

\pagebreak
\section{Obama Weighs Smaller Measures on the Economy}

\lettrine{P}{resident}\mycalendar{Sept.'10}{01} Obama is weighing new steps to bolster the economy,
he said Monday. But any measures he takes seem likely to be small ones, and his options are limited
with Congress showing little appetite for more spending in a hotly contested midterm election year.

On his first workday back in Washington after a 10-day vacation on Martha's Vineyard and a day trip
on Sunday to New Orleans, Mr.~Obama spent part of the morning huddled with his economic team, then
emerged in the Rose Garden for a hastily arranged appearance that was troubled by microphone
difficulties.

He chided Senate Republicans for engaging in ``pure partisan politics'' by holding up a jobs bill
that would offer tax breaks to small businesses and ease credit with a \$30 billion initiative to
channel loans through community banks. ``I ask Senate Republicans to drop the blockade,'' Mr.~Obama
said.

The president also said he and his team were ``hard at work in identifying additional measures,''
including extending tax cuts for the middle class that are scheduled to expire this year, increasing
government investment in clean energy and rebuilding more infrastructure.

None of those steps, however, will come close to the \$787 billion stimulus measure that Democrats
passed at the outset of the Obama presidency. With voters angry about government spending, and
economists divided about just what approach is the correct one, such aggressive steps are by now out
of the question.

``There's a deep frustration among economists that they simply don't know what to do under these
circumstances, at least in terms of fiscal policy,'' said Bruce Bartlett, an economist who advised
Republican presidents.

``I think there are a lot of economists who, in principle, would support some new fiscal stimulus,
perhaps a jobs program where people were directly employed by the government or something of that
sort,'' Mr.~Bartlett said. ``But politically it's simply not possible to do anything remotely like
that under the current circumstances.''

The House has already passed a bill offering tax breaks to small businesses, but the measure is not
the same as the one being considered in the Senate. The majority leader, Senator Harry Reid of
Nevada, has scheduled a series of procedural votes on the Senate bill for when lawmakers return from
their recess on Sept.~13.

But passage ``is not a foregone conclusion,'' said Jim Manley, spokesman for Mr.~Reid. ``We're going
to need Republican votes.''

Republicans countered that Democrats were the ones holding up the measure, by blocking Republican
amendments to the bill and refusing to work with the minority party.

``Instead of growing jobs as promised, Washington Democrats have grown the size of the national
debt, the federal government and the unemployment rate,'' the Senate Republican leader, Mitch
McConnell of Kentucky, wrote in an e-mail.

With unemployment above 9 percent, and some economists warning of a double-dip recession, Mr.~Obama
and his fellow Democrats have been trying to make the case to voters that while the recovery is
slow, the nation is moving in the right direction. But recent economic data have not cooperated:
home sales in July dropped to their lowest level in a decade, and experts expect another bleak jobs
report on Friday.

In his Rose Garden remarks, Mr.~Obama sought to reassure nervous Americans that he is on top of the
economy, reminding them that ``it took nearly a decade to dig the hole that we're in'' and that it
will ``take longer than any of us would like to climb our way out.''

But as the president tried to deliver that message, he had to do several retakes, interrupting
himself to make certain that his microphone was on. ``Can you guys still hear us?'' Mr.~Obama asked.
``O.K. Let me try this one more time.''

\pagebreak
\section{Homeowner's Fight Involves Flag Tied to Tea Party}

\lettrine{D}{on}\mycalendar{Sept.'10}{01}'t tread on Andy C.~McDonel.

This year, Mr.~McDonel began flying a yellow ``Don't Tread on Me'' flag on his roof in this
unincorporated area just outside Phoenix. The historic banner -- which dates to 1775, when it was
hoisted aboard ships during the initial days of the Revolutionary War -- has been adopted by the Tea
Party movement. But Mr.~McDonel said that he had unfurled the flag for its historical significance
and nothing else.

He notes that the banner, the Gadsden flag, has been widely used over the years and was even
featured on the cover of a rock album. ``Am I a Metallica fan because I'm using the flag?'' he
asked.

This month, he received a letter from the homeowners' association ordering him to remove ``the
debris'' from his roof. It threatened fines if the debris (i.e., the flag) did not go within 10
days. But Mr.~McDonel, 32, a logistics operation manager, has vowed to fight the order.

``It's a patriotic gesture,'' he said of his banner. ``It's a historic military flag. It represents
the founding fathers. It shows this nation was born out of an idea.''

The Avalon Village Community Association, which sent the letter, takes a strict interpretation of
the state statute that allows Arizonans the right to fly a variety of flags -- the Stars and
Stripes, the state flag, flags representing Indian nations as well as the official flags of the
Army, Navy, Air Force, Marines and Coast Guard.

The listing of acceptable flags stems from a dispute several years ago in nearby Chandler, Ariz., in
which a woman with a son serving in Iraq was challenged by her homeowners' association for flying
the Marine Corps flag. State legislators intervened.

The Arizona law, says the homeowners' association butting heads with Mr.~McDonel, does not give
residents authorization to fly anything else on their properties. That means no pennants baring
sports team logos, no Jolly Rogers, no rainbow banners celebrating gay pride and no historic flags
showing a coiled rattlesnake bearing its fangs.

As Javier B.~Delgado, a lawyer for the homeowners' association, put it in a statement on the
association's Web site:

``Should the Arizona Legislature expand the Community Association Flag Display Statute to include
the Gadsden Flag, the Association will accommodate Mr.~McDonel's desire to display it. Bottom-line,
anyone considering residing in a community association should carefully review the association's
governing documents beforehand to ensure that the community is a good fit for them.''

Mr.~McDonel knows the rules well since, until July, he was a member of his homeowners' association's
board of directors. He resigned in a dispute with the board's president and shortly thereafter
received his first debris notice. That one concerned a treadmill that he had left on his porch,
which he admits was a violation of the rules. His second debris warning, which came weeks after
that, concerned the flag, which had been up for about six months.

``If this is a grudge, it's sad that the funds that the homeowners put into the association are
being wasted on such a petty matter,'' Mr.~McDonel said.

Mr.~Delgado, whose law firm represents thousands of homeowners' associations, denies that any
dispute among board members led to the citation of Mr.~McDonel's property. ``There is still the
potential for dialogue on both sides,'' he said, indicating that no fines had yet been levied.

The homeowners' association represents a community of tract homes in what had been a sprawling
agricultural area.

A survey of Mr.~McDonel's neighbors after the dispute drew the attention of the local news media
revealed more concern about the television trucks that have been parking in front of his property
than the flag flapping on his roof.

After Mr.~McDonel's standoff was picked up by the media, the American Civil Liberties Union of
Arizona jumped in on Mr.~McDonel's side, arguing that homeowners' associations do not have the right
to ``hijack'' the free speech rights of their members. The A.C.L.U. fired off a letter to the
association on Monday that seeks a meeting with Mr.~Delgado to resolve the matter without going as
far as a lawsuit.

``We're urging the homeowners' association to adopt a less limited interpretation of the statute,''
said Dan Pochoda, the legal director for the civil liberties group. ``The Gadsden flag meets the
spirit of the law. It's a historic military flag. Many consider it the original American flag,
before the Stars and Stripes.''

As for the political significance that the flag has taken on in this election season, Mr.~Pochoda
was uninterested, saying that Mr.~McDonel's motivation for flying the flag was irrelevant to the
dispute. ``We didn't ask him,'' Mr.~Pochoda said.

As the flag becomes more popular -- it was on prominent display on the Washington Mall last weekend
during a rally organized by the conservative commentator Glenn Beck -- more such disputes are
expected. Already, a Colorado homeowner flying the same flag is locked in a standoff with his
homeowners' association. And in Connecticut, a group of retired Marines is challenging the Capitol
Police's decision blocking the Gadsden flag from being flown over the State Capitol.

\pagebreak
\section{History of a Dress, Chinese Style}

\lettrine{A}{ll}\mycalendar{Sept.'10}{01} students of fashion know how the 20th century transformed
women's clothing in the West. Corsets were loosened, hemlines rose, and designers like Coco Chanel
and Yves Saint Laurent famously dressed ladies in trousers and tuxedo jackets.

Less documented was a similar fashion overhaul in China, which is now the subject of an exhibition
at the Hong Kong Museum of History. ``The Evergreen Classic: Transformation of the Qipao,'' showing
until Sept.~13, is a beautifully presented, and sometimes humorous, display of 280 Chinese gowns
created over the last 130 years. The exhibition is augmented by photographs and commentary showing
how the bulky Qing Dynasty robe -- which covered everything but a woman's face and hands -- altered
and shrank until it became the slinky ``cheongsam'' worn today, while retaining the gown's
distinctive diagonal lines. Unlike the Western dress, which has a vertical construction, the qipao
follows the flow of wrapped fabric.

``We wanted to highlight the qipao's role in history, and how it came to have greater meaning,''
said Terence Cheung, assistant curator of the Museum of History. ``The dress changed with the
times.''

The exhibit's earliest examples, Manchu gowns from the late 19th century, are the most striking.
These elaborate creations adorn wide racks in glass cases, not mannequins, to better display their
swaths of heavily embroidered silk. Even a gown tagged as being ``smart casual'' had exquisite
peonies and butterflies hand-sewn onto a crimson backdrop.

The photographs include images from inside the Forbidden City in Beijing before the fall of the
emperors. One undated portrait shows the stern Empress Cixi surrounded by ladies of the court, while
a snapshot captures their more modestly dressed servants. Across the Pacific -- and a world away in
terms of modernity -- Sun Yat-sen, a founding father of Republican China, poses for a 1901 family
picture in Honolulu, with the men in Western suits and bow ties, and the women in traditional qipao.

The photos give a good sense of the silhouette that was fashionable in the imperial court. Extremely
thin women and teenage girls wore heavy, multi-layered garments and enormous headpieces that looked
precariously balanced on their tiny necks. (One comment from several spectators was that the collars
on the qipaos looked impossibly small). The women wearing this rather voluminous combination were
perched on top of unnaturally tiny and painfully bound feet.

Each section of the exhibit, which is divided by historical era, is prefaced with a quote from a
1943 book, ``Chinese Life and Fashions,'' by the Chinese novelist Eileen Chang. She describes the
female models of this period as being ``pleasantly unobtrusive, one of the most desirable qualities
in a woman.''

Change came with the fall of imperial China, which ushered in the Republican period (1912-49), as
well as reforms and more education for women. Smart young women began wearing what was called
``civilized attire'' at the time.

Yet another Sun Yat-sen family picture, from 1912, features young women in wide trousers and high
heels, though still paired with loose Chinese-styled tops. A 1916 photo from an elite Christian
school in Beijing shows schoolgirls in pleated knee-length skirts and jaunty short jackets with
sleeves pushed up the elbows -- outfits that would have been considered revealing several decades
earlier.

In the 1920s, the original qipao, the long Manchurian gown, re-emerged. Long gowns were in style
again, only this time without the layers underneath. The wide bodice was slimmed down to a more
natural A-line, making for a more comfortable garment. Paradoxically, these long gowns reflected a
period style of men's clothing. According to the exhibition, that women could also wear them was
seen as a step toward equality.

The show becomes much more fun around the time of the Jazz Age, with the advent of the Chinese pop
culture aesthetic. Hong Kong calendars and Shanghai advertisements made good use of pretty girls in
the tighter, shorter modern ``cheongsam'' (or ``long shirt'') that evolved from the qipao. Posters
made from free picture cards given away with cigarettes showed qipao-clad girls fishing or playing
tennis.

``There were new elements in the 1950s, like tapered waists, or darts at the bust and waist,''
Mr.~Cheung said. ``The qipao merged with Western dress. In the early 1970s, with the introduction of
the miniskirt, there were some very short qipao.''

``In the 1960s, many more Hong Kong women started to do professional work in offices, and those who
were from the middle class and upper class wore qipao to work,'' Mr.~Cheung added. ``These were
different from the flowery, fussy qipao of the past. They were simple and convenient, with modern
zippers and press studs that were easier than the old flower buttons.''

Earlier in the 20th century, Chinese women innovatively added foreign influences. High-collared,
Chinese silk gowns were paired with stockings, pumps, cardigans, fur stoles. Women styled their hair
in marcelled waves.

Dresses from the 1930s and 1940s had slits up the side, both to free up movement and to show a
little leg. By the 1960s, the cheongsam was oozing with sex appeal, as evidenced by an iconic photo
of two women in Wan Chai, Hong Kong's red light district, in dresses so tight that the viewer can
glimpse the outline of their undergarments from behind.

One of the most eye-catching rooms is the one with dozens of 1950s and '60s dresses in every color
imaginable, displayed on mannequins in a wide arc. With no glass separating the viewer from the
garment, it feels like walking into an amazing retro clothing boutique.

The section dedicated to the 1970s and '80s starts getting campy. A video shows the ``cheongsam
competition'' of the Miss Hong Kong beauty pageant, complete with blue eye shadow, feathered hairdos
and a young Maggie Cheung. (She came in second.)

In the 1980s and '90s, the dress picked up flourishes that, arguably, never should have been added
to a Chinese gown, like shoulder pads and ruffles.

The last part of the exhibit celebrates the modern cheongsam and includes the lovely red and gold
outfit worn by the Beijing Olympic hostesses in 2008. Of note is a Blanc de Chine gown with flowers
in a diagonal pattern to show off the qipao's distinctive structure. But some pieces by local young
designers had so many embellishments that it was hard to discern any Chinese roots at all.

There were a few obvious gaps in this history of the qipao. Missing was Suzie Wong, the fictional
Hong Kong hooker with the heart of gold. She was, for better or worse, one of the best-known wearers
of the cheongsam. It would have also been nice to see segments on Shanghai Tang and Vivienne Tam,
the two brands that have done the most for spreading the popularity of the Chinese dress
internationally.

But the greatest omission is that of a catalog, although the museum is considering publishing one at
the end of the year. The gowns and the archival materials came from a wide range of collections,
both public and private. Given the effort it must have taken to assemble this exhibition, plus the
rarity of comprehensive shows dedicated to Chinese fashion history, it would be a shame if there is
no final document of it.

\pagebreak
\section{In Rare Move, China Court to Hear H.I.V. Case}

\lettrine{I}{n}\mycalendar{Sept.'10}{01} what appears to be a first for China's legal system, a
court in Anhui Province has agreed to hear a complaint by a prospective schoolteacher that he was
illegally denied a job because he is H.I.V. positive, the man's lawyer said Tuesday.

The unidentified man, said to be in his early 20s, brought the case under a 2006 national regulation
that prohibits job discrimination against people with H.I.V., his lawyer, Zheng Jineng, said in a
telephone interview from Hefei, the provincial capital.

Mr.~Zheng said the case would be heard by a district court in Anqing. The plaintiff contends that he
passed a written test and interviews for a teaching job there, but that the city education bureau
rejected him after a physical examination showed he was infected with H.I.V., the virus that causes
AIDS.

``In the past on sensitive cases like this, the court would be very reluctant to accept the case,''
Mr.~Zheng said. ``But this time they accepted it smoothly and quickly. That means the legal system
in China is making progress.''

H.I.V.-positive Chinese suffered official and public discrimination for years after the disease
first surfaced in the country in 1986. Infected students were often forced to leave school and
workers were shunted from their jobs.

More recently, the national government has taken a tolerant approach, offering free antiretroviral
drugs and prenatal care to many people who are H.I.V. positive, as well as screening for those who
suspect that they might be. Many migrants remain unable to receive the services, however, because
they lack the appropriate residence papers.

The National People's Congress, China's legislature, has approved a law that bans employers from
discriminating against job applicants with certain kinds of communicable diseases, as chosen by
state regulators. But the basis for the Anhui lawsuit is a regulation issued in March 2006 by the
State Council, the government's senior management body, which states that ``no institution or
individual shall discriminate against people living with H.I.V., AIDS patients and their
relatives.''

More than four years later, no court had placed an H.I.V. discrimination case on its docket until
Monday's decision, said Yu Fangqiang, the chief coordinator for Yirenping, a Beijing-based
civil-rights advocacy group involved in the Anhui case. The group paid the court fee to file the
lawsuit, and Mr.~Zheng waived his legal fees for the case.

Mr.~Yu said he agreed with Mr.~Zheng that the court's acceptance of the discrimination lawsuit was a
sign of changing legal standards. But he added that news media coverage had probably played a
crucial role in the court's decision, which had been delayed until the Chinese journal Legal Daily
ran an article about the case.

The newspaper, he said, ``is a must-read for a lot of people in the legal system. I think the media
played a role in the court accepting this case.''

Yirenping, the rights group, had filed as many as 15 other lawsuits similar to the Anhui complaint
in the past, Mr.~Yu said, but courts uniformly rejected them. Many other H.I.V.-positive citizens
approached the organization for advice on suing, but later dropped the idea for fear that their
confidentiality would be compromised, he said.

But the Anhui plaintiff, he said, was determined to pursue a lawsuit.

``He was born to a poor family in the countryside,'' Mr.~Yu said, ``and a job as a teacher means a
lot to him -- stable pay and a decent job.''

\pagebreak
\section{China Asks C.E.O.'s to Work for State}

\lettrine{T}{he}\mycalendar{Sept.'10}{01} Chinese government ran an enormous help-wanted
advertisement on Monday seeking professional managers for some of its biggest state-controlled
companies, a novel but not unprecedented move that apparently reflected unhappiness with the
companies' current performance.

The advertisement, two broadsheet-newspaper pages of small type, sought applicants for 20 senior
management jobs in industries like nuclear power, auto manufacturing and textiles. While some of the
positions were restricted to Chinese citizens, many of the posts were open to foreign applicants,
and several required proficiency in languages like English and French.

The advertisements were placed by the Chinese Communist Party's central organization department and
the State-Owned Assets Supervision and Administration Commission, the high-level government body
that oversees the operations of China's 129 biggest state-controlled corporations. They appeared in
major Chinese and English-language newspapers, including the Communist Party newspaper, The People's
Daily, and on most major Internet portals.

Most of the 129 companies, nicknamed the national champions, dominate industries like mining, power
generation and transportation, which are considered by the central government to be vital to China's
national security. But some, including most state-owned auto manufacturers, have struggled to gain
traction against nimbler private competitors, and the assets commission has warned lately that it
plans to merge its weakest companies into its more successful companies.

State-owned companies that monopolize their industries, like those in tobacco, telecommunications
and oil production, are quite profitable. But the profitability of state companies in competitive
industries is generally much poorer than the average, according to a World Bank analysis this year.

The advertisement sought five general managers -- roughly the same as a chief executive officer --
for the Dongfeng Motor Corporation, the China State Construction Engineering Corporation, China
Travel Service (Hong Kong), the State Nuclear Power Technology Corporation and the Chinatex
Corporation, a textile manufacturer.

More junior executives were being sought for posts at mining, communications, construction, nuclear
power and shipping companies, among others.

The solicitation quickly drew hundreds of comments on Internet chat boards, some less than
complimentary. On one major Web site, Sohu.com, the 800 comments were briefly led by one anonymous
posting that referred to the state companies' longstanding practice of giving jobs to relatives of
well-placed party and government officials.

``They are doing this because the sons and daughters of the leaders are now coming back from their
overseas studies,'' that person wrote.

Censors quickly deleted the comment.

\pagebreak
\section{In Britain, Labour Politicians Call for New Look at Scandal}

\lettrine{S}{enior}\mycalendar{Sept.'10}{06} opposition politicians are calling on the government to
respond to renewed accusations that Downing Street's chief communications officer, Andy Coulson,
encouraged reporters to illegally intercept messages from the cellphones of public figures when he
was editor of The News of the World.

At the same time, a number of people whose phone messages may have been intercepted by The News of
the World during Mr.~Coulson's tenure are accusing the Metropolitan Police of failing to fully
examine all the evidence in its criminal investigation in 2006 and 2007.

Lord Prescott, a Labour politician who was the deputy prime minister under Tony Blair and who has
been named as one of hundreds of people whose phones may have been hacked, said the police had never
provided him with a sufficient explanation of what happened.

``I have been far from satisfied with the Metropolitan Police's procedure in dealing with my
requests to uncover the truth about this case,'' Lord Prescott told The Observer newspaper. It was
only after ``repeated requests,'' he said, that he learned that he might have been a victim of phone
hacking. If the police continued to fail to be forthcoming, he said, he would seek a judicial
inquiry into their handling of the matter.

Alan Johnson, a Labour member of Parliament and a former home secretary, announced that he would
review the Home Office papers relating to the case to see whether the matter should be brought to
the Inspectorate of Constabulary, which monitors the police. His recommendation would then go to the
current home secretary, Theresa May.

Lord Prescott was responding to an article published by The New York Times Magazine online Wednesday
and in print Sunday about the scandal. In 2007, The News of the World's royal editor, Clive Goodman,
and an investigator employed by The News of the World, Glenn Mulcaire, were jailed after pleading
guilty to having illegally intercepted voice mail messages of Prince William and Prince Harry and
their chief royal aides.

Mr.~Coulson, who was appointed editor of The News of the World in 2003, said that he had no
knowledge of the hacking and that it was an isolated case, but resigned from the paper in January
2007 nonetheless.

Last year, The Guardian newspaper printed an article saying that hundreds of people might have been
singled out by The News of the World and providing details about some of them, including Gordon
Taylor, former chief executive of the Professional Footballers' Association, who reached a
settlement of \textsterling700,000 with The News of the World over the hacking of his cellphone.

The Times Magazine article provided new details, quoting a former reporter, Sean Hoare, and a
unnamed former editor at The News of the World as saying that Mr.~Coulson was fully aware of the
hacking. In an interview with BBC Radio 4 last week, Mr.~Hoare called Mr.~Coulson's statement to a
parliamentary committee denying that he knew about the phone hacking in his newsroom ``a lie.''

More than a dozen reporters and editors formerly with The News of the World, interviewed for The
Times article said their employer had fostered a culture of recklessness in which reporters were
encouraged to use any means to get exclusive stories. The article also quoted senior Metropolitan
Police officials saying that the police had failed to fully investigate The News of the World's
phone hacking in part because of Scotland Yard's close ties to editors at the paper and executives
at its parent company, News International.

Over the weekend, Tessa Jowell, a former Labour cabinet minister who is still a Parliament member,
said that the police had told her that her phone messages had been intercepted at least 28 times
while she was in the government. And The Independent on Sunday reported that Lord Mandelson, another
senior Labour politician, also had his messages intercepted.

John Yates, the assistant commissioner of the Met, said in a statement Sunday that the police would
consider reopening the criminal inquiry if fresh evidence of wrongdoing emerges and would consult
prosecutors about whether further inquiry was appropriate. Mr.~Yates said the police had asked The
Times for material it collected during its reporting of the magazine story, including notes from its
interviews with Mr.~Hoare.

Bill Keller, the executive editor of The Times, said, ``Scotland Yard has declined our repeated
requests for interviews and refused to release information we requested months ago under the British
freedom of information law. After our story was published, Scotland Yard expressed renewed interest
in the case and asked us to provide interview materials and notes; we declined, as we would with any
such request from police. Our story speaks for itself and makes clear that the police already have
evidence that they have chosen not to pursue.''

Tom Watson, a Labour member of Parliament and a member of the parliamentary committee that
investigated the phone hacking, wrote a letter to the Met Commissioner Sir Paul Stephenson, saying
``the historic continued and mishandling of this affair is beginning to bring your force, and hence
our democracy, into disrepute.''

For its part, the British government said it considers the matter closed and will not investigate
Mr.~Coulson, who was hired as the Conservative Party's chief spokesman in May 2007 after his
resignation from The News of the World. A spokesman at 10 Downing Street said last week that
Mr.~Coulson ``totally and utterly'' denied knowing about phone hacking while he served as editor.
Alan Duncan, the international development minister, appeared on television on Saturday night on
behalf of the government, accusing the Labour Party of acting for purely political reasons.

Speaking of senior Labour leaders who have called for a new investigation, Mr.~Duncan said: ``The
Labour Party -- in a concerted campaign through Ed Miliband, Lord Prescott and Alan Johnson -- have
piled in to attack Andy Coulson about something that happened years ago in order to try to attack
the government.''

Meanwhile, The News of the World denied the Times's allegations and accused it of publishing the
magazine article in an effort to discredit a newspaper belonging to a ``rival group'' -- that is,
the media empire of Rupert Murdoch. Mr.~Murdoch is the chairman of News Corporation, whose many
media holdings include The News of the World, The Times of London and The Wall Street Journal.

Five people whose phones were hacked have filed lawsuits this summer against News of the World's
parent company and Mr.~Mulcaire. And a growing number of public figures who believe their phone
messages may have also been intercepted but who feel the police did not do enough to investigate say
they intend to sue The News of the World. Others, including Brian Paddick, a former deputy assistant
commissioner with the Metropolitan Police, say they intend to seek a judicial review of the police's
handling of the criminal investigation. An application for that review is expected to be filed later
this week.

Senior Labour leaders also said they intended this week to seek a new inquiry by the standards and
privileges committee in the House of Commons.

The publication of the Times Magazine article has starkly exposed the fault lines in the media and
political landscape in Britain. Papers supporting the government -- including The Times of London
and The Sun, both Murdoch-owned -- have devoted little space to the new accusations. But media
outlets critical of the government, including The Guardian, The Independent and the BBC -- which
itself is in a bitter feud with Mr.~Murdoch's company, which has extensive television holdings in
Britain -- have covered The Times's article, and the subsequent calls for new investigations,
extensively.

In an editorial, The Financial Times said that there should be an independent review of The New York
Times's accusation that ``the police may have dropped a valid investigation.''

The Financial Times also called on Prime Minister David Cameron to investigate the matter. ``Was he
not reckless to have employed Mr.~Coulson, given the murkiness of the allegations surrounding The
News of the World?'' the paper asked.

\pagebreak
\section{In Europe, Fund-Raising Lessons From Americans}

\lettrine{I}{t}\mycalendar{Sept.'10}{06}'s tailgating season, when alumni gather in stadium parking
lots to enjoy impromptu picnics, cheer on the college team -- and fend off requests for donations.
But try telling Andrew Hamilton, Oxford University's affable vice chancellor, that part of his
difficulty in raising money may stem from the fact that the British do not tailgate, and he suddenly
turns argumentative.

``I've seen some pretty impressive hampers at the Boat Race,'' he insists.

In office since last October, Dr.~Hamilton, a chemist who earned his doctorate at Cambridge,
returned to Britain after 28 years in the United States. He spent the last 12 years at Yale
University, where he served as provost and a professor of molecular biophysics and biochemistry. His
expertise in Ivy League fund-raising techniques is sure to be very useful in his new job, which
includes raising funds for Oxford Thinking, a \textsterling1.25 billion, or \$2 billion, campaign
launched in 2004.

The Oxford campaign just edged out Cambridge's \textsterling1 billion 800th-anniversary campaign as
the most ambitious fund-raising effort ever undertaken by a university outside the United States.

Both campaigns are part of a broader trend, as universities across Europe realize that in the
current economic slump their customary dependence on government funding not only leaves them exposed
to the politics of austerity but also puts them at a disadvantage in the global competition for
increasingly mobile students and faculty.

According to Jamil Salmi, an education expert at the World Bank and author of a study on the
challenge of establishing world-class universities, the proliferation of international rankings
means that students today are far more aware of how their universities compare with others. ``The
world's best universities enroll and employ large numbers of foreign students and faculty in their
search for the most talented,'' Dr.~Salmi wrote in a report. And though academic prestige cannot
simply be bought, a lack of resources can hamper even the most prestigious universities in their
efforts to recruit faculty or attract the most able students.

Dr.~Salmi points to France and Germany, two prosperous countries, both with a long tradition of
scientific achievement, yet whose best universities ``are hardly recognized as elite institutions.''

``Universities in Germany, France and Scandinavia have traditionally relied on public funding,'' he
said. ``But only the Scandinavian countries and Switzerland have been able to fund their
universities at levels sufficiently generous to allow them to compete.''

This disparity may not have mattered when language and the high cost of travel meant students had
little choice about where to study. ``The European academic mind set has not been much interested in
diversifying the sources of funding,'' Dr.~Salmi said.

Recently, however, that has been changing. Partly in response to decades of ``brain drain'' to the
United States and partly prompted by increased competition from universities in the emerging
economies of India and East Asia, European universities are increasingly turning to American-style
fund-raising methods in an effort to amass endowments that would in turn give them greater economic
independence and stability.

Some have even adopted U.S.~methods of managing their endowments. The British elite universities
have led the way. Last January, Cambridge took another leaf from the Ivy League handbook in raising
\textsterling300 million through the bond market -- the first bond issue ever by a British
university.

But the spread of U.S.~methods extends as far as Russia and even Hong Kong. Oleg Kharkhordin, the
new rector of the European University of St.~Petersburg, cites New York's New School for Social
Research and Georgetown University in Washington as institutions that, despite a relatively late
start in fund-raising, have built substantial endowments.

``There is no point trying to copy Harvard,'' said Dr.~Kharkhordin, who taught at Harvard after
earning his doctorate in political science at Berkeley. ``Harvard is unique.'' And he doesn't think
that, as one of Russia's handful of privately funded universities, the European University of
St.~Petersburg can afford to emulate Berkeley, either. ``In a sense, Berkeley is a big socialist
university.''

The University of Hong Kong is a public university in a country ruled by the Communist Party of
China. Yet Dr.~Salmi says that the university's Stanley Ho Alumni Challenge, in which a graduate who
had made his fortune in the casino industry in Macao offered to match all donations up to 100
million Hong Kong dollars, or \$13 million, a year for a period of five years, triggered a 600
percent increase in alumni donations in the first year alone. It has so far raised 100 million
dollars for each of the three years it has been in existence.

When it comes to university fund-raising, Dr.~Salmi said, ``East Asia has been much more successful
and dynamic'' in adapting American methods. ``In Europe, academics don't want to compromise their
academic integrity'' by engaging in fund-raising, he said, adding that Europe is also hampered by
``ideological hostility.''

``Seeking funds from the private sector has been seen as distasteful,'' he said.

Dr.~Hamilton argues that however widespread such attitudes may have been in the past, ``when I
arrived at Oxford from Yale I found a very strong fund-raising operation in place.'' Ruth Collier, a
spokeswoman for the university, said that the Development Office, which had only 35 staff members in
2005, now employs more than 80 people, including 13 in New York and 2 in Hong Kong. (At Harvard, by
comparison, some 250 people work full time on fund-raising, while the University of Wisconsin
Foundation has a staff of about 130.)

``I spend an increasing amount of time on fund-raising -- and rightly so,'' Dr.~Hamilton said.
Though both Oxford and Cambridge are public institutions, ``it is absolutely essential that we
increase the proportion of our revenues that come from endowments and donations,'' he said.

Dr.~Hamilton said such efforts were almost as old as the university itself, noting that the Bodleian
Library and the Radcliffe Infirmary are named after benefactors from centuries ago.

In countries without a modern tradition of private philanthropy, matters have been far more
complicated. The European University of St.~Petersburg, founded in 1995 during the Yeltsin era, was
one of the first private universities in Russia. Initially funded by American foundations --
principally the Ford Foundation, the MacArthur Foundation and the Open Society Institute -- the
university has been struggling to stand on its own financial feet as its Western donors shift their
attentions elsewhere.

William Rosenberg, an emeritus professor of history at the University of Michigan who is active in
the Russian university's U.S.~fund-raising efforts, said that until quite recently, Russian law
``didn't even allow university endowments. In 2004 a group of us established a fund for the E.U.S.P.
based in the U.S.''

Russian supporters also campaigned for a change in Russian law. ``When an endowment law passed in
2005, E.U.S.P. was the first to register,'' Mr.~Rosenberg said.

But the university's American connections also brought controversy. In 2008, after Prime Minister
Vladimir V.~Putin accused it of being ``an agent of foreign meddling'' the university was
temporarily closed. ``The authorities overreacted,'' Dr.~Kharkhordin said. And though a petition to
allow the university to reopen drew more than 7,000 signatures, not one of the signers made a
contribution to the endowment.

As an institution devoted to graduate study in the social sciences, Mr.~Rosenberg said, the
university ``has a small cadre of alumni, none of whom were capable of contributing the kinds of
amounts needed.''

Besides, Dr.~Kharkhordin said, ``Russian donors had no habit of contributing, it hadn't become
fashionable. But it has become acceptable. When it becomes fashionable, it will be easy to raise
money.'' Until then, the university will continue to rely on the largess of American corporations
doing business in Russia -- and on foreign donors who agree with Dr.~Kharkhordin that while science
and technology have always been well funded by the state, ``noninstrumental knowledge is also
important.''

Such activity, he said, ``contributes to building more places where independent initiative can feel
itself protected.''

The alumni weekend is already well-established at both Cambridge and Oxford, where Dr.~Hamilton
points out that with roughly 12,000 undergraduates in each, the ``number of alumni out in the world
is much larger'' than comparable American universities. So if ``we can increase our level of alumni
support from the current 14 percent'' to something closer to Yale (nearly 50 percent) or Princeton
(nearly 60 percent) ``the potential impact is much larger.''

Tanya Rawlinson, a Dartmouth alumna who heads the University of Bristol's alumni relations office,
says that ``part of the revolution is the simple act of asking. Very few European universities even
ask all their alumni to donate every year.''

Of course there are some American methods European universities are happy to live without. When it
was suggested to Alison Richard, a former Yale provost who has been vice chancellor of Cambridge
since 2003, that Cambridge would find it easier to raise money if the university gave admissions
preference to the children of alumni or donors, she was adamant: ``It's not even worth talking
about, because we won't, and let's make a virtue of that.''

\pagebreak
\section{Former H.P. Chief May Move to Oracle}

\lettrine{M}{ark}\mycalendar{Sept.'10}{06} V.~Hurd, who was forced to resign as Hewlett-Packard's
top executive last month after an investigation into a sexual harassment charge found that he had
manipulated his expenses, is in talks with Oracle about a top executive position there, according a
person briefed on the matter.

Mr.~Hurd, who is close to Lawrence J.~Ellison, Oracle's founder and chief executive, was not
expected to replace him, though it was uncertain what any role at Oracle might be. Mr.~Hurd and
Oracle are close to reaching an agreement, but no deal has been completed, said the person briefed
on the talks, who agreed to speak on the condition of anonymity because the discussions were
supposed to remain confidential.

Oracle did not respond to requests for comment. A spokesman for Mr.~Hurd declined to comment.

For Mr.~Hurd, 53, landing a top role at Oracle would be a quick rebound after his tumultuous exit
from H.P. in early August. Mr.~Hurd was forced out by the board after he settled charges of sexual
harassment brought by Jodie Fisher, a 50-year-old actress who worked as a marketing consultant for
the company.

Both Mr.~Hurd and Ms.~Fisher denied having a sexual relationship, and an H.P. investigation failed
to find any evidence of sexual misconduct by Mr.~Hurd.

But the company has said that Mr.~Hurd's resignation was a result of a break in trust caused by his
falsifying expense reports possibly to conceal the relationship.

Shortly after Mr.~Hurd was forced out, Mr.~Ellison made an unusual and passionate defense of him. In
an e-mail to The New York Times, Mr.~Ellison called the H.P. board's action ``the worst personnel
decision since the idiots on the Apple board fired Steve Jobs many years ago.''

The talks between Mr.~Hurd and Oracle were first reported on the Web site of The Wall Street Journal
on Sunday.

Oracle, which Mr.~Ellison founded 30 years ago, is the world's largest database software maker;
Mr.~Ellison has been its only chief executive. For years, the company has been a close partner with
H.P., which sells computing systems and services to corporations. But since Oracle's acquisition of
Sun Microsystems, in a deal that closed early this year, Oracle and H.P. have become competitors in
the market for computer hardware.

The purchase of Sun caught a number of Oracle's investors off guard, since the company had avoided
the hardware market in the past.

At H.P., Mr.~Hurd helped steer mammoth computer server, storage and services businesses. Such
expertise could come in handy as Oracle continues to try to digest Sun. In particular, Mr.~Hurd
built a reputation as a cost-cutting whiz and could apply those skills to bringing the Sun business
in line.

Sun also has a number of large campuses and an extensive research and development operation. At
H.P., Mr.~Hurd pared back such expenses.

While running the company, Mr.~Hurd passed on trying to acquire Sun, leaving Oracle and I.B.M. to
bid for it.

Mr.~Ellison remains heavily involved in Oracle, but the day-to-day operations are largely overseen
by two presidents, Safra A.~Catz and Charles E.~Phillips Jr.~It was unclear how Mr.~Hurd would fit
into the existing, crowded triumvirate.

Mr.~Hurd took over the top job at H.P. in 2005, succeeding Carly Fiorina, who had been unable to
increase profitability after the company's \$19 billion acquisition of Compaq in 2002.

His tenure was widely seen as a success. Mr.~Hurd brought tight fiscal discipline to the computer
giant and turned it into one of the most reliable performers in the technology sector. During his
tenure, H.P. surpassed I.B.M. as the No.~1 technology company, as revenue increased to \$115 billion
a year, from \$80 billion.

\pagebreak
\section{For G.O.P., Tea Party Wields a Double-Edged Sword}

\lettrine{C}{hristine}\mycalendar{Sept.'10}{06} O'Donnell, running in Delaware to be the Republican
nominee for the Senate seat once held by Vice President Joseph R.~Biden Jr., is a perennial
candidate with a history of financial problems, including unpaid taxes and a home in foreclosure.
Representative Michael N.~Castle, her opponent in the primary, is a veteran congressman and former
governor who has won statewide elections more than 10 times.

But Ms.~O'Donnell has the backing of the Tea Party, and suddenly Delaware has become the latest
Republican civil war battlefield. National Tea Party groups are pouring money into Ms.~O'Donnell's
campaign, while establishment Republicans are attacking her with more ferocity than they have shown
toward the Democrat in the race, worried that Ms.~O'Donnell is the bigger threat to the party's
winning the seat in November.

The battle in Delaware is just the latest reminder that as much as the Tea Party fervor is expected
to help Republicans in November, it may also create problems for them -- and opportunities for the
Democrats.

So far this election season, the Tea Party has brought a huge amount of unexpected energy into the
campaign, and it could drive sufficient Republican turnout to become a major and perhaps decisive
factor in many races. But the movement has also forced Republicans to spend precious time and money
on primary races they should have won easily and has produced some inexperienced candidates who have
stumbled in the early going.

In some House races, Republicans have all but given up hope of winning after local Tea Party groups
helped conservative candidates win primaries in districts that historically prefer moderates. And in
some districts, Tea Party candidates are mounting third-party challenges that could allow the
Democrats to maintain or even win some seats.

``When we talk about what kind of an impact the Tea Party is going to have on the midterms, what I'm
watching are these seats where the Tea Party has nominated candidates over more viable Republican
candidates -- that's my measuring stick,'' said Jennifer Duffy, who watches Senate races for the
nonpartisan Cook Political Report. ``It could be the difference between getting the majority or
not.''

Of the 18 Senate races that The New York Times considers competitive, there are 11 where the Tea
Party stands to be a significant factor. While it is harder to predict the Tea Party's influence in
the House races, given the diffuse nature of thousands of local groups across the country, there are
at least 48 out of 104 competitive seats where it could have a major impact.

In many places, the impact will be from Tea Party groups -- local, national or both -- that are
working to mobilize voters. In others, however, the Tea Party is complicating what should have been
easy Republican primary victories.

In the Eighth Congressional District in Arizona, for instance, a seat held comfortably by
Republicans until 2006, Democrats had worried that their incumbent, Representative Gabrielle
Giffords, would lose. They rejoiced when the Tea Party candidate, Jesse Kelly, beat Jonathan Paton,
who had been backed by Republicans in Washington.

In the 10th Congressional District in Illinois, the Tea Party may have created a rare opportunity
for the Democrats to pick up a seat, helping Robert Dold win against a more moderate Republican in
the primary. The seat is now held by Mark Kirk, a moderate Republican who is running for the Senate
seat vacated by President Obama. Democrats believe they can portray Mr.~Dold as too conservative for
the district.

With the economy still working against Democrats, they say they are hoping to benefit from concerns
about Tea Party extremism.

Allen West, for example, the Republican nominee in Florida's 22nd Congressional District, has become
a Tea Party sensation. He has raised more money than any other House challenger -- and his opponent
-- collecting donations from people across the country who have followed him on YouTube as he
thunders against the ``tyranny'' of the federal government.

But to Democrats, he is an opposition researcher's dream, captured on video rallying his audiences
to ``get your musket, fix your bayonet,'' questioning whether Mr.~Obama is a citizen and urging his
supporters to make his opponent ``scared to come out of his house.''

Democrats said they were trying to make the same case against Tea Party candidates who are the
Republican nominees in Senate races: Rand Paul in Kentucky, Sharron Angle in Nevada, Ken Buck in
Colorado and Ron Johnson in Wisconsin. (The candidates the Tea Party helped nominate in Utah, Mike
Lee, and Alaska, Joe Miller, are considered all but certain to win -- even in a year when
uncertainty is the rule.)

The Democrats are playing up the candidates' support for things that are standard Tea Party
positions, but unpopular among most Americans: getting rid of the Departments of Energy, Commerce
and Education; phasing out Social Security and Medicare; and instituting a 23 percent national sales
tax to replace the income tax.

The arguments seem to be gaining the most traction in Kentucky and Nevada, where the nominations of
Tea Party candidates have helped keep Democrats in the running -- particularly Senator Harry Reid of
Nevada, the Democratic leader, who is considered among the most vulnerable incumbents. In Kentucky,
the Democrat, Jack Conway, gained the support of the state's police union last month after raising
concerns about Mr.~Paul's opposition to federal antidrug efforts, a position unpopular among rural
voters whose communities have been ravaged by methamphetamine.

All of those Senate races are close. And Republicans and, increasingly, independent analysts believe
that the Tea Party candidates can prevail.

Delaware is a different story. Republicans, though they backed Ms.~O'Donnell when she ran against
Mr.~Biden for the Senate, say she is unelectable, and in recent days they have attacked her for
``repeatedly lying to voters'' and ``manipulating her own political history.'' They said they had
considered Mr.~Castle a shoo-in until Mr.~Miller defeated Senator Lisa Murkowski in the Republican
primary in Alaska.

The Tea Party Express, a national group that helped secure Mr.~Miller the nomination and helped
Senator Scott Brown win in Massachusetts, then said it would spend \$250,000 on television and radio
advertisements for Ms.~O'Donnell in advance of the primary Sept.~14.

Other trouble spots for Republicans are Tea Party-backed candidates mounting third-party challenges
in several close House races. These include the Fifth and Second Congressional Districts in
Virginia, where Republicans had controlled the seats until the last couple of years and had hopes of
taking them back.

In Florida's 12th Congressional District, where the seat is being vacated by a Republican,
independent handicappers say that Randy Wilkinson, who registered to run on an official Tea Party
line, could take a sizable chunk of voters from the Republican, handing the election to the
Democrat, Lori Edwards.

And in New York's 23rd Congressional District, Tea Party groups are backing Doug Hoffman, who
created trouble for Republicans in a special election last year by mounting a challenge to the
moderate Republican nominee, ultimately pushing her to quit the race. She threw her support to the
Democrat, Bill Owens, and he won the district despite Republicans having controlled it for 150
years.

This year, Mr.~Hoffman has said that if he does not win the Sept.~14 primary against Matt Doheny,
who has the support of establishment Republicans, he will run as a third-party candidate again. And
many Tea Party activists are supporting his bid even though they understand that this would probably
result in the seat's remaining in Democratic hands.

``Sometimes being principled doesn't always get you what you want,'' said Jennifer Bernstone, an
activist who supported Mr.~Hoffman last year and does again. ``That's part of being principled.''

But analysts say third-party candidates will affect 10 House races, at best, in a year when 100 are
competitive. And Republican strategists say it is wishful thinking on the Democrats' part to hope
that they can combat Tea Party enthusiasm with concerns about Tea Party extremism.

``Enthusiasm in an off year is absolutely critical,'' said Whit Ayres, a Republican pollster. ``I'm
not disagreeing that there are places where the nontraditional Republican nominee may face a greater
challenge than would a traditional Republican nominee, but I do not agree that ergo the Democrat is
going to win.''

\pagebreak
\section{Resentment Simmers in Western Chinese Region}

\lettrine{T}{he}\mycalendar{Sept.'10}{06} five-star hotels are full, bulldozers are making quick
work of dreary slums and billboards for ``French-style villas'' call out to the nouveau riche. In
the year since rioting between the Han and Uighur ethnic groups killed nearly 200 people in this
city in far western China, life appears to be returning to normal.

``Don't worry, everything is peaceful now,'' said the perky bellhop at a hotel in the city's
predominantly Han Chinese quarter.

But before turning away, he had second thoughts. ``You'd better not go to the Uighur part of town at
night,'' he said.

Beneath the gloss and mercantile buzz of Urumqi, the capital of the Xinjiang region, there is a
palpable unease that neither tens of thousands of surveillance cameras nor the patrolling squads of
black-shirted police officers can completely assuage.

Since July 2009, when rampaging Uighur mobs set upon Han Chinese with iron bars and bricks -- a
scene that was reversed for several days when Han vigilantes sought revenge -- the Chinese
authorities have arrested hundreds and tried to soothe frayed nerves with a \$1.5 billion spending
package, a change in local leadership and a barrage of uplifting slogans strung across public buses
and highway overpasses.

But the feel-good propaganda and revved-up economy have so far done little to repair the mutual
distrust. And experts say the government's ``strike hard'' campaign, which has led to the secret
detention of perceived troublemakers and the execution of at least nine people accused of having a
hand in the bloodshed, has worsened tensions.

``I don't think a single Uighur is convinced that the government is acting in their interests,''
said Dru C.~Gladney, a professor of Asian studies at Pomona College in California who studies the
region. ``In fact, the hostile environment is making people feel embattled and resentful.''

Given the heightened surveillance here, it is not always easy to tease out unvarnished sentiments
from either the Uighurs, a Turkic-speaking Muslim minority, or the Han, who make up 96 percent of
China's population. But with patience and the promise of anonymity, raw resentments emerge.

Take the Han Chinese owner of a small restaurant who initially described Uighurs as ``part of our
family'' but later allowed that he found the vast majority of them frightening, untrustworthy and
``savage.''

The man, who would give only his surname, Zhou, said he stopped riding buses in Urumqi last fall
after the city was swept by tales of Uighurs jabbing Han with H.I.V.-infected hypodermic needles.
The police initially detained more than a dozen people in the attacks but later dismissed
suggestions that the needles were contaminated.

Like the hundreds of thousands of Han who migrate to Xinjiang each year, Mr.~Zhou, who left Sichuan
Province in 2004, said he was partly inspired by the notion that he was helping to ``open up''
western China. Although he grew up learning that the Uighurs were Chinese and part of the country's
happy kaleidoscope of 56 ethnicities, he said he quickly discovered otherwise.

``We just don't have much in common,'' he said with a wary glance around him. ``And what's worse is
they don't appreciate what we've done for them.''

Much as it did in Tibet, in an effort to pacify another restive ethnic region, the government has
spent huge sums of money to try to help Xinjiang's economy catch up to eastern China, where income
and production are on average twice as high. In May, the Communist Party announced the first leg of
its ``Love the Great Motherland, Build a Beautiful Homeland'' initiative, which will include six new
airports and 5,000 more miles of rail line linking the far-flung cities in this Alaska-size region
of desert and mountains.

Bowing to popular discontent, Beijing also replaced the region's leader, Wang Lequan, whose 15-year
tenure was marked by a hard-line approach that alienated many Uighurs and in the end failed to
forestall the riots, prompting street protests by Han residents demanding his resignation.

The Uighurs, who make up just under half of Xinjiang's 22 million people -- down from more than 90
percent in 1949 -- harbor their own deeply felt animosities. Beijing is determined to dilute Uighur
culture, they say, while Han migrants often end up with the best jobs, especially in government
bureaucracies or in the factories of the prosperous ``bingtuan,'' the largely segregated Han
outposts carved out of the desert by the People's Liberation Army in the 1950s.

While an inability to speak Mandarin shuts some Uighurs out of Han-run companies, many say the
larger force behind their economic marginalization is naked discrimination. ``It used to be that
state-owned enterprises had Han-only hiring policies, but these days they are more subtle,'' said
Ilham Tohti, a Uighur economist who studies the job market in Xinjiang. ``They reject you after
you've gone in for the interview and they've seen your face.''

Although the uneducated and unskilled have been hardest hit by unemployment, even bilingual Uighurs
who graduated from Chinese universities say they have a hard time finding good jobs.

The frustration many Uighurs have is compounded by a sense that they are trapped, prevented by
bigotry and strict residency rules from moving to more affluent coastal cities, and by tight
passport restrictions from leaving China.

The policy, partly shaped by government fears that Uighurs who travel abroad might become
radicalized and return as terrorists, cuts them off from overseas jobs, academic opportunities and
family reunification. It also frustrates Uighur business owners who seek a bigger slice of China's
trade with its Central Asian neighbors. ``How can we compete with the Han if we can't meet our
customers in their own countries?'' asked a textile trader in Kashgar, a southern oasis city. ``Just
because we are Muslims doesn't mean we're all interested in becoming terrorists.''

Government concerns about the radicalizing influence of Islam play out through a raft of religious
restrictions, including strict limits on the number of Uighurs who can travel to Mecca, Saudi
Arabia, for the annual pilgrimage and rules that force students and government workers to eat during
the monthlong fast of Ramadan.

Fear also governs secular life. Cellphones and e-mail exchanges are frequently monitored, and even
mild criticism of Chinese policies posted online can have dire consequences. In July, four Uighurs
-- three Web site managers and a journalist -- accused of endangering state security were sentenced
to long prison terms during closed trials.

A graduate student from the city of Khotan said he did not dare click on Web sites run by Uighur
exile groups that can be reached only by evading government Internet restrictions, known as the
Great Firewall. ``They will find you,'' he said of the government. ``Sometimes I feel like I'm
living in a giant prison.''

Anu Kultalahti, a researcher at Amnesty International in London who interviewed witnesses of last
July's violence, said the fear extended to those now living in Europe and the United States. ``It
borders on paranoia,'' she said. ``If you promise them confidentiality, they laugh and say: `That's
what you think. The government already knows we're talking.' ''

\pagebreak
\section{China Will Require ID for Cellphone Numbers; Noncompliance Means No Service}

\lettrine{T}{he}\mycalendar{Sept.'10}{06} Chinese government on Wednesday began to require cellphone
users to furnish identification when buying SIM cards, a move officials cast as an effort to rein in
burgeoning cellphone spam, pornography and fraud schemes.

The requirement, which has been in the works for years, is not unlike rules in many developed
nations that ask users to present credit card data or other proof of identification to buy cellphone
numbers. The Ministry of Industry and Information Technology said that about 40 percent of China's
800 million cellphone users were currently unidentified. Those users will be ordered to furnish
identification by 2013 or lose their service, according to The Global Times, a state-run newspaper.

A government center that deals with cellphone complaints said that the average Chinese cellphone
user received a dozen spam messages a week, and that three of every four users received messages
that involved fraud, China Daily, another state newspaper, reported Wednesday.

Some analysts, however, questioned whether the new requirement would substantially reduce illicit
messages. Instead, they warned that the regulation potentially gives the government new tools to
locate and punish individuals who send cellphone messages that censors deem unacceptable.

The Chinese central government has steadily tightened its censorship of the Internet and wireless
communications since 2008, blocking increasing numbers of Web sites, social networking services like
Facebook and Twitter and, most recently, shutting down microblogs that it regards as subversive.

The new regulation will be carried out largely by the three government-controlled companies -- China
Mobile, China Unicom and China Telecom -- that provide cellular service.

``Is China prepared for this?'' David Bandurski, an author and media analyst at the University of
Hong Kong's China Media Project, asked in a telephone interview. ``Does it have the legal framework
and the institutions in place to guarantee they can do this and still protect the privacy of
consumers?''

``People are basically providing their phone numbers and ID numbers'' to the mobile carriers, he
said. ``Those are the two most important pieces of information that most people have.''

In an article posted Wednesday on the China Media Project's Web site, a legal researcher at the
government-sponsored Chinese Academy of Social Sciences, Zhou Hanhua, expressed doubts that
requiring users to register their names with the companies would control spam.

The rules, he wrote, would probably initially create a black market in legally registered SIM cards
that could be used for spam, and then spur hackers to find ways to circumvent the registration
requirement.

``Technology innovation will soon trump the government's control,'' he wrote.

Others were less concerned. A professor at Beijing University of Posts and Telecommunications, Zeng
Jianqiu, said that real-name registration was essential if services now common in other nations,
like payment by cellphone, were to become established in China.

Privacy ``is a problem that needs to be considered seriously,'' he said in a telephone interview on
Wednesday. ``The regulators and mobile operators also need to find ways to protect personal
information. But I think some, like China Mobile and Telecom, are already doing this.''

Under the new policy, convenience stores and street vendors who have been selling anonymous SIM
cards were to suspend sales on Wednesday, until they are trained to register their customers.
Foreigners will also be required to furnish a passport or other identification when establishing
cellphone service.

\pagebreak
\section{Housing Woes Bring New Cry: Let Market Fall}

\lettrine{T}{he}\mycalendar{Sept.'10}{06} unexpectedly deep plunge in home sales this summer is
likely to force the Obama administration to choose between future homeowners and current ones, a
predicament officials had been eager to avoid.

Over the last 18 months, the administration has rolled out just about every program it could think
of to prop up the ailing housing market, using tax credits, mortgage modification programs, low
interest rates, government-backed loans and other assistance intended to keep values up and
delinquent borrowers out of foreclosure. The goal was to stabilize the market until a resurgent
economy created new households that demanded places to live.

As the economy again sputters and potential buyers flee -- July housing sales sank 26 percent from
July 2009 -- there is a growing sense of exhaustion with government intervention. Some economists
and analysts are now urging a dose of shock therapy that would greatly shift the benefits to future
homeowners: Let the housing market crash.

When prices are lower, these experts argue, buyers will pour in, creating the elusive stability the
government has spent billions upon billions trying to achieve.

``Housing needs to go back to reasonable levels,'' said Anthony B.~Sanders, a professor of real
estate finance at George Mason University. ``If we keep trying to stimulate the market, that's the
definition of insanity.''

The further the market descends, however, the more miserable one group -- important both politically
and economically -- will be: the tens of millions of homeowners who have already seen their home
values drop an average of 30 percent.

The poorer these owners feel, the less likely they will indulge in the sort of consumer spending the
economy needs to recover. If they see an identical house down the street going for half what they
owe, the temptation to default might be irresistible. That could make the market's current malaise
seem minor.

Caught in the middle is an administration that gambled on a recovery that is not happening.

``The administration made a bet that a rising economy would solve the housing problem and now they
are out of chips,'' said Howard Glaser, a former Clinton administration housing official with close
ties to policy makers in the administration. ``They are deeply worried and don't really know what to
do.''

That was clear last week, when the secretary of housing and urban development, Shaun Donovan,
appeared to side with current homeowners, telling CNN the administration would ``go everywhere we
can'' to make sure the slumping market recovers.

Mr.~Donovan even opened the door to another housing tax credit like the one that expired last
spring, which paid first-time buyers as much as \$8,000 and buyers who were moving up \$6,500. The
cost to taxpayers was in the neighborhood of \$30 billion, much of which went to people who would
have bought anyway.

Administration press officers quickly backpedaled from Mr.~Donovan's comment, saying a revived
credit was either highly unlikely or flat-out impossible. Mr.~Donovan declined to be interviewed for
this article. In a statement, a White House spokeswoman responded to questions about possible new
stimulus measures by pointing to those already in the works.

``In the weeks ahead, we will focus on successfully getting off the ground programs we have recently
announced,'' the spokeswoman, Amy Brundage, said.

Among those initiatives are \$3 billion to keep the unemployed from losing their homes and a
refinancing program that will try to cut the mortgage balances of owners who owe more than their
property is worth. A previous program with similar goals had limited success.

If last year's tax credit was supposed to be a bridge over a rough patch, it ended with a glimpse of
the abyss. The average home now takes more than a year to sell. Add in the homes that are foreclosed
but not yet for sale and the total is greater still.

Builders are in even worse shape. Sales of new homes are lower than in the depths of the recession
of the early 1980s, when mortgage rates were double what they are now, unemployment was pervasive
and the gloom was at least as thick.

The deteriorating circumstances have given a new voice to the ``do nothing'' chorus, whose members
think the era of trying to buy stability while hoping the market will catch fire -- called ``extend
and pretend'' or ``delay and pray'' -- has run its course.

``We have had enough artificial support and need to let the free market do its thing,'' said the
housing analyst Ivy Zelman.

Michael L.~Moskowitz, president of Equity Now, a direct mortgage lender that operates in New York
and seven other states, also advocates letting the market fall. ``Prices are still artificially
high,'' he said. ``The government is discriminating against the renters who are able to buy at
\$200,000 but can't at \$250,000.''

A small decline in home prices might not make too much of a difference to a slack economy. But an
unchecked drop of 10 percent or more might prove entirely discouraging to the millions of owners
just hanging on, especially those who bought in the last few years under the impression that a
turnaround had already begun.

The government is on the hook for many of these mortgages, another reason policy makers have been
aggressively seeking stability. What helped support the market last year could now cause it to
crumble.

Since 2006, the Federal Housing Administration has insured millions of low down payment loans.
During the first two years, officials concede, the credit quality of the borrowers was too low.

With little at stake and a queasy economy, buyers bailed: nearly 12 percent were delinquent after a
year. Last fall, F.H.A. cash reserves fell below the Congressionally mandated minimum, and the
agency had to shore up its finances.

Government-backed loans in 2009 went to buyers with higher credit scores. Yet the percentage of
first-year defaults was still 5 percent, according to data from the research firm CoreLogic.

``These are at-risk buyers,'' said Sam Khater, a CoreLogic economist. ``They have very little
equity, and that's the largest predictor of default.''

This is the risk policy makers face. ``If home prices begin to fall again with any serious velocity,
borrowers may stay away in such numbers that the market never recovers,'' said Mr.~Glaser, a
consultant whose clients include the National Association of Realtors.

Those sorts of worries have a few people from the world of finance suggesting that the
administration should do much more, not less.

William H.~Gross, managing director at Pimco, a giant manager of bond funds, has proposed the
government refinance at lower rates millions of mortgages it owns or insures. Such a bold action,
Mr.~Gross said in a recent speech, would ``provide a crucial stimulus of \$50 to \$60 billion in
consumption,'' as well as increase housing prices.

The idea has gained little traction. Instead, there is a sense that, even with much more modest
notions, government intervention is not the answer. The National Association of Realtors, the
driving force behind the credit last year, is not calling for a new round of stimulus.

Some members of the National Association of Home Builders say a new credit of \$25,000 would raise
demand but their chances of getting this through Congress are nonexistent.

``Our members are saying that if we can't get a very large tax credit -- one that really brings
people off the bench -- why use our political capital at all?'' said David Crowe, the chief
economist for the home builders.

That might give the Obama administration permission to take the risk of doing nothing.

\pagebreak
\section{Some Newspapers, Tracking Readers Online, Shift}

\lettrine{I}{n}\mycalendar{Sept.'10}{06} most businesses, not knowing how well a particular product
is performing would be almost unthinkable. But newspapers have always been a peculiar business, one
that has stubbornly, proudly clung to a sense that focusing too much on the bottom line can lead
nowhere good.

Now, because of technology that can pinpoint what people online are viewing and commenting on, how
much time they spend with an article and even how much money an article makes in advertising
revenue, newspapers can make more scientific decisions about allocating their ever scarcer
resources.

Such data has never been available with such specificity and timeliness. The reader surveys that
newspapers relied on for decades took months to produce, often leaving editors with stale data.

Looking to the public for insight on how to cover a topic is never comfortable for newsrooms, which
have the deeply held belief that readers come to a newspaper not only for its information but also
for its editorial judgment. But many newsrooms now seem to be re-examining that idea and embracing,
albeit cautiously, a more democratic approach to serving up the news, particularly online.

``How can you say you don't care what your customers think?'' asked Alan Murray, who oversees online
news at The Wall Street Journal. ``We care a lot about what our readers think. But our readers also
care a lot about our editorial judgment. So we're always trying to balance the two.''

Editors at The Journal, like those at other large newspapers, follow the Web traffic metrics
closely. The paper's top editors begin their morning news meetings with a rundown of data points,
including the most popular search terms on WSJ.com, which articles are generating the most traffic
and what posts are generating buzz on Twitter.

At The Washington Post, a television screen with an array of data -- the number of unique visitors
to washingtonpost.com, how many articles those visitors view and where on the Web those visitors
came from -- is on display for the entire newsroom. A red or green marker designates each data
point, indicating whether the Web site's goal for the month on that particular metric has been met.
About 120 people in The Post's newsroom get an e-mail each day laying out how the Web site performed
in the closely watched metrics -- 46 in all.

Rather than corrupt news judgment by causing editors to pander to the most base reader interests,
the availability of this technology so far seems to be leading to more surgical decisions about how
to cover a topic so it becomes more appealing to an online audience.

The Post, which provided extensive coverage of the recent elections in Britain online and in its
print editions, found that online readers were not particularly interested in the topic. One of the
five most viewed items on The Post's Web site in the last year, in fact, was not a political project
at all but a piece on Crocs, the popular foam footwear. Editors attributed that to Yahoo, which
linked to the article.

But that did not translate into more Croc coverage. And coverage of the British elections was not
scaled back.

Raju Narisetti, The Post's managing editor overseeing online operations, said he saw reader metrics
as a tool to help him better determine how to use online resources.

``We ask, `What can we do online to make it more attractive?'' ' Mr.~Narisetti said. ``Can we do
podcasts? Can we do a photo gallery? Can we do any kind of user-generated content?''

He said the data has proved highly useful in today's world of shrinking newsroom budgets.
Mr.~Narisetti said that when he had to reduce his staff last year, he looked at what kind of content
was not performing well with readers. He discovered that long-form video had a low audience, so he
reduced that department by a couple of people.

At The Journal, editors use traffic data to inform decisions on how articles should be presented on
WSJ.com. ``We look at the data, and if things are getting a lot of hits, they'll get better play and
longer play on the home page,'' said Mr.~Murray. Conversely, articles getting low audiences will be
moved down more quickly if there is no compelling news reason to keep them prominent.

But Mr.~Murray explained that the data was not always used as a blunt tool. In the case of a rather
dry business development last month involving the Potash Corporation, the Canadian fertilizer maker,
Journal editors decided to prominently display articles on the subject despite very low traffic
numbers.

``We didn't put it there because it was going to be a big traffic getter. We put it there because
it's big important news in the business world,'' Mr.~Murray said.

The New York Times does not use Web metrics to determine how articles are presented, but it does use
them to make strategic decisions about its online report, said Bill Keller, the executive editor.
``We don't let metrics dictate our assignments and play,'' he said, ``because we believe readers
come to us for our judgment, not the judgment of the crowd. We're not `American Idol.' ''

Mr.~Keller added that the paper would, for example, use the data to determine which blogs to expand,
eliminate or tweak.

As newspaper Web sites use technology to learn more about readers' habits, they are also developing
new ways to persuade readers to tell them more about what they want. The Los Angeles Times features
what it calls a ``personality quiz'' for readers on its Web site. The feature adds a spin to the
personalization options that Web sites have offered for the last few years with a 17-question test
that asks readers things like ``What does success mean to you?'' and has them pick from 12 photos. A
few options include images of a wedding, a gleaming sports car and a man embracing a peasant child.

At the end of the quiz, readers are assigned a personality type like ``dynamo,'' who, as the quiz
explains, is someone ``always seeking new adventures that broaden your horizons and take you out of
your comfort zone.'' A customized news feed then appears each time a reader visits the Web site from
the same computer.

``It helps me understand the readers in a way that I can't with just the metrics,'' said Sean
Gallagher, managing editor for online operations at The Los Angeles Times, explaining that he now
pairs sports articles with food articles because surveys have shown a correlation.

As the technology advances and allows papers to look more deeply at performance metrics, newsrooms
may find that there is just some data they would rather not know.

At a recent meeting with the top online editors of The Los Angeles Times, a consulting group that
helps media companies enhance profits from their Web sites pitched new software that it said could
change the industry. The newsroom would be able to know how much money -- down to the penny -- each
of its articles online was making when readers clicked on ads.

``I could see a business case for it,'' said Mr.~Gallagher, who hastened to add, ``I don't agree
with that business case.''

Software developers acknowledge that the questions can be difficult as newspapers try to reinvent
their business models. But they say the dialogue is ultimately constructive.

``By having this data and making it available, we're spurring the conversations to take place,''
said Tim Ruder, chief revenue officer for Perfect Market, the company that developed the tracking
software for ad clicks. ``And it's especially healthy to have those conversations in the context of
experience and not in an abstract way.''

\pagebreak
\section{Apple Faces Many Rivals for Streaming to TVs}

\lettrine{W}{hen}\mycalendar{Sept.'10}{06} Steven P.~Jobs introduced Apple's first iPod in 2001, it
held more music and was easier to use than the handful of rival products already in the market. But
it had a big shortcoming: it worked only with Macintosh computers, then less than 3 percent of the
PC market.

A few upgrades later and with the creation of the iTunes Store, the iPod emerged as the dominant
digital music player, eclipsing its rivals and turning Apple into the world's largest distributor of
digital music.

So last week when Mr.~Jobs unveiled the second version of Apple TV, a product with limited
capabilities and content, many in the consumer electronics industry wondered whether history would
repeat itself in the market to deliver content to the TV from the Internet.

Most consumers are either unaware of or confused about how to use an Internet-connected TV. There
are no strong brand names but a growing number of competitors. And it is a lucrative market ripe for
a company that understands consumers.

``They are getting more aggressive in the living room, and that's the last big market that can move
the needle for Apple,'' said Gene Munster, an analyst with Piper Jaffray.

The Apple TV box, which fits in the palm of a hand, is elegant and easy to use to rent television
shows for 99 cents, though for now only from ABC and Fox. Owners of iPads, iPhones and
top-of-the-line iPods will be able to transmit videos, photos and other media from their mobile
devices to the television. It will also allow Netflix subscribers to stream thousands of movies and
television shows without an additional fee.

But more than 100 other devices can stream Netflix movies, including game machines, Blu-ray players
and Internet-connected TVs. Indeed, Apple faces an increasing number of competitors that include
upstarts like Roku and Boxee, powerful rivals like Google, and established players like Sony and
Samsung.

And its revamped Apple TV box is, as Mr.~Munster put it, ``an elementary effort.''

Steve Perlman, a former Apple executive and a pioneer in efforts to merge the Internet and
television, described Apple TV's features as ``underwhelming.'' ``Apple is still in a holding
pattern when it comes to television,'' said Mr.~Perlman, now the chief executive of OnLive, an
online gaming company.

Mr.~Perlman and others suggested that Apple TV suffered from shortcomings like limited programming
choices and an inability to use the kinds of applications and games that made the iPhone so popular.
They say it is a box that provides some nice additional features for TV watchers, but not a
replacement for the products and services that consumers now rely on.

Many of the current generation of HDTVs come with Internet capability to stream movies, YouTube
videos, Twitter feeds and music. They do not cost much more than TVs without that capability. John
Revie, senior vice president for home entertainment marketing at Samsung Electronics, estimates the
industry will sell 6.5 million ``smart TVs'' in 2010.

If Mr.~Jobs had bolder ambitions for Apple TV, they were thwarted, in part, by the same challenges
faced by most of its competitors: the established cable operators and content companies that want to
protect their businesses even as they tiptoe into the world of Internet television.

CBS and NBC, for example, have refused to make television shows available to Apple at 99 cents, and
even Fox suggested that it views its agreement with Apple as an experiment.

Mr.~Jobs, a master marketer, tried to cast the device's simplicity as a virtue, saying that Apple TV
delivers many of the things consumers want. ``They want Hollywood movies and TV shows whenever they
want,'' he said. ``They want to pay lower prices for content. They don't want a computer on their
TV. They have computers.'' He added: ``They don't want to manage storage.''

Apple has other advantages as well. The most significant improvement Apple made to the device was
the price. It dropped it to \$99, or \$200 cheaper than its predecessor model. Given Apple's
substantial advertising budget, many consumers who have never heard of Boxee or the movie-streaming
abilities of the Xbox 360 will certainly learn about Apple TV as the holidays approach.

Analysts say that with Apple TV, the company is well positioned to take a bolder step into Internet
television in the next year or two. And they say that its most significant impact, at least in the
short term, is likely to be not on the established television industry's incumbents, but rather on
new players like Roku, Boxee or Google.

Google has yet to enter the market but is expected to release in November its first version of
Google TV. The company's approach is more ambitious, more complex and notably different from
Apple's.

Google wants its software to be at the center of the television viewing experience. The software
system will be built into televisions and will let people have access to any Web site and Web video,
and easily search for programming. Prices for Google TV devices have not been announced.

Boxee is also trying to marry much of the Web with the television set, as well as stream programming
from Netflix and other sources. The company's box is expected to go on sale in the fall for \$199.

``It is really aimed at a different consumer,'' said Avner Ronen, chief executive of Boxee. While
Apple TV will be great for access to some media, he said, ``people want to be able to watch any
video they see on the Internet on the TV.''

Analysts say that Apple TV's closest match in the market is Roku, whose box has been available for
two years and also offers streaming from Netflix and other services. Anthony Wood, the chief
executive of Roku, said he was not worried about competition from Apple. He said sales of the Roku
boxes, which start at \$60, were growing fast, and the company expected to have sold a million
devices by the end of the year.

``We try hard to offer a better product for a lower cost,'' Mr.~Wood said. ``They'll be a competitor
for us, but I don't think they have an Apple-sized success on their hands.''

Apple will have another important advantage over many rivals: the ability to demonstrate Apple's TV
capabilities in its stores. ``I can't walk into Best Buy and ask to see how Roku works,'' said Tim
Bajarin, an analyst with Creative Strategies. Roku is sold only online.

But Apple's biggest challenge may be the same one facing all the new entrants in the market: An
established industry that gives television consumers much of what they want and that it is itself
inching toward the Internet, albeit slowly.

``When Apple introduced the iPod, the music player was broken,'' said Mr.~Munster, the Piper Jaffray
analyst. The software to connect to music players was clumsy and seemed more suited to technophiles
than average consumers, Mr.~Munster said, creating an opening for Apple. ``Now the television works
well,'' he added, ``and a lot of people don't realize that they need Internet TV.''

\pagebreak
\section{Online Giving Meets Social Networking}

\lettrine{L}{ate}\mycalendar{Sept.'10}{06} last month, tens of thousands of runners who are
registered for this year's New York City Marathon got an e-mail from Mary Wittenberg, the president
and chief executive of New York Road Runners.

Ms.~Wittenberg wanted to introduce them to a person whom many had already heard of: the actor Edward
Norton. But the words ``Hollywood movie star'' didn't appear once in her message. Instead, she
implored the runners to join a social networking Web site that Mr.~Norton and three partners started
in May that she says has the potential to revolutionize charitable giving. It's called
Crowdrise.com.

``They've built a phenomenal platform to help us really broaden our reach,'' says Ms.~Wittenberg.
Thanks in part to Crowdrise, she says, the marathon has a shot at raising a record \$26.2 million,
or a million a mile, for charity this year. That would be up from \$24 million in 2009 and \$18.5
million in 2008.

As for Mr.~Norton, she adds: ``You should see him standing in my office at the whiteboard.
Seriously, Edward is as passionate about our vehicle, the marathon, as I am. And I've never said
that about anyone.''

Yes, Mr.~Norton is a runner. More on that in a minute. But this two-time Oscar nominee -- known to
many as the Incredible Hulk's alter ego or the guy to whom Brad Pitt explained the first rule of
Fight Club -- is also a believer in marrying technology and philanthropy.

He knows that a majority of people who now donate to charity don't do so online; they write checks.
But he and his partners contend that Crowdrise, with its mix of edginess, silliness and good-humored
competition, can change that habit, especially for young people.

``The '60s were the era of people realizing they could rally together to express their priorities,''
says Mr.~Norton. But today, he says, social networking offers ``a new way of getting people together
to create power in numbers.'' More than that, he said, it can help users express themselves through
the causes they support.

Mr.~Norton added: ``One of the things we're trying to say at Crowdrise is plant a flag. Raise a
fist. Declare yourself.''

Crowdrise aims to make raising money for a cause not just easy, but also fun. Setting up a page to
support something you care about takes less than a minute. Then, friends and family can be invited
to be sponsors by donating any amount of money, large or small. You don't have to run a marathon.
You can volunteer at a soup kitchen or do whatever strikes your fancy. But Ms.~Wittenberg, who has
already sent her e-mail to 33,000 runners based in the United States and will soon send one to the
27,000 or so based elsewhere, hopes that anyone running in New York on Nov.~7 will use Crowdrise to
do it for charity.

Once your Crowdrise page is up, anyone can donate to it and join your team.

Crowdrise isn't the only site that helps with online fund-raising. There are a handful, with
FirstGiving.com among the best known. But Crowdrise is different, its founders and users say,
because it seeks to build community in much the way that Facebook does.

Irreverent in tone -- one of its slogans is, ``If you don't give back, no one will like you'' --
Crowdrise also appeals to anyone with a gaming sensibility. Users compete for prizes, earning points
for every dollar they raise and more points for every vote they get from members of the Crowdrise
universe.

On July 31, top scorers on the Crowdrise Points Leaderboard won prizes including two MacBook Pros, a
Kindle, a Wii and two \$500 gift cards. ``One of the most unexpected parts of Crowdrise since we
launched is how obsessed people are with their points,'' says Robert Wolfe, another partner in the
venture. (The other two are Mr.~Wolfe's brother, Jeffrey, and the movie producer Shauna Robertson,
whose films include ``Superbad'' and ``Knocked Up.'')

Crowdrise and its partnership with the New York City Marathon both sprang from the same event: last
year's race, in which Mr.~Norton ran to raise money for the Maasai Wilderness Conservation Trust in
Kenya.

The Wolfe brothers, who had recently sold their company, the online retailer Moosejaw.com, and were
looking for a new challenge, helped Mr.~Norton and his team create an interactive Web site. Inspired
by the online fund-raising machine that Barack Obama used in the 2008 presidential campaign, it
sought to make Mr.~Norton's supporters feel engaged and included in the team's quest.

Daily contests for donors were announced via Mr.~Norton's Twitter account. The result: \$1.2 million
raised in two months for the trust, which supports a 280,000-acre wilderness reserve.

MR. NORTON said the experience taught him how social networking could be ``productive, not just
pointless chatter.'' The founding of Crowdrise sought to enable anyone to do what Mr.~Norton did,
and to make new friends in the process.

The site, which is run not as a nonprofit but as a business, takes less than 8 percent of total
money donated, including about 2 percent for credit card fees. ``We need serious scale to make this
business work,'' Robert Wolfe said. ``But we're in this for the long haul.''

The actor Seth Rogen is using Crowdrise to raise money to fund research of Alzheimer's disease.
Flea, the bassist for the Red Hot Chili Peppers, is trying to build support for basic music
education for young people.

But the Crowdrise team is as excited about its nonfamous users -- people like Jaime Haughey, 27, an
out-of-work M.B.A. candidate who lives in Abington, Mass., and Amanda Darby, 33, a model-maker at
Aardman Animations in Bristol, England. Ms.~Haughey and Ms.~Darby, who have a mutual love of Africa,
found each other on the site.

While they haven't yet met -- they talk regularly via Skype -- they are planning a trip to Kenya to
volunteer at the Maasai reserve, probably in 2011.

Crowdrise ``brought the two of us together, to support and help each other,'' Ms.~Haughey, who has
raised nearly \$9,000 for five charities in chunks as small as \$5, said in an e-mail.

A few hours later, when the sun came up in England, Ms.~Darby added, ``Before Crowdrise,
fund-raising was difficult, tedious and you felt like you were fighting to swim upstream.''
Crowdrise, she said, ``makes it personal.''

\pagebreak
\section{City's Efforts Fail to Dent Child Obesity}

\lettrine{N}{ew}\mycalendar{Sept.'10}{06} York City schoolchildren are as heavy, or perhaps even
heavier, than the national average, despite the Bloomberg administration's dogged efforts to improve
the health of city residents, according to new data from the city's health department.

Two out of five, or 40 percent, of the nearly 637,000 children in kindergarten through the eighth
grade were found to be overweight or obese in the 2008-9 school year. Those rates were the same as
in the previous year, according to a survey of both school years that is to be released on Sunday.

That compares with 35.5 percent of 6- to 11-year-olds nationally, according to data from the federal
Centers for Disease Control and Prevention.

Among New York City children who were overweight, 22 percent were obese, compared with 19.6 percent
nationally.

``I'm sorry to say it's in line with the nation, but we're certainly working hard to get it down
from here,'' said Cathy Nonas, the director of physical activity and nutrition for the city's
Department of Health and Mental Hygiene.

The numbers were broken down by ZIP code and showed that less-affluent neighborhoods had the most
severe problems. In the 2008-9 school year, the highest rates were found in Corona, Queens, where 51
percent of schoolchildren were overweight or obese. That was followed by parts of Harlem, with 48 to
49 percent, and Washington Heights, with 47 percent.

In contrast, some of the city's wealthiest areas had the healthiest children. The West 60s near the
Hudson River in Manhattan had the smallest share of overweight or obese children (11.7 percent),
followed by part of TriBeCa (15 percent), SoHo (17.7 percent) and the East 50s, including Turtle Bay
and Sutton Place (18.3 percent).

In the 2007-8 school year, Hunts Point in the Bronx had the highest rates. Other areas with
overweight and obesity rates of 45 percent or higher included East Harlem, parts of Bushwick and
Williamsburg in Brooklyn, and Astoria Heights and part of Jackson Heights in Queens.

Ms.~Nonas suggested that the lack of change from 2007 to 2009 could signal the beginning of a
decline. The numbers were also down slightly from 2003, when 19 percent of city schoolchildren were
overweight and 24 percent were obese.

Yet she said it would be na\"ive to think that measures like banning trans fats and posting the
calories of foods served in restaurants would be enough to bring about a decline in childhood
obesity.

She added that the city would use the data to decide where to concentrate its exercise and nutrition
programs. Since the data was collected, she said, the city had substituted milk that is 1 percent
fat for regular milk, or skim for chocolate milk, and had banned sugar-sweetened beverages from
school vending machines. The school system has also restricted bake sales to once a month.

She said the health department and the school system were introducing a pilot project to train 3,000
teachers, from kindergarten through the third grade, in exercises that children can do during
classroom breaks. One routine has children pretending they are cabdrivers who have to bend down to
go through a tunnel and jump to get over a pothole.

The data was collected through the city's Fitnessgram program, in which children in kindergarten
through eighth grade had their height and weight measured and converted into body mass index. The
individual results, including the results of a physical fitness test, were sent to parents.

\pagebreak
\section{Employers Push Costs for Health on Workers}

\lettrine{A}{s}\mycalendar{Sept.'10}{06} health care costs continue their relentless climb,
companies are increasingly passing on higher premium costs to workers.

The shift is occurring, policy analysts and others say, as employers feel more pressure from the
weak economy and the threat of even more expensive coverage under the new health care law.

In contrast to past practices of absorbing higher prices, some companies chose this year to keep
their costs the same by passing the entire increase in premiums for family coverage onto their
workers, according to a new survey released on Thursday by the Kaiser Family Foundation, a nonprofit
research group.

Workers' share of the cost of a family policy jumped an average of 14 percent, an increase of about
\$500 a year. The cost of a policy rose just 3 percent, to an average of \$13,770.

Workers are now paying nearly \$4,000 for family coverage, according to the survey, and their costs
have increased much faster than those of employers.

Since 2005, while wages have increased just 18 percent, workers' contributions to premiums have
jumped 47 percent, almost twice as fast as the rise in the policy's overall cost.

Workers also increasingly face higher deductibles, forcing them to pay a larger share of their
overall medical bills. ``The long-term trend is pretty clear,'' said Drew E.~Altman, the chief
executive of the Kaiser foundation, which conducted the survey this year with the Health Research
and Educational Trust, a research organization affiliated with the American Hospital Association.
``Insurance is getting stingier and less comprehensive.''

Companies may be at a point where they are no longer willing or able to protect their workers'
health benefits, said Helen Darling, the president of the National Business Group on Health, an
organization representing employers that provide coverage.

She says that companies expect that their costs will only go up more under the new health care law
because it requires them to provide more benefits, like coverage for preventive care.

``There's a sense we can't keep up,'' Ms.~Darling said. ``We can't afford to continue to subsidize
what's happening.'' Her group's own survey, conducted last month, found that almost two-thirds of
employers said they planned to increase the percentage their workers would have to contribute toward
premiums next year.

More employers may be changing their view of providing health benefits, moving toward contributing
only a fixed amount rather than maintaining certain levels of coverage, she said. ``It's a portent
of the future,'' Ms.~Darling said.

But businesses may also have felt less need to protect their workers because the increase in the
cost of premiums was modest, said Nancy-Ann DeParle, who oversees health care for President Obama.
``It's the lowest increase in many years,'' she said.

And Ms.~DeParle pointed to a number of initiatives under the health care legislation that were
likely to help companies better afford insurance, including \$40 billion in tax credits for small
businesses and \$5 billion to help companies pay for retiree health benefits.

The economy may be the dominant influence in forcing employers' hands, said Mr.~Altman of Kaiser.
The decision by companies to pass on the higher costs ``speaks to the depth of the recession and its
impact on employers,'' he said. Businesses may have no other alternative in trying to steady costs,
he said.

Some examples around the country offer examples of the choices being made by employers and their
workers.

Faced with a potential increase in the premiums paid that would bring the cost of family coverage to
about \$1,000 a month, the executives at a trucking business in Salt Lake City chose to switch to a
plan that had a \$6,000 annual deductible.

The company, Utility Trailer Sales of Utah, and a related company were able to reduce their monthly
premiums by nearly \$200, to \$647 a family, according to the chief financial officer, Clair Heslop.

Mr.~Heslop acknowledged that people with chronic conditions or the need for expensive medicines had
felt the impact of the change. ``It's hit them hard,'' he said. ``They're paying the bill because
they're consuming the goods.''

The Kaiser survey found a significant increase in the number of employees who had a deductible of at
least \$1,000, to 27 percent this year, from 22 percent in 2009. Almost half of workers who are
covered by a small employer with fewer than 200 workers have an annual deductible of that amount.

Some employers, however, may be looking for ways to limit their exposure. In Utah, the state is
setting up an insurance exchange that explicitly allows smaller employers to give workers a fixed
amount of money to buy a health policy, much as they might make a defined contribution to a
retirement plan.

Workers choose among about 60 policies offered by four major insurers, paying the difference if the
coverage is more expensive than the amount provided by their employer. State officials this week
opened the exchange to any business with 50 or fewer employees.

They say the exchange offers employers the ability to better manage their health care costs.

``We've given predictability to both the employer and the employee,'' said Spencer Eccles, executive
director of the Governor's Office of Economic Development, which manages the exchange.

The exchange will have to make some changes under the federal law. When the exchanges are up and
running, some workers may be able to get vouchers from their employers under certain conditions to
allow them to shop independently in the exchange if their company's coverage is too expensive.

But some policy analysts are concerned that the movement toward a system in which employers feel
responsible for paying a fixed amount for health care is not an answer to the higher costs.

``We're taking the easy way out,'' said Judi Hillman, the executive director of the Utah Health
Policy Project, which is itself exploring the idea of covering its employees through the exchange
and is trying to form a group of businesses to better understand the dynamics of the exchange.
``We're not thinking structurally and systemically in Utah, but I think federal reform will do
that.''

Once employers have a better handle on the new legislation, they may well pursue different
strategies, including moving toward a system in which they are responsible for only a fixed amount
of the cost of coverage, said Tracy Watts, a partner with Mercer Health and Benefits, which advises
companies about the health benefits they offer. ``There's going to be a lot of studying about what
are the longer-term strategies, what makes sense,'' she said.

\pagebreak
\section{Minister Wavers on Plans to Burn Koran}

\lettrine{F}{irst}\mycalendar{Sept.'10}{10}, Terry Jones, the Florida pastor who set the world on
edge with plans to burn copies of the Koran on Sept.~11, said Thursday that he had canceled his
demonstration because he had won a promise to move the proposed Islamic center near ground zero to a
new location.

Then, hours later, after learning that the project's leaders in New York had said that no such deal
existed, Mr.~Jones backed away from his promise and said the bonfire of sacred texts was simply
``suspended.''

The sudden back and forth suggested that the controversy -- the pastor drew pointed criticisms from
President Obama and an array of leaders, officials and celebrities in the United States and abroad
-- was not yet finished even after multiple appearances before the news media on the lawn of his
small church.

Mr.~Jones seemed to be struggling with how to save face and hold on to the spotlight he has
attracted for an act that could make him a widely reviled figure.

But Mr.~Jones seemed to have been wrong or misled from the start.

Minutes after he announced the cancellation alongside Imam Muhammad Musri, a well-known Islamic
leader in Florida who had been trying to broker a deal, Mr.~Musri contradicted Mr.~Jones's account.
He said that Muslim leaders of the project in New York had not actually agreed to find a new
location. ``The imam committed to meet with us but did not commit to moving the mosque yet,''
Mr.~Musri said.

Even that may not be accurate. Imam Feisal Abdul Rauf, the leader of the New York project, said in a
statement that he had not spoken to Mr.~Jones or Mr.~Musri, who said later that he received the
pledge of a meeting from a staff member in Mr.~Abdul Rauf's office.

The saga of Mr.~Jones appeared likely to continue -- with more pressure likely to come as well. In
just the past week, the list of his critics had come to include Mr.~Obama, the Vatican, Franklin
Graham, Angelina Jolie, Sarah Palin, dozens of members of Congress and Gen.~David H.~Petraeus, who
was among the first to declare that the burning of Korans would put Americans soldiers and civilians
in danger.

That risk of violence seemed to be rising, as large protests against Mr.~Jones were staged over the
past week in Kabul, Afghanistan, and Jakarta, Indonesia. It led the Obama administration to work
furiously to end Mr.~Jones's plans.

On Thursday, F.B.I. officials met with Mr.~Jones, and even Mr.~Obama waded into the fray, sharply
criticizing what he called a ``stunt'' that would be a ``recruitment bonanza for Al Qaeda.''

``I just hope he understands that what he's proposing to do is completely contrary to our values as
Americans,'' Mr.~Obama said on ABC's ``Good Morning America.'' He added that it could ``greatly
endanger our young men and women in uniform who are in Iraq, who are in Afghanistan.''

While Mr.~Jones had told reporters that he would not ``ignore'' a call from the White House,
administration officials decided that an appeal from the military would be more effective. The Obama
administration also had to weigh the desire to stop Mr.~Jones from proceeding with his plans against
the recognition the once-obscure preacher, with a congregation of less than 50, would get from a
direct appeal from the president.

Defense Secretary Robert M.~Gates called Mr.~Jones around 4:15 p.m. Thursday, interrupting a meeting
that Mr.~Jones was having with Mr.~Musri.

The call was brief, Mr.~Jones said, adding that Mr.~Gates was not the key factor in his decision.
What swayed him, he said, was not the risk to Americans or foreigners but rather the promise that
the Islamic center in New York would be moved.

``This is for us a sign from God,'' he said.

As Mr.~Jones walked back into his office, he said that the idea of the Islamic center as a bartering
point came to him only after he had announced his ``International Burn a Koran Day'' in July. He
said he had no regrets.

``We have accomplished what we think God asked us to do,'' he said.

Those involved in the Islamic center project in New York offered contradictory stances and opinions
on Thursday, making it hard to determine if the parties involved had a common front.

In a brief interview on Thursday, minutes before Mr.~Jones made his cancellation announcement,
Mr.~Abdul Rauf, the imam, seemed to suggest that moving the project -- at least the part of it that
he is to lead, which includes a mosque, prayer spaces for other faiths and tolerance education
programs -- was not out of the question.

When asked -- without reference to Mr.~Jones -- whether the comments he made on CNN's ``Larry King
Live'' on Wednesday night, that he would not have proposed the project had he known how much strife
it would cause, indicated a new openness to moving or some other compromise, he said, ``We are
investigating that right now, we are discussing it right now, how we can resolve this issue in a
manner that will defuse the rhetoric and the pain and also reduce the risk'' of emboldening Muslim
radicals.

He added: ``That is the question we are now asking ourselves. We are weighing various options.''

But the imam controls only one part of the project, known as Cordoba House, the interfaith and
Muslim prayer spaces and tolerance programs that are planned as part of the larger community center,
known as Park51.

Sharif el-Gamal, the head of the real estate group that owns the properties where the project is
planned, took a more definite position. ``We're not moving,'' he said in an interview. He later
issued a statement reiterating that.

In Gainesville, Mr.~Jones seemed confused by the differing opinions. At first, after reporters read
him Mr.~Abdul Rauf's statement denying that a deal had been made, Mr.~Jones said he preferred to
believe that the center would be moved.

He said he would be very disappointed if that did not turn out to be the case. As for whether he
would go back to burning Korans, he seemed to go back and forth during multiple appearances before
the news media. At one he said, ``Right now, we are not even entertaining that idea.'' But later he
suggested he might reconsider.

Regardless of whether Mr.~Jones does meet with the mosque leaders in New York, he has already
elevated his profile, which has risen quickly from the small church he has run in Gainesville since
around 2001.

The church has been fairly empty during recent services, with no more than a few dozen congregants,
many of them family members. The smell of dust and mildew wafts out from the piles of used furniture
that Mr.~Jones sells on eBay when he is not preaching.

To most residents of this sprawling college town, where Democrats outnumber Republicans two to one,
Mr.~Jones has generally been a fringe figure, even last year when he put up a sign outside the
church that said ``Islam is of the devil.''

But that began to change when news of his Koran-burning plans reached Muslim countries about a month
ago. Suddenly, there was an overabundance of what Mr.~Jones seemed to want -- attention.

Mr.~Jones, a former hotel manager who calls himself doctor based on an honorary degree from an
unaccredited Bible school, has at times seemed sincerely shocked by the response he has attracted.
But not unhappy.

His church has been in financial trouble for years -- the property is now for sale -- and even
before General Petraeus and the president made him a household name, he said in an interview that he
hoped to become well known as a critic of Islam.

He was in his office at the time, alone, and to his right sat a drawing of a bearded man -- a
terrorist -- that had been used for target practice.

The mix of guns and visions of grandeur would come to embody the church and Mr.~Jones.

On Thursday, several of his parishioners carried pistols on their hips -- the result, they said, of
death threats. That also served as a sign of the outsize role their small group had taken on in
world affairs.

By nightfall, things seemed no closer to an end, as a church member named Stephanie, wearing a pink
shirt with a holstered gun at her hip, arranged for interviews with reporters from all over the
world.

\pagebreak
\section{Chinese Food, the Wines of Germany, and a Debate}

\lettrine{A}{s}\mycalendar{Sept.'10}{10} this country continues its uneasy dialogue about
integration, spurred on by an anti-immigrant book written by an executive of the central bank, the
restaurant owner Jianhua Wu is busy selling wine, promoting wine, eagerly and enthusiastically
sampling and sipping wine. Not just any wine, but German wine.

Mr.~Wu, who came here from China a quarter century ago to study engineering, in many ways represents
the other side of the immigration debate, not the hostile, fearful, anti-immigrant sentiments
stirred up by the shock-book of Thilo Sarrazin, the banker. He and his family instead represent the
emerging Germany that is slowly, painfully becoming a multicultural society, where the spicy snap of
Szechuan dishes and the subtle, flowery sweetness of a riesling can complement each other.

``Riesling and Chinese food, it works,'' said Mr.~Wu, who has become something of a sensation in
this city for his restaurant, Hot Spot, which offers an extensive collection of German wines
alongside his Szechuan- and Shanghai-inspired menu.

``It's my passion,'' Mr.~Wu said as he prepared to play host to five of Germany's top winemakers at
a sold-out buffet on Sunday.

After struggling to make a life here, working in one fast-food Chinese restaurant after another,
after years peddling sweet-and-sour recipes loaded with MSG, Mr.~Wu said he discovered that his
route to financial success in his adopted home was ultimately wine -- or really how his own love of
German wine made Germans feel about him.

``He's a bit of a maniac about German wine,'' said Holger Schwarz, the wine merchant who organized
the get-together at Hot Spot. ``He loves German wine!''

Mr.~Sarrazin's book, ``Germany Does Away With Itself,'' released last week, attacked Germany's
Muslim immigrants for refusing to integrate, saying they were ``dumbing down society.'' It vilifies
Islam and blames Germany's welfare state for being too generous. In response, the central bank asked
the president of Germany to remove him from the board, and Mr.~Sarrazin on Thursday announced his
intention to quit his post by the end of the month.

The book is selling briskly, however, with many Germans saying that Mr.~Sarrazin has a valid point
and that people like Mr.~Wu -- who are willing to make some of the sacrifices that other immigrants
refuse, or fail, to make -- are the proof. ``He named his son Martin; the Turks would never do
that,'' Monica Diel, whose husband, Armin, is a winemaker, said at the Sunday promotion, expressing
a sentiment that had heads nodding in approval.

In fact, Mr.~Wu gave his son two names -- Martin and a Chinese name, Tao. But it seems that Martin
is ascendant, while Tao is fading. This, Mr.~Wu says with a sigh, suggests that he succeeded in
Germany, but not without some cost to his family identity.

That is one of the deepest fault lines in the debate here. Many Germans want to preserve the
nation's cultural identity by having immigrants leave their traditions behind. Many immigrants
refuse, saying they want to hold on to their cultural identities.

In reality, the two are already blending, especially in places like Berlin, and the Hot Spot. Mr.~Wu
kept his Chinese passport, while his wife and son have become naturalized citizens. ``I didn't try
hard to integrate,'' he said in well-spoken German. ``My cultural background is Chinese, that is
where I feel at home. In the back of my head, Germany is still a foreign country for me.''

At home, he and his wife, Huiqin Wang, try to speak mostly Chinese, but switch sometimes to German
because their son expresses himself better in German.

``I am trying to give the basics of Chinese culture and philosophy to my son so he can be Chinese,''
Mr.~Wu said. ``But he lives here, he has to speak perfect German. He likes China, but he feels less
at home there than I do.''

Mr.~Wu, 50, came to Germany in 1984 from Zhejiang. He frequently laughs, the kind of laugh of a man
still amused by his own good fortune. He earned a degree here in engineering but left school and
opened a restaurant that he said was like a thousand other Chinese restaurants.

``The food was all in one direction,'' he said. ``Really, it all tasted the same. It was bad.''

One day in 1995, he saw a leaflet about wine. He was interested, so he went out and bought 10 cases,
all Bordeaux, thinking he could sell the wines in his restaurant. He never sold one bottle because
the expensive wine did not appeal to customers looking for chop suey. So he took the wine home,
bought a reference guide and drank and studied his way to expertise. In 2003 he met a Chinese
businessman who asked him to research German wine for sale in China.

He ended up at Jos.~Christoffel Jr., a vineyard in the Mosel wine region, where he found his
passion. ``The wine struck me, it was the first really beautiful, interesting riesling I ever had,''
Mr.~Wu says with a smile.

Three years ago, Mr.~Wu and his wife rented a failed steakhouse, hoping to sell Chinese food cooked
with fresh vegetables and sophisticated recipes, as well as fine German wines.

For a year, business was dead. ``We were not accepted,'' he said.

Then, by chance, one of Germany's most celebrated chefs, Christian Lohse, noticed a fine bottle of
German wine on a table as he walked by. Mr.~Lohse and Mr.~Wu became friends, and Mr.~Lohse a
frequent customer. Two years ago, Mr.~Lohse visited Mr.~Wu's restaurant with a television crew. The
day after the program was broadcast the place was filled, and it has been ever since, often with
officials and celebrities. In fact, Mr.~Wu said, Mr.~Sarrazin had lunch at Hot Spot the day his book
came out.

``I think that Germans have the impression that Muslims don't like them,'' said Martin von Hermanni,
a computer programmer and wine lover who attended the promotional buffet and was sharply critical of
Mr.~Sarrazin's book. ``With Mr.~Wu, we see he loves Germany, maybe more than the Germans.''

``Everyone wants to be liked,'' he said as he sipped a glass of riesling.

\pagebreak
\section{China's August Trade Surplus Is Second Highest This Year}

\lettrine{C}{hina}\mycalendar{Sept.'10}{10}'s trade surplus hit its second-highest level this year
in August, likely fueling U.S.~demands for Beijing to ease currency controls.

Export growth weakened as global demand cooled while import growth rebounded in a new sign the
slowdown in the world's second-biggest economy might be less severe than expected, government
figures showed Friday.

The \$20 billion trade surplus will fuel demands by Washington and others for Beijing to ease
controls they say keep its yuan undervalued and give Chinese exporters an unfair price advantage.
American lawmakers hold hearings this month and some are pushing for sanctions as they face pressure
to create jobs ahead of November elections.

''There certainly will be pressure from the United States,'' said economist Lu Zhengwei at
Industrial Bank in Shanghai. ''Their appetite is very big.''

In June, Beijing ended an 18-month-old link between the yuan and the dollar and said it would allow
a more flexible exchange rate, but the Chinese currency has risen by less than 1 percent since then.

''The likelihood of congressional action targeting China for pegging its currency to the dollar is
increasing at an alarming rate,'' said economist Derek Scissors of the Heritage Foundation in
Washington in a report.

August export growth fell to 34.4 percent over a year earlier from July's 38.1 percent, the Chinese
customs agency reported. But import growth rebounded to 35.2 percent from the previous month's 22.7
percent.

Strong imports are a rare bright spot for global exporters that are looking to China to help drive
demand for factory machinery, iron ore and other goods. They suggest China's slowdown was moderating
after growth fell from 11.9 percent in the first quarter of the year to 10.3 percent.

Export growth fell short of forecasts, and Lu said it reflected unexpectedly weak orders by foreign
retailers for the peak Christmas selling season.

''That means the outlook is very grim,'' he said.

The trade surplus narrowed from July's 18-month high of \$28.7 billion but was up 28 percent from a
year earlier. It was well below most private sector forecasts, which ranged as high as \$30 billion.

The U.S.~Commerce Department in August declined to launch an investigation of the currency
complaints despite requests by some lawmakers.

A Chinese foreign ministry spokeswoman rejected pressure over currency during a visit this week by
Obama's top economic adviser, Lawrence Summers, saying Beijing will decide the pace of change.

''Did the authorities in Beijing think that the G-20 would fail to notice that the promised yuan
revaluation has not happened?'' said research firm High-Frequency Economics in a report this week.

Also this week, a major U.S.~union filed a trade complaint asking the Obama administration to crack
down on what it said was improper Chinese government aid to clean energy industries such as wind and
solar power.

The United Steelworkers union said the Chinese aid violates global trade agreements and is hurting
American workers.

The administration has 45 days to decide whether to accept the petition and launch an investigation
that could lead to cases being filed against China before the Geneva-based World Trade Organization.

\pagebreak
\section{Takeover Bid Shines Spotlight on Crucial Player in Potash}

\lettrine{I}{t}\mycalendar{Sept.'10}{10} is hardly a household word like, say, the oil cartel OPEC.
That is why Canpotex, an organization with a corner on Canada's potash production, used to go about
its business in nearly complete obscurity.

But that was before the multinational mining and minerals giant BHP Billiton made a \$38.6 billion
unsolicited bid last month for the Potash Corporation of Saskatchewan. The offer cast a spotlight on
hitherto humble potash, an unsung but vital and valuable component of agricultural fertilizer. And
suddenly Canpotex, too, was revealed as a crucial player in the world's potash supply.

The Potash Corporation is the biggest of three Canadian companies that together own Canpotex, which
is the global marketer of potash from Saskatchewan, home to one of the world's biggest deposits.
Many characterize Canpotex as a legalized cartel for controlling potash prices.

BHP, if it wins control of Potash, has promised to pull out of Canpotex and to stop following the
group's policy of limiting production to raise the price of the fertilizer ingredient. While such a
plan could conceivably reduce food prices worldwide, it has provoked alarm in the province of
Saskatchewan, which collects about \$200 million a year in potash royalties.

When food prices soared in 2007 and early 2008, potash prices joined them, rising to about \$1,000 a
metric ton, up from about \$200. Prices in the relatively small spot market are currently at about
\$350 a metric ton. They have been rising recently, but for reasons unrelated to the takeover bid:
farmers who avoided using fertilizer last year to save money have returned to the market before
their soil becomes depleted. Fearing that lower prices will pinch Saskatchewan's treasury, Brad
Wall, the province's premier, said that he would demand that the federal government use foreign
investment laws to require that any purchaser of Potash keep the company in Canpotex.

``We want to be a welcoming place for international investment,'' Mr.~Wall said in an interview
during a visit to Ottawa this week. ``But we do need to balance that with resource stewardship.''

Canpotex did not respond to requests for comment. Potash declined to speak on the record about its
role in Canpotex.

Economists and analysts agree that Canpotex -- whose other two owners are the two other leading
potash producers in Canada, Agrium and Mosaic -- has been good for Saskatchewan. But their views are
more mixed about Canpotex's impact on global food prices.

Rather than trade its product through open markets, Canpotex negotiates long-term, often secret,
pricing agreements with customers. When faced with demands for lower prices its members often
throttle back production to restrict supply.

Jaret Anderson, a fertilizer industry analyst with Salman Partners in Toronto, said that before the
global economic collapse of 2008, which caused a dramatic decline in fertilizer demand, Potash was
operating at only 55 to 60 percent of its capacity. Last year, he estimates, that fell to 26
percent.

``Canpotex has served the producers very well,'' Mr.~Anderson said. ``They've got a system in place
that's allowed them to have consistent profits.''

The story for potash users, not just Canpotex customers, may be another matter.

Murray E.~Fulton, an agricultural economist at the University of Saskatchewan, said that Canpotex's
contract agreements largely determined what the world paid for potash.

``It would be fair to say that they are the price-setter,'' said Professor Fulton. ``But even
without Canpotex you have production concentrated in a very few sellers.''

Canpotex was formed in 1972 and briefly included one producer from Russia, Uralkali, among its
members. (Interconnected companies in Russia and Belarus are the world's other big producers of
potash.)

Canada's competition laws include a specific exemption for exports, and Canpotex's three current
owners are supposed to coordinate production and pricing only for sales outside of North America.

But Professor Fulton and many others argue that their cooperative pricing approach informally
carries over into sales to the United States, the largest market for the Canadian mines.

``While there is no formal cartel for sales to the United States, there is no question that the
discussions these companies have because of Canpotex have a significant impact on the U.S.~market,''
Professor Fulton said. ``Those discussions allow them to understand what each other is doing. ''

Professor Fulton said that other companies had been able to coordinate pricing using far less
information than the three Canpotex owners regularly exchanged with each other.

But Mr.~Wall, in the interview, dismissed suggestions that Canpotex restricted competition to
manipulate the cost of fertilizer and, through that, food. He compared Canpotex to supply management
marketing systems that were used to keep food prices high for the benefit of farmers in several
countries, including Canada. But unlike Canpotex, those arrangements generally affect only domestic
food, and they have long been a target for elimination in international trade negotiations.

Depending on demand, Canpotex controls about 30 to 40 percent of the world market for potash. In
theory, it has a major competitor in another marketing group made up of the potash producers of
Russia and Belarus. But that group largely focuses its export sales on Europe, a market where
Canpotex does not operate.

Iron ore, currently one of BHP's main products, was once largely sold under fixed prices, but that
system has since largely collapsed. Unlike the Canpotex members, BHP tends to run its mines at close
to full capacity regardless of market prices, in the belief that the approach benefits low-cost
producers over the long run.

In 2008, a group of American potash users filed lawsuits in federal courts in Minnesota and Illinois
against the Canpotex members and other potash producers for fixing prices. Their cases are still
pending. A similar case was turned down by a federal appeals courts in 1999. After Mr.~Wall raised
his concerns about Potash and Canpotex, some news reports in Canada concluded that BHP was
reconsidering its plan to pull out of the group.

Ruban Yogarajah, a BHP spokesman, said that the company did not have access to Potash's agreements
with Canpotex and did not know how quickly it could withdraw. But he said that BHP was committed to
going its own way if it acquired Potash.

``Our preference is to market our production ourselves and we will consult the other Canpotex
shareholders and work within the parameters of the Canpotex agreements with that aim in mind,'' he
said. ``Making changes to the current arrangements may take some time.''

\pagebreak
\section{Top Adviser to Lead Panel on Economy}

\lettrine{P}{resident}\mycalendar{Sept.'10}{10} Obama on Friday will promote a longtime economic
adviser, Austan D.~Goolsbee, to chairman of his Council of Economic Advisers, signaling continuity
even as a high unemployment rate has left much of the public dissatisfied with administration
policies.

Mr.~Obama's decision to elevate Mr.~Goolsbee, a left-of-center economist, to succeed Christina
D.~Romer, who returned this month to the University of California, Berkeley, is part of a broader
flux within the White House economic team, as architects of the government's response to the worst
recession in 80 years begin moving up and out and their roles shift.

Mr.~Goolsbee has been serving as a member of the three-person advisory panel since the beginning of
the Obama administration.

No other major changes are expected, officials say, reflecting a theme the president sounded on
Wednesday in an economic address near Cleveland, that the country should ``keep moving forward with
policies that are slowly pulling us out.''

Republican leaders in Congress, and a few endangered Democrats seeking to distance themselves from
the White House before the midterm elections, have called for Mr.~Obama to fire his top advisers,
including the Treasury secretary, Timothy F.~Geithner. But Mr.~Geithner, who did not know Mr.~Obama
previously, has become one of the president's most trusted advisers, credited with successfully
managing the financial bailout and recovering most of the taxpayers' money. He is expected to remain
for some time.

Asked on PBS's ``NewsHour'' this week about the calls for him to be fired, Mr.~Geithner quipped,
``It's an old idea. A lot of people have had it, and my wife had it first, I think.'' He added,
``I'm going to do this as long as the president asks me to do it.''

Similarly, Lawrence H.~Summers, the director of the White House National Economic Council, is not
expected to leave soon, officials say, despite his history of run-ins with other advisers, and
Mr.~Obama's occasional impatience with the policy vetting process that Mr.~Summers oversees.

Mr.~Obama and the other advisers nonetheless value Mr.~Summers's contributions as a renowned
economist and former secretary of the Treasury in the Clinton administration, these officials say.

Yet the other two principals in Mr.~Obama's economic inner circle -- Ms.~Romer and Peter R.~Orszag,
his budget director -- left in recent weeks, largely for personal reasons, giving Mr.~Obama the
opening to remake his team. But for both vacancies, Mr.~Obama has now picked people from within his
administration.

To succeed Mr.~Orszag, who left in July, the president nominated Jack Lew, who has been a deputy
secretary of state and was a budget director in the Clinton administration. Mr.~Lew is not on the
job yet but is awaiting confirmation by the Senate.

Because Mr.~Goolsbee has been confirmed by the Senate as a member of the Council of Economic
Advisers, he does not need approval to become the chairman -- not a small consideration at the White
House, given how often the president's nominees become bogged down in partisans skirmishes.

Another factor initially worked against Mr.~Goolsbee's elevation -- his sex -- and that also played
a part in his being passed over for the chairman post at the start of the administration. Ms.~Romer
was the only woman among Mr.~Obama's top economic advisers, and administration officials considered
whether to name a woman to replace her.

Also, at 41, Mr.~Goolsbee would be the youngest chairman since Arthur M.~Okun held the job from 1968
to 1969 under President Lyndon B.~Johnson. (Mr.~Okun is known for Okun's Law, which describes the
relationship between changes in employment and changes in output.)

But Mr.~Goolsbee, an amateur comic as well as an economist, was a favorite within the White House,
where many colleagues felt he had earned the chairmanship. He has tense relations with Mr.~Summers,
however, after policy disputes in the early crisis-driven debates over the rescues of the financial
industry and Chrysler, among other issues.

Mr.~Goolsbee, who has a free-market bent, opposed bailing out Chrysler. He did not prevail, but
Mr.~Obama personally sought his arguments.

The Council of Economic Advisers traditionally provides analysis of the economy and of the potential
economic impact of proposed policies. But because the office is largely divorced from politics and
located in a building separate from the White House, many past chairmen have had limited influence.
Ms.~Romer, however, was routinely included in the West Wing deliberations of the last 20 months, and
Mr.~Goolsbee is likely to be as well.

And unlike Ms.~Romer or most past chairmen, Mr.~Goolsbee has a previous, friendly relationship with
the president. Mr.~Goolsbee was an economics professor at the University of Chicago when Mr.~Obama
taught at its law school. He provided economic advice when Mr.~Obama ran for the Senate and for
president.

Mr.~Goolsbee was at the center of a controversy during the Democratic race for the presidential
nomination when it was reported that he had told a Canadian official in Chicago that Mr.~Obama's
protectionist campaign talk was ``more reflective of political maneuvering than policy'' he would
support as president. While Mr.~Goolsbee denied the account, as an economist he does espouse a free
trade philosophy.

Mr.~Goolsbee has also been the staff director of the President's Economic Recovery Advisory Board, a
panel of business, labor and academic officials providing outside perspective. As such he worked
closely with Paul A.~Volcker, the former Federal Reserve chairman, and shared with him a preference
for tougher regulation of the financial industry than Mr.~Geithner and others espoused.

\pagebreak
\section{Ron Bloom Is Obama's Manufacturing Emissary}

\lettrine{N}{ot}\mycalendar{Sept.'10}{10} since Ronald Reagan has an American president spoken so
emphatically about the importance of manufacturing. ``We've got to go back to making things,''
Barack Obama says, embedding that view in his oratory. Yet manufacturing's presence in the American
economy continues to shrink, defying the administration's attempts to reverse that trend.

The president has named Ron Bloom, a Harvard M.B.A. who has worked both on Wall Street and in the
labor movement, as a special adviser to help tackle the problem. Mr.~Bloom's tools, however, are
limited. Apart from his persuasiveness, they have consisted mainly of tax credits and subsidies,
many of them flowing to new industries, like the production of wind turbines, solar panels and auto
batteries powerful enough to replace gasoline engines.

The goal is to invigorate private sector initiatives in these industries, then in a host of supplier
companies and eventually throughout manufacturing. That, in a nutshell, is the administration's
manufacturing strategy.

But some, including the United Steelworkers union, say it is not enough. The union is pressing the
administration to challenge China over what it calls unfair subsidies for its clean energy
industries.

``The president has been very clear and articulate that we believe the nation's trade laws must be
respected,'' Mr.~Bloom said in an interview. Although he declined to comment on the complaint filed
Thursday by the steelworkers, claiming that Chinese subsidies violated World Trade Organization
rules, he noted that the administration had imposed tariffs on tubular steel and tires once the
private sector successfully brought antidumping cases against China.

``Our policies assume that the dominant role in manufacturing will continue to be played by the
private sector,'' Mr.~Bloom said. ``We are not willing to accept manufacturing's decline, but it is
simply not feasible to make the government the principal actor in its revival.''

No other sector has lost so much ground relative to the rest of the economy. Manufacturing's share
of gross domestic product topped out at nearly 30 percent in the 1950s. It is 11 percent today. The
fall has accelerated since 2007, with the recession a contributing factor.

The president reiterated the need for manufacturing in a speech on Wednesday in Cleveland. ``We see
a future where we invest in American innovation and American ingenuity; where we export more goods
so we create more jobs here at home; where we make it easier to start a business or patent an
invention; where we build a homegrown clean energy industry, because I don't want to see new solar
panels or electric cars or advanced batteries manufactured in Europe or in Asia. I want to see them
made right here in the U.S.~of A.~by American workers.''

The Obama administration argues, often through Mr.~Bloom, that the United States cannot sustain
itself as a global economic power without a thriving manufacturing sector. Too much research and
development, too many well-paid jobs and too many exports flow from manufacturing.

Mr.~Bloom, speaking more forcefully than others in the administration, challenges the idea that
research and development can be pursued entirely separate from production. In his view, Americans
cannot excel at high-end innovation while factory production continues to decline or to slip
overseas. ``I am deeply afraid that if you lose the ability to make things,'' he said, ``all the
intellectual activity involved in innovation and design will over time erode as well.''

Grumbling by American manufacturers is fairly constant on Mr.~Bloom's trips to the heartland as the
president's senior counselor for manufacturing, his official title. On his most recent trip, to
Cleveland -- he has visited nearly two dozen cities since his appointment a year ago -- more than 60
executives and factory owners in northeastern Ohio listened, and then had their say. In some cases,
they called for a more expansive government role in manufacturing, along the lines of China, Germany
or Japan.

``We don't have to reinvent the wheel,'' said Andre Morrison, an executive at Green Mill Global,
which makes lighting products. ``But why can't we model our policies on those of other countries,
where government and private industry are in bed together?''

Mr.~Morrison's biggest complaint was credit. He said he knew five or six established companies that
wanted to expand within the United States but could not get loans from commercial banks. He asked,
in effect, that the federal government expand its support for such loans, and Mr.~Bloom replied that
the administration was doing just that, but with a caveat. ``It is not government's role to direct
banks to lend to particular companies or industries,'' he explained.

That unwillingness to interfere with the private sector is characteristic of the administration's
industrial policy. Shunning even the term, Mr.~Bloom prefers to call it a ``manufacturing
strategy.''

Still, some of the Ohio executives pushed for more intervention. Several seemed to nod in agreement
when William N.~McCreary, a vice president of the NSG Group, said that private equity firms and
other financiers frequently asked for an American manufacturer's ``China strategy,'' meaning that
having an operation in China made a company more worthy of financial support.

The NSG Group, which makes flat glass of the sort used in auto windshields, operates in many
countries, including China, and is based in Japan. Even so, Mr.~McCreary, who is based in Toledo,
Ohio, sounded perplexed by this lack of faith in American producers.

``The private equity world is heavily on the side that you have to be in China,'' he said. ``It
thinks the U.S.~is not a place you make things.''

Mr.~Bloom, working out of a Treasury Department office with few assistants, aims to reverse that way
of thinking. At 55, he brings to the job the business degree he earned in 1985, followed by 11 years
on Wall Street as an investment banker and a slightly longer stint as a special assistant to the
president of the United Steelworkers.

President Obama drafted him from the Steelworkers to serve as co-chairman of the auto task force in
the bailout of General Motors and Chrysler -- the administration's initial venture into
manufacturing strategy.

The unions say much more is needed. Too many jobs have been lost over the years; too many factories
have shut or moved overseas, and they want a more muscular strategy than the Obama administration
has fielded.

``There are two pieces to what it would take to rejuvenate manufacturing,'' said Thea Lee, deputy
chief of staff to Richard L.~Trumka, president of the A.F.L.-C.I.O. ``One is trade policy, a more
restrictive approach than the free trade, open borders arrangement that we now have. The other is to
reward domestic production. When the government makes a purchase, for example, the presumption
should be that the first crack goes to manufacturers who stay within the United States.''

Mr.~Bloom notes the criticism and, without disputing it, replies that manufacturing strategy in the
Obama administration, modest as it is, is nevertheless ``huge'' in comparison with previous
administrations, which generally accepted the proposition that manufacturing's decline was
inevitable in a mature, service-oriented economy.

Mr.~Reagan, in contrast, responded to the Rust Belt crisis and the onslaught of imported cars, steel
and computer chips, mostly from Japan, by imposing import quotas to protect American manufacturers.
In addition, he negotiated a currency revaluation that enhanced the dollar in trade. His
administration financed Sematech, which brought together semiconductor manufacturers in the
cooperative development of new technology. He even got the Japanese to commit to purchases of a set
amount of American-made computer chips.

``In those days, you had many more U.S.~companies that cared about these issues,'' said Scott Paul,
executive director of the Alliance for American Manufacturing. ``It was the electronics industry
that went to Washington, en masse, and pushed successfully for the creation of Sematech.''

The world has changed since the early 1980s. The giant corporations -- General Electric, Boeing,
Caterpillar, Whirlpool and Intel, among others -- still had most of their production in the United
States. They have since become global behemoths, selling into foreign markets principally from
overseas factories. As those sales rose, so did the nation's trade deficit. And now, when these
multinationals speak out, they mainly aim to support their worldwide operations through free trade
agreements and lowered tariffs.

The Obama administration is counting on sharp increases in exports to buoy the nation's
manufacturers. The president has set a goal of doubling exports in the next five years. But he rules
out subsidies. Instead, he emphasizes stepped-up financing from the Export-Import Bank and ``much
more aggressive diplomacy in representing U.S.~producers,'' which means using embassy personnel as
sales people.

Mr.~Bloom acknowledges that such steps take time to produce results. ``We have been insufficiently
committed to manufacturing for a long time,'' he said. ``We are not going to fix it in an
afternoon.''

\pagebreak
\section{Apple Gives App Developers Its Review Guidelines}

\lettrine{A}{pple}\mycalendar{Sept.'10}{10}, which keeps a tight lid on its products and inner
workings, is loosening up a little.

The company said on Thursday that it would relax its rules on how software developers can build
applications for its iPhone and iPad. And for the first time it published detailed guidelines
explaining how it decides what programs can and cannot be sold through its App Store.

Apple has long faced criticism from developers who say its application review process is opaque and
that it makes seemingly arbitrary decisions about what is acceptable for its customers.

This year, some were incensed when the company purged some sexually suggestive applications it
deemed to be inappropriate -- like a jigsaw puzzle featuring scantily dressed women -- while a
Sports Illustrated application with bikini-clad models remained.

Analysts said the moves on Thursday were a sign that Apple was growing increasingly aware of
competition in the smartphone market, and was trying to be friendlier to the developers whose
applications have helped drive the success of its products.

In particular, Android, the mobile operating system by Google, has steadily increased market share
in the United States and abroad. Shipments of smartphones using Android grew by 8.86 percent during
the second quarter from a year earlier, according to the research firm Canalys. And a wave of
Android tablet computers that will compete with the iPad are on the way.

``Apple is concerned enough about the shifting tide towards Android that it feels it has to loosen
restrictions to keep developers on its side,'' said Jeffrey S.~Hammond, an analyst at Forrester
Research.

Although Mr.~Hammond said that Apple's App Store was outpacing opponents in both the number of
applications available for download and the amount of revenue generated for developers, he said
Apple's announcement was a ``pre-emptive strike.''

``Google has done a better job at leveling the playing field for independent developers, and that
matters,'' he said. ``I don't think Apple can tolerate that, especially with the dozen or more pads
that we're going to see hit the market in eight to nine weeks.''

In the newly released guidelines, which are written in an unusually conversational style, Apple
acknowledges that its App Store policies could give the impression that the company is made up of
``control freaks,'' but adds that ``maybe it's because we're so committed to our users and making
sure they have a quality experience with our products.''

Although the guidelines list many things that will lead to an app's rejection, like the inclusion of
pornography or violent images, or mimicking features that are already on the iPhone, they still
leave much to interpretation. For example, Apple says that ``apps that are not very useful or do not
provide any lasting entertainment value may be rejected.''

Even so, software developers, many of whom have expressed frustration about the app review process,
said the company's newfound candor came as a relief.

``This is gold. This is great,'' said Dom Sagolla, chief executive of Dollar App, a mobile
development company based in the Bay Area. ``It feels like we're finally getting a clue about what
Apple wants.''

``This is a document I've been wanting to see for two years,'' said Raven Zachary, president of
Small Society, a software development firm. ``It's going to foster the creation of better apps
because we know going in what to do and what to avoid.''

Apple also said it would begin to allow developers to use third-party tools to create applications
for its iOS mobile operating system, which is used on the iPhone, iPad and iPod Touch. It had banned
such tools in April.

The change means developers can more easily convert applications written for other types of mobile
phones, including those that use Adobe's Flash technology, which Steven P.~Jobs, Apple's chief
executive, has repeatedly said he will not support. The news bumped shares of Adobe up more than 12
percent to close at \$32.86. Apple shares were little changed.

Apple's new rules also specify that developers can put advertisements in their applications that
come from outside companies. Its earlier rules had raised some questions about whether developers
would be limited to using Apple's iAd service, locking out companies like AdMob, which is owned by
Google.

Omar Hamoui, the former chief executive of AdMob who is now the vice president for product
management at Google, said in a blog post that the changes were ``great news for everyone in the
mobile community.''

``Apple's new terms will keep in-app advertising on the iPhone open to many different mobile ad
competitors and enable advertising solutions that operate across a wide range of platforms,'' he
added.

Good will aside, analysts said Apple may also be trying to keep federal regulators at bay. The ban
on third-party tools and the uncertainty about outside ads had prompted the Federal Trade Commission
to begin asking questions about Apple's approach to competition.

``At some point, Apple asked themselves whether this is doing us more harm than good, between
unhappy developers and the F.T.C. sniffing around,'' Mr.~Hammond said. ``For once, Apple is being a
little bit defensive in their strategy as opposed to being offensive. It just shows how the race is
starting to tighten in this market.''

\pagebreak
\section{Google Unveils Tool to Speed Up Searches}

\lettrine{G}{oogle}\mycalendar{Sept.'10}{10}, which can already feel like an appendage to our
brains, is now predicting what people are thinking before they even type.

On Wednesday, Google introduced Google Instant, which predicts Internet search queries and shows
results as soon as someone begins to type, adjusting the results as each successive letter is typed.

``We want to make Google the third half of your brain,'' said Sergey Brin, Google's co-founder and
president of technology, speaking at a Google press event at the San Francisco Museum of Modern Art.
Marissa Mayer, Google's vice president for search products and user experience, added, ``There's
even a psychic element to it.''

Google's new psychic powers result in much faster searches, but the change might affect the many
businesses that have been built around placing search ads on Google and helping Web sites figure out
how to climb higher in search results to increase revenue.

It is a sign that even as Google expands into other businesses, like display advertising and
cellphones, it remains firmly focused on search, its core business and one that accounts for more
than 90 percent of its revenue. It has faced competition recently from Microsoft's Bing search
engine.

Google has made its new product the default way to search the Web. Instant works with the most
popular modern browsers in the United States and several other countries. It will show up on
cellphones and in browser search bars in a few months.

``It's been awhile since there's been a game changer in search, and this is,'' said Jordan Rohan, an
Internet analyst at Stifel Nicolaus. ``It changes how people search.'' He added that it was a feat
of computing and engineering that could not ``easily be mimicked by Google's competition.''

Google's attempt to greatly change the way people search could throw a wrench in the business models
of the companies that have latched onto the Google ecosystem.

Analysts said that it was too early to tell how Google Instant would affect search engine marketing
and optimization businesses. But because Web pages and ads show up before people finish typing
queries, it could be more challenging and expensive for them to pick keywords that catapult their
sites to the top results, analysts said.

For example, it has been less expensive for a small hotel in Paris to buy ads that show up when
someone searches ``Paris boutique hotels in the Marais'' than when someone searches ``Paris
hotels.'' But now that Google immediately starts showing results, people may type long queries less
often. As a result, advertisers would have to bid for more common terms.

Also, because Google Instant focuses attention on the search box and top results, people could spend
less time looking at the ads that show up on the right side of the page.

``The general murmur is that advertisers are not particularly pleased because they see it as Google
strong-arming people to buy the most expensive terms and make the most money for Google,'' said Matt
Hessler, account director for Trada, a search marketing firm.

Johanna Wright, director of product management at Google, said that Google was not changing the way
it ranked or served ads, and that the instant results were more useful for consumers because ``we're
getting you your answer much quicker and you're not having to scroll and wade through information.''

Before the change, Google's search results probably did not strike anyone as slow. But with Google
Instant, they can easily be twice as fast.

Most people's lives will not change with an extra few milliseconds, but Google calculated that the
new tool would cumulatively save people more than 3.5 billion seconds every day, which works out to
about 11 hours a second.

The results will also be more useful because people can adjust their searches on the fly, based on
what they see, and explore similar queries they might not otherwise have seen, says Ben Gomes, a
Google engineer working on search. ``We want to give you feedback as you type your query, so you can
formulate a better query,'' he said.

Still, some users said they found it distracting to see other queries pop up as they were typing.
And since Google says that a fifth of searches are brand new, it will not always be able to make an
accurate prediction.

``It's not quite psychic, but it is very clever,'' said Othar Hansson, a senior staff software
engineer who helped develop Instant.

To make the predictions, Google relies on search trends, like words that are often searched, were
recently popular or were searched nearby, Ms.~Mayer said.

Some words, like ``nude,'' produce no results because Google Instant filters for violence, hate and
pornography, the company said.

Google, which already handles more than a billion searches a day and has a billion users a week, had
to figure out how to manage the load when suddenly each letter typed was a separate search query.
The solution includes storing frequent searches and sending common ones, like ``Barack,'' back more
quickly than ones that are nearly impossible to predict, like ``Bill.''

``We had to figure out how to do it without melting down our data centers,'' Mr.~Gomes said.

\pagebreak
\section{Scarborough and Kinsley Will Write for Politico}

\lettrine{P}{olitico}\mycalendar{Sept.'10}{10} has built a successful enterprise on the idea that
there is no such thing as too much information when it comes to political news. Now it is going to
apply that concept to political opinion.

Starting on Oct.~1, Politico will run weekly opinion columns by Joe Scarborough, the MSNBC host and
former United States representative, and Michael Kinsley, a columnist for The Atlantic.

In the cacophony of political opinion on blogs and cable television and in newspapers, whether news
media consumers will see the need for yet another set of dueling voices is far from certain.

Politico's executive editor, Jim VandeHei, acknowledged how difficult it could be for pundits to
break through in a news media environment that has become saturated. But he said he thought that
Mr.~Scarborough's and Mr.~Kinsley's commentary was distinct and respected enough that it would speak
to jaded readers.

``Lord knows, there is enough noise and nonsense out there,'' Mr.~VandeHei said. ``What these guys
bring is intellectual muscle and track records of challenging the conventional wisdom.''

The opinion columns will be a departure from Politico's core mission.

``We have been hesitant to have pure opinion voices on our pages because we are so focused on
breaking news and providing reported analysis,'' Mr.~VandeHei added. ``But the chance to get two
exceptionally smart thinkers with reputations for fairness and thoughtfulness was irresistible.''

Initially, Mr.~Scarborough and Mr.~Kinsley will write once a week. But Mr.~VandeHei said his
intention was to have them engage in back-and-forth commentary on big news events like the State of
the Union address.

Mr.~Scarborough, who will remain with MSNBC, said he did not intend to write a column so
ideologically rigid it became predictable. Readers have tired of that model, he said. ``There are
some columns, even at the great papers, that I just skim because I know how they're going to end
even before I read the first sentence,'' he said. ``The last thing I'm going to do is get in an
ideological corner and preach to the choir.''

Mr.~Kinsley, who will leave The Atlantic, could not be reached for comment. His departure will be a
blow to the Atlantic Media Company, which has been establishing a stable of journalistic talent in
an effort to broaden its reach in Washington. The company owns National Journal, which has hired
away journalists from places like The Wall Street Journal and Politico as it prepares to redesign
its magazine and start a breaking news and political analysis Web site that will compete directly
with Politico.

\pagebreak
\section{Stem Cell Financing Ban Ends, for Now}

\lettrine{A}{}\mycalendar{Sept.'10}{10} federal appeals court here ruled Thursday that federal
financing of embryonic stem cell research could continue while the court considers a judge's order
last month that banned the government from underwriting the work.

The ruling by the United States Court of Appeals could save research mice from being euthanized,
cells in petri dishes from starving and scores of scientists from a suspension of paychecks,
according to arguments the Obama administration made in the case.

It could also allow the National Institutes of Health to provide \$78 million to 44 scientists whose
research the agency had previously agreed to finance.

The stay also gives Congress time to consider legislation that would render the ban, and the court
case behind it, largely moot, a prospect that some embattled Democrats have welcomed. Despite
staunch opposition by some critics, embryonic stem cell research is popular, and a legislative fight
on the issue could prove a tonic for Democrats battling a tough political environment.

``I've been inundated with calls from freshman and other vulnerable members saying it's not only the
right issue to work on, but will also be politically good for them,'' said Representative Diana
DeGette, Democrat of Colorado.

The appeals court's ruling was welcomed by advocates of stem cell research and condemned by those
opposed, but it intensifies the uncertainty that has surrounded the research since Aug.~23, when
Chief Judge Royce C.~Lamberth of Federal District Court for the District of Columbia ruled that the
government's stem cell rules violated a legislative ban on federal money being used to destroy
embryos.

In the order on Thursday, the appeals judges wrote that their ruling ``should not be construed in
any way as a ruling on the merits'' of the case. The judges gave both sides until Sept.~20 to file
written arguments.

Carl W.~Tobias, a professor at the University of Richmond School of Law, said the appeals court was
likely to overturn Judge Lamberth's ban in its final ruling.

Judge Lamberth's order ``has been roundly criticized by many people in terms of the legislative
history,'' Professor Tobias said.

Steven Aden, the lawyer for two researchers who have said that embryonic stem cell research is
unethical and are plaintiffs in the case, said the appeals court ruling ``essentially calls a
timeout.''

``We remain confident that the appeals court will recognize that the law is clear and will uphold
the temporary injunction,'' Mr.~Aden said.

Matthew Miller, a spokesman for the Justice Department, said the government was ``pleased with the
court's interim ruling, which will allow this important, lifesaving research to continue while we
present further arguments to the court in the weeks to come.''

The research potential for embryonic stem cells, which were discovered in 1998, arises from their
ability to morph into any cell in the body and possibly form new organs. The stem cells are derived
from human embryos that are several days old, and the embryos are destroyed in the process, leading
some in the anti-abortion movement to liken the research to murder.

The embryos used to create stem cells were donated to research by couples who had had the embryos
created in fertility centers but no longer needed them.

Among the projects whose financing was threatened by Judge Lamberth's order was one overseen by
Dr.~Ira J.~Fox, a professor of surgery at the University of Pittsburgh, who has used embryonic stem
cells to successfully transplant new liver cells into animals.

Another threatened project was one by Dr.~Xuejun Parsons of the University of California, Riverside,
who hopes to use embryonic stem cells to create nerve cells that could replace those damaged by
Parkinson's disease.

President George W.~Bush was the first to allow federal financing of human embryonic stem cell
research, but he limited the research to 21 cell lines already in existence to discourage further
destruction of embryos. President Obama promised in his campaign to expand the research and ordered
the health institutes last year to create rules to do just that.

This year, the health institutes provided \$131 million for the work, a vast expansion.

But Judge Lamberth's ruling last month was so sweeping that the Obama administration interpreted it
as a ban on all stem cell research, including projects that had passed muster during the Bush
administration.

Eight research projects at government labs with a combined annual budget of \$9.5 million were
suspended immediately, and university researchers were warned not to expect any further annual
payments for projects that had already won financing.

At first, the government appealed the ruling to Judge Lamberth, predicting that his ban would
squander the \$546 million in government money already invested in human embryonic research and lead
to the loss of ``unique biological materials that have taken years to develop and that require
ongoing maintenance.''

Judge Lamberth rejected that appeal on Tuesday, writing that the government was ``incorrect about
much of their `parade of horribles' that will supposedly result from this court's preliminary
injunction.''

\pagebreak
\section{The Surgeon's Pact With the Patient}

\lettrine{T}{he}\mycalendar{Sept.'10}{10} patient, in her late 50s with failing kidneys, had come to
the hospital for what she and her doctors thought would be a simple procedure preparing her for
dialysis. But instead of returning home the next day, the woman ended up in the hospital for nearly
half of my internship. Her procedure went awry, she landed in the intensive care unit, and over the
course of the next six months she returned at least a dozen more times to the operating room, all
failed attempts to right what had gone so terribly wrong.

Her bed in the I.C.U. was in plain view to any doctor or nurse walking by. Even today, I can recall
the sickeningly sweet odor of what had become chronic open wounds, the sounds of the bells and
whistles of the small army of machines that kept her alive and the increasingly rancorous
discussions between the lead surgeon and other clinicians as the months dragged on. The surgeon,
ever more haggard, pressed on, convinced that one day he'd send her home. But the others -- nurses,
consultants and eventually the hospital ethics committee too -- began demanding that her care plan
be changed. They wanted to cease all life support interventions and begin comfort care.

One morning, I found the room empty; the woman had died. ``She finally did it on her own, without
any help from you-know-who,'' one of the nurses said grimly, a look of disdain flashing across her
face. ``That's the problem with surgeons,'' she continued. ``Sometimes you guys do the most amazing
things for your patients, and sometimes you just won't let them go.''

That belief -- that surgeons can be both Dr.~Jekyll and Mr.~Hyde when it comes to the doctor-patient
relationship -- has been embraced for generations by more than a few nonsurgical doctors, nurses and
patients. Heroic in their devotion to patients when they are at their best, surgeons inexplicably
seem to transform when they are at their worst. That worst usually comes on the heels of a high-risk
operation and a complicated and protracted postoperative course. The nurses, other doctors and
sometimes even the patient and family request palliation only; in response, the surgeon often
stalls, hesitates or simply refuses.

Since the late 1970s, ethicists and social scientists have tried explain what they viewed as
surgeons' paradoxical behavior with postoperative patients. One of the earliest researchers
attributed to self-protection the surgical imperative to ``do everything possible.'' Inevitably,
this medical sociologist reasoned, all surgeons commit a technical error over the course of their
careers. By doing everything possible ``for the patient,'' surgeons protected themselves against the
emotional distress of failure. The rationale behind this common-sense theory was straightforward: At
least I did all that I could possibly do.

More recent researchers, however, have suggested a more high-minded, almost theological explanation.
With unintended irony, they propose that the surgeon's view of the doctor-patient relationship is
similar to biblical covenants formed between God and his people. I will not abandon you -- this
theory contends -- and I will battle to the death for you.

All of these allusions are intriguing, but none of the research has ever focused on the
surgeon-patient relationship before the operation, on the period of time when the commitment is
actually forged. By overlooking these critical moments, the earlier studies fail to address what is
perhaps the most perplexing question of all: how could any doctor, surgeon or otherwise, be so
confident about pushing for more when the patient and the patient's family seem to be calling for
less?

A study published this year offers an interesting possible answer.

With support from the Greenwall Foundation, researchers from the University of Wisconsin in Madison
and the Medical College of Wisconsin in Milwaukee asked a small group of doctors involved in
high-risk elective operations about withdrawal of life support, advance directives and informed
consent and presented clinical scenarios that involved withdrawing care.

In interview after interview, the surgeons referred to a negotiation and agreement -- what the
researchers called ``surgical buy-in'' -- that occurred during the consent process, long before
these doctors and their patients ever entered the operating room. The surgeons believed that
patients not only consented to the operation itself but also committed themselves to any care after
the operation necessary for successful outcomes. They talked about the operation and postoperative
care as being a ``package deal'' and about a tacit ``two-way agreement'' that included even
well-articulated and well-defined numbers of postoperative days.

``Clearly, surgeons believe something significant is happening during the consent process,'' said
Dr.~Margaret L.~Schwarze, lead author of the study and an assistant professor of surgery and
bioethics at the University of Wisconsin in Madison. For many of the surgeons, dealing with
complications was an important part of their work, and they wanted their patients to have a full
understanding of all the challenges involved before the operation.

``All of the postoperative challenges are tied in with the operation,'' said Dr.~Karen J.~Brasel,
senior author and a professor of surgery and bioethics at the Medical College of Wisconsin.
``Surgeons don't want to invest themselves in a relationship and a technical tour de force, then
have to walk away.''

But while the surgeons believed they conveyed this information clearly during preoperative
conversations, the patients probably heard only part of the story. ``It's entirely possible that
patients are missing this part of their early interaction with surgeons,'' Dr.~Schwarze noted.
Patients may be overwhelmed at the prospect of a high-risk operation and focused solely on getting
through the operation alive. ``The operation is already a lot for patients to take in. I don't know
how much it occurs to them that there may be something worse than survival.''

The result, unfortunately, is the familiar, and potentially devastating, misalignment of surgeon and
patient expectations.

While the researchers concede that their current work is based on a small sampling of surgeons,
their findings have garnered the attention of palliative care experts. Over the next few months,
they hope to complete analysis of a survey involving more than 900 surgeons across the country and
begin a second study focused exclusively on the perspectives of surgical patients.

``Surgical buy-in is a real phenomenon,'' Dr.~Brasel said. ``It is not inherently bad, and it may
even be ethically necessary for surgeons to do what they need to do. But it doesn't preclude patient
autonomy, and it doesn't preclude palliative care.''

She added, ``It just has to be understood from the context of a two-sided relationship, as all
relationships are.''

\pagebreak
\section{A Chinese Advocate Is Freed, but Stays Under Surveillance}

\lettrine{A}{}\mycalendar{Sept.'10}{10} blind, self-taught lawyer imprisoned by the Chinese
authorities in 2006 after years of exposing government abuses was freed Thursday and confined to his
home in Shandong Province, surrounded by guards and watched through closed-circuit surveillance
cameras.

The lawyer, Chen Guangcheng, earned global attention -- and the government's enmity -- by
challenging the legality of government policies that exploited farmers, discriminated against the
disabled and brutally enforced China's one-child policy.

That last crusade documented sterilizations and abortions forced on thousands of peasants in
Mr.~Chen's native Linyi County despite national policies barring such practices. The class-action
lawsuit he prepared also alleged that the authorities had beaten and tortured parents who sought to
escape the procedures.

Security officials detained Mr.~Chen in 2005, hours after he discussed the case in Beijing with
journalists from Time magazine, and kept him under house arrest or in jail for a year before his
trial. He was convicted in August 2006 on charges of destroying property and assembling a crowd to
disrupt traffic, related to an incident in his village that year, and given a 51-month jail
sentence.

His principal defense lawyer was accused shortly before the trial of stealing a wallet, and was
released from custody only after the trial had ended.

Mr.~Chen's imprisonment became a global cause for human rights activists, and international protests
led the government to retry him -- but he was convicted on the same charges.

Mr.~Chen had severe gastrointestinal problems in prison and was denied Braille reading material, his
family has said. A Beijing lawyer and friend, Teng Biao, said by telephone on Thursday that Mr.~Chen
remained heavily guarded at his home in Dongshigu, a village near Linyi City.

``He said that he was escorted home early this morning at 6 a.m.,'' he said. ``There were about 30
people in the village and eight or nine guards in his courtyard to prevent his family from getting
in touch with the outside.''

He continued: ``He also told me that he suffered from chronic diarrhea in prison, and food poisoning
happened very often there. He was using his relative's cellphone, but later I heard that all of
their cellphones are being blocked right now.''

Mr.~Chen, who turns 39 in November, lost his sight as a child after a fever and was trained in
acupressure, a Chinese medical technique that is one of the few professions available to the blind.

Although he never formally attended law school, he audited enough law classes to begin assisting
local villagers with legal problems, and began filing cases in local courts seeking to force
officials to abide by laws protecting the disabled and small farmers.

But it was his documentation of abuses of China's one-child policy that led to his detention.

On Thursday, the advocacy group Human Rights Watch applauded his release in a written statement,
adding that he should never have been imprisoned. A Hong Kong researcher for the organization,
Nicholas Bequelin, said in an interview that Mr.~Chen's continued home confinement under guard, a
practice known in China as ``soft detention,'' has no basis in Chinese law.

``Because his case has attracted international attention, Beijing authorities have given the
instruction to local affiliates that he should not be able to cause embarrassment to China for his
work, and local authorities should not do anything that raises his profile,'' he said. ``So in the
end, what they do is to prevent him from doing anything without arresting him.''

\pagebreak
\section{Philippines Leader Faults Police Over August Siege}

\lettrine{T}{he}\mycalendar{Sept.'10}{10} Philippine government acknowledged for the first time on
Thursday that some of the Hong Kong tourists killed aboard a hijacked bus last month here may have
been hit by police fire.

The disclosure, along with comments from a sometimes defensive news conference on Thursday by
President Benigno S.~Aquino III, may reignite Hong Kong's criticism of the handling of the hostage
standoff and is likely to continue to strain relations between the Philippines and China.

In a 90-minute live interview with the three largest television networks in the Philippines,
Mr.~Aquino said he was finished apologizing for the episode and lashed out at leaders in Hong Kong,
where news outlets have carried reports of the Philippine police and government missteps on a
near-daily basis. Donald Tsang, the chief executive of Hong Kong, had complained specifically that
Mr.~Aquino did not take his telephone calls during the standoff.

Mr.~Aquino said that he did not respond to Mr.~Tsang's calls because they had not been expected that
day and that the government could not immediately confirm the caller's identity.

Mr.~Aquino added that he chose to ignore a letter sent by Hong Kong leaders after the crisis, saying
he found it ``insulting.''

``We were told in very minute detail what we were supposed to do,'' he said.

Even so, Mr.~Aquino acknowledged Thursday that he could have taken a more hands-on approach in
orchestrating police tactics during the hostage crisis, which left eight members of the Hong Kong
tour group dead.

The seven tourists and their tour guide were killed on Aug.~23 when their bus was commandeered by a
former police officer, Rolando Mendoza. Mr.~Mendoza, who had been dismissed by the Manila Police
Department over extortion charges, demanded that he be reinstated in exchange for the release of the
tourists.

As negotiations collapsed, gunfire was heard within the bus, and Mr.~Mendoza was killed by a police
sniper.

The panel investigating the crisis, led by Justice Secretary Leila de Lima, is expected to wrap up
its investigations next week.

On Thursday, Ms.~de Lima said that ``there is a big possibility'' that some of the hostages may have
been hit by ``friendly fire.'' She said ballistic experts had indicated that Mr.~Mendoza could not
have killed all the victims.

Much of Thursday's televised news conference focused on whether Mr.~Aquino should have played a more
direct role in the crisis. In what has become his administration's first test, Mr.~Aquino has tried
to quell the furor from inside and outside the Philippines by assuming ``full responsibility'' for
the botched hostage rescue.

But on Thursday he blamed police officials and sought to justify his actions. ``I presumed that I
could trust officials when they assured me that everything is O.K.,'' Mr.~Aquino said. ``But that
was not the case.''

He later added, ``Perhaps I should have taken a more active role.''

The president said he trusted the police officials on the scene. ``They promised and guaranteed to
me that hostages would be safe, but that did not happen,'' Mr.~Aquino said. He admitted to being
frustrated as the situation deteriorated, saying he had to rush to a restaurant near the scene of
the standoff to learn what was happening.

Mr.~Aquino said he would wait for the results of the government investigation before deciding
whether to discipline any officials.

On Thursday, the Chinese Embassy in Manila said in a statement that it expected the Philippine
government to produce ``a comprehensive and fair report.''

\pagebreak
\section{As North Korea Marks Its 62nd Year, the World Is Left Guessing}

\lettrine{N}{orth}\mycalendar{Sept.'10}{10} Koreans marked the 62nd anniversary of their nation's
founding on Thursday with ceremonies honoring its first leader, Kim Il-sung, as speculation abroad
deepened over political developments in the reclusive nation.

North Korea had said that delegates of the ruling Workers' Party would meet in the capital,
Pyongyang, in early September. The meeting, the largest party gathering since 1980, was widely seen
as a chance for Kim Jong-il, the founder's son and the current leader, to reaffirm his control over
the party and promote one of his sons, Kim Jong-un, as the heir to family rule.

As of late Thursday, however, North Korea had offered no word as to whether the meeting was under
way. Its state-run media have reported only that delegates were arriving in Pyongyang.

Analysts and news media in Seoul had predicted that the party meeting would start by the Thursday
anniversary. They have now begun speculating that North Korea might have delayed the meeting because
roads damaged by recent floods had stranded delegates from provincial towns, or because of Mr.~Kim's
failing health.

The debate in the South has also turned linguistic. Some experts here argued that when North Korea
said the meeting would be in ``sangsun,'' or ``early,'' September, it could have meant the first
half of the month.

In the North, soldiers in uniform and women in colorful traditional dresses presented flowers and
bowed before a giant bronze statue of Kim Il-sung on a hill overlooking Pyongyang, a scene captured
on video by Associated Press Television News. Party officials made pilgrimages to Kumsusan Memorial
Palace in Pyongyang, where Mr.~Kim's embalmed body lies in state, according to North Korean media.

Kim Jong-il appears to prize the mystery surrounding his government. When he met President Kim
Dae-jung of South Korea in 2000, he seemed delighted to learn that the outside world regarded him as
a hermit and his government as an enigma. His rise to power during the cold war years was
accompanied by intense speculation by outside analysts about his leadership ability and personal
quirks.

The South Korean news media are once again rife with speculation as Mr.~Kim prepares to take his
family's rule into a third generation. The North's state-run media have never mentioned Kim Jong-un
by name or published photographs of him.

Few North Korea watchers in Seoul believe that Kim Jong-un, believed to be in his late 20s, would
formally assume power as long as his father was alive and in office. But analysts were watching to
see whether he would assume significant party posts during the coming meeting.

Kim Jong-il, now 68, oversaw important party affairs years before he officially took over after Kim
Il-sung's death in 1994.

No such apprenticeship was reported for Kim Jong-un until after his father had a stroke in 2008.

The North's economy is weaker than when Kim Jong-il inherited power. But his military, with its
nuclear weapons, has become a bigger threat to the region.

In Seoul, Unification Minister Hyun In-taek said at a forum: ``North Korea stands at the crossroads.
Despite its intention, its nuclear weapons development is deepening the instability of its regime
and its economic crisis.''

\pagebreak
\section{Combat Game Goes Too Far for Military}

\lettrine{S}{gt}\mycalendar{Sept.'10}{10}. First Class Brian Hampton knows it is just a video game.
But the details are unnervingly familiar: the uniforms, the weapons, even the military bases and
desert towns where the action is set. Each time Sergeant Hampton, a veteran of the war in
Afghanistan, plays, his heart rate spikes, his breathing quickens and his muscles tense.

``It brings back a real reminder of what it actually felt like to be out there,'' Sergeant Hampton,
31, said Thursday.

That video war games, with ever-greater verisimilitude, provoke such a physical reaction makes the
thought that some people might play at killing American soldiers all the more disturbing for some.

The lifelike simulations of combat are in part the product of a close working relationship between
video game producers and the military. Game makers use access to military facilities and combat
veterans to provide depth of detail. Recruiters, in turn, try to sell teenagers who grew up playing
the games on the idea of signing up to experience the real thing. The games themselves can be found
in stores on military bases and are wildly popular among service members.

But there was an unexpected rupture in that relationship when the organizations that run the stores
on Army, Air Force and Navy bases announced they would refuse to sell a soon-to-be-released combat
simulation game, Medal of Honor by Electronic Arts, one of the world's biggest video game
publishers. (The body that runs the stores -- known as PXs -- for the Marines was still weighing
whether to make the game available.)

At issue is a feature in the game, set in post-Sept.~11 Afghanistan, that allows a user to become a
Taliban fighter and attack American troops.

``Out of respect to those we serve, we will not be stocking this game,'' Maj. Gen.~Bruce Casella,
commander of the Army and Air Force Exchange Service, which runs retail operations, said in a
statement last week. ``We regret any inconvenience this may cause authorized shoppers, but are
optimistic that they will understand the sensitivity to the life-and-death scenarios this product
presents as entertainment.''

Here in the city that abuts Fort Leavenworth, where American flags line the streets and businesses
are accented with the earth tones of camouflage uniforms, the decision provoked a mixed reaction
among soldiers. Some said they supported it, characterizing the idea of Americans pretending to be
Taliban fighters for fun as tasteless. Others suggested that the decision reflected a fundamental
misunderstanding of the moral ambiguities embraced in video games and accused the military of
censorship.

``It's a form of entertainment,'' said Sgt. Danny Waynes, 28, who served in Iraq. ``I think it's
wrong to blatantly censor something, whether it be a book, a movie or a video game.''

Staff Sgt. William Schober, 28, said that he was a big fan of the previous versions of the game but
that he found the new game insensitive.

``You know how many of my friends have been killed by the Taliban?'' Sergeant Schober asked. ``One
of my friends was sniped in the head by them. That's something you want to have fun with?''

Medal of Honor was produced with the blessing and assistance of the military, which allowed
Electronic Arts access to a mock Iraqi village used for training purposes at Fort Irwin in
California. But an Army spokesman said that the Army was not aware that users would have the
capability of fighting against United States troops and added that the review process would be more
thorough in the future.

A spokesman for Electronic Arts said the company respected the decision but stressed that the game
was intended to celebrate the role of American soldiers. He said that video games increasingly give
users the options of embracing the role of bad guy during multiplayer showdowns (when people playing
online can play against others online), noting that the last version of Medal of Honor, set in World
War II, allowed players to fight against the Allied forces.

``Basically it's a cops-and-robbers dynamic,'' said the spokesman, Jeff Brown. ``Someone has to play
the bad guy.''

Even some of those who supported the decision not to sell the games at military stores said they
still planned to go to stores off base to buy a copy.

``I still might end up getting the game,'' Sergeant Hampton said during a lunch break, ``but I won't
play it like that.''

\pagebreak
\section{China and Japan Bristle Over Disputed Chain of Islands}

\lettrine{D}{espite}\mycalendar{Sept.'10}{10} recent efforts to tamp down territorial disputes,
China and Japan are jostling elbows over one of their thorniest such conflicts: control of a tiny,
uninhabited island chain in the East China Sea.

On Wednesday morning, the Chinese Foreign Ministry summoned Japan's ambassador for the second time
in 24 hours to protest Japan's response to a Chinese fishing boat that had entered disputed waters.

On Tuesday, two Japanese naval vessels tried to intercept the Chinese boat, but the three collided.
On Wednesday, the boat's captain was taken to the Japanese island of Okinawa for questioning.

``We demand Japanese patrol boats refrain from so-called law enforcement activities in waters off
the Diaoyu islands,'' the spokeswoman for China's Foreign Ministry, Jiang Yu, said Tuesday at a
weekly news conference, according to the state-run official Xinhua news agency. ``We will closely
follow the situation and reserve our right to take further actions.''

Territorial disputes are common in Asian waters, with some of the most nettlesome surrounding
islands chains between China, Japan, Vietnam and the Koreas. In July, Secretary of State Hillary
Rodham Clinton raised Chinese hackles by saying the United States would try to help solve a dispute
between China and Vietnam over disputed islands in the South China Sea. China claims full
sovereignty over the islands and has insisted that it will resolve any rival claims in one-on-one
negotiations with its neighbors.

The islands at issue in the current dispute, known as Senkaku in Japan and Diaoyu in China, are
claimed by both, as well as by Taiwan, but controlled by Japan.

The islands have been a point of contention among those countries for over four decades. In the
1970s, Taiwanese activists landed on the islands, while Hong Kong groups have staged similar forays.

The issue is tricky for Beijing because it needs to balance nationalists' demands with ties to one
of its most important trading partners. In recent years, China has discouraged nationalists from
pushing the issue and sometimes censored Internet forums, although some chat rooms on Wednesday had
angry calls for boycotts of Japanese goods.

The authorities in Beijing permitted a small protest on Wednesday in front of the Japanese Embassy
by a group demanding Chinese control of the islands.

``This is an issue that no Chinese government can give in to,'' said Zeng Jianhong, a researcher on
Chinese-Japanese relations at the Chinese Academy of Social Sciences. ``But China does not want a
conflict with Japan over this issue.''

On Wednesday the normally bellicose Global Times urged restraint, saying in an editorial: ``China
did not encourage or instigate its people to cruise into the Diaoyu waters. Japan should also
refrain from overreacting to civilian boats occasionally entering this area.''

Japanese officials also sought to play down the issue, with the chief cabinet secretary, Yoshito
Sengoku, saying the issue would not affect ties with China.

``We will handle the matter firmly in accordance with the law,'' he said, according to The
Associated Press. ``It is important that in Japan we not get overly excited.''

\pagebreak
\section{Beijing's Vision for the Future}

\lettrine{B}{eijing}\mycalendar{Sept.'10}{10} -- ``Henry Kissinger sat in this seat,'' the guide
said. ``Henry Paulson'' -- the former U.S.~treasury secretary -- ``sat there.''

The seats in question were equipped with safety belts to ensure against mishaps as they tilted and
turned in the Dynamic Movie Hall, where the moving chairs combine with high-tech images on a screen
to provide a virtual-reality experience, a simulation of speeding through tunnels or soaring over
skyscrapers.

The tunnels zoomed through and the buildings flown over represent Beijing in the year 2020, when the
Beijing City Master Plan is due to be completed. The movie is a featured part of a visit to the
high-tech Beijing Planning Exhibition Hall, which, as the references to Mr.~Kissinger and
Mr.~Paulson showed, has lately been on the itinerary for some of this country's most honored
visitors.

The main theme of the three-story Exhibition Hall is unspoken and yet obvious. It is the intention
of Chinese leaders to make this city of 20 million people a glittering, efficient, eco-friendly,
architecturally distinguished and even beautiful city, appropriate for the great capital of a rising
world power.

To someone like me, who comes here for a month or so every year, the exhibition had a touch of
propaganda about it, or, at least, of advertising. The exhibits are a bit reminiscent of the images
of socialist plenty that decades ago appeared in the official glossy magazines.

In this sense, the Exhibition Hall skips over some of the city's deficiencies, like its frequently
occurring, eye-burning pollution, its monumental traffic jams, or the way its very vastness makes it
generally unfriendly to pedestrians.

This is not an exhibition showing debates over urban philosophy; it's designed to celebrate the
Beijing City Master Plan, approved by the Chinese rubber-stamp Parliament a few years ago. There is
nothing, for example, on the role of the hundreds of thousands of migrant workers who have provided
the low-cost labor involved in the city's frantic pace of construction.

Still, the Exhibition Hall is a stunningly designed and informative museum, right in the center of
the city, just south of Tiananmen Square, next to the Victorian-era former central railroad station,
which has always been a touch of Old Europe in the heart of Old Beijing.

One small display provides a large demonstration of how things have changed here, and how the
attitude toward city planning has changed with it. On the ground floor, there is a row of bronze
sculptures of figures influential in the historical creation of Beijing.

There is Kuai Xiang, for example, who, legend has it, designed the massive Tiananmen Gate at the
entrance to the Forbidden City during the Ming Dynasty, some six hundred years ago.

Alongside the more ancient figures is a bronze bust of Liang Sicheng, who was the pioneering figure
in modern Chinese architecture, famous among other things for having pleaded with Mao Zedong in the
years after the Communists took power to preserve Beijing's old monuments, especially the Ming wall
that encircled the entire city of that time, and its numerous gates.

Mao didn't listen, and Beijing paid a heavy price in the destruction of the wall and much else, a
loss that is at least implicitly acknowledged in the small monument to Liang, who was savagely
persecuted by the Communist Party in the mid-1950s and again during the Cultural Revolution of
1966-76.

The centerpiece of the Exhibition Hall is a scale model of 1,302 square meters, or 14,000 square
feet, that shows in fantastic detail the entire city inside the present-day Fourth Ring Road. This
includes the Bird's Nest and the Ice Cube, main sites of the 2008 Olympic Games, as well as other
newly created signature structures, like the architect Rem Koolhaas's headquarters for China Central
Television and the ellipsoid titanium-and-glass National Center for the Performing Arts, designed by
the French architect Paul Andreu.

Some of these new buildings have stirred their share of controversy, unmentioned in the Exhibition
Hall -- Mr.~Koolhaas's building in particular.

It's called the Big Shorts by some Beijingers, and it does indeed resemble a massive pair of short
pants striding across the North China Plain. Given that it's the headquarters of a national
television network directly supervised by the Communist Party's Propaganda Department, it's also
known, in a touch of Orwellian satire, as the Ministry of Truth.

Still, these buildings, examples of world-class architecture, reflect what the Exhibition Hall
shows, which is the determination of the Chinese authorities to use a portion of this country's
newly earned wealth to transform their capital from the dilapidated, overgrown village it was a few
years ago into a sort of international, planned model city.

There are many ways in which the plan does not conform to current notions in the West about urban
spaces being designed on a human scale. The vast network of ring roads, the annihilation of old
neighborhoods, and the frantic construction of huge new shopping mall complexes all reflect the
out-of-scale monumentalism of the overall plan.

Beijing is not a city of casual strolls through neighborhoods that have grown organically, mixing
living, working, shopping and entertainment spaces. Still, the Dynamic Movie Hall shows a plan for
public transportation that makes New York seem veritably antiquated and sluggish by comparison.
There is already an excellent rapid transit link from the gleaming new international airport to the
center of town, something that New York has simply never managed.

Then there is the nationwide network of high-speed trains, which the United States also doesn't
have. In Beijing, the trains will connect with a network of 22 fast, efficient subway lines, nine of
which have been completed.

There are separate exhibits on local environmentalism as well, showing models for
low-carbon-footprint housing, plans to increase wind and solar power, designs for the recycling of
garbage and water and for the preservation of some old neighborhoods of courtyard houses, with their
elaborate attention to feng shui, the ancient Chinese way of ensuring harmony between man and
nature.

It's an impressive demonstration of the continuing transformation of a very old city into a new one.

\pagebreak
\section{On Clean Energy, China Skirts Rules}

\lettrine{U}{ntil}\mycalendar{Sept.'10}{10} very recently, Hunan Province was known mainly for
lip-searing spicy food, smoggy cities and destitute pig farmers. Mao was born in a village on the
outskirts of Changsha, the provincial capital here in south-central China.

Now, Changsha and two adjacent cities are emerging as a center of clean energy manufacturing. They
are churning out solar panels for the American and European markets, developing new equipment to
manufacture the panels and branching into turbines that generate electricity from wind. By contrast,
clean energy companies in the United States and Europe are struggling. Some have started cutting
jobs and moving operations to China in ventures with local partners.

The booming Chinese clean energy sector, now more than a million jobs strong, is quickly coming to
dominate the production of technologies essential to slowing global warming and other forms of air
pollution. Such technologies are needed to assure adequate energy as the world's population grows by
nearly a third, to nine billion people by the middle of the century, while oil and coal reserves
dwindle.

But much of China's clean energy success lies in aggressive government policies that help this
crucial export industry in ways most other governments do not. These measures risk breaking
international rules to which China and almost all other nations subscribe, according to some trade
experts interviewed by The New York Times.

A visit to one of Changsha's newest success stories offers an example of the government's methods.
Hunan Sunzone Optoelectronics, a two-year-old company, makes solar panels and ships close to 95
percent of them to Europe. Now it is opening sales offices in New York, Chicago and Los Angeles in
preparation for a push into the American market next February.

To help Sunzone, the municipal government transferred to the company 22 acres of valuable urban land
close to downtown at a bargain-basement price. That reduced the company's costs and greatly
increased its worth and attractiveness to investors.

Meanwhile, a state bank is preparing to lend to the company at a low interest rate, and the
provincial government is sweetening the deal by reimbursing the company for most of the interest
payments, to help Sunzone double its production capacity.

Heavily subsidized land and loans for an exporter like Sunzone are the rule, not the exception, for
clean energy businesses in Changsha and across China, Chinese executives said in interviews over the
last three months.

But this kind of help violates World Trade Organization rules banning virtually all subsidies to
exporters, and could be successfully challenged at the agency's tribunals in Geneva, said Charlene
Barshefsky, who was the United States trade representative during the second Clinton administration
and negotiated the terms of China's entry to the organization in 2001.

If the country with the subsidies fails to remove them, other countries can retaliate by imposing
steep tariffs on imports from that country. But multinational companies and trade associations in
the clean energy business, as in many other industries, have been wary of filing trade cases,
fearing Chinese officials' reputation for retaliating against joint ventures in their country and
potentially denying market access to any company that takes sides against China.

W.T.O. rules allow countries to subsidize goods and services in their home markets, as long as those
subsidies do not discriminate against imports. But the rules prohibit export subsidies, to prevent
governments from trying to help their companies gain in world markets.

The W.T.O. also requires countries to declare all national, state and local subsidies every two
years, so that if one country's exports surge suspiciously, other countries' trade officials can
easily check to see if that product is being subsidized.

But China has virtually ignored the requirement since joining the W.T.O. Contending that it is still
a developing country struggling to understand its commitments, China has filed just one list of
subsidies, which were in place between 2001 and 2004. And that one list covered only central
government policies while omitting local or provincial subsidies.

The Chinese mission to the W.T.O., which is part of China's commerce ministry, would not comment for
this article. After reading questions The New York Times submitted by fax last week, mission
officials declined to respond, saying that any comments might affect China's standing in other trade
disputes.

Sunzone and other Chinese clean energy companies also benefit from the fact that the government
spends \$1 billion a day intervening in the currency markets so that Chinese exports become more
affordable in foreign markets. Systematic intervention in currency markets to obtain an advantage in
trade violates the rules of the International Monetary Fund, of which China is a member, although
the I.M.F. has little power to punish violators.

Chinese wind and solar power manufacturers further benefit from the government's imposition of sharp
reductions this summer in exports of raw materials, known as rare earths, that are crucial for solar
panels and wind turbines. China mines almost all of the world's rare earths. W.T.O. rules ban most
export restrictions.

Of course, China's success in clean energy also stems from assets enjoyed by many of the nation's
industries: low labor costs, expanding universities that groom lots of engineering talent,
inexpensive construction and ever-improving transportation and telecommunications networks.

For example, engineers with freshly issued bachelor's degrees can be found here in Hunan Province
for a salary of only about \$2,640 a year -- not significantly more than blue-collar workers with
vocational school degrees can make. But the fuel propelling clean energy companies in China lies in
advantages provided by the government, executives say.

Other countries also try to help their clean energy industries, too, but not to the extent that
China does -- and not, so far at least, to the point of potentially running afoul of W.T.O. rules.

No doubt China's aggressive tactics are making clean energy more affordable. Solar panel prices have
dropped by nearly half in the last two years, and wind turbine prices have fallen by a quarter --
partly because of the global financial crisis but mainly because of China's rapid expansion in these
sectors and the accompanying economies of scale. Large Chinese wind turbines now sell for about
\$685,000 per megawatt of capacity, while Western wind turbines cost \$850,000 a megawatt.

The question is whether China is building this industry in ways that are unfair to overseas
competitors and make other nations overly dependent on a Chinese industry whose approach to the
business may not be economically or politically sustainable.

Because China's clean energy industry has relied so heavily on land deals and cheap state-supported
loans, the industry could be vulnerable if China's real estate bubble bursts, or if the banks' loose
lending creates financial problems of the sort that have plagued Western financial markets in recent
years.

Other countries may also become less enthusiastic about subsidizing renewable energy if it means
importing more goods from China instead of creating jobs at home.

The rapid rise of China's solar and wind industries illuminates how the government helps many export
industries, as well as the challenges for the West now that the country has emerged as the world's
second-largest economy, surpassing Japan and gradually gaining on the United States.

\textbf{Winning Big}

Barely a player in the solar industry five years ago, China is on track to produce more than half
the world's solar panels this year. More than 95 percent of them will be exported to countries like
the United States and Germany that offer generous subsidies for consumers who buy solar panels.

By contrast, the Chinese government has relatively modest solar subsidies for its citizens. Instead
it has devoted more money to helping manufacturers, allowing them to cash in on other countries'
consumer subsidy programs.

China is also on track to make nearly half of the world's wind turbines this year. China offers
financial incentives for utilities to use wind power, which is less costly than solar power, and the
country passed the United States last year as the world's largest wind turbine market.
Government-subsidized turbine makers are now preparing for large-scale exports to the United States
and Europe, which could also result in violations of W.T.O. rules.

Meanwhile, China itself imports virtually no wind turbines or solar panels, instead protecting those
developing industries. For example, China until late last year required that 70 percent of the
content of each wind turbine and 80 percent of the content of each solar panel be made within China.
China quietly dropped that rule after objections from American officials, but also because its own
industries had become the world's largest, lowest-cost producers.

Now China strongly opposes suggestions in Congress that the United States or Europe follow China's
example and impose ``local content'' rules to help their own struggling renewable energy industries.

``Now if the U.S.~sets up that kind of regulation, it will really be a problem'' said Li Junfeng, a
senior Chinese energy policy maker. ``We need to buy from each other.''

China's expansion has been traumatic for American and European solar power manufacturers, and
Western wind turbine makers are now bracing to compete with low-cost Chinese exports. This year, BP
shut down its solar panel manufacturing in Frederick, Md., and in Spain, and laid off most of the
employees while expanding a joint venture in China.

Evergreen Solar of Marlboro, Mass., plans to move the final manufacturing steps for its solar panels
from Devens, Mass., to China next summer, eliminating 300 American jobs, after struggling to borrow
money in the United States and after finding that costs in China were lower.

The Obama administration has begun high-level discussions on how to respond to China's industrial
policies, Treasury Secretary Timothy F.~Geithner said in an interview in Washington in July.

``We are concerned about the depth and breadth of the measures they have taken,'' Mr.~Geithner said,
later adding, ``We will be aggressive on the trade front in terms of fighting anything that is
clearly discriminatory.''

\textbf{Helping Hand}

Here in Changsha, Sunzone's general manager and chief engineer, Zhao Feng, represents a new breed of
Chinese clean energy entrepreneurs. Tall and fit, he is an avid painter, fisherman and golfer.

``If I go to Los Angeles for 10 days, I am on a golf course for eight days,'' he said.

A former professor of semiconductors at Hunan University, he has a daughter studying for a doctorate
in bioengineering at the University of Chicago on a Pentagon grant, and he owns a house in Chicago a
block from President Obama's.

Mr.~Zhao is quick to point out that state and federal governments in the United States have also
encouraged the development of the clean energy industry. ``Our provincial governor has come several
times to our plant, just as Gov.~Arnold Schwarzenegger has made several visits to solar power
companies'' in California, he said.

But the Hunan government's backing of Sunzone is much more extensive than anything in the United
States.

With government help, Sunzone lined up financing and received all the permits necessary to build a
factory in just three months under an expedited approval system for clean energy businesses. It took
only eight more months to build and equip the factory. ``The construction teams worked 24 hours a
day, seven days a week in three shifts,'' Mr.~Zhao said.

Building and equipping a solar panel factory in the United States takes 14 to 16 months, and getting
environmental and other permits can take years, said Tom Zarrella, the former chief executive of GT
Solar in Merrimack, N.H., a big supplier of solar manufacturing equipment to factories in the United
States and China.

A strong symbol of the government's commitment to the clean energy industry in China may be
Sunzone's walled 22-acre compound here.

The company has only 360 employees, who work in a modest two-story building and small factory. Many
of them live in a six-story dormitory. The compound also has a demonstration house powered by solar
panels.

But the government has granted Sunzone enough cheap land to make room for an orchard of orange
trees, a nearly finished golf driving range and winding country lanes -- all of it across the street
from 17-story apartment buildings near the heart of downtown Changsha. A lone trellis-covered swing
that sits on Sunzone's vast plot seems to signal how little occupied the land is.

As a clean energy business, Sunzone was allowed to buy the land two years ago for \$90,000 an acre,
Mr.~Zhao said. That was one-third of the official price then for industrial land from the
government.

Industrial land in this desirable neighborhood now sells for \$720,000 an acre, giving Sunzone an
eightfold profit on paper. The company carries the land on its books at this market price, and can
borrow against it, Mr.~Zhao said. The valuable land also means the company has big assets and little
debt on its balance sheet, which should help attract investors for a planned initial public offering
in 2012.

Executives at three other clean energy companies in and around Changsha said they, too, had been
allowed to buy government land for a third of the regulated price.

Mr.~Zhao defended the size of his corporate park as necessary for his business and said it was not a
real estate investment. The driving range will be made available to all employees for their
relaxation, he said. And he said Sunzone hoped to build a nine-story solar research center on part
of its land someday.

The local government of Zhuzhou, a city near Changsha, is even more generous. ``For really good
projects, we can give them the land for free,'' said He Jianbo, the deputy director of the city's
flourishing high-tech zone, which already makes everything from electric buses to solar panels, and
is preparing to build electric cars. ``This land subsidy is not available to traditional industries,
only high-tech industries.''

Many state and local governments in the United States have also built roads, installed power lines
and made other infrastructure improvements that have increased the value of private land as part of
programs to attract clean energy. Tax holidays for such businesses are common in the United States,
as in China.

But according to Commerce Department experts in Washington, government agencies in the United States
have generally refrained from the sale of deeply discounted government land to export industries,
while infrastructure improvements have been made to benefit all road and telecommunications users,
not just specific export industries. A wide range of international trade agreements, including
W.T.O. rules, allow governments to provide infrastructure and some types of tax breaks, but bar
subsidies in the form of cheap transfers of valuable government assets like land to exporters.

Mr.~Zhao said that whatever the global trade rules might be on export subsidies, the world should
appreciate the generous assistance of Chinese government agencies to the country's clean energy
industries. That support has made possible a sharp drop in the price of renewable energy and has
helped humanity address global warming, he said.

The subsidized land will also help Sunzone afford plans to sell solar panels below cost to poor
people in western China, Mr.~Zhao said, adding that he hoped the effort would build good will and
lead to more sales there.

\textbf{Money for Nothing}

As Sunzone prepares to double its manufacturing capacity by the end of this year, state banks and
the municipal government are ready to help.

The company has reached a tentative deal to borrow \$11 million, to increase employment to 600
workers. The bank will lend the money at an interest rate of about 6 percent, but the provincial
government will then give Sunzone a direct rebate to pay more than half the interest on the loan.
``Just yesterday, the bank general manager brought his staff here to see how they could be of
service to us,'' Mr.~Zhao said. ``We don't need to go to the bank; they come here.''

Low-interest loans from government-run banks are crucial to China's clean energy success, some
experts say, because of the high cost of factory equipment.

``If you change the interest rate half a percent or 1 percent, the difference is amazing, because
the cost is all at the beginning,'' said Dennis Bracy, chief executive of the U.S.-China Clean
Energy Forum, a discussion group of Chinese energy officials and former American cabinet officials.

In the United States, the Obama administration has approved \$10 billion in grants, loan guarantees
and other financing to help new entrants in the industry and aid existing companies, and has
promised another \$10 billion, said Matt Rogers, the senior adviser to Energy Secretary Steven Chu
for economic stimulus programs. Almost all the money is for projects that will generate electricity
for the United States.

But the American clean energy programs carry many time-consuming and difficult requirements.
Companies must show they can repay loans and have innovative technology. The department has given
conditional approval to 18 renewable energy loan guarantees, although only four have led to the
actual issuance of loans so far. But the administration is moving quickly to accelerate the process,
said Jonathan Silver, the executive director of the Energy Department's loan guarantee program.

China has been pumping loans into clean energy so rapidly that even \$23 billion in credit offered
by the China Development Bank to three solar panel exporters and a wind turbine maker since April
has barely raised eyebrows. China Development Bank, owned by the government, exists to lend money
for strategic priorities.

Western clean energy companies complain of much higher financing costs -- when they can raise money
at all. Banks have been cautious about the sector, which leans heavily on venture capitalists and
private equity firms that demand implicit interest rates of up to 9 percent right now in the United
States, said Thomas Maslin, a senior solar analyst at IHS Emerging Energy Research.

Evergreen Solar, the Massachusetts company, struggled for three years to raise money in the States,
but had no trouble doing so in China. Chinese state banks were happy to lend most of the money for
the factory on very attractive terms, like a five-year loan with no payments of interest or
principal until the end of the loan, said Michael El-Hillow, the company's chief financial officer.

``You can't get a penny in the United States, it doesn't matter who you call -- banks, government.
It's awful,'' he said. ``Therein lies the hidden advantage of being in China.''

Many Chinese clean energy executives argue that China should offer more subsidies for its own people
to buy renewable energy, in addition to helping export-oriented manufacturers. But until domestic
demand takes off, government support will remain crucial.

``Who wins this clean energy race,'' Mr.~Zhao of Sunzone said, ``really depends on how much support
the government gives.''

\pagebreak
\section{China Begins Aviation Inquiry After Finding Fake Pilot R\'esum\'es}

\lettrine{T}{he}\mycalendar{Sept.'10}{10} Chinese government is conducting a broad investigation
into the qualifications of the country's pilots, aircraft mechanics, flight trainers and other
aviation personnel after finding that as many as 200 commercial pilots may have falsified their
r\'esum\'es in 2008 and 2009, according to reports in the Chinese news media on Monday and Tuesday.

The investigation started even before the deadly Aug.~24 crash of a Henan Airlines plane that came
down well short of the runway while attempting a night landing in Yichun, a town in northeast
Heilongjiang Province, according to Chinese news media. Regulators at the Civil Aviation
Administration of China could not be reached for comment on Tuesday morning.

The Henan Airlines crash, of an Embraer 190 made in Brazil, killed 42 and injured 54. The crash has
caused such a scandal in China that although it did not occur in Henan Province, the authorities
have ordered the airline to change its name, so as to limit harm to the province's image.

News reports indicated that slightly more than half of the pilots found to have given fake
qualifications on their r\'esum\'es worked for Henan Airlines' fast-growing parent, Shenzhen
Airlines. Officials at Shenzhen Airlines could not be reached for comment.

Peter Harbison, the chairman of the Center for Asia Pacific Aviation in Sydney, said that except for
the Henan Airlines crash last month, China has had a strong reputation for aviation safety in recent
years. ``They certainly do actually, which is not what I would have said five or six years ago,'' he
said.

China's top leaders have given aviation regulators a clear mandate to make safety their top priority
and told the chief executives of the nation's airlines that they would be held personally
responsible for any crashes, Mr.~Harbison said.

The tendency for Chinese carriers to err on the side of safety is sometimes visible at Hong Kong
International Airport. During storms, the Chinese planes tend to be a little quicker to divert to
other airports while the planes of international carriers continue to land.


% \end{multicols}

% \clearpage
% \renewcommand\listfigurename{\textit{Table of Figures}}
% {\footnotesize\textit{\listoffigures}}

\end{document}
