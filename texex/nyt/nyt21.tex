\documentclass[12pt]{article}
\title{Digest of The New York Times}
\author{The New York Times}

\usepackage{config}

\makeindex
\begin{document}
\date{}
\thispagestyle{empty}

\begin{figure}
\includegraphics*[width=0.7\textwidth]{The_New_York_Times_logo.png}
\vspace{-15ex}
\end{figure}
% \renewcommand\contentsname{\textsf{Digest of The New York Times}}
\renewcommand\contentsname{}
{\footnotesize\textsf{\tableofcontents}}

\clearpage
\setcounter{page}{1}

\section{U.N. Reports Rising Afghan Casualties}

\lettrine{T}{he}\mycalendar{Aug.'10}{11} number of civilian casualties in Afghanistan continued to
climb in the first half of 2010, with an increasing number of children in the toll and a spike in
the recently troubled northeast. More than ever, the deaths were caused by insurgents, the United
Nations said in a report released Tuesday.

In its midyear report, the United Nations Assistance Mission in Afghanistan, known as Unama, said
the number of civilians wounded and killed increased by nearly a third in the first six months of
the year, as coalition forces raised the level of military action against insurgents.

In that period, 1,271 civilians were killed and 1,997 were wounded, the report said, with more than
three-quarters attributable to what it called ``antigovernment elements.''

Death and injury to children were up 55 percent, with 176 killed and 389 wounded, the report said,
noting that improvised bombs were often placed in areas frequented by the young, like parks and
markets.

The single biggest cause of the increase in civilian casualties was insurgent bombings, including
both suicide bombings and homemade bombs, which the military calls improvised explosive devices.
Together they caused 557 deaths.

``This is a wake-up call for us,'' the top United Nations official in Kabul, Staffan de Mistura,
said at a news conference. ``By looking at the figures, we suddenly have a trend of increase which
we have the duty to raise publicly, in particularly with those who are causing these deaths.''

Since 2009, when the United States military made it a high priority to reduce civilian casualties,
the trend has been for a far lower percentage of them to be caused by the military, and far more by
the Taliban and other insurgents.

In 2007, less than half of the civilian casualties were caused by the insurgents, according to
United Nations statistics. The new figure, an increase of 53 percent over the same period last year,
was the most significant change to date.

``Nine years into the conflict, measures to protect Afghan civilians effectively and to minimize the
impact of the conflict on basic human rights are more urgent than ever,'' said Georgette Gagnon,
human rights director for Unama.

Mr.~de Mistura, the ranking United Nations official in Afghanistan, had harsh criticism for the
insurgents' conduct, noting their widespread and increased use of indiscriminate weapons like
roadside bombs in civilian areas, and their tendency to fight from civilian cover.

``People who are part of this conflict should not be using human shields, should not be fighting
from where civilians are,'' he said.

Over all, civilian casualties caused by government or coalition forces declined by 30 percent for
the period. Deaths of civilians from NATO aerial bombings, once the leading cause of such
casualties, were down 64 percent over the same period in 2009, for a total of 69 civilian deaths,
the United Nations said.

The report ascribed the decrease to an order in July 2009 from Gen.~Stanley A.~McChrystal, the
United States commander at the time, which greatly limited the use of airstrikes where there was a
risk of civilian casualties.

The overall 31 percent increase in civilian casualties was also attributable to an increase in
military operations, particularly in southern and southeastern Afghanistan, the report said, as
larger numbers of NATO forces have poured into the country and military operations have increased.

In the northeastern part of the country, until recently relatively quiet, the increase in civilian
deaths in the first half of the year was 136 percent over the same period in 2009.

The report also noted an increased use of intimidation and assassination of the civilian population
by the Taliban, singling out anyone ``perceived to be'' connected with the government or
international forces.

In 2009, such assassinations averaged 3.6 per week, increasing to 7 per week in the first four
months of 2010, then increasing to 18 per week in May and June of 2010, the report said.

A statement on Tuesday from NATO's international force welcomed the report's findings, but added a
comment from the new commander, Gen.~David H.~Petraeus, taken from his recent tactical directive to
coalition soldiers.

``Every Afghan death diminishes our cause,'' General Petraeus said. ``While we have made progress in
our efforts to reduce coalition-caused civilian casualties, we know the measure by which our mission
will be judged is protecting the population from harm by either side.''

\section{Chinese Investor Is Said to Be Bidding for English Soccer Club}

\lettrine{W}{hen}\mycalendar{Aug.'10}{11} reports began circulating last week that a Chinese
investor was bidding to take over the Liverpool soccer club in the English Premier League, British
tabloids quickly called him King Kenny.

That little-known investor is Kenny Huang, 46, a globe-trotting sports enthusiast who has made
marketing deals in China with the Cleveland Cavaliers and the Yankees, and has entered into a
business partnership with Leslie Alexander, the owner of the Houston Rockets.

In February, he helped the Yankees tour China with their World Series trophy, trying to drum up
interest in Major League Baseball. He also advised Chinese investors bidding to acquire a 15 percent
stake in the Cavaliers, although the deal did not work out. And in an interview this year, he said
that he would consider acquiring an N.F.L. team.

``If there's an opportunity, we'll do it, but only for the learning process,'' Huang said during the
interview in Shanghai in March. ``The N.F.L. is the most profitable league. The management, the
intellectual property, the ownership structure is the best.''

Friends say Huang made a fortune working on Wall Street and helping clients invest in Chinese
initial public stock offerings. Now, they say, he is pushing China to overhaul its sports system and
develop its own professional sports leagues with corporate sponsors -- a challenge in a country
where the government retains tight control of sports programs.

But his interests are not confined to China. In what could be one of the biggest sports club deals
in history, Huang is said to be considering acquiring the Liverpool club from its owners, Tom Hicks
and George Gillett of the United States, who bought the team in 2007. The English Premier League is
the wealthiest soccer league in the world.

Such a takeover -- which Huang hinted at in a news release last week -- could cost as much as \$600
million, analysts say, including the assumption of debt.

Liverpool said last week that there were six bidders for the club but declined further comment.
Investors from India and the Middle East are reportedly among those who have shown interest.

The planned bidding for Liverpool underscores the growing interest of foreign buyers in professional
sports teams in Europe and the United States. The Manchester City soccer club in England, for
example, is owned by the sovereign wealth fund of Abu Dhabi, and in New Jersey, the Nets are
controlled by the Russian billionaire Mikhail D.~Prokhorov.

Huang, a Chinese-born United States citizen, declined to be interviewed about the potential
Liverpool bid. The Times of London wrote last week that China's sovereign wealth fund was financing
the bid. But a spokesman for that fund, China Investment Corp., told The Financial Times that the
story was preposterous.

That leaves the spotlight on Huang, who colleagues say has close ties to the Chinese government but
also deep-pocketed admirers in the United States.

``Kenny's a very talented guy, and he's very knowledgeable about sports,'' said Red McCombs, a
former owner of the Denver Nuggets and the San Antonio Spurs, and an occasional investor with Huang
in China. ``He's a unique dealmaker. I'll bet on him any day.''

His background, though, is as complex as it is impressive. In the interview in March, Huang -- whose
Chinese name is Huang Jianhua -- said that he grew up in the southern Chinese city of Guangzhou and
that as a 9-year-old, he entered a state-run sports school as a badminton player.

Later, he said, he studied Japanese at Zhongshan University in Guangzhou and then in 1985 moved to
New York, where he earned a master's degree at St.~John's and took courses at Columbia University
and New York University. He is married with two children, he said.

He also spent several years in the 1980s and early 1990s at the New York Stock Exchange working with
Asian stock listings, he said. Then he found more lucrative work.

``After '91, the Japanese economy collapsed,'' Huang recalled of his time at the stock exchange,
saying his Japanese was excellent. ``The billionaires were in trouble. They said, 'Why don't you
quit and become our personal investment adviser?' I was helping them buy and sell assets.''

Court records in the United States show he lived in New York and southern Florida and in the 1990s
formed a series of small companies, including one that sold ``Chinese-made novelty products.'' It
eventually went out of business.

The Web site of another company he helped form, Aspen Infrastructure Investment, lists one of the
partners as George Lindemann, whom Forbes magazine ranks as one of the world's richest men.
Lindemann could not be reached for comment this week.

(Huang was sued by one of his former United States partners in a dispute over \$2.9 million, but
earlier this year, a Florida jury ruled in his favor.)

About eight years ago, he met the consultant Marc Ganis, now his partner at SportsCorp China, a
sports business firm.

Ganis said that years ago, Huang brought Chinese officials planning the Beijing Olympics to visit
sports facilities in the United States. He said Huang's business and government ties and his
knowledge of how to create successful deals in China allowed him to create breakthrough sports
initiatives in the country.

``I've introduced Kenny to some of the most important people in the sports industry in the U.S.,''
Ganis said. ``And they immediately recognize that as China develops, he's going to be able to grow
with it. He has a unique ability to bridge East and West.''

One of his ventures is Rocket Capital, formed with Alexander of the Rockets. The venture and others
have invested more than \$200 million in Chinese I.P.O.'s in Hong Kong over the past five years,
according to securities filings.

Huang has also worked on sponsorship deals in China with the Rockets, the Cavaliers and the Yankees.
Just last week, his company QSL Sports said that a Chinese company had signed a four-year
sponsorship deal with the Yankees in the instant noodle sector.

Now, Huang said, he is working to duplicate Little League baseball and other youth sports leagues in
China. He has already invested in a professional baseball league and a pro basketball league, the
National Basketball League, which he intends to revamp so that it can compete with the much bigger
and more heavily financed Chinese Basketball Association.

Analysts say the Chinese have a growing interest in professional sports teams in the United States
and Europe. Nike and Adidas are investing heavily in expansion here, and China's own sports brands,
like Li Ning and Anta (which Huang invested in and which recently signed Kevin Garnett of the Boston
Celtics basketball team to a sneaker deal) are also thriving.

``I've seen the changes in the post-Olympic period,'' he said in the interview as he lounged on a
sofa at the Westin Hotel in Shanghai, dressed in a track suit and sneakers. ``You now have the
opportunity to invest in sports. The government wants sports to become an industry.''

For Huang, the change in Chinese government policy is drastically altering the sports industry here,
opening new opportunities for sports clubs and marketers. And that is one reason the Yankees are
also looking to Huang.

``He's our man in China and he was incredibly important in creating our connections with the Chinese
Baseball League and getting our first Chinese sponsor,'' said Randy Levine, the president of the
Yankees. ``Basically, he makes sure that the Yankee brand is known in China and he brings
sponsorship opportunities to us.''

\section{South Korean Police Raid Google Office}

\lettrine{T}{he}\mycalendar{Aug.'10}{11} South Korean police raided the offices of Google Korea on
Tuesday as part of an investigation into whether the company had illegally collected and stored
personal wireless data.

The U.S.~search titan is already facing lawsuits and investigations in several countries in
connection with private wireless data collected for its Street View service. Street View, which was
started in 2006, allows users to view panoramic street scenes on Google Maps and take virtual walks
through cities.

From late last year until May, Google Korea dispatched cars topped with cameras to cruise around the
country to photograph neighborhoods before the planned introduction of Street View there this year.

The police suspect that those cars might have illegally captured and stored personal data from
wireless networks while they were mapping streets, the Cyber Terror Response Center of the Korean
National Police Agency said in a statement.

``We will investigate Google Korea officials and scrutinize the data we confiscated today'' to see
whether the company has violated the country's laws on communications and privacy, it said.

``We intend to find out what kinds of data they have collected and how much,'' the response center
said. ``We will try to retrieve all the original data illegally collected and stored through
domestic Wi-Fi networks from the Google headquarters.''

Google, based in Mountain View, California, said, ``We will cooperate with the investigation and
answer any questions they have,'' Bloomberg News reported.

Google said previously that the collection of personal data in other countries had been
unintentional and the company would cooperate with investigations.

Google has had a hard time in South Korea. One domestic search engine, Daum, already runs a popular
service akin to Street View.

Google said Tuesday that it would introduce Street View in Germany before the end of the year, The
Associated Press reported from Berlin.

Google said the feature would be available for the 20 biggest German cities and people could ask to
have the photographs of their houses removed from the database starting next week -- an offer aimed
at dispelling privacy fears.

\section{Obama-Backed Senator Prevails in Colorado Race}

\lettrine{S}{enator}\mycalendar{Aug.'10}{11} Michael Bennet of Colorado, a Democrat who had hitched
his star to the fortunes of President Obama, survived a bitter primary challenge on Tuesday,
suggesting that at least here in this Western swing state, the much touted wrath of the American
voter may not, in the end, be quite ready sweep away all before it.

Mr.~Bennet, who was appointed to the Senate last year to fill a seat vacated by Ken Salazar, now
Mr.~Obama's interior secretary, defeated Andrew Romanoff, a former speaker of the Colorado House.
Mr.~Romanoff conceded about an hour after the polls closed with what a spokesman for Mr.~Bennet
described as ``a very nice call.''

His victory suggests that for all the anger at establishment politicians and officeholders this
year, Democrats and President Obama still retain some political clout in swing states that sided
with the party in the 2008 presidential election.

In a surprising upset in Connecticut, Dannel P.~Malloy, a former mayor of Stamford, Connecticut's
fourth-largest city, defeated Ned Lamont, a multimillionaire businessman who tried unsuccessfully to
oust Senator Joseph I.~Lieberman four years ago, to win the Democratic nomination for governor.

In Colorado, the Republican primary was too close to call, as was the Republican runoff for governor
in Georgia. Karen Handel, a former Georgia secretary of state, was in a tight race late into Tuesday
night. She was endorsed by Sarah Palin, the former Alaska governor turned pundit. The other
candidate, Nathan Deal, a former congressman, received endorsement nods from both Mike Huckabee, the
former governor of Arkansas, and Newt Gingrich, the former speaker of the House. Mr.~Huckabee,
Mr.~Gingrich and Ms.~Palin are all presumed presidential hopefuls in 2012.

At Mile High Station, an events center in Denver, backers of Mr.~Bennett began arriving as a sound
system played ``I'm a Believer'' by the Monkees.

``I've been a supporter since way back,'' gushed Susan Clemens, a retired school teacher. ``This is
a real man. With all his jobs, he's been hired to reform and that's what he's done. But change for
some people is hard to take.''

When asked if she'd support Mr.~Romanoff if he got the nomination, Ms.~Clemens said, ``I would, yes.
But tonight's probably not the night to ask.''

As results began coming in, showing Mr.~Bennet mounting a sturdy lead over his opponent, supporters
of Mr.~Bennet began cheering wildly.

As news of Mr.~Benet's victory began crackling through the crowd, supporters began flowing into the
events center. ``I'm just elated,'' said Eric Duran, a supporter. ``This election was hope against
anger. They tried to portray Michael as the consumate insider when he was really a political
novice.'' .

Mr.~Duran said the election had provoked a lot of ``nasty emotions'' in Denver during the final
stretch.

Many voters in Colorado voted before the actual primary day. State elections officials said the
turnout in Colorado before Tuesday had surged through mail-in balloting. Of the state's 64 counties,
46 conducted elections entirely by mail under a law passed by the state legislature last year aiming
to cut county costs in running elections.

As of Monday, nearly 598,000 votes had already arrived in the offices of county clerks and recorders
around the state. That compares with about 488,000 votes cast in the primary election in 2008 by the
time all the voting was done, and only 397,000 in 2006, according to figures from the Colorado
Secretary of State.

In Minnesota, three people with sharply divergent backgrounds -- and degrees of financial wealth to
back their bids -- were running for the Democratic nomination for governor. The statehouse speaker,
Margaret Anderson Kelliher, who would be the first woman to gain a major party nomination for
governor there, had promised during the campaign to create an ambitious jobs creation package.

Ms.~Kelliher used a compelling personal story -- local farm girl makes good in state politics --
against the deeper financial pockets of her competitors: Matt Entenza, a St.~Paul attorney who spent
than \$5 million of his own money, and a former United States Senator, Mark Dayton, also a
millionaire, who pledged to raise taxes on his own economic class in the state, the wealthy.

\section{Obama's Youthful Voters More Likely to Skip Midterms}

\lettrine{W}{ill}\mycalendar{Aug.'10}{11} all of those young, enthusiastic Obama voters turn out in
2010? If history is any guide, probably not. Older voters are historically more likely to cast
ballots in midterm elections than are voters under the age of 30. And this year, they are already
more enthusiastic than younger voters about the coming campaign.

Those older voters are most likely to say the country is on the wrong track and to disapprove of the
way both Congress and President Obama are doing their jobs, according to a New York Times/CBS News
poll conducted this summer.

Eight in 10 Americans 45 and older disapprove of the job Congress is doing compared with 6 in 10 of
those under age 45. While opinions about Congress differ depending on age, anti-incumbent sentiment
cuts across generational lines, with about 8 in 10 Americans saying it is time to give new people a
chance to serve.

A CNN poll conducted nationwide in mid-July found older voters were significantly more enthusiastic
about voting this year than younger voters. Four in 10 of those aged 65 and older said they were
extremely or very enthusiastic about voting in November while just one-quarter of those under 35
years of age said the same.

Still, while 45 percent of adults surveyed nationwide by the Pew Research Center in mid-June
reported that they were more enthusiastic than usual about voting in this year's election, 36
percent said they were less enthusiastic. Both younger and older voters held similar views about
voting this year.

This is not terribly surprising. A look at historical voting data shows that on the whole, the
public is generally always less excited about voting in midterm elections. Voter turnout in midterm
elections has consistently hovered around 50 percent for the past three decades, according to the
Census Bureau's Current Population Survey. In presidential election years, turnout is significantly
higher, ranging from a low of 58 percent (in 1996) to a high of 68 percent (in 1992) among citizens
18 years of age and older. Turnout in 2008 was 64 percent.

Although presidential contests in the recent past have seen massive voter mobilization efforts,
these programs are not as visible in off years. And this has had a noticeable effect at the voting
booth.

Rock the Vote, aimed at getting out the youth vote, is celebrating its 20th anniversary this year.
The group acknowledges in promotional material that midterm turnout is historically lower, but it is
nonetheless ramping up efforts and has set a goal to register at least 200,000 young voters this
year (four times its 2006 registration).

In contrast, older voters do not need much prodding. Nancy LeaMond, AARP's executive vice president
for social impact, said: ``We have not concentrated specifically on convincing our members that it's
important to vote. We don't have to. That's a core value.''

History shows that a jump in turnout among young people would be unprecedented in an off-year
election. Exit poll data shows that the electorate skews older, particularly in nonpresidential
election years. In 2006, 63 percent of those who cast ballots were 45 or older, and in 2008, that
same age group made up 53 percent of the electorate, according to exit poll data from the National
Election Pool. This group made up about half of the adult population in those years.

``Habitual voters will show up for every election, and the sort of people that are habitual voters
are what political scientists consider to be higher socioeconomic status -- they are more educated
and they also tend to be older,'' said Michael McDonald, a professor of government at George Mason
University and an expert in voter statistics.

In the summer of 1994, Congressional approval was almost as low as it is now, and President Bill
Clinton's approval rating was in the mid-40s, much like President Obama's today. While the nation's
economy was weak, it was not as fragile as it is today. On Election Day 1994, when Democrats lost
control of Congress, only 13 percent of those who voted were under the age of 30.

\section{A Heart Pump Ticks Down, and a Stranger Steps In to Help}

\lettrine{C}{hristian}\mycalendar{Aug.'10}{11} Volpe was shopping with his wife when an alarm
started beeping to warn that only 15 minutes of battery power were left on the implanted heart pump
that was keeping him alive.

Mr.~Volpe, 67, a slight, gray-haired man, looked in his car for the bag he always keeps nearby with
spare batteries. But, no bag. In his mind's eye he saw exactly where he had left it, to make sure he
would not forget it, on a chair near the door back home -- an hour and a half away. He thought of
the clever little hand pump he had been given to keep his mechanical heart going in an emergency.
It, too, was in the missing bag. Standing in the parking lot, he could hear one thing. Beep. Beep.
Beep.

``I have to admit, I panic,'' he said.

Mr.~Volpe is one of thousands of Americans who have had these pumps, called left ventricular assist
devices, surgically implanted to help their failing hearts. Former Vice President Dick Cheney is
another. Sometimes the pumps are used to keep people alive until a transplant becomes available, but
in other cases they are meant to remain as long as the patient lives.

Mr.~Volpe, a retired subway conductor who had had two heart attacks and two bypass operations, had
an assist pump implanted in October 2009 by Dr.~Yoshifumi Naka at NewYork-Presbyterian/Columbia
hospital.

The pump is placed near the patient's own heart. A power line emerges about waist level and connects
to a controller, a mini-computer which plugs into a pair of one-and-a-half-pound, 12-volt batteries.
Patients wear a black mesh vest over their clothing that holds the controller and batteries. The
pump Mr.~Volpe had, a HeartMate XVE, made by Thoratec, could run for about four hours on two
batteries. The pumps cost \$70,000 to \$80,000, usually covered by insurance.

That day in the parking lot in December, in Fishkill, in upstate New York, Mr.~Volpe was too far
from Columbia to get there in time. But his wife phoned its heart-pump clinic, and nurse
practitioners told her to call 911 for an ambulance to the nearest hospital.

Mr.~Volpe knew that if the pump stopped, he was not likely to die immediately; his own heart, though
weak, would probably keep him alive. But he was still in real danger, because clots would form in
the mechanical heart if it quit, and cause a stroke if they escaped into his bloodstream.

Dr.~Donna Mancini, Mr.~Volpe's cardiologist and director of the heart failure and transplant program
at NewYork-Presbyterian/Columbia, said the hospital had not encountered a situation like this
before.

``But with these devices getting more use, it may arise,'' Dr.~Mancini said.

Right now, Dr.~Mancini said, Columbia has 45 patients with pumps who are waiting for transplants.
Just a few years ago, there were only 10.

She said she did not know why, but this year fewer donor hearts have become available than in the
past, leaving more patients dependent on the pumps. Usually, the hospital performs 80 to 100
transplants a year.

``This year we're on a course that will probably yield around 60 transplants,'' Dr.~Mancini said,
adding that there were about 150 patients on Columbia's waiting list.

Nationwide, 3,138 people are waiting for heart transplants, according to the United Network for
Organ Sharing. Last year, 2,211 received new hearts.

Thoratec said that in the past decade or so, a total of 6,000 XVE devices and 5,000 of a newer
model, the HeartMate II (the one Mr.~Cheney has) had been implanted.

An ambulance took Mr.~Volpe to Vassar Brothers Medical Center in Poughkeepsie. But that hospital
does not implant assist pumps, and had no batteries or hand pump. Doctors there, advised by
Columbia, began dripping in a blood-thinning drug, heparin, to prevent clots.

Meanwhile, Khristine Orlanes, a nurse practitioner at Columbia, began trying to find another patient
with an assist pump who was close enough to bring Mr.~Volpe a set of batteries in time.

She called Robert Bump, 61, a building contractor who worked near Poughkeepsie. He had six spare
batteries in a knapsack.

``I'm on my way,'' Mr.~Bump said.

An electrician offered to drive, and they tore off in his pickup truck. The electrician called a
state trooper friend, told him the story and said, ``We're not stopping.''

A police car met them partway to Poughkeepsie and escorted them. They made the half-hour trip in
about 20 minutes.

Mr.~Bump strode into the emergency room and spotted Mr.~Volpe on a gurney, surrounded by doctors,
nurses and his frantic wife. The alarm was still beeping. A doctor, noticing Mr.~Bump's black-mesh
vest and the controller, said, ``Oh, he's got one, too.''

Mr.~Volpe, who had no idea what plans had been hatched on his behalf, said: ``I see this big fellow
walk in. I recognized the outfit right away.''

Mr.~Bump snapped the batteries in place and said, ``O.K., you're good.''

There was a small round of applause in the emergency room. Mr.~Volpe could not stop saying thank
you.

His pump, due to run out in 15 minutes, had somehow lasted nearly an hour, but apparently had just
minutes left when Mr.~Bump arrived.

The two men had different pump models that happened to use the same batteries. If Mr.~Bump had been
using a newer version of the batteries for his model, they would not have been compatible with
Mr.~Volpe's.

``Mr.~Volpe's stars were aligned that day,'' Mr.~Bump said. ``There is some reason that gentleman
needs to be here.''

On July 24, after nearly a year on his assist pump, Mr.~Bump made it to the top of the waiting list
and received a transplant at NewYork-Presbyterian/Columbia. At the hospital, his wife overheard the
spouse of another transplant patient say that she, too, was from upstate.

Mr.~Bump's wife mentioned that her husband had helped another patient from the same area who needed
batteries for an assist pump.

``That was my husband,'' the other woman said.

By coincidence, Mr.~Volpe had also just received a transplant.

Last week, the two were up and about, in good spirits. Both said they owed their lives to the assist
pumps -- but were thrilled to be free of them. Both were desperate for showers, after nearly a year
of sponge baths. They would not miss the vests, either. On more than one occasion, Mr.~Bump had had
to reassure strangers that he was not wearing a bomb.

Leaving a sitting room at the hospital last Thursday, Mr.~Bump rose first and offered Mr.~Volpe a
hand getting up.

``No thanks,'' Mr.~Volpe said softly, smiling. ``You gave me enough help, Robert.''

\section{Win Wars? Today's Generals Must Also Politick and Do P.R.}

\lettrine{A}{fter}\mycalendar{Aug.'10}{13} nine years of fighting in the deserts and mountains of
the Middle East, the military has concluded that the traditional, hard-earned combat skills that
allowed generations of ``muddy boots'' commanders to protect American interests around the world
simply are not enough to assure victory in today's wars -- or career advancement through the top
ranks of the armed forces.

Mastery of battlefield tactics and a knack for leadership are only prerequisites. Generals and other
top officers are now expected to be city managers, cultural ambassadors, public relations whizzes
and politicians as they deal with multiple missions and constituencies in the war zone, in allied
capitals -- and at home.

The increased demands help to explain how the two most recent American commanders in Afghanistan,
among the most respected four-star officers of their generation, lost their jobs. And they are
prompting the military to revamp the way it trains and promotes its top officers.

``They must be `pentathlete' leaders,'' said Gen.~David H.~Petraeus, the senior commander in
Afghanistan. As Iraq and Afghanistan have proved that a commander must stretch to master nuances of
international alliance accord, local governance and tribal politicking, the military has revamped
its training ranges and its curriculum.

Strong scores in mock battle in the deserts of California or in swampy Louisiana are no longer the
lone measurement. Fake villages with irascible, faux tribal leaders and proxies representing the
competing agendas of government agencies and nongovernment organizations are all in play to test a
commander's expanding set of required skills.

But senior officers admit it is much harder to figure out how to prepare their most senior
commanders for managing relationships with civilian masters in Washington, especially if popular
support is waning for both the strategy, and the wars themselves.

In an acknowledgment that the top jobs have become ever more intellectually challenging, physically
exhausting and politically bruising, senior officers confirm that the armed services are looking to
exactly this broader set of skills as they fit their future four-stars with the mask of command worn
by Washington and Grant, Marshall and Eisenhower.

Perhaps no general is in a better position to assess the new challenges of command than General
Petraeus, who was sent to Afghanistan by President Obama to replace Gen.~Stanley A.~McChrystal, who
was relieved as the allied commander there after Rolling Stone magazine reported remarks he and his
aides made that were critical or even disparaging of civilian leaders in Washington. And General
McChrystal was placed in the job when the Obama administration shifted strategy, and removed
Gen.~David D.~McKiernan.

Just as in Iraq, General Petraeus has been ordered to salvage a war effort, this one having suffered
from a shortage of resources after being relegated to a sideshow war for years.

In a telephone interview from his headquarters in Afghanistan, General Petraeus said the shifting
demands of counterinsurgency and coalition warfare presented an array of intricate management
challenges.

In a previous assignment, General Petraeus ran the Army's schoolhouses at Fort Leavenworth, Kan.,
central to the effort to transform the ground force from preparations for what he called ``The Clash
of the Titans'' -- heavy armor-on-armor warfare -- to better succeed in today's counterinsurgency
fight, which focuses as much on protecting the civilian population and enabling local governments as
it does on troops killing enemy fighters.

To that end, he led efforts to write a counterinsurgency manual -- a blueprint for the wars today --
and also a new guide on Army leadership that ``tried to come to grips with all of these attributes
needed in leaders in full-spectrum operations,'' General Petraeus said.

Those complicated tasks, he acknowledged, also must be conducted under ``the magnifying glass that
is applied to them by a 24-hour news cycle.''

General Petraeus will be under that magnifying glass as analysts wonder whether historians who
lament that Afghanistan is the graveyard of empires may have gotten it only partly right -- and
whether, for the American military, Afghanistan may likewise become the symbolic graveyard for the
careers of generals, too.

Which is not to say that past commanders of major theaters of war had it easy.

When Eisenhower was European commander in World War II, he had alliance politics to manage, as well
as the enormous egos of his subordinates.

But, said Kori Schake, a Hoover Institution research fellow who has held senior policy positions at
the National Security Council and at the Departments of State and Defense, Eisenhower's mission was
far more straightforward.

``His orders were to invade Europe, conquer Germany,'' she said. ``He was asked to defeat another
uniformed, organized national army. In comparison, part of the reason we are struggling with the
wars today is that military force cannot so easily achieve the complicated, sophisticated set of
second-order effects we are asking it to achieve.''

The speed of war has increased. Not only have the missions and goals in both Iraq and Afghanistan
shifted over the years of combat, Ms.~Schake said, but in advance of D-Day, the United States
government summoned ``thousands of economists and captains of industry'' to plan the occupation.
Today, though, ``Responsibilities for things that are not traditional areas of military expertise --
reconstruction and economic development -- have migrated into the military realm,'' Ms.~Schake said.

To oversee that broad and expanding portfolio requires a theater commander to establish a wide range
of relationships not demanded of previous officers, said David W.~Barno, a retired lieutenant
general who was ordered to establish a new, three-star headquarters in Kabul in 2003, a command he
held for 19 months.

``I no longer could rely just on relationships inside the military,'' said General Barno, now a
senior fellow at the Center for a New American Security, a nonpartisan policy institute. ``I had to
tremendously expand my network to include other parts of government -- the embassy there and senior
players at Usaid -- and start and then grow relationships with members of the host government to get
anything done. And not to forget allies, Capitol Hill, the media.''

Adm. William J.~Fallon held two of the military's most prestigious theater commands, in charge of
American forces in the Pacific and then the Middle East, before he retired early after making public
statements on Iran, Iraq and other issues that seemed to put him at odds with the Bush
administration.

``A key difference from earlier times of command is the available amount of information and the
demand on time that presents,'' Admiral Fallon said. ``With this 24/7, 365 demand cycle, you lose
the opportunity to sit back and think, to take stock of what is important and what is distracting --
and that is an incredible burden on a senior leader.''

\section{Chinese Hospitals Are Battlegrounds of Discontent}

\lettrine{F}{orget}\mycalendar{Aug.'10}{13} the calls by many Chinese patients for more honest,
better-qualified doctors. What this city's 27 public hospitals really needed, officials decided last
month, was police officers.

And not just at the entrance, but as deputy administrators. The goal: to keep disgruntled patients
and their relatives from attacking the doctors.

The decision was quickly reversed after Chinese health experts assailed it, arguing that the police
were public servants, not doctors' personal bodyguards.

But officials in this northeastern industrial hub of nearly eight million people had a point.
Chinese hospitals are dangerous places to work. In 2006, the last year the Health Ministry published
statistics on hospital violence, attacks by patients or their relatives injured more than 5,500
medical workers.

``I think the police should have a permanent base here,'' said a neurosurgeon at Shengjing Hospital.
``I always feel this element of danger.''

In June alone, a doctor was stabbed to death in Shandong Province by the son of a patient who had
died of liver cancer. Three doctors were severely burned in Shanxi Province when a patient set fire
to a hospital office. A pediatrician in Fujian Province was also injured after leaping out a
fifth-floor window to escape angry relatives of a newborn who had died under his care.

Over the past year, families of deceased patients have forced doctors to don mourning clothes as a
sign of atonement for poor care, and organized protests to bar hospital entrances. Four years ago,
2,000 people rioted at a hospital after reports that a 3-year-old was refused treatment because his
grandfather could not pay \$82 in upfront fees. The child died.

Such episodes are to some extent standard fare in China, where protests over myriad issues have been
on the rise. Officials at all levels of government are on guard against unrest that could spiral and
threaten the Communist Party's power.

Doctors and nurses say the strains in the relations between them and patients' relatives are often
the result of unrealistic expectations by poor families who, having traveled far and exhausted their
savings on care, expect medical miracles.

But the violence also reflects much wider discontent with China's public health care system.
Although the government, under Communist leadership, once offered rudimentary health care at nominal
prices, it pulled back in the 1990s, leaving hospitals largely to fend for themselves in the new
market economy.

By 2000, the World Health Organization ranked China's health system as one of the world's most
inequitable, 188th among 191 nations. Nearly two of every five sick people went untreated. Only one
in 10 had health insurance.

Over the past seven years, the state has intervened anew, with notable results. It has narrowed if
not eliminated the gap in public health care spending with other developing nations of similar
income levels, health experts say, pouring tens of billions of dollars into government insurance
plans and hospital construction.

The World Bank estimates that more than three in four Chinese are now insured, although coverage is
often basic. And far more people are getting care: the World Bank says hospital admissions in rural
counties have doubled in five years.

``That is a steep, steep increase,'' said Jack Langenbrunner, human development coordinator at the
World Bank's Beijing office. ``We haven't seen that in any other country.''

Still, across much of China, the quality of care remains low. Almost half the nation's doctors have
no better than a high school degree, according to the Organization for Economic Cooperation and
Development. Many village doctors did not make it past junior high school.

Primary care is scarce, so public hospitals -- notorious for excessive fees -- are typically
patients' first stop in cities, even for minor ailments. One survey estimated that a fifth of
hospital patients suffer from no more than a cold or flu. Chinese health experts estimate that a
third to a half of patients are hospitalized for no good reason.

Once admitted, patients are at risk of needless surgery; for instance, one of every two Chinese
newborns is delivered by Caesarean sections, a rate three times higher than health experts
recommend.

Patients appear to be even more likely to get useless prescriptions. Drug sales are hospitals'
second biggest source of revenue, and many offer incentives that can lead doctors to overprescribe
or link doctors' salaries to the money they generate from prescriptions and costly diagnostic tests.
Some pharmaceutical companies offer additional under-the-table inducements for prescribing drugs,
doctors and experts say.

An article in November in The Guangzhou Daily in southeastern China cited one particularly egregious
example of unnecessary treatment: a patient paid roughly \$95 for a checkup, several injections and
a dozen different drugs, including pills for liver disease. He had a cold.

The Health Ministry has ordered hospitals to reduce prices of specific drugs 23 times in a decade,
but the World Bank says hospitals have responded, in part, by ordering higher-priced alternatives.

Some experts fear that the newly opened spigot of government insurance money will inspire further
excesses, rather than reduce the financial risk of illness for most Chinese. Indeed, one study shows
only a slight drop in the share of household spending devoted to health care -- 8.2 percent in 2008,
down from 8. 7 percent in 2003.

``Their protection may not really be improving with insurance,'' said Mr.~Langenbrunner of the World
Bank. ``That is the scary part.''

Doctors seem as unhappy as patients. They complain that they are underpaid, undervalued and
mistrusted. One in four suffers from depression, and fewer than two of every three believe that
their patients respect them, a survey by Peking University concluded in October.

In June, more than 100 doctors and nurses in Fujian Province staged their own sit-in after their
hospital paid \$31,000 to the family of a patient who died. The doctors were upset because after the
patient died the relatives took a doctor hostage, setting off a bottle-throwing melee that injured
five employees.

Like some other cities, Shenyang has been seeking ways to ward off disturbances, including setting
up hospital mediation centers. Still, the city reported 152 ``severe conflicts'' between patients
and doctors last year.

At Hospital No.~5, the memory of a January attack remains fresh. After a doctor referred a patient
with a temperature to a fever clinic -- standard practice in China -- frustrated relatives beat the
doctor and several nurses with a mop and sticks.

Now a banner strung across the hospital's main lobby exhorts: ``Everyone participate in the sorting
out of the law and order problem!''

\section{Beckham Fights Against Time to Keep Playing}

\lettrine{C}{alling}\mycalendar{Aug.'10}{13} time on the international playing career of David
Beckham is not a simple sporting decision. He has pushed out managers who tried to retire him in the
past. And although he is now 35 years old, and just starting a comeback from a snapped Achilles
tendon, Beckham sees himself as soccer's Peter Pan -- forever young.

He is more than a player to England. His iconic status crosses over from sports to pop culture. His
presence earns the England Football Association millions in sponsorships, and he is considered worth
his weight in gold in the bidding for the 2018 World Cup.

The country that wins the right to host the tournament -- for which England, Russia, the United
States and the joint candidates Belgium/Netherlands and Portugal/Spain are contenders -- will be
announced in December. It is definitely not a good time for England to declare that Beckham is
finished.

But that is what the coach of England, Fabio Capello, effectively did on national television moments
before Wednesday's game against Hungary at Wembley Stadium in London.

Asked if he saw a future for Beckham in the team, Capello replied: ``No.~I say thank you very much
for helping me at the World Cup, but probably he is a little bit too old.''

Within minutes, Beckham's agent issued a statement via the BBC. ``For your info,'' it read, ``there
has been no discussion of retirement. He will always be available for his country when fit, and if
needed he will be there.''

Beckham has continuously said as much. On the day that Capello publicly closed the door to him with
no advance warning, Beckham returned ahead of schedule to train with the Los Angeles Galaxy. He did
conditioning work only, but told reporters afterwards: ``I've been waiting for this for almost five
months. The doctor said October 1, but I hope it can be a few weeks before that. I'll know
personally when I'm right.''

The injury layoff had brought home to him how much he still loves the game. ``Some people say at my
age that you start not loving football as much as you did when you were 21,'' he said. ``But I still
do.''

Unaware of Capello's comments in London, Beckham repeated what his agent had stated: he would never
retire from playing for his country. There are good young players coming through, but he would
always be available.

The commitment is a part of why fans identify with him. This week, two players selected for England
by Capello -- the goalkeeper Paul Robinson and the defender Wes Brown -- said thanks, but no thanks.
Both aged 30, they opted to concentrate on their lucrative club careers.

Beckham, once again, is different. His global marketing contracts are as much tied to his national
team exposure as the million-dollar salaries that he has been able to command while dividing his
time between the Galaxy team and AC Milan, one of the top clubs in Europe.

On Wednesday, for the first time in three years, Beckham conceded that trying to represent two teams
-- three counting England -- caused his severe injury.

``Hitting it so hard, every month, all year round has been tough on my body,'' he said. ``My focus
now is repairing my body, playing as soon as possible, then taking the time to get the rest in and
be ready for next season.''

He has spoken before of playing until he is 40, just as Milan's great captain Paolo Maldini did. The
difference is that Maldini gave up playing for Italy's national team for the last seven years of his
club career -- and Maldini was a greater talent, arguably one of the best left backs in the history
of soccer.

Beckham, to his credit, made the very best with limited ability. He was never extraordinarily fast,
or able to beat opponents through sheer guile. But he could, with a gift that inspired the film
``Bend it Like Beckham,'' impart pace, precision and swerve to his crosses and free kicks.

He could, as his record has shown, come back from letting himself and his country down after being
sent off for petulance, including at the 1998 World Cup. He could persevere, sometimes through sheer
willpower, through three World Cups and 115 caps, a record for an English outfield player.

Beckham has a simple, enduring and endearing love of the game. And he displays a patriotism that
survives the celebrity cult that he married into when he hitched up with Victoria Adams, the former
Spice Girls singer.

They share a star-studded lifestyle. Their fame has been managed by the same company. And his,
remarkably, has outlasted and out-earned hers.

But England's coach, Capello, came home chastened from the World Cup in South Africa, where England
was humiliated by a younger and faster Germany.

Capello is the highest-paid soccer coach on the planet. England pays him \textsterling6 million a
year, or \$9.3 million, and its fans expect more than what happened this summer at the World Cup.
Capello says he will not quit, and he will take whatever tough decisions are necessary to rebuild
the team.

His comment terminating the career of Beckham for England came across as blunt and ungrateful.
Partly that was because he was taken by surprise by being asked the Beckham question moments before
the game, and partly it was because his grasp of the English language is still incomplete more than
two years after taking the job.

While the match was in play, the Beckham representatives and the F.A. public relations people
realized that an uproar was likely to follow what was a terse dismissal to a long career.

And they were right. The media in Britain, themselves still smarting from such a poor World Cup,
went for Capello after the final whistle.

He was ready. ``I think that David knows he has no future with the international team, because we
have to change,'' Capello said at the postgame news conference. ``David is a fantastic player. He
was really important, but we have to monitor the new players for the future. The age is important.''

There have been calls in England recently for Beckham to have a permanent role with the national
squad, and even calls for Beckham and a former coach of England, Sven-Goran Eriksson, to take charge
of the Premier league club Aston Villa, whose manager quit this week.

If Beckham wants to follow that path, more power to him. Meanwhile, he faces a long, hard slog to
get in the best shape that his aging body will allow, or faces a decline as a celebrity player.

In the real world of competitive sports, there's no room for Peter Pan.

\section{Quick Change in Strategy for a Bookseller}

\lettrine{I}{n}\mycalendar{Aug.'10}{13} the movie ``You've Got Mail,'' Tom Hanks played the
aggressive big-box retailer Joe Fox driving the little bookshop owner played by Meg Ryan out of
business.

Twelve years later, it may be Joe Fox's turn to worry. Readers have gone from skipping small
bookstores to wondering if they need bookstores at all. More people are ordering books online or
plucking them from the best-seller bin at Wal-Mart.

But the threat that has the industry and some readers the most rattled is the growth of e-books. In
the first five months of 2009, e-books made up 2.9 percent of trade book sales. In the same period
in 2010, sales of e-books, which generally cost less than hardcover books, grew to 8.5 percent,
according to the Association of American Publishers, spurred by sales of the Amazon Kindle and the
new Apple iPad. For Barnes \& Noble, long the largest and most powerful bookstore chain in the
country, the new competition has led to declining profits and store traffic. After the company
announced last week that it was putting itself up for sale, Leonard Riggio, Barnes \& Noble's
chairman and largest shareholder, who has declared his confidence in the company's future, hinted
that he might make a play to buy the company himself and take it private.

For readers, e-books have meant a transformation not just of the reading experience, but of the
book-buying tradition of strolling aisles, perusing covers and being able to hold books in their
hands. Many publishers have been astounded by the pace of the e-book popularity and the threat to
print book sales that it represents. If the number of brick-and-mortar stores drops, publishers fear
that sales will go along with it. Some worry that large bookstores will go the way of the record
stores that shut down when the music business went digital.

``The shift from the physical to the digital book can pick up some of the economic slack, but it
can't pick up the loss that is created when you don't have the customers browsing the displays,''
said Laurence J.~Kirshbaum, a literary agent. ``We need people going into stores and seeing a book
they didn't know existed and buying it.''

Carolyn Reidy, the chief executive of Simon \& Schuster, said in an interview that e-books currently
made up about 8 percent of the company's book revenue. She predicted that it could be as high as 40
percent within three to five years.

``E-books are moving faster and faster all of the time, which makes things look harder for
bricks-and-mortar stores,'' said Mike Shatzkin, founder and chief executive of the Idea Logical
Company, which advises book publishers on digital change.

Iris Reeves, a 53-year-old administrative assistant in East Texas, is one of the bookstore holdouts.
Nearly every weekend, she and her husband drive 60 miles to the nearest Barnes \& Noble for a long
browsing session. She buys several paperbacks (thrillers, science fiction and paranormal romance)
and he buys nonfiction (with a few auto magazines thrown in).

She has watched with alarm as dozens of bookstores, both independents and chains like Crown Books,
have disappeared. Beyond Barnes \& Noble and Borders, the only other retailers nearby that sell new
books, she said, are religious bookstores.

``I don't want to lose the option of actually going into a bookstore and handling a book,''
Ms.~Reeves said. ``I like going up and down the aisles, seeing what's there. If I had my druthers,
it would be paper books all the way.''

Whoever ends up in control of Barnes \& Noble's 720 retail stores will have to grapple with the
fundamental changes in the industry -- and if the shift to e-books continues, prove that Barnes \&
Noble can be as successful on the digital side of bookselling as it has been for print.

William Lynch, the chief executive, said in an interview on Friday that the chain was retooling its
stores to build up traffic, add products like educational toys and games, and emphasize its own
e-reader, the Nook.

``We think we've got the right strategy,'' Mr.~Lynch said. ``The growth in our e-books business is
about nine months ahead of our plan.''

It is a rare moment of uncertainty for the company. In the 1990s heyday of the superstore, Barnes \&
Noble reigned supreme, expanding its reach rapidly and dazzling customers with an enormous array of
books and steep discounts that smaller, independent stores could not match. Mr.~Riggio, a tough and
innovative figure, was hailed as the most powerful man in the book business.

``As Barnes \& Noble grew, there was a lot that was very good for publishers and authors,'' said
David Steinberger, the chief executive of the Perseus Books Group. ``They were energetic, they were
aggressive, they were terrific on author events. They were terrific at broadening the selection
available.''

But recently, Barnes \& Noble has had to contend with Amazon.com, which has led on e-books and whose
vast selection of print books is available online. The release of Apple's iPad in April only
increased interest in e-books.

``This company is going to go through a really fundamental existential struggle,'' said Peter Osnos,
the founder and editor at large of PublicAffairs, an independent publisher. ``What you have is this
aggregation of factors -- the changes in the way book buying is taking place, the general
sluggishness of the economy, the management issues at Barnes \& Noble. All of those things together
create a set of problems which are really quite striking.''

At the expansive Barnes \& Noble store in Manhattan's Union Square, the changes sweeping the company
and the industry are on full display. Shelves have been stripped bare to make room for toys and
games, as a sign dangling from the ceiling cheerfully announces.

``I'm in favor of anything that brings traffic in the store,'' said Ms.~Reidy of Simon \& Schuster.
``If it's toys or games that brings a family into the bookstore, then I say fine.''

The company is also taking significant steps to capture the digital market. In September, it will
begin building 1,000-square-foot boutiques to showcase the Nook in all of its outlets.

Samantha Robinson, a 24-year-old student, paused outside the Union Square store last week, a newly
purchased Nook in her hand.

``I'm going to buy as many books as I can on the e-reader, because they're less expensive,''
Ms.~Robinson said.

And if she stopped buying print books altogether? ``I wouldn't miss it,'' she said.

In a twist straight out of the movies, some publishers speculated that many of the independents that
survived the big chains over the last 15 years might be in an unusually stable position. By the
American Booksellers Association's count, there are more than 2,000 independent bookstores in the
United States.

``Being small and privately held allows us to be more nimble,'' said Chris Morrow, owner of the
Northshire Bookstore in Manchester Center, Vt. ``Our competitive advantage has been the curation
aspect -- knowing our customers and picking the right books.

``We still have that competitive advantage,'' he added. ``Barnes \& Noble doesn't have that.''

\section{Eat an Apple (Doctor's Orders)}

\lettrine{T}{he}\mycalendar{Aug.'10}{13} farm stand is becoming the new apothecary, dispensing
apples -- not to mention artichokes, asparagus and arugula -- to fill a novel kind of prescription.

Doctors at three health centers in Massachusetts have begun advising patients to eat ``prescription
produce'' from local farmers' markets, in an effort to fight obesity in children of low-income
families. Now they will give coupons amounting to \$1 a day for each member of a patient's family to
promote healthy meals.

``A lot of these kids have a very limited range of fruits and vegetables that are acceptable and
familiar to them. Potentially, they will try more,'' said Dr.~Suki Tepperberg, a family physician at
Codman Square Health Center in Dorchester, one of the program sites. ``The goal is to get them to
increase their consumption of fruit and vegetables by one serving a day.''

The effort may also help farmers' markets compete with fast-food restaurants selling dollar value
meals. Farmers' markets do more than \$1 billion in annual sales in the United States, according to
the Agriculture Department.

Massachusetts was one of the first states to promote these markets as hubs of preventive health. In
the 1980s, for example, the state began issuing coupons for farmers' markets to low-income women who
were pregnant or breast-feeding or for young children at risk for malnourishment. Thirty-six states
now have such farmers' market nutrition programs aimed at women and young children.

Thomas M.~Menino, the mayor of Boston, said he believed the new children's program, in which doctors
write vegetable ``prescriptions'' to be filled at farmers' markets, was the first of its kind.
Doctors will track participants to determine how the program affects their eating patterns and to
monitor health indicators like weight and body mass index, he said.

``When I go to work in the morning, I see kids standing at the bus stop eating chips and drinking a
soda,'' Mr.~Menino said in a phone interview earlier this week. ``I hope this will help them change
their eating habits and lead to a healthier lifestyle.''

The mayor's attention to healthy eating dates to his days as a city councilman. Most recently he has
appointed a well-known chef as a food policy director to promote local foods in public schools and
to foster market gardens in the city.

Although obesity is a complex problem unlikely to be solved just by eating more vegetables,
supporters of the veggie voucher program hope that physician intervention will spur young people to
adopt the kind of behavioral changes that can help forestall lifelong obesity.

Childhood obesity in the United States costs \$14.1 billion annually in direct health expenses like
prescription drugs and visits to doctors and emergency rooms, according to a recent article on the
economics of childhood obesity published in the journal Health Affairs. Treating obesity-related
illness in adults costs an estimated \$147 billion annually, the article said.

Although the vegetable prescription pilot project is small, its supporters see it as a model for
encouraging obese children and their families to increase the volume and variety of fresh produce
they eat.

``Can we help people in low-income areas, who shop in the center of supermarkets for low-cost
empty-calorie food, to shop at farmers' markets by making fruit and vegetables more affordable?''
said Gus Schumacher, the chairman of Wholesome Wave, a nonprofit group in Bridgeport, Conn., that
supports family farmers and community access to locally grown produce.

If the pilot project is successful, Mr.~Schumacher said, ``farmers' markets would become like a
fruit and vegetable pharmacy for at-risk families.''

The pilot project plans to enroll up to 50 families of four at three health centers in Massachusetts
that already have specialized children's programs called healthy weight clinics.

A foundation called CAVU, for Ceiling and Visibility Unlimited, sponsors the clinics that are
administering the veggie project. The Massachusetts Department of Agriculture and Wholesome Wave
each contributed \$10,000 in seed money. (Another arm of the program, at several health centers in
Maine, is giving fresh produce vouchers to pregnant mothers.) The program is to run until the end of
the farmers' market season in late fall.

One month after Leslie-Ann Ogiste, a certified nursing assistant in Boston, and her 9-year-old son,
Makael Constance, received their first vegetable prescription vouchers at the Codman Center, they
have lost a combined four pounds, she said. A staff member at the center told Ms.~Ogiste about a
farmers' market that is five minutes from her apartment, she said.

``It worked wonders,'' said Ms.~Ogiste, who bought and prepared eggplant, cucumbers, tomatoes,
summer squash, corn, bok choy, parsley, carrots and red onions. ``Just the variety, it did help.''

Ms.~Ogiste said she had minced some vegetables and used them in soup, pasta sauce and rice dishes --
the better to disguise the new good-for-you foods that she served her son.

Makael said he did not mind. ``It's really good,'' he said.

Some nutrition researchers said that the Massachusetts project had a good chance of improving eating
habits in the short term. But, they added, a vegetable prescription program in isolation may not
have a long-term influence on reducing obesity. Families may revert to their former habits in the
winter when the farmers' markets are closed, these researchers said, or they may not be able to
afford fresh produce after the voucher program ends.

Dr.~Shikha Anand, the medical director of CAVU's healthy weight initiative, said the group hoped to
make the veggie prescription project a year-round program through partnerships with grocery stores.

But people tend to overeat junk food in higher proportion than they undereat vegetables, said
Dr.~Deborah A.~Cohen, a senior natural scientist at the RAND Corporation. So, unless people curtail
excessive consumption of salty and sugary snacks, she said, behavioral changes like eating more
fruit and vegetables will have limited effect on obesity.

In a recent study led by Dr.~Cohen, for example, people in southern Louisiana typically exceeded
guidelines for eating salty and sugary foods by 120 percent in the course of a day while falling
short of vegetable and fruit consumption by 20 percent.

The weight clinics in Massachusetts chosen for the vegetable prescription test project already
encourage families to cut down on unhealthy snacks.

Even as Ms.~Ogiste and her son started shopping at the farmers' market and eating more fresh
produce, for example, they also cut back on junk food, she said.

``We have stopped the snacks. We are drinking more water and less soda and less juice too,'' Ms.
Ogiste said. ``All of that helped.''

\section{China's Economy, While Still Surging, Begins Showing Signs of Moderation}

\lettrine{I}{n}\mycalendar{Aug.'10}{13} the latest signs that the Chinese economy is beginning to
cool after setting a torrid pace in the first half of this year, several government indicators
slowed slightly last month, Beijing announced on Wednesday.

The July indicators for industrial output, retail sales, fixed-asset investment and bank lending all
provided a fairly consistent snapshot of a country where economic growth remains the strongest in
the world, but where the nearly manic spending of the last few months is starting to fade.

A gradual slackening in retail sales and fixed-asset investment last month, together with a
weakening of imports that was announced on Tuesday, pointed to more caution by consumers and
investors alike in China. Exports continue to surge, creating the prospect that China may once again
rely on foreign buyers to sustain rapid economic growth and limit unemployment.

The moderation in China's economic growth is no accident. A series of government policies, as varied
as limits on bank loans and restrictions on real estate investments, have been aimed at preventing
the economy from overheating, which could fan inflation.

But an imbalance in China between strong exports and weakening domestic demand could rekindle trade
tensions with industrialized and developing countries, economists said. Unemployment remains a
problem in many other countries, not just the United States, that find themselves buying ever more
from China.

Particularly unexpected in the data on Wednesday was the restraint that Chinese banks showed in
issuing new loans, which in turn contributed to slower growth in the money supply. Banks issued 533
billion renminbi (\$78.7 billion) in new loans last month, down from 603.4 billion renminbi (\$89.1
billion) in June.

A broad measure of money supply, M2, rose 17.6 percent in July compared with a year earlier after
being up 18.5 percent in June.

``We believe this level of broad money supply growth is clearly too restrictive, as it will put more
downward pressure on domestic demand growth in the near future,'' two Goldman Sachs economists, Yu
Song and Helen Qiao, wrote in a research note.

The gradual slowing in China is evident in the factories that have turned the country into the
manufacturing center of the world. Industrial output rose 13.4 percent last month compared with the
same month last year.

By comparison, industrial output had been up 13.7 percent in June from a year earlier, after rising
16.5 percent in May.

Much the same pattern was evident in fixed-asset investment, which was up 24.9 percent last month
compared with a year earlier. It had been ahead by 25.5 percent in June.

Retail sales rose 17.9 percent in July compared with the same month last year, as Chinese consumers
with rising wages continued to flock to stores. But sales had grown 18.3 percent in June compared
with a year earlier, and had been up 18.7 percent in May.

Chinese policy makers have long feared that inflation could lead to social unrest, and have tailored
many policies to prevent it. Consumer prices were up 3.3 percent in July from a year earlier,
accelerating from an increase of 2.9 percent in June.

But there were signs that inflation could begin to moderate. Producer prices were 4.8 percent higher
in July than a year ago, after being up 6.4 percent in June and 7.1 percent in May.

Many businesses are complaining about rising costs, which may hurt their willingness to invest. Alex
Yu, a senior manager at the Dongguan Jet Power Metals and Plastics Factory in Dongguan, China, said
that wages had climbed 20 to 30 percent in the last year while the cost of an indispensable kind of
raw plastic had soared 23 percent.

``Everything is getting more and more expensive, we feel the impact of inflation in all areas of our
business,'' he said.

On Tuesday, China's General Administration of Customs announced that the country's trade surplus
reached \$28.7 billion last month, the highest level since January 2009.

Senator Charles E.~Schumer, Democrat of New York, who has repeatedly called for a more
confrontational trade strategy toward China, said in a statement late Tuesday that China's latest
trade numbers ``show just how little motive China has to end its currency manipulation unless it is
pushed to do so.''

\section{China Passes Japan as Second-Largest Economy}

\lettrine{A}{fter}\mycalendar{Aug.'10}{16} three decades of spectacular growth, China has passed
Japan to become the world's second-largest economy behind the United States, according to government
figures released early Monday.

The milestone, though anticipated for some time, is the most striking evidence yet that China's
ascendance is for real and that the rest of the world will have to reckon with a new economic
superpower.

The milestone was reached early Monday, when Tokyo said that in the second quarter, the Japanese
economy was valued at about \$1.28 trillion, slightly below China's figure of \$1.33 trillion. The
gross domestic product of the United States was roughly \$14 trillion in 2009. Japan's economy grew
0.4 percent in the second quarter, Tokyo said, substantially less than forecast.

Experts say unseating Japan -- and in recent years passing Germany, France and Great Britain --
underscores China's growing clout and bolsters forecasts that China will pass the United States as
the world's biggest economy as early as 2030.

``This has enormous significance,'' said Nicholas R.~Lardy, an economist at the Peterson Institute
for International Economics in Washington. ``It reconfirms what's been happening for the better part
of a decade: China has been eclipsing Japan economically. For everyone in China's region, they're
now the biggest trading partner rather than the U.S.~or Japan.''

For Japan, whose economy has been stagnating for more than a decade, the figures reflect a decline
in economic and political power. Japan has had the world's second-largest economy for much of the
last four decades, according to the World Bank. And during the 1980s, there was even talk about
Japan's economy some day overtaking that of the United States.

But while Japan's economy is mature and its population quickly aging, China is in the throes of
urbanization and is far from developed, analysts say, meaning it has a much lower standard of living
as well as a lot of room to grow. Just five years ago, China's gross domestic product was about
\$2.3 trillion, about half of Japan's.

This country has roughly the same land mass as the United States, but it is burdened with a fifth of
the world's population and insufficient resources. Its per capita income is more on a par with those
of impoverished nations like Algeria, El Salvador and Albania -- which, along with China, are close
to \$3,600 -- than that of the United States, where it is about \$46,000.

Yet there is little disputing that under the direction of the Communist Party, China has begun to
reshape the way the global economy functions by virtue of its growing dominance of trade, its huge
hoard of foreign exchange reserves and United States government debt and its voracious appetite for
oil, coal, iron ore and other natural resources.

China is already a major driver of global growth. The country's leaders have grown more confident on
the international stage and have begun to assert greater influence in Asia, Africa and Latin
America, with things like special trade agreements and multibillion dollar resource deals.

``They're exerting a lot of influence on the global economy and becoming dominant in Asia,'' said
Eswar S.~Prasad, a professor of trade policy at Cornell and former head of the International
Monetary Fund's China division. ``A lot of other economies in the region are essentially riding on
China's coat tails, and this is remarkable for an economy with a low per capita income.''

Beijing is also beginning to shape global dialogues on a range of issues, analysts said, for
instance, last year, when it asserted that the dollar must be phased out as the world's primary
reserve currency.

And while the United States and the European Union are struggling to grow in the wake of the worst
economic crisis in decades, China has continued to climb up the economic league tables by investing
heavily in infrastructure and backing a \$586 billion stimulus plan.

This year, although growth has begun to moderate a bit, China's economy is forecast to expand about
10 percent -- continuing a remarkable three-decade streak of double-digit growth.

``This is just the beginning,'' said Wang Tao, an economist at UBS in Beijing. ``China is still a
developing country. So it has a lot of room to grow. And China has the biggest impact on commodity
prices -- in Russia, India, Australia and Latin America.''

There are huge challenges ahead, though. Economists say that China's economy is too heavily
dependent on exports and investment and that it needs to encourage greater domestic consumption --
something China has struggled to do.

The country's largely state-run banks have recently been criticized for lending far too aggressively
in the last year and even engaging in financial engineering, shifting some loans off their balance
sheet to disguise lending and evade rules meant to curtail lending growth.

China is also locked in a fierce debate over its currency policy, with the United States, European
Union and others accusing Beijing of keeping the Chinese currency, the renminbi, artificially low to
bolster exports -- leading to huge trade surpluses for China but major bilateral trade deficits for
the United States and the European Union. China says that its currency is not substantially
undervalued and that it is moving ahead with currency reform.

Regardless, a fast-growing China suggests that it will continue to compete fiercely with the United
States and Europe for natural resources but also offer big opportunities for global companies and
technology firms eager to tap its market.

Although its economy is still only one-third the size of the American economy, China passed the
United States last year to become the world's largest market for passenger vehicles. China also
passed Germany last year to become the world's biggest exporter.

Global companies like Caterpillar, General Electric, General Motors and Siemens -- as well as scores
of others -- are making a more aggressive push into China, in some cases moving research and
development centers here.

Some analysts, though, say that while China is eager to assert itself as a financial and economic
power -- and to push its state companies to ``go global'' -- it is reluctant to play a greater role
in the debate over climate change or how to slow the growth of greenhouse gases.

China passed the United States in 2006 to become the world's largest emitter of greenhouse gases,
which scientists link to global warming. But China also has an ambitious program to cut the energy
it uses for each unit of economic output by 20 percent by the end of 2010, compared to 2006.

China has imposed stringent automotive fuel-economy standards, built a new generation of more
efficient coal-fired power plants and begun a national program to close antiquated factories or
refit them with more efficient equipment.

Assessing what China's newfound clout means, though, is complicated. While the country is still
relatively poor per capita, it has an authoritarian government that is capable of taking decisive
action -- to stimulate the economy, build new projects and invest in specific industries.

That, Mr.~Lardy at the Peterson Institute said, gives the country unusual power.

``China is already the primary determiner of the price of virtually every major commodity,'' he
said. ``And the Chinese government can be much more decisive in allocating resources in a way that
other governments of this level of per capita income cannot.''

\section{Workers Let Go by China's Banks Putting Up Fight}

\lettrine{T}{hese}\mycalendar{Aug.'10}{16} are heady days for China's state-controlled banks. Last
month, the Agricultural Bank of China made its stock market debut, bringing in \$22 billion for the
largest public offering ever. A sister government-run bank, the Industrial and Commercial Bank of
China, now has the highest stock market value of any bank in the world.

But the windfalls have created an unusual problem for China: white-collar unrest. A few days after
the Agricultural Bank went public, dozens of former bank employees stealthily gathered outside the
headquarters of the country's central bank. There, after distributing small Chinese flags, they
quickly pulled on red and blue T-shirts that read, ``Protect the Rights of Downsized Bank Workers.''
By the time they unfurled their protest banners, the game was over.

Within minutes, a flock of police officers had swept everyone into five waiting public buses. By 8
a.m., when the People's Bank of China opened its doors for business, the only sign of the rally was
a strand of police tape.

During the past two years, these unlikely agitators -- conservatively attired but fiercely
determined -- have staged similar public protests in Beijing and provincial cities. They have
stormed branch offices to mount sit-ins. A few of the more foolhardy have met at Tiananmen Square to
distribute fliers before plainclothes police officers snatched them away.

Strategizing via online message boards and text messages, they speak in code and frequently change
cellphone numbers. Their acts of defiance are never mentioned in state-run news media.

According to one organizer, a scrappy former bank teller named Wu Lijuan, there are at least 70,000
people seeking to regain their old jobs or receive monetary compensation, a sizable wedge of the
400,000 who were laid off during a decade-long purge. Like many other state-owned companies, the
banks slashed payrolls and restructured to raise profitability and make themselves more attractive
to outside investors.

``They tossed us out like garbage,'' Ms.~Wu, 44, said before a recent protest, scanning fellow
restaurant patrons for potential eavesdroppers. ``All we're asking for is justice and maybe to serve
as a model for others who have been wronged.''

For a government determined to maintain social harmony, the protests and petitioning are vexing.
Compared with farmers angry over seized land or retired soldiers seeking fatter pensions, the bank
workers -- educated, organized and knowledgeable about the Internet -- are better equipped to
outsmart the public security agents constantly on their trail.

``What the government fears most are people capable of organizing, and the bank workers have
discovered their power,'' said Renee Xia, international director of Chinese Human Rights Defenders.
``The sad thing is that they're not going to succeed because the more organized you are, the more
harsh the government's reaction.''

Protest organizers are often thrown in ``black jails'' -- extrajudicial holding pens -- where they
are sometimes beaten before local police officers arrive to take them back home. The recalcitrant
and unrepentant sometimes end up in labor camps, where they can spend up to three years without
being prosecuted for a crime.

The years of fruitless protest and economic hardship have taken a toll. According to an informal
tally by protest leaders, dozens of former bank staff members -- most of them unsuccessful at
finding new jobs -- have committed suicide.

``To be middle-aged and live off your elderly parents is humiliating, and it can become
unbearable,'' said Huang Gaoying, 49, a teller who was dismissed from the Industrial and Commercial
Bank known as I.C.B.C., in 2002.

Even if their numbers are smaller, the former bank employees are not unlike the millions of factory
workers shed during the effort to restructure inefficient state-owned enterprises in the late 1990s.
In the years that followed, they, too, clamored for redress but were eventually silenced.

In 2000, the Supreme People's Court put an end to any hope that the legal system might adjudicate
such disputes, saying that plaintiffs from state companies had no standing in Chinese courts.

Like the laid-off factory workers, the former bank employees have no independent trade union or
association to take up their cause.

Yi Xianrong, a scholar at the Financial Research Center, part of the state-backed Chinese Academy of
Social Sciences, said the bank workers were unfortunate victims in the necessary revamping of a
bloated and inefficient sector. ``In a centrally planned economy, people were put into certain units
without regard to need,'' he said. ``It didn't matter if you actually worked or not.''

By Western standards, the banks were -- and arguably still are --overstaffed, a legacy of their role
as the pillars of China's socialist financial system.

Mr.~Yi was not particularly sympathetic to the complaints of the former employees, saying they
signed and accepted buyout packages. Many of the workers would disagree, saying they were often
forced to accept paltry compensation, sometimes just two or three years before planned retirements.

In interviews with nearly two dozen of the aggrieved, the pattern of dismissals was roughly the
same. Workers over 40 were singled out first and there was no room for negotiation. (At I.C.B.C.,
the standard buyout was about \$370 for every year worked.) Those who refused an offer were simply
let go without compensation.

Asked to comment on the plight of laid-off workers, the banks -- I.C.B.C., the Bank of China, the
Agricultural Bank and China Construction Bank -- declined.

The story of Ms.~Wu, the protest organizer, is typical. Hired just out of high school by an I.C.B.C.
branch in central Hubei Province, she said she received numerous ``model worker'' commendations and
had expected lifetime employment. In 2004, as the bank prepared to issue stock, she was among
160,000 employees laid off. The compensation offered, she said, was unacceptable. ``After 20 years
at the bank, I thought I deserved more,'' she said.

The six-year odyssey -- some might say obsession -- has included lawsuits, the petitioning of
China's top leaders and the storming of the branch president's office, which turned into a brawl and
led to a brief jail sentence for Ms.~Wu.

The banks are so powerful that they can enlist the local police to keep an eye on the most
troublesome employees, often following them to Beijing, where their protests and petitioning can
prove embarrassing for executives back home. ``The head of my branch said he would never give me my
money and spend any amount to fight me to the end,'' Ms.~Wu said.

After her husband divorced her, Ms.~Wu moved to Beijing with her teenage son to be closer to the
country's leaders, who she believes would force the banks to make their former employees whole, if
only they knew. She lives in one of the so-called petitioner villages on the outskirts of the
capital and survives by collecting recyclables or working as an artists' model.

In the days after the protest, as the other detainees were released, Ms.~Wu remained in custody. Her
son, Xiao Yang, 22, said the police searched the family's home and left with her computer's hard
drive. He said he had the feeling she might not be coming home for a long time. ``She is a very
stubborn person,'' he said. ``I'm definitely worried about her, but there's nothing I can do.''

\section{Kaymer Wins Wild P.G.A.}

\lettrine{P}{erhaps}\mycalendar{Aug.'10}{16} the only predictable thing about the P.G.A.
Championship was that having the leader board stocked with young players vying for their first major
on Sunday was going to make it a wild ride.

It turned into something closer to a demolition derby.

Martin Kaymer won it in a three-hole playoff over Bubba Watson by avoiding golfing disaster better
than anyone else. Kaymer, the 25-year-old from Germany, seemed like the only one of the first-time
major contenders who was not undone by nerves and swirling winds that appeared ready to tackle all
who dared to win the tournament.

``To be honest with you, I was very nervous during the regular round, especially the last three to
four holes,'' he said. ``But it in the playoff, it was strange. I was very calm, very confident. Now
it feels great.''

Kaymer had made a tricky 15-foot par putt on the 18th to make it into the playoff and then weathered
a drive into the rough on the third hole of the playoffs. To his advantage, Watson was even wilder
on that hole, also the 18th, hitting his second shot into a creek and double-bogeying. Kaymer needed
only a two-putt bogey to win.

Asked if his victory would be big news in Germany, Kaymer said: ``I don't know how big it is even
for me at the moment. I'm having trouble knowing what just happened. I'm getting goose bumps just
talking about it.''

The wildness at Whistling Straits on Sunday started as soon as the third-round leader Nick Watney
teed off. The winds started to swirl and the pressure of a major championship took hold. Watney's
three-shot lead was gone after one hole when he zigzagged from rough to bunker to double bogey.
After 15 holes, he was 11 over par for the day.

Dustin Johnson, Watney's playing partner, came in trying to write a better ending for a major than
the 82 he shot after leading the final round of the United States Open in June. Although he caught
Watney for the lead on that first hole, he went on his own roller-coaster ride with four birdies and
three bogeys, but Johnson appeared to be ending it on an upswing when he birdied 16 from a tough
spot in the rough and added a birdie at 17.

He needed only to par 18 to win, but sent his drive right of the fairway, his second shot into the
rough over the green and after a great chip shot, missed a 5-foot putt that would have won it.

It became worse when Johnson was ruled to have grounded his club on his second shot, which was out
of a bunker. It was hard to tell it was a bunker because Johnson had hit so far into the crowd, the
entire area was trampled down. The two-shot penalty knocked him out of the playoff and gave him a
nightmare finish to match his United States Open flop.

``I don't know if I can describe it,'' Johnson, 26, said when asked about his feelings. ``The only
worse thing could have been if I made that putt on 18.''

Johnson said he had no idea the area was a bunker. ``I figured it was just a piece of dirt that the
crowd had trampled down,'' he said.

Quietly, Kaymer was one of the few playing a steady round. He put together birdies on Nos. 2, 4 and
10 to take the lead at 12 under. He calmly saved par after a drive went wide on No.~8 while nearly
everyone around him seemed to capsize under similar circumstances. He did not stumble until his
first bogey on No.~15, and he made a great par save on 18 from the greenside rough to get himself in
the playoff.

Watson had a chance to take the lead, but on No.~17, he sent his tee shot off the cliff to the left
of the green, chipped up and settled for bogey. Steve Elkington could have led too, but he barely
missed a birdie putt at No.~15 and a 6-foot eagle putt on 16 and missed the green on 17 and bogeyed.

Despite having won a major, Elkington, 47, was an unlikely contender. Not only was his notable win
15 years ago at the 1995 P.G.A., he is currently ranked 300th. Watson, 31, managed his first PGA
Tour win this year, an emotional one he dedicated to his father, who has cancer.

Kaymer might be little known to a wider audience, but he has certainly worked his way into this spot
methodically. He finished seventh at the British Open last month, eighth at the United States Open
and sixth at last year's P.G.A. He has 13 victories on the European Tour. Finishing at 10 under here
was Rory McIlroy, the 21-year-old phenom from Northern Ireland, and Zach Johnson, the 2007 Masters
winner.

Perfect weather lured plenty of spectators to Whistling Straits early. After maddening fog and the
threat of rain marred the first two days, the sun finally came out Saturday, and conditions were
even more ideal Sunday.

They were treated to a terrific round by Phil Mickelson, who came into the tournament playing
unsteadily and with the news that he has psoriatic arthritis and had not put his game together
through the first three rounds. But on Sunday, he finally got hot. He eagled the par-5 No.~5 early
and then had birdies on three consecutive holes. He hit nice putts on Nos. 12 and 13 and hit his
approach within two feet for birdie on No.~14. That put him at seven under. He stayed there until a
wild ride on 18, which led to a bogey and a round of 67.

Tiger Woods, who had also come in on a slide, could not match Mickelson's momentum. His front nine
included four birdies and three bogeys, but he started his back nine with a double bogey on 10 and
made little noise after that and finished at two under. In their place, a first-time major winner
would emerge again, following in the footsteps of the 27-year-old South African Louis Oosthuizen at
the British Open and Graeme McDowell at the United States Open. In all, a first-timer has won six of
the past seven majors, the only exception being Mickelson's victory at the Masters in April.

\section{Petraeus Opposes a Hasty Pullout in Afghanistan}

\lettrine{G}{en}\mycalendar{Aug.'10}{16}. David H.~Petraeus, the commander of American and NATO
forces, began a campaign on Sunday to convince an increasingly skeptical public that the
American-led coalition can still succeed here despite months of setbacks, saying he had not come to
Afghanistan to preside over a ``graceful exit.''

In an hourlong interview with The New York Times, the general argued against any precipitous
withdrawal of forces in July 2011, the date set by President Obama to begin at least a gradual
reduction of the 100,000 troops on the ground. General Petraeus said that it was only in the last
few weeks that the war plan has been fine-tuned and given the resources that it required. ``For the
first time,'' he said. ``we will have what we have been working to put in place for the last year
and a half.''

In another in a series of interviews, on ``Meet the Press,'' General Petraeus even appeared to leave
open the possibility that he would recommend against any withdrawal of American forces next summer.
``Certainly, yes,'' he said when the show's host, David Gregory, asked him if, depending on how the
war was proceeding, he might tell the president that a drawdown should be delayed. ``The President
and I sat down in the Oval Office, and he expressed very clearly that what he wants from me is my
best professional military advice.''

The statement offered a preview of what promised to be an intense political battle over the future
of the American-led war in Afghanistan, which has deteriorated on the ground and turned unpopular at
home. Already, some Democrats in Congress are pushing for steep withdrawals early on, while
supporters of the war say that a precipitous draw-down could endanger the Afghan mission altogether.

General Petraeus, in his interview with The Times, said American and NATO troops were making
progress on a number of fronts, including routing Taliban insurgents from their sanctuaries,
reforming the Afghan government and preparing Afghan soldiers to fight on their own.

General Petraeus, who took over last month after Gen.~Stanley A.~McChrystal was fired for making
disparaging remarks about civilian leaders, said he believed that he would be given the time and
mat\'eriel necessary to prevail here. He expressed that confidence despite the fact that nearly
every phase of the war is going badly -- and even though some inside the Obama administration have
turned against it.

``The president didn't send me over here to seek a graceful exit,'' General Petraeus said at his
office at NATO headquarters in downtown Kabul. ``My marching orders are to do all that is humanly
possible to help us achieve our objectives.''

General Petraeus' public remarks, his first since taking over, highlight the extraordinary
challenges, both military and political, that loom in the coming months. American soldiers and
Marines are dying at a faster rate than at any time since 2001. The Afghan in whom the United States
has placed its hopes, President Hamid Karzai, has demonstrated little resolve in rooting out the
corruption that pervades every corner of his government.

And perhaps most important, the general will be trying to demonstrate progress in the 11 months
until Mr.~Obama's deadline to begin withdrawing troops.

The date was chosen in part to win over critics of the war and to push the Afghan government to
reform more quickly. But as critical battles to reclaim parts of the Taliban heartland have
faltered, military commanders have begun preparing to ask the White House to keep any withdrawals
next year to a minimum.

In the interview with The Times, General Petraeus also suggested that he would resist any
large-scale or rapid drawdown of American forces. If the Taliban believes that will happen, he said,
they are mistaken.

``Clearly the enemy is fighting back, sees this as a very pivotal moment, believes that all he has
to do is outlast us through this fighting season,'' General Petraeus said. ``That is just not the
case.''

The public campaign begun Sunday echoes the similarly high-profile efforts the general undertook at
the bloodiest phase of the war in Iraq. In early 2007, joining a group of defense intellectuals and
retired generals, General Petraeus asserted that the anarchic situation in Iraq could be stabilized
with an infusion of tens of thousands additional American troops.

Then-President George W.~Bush endorsed the effort and chose General Petraeus to lead it. And, to the
surprise of many, the campaign, known as ``the surge,'' helped bring about a dramatic drop in
violence that has largely held to this day. During the surge, General Petraeus sometimes skirted the
traditional lines separating the military and political worlds, testifying before Congress and
speaking almost weekly to Mr.~Bush.

General Petraeus has taken a lower public profile since Mr.~Obama's inauguration. His efforts on
Sunday -- which will continue with more interviews in the coming days -- represent his first attempt
to convince the American people that his efforts and those of the American soldiers and Marines
deployed here can succeed.

The campaign that began Sunday, and which included an interview with The Washington Post,
highlighted General Petraeus's political strengths as much as his military ones. He was careful,
patient and disciplined -- sticking carefully to his main points -- traits that have won him
widespread respect in Washington.

Among other things, the general is fighting to preserve his own legacy, based on the dramatic
turnaround he helped orchestrate during the war in Iraq. The hallmark of that strategy was its focus
on protecting civilians, even at the expense of letting insurgents walk away.

In Afghanistan, that approach is coming under growing criticism, mainly from people who regard it
has too expensive and open ended. Some in the Obama administration have been advocating a move away
from counterinsurgency toward a strategy that is focused on hunting and killing terrorists.

To put his plan into effect, General Petraeus has imported some hands from his Iraq days to help
him. Gen.~H. R.~McMaster, one of the most innovative officers in the Iraq war, has taken charge of a
task force assigned to attack public corruption. Frederick W.~Kagan, one of the fathers of the surge
-- and more recently a critic of the Afghan government -- has come to help as well.

The drafting of those experts suggests that General Petraeus intends to take a harder line against
corruption in the Karzai government, which ranks among the biggest factors driving Afghans to the
Taliban.

Mr.~Karzai has promised over the years to root out corruption but has largely failed to do so. He
has refused requests from American officials to remove his brother, Ahmed Wali Karzai, as chairman
of the provincial council in Kandahar Province, despite widespread reports of corruption. Last week,
the president tried to assert control over two American-backed Afghan anticorruption units that are
investigating Afghan officials.

General Petraeus declined to discuss the status of Ahmed Wali Karzai, and he praised President
Karzai's efforts to attack corruption. In any case, he suggested, American leverage over Mr.~Karzai
is limited. ``President Karzai is the elected leader of a sovereign country,'' he said. ``That is
how the people see him by and large; he is therefore -- and has to be, for sure -- our partner.''

\section{Nokia Seeks to Reconnect With the U.S.~Market}

\lettrine{M}{ary}\mycalendar{Aug.'10}{16} T.~McDowell, a 46-year-old Chicago native, is acutely
aware that her employer, Nokia, the mobile phone maker from Finland, has a lot of work to do.

The task is particularly daunting in the United States, where Nokia shows little of the swagger of a
global powerhouse that sells about 11 cellphones a second. With a skimpy 8 percent share of the
U.S.~market, according to the research firm comScore, Nokia trails Samsung, LG, Motorola and
Research In Motion, the maker of the BlackBerry.

As Nokia seeks to turn around its smartphone lineup to try to compete with the BlackBerry and with
Apple's iPhone, the company is also trying to shore up its bulk cellphone business, which accounts
for 84 percent of the 432 million devices Nokia sold last year. In July, it put Ms.~McDowell, a
former line engineer who kept Compaq and Hewlett-Packard atop the computer server market in the
1990s, in charge of its mass-market phone operations.

``People tell me: `Oh yes, I remember my Nokia device. It was very reliable and very good. I haven't
bought one in a while,' '' Ms.~McDowell said. ``So I think the opportunity for re-entry is there.''

Ms.~McDowell, who in 2004 became the first American hired directly onto Nokia's management board,
has taken over Nokia's bread-and-butter business at a crucial time. The company's market lead is
under assault, its profit and share price are sliding and its chief executive, Olli-Pekka Kallasvuo,
was recently called on to deny speculation that he would soon be replaced.

Long faulted for an insular corporate culture, Nokia is increasingly turning to Americans like
Ms.~McDowell, who will be relocating to London from Nokia's offices in White Plains, New York, and
Jo Harlow, a native of Harrodsburg, Kentucky, a senior executive in the smartphone division, to
oversee its global business and reconnect with U.S.~consumers. Another American, Richard Green, a
former Sun Microsystems executive, is Nokia's chief technology officer.

Asked last week whether Nokia was in crisis, Ms.~McDowell, the highest-ranking American within the
corporate hierarchy, said Nokia's top managers knew that its trusted recipe for success -- the
production of solid, dependable devices for a global mass market -- must become more flexible,
innovative and timely. She declined to comment on speculation about a change at the top.

``What I would say is, we have something to prove, particularly at the high end of the product
line,'' Ms.~McDowell said. ``I think it is a business that has good momentum, but obviously there is
more work to be done. There's a richness of innovation at Nokia.''

She added that Nokia's problems in the United States stemmed from its failure to work closely with
U.S.~mobile operators to tailor devices to their needs, rather than from any shortcomings in the
phones themselves.

Signs of the new, more urgent Nokia attitude can be seen in Berlin, where 400 Nokia employees work
on mapping and location-based services, two pillars of its online services business, Ovi. At Nokia's
reception desk in Berlin, a computer screen displayed a tally last week of 120.5 million registered
Ovi users, 300,000 more than a week earlier.

Nokia has a goal of signing up 300 million Ovi users by the end of 2011, customers who would also be
in a position to buy Nokia mobile services like banking, e-mail or even agricultural education, part
of a service called Nokia Life Tools in developing markets.

To make this a reality, Nokia plans to democratize the smartphone, Ms.~McDowell said, making it
affordable to more people, particularly in emerging markets, by offering premium features like touch
screens and enhanced Internet connectivity on lower-end phones. The company is expected to announce
some new features for its nonsmartphone line this week.

``Consumers across the price spectrum are starting to demand more from their mobile phones,'' she
said. Even when prices are low, ``people are looking up at smartphones and thinking, `What sort of
features would be relevant to me?' How do you tie that in a package that is both appealing and the
right price? That is the central challenge.''

In her 17-year tenure at Compaq and later at H.P., which bought Compaq in 2002, Ms.~McDowell
established a record of defending her employer's lead in the critical server market.

``She consistently drove bold changes that kept us one step ahead with a mantra of innovation on top
of standards netting differentiation on a large scale,'' said James Mouton, an H.P. senior vice
president and general manager, who worked with Ms.~McDowell.

Nokia hired Ms.~McDowell to oversee its line of business devices, and in 2008, promoted her to chief
development officer, responsible for global research and development. Ms.~McDowell said Nokia was
preparing new features for its mass-market phones, aiming at major markets like China, where it is
the market leader.

Some recent pragmatic innovations include the Nuron, a touch-screen phone sold by T-Mobile in the
United States for \$179 without a calling plan, and the C3, an e-mail device with full keyboard that
costs about \$100 before subsidies, made for the global market. Last week, hundreds of consumers
lined up to buy the C3 at its introduction in Hanoi.

Antii Vasara, a senior vice president for Nokia's Symbian devices who works for Ms.~McDowell, said
she liked to set clear, precise goals and track results. ``When something doesn't go as expected,
she has the ability to change and adapt quickly,'' Mr.~Vasara said.

In her new role, Ms.~McDowell will have to think on her feet. In the three months through May,
Nokia's U.S.~market share slipped 0.6 percentage point from the level of the previous quarter, to
8.1 percent, according to comScore. Globally, Nokia remains dominant, with 49 percent of the market
for phones costing less than \$100, according to Strategy Analytics. But even there, competitors
like ZTE of China and Samsung and LG are in hot pursuit.

\section{Asia Stops to Remember End of World War II}

\lettrine{A}{sia}\mycalendar{Aug.'10}{16} paused on Sunday to remember Japan's surrender to the
allied forces which ended World War II 65 years ago, as the Japanese prime minister apologized for
wreaking suffering on the region and the South Korean president said Tokyo's remorse was a step in
the right direction.

From Nanjing -- the site of a 1937 massacre by Japanese troops -- to Tokyo's Yasukuni Shrine, which
has drawn outrage from Asia for honoring Class A war criminals, people prayed for the millions who
died in war and expressed hopes for peace.

The reckoning with history has taken special meaning this year as it comes amid a global effort to
realize a world without nuclear weapons, a resolve backed by President Barack Obama. But there were
reminders of lingering tensions.

In Seoul, President Lee Myung-bak, dressed in traditional robes, led a ceremony celebrating the
liberation of the Korean peninsula from Japan's 1910-45 colonial rule with the Aug.~15 surrender.

He also urged North Korea to abandon military provocations and make a ''courageous change'' toward
peace. Relations with North Korea have nose-dived after the March sinking of a South Korean warship
and Pyongyang's firing last week of a barrage of artillery into South Korean waters.

In Tokyo, at a ceremony for the war dead, Prime Minister Naoto Kan reiterated his apology to South
Korea for wartime atrocities, and this time offered his regret to all of Asia.

Last week, Kan offered ''deep remorse'' in an apology issued ahead of the 100th anniversary of the
Japanese annexation of the Korean peninsula on Aug.~29, 1910.

''We caused great damage and suffering to many nations during the war, especially to the people of
Asia,'' Kan said Sunday before a crowd of about 6,000, including Emperor Akihito, at Budokan hall.

''We feel a deep regret, and we offer our sincere feelings of condolence to those who suffered and
their families,'' Kan said.

Lee said history should not be forgotten, but that Kan's apology last week marked progress.

''I have taken note of Japan's effort, which represents one step forward,'' Lee said.

''However, there still remain issues that have to be resolved,'' he said, without elaborating. ''The
two countries are called upon to take concrete measures to forge a new relationship for another 100
years.''

Many older Koreans still harbor resentment against Japan over the colonization. Hundreds of
thousands of Koreans were forced to fight as front-line soldiers, work in slave-labor conditions or
serve as prostitutes called ''comfort women'' in brothels operated by the military.

Later Sunday, about 50 women rallied in front of the Japanese Embassy in Seoul, chanting slogans and
demanding compensation for former comfort women and other Korean victims of colonial rule.

Reflecting a common sentiment among Koreans, activist Lee Kang-sil criticized Japan's apology as
''lacking in action.''

Such hard feelings were also evident in China, where about 300 people gathered in the eastern city
of Nanjing, to remember the victims of the 1937 ''Nanjing Massacre,'' known in the West as the
''Rape of Nanking,'' a ramgage by Japanese troops that many historians generally agree ended with
the slaughter of at least 150,000 civilians and disarmed soldiers and the rape of tens of thousands
of women. The estimates are debated, with China saying the figures are far higher, while some
Japanese historians claim they are lower.

In Australia, World War II veterans and representatives from New Zealand, the U.S.~and Asian
countries were among more than 300 people gathered in downtown Sydney to mark the anniversary.

The group placed wreaths at the foot of the Cenotaph war memorial to mark Japan's surrender and
observed a minute of silence.

More than 27,000 Australians were killed or died as prisoners of war during World War II.

New South Wales Returned and Services League President Don Rowe said Australians at home and
overseas were fighting for victory and peace.

''And when peace came some 65 years ago today, it was also a sad time for many, many families whose
loved ones never returned,'' Rowe said. ''So today, ladies and gentlemen, we remember that victory
but we also remember those who laid down their lives.''

Kan and his Cabinet broke from the past by staying away from Yasukuni Shrine, while members of the
opposition continued with their visits, including Liberal Democratic leader Sadakazu Tanigaki and
former Prime Minister Shinzo Abe.

The national Mainichi newspaper expressed hopes for a world without nuclear weapons, highlighted by
Obama's promise to work toward nuclear disarmament.

''We must never repeat the tragedy of war, and we must continue to build peace. This anniversary
should be a time for each of us to reflect,'' it said in an editorial Sunday.

Memorials were held earlier this month in Hiroshima and Nagasaki, the two Japanese cities devastated
by U.S.~nuclear attacks at the end of World War II.

U.S.~Ambassador John Roos became the first official U.S.~representative to attend the Hiroshima
commemoration this year. Hopes are high Obama will also go to Hiroshima during his trip to Japan set
for later this year.

At Sunday's ceremony, Akihito led a moment of silence at noon, bowing before a stage filled with
yellow and white chrysanthemums.

It was the his father Hirohito's radio broadcast 65 years ago that announced the end of World War II
-- the first time the Japanese public had ever heard the voice of the emperor, who had been revered
as a living god.

''I feel once again a deep sadness for those many who lost their precious lives and for their
families,'' Akihito said. ''I pray for the continued prosperity of our nation and for world peace.''

\section{Japan, Checking on Its Oldest, Finds Many Gone}

\lettrine{J}{apan}\mycalendar{Aug.'10}{16} has long boasted of having many of the world's oldest
people -- testament, many here say, to a society with a superior diet and a commitment to its
elderly that is unrivaled in the West.

That was before the police found the body of a man thought to be one of Japan's oldest, at 111
years, mummified in his bed, dead for more than three decades. His daughter, now 81, hid his death
to continue collecting his monthly pension payments, the police said.

Alarmed, local governments began sending teams to check on other elderly residents. What they found
so far has been anything but encouraging.

A woman thought to be Tokyo's oldest, who would be 113, was last seen in the 1980s. Another woman,
who would be the oldest in the world at 125, is also missing, and probably has been for a long time.
When city officials tried to visit her at her registered address, they discovered that the site had
been turned into a city park, in 1981.

To date, the authorities have been unable to find more than 281 Japanese who had been listed in
records as 100 years old or older. Facing a growing public outcry, the country's health minister,
Akira Nagatsuma, said officials would meet with every person listed as 110 or older to verify that
they are alive; Tokyo officials made the same promise for the 3,000 or so residents listed as 100
and up.

The national hand-wringing over the revelations has reached such proportions that the rising toll of
people missing has merited daily, and mournful, media coverage. ``Is this the reality of a longevity
nation?'' lamented an editorial last week in The Mainichi newspaper, one of Japan's biggest dailies.

Among those who officials have confirmed is alive: a 113-year-old woman in the southern prefecture
of Saga believed to be the country's oldest person, at least for now.

The soul-searching over the missing old people has hit this rapidly graying country -- and tested
its sense of self -- when it is already grappling with overburdened care facilities for the elderly,
criminal schemes that prey on them and the nearly daily discovery of old people who have died alone
in their homes.

For the moment, there are no clear answers about what happened to most of the missing centenarians.
Is the country witnessing the results of pension fraud on a large scale, or, as most officials
maintain, was most of the problem a result of sloppy record keeping? Or was the whole sordid affair,
as the gloomiest commentators here are saying, a reflection of disintegrating family ties, as an
indifferent younger generation lets its elders drift away into obscurity?

``This is a type of abandonment, through disinterest,'' said Hiroshi Takahashi, a professor at the
International University of Health and Welfare in Tokyo. ``Now we see the reality of aging in a more
urbanized society where communal bonds are deteriorating.''

Officials here tend to play down the psychosocial explanations. While some older people may have
simply moved into care facilities, they say, there is a growing suspicion that, as in the case of
the mummified corpse, many may already have died.

Officials in the Adachi ward of Tokyo, where the body was found, said they grew suspicious after
trying to pay a visit to the man, Sogen Kato. (They were visiting him because the man previously
thought to be Tokyo's oldest had died and they wished to congratulate Mr.~Kato on his new status.)

They said his daughter gave conflicting excuses, saying at first that he did not want to meet them,
and then that he was elsewhere in Japan giving Buddhist sermons. The police moved in after a
granddaughter, who also shared the house, admitted that Mr.~Kato had not emerged from his bedroom
since about 1978.

In a more typical case that took place just blocks from the Mr.~Kato's house, relatives of a man
listed as 103 years old said he had left home 38 years ago and never returned. The man's son, now
73, told officials that he continued to collect his father's pension ``in case he returned one
day.''

``No one really suspects foul play in these cases,'' said Manabu Hajikano, director of Adachi's
resident registration section. ``But it is still a crime if you fail to report a disappearance or
death in order to collect pension money.''

Some health experts say these cases reflect strains in a society that expects children to care for
their parents, instead of placing them in care facilities. They point out that longer life spans
mean that children are called upon to take care of their elderly parents at a time when the children
are reaching their 70s and are possibly in need of care themselves.

In at least some of the cases, local officials have said, an aged parent disappeared after leaving
home under murky circumstances. Experts say that the parents appeared to have suffered from dementia
or some other condition that made their care too demanding, and the overburdened family members
simply gave up, failing to chase after the elderly people or report their disappearance to the
police.

While the authorities have turned up a large number of missing centenarians, demographic experts say
they doubt that discoveries of the living or the dead would have much impact on Japan's vaunted life
expectancy figures; the country has the world's highest life expectancy -- nearly 83 years --
according to the World Bank. But officials admit that Japan may have far fewer centenarians than it
thought.

``Living until 150 years old is impossible in the natural world,'' said Akira Nemoto, director of
the elderly services section of the Adachi ward office. ``But it is not impossible in the world of
Japanese public administration.''

\section{Tensions Over Chinese Mining Venture in Peru}

\lettrine{I}{n}\mycalendar{Aug.'10}{16} its worldwide quest for commodities, China has scoured South
America for everything from Brazilian soybeans to Guyanese timber and Venezuelan oil. But long
before it made any of those forays, China put down stakes in this desolate mining town in Peru's
southern desert.

The year was 1992. Chinese companies had begun to look abroad. One steelmaker, the Shougang
Corporation of Beijing, set its sights on an iron ore mine here and bought it in a move that seemed
particularly bold. At the time, Peru was still plagued by attacks by the Maoist guerrillas of the
Shining Path.

But the hero's welcome for Shougang soon faded. Workers at the mine, which was founded by Americans
in the 1950s and nationalized by leftist generals in the 1970s, began fomenting the unexpected: a
revolt that has endured to this day, marked by repeated strikes, clashes with the police and even
arson attacks against their nominally Communist bosses from China.

``We quickly realized that we were being exploited to help build the new China, but without seeing
any of the rewards for doing so,'' said Honorato Quispe, 63, a longtime union official at the mine,
where workers have held three strikes this year alone, including an 11-day stoppage last month.

The long-festering conflict with Shougang over wages, environmental pollution and Shougang's
treatment of residents of this company town does not square well with China's celebratory vision of
its rising profile in Latin America, in which everyone benefits and a ``win-win'' is ``the
consensus.'' Latin America, as this idea of so-called South-South cooperation goes, sells China raw
materials like copper, oil or iron; in return, the region buys goods like cellphones, cars and cheap
plastic toys.

The tension in Marcona, one of the most conflict-ridden towns in a country increasingly prone to
conflict over mining and energy projects, suggests that China's engagement in the region -- like
that of the United States, Britain and other powers that preceded it in Latin America -- is not
without pitfalls.

While not the dominant theme in the region's relations with China, a wariness is crystallizing in
some countries over the booming trade with China.

Reactions to this surge largely focus on cheap Chinese imports or on China's assertive efforts to
win access to energy reserves. In both Brazil and Argentina, for instance, manufacturers accused
Chinese companies of unfairly dumping Chinese products in their markets, prompting new tariffs
against some Chinese imports.

But perhaps nowhere in the region has wariness and regret over Chinese investment coalesced as much
as in Marcona. With about 15,000 residents, it still has the look of a mining town in the American
Southwest, a legacy of its construction in the 1950s by engineers from the United States.

The Americans are long gone, but the Chinese managers now live in the same ranch-style houses built
for their predecessors in a district called Playa Hermosa (Beautiful Beach). They drive sport
utility vehicles and talk to subordinates through translators. They eat meals at their own
cafeteria, avoiding mixing with Peruvians in town.

Workers here said the problems with Shougang began in the 1990s, when the company slashed the mine's
work force to 1,700 from 3,000 and brought in some Chinese workers. Resistance in the form of
strikes soon convinced the managers to return their workers to China.

Resentment also emerged when Shougang did not invest a promised \$150 million in the mine and the
town's infrastructure, opting instead to pay a \$14 million fine for failing to do so, and left
blocks of housing once occupied by workers vacant in a town with an acute housing shortage.

At a union building, workers spoke of low wages and company resistance to enacting
government-mandated raises, and they claimed that Shougang had dumped chemical waste into the sea.

On the other side of Marcona from Playa Hermosa, some workers at the mine live in bleak company
housing. Others rent squalid rooms in the town. A lower class of squatters subsists on Marcona's
edge in a driftwood shantytown, Ruta del Sol.

``The Chinese see us as little more than slaves,'' said Hermilia Zamudio, 58, a resident of Ruta del
Sol, whose husband was fired from the mine after working there for almost 30 years. ``They deem it
beneath them to talk to us, and when they need to address problems here, they do so with their
thugs.''

Clashes with private security guards and with the police, who receive a monthly stipend paid by
Shougang, are common in Ruta del Sol, on land where Shougang says it has concessionary rights to
exploit deposits of dolomite, a mineral it hopes to extract for smelting iron and steel.

At one clash last year, Wilber Huaman\~{n}ahui, 21, a construction worker, was shot dead as he and
dozens of others tried to take possession of land controlled by Shougang. The case remains unsolved.
``I know there will never be justice for his killing,'' said his widow, Zoila Benites, 18.

Elected officials here still express dismay over the inability to punish those responsible for
Mr.~Huaman\~{n}ahui's killing. ``We think there's an effort by judicial authorities to delay the
process for four or five years until the matter is forgotten,'' Joel Rosales, the mayor of San Juan
de Marcona, said this month.

Shougang, which keeps its Chinese managers cloistered away from the news media, has generally
responded to such statements with silence. An effort to approach Chinese executives at their private
cafeteria here was met by a threat of forceful expulsion by a guard.

Raúl Vera la Torre, a Peruvian executive for Shougang who handles relations with the government and
journalists, acknowledged in an interview in Lima that the company faced complaints over issues like
the housing shortage, water scarcity and expulsions of squatters. He contended this month that
Shougang had carried out projects to improve the quality of life in the town, like providing potable
water to many residents.

Still, he said, ``a company cannot take on duties that are those of a government.''

For now, Shougang seems prepared to manage from crisis to crisis. The mine here has been the focus
of one to four significant strikes annually in recent years, according to Evan Ellis, a specialist
in Chinese-Latin American relations at the Center for Hemispheric Defense Studies in Washington.

Mr.~Vera la Torre, Shougang's Peruvian executive, said he preferred to focus on Marcona's potential.
Pointing to China's long-term view, he said Shougang planned to invest \$1 billion to raise
production to 18 million tons of iron ore by 2012 from 8 million tons today.

Geography blessed Marcona, he said, with a location at the end of a planned highway link to Brazil.
Others are also eyeing Marcona's location, including an American fertilizer manufacturer that plans
to build a \$1 billion plant here. Large ships could easily dock in a nearby port, which Shougang
also owns.

But Marcona's workers suggest that unlocking that potential could do little to ease tension here.

``After nearly two decades of this experiment, the answer is no,'' said F\'elix Díaz, 66, a senior
union official. ``When the Chinese arrived, they talked about things like solidarity and the
equality of man. If this is the brotherhood they praise, then one day sooner or later, the Chinese
must be made to leave.''

\section{A Better Way to Keep the Net Open and Accessible}

\lettrine{N}{eutrality}\mycalendar{Aug.'10}{16} has been great for Switzerland -- and it could be
for the Internet, too, say supporters of the idea that broadband providers should give equal
priority to all digital traffic, from e-mail to bandwidth-hungry video.

U.S.~Web companies and consumer groups say that so-called network neutrality is essential to keeping
the Internet open and wondrous. Without rules guaranteeing neutrality, they add, telecommunications
providers might erect online tollbooths or obstruct Internet traffic in other ways.

Network operators argue that technology companies are getting a free ride; without the power to
manage an ever-increasing flow of digital data, they say, the Internet will grind to a halt.

So, when news emerged of a nonaggression pact between Google, the biggest Internet company, and
Verizon, a U.S.~network operator, in which they endorsed some kinds of digital traffic management,
the Internet erupted with calls for government action to guarantee network neutrality.

But instead of demanding new regulations, U.S.~advocates of network neutrality ought to take a look
abroad. They might find that there is another way to ensure that the Internet remains accessible,
one that is more consistent with American laissez-faire business principles: competition.

Across much of Europe, consumers can choose among dozens of broadband providers, offering faster and
less expensive Internet access than is available to most Americans. The situation is similar in
Australia and in some advanced broadband markets in Asia, like Hong Kong and Singapore. Consumers
who are unhappy with their broadband providers -- if, for example, they suspect that their Internet
use is not getting priority treatment -- can simply switch.

In the United States, by contrast, many consumers can choose between only two broadband providers --
one offering service over the phone lines, the other via cable. Others have no choice at all.
U.S.~regulators, unlike their counterparts elsewhere, have not generally required broadband
providers to open their networks to competitors.

This fuels suspicions about the intentions of broadband providers, as well as the demands for
network neutrality.

In Australia, by contrast, broadband providers openly flout network neutrality, and few people
object. Broadband providers there manage traffic by putting caps on some kinds of Web use and by
encouraging consumers to favor the services of selected Internet companies.

The debate has also been more muted in Europe. Yes, the European Commission recently opened a public
consultation on the matter, and regulators in some European countries, like Britain and France, have
followed suit.

But comments from European regulators suggest that there is little enthusiasm for neutrality
mandates.

``Consumers should be able to access the content they want,'' Neelie Kroes, the European
commissioner in charge of digital policies, said in introducing the consultation. ``Content
providers and operators should have the right incentives to keep innovating. But traffic management
and net neutrality are highly complex issues. I do not assume that one approach or another should
prevail.''

In opening its study, the British Internet regulator, Ofcom, said, ``We believe that there is
insufficient evidence at present to justify the setting of blanket restrictions on all forms of
traffic management.''

The Google-Verizon proposal acknowledges the power of competition, arguing that there is no need for
neutrality on the wireless Internet because there is more rivalry among U.S.~mobile network
operators than among fixed-line broadband providers. (By contrast, Google and Verizon call for a
neutrality mandate on the fixed-line Internet, with the exception of vaguely defined new broadband
services.)

Cynics say this is just a convenient carving up that would protect both companies' vested interests.
But it highlights a bigger question: if competition is so effective, why not give the U.S.~broadband
business a taste of it, and leave neutrality to the Swiss?

\section{Double Dip? A Tipping Point May Be Near}

\lettrine{L}{ike}\mycalendar{Aug.'10}{16} a car spinning its wheels, the American economy hasn't
been getting much traction. Many financial indicators are issuing worrisome signals, millions of
people are still out of work, and growth is slowing.

Will the economy pick up momentum or slip back into recession? Unfortunately, the answer is very
much in doubt.

``We are at a very critical moment in the business cycle,'' said Lakshman Achuthan, the managing
director of the Economic Cycle Research Institute, a private forecasting group with an excellent
track record. After the economy began to recover last summer, in his estimation, ``growth has
definitely slowed.'' But he said he wouldn't have enough data until at least the fall to know
``whether we're dipping back into recession.''

Even the Federal Reserve chairman, Ben S.~Bernanke, calls the current economic outlook ``unusually
uncertain.'' And the financial markets have certainly noticed.

``The market is favoring investments thought to have no risk, like Treasuries, as opposed to assets
with risk, like stocks -- and that's been becoming more and more pronounced,'' said William
H.~Gross, co-chief investment officer of the Pacific Investment Management Company, or Pimco, the
big bond manager.

In the bond market last week, safety-seeking investors bid up the prices of Treasury securities,
driving down yields to extraordinarily low levels along a broad range of maturities -- and extending
the astonishing rally of long-term Treasury bonds, which continue to outperform the stock market and
nearly every other asset class.

For the year, for example, long-term Treasuries have returned more than 15 percent, as gauged by the
Barclays Capital U.S.~20+ Year Treasury Bond index. By comparison, the Standard \& Poor's 500-stock
index has lost more than 2 percent, including dividends. And over the decade through July, long-term
Treasuries outperformed the S.\& P.~500 by more than 8.6 percentage points, annualized, according to
Morningstar data.

The stock market was particularly shaky last week. On Wednesday, the Dow Jones industrial average
dropped 265.42 points, or 2.5 percent, its biggest decline since June. For the week, the Dow lost
more than 350 points, or 3.3 percent, closing at 10,303.15, while the S.\& P.~fell more than 42
points, or 3.8 percent, to 1,079.25.

This preference for bonds over stocks, which has emerged periodically since the financial crisis
began in 2008, came into high relief after the Fed's announcement on Tuesday that it would recycle
the proceeds of its mortgage bond portfolio into Treasuries.

Considering the Fed balance sheet -- now approximately \$2.3 trillion, up from \$900 billion before
the financial crisis -- the central bank's latest move was modest. It will translate into \$200
million to \$400 million in purchases of Treasuries over 12 months, according to an estimate by Bank
of America Merrill Lynch. But it was the direction the Fed was taking -- its decision to maintain an
expansive monetary policy through ``quantitative easing,'' rather than to begin monetary tightening
-- that was most striking.

``Ben Bernanke's motto seems to be that he will do whatever it takes to avoid a depression and avert
deflation,'' said Edward Yardeni, president of Yardeni Research. ``He has been very successful so
far,'' lowering interest rates along the yield curve and thereby stimulating the economy.

The Fed may need to go it alone over the next several months, several analysts said, because with
midterm elections approaching, additional fiscal stimulus legislation may be too controversial.

Mr.~Achuthan of the Economic Cycle institute said a weakening in the economy's growth rate began to
appear in the institute's leading indicators early in the winter. ``Now, that slowdown is baked in
the cake, it's already happening,'' he said.

The Fed acknowledged as much last week. In a statement on Tuesday, its Open Market Committee said
that the recovery pace had ``slowed'' and that growth ``is likely to be more modest in the near term
than had been anticipated.''

It's too early to know whether the economy will tip into an actual decline, Mr.~Achuthan said. The
institute's Weekly Leading Index ticked up again on Friday, after a steep plunge through late June.
It's possible that the Fed ``will be able to use its influence'' to engineer a soft landing and
avert another, immediate recession, he said. In September or October, he said, there may be more
clarity.

MR. GROSS of Pimco said the growing American trade deficit -- it jumped to \$49.9 billion in June,
its highest level since October 2008, according to a report on Wednesday -- is cutting into the
G.D.P., ``implying that second-quarter G.D.P. is going to be in the 1 to 2 percent range.'' In turn,
he said, ``that implies that the momentum of the economy from the first to the second quarter was
downhill, and therefore we think ``it's quite possible that we might actually be somewhere between
zero and 1 percent -- in other words, very close to a double-dip recession.''

He put the probability of a recession -- and of an accompanying bout of deflation -- at 25 to 35
percent.

Still, the economic signs are ambiguous. After falling in the spring, industrial commodity prices
have risen, suggesting that there is still surging demand from manufacturers. Shipping indexes,
which had declined, have ticked upward again over the last month. Mortgage rates are dropping. And
thanks in large part to fierce cost-cutting, very low interest rates, and strong demand in
developing countries, corporate profits have been strong.

What's been lacking is broad consumer demand, a revival of the housing market and sufficient
business confidence in large-scale hiring. And, of course, there are deep structural economic
problems -- the highest ratio of public debt to gross domestic product since World War II, for
example -- that will need to be dealt with over many years.

Nonetheless, if there is good news in the fall -- if it becomes clear that the pace of recovery has
begun to accelerate -- there could easily be another burst of stock market exuberance, Mr.~Achuthan
said. ``Interest rates are very low,'' he said, ``and if you were to combine that with economic
growth, you've got a wonderful recipe for profits.''

In the meantime, Mr.~Gross said, with the Fed funneling money into Treasuries, the improbable rally
in the bond market may still have a way to go.

\section{Apple Employee Accused of Accepting Kickbacks}

\lettrine{A}{}\mycalendar{Aug.'10}{16} manager at Apple has been indicted by a federal grand jury,
accused of taking kickbacks in a scheme involving suppliers of iPhone and iPod accessories.

The manager, Paul Shin Devine, of Sunnyvale, Calif., was arrested on Friday and charged with wire
fraud, money laundering and taking kickbacks, the authorities said. The indictment had been filed
under seal on Wednesday.

Mr.~Devine, 37, is accused of accepting more than \$1 million in exchange for providing confidential
information to Apple suppliers in Asia. The suppliers are thought to have used the information to
negotiate favorable contracts with Apple.

``Apple is committed to the highest ethical standards in the way we do business,'' a company
spokesman, Steve Dowling, said in a statement. ``We have zero tolerance for dishonest behavior
inside or outside the company.''

The authorities say a Singapore resident, Andrew Ang, was one of the suppliers involved. He is also
named in the federal indictment.

The authorities have declined to comment on Mr.~Ang's whereabouts.

Mr.~Devine is being held by federal marshals and is scheduled to appear in federal court in San
Jose, Calif., on Monday afternoon.

\section{Innovate, Yes, but Make It Practical}

\lettrine{B}{usiness}\mycalendar{Aug.'10}{16} is a field not of theory but of practice. The central
intellectual inquiry of the science of management is simply this: What works?

That, it seems, is the best way to examine the steady rise in the practice of innovation management.
A search of the database of the professional networking site LinkedIn found that more than 700
people listed their current job title as ``chief innovation officer'' and that nearly 25,000 had the
word ``innovation'' in their job title. Many others may not have the word in their titles, but their
job is to pursue opportunities that result in new products, services and more efficient ways of
doing things.

So what does work in the innovation game? No single formula, to be sure. But some recent interviews
with executives, consultants and academics can be distilled into three recommendations: think
broadly, borrow from the entrepreneurial Silicon Valley model, and pay close attention to customers
and to emerging user needs.

Here, then, are three innovation works in progress that include those ingredients, whether or not
the efforts will ultimately prove to be winners:

\emph{$\bullet$ Marching Into New Markets}

John Tao joined Weyerhaeuser, the wood and pulp producer, two years ago as its vice president for
open innovation, coming from Air Products and Chemicals. At Weyerhaeuser, Mr.~Tao has led an
initiative to find new markets for lignin, a chemical compound that binds cellulose fibers together
in trees. Lignin is extracted during pulp-making as a black liquor, and is typically recycled as a
fuel for pulp plants.

Yet lignin can also be converted to a solid and serve as a chemical feedstock for making a range of
products. Mr.~Tao, a Ph.D. chemical engineer, and his staff studied the market, including the curbs
on carbon emissions that chemical producers will likely face in the future.

Lignin can be a nonpolluting alternative for producing goods as different as seat cushions and
carbon fiber. Automakers, for example, are beginning to use carbon fiber as a lightweight but strong
substitute for metal to improve fuel efficiency.

As a chemical feedstock, lignin is worth 10 to 20 times its value as a pulp-plant fuel, Mr.~Tao
said. Weyerhaeuser has a pilot plant in North Carolina to produce specialized lignin chemicals.
Mr.~Tao has met with chemical companies, carbon fiber makers and the Department of Energy to try to
nurture new lignin markets. ``You have to have some technical background,'' he said, ``but a lot of
this work is market analysis, communications and networking with industry partners.''

\emph{$\bullet$ Customized Discounts}

For innovation champions, titles matter far less than their independence, breadth of knowledge and
corporate clout, experts say. ``Whatever you call it, there is a real need for a senior-level
executive to be able to reach across a company and beyond to tap ideas, skills and resources,'' said
Henry Chesbrough, executive director of the Center for Open Innovation at the University of
California, Berkeley. ``It is this systems integration aspect that is central to innovation as a
field and a discipline.''

Money helps too. Rick Rommel, a senior vice president of the new-business group at Best Buy, says
his unit has ``an internal venture capital mind-set.'' Best Buy gave his group additional financing
this year to sharply increase investments in experimental ventures that, he said, ``explore what
customers think and what technologies are ready for widespread adoption.''

The new-business group has been working with a start-up, Shopkick, which is introducing an
application for iPhones, and later for other smartphones, that retailers can use to track when
shoppers have entered a store and reward them with discounts.

When linked to other online browsing and buying data, the discount offers can be not only immediate,
when a person is in the store, but also tailored to individual interests. A person who has browsed
computer Web sites, for example, might be offered a 10 percent discount on a notebook computer.

``This really moves toward one-to-one marketing,'' Mr.~Rommel said.

\emph{$\bullet$ Banks of the Future}

At Citigroup, Deborah Hopkins, chief innovation officer, is also in charge of the bank's venture
investing arm. This year, she decided to move from New York to Silicon Valley to be close to its
entrepreneurial networks. ``It's a small community out there,'' she explained.

One Citigroup investment is in Bundle.com, a social media start-up where users can compare their
spending and saving habits with those of others. The idea came from the Citigroup innovation unit,
and Bundle's C.E.O., Jaidev Shergill, came from Citigroup. The other investors in Bundle are
Microsoft and Morningstar. ``The whole social networking phenomenon is moving so fast, and we need
to be invested in some way,'' said Don Callahan, Citigroup's chief administrative officer, who
oversees the innovation unit. ``Whatever the outcome, we're going to learn a lot.''

Ms.~Hopkins sees her role as ``being a catalyst, to challenge people to think differently, but also
pursue new ideas with a lot of rigor.'' An example of that systematic approach to innovation is
Citigroup's ``bank of the future'' project. The first two redesigned bank branches opened in April
in Japan, but the concepts will eventually be transplanted to America, tailored to local markets.

The overhaul began with a shift in mind-set, from one oriented around banking products to one
focused on customers. Months of extensive customer and demographic research resulted in personality
profiles of four customer types, from up-and-comers in their 30s to retiring baby boomers. Customer
service and marketing were geared toward those four affluent groups.

The branches have been remade as digital banks, with touch-screen work stations and
videoconferencing links to financial experts. Traditional banks have up to 100 paper forms, while
the redesigned branches are almost paperless, says Darren Buckley, president of Citibank Japan. The
design imprint of Eight Inc., a firm that worked on Apple's stores, is evident in the open,
minimalist interiors of the new branches.

``We're incubating ideas, but what we're doing in Japan is absolutely something that can be scaled
out elsewhere,'' said Chris Kay, a managing director of Citigroup's innovation arm.

\section{Activists Take Fight on Immigration to Border}

\lettrine{N}{o}\mycalendar{Aug.'10}{16} migrant would have dared cross from Mexico into this
particular stretch of Arizona on Sunday.

Hundreds of Tea Party activists converged on the border fence here in what is typically a desolate
area popular with traffickers to rally for conservative political candidates and to denounce what
they called lax federal enforcement of immigration laws. The rally brought a significant law
enforcement presence as well as numerous private patrols by advocates of a more secure border.

But rallies, even daylong ones, are no way to seal the border. The Obama administration insists that
its statistics show that significant financing increases in the federal Border Patrol have helped
bring down crime at the border and make the smuggling of immigrants and drugs harder than ever.

But the activists who gathered Sunday had a decidedly different take. The border, in their view, is
still far too easy to get across and has become so dangerous that some of them brought their
sidearms for protection. Organizers urged participants to leave rifles in their cars.

``Instead of finding bugs in our beds, we're finding home invaders,'' said Tony Venuti, a Tucson
radio host who attached a huge sign to the fence that told immigrants to head to Los Angeles, where
they will be more welcome, and even offered directions for getting there.

Addressing the crowd, Sheriff Joe Arpaio, who conducts controversial sweeps in immigrant
neighborhoods in Phoenix and other parts of Maricopa County, said the problem could be solved if the
Border Patrol was given permission to track down migrants on the Mexican side before they crossed.

``If I had all the national TV here, I'd probably climb the fence to show you how easy it is,''
Sheriff Arpaio said from the rally's stage, a flag with the words ``Don't Tread on Me'' flapping
behind him.

Also among the speakers was Russell Pearce, the state senator who sponsored Arizona's controversial
immigration law known as 1070, part of which was blocked by a federal judge last month.

The event was monitored on the Mexican side. A rally participant spotted a group of people in the
rugged terrain in Mexico and alerted Border Patrol officers, who identified them with binoculars as
members of Grupo Beta, a Mexican agency that aids migrants in distress.

Sheriff Larry A.~Dever of Cochise County, where the event was held, said the area was a hotspot for
traffickers.

``We're right at the point of the spear where human and dope smuggling takes place,'' Sheriff Dever
said. ``These mountains are a beehive of activity.''

He said he had no doubt that migrants and drug smugglers were using lookouts to keep track of the
rally.

``They know this rally is going on,'' he said. ``They are not fools. They're experts. They probably
know more about this than we do standing here.''

J.~D. Hayworth, who is challenging Senator John McCain in the Republican primary to be held later
this month, used the event to question Mr.~McCain's commitment to fighting illegal immigration.
Trying to outflank Mr.~Hayworth, Mr.~McCain has made several stops in the border region recently.

The Obama administration has similarly started a defense of its border policies in recent days.

``Is there more work to be done? Absolutely. Is the problem a significant one, a challenging one for
the nation? Absolutely,'' John T.~Morton, director of federal Immigration and Customs Enforcement,
said in Phoenix last week, vowing that his agency was committed to securing the border.

The rally was held on private land, not far from where a popular Arizona rancher died in late March
in a killing that helped fuel the immigration debate in the state.

Cindy Kolb, a border activist who lives nearby, yelled out through the thick metal slates in the
border fence, which had been decorated on the American side with tiny flags, ``Hey, don't come over
here anymore.''

She added: ``We don't like illegals hiding under bushes when our kids wait for the school bus. This
border needs to be secure.''

\section{Pharmacists Take Larger Role on Health Team}

\lettrine{E}{loise}\mycalendar{Aug.'10}{16} Gelinas depends on a personal health coach.

At Barney's Pharmacy, her local drugstore in Augusta, Ga., the pharmacist outlines all her
medications, teaching her what times of day to take the drugs that will help control her diabetes.

Ms.~Gelinas, a retired nurse, also attends classes at the store once a month on how to manage her
disease with drugs, diet and exercise. Since she started working with the Barney's pharmacists, she
boasts that her blood sugar, bad cholesterol and blood pressure have all decreased. ``It's my home
away from home,'' she says.

While some of the services being offered to Ms.~Gelinas resemble those found in an old-fashioned
neighborhood drugstore, others reflect the expanding role of the nation's pharmacists in ways that
may benefit their customers and also represent a new source of revenue for the profession. Some
health plans are even paying pharmacists to monitor patients taking regular medications for chronic
illnesses like diabetes or heart disease.

``We are not just going to dispense your drugs,'' said David Pope, a pharmacist at Barney's. ``We
are going to partner with you to improve your health as well.''

At independent drugstores and some national chains like Walgreens and the Medicine Shoppe and even
supermarkets like Kroger, pharmacists work with doctors and nurses to care for people with long-term
illnesses.

They are being enlisted by some health insurers and large employers to address one of the
fundamental problems in health care: as many as half of the nation's patients do not take their
medications as prescribed, costing nearly \$300 billion a year in emergency room visits, hospital
stays and other medical expenditures, by some estimates.

The pharmacists represent the front line of detecting prescription overlap or dangerous interaction
between drugs and for recommending cheaper options to expensive medicines. This evolving use of
pharmacists also holds promise as a buffer against an anticipated shortage of primary care doctors.

``We're going to need to get creative,'' said Dr.~Andrew Halpert, senior medical director for Blue
Shield of California, which has just begun a pilot program with pharmacists at Raley's, a local
grocery store chain, to help some diabetic patients in Northern California insured through the
California Public Employees' Retirement System.

Like other health plans, Blue Shield views pharmacists as having the education, expertise, free time
and plain-spoken approach to talk to patients at length about what medicines they are taking and to
keep close tabs on their well-being. The pharmacists ``could do as well and better than a
physician'' for less money, Dr.~Halpert said.

Some health insurers and large employers already pay for programs called medication therapy
management, which typically involve face-to-face sessions between pharmacists and patients in retail
stores or clinics. Pharmacists can be paid to track patients, monitoring cholesterol or blood
glucose levels, for example, or prodding customers to change their diets or exercise. UnitedHealth
Group has recently started working with pharmacists and health coaches at the Y.M.C.A. to counsel
diabetic patients.

The idea of using pharmacists in this way began to gain popularity in 2006 when some Medicare plans
started covering medication therapy management programs, paying \$1 to \$2 a minute to pharmacists
to review patients' medicines with them; this year, about one in four people covered by Medicare
Part D prescription drug plans will be eligible, according to agency estimates. For example, a
Medicare Part D plan covered Ms.~Gelinas's medication management session at Barney's pharmacy.

More employers and insurers also pay for pharmacists to advise patients, a role that the new health
care law encourages with potential grants for such programs. In Wisconsin, for example, community
pharmacists and some health plans have banded together to create a joint program, the Wisconsin
Pharmacy Quality Collaborative, to standardize medication therapy management and ensure quality
care.

Meanwhile Humana, which first paid for pharmacists to work with Medicare patients, expanded its
coverage a few years ago. About a third of the 62,000 pharmacies in its network offer these
services, and the insurer says it is studying whether a pharmacist seeing a patient in person has
more impact than a phone call.

The advent of these services has spawned a new industry of medication therapy management companies
to run clinical pharmacy programs for health insurers, contracting with pharmacists and tracking the
financial and health outcomes of their services. One such company, Mirixa, founded in 2006 by the
National Community Pharmacists Association, does business with more than 40,000 pharmacies
nationwide. Pharmacists and others see these joint efforts as vital to remain competitive with
mail-order pharmacies.

One of the first places where retail pharmacists began to expand their role was Asheville, N.C.,
where studies validated the services. ``We really positioned the pharmacist as coach,'' said Fred
Eckel, executive director of the state's pharmacist group.

In one recent study of 573 people with diabetes, 30 employers in 10 cities waived co-payments for
diabetes drugs and supplies for those employees or family members willing to meet regularly with a
pharmacist. People in the study, financed by the drug maker GlaxoSmithKline, took part in at least
two sessions with pharmacists who helped them track their blood sugar, blood pressure and
cholesterol levels and offered diet and exercise advice. After a year, blood pressure, blood sugar
and cholesterol levels typically improved -- and saved an average \$593 a person on diabetes drugs
and supplies.

But the new relationships have stirred concerns. Federal regulators have recently accused chains
like Rite Aid and CVS Caremark of inadequately protecting health records.

And groups like the American Academy of Family Physicians, say pharmacists should be careful not to
usurp the physician's role. ``I'm concerned that people are thinking about this in terms of `either
or,' and that's the wrong approach,'' said Dr.~Lori J.~Heim, the academy's president. ``It's an
`and' approach.''

Michelle A.~Chui, an assistant professor at the University of Wisconsin School of Pharmacy, said
that pharmacists do not want to compete with doctors, but merely provide more information ``so the
physician has a more in-depth picture.''

Still, the pharmacy business benefits. Barry S.~Bryant, owner of Barney's in Augusta, said expanding
to include a wellness center where pharmacists hold medication management sessions and monthly
health classes attracted more customers.

Today, Barney's fills an average of 1,000 prescriptions a day, up from 300 seven years ago, with
about a third of his customers covered by Medicaid and another third by Medicare, he said.

The business growth at Barney's has even prompted Mr.~Bryant and Mr.~Pope to start their own
education company, CreativePharmacist.com, that teaches other pharmacies how to introduce in-store
services.

``When we get involved with chronic care patients, their outcomes improve,'' Mr.~Pope said. ``But,
at the same time, they are improving our bottom line.''

\section{In Bold Display, Taliban Order Stoning Deaths}

\lettrine{T}{he}\mycalendar{Aug.'10}{17} Taliban on Sunday ordered their first public executions by
stoning since their fall from power nine years ago, killing a young couple who had eloped, according
to Afghan officials and a witness.

The punishment was carried out by hundreds of the victims' neighbors in a village in northern Kunduz
Province, according to Nadir Khan, 40, a local farmer and Taliban sympathizer, who was interviewed
by telephone. Even family members were involved, both in the stoning and in tricking the couple into
returning after they had fled.

Mr.~Khan said that as a Taliban mullah prepared to read the judgment of a religious court, the
lovers, a 25-year-old man named Khayyam and a 19-year-old woman named Siddiqa, defiantly confessed
in public to their relationship. ``They said, 'We love each other no matter what happens,' ''
Mr.~Khan said.

The executions were the latest in a series of cases where the Taliban have imposed their harsh
version of Shariah law for social crimes, reminiscent of their behavior during their decade of
ruling the country. In recent years, Taliban officials have sought to play down their bloody
punishments of the past, as they concentrated on building up popular support.

``We see it as a sign of a new confidence on the part of the Taliban in the application of their
rules, like they did in the '90s,'' said Nader Nadery, a senior commissioner on the Afghanistan
Independent Human Rights Commission. ``We do see it as a trend. They're showing more strength in
recent months, not just in attacks, but including their own way of implementing laws, arbitrary and
extrajudicial killings.''

The stoning deaths, along with similarly brazen attacks in northern Afghanistan, were also a sign of
growing Taliban strength in parts of the country where, until recently, they had been weak or
absent. In their home regions in southern Afghanistan, Mr.~Nadery said, the Taliban have already
been cracking down.

``We've seen a big increase in intimidation of women and more strict rules on women,'' he said.

Perhaps most worrisome were signs of support for the action from mainstream religious authorities in
Afghanistan. The head of the Ulema Council in Kunduz Province, Mawlawi Abdul Yaqub, interviewed by
telephone, said Monday that stoning to death was the appropriate punishment for an illegal sexual
relationship, although he declined to give his view on this particular case. An Ulema Council is a
body of Islamic clerics with religious authority in a region.

And less than a week earlier, the national Ulema Council brought together 350 religious scholars in
a meeting with government religious officials, who issued a joint statement on Aug.~10 calling for
more punishment under Shariah law, apparently referring to stoning, amputations and lashings.

Failure to carry out such ``Islamic provisions,'' the council statement said, was hindering the
peace process and encouraging crime.

The controversy could have implications for efforts by Afghan officials to reconcile with Taliban
leaders and draw them into power-sharing talks.

Afghan officials, supported by Western countries, have insisted that Taliban leaders would have to
accept the Afghan Constitution, which guarantees women's rights, and not expect a return to Shariah
law.

The stoning deaths were confirmed by Afghan officials in the area on Monday. Mahbubullah Sayedi, a
spokesman for the Kunduz governor's office, condemned the executions, and said there was ample
provision in Afghan law for prosecuting someone if they were accused of adultery or other social
crimes.

``We have courts here, and we can solve such cases through our judicial organizations,'' he said.
``This act is against human rights and against our national Constitution.''

The couple eloped when the man was unable to persuade family members to allow him to marry the young
woman. She was engaged to marry a relative of her lover, but was unwilling to do so, according to
Mr.~Khan.

Mohammed Ayub, the governor of nearby Imam Sahib district, also confirmed the stoning deaths, which
took place in the local bazaar in Mullah Quli village, in Archi district, a remote corner of Kunduz
Province close to Tajikistan.

The couple eloped to Kunar Province, in eastern Afghanistan, staying with distant relatives, but
family members persuaded them to return to their village, promising to allow them to marry. (Afghan
men are legally allowed to marry up to four wives). Once back in Kunduz, however, they were seized
by the Taliban, who convened local mullahs from surrounding villages for a religious court.

After the Taliban proclaimed the sentence, Siddiqa, dressed in the head-to-toe Afghan burqa, and
Khayyam, who had a wife and two young children, were encircled by the male-only crowd in the bazaar.
Taliban activists began stoning them first, then villagers joined in until they killed first Siddiqa
and then Khayyam, Mr.~Khan said. No women were allowed to attend, he said.

Mr.~Khan estimated that about 200 villagers participated in the executions, including Khayyam's
father and brother, and Siddiqa's brother, as well as other relatives, with a larger crowd of
onlookers who did not take part.

``People were very happy seeing this,'' Mr.~Khan maintained, saying the crowd was festive and
cheered during the stoning. The couple, he said, ``did a bad thing.''

A spokesman for the Taliban, Zabiullah Mujahid, praised the action. ``We have heard about this
report,'' he said, interviewed by cellphone. ``But let me tell you that according to Shariah law, if
someone commits a crime like that, we have our courts and we deal with such crimes based on Islamic
law.''

Mr.~Nadery, from the human rights commission, pointed to a string of recent such cases of summary
justice by the Taliban. In northwestern Badghis Province on Aug.~8, a 41-year-old widow, who was
made pregnant by a man she said promised to marry her, was convicted of fornication by a Taliban
court. She was given 200 lashes with a whip and then shot to death, according to Col.~Abdul Jabar, a
provincial police official, who said the killing was ordered by the local Taliban commander, Mullah
Yousef, in Qadis district.

President Hamid Karzai's spokesman, Waheed Omer, said: ````President Karzai was deeply saddened and
grieved when he heard that news. Nine years ago and we still see the Taliban doing events like that
in Badghis.''

Time magazine focused widespread indignation on Afghanistan recently by putting on its cover a
picture of an 18-year-old woman from Oruzgan Province whose nose and ears were cut off by her
Taliban husband after she had fled her child marriage to him.

Amnesty International condemned the latest stonings, calling them the first such executions since
the fall of the Taliban in 2001. ``The Taliban and other insurgent groups are growing increasingly
brutal in their abuses against Afghans,'' said Sam Zarifi, an Amnesty International official.

\section{U.S.~Said to Plan Easing Rules for Travel to Cuba}

\lettrine{T}{he}\mycalendar{Aug.'10}{17} Obama administration is planning to expand opportunities
for Americans to travel to Cuba, the latest step aimed at encouraging more contact between people in
both countries, while leaving intact the decades-old embargo against the island's Communist
government, according to Congressional and administration officials.

The officials, who asked not to be identified because they had not been authorized to discuss the
policy before it was announced, said it was meant to loosen restrictions on academic, religious and
cultural groups that were adopted under President George W.~Bush, and return to the ``people to
people'' policies followed under President Bill Clinton.

Those policies, officials said, fostered robust exchanges between the United States and Cuba,
allowing groups -- including universities, sports teams, museums and chambers of commerce -- to
share expertise as well as life experiences.

Policy analysts said the intended changes would mark a significant shift in Cuba policy. In early
2009, President Obama lifted restrictions on travel and remittances only for Americans with
relatives on the island.

Congressional aides cautioned that some administration officials still saw the proposals as too
politically volatile to announce until after the coming midterm elections, and they said revisions
could still be made.

But others said the policy, which does not need legislative approval, would be announced before
Congress returned from its break in mid-September, partly to avoid a political backlash from
outspoken groups within the Cuban American lobby -- backed by Senator Robert Menendez, Democrat of
New Jersey -- that oppose any softening in Washington's position toward Havana.

Those favoring the change said that with a growing number of polls showing that Cuban-Americans'
attitudes toward Cuba had softened as well, the administration did not expect much of a backlash.

``They have made the calculation that if you put a smarter Cuba policy on the table, it will not
harm us in the election cycle,'' said one Democratic Congressional aide who has been working with
the administration on the policy. ``That, I think, is what animates this.''

Mr.~Menendez, in a statement, objected to the anticipated changes. ``This is not the time to ease
pressure on the Castro regime,'' he said, referring to President Raúl Castro of Cuba, who took
office in 2006 after his brother, Fidel, fell ill. Mr.~Menendez added that promoting travel would
give Havana a ``much needed infusion of dollars that will only allow the Castro brothers to extend
their reign of oppression.''

In effect, the new policy would expand current channels for travel to Cuba, rather than create new
ones. Academic, religious and cultural groups are now allowed to travel under very tight rules. For
example, students wanting to study in Cuba are required to stay at least 10 weeks. And only
accredited universities can apply for academic visas.

Under the new policy, such restrictions would be eased, officials said. And academic institutions,
including research and advocacy groups and museums, would be able to seek licenses for as long as
two years.

In addition, the administration is also planning to allow flights to Cuba from more cities than the
three -- Miami, New York and Los Angeles -- currently permitted. And there are proposals, the
officials said, to allow all Americans to send remittances or charitable donations to churches,
schools and human rights groups in Cuba.

Some analysts said the measures were partly a response to pressure from an unlikely alliance of
liberal political groups and conservative business associations -- led by Senator John Kerry, head
of the Senate Foreign Relations Committee -- who have been pushing Congress to lift all restrictions
on travel to Cuba.

Others described it as a nod to President Castro's stunning decision last month to begin releasing
dozens of political prisoners.

``It's a way of fostering greater opening and exchange without a bruising battle with a much-needed
political ally in an election year,'' said Christopher Sabatini, senior policy director at the
Council of the Americas. ``But it can still be legitimately couched as a way of supporting democracy
and human rights by allowing independent exchange and thought.''

As with everything concerning Cuba, the new policy seems fraught with complications. President
Obama, who came to office promising to open new channels of engagement with Cuba, has so far had
limited those new openings to Cuban-Americans, partly because of political concerns, and also
because his administration's attention had been focused on more pressing foreign policy matters,
including two wars.

``I don't think the administration believes this will produce palpable change in the short term,''
said Julia Sweig of the Council on Foreign Relations. ``But it's a way over the long term to allow
Americans and Cubans to have contact, even as their governments continue to hash out a lot of
seriously thorny issues.''

High on the United States' list of issues is winning the release of an American contractor who was
detained in Cuba nine months ago when the authorities said they caught him distributing satellite
telephones to Jewish dissidents. The contractor, Alan P.~Gross, had gone to Cuba without the proper
visa as part of longstanding program by the organization Usaid, in which development workers conduct
activities aimed at strengthening groups that oppose the Castro government.

``We're dealing with a relationship that's so contorted, it would take another 50 years of
incremental steps to pull it apart and reassemble it in a constructive way,'' said Robert Pastor, a
professor of international relations at American University. ``Even then, we're having trouble
taking baby steps, when what we need is a giant leap.''

\section{Nonprofits Review Technology Failures}

\lettrine{A}{t}\mycalendar{Aug.'10}{17} a gathering last month over drinks and finger food, a
specialist at the World Bank related the story of how female weavers in a remote Amazonian region of
Guyana had against all odds built themselves a thriving global online business selling intricately
woven hammocks for \$1,000 apiece.

The state phone company had donated a communications center that helped the women find buyers around
the world, selling to places like the British Museum. Within short order, though, their husbands
pulled the plug, worried that their wives' sudden increase in income was a threat to the traditional
male domination in their society.

Technology's potential to bring about social good is widely extolled, but its failures, until now,
have rarely been discussed by nonprofits who deploy it. The experience in Guyana might never have
come to light without FailFaire, a recurring party whose participants revel in revealing
technology's shortcomings.

``We are taking technology embedded with our values and our culture and embedding it in the
developing world, which has very different values and cultures,'' Soren Gigler, the World Bank
specialist, told those at the FailFaire event here in July.

Behind the events is a Manhattan-based nonprofit group, MobileActive, a network of people and
organizations trying to improve the lives of the poor through technology. Its members hope
light-hearted examinations of failures will turn into learning experiences -- and prevent others
from making the same mistakes.

``I absolutely think we learn from failure, but getting people to talk about it honestly is not so
easy,'' said Katrin Verclas, a founder of MobileActive. ``So I thought, why not try to start
conversations about failure through an evening event with drinks and finger foods in a relaxed,
informal atmosphere that would make it seem more like a party than a debriefing.''

There is also a prize for the worst failure, a garish green-and-white child's computer nicknamed the
O.L.P.C. -- for One Laptop Per Child -- a program that MobileActive members regard as the emblem of
the failure of technology to achieve change for the better. When Ms.~Verclas held it up during last
month's party, the room erupted in laughter. (Jackie Lustig, a spokeswoman for O.L.P.C., said the
organization did not consider its program a failure.)

With the prize in his sights, Tim Kelly, a technology specialist at the World Bank who had just
flown in from South Africa, found himself in front of a screen displaying what looked like a line
drawing of a bowl of spaghetti and meatballs but in fact was an effort to explain the roles and
relationships of the many partners in the Global Capacity Building Initiative, a program aimed at
building strong policies and regulatory environments to foster the expansion of the Internet in
developing countries. ``This is the point in the evening where I'm suddenly asking myself why I let
myself be talked into this,'' Mr.~Kelly said.

He nonetheless gamely continued. One big problem with the project is that three groups raising money
for it were more interested in raising money for themselves, Mr.~Kelly said. ``One raised money and
when it finished doing that, took the money and went off and did its own work,'' Mr.~Kelly said.

The initiative had too many ``players,'' he continued. Donor countries wanted vastly different
things. It was way too complex, he said, gesturing at the spaghetti bowl.

Next time, he said, he would advocate for an initiative that matched specific donors to specific
projects and not work so hard to be all things to all people.

His eight minutes of torture over, Mr.~Kelly returned to his chair, looking somewhat relieved.

Mr.~Kelly's employer, the World Bank, sponsored the event here last month.

``The idea is that not only should we be open about what we're doing, but we should also be open
about where we learn and our mistakes,'' said Aleem Walji, practice manager for innovation at the
World Bank. ``The cost of not doing so is too high.''

Mr.~Walji said he was surprised to find, when he joined the bank from Google last fall, that
mistakes were rarely discussed, so different from the for-profit world, where failures are used to
spur innovation.

Google, for example, has blogged about the failure of its Google Wave application on Aug.~4., saying
that while it had ``numerous loyal fans, Wave has not seen the user adoption we would have liked.''

``Wave has taught us a lot,'' wrote Urs Hölzle, senior vice president for operations at Google.

Mr.~Walji pointed out that ``the private sector talks about failure freely and candidly,'' while the
nonprofit world ``has to worry about donors who don't want to be associated with failure and
beneficiaries who may not benefit from admissions of failure.''

Next up, after Mr.~Kelly, was Mahad Ibrahim, a researcher who had been hired to assess an Egyptian
government program to roll out telecenters across the country to increase access to the Internet.
The program has grown to more than 2,000 such centers, from 300 in 2001.

But numbers alone can be deceiving. Mr.~Ibrahim started his research by calling the centers. ``The
phones didn't work, or you got a grocery store,'' he said.

He headed for Aswan, where government records showed 23 telecenters. He found four actually working.

Mr.~Ibrahim concluded that the program had failed because it did not take into account the rise of
Internet cafes in Egypt and because the government had, in most cases, picked as partners nonprofit
groups whose primary mission had little or nothing to do with the Internet, communications or
technology.

The failure, in other words, was in not understanding the ecosystem in which the telecenters would
be operating. ``We dump hardware down and hope magic will happen,'' said Michael Trucano, senior
information and education specialist at the World Bank, whose offering to FailFaire was a list of
the 10 worst practices he had encountered in his job.

His presentation clearly resonated with the attendees, who voted him the winner of the O.L.P.C.

``I guess it's a dubious distinction,'' Mr.~Trucano said later, ``but I thought it was an enjoyable
evening and a useful way to talk about a lot of things that civil servants don't like to talk
about.''

\section{Denmark Starts to Trim Its Admired Safety Net}

\lettrine{H}{ow}\mycalendar{Aug.'10}{17} long is too long to be paid to go without a job?

As extended unemployment swells almost everywhere across the advanced industrial world, that
question is turning into a lightning rod for governments.

For years, Denmark was held out as a model to countries with high unemployment and as a progressive
touchstone to liberals in the United States. The Danes, despite their lavish social welfare state,
managed to keep joblessness remarkably low.

But now Denmark, which allows employers to hire and fire at will while relying on an elaborate
system of training, subsidies for those between jobs and aggressive measures to press the unemployed
into available openings, is facing its own strains. As a result, it is beginning to tighten up.

Struggling to keep its budget under control after the financial crisis, the government in June cut
into its benefits system, the world's most generous, by limiting unemployment payments to two years
instead of four. Having found that recipients either get work right away or take any job as their
checks run out, officials are also redoubling longstanding efforts to move Danes more quickly out of
the safety net.

``The cold fact is that the longer you are out of a job, the more difficult it is to get a job,''
Claus Hjort Frederiksen, the Danish finance minister, said during an interview. ``Four years of
unemployment is a luxury we can no longer allow ourselves.''

In the United States, where the Senate passed an unemployment insurance extension last month only
after a long battle, the debate over how to treat persistent joblessness has mounted as well.

It pits those who argue that decent benefits are necessary to support workers and their families
when companies are doing little hiring against those who contend that longer benefits periods
discourage job-seeking. Another fight is brewing over putting more federal dollars toward
retraining.

Similar concerns loom in debt-ridden countries like Spain and Portugal, where the costs of high
long-term unemployment have governments straining for a solution.

Such European countries could profit, many economists say, from adopting the more dynamic parts of
Denmark's ``flexicurity'' system. But now that the global recession has exposed chinks in its armor,
Denmark's efforts to find a new balance between job market flexibility and security for workers are
setting off alarm bells in the country.

``We have a famous flexicurity model, but now it's all flex and no security,'' complained Kim
Simonsen, chairman of HK, one of Denmark's largest trade unions.

To be sure, Denmark is not abandoning the welfare state. Government spending accounts for about half
of gross domestic product, and few Danes complain about a top income tax rate of 50 percent that
generously finances unemployment, pensions, health care and other accoutrements that, studies claim,
make Danes the happiest people on earth.

Hardly anyone in Denmark, a small, tranquil country of 5.5 million people, falls through the cracks.
The constitution even guarantees Danes the right to work and to receive public assistance if they
stumble. But sustaining a benevolent nanny state is proving to be challenging even for the notably
generous Danes.

``It's no surprise the government is saying that programs that are highly expensive and give a
Rolls-Royce treatment to citizens have to be trimmed,'' said Iain Begg, a professor at the London
School of Economics. ``So the search will now be on for labor market policies that deliver more
people in work with less money, which has an inevitable air of the holy grail about it.''

In Denmark, employers have carte blanche to hire and fire, and in most cases laid-off people are
guaranteed about 80 percent of their wages in benefits, a figure capped for high earners. In turn,
they must participate in retraining and job placement programs tailored to get them back to work,
which the government has intensified.

Each year, a remarkable 30 percent of Danes change jobs, knowing the system will allow them to pay
rent and buy food so they can focus on landing a new position. About 80 percent belong to unions,
which manage the workplace, help run the unemployment insurance program and press the laid-off into
retraining.

But as the financial crisis erased jobs, the government, Denmark's largest employer, has had to
provide more temporary work and intensify coaching. Unemployment is at 4.2 percent today, lower than
most European countries, though more than twice the 1.7 percent rate two years ago.

As in Germany and some other European countries, hundreds of Danish employers have also embraced
government-subsidized short-work programs, a tactic adopted to keep a lid on unemployment. The plans
allow companies to cut working hours to hold onto highly skilled workers, rather than laying them
off when times are tough.

Danish politicians say their program is still working well.

But unions argue that the cutbacks in the safety net go too far, and they are planning to press
companies to lengthen the typical one- to three-month notice period before dismissals.

Business leaders fear that would push Denmark toward the type of rigid systems found in Spain, Italy
and France, where it can take a year or more to lay off most employees, which drains finances and
raises the danger of more job cuts.

``If unions start requiring longer job-cut notices in exchange for reduced benefits, you'll lose the
flexibility to adapt to changes in the economy,'' said Stine Pilegaard Jespersen, head of labor
market policy at the Danish Chamber of Commerce.

Inger Skouby, at 58 a longtime nurse, has seen the system shift from the inside. She was in and out
of unemployment for nearly four years after she fell ill. She took a year off for treatment, paid
for by the state. She then tapped jobless pay, receiving about 80 percent of her former wage.

To get with the times, she received information technology training, leading to a telemarketing
position until the financial crisis hit.

When she returned to unemployment, she said, the government had tightened up, requiring weekly job
applications, meetings with job counselors, and repetitive training that produced scant results. She
was put into a work program as a school secretary, until something better came along.

Many others spoke in interviews about being required to take make-work or menial jobs that have
eroded their morale.

``Before, it wasn't like this,'' Mrs.~Skouby said. ``Now, it's about controlling people.''

Lisbeth Halvorsen, 30, had her first brush with unemployment last month, after her part-time
teaching job expired. She will get 70 percent of her salary, but is frantically sending r\'esum\'es
to get out of the system as soon as possible.

The government has created a lot of incentive to do so, she said. To improve its job activation
program, Denmark has outsourced some of it to private companies, which receive bonus payments for
every person placed into job training or a new job.

That has led to cases where laid-off workers spend an entire month in courses to improve their
r\'esum\'es, or tied up in ``sit-around-and-drink-coffee meetings'' to obtain unemployment checks.

Occasionally, they offend Danish sensibilities. Torben Frederiksen, 32, a plumber out of work for
three months, said his employment center forbade him from attending his mother's funeral because it
conflicted with a meeting with his counselor.

``They told me that a funeral was no excuse for missing my appointment,'' he said. Mr.~Frederiksen
went anyway, and was granted another meeting.

From the perspective of Claus Hjort Frederiksen, the finance minister, Denmark is carefully laying
the groundwork for the future by changing its policies to make more people eligible for work when
the economy picks up.

``In two years, we expect to be out of the crisis,'' he said, ``and we'll need to be ready.''

\section{Despite H.P.'s Efforts, Spectacle of a Chief Goes On}

\lettrine{E}{ven}\mycalendar{Aug.'10}{17} after a former actress in erotic films had accused
Hewlett-Packard's chief executive, Mark V.~Hurd, of sexual harassment, the company's board stood
behind him.

The directors had often talked with him about taking the world's biggest technology company back to
its roots as an innovator after the big-ticket acquisitions of 3Com and Palm. They doubted that
Mr.~Hurd, ever meticulous and boastful of his integrity, could commit such unscrupulous acts,
according to people with knowledge of the board's thinking.

But when he settled the woman's harassment complaint in a late-night meeting before the board's
investigators had a chance to speak with her, the directors deemed his behavior just too troubling.

H.P.'s board rushed out Mr.~Hurd's resignation the next day, on Aug.~6. What has followed is a
stream of leaks from both sides resulting in a very public imbroglio. The drama between staid H.P.
and equally staid Mr.~Hurd continues in a fashion quite unlike executive departures of its kind.

The company has uncovered communications between Mr.~Hurd and Jodie Fisher, the occasional H.P.
contractor who accused him of sexual harassment, that seemed cordial, even after a last meeting in a
hotel room in Boise, Idaho, a person with knowledge of Mr.~Hurd's e-mails said.

But the board was increasingly troubled that Mr.~Hurd had been so willing to put Ms.~Fisher, whose
job was to introduce Mr.~Hurd to customers at H.P. marketing gatherings, in front of top customers
at upscale company events. Nor could they fathom how Mr.~Hurd could authorize more than \$75,000 in
payments and expenses, including first-class travel and stays in luxury hotels, for Ms.~Fisher when
the company's employees traveled under more austere conditions, according to several people with
knowledge of the expenses.

Mr.~Hurd has been portrayed as engaging in increasingly questionable behavior as H.P. examined his
relationship with Ms.~Fisher. The directors could not grasp how he could defend having a close
personal relationship with an expensive contractor and play down her background in sexually charged
films like ``Intimate Obsession'' and ``Body of Influence 2.'' She also posed partly nude in Playboy
while in college in the early 1980s.

Simply put, he lost the board's trust, said people with knowledge of the board's thinking.

A person close to Mr.~Hurd counters that H.P. officials are trumping up a version of events in
conversations with reporters.

``Mark Hurd is shocked and enraged by allegations from an H.P. board source that by settling an
unfounded sexual harassment claim he was impeding the board from learning the truth,'' said this
person close to Mr.~Hurd.

Mr.~Hurd had been encouraged by the directors to settle the claims for weeks, the person said.

People close to Mr.~Hurd have argued that certain board members overreacted to the sexual harassment
charges and the possible negative publicity if word of them leaked.

Company officials, meanwhile, have publicly stated that Mr.~Hurd's behavior ran far afoul of company
policy. In particular, Mr.~Hurd has been accused of trying to hide an inappropriate relationship
with Ms.~Fisher by altering his expense reports.

An H.P. spokeswoman, Christina Schneider, declined to comment on the matter.

Regardless, the company is getting exactly what it had hoped to avoid. It hired Kent Jarrell, a
crisis management expert from the public relations firm APCO Worldwide, soon after the board
received the harassment complaint at the end of June. Mr.~Jarrell wrote a mock news story for the
board, saying it could expect headlines about H.P.'s having a soft-core pornography star working its
events, people with knowledge of Mr.~Jarrell's work said.

``This history here is important,'' said Scott Stern, a business professor at the Sloan School of
Management at M.I.T.

Federal regulators are keeping a close eye on the H.P. board's behavior after a scandal several
years ago when factions of H.P.'s board spied on other members of the board, journalists and
employees in an attempt to stop leaks. ``This is a very proud company with a great history,''
Mr.~Stern said. ``The fact is that you could understand that the board would be trigger-shy about
public controversy.''

Mr.~Hurd, according to those close to him, found it baffling that the company would disclose the
sexual harassment case after a company investigation turned up no wrongdoing.

The accusations linking Mr.~Hurd to improper spending resonate because he had built up a culture of
severe financial accountability at H.P. A former H.P. executive said that Mr.~Hurd's meetings were
known internally as ``rectal exams'' because of the fierce questioning.

``If you put up a slide with lower financial forecasts, he would spot it right away,'' the executive
said, requesting anonymity because he still deals with H.P. ``Then, he would demand that you fix the
situation.''

Stories abound of Mr.~Hurd's slicing into marketing costs and making employees fight for every
dollar in the budget (although Mr.~Hurd often found marketing money to sponsor tennis events,
uniting his love of the sport with H.P.).

It's this history that made Mr.~Hurd's recent behavior stand out to the board.

The situation was made worse after H.P. discovered that Mr.~Hurd had viewed some of Ms.~Fisher's
racy acting on his work computer, signaling that he was aware of her past.

Mr.~Hurd has told people that he did a brief Google search of Ms.~Fisher in April or May of 2009,
nearly two years after she started contract work for H.P.

Lawrence J.~Ellison, the chief executive of Oracle, called H.P.'s board cowardly in an e-mail to The
New York Times. And plenty of pundits went on to say that H.P. should not have let Mr.~Hurd go
simply because he left Ms.~Fisher's name off some expense report filings.

A number of outsiders have complimented H.P.'s board for its decisive action, noting that the
company's board has operated under a microscope because of past transgressions.

William W.~George, a professor of management practices at the Harvard Business School, said H.P. had
made the right decision and that Mr.~Hurd needed to move on. ``I think once it's settled, you need
to walk away,'' Mr.~George said. ``If he comes to grips with this, he is going to come back and have
great leadership roles.''

\section{Obama Stumps for Democrats and Economic Agenda}

\lettrine{P}{resident}\mycalendar{Aug.'10}{17} Obama excoriated Congressional Republicans on Monday
for seeking to stymie his economic agenda, delivering a harsh critique of what he said was
obstructionism that put political calculations ahead of the welfare of average Americans.

In a rousing political speech before a crowd of Wisconsin Democrats who had each paid \$250 to hear
from their party's leader, Mr.~Obama called for a renewed effort to push for his economic agenda
this election year.

Deriding his Congressional opponents as the ``no, we can't'' crowd, Mr.~Obama called his critics
``more concerned with the next election than the next generation.''

The speech came as Mr.~Obama began a three-day campaign swing through five states that is intended
to shore up the fortunes of Democratic candidates still willing to stand beside him and his
deteriorating approval ratings.

Mr.~Obama chided Republicans for ``offering the exact same policies that you rejected in 2006 and
2008 because you knew they weren't working.''

The trip combines job-related events and political fund-raising.

At his first stop, Mr.~Obama maintained that his economic policies had put the country on the road
to a recovery.

``The worst mistake we could make now would be to turn back,'' he told a small crowd of workers at
ZBB Energy, a company in Menomonee Falls, Wis., that makes batteries and power systems that use
renewable energy. The company received a \$1.3 million federal stimulus loan, which officials said
would triple its manufacturing capacity and could lead to 80 new jobs.

After touring the plant, Mr.~Obama headed to Milwaukee for the fund-raiser, for Mayor Tom Barrett,
who is running for governor and trying to keep the office in Democratic hands. Organizers said the
event raised about \$325,000.

On Wednesday, Mr.~Obama will try to do the same in Ohio for Gov.~Ted Strickland, a Democrat in a
tight re-election race. The White House has already sent other officials to campaign for
Mr.~Strickland, including Vice President Joseph R.~Biden Jr.

The fight for the Midwestern governorships is important because the new crop of governors will play
a big role next year as state legislatures take up the fight over Congressional redistricting that
comes along once every decade. The results in the Midwest will also help to define crucial party
organizing efforts leading up to the 2012 presidential campaign in some of the bigger states,
including Michigan, Minnesota, Ohio and Wisconsin.

At the fund-raiser, attended by 1,300 people, Mr.~Barrett acknowledged that the economy was still
struggling. ``All of us recognize that we still have a long way to go,'' he said. But he added, ``We
all need to remember the direction this country was falling when President Obama took over.''

Mr.~Barrett shot to national attention last year when he was injured after intervening to help a
woman and her grandchild in a domestic dispute with a young man near the Wisconsin State Fair.

Mr.~Obama referred to the incident on Monday, saying: ``I've heard stories about mayors. I've never
heard of a mayor who risked his life to respond to an actual cry for help. That is some serious
customer service from this man right here.''

Mr.~Obama's ability -- or inability -- to help here will be crucial in forecasting his own chances
in 2012. Even as Mr.~Barrett was preparing to introduce the president, he emphasized his independent
credentials. ``If you're looking for the ideologue in this race, it's not me,'' Mr.~Barrett said.
``The problems of this state will not be solved by bumper-sticker slogans.''

Even before Air Force One had touched down in Milwaukee on Monday, statewide Republicans were
playing down the chances that the president, whose approval ratings have fallen in recent months,
would be helpful.

The Associated Press reported that the state Republican chairman, Reince Priebus, said Monday that
Mr.~Obama's visit would hurt local Democrats. Mr.~Priebus said that an \$810 million high-speed
train, to be built with federal stimulus money between Madison and Milwaukee, is an example of the
kind of irresponsible spending that has upset Wisconsin residents.

The deputy White House press secretary, Bill Burton, took issue with the characterization of the
high-speed train as a boondoggle.

``I'm not an expert on transportation issues,'' Mr.~Burton told reporters aboard Air Force One to
Wisconsin. ``But I do think that these grants -- the investments that we've made on renewable
energy, on infrastructure -- are helping to make our economy stronger and are helping to create an
environment where jobs are being created and people getting back to work.''

\section{Pentagon Cites Concerns in China Military Growth}

\lettrine{C}{hina}\mycalendar{Aug.'10}{17} has increased spending on a military that is becoming
larger and more effective even as Beijing has rebuffed exchanges with the Defense Department that
could improve stability, according to a Pentagon study released Monday.

Senior Pentagon officials acknowledged that much of the Chinese military modernization program may
reflect the rational ambition of a rising global power, albeit one that may be a worrisome rival to
American interests in the Pacific region.

But across the American government -- from the White House to the Pentagon to Congress -- officials
express concern that China's lack of openness about the growth, capabilities and intentions of its
military injects instability to a vital region of the globe.

China's overall spending on national defense for 2009 was estimated at \$150 billion, an increase of
7.5 percent but only about one-fifth of what the Pentagon spent to operate and carry out the wars in
Iraq and Afghanistan, according to the study, required each year by Congress.

China's arsenal of missiles arrayed across a strait from Taiwan, an American ally considered a
wayward province by Beijing, did not substantially grow in numbers but is being upgraded to be more
capable, according to the review.

Of the many potential points of conflict, Taiwan remains the most notable, as China froze
military-to-military relations with the Defense Department earlier this year after an announcement
that the United States was selling more than \$6 billion in weapons to Taiwan.

Administration officials say that while ties between Washington and Beijing in the areas of
diplomacy and economics are improving, the military-to-military relationship is prickly and a reason
for concern.

Another cause of worry, according to the study, is China's emphasis on weapons that could deny the
ability of American warships to operate in international waters off the coast; those weapons include
precision, long-range missiles and a growing fleet of submarines and warships.

The Pentagon study said that China had an active program to develop and build several aircraft
carriers, and could start construction by the end of this year. China also appears intent on
expanding its arsenal of nuclear-powered submarines, with one missile-launching submarine and
several hunter-killer submarines already at sea, all nuclear-powered for greater range. These
nuclear-powered submarines are in addition to larger and growing numbers of diesel-powered
hunter-killer submarines in the Chinese Navy, according to the study.

Administration and military officials also criticized China's actions beyond its territorial waters,
particularly in the South China Sea. Pentagon officials say China's military appears intent on
extending claims for maritime jurisdiction beyond the range accepted by international law.

Senior Defense Department officials who released the study declined to be drawn into a discussion of
politics, but Representative Ike Skelton of Missouri, the Democrat who is chairman of the House
Armed Services Committee, expressed a view shared by the Pentagon.

In a statement released Monday, Mr.~Skelton said he was concerned by ``ambiguities regarding China's
military modernization, including its missile buildup across from Taiwan, its maritime activities in
the South China Sea, and the steady increase of its power-projection capabilities, which do not
obviously support China's stated national security objectives.''

While ``China has taken some steps toward increasing transparency and openness regarding its defense
strategy and expenditures in recent years,'' Mr.~Skelton said, ``such steps are modest. China's most
recent military budget continues a trend of sustained annual increases, and China's strategic
intentions remain opaque.''

The Pentagon review comes as China surpassed Japan in the second quarter of the year to become the
world's No.~2 economy, after the United States.

\section{New China Search Engine Will Be State-Controlled}

\lettrine{I}{n}\mycalendar{Aug.'10}{17} an apparent bid to extend its control over the Internet and
cash in on the rapid growth of mobile devices, China plans to create a government-controlled search
engine.

The new venture would compete with Baidu.com, a private company that runs China's dominant search
engine. Baidu's market has grown since Google retreated from the mainland earlier this year.

The state-owned China Mobile -- the world's biggest cellphone carrier -- and Xinhua, China's
official state-run news agency, signed an agreement on Thursday to create a joint venture called the
Search Engine New Media International Communications Company.

China already has the world's largest number of Internet users, more than 420 million, and also the
largest number of mobile phone subscribers, with more than 800 million.

Private start-up companies play a big role on the Web in China, but the government maintains tight
control over Internet companies and censors content that it deems dangerous or sensitive.

Now, though, analysts say that Beijing is pushing state-run companies to take a more active role
online. China Central Television, the nation's dominant broadcaster, is trying to develop an online
video site. Xinhua News Agency is trying to build a global platform of news providers using
television and the Internet.

At the announcement of the joint venture in Beijing on Thursday, Zhou Xisheng, vice president of
Xinhua, said the new company would build a leading search engine platform. But he also said the move
was ``part of the country's broader efforts to safeguard its information security and push forward
the robust, healthy and orderly development of China's new media industry.''

Representatives of Baidu could not be reached for comment.

For years, Baidu has dominated Internet searches in China, holding a sizable lead over Google, which
entered the market late. Earlier this year, Google pulled its search engine out of Beijing after
complaining about censorship and online attacks that appeared to be coming from hackers within
China.

Google now operates its Chinese-language search engine from Hong Kong; it is accessible from China
but some results are censored by the government.

Most of China's other big, private Internet companies are involved in online games and
entertainment. But on Monday, Alibaba.com, one of the country's biggest e-commerce sites, said the
company and a fund co-founded by its chairman would acquire a 16 percent stake in the search engine
Sogou, which is owned by the Chinese portal Sohu.com.

Yahoo, the United States portal, holds a 40 percent stake in the Alibaba Group.

\section{Japanese Sushi Students Aim for a Job Overseas}

\lettrine{A}{t}\mycalendar{Aug.'10}{17} the sound of the buzzer, Kensuke Aoki starts furiously
pressing rice into shape in his left palm. Adding dabs of bright green wasabi paste and quickly
placing slices of mackerel on top of the rice, he completes each nigiri one after another. Three
minutes and 18 nigiri later, time is up, and his instructor comes around to sort the well-shaped
ones from the ill-shaped, as Mr.~Aoki wistfully looks on. Just 12 have made the grade.

``It's hard, but I am getting better every time I do it,'' said Mr.~Aoki, a student at the Tokyo
Sushi Academy. ``Speed is as crucial as quality because efficiency is what they will seek in the
real world.''

Mr.~Aoki, 30, believes that having a real-world sense of things will help him and his classmates
gain the skills they need to plunge into the competitive market for sushi chefs in places like
Germany, the United States and Australia.

Students at the privately owned academy plan to join the growing ranks of professional Japanese
chefs eager to serve a growing overseas appetite for sushi. Their plan to seek jobs abroad comes as
revenue is declining within the Japanese sushi sector amid a cutthroat price war within the
restaurant industry overall, which means consumers expect to pay less and receive more.

By contrast, the sushi restaurant market overseas is a rich source of entrepreneurial opportunity
for young Japanese, said Hiromi Sugiyama, a director of the academy, which also hosts a Web site,
www.sushijob.com, for chefs seeking jobs in other countries.

``Graduates of this school often earn much more than they would make here in Japan,'' she said, as
more sushi restaurants open in Europe and Latin America.

The academy enrolls about 100 students each year in either intensive, two-month programs or yearlong
diploma courses. More than 700 students have graduated from the academy since it opened in 2002, its
executives say. Of the 10 people in Mr.~Aoki's yearlong program, 9 said they were preparing for
careers abroad.

Mr.~Aoki sees his training as a way to return to the United States, where he spent two years as a
college student in Reno, Nevada.

``I liked the lifestyle there,'' he said. ``The working environment seems better, and the nature and
the wildlife is terrific.''

A more fragile job market in Japan may also be a factor. The ¥1.5 trillion, or \$17.4 billion, sushi
restaurant industry in Japan is changing, says Akihiro Nisugi, a restaurant consultant at Funai
Consulting, based in Tokyo. ``The fast-food 'rotating conveyor belt' sushi chains are growing,'' he
said, ``but the traditional sushi restaurant is facing contraction.''

Revenue at traditional sushi restaurants in Japan declined to ¥1.049 trillion in 2009 from ¥1.08
trillion in 2006, according to Funai. Sales at fast-food sushi places amounted to ¥428 billion in
2009.

Decades ago, an aspiring chef would have joined a traditional sushi restaurant as an apprentice,
dreaming one day of becoming a taisho, or a sushi restaurant owner, in places like the chic Ginza
district of Tokyo, where restaurants are renowned for providing the highest-quality sushi. But with
that sort of job security crumbling, along with the concept of long-term loyalty to one employer,
chefs are directing their attention abroad.

Taira Matsuki, 39, who completed a short-term program at the Tokyo Sushi Academy two years ago, set
up a catering business in Warsaw last year after working at a sushi restaurant in Poland.

``Here, sushi and pizza are two categories that are growing strongly and where people are making
money,'' said Mr.~Matsuki, who now employs five Polish workers. His mainstay product is a \$7 sushi
lunchbox aimed at business executives in downtown Warsaw.

The speed with which he was able to open his own business contrasts with the centuries-old
traditions of the Japanese sushi apprenticeship. Young people are shunning the profession at home
partly because the industry requires years of mopping floors and washing dishes before an apprentice
is even allowed to touch rice.

``People say it takes three years before you can master the nigiri, and five years before you
perfect maki sushi, the roll, and you need 10 years before you become a full-fledged sushi master,''
said Ken Kawasumi, chief instructor at the Sushi Academy and a former sushi chef. ``That's not a
valid approach anymore.''

Plus, his students are not willing to wait that long. Koji Ohno, 30, another student at the Tokyo
Sushi Academy, has worked as an information technology engineer and wants to reinvent himself as a
sushi cook overseas. Going the traditional apprentice route is not part of his plan. ``That's a risk
I am not willing to take at this age,'' he said. ``I wanted to get right to the training.'' He plans
to move to Munich, where he once studied German.

But traditionalists say the regime of years of training is integral to achieving the right mental
attitude as a professional who works directly, almost intimately, with consumers. ``In sushi,'' said
Issei Kurimoto, a head chef at a sushi restaurant in the Yurakucho area of Tokyo, ``you work in an
open kitchen counter, facing your customers and serving them directly. You need to develop direct
hospitality skills.''

Mr.~Kurimoto, who has hired sushi school graduates over his 40-year career, says students need more
than a diploma. The true skills of a sushi chef are learned at a tension-filled counter, repeating
the routines over and over ``until your body knows it,'' he said.

On-the-job experience also adds to subtle knowledge of the product -- the seasonality of the fish,
for example. Some fish are only available during a particular season, he said, while an item like
squid, which is available year-round, ``is different in condition between summer and winter, and
requires a different slicing technique.''

``You have to go through several summers or several winters before you are fully equipped with that
kind of skill,'' he said.

Some restaurants, however, acknowledge that they need to revise the system to attract the sushi
chefs of the future.

``Young people want visibility in terms of a career, and it's difficult to get that if you have to
wait three years before you are allowed to touch the knife,'' said Hiroshi Umehara, a spokesman at
Kiyomura, which operates 30 sushi shops in Tokyo.

The company opened its own sushi school four years ago, making sure that ``you touch rice the first
day,'' Mr.~Umehara said.

The school enrolls 20 students for a three-month program, which is repeated twice a year, he said,
and students increasingly seek career opportunities overseas.

Even with the training, the skills needed in overseas markets can vary from those required in Japan.

``There are certain skills that are not useful overseas,'' said Suehiko Shimizu, 70, another
instructor at the Sushi Academy. Some Japanese favorites, like conger eel, are not available in most
overseas markets, he said, while many consumers in European markets turn up their noses at raw
shellfish like clams, or salmon roe, which are standard fare in Japan.

In fact, Mr.~Matsuki, in Warsaw, says that half of the contents of the lunchbox he serves are fried.
``Many Polish people aren't used to raw fish yet,'' he said. ``I also tweak the sauce so it is a bit
sweet, which is to the liking of the locals.''

\section{Chinese Health Ministry Clears Milk Powder in Latest Scare}

\lettrine{C}{hina}\mycalendar{Aug.'10}{17}'s health ministry announced Sunday that it had found no
clinical evidence that milk powder had been responsible for apparent breast development in three
infant girls in south-central China.

The ministry said that tests on 73 samples of infant formula had uncovered no sign of inappropriate
hormones, after Chinese media had speculated that such hormones might be responsible for the
infants' development.

The official Xinhua news agency also quoted Chinese medical specialists as saying that the infants'
physiological development was not unheard of.

Widespread speculation in the Chinese media in recent days about the infants' health, without the
benefit of scientific analysis, has underlined the increasingly free-wheeling nature of Chinese
media on health and business issues, as well as a national preoccupation with food safety. Milk
powder contaminated with melamine, an industrial chemical used to hide the effect of diluting raw
milk with water, killed six babies and sickened 300,000 two years ago.

A Chinese milk powder company, Synutra, was at the center of suspicions regarding the three infant
girls, who were fed milk prepared from its powder. The health ministry said that 42 of the 73
samples that it had tested without finding unusual levels of hormones had been Synutra's milk,
including one sample of a milk powder residue from one of the infant girls' homes.

Synutra has repeatedly denied that there is any problem with its infant formula.

\section{A Dutch City Seeks to End Drug Tourism}

\lettrine{O}{n}\mycalendar{Aug.'10}{18} a recent summer night, Marc Josemans's Easy Going Coffee
Shop was packed. The lines to buy marijuana and hashish stretched to the reception area where
customers waited behind glass barriers.

Most were young. Few were Dutch.

Thousands of ``drug tourists'' sweep into this small, picturesque city in the southeastern part of
the Netherlands every day -- as many as two million a year, city officials say. Their sole purpose
is to visit the city's 13 ``coffee shops,'' where they can buy varieties of marijuana with names
like Big Bud, Amnesia and Gold Palm without fear of prosecution.

It is an attraction Maastricht and other Dutch border cities would now gladly do without. Struggling
to reduce traffic jams and a high crime rate, the city is pushing to make its legalized use of
recreational drugs a Dutch-only policy, banning sales to foreigners who cross the border to indulge.
But whether the European Union's free trade laws will allow that is another matter.

The case, now wending its way through the courts, is being closely watched by legal scholars as a
test of whether the European Court of Justice will carve out an exception to trade rules -- allowing
one country's security concerns to override the European Union's guarantee of a unified and
unfettered market for goods and services.

City officials say they have watched with horror as a drug tolerance policy intended to keep Dutch
youth safe -- and established long before Europe's borders became so porous -- has morphed into
something else entirely. Municipalities like Maastricht, in easy driving distance from Belgium,
France and Germany, have become regional drug supply hubs.

Maastricht now has a crime rate three times that of similar-size Dutch cities farther from the
border. ``They come with their cars and they make a lot of noise and so on,'' said Gerd Leers, who
was mayor of Maastricht for eight years. ``But the worst part is that this group, this enormous
group, is such an attractive target for criminals who want to sell their own stuff, hard stuff, and
they are here too now.''

In recent years, crime in Maastricht, a city of cobblestone lanes and medieval structures, has
included a shootout on the highway, involving a Bulgarian assassin hired to kill a rival drug
producer.

Mr.~Leers used to call the possibility of banning sales to foreigners a long shot. But last month,
Maastricht won an early round. The advocate general for the European Court of Justice, Yves Bot,
issued a finding that ``narcotics, including cannabis, are not goods like others and their sale does
not benefit from the freedoms of movement guaranteed by European law.''

Mr.~Leers called the ruling ``very encouraging.'' Coffee shop owners saw it differently.

``There is no way this will hold up,'' said John Deckers, a spokesman for the Maastricht coffee shop
owners' association. ``It is discrimination against other European Union citizens.''

If Maastricht gets its way, many other Dutch municipalities will doubtless follow. Last year, two
small Dutch towns, Rosendal and Bergen op Zoom, decided to close all their coffee shops after
surveys showed that most of their customers were foreigners.

The situation has not made for good neighborly feelings. Many residents of border towns criticize
Belgium, France and Germany for tolerating recreational drug use but banning the sale of drugs.
``They don't punish small buyers,'' said Cyrille Fijnaut, a professor at the University of Tilburg
law school. ``But they also don't have their own coffee shops, so that leaves us as the suppliers.
Our policy has been abused, misused, totally perverted.''

As business has boomed, many of the Dutch coffee shops -- dingy, hippie establishments in the '80s
and '90s with a few plastic tubs of marijuana on the shelves -- have become slick shops serving
freshly squeezed orange juice and coffee in fine china.

The Easy Going Coffee Shop has a computer console at the door where identification documents proving
that customers are 18 or older are scanned and recorded. Tiny pictures on driver's licenses are
blown up to life-size on a screen, so guards can get a good look at them. Behind the teller windows,
workers still cut the hashish with a big kitchen knife, but all sales are recorded on computerized
cash registers.

Mr.~Bot's ruling last month is only an early step in determining whether Maastricht can enforce a
Dutch-only policy. A final ruling by the full court is expected by the end of the year.

But Mr.~Bot's finding, a veritable tirade on the evils of drugs, surprised many legal scholars, who
expected the European Union's open market rules to trump any public order arguments, as they have in
other cases. Sweden, for instance, which has a long history of struggling with alcohol abuse, was
obliged to take down most of its anti-alcohol laws restricting store hours and sales, as they were
seen as impinging on free trade.

Polls show that a majority of the Dutch still believe that the coffee shops should exist. But the
Netherlands once had 1,500 of them; now, there are about 700. And every year, the numbers decline,
according to Nicole Maalste, a professor at the University of Tilburg who has written a book on the
subject. ``Slowly, slowly they are being closed down by inventing new rules, and new rules,''
Ms.~Maalste said.

Much of the criminality associated with the coffee shops, experts say, revolves around what people
here call the ``back door'' problem. The government regulates what goes on in coffee shops. But it
has never legalized or regulated how the stores get the drugs they sell -- an issue that states in
the United States that have legalized medical marijuana are just beginning to grapple with.

In recent years, the tremendous volume of sales created by foreigners has prompted an industry of
cultivating cannabis and other drugs within the Netherlands -- some estimate that it is now a \$2
billion a year business -- much of it tangled in organized crime and money laundering operations,
experts say.

Advocates for legalized sales and coffee shop owners argue that trying to restrict foreigners will
only encourage them to buy illegally in the streets. They also say that coffee shops have other
selling points: they pay 450 million euros a year in taxes and provide thousands of jobs.

Mr.~Deckers, the shop association spokesman, said coffee shop owners were so skeptical that the
European Union would allow restrictions on sales based on nationality that they encouraged the city
to get a ruling on the subject. They doubt Mr.~Bot's arguments will stand. ``We know he is wrong,''
Mr.~Deckers said.

\section{`Straddling Bus' Offered as a Traffic Fix in China}

\lettrine{W}{hat}\mycalendar{Aug.'10}{18} do you do if your roads are congested and polluted? Try
designing a vehicle that takes up no road space. And make it partly solar powered.

A company in the southern Chinese town of Shenzhen has done just that. To address the country's
problems with traffic and air quality, Shenzhen Huashi Future Parking Equipment has developed a
decidedly odd-looking, extra-wide and extra-tall vehicle that can carry up to 1,200 passengers.

Though it is called the ``straddling bus,'' Huashi's invention resembles a train in many respects --
but it requires neither elevated tracks nor extensive tunneling. Its passenger compartment spans the
width of two traffic lanes and sits high above the road surface, on a pair of fencelike stilts that
leave the road clear for ordinary cars to pass underneath. It runs along a fixed route.

Huashi Future Parking's outsize invention -- six meters, or about 20 feet, wide -- is to be powered
by a combination of municipal electricity and solar power derived from panels mounted on the roofs
of the vehicles and at bus stops.

A pilot project for the vehicle is in the works in Beijing, and several other Chinese cities have
shown interest.

The company says the vehicle -- which will travel at an average speed of 40 kilometers an hour, or
about 25 m.p.h. -- could reduce traffic jams by 25 to 30 percent on main routes.

The straddling bus could replace up to 40 conventional buses, potentially saving the 860 tons of
fuel that 40 buses would consume annually, and preventing 2,640 tons of carbon emissions, said
Youzhou Song, the vehicle's designer.

``I had the idea when I was doing research on the road for the designs of innovative parking slots
for bikes and cars,'' Mr.~Song, who founded the company with several partners in 2009, said by phone
last week. ``I saw the traffic jams and wondered if it's possible to make buses high up in the air
as well.''

The design highlights a range of issues that have come with China's explosive economic growth.

The nation's urban population has expanded rapidly in recent years. In a report last year, the
consulting firm McKinsey estimated that an additional 350 million people -- more than the population
of the United States -- would move to the cities by 2015. More than 220 cities will have more than
one million people. By comparison, Europe has 35 such cities now.

All this has caused a vast need for urban infrastructure, with McKinsey estimating that 170 new mass
transit systems could be built in China by 2025.

At the same time, rising affluence has caused the number of cars -- and traffic jams -- to soar.

China is the world's largest polluter, and Beijing is eager to reduce carbon emissions. The
authorities have been pushing solar power and fuel-efficient transportation.

Huashi's invention appears to have received a preliminary seal of approval from Beijing. The
capital's Mentougou district is testing the technology and plans to start building nine kilometers
of route at the end of this year. If the test is successful, about 116 miles would be put in place.

``Mr.~Song's design is in line with our concept of green transportation and our vision of the
future. We hope to start the construction and operation as soon as possible,'' said Wenbo Zhang,
head of the science and technology commission of Mentougou district, though he added that the
necessary approvals would take time and investment.

Shijiazhuang, in Hebei Province, and Wuhu, in Anhui Province, have also applied to obtain financing
for straddling bus systems, Mr.~Song said, while Luzhou in Sichuan Province has shown interest.

The vehicles will be built by the China South Locomotive and Rolling Stock Corporation starting at
the end of this month, Mr.~Song said.

The cost of construction -- 50 million renminbi, or \$7.4 million, for one bus and about 25 miles of
route facilities -- is roughly one-tenth what it costs to build a subway of the same length, he
said.

Huashi Future Parking's more modest inventions include space-saving vertical bicycle-parking sheds.
The bike sheds have been sold to the municipal government of Nanchang, in Jiangxi Province, and to a
factory in Dongguan, in the southern province of Guangdong, Mr.~Song said.

\section{China Sets Strict Rules on Off-Book Loans}

\lettrine{D}{isappointed}\mycalendar{Aug.'10}{18} with his low savings deposit rate, a 29-year-old
chemical company salesman named Zhang Zhenlei says he took \$19,000 out of his savings account last
November and bought into a higher-yielding investment trust through his bank's wealth management
division.

The money, the bank told him, would help finance two government highway and infrastructure projects.

``They told me the details and which companies would get the loans,'' Mr.~Zhang said. ``And they
told me the risk was under control.''

But last month, the China Banking Regulatory Commission issued a sharp warning, ordering investment
trust companies to stop selling such products in cooperation with banks. Regulators were apparently
worried that banks and trusts were forming partnerships and using products like the one sold to
Mr.~Zhang to, in effect, finance loans without calling them loans.

The government evidently suspected that banks were using such maneuvers to evade rules put in place
this year to rein in rampant lending and excess credit. Those conditions have been cited as a reason
for rising property prices and overall inflation.

Regulators, suspecting that banks and trusts are secretly repackaging old loans and moving them off
bank balance sheets, are concerned that financial institutions in China may have engaged in the same
sort of financial engineering that got Western banks into trouble.

On Aug.~10, government overseers acted again, ordering banks to move any off-balance-sheet loans
back onto their books and to make provisions to safeguard against a rise in bad loans, according to
a copy of the government order given to The New York Times by an industry expert.

Several weeks ago, the Fitch credit rating agency warned that such off-balance-sheet deals were
understating the size of bank lending in China and thereby masking the risks associated with an
increase in dodgy loans.

Fitch estimates that Chinese banks had about \$350 billion in trust-related products on their
balance sheets at the end of June, and that much of that lending had not been publicly disclosed.

``Regardless of how the transaction is structured, credit is disappearing from bank balance sheets,
resulting in pervasive understatement of credit growth,'' a Fitch analyst, Charlene Chu, wrote in a
June presentation. ``Credit risk has not disappeared but merely been transferred to investors.''

The off-balance-sheet deals are raising warning flags about a possible slowdown here. While China's
economy remains robust -- it overtook Japan in the second quarter to become the world's
second-largest economy, behind the United States -- analysts worry that a surge in bank lending last
year and early this year might have led to wasteful spending on infrastructure and real estate
projects.

In recent years, the government has been trying to crack down on what it deems wasteful spending on
``luxurious'' local government buildings, highways to nowhere and so-called image projects that are
constructed in poverty-stricken areas.

The basis for these worries was laid last year, after the government encouraged aggressive lending
as part of a huge economic stimulus package. The result was a record \$1.4 trillion in new bank
loans in 2009, about double the previous year. Some analysts fear the sharp increase in lending
included many bad loans that will begin to show up over the next few years.

Aware of the risks, Chinese regulators are pressing state-run banks to raise billions of dollars in
capital to cushion any downturn -- a task that could be complicated by any perceptions among private
investors that the banks are exposed to a lot of risky debt. The government is also conducting
stress tests to determine how banks will perform if property prices plummet.

Beijing is now trying to restrict lending and ease asset price inflation without setting off a
slowdown.

``They're stuck in a policy bind,'' says Michael Pettis, a professor of finance at Peking University
in Beijing, noting China's heavy dependence on investment-driven growth. ``They have to choose
between cleaning things up and maintaining high growth.''

Pushing in the opposite direction are banks and investment trusts, which want to continue pumping
money into the economy to bolster their profits. Analysts say they have been adept at evading the
rules with clever and complex financial products.

Even though most banks contacted in recent weeks said they had stopped offering such products,
several said they had found ways to continue to sell them. Analysts say it is unclear just how
pervasive such products are.

Something similar happened here in the 1990s, when aggressive financing by banks and investment
trusts led to big losses, huge bankruptcies and new regulations.

Chinese banks appear to be stronger this time around. Over the last decade, huge restructuring and
government recapitalization efforts allowed the big state banks to clean up their balance sheets and
eventually raise billions of dollars in public stock offerings.

Still, analysts say the lack of transparency about lending makes it difficult for investors and
regulators to assess the risks facing some banks.

``Essentially what you had was a bank using a trust company to package that bank's own loans into a
wealth management product, which was then sold to its own customers,'' says Jason Bedford, a manager
at KPMG in Beijing. ``The problem is you don't have a clear transfer of risk off of the balance
sheet.''

With the help of trusts, banks are repackaging loans as investments, analysts say, thereby making
room to issue additional loans. And trusts are turning to wealthy bank clients to raise new capital.

Banking customers are also contributing to the continued use of partnerships between banks and
investment trusts, analysts say. These customers, which include corporate clients, have been
frustrated with the low yields they earn on savings deposit -- close to 2 percent -- and with the
dearth of alternative investment options. Higher-yielding investment trusts were seen as ideal.

``Banks have a lot of demands from high-net-worth depositors,'' particularly big corporate accounts,
Mr.~Bedford at KPMG said.

Stephen Green, a Shanghai-based analyst at Standard Chartered Bank, describes investment trusts as
financial intermediaries, filling in gaps in the financial markets and acting as a jack of all
trades -- part hedge fund, mutual fund, private equity firm and bank lender.

Trusts are also a vital source of financing for private companies, and lately real estate
developers, which often have a difficult time securing loans from state-owned banks -- largely
because the government is trying to restrain real estate development.

But many investment trusts are state-owned, and they often finance state infrastructure projects.
For instance, the Xi'an Trust, which is owned by the government in the city of Xi'an, is providing
money for land, water and electricity projects to build a new high-tech base for the city.

For wealthy investors like Zhang Zhenlei, though, investment trusts offer attractive interest rates,
about double the savings deposit rate.

Mr.~Zhang, who is a V.I.P. client at his state-owned bank, says he spent a half hour filling out
paperwork so that he could invest in a trust and help finance an infrastructure company in Inner
Mongolia and also a highway development company in southern China's Guangdong Province.

``It turned out very well,'' says Mr.~Zhang, who has already gotten his money back, with interest
payments, after the six-month lockup period ended.

As for the crackdown on trust products, regulators may have difficulty, analysts say. A wealth
manager at I.C.B.C., one of China's biggest state-owned banks, was offering investment trusts to
banking clients two weeks ago.

``Now we have a 57-day trust product with a yield rate of 2.6 percent,'' the manager told one
prospective client over the phone. ``It's available from today. Though financial products aren't
authorized for principal to be guaranteed, usually we can orally guarantee your principal.''

\section{India May Soon Resolve BlackBerry Dispute}

\lettrine{I}{ndian}\mycalendar{Aug.'10}{18} officials and Research In Motion, the maker of popular
BlackBerry devices, appear to be making progress toward resolving a battle over the government's
ability to monitor encrypted e-mail and instant messages.

Government officials, wireless phone companies and R.I.M. have been hashing out details of how the
monitoring would work and are hopeful that they will have a plan in place by an Aug.~31 deadline set
by New Delhi, a representative for cellular companies said Tuesday.

India has said it wants to be able to monitor encrypted corporate e-mail and messages sent over
BlackBerry Messenger, a service that allows users to chat with one another. Officials here and in
other countries like the United Arab Emirates have expressed concern that the services could be used
by terrorists to plan and carry out attacks.

R.I.M., which is based in Waterloo, Canada, has indicated that it will provide a way to monitor
BlackBerry Messenger chats by the government-mandated deadline, said Rajan Mathews, director general
of the Cellular Operators Association of India, a trade group.

For corporate e-mail, R.I.M. would not directly provide the government with messages but would
identify corporations whose servers hold readable, or unencrypted, versions of messages, Mr.~Mathews
said. Indian authorities could then seek access to the messages from the corporation through a court
order or other legal processes.

R.I.M. is proposing that ``we can help identify the enterprise and location of the enterprise, but
then you have to go to the enterprise,'' Mr.~Mathews said. Officials are working to develop a
streamlined process to ensure that the government gets access to messages that it wants quickly, he
added.

This approach, if accepted by India, would allow R.I.M. to live by its public commitment to not
directly help governments decrypt corporate e-mail messages sent over its devices and servers.
Analysts say that the company has been successful in large part because corporate and government
clients trust that it closely guards the security of their messages.

Reaching a deal with India would help ensure that R.I.M. did not lose access to an important and
fast-growing market. There are an estimated one million BlackBerry users here, and the popularity of
the devices is growing as more Indians use e-mail and smartphones. India is already the world's
second-largest wireless market after China.

An official at the Home Ministry, which is responsible for domestic security, declined to comment on
Tuesday. A spokesman for R.I.M. in India declined to comment.

Indian law and stipulations in wireless licenses give law enforcement officials here explicit and
broad powers to monitor voice and data communications that could be useful in investigating crimes.
The government said last week, for instance, that it was not concerned about consumer e-mail on
BlackBerry devices because it can already monitor that.

Also on Tuesday, wireless phone companies said they had received a formal notice from the government
to shut off BlackBerry Messenger and corporate e-mail services on Aug.~31 -- which appeared to be
preparation for the possibility that the government and R.I.M. were not able to reach an agreement
by then.

\section{Taiwan Parliament Approves China Trade Deal}

\lettrine{T}{aiwan}\mycalendar{Aug.'10}{18}'s Parliament passed a trade deal with China on Tuesday,
the most significant agreement between the political foes of 60 years and one that binds Taiwan's
economy to the mainland while opening doors to other countries.

Legislators approved all but one piece of the economic cooperation framework agreement after a day
of debate and protest, meaning the deal will become law on Jan.~1. Negotiators initially signed it
in June.

The deal, which slashes tariffs on about 800 products, is considered a catalyst for similar
agreements with other countries that could ease Taiwan's diplomatic isolation, imposed by China, and
raise its competitiveness as an export-dependent economy.

The government has said the deal will create 260,000 jobs, while one private forecast has predicted
that the net effect will be a 5.3 percent improvement in Taiwan's gross domestic product by 2020.

China sees the deal as part of its long-term plan to draw Taiwan under its rule, luring the island
with economic sweeteners even as it continues a military buildup against a territory it has regarded
as its own since the Chinese civil war ended in 1949.

As part of that charm offensive, Beijing is unlikely to stop Taiwan from seeking deals with other
trade partners to help its economy. China's apparent blessing for an agreement with Singapore,
Taiwan's sixth-largest trading partner, is a strong first sign of agreements to come.

Approval of the trade pact, which came after a brawl last month in a divided Parliament, gives
Taiwan negotiators a stronger mandate to return to the table with China for talks on further tariff
reductions and other economic cooperation.

Opponents in Taiwan had feared that China wanted the trade pact to assert its claim of sovereignty
over the self-ruled, democratic island of Taiwan by making the economies more interdependent.

Dozens of opponents protested Tuesday outside Parliament, with television showing some in their
underwear, while opposition legislators yelled and displayed giant placards during the debate.

``People are clear this isn't going to be good for Taiwan,'' Chen Ming-wen, a legislator, told
Parliament. ``It's going to steer our economy closer to China.''

\section{From the Streets To Old Trafford}

\lettrine{T}{he}\mycalendar{Aug.'10}{18} most intriguing move so far in this summer's soccer
transfer period has not been any of the household names stockpiled by Manchester City, Chelsea,
Barcelona or Real Madrid.

Rather, it is the gamble that Manchester United has made in paying more than \$11.5 million for a
street player who has not yet played a game in Portugal's top league.

The first-division club Vit\'oria Guimarães had signed Tiago Manuel Dias Correia for nothing five
weeks earlier. But Manchester United paid the buyout clause in his contract last week when it became
known that other big clubs, Real Madrid among them, were sniffing around the striker known simply as
B\'eb\'e.

The nickname has stuck since childhood, when an older brother called him ``baby.'' Born to
immigrants from Cape Verde, picked up off the streets where they lived a rough life, they were taken
as youngsters into a charitable institution, a refuge and orphanage in Loures, north of Lisbon.

By then, the streetwise skills, the improvisation that was so scarce at the World Cup this summer,
were bred into B\'eb\'e.

``He's raw material, but we can work with that,'' Manchester United Manager Alex Ferguson said.
``It's a fairy tale when you read about his background.

``When we identify someone with potential, we normally assess that over a period of time. But we
have a good scout in Portugal, and he's very bright. He's got a tremendous instinct about the boy
and other clubs were starting to hover on the boy, so we made a quick decision.''

``Sometimes you have to work on an impulse, and we're good at developing young players,'' Ferguson
said.

B\'eb\'e is not so young. He turned 20 last month, and it was his agent, Jorge Mendes, who guided
Cristiano Ronaldo from Manchester United to Real Madrid. Mendes also represents Jos\'e Mourinho, who
became Real Madrid's coach this summer.

It would have been Mendes who inserted the buyout clause in B\'eb\'e's contract when he stepped up
to Guimarães this summer. Among those whom Ferguson quizzed about B\'eb\'e's potential was Carlos
Queiroz, Portugal's national team coach and a former assistant to Ferguson at Manchester United.

When Ferguson talks of impulse or instinct, he is hoodwinking his audience. Before he signs any
player, from a schoolboy to a world-renowned star, he badgers people about the lifestyle, the
family, the roots of that individual. And no deal is completed until Manchester United's doctors
have run a battery of tests on the player. Ferguson flew to Lisbon the day before he authorized the
signing of B\'eb\'e. He was there to provide a character reference for Queiroz, who was accused of
having intimidated and insulted doping officials who arrived unannounced in May at Portugal's
training camp before the World Cup.

``He's a fantastic coach and teacher,'' Ferguson told the tribunal. ``His main purpose in life is to
develop young people, to inspire them and to make sure they turn out good human beings. He's one of
the good guys.''

And a good source of players. Ferguson signed Ronaldo as a teenager from Sporting Lisbon, as well as
Anderson and Nani, who are current first-team players at Manchester United. But when Ferguson says
his club is good at developing young players, the greatest testimony to that came again at Old
Trafford on Monday night.

Manchester United overwhelmed Newcastle, 3-0, in its first match of the season. Paul Scholes, now
35, was the complete playmaker, a man developed from a boy born at the nearby Hope Hospital, now
embarking on his 16th season with the club.

When Scholes made his final pass, and Ryan Giggs volleyed the third goal, it was a glorious example
of working with raw material. Giggs is 37, starting his 21st season; he has scored in every one of
them.

Some legacy, then, for B\'eb\'e to live up to.

It is still some journey for a young man who two years ago was hoping to play in the Homeless World
Cup. That tournament is for enthusiastic amateurs. The next edition will feature 30,000 players from
64 nations on Copacabana Beach in Rio de Janeiro from Sept.~19-26.

``I had the dream of playing one day for a major club,'' B\'eb\'e said last week. ``Football can
change lives, very much.''

He might have spoken, in that simple sentence, for millions around the world who are being drawn
into projects that use soccer to give street youths hope. The Mathare Youth Sports Association in
Kenya has been running since 1987 as a way of diverting youngsters away from hopelessness, drugs,
violence and AIDS.

By the sound of it, B\'eb\'e could have lost his way but for his talent. ``He's a player who is the
fruit of street football,'' Jorge Paixão said on Radio Antena 1 last week. Paixão coached B\'eb\'e
in his one season at Estrela da Amadora in Portugal's third division before Guimarães signed him as
a free agent.

``Nowadays,'' Paixão added, ``players are schooled in the clubs. He has none of this. He's old
school. He learned to play in the street and has that natural creativity. He improvises because he
has quality and a set of characteristics that are difficult to find in a single player -- he is
tall, good in the air, technically gifted, fast and strong.''

B\'eb\'e, who is 6 feet 2 inches, can play wing or striker.

Time will tell whether he adapts to Manchester's team play. He must learn his first words of
English. He must adapt to climate, training and expectations.

Paixão, his former coach, has identified the prime asset: improvisation. I remember Sandor Barcs,
the president of Hungary's soccer federation during the Magical Magyars era of the 1950s, explaining
decades later why Hungary lost that magic.

``Listen,'' Barcs said in a Budapest street. ``What do you hear?'' There was nothing but the sound
of traffic. ``What we had back then was boom, boom, boom. Everywhere, you would see kids kicking the
ball, or an improvised ball, against the walls or railings. When you lose that, you lose the level
of play.''

\section{A Chip That Digests Data and Calculates the Odds}

\lettrine{C}{omplex}\mycalendar{Aug.'10}{18} as they may seem, traditional computers deal in a
simple art. They rely on tiny switches that turn on and off, producing the streams of ones and zeros
that software eventually translates into something meaningful to a human.

Some computer scientists find solace in the degree of certainty that comes from trading in yes-or-no
operations.

Lyric Semiconductor, a start-up that emerged from work at the Massachusetts Institute of Technology,
looks to forgo this certainty in favor of probability. It unveiled plans this week to build a chip
that can compute likelihoods. Such technology may help figure out which book someone will want to
buy on Amazon.com or help create a better gene-sequencing machine.

``We decided there are lots of probability problems out there that are so important they deserved
their own hardware,'' said Ben Vigoda, the co-founder and chief executive at Lyric.

Most of Lyric's nearly \$20 million in financing has come from the Department of Defense and what
the company will only refer to as a three-letter government agency. The military interest revolves
around Lyric's approach to determining the relationship between bits of information in a stream of
communications and separating noise from useful data.

Over time, Lyric plans to step out from this military work and offer its wares to corporations
working with large sets of data.

Mr.~Vigoda pointed to companies like Amazon and Google as possible beneficiaries of Lyric's
technology.

Today, retail Web sites throw lots of computing horsepower and algorithms at the prediction engines
that try to determine which product someone might want to buy based on their past purchases and
ratings. This is a grand probability puzzle, and it taxes traditional computers because they are
made to deal with black-and-white questions.

Lyric, by contrast, can tell a computer that someone buys a mystery book 60 percent of the time and
a science fiction book 30 percent of the time, and then hunt for the probability that the reader
will like a new title that touches on these interests.

Similarly, the technology could be used to determine the best search results for an individual, or
the likelihood that an e-mail message is spam. It could determine if a recent credit card purchase
is fraudulent by comparing it with past purchases.

Crucially, Lyric claims it can perform these calculations using just a handful of the transistors
inside a chip rather than the hundreds it takes today because it has created algorithms and chip
designs geared toward probability. This means that companies could spend far less on computing gear
for these complex tasks and save energy and space.

Lyric's approach is similar to that of other chip companies that have filled niches in the past.
Nvidia, for example, boomed after it created graphics chips that made things like video games and
nuclear weapons simulations run faster.

Jon Stokes, the author of ``Inside the Machine: An Illustrated Introduction to Microprocessors and
Computer Architecture,'' said Lyric faced ``a really tough road ahead.''

``To take it to the next level requires an investment that reaches into the hundreds of millions of
dollars, and it's very risky and very hard to raise money for that sort of play right now,''
Mr.~Stokes said.

This week, the company unveiled one take on the technology that it hoped to license to makers of the
memory that went into gadgets and computers. Lyric produced a small chip that could help catch
errors that arose as those devices read and wrote data.

\section{Blagojevich, Guilty on 1 of 24 Counts, Faces Retrial}

\lettrine{P}{rosecutors}\mycalendar{Aug.'10}{18} here once said that the conduct of Rod
R.~Blagojevich, the former governor of Illinois, was so despicable it would make Abraham Lincoln
``roll over in his grave,'' but 12 jurors in the federal corruption case against him were apparently
not all so certain.

After deliberating for 14 days, the jury found Mr.~Blagojevich guilty of a single criminal count --
making false statements to the Federal Bureau of Investigation, which carries a maximum sentence of
five years in prison, one of the least severe penalties in the charges against him.

The jurors also said they could not reach a unanimous verdict on 23 of the 24 counts against him,
including an accusation that he had tried to sell an appointment to fill the Senate seat once held
by President Obama.

Prosecutors immediately announced plans for a retrial, but the outcome was seen as something of a
victory, at least for now, for Mr.~Blagojevich, a Democrat and lifelong politician from this city,
who had always proclaimed his innocence and did so again as he left court.

``We have a prosecutor who has wasted millions and wants to keep spending money to persecute me,''
Mr.~Blagojevich told a swarm of reporters in the lobby of the courthouse after the judge declared a
mistrial on the bulk of charges against him, including racketeering, wire fraud, bribery and
extortion conspiracy.

Inside the courtroom, as the jurors filed in to deliver their findings, Mr.~Blagojevich clasped his
hands together so tightly that the veins on the back of his hands bulged, and his wife, Patti,
waited with her eyes closed at times, exhaling deep breaths.

As the false statements conviction was read aloud, Mr.~Blagojevich shook his head. He later said he
was innocent of lying to federal agents, too, and would soon appeal.

During an interview with agents, Mr.~Blagojevich once said that he did not keep track of or want to
know who was contributing to his campaign or how much was being given -- comments jurors found to be
a lie.

As for the more serious counts against Mr.~Blagojevich, it was hard to know precisely what had led
the jurors, who seemed during the trial to be a particularly attentive, studious bunch, to find
themselves unable to reach an agreement. One of the jurors told The Associated Press that the panel
was deadlocked 11 to 1 in favor of conviction on the charge of trying to sell or trade the Senate
seat.

After nearly two months of testimony and several weeks of deliberation, the overall result here was
seen as a significant setback for federal authorities, who arrested Mr.~Blagojevich almost two years
ago to stop what they described unambiguously as ``a political corruption crime spree.''

Patrick J.~Fitzgerald, the United States attorney, declined to answer questions, but gave a brief
statement calling for respect for the jurors and their service, then added, ``We're about to get
ready for a retrial.''

A spokesman for Mr.~Fitzgerald said he could not comment on the cost of the prosecution, an effort
that had spanned several years.

In broad terms, prosecutors had accused Mr.~Blagojevich of turning his state office into a criminal
enterprise to benefit himself, citing what they said were brazen efforts to get political
contributions in exchange for everything from legislation to help a local pediatric hospital, to
state money for a school, to a law to benefit the horse track industry to, most infamously, an
appointment to fill Mr.~Obama's Senate seat.

The trial featured prosecution testimony from Mr.~Blagojevich's former chiefs of staff (some of whom
pleaded guilty to charges in the case) as well as numerous secretly recorded telephone calls in
which Mr.~Blagojevich or his aides seemed to seek financial benefits for official state actions.

Still, the trial had also exposed two notions that seemed to favor the defense: Mr.~Blagojevich
rarely managed to succeed in getting whatever financial benefits he might have sought, and much of
what he did might also be viewed by some as common, if especially ugly, political deal making.

Jurors were unable to reach a verdict in any of four counts against Mr.~Blagojevich's older brother,
Robert, who had briefly served as a fund-raiser for the former governor and who left the courthouse
Tuesday saying, as he has all along, that he did nothing wrong and intends to prove that in the next
trial.

For months, the events surrounding the former governor mortified Illinois, a state that is hardly
na\"ive about politics, setting off a wave of commissions and committees with proposals to reform
state ethics rules, campaign finance laws and policies on making state business open to public view.

Some political leaders and others pointed to Mr.~Blagojevich's conviction for false statements as a
satisfactory conclusion to the case, and a repudiation of bad government. But others, including some
most involved in the state's reform efforts, said Tuesday evening that they were disheartened by the
unresolved 23 counts, and concerned about what the lessons in any of this might now be.

``It's been a painful trial,'' said Cindi Canary, director of the Illinois Campaign for Political
Reform. ``It's been very distracting in a state that's up to its eyeballs with problems.''

She went on, ``The big question now is, if we draw up 12 new people, will they find something
different?''

Andy Shaw, the executive director of the Better Government Association, a local watchdog group, said
he had not changed his view of the former governor.

``He and his cabal of thugs hijacked our government,'' Mr.~Shaw said. ``It was disgraceful and
reprehensible. But, at least to these 12 jurors, not illegal.''

Mr.~Blagojevich, 53, was the fourth Illinois governor in recent memory to face the possibility of
prison. It seemed a remarkable shift for a politician who won his first race for governor, in 2002,
by portraying himself as a reformer who would clean up state politics after his predecessor, George
Ryan, a Republican, was convicted of corruption and sent to federal prison.

On Dec.~9, 2008, Governor Blagojevich, then in his second term, was awoken around dawn at his
Chicago home and arrested. Government agents had secretly recorded some 500 hours of telephone calls
with Mr.~Blagojevich and his advisers, and a portion of those recordings became a crucial element of
the prosecution's case.

As a whole, the recordings painted Mr.~Blagojevich as an insecure, isolated man who was jealous of
Mr.~Obama, obsessed with schemes for getting out of being governor as quickly as possible, worried
about deteriorating family finances, who rarely went to his office and cursed a lot.

For months, Mr.~Blagojevich, a former congressman who came up through the political operation of his
father-in-law, an alderman from Chicago's Northwest Side, proclaimed his innocence to all who would
listen. He was impeached and removed by state lawmakers, but he refused to fade away.

Leading up to his trial, the former governor, who is known for his thick mop of jet-black hair,
transformed from someone who once imagined himself as a serious contender for president to something
of a national spectacle, appearing on the television show ``Celebrity Apprentice,'' impersonating
Elvis Presley (a hero of his), writing a memoir, hosting a weekly radio talk show and even turning
down an offer to play for a minor league baseball team in his home state, the Joliet JackHammers.

Mr.~Blagojevich pleaded not guilty and often said that he would testify at his trial. In the end, he
did not. The defense presented no witnesses.

The prospect of a new trial on the 23 undecided counts, just as the fall political season arrives,
was sure to alarm Democrats here, who are locked in tight races for governor (the party's nominee is
Pat Quinn, the former lieutenant governor who replaced Mr.~Blagojevich after he was impeached) and
for the Senate seat that Mr.~Blagojevich was accused of trying to peddle (Alexi Giannoulias, the
Democratic nominee, was subpoenaed but did not testify at the trial).

A new trial also threatens once more to draw in officials from the White House and leaders in
Washington, mainly because of the charges tied to the Senate seat that Mr.~Blagojevich was required
by law to fill when Mr.~Obama was elected president.

\section{Tea Party Choice Scrambles in Taking On Reid}

\lettrine{S}{harron}\mycalendar{Aug.'10}{18} Angle leaned across a table in her campaign office
here, defending her suddenly embattled campaign to defeat Senator Harry Reid, the Democratic
majority leader, under the gaze of a half-dozen advisers and an official videographer packed into
the room.

``We always have known that this was going to be a close race,'' Ms.~Angle said in a rare interview.
``Senator Reid has never just given up the fight, you know? He has always been a tough opponent.''

But after a year in which Mr.~Reid has been at the top of every list of endangered Democrats --
there is no one Republicans would like more to unseat -- Mr.~Reid sees a lifeline in Ms.~Angle, a
former state assemblywoman who, with the backing of the Tea Party, overcame a field of establishment
Republicans to win the party's nomination.

Since Ms.~Angle won, her campaign has been rocked by a series of politically intemperate remarks and
awkward efforts to retreat from hard-line positions she has embraced in the past, like phasing out
Social Security. There have also been a staff shake-up and run-ins with Nevada journalists,
including one in which a television reporter chased her through a parking lot trying to get her to
answer a question.

Republicans in this state are concerned that what had once seemed a relatively easy victory is
suddenly in doubt, with signs that Ms.~Angle's campaign is scrambling to regroup.

``Reid had no chance to win before,'' said Danny Tarkanian, one of the Republicans who lost to
Ms.~Angle. ``He has a shot to win now. He could still lose, but I have to say he is favored.''

Ms.~Angle's primary victory was a testimony to the power of the Tea Party. And if she wins in
November, it will have huge national implications: a Tea Party candidate will have taken down the
most powerful member of the Senate and a close ally and friend of President Obama.

But some of her conservative positions could prove a hurdle come November. She has, for example,
called for the elimination of the Energy Department and the Environmental Protection Agency,
denounced the BP compensation fund for victims of the oil spill as a slush fund, and suggested that
her candidacy was a mission for God.

``Our focus groups that we've done, my staff tells me this -- her statements, they can't believe
that anyone would say that,'' Mr.~Reid said. ``That's why most of our TV just has her talking.''

For Ms.~Angle, this has been the political equivalent of a bungee jump. In Mr.~Reid, she has an
opponent who has a reputation as something of a knife-fighter and who appears to have access to
unlimited money and a campaign staff that includes the consultant who made advertisements for
President Obama.

In recent days, there have been a number of events that suggest Ms.~Angle is trying to change the
way she is doing things. The interview itself was a break from past practice, and after that, she
took a few questions from reporters at a brief news conference. Her appearance was filled with sharp
attacks on Mr.~Reid, including an assertion that it was Mr.~Reid who was a threat to Social Security
(though she offered that line only upon being prompted by an aide after she said she was done with
her remarks).

Her recent advertisements have been sharper and more focused, including one accusing Mr.~Reid of
wanting to raid the Social Security Trust Fund. Jon Ralston, a political columnist for The Las Vegas
Sun, wrote that the advertisement was ``quite well done. It's also totally disingenuous.''

In the interview -- which came as she has been criticized by Republicans for avoiding the Nevada
press, and as Washington Republicans have nudged her to retool her campaign -- Ms.~Angle said she
was in step with most Nevada voters and dismissed Mr.~Reid's contention that she was too
conservative.

``I'm sure that they probably said that about Thomas Jefferson and George Washington and Benjamin
Franklin,'' she said. ``And truly, when you look at the Constitution and our founding fathers and
their writings, the things that made this country great, you might draw those conclusions: That they
were conservative. They were fiscally conservative and socially conservative.''

In the course of the interview, Ms.~Angle spoke slowly and cautiously. She appeared reluctant to
engage, frequently reciting stock answers to questions.

Still, even as she criticized Mr.~Reid, she also criticized Republicans, including former President
George W.~Bush, for excessive spending. ``There is a lot of responsibility to be spread around --
plenty,'' she said. ``Harry Reid has been around for 27 years, through several administrations. He
was there for the stimulus before Obama and voted for it.''

Ms.~Angle did not elaborate on what kind of stimulus she was referring to, though Tea Party leaders
have been critical of Mr.~Bush for allowing the deficit to balloon under his watch, the result of
both the unfunded Medicare drug benefit and a series of deep tax cuts. An aide to Ms.~Angle, Jarrod
Agen, later explained that in talking about ``the stimulus before Obama,'' Ms.~Angle was referring
to the bank bailout plan that Congress passed under Mr.~Bush.

If Mr.~Reid is doing better than he once was, it is still relative; he is a politician in deep
trouble. A Mason-Dixon poll last week found that 51 percent of Nevadans held an unfavorable opinion
of him, a toxic number for an incumbent. That poll found Mr.~Reid and Ms.~Angle in an effective tie.

``I'll say this about Angle: I still think when we get to the end, it's still going to be about
Harry Reid and whether Nevada voters want to get rid of him and send a message to Washington,'' said
Brad Coker, managing director of Mason-Dixon. ``They may still hold their nose and vote for Sharron
Angle even if they don't agree with a lot of things that she says and does.''

Mr.~Reid's advisers made clear that the only way they could win was to make Ms.~Angle so distasteful
to Nevada voters that they would vote for Mr.~Reid or someone else -- it is possible here to vote
for ``none of the above'' -- or stay home

``I'm not discounting her,'' Mr.~Reid said. ``In the spite of the work we've done, people need to
understand more about her. There are some unusual stands she has.''

Republican leaders in Washington and Nevada said their best hopes now were that Ms.~Angle would not
make any mistakes and that Mr.~Reid would simply sink under the weight of his unpopularity, his
association with widely scorned Washington positions and his tendency to stumble.

Yet Republicans have found that Ms.~Angle is not particularly open to suggestions from outsiders.
Asked if her campaign had done anything wrong, she responded, ``I don't think so.''

And asked if she would be doing anything differently going forward, Ms.~Angle paused again and said,
``I can't think of anything like that.''

\section{Obama Rolls Out Midterm Metaphor}

\lettrine{I}{n}\mycalendar{Aug.'10}{18} road-testing his stump speech for the midterm elections,
President Obama would like American voters to think of a car that has been driven into a ditch.
(Take a wild guess who was at the wheel.)

``You had a group of folks who drove the economy, drove the country, drove our car into the ditch,''
Mr.~Obama told some 200 party faithful sipping sauvignon blanc by the pool Monday night at a Los
Angeles fund-raiser.

The president may change a word here or refer to a different member of Congress there, but the basic
arc of the anecdote has been the same at his stops so far this week on a three-day, five-state
swing.

The White House wants to frame the November elections as a clear choice between the Democrats who
have put the country on the road to recovery, anemic though it may be, and the Republicans who, in
the White House view, got the country into the economic mess to begin with.

``We put on our boots and walked into the ditch -- it's muddy and hot and dusty and bugs everywhere
-- and we're pushing,'' Mr.~Obama said of the efforts of the White House and its Democratic partners
in Congress.

``And we're slipping and sliding and sweating, and the other side, the Republicans, they're standing
there with their Slurpees watching us,'' Mr.~Obama said, building up to the punch line, which he has
been refining (minus the Slurpees) for several months. ``Finally we get this car to level ground.
Finally we're ready to move forward, go down that road once again to American prosperity, and what
happens? They want the keys back.''

``Well, you can't have the keys back,'' he said to cheers in Milwaukee on Monday. ``You don't know
how to drive. You got us into the ditch.''

It is not an easy sales pitch, in no small part because Republicans have countered that Mr.~Obama's
stimulus spending and health care and financial overhauls are slowing the recovery and keeping the
unemployment rate high.

Indeed, far from running from it, Republicans are embracing the obstructionist label that Mr.~Obama
has been using liberally.

When White House officials started pointing out to reporters recent remarks by the Republican
leader, Senator Mitch McConnell of Kentucky, that he wished Republicans had been able to obstruct
more, Mr.~McConnell's office fired off e-mails standing behind the remarks.

``Like most Americans, Senate Republicans opposed a government takeover of health care, a bill that
nearly every week we learn won't reduce health care costs for families, but will take a
half-trillion dollars out of Medicare while kicking seniors off the plans they like and raising
taxes on small business,'' Don Stewart, Mr.~McConnell's spokesman, wrote in an e-mail.

The president and the majority leader have also tangled over legislation to give small businesses a
variety of tax breaks and incentives, including easier access to loans. ``They won't even let it go
to a vote,'' Mr.~Obama said in Seattle.

Republicans counter that the bill is misguided and costly, and that it will not help small
businesses. And they blame the Democrats for delaying a vote on it.

``From the beginning,'' an e-mail from Mr.~Stewart said, ``this bill clearly wasn't a priority to
them until they realized that they didn't have anything to talk about when they go home in August.''

White House officials nonetheless expressed optimism Tuesday that the effort to frame the midterm
elections as a battle between hope and obstructionism would be successful.

Mr.~Obama's speechwriter, Jon Favreau, was traveling with him on the three-day swing as the
president worked on his message, and the two men have tossed ideas back and forth over what works
and what does not. The ``you can't drive'' anecdote, which made its debut as just two lines in a
speech a few months ago, has increasingly found favor with the president as audiences have responded
to it, administration officials said.

Of course, on this trip, Mr.~Obama has been preaching to the choir.

``You notice that when you move forward in your car, you put it in 'D'; when you want to go
backwards, you put it in 'R?' '' Mr.~Obama said to wild cheers in Seattle, where, for the first
time, he pantomimed drinking a Slurpee as he caricatured his Republican opponents. ``Back into that
ditch! Keep that in mind in November. That's not a coincidence.''

Jokes aside, the effort to position Congressional Democrats as more concerned about the economy than
their Republican counterparts is crucial to Mr.~Obama's hopes that his party will retain control of
Congress in November. With that goal in mind, Mr.~Obama is trying to harness the success of his 2008
drive toward the White House.

``You remember our slogan during the campaign -- 'Yes, we can?' '' Mr.~Obama asked in Seattle.
``Their slogan is, 'No, we can't.' ''

Clearly enjoying himself, he started to grin. ``It's really inspiring, that vision they have for the
future -- gives you a little pep in your step when you hear it, doesn't it?''

Crowding Air Force One

SEATTLE (AP) -- The North American Aerospace Defense Command said Tuesday that military fighter jets
were scrambled to respond to an air-space violation near Air Force One here.

John Cornelio, a spokesman, said the jets were sent from Portland, Ore., as the president was
visiting here after a report that an aircraft had entered the restricted airspace. Mr.~Cornelio said
the aircraft left the restricted area before the Air National Guard jets arrived.

\section{Moose Offer Trail of Clues on Arthritis}

\lettrine{I}{n}\mycalendar{Aug.'10}{18} the 100 years since the first moose swam into Lake Superior
and set up shop on an island, they have mostly minded their moosely business, munching balsam fir
and trying to evade hungry gray wolves.

But now the moose of Isle Royale have something to say -- well, their bones do. Many of the moose,
it turns out, have arthritis. And scientists believe their condition's origin can help explain human
osteoarthritis -- by far the most common type of arthritis, affecting one of every seven adults 25
and older and becoming increasingly prevalent.

The arthritic Bullwinkles got that way because of poor nutrition early in life, an extraordinary
50-year research project has discovered. That could mean, scientists say, that some people's
arthritis can be linked in part to nutritional deficits, in the womb and possibly throughout
childhood.

The moose conclusion bolsters a small but growing body of research connecting early development to
chronic conditions like osteoarthritis, which currently affects 27 million Americans, up from 21
million in 1990.

Osteoarthritis's exact cause remains unknown, but it is generally thought to stem from aging and
wear and tear on joints, exacerbated for some by genes. Overweight or obese people have greater
arthritis risk, usually attributed to the load their joints carry, and the number of cases is
increasing as people live longer and weigh more.

But the moose work, along with some human research, suggests arthritis's origins are more complex,
probably influenced by early exposures to nutrients and other factors while our bodies are
developing. Even obesity's link to arthritis probably goes beyond extra pounds, experts say, to
include the impact on the body of eating the wrong things.

Nutrients, experts say, might influence composition or shape of bones, joints or cartilage.
Nutrition might also affect hormones, the likelihood of later inflammation or oxidative stress, even
how a genetic predisposition for arthritis is expressed or suppressed.

``It makes perfect sense,'' said Dr.~Joanne Jordan, director of the Thurston Arthritis Research
Center at the University of North Carolina. ``Osteoarthritis starts way before the person knows it,
way before their knee hurts or their hand hurts. It's very clear that we're going to have to start
looking back'' at ``things in the early life course.''

Such research could lead to nutritional steps people can take to protect against osteoarthritis, a
condition that is often painful or debilitating, and according to federal data, costs billions of
dollars annually in knee and hip replacements alone.

``It would be helpful to know if we want to make sure pregnant moms are taking certain vitamins or
if you need to supplement with such and such nutrition,'' said Dr.~David Felson, an arthritis expert
at Boston University School of Medicine. ``The moose guy is right in that we probably should study
weight or some other nutritional factor almost through adolescence when the bones or joints have
stopped forming.''

The ``moose guy'' is Rolf Peterson, a Michigan Technological University scientist on the Isle Royale
project, which began in 1958 and is reportedly the longest-running predator-prey study.

For half the year, Dr.~Peterson and his colleagues are the only humans allowed on the 45-mile-long
island, part of a national park. They stay in yurts, a log cabin or a wood-stove-heated lodge,
navigate the wilderness without roads or cars, and share a single staticky phone line. They analyze
everything from wolves' moose-hunting strategies to moose feces. Collecting bones of more than 4,000
moose, they noticed that out of 1,200 carcasses they analyzed, more than half had arthritis,
virtually identical to the human kind. It usually attacked the hip and instantly made the moose
vulnerable.

``Arthritis is a death sentence around here -- you need all four legs,'' Dr.~Peterson said. ``Wolves
pick them off so quickly that you don't even see them limping.''

What is more, the arthritic moose were often small, measured by the length of the metatarsal bone in
the foot. Small metatarsals indicate poor early nutrition, and scientists determined that the
arthritic moose were born during times when food was scarce, so their mothers could not produce
enough milk.

Dr.~Peterson said if the arthritis were caused by excess wear and tear on the moose's joints, that
would have meant that times of food scarcity occurred when the moose were already grown, since the
extra wear would have happened to moose walking farther to find edible plants. But the arthritic
moose had had plentiful food as adults.

For people, several historical cases may suggest a nutritional link. Bones of 16th-century American
Indians in Florida and Georgia showed significant increases in osteoarthritis after Spanish
missionaries arrived and tribes adopted farming, increasing their workload but also shifting their
diet from fish and wild plants to corn, which ``lacks a couple of essential amino acids and is iron
deficient,'' said Clark Larsen, an Ohio State University anthropologist collaborating with
Dr.~Peterson. Many children and young adults were smaller and died earlier, Dr.~Larsen said, and
similar patterns occurred when an earlier American Indian population in the Midwest began farming
maize.

British scientists studying people born in the 1940s found low birth weight (indicating poor
prenatal nutrition) linked to osteoarthritis in the men's hands, Dr.~Felson said. And Dr.~David
Barker, a British expert on how nutrition and early development influence cardiac and other
conditions, said ``studies of people in utero during the Great Chinese Famine'' of the late 1950s
found that ``40, 50 years later, those people have got disabilities.''

Overeating can be as problematic as undereating. Dr.~Lisa A.~Fortier, a large-animal orthopedist at
Cornell University's College of Veterinary Medicine, said she saw ``abnormal joint and tendon
development from excessive nutrition'' in horses overfed ``in utero or in the postnatal life,''
probably ingesting ``too much of the wrong type of sugar that may cause levels of inflammation.''

Dr.~Peter Bales, an orthopedic surgeon affiliated with University of California, Davis, Medical
Center, who has written about nutrition and arthritis, sees similar problems in overweight patients.
He said the causes were not as ``simplistic'' as ``carrying more weight around,'' but might involve
nutritional imbalances that could hurt joints and erode cartilage. Much is unknown about nutrition's
relevance. Isle Royale moose, for example, also seem to have genetic predispositions for arthritis,
suggesting that nutrition might be amplifying or jump-starting the genes.

``Genes are not Stalinist dictators,'' said Dr.~Barker, now at Oregon Health and Science University.
``What they do, how they're expressed, is conditional on the rest of the body. The human being is a
product of a general recipe, and the specific nutrients you get or don't get.''

Studying nutrition in people is much more complicated than in moose. Dr.~Peterson said the early
moosehood developmental window occurred in utero through 28 months, but humans' developmental time
frame lasted into the teens. Some experts say prenatal nutrition is most critical; others see roles
for nutrients after birth and beyond.

``Up until the growth plates close, which is through adolescence and even early adulthood, the
effects of nutrition are magnified,'' said Dr.~Constance R.~Chu, director of the Cartilage
Restoration Center at the University of Pittsburgh, who said nutrients might affect the number of
healthy cells in cartilage and its thickness. ``But in my opinion, it's relevant throughout life.''

\section{A Quest to Make the Morgan Seaworthy}

\lettrine{T}{he}\mycalendar{Aug.'10}{18} shipbuilders are long dead, their knowledge gone. The
shipyard no longer exists. No blueprints survive, nor ship's models.

But the Charles W.~Morgan is still here -- the world's last surviving wooden whaling vessel, built
in 1841. And restorers are spending \$10 million to turn the museum piece into a working ship able
to ply the unruly sea. They plan to sail the ship on its first voyage in nearly a century, opening a
new chapter in its long career.

Built in New Bedford, Mass., a bustling port known as the whaling capital of the world, the Morgan
sailed the globe for eight decades in pursuit of leviathans, escaping fire and cannibals,
Confederate raiders and Arctic ice. She brought home thousands of barrels of whale oil that lighted
homes and cities. She also delivered tons of baleen, the horny material from the mouths of certain
whales that was made into buggy whips and corset stays. In 1941, its centenary, the Morgan was towed
to Mystic Seaport for museum display and in 1966 was named a national historic landmark.

To learn as much as possible about the old ship and ensure its successful restoration, the
specialists here are turning to the art and science of imaging.

They are deploying lasers and portable X-ray machines, laptops and forensic specialists, cameras and
recorders, historians and graphic artists to tease out hidden details of the ship's construction and
condition. The project, begun in 2008, is producing a revealing portrait. It shows the exact
placement and status of many thousands of planks, ribs, beams, nails, reinforcing pins, wooden pegs
and other vital parts of the Morgan, giving shipwrights a high-tech guide for the rebuilding of the
historic vessel.

``When we're done, she'll be as strong or stronger as the last time she went to sea,'' Quentin
Snediker, director of the shipyard here, said during a restoration tour. ``So why not sail her?''

Minutes later, a specialist was firing X-rays through the ship's keel -- a massive oak spine
composed of several timbers, its length more than 90 feet. He was hunting for the large bronze pins
that hold the keel together. The restorers want to assess the so-called drift pins 169 years after
their installation and plan to replace or reinforce those that show deterioration. The pins are
between one and two feet long.

In a more sweeping assessment, specialists have sent laser beams racing across the Morgan, inside
and out, seeking to record inconspicuous details and form a digital archive of exact measurements.
The laser scans can track details as small as an eighth of an inch and have swept the entire ship
across its 114-foot length and 28-foot width -- once a cramped home to a crew of 35.

The scans have produced ``millions of points of information'' and a wealth of three-dimensional
images, said Kane Borden, research coordinator of the restoration. ``The results are pretty
spectacular to look at.''

Historians here say the restoration, for all its high-tech sophistication, is fundamentally about
remembering and honoring the past. The Morgan is the last representative of a fleet of 2,700
American whaling vessels that put the young country on the map and nourished its growing economy.
The industry was so important that the whaling life became a distinctive part of the American
experience.

``The scope and scale of it is something that people have no idea of today,'' said Matthew
Stackpole, a Mystic Seaport official. ``It was the first time the U.S.~presence was felt around the
world.''

The Morgan was built in the shipyard of Jethro and Zachariah Hillman and named after Charles Waln
Morgan, a Philadelphia Quaker who was its first main owner. The year of its inaugural voyage, 1841,
also marked the departure from New Bedford of another ship, carrying an aspiring author by the name
of Herman Melville. His whaling experience resulted in ``Moby Dick,'' and his realistic portrayals
of the industry gave it new visibility and status.

The Morgan completed 37 voyages from her home ports of New Bedford and San Francisco and sailed
farther than any other American whaler, according to historians. Near a remote Pacific isle, the
crew took up firearms to fend off canoes full of cannibals.

Captains could bring along their wives, and two of them served as expert navigators. The logs of
Charlotte Church, the wife of Capt.~Charles S.~Church, who sailed on the Morgan from 1909 to 1913,
recorded not only latitude, longitude, heading, distance and barometric pressure but the death of a
pet cat.

Dry humor marked her entries. ``We have two live pigs, one rooster, four cats and almost twenty
canary bird -- no fear of starving for a while.''

The ship's great luck in escaping from serious threats translated into bad luck for whales. Over the
decades, the ship's harpooners took in more than 2,500 of the behemoths to dismember their carcasses
and boil their blubber down into more than 50,000 barrels of oil. Whales were the petroleum wells of
the day.

``We're not apologizing for the past,'' Mr.~Stackpole, the seaport official, said of the hunt. ``But
we have to understand what happened and do better,'' especially in protecting whales threatened with
extinction because of the long decades of concentrated whaling.

On a beautiful day in late July, the restoration shipyard here was alive with specialists. Some cut
wood in the sawmill. The tons of replacement timbers include some carved from colossal hunks of live
oak salvaged after Hurricane Katrina in 2005 uprooted many trees along the Gulf Coast.

The Morgan rested high and dry on supporting beams, stripped of masts and most gear. Experts
scrutinized her for construction details to add to the growing library.

Bill Movalson, an official with Allpro Imaging, a company in Melville, N.Y., that makes portable
X-ray machines, took aim at the thick keel.

``Stand back,'' he advised.

The X-ray source, mounted on a tripod and looking like a large video camera, emitted a series of
beeps and then a continuous hum.

Mr.~Movalson stepped behind the keel to retrieve the exposed plate, which he fed into a small
machine. It read the plate by a method known as computed radiography, using lasers and electronics
rather than chemicals to uncover the invisible. In a few seconds, a ghostly image appeared on his
laptop screen.

``We got it,'' Mr.~Movalson said, pointing to the image of a bronze drift pin. ``Look, it's all
eaten away right at the seam.'' Decades of exposure to seawater had corroded the pin at the area
where it connected the keel to a protective timber known as the false keel.

``This is what we'd expect,'' Mr.~Snediker, the shipyard director, said while examining the image.
He added that the discovery of the corrosion ``validates the technique,'' showing that the X-ray
exposures are sensitive enough to distinguish faulty drift pins from those that show no
deterioration.

Mr.~Snediker said the project was seeking to harness every conceivable tool and method ``to learn as
much as we can.'' The restorers are even recording the comments of shipwrights who dismantle old
wooden structures in an effort to capture subtle insights into ship construction.

``We get layer upon layer of information,'' he said. The least technical of the methods centers on a
group of young artists who are sketching the various stages of the Morgan's disassembly and repair.
``It's great to see it through their eyes,'' Mr.~Snediker said.

Nearby was the project's headquarters, lodged in ramshackle offices full of files, papers and
computers. At his desk, Mr.~Borden, the research coordinator, showed how a computer served as the
library for the accumulated information. A blueprint-style image of the Morgan glowed on his
computer screen, as did a series of X-ray icons.

``You can zoom in,'' Mr.~Borden said, clicking on one. The underlying image revealed a central
joint.

If all goes as planned, the refurbished Morgan will be outfitted with new rigging in late 2012.

And the next year, in early summer, if the weather proves fair, the whaler will pull away from the
granite pier at the seaport and once again sail with the wind at her back, rocking though the waves,
making history. The goal is to visit places along the New England Coast of special significance to
whaling, like New Bedford.

``She's the last of her kind,'' Mr.~Stackpole said of the Morgan. ``We want her to be a living issue
rather than a dusty old artifact,'' her voyages a new page in the story of American whaling.

\section{Civilians to Take U.S.~Lead After Military Leaves Iraq}

\lettrine{A}{s}\mycalendar{Aug.'10}{19} the United States military prepares to leave Iraq by the end
of 2011, the Obama administration is planning a remarkable civilian effort, buttressed by a small
army of contractors, to fill the void.

By October 2011, the State Department will assume responsibility for training the Iraqi police, a
task that will largely be carried out by contractors. With no American soldiers to defuse sectarian
tensions in northern Iraq, it will be up to American diplomats in two new \$100 million outposts to
head off potential confrontations between the Iraqi Army and Kurdish pesh merga forces.

To protect the civilians in a country that is still home to insurgents with Al Qaeda and
Iranian-backed militias, the State Department is planning to more than double its private security
guards, up to about 7,000, according to administration officials who disclosed new details of the
plan. Defending five fortified compounds across the country, the security contractors would operate
radars to warn of enemy rocket attacks, search for roadside bombs, fly reconnaissance drones and
even staff quick reaction forces to aid civilians in distress, the officials said.

``I don't think State has ever operated on its own, independent of the U.S.~military, in an
environment that is quite as threatening on such a large scale,'' said James Dobbins, a former
ambassador who has seen his share of trouble spots as a special envoy for Afghanistan, Bosnia,
Haiti, Kosovo and Somalia. ``It is unprecedented in scale.''

White House officials expressed confidence that the transfer to civilians -- about 2,400 people who
would work at the Baghdad embassy and other diplomatic sites -- would be carried out on schedule,
and that they could fulfill their mission of helping bring stability to Iraq.

``The really big picture that we have seen in Iraq over the last year and half to two years is this:
the number of violent incidents is significantly down, the competence of Iraqi security forces is
significantly up, and politics has emerged as the basic way of doing business in Iraq,'' said
Anthony J.~Blinken, the national security adviser to Vice President Joseph R.~Biden Jr.~``If that
trend continues, and I acknowledge it is an `if,' that creates a much better context for dealing
with the very significant and serious problems that remain in Iraq.''

But the tiny military presence under the Obama administration's plan-- limited to several dozen to
several hundred officers in an embassy office who would help the Iraqis purchase and field new
American military equipment -- and the civilians' growing portfolio have led some veteran Iraq hands
to suggest that thousands of additional troops will be needed after 2011.

``We need strategic patience here,'' Ryan C.~Crocker, who served as ambassador in Iraq in from 2007
until early 2009, said in an interview. ``Our timetables are getting out ahead of Iraqi reality. We
do have an Iraqi partner in this. We certainly are not the ones making unilateral decisions anymore.
But if they come to us later on this year requesting that we jointly relook at the post-2011 period,
it is going to be in our strategic interest to be responsive.''

The array of tasks that military experts and some Iraqi officials believe American troops likely
will be needed for include training Iraqi forces to operate and support logistically new M-1 tanks,
artillery and F-16s they intend to acquire from the Americans, protecting Iraq's airspace until the
country can rebuild its air force and perhaps assisting Iraq's special operations units in carrying
out counterterrorism operations.

Such an arrangement would need to be negotiated with Iraqi officials, who insisted on the 2011
deadline in the agreement with the Bush administration for removing American forces. With the Obama
administration in campaign mode for the coming midterm elections and Iraqi politicians yet to form a
government, the question of what future military presence might be needed has been all but banished
from public discussion.

``The administration does not want to touch this question right now,'' said one administration
official involved in Iraq issues, adding that military officers had suggested that 5,000 to 10,000
troops might be needed. ``It runs counter to their political argument that we are getting out these
messy places,'' the official, speaking only on condition of anonymity, added. ``And it would be
quite counterproductive to talk this way in front of the Iraqis. If the Iraqis want us, they should
be the demandeur.''

The Obama administration had already committed itself to reducing America troops in Iraq to 50,000
by the end of August, a goal the White House on Wednesday said would be met. Administration
officials and experts outside government say, however, that implementing the agreement that calls
for removing all American forces by the end of 2011 will be far more challenging.

The progress or difficulties in transferring responsibility to the civilians will not only influence
events in Iraq but will provide something of a test case for the Obama administration's longer-term
strategy in Afghanistan.

The preparations for the civilian mission have been under way for months. One American official said
that more than 1,200 specific tasks carried out by the American military in Iraq had been identified
to be handed over to the civilians, transferred to the Iraqis or phased out.

To move around Iraq without United States troops, the State Department plans to acquire 60
mine-resistant, ambush-protected vehicles, called MRAPs, from the Pentagon; expand its inventory of
armored cars to 1,320; and create a mini-air fleet by buying three planes to add to its lone
aircraft. Its helicopter fleet, which will be piloted by contractors, will grow to 29 choppers from
17.

The department's plans to rely on 6,000 to 7,000 security contractors, who are also expected to form
``quick reaction forces'' to rescue civilians in trouble, is a sensitive issue, given Iraqi fury
about shootings of civilians by American private guards in recent years. Administration officials
said that security contractors would have no special immunity and would be required to register with
the Iraqi government. In addition, one of the State Department's regional security officers, agents
who oversee security at diplomatic outposts, will be required to approve and accompany every
civilian convoy, providing additional oversight.

The startup cost of building and sustaining two embassy branch offices, one in Kirkuk and the other
in Mosul, of hiring security contractors, buying new equipment, and setting up two consulates in
Basra and Erbil, is about \$1 billion. It will cost another \$500 million or so to make the two
consulates permanent. And getting the police training program under way will cost more than \$800
million.

Among the trickiest missions for the civilians will be dealing with lingering Kurdish and Arab
tensions. To tamp down potential conflicts in disputed areas, Gen.~Ray Odierno, the senior American
commander in Iraq, established a series of checkpoints made up of American soldiers, Iraqi Army
troops and pesh merga fighters. But those checkpoints may be phased out when the American troops
leave. Instead, the United States is counting on the new embassy branch offices in Mosul and Kirkuk.

Administration officials initially considered building five such branch offices in Iraq, but
abandoned those plans as cost estimates grew.

``They will be eyes and ears on the ground to see if progress is being made or problems are
developing,'' Mr.~Blinken said.

But Daniel Serwer, the vice president of the United States Institute of Peace, a Congressionally
financed research center, questioned whether this would be sufficient. ``There is a risk it will
open the door to real problems. Our soldiers have been out there in the field with the Kurds and
Arabs. Now they are talking about two embassy branch offices, and the officials there may need to
stay around the quad if it is not safe enough to be outside.''

Another area that has prompted concern is police training, which the civilians are to take over by
October 2011. That will primarily be done by contractors with State Department oversight and is to
be carried out at three main hubs with visits to other sites. Administration officials say that the
program has been designed with input from the Iraqis and will help Iraqi police officers develop the
skills to move from counterinsurgency operations to crime solving. The aim is ``focus on the
higher-end skill set,'' Colin Kahl, a deputy assistant secretary of defense, told reporters this
week.

But James M.~Dubik, a retired Army three-star general who oversaw the training of Iraqi security
forces in 2007 and 2008, questioned whether the State Department was fully up to the mission. ``The
task is much more than just developing skills,'' he said. ``It is developing the Ministry of
Interior and law enforcement systems at the national to local levels, and the State Department has
little experience in doing that.''

Mr.~Crocker said that however capable the State Department was in carrying out its tasks, it was
important for the American military to keep enough of a presence in Iraq to encourage Iraq's
generals to stay out of politics.

``We need an intense, sustained military-to-military engagement,'' he said. ``If military commanders
start asking themselves, `Why are we fighting and dying to hold this country together while the
civilians fiddle away our future?', that can get dangerous.''

\section{Court Strikes Down Berenson's Parole}

\lettrine{A}{}\mycalendar{Aug.'10}{19} Peruvian court said Wednesday that it had struck down a
decision granting parole to Lori Berenson, the New Yorker imprisoned in the 1990s on charges of
collaborating with a Marxist revolutionary group. The court ordered her to be returned to prison to
complete the five years left in her 20-year sentence.

The decision was an abrupt turnabout for Ms.~Berenson, 40, who turned herself in after the
announcement, according to the United States Embassy in Lima, Peru's capital.

After 14 and a half years in Peruvian prisons, including time spent in isolation in high-altitude
Andean facilities, Ms.~Berenson was released on parole in May when a judge decided that she had
exhibited good behavior during her incarceration.

Ms.~Berenson had a son while in prison about a year ago, another factor thought to have weighed on
the decision to grant her parole. She and her son, Salvador, who had lived with her in prison since
his birth, had recently moved into a rented apartment in Miraflores, an upscale district of Lima
where she was planning to live and work as a baker.

But the Peruvian court said it voided her parole because there were discrepancies in the addresses
Ms.~Berenson had provided for her planned residence.

Beyond the court's official reasoning, it was also clear that broad resentment over the release of
Ms.~Berenson -- who is widely viewed in Peru as a symbol of the insurgents and the war against them
that cost nearly 70,000 lives between 1980 and 2000 -- influenced the decision.

``Berenson was sent to prison for 20 years, and she only served 14 years, five months and 25 days,''
said Luis Marill del \'Aguila, an official in Peru's Justice Ministry who announced the court's
ruling.

``She hasn't even served three-fourths of her sentence,'' said Mr.~Marill del \'Aguila, adding that
Ms.~Berenson's immediate arrest had been ordered.

The Justice Ministry said in a statement that the court's decision reversing her parole could not be
appealed.

But Guillermo Gonz\'alez, a spokesman for Peru's Supreme Court, said that the case would return to
Judge J\'essica Le\'on, who granted parole to Ms.~Berenson and that it would ultimately be up to her
whether Ms.~Berenson and her son would remain in prison.

``The main reason for this is in the issue of her domicile,'' Mr.~Gonz\'alez said.

Ms.~Berenson's parents, Rhoda and Mark, said their daughter's parole had been ``annulled on a
technicality'' that they hoped could be sorted out. ``We are still in an unsure position,'' they
said in an e-mail.

Ms.~Berenson learned of the decision while she was at the embassy in Lima, where she was at a
previously scheduled consular meeting, an official at the embassy said.

``This was a Peruvian judicial process,'' James Fennell, a spokesman for the embassy, said in a
statement. ``Berenson voluntarily turned herself in today to Peruvian authorities in order to comply
with the court's order. With her consent, she was picked up at the embassy this afternoon by
Peruvian authorities.''

Ms.~Berenson's lawyer, Aníbal Apari, told local reporters that Ms.~Berenson had made no effort to
flee the country, despite disagreeing with the court's decision. Mr.~Apari, a former rebel who is
also the father of Ms.~Berenson's son, said she and Salvador would return to prison. In Peru,
children are allowed to remain with their mother until age 3 if the mother is incarcerated.

Mr.~Apari also questioned the court's decision to base its ruling on doubts surrounding information
about Ms.~Berenson's address. ``Everyone knows where she lives,'' he said.

She was initially arrested in 1995 in Peru while enmeshed with members of the Túpac Amaru
Revolutionary Movement, a group that became known internationally in 1996 when it raided the
Japanese Embassy in Lima and took hundreds of diplomats and senior government officials hostage.

In an appearance before court officials on Monday, Ms.~Berenson acknowledged collaborating with the
Túpac Amaru group but said she had never killed anyone or took part in violent actions.

``If my participation contributed to societal violence I am very sorry for this,'' she said at the
televised hearing.

But Julio Galindo, the prosecutor who sought the annulment of Ms.~Berenson's parole, said such
statements were a tactic aimed at keeping her out of prison. ``For me,'' Mr.~Galindo said, ``all
such apologies are natural acts used by those condemned for terrorism as strategy.''

\section{Facebook Unveils a Service to Announce Where Users Are}

\lettrine{F}{acebook}\mycalendar{Aug.'10}{19} introduced a long-anticipated service called Places on
Wednesday that could help the company tap local and small-business advertisers and sharpen its
competition with Google.

Facebook's Places borrows heavily from location-based social networks like Foursquare and Gowalla,
which allow users to check in at places and broadcast their location to friends. But those
companies, as well as others like Yelp, said they saw Facebook's Places as a complement to their own
services and as an opportunity to gain additional distribution.

Using Places on hand-held devices like the iPhone and other smartphones, users will be able to check
in, for example, at a restaurant, bar or museum, alerting their friends on Facebook of their
presence. They will then be able to see, using the service, any friends that are nearby, as well as
other people who have checked in at the same location and have agreed to have their location
broadcast widely.

Users will also be able to tag friends who are with them, and the service will suggest other nearby
locations that users may be interested in. Check-ins will be broadcast in status updates that will
appear in a variety of places, including the pages of a user's friends and the Places page for that
location.

``This is not a service to broadcast your location at all times, but rather one to share where you
are, who you are with, when you want to,'' said Michael Sharon, product manager for Places. ``It
lets you find friends that are nearby and help you discover nearby places.''

People will be able to see Places pages on the Web and who has checked in. But for now, the ability
to check in at a particular location will be available only on mobile phones.

Mr.~Sharon said that Places was meant to get the 500 million existing users of Facebook to increase
their usage of the social networking service.

Facebook's initiative heightens its competition with Google as they battle to emerge as the dominant
Internet hub. With that comes more advertising dollars. Facebook's long-term goal with Places
appears to be to capture the largely untapped advertising opportunity that local and small
businesses offer.

In recent months, Google has also expanded its efforts to tap the market for small-business
advertisers with a series of initiatives aimed at bringing more small businesses online and services
that allow users to find those businesses quickly. Google is also planning to add social features to
many of its services.

Analysts say both companies see the potential market for mobile advertising and search. For
retailers, seeing where people check in offers a way for companies to see who their clients are and
how they behave.

``Location gives marketers a great way to target customers,'' said Debra Williamson, an analyst with
eMarketer. ``The whole idea is to reach people at the point of decision before they have to clip a
coupon or perform a search.''

Mr.~Sharon said Facebook did not have any new advertising products tied to Places for now.

Ray Valdes, an analyst with Forrester Research, said that Facebook's move mirrored the broader trend
online to using location, particularly on mobile phones, to create new applications and services.

But Facebook will have to tread carefully regarding privacy concerns of its users, said Mr.~Valdes.

Earlier this year, the company faced harsh scrutiny over revisions to its platform and privacy
policy that encouraged members to make personal details more broadly accessible on the Internet.

Mr.~Sharon said that Places would have a series of privacy controls intended to give users the power
to decide with whom they would share check-ins. Users will, for instance, be able to remove
themselves after they are tagged. There will also be special privacy controls to protect minors, he
added.

Foursquare, a mobile social network based in New York, is one of the companies that joined Facebook
for the release of Places at a news conference at Facebook's headquarters here. ``We'll wait to see
what the product looks like, play around with it and see if its something we want to integrate
with,'' said Dennis Crowley, the chief executive of Foursquare.

Josh Williams, the chief executive of Gowalla, said he saw Facebook Places as ``an opportunity for
additional distribution of our service.''

\section{Given Money, Schools Wait on Rehiring Teachers}

\lettrine{A}{s}\mycalendar{Aug.'10}{19} schools handed out pink slips to teachers this spring,
states made a beeline to Washington to plead for money for their ravaged education budgets. But now
that the federal government has come through with \$10 billion, some of the nation's biggest school
districts are balking at using their share of the money to hire teachers right away.

With the economic outlook weakening, they argue that big deficits are looming for the next academic
year and that they need to preserve the funds to prevent future layoffs. Los Angeles, for example,
is projecting a \$280 million budget shortfall next year that could threaten more jobs.

``You've got this herculean task to deal with next year's deficit,'' said Lydia L.~Ramos, a
spokeswoman for the Los Angeles Unified School District, the nation's second-largest after New York
City.

``So if there's a way that you can lessen the blow for next year,'' she said, ``we feel like it
would be responsible to try to do that.''

The district laid off 682 teachers and counselors and about 2,000 support workers this spring and
was not sure it would be able to hire any of them back with the stimulus money. The district says it
could be forced to cut 4,500 more people next year.

In New York City, Mayor Michael R.~Bloomberg committed to no teacher layoffs this year in exchange
for not offering raises. A spokeswoman said the city's budget had already taken the federal aid into
account.

In New Jersey, where about 3,000 teachers were let go in May, Gov.~Chris Christie's administration
worries that the federal aid will only forestall difficult decisions later, and it is unclear how
much will be spent immediately.

``It's a real double-edged sword,'' said Michael Drewniak, a spokesman for the governor. ``This
money will not be there next year, and we're not going to get back up to the funding that they had
previously been used to.''

A \$26 billion federal aid package, signed by President Obama on Aug.~10, allocates \$10 billion for
school districts to retain or rehire teachers, counselors, classroom aides, cafeteria workers, bus
drivers and others -- with the remainder of the money directed toward health care for the poor,
emergency personnel and other state purposes.

The education measure requires states to distribute the money for the current school year, but
allows school districts to spend it as late as September 2012. It also allows schools to roll back
furlough days. The education department estimates it could salvage about 160,000 jobs.

``We can't stand by and do nothing while pink slips are given to the men and women who educate our
children or keep our communities safe,'' President Obama said last week. ``That doesn't make
sense.''

Though preserving jobs will be good for the economy, it will disappoint out-of-work teachers and
parents who have been expecting a surge in rehiring. Many districts, like Kansas City, Kan., face
the likelihood of midyear cuts, and administrators will count themselves lucky to save jobs. In the
nation's fifth-largest district in Clark County in Las Vegas, administrators are eager to hire some
teachers, though they wonder what they will do when the federal money runs out.

``We're a little wary about hiring people if we only have money for a year, but we know that's the
intent of this bill,'' said Jeff Weiler, chief financial officer for Clark County schools.

In Texas, Republican Gov.~Rick Perry so far has rejected the new federal education dollars. Should
he relent, Houston's superintendent, Terry B.~Grier, proposes to use \$40 million to \$70 million of
it to extend the school day and year, and to hire tutors. He does not plan to rehire 414 people --
including quite a few certified teachers -- laid off from the central office staff.

``We can't treat this money as if it's a supplement to a jobs bill,'' Mr.~Grier said. ``I want to
put people to work to help children.''

Still other obstacles loom for districts, not the least of which is timing. School has resumed in
many districts in struggling states, including Arizona, California and Illinois. Assigning new
teachers and juggling classrooms could disrupt students. In California, the budget picture is
further clouded by the state's failure to pass its own budget for the coming year.

Even administrators in districts that start school after Labor Day have only weeks to rearrange
class rosters. And with classes largely set in many places, they might more quickly deploy the money
by hiring support personnel, like those tutors in Houston.

In Arizona, where most schools opened this month, nonteaching employees are more likely to be
recalled. ``It would be hard to add teachers this year,'' said Paul Senseman, a spokesman for
Gov.~Jan Brewer. ``But the funds could be used on any school-level position like counselors,
after-school programs, aides, nurses or coaches.''

Teachers' unions are strongly urging districts to use the money right away to keep class sizes
manageable and to reduce the jobless rolls. ``The intent is to help districts avert layoffs now,''
said Randi Weingarten, president of the American Federation of Teachers. ``Kids don't have a pause
button.''

Joelle Beck, a 25-year-old high school English teacher in O'Fallon, Ill., received notice in March
that she would be laid off at the end of the school year. She recently was hired to oversee an
in-school suspension program for just over half the pay she received as a classroom teacher.

``When the economy first started going downhill,'' Ms.~Beck said, ``I na\"ively told my husband,
`Well, they're always going to need teachers.' ''

With the national unemployment rate stuck at 9.5 percent and private sector companies hiring
cautiously, local governments are an important source of jobs and consumer spending power.

State and local governments have let go 102,000 more employees than they have added in the last
three months, and economists are concerned that with revenue so depressed, school payrolls could
shrink more in coming months.

Though grateful for the aid, districts like Los Angeles are worried about how to create some budget
stability year to year. In Pomona, Calif., the district has yet to decide whether to hire back about
68 teachers laid off in the spring.

``We're also looking at a pretty bad budget, so we may decide to hold all or some of the money for
the next year,'' said Steve Horowitz, assistant superintendent of personnel services at the Pomona
Unified School District. He added that the money might be used for bus drivers or custodians, or to
roll back five furlough days for teachers.

Administrators in South Florida hope that an economic upturn, particularly in travel and tourism,
will help close their future budget gap and are planning to bring back teachers. At the Broward
County Public Schools, an operating deficit of at least \$145 million is expected next school year.

``Frankly, from my perspective, it's better to hire them now,'' said James F.~Notter, superintendent
for Broward. Of the 1,300 pink slips to school workers in the spring, about 555 went to teachers.
The district has recalled nearly 400 of them and now hopes to use the federal aid to rehire the
remaining 155.

Teachers who spent the summer in limbo are painfully aware that at best, the new federal aid may be
a temporary lifeline. Latravis Bernard, who was laid off last spring as a physical education teacher
at an elementary school in Miramar, Fla., for the second year in a row, is holding out hope he will
be recalled.

In the meantime, Mr.~Bernard, a 33-year-old father of four, has accepted a post as a special
education intern, for half his previous pay, at a different school in the Broward district.

``Even if I get brought back this year,'' he said, ``what's going to happen next year? It's really
discouraging.''

\section{Obama Campaigns for Democrats in Swing States}

\lettrine{P}{resident}\mycalendar{Aug.'10}{19} Obama threw his weight behind Democratic governors in
the battleground states of Florida and Ohio on Wednesday, while leveling blistering critiques of
Republican policies that he said would put the country in further economic peril.

On a three-day, five-state campaign swing that has hit almost every corner of the country before
ending Wednesday evening at the ritzy Fontainebleau Hotel in Miami Beach, Mr.~Obama's message has
been largely the same.

``I don't want to relive the past,'' he told a cheering crowd of around 700 at a fund-raiser in
Columbus on Wednesday for Gov.~Ted Strickland and other Ohio Democrats.

``It would be one thing if Republicans had said, 'You know, our philosophy doesn't work,' ''
Mr.~Obama said. Or, he suggested, they could have ``gone off to the desert to meditate for a while''
before returning with new ideas to run the country. But instead, Mr.~Obama said his opponents wanted
to return the country to the failed policies of the past.

That theme has emerged clearly as one that the White House will be trumpeting in the fall as
Mr.~Obama seeks to give as much help -- and as little trouble -- as he can to Democratic candidates
who are fighting to cling to their seats.

On Monday, Mr.~Obama was in Wisconsin raising money for Mayor Tom Barrett of Milwaukee, who is
running for governor. On Wednesday, he was doing the same thing in Columbus for Mr.~Strickland, who
is in a tight re-election race against former Representative John Kasich.

In Miami late Wednesday, Mr.~Obama stumped for Alex Sink, Florida's chief financial officer, for
whom Democrats have high hopes in the governor's race.

Mr.~Obama began his day talking up Democratic economic policies in a setting as intimate as the
White House could manage, showing up -- literally -- in the backyard of an Ohio family to talk with
them and their neighbors. The presidential motorcade arrived at the home of Rhonda and Joe Weithman.
Mr.~Weithman is a co-owner of an architectural company that has benefited from federal stimulus
money, while Mrs.~Weithman was able to keep her health insurance even though she lost her job last
year, White House officials said, thanks to aid provided through the stimulus program.

After chatting at the kitchen table with the Weithmans and their two children, Mr.~Obama moved to
the backyard for a town-hall style meeting with neighbors.

All 435 House seats, along with one third of the Senate and many governorships are up for grabs in
November, and Mr.~Obama and Democrats are fighting to avoid heavy losses that could lead to a
Republican takeover of the Senate or House.

The fund-raising is also crucial to that effort. A Los Angeles fund-raiser that Mr.~Obama attended
on Monday raised \$1 million for the Democratic Congressional Campaign Committee, and two
fund-raisers in Seattle on Tuesday raised \$1.3 million for Senator Patty Murray and other
Washington Democrats.

Mr.~Strickland's office would not say how much money was raised at the Columbus fund-raiser. Tickets
for the lunch, attended by 600, were priced at \$1,000, \$1,500 and \$2,000, with another 100 people
buying standing-room only tickets for \$100.

Organizers said the Miami affair was expected to raise \$700,000, with ticket costs between \$250
and \$10,000.

\section{In Defining Obama, Misperceptions Stick}

\lettrine{A}{mericans}\mycalendar{Aug.'10}{19} need only stand in line at the grocery checkout
counter to glimpse the conspiracy theories percolating about President Obama. ``Birthplace
Cover-Up,'' screams the current issue of the racy tabloid Globe. ``Obama's Secret Life Exposed!''

The article claims, without proof, that Mr.~Obama uses a phony Social Security number as ``part of
an elaborate scheme to conceal that he is not a natural-born U.S.~citizen.'' Despite repeated
denials by Obama aides -- they posted his birth certificate, from Hawaii, on the Internet during his
presidential campaign -- polls show that as many as one quarter of Americans believe Mr.~Obama was
born outside the United States.

Now comes fresh evidence of misperceptions about the president taking root in the public mind: a new
poll by the Pew Research Center finds a substantial rise in the percentage of Americans who believe,
incorrectly, that Mr.~Obama is Muslim. The president is Christian, but 18 percent now believe he is
Muslim, up from 12 percent when he ran for the presidency and 11 percent after he was inaugurated.

The findings suggest that, nearly two years into Mr.~Obama's presidency, the White House is
struggling with the perception of ``otherness'' that Candidate Obama sought so hard to overcome --
in part because of an aggressive misinformation campaign by critics and in part, some Democratic
allies say, because Mr.~Obama is doing a poor job of communicating who he is and what he believes.

The president's recent comments on the controversy over whether to build an Islamic community center
and mosque near ground zero in Lower Manhattan have most likely intensified suspicions about him.
Yet the Pew survey, completed before Mr.~Obama spoke out in favor of the right of Muslims to build
the center, shows that misperceptions were building even before then.

``This is an expression of the people who are opposed to Obama having an increasingly negative view
of him,'' said Andrew Kohut, the Pew center's director.

But Mr.~Kohut also said the numbers reflected that Mr.~Obama had ``not made religion a part of his
public persona'' as much as he did during his presidential campaign -- so much so that even his own
supporters are confused.

Among Democrats, for example, just 46 percent said Mr.~Obama was Christian, down from 55 percent in
March 2009, two months after he took office. As to the issue of his birthplace, a CNN poll released
this month when the president turned 49 found that 27 percent of Americans doubted he was born in
the United States. A New York Times/ CBS News poll in April put the figure at 20 percent.

The White House has at times seemed to throw up its hands at the so-called birther conspiracy. ``We
don't spend a lot of time worrying about what to do about people that don't think the president was
born here,'' Robert Gibbs, Mr.~Obama's press secretary, said in April.

But Dan Pfeiffer, the White House communications director, said aides did work hard to push back
against misinformation in a news media environment in which ``the tweets of discredited
rabble-rousers have as much credence to many as the pronouncements of the paper of record.''

Some allies say the White House could be doing a better job. Mr.~Obama spoke out about his faith
during his 2008 campaign -- he had little choice amid controversy over his former pastor, Jeremiah
A.~Wright Jr.~-- and pleased Christian voters by having the evangelical preacher Rick Warren deliver
the invocation at his inauguration.

``This is a president who gave really compelling speeches about faith and values, memorable stuff,''
said the Rev. Dr.~David P.~Gushee, a professor of Christian ethics at Mercer University who has
advised Mr.~Obama on religious matters. ``And you're not hearing that voice right now.''

The White House says the public -- and the press -- are not listening. Since taking office,
Mr.~Obama has given six speeches either from a church pulpit or addressing religion in public life
-- including an Easter prayer breakfast where he ``offered a very personal and candid reflection of
what the Resurrection means to him,'' said Joshua DuBois, who runs the White House Office of
Faith-Based and Neighborhood Partnerships.

But the Easter address attracted scant attention in the news media. And the fact that the Obama
family has not joined a church in Washington -- the president has said his presence would be too
disruptive -- has not helped, because the public rarely sees images of them attending services.

The White House says Mr.~Obama prays daily, sometimes in person or over the telephone with a small
circle of Christian pastors. One of them, the Rev. Kirbyjon Caldwell, who was also a spiritual
adviser to former President George W.~Bush, telephoned a reporter on Wednesday, at the White House's
behest. He said he was surprised that the number of Americans who say Mr.~Obama is Muslim is
growing.

``I must say,'' Mr.~Caldwell said, ``never in the history of modern-day presidential politics has a
president confessed his faith in the Lord, and folks basically call him a liar.''

\section{Doubt on Tactic in Alzheimer's Battle}

\lettrine{T}{he}\mycalendar{Aug.'10}{19} failure of a promising Alzheimer's drug in clinical trials
highlights the gap between diagnosis -- where real progress has recently been made -- and treatment
of the disease.

It was not just that the drug, made by Eli Lilly, did not work -- maybe that could be explained by
saying the patients' illness was too far advanced when they received it. It was that the drug
actually made them worse, the company said. And the larger the dose they took, the worse were
patients' symptoms of memory loss and inability to care for themselves. Not only that, the drug also
increased the risk of skin cancer.

So when Lilly announced on Tuesday that it was ending its large clinical trials of that drug,
semagacestat, researchers were dismayed.

``Obviously, this is disappointing news, to say the least,'' said Dr.~Steven Paul, an Alzheimer's
researcher and a recently retired executive vice president at Lilly.

Beyond the setback for Lilly, the study raises questions about a leading hypothesis of the cause of
Alzheimer's and how to treat it. The idea, known as the amyloid hypothesis, says the disease occurs
when a toxic protein, beta amyloid, accumulates in the brain. The idea is that if beta amyloid
levels are reduced, the disease might be slowed, halted or even prevented if treatment starts early
enough.

The Lilly drug, like most of the more than 100 Alzheimer's drugs under development, blocks an
enzyme, gamma secretase, needed to make beta amyloid. It was among the first shown to breach the
blood-brain barrier and reduce levels of beta amyloid in the brain. And, company studies showed, it
did reduce amyloid production.

``We did get enough in the brain to have an effect,'' said Dr.~Eric Siemers, medical director of
Lilly's Alzheimer's disease team. ``Unfortunately, the effect was not what we wanted.''

Now researchers are focused on what went wrong, and why.

Some, like Dr.~Lon Schneider, an Alzheimer's researcher at the University of Southern California,
say the drug's failure may mean the field is rushing off a cliff in its near single-minded focus on
blocking the production of amyloid. Dr.~Schneider, like most leading Alzheimer's researchers,
consults for a number of drug companies, including Lilly.

The Lilly study's failure, he said, ``chips away at that approach to testing the amyloid
hypothesis.''

``We don't know what the drug targets for Alzheimer's disease are,'' Dr.~Schneider said. ``We don't
know because we don't know the causes of Alzheimer's.''

At the very least, said Dr.~P. Murali Doraiswamy, an Alzheimer's researcher at Duke University, the
Lilly result ``clearly tells us that our current views may be too simplistic.''

Dr.~Doraiswamy said he was not abandoning the amyloid hypothesis. But, he said, ``this is a time of
major soul-searching in the field.''

``What worries me is that we don't know if this was a toxicity unique to Lilly's drug and this
late-stage population or whether it also applies to similar anti-amyloid therapies given at earlier
stages of the disease,'' Dr.~Doraiswamy said.

The bad news came on the heels of what researchers see as a resurgence of hope in this challenging
field. With new cooperation in research they have made advances in diagnosing Alzheimer's, a disease
that used to be uncertain until autopsy. And those new diagnostic tests are still exciting,
researchers said.

PET scans of amyloid plaques in the brain and tests of cerebrospinal fluid can show amyloid
accumulation long before people have symptoms of Alzheimer's disease and, as recently reported,
appear to identify people at high risk of the disease. Researchers believe the best time to try to
alter the course of the disease is before memory loss. By then, brain cells are dead or dying and
are unlikely to be restored.

At this point, though, when there is no treatment, those tests are primarily a benefit for companies
testing new therapies and researchers trying to understand the disease's progress. .

The long journey of semagacestat began more than a decade ago when Lilly scientists discovered it
could block gamma secretase in laboratory experiments. Years of work followed, showing it appeared
safe, that it got into the brains of people, that it reduced the production of amyloid in the brain.

Finally, in 2008, Lilly began two large studies of semagacestat, enrolling more than 2,600 people
with Alzheimer's disease. The company did not expect its drug to reverse the disease -- patients'
brains were too ravaged for that, said Richard Mohs, Lilly's team leader in Alzheimer's research.
But it did hope to slow the disease's progression.

Now, with the abrupt end of the studies, patients will continue to be followed but no one will be
taking any more of the drug.

``The fact that people got worse means there is biology we don't understand,'' Dr.~Mohs said.

There are several possible explanations.

One is that the drug altered the functioning of other proteins in the brain and body -- it now
appears that gamma secretase is involved in the production of about 20 proteins in addition to beta
amyloid. Companies, including Lilly, are developing drugs that block gamma secretase from making
amyloid but have little effect on other proteins.

One company, Bristol-Myers Squibb, says that is what its drug does. Its drug is now being tested in
two clinical trials. In one, the participants have Alzheimer's. In the other, they have lesser
memory impairment and have brain amyloid PET scans and tests of cerebrospinal fluid showing amyloid
is accumulating in their brains, indicating that they are likely to develop Alzheimer's.

``We still like the amyloid hypothesis,'' said Charlie Albright, a Bristol-Myers group director in
neuroscience biology. The Lilly drug failure ``doesn't affect our enthusiasm about going forward.''

Another possibility is that the enzyme is decreasing production not just of a dangerous form of
amyloid, known as a beta 42, but also of another form, a beta 40, that may protect the brain.
Companies are developing so-called selective gamma secretase inhibitors, Dr.~Paul said, which only
block the production of a beta 42.

Lilly and other companies are also testing monoclonal antibodies to reduce amyloid levels.

And companies are pursuing a more difficult target -- blocking a protein, tau, that accumulates in
dead and dying nerve cells after the disease is under way.

But Alzheimer's experts worry about the future. The research is extremely expensive -- Lilly spent
hundreds of millions of dollars on its failed drug -- and it can take a decade or more to know if a
drug works. It can take even longer if drugs are tested in people with mild symptoms of Alzheimer's
disease or in people who are at high risk but have no symptoms yet -- a direction many think is
necessary to really make a difference.

``Failures certainly don't build energy and enthusiasm,'' said Dr.~Samuel Gandy, an Alzheimer's
researcher at Mount Sinai Medical Center. ``The market is still there, but failures do take their
toll.''

\section{Pilot in Crash In China Fled North Korea, Report Says}

\lettrine{A}{}\mycalendar{Aug.'10}{19} North Korean plane crashed in northern China on Tuesday,
killing the pilot, the only person on board, according to China's state-run Xinhua News Agency.

Yonhap, the South Korean news agency, citing unnamed intelligence sources, said that the plane
appeared to be a Soviet-era MiG-21 jet, a craft used by the North's air force, and that the pilot
might have been trying to defect to Russia but lost course.

Although thousands of North Koreans have fled their repressive home country in the past decade and a
half, it is highly unusual for an elite pilot to defect. A North Korean pilot flew his MiG-19 to
defect to South Korea in 1983. Another North Korean pilot did the same in 1996. China's official
policy is to return North Korean defectors, though in practice it allows many to stay quietly.

Quoting an unidentified South Korean military official, Yonhap said the jet took off from an
airfield in Sinuiju, a North Korean town on the far western border with China about 125 miles from
the site of the crash. Xinhua said the plane crashed into a house in a rural area of Liaoning
Province, which borders North Korea. The Chinese report, which described the crash as an accident,
said that no one on the ground was killed or injured.

Cao Yunjuan, a 54-year-old farmer in Fushun County, where the crash occurred, said that she saw the
plane going down but that she heard no explosion.

``Around 3 p.m. yesterday, I saw a small plane going down and soon it disappeared from my view,''
she said in a telephone interview on Wednesday. ``There was no blast, though.''

Ms.~Cao said that she lived less than a mile from the crash site and that she and other villagers
went to see the wreckage before the area was cordoned off by the police. Many saw a North Korean
emblem on the plane's tail. Photographs of what appear to be the crash site show a North Korean star
on the wreckage.

\section{G.M. to Develop Small Engines With China Partner}

\lettrine{D}{eepening}\mycalendar{Aug.'10}{19} cooperation with one of its major partners in China,
General Motors said Wednesday that it planned to jointly develop small, fuel-efficient engines and
advanced transmissions here with S.A.I.C. Motor Corp.

The agreement is part of the American automaker's plan to create more environmentally friendly
technologies and expand its range of offerings in China's fast-growing auto market, which has
overtaken the United States as the world's largest.

China is also General Motors' biggest market. The company sold more vehicles in the first half of
this year in China than in the United States. For 13 years, the company has been producing and
selling cars here in a joint venture that is now majority owned by S.A.I.C., formerly known as
Shanghai Automotive Industry Corp.

S.A.I.C. is one of China's biggest auto makers and has a joint venture with Volkswagen as well.

As sales of G.M. cars have slowed in the United States and the rest of the world, its Buick division
in China has been stellar. Analysts say this country's fast-growing auto market has been one of the
few bright spots for G.M., which is now majority owned by the U.S.~government.

G.M. is expected to soon announce an initial public offering that could raise billions of dollars
for the auto maker and help it repay some of funds it received from the U.S.~government.

Much of General Motors' value now depends on whether it can continue to expand overseas, while
revamping its American division.

G.M. and S.A.I.C. have already agreed to make small cars and commercial vehicles together in India.
And the decision to jointly develop engines with S.A.I.C. should help solidify G.M.'s ties to the
Chinese government, which controls S.A.I.C. -- perhaps giving G.M. an edge here.

In their announcement Wednesday, G.M. and S.A.I.C. said the engines and transmissions would be
jointly developed by engineers in Detroit and Shanghai and they would be used in vehicles in China
and the rest of the world.

For China, the arrangement may help the country's engineers become more sophisticated when it comes
to engine design -- which has long been a weakness of Chinese car makers.

``These development agreements open an exciting new chapter in the partnership between S.A.I.C. and
G.M.,'' Hu Maoyuan, chairman of S.A.I.C. Motor, said in a statement. ``Not only will they add
critical green technologies to our next generation vehicles, they will also build on the strong
engineering capabilities forged as part of G.M. and S.A.I.C.'s corporate responsibility.''

The companies did not provide any financial figures for the new project.


% \clearpage
% \renewcommand\listfigurename{\textit{Table of Figures}}
% {\footnotesize\textit{\listoffigures}}

\end{document}
