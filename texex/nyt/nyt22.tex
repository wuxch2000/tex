\documentclass[12pt]{article}
\title{Digest of The New York Times}
\author{The New York Times}

\usepackage{config}

% \makeindex
\begin{document}
\date{}
\thispagestyle{empty}

\begin{figure}
\includegraphics*[width=0.3\textwidth]{The_New_York_Times_logo.png}
\vspace{-20ex}
\end{figure}
% \renewcommand\contentsname{\textsf{Digest of The New York Times}}
\renewcommand\contentsname{}
{\footnotesize\textsf{\tableofcontents}}

\clearpage
\setcounter{page}{1}

% \begin{multicols}{2}

\section{Despite Risks, an Addictive Treat Fuels a Chinese City}

\lettrine{I}{f}\mycalendar{Aug.'10}{23} the residents of this city seem a bit lively on even the
hottest days or most frigid nights, check their mouths. That minty scent and cracking sound come
from a fragrant pick-me-up that men, women and even children chew from breakfast until bedtime.

The source of their euphoria is ``binglang,'' the dried fruit of the areca palm, sometimes referred
to as betel nut, which sets the nervous system buzzing and warms the body, especially after a large
banquet.

``It helps with digestion and sobers you up,'' said Xie Shuo, a cellphone repairman who added he
consumed 100 pods a day. He smiled to reveal blackened gums and stained teeth, one of the less
attractive side effects of chewing the fruit. ``I'm addicted to binglang, but I really love it so
it's not a problem,'' he said.

That is a sentiment shared by many of Xiangtan's one million residents, whose fondness for the treat
has fueled the city's prosperity. The city, in Hunan Province, is China's leading commercial
producer of binglang. Its manufacturers import the fresh ingredients, mostly from the island
province of Hainan, and sell the dried husks across Hunan -- and to a lesser extent elsewhere.

The \$1.18 billion industry employs more than 100,000 people in Xiangtan County. No wonder the city
government has asked for the area's seven factories and nearly 50 workshops to increase production
to keep the local economy humming through the global financial crisis.

But the addictive treat has a downside, ruining mouths when chewed and soiling sidewalks when spat
out.

In many parts of Asia, the fresh areca nut is chewed, sometimes wrapped in betel leaf or with
tobacco, a habit that has been linked to higher rates of oral cancer. Hunan residents prefer to eat
just the preserved husk, which seems to deter the fruit's carcinogenic effects, according to recent
studies. But local dentists and doctors denounce the practice, saying it can cause other mouth
diseases.

Still, binglang is a symbol of pride to residents, business leaders and local officials, who wave
off suggestions that the fruit is harmful.

``Binglang has helped Xiangtan make a name for itself,'' said Li Lihua, 60, a veteran chewer of 40
years who sells bags of binglang for 75 cents each. ``It's like chewing gum but stronger, plus it
kills parasites.''

Experts affiliated with the World Health Organization estimate that 80 percent of Xiangtan's
residents chew binglang. Many start around 10 years old, picking up the habit from parents who
believe that it has powerful health benefits. According to traditional Chinese medicine, it aids
digestion, removes plaque and expels worms.

Standing outside a boisterous arcade, Pan Bozhe, 24, was gleefully chewing some ``Big Brother''
binglang during a visit home from Guangdong Province, where he works as an electronics salesman. ``I
really miss this stuff the rest of the year, since they don't sell it in Shenzhen,'' he said, his
jaw working furiously. ``So whenever I come back to Xiangtan I chew as much as I can.''

Hunan's love affair with binglang began more than 300 years ago, when, according to local lore, a
magistrate promoted it as a cure for a plague that was sweeping the region. During China's economic
reforms of the early 1980s, Xiangtan began turning the pastime into profit, and these days processes
more than 700 tons each year, according to the Hunan Binglang Association.

The city fathers speak of binglang as if it were a panacea for all of Xiangtan's ills, from curing
tapeworm to solving unemployment.

Just ask Li Xuejun 46, a binglang manufacturer who named his brand Enormous Blessing. His
grandmother's fondness for the desiccated fruit inspired him to break into the business in the
mid-1990s, and he now sells \$410,000 of binglang annually, employing 200 people. Sales have grown
about 50 percent a year, he said, which he attributes to heavier advertising and more children
taking up the habit of chewing binglang in mint, cinnamon and orange flavors.

With a cigarette hanging from his lips and binglang shells crunching between his molars, Mr.~Li
lounged in the back room of a restaurant here and extolled the virtues of the plant that has
showered the city with jobs and sport utility vehicles and has sent its children to top
universities.

Local medical experts have long been nudging the Xiangtan government to make binglang safer. Tang
Jieqing, the vice director of the Xiangtan Stomatological Hospital, who has studied the area's
binglang habits for more than 10 years, believes that rates of mouth diseases like oral submucous
fibrosis are higher in Xiangtan than elsewhere in China, and he wants the government to publicize
binglang's health risks.

While the local government set a production standard in 2004, ``to be honest, it hasn't been very
supportive in a practical way,'' Dr.~Tang said, citing the lack of package warnings or other
consumer information.

Noting the industry's economic value, he added, ``our government can be supportive of binglang, but
one should never say that it's very healthy.''

\pagebreak
\section{Western Schools Sprout in South Korea}

\lettrine{H}{ere}\mycalendar{Aug.'10}{23} on Jeju Island, famous for its tangerine groves, pearly
beaches and honeymoon resorts, South Korea is conducting a bold educational experiment, one intended
to bolster opportunity at home and attract investment from abroad.

By 2015, if all goes according to plan, 12 prestigious Western schools will have opened branch
campuses in a government-financed, 940-acre Jeju Global Education City, a self-contained community
within Seogwipo, where everyone -- students, teachers, administrators, doctors, store clerks -- will
speak only English. The first school, North London Collegiate, broke ground for its campus this
month.

While this is the country's first enclave constructed expressly around foreign-style education,
individual campuses are opening elsewhere. Dulwich College, a private British school, is scheduled
to open a branch in Seoul, the capital, in a few weeks. And the Chadwick School of California is set
to open a branch in Songdo, a new town rising west of Seoul, around the same time.

What is happening in South Korea is part of the global expansion of Western schools -- a complex
trend fueled by parents in Asia and elsewhere who want to be able to keep their families together
while giving their children a more global and English-language curriculum beginning with elementary
school, and by governments hoping for economic rewards from making their countries more attractive
to foreigners with money to invest.

``We will do everything humanly possible to create an environment where your children must speak
English, even if they are not abroad,'' Jang Tae-young, a Jeju official, recently told a group of
Korean parents.

By inviting leading Western schools, the government is hoping to address one of the notorious stress
points in South Korean society. Many parents want to send children abroad so they can learn English
and avoid the crushing pressure and narrow focus of the Korean educational system. The number of
South Korean students from elementary school through high school who go abroad for education
increased to 27,350 in 2008 from 1,840 in 1999, according to government data.

But this arrangement often resulted in the fracturing of families, with the mother accompanying the
children abroad and the father becoming a ``goose'' -- by staying behind to earn the money to
finance these ventures and taking occasional transoceanic flights to visit.

This trend has raised alarms about broken families and a brain drain from a country that is already
suffering from one of the world's lowest birthrates. Many of the children who study abroad end up
staying abroad; those who return often have trouble finding jobs at Korean companies, regaining
their language fluency or adapting to the Korean way of doing business.

Lee Kyung-min, 42, a pharmacist in Seoul whose 12-year-old daughter, Jeong Min-joo, attended a
private school in Canada for a year and a half, said she knew why families were willing to make
sacrifices to send their children away.

``In South Korea, it's all rote learning for college entrance exams,'' Ms.~Lee said. ``A student's
worth is determined solely by what grades she gets.'' She added that competition among parents
forced their children to sign up for extracurricular cram sessions that left them with little free
time to develop their creativity. ``Children wither in our education system,'' she said.

So Min-joo's parents believed that exposing her to a Western school system was worth the \$5,000
they paid each month for her tuition and board, 10 times what they would have spent had she studied
at home.

But Ms.~Lee said her heart sank when Min-joo began forgetting her Korean grammar and stopped calling
home. Still, she did not want to leave her husband behind to join her daughter, because she had
witnessed in her own neighborhood how often the loneliness of ``goose'' fathers led to broken
marriages.

``Our family was losing its bonds, becoming just a shell,'' she said.

In June, they brought Min-joo home, and they plan to enroll her in one of the international boarding
schools in Jeju, often romanized as Cheju, next year. For Ms.~Lee, this is the closest she can get
to sending her daughter abroad without leaving the country.

``There is an expressed desire in Korea to seek the benefits of a 'Western' or 'American' approach
to pre-collegiate education,'' said Ted Hill, headmaster of the Chadwick School, whose Songdo campus
has been deluged with applicants to fill the 30 percent of slots reserved for Korean students. The
balance of the student body will be recruited from expatriate families living in South Korea and
China.

``When we explain to Korean parents what we try to do in the classroom, we see their eyes light
up,'' said Chris DeMarino, business development director at Dulwich College Management
International, which has a government-set 25 percent ceiling on Korean students at its Seoul school.
``There is a tremendous demand for what we offer, but, unfortunately, we have to turn many of them
away.''

In South Korea, English proficiency and a diploma from a top American university are such important
status markers that some deliberately sprinkle their Korean conversation with English phrases.

The country sends more nonimmigrant students -- 113,519 in the fiscal year that ended on Sept.~30,
2009 -- to the United States than any other country except China, according to the United States
Office of Immigration Statistics.

In a 2008 survey by South Korea's National Statistical Office, 48.3 percent of South Korean parents
said they wanted to send their children abroad to ``develop global perspectives,'' avoid the rigid
domestic school system or learn English. More than 12 percent wanted it for their children as early
as elementary school.

Critics say that the Jeju schools -- with annual tuition fees of \$17,000 to \$25,800 and their
English-language curriculum, aside from the Korean language and history classes for Korean students
-- will create ``schools for the rich.'' But Kwon Do-yeop, a vice minister of land, transport and
maritime affairs whose department oversees the project, said it could save South Korea \$500 million
annually in what is now being spent to educate children overseas.

``Jeju schools cost half what you spend when you have your children studying in the United States,''
said Byon Jong-il, the chief of the Jeju Free International City Development Center, which is
managing the education project as part of an overall plan for the island. ``Not everything goes
right when you send your children abroad.''

Some of the things that can go wrong have been highlighted by the economic downturn.

``Many of the students who were sent abroad in the 1990s have since returned home,'' said Shin
Hyun-man, the president of CareerCare, a job placement company. ``Despite their foreign diplomas,
they were unable to find jobs abroad because of the global recession. But their Korean isn't good
enough, and they don't adapt well to the corporate culture here.''

Jimmy Y.~Hong, a graduate of Middlesex University in London and now a marketing official at LG
Electronics in Seoul, said that when he returned to South Korea in 2008, he enrolled in a business
master's degree program at Yonsei University in Seoul to help compensate for his lack of local
school connections, which can be critical to making friends, landing jobs and closing deals.

``I feared I might be ostracized for studying abroad,'' he said.

\pagebreak
\section{Scrutiny for Chinese Telecom Bid}

\lettrine{W}{arning}\mycalendar{Aug.'10}{23} about a potential threat to national security, eight
Republican lawmakers have asked the Obama administration to scrutinize a bid by one of the biggest
corporations in China to supply telecommunications equipment to Sprint Nextel in the United States.

In a letter sent last week to top administration officials, including Treasury Secretary Timothy
F.~Geithner and the director of national intelligence, Lt.~Gen.~James R.~Clapper Jr., the senators
expressed concern over claims that the company had sold equipment to the regime of Saddam Hussein
and had a close business relationship with the Islamic Revolutionary Guard in Iran.

The senators also said the company, Huawei Inc., had close ties to the People's Liberation Army in
China.

``Sprint Nextel supplies important equipment to the U.S.~military and law enforcement agencies, and
it offers a broad array of devices, systems, software and services to the private sector,'' wrote
the group of senators, including Jon Kyl of Arizona, Christopher S.~Bond of Missouri and Susan
Collins of Maine. ``We are concerned that Huawei's position as a supplier of Sprint Nextel could
create substantial risk for U.S.~companies and possibly undermine U.S.~national security.''

A campaign to block Huawei's bid to sell equipment in the United States would almost certainly
aggravate American-Chinese trade relations and intensify a longstanding debate over whether big
Chinese companies will be allowed to invest in sensitive industries in the United States.

Several Chinese companies, including Huawei, have repeatedly been discouraged or blocked in recent
years from doing business with American companies because of national security concerns, decisions
that have angered Chinese officials and business leaders.

A Treasury spokeswoman, Natalie Wyeth, said the department had received the letter and was reviewing
it. Ms.~Wyeth said the government welcomed foreign investment in general, including from China.
``Foreign investment creates significant economic benefits for American workers, including millions
of well-paying jobs,'' she said.

The letter, dated Aug.~18, drew sharp responses from both the Chinese government and Huawei. Wang
Baodong, a spokesman for the Chinese Embassy in Washington, issued a statement saying Huawei was a
private company that simply wanted to do business in the United States.

``We hope that some people in the U.S.~will take a rational approach toward these normal commercial
activities rather than do anything to stand in the way by abusing national security concerns,''
Mr.~Wang said in the statement.

Huawei, which is based in Shenzhen in Southern China, issued its own statement late Friday saying
that the Chinese government and military had no control over the company.

Huawei also defended itself against past charges that it had infringed on the intellectual property
rights of American companies.

``Huawei is disappointed to learn that old mischaracterizations about the company still linger,''
Ross Gan, a company spokesman, said in the statement.

Under American law, an interagency panel called the Committee on Foreign Investments can block
foreign companies from acquiring or investing in companies.

In this case, the senators, in their letter, asked the Obama administration and national security
agencies to fully investigate Huawei and assess the risks posed by allowing it to sell equipment to
Sprint Nextel.

In raising their concerns, the senators cited media reports; a press release from the Chinese
embassy in Iran; a report by the Iraq Survey Group, a committee set up after the fall of Iraq; and
several other documents.

The senators wrote that Huawei's presence in Iran, ``especially with military industries, suggests
that under U.S.~sanctions on Iran, Huawei should be prohibited from doing business with the
U.S.~government.''

The letter also highlighted the firm's ``reported relationship'' with the People's Liberation Army,
citing what it called ``China's well-documented focus on developing cyber warfare capabilities.''

It concluded with a series of questions in which the senators asked whether the Treasury Department
was currently negotiating a deal to allow Huawei to acquire or invest in American companies, and
whether Huawei's potential control over a company involved with sensitive government contracts would
``present a national security threat for technology leakage or enhanced espionage against the United
States.''

A spokesman for Sprint Nextel, one of the largest mobile phone operators in the United States,
declined to comment late last week. Efforts to reach several of the senators for elaboration on the
letter were unsuccessful.

Although Huawei is not well known in the United States, it is already one of the biggest telecom
equipment suppliers worldwide, with major operations in China, Europe and Africa and revenue last
year of over \$20 billion

The company is now eager to enter the American telecom equipment market. But that move has been
slowed by nagging questions about the company's history -- whether it has sold equipment to
countries like Iraq and Iran; received financing from the Chinese government; or stolen technology
from global companies.

The company was founded in the 1980s by Ren Zhengfei, a former officer in the People's Liberation
Army. Some analysts say the Chinese government has helped Huawei win contracts in Africa and other
developing nations and often helped finance government purchases of Huawei equipment.

Cisco Systems sued Huawei in 2003, saying that the company had illegally copied its software and
infringed on several patents. The companies settled the case out of court.

Last month, Motorola sued Huawei in the United States, accusing it of plotting to steal trade
secrets with the help of a group of former Motorola employees. Huawei insists the case has no merit
and said it planned to fight it in court.

Huawei did not say what Sprint Nextel contract it was bidding on. But Huawei has had trouble
entering the United States market.

In 2008, Huawei dropped a bid to take over the 3Com Corporation after the federal government began
investigating whether the acquisition would pose national security risks. And earlier this year,
Huawei lost bids to buy the assets of the software maker 2Wire and Motorola's wireless equipment
unit, according to Bloomberg News. Bloomberg said Huawei lost both bids despite offering more than
the winning bidders because the sellers doubted Huawei could win federal approval.

\pagebreak
\section{Severe Flooding Hits Northeast China}

\lettrine{M}{ore}\mycalendar{Aug.'10}{23} than 250,000 people were evacuated in northeastern China
over the weekend after torrential rains battered the area and led to severe flooding along the
border with North Korea, Chinese state media reported on Monday.

The government said that four people were killed and one was missing near the port city of Dandong
in the northeastern province of Liaoning after some of the worst flooding to hit the region in
decades.

Emergency crews worked beginning Saturday into Sunday to move the estimated 253,500 people, the
Xinhua news agency reported.

China has been suffering from severe flooding in various parts of the country for months, and is
still trying to cope with massive mudslides that killed at least 1,400 people this month in Gansu
Province, in the northwestern part of the country.

The heavy rains in North China over the weekend flooded the Yalu River, which separates China from
North Korea, forcing the river to breach its banks, China's state-run news media reported.

In North Korea, flooding submerged much of Sinuiju. The North Korean state-run media said Sunday
that the country's leader, Kim Jong-il, had mobilized military forces to rescue and evacuate
thousands of North Koreans from floods that hit Sinuiju, the isolated country's major trading gate
on its border with China.

The North's Korean Central News Agency said that about a foot of rain had fallen around Sinuiju from
midnight until 9 a.m. Saturday. The agency reported ``severe damage'' and said that 5,150 people had
been evacuated to higher ground. It reported no deaths.

Sinuiju forms a vital lifeline for the North's impoverished economy. Much of the country's land
traffic with China, its main trading partner, travels trough Sinuiju.

Since the mid-1990s, North Korea's agricultural sector has often been devastated by both floods and
drought. After decades of denuding its hills for firewood, North Korea remains vulnerable to
landslides and flash floods.

In the Chinese province of Liaoning, the floodwaters damaged five border cities, destroying or
damaging thousands of homes and buildings and causing at least \$100 million in losses, the
government said.

The heavy rains began pounding Liaoning Province on Thursday and did not let up until Saturday. But,
the government said Sunday, another wave of heavy rains was expected to worsen the situation.

\pagebreak
\section{Inside the Knockoff-Tennis-Shoe Factory}

\lettrine{A}{}\mycalendar{Aug.'10}{23} shopkeeper in Italy placed an order with a Chinese sneaker
factory in Putian for 3,000 pairs of white Nike Tiempo indoor soccer shoes. It was early February,
and the shopkeeper wanted the Tiempos pronto. Neither he nor Lin, the factory manager, were
authorized to make Nikes. They would have no blueprints or instructions to follow. But Lin didn't
mind. He was used to working from scratch. A week later, Lin, who asked that I only use his first
name, received a pair of authentic Tiempos, took them apart, studied their stitching and molding,
drew up his own design and oversaw the production of 3,000 Nike clones. A month later, he shipped
the shoes to Italy. ``He'll order more when there's none left,'' Lin told me recently, with
confidence.

Lin has spent most of his adult life making sneakers, though he only entered the counterfeit
business about five years ago. ``What we make depends on the order,'' Lin said. ``But if someone
wants Nikes, we'll make them Nikes.'' Putian, a ``nest'' for counterfeit-sneaker manufacturing, as
one China-based intellectual-property lawyer put it, is in the south eastern Chinese province of
Fujian, just across the strait from Taiwan. In the late 1980s, multinational companies from all
industries started outsourcing production to factories in the coastal provinces of Fujian, Guangdong
and Zhejiang. Industries tended to cluster in specific cities and sub regions. For Putian, it was
sneakers. By the mid-1990s, a new brand of factory, specializing in fakes, began copying authentic
Nike, Adidas, Puma and Reebok shoes. Counterfeiters played a low-budget game of industrial
espionage, bribing employees at the licensed factories to lift samples or copy blueprints. Shoes
were even chucked over a factory wall, according to a worker at one of Nike's Putian factories. It
wasn't unusual for counterfeit models to show up in stores before the real ones did.

``There's no way to get inside anymore,'' Lin told me, describing the enhanced security measures at
the licensed factories: guards, cameras and secondary outer walls. ``Now we just go to a shop that
sells the real shoes, buy a pair from the store and duplicate them.'' Counterfeits come in varying
levels of quality depending on their intended market. Shoes from Putian are designed primarily for
export, and in corporate-footwear and intellectual-property-rights circles, Putian has become
synonymous with high-end fakes, shoes so sophisticated that it is difficult to distinguish the real
ones from the counterfeits.

In the last fiscal year, U.S.~Customs and Border Protection seized more than \$260 million worth of
counterfeit goods. The goods included counterfeit Snuggies, DVDs, brake pads, computer parts and
baby formula. But for four years, counterfeit footwear has topped the seizure list of the customs
service; in the last fiscal year it accounted for nearly 40 percent of total seizures. (Electronics
made up the second-largest share in that year, with about 12 percent of the total.) The customs
service doesn't break down seizures by brand, but demand for the fake reflects demand for the real,
and Nike is widely considered to be the most counterfeited brand. One Nike employee estimated that
there was one fake Nike item for every two authentic ones. But Peter Koehler, Nike's global counsel
for brand and litigation, told me that ``counting the number of counterfeits is frankly
impossible.''

The factory is off-white, five stories tall and fronted by a brown metal gate. It was a seasonable
summer afternoon when I visited. Lin is 32, with a wispy mustache and a disarming smirk. He met me
outside the factory and took me through the gate. We scaled two flights of aluminum stairs and
entered a production floor echoing with the grinding and hissing noises of industrial labor. A few
dozen workers stuffed shoe tongues with padding, brushed glue onto foot molds and ran laces through
nearly finished sneakers. Nike and Adidas boxes were stacked in one corner, a pile of Asics uppers
in another. On this particular day, the factory was churning out hundreds of trail runners.

A help-wanted notice on the wall beside the gated entrance sought individuals with stitching skills
for all shifts; the bulletin made no mention that the work was illegal. Such things are often just
assumed in Putian. Managing a fake-shoe factory puts Lin in the middle of a multibillion-dollar
transnational enterprise that produces, distributes and sells counterfeits. Of course, like coca
farmers in Bolivia and opium croppers in Afghanistan, Lin doesn't make the big money; that's for the
networks running importation and distribution. Last year, for example, the F.B.I. arrested several
people of Balkan origin in New York and New Jersey for their suspected roles in ``the importation of
large amounts of cocaine, heroin, marijuana, oxycodone, anabolic steroids, over a million pills of
Ecstasy and counterfeit sneakers.'' Dean Phillips, the chief of the F.B.I.'s Asian/African Criminal
Enterprise Unit, describes counterfeiting as a ``smart play'' for criminals. The profits are high
while the penalties are low. An Interpol analyst added: ``If they get caught with a container of
counterfeit sneakers, they lose their goods and get a mark on their customs records. But if they get
caught with three kilos of coke, they're going down for four to six years. That's why you
diversify.''

In September 2007, police officers in New York City seized 291,699 pairs of fake Nikes from two
warehouses in Brooklyn. The early--morning raids were part of a simultaneous crackdown on a
counterfeiting ring with tentacles in China, New York and at least six other American states.
Employing undercover agents and wiretapping, the joint operation -- run by Immigration and Customs
Enforcement, the New York State Police, the Niagara Falls Police Department and the New York Police
Department -- exposed a scheme in which counterfeit Nikes arrived from China, were stored in
Brooklyn and then shipped, often via UPS, to stores in Buffalo, Rochester, Pittsburgh, Dallas,
Milwaukee, Chicago, Newark, Pawtucket, R.I., and Indian apolis. Lev J.~Kubiak, an immigration agent
involved in the case, said the total street value of the seized goods (had they been legitimately
trademarked) ``turned out to be just over \$31 million.'' Establishing provenance on the sneakers
proved difficult. ``Naturally the importation docs were not truthful,'' an immigration spokeswoman
wrote in an e-mail message, when I asked her where the shoes originated. ``But probably in or near
Putian.''

After touring the assembly line, Lin and I walked up another flight of stairs to the roof of the
factory. A mild breeze blew off the creek that snaked behind the building. Half-constructed
high-rise apartments, ensconced in scaffolding and green mesh, stood beside towering cranes. The
pace of development in Putian, a secondary provincial city with a population of about three million,
was dizzying. A cluster of unfinished apartment buildings visible from my hotel window seemed to be
a floor higher every morning.

We sat in Lin's rooftop office around a small table topped with a chessboard-size tea-making
contraption. Lin proceeded to sweep the excess water off the tea table with a paint brush and then
make a pot of green tea while recounting the transaction with the Italian shopkeeper earlier this
year. After pouring cups for my translator and me, Lin excused himself and ran downstairs. He
returned with three samples, including a single fake Nike Tiempo, the first of the batch, which was
sent to the Italian buyer to make sure it met his standards. Scribbled on the side of the shoe in
navy blue pen was a date and the man's signature. While looking the shoes over myself, I noticed the
label on the inside of the tongue read ``Made in Vietnam.'' That was all part of the subterfuge, Lin
said, adding that there are ``different levels of counterfeit. Some are low quality and don't look
anything like the originals. But some are high quality and look just like the real ones. The only
way to tell the difference between the real ones and ours is by the smell of the glue.'' He took
back the shoe, buried his nose in the footbed and inhaled.

National Intellectual Property Rights Coordination Center is the anticounterfeiting headquarters in
the United States. Situated among short stacks of concrete office buildings in Arlington, Va., the
center brings together representatives from Immigration and Customs Enforcement, Customs and Border
Protection, the Food and Drug Administration, the F.B.I., the Patent and Trademark Office, the
United States Postal Service, the Defense Criminal Investigative Service, the Naval Criminal
Investigative Service and other government agencies. J.~Scott Ballman, an immigration agent with
short, sandy hair and a Tennessee accent, is the center's deputy director. Since joining customs in
the early 1980s, Ballman has tracked the evolution of law enforcement's response to
intellectual-property violators as closely as anyone. (Customs split after 9/11 into Customs and
Border Protection, which handles interdiction, and Immigration and Customs Enforcement, which deals
with investigations.) He worked on what he says was the first undercover intellectual-property case
for the customs service when he and a team of agents investigated and ultimately arrested a group in
Miami for assembling counterfeit watches in 1985. ``Most production of this stuff has since been
pushed out of the United States,'' he told me.

In 1998, the National Security Council studied the impact of intellectual- property crimes and
concluded that federal law-enforcement efforts lacked coordination. An executive order soon
followed, sketching out the role of the National Intellectual Property Rights Coordination Center.
Two years later a makeshift office opened in Washington, but after 9/11, chasing counterfeit goods
lost priority. Ballman said: ``Resources and focus changed overnight. Agents were detailed elsewhere
and moved away from thinking about I.P. to counterterrorism and weapons of mass destruction.''

The Obama administration has made intellectual property more of a focus. ``Our single greatest asset
is the innovation and the ingenuity and creativity of the American people,'' President Obama said in
a speech in March. ``But it's only a competitive advantage if our companies know that someone else
can't just steal that idea and duplicate it with cheaper inputs and labor.'' To implement his
intellectual-property strategy, Obama appointed an intellectual-property-enforcement coordinator,
while Immigration and Customs Enforcement invigorated the property-rights coordination center.

Can such efforts make a difference? ``You're not going to arrest your way out of this,'' Bob
Barchiesi, president of the International Anticounterfeiting Coalition, told me in a despairing tone
this past spring. As long as there is a demand, he insisted, there will be supply. He had just
returned from a trip to China, the point of origin for nearly 80 percent of all goods seized by
Customs and Border Protection in the previous fiscal year. One day, Barchiesi observed a factory
raid where counterfeit jeans were seized by the Chinese authorities. The factory, its employees and
all its equipment remained in place. Barchiesi called the raid a ``propaganda show.''

Efforts to have intellectual-property rights honored in China are not new. Soon after Gilbert Stuart
completed his Athenaeum portrait of George Washington in 1796, the one that's reproduced today on
the front of every \$1 bill, a Philadelphia ship captain named John Swords set sail for southeast
China. Once in Canton, in modern-day Guangdong province, Swords ordered 100 unauthorized replicas of
the Washington portrait, which were painted on glass. (Two replicas had somehow already made their
way to China and served as the template.) Stuart was furious when he learned of Swords's activities
and, in 1801, he sued Swords in a Pennsylvania court and won. The damage was probably done, however.
Even more than a century later, Antiques Magazine observed, ``a good many portraits of George
Washington painted on glass are knocking about the country.''

But China's counterfeiting dynamic is more complicated than foreign goods being copied in places
like Putian. Chinese sneaker brands, for instance, are also counterfeited. And the domestic debate
about ensuring intellectual-property rights dates to at least the middle of the 19th century, said
Mark Cohen, who moved to Beijing in 2004 to be the U.S.~Patent and Trademark Office's first
permanent intellectual-property representative at the American Embassy. (He has since become
co-chairman of the American Chamber of Commerce's intellectual-property committee.) One initiative
of the Taiping Rebellion during the 1850s, Cohen told me, was to ``draft a patent law to encourage
Chinese innovation.'' Over a cappuccino one morning at an upscale cafe in Beijing, Cohen criticized
the notion of Chinese government negligence, which he called overly simplistic. ``People come to
this environment with certain assumptions that all this counterfeiting must mean that there's no one
enforcing,'' he said. ``But there's loads of people enforcing! There's enough I.P. officials'' -- at
least several hundred thousand by his estimate -- ``to make a small European country.''

Numbers don't necessarily spell efficiency, of course. Joe Simone, an intellectual-property lawyer
with Baker \& McKenzie in China, said: ``This is police work, but $[$the Chinese government$]$ isn't
putting enough police on it. Ninety-nine percent of the enforcement work is nothing but
bureaucrats.'' He questioned whether the current enforcement system was effective. Lin, the
counterfeiter from Putian, told me about instances in which local authorities had searched his
factory or even forced him to close in daytime, leaving him to run the factory at night. But
production always goes on.

Beijing's top intellectual-property officials, meanwhile, seem to disagree over what even
constitutes counterfeiting. Last year, a debate occurred between the heads of the State Intellectual
Property Office and the National Copyright Administration. The dispute revolved around shanzhai, a
term that translates literally into ``mountain fortress''; in contemporary usage, it connotes
counterfeiting that you should take pride in. There are shanzhai iPhones and shanzhai Porsches.

In February 2009, a reporter asked Tian Lipu, the commissioner of the State Intellectual Property
Office, whether shanzhai was something to be esteemed. ``I am an intellectual-property-rights
worker,'' Tian curtly replied. ``Using other people's intellectual property without authorization is
against the law.'' Chinese culture, he added, was not about imitating and plagiarizing others. But
one month later, Liu Binjie, from the National Copyright Administration, drew a distinction between
shanzhai and counterfeiting. ``Shanzhai shows the cultural creativity of the common people,'' Liu
said. ``It fits a market need, and people like it. We have to guide shanzhai culture and regulate
it.'' Soon after that, the mayor of Shenzhen, an industrial city near Hong Kong, reportedly urged
local businessmen to ignore lofty debates about what is and isn't defined as counterfeiting and to
``not worry about the problem of fighting against plagiarism'' and ``just focus on doing business.''

This contradictory political environment parallels -- or perhaps fosters -- a seemingly confused
corporate response. There is no doubt that, as with Washington's Athenaeum portrait, there are today
a ``good many'' fake sneakers ``knocking about'' China, the United States, Italy and the rest of the
world. But none of the major footwear companies I contacted ventured an estimate of the scale of
their counterfeiting problems. For them, it's something better not discussed. Peter Humphrey, the
founder of a risk consultancy firm in Beijing called ChinaWhys, suggested this could be for one of
two reasons: a wariness of ``upsetting the Chinese authorities'' or being ``afraid to admit publicly
too loud'' that they have a counterfeiting problem. ``Because when word gets around the consumer
market,'' Humphrey said, ``then everyone starts wondering if their shoes are real or not.''

How do counterfeit products translate to the bottom line of the legitimate company? Is each fake
Nike or Adidas tennis shoe a lost sale? A senior employee at a major athletic-footwear company,
speaking on condition of anonymity, reflected on counterfeiting as a simple fact of industrial life:
``Does it cut into our business? Probably not. Is it frustrating? \ldots Of course. But we put it as
a form of flattery, I guess.''

It could also be a form of industrial training. In Putian, Lin told me of his real ambitions.
``Making counterfeit shoes is a transitional choice,'' he said. ``We are developing our own brand
now. In the longer term we want to make all our own brands, to make our own reputation.'' Lin's
goals seemed in line with China's de facto counterfeiting policy: to discourage it as a matter of
law, but also to hope, as a matter of laissez-faire industrial-development policy, that the skills
being acquired will eventually result in strong legitimate businesses.

Putian's counterfeit-sneaker industry operates in the open. Just type ``Putian Nike'' into any
Internet search engine, and hundreds of results immediately turn up, directing you to Putian-based
Web sites selling fake shoes. (Putian's counterfeit-sneaker business has become so renowned that
Alibaba.com, an online marketplace, offers a page warning buyers to exercise caution when dealing
with suppliers from Putian.) ``People who make the product and sell the product are no longer
secret,'' says Harley Lewin, an intellectual-property lawyer at the firm McCarter \& English.
``Where sellers in the past were unwilling to disclose who they were, these days it's a piece of
cake'' to find them.

Student Street in downtown Putian is a leafy, two-lane road lined with stores stocked with nothing
but fake tennis shoes. I spent an afternoon browsing their wares. Like the products inside, the
stores varied in quality. One resembled an Urban Outfitters -- exposed brick and ductwork, sunlight
beaming through a windowed facade, down-tempo electronica playing in the background -- but the
majority of the stores appeared to value enterprise over aesthetics, with storefronts made of metal
shutters left ajar to indicate they were open for business. I ducked into one and discovered a
single room with two opposing walls covered in sneakers shrink-wrapped in clear plastic: Air
Jordans, the latest LeBron James models, Vibram FiveFingers and more. It was like a Foot Locker for
fakes.

I pulled a pair of black Nike Frees from the rack, spun them in my hands, folded the sole back and
forth, tugged at the stitching and sniffed the glue; every budding aficionado has their tasting
routine. (I never could detect the smell of ``bad'' glue.) The shoes, which cost about \$12 at the
Student Street shops, seemed indistinguishable from the pair my wife bought for \$85 in the United
States. ``I don't know if I could tell a $[$fake$]$ shoe right off the bat,'' Ballman, the deputy
director of the National Intellectual Property Rights Coordination Center, told me. If someone who
specialized in intellectual-property-rights enforcement most of his career wasn't sure he could tell
the difference, how could I? (Ballman said the key was that fake shoes have a ``heavy'' glue smell.)
As one Chinese salesman selling counterfeits in Beijing told me: ``The shoes are original. It's just
the brands that are fake.''

``Are you looking to buy or sell?'' a tall, 30-something woman with bangs asked as I examined the
Nike Frees. Her husband sat behind her, facing a large desktop-computer monitor. Their young
daughter sat at another computer, wearing a headset and playing video games. The shop doubled as a
wholesaler. The woman later confided that she and her husband ran a small factory as well as the
store. They were on the lookout for ways to get their sneakers to market and for sales agents who
could sell their shoes in the West. ``We can give a discount if you order in bulk,'' she said.

I asked how long it would take to make 2,000 pairs. ``Once you send us the model, about a month,''
she said. Her husband spoke up and assured me that the shoes ``would be the highest quality,''
adding, ``we'll use all the same materials. All the best materials are available in Putian.'' (Lin,
however, disputed that and said that using the same materials would quickly drive the price up.)

``How would I get 2,000 pairs of counterfeits past customs agents in the United States?'' I asked.

``They won't come from Putian,'' he said. Or at least the documents wouldn't indicate that. ``We
usually ship through Hong Kong on our way to America. Don't worry. We do this all the time.''

A week later, I flew to Hong Kong to meet with a private detective named Ted Kavowras. Kavowras runs
Panoramic Consulting, an investigative firm employing 30 people in China and Hong Kong. (He is also
the China and Hong Kong ambassador for the World Association of Detectives.) His forte is
investigating counterfeit factories and distribution networks. ``Until seven years ago, to export
from China was much more complex, because you didn't have the Internet and didn't have that window
into the world,'' he told me one evening over Diet Cokes and skewers of grilled octopus at a small
Japanese restaurant near his office. ``So most of the exports that came out of China had to go
through these state-owned shipping companies. It was all pretty centralized. Now it's pretty much a
free-for-all.''

Kavowras is a pear-shaped 48-year-old with pasty skin and a brash demeanor. The night after we met
for Japanese food, he showed up at a fancy steakhouse wearing a black velour Fila tracksuit.
(``What? I'm from Brooklyn,'' he said with a shoulder shrug and pursed lips.) Kavowras grew up in
New York City and joined the New York Police Department soon after graduating from high school.
Three years later he retired on disability. He ended up working ``a lot of law-enforcement stuff,''
including security- guard duties, but he found it unrewarding. ``When you're not the real thing,
you're not the real thing,'' he said. Kavowras then worked in production with The New York Times but
quit after five years and moved to Asia. In 1994, Pinkerton offered him a job in Guangzhou, China.
``I was at the right place at the right time with the right skill set,'' he said. Five years later,
Kavowras formed Panoramic.

Kavowras estimates that he works about 800 cases a year, encompassing everything from sneakers to
watches to industrial mining pumps. In 2002, New Balance hired him to nose around a factory run by
one of its former licensees in China, a Taiwanese businessman named Horace Chang. According to press
reports, Chang had more or less gone rogue. Though he had been previously contracted by New Balance
to make and distribute sneakers, relations turned bad, and New Balance canceled the contract. But
Chang continued making shoes that bore the New Balance trademark without permission. New Balance
asked Kavowras to get inside Chang's operation and report back. ``I use a wonderful investigative
methodology that works like a dream,'' Kavowras said when I asked him how a former street cop from
Brooklyn goes undercover in China. ``Drug dealers have to deal drugs, and counterfeiters have to
sell their goods. When I show up at a counterfeit factory, I look like a pretty girl on prom night.
I look like a big buyer who they can export a lot of goods to.'' Chang eventually quit making
counterfeit New Balance shoes.

If there's one commonality throughout the counterfeit world, it's deception. Along the top of a file
cabinet in Kavowras's office, located at the end of a hallway on an upper floor of a quiet building,
was a row of putty heads that a Hollywood makeup artist had designed so that Kavowras and his staff
could experiment with disguises: hats, sunglasses, beards and mustaches, fake teeth. ``I'm the only
working actor who's not waiting tables on the weekend,'' Kavowras joked. A half-dozen fax machines
were programmed to display the country codes and phone numbers of the overseas companies that
Kavowras and his colleagues pretended to represent. Each employee kept a tray stacked with various
business cards to corroborate their multiple identities. ``The bigger the lie, the more they
believe,'' said Kavowras, who also rents four shell offices around Hong Kong where he meets
``targets.''

Kavowras crossed the office to a shelf piled with purses and backpacks embedded with hidden cameras.
I asked him how the recession had affected the detective business. ``Business definitely slowed down
last year,'' he said. Corporate brand-protection budgets were slashed, and Kavowras's caseload
dropped. ``But we've been twice as busy this year. Whatever companies avoided last year came back to
haunt them this year. You can't run away from these issues. Some people say, 'Oh, it's just China,
we don't really have a market in China.' But if it's in China, it's going to get out. It's going to
wash up on beaches all over the world.''

Where did he see the counterfeit industry going next?

``It's a constant battle,'' he said.

``Like `the War on Drugs' -kind of constant battle?'' I asked.

``That's different,'' he said. Kavowras popped in a set of fake teeth and smiled. ``I see the battle
staying the same, just the battleground changing. More and more industrial work is shifting to
Vietnam. Cambodia too, though it's still a bit messy there. It's going to become more
international.'' And that, in all likelihood, will mean more agents, more detectives and more money
spent to pursue fake sneakers that no one is quite sure they can identify.

\pagebreak
\section{In Case of Emergency: What Not to Do}

\lettrine{W}{hoever}\mycalendar{Aug.'10}{23} suggested that all publicity is good publicity clearly
never envisioned the wave of catastrophe engulfing high-profile corporations over the last year,
laying waste to some of the most meticulously tailored reputations on earth.

Toyota, celebrated for engineering cars so utterly reliable that they seemed boring, endured
revelations that its most popular models sometimes accelerated for mysterious reasons. The energy
giant BP, which once packaged itself as an environmental visionary, now confronts the future with a
new identity: progenitor of the worst oil spill in American history. And the Wall Street icon
Goldman Sachs, an elite player in the white-collar-and-suspenders set, found itself derided in
Rolling Stone as ``a great vampire squid wrapped around the face of humanity, relentlessly jamming
its blood funnel into anything that smells like money.'' Last month, Goldman agreed to pay \$550
million to settle federal securities fraud charges.

``These were real reputational implosions,'' says Howard Rubenstein, the public relations luminary
who represents the New York Yankees and the News Corporation. ``In all three cases, the companies
found themselves under attack over the very traits that were central to their strong global brands
and corporate identities.''

Image implosions, of course, haven't been confined to the business world. The basketball wizard
LeBron James found himself scorned as a narcissist after his nationally televised abandonment of
Cleveland. Taped conversations of the Hollywood star Mel Gibson with his former girlfriend have
secured him seemingly permanent billing as The Worst Guy Ever.

But for members of the protective tribe known as the crisis management industry, the scandals
capturing headlines in the corporate realm involve far higher stakes, threatening the lifeblood of
global behemoths worth hundreds of billions of dollars. The calamities have served up a lifetime
supply of case studies to be mined for lessons on best practices, as well as pitfalls to avoid when
disaster arrives.

As conventional wisdom has it, the three companies at the center of these fiascos worsened their
problems by failing to heed established protocol: When the story is bad, disclose it immediately --
awful parts included -- lest you be forced to backtrack and slide into the death spiral of lost
credibility.

Exhibit A in the lesson book on forthright crisis management is the mass recall of Tylenol in 1982,
after the deaths of seven people who ingested tainted painkillers. Johnson \& Johnson promptly
acknowledged that some of its product had been poisoned and pulled bottles off store shelves.

In the view of many who are paid to extract corporations from terrible situations, Toyota, BP and
Goldman exacerbated their woes by either declining to fess up promptly, casting blame elsewhere or
striking adversarial postures with the public, the government and the news media.

``Companies that typically handle crises well, you never hear about them,'' says James Donnelly,
senior vice president for crisis management at the public relations colossus Ketchum, who -- like
many practitioners contacted for this article -- required elaborate promises that he would not be
portrayed as speaking about any particular company. ``There's not a lot of news when the company
takes responsibility and moves on. The good crisis-management examples rarely end waving the flag of
victory. They end with a whisper, and it's over in a day or two.''

Alas, recent months have featured little whispering and a good deal of high-decibel theatrics:
sirens headed to another Toyota accident; recriminations over how birds in the Gulf of Mexico became
covered in black goo; debate over the propriety of Goldman selling investments engineered to fail.
The basic facts were so unpalatable that they subdued the cleansing power of the American industrial
additive known as spin.

Which raises a question: Are some crises so dire that public relations victory is simply not on the
menu? And, if so, what's an embattled company to do?

Eric Dezenhall, a communications strategist in Washington who worked in the White House for
President Ronald Reagan, argues that the standard playbook is useless when the facts are
sufficiently distasteful. (He would know. He once represented Michael Jackson after allegations of
child molestation.)

Mr.~Dezenhall is particularly scornful of the classic imperative to ``get out in front of the
story,'' as if swift disclosure provides inoculation against all ugly realities. When the facts are
horrible, he argues, the best P.R. fix may simply be to absorb the pounding and get back to
business, while eschewing the sort of foolish communications gimmicks that can make things worse.

Consider Tiger Woods. His now-infamous fondness for women other than his wife enthralled the nation,
all the while torturing corporate sponsors who paid gargantuan sums to associate their brands with
his winning image.

``What was Woods supposed to do?'' Mr.~Dezenhall asks in an essay in the publication Ethical
Corporation. ``Call an immediate press conference and rattle through a list of lady friends
declaring, `Tiffany, yes; Trixy, no, Amber, don't remember \ldots'? And if Woods had pre-empted with
a confession, would this have caused the news media, bloggers, pundits, Hooters waitresses and
everyone else to collectively reward him with their silence? Not a chance.''

Much the same can be said for BP, Toyota and Goldman, he suggests, with attempts to win public
affection almost certain to be viewed as insincere so long as real problems persisted -- oil
spilling into the ocean, cars crashing, Wall Street profiting while ordinary people suffered.

``The two things that are very hard to survive are hypocrisy and ridicule,'' Mr.~Dezenhall says.
``It's the height of arrogance to assume that in the middle of a crisis the public yearns for
chestnuts of wisdom from people they want to kill. The goal is not to get people not to hate them.
It's to get people to hate them less.''

A Disaster, Made Worse

Putting aside the limitations of crisis management, those in the trade generally share a sense that
the companies at the center of recent events committed grievous errors. At the top of the list is
BP.

``It was one of the worst P.R. approaches that I've seen in my 56 years of business,'' says
Mr.~Rubenstein. ``They tried to be opaque. They had every excuse in the book. Right away they should
have accepted responsibility and recognized what a disaster they faced. They basically thought they
could spin their way out of catastrophe. It doesn't work that way.''

Not so, protests the energy company. ``From the beginning of the crisis, BP stepped forward and
accepted responsibility,'' says Andrew Gowers, who oversees BP's communications. ``We mounted a
massive maritime response. It was the largest response there's ever been.''

For BP, any spill threatened to undermine its image as a new breed of energy company, one sensitive
to the environment. In an advertising campaign started in 2000, BP reinterpreted its initials to
stand for Beyond Petroleum, featuring solar panels and wind turbines.

``BP made a decision years ago to hold themselves up as a paragon in a pretty controversial
industry,'' says Eddie Reeves, a former vice president for media relations at Merrill Lynch. ``I
remember hearing somebody talking about how Beyond Petroleum was going to be the poster child for
this new environmental mind-set, and somebody else saying, 'Yeah, that'll work until they have their
own plant explosion.' ''

BP was no stranger to accidents. An explosion at a refinery in Texas in 2005 killed 15 workers,
prompting a federal fine of \$50.6 million. The next year, an oil pipeline leak in Alaska brought BP
\$20 million in penalties. But those events became footnotes after the lethal explosion at an
offshore oil rig in the gulf in April.

The company issued early estimates suggesting that little oil was reaching the water, perhaps 1,000
barrels a day. Outside scientists challenged that figure and pressured the federal government to
expand the estimate to five times as much. So it went for weeks, the numbers swelling and each
revision calling into question the veracity of anything BP might say.

In the end, consensus estimates assume that nearly 60,000 barrels a day were spilling into the
water, with the total volume close to five million barrels.

``BP lost a lot of credibility when it turned out they weren't being forthright about how much oil
was spilling out,'' says Lucio Guerrero, who, as former spokesman for Rod R.~Blagojevich, the
impeached governor of Illinois, has intimate knowledge of the art of trust management. ``Once you
lose credibility, that's the kiss of death.''

BP counters that it offered the best real-time numbers it could, while always stressing that
estimates were subject to change.

``There was enormous pressure to mention a number,'' says Mr.~Gowers. ``We did not underestimate the
scope of the spill.''

On the highlight reel of BP's missteps, strategists cite its effort to deflect blame for the spill
by pinning responsibility on contractors. That made BP appear callous, as if it were focused on
avoiding legal liability rather than doing right by those whose lives had been upended -- the
families of the 11 rig workers who died in the explosion, and communities that draw their
livelihoods from the gulf. (BP declined to comment on such assertions.)

The company had to contend with a classic corporate quandary of balancing advice from counselors
with starkly different considerations, according to people familiar with BP's deliberations who
requested anonymity because the advice was confidential.

In times of crisis, communications professionals and lawyers often pursue conflicting agendas.
Communications strategists are inclined to mollify public anger with expressions of concern, while
lawyers warn that contrition can be construed as admissions of guilt in potentially expensive
lawsuits.

For BP, this tension burst into view in May, when executives went to Capitol Hill with officials
from two of its contractors: Transocean, which owned the offshore rig that exploded, and
Halliburton, which aided BP in drilling. Executives from the three companies each disowned
culpability while pointing fingers at one another.

``What that screamed is the lawyers are in control,'' says Mr.~Reeves. ``All it did was get
everybody all the more peeved at them.''

Many analysts say BP erred in putting its message in the hands of its C.E.O., Tony Hayward. Inclined
toward pinstriped suits, Mr.~Hayward found himself in coastal communities in the American South,
where shrimpers donned stained coveralls in pursuit of a catch now polluted by his company's gushing
inventory. His words of regret were delivered with a British accent, and he complicated his task
with a series of tin-eared utterances.

``The Gulf of Mexico is a very big ocean,'' Mr.~Hayward told The Guardian amid debate over the
extent of the spill. ``The amount of volume of oil and dispersant we are putting into it is tiny in
relation to the total water volume.''

Four days later, he told a TV reporter that ``the environmental impact of this disaster is likely to
have been very, very modest.''

When he apologized to those harmed by the spill -- telling a television interviewer, ``We're sorry
for the massive disruption it's caused to their lives'' -- he tacked on two sentences that would
overshadow all else. ``There's no one who wants this thing over more than I do,'' he said. ``I'd
like my life back.''

With this, Mr.~Hayward opened the gates to Sound-Bite Hell. Gangs of reporters deployed to the spill
now had a cartoonish narrative to lean on, instead of the discomfiting m\'elange of scientific
conjecture that had been their story before: Here was the evil corporation, headed by an unfeeling
rich guy with a fancy accent (no matter that Mr.~Hayward wasn't born to wealth and attended none of
Britain's patrician schools).

Pundits gleefully pointed out that the life he yearned to reclaim featured millions of dollars in
compensation, a point reinforced when he took leave of the disaster to spend time with the family --
at a yacht race.

Eventually, BP minimized Mr.~Hayward's presence, unleashing ads featuring BP employees from gulf
communities who were working diligently to clean up the mess -- a decision that draws praise from
crisis managers as a better way to convey concern.

``The ads they have up with the New Orleans-area natives who work for BP are much better,'' says
Mr.~Reeves, the former Merrill Lynch executive.

Later, Mr.~Hayward handed the reins of BP to Robert Dudley, who brought street cred as a competent
operator along with a useful pedigree as a Mississippi native with an accent to prove it.

Mr.~Hayward's travails illustrate another perilous lesson from the crisis-management handbook:
Although strategists constantly hector executives to stick to the script, ad-libbed gaffes are
common. Some of this is rooted in the culture of C.E.O.-dom. Bosses are accustomed to being obeyed
as authorities, making them confident in their abilities to charm and persuade -- a trait that can
cause trouble when the audience is no longer a room full of underlings but a panel of Congressional
interrogators or a pack of reporters wielding recording devices.

Yet, in the end, there may have been little that BP could have said to ameliorate public anger while
the oil continued to spew, flooding the national consciousness -- particularly given the modern
realities of 24-hour cable newsMore than any other company in crisis, BP had to contend with the
power of a stark and ubiquitous image: oil gushing into water. As it tried and failed to plug the
leak with various Rube Goldberg contraptions, any communications strategy was prone to backfiring.

``BP could apologize every day,'' says Keith Michael Hearit, a communications professor at Western
Michigan University. ``They could have a situation where the C.E.O. goes on an environmental
pilgrimage and falls on his knees going up a mountain, and it wouldn't do them any good. Until the
oil stopped, there was nothing that could be done to make it better, but there was plenty that could
be said to make it worse.''

Squandered Good Will

Toyota had a stronger hand to play, strategists say. Rare is the consumer with affection for an oil
company, but Toyota enjoyed immense good will. It was the world's largest automaker, with reasonably
priced cars that delivered excellent gas mileage and performance. It had cultural stereotypes
working in its favor -- the image of the disciplined Japanese corporation with uncompromising
quality standards.

All of this suggested that Toyota might have won the public's understanding had it been seen as
forthright.

``When you're in the mix of these really obtuse situations where nobody really knows the facts, in
some sense the facts are less important than your posture toward the facts,'' says Mr.~Reeves, the
former Merrill Lynch media relations executive. ``People are reasonable. They know companies make
mistakes, and people will forgive an honest mistake. They will not forgive a dishonest cover-up.''

Strategists express amazement that a company with Toyota's reputational advantages repeatedly
dismissed reports of problems, communicated mainly through press releases -- which underscored a
fortresslike position -- and then begrudgingly confirmed bits of evidence, ensuring that the story
would play out in damaging increments.

``Toyota blew it,'' says Brad Burns, who ran communications at WorldCom, the telecommunications
giant leveled by a 2001 accounting scandal. ``It's been the proverbial death by a thousand cuts.
They knew they had problems long ago, whether it was a mechanical issue or operator error, but they
knew they had an issue they had to deal with.

``And rather than put public safety over profits, they appear to have listened to the product
liability lawyers and they totally lost it. It's brand damage.''

Mr.~Burns says his experience at WorldCom -- which reckoned with the scandal before selling itself
to Verizon in 2005 -- laid bare the benefits of transparency.

``The quicker you apologize and make it right, the faster it goes away,'' he says. ``The longer you
stonewall, the worse it gets.'' (``Oh, by the way,'' he adds, ``it didn't help when your former
management team is walking around in handcuffs.'')

Toyota's communications coordinators say the company never ducked problems, even as they wish they
had been quicker to deploy top executives.

``We made lots of mistakes,'' says Jim Wiseman, a communications executive at Toyota Motor North
America. ``We could have been out there even more. We should have been. We're trying to take various
actions to learn from this.''

The crisis began last September, as the National Highway Traffic Safety Administration issued an
alert that floor mats in some Toyota vehicles appeared to ensnare gas pedals, causing surges in
acceleration. The same day, Toyota announced the recall of 3.8 million vehicles.

The administration concluded that the mat problem was probably to blame in an earlier crash of a
Lexus -- an accident that killed four people.

A week after the recall, Toyota's C.E.O., Akio Toyoda, issued the first in a string of apologies.
``Four precious lives have been lost,'' he told reporters. ``Customers bought our cars because they
thought they were the safest. But now we have given them cause for grave concern. I can't begin to
express my remorse.''

Yet even as Mr.~Toyoda apologized, the company offered what later looked like false reassurance,
emphasizing that if customers removed questionable mats, there was nothing else to worry about.
Toyota asserted in a press release that a review by the safety administration had turned up no other
problems with Toyota vehicles. That brought an immediate rebuke from the federal agency, which
branded the claim ``inaccurate and misleading.''

By late November, Toyota had announced that it would fix gas pedals on four million vehicles to
further prevent them from catching on floor mats. The company again declared the problem contained.
``We are very confident that we have addressed this issue,'' a Toyota spokesman, Irv Miller, said at
the time.

Yet the next month, four people died in Texas when a Toyota Avalon sped off a roadway and into a
pond. The police found the floor mats in the trunk of the car. Toyota says that the cause of the
crash has yet to be determined.

As reports mounted that gas pedals might be sticking even when floor mats were not implicated,
Toyota wrestled internally with how much to disclose, according to documents later given to
government investigators.

On Jan.~16, as Toyota executives prepared to travel to Washington for talks with regulators,
Katsuhiko Koganei, a communications coordinator at Toyota Motor Sales USA, sent a note to his
colleagues urging that the company not discuss mechanical problems in gas pedals aside from the
floor-mat issue, because neither the cause nor the proper remedy for such troubles had been
identified. He added that if executives broached the subject, that ``might raise another uneasiness
of customers.''

Mr.~Koganei's e-mail prompted an agitated response from Mr.~Miller. ``We have a tendency for
mechanical failure in accelerator pedals of a certain manufacturer on certain models,'' he fired
back. ``The time to hide on this one is over. We need to come clean.''

Later that month, as Toyota announced the recall of an additional 2.3 million cars, it acknowledged
that gas pedals were sometimes sticking even when floor mats were not an issue, while emphasizing
that no serious accidents were linked to the problem.

``Our investigation indicates that there is a possibility that certain accelerator pedal mechanisms
may, in rare instances, mechanically stick in a partially depressed position,'' Mr.~Miller said in a
press release.

According to another Toyota spokesman, Mike Michels, the company's carefully calibrated language was
written with the misfortunes of another automaker in mind: Audi.

In the mid-1980s, Audi 5000 vehicles were recalled after reports of sudden accelerations blamed for
hundreds of crashes. Audi went on the offensive, contending that drivers had errantly hit the
accelerator when they intended to hit the brake. Audi later claimed vindication from a National
Highway Traffic Safety Administration review, but blaming the customer proved a costly public
relations strategy.

A quarter-century later, as Toyota acknowledged rare problems with its accelerators, it was leaving
itself room to argue later that sometimes drivers were at fault -- while pointedly avoiding saying
so directly.

``The lesson from Audi is no matter how convinced you are that the driver has made a mistake, it is
very sensitive to suggest,'' says Mr.~Michels.

Like BP, Toyota had a corporate steward from another country who sometimes stumbled before an
American audience. As he apologized to lawmakers on Capitol Hill, Mr.~Toyoda spoke heavily accented
English and read haltingly from his notes as a gray-suited assemblage of minders sat behind him,
shifting nervously as he struggled to pronounce ``condolences.''

``I said at the time, `Oh my God, he's not communicating with us,' '' recalls Mr.~Rubenstein, the
longtime P.R. executive. ``He should have somebody that really knows our American society.''

Mr.~Michels argues that no one could have substituted for the chief on Capitol Hill. ``You would
have had members of Congress pointing out that he wasn't there,'' he says.

But Mr.~Toyoda's Congressional performance may have also supplanted a positive image of the
supremely efficient Japanese corporation with another Japanese stereotype: a deep-seated aversion to
shame.

``They believed in the Japanese strategy of putting a lid on it and maybe it will blow over,'' says
Howard Anderson, a crisis management expert at M.I.T.'s Sloan School of Management. ``Stonewalling
only gets you into more trouble.''

Yet seeking a way around a painful public reckoning appears to be a nearly universal approach to
corporate crises. In the long run, the best course for an embattled company may be swiftly owning up
to its errors. But to human beings stuck with the task of disclosing embarrassing details here and
now, dissembling and delaying may beckon as the easiest way to get through the day.

One strength of human beings is their ability to learn from mistakes -- governed, as they are, by
fear and shame. But corporate bureaucracies are essentially structured to tune out sentiment in the
pursuit of profit, making them less prone to absorbing lessons of previous disasters.

``We tend to anthropomorphize corporations,'' says Mr.~Hearit at Western Michigan. ``There's this
myth of managerial rationality, the idea that corporations can learn. Well, they have no soul to
kick. You can't put a corporation in jail.''

Above all, crisis management is conducted with stress and sleeplessness layered atop the usual
factionalism and politics afflicting any big organization. Mr.~Dezenhall, the strategist, is amused
by crises as glimpsed in movies, where people sit at banks of synchronized computers, speaking
calmly into headsets.

``The reality is absolute chaos,'' he says. ``Nobody knows what the facts are. The lawyers are
trying to get the P.R. consultants fired and the P.R. consultants are criticizing the lawyers.
Everybody despises each other. It's a totally unmanageable situation. A corporation in crisis is not
a corporation. It is a collection of panicked individuals motivated by self-preservation.''

A Laser Focus on Profits

If there was panic and chaos inside Goldman Sachs, the company kept it hidden, maintaining a
consistent communications posture throughout its brush with unwanted scrutiny: Yield little.

Yet in opting to mount an aggressive defense, the company appears to have intensified criticism. As
many communications experts see it, Goldman took a series of unsavory but not crippling disclosures
about how it profited before, during and after a global financial crisis and -- through a public
relations strategy built on arrogance and insensitivity to the national mood -- turned itself into a
symbol of Wall Street shenanigans.

Given the loss of jobs, homes and savings attendant to the financial crisis, Wall Street was bound
to face sharp questioning about its role in the collapse. Goldman, in particular, seemed certain to
confront special scrutiny by dint of its hefty profits, its proximity to pivotal moments in the
crisis, and its former executives' ubiquity in Washington's corridors of power.

Goldman's first round of questioning began in the wake of the \$85 billion federal bailout of the
American International Group, the insurance giant, in 2008. Goldman owned insurance policies from
A.I.G. on some of its mortgage investments. Analysts, journalists and federal authorities all raised
questions about whether Goldman unfairly benefited from taxpayer funds used to bail out A.I.G.

Such questions were fueled by the r\'esum\'e of a prime architect of the bailout, the Treasury
secretary Henry M.~Paulson Jr., who had formerly led Goldman. (Mr.~Paulson has said that he never
took action to specifically benefit Goldman, seeking only to buttress the financial system as a
whole.)

Like Toyota, Goldman has had internal debates about how forthcoming to be in confronting sharp
questioning, with some insiders advocating a swift, unabashed disclosure of its dealings with A.I.G.
to avoid inflaming public anger, according to people familiar with the deliberations who requested
anonymity because the talks were confidential.

As Congress kept examining Wall Street's role in the crisis, federal investigators made Goldman the
first major firm to face legal charges associated with the nation's housing debacle.

In April, the Securities and Exchange Commission filed a lawsuit accusing Goldman of securities
fraud in marketing a mortgage investment that was built to fail without fully informing clients of
its provenance. The hedge fund manager who selected contents of the investment, John A.~Paulson,
profited mightily, but the banks on the other side of the deal lost more than \$1 billion, according
to the S.E.C. suit.

Some analysts contend that Goldman, in adopting a defensive posture to queries about how it snared a
large piece of the A.I.G. bailout, effectively put itself in front of an onrushing train: an S.E.C.
determined to find a high-profile case that would eclipse criticism that it had failed to police
Wall Street effectively.

Once the S.E.C. filed its case, bloggers and pundits seized it as proof that Goldman was rigging the
financial game. Others criticized the lawsuit as flimsy.

Goldman maintained that it had done nothing wrong, asserting it had merely enabled sophisticated
investors to make opposing bets on the mortgage market.

Analysts say that this may have been a smart legal defense -- Goldman ultimately settled the case
for a sliver of its profits -- but in the public eye, it intensified views that Goldman hadn't
played fair. Never mind its supposed role in enabling American commerce: Goldman was now dismissing
allegations of fraud by arguing that, in essence, it had been running a giant casino, in which it
had inside information.

How badly did these revelations play? Mr.~Anderson, at M.I.T., sees parallels to the sexual abuse
scandal in the Roman Catholic Church.

``The priests thought they should be protecting one another rather than the children in their
care,'' he says. ``Goldman now has the same problem. It turns out it's, 'We make money for ourselves
first, and our customers second,' when it should be the other way around. This is going to hurt them
for years.''

Goldman faced an added complication in that its business was arcane and many of its products were
exotic. Toyota and BP faced more clear-cut crises that could be conveyed with a simple image or
sentence: you could see the oil sullying the gulf, the cars careering out of control. Goldman was
accused of failing to disclose details of something called a synthetic collateralized debt
obligation, a term that made readers seek refuge in the sports section.

Experts say murkiness could have worked in Goldman's favor, had the company adopted a conciliatory
stance. But Goldman has concluded that the obscurity of the bank's business may be the source of its
problems, says its head of corporate communications, Lucas van Praag.

``The issues we have aren't rooted in bad communications,'' he says. ``They're a direct function of
our business model. One of the things we've learned is it may be perfectly legal but if it's too
complicated to explain to ordinary, rational people, then maybe it's a business we ought not be
in.''

Like BP, Goldman saw its troubles aggravated by ill-advised sarcasm from its C.E.O., Lloyd
C.~Blankfein, who told a reporter for The Sunday Times of London that he considered banking as
``God's work.'' Here was another story kit thrown to the press corps, an easily digested narrative
to replace the tedious work of explicating Goldman's mind-bending financial arrangements.

Some argue that Goldman's notoriety may matter little to the bank's main constituents: shareholders
and clients. Toyota and BP both sell products to ordinary consumers, making image maintenance a
crucial concern. Goldman, on the other hand, generally confines its work to serving extremely
wealthy people, companies and governments. Its success depends not on being liked, but on being
respected for a pursuit that sometimes pulls in the opposite direction -- racking up profits, even
amid calamity.

Analysts say Goldman's P.R. woes are compounded by a reality bigger than any one institution: it is
a leading investment bank at a time when investment banks are as beloved as bedbugs, airlines and
Congress.

``It's a problem of making obscene amounts of money in ways that no one understands in the middle of
an economic meltdown,'' suggests Mr.~Dezenhall. ``I don't think there is a way to make people love a
bank that earns money in the middle of a recession.''

Back to Their Basics

Children stuck on scary roller coasters sometimes close their eyes and wait for the ride to end. So,
apparently, do grown-ups heading giant corporations in crisis. This is the conventional explanation
for how three enormously successful enterprises managed to prolong and deepen their public relations
agony.

``These companies made the same mistakes,'' says Mr.~Rubenstein. ``They broke the cardinal rule of
crisis management: They didn't seem to have a crisis plan in hand. They sought to minimize the
extent of their problems, and they never seemed to display an understanding for the situation they
were in.''

But maybe they did understand, and what they grasped keenly was that all of their options were poor.
There was nothing they could do short of returning to their fundamental reasons for existence:
making petroleum products, making cars, making money.

``We can sit in front of cameras and beat our breasts all we want, but what's going to count is
whether we're able to keep making high-quality products,'' says Mr.~Wiseman at Toyota.

Five years ago, Tyco International, the electronic security and financial services company, was
defined in the public realm by the extravagances of its C.E.O., L.~Dennis Kozlowski, and his finance
chief, who together were convicted of looting millions from company coffers. For his wife's
birthday, Mr.~Kozlowski spent some \$2 million on a bacchanal at a Sardinian estate, where an
ice-sculpture replica of Michelangelo's David served a stream of vodka from its genitalia.

Here was a seemingly fatal public relations issue. Yet, today, Tyco is still going. It endured not
by spinning the unspinnable, but by quietly amassing contracts.

``It wasn't getting the chief executive on TV holding a puppy,'' says Mr.~Dezenhall. ``You have to
have a realistic expectation of what communications can accomplish. Nobody ever says: 'Oh, that's
wonderful communications. We feel good now.' ''

\pagebreak
\section{TV Makers Predicting a Bright Future for 3-D Sets}

\lettrine{N}{ow}\mycalendar{Aug.'10}{23} that almost two-thirds of American homes have ditched their
old tube televisions for flat-screens and high-definition sets, TV makers are trying to lure
consumers back into the stores for the next big thing -- 3-D TV.

To succeed, they want to entice consumers with bundles of 3-D glasses and Blu-ray discs, discounts
and special broadcasts of sports events. If all goes as analysts predict, 3-D TV could account for
half of all television sales within five years.

So far, 3-D TV is a sliver of the overall market, accounting for about 2.5 percent of new television
sales in the United States in the last quarter, according to a survey by the market researcher
iSuppli.

Riddhi Patel, iSuppli's director of television services, said 83 percent of survey respondents who
bought 3-D TVs were professed early adopters -- people who like to own the latest technology -- and
half of them had annual household incomes of \$100,000 or more.

A survey by Frank N.~Magid Associates, an entertainment industry consulting firm, shows that a
quarter of consumers plan to buy a 3-D-enabled set in the next 12 months. (Among men 18 to 24, the
rate is almost 40 percent, the firm said.)

The optimism may not be misplaced. ``We're seeing a similar adoption curve as high definition,''
Chris Fawcett, vice president of Sony's television business, wrote in an e-mail message. The
Consumer Electronics Association said HDTV was in 6 percent of homes in 2004, the first year the
group tracked this statistic.

American consumers are familiar with 3-D. More than two-thirds have seen a 3-D movie in a theater,
according to Magid Associates. Electronics stores and big-box retailers like Costco are prominently
displaying the 3-D models.

Daniel Camin, a 23-year-old building project engineer, watched a 3-D TV at a Best Buy store in Los
Angeles and said, ``I have to admit it was better than I thought it would be. Honestly, it looked
like going to see it in the theater. ``

If the technology is familiar, the price is still a barrier. When Mr.~Camin saw the price of a
50-inch Panasonic, he said, ``Two thousand three hundred bucks? I'm not sure I'd buy one. If all the
networks were in 3-D, I'd pay a premium for it, but probably not more than 25 percent.''

Now, 3-D TVs cost \$1,200 more than flat-panel 2-D televisions when averaged across all sizes,
Ms.~Patel said. By 2014, however, iSuppli expects 3-D televisions to cost \$325 more than regular
HDTV.

But there are significant hidden costs. Panasonic, Sony and Samsung charge \$150 for each pair of
3-D glasses, and a new television set may not come with a pair. Glasses from different television
brands are not necessarily compatible. Watching 3-D movies also requires a 3-D Blu-ray player,
starting at \$200 for a basic model.

Add that up for a family of four and the 3-D experience at home costs an additional \$800, about as
much as it costs to buy a regular 42-inch LCD TV. Consumers will have to worry about losing glasses,
keeping them clean and protecting them.

At a Costco store in San Francisco where shoppers were encouraged to test the 3-D TVs, the frames of
the glasses were broken, which was not a compelling display of durability.

If manufacturers can agree on a technology standard for glasses, the price of those items will be
driven down through competition. But there may be another way.

Vizio, a leading maker, is developing a 3-D TV that uses the cheaper kind of glasses, like those
used in movie theaters. While this so-called passive technology will require more expensive
televisions, the glasses will practically become disposable, said Ross Rubin, executive director of
industry analysis at the market researcher NPD Group.

Glasses-free 3-D TV is a distant possibility, but the technology is unrealistic for large screens
until manufacturers figure out how to widen viewing angles, Mr.~Rubin said.

Price aside, television makers have to convince consumers that 3-D is safe and comfortable. In
April, Samsung caused a stir when its Australian Web site warned that 3-D TV could be dangerous for
people who are intoxicated, elderly or tired, as well as for pregnant women. ``I think manufacturers
are being overly cautious to make sure they're covering themselves legally,'' said Robert Scaglione,
senior vice president and group manager of Sharp's product and marketing group.

The industry will be pushing the TVs as the holiday shopping season approaches. Already, Sony is
giving away ``starter kits'' that include two pairs of glasses and 3-D Blu-ray discs with 3-D
television purchases.

Along with the free starter kits, Sony is offering discounts of \$350 or more this month on its 3-D
TVs.

The first big fall sales push comes on the second weekend of September when Best Buy and other
retailers around the country will host broadcasts of ESPN programming in 3-D.

\pagebreak
\section{Crowded Field for Bringing Web Video to TVs}

\lettrine{I}{f}\mycalendar{Aug.'10}{23} you want to watch Internet video on most televisions, you
need a gadget that pulls it in. And a growing number of technology companies want to sell it to you.

Start-ups and tech giants alike are offering what they say are easy ways to pipe shows and movies to
a TV, hoping to win over people who might want a cheaper or more diverse alternative to cable and
satellite service.

These companies have a lot of convincing to do. Most people do not have the tech-savviness to tackle
the hardware and software setup that these products often require. And the companies are not able to
offer access to many shows and channels that are on traditional pay TV, nor bundle services like
phone service and Internet access at a discounted rate, as TV service providers do.

But there are also several perks, beginning with the cost. Many of these products do not require
monthly subscriptions, and those that do rarely cost more than \$20 a month. And they try to make up
for the lack of some programming by organizing the Internet's offerings through an easy-to-navigate
menu.

``People don't want 400 overpriced channels,'' said Phil Wiser, co-founder and president of Sezmi, a
start-up that thinks it has a shot at the big screen in the living room. ``Consumers are ready to
make a new decision about how they are paying for television.''

Sezmi, based in Belmont, Calif., offers a hybrid system that delivers content in several ways. The
system, which sells for \$150 at Best Buy, has a DVR and pulls in both over-the-air TV broadcasts
and on-demand content from the Internet.

Sezmi, which is slowly being introduced in major cities like San Francisco, Washington and Los
Angeles before a national rollout, offers two service plans. Customers can pay \$5 a month for a
programming guide and access to a catalog of pay-as-you-go shows and movies. A \$20 plan adds cable
programming, including channels like Bravo, Comedy Central and Nickelodeon.

One major hurdle for a company like Sezmi, analysts say, is that many consumers are now used to
buying their cable, telephone and Internet services in a bundle.

``These services by and large will be limited by the fact that they are one-trick ponies,'' said
Mike Jude, an analyst with Frost and Sullivan.

But Mr.~Wiser says Sezmi is working with Internet service providers to try to offer Sezmi packaged
with broadband and telephone service: ``We're trying to be everything.''

Others companies, like Boxee, think they can draw a sizeable audience without having to offer
prime-time programming. Boxee's free software pulls in online video from many sources around the
Internet, including CNN.com. But the software requires viewers to watch on a computer, or hook the
computer to a TV. In November the company will get around that problem by introducing a set-top box
that runs its software.

Then there are companies that are taking a more symbiotic approach. Roku, for example, makes a slim
box starting at \$70 that can wirelessly stream movies and TV shows from Netflix and other sources,
but does not aim to be a cable replacement.

``We're more complementary, for people who are shaving down their cable services or trimming the
breadth of what they get,'' said Brian Jaquet, a spokesman for the company.

A service called Kylo, introduced in February, is gearing up for a wave of Internet-connected
television sets with free software that allows users to search for online video using a browser on
their television screens.

All of these services are relatively new, so most have not yet gained any significant traction. But
analysts say that even the larger companies that are hoping to make inroads in this area have not
found much success.

That is in part because many big media companies have been reluctant to make their best programming
available online. That would give people fewer reasons to pay for expensive monthly cable and
satellite services, which would in turn hurt the content companies.

The video site Hulu does not want people viewing its content on TV sets, so it has used
technological means to block Kylo and Boxee users. Boxee has found ways to circumvent this. Hulu is
a joint venture of the News Corporation, the Walt Disney Company, NBC Universal and Providence
Equity Partners.

Even Apple has struggled with Apple TV, a \$229 set-top box that is its attempt to bring its iTunes
software and store to the heart of home entertainment.

``People love Apple, but we've seen a low adoption of their Apple TV,'' said Jonathan Hurd, director
of Altman Vilandrie, a Boston consulting firm that studies media habits. ``Setup is a big factor.
It's typically more complicated than the average consumer is willing to put up with.''

There is speculation that Apple may be gearing up to take another shot at the market with a new
device. And Google is diving in this fall. It has teamed up with several partners to develop its
Google TV platform.

The Google software, which will pull together Web content and television channels in one programming
guide, will be built into high-definition televisions made by Sony and set-top boxes from Logitech.
It will be powered by a chip from Intel and by Google's Android software, originally designed for
smartphones.

Of course, many living rooms already have all the technology necessary for watching free online
shows. Game consoles like the PlayStation 3, the Wii and the Xbox can be connected to the Internet
and can funnel online videos to the television.

Consumers might be more prone to making the Internet switch than they were a few years ago, Mr.~Hurd
said. According to a recent survey conducted by his company, fewer than 40 percent of viewers under
the age of 24 watch television during prime time. And the number of people watching television shows
on the Internet has doubled in the last year.

``The biggest threat to the traditional companies is on-demand video,'' said Mr.~Hurd. ``The
opportunity is there for a new entrant to come up with a compelling slice of on-demand content.''

\pagebreak
\section{In the Living Room, Hooked on Pay TV}

\lettrine{I}{t}\mycalendar{Aug.'10}{23} is a fantasy shared by many Americans: dropping cable
television and its fat monthly bills and turning instead to the wide-open frontier of Internet
video.

Some are finding that the reality is not that simple.

Just ask Bill Mitchell, a 40-year-old engineer in Winston-Salem, N.C. He canceled his Time Warner
cable service and connected his flat-panel television to the Internet to watch sitcoms and his other
favorite shows, using products from Apple and Boxee.

His experiment lasted 12 months. Recently, grudgingly, he returned to his \$130-a-month cable
subscription, partly because his family wanted programming that was not available online.

``The problem is, we're hooked on shows on HBO and Showtime, like 'True Blood' and 'Dexter,' '' he
said, adding that he wishes he could buy only the shows he wants instead of big bundles of channels
he doesn't. ``It's so frustrating.''

These are confusing times in the living room. The proliferation of Internet video has led to much
talk of ``cord-cutting'' -- a term that has come to mean canceling traditional pay TV and replacing
it with programming from a grab bag of online sources.

But so far Americans are not doing this in any meaningful numbers. ``Nor is there any evidence of it
emerging in the near future,'' said Bruce Leichtman, the president of Leichtman Research Group,
which studies consumer media habits.

This is all the more remarkable, industry analysts say, because it seems to defy the way the
Internet has disrupted and challenged virtually every other major form of media -- from music to
newspapers to books.

In part that is because the television business took action to avoid the same fate. Heavyweight
distributors and producers have protected their business models by ensuring that some must-see shows
and live sporting events cannot legally be seen online.

Technology companies are pushing alternatives like Web-connected set-top boxes. But these are still
not as easy as signing up for cable or satellite service, particularly for those who want to watch
on a big flat-screen TV and not a computer.

And so, in the battle for the living room, 2010 seems to be the year that the incumbent is
strengthening its foothold.

A New York Times/CBS News poll this month found that 88 percent of respondents paid for traditional
TV service. Just 15 percent of those subscribers had considered replacing it with Internet video
services like Hulu and YouTube.

Younger people, though, are more intrigued by the possibility: respondents under the age of 45 were
significantly more likely than older ones to say they had considered replacing their pay TV service.
The poll was conducted Aug.~3-5 with 847 respondents and has a margin of sampling error of plus or
minus three percentage points.

Even through the downturn, the number of people subscribing to pay TV continued to grow. Cable,
satellite and fiber-optic providers added 677,000 customers in the first quarter of this year,
according to the investment firm Sanford C.~Bernstein.

The firm's preliminary numbers for the second quarter, which is traditionally weak, show a slight
drop in subscribers. Satellite providers and Verizon's FiOS service have been stealing market share
from cable.

The cable and satellite companies say that their customers are reluctant to pay more -- the Comcast
chief executive, Brian L.~Roberts, described customers who paid only for video, without a bundle of
other services, as ``very price-sensitive'' -- but insist that cord-cutting has not been an
especially disruptive trend.

To keep customers, especially the price-sensitive ones, the carriers are getting creative. They are
trying to bring the living-room experience to every other screen in a customer's home, including
laptops and tablets. Last week Verizon became the latest carrier to announce plans for an app that
puts live TV on the iPad, pushing out the walls of cable TV's walled garden a bit.

Cablevision, in the New York City area, is running similar trials for tablets and smartphones, Tom
Rutledge, the chief operating officer, said in an earnings call this month. ``Our vision is that we
will provide our full service, everything we offer in the home,'' Mr.~Rutledge said, ``on any device
that can display TV or act as a TV in the home.''

Craig Moffett, a cable industry analyst at Bernstein, says the fortunes of pay TV companies are
nevertheless destined to flag, given customers' dissatisfaction with prices, as well as the
persistent efforts by technology companies to come up with alternatives.

Entrepreneurs will ``keep storming the castle until somebody figures it out,'' Mr.~Moffett wrote in
a recent note to investors. But he also called cord-cutting ``perhaps the most overhyped and
overanticipated phenomenon in tech history.''

Plenty of people say they have foresworn cable for good. They are largely young adults who know
their way around the Internet and have grown accustomed to watching video on computers and other
devices.

The Times/CBS News survey found that people under the age of 45 were about four times as likely as
those 45 and over to say Internet video services could effectively replace cable.

``I pay for the Internet; why would I pay for cable?'' said Breck Yunits, 26, who lives in the
Mission neighborhood in San Francisco in a house he shares with four roommates. They regularly
gather at night around the dinner table and use his Hewlett-Packard laptop to watch ``The Office,''
``Arrested Development'' and other shows on sites like Hulu, NBC.com and MTV.com.

The roommates split a \$40 Internet bill, and one roommate pays around \$10 a month for a Netflix
subscription. In the past they have also grabbed illegal copies of movies using BitTorrent, a
file-sharing system. Mr.~Yunits, a freelance computer programmer, said his girlfriend liked some of
what was legally available only on pay TV, so he might eventually ``be a candidate for it.''

``Personally I would never get cable because the shows online have gotten so good,'' he said. ``You
get to watch what you want to watch, when you want to watch it.''

But people like Mr.~Yunits are still few and far between, in part because it is simply too hard for
most people to cobble together an alternative to pay TV's offerings.

Take Hulu, which attracts tens of millions of users a month. People can watch most broadcast shows
on Hulu, but few cable shows, because they are carefully protected by media companies that rely on
subscriber revenue. ``American Idol,'' the biggest entertainment show in the country, cannot legally
be watched online at all.

Meanwhile there has not been much traction for a host of services like Boxee, which aim to connect
televisions to the Internet and help people find things to watch. There is, though, interest in the
possibilities: a global survey by Nielsen published this month found that one in five people was
eager to buy a Net-connected TV in the next year.

Another alternative comes from video game consoles, like the Xbox 360 from Microsoft, the Nintendo
Wii and the Sony PlayStation 3. They are in millions of living rooms and are adding video
programming, like Netflix's streaming service. But to date, research shows, they are not undermining
cable or satellite -- merely complementing it with pay-per-view TV episodes and some movie streams.

Instead of cord-cutting, Mr.~Leichtman said there was a greater potential for what he called
cord-slicing, cutting back on one part of the monthly cable bill, like pay-per-view movies, while
keeping the basic service.

Any relief would be welcome to John Akerson, 46, a friend of Mr.~Mitchell's in Winston-Salem, who
pays \$100 a month to Time Warner Cable so he and his wife, Jen, can spend two to three hours a
night watching ESPN highlights, ``The Real Housewives of Atlanta'' and local newscasts.

Because of the cost and periodic service glitches, he says that he has ``a deep resentment bordering
on hate'' for the cable company. But other options have not worked out, even though Mr.~Mitchell
gave him a tutorial on using an Apple TV device.

Mr.~Akerson finds it almost laughable that he and his wife have not been able to figure out an
alternative, given that both of them are technically adept.

``I guess it comes down to laziness on my part,'' he said of his failure to find an alternative to
Time Warner Cable. That and, if he switches, ``I'm definitely not going to be able to get 'True
Blood.' ''

\pagebreak
\section{In New Approach to Titanic, an Exhibitor Aids Scientists}

\lettrine{I}{n}\mycalendar{Aug.'10}{23} the 23 years since divers first reached the wreckage of the
Titanic, commercial efforts to salvage artifacts from the doomed ocean liner have aroused as much
scientific dispute as public curiosity.

Many archaeologists and others -- including Robert D.~Ballard of the Woods Hole Oceanographic
Institution, who led an American-French team that discovered the remains 25 years ago -- wanted the
site left untouched as a memorial. Some of them compared salvage efforts to grave robbing.

Now, R.M.S. Titanic, the American company that has removed about 4,650 artifacts from the Titanic,
will try to mend fences with the scientific community by sponsoring two voyages, the first of which
sets sail on Sunday from St.~John's, Newfoundland.

Instead of stripping the wreckage, these trips will include archeologists who will carefully
document and map the site for the first time as a step toward creating a long-term archaeological
management plan for it.

``This is a very different approach for my company,'' said Chris Davino, the president of Premier
Exhibitions, the parent firm of R.M.S. Titanic. ``There was some skepticism among a number of groups
given the record Premier and R.M.S. Titanic have had with the broad archaeological community. And
that skepticism was warranted.''

Other than a few samples from the hull that researchers will use to study the bacteria that are
slowly consuming the ocean liner, nothing will be removed from the wreckage, which sits about 2.5
miles below the sea. Instead, the research group plans to carefully document the area, hoping that
precise measurements will create a baseline for calculating the rate at which is it succumbing to
the bacterial feasting.

While images of the wreckage on the ocean floor have become common over the past two decades, David
Gallo, director of special projects at Woods Hole, said that beyond the stern and bow much of it
remains unrecorded.

``In fact, only about 40 percent of the site has been looked at,'' Mr.~Gallo said by telephone while
traveling to Newfoundland on Friday. ``Some of the images that stick in most people's minds are not
real photographs but paintings.''

Advances in digital photography, sonar and computer imaging software over the last two decades will
obviously aid the documentation. But P.~H. Nargeolet, director of Underwater Research for R.M.S.
Titanic, said that improvements in robotic submarines would be the single most important factor.

Those submarines, which carry two different kinds of high resolution sonar, have guidance systems
that enable them to precisely trace a detailed grid measuring about two by three miles,
Mr.~Nargeolet said. Software can then convert that into a 3-D, digital map of the wreckage.

Using that map for guidance, cameras on other submarines will then take about 80,000 photographs.
Finally, those images will be digitally pasted onto the sonar map to create a 3-D photo.

Mr.~Nargeolet acknowledged that the imaging would show damage not only from the sinking but also
from earlier salvage trips (including the floating of a 15-ton portion of the hull in 1998). Some
critics of his company have said that the salvage efforts have also left the site littered with
debris.

The voyage was prompted by a change of management at R.M.S. Titanic, which has been arguing in court
for 17 years to be granted ownership of the artifacts it collected after 1987 or to be compensated
for salvaging them. Rather than battle the archaeologists, the company's new management met with a
group of them over a year ago and learned that carefully mapping the wreckage site was the
scientific community's priority.

``A lot of decisions in the past have been decided by a court saying you need to go and pick up
things in order to maintain sovereign possession,'' said James P.~Delgado, the president and chief
executive of the Institute of Nautical Archaeology, a former critic of the company whose group is
participating in this trip. ``The level of intervention in the site in the future needs to be
dictated by hard science.''

\pagebreak
\section{Technology Leads More Park Visitors Into Trouble}

\lettrine{C}{athy}\mycalendar{Aug.'10}{23} Hayes was cracking jokes as she recorded a close
encounter with a buffalo on her camera in a recent visit to Yellowstone National Park.

``Watch Donald get gored,'' she said as her companion hustled toward a grazing one-ton beast for a
closer shot with his own camera.

Seconds later, as if on cue, the buffalo lowered its head, pawed the ground and charged, injuring,
as it turns out, Ms.~Hayes.

``We were about 30, 35 feet, and I zoomed in on him, but that wasn't far enough, because they are
fast,'' she recounted later in a YouTube video displaying her bruised and cut legs.

The national parks' history is full of examples of misguided visitors feeding bears, putting
children on buffalos for photos and dipping into geysers despite signs warning of scalding
temperatures.

But today, as an ever more wired and interconnected public visits the parks in rising numbers --
July was a record month for visitors at Yellowstone -- rangers say that technology often figures
into such mishaps.

People with cellphones call rangers from mountaintops to request refreshments or a guide; in Jackson
Hole, Wyo., one lost hiker even asked for hot chocolate.

A French teenager was injured after plunging 75 feet this month from the South Rim of the Grand
Canyon when he backed up while taking pictures. And last fall, a group of hikers in the canyon
called in rescue helicopters three times by pressing the emergency button on their satellite
location device. When rangers arrived the second time, the hikers explained that their water supply
``tasted salty.''

``Because of having that electronic device, people have an expectation that they can do something
stupid and be rescued,'' said Jackie Skaggs, spokeswoman for Grand Teton National Park in Wyoming.

``Every once in a while we get a call from someone who has gone to the top of a peak, the weather
has turned and they are confused about how to get down and they want someone to personally escort
them,'' Ms.~Skaggs said. ``The answer is that you are up there for the night.''

The National Park Service does not keep track of what percentage of its search and rescue missions,
which have been climbing for the last five years and topped 3,500 in 2009, are technology related.
But in an effort to home in on ``contributing factors'' to park accidents, the service recently felt
compelled to add ``inattention to surroundings'' to more old-fashioned causes like ``darkness'' and
``animals.''

The service acknowledges that the new technologies have benefits as well. They can and do save lives
when calls come from people who really are in trouble.

The park service itself has put technology to good use in countering the occasional unruliness of
visitors. Last summer, several men who thought they had managed to urinate undetected into the Old
Faithful geyser in Yellowstone were surprised to be confronted by rangers shortly after their stunt.
It turns out that the park had installed a 24-hour camera so people could experience Old Faithful's
majesty online. Viewers spotted the men in action and called to alert the park.

In an era when most people experience the wild mostly through television shows that may push the
boundaries of appropriateness for entertainment, rangers say people can wildly miscalculate the
risks of their antics.

In an extreme instance in April, two young men from Las Vegas were killed in Zion National Park in
Utah while trying to float a hand-built log raft down the Virgin River. A park investigation found
that the men ``did not have whitewater rafting experience, and had limited camping experience,
little food and no overnight gear.''

``They told their father that they intended to record their entire trip on video camera as an entry
into the 'Man vs. Wild'competition'' on television, investigators wrote.

Far more common but no less perilous, park workers say, are visitors who arrive with cellphones or
GPS devices and little else -- sometimes not even water -- and find themselves in trouble. Such
visitors often acknowledge that they have pushed themselves too far because they believe that in a
bind, the technology can save them.

It does not always work out that way. ``We have seen people who have solely relied on GPS technology
but were not using common sense or maps and compasses, and it leads them astray,'' said Kyle
Patterson, a spokesman for Rocky Mountain National Park, just outside Denver.

Like a lot of other national parks, Rocky Mountain does not allow cellphone towers, so service that
visitors may take for granted is spotty at best. ``Sometimes when they call 911, it goes to a
communications center in Nebraska or Wyoming,'' Mr.~Patterson said. ``And that can take a long time
to sort out.''

One of the most frustrating new technologies for the parks to deal with, rangers say, are the
personal satellite messaging devices that can send out an emergency signal but are not capable of
two-way communication. (Globalstar Inc., the manufacturer of SPOT brand devices, says new models
allow owners to send a message with the help request.)

In some cases, said Keith Lober, the ranger in charge of search and rescue at Yosemite National Park
in California, the calls ``come from people who don't need the 911 service, but they take the SPOT
and at the first sign of trouble, they hit the panic button.''

But without two-way communication, the rangers cannot evaluate the seriousness of the call, so they
respond as if it were an emergency.

Last fall, two men with teenage sons pressed the help button on a device they were carrying as they
hiked the challenging backcountry of Grand Canyon National Park. Search and rescue sent a
helicopter, but the men declined to board, saying they had activated the device because they were
short on water.

The group's leader had hiked the Grand Canyon once before, but the other man had little backpacking
experience. Rangers reported that the leader told them that without the device, ``we would have
never attempted this hike.''

The group activated the device again the next evening. Darkness prevented a park helicopter from
flying in, but the Arizona Department of Public Safety sent in a helicopter whose crew could use
night vision equipment.

The hikers were found and again refused rescue. They said they had been afraid of dehydration
because the local water ``tasted salty.'' They were provided with water.

Helicopter trips into the park can cost as much as \$3,400 an hour, said Maureen Oltrogge, a
spokeswoman for Grand Canyon National Park.

So perhaps it is no surprise that when the hikers pressed the button again the following morning,
park personnel gave them no choice but to return home. The leader was issued a citation for creating
hazardous conditions in the parks.

\pagebreak
\section{After Close Australia Vote, Parties Seek Coalition}

\lettrine{T}{he}\mycalendar{Aug.'10}{23} two main candidates in Australia's cliffhanger election
began scrambling Sunday to win the support of three independent lawmakers and a newly elected Greens
party representative who will decide which party forms a minority government in one of the tightest
races ever seen here.

By late Sunday, with nearly 80 percent of the vote counted, neither the governing center-left Labor
Party nor a conservative opposition coalition had captured enough votes to claim a majority in the
150-member House of Representatives.

About two million mail ballots have not yet been counted, and it could take at least a week before
the final result is known.

Prime Minister Julia Gillard and her conservative rival, Tony Abbott, began what many analysts
expect to be a lengthy period of negotiations to win over the crucial bloc. Australia has not had a
minority government since 1940.

``We are clearly at a historic moment in our country,'' Ms.~Gillard said in Melbourne.

Labor, which held 83 seats before the election on Saturday, lost at least 13 and is at risk of
becoming the first government in nearly 80 years to be turned from office after just one three-year
term.

Talks were expected to begin in earnest in Canberra, the capital, on Monday.

The Greens drew a record number of votes, mainly from traditional Labor supporters angry about the
government's decision to suspend its cap-and-trade policy on carbon emissions until 2013. The three
independent lawmakers represent rural constituencies and have links to parties within the
conservative coalition.

\pagebreak
\section{Gunman and 8 Hostages Dead in the Philippines}

\lettrine{A}{}\mycalendar{Aug.'10}{24} former police officer took a busload of tourists hostage in
downtown Manila on Monday morning, opening a 12-hour standoff that was broadcast live on television,
including its end as police commandos stormed the bus before a watching crowd.

Eight tourists, all from Hong Kong, were killed, along with the hostage-taker. He was identified as
Rolando Mendoza, a 55-year-old officer who had been accused of robbery and extortion and was fired
last year.

There were indications that Mr.~Mendoza, who news reports said was armed with an M-16 assault rifle,
was watching the live news broadcasts of the scene on a monitor inside the bus as it sat for hours,
curtains drawn, at a major public plaza.

Throughout the day, the gunman wrote his demands for the return of his job and benefits on cardboard
and pasted them on the windshield of the bus. One message read, ``Big mistake to correct a big wrong
decision.''

Late in the afternoon, he posted a message saying, ``Media now,'' apparently telling journalists to
come to him. But by then the police prevented reporters from even getting near the bus.

At one point, the gunman's brother complained to reporters near the scene that the police were
threatening him; the cameras then showed him being detained by officers and shoved into a police
car, his relatives wailing behind him. Shortly after, shots were heard from inside the bus.

Gunfire also broke out when the police tried to break the windshield and glass windows of the bus
with sledgehammers. A bystander was hit in the leg by one of the bullets.

Mr.~Mendoza gave an interview to Radio Mindanao Network, a Manila station, in which he admitted
shooting two of the hostages and threatened to kill more.

``I shot two Chinese,'' he told the station in Tagalog. ``I will finish them all if they do not
stop.''

The commandos struck after the bus driver jumped through a window and ran from the bus screaming,
``Everybody is dead!''

The cameras captured the commandos, armed with rifles, surrounding the bus and opening an emergency
exit, as emergency vehicles converged at the scene in heavy rain.

Police officers threw tear gas inside the bus, apparently forcing Mr.~Mendoza to go near the bus's
main door, which they had torn down. Shots were heard and in a split second the body of a man --
presumably Mr.~Mendoza -- was seen slumped by the door.

Several of the unharmed hostages, visibly shaken and some crying, were taken off the bus through the
emergency exit.

President Benigno Aquino, in a news briefing around midnight, said Mr.~Mendoza might have gained
some advantage from the coverage. ``To a certain extent, he may have had a bird's eye view of the
developments, which might not have helped,'' the president said.

The case captivated -- and angered -- Filipinos, with many blaming the news coverage for the
disastrous end.

The chief executive of Hong Kong, Donald Tsang, called the case a ``major tragedy,'' and criticized
the Filipino authorities.

``The way it was handled, particularly the outcome, I find is disappointing,'' Mr.~Tsang said at a
news conference in Hong Kong, Reuters reported.

But Mr.~Aquino defended the actions of the authorities: ``The idea was to let the ground commanders
who are the experts in this field handle the operation with minimal interference from people who are
less expert.''

\pagebreak
\section{Germany Plans Limits on Facebook Use in Hiring}

\lettrine{A}{s}\mycalendar{Aug.'10}{26} part of the draft of a law governing workplace privacy, the
German government on Wednesday proposed placing restrictions on employers who want to use Facebook
profiles when recruiting.

The bill would allow managers to search for publicly accessible information about prospective
employees on the Web and to view their pages on job networking sites, like LinkedIn or Xing. But it
would draw the line at purely social networking sites like Facebook, said Philipp Spauschus, a
spokesman for the Interior Minister, Thomas de Maizi\`ere.

Chancellor Angela Merkel's cabinet on Wednesday gave its backing to the proposed law. The bill will
now go to Parliament for discussion, and could be passed as early as this year, Mr.~Spauschus said.

The law also would prohibit companies from secretly videotaping employees, though they could still
videotape in certain areas as long as they disclosed the fact. It would also allow employers to hold
secret investigations when they suspected a crime had been committed.

Germany's Nazi-era history has made the country extremely cautious on matters of individual privacy.
Concerns have been heightened in recent years by scandals involving companies' secret videotaping of
employees, as well as intercepting their e-mail and bank data. The explosion of Web-based
information tools has added to the unease.

The German authorities are investigating Google for having collected private Internet information
while doing research for its Street View mapping service, and they have asked Apple to explain its
data-collection policies for the iPhone.

Facebook, which says it has more than 500 million users worldwide, with about 10 million in Germany,
has come under fire for what some consider privacy shortcomings, as when the site changed its
default settings to reveal more of individuals' personal data. The German proposal, however, is
aimed squarely at employers.

Peter Schaar, the German commissioner for data protection and freedom of information, told The
Associated Press that the proposal was ``a substantial improvement on the status quo in dealing with
employees' data.''

There are currently no rules governing how companies use Facebook data, Mr.~Spauschus said. The
proposal is meant to create guidelines for the courts in handling the cases that will inevitably
arise as social networking penetrates further into everyday life, he said, and companies would also
benefit from clear rules.

Sarah Roy, a spokeswoman in Paris for Facebook, said the company generally did not comment on
legislation as a matter of policy. But she said that the Web site's privacy settings allowed users
to share information as broadly or as narrowly as they liked, either with entire networks or with a
limited number of participants.

\pagebreak
\section{Chinese Airport Unsafe Before Crash}

\lettrine{C}{hina}\mycalendar{Aug.'10}{26}'s largest passenger airline deemed nighttime landings at
a new airport in northeastern China unsafe a year before a Henan Airlines jet crashed there on
Tuesday night, killing 42 passengers and injuring 54 others.

China Southern Airlines, the country's largest passenger carrier, concluded that the Lindu airport,
outside Yichun, a city of one million people in Heilongjiang Province, was ``in principle not
suitable for night flights,'' according to a safety notice posted on a Chinese news organization's
Web site. Daytime landings in rainy conditions were also ruled out for the airline, the notice said.

The small airport, nestled in a thickly forested valley, opened last year, according to Chinese news
media.

The crash Tuesday at the Lindu airport was China's first major passenger airline disaster since a
China Eastern Airlines plane crashed into a lake in northern China in 2004, killing 55 people. China
has recently made a concerted effort to tighten safety rules and improve training.

It has also been rapidly opening new airports to help spur economic development and satisfy surging
demand. The number of airline passengers in China more than tripled between 2000 and 2009,
government statistics show.

On Wednesday, investigators continued to search for clues that would explain why the plane, a
Brazilian-made Embraer E-190, crashed into a grassy area and burst into flames at 9:36 p.m. Tuesday
while trying to land on a fog-shrouded runway at the Lindu airport. The jet had taken off about 40
minutes earlier from the provincial capital of Harbin.

A team of Chinese officials, led by Deputy Prime Minister Zhang Dejiang, headed for Yichun for the
inquiry. Technicians from Embraer also flew to China to investigate. Henan Airlines grounded flights
for three days.

The state-run Xinhua news agency reported that Chinese carriers had previously complained of
problems with E-190 aircraft, including cracks in the turbine plates and flight control system
errors. China's Civil Aviation Administration organized a workshop in June to discuss the concerns,
Xinhua said.

Embraer has 650 E-170 and E-190 model jets operating in 39 countries according to the company, and
until the crash of Henan Airlines on Tuesday, there have been no fatal crashes of the 190. JetBlue,
USAirways, and Air Canada operate E-190s.

Embraer officials confirmed that the company was scheduled to meet with Henan next week.A local
official told Xinhua that crew members of the Henan Airlines flight reported that they could see
lights on the ground and requested a normal landing.

Survivors said the plane jolted so violently while trying to land that luggage flew off the overhead
racks. One middle-aged man, interviewed in a hospital bed by the state-controlled CCTV television
network, said that after the jet hit the ground, smoke billowed from the rear of the plane and he
feared suffocation.

``It was very strong,'' he said. ``It looked like we had only two or three minutes left. I knew
something bad was going to happen.''

Hua Jingwei, head of the Communist Party's propaganda department in Yichun, told Xinhua that the
plane broke in two as it approached the runway, throwing some passengers out of the cabin. That
account was not immediately confirmed by survivors.

Henan Airlines primarily operates regional flights. Passengers on the Tuesday flight included five
children and a team of officials from Beijing, among them the vice minister for human resources, Sun
Baoshu. He was described as seriously injured.

\pagebreak
\section{Carter Arrives in North Korea}

\lettrine{F}{ormer}\mycalendar{Aug.'10}{26} President Jimmy Carter arrived in North Korea on
Wednesday to seek the release of an American held by the North, its state-run media reported.

Analysts in Seoul said Mr.~Carter, who helped defuse a Korean nuclear crisis more than 16 years ago,
could also try to help end the two countries' impasse.

The man Mr.~Carter is seeking to free is Aijalon Mahli Gomes, a 30-year-old Christian from Boston
who was arrested in January for crossing into North Korea and sentenced in April to eight years of
hard labor and fined \$700,000. Last month, North Korea said he tried to kill himself out of
``frustration with the U.S.~government's failure to free him.''

The visit by Mr.~Carter, an evangelical Christian, is the second to North Korea by a former American
president in a year on what the United States described as a private humanitarian missions. Last
August, Bill Clinton flew there and met with the reclusive North Korean leader, Kim Jong-il, to
secure the release of two American journalists held for five months for illegal entry.

The Obama administration kept its distance, emphasizing that Mr.~Carter not an envoy. ``I'll just
say that President Carter is on a private humanitarian mission and I'm not going to comment more
beyond that,'' said Mark Toner, a State Department spokesman.

But as with Mr.~Clinton's visit, Mr.~Carter's has deeper diplomatic undercurrents. The North Koreans
have used the captive Americans as bargaining chips, promising to release them in exchange for
visits from specific high-profile Americans. North Korea can portray the meetings domestically as
evidence of its international importance, while the United States has a high-level direct encounter
that it cannot officially engage in.

But Mr.~Carter has a long history as an independent agent, and some administration officials worried
that he might undercut their policy in some way and make it harder to keep up the pressure on
Pyongyang to give up its nuclear program.

It was not immediately clear who among the North Koreans would meet with Mr.~Carter. The North
Korean media reports said that he was greeted at the airport in Pyongyang, the capital, by Kim
Kye-gwan, a senior diplomat who has been the North's main envoy to the six-nation talks on its
nuclear weapons program. The talks have been stalled for more than two years, but the North recently
said it was willing to return to the discussions.

Higher-level meetings would appear to be likely, since Mr.~Carter's visit comes at a fraught time
for North Korea. Its economy remains deeply troubled, and its ravaged agricultural sector has been
further damaged by recent flooding. A March torpedo attack that sank one of the South's warships,
killing 46 sailors, drove inter-Korean relations to their lowest point in years and added to
tensions with the United States. In addition, there may be a struggle over succession within the
government of Kim Jong-il, who has had serious health problems.

The case of Mr.~Gomes also touches on efforts of Christians in South Korea and the United States on
behalf of North Koreans. His illegal entry was made in support of Robert Park, a fellow Christian
from the United States who crossed from China in December to call attention to the dismal conditions
in the North's prison camps. Mr.~Park was expelled after about 40 days.

Mr.~Carter has been a contentious figure among South Korean conservatives. ``Carter is idealistic,
not realistic when it comes to North Korea,'' said Hong Kwan-hee, director of the Institute for
Security Strategy in Seoul. ``North Korea always has tried to use prominent Americans, preferably
Democrats, as a medium to engage the United States and drive a wedge between Seoul and Washington.
At home, they tell their people that the former U.S.~president came to pay respect to their
general.''

Michael J.~Green, a regional scholar at Georgetown University who was President George W.~Bush's
Asia adviser, said that, despite the skepticism regarding North Korea, the White House was also
concerned about remaining completely out of touch with such a volatile and unpredictable state.

There is a feeling ``that the lack of communication with North Korea is unsafe,'' he said. ``There's
a consensus taking shape that at some point they have to have some contact and the Carter visit
could help. But he's a risky messenger.''

During Mr.~Carter's first trip to Pyongyang, in 1994, he met with Kim Il-sung, the father of the
current leader. The trip helped restart nuclear talks that led to a disarmament deal. But it fell
apart around 2002 as the United States accused the North of running a secret uranium enrichment
program, touching off the enduring nuclear standoff.

\pagebreak
\section{China Gives Ex-Official Suspended Death Sentence}

\lettrine{A}{}\mycalendar{Aug.'10}{26} former assistant minister of public security linked to a
major corruption case in China has been given a suspended death sentence for taking more than \$1
million in bribes over six years, the Chinese news media reported Wednesday.

The defendant, Zheng Shaodong, was sentenced to death with a two-year reprieve -- a penalty that
typically is commuted to life in prison.

The state news media have reported that Mr.~Zheng was implicated in a corruption inquiry that
centered on Huang Guangyu, who built a home appliance chain, Gome, into a multi-billion-dollar
empire. Mr.~Huang was sentenced in May to 14 years in prison for insider trading, bribery and other
crimes.

According to China Daily, the state-run English language newspaper, Mr.~Zheng was accused of
enriching himself by granting jobs and promotions to workers and granting favors to subjects of
criminal investigations.

From 2001 to 2005, Mr.~Zheng was the official of the Ministry of Public Security in charge of
investigations of economic crimes. He was then appointed an assistant minister and listed as the
10th-ranking official. In 2006, Mr.~Huang asked Mr.~Zheng to protect him from investigators looking
into allegations of a fraud scheme involving the Bank of China's Beijing branch, the newspaper
Beijing Morning News reported, adding that Mr.~Zheng instructed a lower-ranking official to ``sort
it out.'' Mr.~Huang was eventually cleared of suspicions in that inquiry, the newspaper said.

Mr.~Zheng was arrested in January 2009.

The month before, in his last public appearance, he said the government should exercise caution when
investigating allegations of economic crimes.

\pagebreak
\section{N.H.L. Warned Not to Expand in Europe}

\lettrine{T}{he}\mycalendar{Aug.'10}{26} National Hockey League should forget about any idea of
expanding into Europe, the head of the International Ice Hockey Federation said at the World Hockey
Summit.

Speaking to a gathering of some of the sport's top executives on Tuesday, Ren\'e Fasel, the
president of the governing body for international ice hockey, blasted the idea of the N.H.L.'s
setting up operations in Europe, told the league to stay away and said that he would never allow it
to happen.

``Try to come -- good luck,'' Fasel said during a question and answer session at the four-day
meeting in Toronto. ``This is our territory, and I will fight like hell to not allow anybody to come
from abroad.''

The N.H.L. has long flirted with the idea of one day adding a European division to the league, with
the N.H.L. deputy commissioner, Bill Daly, calling it a dream that could become reality within the
next 10 years.

But while the N.H.L. has repeatedly stated the stability of its 30 teams is its top priority, the
league has been working hard to increase its presence in Europe.

This year, a record six N.H.L. teams will open the season in Europe, with each team playing two
games in Helsinki, Stockholm or Prague.

Fasel also said that if the National Basketball Association thought expanding into Europe would be
successful, then the league would have tried it by now.

``There are 500 million people in the European community, different languages, different cultures,
different view of sport. It's different,'' Fasel said. ``Let us do the job in Europe, and in the
end, let us come together. Keep our identity in Europe; you keep your identity here.''

Europe is becoming the new battleground for ice hockey, with the I.I.H.F. eager to create a
Champions League-style competition that soccer has developed with massive success.

Fasel, who is a member of the International Olympic Committee's executive board, said his dream was
for a tournament involving teams from Sweden, Russia, Germany, Finland and Switzerland, with the
European champion playing the Stanley Cup winner from the N.H.L.

\pagebreak
\section{Digital Devices Deprive Brain of Needed Downtime}

\lettrine{I}{t}\mycalendar{Aug.'10}{26}'s 1 p.m. on a Thursday and Dianne Bates, 40, juggles three
screens. She listens to a few songs on her iPod, then taps out a quick e-mail on her iPhone and
turns her attention to the high-definition television.

Just another day at the gym.

As Ms.~Bates multitasks, she is also churning her legs in fast loops on an elliptical machine in a
downtown fitness center. She is in good company. In gyms and elsewhere, people use phones and other
electronic devices to get work done -- and as a reliable antidote to boredom.

Cellphones, which in the last few years have become full-fledged computers with high-speed Internet
connections, let people relieve the tedium of exercising, the grocery store line, stoplights or
lulls in the dinner conversation.

The technology makes the tiniest windows of time entertaining, and potentially productive. But
scientists point to an unanticipated side effect: when people keep their brains busy with digital
input, they are forfeiting downtime that could allow them to better learn and remember information,
or come up with new ideas.

Ms.~Bates, for example, might be clearer-headed if she went for a run outside, away from her
devices, research suggests.

At the University of California, San Francisco, scientists have found that when rats have a new
experience, like exploring an unfamiliar area, their brains show new patterns of activity. But only
when the rats take a break from their exploration do they process those patterns in a way that seems
to create a persistent memory of the experience.

The researchers suspect that the findings also apply to how humans learn.

``Almost certainly, downtime lets the brain go over experiences it's had, solidify them and turn
them into permanent long-term memories,'' said Loren Frank, assistant professor in the department of
physiology at the university, where he specializes in learning and memory. He said he believed that
when the brain was constantly stimulated, ``you prevent this learning process.''

At the University of Michigan, a study found that people learned significantly better after a walk
in nature than after a walk in a dense urban environment, suggesting that processing a barrage of
information leaves people fatigued.

Even though people feel entertained, even relaxed, when they multitask while exercising, or pass a
moment at the bus stop by catching a quick video clip, they might be taxing their brains, scientists
say.

``People think they're refreshing themselves, but they're fatiguing themselves,'' said Marc Berman,
a University of Michigan neuroscientist.

Regardless, there is now a whole industry of mobile software developers competing to help people
scratch the entertainment itch. Flurry, a company that tracks the use of apps, has found that mobile
games are typically played for 6.3 minutes, but that many are played for much shorter intervals. One
popular game that involves stacking blocks gets played for 2.2 minutes on average.

Today's game makers are trying to fill small bits of free time, said Sebastien de Halleux, a
co-founder of PlayFish, a game company owned by the industry giant Electronic Arts.

``Instead of having long relaxing breaks, like taking two hours for lunch, we have a lot of these
micro-moments,'' he said. Game makers like Electronic Arts, he added, ``have reinvented the game
experience to fit into micro-moments.''

Many business people, of course, have good reason to be constantly checking their phones. But this
can take a mental toll. Henry Chen, 26, a self-employed auto mechanic in San Francisco, has mixed
feelings about his BlackBerry habits.

``I check it a lot, whenever there is downtime,'' Mr.~Chen said. Moments earlier, he was texting
with a friend while he stood in line at a bagel shop; he stopped only when the woman behind the
counter interrupted him to ask for his order.

Mr.~Chen, who recently started his business, doesn't want to miss a potential customer. Yet he says
that since he upgraded his phone a year ago to a feature-rich BlackBerry, he can feel stressed out
by what he described as internal pressure to constantly stay in contact.

``It's become a demand. Not necessarily a demand of the customer, but a demand of my head,'' he
said. ``I told my girlfriend that I'm more tired since I got this thing.''

In the parking lot outside the bagel shop, others were filling up moments with their phones. While
Eddie Umadhay, 59, a construction inspector, sat in his car waiting for his wife to grocery shop, he
deleted old e-mail while listening to news on the radio. On a bench outside a coffee house, Ossie
Gabriel, 44, a nurse practitioner, waited for a friend and checked e-mail ``to kill time.''

Crossing the street from the grocery store to his car, David Alvarado pushed his 2-year-old daughter
in a cart filled with shopping bags, his phone pressed to his ear.

He was talking to a colleague about work scheduling, noting that he wanted to steal a moment to make
the call between paying for the groceries and driving.

``I wanted to take advantage of the little gap,'' said Mr.~Alvarado, 30, a facilities manager at a
community center.

For many such people, the little digital asides come on top of heavy use of computers during the
day. Take Ms.~Bates, the exercising multitasker at the expansive Bakar Fitness and Recreation
Center. She wakes up and peeks at her iPhone before she gets out of bed. At her job in advertising,
she spends all day in front of her laptop.

But, far from wanting a break from screens when she exercises, she says she couldn't possibly spend
55 minutes on the elliptical machine without ``lots of things to do.'' This includes relentless
channel surfing.

``I switch constantly,'' she said. ``I can't stand commercials. I have to flip around unless I'm
watching 'Project Runway' or something I'm really into.''

Some researchers say that whatever downside there is to not resting the brain, it pales in
comparison to the benefits technology can bring in motivating people to sweat.

``Exercise needs to be part of our lives in the sedentary world we're immersed in. Anything that
helps us move is beneficial,'' said John J.~Ratey, associate clinical professor of psychiatry at the
Harvard Medical School and author of ``Spark: The Revolutionary New Science of Exercise and the
Brain.''

But all things being equal, Mr.~Ratey said, he would prefer to see people do their workouts away
from their devices: ``There is more bang for your buck doing it outside, for your mood and working
memory.''

Of the 70 cardio machines on the main floor at Bakar Fitness, 67 have televisions attached. Most of
them also have iPod docks and displays showing workout performance, and a few have games, like a
rope-climbing machine that shows an animated character climbing the rope while the live human does
so too.

A few months ago, the cable TV went out and some patrons were apoplectic. ``It was an uproar. People
said: 'That's what we're paying for,' '' said Leeane Jensen, 28, the fitness manager.

At least one exerciser has a different take. Two stories up from the main floor, Peter Colley, 23,
churns away on one of the several dozen elliptical machines without a TV. Instead, they are bathed
in sunlight, looking out onto the pool and palm trees.

``I look at the wind on the trees. I watch the swimmers go back and forth,'' Mr.~Colley said. ``I
usually come here to clear my head.''

\pagebreak
\section{Shaping Tea Party Passion Into Campaign Force}

\lettrine{O}{n}\mycalendar{Aug.'10}{26} a Saturday in August when most of the political class has
escaped this city's swelter, 50 Tea Party leaders have flown in from across the country to jam into
a conference room in an office building on Pennsylvania Avenue, apparently unconcerned that the
fancy address does not guarantee air-conditioning on weekends. They have come to learn how to take
over the country, voter by voter.

Look for houses with flags, they are instructed; their residents tend to be patriotic conservatives.
Marine flags or religious symbols, ditto. Take doggie treats with you as you canvass neighborhoods
-- ``Now they are your best friend; it's dog person to dog person.'' Don't just hand out yard signs
and bumper stickers for your candidate -- offer to plant them on the lawn or paste them on the
bumper (front driver's side works best.) Follow up with thank you notes, the handwritten kind. Be
polite, and don't take rejection personally: ``Remember, it's for freedom!''

This is a three-day ``boot camp'' at FreedomWorks, the Washington advocacy group that has done more
than any other organization to build the Tea Party movement. For 18 months, the group's young staff
has been conducting training sessions like this one across the country, in hotel conference rooms or
basements of bars, shaping the inchoate anger of the Tea Party with its libertarian ideology and
leftist organizing tactics.

The goal is to turn local Tea Party groups into a standing get-out-the-vote operation in
Congressional districts across the country. Sarah Palin made community organizing a term of derision
during the 2008 presidential campaign; FreedomWorks has made Tea Party conservatives the surprise
community organizing force of the 2010 midterm elections, showing on-the-ground strength in races
like the Republican primary for the Senate in Alaska on Tuesday, where the upstart Joe Miller was
leading Senator Lisa Murkowski in a race that may take weeks to call.

``This movement, if we can turn out hundreds or thousands to the streets to protest and wave signs
and yell and make an impact on public policy debate, then we can make a lot of difference,'' Brendan
Steinhauser, FreedomWorks's chief organizer for the Tea Party groups, told the leaders gathered
here. ``But if those same people go and walk neighborhoods and do all the things we're talking
about, put up the door-hangers in the final 72 hours and make the phone calls, we may crush some of
these guys.''

In recent months, FreedomWorks has teamed up with Glenn Beck, the biggest celebrity of the Tea Party
movement to promote it. This weekend, with many Tea Party supporters descending on Washington for a
rally that Mr.~Beck is holding at the Lincoln Memorial, FreedomWorks is staging a convention where
Tea Party candidates will address 1,600 activists.

Through its political action committee, FreedomWorks plans to spend \$10 million on the midterm
elections, on campaign paraphernalia -- signs for candidates like Rand Paul of Kentucky and Marco
Rubio of Florida are stacked around the offices here -- voter lists, and a phone system that allows
volunteers to make calls for candidates around the country from their home computers. With
``microfinancing'' grants, it will steer money from FreedomWorks donors -- the tax code protects
their anonymity -- to local Tea Parties.

Other groups will spend more. On the left, a coalition of unions plans to spend at least \$88
million; on the right, Americans for Prosperity will spend \$45 million.

But FreedomWorks's pitch to activists is that the money is not really the point. It is about
convincing friends, neighbors and strangers in Congressional districts where 100 or 1,000 votes can
make all the difference. The activists tend to be a zealous lot to start with; FreedomWorks urges
them to channel that energy by becoming precinct captains, knocking on doors and learning from the
way that Barack Obama -- not someone Tea Party supporters generally admire -- wrapped up the
Democratic nomination for president by organizing the caucus states.

FreedomWorks was founded in 1984 as Citizens for a Sound Economy, which was financed by the Koch
Foundation, the underwriter for many libertarian causes. In 2003, it hired as its chairman Dick
Armey, the former Texas congressman and House majority leader who was a force behind the 1994
Republican takeover of Congress.

While Mr.~Armey serves as a kind of ambassador for FreedomWorks, the day-to-day task of organizing
Tea Party groups has gone to a staff of about 20 hard-charging conservatives in their 20s and 30s --
a striking contrast to a movement that is made up largely of people twice their age and more. Tea
Party leaders at the boot camp gasped when Mr.~Steinhauser emphasized the importance of going after
so-called Reagan Democrats and then noted that he himself was not born until 1981, after Ronald
Reagan's first inauguration.

Staff members like to say that they model FreedomWorks on the Grateful Dead or Virgin Atlantic
Airways: they want to build a like-minded community, an endeavor that is as much fun as work.

But they are also deeply ideological; a portrait of Ayn Rand hangs on the office walls along with
one of Jerry Garcia. FreedomWorks was founded to promote the theories of the Austrian economic
school, which argues that economic models are useless because they cannot account for all the
variables of human behavior, and that markets must be unfettered to succeed.

New employees receive a required-reading list that includes ``Rules for Radicals,'' by Saul Alinsky,
the father of modern community organizing, and ``A Force More Powerful,'' about 20th-century social
movements, as well as Fr\'ed\'eric Bastiat's ``The Law,'' which argues that governments are
essentially stealing when they tax their citizens to spend on welfare, infrastructure or public
education. FreedomWorks urges Tea Party groups to read the same works. (``It's better than 'Going
Rogue,' '' said Mr.~Steinhauser, referring to Ms.~Palin's memoir.)

While other conservative groups have tried to mobilize the Tea Party energy, FreedomWorks moved
first, and most aggressively. Hours after Rick Santelli called for ``a Chicago tea party'' in a
widely viewed rant on CNBC in February 2009, it put up a Web site with tips on how to hold a tea
party, then a Google map of events. As more people found the map on Web searches, they e-mailed
FreedomWorks information on their own events, ultimately allowing Mr.~Steinhauser to compile a list
of thousands of Tea Party contacts across the country.

That list allowed the group to mobilize volunteers to Massachusetts in January to campaign for Scott
P.~Brown, who won the United States Senate seat that had been occupied by Edward M.~Kennedy for
nearly 50 years, and to Utah to elect Mike Lee as the Republican nominee for Senate after Tea Party
groups deposed the three-term incumbent Robert F.~Bennett. About 180,000 people voted in the primary
that Mr.~Lee won; FreedomWorks says 30,000 had received a phone call or a visit from its volunteers.

Its candidates are libertarians and economic conservatives, but in the 2010 midterm elections,
FreedomWorks is urging Tea Party groups to work for any Republican, on the theory that a compromised
Republican is better than Democratic control of Congress.

Mr.~Steinhauser has traveled to 42 states to train local groups or meet with leaders in races where
FreedomWorks hopes to make a difference. But the Tea Parties like to think of themselves as
leaderless organizations, and are suspicious of attempts to co-opt their energy.

In a swing through New England last month, he met with activists eager to defeat Charlie Bass, a
former Republican congressman from New Hampshire who is running again in the Sept.~14 primary. But
they did not want to endorse either of the Tea Party candidates because they feared their membership
would resent anything that looked like top-down control. ``You have to endorse,'' Mr.~Steinhauser
told them. ``If you don't, the bad guys will.'' Each group should endorse separately, he advised, so
that the local news media would write a new story each time.

Still, the activists were eager for outside advice.

``If you give us the education, we'll do the work,'' Robert Horr, the chairman of the Cumberland
County Tea Party, in Maine, told him. ``Just aim us.''

Mr.~Steinhauser encouraged the Maine activists to start getting behind candidates to challenge
Senator Olympia J.~Snowe, a Republican up for re-election in 2012.

FreedomWorks is focused particularly on midterm races in Florida, New York, Ohio and Pennsylvania.
For the boot camp in Washington, it had flown in representatives from those states.

Nan Swift, FreedomWorks's campaign manager, encouraged them to stage dramatic events to call
attention to their candidates -- ``Everyone already thinks we're crazy, embrace it!'' -- and to sign
up for their opponents' e-mails, then show up to their events and swamp them with signs.

Mr.~Steinhauser urged them not to waste their energy on districts so deeply Democratic that they
cannot be won.

Still, he did not cut off any opportunity; after all, he noted, no one thought Scott Brown could
win. ``This year, if there's one message you can take away,'' he said, ``it's that nothing is
impossible for us.''

\pagebreak
\section{Wal-Mart Asks Supreme Court to Hear Bias Suit}

\lettrine{W}{al-Mart}\mycalendar{Aug.'10}{26} Stores asked the Supreme Court on Wednesday to review
the largest employment discrimination lawsuit in American history, involving more than a million
female workers, current and former, at Wal-Mart and Sam's Club stores.

Nine years after the suit was filed, the central issue before the Supreme Court will not be whether
any discrimination occurred, but whether more than a million people can even make this joint claim
through a class-action lawsuit, as opposed to filing claims individually or in smaller groups.

In April, the United States Court of Appeals for the Ninth Circuit in San Francisco ruled 6-5 that
the lawsuit could proceed as a jumbo class action -- the fourth judicial decision upholding a class
action.

The stakes are huge. If the Supreme Court allows the suit to proceed as a class action, that could
easily cost Wal-Mart \$1 billion or more in damages, legal experts say.

More significant, the court's ruling could set guidelines for other types of class-action suits.
``This is the big one that will set the standards for all other class actions,'' said Robin
S.~Conrad, executive vice president of the National Chamber Litigation Center, an arm of the United
States Chamber of Commerce, which has filed several amicus briefs backing Wal-Mart.

Meanwhile, the women at the core of the original lawsuit, known as Dukes v. Wal-Mart, have tried to
move on with their lives. Some still work at Wal-Mart and have been promoted or received raises. One
still works as a greeter there. Others have left Wal-Mart.

The case began nearly a decade ago with one woman, Stephanie Odle, who was upset to discover that
the top manager at the Sam's Club where she worked as an assistant store manager had been
administering a promotion test to the three male assistant store managers but not to her.

That came after Ms.~Odle discovered that a male assistant manager at a previous Sam's Club where she
worked had been earning \$23,000 more a year than she was. When she complained, she said, the
district manager responded, ``Stephanie, that assistant manager has a family and two children to
support.''

``I told him, 'I'm a single mother, and I have a 6-month-old child to support,' '' she recalled in
an interview.

Lawyers representing the plaintiffs recruited Ms.~Odle after obtaining a data showing that just a
third of Wal-Mart's managers were women even though two-thirds of its employees were. The lawyers
wanted to enlist a Wal-Mart employee whose complaints about pay and promotions would be a base from
which to build a broader sex discrimination case.

Ms.~Odle's story, along with those of six other women, became the seed of the 2001 lawsuit that
accused Wal-Mart of systematic discrimination against women in pay and promotions. No one expected
it to become such a drawn-out battle.

In its appeal, Wal-Mart said the Ninth Circuit's decision had contradicted earlier decisions of the
Supreme Court and other appeals courts and had wrongly relieved the plaintiffs of the burden of
proving individual injury.

``This conflict and confusion in class-action law is harmful for everyone -- employers, employees,
businesses of all types and sizes, and the civil justice system,'' said Theodore Boutrous, a lawyer
for Wal-Mart.

In its filing, Wal-Mart argued that while a class action might be appropriate for plaintiffs seeking
changes to the retailer's behavior, the status was improper for seeking monetary damages.

The company said the complaints of the seven women were not typical of the more than one million
women who have worked at Wal-Mart in the last decade. In a statement Wednesday, Wal-Mart said that
it ``has been recognized as a leader in fostering the advancement and success of women in the
workplace.''

Brad Seligman, a lawyer for the women, disputed Wal-Mart's legal analysis. ``The ruling upholding
the class in this case is well within the mainstream that courts at all levels have recognized for
decades,'' he said in an e-mail Wednesday. ``Only the size of the case is unusual, and that is a
product of Wal-Mart's size and the breadth of the discrimination we documented. There is no `too big
to be liable' exception in civil rights laws.''

The slow grind of the legal process has taken its toll on the plaintiffs.

Patricia Surgeson said she had quit Wal-Mart in frustration after being repeatedly denied a
promotion and discovering that male employees at her store were typically paid more than women. She
is now at home raising three children.

Cleo Page resigned from the Wal-Mart store in Union City, Calif., and became a teacher for disabled
students. She said she had grown angry because she was never promoted to management trainee and
because the store manager had only considered men to head the sporting goods department.

Ms.~Odle, who said Sam's Club had fired her because she kept speaking out against discrimination,
moved to Old Navy and then to A\'eropostale. At both, she said, managers threatened to fire her
after discovering that she had appeared on television criticizing Wal-Mart.

Tired of worrying about dismissal, she went into business for herself. For the last five years, she
has been selling country pecan pork chops and chicken and dumplings at Dishing It Up, her take-out
meals shop in Norman, Okla.

``This way it's better, because now no one can fire me,'' Ms.~Odle said.

Still, her customers frequently comment on her role in the lawsuit. ``You have people who say, `You
go, girl,' and you have other people saying, `Oh, you're that girl,' '' she said. (Ms.~Odle is no
longer one of the named plaintiffs; that group is limited to California residents.)

David Tovar, a Wal-Mart spokesman, denied that there was any companywide discrimination, saying that
conditions had steadily improved for female employees. He said 46 percent of Wal-Mart's assistant
store managers were women, a position that is a pipeline to higher positions.

Mr.~Tovar pointed to a company-sponsored expert study indicating that in 90 percent of its stores,
there were no statistically significant pay disparities between men and women. But the plaintiffs'
experts said they found sex discrimination in all 46 Wal-Mart regions.

Mr.~Boutrous said that even if the seven lead plaintiffs had suffered discrimination, that did not
mean there was across-the-board bias at thousands of stores nationwide. He said the women's claims
should be tried individually, or if a manager discriminated against a store's 200 women employees,
then perhaps as a 200-member class action for those women.

Joseph Sellers, a lawyer for the plaintiffs, said the case should be a class action because Wal-Mart
had and still has a common set of personnel policies at all of its stores. ``We regard them as
cookie-cutter operations that are similar to each other,'' he said.

The seven lead plaintiffs disagree on one important matter: whether Wal-Mart has improved its
policies toward women.

Betty Dukes, the woman for whom the case is named, is not convinced that conditions are any better.
She began working for Wal-Mart in Pittsburg, Calif., in 1994 and is still a greeter there. She said
she had stayed because it was hard to find another job while in the spotlight.

Ms.~Dukes originally complained that she had been repeatedly passed over for promotions and that
management had not even posted openings. Moreover, she said, women were paid less than men for the
same job. When the lawsuit was filed, she was earning \$8.44 an hour, despite nine years of service.

When the news media began covering the lawsuit and writing about her, she said, Wal-Mart grew
embarrassed and raised her pay by nearly 50 percent within a year. After 16 years at Wal-Mart,
Ms.~Dukes said she earns about \$31,000 a year. ``I'm still struggling to get by,'' she said.

Another plaintiff, Deborah Gunter, grew upset that she had not been promoted to photo lab or pet
department manager even though she said she had considerable experience with photography and pets.

She was later denied promotions to become a tire and lube manager, she said, adding that several men
she had trained were promoted over her. Her boss then cut her hours back, reducing her pay, and when
she complained about that, she was fired, she said.

Edith Arana is still fuming at Wal-Mart. During her six years there, she said, she had been
repeatedly passed over for promotions even though she often worked grueling hours to impress
management, frequently shortchanging her children.

``There are some women who are afraid to speak up,'' said Ms.~Arana, who now works at the Los
Angeles Public Library. ``Someone needs to speak up for them. I'm willing to take on the fight.''

Christine Kwapnoski sees signs that the suit has had an impact.

She complained that while working at the freezer department at a Sam's Club in Concord, Calif.,
several men she had trained were promoted over her. Soon after the suit was filed, though, Sam's
promoted her to assistant store manager -- she believes to make the company look better in court.

``The influx of women into management after the lawsuit was brought was phenomenal,'' Ms.~Kwapnoski
said.

Ms.~Odle also said opportunities for women had improved, even though the broader legal questions
remained unresolved.

``We've already won because they already had to change their policies toward women because of us,''
Ms.~Odle said.

\pagebreak
\section{Google Is Offering Phone Calls via Gmail}

\lettrine{G}{oogle}\mycalendar{Aug.'10}{26} entered a new businesses beyond Internet search on
Wednesday with a service within Gmail to make phone calls over the Web to landlines or cellphones.

The service will thrust Google into direct competition with Skype, the Internet telephone company,
and with telecommunications providers. It could also make Google a more ubiquitous part of people's
social interactions by uniting the service for phone calls with e-mail, text messages and video
chats.

``It's one place where you can get in touch with the people that you care about, and how that
happens from a network perspective is less important,'' said Charles S.~Golvin, a telecommunications
analyst at Forrester Research.

Gmail has offered voice and video chat for two years, but both parties must be at their computers.
Google said the new service would work well for people in a spot with poor cellphone reception or
for those making a quick call from their desk.

After Gmail users install a voice and video chat plug-in to their browsers, they can make a call
using their computer's microphone and speakers or a headset. Calls to numbers in the United States
and Canada will be free at least through the end of the year. International calls range from 2 cents
a minute to many countries to 98 cents a minute to call Cuba.

Skype, which was founded in 2003, lets people call phone numbers in the United States and Canada for
2.1 cents a minute or make unlimited calls for \$3 a month. For \$14 a month, Skype users can make
unlimited calls to people in 40 countries.

This month, Skype filed for an initial public offering of stock with the Securities and Exchange
Commission. Skype was acquired by eBay for \$2.6 billion in 2005 and was sold last year to an
investor group led by the private equity firm Silver Lake.

Skype has an average 124 million users a month worldwide, according to the filing, of which 8.1
million pay for the service to call mobile phones and landlines. Google does not say how many people
use Gmail, but analysts estimate that Gmail has 200 million users.

\pagebreak
\section{3 Men in Canada Charged With Terrorism}

\lettrine{A}{}\mycalendar{Aug.'10}{27} man who appeared on Canada's version of ``American Idol'' was
the third person arrested as part of an alleged plot against targets in Canada and abroad, police
said Thursday.

The two other suspects made a brief appearance in court on Thursday on charges they had plans to
make bombs and to use them.

Hiva Alizadeh, 30, and Misbahuddin Ahmed, 26, appeared in court after their arrests on Wednesday in
Ottawa. They are to appear again, by video, next Wednesday. Dr.~Khurram Syed Sher was taken into
custody in London, Ontario, on Thursday. All three are Canadian.

Sher, 28, appeared on the reality show ``Canadian Idol'' in 2008 singing a comical version of Avril
Lavigne's ``Complicated,'' complete with dance moves that include a moonwalk. He told the judges
he's from Pakistan and likes hockey, music and acting.

Police allege the men had plans and schematics to make improvised explosive devices. Police seized
50 electronic circuit boards which they say could be used as remote-control triggers for bombs. They
said one of the men was trained overseas to make explosive booby traps, but did not specify which
one.

Police say they moved in on the men to prevent them from sending money to terror groups in
Afghanistan.

``The arrests have prevented the gathering of bombs and the execution of one or many terrorist
attacks,'' RCMP Chief Supt. Serge Therriault said.

Therriault said details on the targets would be released in court.

Police alleged that they conspired with three other individuals to ``knowingly facilitate terrorist
activities'' in Canada and abroad. Police say the plot ranged from Canada to Iran, Afghanistan,
Dubai and Pakistan, but did not elaborate.

Canadian Prime Minister Stephen Harper said the arrests should remind Canadians that they are not
immune to terrorism.

``The networks that threaten us are worldwide. They exist not only in remote countries but they have
-- through globalization and through the Internet -- they have links through our country and all
through the world,'' Harper said.

Sher is a doctor in Ontario and reportedly started a new job on Aug.~3 at St.~Thomas Elgin General
Hospital.

``I'm really devastated,'' Dr.~Syed Wasty, the chief of pathology and Sher's supervisor, said of the
arrest. ``I cannot say any more.''

Sher spent time in Pakistan in 2006 as part of a relief effort after an earthquake. In 2007, he
wrote the Canadian government protesting the treatment of three Muslims at Kingston prison.

Police, who made the arrests after a yearlong investigation, said the three suspects had been
working together since February 2008.

Ahmed is an X-ray technician in Ottawa. Alizadeh studied English as an additional language and
electrical engineering technology at Red River College in Winnipeg, Manitoba.

Ahmed's lawyer, Ian Carter, said the charges are serious and his client, a husband and father, could
be put away ``for a long time.''

``He is in shock. That's all I can say,'' Carter said.

The arrests come four years after the arrest of the so-called Toronto 18, suspects in a homegrown
terror plot that involved the attempted setting off of truck bombs in front of Canada's main stock
exchange and two government buildings. The ringleaders and others have been convicted.

\pagebreak
\section{In Japan, Party Ex-Leader Will Challenge Premier}

\lettrine{A}{fter}\mycalendar{Aug.'10}{27} just three months in office, Prime Minister Naoto Kan of
Japan faces a challenge from a scandal-tainted power broker within his own party in a leadership
race that could hamper the government's response to a debilitating economic slowdown.

The power broker, Ichiro Ozawa, 68, said Thursday that he planned to run against Mr.~Kan for the
presidency of the Democratic Party on Sept.~14, a position that would ensure his appointment as
prime minister. Mr.~Ozawa, who is credited with the Democrats' rise to power last year, has been
increasingly critical of Mr.~Kan since the party lost ground in elections last month.

Mr.~Ozawa's challenge could bring upheaval to the Democrats just as Japan's recovery from a painful
recession shows signs of sputtering. Growth slowed to just 0.1 percent in the most recent quarter,
and the strength of the yen has threatened to erode earnings at Japan's exporters because it makes
their products more expensive abroad and therefore less competitive.

``Although I am unworthy, I have decided to run in the leadership election,'' Mr.~Ozawa said.

In a thinly veiled censure of Mr.~Ozawa's old-school politics, Mr.~Kan said Thursday, ``We must
first break down Japan's political structure, then construct a new one together.''

A divisive figure, Mr.~Ozawa could also bring more political uncertainty to a country that has had
five prime ministers in three years.

The Democrats ousted the Liberal Democratic Party last August after a string of unpopular prime
ministers finally compelled voters to reject the L.D.P., a party that had governed Japan for most of
the past half-century. Yukio Hatoyama of the Democrats took office as prime minister in September,
only to resign nine months later over a broken campaign promise to move a United States Marine base
from Okinawa.

His successor, Mr.~Kan, first appeared to be winning back popular support for the party, but he has
seen his ratings nosedive after suggesting that Japan may raise its consumption tax to tackle its
mounting public debt.

Mr.~Ozawa, the leader of the Democrats from 2006 to 2009, was forced to step down over a political
financing scandal just months before his party's rise to power. The scandal still clouds his future.
He remains under criminal investigation, and a citizens' judicial panel is expected to rule on
whether he will be indicted after the Sept.~14 vote.

\pagebreak
\section{H.P. Raises Its 3Par Bid, Leapfrogging Dell Again}

\lettrine{I}{n}\mycalendar{Aug.'10}{27} a rapidly escalating bidding war, Hewlett-Packard raised its
offer for the data storage company 3Par to \$1.8 billion after the close of regular trading on
Thursday, topping a \$1.6 billion bid by Dell that came just hours earlier.

Shares of 3Par, considered a crucial strategic asset by both companies as they look to shore up
their corporate solutions businesses, jumped 6 percent in after-hours trading. They had fallen 2.7
percent during the regular session on disappointment that Dell's revised offer, announced on
Thursday morning, came in just a sliver above H.P.'s first bid.

Hewlett-Packard, which analysts say has the upper hand because of its much larger revenue and cash
position, is now offering \$27 a share in cash, trumping Dell's latest of \$24.30 a share.

Dell's new offer was just 30 cents a share above H.P.'s opening bid. Some analysts had said that the
meager difference suggested Dell might give up if H.P. were to retaliate with a much higher price.

H.P. reported \$115 billion of annual revenue last year compared with \$53 billion at Dell.

``Even though Dell has the balance sheet to step up the offer, they're probably reaching the upper
limits of what they can offer,'' said Ashok Kumar, an analyst at Rodman \& Renshaw. ``At the end of
the day, Hewlett-Packard is in a better position to close the deal.''

On Thursday morning, 3Par accepted Dell's revised offer and said that the two companies had raised
their termination fee to \$72 million from \$53.5 million.

The pursuit of 3Par comes as Hewlett-Packard and Dell, as well as other large technology vendors
like I.B.M. and Cisco Systems, are trying to expand into new products and services.

3Par specializes in high-end data storage, a crucial part of ``cloud computing'' -- an increasingly
popular technology that enables computer users to access data and software over the Internet,
allowing companies to save costs.

The company competes with EMC as well as I.B.M. and other data storage companies, and 3Par's
expertise in the high end has made it attractive.

August has been an unusually active month for deals, after Intel's \$7.7 billion bid for McAfee, the
maker of security software.

On Thursday alone, Cisco announced it would buy the online video software company ExtendMedia, while
Hewlett-Packard said it would buy Stratavia, a software company involved in cloud computing.

Many analysts say the bidding war has driven up valuations to unreasonable heights. 3Par now trades
at around 135 times forward earnings, according to the research firm StarMine's SmartEstimates
forecast, compared with less than 15 times for EMC and other competitors.

H.P.'s latest, sharply higher offer appeared to dash speculation that it might decide to back off
given increasing concerns about the global economy. The lack of a chief executive after Mark
V.~Hurd's sudden resignation may also be a factor.

A survey by Reuters of nine fund managers and analysts found that most expected another bid or two,
and a final price of about \$29 a share. Dell shares closed at \$11.75, but dipped to \$11.70 after
H.P.'s revised offer was announced. H.P. shares were little changed at \$38.22.

\pagebreak
\section{Parade of Super Cars Inspires Mixed Feelings in London}

\lettrine{W}{ith}\mycalendar{Aug.'10}{27} his canary yellow Ferrari at rest in the forecourt of one
of the most expensive hotels in London's upscale West End, its eight-cylinder, race-bred engine
burbling, a young Arab man who gave his name as Khalefa spoke with whimsical regret about the array
of even faster, more expensive super cars parked nearby that belonged to other young men like
himself from the Persian Gulf oil states.

``Me, I only have the Ferrari,'' he said. ``I am a poor man.''

With Britain still struggling to climb out of recession, and the new governing coalition embarked on
a historic campaign of budget austerity, wealthy young men like Khalefa -- who declined to give his
full name, or his nationality -- encounter a conflicted reception when they flee the conservative
social mores and the 130-degree heat of the Middle East in high summer to enjoy the cool breezes of
millionaires' row districts of London like Belgravia, Mayfair and Knightsbridge.

On one hand, the young men and their families get an eager welcome in the hotels, department stores,
and jewelry and fashion boutiques that rely heavily on the visitors' wealth in otherwise lean
economic times. Many local people, too, at least in London, enjoy the excitement and panache that
come with a parade of exotic cars with Arabic license plates from gulf oil states like Saudi Arabia,
Kuwait, Qatar and, especially, the United Arab Emirates.

But a display of vehicles that rival anything to be seen at the Pebble Beach Concours d'Él\'egance
or Monaco's Casino Square at Grand Prix time has its downside, too, with residents complaining that
some visitors have exploited the easygoing atmosphere of the West End in summer to flout parking
regulations and stage what have amounted to races of their own in the deserted streets after
midnight, in what one of Britain's tabloids called the Knightsbridge Grand Prix.

Arab visitors to Britain -- more than five million last year, according to official figures -- have
increased since the Sept.~11 attacks in the United States, which some Arab commentators say made
Arab visitors feel less comfortable in New York.

``Arab people are scared, in a way, of becoming figures in an exhibition, of being associated in
some way with what happened in those attacks,'' said Abdel Bari Atwan, editor in chief of Al Quds Al
Arabi, a London-based Arabic-language newspaper distributed widely in the Middle East. ``They feel
the people of London are more tolerant.''

But this year, the display of wealth has been more conspicuous than ever, especially the cars, and
it has drawn mixed reviews. The cars are flown and shipped in from the Middle East for a few weeks,
then returned home as the end of the European summer approaches, with costs many times what the
owners might pay if they leased luxury cars in London. One carrier, the Gulf Agency Company, said it
had flown 15 cars to London this summer, at an average cost of about \$15,000 one way.

Enthusiasts gather wherever the Ferraris, Lamborghinis, Mercedes-Benz SLRs and Rolls-Royces are
parked. Their gulf origins are evident from their license plates; that discretion is not the favored
mode is suggested by the customized paint jobs, of silver plate and gold lam\'e and black matte,
and, for Rolls-Royces, bright red or yellow or other arresting color schemes barely imagined when
Charles Rolls and Henry Royce began selling their cars to Britain's aristocratic elite in 1906.

These cars have included several examples of the Mercedes SLR McLaren; a number of French-made
Bugatti Veyrons, with 1,000-horsepower engines said to be capable of nearly 270 miles per hour, and
a \$1.3 million sticker price; and the Italian-made Pagani Zonda Cinque, base price \$1 million,
that the manufacturers say is so rare that only half a dozen have been sold.

But perhaps the rarest of all is the Swedish-built Koenigsegg CCXR, retailing at \$1.9 million and
said by its manufacturers to be the world's first ``green'' super car, capable of achieving speeds
of 250 miles an hour on biofuel. A powder-blue Koenigsegg became an inadvertent icon for the super
car invasion when, together with a similarly painted Lamborghini Murcielago, it was ``clamped'' by
parking wardens -- in American terms, immobilized by having a steel ``boot'' locked to one of its
front wheels -- outside Harrods department store last month.

The cars' owners, members of the royal family of Qatar, bought the store in May for \$2.2 billion
from Mohamed al-Fayed, the 77-year-old Egyptian-born entrepreneur whose son, Dodi, died in a Paris
car crash in 1997 with Diana, Princess of Wales. Perhaps they believed that a parking space should
have come along with the purchase.

For many gulf visitors, clamping has become a hazard of the summer, only partly because of what the
wardens have described as their carelessness about parking restrictions. Traffic cameras and
handheld computers used by the traffic police cannot handle Arabic script, so a clamp sometimes
becomes the only way to enforce a parking penalty or to assure that violators are brought to
justice.

With the arrival of the Muslim holy month of Ramadan, many of the super cars began disappearing last
week. But the end-of-season mood was punctured by a spectacular crash last month in the heart of
Knightsbridge.

Two wealthy young men from Abu Dhabi were arrested after a Lamborghini spun out of control on a dash
around a square shortly before 2 a.m., wrecking four other expensive vehicles, one of them a BMW
that was flipped over by the impact. Witnesses were quoted in The Evening Standard as saying that
one of the men called out as he walked away from the impact, saying, ``It's all right, we'll pay for
the damage.''

It is not only the British who have cringed at the flamboyant excesses of some of the visitors.
Mr.~Atwan, the newspaper editor, described the ``show off'' behavior of the young super car owners
as ``very annoying and very provoking, and not only to the British, but other Arabs.''

Nevertheless, he said, it would be a mistake to condemn all the young Arab men for the actions of a
few. He said he was sitting at a Knightsbridge traffic light in his middle-market family sedan
recently when a yellow Ferrari pulled up alongside him with its engine growling.

``I rolled down my window to say something to him before the light changed, expecting him to rev his
engine and zoom away,'' Mr.~Atwan said. ``But then I decided to be polite, and said, 'What a nice
car! Good luck!' and do you know what? He pulled away very quietly, waving his hand to say goodbye.
He couldn't have been more polite.''

\pagebreak
\section{Charges Settled Over Fake Reviews on iTunes}

\lettrine{D}{iscerning}\mycalendar{Aug.'10}{27} Internet users know that glowing online reviews of
things like books or restaurants cannot always be trusted. But federal regulators are serving notice
that if you stand to gain financially from the review you are writing, you should be upfront about
it.

The Federal Trade Commission said on Thursday that a California marketing company had settled
charges that it engaged in deceptive advertising by having its employees write and post positive
reviews of clients' games in the Apple iTunes Store, without disclosing that they were being paid to
do so.

The charges were the first to be brought under a new set of guidelines for Internet endorsements
that the agency introduced last year. The guidelines have often been described as rules for
bloggers, but they also cover anyone writing reviews on Web sites or promoting products through
Facebook or Twitter.

They are meant to impose on the Internet the same kind of truth-in-advertising principles that have
long existed offline.

Under the settlement, Reverb Communications and one of its executives, Tracie Snitker, agreed to
remove all of the iTunes reviews that appeared to be written by ordinary people but were actually by
employees of the company, which is based in Twain Harte, Calif.

The settlement also bars Reverb and Ms.~Snitker from making similar endorsements of any product or
service without disclosing any relevant connections. The settlement did not involve any monetary
penalties.

``We hope that this case will show advertisers that they have to be transparent in their practices
and help guide other ad agencies,'' said Stacey Ferguson, a lawyer in the advertising practices
division of the trade commission's Bureau of Consumer Protection.

Ms.~Snitker declined to be interviewed, but in a statement she said that in discussions with the
trade commission, ``it became apparent that we would never agree on the facts of the situation.''

``Rather than continuing to spend time and money arguing, and laying off employees to fight what we
believed was a frivolous matter, we settled this case and ended the discussion,'' she said.
Ms.~Snitker said that the settlement did not involve any admission of lawbreaking.

When the guidelines were announced, many bloggers and users of services like Twitter complained of
government overreach, and worried that they would have to disclose even tenuous connections with
companies or services they wrote about.

But Jonathan Zittrain, a professor at Harvard Law School and co-founder of the Berkman Center for
Internet and Society, said the commission's first enforcement action under the guidelines should be
seen as good news by those who were concerned.

``This case sort of shows that what they have in mind is not the individual blogger or Twitterer,
but rather a professional endorser,'' Professor Zittrain said.

The action could be useful to public relations companies that want to resist requests from clients
that they play dirty, he said.

``When a client says 'Where are my good reviews? I am paying for them,' you can say, 'We can't do it
because it is illegal,' '' Professor Zittrain said.

According to the commission's complaint, Reverb employees, including Ms.~Snitker, posted positive
reviews about clients' games from November 2008 to May 2009. The reviews were posted under account
names that would give readers the impression that they had been placed by ordinary consumers, the
complaint says.

The reviews typically gave the games four or five stars and included comments like ``Amazing new
game'' and ``One of the best apps just got better.''

The complaint does not identify the game developers whose work was reviewed. Reverb's Web site lists
more than 60 current and former clients, including Digital Leisure, Harmonix and MTV Games. The
complaint said Reverb was paid a commission of a portion of sales by its game developer clients.

Given that fake reviews are widely understood to be common in the iTunes Store and on many Web
sites, it was not clear why the trade commission had singled out Reverb. But the blog MobileCrunch
reported last August that it had obtained a company document in which Reverb said it had hired ``a
small team of interns'' whose tasks included ``writing influential game reviews.''

Eric Goldman, former general counsel of Epinions.com, which reviews consumer products, said fake
reviews were ``a pervasive problem on the Internet.''

``It is a problem that every review site has to grapple with,'' said Mr.~Goldman, who is now a law
professor at Santa Clara University and director of the school's High Tech Law Center. ``Many sites
don't have rigorous policing mechanisms, so it is very hard to verify the trustworthiness of
reviews.''

While the case against Reverb is the first brought under the commission's new guidelines, it is not
the first of its kind. Last year, Attorney General Andrew M.~Cuomo announced that the State of New
York had reached a \$300,000 settlement with Lifestyle Lift, a cosmetic surgery outfit, over faked
reviews of its products on the Internet.

In a press release, the attorney general said the action was ``a strike against the growing practice
of 'astroturfing,' in which employees pose as independent consumers to post positive reviews and
commentary to Web sites and Internet message boards about their own company.''

\pagebreak
\section{Moscow's New Dress Code: Less}

\lettrine{O}{n}\mycalendar{Aug.'10}{27} any given day, the clothing seen on that once-sacred spot
where somber citizens used to gather in long lines to pay homage to the founder of the Soviet state
offers a whirlwind tour of Russia's fashion preferences.

They can range from a woman in short shorts and stiletto heels with small children in tow, to a
teenage girl in a micro-mini skirt or a maxidress, to a consciously stylish young man in a
sleeveless shirt with a flower-framed portrait of Vladimir Putin, black cargo pants, flip-flops and
straw fedora.

Muscovites and tourists now wander past Lenin's Mausoleum in fashions that until recently might have
been considered more appropriate for a beach resort. And around the city, women have been seen
sporting bikini tops and men are going shirtless.

``Everything is changing,'' said Natalya Osadchuk, 37, a schoolteacher from the Moscow region who
wore white shorts and wedge heels as she walked past the mausoleum on Red Square with her family.
``Now there are no strict rules in fashion. There's a much simpler attitude. Of course, it's the
heat, too.''

This summer, Russians have battled temperatures of up to 40 degrees Celsius, or 104 degrees
Fahrenheit, along with smoke from nearby fires. But even before the worst of the summer's heat,
there was a move toward informal fashion, best seen in the trend for shorts, which used to be
largely taboo beyond a certain very young age.

Tatyana Makulova, 74, a retired theater director, said the only thing keeping her from wearing
shorts on the city streets these days is the fact that her legs are not yet tan enough.

``It used to be considered indecent to leave the house in the summer without stockings,''
Mrs.~Makulova said. ``My mother never in her life went out without stockings.''

She wore white shorts with no shame at a friend's dacha near Moscow, and her 81-year-old husband,
Nikita, walked around shirtless and in shorts.

Mr.~Makulova said that when he and his friends used to camp on the banks of the Volga River near his
hometown of Saratov, they would take off their shirts and change into shorts or swimming trunks only
once they neared the water and there were no strangers around.

``The Soviet attitude towards propriety was completely different,'' said Dmitri Chernetsky, a former
documentary filmmaker.

Indeed, while this summer's heat is a factor, the embrace of informal dress highlights deeper
changes in society.

The ingrained formality of the Soviet era, and the over-the-top glamour of the oil boom years, have
given way to a fashion free-for-all. Russians are daring to wear shorts in former Soviet shrines and
even in Russian Orthodox churches, where for years women were expected to wear headscarves and long
skirts.

On Tverskaya Street, Moscow's main thoroughfare leading from Red Square, informality was the norm in
August near Pushkin Square, a popular meeting spot near a monument to Russia's most revered poet,
who is dressed in a ruffled shirt and topcoat.

Yulia Lunyova, a 29-year-old strawberry-blonde cosmetology student, was dressed in Adidas shorts,
Adidas sneakers and a tank top.

``Before I would have been shy to come to Moscow like this,'' said Ms.~Lunyova, who is originally
from Chelyabinsk in the Ural Mountains region.

``Now I go to class like this,'' she said. ``No one is looking at who is wearing what.''

Sergei Subachyov, 22, was wearing light track pants, but only because all of his shorts were in the
wash, he said.

``We're children of perestroika,'' he said, referring to the era of openness as Communism was
collapsing.

This is not to say, said Djurdja Bartlett, an expert in socialist and post-socialist fashion at the
London College of Fashion, that the skin being shown on Russia's streets today is cloaked in
ideology.

``I should think that the phenomenon of the people walking half-naked in the center of Moscow marks
the definitive end of the concept of regimented bodies, organized and managed by the state,
following the principles of socialist puritanical ideology,'' said Ms.~Bartlett, whose book
``FashionEast: The Spectre That Haunted Socialism,'' will be published in October.

The changes have exposed, however, something of a generation gap. Yuri Alimov, a retired defense
industry worker dressed in black shorts, said he started to feel it was acceptable to wear shorts in
Moscow about five to seven years ago. In his day, he said, only children wore shorts and it was part
of the Pioneer camp uniform of Soviet youth.

He tells how his 93-year-old mother criticizes him when she sees him in shorts. `` `How old are
you?' she says. I say, `I'm 66.' `Then why are you in shorts?' ''

Still, the trend is widespread. TsUM, a Moscow luxury department store, told Vedomosti, a Russian
business daily, in July that it had sold out its supply of Prada, Gucci and Dolce \& Gabbana shorts.

But the newfound prevalence of shorts has also highlighted the dichotomy in dress. At a recent
Friday-night showing of the film ``Inception'' at a multiplex on Novy Arbat, a Moscow avenue with
the neon glow of Las Vegas, some young women were in shorts, while others were dressed as if for a
Hollywood premier, in sequined dresses, impossibly high heels and bouffant hairdos.

Alexandre Vassiliev, Russia's leading fashion historian and host of the television program called
``Modny Prigovor,'' or ``Fashion Verdict,'' a Russian variation on ``What Not To Wear,'' says the
extremes in women's fashion, from short shorts to overdressing out of context, are the consequences
of what he said was Russia's man shortage and a ``wild sexual revolution'' among women here.

``Women are on a constant hunt,'' he said, adding that if women already have a man, ``they're
convinced if they aren't pin-up girls their husbands will run away from them and they won't snap up
a new one.''

Nina Khrushcheva, the great-granddaughter of Nikita Khrushchev and now an associate professor at the
New School in New York, said the sense that Russians are all over the place in fashion reflects the
disordered state of Russian souls in general today.

``It's sort of like a palace in Uspenskoye and a rusty car right next to it in a ditch, and no one
even notices,'' she said, referring to a suburb where elite Russians live.

``It's the same with fashion,'' she said. ``It's more the idea of rich'' than actually being rich,
``the idea of fashion, than fashion.''

Other factors have had an impact on Russian dress habits, including a rebellion against religious
dictates that women must be covered, Mr.~Vassiliev said.

Patriarch Kirill I of the Russian Orthodox Church, who has made outreach to young people and secular
society a priority, has spoken out against gloomy women's clothes and called on women to pay more
attention to their appearance in church and to present a more joyous and fashionable image of
Orthodoxy.

The chairman of the patriarchate's department on relations between church and society, the
Rev.~Vsevolod Chaplin, said worshippers should dress respectfully.

But he also made light of the headscarf obsession in Russian Orthodox churches and shared a favorite
anecdote.

``Two young women in bikinis leave the beach and decide to go to church, arguing whether it's
allowed to enter like that or not,'' he said. ``One says, `You know it's probably not really
appropriate to go to a men's monastery like this.' They go in. Someone hisses at them. And one girl
says to the other: 'See, I told you we should have worn headscarves!'''

\pagebreak
\section{Kim Jong-il Reportedly Absent During Carter Visit}

\lettrine{A}{}\mycalendar{Aug.'10}{27} special train believed to be carrying the North Korean
leader, Kim Jong-il, entered China around midnight on Wednesday, South Korean officials said,
setting off speculation over what might have compelled him to travel to his isolated government's
closest ally while former President Jimmy Carter was visiting at the North's invitation.

After watching Mr.~Kim's movements for the past few days, the South Korean authorities said his
train had crossed the border with China, traveling from the North Korean town of Manpo to Jian in
China, according to an official at the presidential Blue House in Seoul.

Two South Korean intelligence sources who, like the presidential aide, spoke on condition of
anonymity because of the delicacy of the matter, said Mr.~Kim might be taking his son with him to
introduce him formally to Chinese leaders. South Korean news outlets raised the same possibility.

Mr.~Kim is grooming his youngest son Kim Jong-un, as successor, according to South Korean officials.
North Korea is to convene a congress of its ruling Workers' Party early next month, where Mr.~Kim is
expected to rally popular support for his succession plans.

If confirmed, this would be Mr.~Kim's sixth trip to China, his impoverished country's largest
trading partner and aid provider. His last trip was in May, when he met President Hu Jintao during a
five-day visit. North Korea and China usually do not confirm a trip by Mr.~Kim until it is over.

Reports of Mr.~Kim's travels came a day after Mr.~Carter arrived in Pyongyang, the North Korean
capital. It would be highly unusual for Mr.~Kim to leave while an important guest was visiting.

Officially, Mr.~Carter is visiting Pyongyang on a private humanitarian mission to win the release of
Aijalon Mahli Gomes, who was sentenced in April to eight years of hard labor in a North Korean
prison and fined some \$700,000 for entering the country illegally. There is speculation that North
Korea might also try to use Mr.~Carter as a conduit to ease tensions with the United States.

There was no indication on Thursday from North Korea on whether Mr.~Carter had met Mr.~Kim or
secured Mr.~Gomes's release.

News of the possible trip by Mr.~Kim led to rampant speculation in South Korea. Possible motives
cited by analysts in Seoul included the North's need for Chinese aid because of flooding and the
possibility of a decline in Mr.~Kim's health, which might have forced aides to take him to China for
treatment. Many intelligence officials believe Mr.~Kim had a stroke in 2008. Around the time that
Mr.~Kim's train crossed the border, North Korean media reported that China would provide emergency
flood relief.

With North Korea's relations with the South and the United States at a low point, ``China is the
only one Kim Jong-il can go to to seek aid,'' said Kim Keun-sik, an analyst at the University of
North Korean Studies in Seoul. ``He badly needs aid before the party meeting to make it a national
festival, as it is meant to be.''

Even so, leaving North Korea without meeting Mr.~Carter would be a notable breach of diplomatic
etiquette, the analyst said. ``A possible political message of this is that North Korea gives its
priority to China over the United States,'' he said.

China's Foreign Ministry had no comment on the visit. Two teachers told The Associated Press that
Mr.~Kim spent 20 minutes Thursday at Yuwen Middle School in Jilin, in the northeast, where his
father, Kim Il-sung, attended classes from 1927 to 1930.

A secretary who answered the phone Thursday afternoon said ``an important person'' visited but said
she did not know who it was.

Shi Yinhong, a professor of international relations at People's University in Beijing, said a visit
by Kim Jong-il could offer a diplomatic respite from the pressure being exerted by South Korea and
the United States.

John Delury, senior fellow of the Center on U.S.-China Relations of the Asia Society in New York,
said Mr.~Kim's reported trip begged for explanation.

``Some will say the trip is related to North Korean succession, but why seek Beijing's blessing for
transferring power to his son in a last-minute trip before the momentous Korean Workers' Party
meeting expected in early September, rather than do so more discreetly during his previous trip in
May?'' he said.

Mr.~Delury also was skeptical that Mr.~Kim would have gone to China in person to seek a breakthrough
in the stalled six-nation talks on ending his country's nuclear weapons programs.

``That's what diplomats are for,'' he said.

\pagebreak
\section{New Kindle Leaves Rivals Farther Back}

\lettrine{T}{oo}\mycalendar{Aug.'10}{27} bad there's not a reality TV show called ``America's Most
Freaked-Out Tech-Company Meetings,'' where you watch classic panicked board meetings. For example,
when the Apple employee left an iPhone 4 prototype in a bar. Or when Intel learned that its Pentium
chip contained a math error. Or when Microsoft was caught bribing bloggers with \$2,500 laptops to
promote Windows Vista.

One of the most exciting episodes, though, could have been shot at Amazon the day Apple announced
the iPad.

Amazon makes the popular Kindle e-book reader. For a while, it was pretty much the only game in
e-book town. But the iPad has a touch screen, color, prettier software, audio and video playback,
100,000 apps -- and at the time, it didn't cost much more than the Kindle. For the Kindle, with its
six-inch monochrome nontouch screen, the iPad was your basic (full-color) nightmare.

This week, Amazon unveiled what everyone (except Amazon) is calling the Kindle 3. You might call it
Amazon's iPad response. The Kindle 3 is ingeniously designed to be everything the iPad will never
be: small, light and inexpensive.

The smallness comes in the form of a 21 percent reduction in the dimensions from the previous
Kindle. The new one measures 7.5 by 4.8 by 0.3 inches, yet the screen has the same six-inch diagonal
measurements as always. Amazon's designers did what they should have done a long time ago: they
shaved away a lot of that empty beige (or now dark gray) plastic margin.

Now, the Kindle is almost ridiculously lightweight; at 8.5 ounces, it's a third the weight of the
iPad. That's a big deal for a machine that you want to hold in your hands for hours.

Then there is the \$140 price. That's for the model with Wi-Fi -- a feature new to the Kindle that
plays catch-up to the Barnes \& Noble Nook. A Kindle model that can also get online using the
cellular network, as earlier models do, costs \$50 more. But the main thing you do with the wireless
feature is download new books, so Wi-Fi is probably plenty for most people.

That \$140 is quite a tumble from the Kindle's original \$400 price, and a tiny sliver of what you
would pay for an iPad (\$500 and way, way up).

Yes, of course, it's a little silly to compare the Kindle with the iPad, a full-blown computer with
infinitely greater powers. Although it's worth pointing out, just in case you were indeed
considering the iPad primarily for its e-book features, that the Kindle's catalog of 630,000 current
books is 10 times the size of Apple's.

No, the Kindle's real competition is the gaggle of extremely similar, rival e-book readers, all of
which use the same E Ink screen technology.

E Ink is satisfying to read but deeply flawed technology for e-book screens. It works by applying an
electrical charge to millions of tiny black particles, causing them to freeze in a pattern of
letters or grayscale images. The result really looks like ink on paper, because the black stuff is
so close to the surface.

E Ink is great for battery life, too, since only turning pages uses power; otherwise, the image
could sit on the screen forever without requiring any additional juice. (Amazon says that on the new
Kindle, if you turn off the wireless features, you can read for a month on a single charge.)

But E Ink has plenty of drawbacks, too. It's slow to change the page image, for example. The new
Kindle reduces the page-turn wait to well under a second. It's the fastest page-turner among
e-readers, leaving its rivals in the dust (especially the Nook, which, despite five software
upgrades since its debut, still lags). But the page turn moment still features a bizarre,
black-white-black flashing sequence -- a nonnegotiable characteristic of E Ink.

E Ink's speed problems mean that it can never display video, either. And, of course, it can't
display color. Last month, Jeff Bezos, Amazon's chief executive, responded to this point this way:
``You are not going to improve Hemingway by adding video snippets.'' Yes, but color and video may
well improve a new era of livelier e-books.

Still, Amazon has clearly put a lot of time into refining the new Kindle's E Ink screen. The
background gray is a few shades lighter than on any other reader, producing much better contrast
behind the black text.

In the world of copy-protected e-books, choosing a reader is a particularly momentous decision.
You're not just buying a portable reader. You're also committing to a particular online e-book
store, since in general, each company's e-books don't work on other companies' readers. (The one
exception: Sony and the Nook use the same copy-protection scheme.) Even on the new Kindle, you can't
read nonprotected books in the popular ePub format, as you can on its rivals.

(However, Amazon and Barnes \& Noble each offer excellent reader programs for Mac, Windows, iPhone,
iPad and Android; in other words, you don't actually have to buy a Kindle or a Nook to read those
companies' e-books. Buy a book once, read it on all your gadgets. Kindle books even wirelessly sync
up, so each gadget remembers where you stopped -- a feature that's still on the Nook's to-do list.)

Fortunately, the online stores are all pretty good (except Apple's, whose book selection is still
puny). The pricing seems to have evened out, too; in general, Amazon, Barnes \& Noble and Sony have
exactly the same prices for New York Times best sellers. Sadly, lots of them are now \$13, up from
the flat \$10 that Amazon used to charge for all best sellers.

Those prices seem high. The fact that e-books involve no printing, binding, shipping, distributing
or taking back and shredding unsold copies ought to save you something. And it's outrageous that
that you can't sell or even give away an e-book when you're finished with it. You paid for it; why
shouldn't you be allowed to pass it on? (End of rant.)

The new Kindle's nonremovable storage now holds twice as many books: 3,500 of them, which should
just about cover your next flight delay. The tiny joystick has been replaced by cellphone-like
four-way control buttons, and the page-turn Forward and Back buttons, which flank both edges, are
silent now, for the benefit of sleeping spouses. And the new Kindle handles PDF documents much
better now; you can even add notes to them and magnify them.

Of course, the Kindle's rivals have their own attractive features. The Nook, for example, has a
balky color touch screen beneath the E Ink screen, which you use for navigation. You can read any
Nook book at no charge, one hour a day, when you're in a Barnes \& Noble store. You can even
``lend'' a book to a friend -- although held to a two-week maximum, one time a title and only on
books whose publishers have permitted this feature.

Sony Readers have touch screens; some models even have built-in illumination screens. (If you
believe the rumors, new Reader models are on the way.)

Really, though, what makes the Kindle so successful isn't what Amazon added to it; it's what Amazon
subtracted: size, weight and price. Nook's two-screen setup makes it fussy and complicated. Sony's
additional screen layers make the E Ink less sharp.

In the meantime, certain facts are unassailable: that the new Kindle offers the best E Ink screen,
the fastest page turns, the smallest, lightest, thinnest body and the lowest price tag of any
e-reader. It's also the most refined and comfortable.

No doubt about it -- the next episodes of ``America's Most Freaked-Out Tech-Company Board Meetings''
won't be filmed at Amazon. They'll be set at Amazon's rivals.

\pagebreak
\section{Foreign Language Courses, Brushing Up or Immersion}

\lettrine{T}{hey}\mycalendar{Aug.'10}{27} may be preparing for a vacation in Europe, trying to
communicate with colleagues abroad or immigrant clients at home or unlocking the skills, learned in
college, that have retreated to an inaccessible part of the brain. For those aiming to learn a
foreign language, continuing education courses can lead people toward fluency -- or at least help
them get by.

These days, online programs and CDs like Rosetta Stone and Pimsleur are grabbing the interest of
people attracted by their convenience and relatively low cost. But more schools are offering their
own online-only language courses as part of extension programs.

At the University of California, Los Angeles, traditional, three-month language classes cost \$480,
and online classes cost \$550. The online courses include video lectures, readings, exercises and
assignments, which the instructors can correct and return to the student via e-mail. Students can
practice with one another via chatrooms, and instructors and students can also talk on the phone to
work on pronunciation, said Krista K.~Loretto, program manager for U.C.L.A. Extension.

The biggest weakness of the online courses is the conversational element, Ms.~Loretto said, although
online students may soon be able to have real-time conversations thanks to technological advances.

The school also started offering combination online and classroom classes, which are especially
helpful for those who have trouble making time for a class or live a long way from U.C.L.A.,
Ms.~Loretto said.

Rosetta Stone, too, has gotten in on the classroom act. It does not consider itself a competitor,
but rather a supplement to traditional language classes, said Cathy Quenzer, the company's education
director. It provides online courses for college instructors who want to augment their classroom
lessons, she said. Students can learn through the program's image-based format at their own pace at
home, ``and then come together to share and practice in the classroom,'' she said.

There is no substitute for the traditional language class, with its emphasis on conversation and
human interaction, said Florence Leclerc-Dickler, chairwoman of the foreign language department at
the New School in New York and an assistant French professor.

Online-only courses are ``good for people who are extremely self-disciplined,'' Ms.~Leclerc-Dickler
said, comparing them to having a treadmill at home, whereas attending a class is like going to a
gym.

The New School offers continuing education courses in 17 languages, with placement exams available
for those not sure how far their rusty college skills will take them. French is by far the most
popular language, which may reflect its cultural appeal as well as France's popularity as a travel
destination, Ms.~Leclerc-Dickler said.

Spanish is second in popularity, with Arabic, German, Italian and Portuguese coming next in roughly
similar numbers, she said. While the Tibetan course is offered less frequently, the one to be
offered this fall is full, she said, as was the class in Nepali last spring.

Many classes meet once a week for an hour and 50 minutes and last 13 weeks, at a cost of \$590. But
Ms.~Leclerc-Dickler recognizes that it can be hard for busy professionals to commit to a set time
every week. That is why she also organizes weekend language immersion courses. These start on Friday
evening and run through Sunday for a total of 14 hours, at a cost of \$350.

Ellen Golub, a mortgage broker in Manhattan, took the weekend French class last spring a few weeks
before her vacation in Paris. She called it ``a crash course where you learn the basics.''

Though she did not come near to mastering French, she said the class, with its heavy emphasis on
conversation, helped her feel more comfortable doing things like ordering food and navigating the
M\'etro in Paris.

Professionals often take foreign language classes for personal reasons and enjoyment,
Ms.~Leclerc-Dickler said. But the needs of a global economy are also causing more people to learn
languages for work-related purposes, Ms.~Loretto said.

Being able to make a presentation in a foreign language, whether in person or through a
teleconference, can give an English-speaking employee a serious edge, she said. Foreign languages
are also very useful for workers who interact with immigrants, she said.

Ms.~Loretto said Spanish had long been the most popular language at U.C.L.A. Extension, but she said
that demand for Mandarin had been growing every year, ``and I wouldn't be surprised if it ends up
being neck and neck with Spanish in popularity.''

Extension offerings often reflect the needs of the surrounding community. Los Angeles has a large
Korean population, and court officials have found Korean classes valuable in helping them to
communicate with those who pass through the court system, Ms.~Loretto said.

Valentina Zaitseva, who teaches at the University of Washington, has had among her students medical
professionals who treat Seattle's large Russian community.

In addition to traditional classes, she teaches a summer course that packs a year's worth of Russian
language study into two intensive months. Now she is teaching a summer interdisciplinary class in
Sochi, site of the 2014 Winter Olympics, in partnership with Sochi State University.

Although many of her summer intensive students take her class to obtain credits toward a degree,
some are professionals who may want to develop and maintain business contacts in Russia, she says --
forming a contrast to others who mainly want to read authors like Dostoyevsky in the original
Russian.


% \end{multicols}

% \clearpage
% \renewcommand\listfigurename{\textit{Table of Figures}}
% {\footnotesize\textit{\listoffigures}}

\end{document}
