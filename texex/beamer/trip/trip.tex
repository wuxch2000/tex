\documentclass[hyperref={bookmarks=ture}xcolor=pdflatex,svgnames,table,compress]{beamer}

\usepackage{config}

\title{08年一次徒步的回忆\\ \myrule 公司摄影论坛一帖}
\author{}
\date{}

\begin{document}
\begin{frame}
\titlepage
\end{frame}

\section{前言,第一天}
\subsection{}
\begin{frame}
\begin{ztebox}
  很久没摸相机了,某天翻出08贡嘎穿越的照片,回忆起来仍觉有趣,PP很烂,文字水平也很烂,正犹豫要不要发
  上来,从一个同事口中得知,另一个同事因为看了偶云南哈巴的照片,去年十一去云南旅行了,更坚定了偶发贴
  的决心。此贴纯粹是从记事的角度分享那次穿越的历程,希望能引起同样喜欢户外或旅行的筒子们的共鸣,没有
  别的意思。
\end{ztebox}
\end{frame}

\subsection{}
\begin{frame}
\begin{ztebox}
  第一次听说贡嘎线路,是07年西藏回来,我抱怨坐车的时间太长,希望能找一条可以徒步走走的线路,朋友推荐
  我:为什么不试试贡嘎徒步?于是开始关注、了解,直到08年十一,看到网上有人召集,毫不犹豫报名了,开始
  了这次愉悦、难忘而又惊心动魄的贡嘎之旅。
\end{ztebox}
\end{frame}

\subsection{}
\begin{frame}
  \photoportrait{day1_image001.jpg}{出发前,花了两天时间才把包打好,这是装得最紧凑的一次,70+10的包被塞得
    满满的,红色的OSPREY“魔爪”是软背负,塞在大包外的隔层刚刚好,徒步的时候作为随身包。}
\end{frame}


\subsection{}
\begin{frame}
\photolandscape{day1_image002.jpg}{在成都入住交通饭店,距离新南门车站很近,三人间120,四人间也是120,卫生
  间虽是公用,但很干净,热水也很大,值得推荐。}
\end{frame}

\subsection{}
\begin{frame}
\begin{center}
\ztecalendar{Day}{1}
\end{center}
\end{frame}

\subsection{}
\begin{frame}
  \photolandscape{day1_image003.jpg}{前往康定的路上,因为修路实行单向行车,大家下来闲聊}
\end{frame}

\subsection{}
\begin{frame}
  \photolandscape{day1_image004.jpg}{快到二郎山隧道的时候,已接近中午,下车吃饭,正宗的四川烧菜哦}
\end{frame}

\subsection{}
\begin{frame}
  \photolandscape{day1_image005.jpg}{下午到了康定,等向导老大--多吉派车来接我们。康定是去很多地方的中转
    站,因此路上到处可以看到穿得花花绿绿的驴友。因为十一是旺季,不少做旅游的藏族司机跑到大街上拉客。
    我抽空找了家户外店买了个气罐,竟然要35元(深圳的普通扁罐气一般是18、19元),老板说是高山气,但没
    看到有丙烷的成分。这让我在徒步的过程中可以有热水刷牙、洗脸,在那样的环境下可算是奢侈的享受了。呵
    呵}
\end{frame}

\subsection{}
\begin{frame}
  \photolandscape{day1_image006.jpg}{晚上在多吉家饱餐一顿,听说多吉家附近有个温泉,饭后冒雨走过去,结果大
    失所望,所谓的温泉,不过是一个很小的水泥池子,池底布满了淤泥,水也仅能没过脚踝。不过是千真万确的
    温泉哦,大家还装得很happy地在里面边泡边聊。}
\end{frame}

\subsection{}
\begin{frame}
  \photolandscape{day1_image007.jpg}{晚上住多吉家的队伍很多,因为怕被别人的鼾声影响,我独自找了个小阁楼铺
    了地布对付了一晚。第二天早上起来上厕所,发现门口站了个MM,对着我喊:别过来、别过来!我稍一纳
    闷,旋即明白,原来是厕所位置少,连男厕也被MM占领了,呵呵,没办法,女士优先。照片里就是多吉家,高
    的那栋楼是专门招待驴友的,照片里站在围栏上的GG是一个人从杭州来的,他后来跟了另一支队伍出发了。}
\end{frame}

\subsection{}
\begin{frame}
  \photolandscape{day1_image008.jpg}{出发前先来张合影。向导们在后面给马匹装辎重。}
\end{frame}

\subsection{}
\begin{frame}
  \photolandscape{day1_image009.jpg}{出发点可以是多吉家,我们选择坐车到水电站,车开得很快,风很大,我们几
    个坐货车尾箱的被冻得不行。下了车,整理行装准备出发。}
\end{frame}

\subsection{}
\begin{frame}
  \photolandscape{day1_image010.jpg}{开始沿着河谷往上走。}
\end{frame}

\subsection{}
\begin{frame}
  \photolandscape{day1_image011.jpg}{中间经过一个大草坪可以休息,草坪应该有个名字的,我忘了,呵呵}
\end{frame}

\subsection{}
\begin{frame}
  \photolandscape{day1_image012.jpg}{中午饭是在康定买的大饼,领队何老大把它叫做馕(新疆的叫法),开始吃的
    感觉还可以,到了后面几天水分越来越干,就变得难以下咽了,那帮FB分子到后来干脆不吃了,中午也开炉煮
    面条吃。我一直保持着艰苦朴素的作风(其实是我懒得洗碗),就着果珍吃下去,还能接受。}
\end{frame}

\subsection{}
\begin{frame}
  \photolandscape{day1_image013.jpg}{典型的西部风光}
\end{frame}

\subsection{}
\begin{frame}
  \photoportrait{day1_image014.jpg}{同样是典型的西部风光}
\end{frame}

\subsection{}
\begin{frame}
  \photoportrait{day1_image015.jpg}{和我结组的队友江月没上过高原,在出发没多久就出现高反,我看她走得很辛
    苦,也没有办法,只能不停地鼓励和安慰,后来前队让向导牵了匹马回来,把她载到了第一天的营地。}
\end{frame}

\subsection{}
\begin{frame}
  \photoportrait{day1_image016.jpg}{走这线路的队伍很多,路上偶遇不少其他队伍。}
\end{frame}

\subsection{}
\begin{frame}
  \photolandscape{day1_image017.jpg}{到了营地,大家忙着支帐篷。这个地方叫两河口(?),是一个大草坪,营地
    附近有条河,但是到河边取水要经过一个积水洼地,还好,我带了折叠水桶。}
\end{frame}

\subsection{}
\begin{frame}
  \photoportrait{day1_image018.jpg}{第一天的晚餐是火锅,还有新鲜的牛肉哦,但是第一天走得太饿了,往往食
    物还没煮熟就被大家抢着吞下去了。天幕是队友带的,用登山杖支起来既遮风又挡雨}
\end{frame}

\subsection{}
\begin{frame}
\begin{ztebox}
  这天晚上下了很大雨,雨点打在帐篷上让人睡不着,我不时爬起来看看帐篷有没有漏水,帘布下的登山鞋有没有
  被打湿,在这样的线路中鞋子湿了是灰常严重的事情,还好,除了帐底有点渗水之外,一切都安然无恙。在这样
  的忐忑与不安中过了一个不眠之夜,然而第二天最不希望发生的事情还是发生了……
\end{ztebox}
\end{frame}

\section{第二,三天}

\subsection{}
\begin{frame}
\begin{center}
\ztecalendar{Day}{2}
\end{center}
\end{frame}

\subsection{}
\begin{frame}
\begin{ztebox}
  第二天吃早饭的时候,少了“江月”、“十三”两个人,江月我知道状态不好,而十三、南瓜、LV都感冒了,十
  三的早饭是送进帐篷给他吃的。要知道感冒在高原是可大可小的事情,引起高原肺气肿更是有生命危险,不敢想
  了。而江月根本吃不下,心里掠过一丝不祥的预感。等她出了帐篷,我关切地问她状态怎么样,她看上去没有精
  神,脸色也不好,轻声地说:“我可能要下撤了”。这让我非常震惊,没想到第二天就有队友下撤,这是自己以
  及其他队友所不愿看到的。不过后来的事实证明,选择下撤是非常正确的,而同样状态不佳但选择坚持的十三后
  来则走得非常惨,这是后话。何老大安排了一个向导和一匹马帮江月下撤。这样一折腾,我们这天出发的时间就
  很晚了
\end{ztebox}
\end{frame}

\subsection{}
\begin{frame}
  \photolandscape{day23_image001.jpg}{}
\end{frame}

\subsection{}
\begin{frame}
  \photolandscape{day23_image002.jpg}{这一天的安排是从两河口到上热乌且营地,行程相对比较轻松,一路都是
    高山草甸。}
\end{frame}

\subsection{}
\begin{frame}
  \photolandscape{day23_image003.jpg}{或者沿着河边的乱石滩走}
\end{frame}

\subsection{}
\begin{frame}
  \photolandscape{day23_image004.jpg}{随着海拔的逐渐上升,风景也变得更加迷人}
\end{frame}

\subsection{}
\begin{frame}
  \photolandscape{day23_image005.jpg}{快到营地的时候,雪山已经忽隐忽现,大家的情绪开始高涨起来,走了两
    天,终于看到雪山了}
\end{frame}

\subsection{}
\begin{frame}
  \photolandscape{day23_image006.jpg}{下午3、4点就到了上热乌且营地,这是夹在两山之间的一个河谷地带,上
    面有雪山融水形成的河流,扎营再合适不过。}
\end{frame}

% \section{第三天}

\subsection{}
\begin{frame}
\begin{center}
\ztecalendar{Day}{3}
\end{center}
\end{frame}

\subsection{}
\begin{frame}
  \photolandscape{day23_image007.jpg}{早上起来,营地后面的热乌且雪山显得异常壮观。今天的目标是翻
    过4900米的热乌且垭口到达莫溪沟营地。}
\end{frame}

\subsection{}
\begin{frame}
  \photolandscape{day23_image008.jpg}{从营地右侧的山地往上爬,过了一座山头,是连续较长的乱石坡,因为海
    拔高(已经过了4100米),石头也比较松,走得甚为吃力。前两天和尚一直走在前面,今天不知什么原因落到
    后面了,后来才知道他今天高反明显。而十三因为状态不好,这一天是骑着马走的。过了乱石坡,回望来的路
    和身后的雪山,让人惊叹于大自然的神奇与魅力。每次去高原,都喜欢带上一个望远镜,可以偷窥雪山的每一
    吋迷人肌肤}
\end{frame}

\subsection{}
\begin{frame}
  \photolandscape{day23_image009.jpg}{爬上一个土坡,突然两个巨大的冰川出现在眼前,冰川下面是融水形成的
    高山海子,是此行中最壮观的风景之一了,大家纷纷在冰川前留影。自曝一张}
\end{frame}

\subsection{}
\begin{frame}
  \photolandscape{day23_image010.jpg}{随着海拔上升,高反的症状开始明显起来,主要的表现是头会出现阵
    痛,头痛的时候就停下来缓一下,虽然早上已经吃了一颗止痛药,但作用似乎不大。在这样的海拔,高反也是
    很正常的事情,忍一忍就过去了。}
\end{frame}

\subsection{}
\begin{frame}
  \photolandscape{day23_image011.jpg}{从老榆林到子梅垭口徒步,可以走热乌且垭口,也可以走盘盘山垭口,我
    们走的是前者}
\end{frame}

\subsection{}
\begin{frame}
  \photolandscape{day23_image012.jpg}{越往上走,地面的积雪越明显,还好,雪并不深,一脚踩下去已触及土
    层,并不显滑}
\end{frame}

\subsection{}
\begin{frame}
  \photolandscape{day23_image013.jpg}{望过去,垭口就在前方,但走起来却是那么遥远。这是此行最高海拔
    点――4900米的热乌且垭口}
\end{frame}

\subsection{}
\begin{frame}
  \photolandscape{day23_image014.jpg}{过了垭口,在另一面,也是个陡坡,由于雨越下越大,土坡显得非常湿
    滑,根本站不稳,全靠登山杖做支撑}
\end{frame}

\subsection{}
\begin{frame}
  \photolandscape{day23_image015.jpg}{下到草甸,所有队伍都在这休息,马上就到莫溪沟了。向导嘎玛指着左边
    那座雪山告诉我,那叫勒多漫因。过垭口之前,马队是沿着靠近冰川的那条路走的,十三虽然可以骑马,但是
    在坡陡的路段,就很危险了,听向导说他差点从马背上掉下来,是向导和其他队友搀扶着过垭口的。休息当
    中,其他队友开始煮面条,我啃完难以下咽的“馕”,和小四先走了。}
\end{frame}

\subsection{}
\begin{frame}
  \photolandscape{day23_image016.jpg}{沿着莫西沟往前走,山坡的树很奇怪地偏向一边生长。}
\end{frame}

\subsection{}
\begin{frame}
  \photolandscape{day23_image017.jpg}{突然,路边冒出一道彩虹,似乎触手可及。}
\end{frame}

\subsection{}
\begin{frame}
  \photolandscape{day23_image018.jpg}{天黑前,终于在一块相对平坦的草坡安营扎寨。}
\end{frame}

\subsection{}
\begin{frame}
  \photolandscape{day23_image019.jpg}{我故意把帐门对着雪山,这样躺在帐篷里都能对着雪山发呆。我带的是普
    通铝杆三季帐,第一天已经受过风雨洗礼,说明在这样的线路和气候条件下,三季帐已足够应付。}
\end{frame}

\subsection{}
\begin{frame}
\begin{ztebox}
  大家把公用物质卸下来,向导降措开始烧酥油茶了,然而忙活到天黑,十三他们竟然还没到,和十三在一起
  的,还有两个向导嘎玛、吉仁以及两个队友“和尚”和“鬼”。渐渐地我们越发觉得不对劲,时间在一分一分地
  过去,我们开始着急起来,我不停地用对讲机呼叫着“和尚”,但山的那头仍然是寂静一片……
\end{ztebox}
\end{frame}

\section{第四天}


\subsection{}
\begin{frame}
\begin{center}
\ztecalendar{Day}{4}
\end{center}
\end{frame}

\subsection{}
\begin{frame}
\begin{ztebox} 
{\footnotesize
  情急之下,何老大提出让向导降措骑马回去找十三,看究竟出了什么事。但降措因为照看着几匹马,脱不开
  身,他让我们去找其他队的向导熊英,花点钱请他帮忙。我们看看南瓜,因为出钱请人还需要她来决定。这里有
  必要解释一下,南瓜和十三是情侣,虽然是MM,南瓜的体能比十三要好得多。南瓜答应了,除了这样还能有什么
  办法呢?我们让降措找来熊英,问好价钱,他骑着马出发了。
}
\end{ztebox}
\begin{ztebox}
{\footnotesize
  我们在营地的几个准备埋锅造饭,这样十三他们一到就能吃到可口的饭菜。不知过了多久,终于听到路那边传来
  人声,是十三!他们终于回来了,十三几乎是被架着回来的,回到营地,他几乎连坐着的力气都没有,处于半休
  克状态,我们赶紧让他进帐篷休息。陪着十三回来的向导嘎玛和吉仁一声不吭扎到帐篷里没再出来。队友和尚、
  鬼也显得疲惫不堪。和尚和鬼一路照顾着十三,不离不弃,真是非常难得。在这支队伍里,和尚、鬼、南瓜、LV、
  十三在深圳就是一起爬山的山友,所以他们彼此关系都很好。和尚是我们的大厨,我看他比较累,这天是我做的
  饭,但大家似乎都有心事,吃得都不多,向导嘎玛和吉仁连饭都没吃。
}
\end{ztebox}
\end{frame}

\subsection{}
\begin{frame}
\begin{ztebox}
{\footnotesize
  此后,我明显觉得,两个向导都不愿多搭理十三,明显有气,也难怪,为了照顾他,两个向导确实花了不少精力。
  不过后来从南瓜口中得知,两个向导生气,其实还有别的原因。当时请熊英回去找十三,说好价钱是100块的,把
  十三接回来后,熊英说太累了,十三几乎是他背回来的,要把价钱提高到300,南瓜同意了,当时嘎玛和吉仁也在
  场。第二天,南瓜看到嘎玛、吉仁和熊英似乎在争执什么,猜想可能是为那300块钱的事,毕竟把人平安送到营
  地,嘎玛和吉仁也出了不少力,钱也应该三个人分才是。所以南瓜后来让十三给嘎玛和吉仁各送了100块钱,不过
  嘎玛和吉仁没有收,听说嘎玛当时还流泪了,说:这不是我们和你之间的事情,而是我们和他(熊英)之间的事
  情。
}
\end{ztebox}
\begin{ztebox}
{\footnotesize
  其实话说回来,这次徒步所雇的三个向导,都很好人,很纯朴,何老大吩咐他们做的一些额外的事情都很爽快地
  答应了,他们也常常给体力不好的队友背包,为了这样一件事把他们搞得不愉快是大家都不愿意看到的。
}
\end{ztebox}
\end{frame}


\subsection{}
\begin{frame}
  \photolandscape{day4_image001.jpg}{我们早早地出发了,因为前几天耽误了不少时间,这天将是最辛苦的。}
\end{frame}

\subsection{}
\begin{frame}
  \photolandscape{day4_image002.jpg}{路过一个冰川,藏民在前面垒起了玛尼堆。}
\end{frame}

\subsection{}
\begin{frame}
  \photolandscape{day4_image003.jpg}{走出没多久,远远看见一个河滩上站了不少人,走近一看,一条小河横在
    前面,要涉水过河,河水来自冰川融雪,冰冷刺骨。}
\end{frame}

\subsection{}
\begin{frame}
  \photolandscape{day4_image004.jpg}{过河的办法有多种,最简单的就是骑马过去,向导只答应MM骑马过去,GG要
    骑马的话,就要自掏腰包}
\end{frame}

\subsection{}
\begin{frame}
  \photolandscape{day4_image005.jpg}{直接涉水过去,把鞋子先扔到对岸,臂力不够就惨了。当然你如果有足够
    魅力,也可以找人背过去}
\end{frame}

\subsection{}
\begin{frame}
  \photolandscape{day4_image006.jpg}{相互帮助是最好的办法。}
\end{frame}

\subsection{}
\begin{frame}
  \photolandscape{day4_image007.jpg}{何老大想了个绝招,他把防水袋套在鞋子和裤腿上,扎紧,直接淌过去。
    我很是为他捏了把汗,万一漏水可不得了,不过,他很幸运。}
\end{frame}

\subsection{}
\begin{frame}
  \photolandscape{day4_image008.jpg}{小四是最后一个过的,他原想等着骑马过去,但我们的马队一直没到。}
\end{frame}

\subsection{}
\begin{frame}
  \photolandscape{day4_image009.jpg}{在河滩休息片刻,继续赶路。}
\end{frame}

\subsection{}
\begin{frame}
  \photolandscape{day4_image010.jpg}{此后
是在这样的河谷开阔地穿越,地形起伏不大。}
\end{frame}

\subsection{}
\begin{frame}
  \photolandscape{day4_image011.jpg}{然而前面又有一条溪流横在路前,虽然水不深,但水流急,硬底登山鞋踩
    在被水淹没的石头上特别滑,需非常小心。我好不容易找了相对较窄的水面,用登山杖支撑小心越过。}
\end{frame}

\subsection{}
\begin{frame}
  \photolandscape{day4_image012.jpg}{不久,我们的马队过来了。}
\end{frame}

\subsection{}
\begin{frame}
  \photolandscape{day4_image013.jpg}{队友们相互帮助着安全渡过了难关。}
\end{frame}

\subsection{}
\begin{frame}
  \photolandscape{day4_image014.jpg}{过了开阔地,进入林间小道,流水惊奇地发现,路边一棵松树上的松萝竟
    然是蓝色的。}
\end{frame}

\subsection{}
\begin{frame}
  \photolandscape{day4_image015.jpg}{这一天穿越的地形是最丰富的,这就是有名的“原始森林”。张牙舞爪的
    树木极具震撼力。}
\end{frame}

\subsection{}
\begin{frame}
  \photolandscape{day4_image016.jpg}{下午快三点,终于到达冬季牧场,向导建议我们在此扎营,何老大极力说
    服大家直接赶到上子梅村,因为前面耽误了不少时间,今天要把原计划两天的路程走完。而冬季牧场到上子梅
    村至少要走}
\end{frame}

\subsection{}
\begin{frame}
  \photolandscape{day4_image017.jpg}{从冬季牧场到贡嘎寺和子梅村的分岔路口,我都记不清翻了多少个山
    头,只记得每上一个斜坡都要耗费不少体力。我给自己设立一个个目标点,到达目标点就休息一会。路上碰到
    两个负重穿越的驴友,疲惫的眼神中流露着绝望。}
\end{frame}

\subsection{}
\begin{frame}
  \photolandscape{day4_image018.jpg}{十三的状态依然很差,因为骑马上下坡比较危险,他只有下来徒步,多亏
    了和尚和鬼一路无微不至地照顾他。}
\end{frame}

\subsection{}
\begin{frame}
  \photolandscape{day4_image019.jpg}{路过一个木牌坊,路边有个木牌用汉藏文写着:大圣贡呷坚赞驱魔祭
    品,看了让人毛骨悚然。}
\end{frame}

\subsection{}
\begin{frame}
  \begin{ztebox} {\footnotesize 眼看着天色越来越暗,我加快了步伐,把何老大、流水和野豌豆落在了后
      面,我追上了前面的LV,但丝毫没有放慢脚步,LV也越走越快。大家都是经常出来徒步爬山的,心里都有一
      股不服输的劲头。终于在下午6点左右到达贡嘎寺和子梅村的分岔口,这里有玛尼堆和经幡,向左到贡嘎
      寺,向右到子梅村。我其实是很想去贡嘎寺看看的,虽然只是个小寺庙,却因了贡嘎山的原因而声名远扬。
      何老大决定放弃贡嘎寺,直接去上子梅村。后来听说这几天贡嘎寺人满为患,打地铺都要75元每人,不禁暗
      自庆幸。}
\end{ztebox}
\begin{ztebox} {\footnotesize 不幸的是,雨又开始下起来,我和LV走散了,一个人走在林荫下泥泞的马道上,由
    于天色将黑,前面的路已经看不清,我又懒得拿出头灯,就这么摸黑地走着,除了自己的脚步声,周围显得非
    常安静,我不禁左顾右盼,生怕路两旁跳出什么骇人的动物。路过一个木牌坊,路边有个木牌用汉藏文写
    着:大圣贡呷坚赞驱魔祭品,看了让人毛骨悚然。}
\end{ztebox}
\end{frame}

\subsection{}
\begin{frame}
  \begin{ztebox} {\footnotesize 不知走了多久,仍然看不到前面有村落,雨越下越大,虽然穿着冲锋衣,浑身内
      外都湿了,天色已经全黑,我亮起头灯,似乎听说过这地方有熊,一边走一边还用头灯照照路边的树丛,生
      怕树后面真的藏了一头狗熊。路上满是泥泞的积水,鞋子外面裹满了泥浆,还好里面没怎么湿,有“狗太
      死”防护的登山鞋果然不错!突然,我仿佛听到了人声,是有人在说话,我学着狼“呜呜”叫,给对方打信
      号,对方也以同样的方式给我回答,我喜出望外,似乎快要到了!当然我并不知道给我回答的是什么人,可
      能是子梅村的村民,也可能是其他旅行者。过了一座木桥,我看到前面山脚下冒雨站了一个人,是当地人的
      打扮,我问他前面是上子梅村吗?他回答是。我猜刚才给我打信号的是他。他二话没说,拿起我的随身包转
      身就走,我跟着他心里泛起阵阵惊恐,如果他是坏人怎么办,如果他还有同伙在前面,在这慌无人烟的地
      方,我是怎么也跑不掉的,只好听天由命了。\ldots }
\end{ztebox}
\end{frame}

\subsection{}
\begin{frame}
  \begin{ztebox} {\footnotesize 还好,走了没多远,我看到了LV和我们可爱的向导嘎玛,谢天谢地,终于走到上
      子梅村了!向导告诉我们,小四已经到了,但南瓜还没到。我当时也没在意,只顾用对讲机跟何老大联
      系,招呼他该怎么走。十三、和尚、鬼三个走在最后,对讲机联系不上,不知他们怎么样了,因为雨下得很
      大,十三状态一直不好,真担心会出什么事。何老大让我找一辆摩托车去接十三,我办妥后,跟LV坐了另一
      辆摩托车回到上子梅村村民家中,我们将在这里住一晚,我看了一下表,今天走了13个小时。小四是第一个
      到达的,他告诉我,南瓜失踪了\ldots }
\end{ztebox}
\end{frame}


\section{第五,六天}
\subsection{}
\begin{frame}
  \begin{ztebox} {\footnotesize 南瓜没有对讲机,手机在这山旮旯里也没有信号,根本无法联系上。我们判断
      最大的可能是南瓜在哪个分岔路走错了方向。据小四回忆,他最后一次看到南瓜是在贡嘎寺和子梅村岔口不
      远的地方,他在拍照,南瓜超过了他。照理南瓜应该是第一个到达的。在冬季牧场,何老大就问过向导,从
      分岔口到上子梅村应该沿着大路走,难道南瓜走到小道去了?我们决定等何老大他们到齐了再商量。外面黑
      灯瞎火的,又下着雨,贸然出去寻找,只会徒增麻烦,而且走了一天,大家又累又饿。这时,村民家开始吃
      饭了,我们三个先到的吃了点饭,觉得暖和多了。吃过饭没多久,我们仨正聊着天,南瓜竟若无其事地进来
      了。我们先是一愣,纷纷问起她“失踪”的原因。果不其然,她走错了方向。南瓜回忆说,她在贡嘎寺和子
      梅村的岔路遇到了一对反穿的情侣,问他们子梅村怎么走,得到的回答是:一路下坡就到了。结果,她顺着
      向下的路走到了下子梅村。在那里她遇到一支扎营的队伍,问起来才知道还有个上子梅村,重新返回,上
      山,又走了一个多小时,终于回到上子梅村。}
\end{ztebox}
\end{frame}

\subsection{}
\begin{frame}
\begin{ztebox} 
{\footnotesize
不久,何老大、流水、野豌豆和奶茶都到了,十三、鬼、和尚三个随后也搭乘摩托车顺利回到。我想到他们几个还
没吃饭,记起背包里还有几个罐头,在炉头上加热了一下,正好可以消灭掉,村民也煮好了酥油茶招待我们。我们
在火炉旁烘烤着冲锋衣和登山鞋,聊起这天的经历,大家似乎有吐不完的言语。

这天晚上,在上子梅村的村民家,睡了个安稳觉。}
\end{ztebox}
\end{frame}

\subsection{}
\begin{frame}
\begin{center}
\ztecalendar{Day}{5}
\end{center}
\end{frame}

\subsection{}
\begin{frame}
  \photolandscape{fin_image001.jpg}{早上起来,向导和村民在火炉旁吃糌粑。所谓的糌粑,就是把炒熟的青稞
    面和上酥油茶,用手指在碗里揉捏,直至和成面团,然后用手捏着吃。我试了一下,味道很香,加上糖口感很
    好,有炒芝麻的味道,不过用手捏着吃还是不太习惯。向导还让我试了一种乳制品,名字忘了,象拉长的牛皮
    糖,味道则实在不感恭维,又酸又馊,向导问我感觉怎么样,我皱着眉头说还可以,向导们都笑了,他们知道
    我在说谎}
\end{frame}

\subsection{}
\begin{frame}
  \photolandscape{fin_image002.jpg}{这天大伙的行动就比较随意了,野豌豆、奶茶、十三和南瓜包了摩托车去
    泉华滩看钙化池,、流水和小四则坐摩托车到子梅垭口。在这点上我很欣赏小四,虽然体力很好,但并不固执
    于过程,有条件就利用条件达到目的。而我跟何老大则属于顽固派,似乎不借助自己的力量走全程就是受了什
    么奇耻大辱。所以徒步上子梅垭口的只有我、何老大、和尚和鬼四个。虽然这天要垂直爬高1100米,由于有一
    整天时间,即使慢慢爬也能爬到子梅垭口,我们这样分析。}
\end{frame}

\subsection{}
\begin{frame}
  \photolandscape{fin_image003.jpg}{从子梅村出来就一路爬升,开始地形相对平坦,路也好走。路边是茂密的
    高山针叶林和清澈溪涧,空气很清新。我们走得很悠闲,边走边拍照。}
\end{frame}

\subsection{}
\begin{frame}
  \photolandscape{fin_image004.jpg}{走到山腰,回望,可以看到贡嘎了,主峰却一直藏在云层后面。网络
    上,流传着一个叫墨汁的驴友拍的贡嘎照片:蓝天之下,白云之上,一个徒步的驴友回望身后熠熠生辉的贡嘎
    主峰,给人极其震撼的印象。正是这张照片,吸引着无数驴友前赴后继,一探贡嘎神秘面纱后的真实面目。我
    们边走边在猜想,墨汁的照片究竟是在哪个角度照的。}
\end{frame}

\subsection{}
\begin{frame}
  \photolandscape{fin_image005.jpg}{在一个山坡休息,面对着贡嘎静静发呆。山坡下的云涌过来,竟开始下雨
    了。}
\end{frame}

\subsection{}
\begin{frame}
  \photolandscape{fin_image006.jpg}{后面的路越来越陡,走起来也越发吃力,快到垭口的时候,下起雨加雪。}
\end{frame}

\subsection{}
\begin{frame}
  \photolandscape{fin_image007.jpg}{下午3、4点就到了垭口,LV、流水和小四早已到达,帮我们把床铺占好了。
    子梅垭口有简易的大通铺,是贡嘎寺的喇嘛开设的,价格好像是每人每天25元。我放下行李,第一时间跑到旁
    边的小卖部,要了一支可乐,喝下去,冰冷提神,太爽了!早在几天前,就听何老大说过子梅垭口能喝到可乐
    饮料。徒步这些天来,仿佛与现代工业文明失去了联系,走的是原始的山林河谷,吃的是制作简单的食物,没
    有电视、电话,没有网络、汉堡,真正体会了眼睛在天堂,肉体在地狱的感觉,现在,一支可乐下去,肉体也
    回到了人间。虽然,不少人痛恨这种西洋的垃圾饮料,但,我喜欢,That's enough!}
\end{frame}

\subsection{}
\begin{frame}
  \photolandscape{fin_image008.jpg}{子梅垭口,海拔4550米。}
\end{frame}

\subsection{}
\begin{frame}
  \photolandscape{fin_image009.jpg}{在子梅垭口,所有人的焦点只有一个:贡嘎主峰。照片中间那位戴白帽子
    的色驴在上面苦等了6天,也未能有缘相见。}
\end{frame}

\subsection{}
\begin{frame}
  \photoportrait{fin_image010.jpg}{贡嘎始终躲在云后不肯露面,等了许久,这已经是现身最多的一刻。高原
    辐射强烈,多日的徒步使肤色宛如古钟,队友们笑我可以冒充藏族同胞了}
\end{frame}

\subsection{}
\begin{frame}
  \photolandscapesmall{fin_image011.jpg}{晚上住通铺我认为是个错误选择。想想,几十个不相识的旅行者一个挨一
    个躺在简单铺就的被褥里,半夜梦者鼾声一片,醒者辗转无眠,如厕者来去自便,有人取物而响不停的背包拉
    链,让所有睡意化作青烟。我一夜翻来覆去,几乎没合眼,早知道拿帐篷到外面扎营了。}
\end{frame}


\subsection{}
\begin{frame}
\begin{center}
\ztecalendar{Day}{6}
\end{center}
\end{frame}


\subsection{}
\begin{frame}
  \photolandscape{fin_image012.jpg}{第二天早上,睡意朦胧间,流水跑进来告诉我,可以看到贡嘎了。我赶紧
    爬起来,打起头灯穿上外套飞身屋外,我惊呆了,几天的艰苦跋涉没有白费\ldots}
\end{frame}

\subsection{}
\begin{frame}
  \photolandscape{fin_image013.jpg}{米的“蜀山之王”如大鹏展翅般屹立在面前。子梅垭口几乎沸腾了,到处
    是此起彼伏的快门声。}
\end{frame}

\subsection{}
\begin{frame}
  \photolandscape{fin_image014.jpg}{在神山前驻足良久,看够了拍够了,我们决定下撤,下到上木居包车回康
    定。}
\end{frame}

\subsection{}
\begin{frame}
  \photolandscape{fin_image015.jpg}{从子梅垭口到上木居近20公里,一路下坡,很好走,路上碰到前一天包车
    去泉华滩的奶茶、野豌豆、十三和南瓜,他们包了辆拖拉机上子梅垭口看贡嘎。其他队友走得很快,我和何老
    大在后面边走边聊。}
\end{frame}

\subsection{}
\begin{frame}
  \photolandscape{fin_image016.jpg}{中午到上木居,向导降措把我们接到当地村民家休息吃饭。向导嘎玛和吉
    仁把我们的包驮到村民家就赶着马匹回去了,没有跟他们道别甚是遗憾。我猜他们可能还在生着气。}
\end{frame}

\subsection{}
\begin{frame}
  \photolandscape{fin_image017.jpg}{在村民家主人给我们炒了几个小菜,被我们消灭得一干二净。主人还给我
    们现做酥油茶,把一个细长的木桶装上酥油和茶水,用一根长长的木棍不停地捣,倒出来就是传说中的酥油茶
    了。}
\end{frame}

\subsection{}
\begin{frame}
  \photolandscape{fin_image018.jpg}{在上木居包了两辆面的去康定,每辆650。所有大包被绑在车顶上。}
\end{frame}

\subsection{}
\begin{frame}
  \photolandscape{fin_image019.jpg}{平安走完全程,大家心情都很放松。}
\end{frame}

\subsection{}
\begin{frame}
\begin{ztebox} 
{\footnotesize 
上木居到康定有200多公里,中间经过新都桥、折多山。我们这辆车的司机比较鲁莽,路
      上几乎撞死一条狗,新都桥之后路很烂,司机仍旧鲁莽驾驶,晚上车子终于在折多山歇菜了。为此小四和司
      机几乎吵起来。我们的御寒衣物在车顶上,坐在车厢里越发觉得寒冷,只有掏出零食解闷。小四给讲了几个
      昏段子把大家逗得哈哈大笑。我给旅行西藏的朋友发了条短信,他们正好也堵在青藏公路唐古拉山至沱沱河
      之间,当天当雄发生地震,他们躲过了一劫。我们的司机下车折腾了半天,终究没辙,打电话让康定的车返
      回来接我们。等我们回到康定已经是凌晨2点多,先到的队友给我们订好了房间并打包了宵夜。然而我们回到
      旅店做的第一件事是――洗头冲凉。自从离开成都,已整整7天没洗澡,加上连续徒步,一个个已是蓬头垢
      脸,拖出去也跟叫化子无异了。}
\end{ztebox}
\end{frame}

\subsection{}
\begin{frame}
\begin{ztebox} 
{\footnotesize
天亮后,其他队友坐班车回成都,我则去海螺沟看冰川,大家就此惜别。

事情过去了一年多,岁月渐渐为记忆蒙上了尘土,吹开这片尘土,会发现记忆深处有个宝贝在闪闪发亮。这宝贝就
象陈年老酒般愈久弥新,即使多年以后品上一口仍觉芳香扑鼻,回味无穷。当然,个中滋味,只有了解酒的来历的
人才能真正体会到 \ldots
}
\end{ztebox}
\begin{ztebox} 
{\footnotesize
\begin{center}
【全文结束】
\end{center}
}
\end{ztebox}
\end{frame}

\section{贡嘎线路图}
\subsection{}
\begin{frame}
  \photolandscape{fin_image020.jpg}{提示:贡嘎穿越经历多日高原徒步,需面对复杂多变的气候和地形条
    件,对体力和装备有一定要求,计划参与者请慎行}
\end{frame}

\end{document}
