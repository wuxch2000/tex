\documentclass[12pt]{article}
\title{李逵日记}
\author{}
\date{}

\usepackage{config}

\linespread{1.2}

\begin{document}

% \pagenumbering{gobble}
\twocolumn[
\begin{@twocolumnfalse}
\maketitle\thispagestyle{empty}
% \tableofcontents
\end{@twocolumnfalse}
]

\clearpage

\pageaofbfoot{1}

\section{}

扈三娘生了,是个大胖小子,我心里十分纳闷,二月份才结婚,这才刚刚进八月\dldots 这里面肯定有古怪\dldots

聚义厅照例聚会,烦透了,本不想去,但强盗圈就这么大,抬头不见低头见,不去说不过去,去了就得随礼,哎!我区区一
个堂级干部,一月俸禄才二十两银子,前几天秦明结婚随了十两,他是厅级干部,给少了不好看,何况我以后可能要归他
大舅子花荣管。不过心里想想,秦明这厮忒不要脸,二婚还搞的这么隆重,咒你生儿子没屁眼。

扈三娘和王矮虎都是堂级干部,跟我平级,王矮虎武艺有限,人品也不咋地,估计没多大前途,本来想给二两银子意思意
思行了,不过扈三娘好像在宋大哥那边说得上话,最近中层干部要调整,这是关键时刻,舍不得孩子套不着狼,给五两吧。

听说张顺的爹快死了,剩下的五两得给预备着。

幸亏这个月下山干了票大的,山寨规定按百分之十提成,估计有十两银子分红,明天先预支一下,不然得喝西北风了。

王矮虎那厮脸笑的跟花似的,越看越恶心,扈三娘怎么嫁给他了那?要长相没长相,要内涵没内涵!哎!好菜都让猪拱了。

会上发生了点小小不快,晁天王和宋大哥又吵了起来,其实也不是啥原则性分歧,晁天王说孩子像爸爸,宋大哥说像妈
妈,两人总爱为这样鸡毛蒜皮的小事较劲。

两人争执不下,脸红脖子粗,像发情的公鸡,每当此时最讨厌,两人非得让手下表态,林冲借口喝醉了狂奔出去呕吐,戴
宗犯了间歇性耳聋,公孙胜、刘唐和阮家三兄弟支持晁天王,花荣、武松和鲁智深支持宋大哥,吴用这厮最狡猾,说鼻
子像爸爸,眼睛像妈妈,读书人花花肠子就是多,轮到我了,我慢条斯理的说,都不像,像我!扈三娘大怒,拿起酒碗泼了
我一身,众人哈哈大笑,才算过去了。

其实,那孩子,像宋大哥,黑不溜秋的,但是我没敢说

\section{}

酒,真是好东西,它可以让人忘记烦恼。

晁天王喝多了,宋大哥也喝多了。两人刚刚还脸红脖子粗,仿佛有不共戴天之仇,转眼间就像亲兄弟一样,手拉着手,痛
说革命家史,翻脸的速度比翻书还快,看来老大还真不是一般人能当的

晁天王醉醺醺的说,抢劫生辰纲那次,多亏贤弟及时报信,不然我们兄弟几个就折进去了,你是梁山泊的大恩人,这头把
交椅该你坐\dldots

宋大哥连连摆手说,江州劫法场那次,若不是老哥你带着兄弟及时赶到,恐怕小弟早就沦为刀下之鬼了,这头把交椅还
是大哥你坐\dldots

这两件事都叨叨八百遍了,耳朵都起茧子了,朱武在一旁冷笑,我想,其中内幕绝非``义气''二字那么简单\dldots

吴用拿着把四处漏风的破蒲扇,一边摇一边念念有词:安得广厦千万间,大辟天下寒士尽欢颜\dldots 那表情,那神态,
很是悲伤,跟死了爹似地

我心想,文化人真他妈的虚伪,咱是什么?强盗啊!老百姓的房屋就是咱烧的,老婆孩子也是咱杀的,你还在这里充什么
大陷包子?真不害臊!不过这话不能明说,毕竟人家是领导嘛,领导天生就是虚伪动物,宋大哥和晁天王天天都在背后问
候对方的八辈祖宗,见了面不照样称兄道弟?

公孙胜是道家,按说出家人不该喝酒,这厮非得喝米酒,说什么米酒是素酒,不算破戒,杀人放火的事你都干了,还在乎
这点小事?又想当婊子又要立牌坊,真没意思!看人家鲁智深,也是出家人,人家就敞亮多了,该喝酒喝酒,该吃肉吃肉,
也没人笑话他!

\section{}

酒场上男人的三大尴尬:自己喝醉了缠着兄弟的老婆,老婆喝醉了缠着自己兄弟,兄弟的老婆喝醉了缠着自己。

第一句话是鲁智深总结的,据说有一次他喝醉后曾拉着林冲的娘子叨叨个不停,不过那是上山之前的事了。

第二句话是张青总结的,他老婆喝醉了就爱缠着别的男人没完,每当此时,他坐在那里,脸青的跟萝卜似的。

第三句话是武松总结的,纠缠他的女人海了去了,其中,曾经有个女人是他的亲嫂嫂,而这个女人,也是被他亲手杀的,
他自己从来不提这事,当然,也没人敢问。

我从没有类似经历,原因有三:一、我没老婆,二、我喝醉了只会抱着树哭,绝不会抱女人,当然,最主要的是女人也不
会让我抱,三、从没有一个女人喝醉后缠着我,哪怕醉的不省人事,见了我,立马就醒了。

\section{}

我发现一个规律,男人的相貌会影响女人的酒量。比如,如果我坐旁边,那么女人个个都是女中豪杰,揎拳
捋袖,千杯不醉,如果换成武松,那旁边的女人抿两口就脸色绯红,直喊头疼,甚至步履踉跄,真他妈邪了门了。

孙二娘又喝多了,大红裙子系腰间,一只脚踏在板凳上,唾沫横飞的拽着武松拼酒,武松喝也不是,不喝也不是,脸涨得
通红,看来长的帅也是种负担。

女人这东西,三天不打,上房揭瓦,这是王矮虎教育张青的话,张青哭丧着脸说,他也经常打,不过,是被打。

张青也是倒霉,怎么娶了这么一个女人,休又不敢休,活脱脱受罪,要是我,早就大耳瓜子煽上了。

\section{}

林冲一个人在自酌自饮,我过去跟他碰杯,其实我不喜欢他这种墙头草性格,风一吹立马就倒。

不过每个人都很忙,只有他闲着,有时候两个男人在一起喝酒,是不需要任何理由的。

这时,王矮虎哭丧着脸从旁边经过,林冲叫住他问,喜事你怎么摆了副丧事的脸?我们又不白吃,看你弄的这几个菜,今
天没少赚吧?

王矮虎讪讪的说,别说赚了,赔大发了,贺礼收了一千多两银子,可光酒席就花了两千两银子

林冲擂着桌子说道,你别瞎说,菜全是山上的野菜,鱼是湖里捞的,兔子肯定也是从山上打的,没啥本钱,怎么会花那么
多银子

林冲声音有点大,旁边有人看过来,王矮虎食指放嘴唇,做了个``嘘''的手势说道:酒席是宋氏酒楼操办的

林冲``啪''的把筷子一撂,``哪个酒楼办的也不能漫天要价''

我捅捅他的腰,小声告诉他,酒楼老板是宋青,宋大哥的亲弟弟。

林冲的脸像开了个水彩铺,红了又白,白了又青,满腔怒火立马化为乌有,板着脸开始训王矮虎:今天的酒席真不错,你
看这野菜多新鲜,你尝这鱼汤,口感多好,这野兔,一看就是精心烹制的,收你两千两银子算你赚了\dldots

``那是、那是''王矮虎苦笑两声离去。

做人难,做强盗难上加难!

\section{}

夜深了,我还不敢睡,我在等宋大哥。

上梁山后,宋大哥从没有在一个地方睡超过两晚上,有时半夜敲开我的门,有时去花荣那里,有时去武松那里,极少在他
自己房子里住。他那个房子也邪了门了,不是突然起火,就是莫名被砸,真不知道他上辈子做了啥孽。

来这里睡,我没意见,宋大哥从不睡床,来了就爬屋梁上,说是小时候睡习惯了,真不知道他有这癖好。不过早上醒来
时,他总是趴在地上,鼻青脸肿,口水遍地。

今天喝醉后,我当众问他来不来我这里睡,他阴沉着脸说不来,我知道,他肯定会来,因为他说不来的时候肯定会来,他
说来的时候,肯定会不来,我早就摸透了。

半夜时分,宋大哥果然来了,咧着大嘴笑的很灿烂:黑厮,没想到吧?

大哥如此高兴,做小弟的也不能拂了美意,我装出意外的样子说:宋大哥,你不是说不来吗?

宋大哥哈哈大笑说,孙子兵法云:虚则实之、实则虚之,虚虚实实,鬼神莫测,这是兵法的最高境界

我想,要么是我太聪明,要么就是那个叫孙子的太蠢。

我突然又想,每当我自以为很聪明的时候,是不是别人也在装着很蠢?

\section{}

遇到一个人,改变一生的命运,这是军师吴用说的,不过我深以为然。

林冲若遇不到高俅,现在还在开封当禁军教头,鲁智深若遇不到金翠莲,还会在经略府当提辖。

命运,真让人捉摸不透!我以前在江州当狱警,虽赚不了大钱,但也是朝廷在编人员,按月领工资,偶尔收点小贿捞个外
快,赌赌博,喝喝酒,日子也过的逍遥自在。照这样发展下去,娶个老婆、生个孩子,很快就能奔上小康生活。可惜,后
来遇上宋大哥。

初见宋大哥时,心中大喜,活了二十多年终于碰到比我还丑的了。宋大哥五短三粗,咋看咋像个黑茄子,不过出手阔绰,前
后送给我几百两银子,当然,我也不白收,他不用打杀威棒,不用干体力活,可以随意出入监狱\dldots

让我万万没料到的是,这厮酒后居然题了反诗,其实题反诗也没啥,每个男人心中都有个造反 情节,喝醉后骂骂朝廷
也是常事,但千不该万不该不该在浔阳楼上题反诗,那可是州府指定合作单位,来往的都是官场上的人,这不是老虎嘴
里夺肉,怡红楼里泡妞!

题了也就题了,你他妈还署上名,署了也就署了,署的竟然还是真名,真是没事找抽型的!

结果被抓起来审问,这厮装疯卖傻,吃大粪跟吃馒头似的,差点就蒙混过关,不过关键时刻没顶住,竹筒倒豆子般从小时
候偷邻居茄子到长大后偷看女人洗澡全招了,当然包括给我送钱的事,哎!可惜那堆大粪了!

那时我还不怎么懂法,以为几百两银子要掉脑袋,头脑一发热,干脆反了,后来知道朝廷有政策,贪污犯不判死刑,肠子
都悔青了。

\section{}

这两天没啥大事。

晁天王三天两头请人喝酒,宋大哥隔三岔五找人谈心。

林冲请公孙胜弄了个草人,写上高俅的名字,每天早中晚各扎上一针。

武松和鲁智深干了一架,两人闲的无聊,猜筛子大小,赢了的煽输了的一巴掌,鲁智深连输十八场,脸肿的跟烧饼似地。
鲁智深说武松出老千,武松说鲁智深太笨,两人你一言我一语,说着说着就打了起来\dldots

南拳北腿,你来我往,众兄弟一听说打架,哗的围了半山人,有喝彩的,有加油的,有敲锣打鼓的,还有开盘口下赌注的
\dldots

两人打了半天,看没人拉架,只好自己罢手。鲁智深气呼呼的说再跟武松说话他就是王八,武松说他再搭理鲁智深他就
是狗娘养的。

没半个时辰,王八自己说话了,说只要武松不再跟他要赌债,就原谅他,武松也不管他娘同不同意,当下说赌债不要了,
两人和好如初。

秦明和大舅子花荣掐了一架,好像是因为秦明喝醉后跟花二妹亲热时念叨着前妻的名字,被花二妹凌空一脚踹下床,跌
的头破血流,事后据王矮虎推测,两人当时用的很可能是江湖中失传已久的猿搏式。

秦明是个火爆脾气,爬起来就给了花二妹一巴掌,花荣得知后鞋子都没穿一溜小跑进门就``啪啪''给了秦明两巴掌。

两人都是武将,武艺不相上下,又都是厅级领导,谁也不服谁,当下对打了起来,刚开始时秦明骑着花荣打,花二妹看哥
哥吃亏上去拽着秦明的头发来了一通虎鹤双形,后来花荣骑着秦明打,花二妹又心疼丈夫,拉着花荣胳膊不让打,场面
乱成一锅粥。

领导打架,我们做下属的不敢贸然劝架,再说了,人家是亲家,上去拉架说不定会被两人合揍,都在一边傻站着看,嘴里
嚷嚷着别打了别打了,心里美滋滋的盼望两人能够多打一会,不然漫漫长夜没啥消遣真是无聊死了,后来宋大哥赶到,
一人给了一巴掌,两人才算消停。

王矮虎凑的太近,也被宋大哥煽了一巴掌,活该,谁让你离那么近!

秦明成了熊猫眼,花荣成了歪嘴巴,两人窝在家里都不出门,花二妹也搬回了哥哥家住。秦明天天在家捂着脸哀叹:唯
女子与小人难养也!

阮家三兄弟因为赡养老父亲的事又大打出手,三人上梁山时都带着家眷,晁天王照顾他们,特批给一套四合院,一家人
一直在一口锅里吃饭,天天因为鸡毛蒜皮的事吵架。老二媳妇嫌老五媳妇吃完饭不刷碗,老五媳妇嫌老七媳妇不打讲
究卫生随地吐痰,老七媳妇嫌老二媳妇的做的饭太咸\dldots

后来请吴用去给分家,约好老父亲一家呆一个月,六月份阮小二,七月份阮小五,八月份阮小七,为了防止反悔,特立字
为据,签字画押。

七月份过完才发现当年是闰七月,老五媳妇说他们已经养了一个月,该轮到老七家,老七媳妇拿出当初签的字据说非得
等八月份才肯接老父亲过去\dldots

不论晁天王还是宋大哥,都嫌这事太丢人,不愿意管,吴用当初也签了字,不愿意自己抽自己脸,只好说得了痔疮在家养
病。

阮老爷子天天在山前大骂瞎了眼生了三个白眼狼\dldots
\section{}

今天来说说鲁智深,鲁智深号称梁山泊三大猛人之一,另外两个一个是武松,一个是我。

不过,我对把我和鲁智深相提并论一直很不满,觉得跟他齐名是对我本来就不高的智商的侮辱,所以每当别人提起他
时,我总是强调:别在我面前提他,我跟他不熟,谢谢!

我虽然极力跟他拉开距离,但不知为什么,在别人眼中,我们仍是一路货色,但我认为,我跟他还是有本质区别的。

我莽撞是因为做事不计后果,由着性子胡来,他莽撞纯粹是没脑子,三岁小孩都能把他忽悠的团团转。

武松曾对他做过简短评价:实在。

但我觉得,用另外一个字来形容更加贴切:蠢!

看看他做的那些鸟事,我都替他汗颜。

在渭州当提辖时,被金翠莲父女忽悠,金翠莲本是郑屠的小妾,因争风吃醋被郑夫人赶出家门,这事无论怎么说都是原
配和小三之间的家庭恩怨,不干别人事。可是金翠莲一番哭诉,这厮就头脑发热,找郑屠算账,结果下手没轻没重,失手
把对方打死了。郑屠虽不是啥好鸟,欺男霸女无恶不作,但人家有合法外衣,是关西著名企业家,跟州府关系密切,光天
化日之下被打死,社会影响极其恶劣,经略相公也救不得他,只能跑路。

苍天有眼,这厮在雁门县又碰到金氏父女,结果还没等弄明白金翠莲拿着他白送的银子为啥不回东京反而跑到雁门县
当小三,就被三言两语忽悠去五台山当了和尚。

在五台山屁股没坐热,就被智清长老忽悠去大相国寺去当什么执事僧;到了大相国寺,执事僧没当上,被智真长老三言
两语打发去看守菜园子;华州救史进时,被鸟太守忽悠的连兵器都主动扔了\dldots

被忽悠一次,可能是大意,但次次被忽悠,说明这个人脑子有问题。

林冲娘子遭人调戏那次,他的表现也让人目瞪口呆。

但凡妻子被流氓调戏,无非是丈夫在发飙,``我要杀了你'',丈夫的朋友在一边劝,``算了算了,反正没进去'',流氓抱
头鼠窜,``误会误会,再也不敢了''。

到了鲁智深这里就乱了套了:高衙内抱头鼠窜,林冲在一边劝,鲁智深在发飙,看热闹的都糊涂了,到底谁家娘子遭人调
戏了? 
\section{}

张顺的爹病危

我跟张顺是老乡,平常关系不错,多次从他关卡偷偷下山,他总是睁一只眼闭一只眼,这种时候自然得去看望。

听说他爹喜欢吃蜜枣,本打算弄两斤,去宋氏酒楼一问,得二两银子一斤,忒黑!我拼死拼活砍个人头奖金不才二两?

况且这个月俸禄只剩五两,心想还是留着他爹死后给凑个整数吧!但又不能空手去,只好去山上采了些野蘑菇趁手。

张老爹已经病入膏肓,只有进的气,没有出的气。

很多兄弟在,我看了一下,大体分两种:一、是张顺的部下,二、是爹还没死的。像我这种死了爹又不归他管的人很少。

众兄弟有拎野味的、有提海鲜的、有送水果的\dldots 唯有鲁智深两手空空,这厮倒很是大方,甩给张顺十两银子,张
顺哪里肯收,再三推辞

鲁智深急了,冒出石破天惊的一句话:反正你爹就这两天的事了,就当我随礼了,行不?

众人目瞪口呆,哑口无言。

张顺手僵在半空,递出去也不是,缩回来也不是。

我看场面有些尴尬,想圆圆场,搞活下气氛,接过来说道:银子先放我这边吧,等过两天给你一起随。

估计张顺也晕乎了,看到坡立马往下滚:那也好,先放你这里。

刚说完,两人都觉得不大对劲,一琢磨,这是说的啥话啊!张顺和我大眼瞪小眼,无语了。

不知道张老爹是被我们气的还了魂,还是回光返照,颤巍巍坐起来,问道:你那边有啥亲人没?

鲁智深赶紧说他全家除了他都在那边。

张老爹说那就好办,银子他走时捎给鲁智深家人,他死后随礼钱另算。

鲁智深哭丧着脸说,那好那好,张老爹直接捎过去能省不少买路钱。

\section{}

宁跟明白人打一架,不跟SB说句话,这是众兄弟对鲁智深的评价。

我有时怀疑,这厮小时候是不是脑袋被驴踢过,满脑子浆糊,而且口无遮拦,说话不经过大脑,一句话能噎死一群人。

刚上山时,见林冲第一句话就是,阿嫂如何?

林冲当时脸刷的一下就黑了,冷冰冰的说:``死了''

鲁智深锲而不舍``怎么死的?''

林冲气呼呼的说:吃饭噎死的!

``吃鱼还是吃肉?''

``\dldots ''

有次喝酒,吴用喝多了,又在吹嘘自己学富五车、才高八斗、谋略过人\dldots

领导吹 牛 逼 吹到忘我时,正是下属们拍马屁的最佳时机,众人抓住机会,纷纷附和,你一言我一语,什么肉麻话都出
来了,什么空前绝后啦,什么千古文豪啦\dldots 乐和拍的最有水平,说什么``天下才分一石,吴军师独占九斗半'',这
个马屁无论力度还是着力点均恰到好处,众人无不叹服

吴用一时间飘飘然,嘴笑的都合不拢\dldots

眼看晚宴就功德圆满,这厮没头没脑的来了句:军师文采这么高,怎么连个举人都没中?

这一砖直接把吴用从云端拍到了人间,阴沉着脸不说话,宴会不欢而散\dldots

这样的例子数不胜数,众兄弟没一个不骂他脑袋缺根筋,要不是他武功高强,估计早就被黑了

\section{}

快到中秋了,开溜的兄弟越来越多。

也不怪兄弟们意志不坚定,毕竟强盗也是人,也是妈生爹养的,长年累月在外飘荡,生死连个信都没有,谁不想中秋时跟
家人团聚一下?好歹报个平安,省的猴年马月回家时自己坟上草都老高了,老婆领着孩子站在旁边,唠叨着说这土里埋
的虽然不是你亲爹但胜似亲爹\dldots

但强盗这个行业特殊,散开容易,聚起来难,所以不论哪个山头都有同样的规定:私自下山者,杀无赦!

为了阻止弟兄们开溜,晁天王和宋大哥捐弃前嫌,一致对内,毕竟再这样下去,两人就成光杆司令了

晁天王的长项是打劫,宋大哥的长项是拉拢人心,吴军师卖弄文骚别有一套,三人均对当前情况束手无策,只好召开紧
急扩大会议,堂级以上干部参加,要求大家出谋划策。

杀人放火我在行,遇到动脑筋的事,我从来都是只听不说,因为我自己几斤几两心里还是清楚的。

公孙胜献计,看守法:在河边设置岗哨,结果,还没等半夜,岗哨自己溜了。

武松献计,连坐法:分小组,十人一组,若一人逃跑,剩下的九人都砍头,武松很天真,以为人人都跟他一样讲义气,不肯
连累兄弟。大错特错,当强盗的连爹妈都肯连累,还会在乎兄弟?这个法子仅用了一天就不得不紧急废除,照这样下去,没
等跑光倒先砍光了。

王矮虎受武松启发,心生奇计,提议捆绑法:每个组选个组长,睡觉前把剩下的九人绑起来,结果第二天一早,组长跑了
\dldots

孙二娘献上一计,押金法:扣发当月俸禄和奖金,等过了节补上,不过这法子都老掉牙了,屁用没有。你想想,在山寨里
干,干一票拿一成,剩下的悉数上交。自己拉出去单干,干一票顶在山寨里干十票。那点破工资,还不够塞牙缝的,没人
稀罕。

鲁智深献计,吓唬法:抓住逃跑的人,砍了头挂在桅杆上示众。这个也没用,出来当强盗的,都是看惯生死的主,都信奉
脑袋掉了碗大的疤,二十年后又是条好汉,没人把命当回事。

眼见人一天天减少,晁天王和宋大哥一个劲的上火,嘴唇都起泡了,仍无济于事。

\section{}

今天厅级干部开会,晁天王下了军令状,谁部下再开溜,就砍谁的狗头。

花荣是我的上级,回来就召集我们堂级干部开会,杀气腾腾的说:我的狗头保不住了,你们的狗头也别想保住。

我突然想到一个问题,那厅级干部跑了,砍谁的头?不过张了张嘴,没敢问。

散会后我立马召焦廷、鲍旭和李兖开会,他们三人是我统辖的地级干部,照例把领导指示传达一遍,这三个鸟人吓的汗
都出来了。

散会后三人一溜烟跑了,肯定是去召集下属开会,不用猜我都知道他们说什么\dldots

我以前总觉得自己不够沉稳,遇事慌里慌张,很羡慕那些办事沉稳的人。

现在我算明白了,火烧房顶还在喝茶聊天的,烧的必定不是自家房子,孩子掉到水井里还不急不躁的,肯定不是自家孩
子。

一贯沉稳的晁天王这两天变得急躁,走路四平八稳的宋大哥,现在也开始小跑。

我想,若能想出办法解了山寨的燃眉之急,那我这次升厅级干部的事就十拿九稳了。

不过我这榆木脑袋肯定想不出,但我知道有个人肯定有办法,朱武。

朱武平常独来独往,不显山不露水,天天一副憨样,表态时从来都是双手赞同,发言时总是高呼领导英明,提意见时从来
都是领导注意身体之类的话\dldots 从不表露自己的真实想法。

但我认为,山寨里最聪明的非他莫属,其谋略比吴用高出一筹,其心机甚至比宋大哥还要深\dldots

不知咋回事,吴用似乎不大喜欢他,看他的眼神总是怪怪的,还处处给他难堪。

宋大哥似乎也不怎么待见他,提起他来总是冷笑\dldots

领导不喜欢的人,众兄弟也都识趣的离得远远的。不过我经常去找他,倒不是我喜欢他,而是没办法,我不识数,后勤处
那帮王八蛋经常克扣我俸禄,所以我每次去领俸禄总是喊他一起,好给我把把关。

有次我下山抢了二百两银子,提成按一成算,竟然给我算成十两,幸好朱武在,当即指出数目不对,重新算了一下,应该
是十五两。那次他干了票小的,抢了五十两银子,领了十两提成,我又送给他一两当做感谢。

这厮正喝着小酒,唱着小曲:虎为百兽尊,百兽伏不动,若逢狮子吼,虎又全没用\dldots

那表情,很享受,像射了似地,或者正在射似地。

我说现在领导们束手无策,都准备拿行李散伙,你有啥好办法?

朱武说,士兵们逃走,无非是觉得在山寨因为没有希望,只有给他们希望,他们才会留下来\dldots 唯一的办法就是:招
安! 

\section{}

当我兴冲冲的跑去告诉宋大哥时,宋大哥脸上阴晴不定,问是谁想出来的,我不想出卖兄弟,就说我自己想出来的。

宋大哥不信,这个正常,换做是我,我也不信。一番威逼利诱,又讲了一通大道理,我还是不招,宋大哥最后问我兄弟是
干啥用的?

我虽然粗鲁,但还是很讲义气的,说兄弟就是两肋插刀,肝胆相照

宋大哥摇摇头说,所谓兄弟,就是平常在一起喝酒解闷,关键时刻用来出卖的。

我一想,似乎有点道理,读书人看问题就是不一样,就把朱武卖了。

宋大哥冷哼一声,说果然是这厮,那一刻,宋大哥眼中似乎有杀气。

出来后,心中越想越不对,我跟宋大哥是不是也是兄弟?\dldots

聚义厅贴出告示,中秋节后,朝廷会招安,到时候兄弟们都加官进爵,荣归故里,这一招果然奏效,不但没人再开溜,以前
开溜的又跑了回来,梁山泊一时人满为患。

\section{}

今天宋大哥请客吃饭,自然少不了我,不过他没请我,是我主动去的,反正是公款,不吃白不吃,吃了也白吃。

赴席的有吴用、蒋敬、乐和、顾大嫂和王矮虎等一干鸟人

以前无论谁请客,吴用必不可少,请他吃饭的人都挤破门槛,他每天不是正在酒场上,就是正在赶往酒场的路上,号称梁
山泊头号饭桶。

我很是羡慕,以为是因为吴用有文化,懂礼数,跟兄弟们感情好。后来山寨高层领导职务调整,吴用专管兵马调配,钱粮
报销划归蒋敬,吴用就不吃香了,现在逢宴必请的成了蒋敬。

吴大饭桶郁闷了好长时间,天天嘟囔世态炎凉、人心不古\dldots 诅咒蒋敬吃鱼卡死、吃饭噎死、喝酒呛死\dldots

不过我很高兴,终于解开了心中的疙瘩:别人不请我不是因为我上完厕所不洗手,也不是以为我吃饭不用筷子,更不是
因为我长的丑,而是因为,我不在那个位置上!

有时我恶毒的想,是不是放条狗在那个位置上,也会如此受欢迎?

\section{}

顾大嫂显然有备而来,脸上厚厚一层白粉,黑底白边,像是驴粪蛋上涂了一层霜,脖子上挂串指头粗的金链子,活脱脱一
个地主婆。

王矮虎悄悄告诉我,那串金项链是假的,我问他怎么知道,他说顾大嫂洗澡时,金链子竟然漂在水面上。我觉得有些不
对劲,但哪里不对劲,又一时想不起来

顾大嫂虽然刚上山,但人缘极差,很多兄弟都恨她,而且不是一般的恨,是恨之入骨的恨。

事情还得从攻打祝家庄说起,那时,宋大哥刚上梁山,对周围敌情不熟,被晁天王忽悠去打祝家庄。

宋大哥眼高于顶,自诩熟读兵书,拿下小小祝家庄不在话下,带着一干小弟屁颠屁颠去了。

没想到祝家庄是块硬骨头,众兄弟被打的屁股尿流,宋大哥先锋印也扔了,兵器也不要了,骑马狂奔三十里,鞋子都跑丢
了一只,差点做了俘虏,幸亏我及时赶到救了他。

不过宋大哥每次提起这茬时,总说那是诱敌深入,真他妈没意思,逃跑就是逃跑,还装什么大头蒜!不过宋大哥有这本
事,打死不认,那怕被人堵在被窝里,也会大喊一声:老子还没进去!

山寨的士兵大多都是地痞流氓,什么阵法、战法根本不懂,只知道举红旗时一窝蜂的往前冲,举白旗时扔了武器就溜
\dldots 打群架还可以,打仗基本没戏

我们屡战屡败,最后吴用出了个绝妙主意,在军中散布流言,说祝家庄的女人个个前凸后掘、貌美如花。最后宣布,攻
破祝家庄后,谁抓到的女人算谁的。同时派顾大嫂潜入祝家庄做内应。

消息一公布,兄弟们立马跟打了鸡冠血似的兴奋,再次交战,个个奋力杀敌,死战不退,时迁够爷们,被打断了两条腿,还
不停的往里爬\dldots

当众兄弟浑身血污挺枪冲进去时,晚了一步,顾大嫂不知是嫉妒那些女人漂亮还是想独自伺候众兄弟,已经把祝家庄下
自八岁上至八十岁的女人全砍了

那时众兄弟还不认识顾大嫂,满腔欲火无处发泄,一下子炸了营,蜂拥到中军大帐,扬言要轮了顾大嫂。

宋大哥和吴军师急得团团转,手下这帮流氓已经红了眼,估计弄几头大象来都不能幸免。两人正在考虑要不要抓个烟
花女子来顶包平息众怒时,顾大嫂自顾自爬到中军大帐顶棚上,面对黑压压的挺枪流氓,一声狮吼:老娘就是顾大嫂,听
说你们想轮我,来啊,你们是一个一个来,还是一起来?

众兄弟抬头一看,只见一中年妇女威风凛凛,叉腰而立,黄牛眼、鹰钩鼻、招风耳\dldots 要多难看有多难看

众兄弟大倒胃口,欲火顿泄,垂头丧气的离去。

王矮虎当众大喊:你他妈脱光衣服追八公里,回头看一眼算我流氓\dldots

一个在山寨呆了十几年的老兵,冲锋时最勇猛,被打的也最惨,头眼歪斜,浑身血窟窿,奄奄一息,说死前有最后一个请
求,摸一下女人,也不枉白活一生。

顾大嫂同情心大起,放言摸哪都行,满脸悲壮的把身体凑过去,老兵只看了一眼,登时死绝,临终遗愿:我操,白日见鬼了。

顾大嫂成了全山寨男人的公敌,众兄弟喝醉酒就问候她姐她妈她姥姥,惟独不敢问候她本人。

\section{}

初次喝酒,客套话是免不了的。

宋大哥领了三杯酒,第一杯酒,祝愿当今天子福寿安康,高俅等四奸臣断子绝孙早日死光光。

这是宋大哥每逢喝酒必说的一句祝酒词,我每次听到后都忍不住想笑,真不知道皇帝老儿听到后会有啥感想。

第二杯酒,欢迎顾大嫂加入前途无量的强盗行业,说这个行业是撑死胆大的饿死胆小的,鼓励她硬起心肠,努力杀人,使
劲放火\dldots

顾大嫂说她以前只是个开酒店的,隔行如隔山,对这行的规矩不大熟,希望宋大哥以后多多指教。

第三杯酒,欢迎顾大嫂在竞争激烈的山头中选择了梁山,宋大哥列举了梁山近几年的业绩,打败了曾头市、祝家庄等朝
廷认证单位,先后兼并了清风寨、二龙山、桃花山等山头,下一步的目标是争取得到朝廷认证

顾大嫂感谢梁山能够收留她,给她一个展示自我能力的机会,发誓一定不辜负宋大哥期望,早起抢劫,晚睡偷窃,尽早成
为一名合格的强盗酒过三巡,菜过五味,你来我往,酒越喝越高,一群人开始瞎扯。

喝醉了有个好处,可以趴在那里静静的看别人的丑态

蒋敬搂着吴用的肩醉醺醺的说,他对山寨的安排也很意外,不是故意抢吴用饭碗,吴用一脸真诚的说他本来管事就多,
忙不过来,蒋敬帮他分担他正好清闲\dldots

王矮虎拉着乐和的手痛说女人不可靠\dldots

顾大嫂称赞宋大哥仗义疏财、义薄云天,宋大哥恶心死人不偿命,称赞顾大嫂福态端庄、心广体胖,我一时忍不住去外
面吐的昏天暗地\dldots

等回屋时,宋大哥又喝高了,这厮贼性不改,喝高了就要作诗,想起来就来气,要不是他乱作诗,估计我现在都升成院长
了

王矮虎把早就准备好的纸笔递过去,满脸谄笑

宋大哥借着酒劲作诗一首,``星空很蓝,一道银河分两边,这边是天,那边也是天\dldots ''

吴用率先叫好,说此诗对仗工整、意味深长,有李太白之风

顾大嫂和王矮虎也齐声称赞

乐和当即谱上曲编成歌

我心里暗骂王矮虎:你他妈大字不识一个,懂个屁,就知道拍马屁!

晚上临走,我把诗要过来,说回去好好研究研究,宋大哥很高兴,拍着我肩膀说年轻人要好好学习,我心想老子我都快奔
三的人了,还学个屁。不过话到嘴边变成:那是那是,我一定不辜负大哥栽培!

出来后径直去了朱武那里,这厮正在看春宫图,我把诗递给他,让他点评一下

朱武瞟了两眼,骂道:这是什么狗屁诗词,要对仗不对仗,要押韵不押韵,文不文古不古,简直狗屁不通\dldots

骂了半天,这厮才似乎感觉哪里不对,问是哪个鸟人做的?

我说是宋大哥那鸟人作的

这厮楞在原地足足有一分钟,嘴巴张的大大的,像濒死的鲶鱼,低头把诗词又看了一遍,突然一拍桌子,惊道:哎呀!刚刚
一时迷糊没看出来,这首诗词果然奥妙无穷,表面浅显易懂其实内藏玄机,不但蕴含哲理,而且\dldots

我哈哈大笑,告辞出来,抬眼看天,今夜的天空的确很蓝

\section{}

今天聚义厅开会,晁天王提了两项建议:一、中层干部调整延期到中秋节后,二、中秋节期间严禁送礼。

第一项提议老掉牙了,意料之中,每次临近过节,总是传言要动干部,弄的人心惶惶、个个上蹿下跳。

我现在是堂级干部,前几天又立了一功,估计这次提拔厅级干部的事有戏,不过不能大意,得抓住这次机会好好加把劲。

送点什么好哪?

朱武那厮曾说过,送礼是门高深的学问,虽然人人都学过,但大部分人只知皮毛,极少有人能掌握其精髓。

送礼一定要恰到好处,十两银子办到的事,你三两,诚心恶心人,送九两,功亏一篑,送二十两,明显不划算。

朱武将送礼分三种境界,最高境界:雪中送炭,中等境界:锦上添花,最下境界:适得其反。

雪中送炭,就是缺什么送什么,掉井里送绳子,掉海里送木头,他一定会铭记在心,这是最高境界,不花钱,但能让对方记
你一辈子。据朱武说,达到这种境界的人万中无一,他所认识的人中只有一人,那人叫高球,现在已经是太尉,总管全国
兵马调配。

锦上添花,别人送女儿红,你也送女儿红,别人送狗肉,你也送狗肉,送了等于没送,没啥新意。大多数人只能停留在此
种境界,一生难以突破。

适得其反,人家孩子前脚被狼撕了你后脚就去送糖葫芦,老娘刚被和尚拐跑了你就去送贞节牌坊,还特别强调是纯金
的,钻石的都没用!这就是典型的没事找抽型的,鲁智深就是这个类型,本来还想不起你这号人,自己倒贴上去了。

我听后大受启发,真是听君闲扯淡,胜读十年书。

晁夫人这两天总是跳脚骂娘,晁天王唉声叹气,估计是内功不行,送两只千年王八给补补。

宋大哥那边不知缺点啥,这两天得找宋青摸摸底,得送到点子上,不能花冤枉钱。

吴用这次就不送了,这次调整干部他说不上话\dldots

晁天王脸皮也忒厚,提第二项建议时,摆出一副大义凛然的鸟样,瞪着眼睛,拍着桌子,警告在坐的干部中秋期间禁止收
礼,否则严惩不贷。但没人当真,说的人当放屁,听的人也当放屁,谁当真谁是棒槌。但棒槌还真有,比如刚上山的顾大
嫂,会后逢人就说上梁山是她这辈子最正确的选择,没想到梁山是如此的清明\dldots 本想点播点播她,但突然想到他
丈夫孙新也在提拔的红线上,也就忍住没说\dldots

宋大哥今天很奇怪,以前凡是晁天王赞成的他都反对,今天不但没反对,反而补充了两条提议,中秋节期间禁止打架动
刀,严禁喝醉酒后下山强奸妇女。

很多兄弟上山前都是当地一霸,横着走惯了的主,脾气火爆,凑在一起一言不合就大打出手,为骡子他爹是马还是驴的
屁事都能打一架。急了就动刀子拼个你死我活。比如鲁智深,见谁都自称洒家,这是关西话,就是``老子''的意思,见
面就自称老子,谁也不乐意,为这个口头禅打了无数架,跟杨志拼过、跟史进干过、跟武松单挑过,总之,打平了的都成
了兄弟,输了的都被他咔嚓一刀结果了,要不是他武功高强,估计早就被打的生活不能自理。

山寨屡禁不止,后来只能规定,打架可以,但不能动刀。

逢年过节,山寨必有酒会,很多兄弟憋了大半年,喝醉后就跑下山伺机强奸良家妇女。

每次过完节,山寨消停了,山下热闹了,方圆几十里的女人哭喊着上吊自杀的,寻死觅活跳河的\dldots 我一直搞不懂,
她们是晚上被强奸的,为啥非得等到大白天的再嚷嚷\dldots 想不明白

宋大哥嫌坏了山寨``替天行道''的名声,想了很多办法,还是禁不住\dldots

这帮鸟人,改天让我碰到非剁了他们不可,最恨这种欺负妇女的流氓了\dldots

\section{}

今天请宋青吃饭,在宋氏酒楼请的,点的最好的菜,霸王别鸡、蚂蚁上树、活烧猴脑\dldots 酒是三十年的绍兴女儿红,够
味道

其实我很烦这厮,屁本事没有,而且笨的要命,不过今天有求于他,得打听打听宋大哥缺什么,中秋送东西也好送到点子
上,只好硬着头皮跟他扯淡。

这厮刚上山时,宋大哥想把他送到战场上锻炼锻炼,日后也好提拔重用,各将领纷纷推辞,这也难怪,刀枪无眼,这厮武
艺又平常,万一有个闪失,谁也负不起这个责任。

后来宋大哥找到我,我本想推脱,宋大哥又是套近乎又是拉关系,说了一番掏心窝的话,眼泪都流出来了,我开不了口,
只好应承下来。

虽然答应下来,我可不敢冒险让他去冲锋,只是让他当伙夫。

伙夫虽然累点,但没啥危险,早起做做饭,晚睡刷刷锅,还可以打打偏手,看谁不顺眼就往他碗里吐口唾沫,当兵的也不
讲究,只要吃不死人就行。打仗时跟在后边吆喝两声,吃了败仗溜的也快。晁盖的小舅子,花容的外甥都在当伙夫。

没想到这厮伙夫都当不好,有次烧火把营帐给烧着了,气得我要死,要不是看在宋大哥面上,早就把他乱刀剁了喂了王
八。

后来这厮突生豪气,要求上阵杀敌,我想了想同意了,特地嘱咐他不要冲的太靠前,到时跟在众兄弟后面,偷偷往死人身
上插两只箭,为防止兄弟们争功,箭上都标着名字,我再跟执法处杨雄打声招呼,到时候都算在他头上,给他记一大功,
然后风光的调离部队,也不负宋大哥所托。

结果,这厮是个软蛋,上阵前豪情万丈,放言万军中取上将首级如探囊取物,上阵后,看着黑压压的人群,立马就憋了,还
未等冲锋,腿都软了,瘫倒在地,拉都拉不起来。一仗下来,刀未出鞘,箭未出鞍,腿倒折了,被自己人踩折的。

后来安排他当更夫,晚上按点敲敲鼓,不费神不费力。

一次,我跟武松、鲁智深约好五更去偷袭敌人,结果这厮把更敲错了,三更就出发,阴差阳错,碰到打埋伏的敌人,一场
混战,把敌人打跑了,众兄弟强烈要求给他记一功,让他赶快滚蛋。我借机给记了一大功,调回梁山当后勤,专管安排酒
席。

哎!有时候想想,佛祖真是公平的,有的人给了满腹韬略,有的人给了一身武艺,有的人给了溜须拍马的本事,那些百无
一用的,不是有个好爹就是有个漂亮的妈,要不就有个好哥!

喝完酒又请他掷骰子,二两银子起底,上不封顶,本想故意输给他,结果我开大,他猜小,我开小,他猜大,没半个时辰,我
倒赢了二十两,这厮脸色立马黑了下来。

我也急了,最后一局,二十两都押上,心想不管怎样都要把银子输给他,为防意外,我偷看了一眼,是大,结果他又猜小。

我想点播点播他,说道:你确定?要不咱再改改?

这厮咬牙不松,还是猜小。

我最后实在忍不住了,说道:宋青,这次真的是大,你就猜大吧!

这厮一脸欠揍的表情,说什么我是跟他玩虚的,是诈他,咬定是小

我脑袋发热,心一横,他娘的,厅级干部老子不当了,当下开了,揣着四十两银子就走了

见过笨的,没见过你他妈这么笨的!笨就笨吧,还自以为聪明!是可忍,孰不可忍。

\section{}

梁山泊的头领共分四级,晁天王、宋大哥、吴用和公孙胜是一级,下面依次是厅级、堂级和地级。

开会时厅级干部在大厅里坐,堂级干部在大堂里坐,地级干部只能坐在地上。

晁天王等四人在聚义厅附近有专门的房子,四周把守的都是晁天王的心腹,不过宋大哥很少在那里住。

从山下到山上分三关,厅级干部住在一关内,堂级干部住在二关内,地级干部住在三关内,没级别的兄弟只能睡山下的
通铺。

在山寨,由小兵升地级容易,由地级升堂级也容易,只要你有武艺,够狠,打仗不要命,很容易坐到堂级。但由堂级升厅
级很难,厅级干部不但要武艺高强,还要出身好,又要有一定的威望,如林冲、秦明、花荣\dldots 以前都是朝廷军官,
都是响当当的人物。

厅级是个门槛,山寨规定,只有厅级以上干部才能娶亲。

娶不娶亲倒无所谓,主要是每次开会都坐在人后,看人后脑勺,听人吹牛逼,还闻人放臭屁,感觉忒不爽!

宋大哥曾说过,当强盗不可耻,可耻的是丧失追求,所以我一直想当个有追求的强盗,梦想着有朝一日能当上厅级强盗

对某些东西看的过重,就容易患得患失,为了早日实现追求,我打仗冲锋在前,从不惜身。平常还要讨好领导、领导他
老婆、领导他弟弟\dldots 忒他妈累

当初以为当强盗能轻松些,快意恩仇,大碗喝酒、大块吃肉、论称分金银\dldots 看谁不顺眼就给他一鸟斧,现在倒好,竟
然给宋青这样的憨货当孙子\dldots 忒他妈的窝囊

刚刚碰到晁天王的小舅子,这厮平常仗着有他姐夫撑腰,在山寨横行霸道,到山下顶着梁山好汉的名头白吃白喝,嫖完
不给钱,吃完抹嘴就溜,口头禅是:你知不知道我姐夫是谁?

平常除了厅级以上干部,谁见面都得叫我一声黑哥,这家伙倒好,张口闭口``黑厮'',妈的,黑厮也是你叫的?找个机会
看我怎么收拾你!

老子我现在不在乎了,啥厅级干部、堂级干部,老子不稀罕当,惹恼了我,反下山去,投别的山头去,此处不留爷自有留
爷处,反正我这种天才强盗到哪里都吃香
\section{}

八月十五

每年这时候,晁天王、宋大哥和吴用无论发生什么事,就是天塌下来,都会在家候着。

那套说辞我都听出老茧了:怎么能这样?不是说不让送礼吗\dldots 再这样我就生气了啊\dldots 那放这吧\dldots 下不
为例啊\dldots

今年稍有不同,晁天王和宋大哥照例在家候着,吴用拿个板凳坐在去聚义厅的必经之路上,见人就笑容可掬的问:来送
礼啊!

这可苦了众兄弟,今年干部调整他说不上话,大家都没准备他的礼,绕又绕不过去,真是一夫当关万夫莫开

很多兄弟没辙,只好回去重新准备一份。

时迁手脚灵便,把礼物拴在腰里,从后山攀着悬崖上去,送完后又悄悄攀着悬崖溜了

谢珍、谢宝哥俩也想学时迁从后山攀上去,没想到没那两下子,爬到半山腰被树枝挂住了,上不能上,下不能下,又不敢
高声吆喝,挂在后山大半天,让山风吹的脸都裂了口子

燕顺拎了只巴掌大的乌龟来,一看就知道刚从河里捞的,说什么过节了来看看我,真贼娘,我跟他八竿子打不到一块,平
时也没啥来往,他升堂级干部时我没送贺礼,我妈死了时他也没随份,无事献殷勤非奸即盗。

他那点花花肠子我清楚的很,我们两人都是厅级干部的热门门选,这厮肯定是探风来了,果不然,话没说两句,就问我今
年送啥礼。

我想反正得罪宋大哥了,也不想再凑那热闹,干脆做个顺水人情,就说领导都他妈的棒槌,送给领导还不如送给条狗,顺
手把两条千年王八送给了他

这厮拎着王八一脸感动,连说黑哥够义气,够爽快,我想要是我继续给你争,你指不定背后骂我啥那!

\section{}

中秋之夜

山上山下披红挂绿,关内关外喜气洋洋

众兄弟扶老携幼依次落座,众人就座毕,鼓掌欢迎晁天王、宋大哥、吴军师落座。

晁天王黄袍黄马褂,宋大哥身披朱红甲,内着青锦袄,吴军师青缁灰袍,头戴秀才帽,绿的。

晁天王领第一杯酒,深情缅怀往事,追忆了七人智劫生辰纲,上梁山火拼王伦,壮大山寨的艰辛\dldots

宋大哥领第二杯酒,祝那个恨他恨到牙痒痒的当今皇上,洪福齐天,万寿无疆,回忆了大闹清风山、江州劫法场、三打
祝家庄的光辉事迹\dldots

吴用领第三杯酒,号召大家团结在晁宋两位头领的周围,努力打家劫舍,尽力杀人放火,将强盗事业发扬光大\dldots

下面依次是,厅级干部敬三位头领,堂级干部敬三位头领、地级干部敬三位头领\dldots

十八碗酒过后,开始串场\dldots

往年我还上前敬几位头领酒,表表忠心,装装孙子,今年谁也不敬,专心喝我的酒\dldots

众人很快喝多了,晁天王黄袍也脱了,宋大哥朱红甲也扔了,吴用绿帽子也摘了,个个喝的脸红脖子粗

众兄弟有摔倒在地爬不起来的,有当场吐了的,有唱十八摸的\dldots

人人看似很高兴,吆五喝六、举杯痛饮\dldots 说着自以为清醒的醉话,互相敷衍吹捧\dldots

我突然感到一阵阵的孤独,想起朱武说过的一句话,狂欢是一群人的寂寞,那时我笑他装逼,现在想想,似乎有点道理。

抱着坛酒,揣只猪腿,离开嘈杂的大厅,找了一个无人的山头,坐了下来

不远处鲁智深正吐得昏天暗地\dldots 吐着吐着,竟然开始哭泣\dldots

吴用喝醉了,站在聚义厅门口,对着替天行道的大旗,大声的念叨:十年寒窗无人问、一举成名天下知\dldots 没想到我
吴用竟然也成了强盗\dldots

林冲靠在树旁,抬头望天\dldots 月光下,他棱角分明的脸上,泪水止不住的滑过,一滴一滴,接着汹涌成河、、他在想
什么哪,是不是想起了惨死的林娘子\dldots 是不是后悔当初的软弱\dldots

武松躺在草坪上,双手叠在脑后,怔怔的盯着月空,一动不动\dldots 这个面无表情,心如冰石的好汉,是不是也想起了
如风往事?\dldots 是不是想起了他那个窝囊一生的哥哥?想起了那个被他亲手杀死的风情万种的嫂嫂?

再坚强的男人,心中都有一块禁地,不许任何人触摸\dldots

你不能问吴用为啥他自诩学富五车才高八斗却没能中举人

你不能问林冲他娘子被高衙内凌辱时用的是老汉推车还是隔山取火

你也不能问武松他杀潘金莲时心中到底有没有过一丝心痛\dldots

碧空万里,月光皎洁,桂花树是那么的清晰,仿佛触手可及\dldots 我想起了小时候,母亲抱着我和哥哥在天井的葡萄架
下赏月,我指着天空问,哪个是牛郎、哪个是织女\dldots 母亲不厌其烦的一遍遍解答\dldots

如今,物是人非,当初的小孩变成了杀人不眨眼的恶魔,当初慈祥的母亲,已经去了天国\dldots

母亲曾说过,人死后,会成为天上的星星,可如今,天上星星如此之多,我怎么分得清是哪颗\dldots

\section{}

盛宴结束,晁天王走了,宋大哥走了,吴军师也被人搀回去了

众兄弟该散的散,该撤的撤,几个贼头贼脑的家伙又偷偷向山下溜去\dldots

我提上板斧,朝山下走去

今晚,我要替天行道

没想到半路碰上了武松和鲁智深,三人相视一笑,都是同道中人\dldots

我和鲁智深经常笑,是那种没心没肺的哈哈大笑,蚂蚁打架的屁事都能让我们乐半天,兄弟们说蠢人都这样\dldots

而武松,自从上山,从没有见他笑过,总是一副拒人千里之外的冰冷模样,我虽然敬服他人品,但那副鸟脸一直看着不耐
烦,要不是打不过他,早就给他两巴掌\dldots

今天,他笑了,但依然冷气逼人。我想,可能是他太聪明的缘故

人一旦聪明了,很多事就看的透彻,也就失去了快乐

三人埋伏在金沙滩外。

宋大哥已经下了禁山令,今夜所有兄弟不得外出,现在偷跑出来的,非奸即盗,杀之有名

不多久,一个家伙哼着小曲过来了:

不吃苦、不受累、良家妇女咱白睡;不花钱、不受罪、免费的小酒天天醉;怡红楼、翠红院,所有的姑娘都白干;揍他
爹,日他娘,谁让咱姐夫是晁天王

冤家路窄,这厮是晁天王的小舅子,最近几年被他糟蹋的黄花大闺女不计其数,而且口味特重,不挑不捡,弄的十里八庄
的老太婆都不敢出门\dldots

武松跳起来把戒刀架在他脖子上,这厮吓了一跳,当场跪倒,大喊爷爷饶命,银子在口袋里,不够写个绑票跟我姐夫要

三人看着他不说话,这厮抬头一看是我们,胆气立马壮了,拍拍屁股爬起来,怒气冲冲的问了三个问题:你们是不是找死
啊?知不知道我姐夫是谁?你们还想不想当强盗了?

我觉得他这话问的太蠢,如果他平常多花点时间了解一下我们中随便一人的过去的话,就不会问这么多废话

武松回答一个问题给一刀

``是'',砍掉左腿

``知道'',砍掉右腿

``想'',砍掉了脑袋

我真为这厮可惜,多问了一个问题。
\section{}

当夜,血流成河,尸横遍地。

我平生别无爱好,唯好喝酒、赌博、杀人,吹牛逼不算,因为我觉得那只是个习惯,不算爱好。

喝酒,可以让我忘记很多烦恼,能够想明白很多清醒时想不明白的事情,虽然清醒后依旧糊涂,但至少,我知道,我曾明
白过。

赌博,是件很有意思的事情,开盘前,人人大呼小叫,连连加注,都自以为胜券在握,开盘后,赢者兴高采烈,输的垂头丧
气。

我喜欢开局前那短暂的沉寂,当你把所有俸禄都押上时,你不知道你下个月是吃肉,还是喝西北风\dldots

杀人,在江州劫法场前,我从没有想过,并且对杀人充满着深深的恐惧,我虽然没文化,但也知道杀人偿命。我不怕死,
并不代表我可以随便去死。

但很多事,有了开始,就很难再结束,当我杀第一人时,看到他看我的眼神由不屑变成恐惧,我感到莫名的兴奋,从那时
便喜欢上了杀人,一发不可收拾。

鲁智深趴在树顶望风,五更时分,再无人影,正准备回山,恰好碰到王矮虎从外面回来,怪不得这厮喝酒喝到一半就溜
了,原来又去做这种勾当

王矮虎为人好色,是众兄弟最看不起的一个,在清风山时,就是方圆几十里的色魔,而且有个怪癖,喜欢吃动物老二,现
在山上的猪牛羊基本都成了太监,满山的动物见到他就夹紧大腿,战战兢兢,他在畜生界的威望比宋大哥在梁山的威望
大多了。

武松等人一直不肯跟他同桌喝酒,嫌他埋汰。

王矮虎一看这阵势,立马明白咋回事,他虽然好色,但不蠢。他了解我,也了解武松,知道我们是那种一瞪眼连天王老子
都敢砍的人,当场吓的就尿了裤子,磕头如捣蒜

我正在想要不要留他一命,毕竟他跟晁盖小舅子不一回事,晁盖小舅子不过是个小混混,上不了台面,而他则是名正言
顺的堂级干部,是当初一起喝过结义酒对天发过誓的。

为时已晚,武松已经走了过去,武松有个习惯,杀人爱灭门,遇到一个算一个,不杀干净绝不罢手,血溅鸳鸯楼、大闹飞
云浦,莫不如此。

月光下,武松鬓影凌乱,面色冷峻,犹如天人,镔花刀闪着寒光

``今天你又强奸了谁家闺女?''

王矮虎急的双手乱摆:我没强奸谁家闺女\dldots 我是通奸\dldots 真的\dldots 通奸\dldots 跟东风屯的刘寡妇\dldots
不信你们去问她\dldots

他如果说强奸,有可能侥幸不死,但一说通奸,必死无疑,因为武松最恨通奸

``明年今天是你忌日!''

有些话,只有某些特定的人说才有气势,刚刚的话,换做是我,或者是鲁智深,也会有一定效果,但绝不会有武松那种摄
人心魄的气势。

武松举起了镔花刀,说时迟,那时快,一滩秽物从天而降,淋了武松一身。

鲁智深正在武松头顶的树上,吐了。

很多事,是需要心情的,我想武松当时的心情应该跟上次我在战前动员会上正讲的激情澎湃时鲍旭跟我说拉链开了时
心情差不多。

武松当下也不管王矮虎,喊了声``我操!''扔了镔花刀一溜烟跑到河边跳河里去了

王矮虎可怜巴巴的看着我,这等鸟人懒得杀他,让他滚了!

\section{}

这一觉睡的特别香,梦到娶媳妇,锣鼓喧天,鞭炮齐鸣,大花轿抬进门,拜完天地进洞房,新娘子一身红妆坐在炕头,正要
走过去掀开红盖头\dldots

门``哐当''一声被撞开,梦醒了,鲍旭慌里慌张的跑进来

我说鲍旭就是属三岁小孩的,刚要喝粥你拉稀,刚要上床你喊娘

鲍旭说,几十个兄弟被杀死在金沙滩,现在山寨炸了锅\dldots

我心里直乐,跟鲍旭来到金沙滩上,鲁智深和武松也在,躲在人群中冷笑

晁夫人不顾体面,抱着她弟弟残缺不全的身体大哭,大骂那个杀千刀的该死,说他弟弟三岁没了妈,六岁死了爹,八岁全
家死绝,要多可怜有多可怜,而且从小就特听话,从不偷鸡摸狗,看到女孩就脸红\dldots 到底哪个王八蛋如此狠心
\dldots

众人都摆出一副丈母娘死了的模样,哭又哭不出,笑又不合适,绷着脸,摇头叹息\dldots 哎!、、死的忒惨了\dldots

顾大嫂和孙二娘在一边劝,什么人已经死了,节哀顺变,保重身体之类的套话\dldots

晁天王小舅子死了,这可是山寨头顶大事,聚义厅开会,迅速成立专案组,吴用全权负责,限期破案\dldots

\section{}

戴宗来叫我,说宋大哥有急事找我

戴宗是我老领导,在江州时我当狱警,他当监狱长,关系一直不是太好,主要是这厮手太黑,有钱没钱敲一竹杠,蚊子都
能让他挤出摊屎来,逢年过节的份钱给少了就找茬,经常给我穿小鞋,弄得我苦不堪言,睡觉都问候他八辈祖宗。

那时他最常做的一件事就是,给我一文钱,让去买新鲜大鲤鱼,浔阳江的鲤鱼是稀罕物,一般都得两斤以上,一斤得200
文,每次我都倒贴钱,这还不算,每当我把鲤鱼给他时,他还问我找零了没有。

当我忍着一肚子气要把一文钱还给他时,这厮故作大方的挥挥手:不用给了,去帮我买瓶绍兴女儿红,要三十年的。

气得我每次都往鱼嘴里吐口水。

到宋大哥门口时,武松刚出来,眼睛通红,我想,能让铁石心肠的武松落泪的,也就只有宋大哥了。

在梁山,我最佩服两个人,一个是宋大哥,另外一个是武松

宋大哥自不必说,刚见他时只是觉得此人豪爽仗义,不拘小节,直到那次他被抓住后,装疯卖傻,吃大粪跟吃馒头似地,
看的旁人隔夜饭都吐了,他还在不停的咂摸嘴,我就佩服的五体投地,知道他以后一定能成大事。

宋大哥还有项特殊的本事,会哭,眼泪挥之即来招之即去,比起卖草鞋的刘玄德来有过之而无不及,刘玄德只是自己干
嚎两声,而宋大哥则能忽悠的大家跟他一起哭。

在梁山,我天不怕地不怕,就是天王老子我都敢跟他拼三百回合,急了眼,一个人敢打东京,但对于武松,我从心底里感
到一丝恐惧,这个人太冷静了,冷静的让人不可思议。印象中,除了被鲁智深兜头吐了一身那次,我从没有见他慌乱过。

我、武松和鲁智深虽然都是莽人,但行事风格有明显的区别,武松是那种凡事想了再干的人,而我是那种干了再想的
人,鲁智深这憨货是干了都不想的人。

梁山虽然经常搞什么排行榜,又是马上功夫排行榜,又是地上功夫排行榜,但我认为,真要打红了眼,没人是武松对手。

\section{}

我进去时,宋大哥正背着手,仰着头,盯着屋顶看,叫他也不应,不知道在想啥

山寨头领都有这毛病,晁天王爱这样,吴军师也爱这样,要么站在山头,要么站在河边,望着远处,摆出副高深莫测的鸟
样\dldots

朱武说,这叫``拿架'',是当头领必须要学会的三大本领之一

不过我从来不学,因为``拿架''忒危险。记得去年夏天的一个傍晚,我去梁山最高的那个山头抓知了,恰好吴军师也
在,正背着手看着远处金沙滩的方向,摆出副死了爹的鸟样、、

我凑过去问他看啥,问了八遍,这厮才慢条斯理的挤出两个字,``前程''

我顺着他的目光往远处看,哪有什么前程,除了几个洗衣服的村妇啥都没有

七月的天,娃娃脸,说变就变。天一下子阴下来,黑压压的乌云像脸盆一样挂在头顶,不一会就下起了小雨,山风一吹,
忒冷,我冻的双手抱肩浑身打哆嗦,问他冷不冷,说不冷,肯定是说谎,我看到他背在身后的胳膊都起了鸡皮疙瘩,腿也
在哆嗦

我心想你不冷就呆着吧,我可不奉陪,就说那我先走了,这厮连话都懒得说,背着身抽出右手摆了摆

我刚不走远,就听到他在作诗,``知我者谓我心忧,不知我者谓我何求''

我心里暗骂一声``装逼遭雷劈''

只听``求''字尚未说完,一道闪电掠过,接着晴空一声霹雳,我立马双手抱头趴倒在地,毕竟武功再高也经不住雷劈

霹雳过后,我抹抹头,还在,放下心来,突然想起军师,忙爬起来一看,这厮一身焦炭,头发跟鸟窝似的向上竖着,还冒着
青烟,慢慢转过身来,脸跟锅底似的,嘴一咧,一口白牙

那次他在床上足足躺了一个月才能下地,从那之后他再也不敢去山顶,也很少看到他作诗了,每逢下雨就在门口挂串佛
珠,在屋里念四字经,``阿弥陀佛''
\section{}

许久,宋大哥``拿架''完毕,转过身来,一脸沉重,问我人是不是我杀的

我想,人都已经死了,肯定不会再从地下爬出来跟我对质,王矮虎欺负女人的胆量有,但借他三个胆都不敢惹多嘴。

正想抵死不认,突然想到武松刚刚来过,这厮有个习惯,杀人爱留名,别人杀人后也留名,不过都是留仇家的名,他倒好,专
留自己名。血溅鸳鸯楼时,他要不在墙上写名,谁能知道是他干的?

州府破案那一套我早就摸透了,先把案发现场附近的邻居抓起来,不分青红皂白毒打一顿,有承认的最好,没承认的就
作个海捕文书,限期破案,万一到期破不了,从死囚牢里拉个犯人顶包,咔嚓一刀,结案了事,我当狱警时没少操持这事。

昨晚,这厮非要在沙滩上写上三人名字,被我劝住了。

估计刚刚他是认了,就是他没认,鲁智深早晚也得露了,以鲁智深的智商,不超过三句就能露陷。你若直接问他,昨晚是
不是他干的,他会非常坚决的说不是,但你若装出胸有成竹的模样自言自语说应该是一个人干的,他就会哈哈大笑说,
错,是三个人干的!

真贼娘,怎么跟这两个鸟人一起杀人!

哎!不怕虎一样的对手,就怕猪一样的队友!我算栽了。

我低头不说话,权当默认

宋大哥叹口气说,你杀谁不好,非得杀晁天王的小舅子,这下我怎么保你

我心中豪气陡升,说大哥你不用保我,一人做事一人当,大不了抵命给他,脑袋掉了碗大的疤

宋大哥摇摇头,伤感的说,当初你老娘也不要了,官职也不要了,提着板斧劫了法场,救了我一命,老娘不要了说得过去,你
竟然官职也不要了,这是天大的恩情,我怎么能看着你送死哪,咱们不是亲兄弟胜似亲兄弟。

我心里一阵感动,眼泪在眼中打转

宋大哥继续说,前几天宋青来找我,说你坏话,说你故意赢他钱,被我煽了一巴掌,这个笨蛋,就他那脑子,别人想不赢都
难,要不是当初你抬举他,他怎么会有今天!

知铁牛者,宋大哥也!太感动了,眼泪哗哗地

宋大哥最后擂着桌子说,你放心,只要我在,他们就甭想破案。

我当时差点就哭出来,我知道,我这辈子都跟定宋大哥了,不为别的,就为``情义''二字。

聚义厅开会,商议梁山兄弟被杀一事。

吴用摇身一变,成为断案高手,说什么杀人无外乎两种原因,财杀和仇杀,并且逐一分析:现场银两分文未取,应不是财
杀,凶手下手狠毒,招招致命,仇杀的可能性较大,下一步应从与死者结怨的人开始查

鲁智深突然哈哈大笑,幸亏武松早有准备,一巴掌轮上,这厮才闭嘴。

宋大哥说,被杀兄弟得罪的人太多,方圆几十里的男人都跟他们有仇,而且是不共戴天的共妻之仇,人人都有嫌疑,真要
查起来,非一朝一夕的事,建议先把人给火葬了事

晁天王反对,说案子没破,最好是土葬,将来再扒出来,也能有个见证。

兄弟们有说土葬好的,有说火葬好的,还有沉默不语的,乱成一锅粥

会议从上午开到中午,从中午开到下午,兄弟们饿的肚子都扁了,还没商量出个结果来

晁天王跟宋大哥吵了起来,晁天王说宋大哥心里有鬼,宋大哥说晁天王无理取闹。晁天王这两天被老婆折腾病了,一上
火,晕了过去

等他醒来时,人都已经烧成灰了

晁天王一个劲的感叹:男人,关键时刻一定要挺住!
\section{}

第二天一大早,宿醉还没醒,外面鞭炮齐鸣,一行人敲锣打鼓走上关来

我正纳闷,今天谁娶亲?我怎么没收到请柬?山寨规定只有厅级干部才能娶亲,不过现在的厅级干部该娶的都娶
了\dldots

莫非林冲要开第二春?不像,昨天去找他借钱,这厮正对着林娘子的牌位发誓要孤独终老永不再娶\dldots

莫非公孙胜要还俗?这个有可能,别看这厮穿上道袍人模狗样,其实一肚子男盗女娼,每次喝醉了都盯着扈三娘胸前看,眼
珠子都能瞪出来\dldots 真是个老流氓,人家胸大,你也不能随便看啊\dldots

转念一想,不管谁结婚,反正没通知我,正好不用随份,酒席照吃,等第二天找上门去做做样子,摆出气呼呼的模样:你娶
亲竟然不请我,咱还是不是兄弟了?\dldots 你真不够义气\dldots 要是手边有桌子,再拍两下,就算完美了

我挤到前面一看,不是娶亲,是东风屯的葛老爷子,领着一帮村夫,抬块大匾,直送到聚义厅问口,匾上四个溜金大字:替
天行道

我挨过去摸了一下,一手金粉,妈的,假的!

葛老爷子在方圆几十里威望很高,给死人树碑、给寡妇立牌坊的事都由他操持。

葛老爷子率众人跪倒在地,高呼晁宋两位头领替天行道,替他们除了作恶多端的流氓\dldots

晁天王哑巴吃黄连有口说不出,牙齿咬得咯咯响,狰狞的脸上勉强挤出一丝笑容,说都是分内的事,本该如此

宋大哥不失时机的说要把匾挂到晁天王家里,晁天王拒绝了,最后挂到了聚义厅大堂上

据说晁天王回到家就把桌碗瓢盆全砸了

\section{}

今晚又喝多了,翻来覆去睡不着

宋大哥在屋梁上打呼噜,花容蹲在房顶上看月亮,宋大哥说晁大愣今天急了眼,让花容过来以防万一。

晁大愣是宋大哥给晁天王起的外号,当然,也只有他自己敢叫,就像黑胖子只有晁天王敢叫一样。

有花容在,我总是很放心,这个面容清秀的男子,箭法超群,百步穿杨,说射牙齿绝碰不到嘴唇,说射眼睛绝碰不到睫毛,而
且从未失手。

花容儒雅风流,吃肉从不下手抓,喝汤从不出声,也从不酗酒闹事,比读书人还读书人。

他刚上山时,兄弟们总爱跟他赌箭,无不落败,白白让他赢了许多利物。

他唯一输的一次,是输给了朱武。

那次,众兄弟在金沙滩乘凉,朱武指着百步外蹲在树梢上的麻雀说,咱们赌箭,你射中麻雀的老二就算我输,不然,就算
你输,花容当场答应,众兄弟纷纷押注,都押给了花容

我知道朱武这人外表蠢笨其实内心敞亮,没把握的事绝不会做,但这么近的距离花容也绝不会失手

笨人自然有笨办法,我把赌注分成两份,一人押一份

花容从容的拈弓搭箭,弓满如月,麻雀是蹲着的,花容用小拇指一勾弓弦,锃然作响,麻雀惊慌失措,展翅欲飞,将飞未飞
时,一声响迪,麻雀中箭落地,众人齐声喝彩。

一兄弟飞快捡来麻雀,众人一看,愣了,母的。

那次是我赌博生涯中赢的最多的一次。
\section{}

宋大哥喝醉了喜欢作诗,晁天王喝醉了喜欢骂娘,王矮虎喝醉了喜欢逛窑子\dldots

我喝醉后总爱思考人生,酒可以让我变得聪明,能够想明白很多事情。

人生大致分两种,成功的,失败的。成功的人生,大致形同,要么有个好爹,要么有个赏识你的领导,失败的人生,各有各
的不幸。

我混了二十多年,官场混过、赌场也混过、现在落草当了强盗,人生很不如意

身边的人,没有最惨,只有更惨,作为失败的典型,都很成功。

我们身上虽然没啥成功的经验,但还是能总结出不少失败的教训。

不要跟豪爽仗义的人走的太近,这种人,当你落难时,会帮你一把,但当他落难时,也会倒打一耙,拉你下水,我和花荣就
是活生生的例子。

不要娶漂亮的老婆,老话说得好,女人是祸水,漂亮的女人更是如此。林冲原来是八十万禁军教头,响当当的一条好汉,就
因为老婆太漂亮,被高衙内看中,弄的是有家难回,有国难奔;

这个世界上,大部分男人,并不是``别人妻,不可骑''的人,而是``别人妻,不骑白不骑''的人

不要相信所谓的兄弟义气,山上的兄弟,有一半是被自己兄弟骗上山的,军师说那不是骗,是赚,但我实在看不出有啥区
别。
\section{}

今天把时迁揍了一顿。

这厮专门恶心人,你跟他谈义气,他跟你谈银子,还了他银子,又给你谈利息,还了利息,他竟然能一脸真诚的跟你谈义
气,操!

对了,鲁智深上月借了我二两银子,说月初还,现在都月底了,还没还。这厮忘性忒大,不跟他要永远想不起来还。

不过直接张口要吧,显得自己不大方,不要吧,天天在心里惦记,睡觉都不踏实\dldots

今天想暗示他一下,说我有两把板斧,不是三把,也不是一把,恰好两把,问他为啥,这厮抱着脑袋楞楞的琢磨了半天,最
后一拍大腿,我当时心里一阵激动,以为终于想起来了,没想到这厮竟然说是因为我恰好有两只手\dldots

哎!明天还是跟他直接说道说道吧\dldots 大不了不要利息了\dldots

还是有个娘子好,王矮虎要账从不用自己出面,谁要欠了他钱,扈三娘第二天就找上门去\dldots

实在不行娶个娘子吧,以后借出银子去也不用天天惦记了\dldots

\section{}

很多人说我粗鲁,脾气暴躁,这也怪不得我,上山前我跟牢犯打交道,上山后跟强盗打交道,天天混在一起的不是地痞就
是流氓,跟这些鸟人呆久了,由不得你不粗鲁,就是孔老夫子来,估计用不了两天,也会满嘴脏话。

当周围人开口闭口都是`` 草尼妈时'',你最后的回答就是``草尼姥姥''。
\section{}

一个叫徐宁的军官被骗上山,汤隆骗的,两人好像还是姑表兄弟。

落草当强盗这事,除了王矮虎之流除了当强盗别无出路的,没人愿意干,更何况朝廷干部,你想啊,人家日子本来过的好
好的,每天去衙门喝喝茶,吹吹牛逼,踢踢皮球,汇报工作时糊弄糊弄上级,布置任务时训斥训斥下级,下班后去翠红楼
洗洗澡,吹吹萧,一天就算完事了,又安稳又舒服。定期拿着朝廷银子,偶尔骑一下别人娘子,人生如此,夫复何求?

突然抛妻别子来梁山这兔子不拉屎的地方当强盗,打死也没人愿意来,条件忒艰苦!不但天天要跟一帮上完厕所从不洗
手的臭流氓拼酒,还要定期下山借粮,说是借,压根就没还过。不给就得动手明抢,危险系数忒高!

为朝廷效力,战死了,算烈士,朝廷有抚恤金,还会给你建忠烈祠,发个大匾``精忠报国'';为山寨效力,战死了,孩子管
别人叫爹,老婆躺别人被窝,坟头上还给挂个匾,``最佳强盗''

若是晁天王和宋大哥一较劲,你都不知道是土葬还是火葬。

山寨规定,骗一个人上山,奖五两银子,若是朝廷干部,二十两

大部分兄弟只是骗以前的同事,汤隆这厮心忒黑,连表兄弟都要骗,问他是不是跟徐宁有血海深仇,这厮竟然咬牙切齿
的说他爹死了徐宁没去吊唁

据说徐宁为人圆滑,从不得罪人,凡事明哲保身,这下真是祸从天降,多少年不来往的表兄弟把他吭了。

这个世界,没人能做成好人。

有时候,不经意间,你会得罪很多人。

老婆生了个儿子,你放支鞭炮庆祝,天经地义,但那些老婆生不出孩子的人心里恨不得掐死你儿子。

吃饭时给媳妇夹菜忘了给老娘夹,老娘会在背后骂你白眼狼,若给老娘夹不给媳妇夹,晚上甭想上床。

很多事,你不做,也会得罪人。邻居墙倒了,你虽然没去推,但邻居心里恨上你了,因为你没去扶。

做好事,也会得罪人,你甚至不知道究竟哪里得罪了人。两个乞丐,你给了一个一两银子,给了另外一个二两,拿一两银
子的那个不但不会感激,反而会骂你小气鬼。

所以,这个世界只有自认为是好人的人,而没有众人都认为是好人的人!
\section{}

徐宁寻死觅活,宁死不当强盗。

但你一旦上了梁山,就由不得你了,要么坐把交椅,要么咔嚓一刀丢到河里喂鱼,想回头,没门!

下面好戏开始,先来硬的,一般都是我来,升的化妆。以前鲁智深也干过,不过有一次演的太投入,忘了是演戏,一禅杖
把人打死了,自那之后再也不敢让他出场。

我酝酿了下情绪,呲着牙,咧着嘴,拎着板斧去踹门。

本来打算的好好的,一脚把门踹烂,最好是木屑满天飞,然后提着板斧猛冲过去,再把桌子砸个稀巴烂,估计他就是不服
软,也能被我这气势吓个半死

谁成想门框太结实,一脚只踹了个窟窿,卡住了,好不容易抽出来,鞋没了。

只好抄家伙,用力劈了十几下,方才劈开,当时心里唯一的想法就是回头把造门的木匠给剁了

我进门时肯定特狼狈,赤着脚,满头大汗,气喘吁吁

我本还想弯腰捡鞋,想了想,算了,太寒碜。

徐宁正坐在椅子上,冷眼看着我,一副宁死不从的模样

我举起板斧,呐喊一声,猛冲过去

预料中徐宁就算不大喊``好汉饶命'',也得缩成一团,没想到这厮突然站起来,一脚踹翻桌子,伸着头说``你砍死我吧''

我当时愣住了,板斧举在空中,可是桌子已经踹翻了,也不能老举着,总得劈点东西吧,不然自己再放下来太丢份
了\dldots 可四周空无一物\dldots

这时,候在门外的兄弟跑进来,拉着我的胳膊说,宋大哥有令,决不能伤了徐宁一根寒毛

我只好借坡下驴,一边叫骂一边跟众人出去喝酒了

娘地!太丢人了!

\section{}

接下来轮到林冲出场,现身说法.

两人在东京时曾是同事,一起扛过枪,一起嫖过娼,关系相当铁。

两人回忆了曾经的光辉岁月,当初两人同中武举,骑马并辔游遍东京,是人人羡慕的青年才俊,都怀着一腔热血,发誓要
忠君报国,保境安民。时过境迁,如今竟在强盗窝里相见,一时唏嘘不已

哭完了,擦干眼泪,步入正题,林冲盛赞晁宋两位头领宽仁宅厚,义薄云天,说其实当强盗没啥不好,当着当着也就习惯
了\dldots

任林冲说的天花乱坠,徐宁不为所动,这没啥奇怪,就像林冲在吃一堆大粪,任他脸上的表情多享受,在徐宁看来,仍是
一堆大粪而已

林冲没辙,拍拍屁股走了

接着是花容、秦明、孙新\dldots

一个个乘兴而来败兴而归\dldots

没辙,只好宋大哥亲自出马

宋大哥的忽悠本领在大宋朝那可是无人可比,他要自称榜眼,绝无人敢当状元。

宋大哥在郓城县当押司时,别人孩子都会打酱油了,他还拒不娶妻,任媒婆踏破门槛都无动于衷,宋老爹急了,把棺材板
卖了,凑了一百斤黄金,请动济州府大名鼎鼎的张媒婆。

张媒婆号称``说破天'',能把王母忽悠的下凡,把道士忽悠的信佛,只要她撮合的姻缘,无一不成,任你铁石心肠,也经
不住她的三寸不烂之舌

据说张媒婆和宋大哥聊了整整一下午,聊的内容旁人不得而知,聊完后,宋大哥照常去官府当差,将单身进行到底,张媒
婆家也没回,直接去附近的尼姑庵削发为尼。

前段时间,朝廷屡派兵马攻打梁山,结果屡战屡败,弄得狼狈不堪,最后蔡太师出奇招,花重金从西域请来得道高僧智洪
禅师,让他上梁山点化宋大哥,想让宋大哥放下屠刀,立地成佛。

智洪禅师一脸仙风道骨,跟宋大哥谈人生、谈佛法、谈轮回,一席话下来,禅师脱掉袈裟,抄起戒刀入了伙。

\section{}

赌场开了盘口,一赔十,一时盛况空前,鲁智深、刘唐、鲍旭这些老赌棍自然少不了,就连扈三娘的丫鬟、公孙胜的道
童、吴用的书童都扛着大锭银子赶来下注\dldots

以前这三人从不踏进赌场半步,扈三娘自诩是大家闺秀,公孙胜奉行道家的清静无为,吴用秉承孔孟之道,都对赌博很
反感\dldots

这次,三人也坐不住了,毕竟装逼换不来银子,稳赚不赔的买卖不做是傻帽

我把压箱底的银子全部押上,形势一边倒,都押宋大哥赢

不知道哪个傻帽押宋大哥输,押了一千金,真是脑袋被门挤了

整个梁山泊只有一人未下注,朱武,这厮躲在一边冷笑\dldots

朱武冷笑准没好事,兄弟们谁也不愿意搭理他,嫌晦气,宁听夜猫子半夜叫,不听朱武冷笑。

一个时辰过去了,众人淡定自若\dldots

两个时辰过去了,众人坐立不宁\dldots

三个时辰过去了,众人焦躁不安\dldots

宋大哥跟徐宁从日初谈到日中,从日中谈到日落

夕阳西下,宋大哥阴沉着脸走出来,无奈的摇了摇头

众人犹如三伏天兜头泼了一桶凉水,呆住了

反应过来后,一下子炸了窝,有破口大骂的,有哭爹喊娘的,有寻死觅活的\dldots

我和李兖、刘二抱头痛哭,这次三人赔的血本无归,还欠了一屁股高利贷

三人来到金沙滩,想投河,河边人太多,没挤进去;又到树林里,想上吊,结果树都被人占了,有几个兄弟为争位置打的头
破血流,只好来到后山悬崖处

下面是万丈深渊,跳下去绝无生还的希望

一个人自杀容易,一群人自杀很难,大家都有自杀的勇气,但没人希望死在别人前面。

三人站在悬崖边,犹犹豫豫,你推我让,谁都不想第一个见阎王。

最后,李兖提议,闭上眼睛,喊一二三,然后一起跳。

这个方法不错,我和刘二同意,闭上眼睛,喊完一二三,我没跳,睁开眼睛,李兖这厮也没跳,刘二跳了

刘二在半空中骂了一句``李兖,我日你姥姥!''

我问李兖为啥不跳,李兖转过头对我说,黑哥,我还有很多话憋在心里难受

我说我们都是要死的人了,有啥话你就直接说吧

李兖说:黑哥,以前我总说你长的板正,其实,你长的真丑

我说,美也好,丑也罢,死后不都得烧成一撮灰?咱不墨迹了,喊完一二三,一起跳吧

这次我没闭眼,刚喊完三,我抬起一脚,把他踹下去了。

娘的,我最讨厌别人说我长的丑
\section{}

活着,需要勇气;死,更需要勇气。

我深吸一口气,正准备跳,戴宗赶来了,说不用跳了,宋大哥有令,所有输的银子当做奖金如数返给各位兄弟

有时候,一刹那,生死两重天

上梁山后,我跟戴宗一向不合,经常互相骂娘,他要是藏着死心,晚喊几秒,我就算交代了,我一阵感动,忙不迭的道谢

戴宗说,别谢,要不是你还欠我二十两银子没还,我才懒得跑这么远救你。

哎!有时欠钱也是一种幸福!

原来押一千金赌宋大哥输的,是宋青

山寨上下感激涕零,盛赞宋大哥仗义疏财,及时雨的称号果然名不虚传

我想,若是再及时点,刘二就不用跳崖了

吴用专门召开会议,高度赞扬了这次赌博,说这次赌博吸引赌资黄金一万两,极大的拉动了山寨经济发展,增长指数翻
了几十番,一天时间就完成了全年的任务,是古今未有的奇迹\dldots 山寨又给各位兄弟发了一万两的奖金,各位兄弟
的收入水平也翻了几十倍,已经踏入大康生活\dldots

反正说了一大堆,我也不懂,我感觉啥都没变,输了的银子又发了下来,不多不少,怎么会拉动经济发展?我的银子又还
给我,收入怎么翻了上百倍?想不懂,可能是我太笨了。

宋大哥让我把徐宁拉到乱坟岗给咔嚓了

我憋了一肚子气,这厮害的我差点跳崖,正想狠狠的多砍几斧,朱武来了

朱武说他要跟徐宁聊聊

我想,聊就聊吧,宋大哥都忽悠不了的人,你还能忽悠出个景来?反正徐宁现在是案板上的鱼,等着他的,只是清蒸还是
油泼的区别

朱武问徐宁为啥不愿意落草?

徐宁说他不愿意当强盗

朱武冷笑一声说,当今朝廷,人人都是强盗

徐宁不解,我也很惊奇

朱武说,朝堂上那些公卿大臣,个个肥头大耳,却不知忧国忧民,只会欺君误国,贪污受贿,剥削百姓,难道不是强盗?

那些举人秀才,呼朋引伴,把持衙门,吃了原告吃被告,歪曲法例,谄媚权势,颠倒黑白,弄得人家破人亡,难道不是强盗?

那些公子哥,依仗父兄权势,横行乡里,欺男霸女,巧取豪夺,百姓无处申冤,官府不敢盘问,难道不是强盗?

那些商人富贾,虽锦罗绸缎,但心如蛇蝎,囤压居奇,克薄伙计,以次充好,以旧充新,大斗进小斗出,难道不是强盗?

最后朱武说,既然处处是强盗,何不栖身水泊,留有用之身?

一席话,徐宁当场愣住了,最后情愿纳降。

\section{}

宋大哥趴在屋梁上睡着了,呼噜震天响。

我打心眼里佩服他,上万两的银子,挥挥手,说不要就不要了,眼睛眨都没眨。

我靠在窗栏上,月光从窗户钻进来,很亮。

我琢磨着朱武白天的话,似乎有点道理,但具体哪里有道理,却又想不明白。

我琢磨问题的时候,喜欢喝酒,就像军师琢磨问题时喜欢摇扇子,鲁智深琢磨问题时喜欢拿头撞墙一样。

喝了两坛绍兴女人红,心中渐渐有些敞亮,对啊,那些王八蛋的确是他妈的强盗!

又喝了两坛,又有些迷糊,为啥都是强盗,他们却高高在山,人五人六,我们却被逼的东躲西藏?为啥别人都拼死拼活的
要加入他们,却宁肯杀头都不肯加入我们?

这么深奥的问题我一般不问宋大哥,问了也白问,他自从当了头领后,只知道喝酒应酬,到处赶场,三句话离不了梁山大
业,没劲!

我虽然笨,但有个优点,想不明白的问题就要问,为这习惯小时候没少挨揍,有次我缠着母亲问驴的老二为啥一会大,一
会小,母亲二话没说,给了我一巴掌。还有一次,我问父亲,为啥母亲不在的时候,他总爱往隔壁王婶家跑,回答我的,仍
是一巴掌。

我提着坛酒,去找朱武

梁山的夜晚很热闹,鲁智深等人在赌博,武松在喝酒,林冲在练功,他也就这爱好了\dldots

朱武正在看书,边看边骂娘。

我有些糊涂,在我心中,书是很神圣的东西,记载的都是圣人之言,前朝之事,我每次帮宋大哥搬书都双手捧着,生怕弄
皱一点边

我问他为啥生气

朱武合上书气呼呼的说,写书的人也是强盗

我更糊涂了

朱武说,写书的人颠倒黑白,歪曲事实,抹黑前朝,粉饰本朝,难道不是强盗?

恩,有点道理,我问他既然都是强盗,为啥我们犯法,他们不犯法。

朱武说,强盗分两种,一种是合法强盗,一种是违法强盗

我问哪些是合法,哪些是违法。

朱武说,谁得了天下,谁就是合法,得了天下的人说谁是违法,谁就是违法

朱武接着说,历史上那些开国之君,没一个不是强盗,我朝太祖赵匡胤,本是柴世宗的臣子,捧着大周朝的饭碗,拿着大
周朝的俸禄,却夺了人家位子,是强盗无疑。

前朝太宗皇帝,本是隋朝臣子,国家有难时不知匡扶社稷,却趁机谋反,后来为了争皇位,射死哥哥李建成,又把他五个
儿子一刀一个全剁了,又杀死亲弟弟李元吉,然后把弟媳妇抢过来自己用,连强盗都不如。

所谓改朝换代,不过是一伙强盗打跑了另一伙强盗而已,得了天下的,就是功臣元勋,自然有一帮文人来捧臭脚拍马屁,写
书立传,流芳百世;败了的,就是乱臣贼子,口诛笔伐,遗臭万年,永世不得翻身

我突然想到一个问题,那我们哪?会流芳百世,还是遗臭万年?

朱武说要看天意\dldots

\section{}

最近无事,总喜欢去山后看老黄,老黄是头牛,名字我给起的。

老黄是我的战利品,有次下山抢劫,遇到一财主驾牛车载着女儿去赶庙会,财主肥头大耳坐在前面,女儿头戴围巾坐在
后面\dldots

武松二话不说,当头一刀砍了财主,鲁智深抢了财物,我拿着斧头对着女儿犹豫,我跟武松不同,他是人人可杀,我则没
有杀女人的习惯,尤其是年轻的女人\dldots

女人坐在车后,埋着头,瑟瑟发抖,看背影,应该很漂亮\dldots

我用斧头挑开女人的围巾,当即愣住了,半边脸全是疙瘩,密密麻麻,我不由的想起了癞蛤蟆\dldots

武松挺刀赶了过来,也愣住了,他喜欢杀漂亮女人,如此丑的,也下不去手

兄弟们起哄说那个女人跟我挺般配,让我干脆娶了她吧,我拒绝了,我长得丑,并不代表我喜欢丑

山上的规矩,不留活口,兄弟们商量半天,一致认为,让她活着就是对她最大的惩罚,决定带她上山,谁看上就给谁做压
寨夫人

兄弟们要回山,没有多余的马,谁也不肯跟她同骑一匹,说谁抢的算谁的,我也不愿意跟她一起,只好把马让给了她

那天兄弟们都满载而归,就我,因为被她吓了一跳,一愣神的功夫啥都没抢到,心里很窝火。

现场只剩一头牛,兄弟们抢劫时,极少抢牛,块头太大,杀了扛不动,牵着又太慢,赶不上回去的庆功宴。

看着兄弟们的背影,越想越气,提着板斧正要回去,这时老黄``哞哞''的叫了两声,我正心烦,回身``啪啪''给了它两巴
掌,这厮除了眯了眯眼,居然不跑不跳\dldots

我正准备一斧头结果了这厮,突然一想,上山的路太远,还拎着两把板斧,太累,干脆骑着吧,回去杀了下酒

这头牛属驴的,不打不动弹,我都打累了,它还慢悠悠的,索性不去管它,走多快算多快

从午时走到酉时,从酉时走到戌时,太阳落山时,终于到了半山腰,又碰到那个女人,正独自下山,我叫住她,问她怎么回
事

她说山上的头领都看不上她,让她快点下山,我笑笑,意料之中。

她的眼神像极了受伤的小猫,我突然明白了王矮虎曾讲过的笑话:一美一丑两女深夜出门,途中遇到歹徒,歹徒放走了
丑的,强奸了美的,他问我,是美的痛苦还是丑的痛苦?我说当然是美的痛苦,他说,错,是那个丑的\dldots

我一直想不明白,今天,似乎有些明白了。

我摸摸口袋,只有五两银子,全给她了

她千恩万谢,说不会忘了我的大恩大德,这套说辞都听出老茧来了,刚当强盗时,我也偷偷放过几个活口,他们当面感谢
我八辈祖宗,转身就去官府里告我一状,还详细描绘我的相貌,这个倒也罢了,让我气愤的是竟然把我描绘的那么丑,真
是岂有此理,希望这个女人不要把我说的太丑。

我问她叫啥名字,她说叫槐花,我点点头,让她慢点下山,不用急,绝对没人打她主意

我把老黄扔到后山,赶到聚义厅时,庆功宴已经结束,吃了口剩饭,喝了两杯冷酒,窝着一肚子火,翻来覆去睡不着,爬起
来跑到后山又给了它两巴掌,才心满意足的睡下

我这人特爱管闲事,经常惹一肚子气,一生气就睡不着。吴用给出主意,数羊,说数到一千只就睡着了,我表面上说军师
神机妙计,是好主意,但心里直骂娘,老子连十只羊都能数错,上哪数一千只去

王矮虎出主意,说娶房媳妇就睡着了,不过看到张青天天鼻青脸肿的鸟样,也就心懒了

这下好了,生气时,就去后山煽老黄两巴掌,煽完气就顺了,也就睡踏实了。

\section{}

梁山别无他事,喝酒、抢劫、吹牛逼,唯此而已

今天聚会喝酒,有文化的行酒令,没文化的答题,我很自觉,乖乖的坐到答题那一桌。

王矮虎这厮,仅念过两天书,三字经都背不全,就敢跑到有文化那桌装逼,每次都醉的抱着老黄叫娘,醒酒后还大言不惭
的炫耀``兄弟我可是在有文化那桌喝醉的''。草!

他那行酒令,翻来覆去就那一句,我早就摸透了,什么``酒是米做,不喝是错,B是肉做,不日罪过'',要多下流有多下
流,呸!人至贱则无敌,王矮虎算是贱到家了。

晁天王领第一杯酒,说``女人大点口,男人全进去''。

众人哄堂大笑,我也跟着哈哈大笑,其实,我真不觉得有啥好笑,既然众人都笑,我也只好附和两声。

旁边鲍旭笑的前仰后合,眼泪都流出来了,我很奇怪,这厮大字不识一个,没理由比我还聪明,我悄悄问他到底哪里好
笑,这厮擦擦眼泪说,他也不知道。

宋大哥领第二杯酒,说``今朝有酒今朝醉,省的将来徒伤悲'',众人叫好,干了

吴军师领了第三杯酒,说``读万卷书,行万里路,不如陪领导尽情一吐!'',众人叫好,干了

下面开始分桌喝,我和鲍旭、鲁智深、刘唐、阮家兄弟等人在一桌,我们桌是顾大嫂出题,真是冤家路窄,前几天我把
她得罪了,那天喝醉了,说她长的像母猪,她拿着刀狂追了我八公里,妈的,猪都还没生气她倒先生气了。

我也不怕,我虽然跟吴用等人比起来是笨,但跟鲁智深等憨货相比,还是相当有自信的。

顾大嫂出题,问鲍旭,天上几个太阳,鲍旭回答一个;问鲁智深,天上几个月亮,鲁智深回答一个;轮到我了,天上几颗星
星?

妈的,喝酒!

顾大嫂继续出题,问阮小二,蜈蚣有几颗头,回答一颗,问阮小五,蜈蚣有几只眼睛,回答两只;

我突然明白过来,忙说,你不能问我蜈蚣有多少条腿

顾大嫂冷哼了一声,问我还有啥不能问,我想了想说,只要是数字的,都不能问

顾大嫂问,蜈蚣总共三十四条退,走路时先迈哪条?

``\dldots ''

妈的,喝!

顾大嫂继续出题,问阮小七,我朝太祖派哪位将领攻破后唐,答曹彬;问时迁,曹彬帅多少人攻打后唐,答十万;

顾大嫂看着我,一脸阴险,问我还有啥不能问

我想了想说,数字不能问,曹彬是哪条腿先踏进城门也不能问

顾大嫂点点头,问道,十万士兵都叫啥名字?

``\dldots ''

妈的!喝! 赌有赌品,酒有酒品。

赌场上讲究愿赌服输,哪怕押的是老婆孩子,也只有咬牙认了。不过山寨兄弟虽然粗鲁,但极少押老婆孩子,孙新除外,这
厮每当输的差不多时,就把他老婆押上,诚心恶心人。

酒桌上罚酒,哪怕喝到吐血,该你喝的,一杯不能少。

我虽然人品不咋地,但赌品和酒品没得说,顾大嫂连阴我十九碗,我一碗都没磨叽。要知道武松打虎时才喝了十八
碗,还是村醪白酒,我喝的可是正宗的绍兴女儿红

第二十圈,顾大嫂先问鲁智深,扈三娘的马褂颜色,红的,问鲍旭,裙子颜色,灰的,轮到我了,问抹胸啥颜色?

我低头看看酒,胃里直冒酸水,扭头看看扈三娘,双手挡在胸前,神色紧张,抬头看看众兄弟,都在盯着我,眼睛冒着绿
光,一脸坏笑

顾大嫂跟扈三娘一向不和,顾大嫂背后骂扈三娘是骚货,一双媚眼专门勾引男人,扈三娘背后骂顾大嫂是泼妇,成天骂
街有失体统\dldots

真奇怪,两人上山前压根就不认识,上山后也没啥过节,就是凑不到一块,相反,顾大嫂和孙二娘关系好的很,天天姐妹
长姐妹短的,经常腻在一块骂扈三娘。

宁得罪十个君子,不得罪一个小人,宁得罪十个小人,不得罪一个女人,我算是栽了。

咬咬牙,举起碗,一饮而尽

顾大嫂笑吟吟的凑到我面前,问我还能不能喝

我越看她那张脸越恶心,喉头一紧,喷涌而出\dldots

\section{}

我踉踉跄跄的去了后山,不停的大骂顾大嫂的父母,作了几辈子孽才弄出这么个破玩意来,我要是她爹,当初就再给塞
回去

要是说子孙不贤是因为祖宗无德,那顾大嫂的祖宗一定缺了大德

孙新够倒霉,长的一表人才,就算娶不到大家闺秀,娶个小家碧玉总不成问题,结果摊上了这么个母老虎,休又休不
掉,打又打不过,苦不堪言。上山前,喝醉了就跑祖坟上,大骂苍天无眼,现在喝醉了就跟张青抱头痛哭,两人算是难兄
难弟,时常在一起交流跪搓衣板的经验。

当初顾大嫂在登州,可谓是无人不知无人不晓,天天穿件大红袍,头上插根银钗,招摇过市,自称``拳打城东好汉,技压
江北群芳'',人称``斜眼歪嘴一獠牙,登州头号豆腐渣''

年方三八,仍嫁不出去,被众媒婆评为说媒史上一道无法逾越的鸿沟。

佛祖对女人是仁慈的,有的给了美貌,有的给了贤惠,有的给了善良,有的给了勤俭\dldots 总之,身上总会有打动男人
的东西。如此看来,顾大嫂肯定是佛祖跟人间开的一个玩笑。

顾大娘对女儿缺乏清醒的认识,刚开始定下择婿条件:年龄相当,相貌堂堂,最好是乡绅公子。

无人问津。

后来降低条件:年龄相当,相貌放宽,家境殷实。

仍无人问津。

后来,年龄不限,四肢健全,家境一般,实在不行给人当二房也行。

还是无人问津。

再后来,顾大娘拉着女儿手直唠叨,只要是个男的就嫁了吧。

如此条件,仍嫁不出去,不论谁家,听到是给顾大嫂提亲来了,立马轰出去,连水都不让喝一口,众媒婆无人再管这闲事

顾大嫂一怒之下,放言抢亲,挑个庙会的日子,沐浴更衣,穿戴整齐,拿着条绳子去一看,清一色的女子,别说青壮年,就
是天天在马路边晒太阳的七八十岁的老头子都不见了踪影
\section{}

正当顾大嫂心灰意冷时,倒霉鬼孙新出现了。

孙新不是本地人,他哥哥孙立调任登州提辖,他也跟着过来了,孙提辖图吉利,花十两银子请大名鼎鼎的贾半仙挑了个
黄道吉日,据说,当天大吉,诸事皆宜。

孙提辖去衙门报道,孙新闲着无聊,去逛庙会,正当他纳闷为啥庙会上全是女人时,披铠执甲、红袍马褂、状若门神的
顾大嫂出现在面前

一看这阵势,孙新立马明白咋回事,但他真没害怕,要知道他打小习武,拳脚了得,也是横着走惯了的主,压根没把眼前
这个腰围三尺胸围三尺臀围三尺的``水桶''放在眼里。

据说当时两人``深情''对视了一炷香时间

后来据顾大嫂说,她当时是心花怒放,春心荡漾,认为眼前这个白面郎君就是她苦苦等待的意中人

不过据孙新酒后说,他当时就是觉得眼前这个丑八怪面熟,特像他家狮子狗

孙新楞着眼睛问了一句,长的丑就牛逼啊?

后来王矮虎说,一句话就看出孙新年幼不懂事,其实长得丑不牛逼,但逼特牛。

接下来开练,孙新本来跟顾大嫂武艺不相上下,但没有跟女人打架的经验,不知道该从哪里下手,武艺打了折扣

顾大嫂则不然,全没有顾忌,除了猴子摘桃不用,什么招式都用,锁喉功、鹰爪功\dldots 再说也急红了眼,越战越勇,几
十个回合下来,孙新就招架不住,被五花大绑抬去拜了天地

孙新街头酣战时,孙立也得到消息,不过他高兴的很,这个弟弟早该成家了,这下被抢了亲,正好了了他一桩心事

孙新成亲当夜,宁死不从,洞房里又大战一晚,床塌了,桌子碎了,椅子烂了\dldots

别人洞房都是女人叫,孙新洞房他自己叫,嗓子都吼哑了

第二天,孙立在家等候新人回门拜访,当他看到鼻青脸肿的弟弟和脸含娇羞的顾大嫂时,端着茶杯的手,僵在空中,半晌
无语

最后,兄弟两人抱头痛哭,孙立从那之后,逢人就说,要知道顾大嫂长那个鸟样,天上就是下刀子,他都会去救他弟
弟\dldots

孙新上梁山前,特地去把贾半仙一刀剁了

新婚之夜,孙新到底有没有失身,是梁山兄弟一直八卦的问题,在八卦排行榜上高居第三,第二是关于武松的,这个我知
道,排名第一的是啥就不清楚,每当我问,众人都怪怪的笑,直娘贼!

\section{}

老黄在后山吃草,优哉游哉,有时候挺羡慕它的,不用上阵厮杀,不用到处拉关系随份子,偶尔挨两巴掌,就有吃有喝,高
兴了叫两声,不高兴了就睡它娘的,要多快活就有多快活\dldots

今天竟然被顾大嫂连阴二十碗,越想越气,正准备煽老黄两巴掌解气,突然发现,老黄的牛脸肿了,眼睛眯成一条缝

我纳闷,最近没煽它啊,咋回事?谁他娘的下这么重的手,不把牲畜当人看\dldots

我喝醉了总是很聪明,没有声张,决定躲起来看个究竟,四周空阔,没啥遮挡,不远处有个水坑,水草半人高,是个藏身的
绝妙好地,我悄悄猫在里面

月上头顶

晁天王摇摇晃晃的来了,看看四处无人,指着老黄大骂:你这黑胖子,丢到煤渣里就找不到的狗东西,晚上不穿衣服围着
梁山跑八圈都没人看见的腌臜货,脸比锅底黑心比脸黑的王八蛋,当初要不是我带着兄弟们劫了法场,你早去阴曹地府
报道了\dldots 现在倒好,到处拉拢人心,处处跟我作对\dldots 别人都说你仗义,仗义个屁!要不是你怕我把给你送银
子的事抖搂出来,你能有那么好心吗\dldots 你说说,你当押司时收了我多少银子\dldots

晁天王骂完,伸手煽了老黄两巴掌,出手忒狠,啪啪的响,听的我都心焦

煽完径直走过来,我心怦怦乱跳,以为被发现了,没想到这厮走到水坑边,解开裤腰带,一阵噼里哗啦\dldots

我只觉得一股热流从头顶浇下\dldots

晁天王走远了,我抹抹脸,准备起身,宋大哥来了

宋大哥四处转悠一圈,突然说,我看到你了,出来吧!

吓了我一大跳,以为被他看到了,正准备跳出来,突然想到,这厮经常玩诈术,很可能又是唬人

果然,宋大哥看无人回答,转过身去,对着老黄开骂:晁大愣,你有啥本事,还自称天王,我呸!真不要脸,当初在郓城
县,天天跟在我屁股后面,左一口押司右一口押司,今天一颗枣,明天一筐蛋,把我拉下水,现在倒好,人模狗样的装大头
蒜\dldots 你懂个啥?昨天我问你知不知道杜甫,你说没跟他喝过酒,今天我夸李白文采高,你竟然说要请人上山,人家
都死二百年了,你上哪请去?

宋大哥骂完,煽了老黄两巴掌,伸个懒腰,四处看看,径直朝我走过来

\dldots

好不容易把宋大哥盼走,我爬出来,走到老黄面前,``啪''``啪''给了它两巴掌,恨恨的下山

路上碰到了吴军师、王矮虎、杨志\dldots
\section{}

今天又有兄弟上山,每当这时,山上最积极就是晁天王、宋大哥和杨志。

晁天王拉人入伙翻来覆去就那几招,许位子、送银子、拜把子,吹嘘一人入伙,全家光荣,忒低级!典型的脑袋被门板挤
了,捧着大粪当馒头,认不清形势。

你看重梁山泊的位子,人家大宋朝的官吏却不放在眼里;银子,更不在意,当官的谁家还缺个万儿八千的银子;至于跟强
盗拜把子,等于污了清白之躯,会耽误前程,更没人稀罕。

宋大哥就不同,见人说人话,见鬼说鬼话,跟军官在一起,探讨探讨报国杀敌,动情处拍案骂娘\dldots 跟官吏在一起,议
论议论奸臣误国,时不时来段官场荤段子\dldots 跟读书人在一起,讨论讨论孔孟之道,诌两句之乎者也,一席话,能跟
人结成生死之交

宋大哥今天跟人谈起枪棒功夫,指手画脚,一副内行人的鸟样,仿佛自己是不世出的盖世高手\dldots 把人唬的一愣一
愣的,当场要拜他为师

我心中暗笑,他要是真有这么好的功夫,也不至于跟阎婆惜一个小娘们大战三十回合,还不分胜负,逼着用刀才解决

不过宋大哥有这本事,谎话说的一本正经,你根本看不出一丝破绽,不由你不信,有时候连自己都信,他已经达到了说谎
的最高境界,连自己都分不清哪是真哪是假。

杨志这厮,青面獠牙,一副野猪模样,不论见谁,三句话不到,保准就问人家祖上是干啥的。问完后,不论人家问不问他,他
都要拉出他祖宗炫耀,三代将门之后,王侯杨令公嫡孙

跟他一起喝酒忒窝火,要是比子孙,自己子孙不争气比输了,还能回去抽两巴掌解解气,教育一番,他偏偏跟你比祖
宗,你总不能把祖宗从坟里拉出来重活一次\dldots

我祖宗没啥吹嘘的,上数八代都是清一色的穷光蛋,晁天王祖上是强盗,宋大哥祖上是地主,吴军师祖上是农民,都没啥
好炫耀的\dldots

有个好祖宗了不起啊?你祖宗再强,吃完饭能当银子使吗?你祖宗再厉害,不照样生出你这个不成器的鸟人?

后来关胜上山后,这厮才不那么张狂

关胜祖上不是人,是神,你祖宗再厉害,能跟神比吗?

惹恼了关二爷,跟阎王打个招呼,把孟婆汤给你换成二锅头,轮回时让你下辈子做奶牛,天天有人摸没人骑,憋屈死你

杨志的三代将门祖宗,没给他带来一丁点好运气,他跟林冲一样,点背到家,靠山山倒,靠人人跑。

林冲是,人在家中坐,没招谁没惹谁,横祸从天落,最后被逼上梁山。

杨志是,人在江湖漂,大事小事最后时刻准能挨一刀。

杨志本来是高干子弟,领导高看一眼,年纪轻轻抬举他坐到制使,下一步就是去基层历练历练,镀镀金,回头好提拔重
用。

朝廷派他押运花石纲,这可是个手到擒来的功劳,一堆破石头,又重又不值钱,强盗都不入眼。

十艘船,浩浩荡荡,从太湖驶入江南运河,接着转道长江,最后进入黄河。

朝廷公干,地方政府自然殷勤招待,一路喝喝小酒,看看美景,泡泡小妞,悠然自得。

眼看就到东京,杨志激动万分,船一到岸,他就该官升一级成团练使了

杨志这厮,见缝插针,抓住最后的机会,把所有人召集起来吹牛逼,当然,基本上都是他吹,别人听

第一百零八次吹嘘他老祖宗大战陈家峪的往事,当吹到力战不支,杨业死节的关头,慷慨激昂,唾沫横飞,估计河神爷这
一路也听腻歪了,平地一阵狂风

一船人端坐一侧,本就不稳,风一吹,船翻了。

吹牛逼害死人啊!

失落生辰纲可是要治罪的,杨志欲哭无泪,没有他祖宗死节的勇气,拍拍屁股,溜了

后来,大赦天下,杨志张罗了一担财物,上东京行贿,欲见高俅,但阎王好见小鬼难缠,文书要打点,衙役要打点,管家也
要打点\dldots 这厮算数没学好,打点完时,没了给高俅的礼,挨了一通义正言辞的臭骂,轰了出来。

杨志身无分文,流落街头,毕竟祖宗再厉害也不能当饭吃,只好卖刀,一时兴起,杀了挑衅的泼皮,最后迭配大名府充
军。

后来得到梁中书赏识,提拔他当了提辖,让他押送生辰纲,再次回京送礼,可惜在黄泥岗,又被晁盖等人抢了去。

这厮拍拍屁股,又溜了。

\section{}

最近山寨流行收小弟,山寨头领基本上都收了,武松除外,他独来独往惯了,不喜欢有人跟着。

有什么样的大哥,就有什么样的小弟,宋大哥的小弟比他还黑,王矮虎的小弟比他还矮,刘唐的小弟比他还猥琐

我也收了个小弟,名字挺怪,叫焦挺,当然,长的比我丑多了,牛头马面三角眼,兄弟们都说他长的像庙里的金刚。

有个小弟方便多了,省了很多尴尬。

以前,我一直以好汉自居,到处跟人搦战,每当威风凛凛的使出三十六路地煞斧,将对手砍倒在地时,立马就丢掉板斧去
人口袋里摸银子,一点好汉形象都没有,忒掉价。

现在好了,摸人口袋的事自然有焦挺代劳,我只在旁边提醒一下:戒指、手镯\dldots 好汉形象保持住了,银子也不少
拿,一举两得。

以前吃饭时,一顿饭花了多少银子我也不会算,当然,也不好意思算,强盗就是大碗喝酒,大碗吃肉,豪爽大气,视金钱如
粪土,吃完饭把大锭银子往桌子上一``掷'',来一句:店小二,甭找了。要多威风有多威风。

你要是前脚吃饱喝足吹完牛逼,后脚再打着算盘,对着账单一道菜一道菜的对价钱,要多丢份有多丢份!别说兄弟们瞧
不起,就是店小二都瞧不起。

现在好了,有焦挺代劳,等他算完没错后,再板起脸来埋怨两句:算啥算,不就那点银子嘛,老子还没放在眼里!

注意,上文说的是``掷'',不是``递'',也不是``拍''。

别看这轻轻一掷,学问大着那,要眼观六路,不能掷汤碗里,不然溅大家一身汤水,忒不好看;力道也不能大了,万一滚桌
子底下,刚刚还豪气十足,转眼再缩头撅屁股钻到桌底下满地乱摸,气氛就全没了。

为了练这一掷,我在家足足练了一个多月,打碎了几十只粗瓷大碗,才练到炉火纯青的地步,这可是我的拿手绝活,山上
兄弟掌握的不多,武松那么牛逼,结账时也只敢拿银子往桌子上一拍,也不敢乱掷。

\section{}

曹正把朱武的书童打了。

曹正是林冲的徒弟,武功一般,但杀猪宰羊别有一套,手艺纯熟,取猪命从来只用一刀,猪连哼都来不及哼一声,外号操
刀鬼,我们都叫他曹一刀

其实杀猪比杀人难多了,杀人很快当,照头一鸟斧,立马死翘翘,杀猪就没那么轻快。

上次我闲的无聊,去帮曹正杀猪,挑了头最大的,站定,提气,凝神,照着脖艮猛一板斧,这是我平生绝学,本以为十拿九
稳。

没想到,这厮头都砍掉一半,在脖子下面耷拉着,血蹭蹭乱喷,竟然还没死,冲我猛冲过来

俄的娘哎!忒吓人了,我调头就逃,板斧也扔了,鞋子也丢了,不知摔了多少跤,这厮还不依不饶,一路狂追

两条腿的怎么能跑过四条腿的,眼瞅着就要追上,我急中生智,想去骑马,没想到马群往这边一瞥,立马炸了群,四散逃
命,马棚都被拽塌了。

最后没辙,只好爬上聚义厅前的``替天行道''大旗,这厮在下面转悠了一炷香时间,才血尽而亡。

太丢人了,在江州杀的血流成河,眼睛眨都没眨一下,打祝家庄时杀的遍身血污,也没害怕过,竟然被一头猪追着跑!自
己都瞧不起自己!

哎!从那之后,在酒桌上一吹牛逼,兄弟们就拿这事挤兑我,说我在牲畜界给人丢脸了。曹正虽然杀猪很有一套,但办
事不靠谱,交给他的事没有办不错的时候。

鲁智深和杨志取二龙山时,攻打数天,毫无进展,倒不是二人武艺不济,而是邓龙那厮吓破了胆,任你在山下问候他八辈
祖宗,压根就不敢下来交战,二龙山易守难攻,急的两人团团转。

曹正献计,把鲁智深绑起来,做个活死扣,就说喝醉了抓住的,要献给大王,赚开山隘,送上山去,等见到邓龙时,把活扣
打开,趁其不备,袭杀邓龙,夺取山寨。

鲁智深和杨志欣然同意,曹正和自己小舅子押着鲁智深,杨志拿着戒刀和鲁智深的禅杖,一行人上了山,一切顺利

其实计是好计,没啥破绽,但有个最关键环节,那就是,曹正一定要在第一时间解开活扣。

我想,如果鲁智深和杨志了解曹正的话,打死都不会让他去解扣。

关键时刻,曹正大喊一声,动手,同时伸手去拽活扣。鲁智深大骂一声,``兀那挫鸟,哪里跑'',腾地从地下跳起来,杨志
也大喝一声,把禅杖扔向鲁智深,抽刀战邓龙。

但接下来,跟预想的截然不同,鲁智深站是站起来了,但挣了两下,绳索却没解开,曹正拽错了,他拽的是,死扣

鲁智深腾不出手去接禅杖,杨志又扔的太正,``梆''的一下,脑袋上被砸了个偌大血窟窿

形势急转直下,杨志独虎难敌群狼,曹正和他小舅子武艺泛泛,眼看一行人就命丧于此,关键时刻,曹小舅掏出杀猪刀三
下五除二割断绳索,鲁智深拾起禅杖奋力打死邓龙,才救了众人一命。

鲁智深擦着脸上的冷汗,问曹小舅怎么会随身带着杀猪刀,曹小舅淡然的说,我姐夫拽错已经不是一次两次了,早有准
备!

\section{}

曹正混的很不如意,山寨大小头领都有自己的一片天地,连偷鸡摸狗的时迁,都有了自己的码头,上山下山都有小弟跟
着,出门有人拎包,进门有人捶背,很是威风。

再不济,如菜园子张青,现在也管着偌大的酒店,手底下有二十几号人,不用亲自动手干活,没事时躺太师椅上晒晒太
阳,有事时板起脸来训训手下,也很快活。

曹正就惨多了,天天蹲在屠宰场,拿着把杀猪刀,白刀子进,红刀子出,天天一身腥味,连酒店的厨师都不如,厨师炒完菜
还能先尝一口,他连这个口福都没有。

兄弟们聚会时,各自吹能道会,他只有坐那里听的份,落落寡欢,唾沫蛋子飞脸上都不好意思去擦。

曹正上头有人,林冲是他师傅,按说不该混到如此地步,主要是因为他曾惹恼了宋大哥,宋大哥说他是烂泥扶不上墙,难
堪大任。曹正刚上山时,林冲把他推荐给宋大哥,林冲是山寨元老,说话有分量,宋大哥很给面子,当即予以重任,让
他操持跟李家庄歃血为盟的事。

盟誓地点定在金沙滩,那天我也去了,我和宋大哥先到。

宋大哥反复问曹正,李大庄主长啥鸟样\dldots

曹正这厮说长的五大三粗,潇洒魁梧\dldots

日过杆头,晴空万里,马蹄声急

对方来了七匹马,五马殿后,两马当先,左边是一中年清瘦汉子,右边是一彪形大汉,两人同时下马,走过来

宋大哥瞟了曹正一眼,这厮瞪着茫然的小眼,没啥反应。

宋大哥犹豫了一下,大踏步朝彪形大汉走去,老远就伸出双手,握着彪形大汉的手直哆嗦,笑容可掬的说:李大庄主,久
仰大名,今日一见,果然堂堂一表,凛凛一躯,大慰小可平生之思。

我注意到,彪形大汉像是进错洞房的新媳妇,张着嘴,想说什么又不敢说,想抽回手又不好意思。

我当时心就咯噔一下,坏了,握错了。

果然,旁边的清瘦汉子拉拉宋大哥的袖子,说:宋头领,他是我的小弟,我是李庄主。

``\dldots ''

宋大哥不愧是做头领的,稍一愣神的功夫,立马反应过来,松开大汉的手,转而握住李庄主的手,热情的说:李庄主天庭
饱满地阁方圆,果然是人中豪杰,今日一见,不枉平生

李庄主也恭维宋大哥义薄云天,仗义疏财,远近闻名。
\section{}

寒暄完,准备歃血为盟,盟誓必备三伙什:乌鸡、白马、誓箭

宋大哥和李庄主对着供桌拜了三拜。

曹正把乌鸡拎过来,凌空一刀划开气管,正准备往碗里沥血,结果一个没拿稳,那鸡``蹭''的一下飞了,扑蹬着翅膀满地
乱跑。

曹正楞了一下,撒丫子就追,我本来冷着脸提着板斧扮酷,这下没辙,形象不要了,扔了板斧一溜烟跟着追,彪形大汉也
扔了大刀加入进来

乌鸡一会上天,一会下地,两只小腿跑的溜溜的\dldots

金沙滩上全是泥沙,三个大老爷们深一脚浅一脚,又扑又跳,弄的浑身脏兮兮的,折腾半天,终于抓到了,但鸡血流没了

宋大哥脸色一沉,李庄主忙圆场说,不用鸡血也无所谓,不还有白马吗?

宋大哥左右看看,问曹正,白马哪?

曹正指了指远处,说:白马没淘换到,只有这个了。

我一看,娘哎,一头黑驴!

没白马你弄头灰马红马也行啊,至少凑活,你竟然弄头黑驴。

宋大哥脸色``唰''的一下,变的铁青。

李庄主忙说用啥都一样,有这份心就行。

两人同饮了一碗驴血,宋大哥表情淡定,但李庄主就没这份能耐,呲牙咧嘴,喉头动了几下,打了几个饱嗝,强忍着没吐
出来,看那样,比喝毒药都难受。

下面起誓,曹正忙递上准备好的誓箭

宋大哥和李庄主并跪在地,一脸庄重,宋大哥擎着誓箭,起誓说:皇天后土,实所共鉴,我与李庄主,在此杀黑驴盟誓,结
为异性兄弟,互通友好,永不为敌,若违此誓,当如此箭。

宋大哥用力一掰,箭纹丝未动。

宋大哥沉沉气,再次发力,还是没断

宋大哥有些窘迫,收回胳膊,把箭担在腰间,上身弯曲,两臂用力,脸胀的通红\dldots

誓箭只弯了弯,又弹回原样

我心里纳闷,箭柄都是木头做的,誓箭还要在当中切个豁口,让折箭的人稍微用力即可折断,这只箭怎么了?

我悄悄问曹正,那只箭怎么回事?

这厮竟然说他特地请汤隆花两天时间做了支铁箭,还是千年寒铁做的

宋大哥和李庄主累的气喘吁吁,头上直冒热气,还未折断

双方都没带弓箭,但盟誓完不弄坏点东西忒不像话,就像大便完不擦屁股,忒难受。

最后,宋大哥灵机一动,拿起桌子上的粗瓷大碗,朝地下猛摔下去,说``若违誓,有如此碗''

碗落地,完好无缺

地是沙地,砸出好大一个坑。

众人面面相觑,想笑又不敢笑

做小弟的自然不能让老大难看,我忙把板斧垫在沙地上,捡起碗,又让宋大哥摔了一次,这才完事。

回去的路上,宋大哥脸阴的吓人。

回山寨第一件事,就是撤了曹正的头领,发配屠宰场干活。

\section{}

鲁智深刚升了厅级干部,分了套房子,打发曹正去找朱武,想讨副对联。恰好朱武不在,曹正就在书房等,不知朱武的书
童是看曹正不顺眼,还是被他身上的骚味恶心着了,想让他快点滚蛋,就顺手给写了副对联。

曹正不认识字,拿着白底黑字的对联,如获至宝,兴冲冲的跑回去,帮鲁智深挂门楹上。

鲁智深高升,众兄弟不管乐不乐意,都得去随份

王矮虎先到,这厮大字不识几个,但特爱装文化人,看到带字的就往前凑,背着手,昂着头,看完左边看右边,看完右边看
左边,别人看对联都是从上往下看,他倒好,从下往上看,边看边点头:好对子,寓意深远,千古绝对,字里行间有步步高
升之意、、

这厮看完还似笑非笑的看着我,问我懂不懂

我想,再不济也不能输给这鸟人,也装模作样的审了一下,连夸好对

说话功夫,花荣来了,刚要进门,看到对子后,摇摇头,把银子给了鲁智深,不顾再三挽留,说孩子病了,需要照顾,闪了

秦明来了,看到对联后,笑笑,离得门槛远远地,把银子抛给了鲁智深,说老婆怀孕了,需要伺候,溜了

先后又来了几拨兄弟,看到对联后都摇摇头走了

武松来了,看到对联后冷笑,原来对联是这么写的

上联:进门全家死绝 下联:出门七窍流血 横批:满门忠烈

鲁智深气炸了肺,大骂曹正屁大点事都办不好

曹正脸青一块紫一块,二话不说,立马去把书童捆来,当众煽了二十多嘴巴子,书童的脸当时就肿的老高,白里透红,跟
腚似地。

\section{}

中午又喝多了,鲁智深喝一碗酒骂一句娘,把书童祖坟里躺着的几位先人都问候了百八十遍,内容没啥新意,无非跟书
童先人发生超越男女的不正当关系。

我想,一件事,要伤害你,只有敌人不行,还得有朋友。

鲁智深脾气暴躁,动辄问候人老母,在山上的朋友不多,武松算一个,林冲算一个,我也算一个。

我不识字,若武松不说,林冲也不说,鲁智深永远也不会知道对联内容,也就不会生气

敌人背后诋毁你,朋友再把话传给你,你再生气伤身,朋友是帮了你,还是害了你? 

\section{}

我还没想明白这个问题,朱武来了,他书童被打成包子样,不能不管。

人有时候很奇怪,你打他一巴掌,他可能就忍了,但你要是打他身边人一巴掌,他可能会跟你拼命。

朱武不吵不闹,表情淡定,随了十两银子,还有一副对联。

武松把对联念了一遍

上联:碧沙好倚透夏天 下联:征槽喝打陆十舅 横联:卧石 竹贲

众人茫然,无人知道是啥意思

朱武说这幅对联暗含玄机,挂在门楹上可驱灾避祸,鲁智深乐呵呵的让曹正挂上了。(天涯的网友们,有人知道对联是
啥意思吗?)

朱武问鲁智深他书童的事咋解决,鲁智深说书童是自己找打,朱武说打狗也得看主人,坚持鲁智深当众道歉,鲁智深说
道歉不可能,但可以让书童打回来,朱武说他那张脸厚的锥子都能捅折了,打了等于白打,两人越说越僵

最后,朱武提议,两人比赛,比力气,赢了的说了算。

鲁智深最得意的就是他那身蛮力,没事就到处显摆,山上的树都被他拔了一半,一听比力气,欣然同意

朱武说谁能把自己提起来,谁就算赢。

鲁智深一听,破口大骂,这不是难为我吗,老子又没头发,怎么把自己提起来?

朱武冷哼一声,拿根绳子,一头拴腰上,另一头搭树杈上,三下两下,把自己提了起来。

鲁智深当众道歉,嘴里不停的嘟囔,要是我有头发,也不会输

其实,在两人比赛前,我就知道朱武一定会赢。

我这人笨,很多事情看不透,但我有个绝招,对人不对事。

因为我知道,事,没有好坏之分,得看谁去做

比如,站在山顶装深沉,如果是宋大哥,那就叫忧国忧民,如果是我,就是装逼

再比如,宋大哥提议过节开灯会,众兄弟欢呼雀跃,说宋大哥一团和气,与民同乐,如果是我提,众兄弟会骂我脑袋抽风,浪
费山寨粮食

所以,我从来都是对人不对事,每当开会,宋大哥赞成的,我都高呼万岁,宋大哥反对的,我就当放屁

每当朱武打赌,无论赌什么,就是赌母猪上树,我都赌他赢!

写在题外:

我时常想,人活着的意义是什么?

在古代,可能是薄田两亩,贤妻一名,美妾无数,儿孙满堂\dldots

而在当代,可能是别墅住着,宝马开着,女子搂着,票子数着\dldots

总之,离不开物质。

但我想,总会有一种价值观可以超越古今,永恒存在,但这种价值观是在喧嚣的闹市中无法体会到的。

每天苦苦努力,只为了一栋安身立命的房子,这样的生活是否有意义?

人活着,应该有更重要的事情

一直想跳出现在的生活框架,但始终无法挣脱,毕竟人活在世上,有很多不得不去担当的责任,你不能逃避\dldots

但现在,客观条件已经具备

等写完《李逵日记》,我就孤身入藏,寻找那亘古不变的光明

\section{}

山寨乏闷,没啥娱乐,众兄弟无聊时就凑一块喝喝酒赌赌钱,当然,也是分圈子的,小兵跟小兵玩,堂级干部跟堂级干部
玩,厅级干部跟厅级干部玩。

大家都平级,玩起来爽快,没有领导,不用故意输,没有下级,也不用在乎面子。别看时迁那厮,在下属面前昂首挺胸,道
貌盎然,一副不食人间烟火的鸟样,跟我们一起赌时,输了照旧帽子一摘,额头上画只乌龟,钻人裤裆。

王矮虎这厮最他妈的无耻,专门跟下级玩,他那几个部下输的咸菜都吃不起了,前天还看到宋万蹲在大门口,对着西北
风,左手拿馒头右手端水壶,啃一口馒头,就一口凉水,就一口凉水,啃一口馒头\dldots 忒惨了。你说人家来当强盗容
易吗,抛妻离子的,来受这份鸟罪。

宋大哥刚上山时,还跟晁天王一起玩玩筛子,现在两人见面都恨不得平吞了对方,早就尿不到一块了。晁天王天天盼着
宋大哥骑马摔死,宋大哥天天诅咒晁天王精尽而亡\dldots

宋大哥说今夜有要紧事,让我三更时分叫醒他,反复叮嘱,还说误了时辰砍我狗头。

草他妈的,你当领导的晚上睡觉,偏我当下级的是属夜猫子的?不拿下级当人看!

我打小有个习惯,身子一挨床,一觉到天亮,屋顶塌了都听不见,这可咋办?总不能一夜不睡

我想了想,叫来鲍旭,告诉他,三更前必须叫醒我,否则砍他狗头,吩咐完躺下就睡了

还是当领导好,有啥事动动嘴就行。
\section{}

没心没肺睡的香,做了两个梦,一个好梦,一个恶梦,好梦是梦到吃咸鱼,恶梦是梦到小时候往邻居烟筒里塞稻草,邻居
告我爹,我爹拿大耳瓜子煽我,啪啪的,稀里糊涂就醒了,脸上似乎还火辣辣的。

鲍旭瞪着通红的熊猫眼,凑到眼前说,黑哥,你可醒了!

还不到三更,这厮办事稳当,我拍拍他肩膀让他回去睡了。

宋大哥趴在地上,睡的跟死猪似的,呼噜震天响

我过去喊了他半天,连哼都不哼一声,摇摇胳膊,还没反应。我突然心想,你这狗日的,天天人模狗样装清高,我天天夹
着尾巴装孙子,这下看我怎么整治你。

我把半年没洗的臭袜子塞他嘴里,这厮竟然不停的吧唧嘴,娘的,给你二两颜料你还开起染坊,我把袜子扯出来,甩手给
他两巴掌,可能下手重了,黑脸上浮起十道指痕,黑亮亮的

这厮终于醒了,睁开眼,抹抹嘴,再揉揉脸,说梦到偷吃咸鱼,被人抓住煽了两巴掌,还夸我办事牢靠!

我心中好一阵乐。
\section{}

今夜朔月,天无星,地无灯,伸手不见五指。

宋大哥要我跟他下山,别人晚上下山都要穿夜行衣,我和宋大哥就省事多了,把衣服一扒,比什么夜行衣都管用,从来没
被人看到过。

出了金沙滩,进了济州府,从后墙翻进府尹张叔夜家

张叔夜早已在书房等候,两人关门嘀咕了半天,不知道说啥。

我发现一个有趣的规律,每次宋大哥跟张叔夜嘀咕完,过不了几天,梁山就起兵马攻打济州府,奇怪的是,每次都打不下
来,有几次明明都快攻破了,几十号兄弟都已爬上了城墙,关键时刻,宋大哥总是鸣锣收兵。

过不久,朝廷就会调拨粮草来,宋大哥带人截了粮草,张叔夜再领兵杀出来,混战一阵,宋大哥扔掉一部分粮草,引兵回
山,张叔夜押着粮草耀武扬威的回城。

山寨有了粮草,宋大哥威望日高,张叔夜也因独守孤城,剿贼有功,多次得到朝廷嘉奖,成为大宋朝赫赫有名的忠臣。

只是可惜了那些打仗最勇猛的兄弟,不是跳下城墙摔断腿,就是被抓住,一刀一个剁了,脑袋挂在城墙上,成了张叔夜的
活牌坊。
\section{}

死的兄弟,梁山泊专门在半山腰给建了座庙,牌位供里面,香火不断,取名忠烈祠。

每年过节,宋大哥都要带着众兄弟前去祭奠,每次都慷慨激昂,泣涕泗流,有几次竟然哭昏过去。

众兄弟无不感动的热泪狂奔,哀嚎一片,盛赞宋大哥仁义道德,义薄云天,宋大哥趁机醒来,号召大家以埋在地下的兄弟
为榜样,努力杀贼,忠心报国。

我基本不哭,太假!对着几块木头有啥可哭的?况且平常也没啥交情,甚至叫啥名字都不知道。

我每次都站在一边观察众兄弟的哭相。

鲁智深是哇哇大哭,鼻涕眼泪一大把,一边哭一边用袖子抹;王矮虎是干嚎,声音数他高,一滴眼泪都没有;武松眼圈红
红的,冷着脸,不作声;顾大嫂用袖子蒙着脸,哭的前仰后合,声音抑扬顿挫,忽高忽低,哭的比乐和唱的都好听;孙二娘
最逗,趴着身子,撅着屁股,一边哭一边用巴掌拍地,声音拉的老长,不像哭丧,倒像唱戏。

哭,绝对是门学问,不但要分场合,还要分对象。

兄弟老婆死了,兄弟在一边默默流泪,你嚎啕大哭,不合适,容易让人误会。

邻居孩子被狼叼走了,你要是比他爹哭的还厉害,也不太合适。

宋大哥对哭运用的炉火纯青,想怎么哭就怎么哭,哭到什么程度,恰到好处,既能让人感到他的诚意,也不会让人误会。

鲁智深这憨货,每次哭完,宋大哥说结束了,众人都起身拍拍屁股走了,他还在哭,真他们二百五!
\section{}

张叔夜在济州府府尹位子上,一呆就是几十个年头,早过了退休年龄,身子也埋黄土半截了,还赖在位子上。

倒不是他主动赖着不走,而是朝廷离不开他,往往朝廷刚下调令,宋大哥就带着兄弟下山闹腾,皇帝老儿只好收回成令,让
他勉为其难,继续支撑危局,发挥余热。

有次,他都打包滚回老家了,结果新任府尹一天夜里被人黑了,死的忒惨,全家老少十八口,一刀一个,无人幸免。

又调了个府尹来,人还没认全,全家死绝。

济州府尹成了烫手山芋,谁也不来,打死都不来,这个可以理解,千里做官只为财,没必要连命都搭上。

皇帝老儿决定从京东路选一名闲官任职,符合条件的有四名:张吾能,李巨谈,王本竹,刘扯丹

张吾能首先得到消息,连夜上书,说他前年骑马摔折了腿,如今旧伤复发,不能视事,为免耽误朝政,特辞官引归故里

皇帝老儿假惺惺的慰问两句,准奏!

李巨谈紧跟着上书,说他老娘五年前归西,朝廷有制,至亲去世可服丧三年,他为朝廷大事,一直忍痛坚持工作,丧假一
直未用,近来思母日切,心内伤悲,特请假三年,回家守孝,云云

皇帝老儿气的大笔一挥,革职滚蛋,同时下诏,无特殊原因,一律不准请假。

刘扯丹一看不妙,急得团团转,求爷爷告奶奶,上蹿下跳,但谁也帮不了他,皇帝金口,岂能出尔反尔?刘扯丹心生一计,
牙一咬,心一横,挥刀把腿砍断了,皇帝一看,没辙,只能王本竹了

王本竹真倒霉,刚花一千两银子买了个闲官,本以为填个空缺,好好捞一把,没想到碰上这么摊子破事。家里该死的都
已死过了,丧假请不了,想来想去,只能玩苦肉计,这厮更狠,拿着板状照脑袋来了一下,结果力道没掌握好,不小心把自
己拍死了。
\section{}

济州府尹也不能没有人当啊,捕快们天天拿衙门当赌场,十里八村的破落户,二流子,纷纷赶来下注,熙熙攘攘,吆五喝
六,比菜市场都热闹,人太多,一桌不够,``明镜高悬''的大匾摘下来又凑一桌

偷盗事件屡有发生,别说偷鸡摸狗的,连惊堂木、威武棍都被人顺走当了柴火

皇帝老儿愁得坐卧不宁,茶饭不思,最后,蔡太师推荐了刘义学

刘义学自幼聪明,过目成诵,被誉为神童,刷新了大宋朝的几项记录,十五岁中秀才,十八岁中举人,二十一岁中状元,一
时朝野闻名。

考的好不一定分的好,刘义学家徒四壁,没钱上下打点,结果只分到个翰林院编修的虚职,天天拿着四书五经找错别字,找
不到还扣工钱。

看到高廉、蔡九这帮不学无术的花花太岁,竟然坐上知府,心中很不是滋味。刘义学是个碰头就弯腰的人,他想明白
了,要想做官,就得有靠山,但他八辈子打的着的亲戚全是穷光蛋,没啥靠头,想了许久,决定向领导靠拢

但领导的马屁不是好拍的,屁股就那么大,有那么多人在一边举着巴掌伺候着,哪能轮得到你?

一个偶然的机会,他得到消息,蔡太师有一个小女儿,名叫蔡十九,失散多年,不知所踪

刘义学眼前一亮,要是帮蔡太师了了这桩心病,那他就算傍到大树,这辈子有奔头了,况且也不是太难,蔡十九胸前有三
颗大痣,好认!

说干就干,他开始了漫长的寻人之路,冬顶寒雪,夏冒酷暑,走南闯北,逢人三句话不到就问人胸前有痣否?为此挨了无
数巴掌,脸上都磨出一层老茧,有好几次都被当做流氓投入大牢

在阴暗潮湿的牢房里,他咬牙发誓,一定要在蔡太师退休前找到蔡十九

可能是他的诚心感动了佛祖,三年后,他找到了。

他将蔡十九带回东京时,正是中秋之夜

蔡府合家团圆,独缺十九,一家人正唏嘘不已

这时,刘义学出场了,他矜持着,得瑟着,把十九请了出来

众人喜极而泣,称赞他是蔡家的大恩人

蔡太师拍着肩膀夸他精明干练,是个人才

他似乎看到,荣华富贵正在向他招手,前途,一片光明。

这时,意外,发生了

蔡太师的八旬老母,因高兴过头,一口痰没上来,一命呜呼

这下糟了,喜事变丧事,拍马屁拍出人命来了,大恩人转眼变成了大仇人。

蔡太师从那之后一直对他心存芥蒂,这不正好济州府缺人,就把他推荐给皇上

皇上立马准奏,但刘义学死活不干,拒不赴命。

皇上一生气,威胁说不去就抄家问斩。

刘义学没辙,回祖坟上磕个头,交代完后事,一路哭哭啼啼的来了济州府。

\section{}

刘义学到济州府当晚,惨死住处,身体大卸八块,脑袋被砍了十数刀,面目全非,入殓师折腾一整天,都没把肢节凑齐,最
后用笤帚划拉划拉,囫囵埋了。

案子是我和武松做的

那夜,宋大哥派我们两人下山,取刘义学狗命

我们到时,屋门四敞八开,打眼一看,客厅中央摆一棺材,刘义学一身敛服,坐在椅子上,旁边桌子上放一白练

刘义学对我们的到来一点都不惊讶,只是淡淡的说了句,你们终于来了

我一愣,隐隐约约觉得不妙,武松反应奇快,大喝一声:不好,有埋伏,说罢朝窗户奔去,凌空飞起踹碎窗户,扑通一声,跳
入窗外的池塘中

我紧随其后,也想学他凌空一跳,结果高度不够,两脚被窗框一绊,头重脚轻,一头栽进去

爬出来一看,没啥伏兵,放下心来,我把脸上的淤泥洗掉,武松把耳朵上挂的水草摘掉,毕竟都是强盗圈里有头有脸的
人,得注意形象。

两人重新进入客厅,刘义学端坐没动,说没啥埋伏,早就料到这一刻,希望我们给他个全尸。

我们是讲仁义的强盗,同意了

这厮拿起白练,踩着凳子,在门框上打个结,长叹一声,说他从小立志做个好官,上报皇恩、下安黎庶,为百姓伸张正义,没
想到如今却落个如此下场\dldots 最后这厮大骂苍天无眼,哀叹好官难做\dldots

椅子倒地,刘义学身体悬空

武松叹口气,说这厮是个好官,话未说完,只听咔嚓一声,门梁断了,刘义学摔倒在地,呲牙咧嘴

环顾四周,只能吊屋梁上了,刘义学够不着,希望我们能帮他把白练挂上去,还说我们帮他上吊的大恩大德他没齿不忘,若
来生有机会,一定帮回来

话说到这份上,不忙不行,武松站椅子上,我踩他肩膀上,够不着,他说他比我高,让我站椅子上,他踩我肩膀上,结果还
是够不着

折腾了半天,都没挂上,武松有些不耐烦了,说直接来一刀,他刀法准,一刀扎心脏,也是全尸

我说还是来一鸟斧吧,对准脖艮,力道掌握好,脑袋掉不下来,顶多流点血,两人争了起来

这时,刘义学脸色苍白,跪在地上说:你们掐死我吧,太他妈的吓人了!

我们没杀他,让他走了,永远不要再出现,那个死的,另有其人

皇帝老儿最后没辙,只好重金请回张叔夜,让他鞠躬精粹,争取死在位置上。


\section{}

从张叔夜家出来,已是五更天,宋大哥直喊累,问哪里可以休息,我想深更半夜的,客栈都打烊了,整个济州府,估计只有
翠红楼开着。

朱武说过,陪领导干十件好事不如陪他干一件坏事关系来的铁。他还编了个顺口溜,说什么陪领导工作受累,不如陪他
闲扯开会,陪他闲扯开会,不如陪他尽情一醉,陪他尽情一醉,不如陪他贪污受贿,陪他贪污受贿,不如陪他一女同睡。

要不陪宋大哥去逛逛青楼吧。

我小心翼翼的说,这个时辰只有翠红楼还开着,要不咱去看看?

宋大哥脸一沉,劈头盖脸训了我一顿,说山寨头领怎么能做这等鸟事?我心想,你他娘的在郓城县养小妾时,鸟也没少干
事,现在倒装清高!

不过这话只能在心里说说,我摆出副惭愧的表情,正要夸他人品高尚,话还未出口,这厮换了副万般无奈的表情,叹口气
说,既然没别的地方去,也只好如此了

我心中暗笑,生生把话咽了回去。

\section{}

翠红楼是济州府最有名的青楼,是无数男人心中的圣地。

据兄弟们说,里面的姑娘不但个顶个的娇嫩水灵,而且琴棋书画样样精通。头牌玉如意更是色艺俱全,据说前年,大宋
朝新科状元来这里千金买笑,与玉如意行酒令,结果玉如意酒未沾唇,这厮已经醉的一塌糊涂,床都爬不上去,在地上躺
了一晚,第二天嚷嚷着要打折,把读书人的脸都丢尽了。

山上日子枯燥,兄弟们本来懒懒散散,打劫也不积极,东一榔头西一棒槌,够吃够喝就行,天天得过且过,任宋大哥把
``替天行道''吹的天花乱坠,也不见丝毫起色,毕竟口号再响,也不能当馒头吃,也不能当女人睡

但自从玉如意投身风尘,兄弟们立马找到了奋斗目标,个个跟打了鸡冠血似地兴奋,大白天就敢去闹市抢劫,没办法,玉
如意的标价忒高,两腿一翘,收价比堂级干部一年的俸禄都多。

兄弟们太过积极,闹得方圆几十里鸡犬不宁,衙门天天挤满告状的人,最后府尹张叔夜只好出面干预,玉如意去了东京
发展,据说后来改名李师师,现在跟皇上打的火热。

虽然玉如意走了,但翠红楼的生意依然红火。

在济州,人气最旺的就数翠红楼和对面的小相国寺,两者隔着二里地遥遥相望。

翠红楼以姑娘美色声名远播,小相国寺以佛法精深闻名朝野,很多人千里迢迢赶来翠红楼春宵一度,第二天满怀内疚的
前往小相国寺参禅悟道,他们见到高僧的第一句话,惊人的相似,大师,我们昨夜似曾相遇

后来,翠红楼请府尹张叔夜写了幅超牛逼的对联

上联,妙口一开广迎八方客,下联,红唇两片吹尽天下箫,横联,欢迎来日

小相国寺不甘示弱,也请人写了幅对联

上联,金口一开点播四方痴迷客,下联,玉音天降普度众生有缘人

横联,回头是岸

但自从这幅对联挂上后,很多香客看了,连庙门也不进,直接回头钻进了对面的翠红楼

\section{}

我和宋大哥大摇大摆的进了翠红楼,老鸨很是热情,一边往雅间领,一边跟宋大哥攀谈,客官哪里来?怎么称呼?

宋大哥豪爽的说,姓晁名盖,打梁山泊来

老鸨``哎呀''一声说道,真不巧,你兄弟宋江前脚刚走

宋大哥一愣,问那个宋江长啥模样

老鸨气呼呼的说,五大三粗,直愣愣的,每次干完提上裤子就走,从来没给过一两银子,还扬言他是宋江他怕谁,让我有
本事上梁山找他要去\dldots 忒他娘的缺德\dldots 惹恼了老娘,去把替天行道的大旗扯下来当鞍马布\dldots 我们做
皮肉生意容易吗?千人骑万人入,起早贪黑挣点辛苦钱\dldots 每天迎来送往的人多了去了,就没见过他这么不要脸的
\dldots 晁大哥你说是吧,宋江这厮是不是忒不要脸?

宋大哥黑脸涨的紫红,点头也不是,摇头也不是

落座定,水果点心摆上,老鸨喊来两个姑娘,一个婀娜多姿,胸大如奶牛,一个小巧玲珑,嘴唇涂得鲜红

这种场合,当然是领导先来,跟领导抢女人,是大忌讳,宋大哥点了奶牛,我只好要了红嘴唇

宋大哥刚开始正襟危坐,摆出副正人君子的鸟样,三杯两盏下肚,放开手脚,露了原形,跟奶牛黏糊起来,百忙中腾出左
手指着我说,别看这厮长的丑,打起仗来卖力,是我的左右手

我心想,别他娘的别埋汰我,你右手都伸人裙子底下了
\section{}

宋大哥喝酒很有规律,前三杯酒,正襟危坐,祝愿皇上洪福齐天,百姓安居乐业,鼓励各位忠心报国;后三杯酒,微醉,揎
拳捋袖,指指点点,吹嘘自己权谋堪比萧何;又三杯酒,小醉,开始挤兑晁天王文武不通,无能无德;再三杯酒,大醉,大骂
朝廷昏暗腐败,诅咒高俅等四奸臣早日死绝;再喝,烂醉,别人烂醉时都是抱着树喊娘,他烂醉时抱着树喊爹。

我看宋大哥开始问候高俅的八辈祖宗,知道喝的差不多了,让老鸨安排房间休息,我提着酒坛继续喝

楼上楼下叫声一片,我充耳不闻,自顾自的喝酒,红嘴唇在旁边呆坐许久,有些不耐烦

我拿出一大锭银子掷桌子上,她眼睛一亮,表情立马像面条下在开水里,活泛起来,直往我身上凑,说让她干啥都行

我说那你就陪我说会话吧

红嘴唇看我的眼神充满狐疑,说我跟别的男人不一样,我问哪里不一样

她说以她的经验,男人无外乎两种,一种是禽兽,另外一种是衣冠禽兽

我问怎么讲

她说,梁山上一个叫王矮虎的,大白天敢在闹市调戏她,晚上来了二话不说抬腿就往她身上爬,这就叫禽兽

有的人,大白天道貌岸然,一副正人君子的鸟样,大骂你伤风败俗,淫荡无耻,晚上来了还给你讲一通大道理,讲完伸手
扯你裤腰带,这就叫衣冠禽兽。梁山上有一个叫吴用的,就是衣冠禽兽中的极品,每次来都要讲半个小时的三从四德,
忠义节操,完事后还嫌你叫声不够高。

我听得哈哈大笑,问她那我是哪种?她嗫喏半天没敢说话。

\section{}

我问她,平常来这里的人多吗,她说,多得很,甭管是街头乞丐、贩夫走卒,还是满口仁义道德的读书人,抑或是高高在
上的官员,只要是男人,没有不来嫖的,就连对面小相国寺的和尚,都隔三岔五的来拯救她们这些肮脏不堪的灵魂。

我不怎么信,说小相国寺还是有很多得道高僧的

红嘴唇嘴一撇,一脸不屑,告诉我说,你现在去敲门,开门后说阿弥陀佛的就算是高僧。

我想,这很简单,哪个和尚见人不是双手合十来一句阿弥陀佛,反正无事,我出了翠红楼直奔对面的小相国寺

天还黑着,无风无月

我踹了半天门,才有人应,还没等我说话,这厮开门第一句就是:我曰你妈,佛门净地,你大半夜的嚷嚷什么?

娘的,照头一鸟斧!

挨个去敲门,连敲了八九个,连砍了八九斧

最后,到了主持方丈门前,举手欲敲,转念一想,万一这厮也来一句三字经,今晚就白费力气了

我脑袋一转,有了办法,敲门的同时,把板斧举在面前,这样对方看到后态度可能要好一些

结果,方丈穿着大裤衩,睡眼惺忪的打开门,看到明晃晃的板斧后,一声``我操'',当即瘫倒在地。

哎!守着金山找不到金子,在庙里想听句``阿弥陀佛''竟如此难!

大雄宝殿里,我放下板斧,扑倒在地,对着佛祖拜了三拜,祈求他原谅我佛门净地大开杀戒的罪孽,转身离去。

我虽然不敬和尚,但对佛祖,一直心存敬畏!
\section{}

宋大哥回山后蒙头就睡,聚义厅开扯淡会也没去,众兄弟们以为他病了,纷纷提着礼物前去探望,个个一脸关切,叮嘱宋
大哥工作别太拼命,好好保重身体\dldots

扈三娘摸着宋大哥的手,流下了伤心的泪水,说宋大哥是为山寨操劳过度才累病的\dldots 说到动情处,哽咽难言
\dldots

神医安道全把熬好的补药双手呈到宋大哥床前,还特地强调是亲手熬得,嘱咐宋大哥趁热喝了\dldots

草他妈的,他亲爹生病时都没见他这么勤快

时迁跟郁保四在宋大哥床前结了死仇。

时迁这厮一直想升厅级干部,但条件不够,历史上有污点,很多人不同意。

这厮下狠心花五百两银子买了只千年东北参,想送给宋大哥,从三个月前就盼着合适机会,这不好不容易盼到宋大哥病
了,高兴坏了,屁颠屁颠跑去。

床前围了一圈兄弟,时迁正好排在郁保四后面,别的兄弟都是提着礼物站在床前说两句不痛不痒的吉利话,然后放下礼
物让给后面的兄弟表现

郁保四这厮不知道是想让宋大哥对他加深印象,还是真的从心底里关心,从开始说到结束,任时迁捅了他七八十下都不
挪位置

郁保四身高一丈,腰围三尺,足足顶时迁四个大,遮的严严实实的,从前到后,时迁压根就没见上宋大哥的面

礼物提来,不好再提回去,只好放下,时迁算倒霉到家了,送了重礼人家还不知道谁送的

出了门,两人就打一块去了,时迁瘦小干巴,被郁保四提着衣领直接从墙内扔到墙外\dldots

我越想越气,我他娘的睡了一天一夜,也没去开会,山上兄弟除了鲍旭那小子,连个问问的都没有\dldots

不过转念一想,又高兴了,至少我知道谁是虚情假意,谁是真正关心我,还是鲍旭这小子实在,回头得提拔提拔\dldots
\section{}

山寨头领,婚丧嫁娶,迎来送往,必不可少,牛逼如武松,也屁颠屁颠跟在别人后面随份子,鲁智深这憨货,也乐呵呵的赶
去帮忙

朱武曾对送礼发表过一番感慨

礼尚往来,礼``上''往来,人一旦降生,谁也挣脱不开。

往卑微里说,前街的刘乞丐,在后街王乞丐结婚时,不也送了两块大馒头?

往高贵里说,跳出红尘的文殊院智真长老,在大相国寺智清长老七十大寿时,不也千里迢迢的送了贺仪?

你出生时,别人给你爹妈送礼,你生儿子时,别人给你送礼,你咽气时,别人给你儿子送礼?谁能免俗?

朝代更迭,礼仪不变,五千年文明史,就是五千年礼仪式\dldots 新媳妇哪条腿先下轿,人死了头朝哪个方向摆\dldots
都有讲究

婚丧嫁娶,讲究热热闹闹,路人皆知

我问朱武,为啥要弄这么大场面?

这厮苦思良久,说可能是因为

第一、一个人抬不动轿子

第二、谁也不想自己好兄弟拉着自己手说村里最近来了个骚娘们一起去调戏时却发现是自己娘子

\section{}

送礼绝对是门学问,赶早不赶晚,你想想,领导一天得迎来送往多少人?哪能个个记得住?但送早也有风险

上次,晁夫人怀胎八月,晁盖天天没事往扈三娘家跑,晁夫人气的无处发泄,跑到后山煽老黄出气,别人都是煽牛脸,她
倒好,专煽牛腚,被老黄一蹄子踢流了产

去帮忙回来的顾大嫂在半山腰跟孙二娘说,流了,还是个大胖小子,忒可怜

杨志这厮正从旁边经过,没听到前一句,听到了当中一句,一溜烟跑了,后一句听到没听到不清楚,但从后来事情看,应
该是没听到

这厮领了几个手下,抬着两筐鸡蛋,跑到晁盖家门口,敲锣打鼓,放起鞭炮,把晁盖气得脸都绿了

这厮名气有,武艺也有,出身也好,按说早就该升厅级干部,晁盖压着就不给他升,现在还在堂级的位置上原地踏步

活该!让你再吹嘘你祖宗!

活该!让你送礼老跑我前面!
\section{}

今天山寨乱糟糟的,一大早,几个一身血污的兄弟蹲在聚义厅前哭诉,说在山下老老实实的巡逻,济州府不讲道义,派人
偷袭,将他们打成重伤\dldots

下午,葛老爷子领着几个老头子,一路哭着上山,大骂济州府衙门黑暗,鱼肉百姓,欺压良民,请梁山英雄们替天行道,解
救水深火热中的济州百姓\dldots

鲁智深气得高声骂娘,武松牙齿咬得格格响,兄弟们无不义愤填膺,纷纷要求攻打济州府,一小兵当场咬破食指写了血
书,立马被宋大哥提拔为地级干部,一时血书满天飞\dldots

群情激奋,最后晁宋两位头领顺应民意,贴出告示,列数济州府二十大罪状,即日起兵攻打济州\dldots

王矮虎悄悄告诉我说,那几个兄弟不是被济州府打的,是去翠红楼嫖完不给钱被翠红楼的小梭罗打的\dldots

还有个问题一直想不明白,葛五叔不种地不做生意,平日里提个鸟笼下下棋,吹吹牛逼,逛逛妓院,竟然住着高宅大院,
花钱如流水,哪来的银子?
\section{}

晚上越想越气,这几个家伙不是忽悠众兄弟吗,我最恨这种鸟人,爬起来去找宋大哥,要拆穿他们。

路上碰到朱武,问我干啥去,我把前后讲了讲

朱武沉默许久,反问我,你以为宋大哥就不知道?

我有些糊涂了,既然知道为啥还相信?

朱武没回答我的问题,反而给我扯历史

他说,自夏朝建立以来,历经数千年,朝廷和老百姓的关系就是忽悠和被忽悠的关系。忽悠的好,天下太平,忽悠的不
好,改朝换代,换一拨人继续忽悠。

听朝廷忽悠,本本分分,埋头干活,风里来雨里去,好不容易攒了点银子,结果朝廷多征两种赋税,交上之后,发现,啥都
没剩下!白干!

听叛贼忽悠,拿起锄头反他娘的,脑袋掉了碗大的疤\dldots 到头来发现:我草!除了骑在脖子上的人换了一拨,别的啥
都没变!

我本来就笨,听了朱武一段鸟话,有些迷糊,难道老百姓都像鲁智深那憨货那么傻,任人忽悠?

朱武举了个例子,隋朝末年,隋炀帝移驾江都,李渊暗中勾结突厥,阴谋叛乱,怕副留守高君雅不从,倒打一耙,诬告其勾
结突厥当叛贼。

叛贼最可恨,太原百姓怒火中烧,蜂拥到留守府,要求处死高君雅。民意难违,忠臣被一刀砍作两半,众人欢呼万岁,殊
不知真正的叛贼已经将他们的妻女许诺给突厥当做起兵的条件\dldots

最后,朱武问,懂了没?

我点点头,又摇摇头!好像懂点,又好像不懂!

这厮摇着头走了,说啥朽木不可雕也!

娘的!说我笨就直说,尽整些文明词! 

\section{}

我琢磨了大半天,还是没搞明白啥意思!

头脑发昏,干脆不去想了,知道的越多越痛苦,别看这厮天天嘲笑我笨,其实我活的比他自在多了,该吃吃,该喝喝,该睡
睡,你看他,天天哭丧着脸,像谁都欠他银子不还似地,谁见谁烦!

路过顾大嫂家,顾大嫂又在求神拜佛,这婆娘忒逗,每次有事求着神仙时,就摆个八仙桌,弄个猪头,置些水果,插上三炷
香

上面挂一溜神仙:如来佛祖,观音菩萨,玉皇大帝,灶王爷,财神爷,土地爷\dldots 足足有十多个,也不怕贡品不够神仙
们打起来

顾大嫂每次都特虔诚,跪倒在地,双手合十,嘴里念念有词,祈求神仙们保佑。

每念叨一个神仙,磕三个响头,实打实的磕,梆梆响\dldots

一圈下来,额头都能肿成馒头。

若是灵验了,万事好说,鞭炮齐鸣,捧个猪头千恩万谢,逢人就夸老天有眼,神仙保佑。

若不灵验,那可翻了天,这婆娘会跳脚大骂:你瞎了眼啊?白给你猪头吃了?咋不灵验?

\section{}

又要打仗了。

在江州时,对行军打仗很是向往,做梦都想当将军,提十万虎狼师,纵横南北,青史留名。

后来跟着宋大哥上了梁山,做了头领,才发现跟想象中的完全不一样。

第一次领兵打仗是攻打祝家庄,宋大哥一手翻兵书,一手排兵布阵,煞有介事的将兄弟们分为三队,先锋队,中军队,弓
箭队

先锋队最危险,冲锋时在最前,逃跑时在最后,危险系数忒高,基本上都是一帮亡命徒

弓箭队最安全,进攻时,跟在中军后面放放箭,战事不利,抬脚就溜。

能进弓箭队的都是关系户,晁盖的小舅子,花荣的外甥,吴用的侄子\dldots

约好,先锋冲阵,中军压住阵脚,弓箭队在后面放箭

我任先锋,领着上百号人,浩浩荡荡的杀到阵前,兄弟们原本都是流氓,没打仗经验,也不懂啥阵法,你推我挤,乱成一锅
粥\dldots 队伍过处,地上一溜草鞋\dldots

打仗总要先礼后兵,双方将领到阵前,互通姓名,讲一下为啥来打你,顺便问候一下对方老母,接着开打\dldots

那次,我词都想好了,本想好好吹吹牛逼,但自始至终压根就没说话机会,因为根本停不住,人推人\dldots

只好冲锋,我领着兄弟们呐声喊,一窝蜂往前冲,结果,还没跟敌人照面,天上嗖嗖的往下射弓箭,全他妈的背后射的

弓箭队那帮王八蛋压根没几个人摸过弓箭,全他妈的一帮鸟人,连弓都拉不动,弓箭全掉自己人头上

宋大哥坐在高头大马上,戴着大高帽,举着令旗,装模作样的指挥,一支冷箭嗖的把他帽子射了个对穿,这厮趴在马上再
不敢起身

仗没法打了,败军如山崩,兄弟们四散逃跑,祝家庄的人纹丝未动,都在哈哈大笑

事后清点,百余个兄弟阵亡,一半是被射死的,一半是被踩死的

我中了三箭,全他妈背后中的。
\section{}

每次打仗前,兄弟们都很忙,先聚一起,大块吃肉,大碗喝酒,大呼小叫,人人似乎都很开心,其实,谁心里都明白,过了明
天,很多人的牌位,都将摆在忠烈祠中

宋大哥黑脸喝的通红,语含悲伤的对晁天王说,晁大哥,咱哥俩平常有点不对付,但今天老弟给你说句掏心窝子的话,万
一我不小心交代了,你一定要照顾我妻儿老小

晁天王一脸仗义,兄弟,你放心去死就行,你的老爹就是我的老爹,你平常怎么孝顺我就怎么孝顺,你老婆孩子就是我老
婆孩子,你平常怎么照顾,我就怎么照顾\dldots

时迁搂着朱武的肩膀,再三叮嘱,时是时迁的时,迁是时迁的迁,万一明天他战死了,牌位一定不能刻错了

王矮虎拉着乐和的手,反复哀求,给他写祷文时,千万不要写他好色,切记!切记!

顾大嫂拿出锭二十两的银子,``啪''的拍桌子上,让鲍旭管他叫娘,鲍旭二话不说,当即跪倒在地,脆生生喊了一声娘,
顾大嫂哈哈大笑,没等她笑完,鲍旭跑到门口,抱住看门狗的后腿,大声叫爹\dldots

吃饱喝足,有老婆的,回家腻歪,没老婆的,跑山下翠红楼腻歪

人人都在放纵,花钱如流水。

明天,谁也不知道自己是趴在兄弟尸体上翻兄弟口袋,还是直挺挺的躺在那里被自己兄弟翻口袋

偌大的聚义厅,刚刚还热闹非凡,转眼就空空荡荡,只剩我、鲁智深和武松

我和鲁智深玩骰子,老规矩,赢了的煽输了的一巴掌,这厮脸肿的老高,赌气不跟我玩,跑去跟武松玩

我心下暗笑,真他娘的笨,换个人煽你就高兴了?
\section{}

又喝多了。

酒,可以轻易改变一个人。

别看乐和这厮平日里斯斯文文,见了谁都满脸堆笑抱拳作揖,放屁都要抬屁股,三碗酒下肚,立马换副鸟脸,腿往凳子上
一翘,袖子一捋,开口就操你妈

林冲,平日里在晁宋两位头领面前,唯唯诺诺,像个裹脚小媳妇,蚂蚁放屁的事都要请示一下,从不敢乱说话,喝醉后,脸
红脖子粗,唾沫乱飞,张嘴闭嘴都是老子当教头的时候\dldots

扈三娘,平日里自诩大家闺秀,知书达理,兄弟们多看她两眼,就杏目圆睁,一脸不快,喝醉后,也一脸娇羞,眼神迷离,一
个劲的往武松身边凑,王矮虎拉都拉不住

我小醉时,喜欢跟鲁智深掷骰子,大醉时,喜欢考虑问题,烂醉时,喜欢骑墙上看星星

我应该是大醉了,因为我在思考,为啥今天这堵墙怎么爬不上去
\section{}

宋大哥起兵攻打济州府,出征前照例召集众兄弟祭旗,祷告天地,祈求旗开得胜。

祭旗毕,求神问卦。

这种事以前都是由小相国寺的智善长老操持,智善跟鲁智深同是智字辈高僧,关系不错,经常在一起喝酒讨论佛法。

智善为了揽下梁山所有祭祀业务,舍身忘佛,跟宋大哥拼过酒,陪晁天王骂过娘,请蒋敬去五台山旅过游

这次清风观的一尘道长找到公孙胜,请他看在同门的份上照顾一下生意,公孙胜很讲义气,当即跟蒋敬打了招呼

一尘是个二把刀,本来是个破落户,读过半年私塾,字都认不全,专职坑蒙拐骗,后来不知是混不下去还是良心发现,皈
依了道教。

蒋敬很为难,自己虽不归公孙胜管,但人家排名靠前,说的话不能全当放屁。不过拿人手短,佛祖的面子也不能不给。

最后,这厮想了个办法,一分为二,祭祀天地,由智善长老操持,解卦交给一尘长老,利物五五分成。

智善长老祭祀毕,道声阿弥陀佛,退在一边。

众人单膝跪地,聆听上天告示

一尘这厮背着手,装模作样,一步一步踱上道坛\dldots 不是走,也不是迈,而是踱\dldots 他娘的,要是在马路上碰到
他,照头就一鸟斧!

这厮闭着眼睛念念有词,念叨完,拿出卦盒请宋大哥请卦。

宋大哥双手擎着卦盒,上摇三圈,下摇三摇,左摇三圈,右摇三圈,每摇一圈都无比虔诚的念叨一番,最后往面前猛一抻

啪!

签请出来了!

两根!

众人面面相觑,宋大哥死死盯着两根竹签,汗都出来了。

我咬着嘴唇,强忍着没笑出来。

还是一尘道长修行高,高深莫测的表情丝毫未变,左脚偷偷一伸,把一根签勾到道袍底下!

脚法熟练,力度拿捏的恰到好处,估计平常没少练!

一尘俯身拾起另一根签,摇头晃脑看了半天,一脸惊讶!

众兄弟心都提到嗓子眼,大军出征,若抽个下下签,影响军心,忒不吉利!

这厮金口一开,语出惊人:不是上签,也不是下签,签文奇形怪状,不似人间文字,应是神签!

吴用糊涂了,卦文是他昨晚亲自刻好的,全是他妈的上上签,哪来的神签?

当即也不顾上天恼怒,一骨碌爬起来,走上神坛,探头一看,凑在一尘耳边说:道长,你他妈的拿倒了!

一尘脸一红,忙把签倒过来:上上签,大吉大利,旗开得胜!
\section{}

一尘道长一脸尴尬,当下也不装了,匆匆溜下道坛。结果一不小心踩歪了台阶,一个趔趄,摔倒在智真长老面前。

智真长老俯身将他扶起,说声阿弥陀佛,道长请起,贫僧受用不起!

一尘讪讪的爬起,站在一边,再不敢多言!

宋大哥起身,发表战前动员,号召兄弟们拼命杀敌,勇往直前,说什么只要打仗勇敢,就会得到提拔。

他这话也只能哄哄那些新上山的傻蛋,老兵油子谁也不信,谁信谁死得快!

一战下来,打仗最勇猛、冲在最前面的基本都战死了,而得到提拔的,大都是那些喊声最高、冲锋最慢、逃跑最快的软
蛋

每次打仗,头领极少阵亡,因为冲锋时他们大都站在一边,举着大旗,一脸悲壮的大喊:为了山寨的明天,兄弟们冲啊!

兄弟们在前面杀的昏天暗地,他在后面声竭力嘶:狭路相逢勇者胜,兄弟们一定要顶住!顶住!!顶住!!!

万一战况不利,抬脚就溜,一旦攻破城池,没啥危险了,他扛杆大旗跑城墙上,庄严地宣布:我们胜利了!

山寨上,敢带头冲锋的头领,唯我一人。

倒不是我高尚,而是我喜欢砍头如切菜的快感,喜欢鲜血横飞的场面,锋利的斧头将敌人一挥两半时,我会感到莫名的
满足。

杀人,也是有瘾的。宋大哥又拿时迁的例子鼓励大家,时迁从一名跑堂小兵升为堂级干部,仅用了半年时间,被普
通小兵奉为奋斗的榜样,床头都贴着他的画像。

时迁这厮,屁本事没有,人品低劣,总爱干点偷鸡摸狗的事,山寨兄弟都瞧不起他,是那种一起吃饭点菜时问他想吃点什
么没等他答话就已点好了的人。

这样的鸟人,按说在山寨永无出头之日,却被提拔为堂级干部,跟我平起平坐,忒他娘的气人!虽然一百个不情愿,但也
没办法,谁让人参加了金沙滩遭遇战那!

那次战斗打的异常惨烈,朝廷从禁卫军挑了上百名高手,星夜奔袭梁山泊,都过了金沙滩,还无人知觉。

眼看梁山大业危在旦夕,这时,恰好碰到巡山队,双方当即厮杀起来

那一战,尸横遍地,血流成河,巡山队绝大多数人当场战死,却成功挡住了敌人的偷袭,为山寨调配兵马赢得了时间。

白胜这厮,躺在兄弟尸体下装死,才捡回一条命,后来聚义厅论功行赏,别人都战死了,只好赏他! 

\section{}

现在的梁山,兵强马壮,已非当初的乌合之众。

宋大哥把兄弟们分成八队,马兵五队,步兵三队,林冲、关胜、花荣、秦明、杨志为马兵头领,鲁智深、武松和我为步
兵头领

约好令旗指挥,令旗共分七种,青龙旗、白虎旗、赤霞旗、土黄旗、黑水旗、紫云旗、凤绿旗,一队对应一种令旗。

宋大哥记性不太好,老弄混,就事先记在纸上藏袖子里。

我不懂那些乱七八糟的旗语,我跟宋大哥约好,要我冲,就举剑朝前指,要我撤,拿剑朝后指,原地不动,就拿箭朝地下
指,简单明了

交战前,众头领聚在中军大帐讨论计策,军师献计,把队伍按斗、牛、女、虚、危、室、壁七个方位摆成北斗七星阵,
引诱官军来打。

只要是吴军师献计,宋大哥总是两个字``好计!''

攻打高唐州那次,军师不知得了啥鸟病,骑不得马,坐不得轿,天天在家趴着,未能随军下山。

宋大哥临行前匆匆赶去请教,军师说,欲破此城,必用火攻\dldots

宋大哥喜出望外,道声好计,一溜烟跑了\dldots

当夜,宋大哥不分青红皂白,在上风口放了一把火,结果,把自己大营给烧了\dldots

宋大哥眉毛胡子烧掉一大把,跑回去问军师咋办,军师徐徐说,上次还没说完:欲用火攻,必借东风\dldots

宋大哥道声好计,又一溜烟跑了\dldots

众兄弟抱着柴火,喝了三天西北风\dldots 脸都吹绿了\dldots

宋大哥撑不住了,灰溜溜跑回山寨,军师叹口气说,上次又没说完,若无东风,转到城西,继续火攻\dldots

打那之后,军师每献一计,宋大哥总要先问,说完了?等军师点头后,再连声赞叹,好计!
\section{}

第二天,按计划行事。

人饱食,马喂足,济州城下,摆好阵势。

宋大哥高头大马,锦衣玉袄,意气风发;吴军师书生打扮,青衣紫袍,信心十足

宋大哥为了鼓舞士气,夺人之威,挥着令旗演练阵法,三军将士步调齐整,疾而不乱,奔腾往来,一会排成S,一会排成B。

满城百姓提老携幼,纷纷涌到城墙上看热闹,指指点点,前仰后合,仿佛我们不是来攻城的,而是来耍杂戏的\dldots

演练完阵法,上万号人在济州城下傻傻的站着

计是好计,阵是好阵,但搞错了一件事情,是我们攻打济州,不是济州攻打我们,而北斗七星阵是个守阵

太阳忒毒,一站就是他娘的半天,腿都麻了,官军压根不出来

日头偏西时,城门吱呀一声开了,众人立马来了精神,准备厮杀,但没有预料中的大队人马,只有一个小兵,战战兢兢的
走到中军大帐

宋大哥以为自己阵法高深,把官军给震住了,对方是来求和的,得意洋洋的问,为何不敢出来交战?

小兵怯生生的说,知府说今天太热,改天再战!!!

宋大哥脸一下子胀的紫红,这不是拿我们万把人当猴耍吗,气呼呼的问,你们知府还说啥了?

小兵冷汗都出来了,结结巴巴的说,我们知府还问你们喝不喝水?二文钱一碗?

宋大哥大手一挥,推出去咔嚓一刀砍了!
\section{}

当夜无话,第二天,官军还是死活不出城。

对付缩头乌龟,只能骂阵!

骂阵有讲究,首先,站的距离要适中,离城太近,一箭把你射个透明窟窿,忒危险!骂阵把自己骂死的兄弟不在少数。若
离城太远,敌人又听不见,任你骂的嗓子冒烟,无济于事。

骂的内容还要有新意,不能老骂三字经,没啥效果,得骂出花样来,骂到对方心坎里去,让他恼火,不得不出战。

这活一般都是我和鲁智深来干。我嘴笨,把对方十八辈祖宗问候一遍,就没词了,后来王矮虎给出主意,把对方家养的
狗啊猪啊再问候一边\dldots 不过没啥效果,我骂的越恨对方越高兴,有时还冷不丁扔下两头猪来\dldots

鲁智深翻来覆去就那一句``兀那挫鸟,还不出战?''骂四五个时辰,不带改一个字的,能把人给骂睡了。有次,对方还没
咋地,先把宋大哥骂火了,在后面大喊:大师,你他娘的能不能换一句?

跟别的强盗骂阵,勉强骂个平嘴,但跟朝廷骂阵,水平远远不如。

这帮孙子忒沉得住气,端壶茶往城墙边一站,任你骂翻天,一直保持笑眯眯的鸟样,不急不躁,等你骂累了,慢条斯理的
夸你两句。你一听,貌似好话,转念一想,不大对劲,再一琢磨,我草,这是在骂我啊!!!

后来,牛人顾大嫂上山了,顾大嫂的嘴上功夫,上数五百年下数五百年,无出其右者。

上次,扈三娘惹恼了她,堵在门口从清晨骂到黄昏,没带一句重复的。

打她上山,梁山骂阵从未输过,十二次出场,十一次成功骂出敌人,剩下那一次,把对方将领骂的当场吐血而亡,不战而
降。

顾大嫂骂出了名气,一时江湖闻名,很多山寨打仗前不惜花重金请她去骂阵,后来朝廷想招安她,宋大哥看她是个难得
的人才,没放她走。
\section{}

兄弟们摆好阵势,顾大嫂喝口茶水,润润嗓子,咳嗽两声,大摇大摆走到阵前,叉腰而立。

只见顾大嫂气沉丹心,众兄弟纷纷扔下兵器,堵住耳朵,城头的官军正莫名其妙

这时,顾大嫂突发一声长啸,犹如平地起声炸雷\dldots

城头上士兵突闻此声,手中兵器惊落无数,远近万籁俱无,阒然无声,众人瞠目结舌\dldots

这是顾大嫂的拿手好戏,当头一棒子,从气势上压倒对方

接下来顾大嫂舌绽春雷,中气十足,骂声犹如滔滔江水连绵不绝\dldots

那帮孙子还没从当头一棒子中回过神来,就被骂的狗血淋头,七荤八素\dldots

顾大嫂骂的内容也很磕碜,什么难听骂什么,什么恶心骂什么\dldots

骂了足足一炷香时间,那帮孙子一句话都插不上,端茶杯的手直哆嗦,张叔夜气的当场吐了血\dldots

济州城立马颁布招贤令,能骂过顾大嫂者,赏银五百!

重赏之下,必有悍妇。

济州城数得着的泼妇纷纷涌上城头,乌压压一片,以前骂街是为了出气,现在骂街是为朝廷争光,骂的好还有银子拿
\dldots

顾大嫂不慌不忙,舌战群妇,骂完左边骂右边,骂完右边骂左边,间隙里不忘骂张叔夜这老而不死的王八蛋\dldots 丝毫
不落下风

众泼妇纷纷败下阵来,自愧不如\dldots

张叔夜连吐数口血,再也撑不住了,再吐就把自己给交代了,性命是小,名节为大。

死在杀场上,朝廷会行文表彰,史书上也会留下浓重一笔:叔夜,年近古稀,尽忠王事,提兵杀贼,来回冲突数十遭,身披
十余创,终因力战不支,战死沙场,临死犹大嚎:杀贼!

若是被骂死,史官会这么埋汰他:叔夜,龟缩城中,不敢战,被悍妇辱骂,气绝身亡!

你战死了,别人问起时,你家人会挺起胸脯自豪的说,上阵杀敌,死在沙场上

听者会肃然起敬:英雄!国之栋梁!

若你被骂死了,家人门都不好意思出,总不能大言不惭的到处炫耀:我爹是被人骂死的!

\section{}

城门打开,一队人马气势汹汹的杀出城来

宋大哥拿剑朝前一指,我领着三百兄弟迎上去

别的将领上阵,里三层外三层,又是黄金锁子甲,又是天山金蝉丝,包裹的严严实实。

我上阵有个习惯,不穿衣服,赤条条的。

倒不是我喜欢装另类,而是我发现,对敌时不穿衣服比穿衣服效果要好的多。

以前我穿衣服,往往要大战几十个回合,才能把敌将砍倒,身上还会留下不少刀伤。不穿衣服时,取敌人首级如探囊取
物。

并不是我扒了衣服武艺就高,而是我赤身时敌将武艺打了折扣。

厮杀讲究凝神静气,全神贯注,而跟我对敌的将领,眼神游移不定,往往先往我下面一瞟,``哇!''

等他目光上移时,板斧已到面前,只来得及``啊!''

极少有人能在我手下走上三招,女将更是如此!

鲍旭这厮,有段时间也学我裸体,效果更佳,敌将极少能在他手下走一招,因为敌将往往要朝他下面瞟两眼

第一眼,``咦???男女?''

第二眼:``噢!找到了!''

等抬头时,脑袋已经搬家。

赤身上阵两次后,这厮再不敢扒衣服,天天被兄弟们嘲笑,媳妇都娶不到!

你看我,天天有人追着给我说亲!

没那二两本钱,还学我赤身,弄的现在还是光棍一条!
\section{}

一队兵将直冲左阵,宋大哥指挥若定,掏出书帛看了一眼,把青龙旗一举,林冲挺着丈八长矛,跃马出阵,截住厮杀

林冲这厮,典型的墙头草,跟晁天王在一起,吹嘘晁天王盖世无双,跟宋大哥在一起,恭维宋大哥举世无对,不怎么招人
待见,不过行军打仗很有一套,功夫也没的说,号称梁山泊头号猛将,虽然谁也不喜欢他,但谁也离不开他,都指着他打
胜仗那!

一队兵将直奔右阵,宋大哥把白虎旗一挥,关胜大喝一声,瞪着三角眼,倒拖青龙刀,抵住来将

几对兵马正捉对厮杀。这时,一彪军马直奔宋大哥而去,势如闪电

宋大哥心一慌,手有些抖,正好阵前刮起一阵风,一个没拿稳,书帛飞了

这可咋办?

来将越来越近,宋大哥也不分青红皂白,抓起赤霞旗就挥

鲁智深远远看见赤霞旗就有些晕,他在左后侧掠阵,要去中军就得打乱阵形,当下也顾不得多想,要从杨志军中穿过,杨
志不让,说乱了阵形是要砍脑袋的,两人当即闹起来,连在二龙山时陈芝麻烂谷子的事都翻了出来。

宋大哥眼看无望,又抓起起黑水旗乱挥

花荣正在厮杀,当下也顾不得敌人,调转马头往回赶,但远水救不得近火,那支军马已到跟前

吴军师大喊一声快跑,自己抬脚溜了,宋大哥令旗也不要了,拍马就逃,逃跑面前,人人平等,到处是人,根本没有路,敌
将已到身后

眼看宋大哥的老命就要交代了,关键时刻,只见人丛中一小兵拍马舞刀杀出来,小兵身着宝蓝直裰,绯袍锦带,勇猛异
常,所过之处,人仰马翻

小兵杀到宋大哥身边,替宋大哥挡了三刀,之后一人力敌五将,不落下风,血满征袍,兀自死战

众兄弟匆匆赶到,杀退敌兵,救了宋大哥一命,小兵晕倒在地

宋大哥下马抱着小兵大哭:赵兄弟,我的好兄弟,你可不能死啊?、、你醒醒啊!!!

宋大哥眼泪鼻涕一大把,流了小兵一脸\dldots

可能宋大哥的真情感动了上天,小兵悠悠醒来,张了张嘴,挤出几个字:``大哥,俺姓刘!''
\section{}

小兵叫刘彦,当下官升三级,被宋大哥提拔为厅级干部。

刘彦回山后刚躺下,蒋敬屁颠屁颠跑去了,老脸堆成朵花,先一脸关切的问伤的怎么样,要不要紧?接着叹口气说以前不
知道刘头领住的条件如此差,是他的失职,刚刚给安置了座独门大院,三进三出的,请刘头领过去安歇,还说明年准备在
金沙滩开个酒店,到时请刘娘子前去掌管\dldots

扈三娘风风火火的赶到,蒋敬恰好出院门,扈三娘问他新升的头领叫啥,蒋敬说叫刘彦。

扈三娘还未进门,先咋呼开了,带着哭腔,一口一个刘彦兄弟,进门后,夸张的长舒一口气,拿袖子擦擦额头,双手一拍大
腿,说什么还以为刘彦受了重伤,她急得团团转,这下看到刘兄弟安好无恙,就放心了\dldots

公孙胜也赶去凑热闹,拉住刘彦的手,絮絮叨叨的说,上次你找我算命,我说你是一身狗贱骨,话没说完你就走了,其实
后面还有一句,你还有一口宰相牙,是封侯拜相的贵人之命\dldots 临走还说,你的弟弟刘璋我已经从先锋队给调到弓
箭队了\dldots 以后有什么事尽管开口就行\dldots

\section{}

刘彦上山时带着老娘,山寨上,地级以上头领有自己房子,老小自然不用愁。小兵住大通铺,老小没地住,山寨给盖了一
排泥胚房,统一安放。

刘大娘前几年摔了一跤,半身瘫痪,一直卧病在床,屎尿都拉在被窝里,别家老小嫌气味难闻,不愿意跟她同住,刘彦只
好在半山腰给搭了个茅草屋,平常冷冷清清,没人来往。

这下热闹了,顾大嫂跑去,二话不说,扯起床单放盆里,挽起袖子,洗的那叫一个带劲,也不嫌臭也不喊脏,弄得浑身湿漉
漉的\dldots

孙二娘把被褥子拆了,换了新瓤子,一针一线,缝的有板有眼\dldots

安道全拖着老胳膊老腿,柱个拐棍,背个药匣子,也去凑热闹,老姐长老姐短的叫着\dldots 草他妈的,你大牙都快掉没
了,比人刘大娘要大一旬,还叫人老姐!

这厮给刘大娘从头到脚做了个全身检查,连妇科病都给查了\dldots

查完就跑刘廷面前,一脸沉痛的说,老姐这个病,本无大碍,但因为拖延久了,所以治起来比较麻烦,最后表示,一定会尽
力而为。

讲完病情拉关系,这老不死的说他跟刘老爹原本是至亲兄弟,经常在一起喝酒扯淡,后来刘老爹不幸西去,他怕刘大娘
一个人寂寞,不分白天黑夜的往刘家跑,陪刘大娘聊天解闷\dldots

草他妈的,你上山前人家刘老爹就已经死了,你跟鬼扯淡?

皇甫端站在院子里干着急,他只会医马,不会医人,刘大娘的病他插不上手,这可咋办?

办法要想总是有的,这厮趁人不注意,在草料中掺了些巴豆,然后像模像样的看看嚼口,听听肚子,最后,站到马屁股后
面,拿手东戳戳西摸摸\dldots 估计马被弄烦了,肚子咕噜一响,一个响屁喷出来,这厮躲避不及,兜头喷了满脸\dldots
我的娘哎,一脸绿水\dldots

刘大娘又要解手,时迁眼疾手快,把尿盆抢在怀中,众人七手八脚把刘大娘扶起来,时迁端着尿盆伸过去\dldots 眉头都
不皱一下,而且摆出副享受的鸟样,众人没被赃物恶心到,倒被时迁的表情恶心的直想吐

\section{}

聚义厅聚会,宋大哥喝醉了,作诗一首:
{\footnotesize \it
\begin{verse}
高俅\\

高俅的贪,是纯朝廷官吏的贪\\
我一直想给他送礼,至今未能如愿\\
其实小时候我和他住的特近\\
一江之遥\\
他住江头,我住江尾\\
后来他遇见了皇帝\\
拍了一个又一个技惊四座的马屁\\
青云直步、让人始料不及\\
我喜欢他拍的马屁\\
大道无形不露痕迹\\
我有时候搞不明白\\
我们都是百无一用只会拍马屁\\
为何他当了大官\\
我却窝在这水窝里整天唉声叹气\\
我想,前途其实是可以用银子塑造的  \\
最近去东京看送礼三十六计\\
兄弟们问我,你送了吗\\
我说:本不想送,就是两只手不守纪\\
银子太少四处碰壁\\
心,沉到锅底\\
吴用说,你可以找高俅,让他给你拿个主意\\
我说:你扯淡吧,人家可是大官\\
吴用说:大官怎么了\\
大官更应该知道朝廷的规矩:不收礼,怎么当官吏 \\
我觉得有理,真去找高俅\\
高俅端一杯龙井坐在那里\\
盯着我手里两只王八,眼神婆娑迷离\\
我跪迎过去,近了 \\
他突然那么雷霆一怒,怒的惊天动地\\
毫无准备的我\\
夹紧屁股,浑身战栗\\
高俅说,办这么大的事,你送这点礼,你当我是SB\\
哎!招安,真他妈不容易\\
\end{verse}
}

\section{}

兄弟们上山的原因五花八门,大体可分为四种:

一是犯了案子遭通缉无处容身,如武松

二是打了败仗有家难回,如秦明

三是被自己亲戚朋友骗上山的,如徐宁

四是烂泥扶不上墙以当强盗为乐,如王矮虎

刘彦特殊,不属于任何一种,他出身书香门第,文武双全,二十岁中举,做过两年县令,官声也不错,深得蔡京赏识,可谓
前程远大,可惜这厮不学好,放着好端端的乌纱帽不戴,非得跑这兔子不拉屎的地当强盗!

问他为啥?这厮竟然说什么要替天行道!

草!宋大哥为啥要替天行道?还不是想有朝一日招安后戴顶乌纱帽?

哎!十几年的圣贤书都读狗肚子去了,我们当强盗的自己都不信,你他娘的倒信了!

我没读过书,本来很后悔,喝醉了就骂我那死鬼老爹,当年为了省那二两油钱不让我进私塾。自从认识刘彦,倒也看开
了,不读书,被读书人骗,读书,被书本骗。被别人骗,糊涂一时,被书本骗,则能糊涂一世,想想,还是不读书为好!

\section{}

这个社会,干啥都得讲关系,做官如此,做强盗亦是如此\dldots

你若想做官,就得有个好爹,如蔡九,这厮笨的出奇,他老爹为他请遍天下名师,都没撬开他那榆木疙瘩,众老师直感叹,让
他开窍比当年洞房都累,对他的评价跟智真长老对鲁智深的评价出奇的一致:灵光一犀,价值千金,意思是,一旦开窍,
光芒如一千斤金子发出的光芒一样耀眼,但我认为,这句话还有另外一层意思,那就是,拿一千斤金子砸死他,都砸不出
个屁来。一本三字经,足足学了两年都没学全,平常吃喝嫖赌,十足的二世祖,但人生的好,照样知府当着、小钱收着、
小妾养着\dldots

若没个好爹,有个好哥也可,像高廉,这厮琴棋书画不会,升堂断案嫌累,原本就是一街头跳大神的,专职糊弄妖魔鬼怪,后
来他哥当上了太尉,他摇身一变,八卦衣一脱,穿上官袍,转身糊弄百姓。升堂就那一招,一拍惊堂木,大喝一声大刑伺
候,招了的,画押打死,没招的,打死后画押,破案率百分之百,道君皇帝甚是欣慰,大笔一挥,赠匾一枚:国之栋梁!

你若生的不好,既没摊上个好爹,又没个有本事的好哥,那别灰心,还是有机会的,打听打听哪府小姐还未出嫁,花几两
银子雇几个流氓,趁她外出时半路打劫,千钧一发时,你再从天而降,英雄救美,说不定人家小姐对你一见倾心,你再趁
机生米煮成熟饭\dldots 也就攀上个好岳丈,前途自然无量

啥?以上都没有?那你他妈的还想当官?你丫脑袋秀逗了吧? 做强盗也得讲关系,白胜是晁天王的关系,武松是宋大
哥的关系,杨雄是戴宗的关系\dldots 有了关系,自然高看一眼,诸事好办\dldots

若没关系也想当强盗,也可,不过得纳投名状,砍人头表忠心!

刘彦跟宋大哥不是老乡,跟吴军师也不是同窗,跟山上的兄弟八竿子划拉不着,再说历史上有些问题,不清不楚,属于限
制使用那类人,必须得纳投名状

纳就纳吧,砍个人头还不是手到擒来的功劳,当强盗的,杀个把人那还不跟喝凉水似地

万万没想到,这厮在山下吹了三天西北风,头发都吹成鸟窝了,竟然一个人头都没砍到,倒不是武艺不济,而是这厮缺心
眼,都不希的说他,瞧瞧他办的那些鸟事

第一天,运气不错,碰到个过路的老头,要我说,甭废话,咔嚓一刀,割下头来交差,一了百了。这厮倒好,搀着老头送出
二十多里地,直送到济州府门口才罢休

第二天,碰到个去东京看病的瘸子,这厮又没下手,不但没下手,还生怕人跑的不快,把马送给人家,另外还倒贴了十多
两银子

第三天,碰到个青年,刚要下手,经人一番哭诉,那套说词都老掉牙了,什么上有八十老母下有三岁幼子\dldots 这厮心
一软,又给放了,他娘的,就不会动动脑子,那青年顶多二十来岁,她妈六十岁生的他啊?

当强盗当到他这个份,拔根鸟毛吊死算了。

你说他,他还来劲,说什么只杀贪官不杀良民,不能滥杀无辜之类的鸟话\dldots 草!

第三天傍晚,这厮两手空空回到山上,宋大哥拉着他的手说,刘县令啊,感谢你弃明投暗,投奔梁山,但我觉得,你不大适
合当强盗,你还是回头坐你的县令吧\dldots

没想到这厮自甘堕落,当强盗的决心很大,当天夜里,一个人跑寿张县把县令给砍了,这才勉强入伙。

\section{}

刘彦刚开始挺受兄弟们欢迎,打仗时冲在前面,论功时站在后面,也很讲义气,找他借钱从不扭扭捏捏,甩手就一大锭银
子,从不用秤称,还他时也不验成色,但就一毛病,总爱给人掰扯大道理。

兄弟们没事就凑一圈扯淡,吹吹牛逼,聊聊女人,骂骂娘,这厮一去,张口就什么伐无道、济苍生,要建立什么劳什子清
平天下\dldots 把人说的一愣一愣的\dldots

时间久了,没人愿意搭理他,本来牛逼吹的热火朝天,唾沫星子满天飞,他一去,立马都拍拍屁股走人

这厮典型的书呆子,读书读傻了,几千年的破事都看不透,什么尧舜禹汤,什么王侯将相,什么替天行道,全他妈扯淡,目
的只有一个:争位子,捞银子,抢女子。

在朝廷当差,那总得说点冠冕堂皇的话,什么苟利国家生死以啦,什么岂因祸福避趋之啦,说到动情处,拍两下桌子,掉
两滴眼泪\dldots 这个可以理解,当官吗,不就是上糊弄朝廷,下糊弄百姓,顺便糊弄糊弄自己\dldots 你他妈都当强盗
了,还装什么大头蒜?

这厮恶心兄弟们还不够,还跑去恶心宋大哥,宋大哥虽然不耐烦,但人涵养好,假惺惺的夸他两句,然后叹口气说自己不
在其职不谋其政,现在是晁天王当家,你还是去给晁天王讲吧。

晁天王是个二杆子脾气,本来就没啥文化,最烦别人在他面前掰扯大道理,没听两句就不耐烦摆摆手,说好好好,我知道
了,领导说这句话的意思就是,你他妈有完没完,赶快去死!这厮还看不出眉眼高低,唠叨个不停,把晁天王气的直骂娘

这厮在山寨混了大半年,别说连个头领都没当上,倒把山寨大小头领得罪了一个遍。

攻打青州府那次,王矮虎要强暴民女,这是他习惯,每次攻破城池都要糟蹋几个良家妇女泄泄火,宋大哥也睁一只眼闭
一只眼,不怎么管他,裙子都给扒下来了,这厮硬生生把王矮虎从炕上揪下来,王矮虎差点跟他拼了老命

打破高唐州那次,刘唐带着小啰啰趁机去百姓家打劫,这个都理解,当强盗吗,不就是这点鸟事?不然拼死拼活为了啥?
结果银子都装口袋里了,这厮硬生生给掏出来还回去,气得刘唐拿着朴刀要跟他火拼\dldots

要不是这次救了宋大哥,估计他也就老死山寨了,死了都不一定有人埋!

\section{}

今天摆酒庆功,众人喝的七倒八歪。

宋大哥最近有一习惯,啥事都分个三六九等,比如逛窑子,谁去都是那点鸟事,上床、付钱、拍屁股走人,他逛了两次倒
逛出景来了,非得分什么风流下流,说什么重情是风流,重欲为下流,不懂那些鸟语,反正他去就是风流,别人去就是下
流。

还有,大家都是强盗,都是兄弟,他非得分什么大盗、中盗、小盗,说什么大盗盗国、中盗盗名、小盗盗财,他自然是大
盗,我们这帮头领承他看得起,算中盗,那些到处打家劫舍的兄弟只能算小盗\dldots

众兄弟纷纷赞叹宋大哥高见,我心想,要是没有那些风里来雨里去杀人打劫的小盗,你他妈还大盗?你连咸菜都啃不上!

刘彦做了厅级干部,众兄弟虽然不喜欢他,但也堆起笑脸纷纷前去奉承。

时迁端着酒碗,抱着刘彦的腰,他那五短身材也只能勾到人腰,仰着头一脸真诚的说,刘兄,你两个月前给我说的话,我
日思夜想,终于想明白了,太他妈有道理了

王矮虎也凑上去,说什么他早就看出刘彦绝非池中之物,迟早坐上头领之类\dldots

刘彦今天情绪有些不对,以前从不敬人酒,别人敬他,也只是在嘴唇上碰碰,意思一下,今天来者不拒,干了一碗又一碗,连
喝了十多碗,眼睛通红

刘彦拉着我的手,像是跟我说话,又像是自言自语,黑哥,今天我当上了头领,大家都来恭维我,来敬我酒,可是那些战死
的普通士兵哪?他们也是为山寨打仗而死,也是妈生爹养的,也曾是有血有肉活生生的人,现在却连名字都没有\dldots
尸体暴尸荒野,任豺狼撕咬\dldots 山寨有了粮草,头领们得到奖赏,都心满意足,可是他们却躺在战场上,冷冰冰的,他
们的命就那么贱吗?他们为何而来?是什么让他们成了异乡冤魂?是是梁山?是朝廷?还是门外那替天行道的大旗? 

\section{}

刘彦喝的酩酊大醉,逮谁都问人理想

晁天王笑而不语,其实不说大家也知道,他天天在家摆张香台贡个猪头,一日三拜,祝愿宋大哥早升仙界。

宋大哥一脸深沉,说他的理想是天下太平,他的话鬼都不信,十句里九句半虚的!他真要这理想,也不难,找块石头一头
碰死,立马实现。

吴军师摇着蒲扇说,他的理想是进翰林院,每年科举由他主持,想让谁中就让谁中,想让谁中几等就让谁中几等,不顺眼
的直接给两巴掌

林冲恶狠狠地说,他的理想是当太尉,寻个过失,把高俅刺配边疆,然后把他娘子发配官窑,天天被人上

王矮虎一脸猥琐,说他的理想是,捞个一官半职,看谁家娘子漂亮,想上就上

我以前就盼着戴宗快退休,然后我当院长,看谁不服就给他二文钱让他去买大鲤鱼,回头再问他找零,买不回来就给他
穿小鞋\dldots 现在没啥理想,有酒喝,有银子赌,还有王矮虎这厮别爬我头上就行\dldots

刘彦听的目瞪口呆,直叹气,说什么后悔了,还说都他妈一个鸟样,也不知道这厮骂的是谁
\section{}

今天聚义厅开会,晁天王和宋大哥吵得不可开交,宋大哥想把聚义厅改成忠义堂,晁天王死活不同意,聚义厅这名字可
是他想破脑袋才想出来的,人还没死哪,就想换天,没门。

宋大哥的意思,我们只反贪官不反皇帝,对朝廷,还是要讲忠心的,改成忠义堂,很有必要。

晁天王的意思,我们都当了婊子,就不要再立牌坊了,天天在忠义堂里面讨造反的事,不大合适,自己都觉得脸红。

林冲这墙头草,说话忒不爽快,唾沫横飞,足足说了一个时辰,从朝廷法度扯到个人卫生,从四书五经扯到不准随地大小
便,天南海北,胡侃一通,听得众兄弟昏昏欲睡,也没听出到底是个啥意思!朱武用两个字对他的发言做了高度概括:废
话!

戴宗今天特滑稽,脸肿的老高,嘴上缠着绷带,封的严严实实,别说表态,连口都开不了,众兄弟问他咋了,他眼泪汪汪,
头摇的跟拨浪鼓似地,我心中暗笑,这厮昨晚请我喝了顿酒,非得让我煽他两巴掌,当时我还不知道为啥,原来是在这等
着那\dldots

支持晁天王的,大部分都是山寨老兄弟,这帮鸟人没啥文化,也说不出个一二三四五来,总是怒气冲冲的拍桌子。

支持宋大哥的人就海了去了,花荣和黄信一唱一和,配合的天衣无缝,本来以前都是和秦明配合,这两天秦明和花二妹
又吵架了,秦明赌气不搭他话茬,只好黄信顶上。

乐和引经据典,滔滔不绝,历数忠义堂好处一百零八条,说的是有理有据\dldots 把晁天王气的直翻白眼

大会吵,小会骂,也没议论出个结果来,最后两人大笔一挥:望军师酌办!

现在山上最难坐的就是吴用那把椅子,他现在是风箱里的老鼠,两头受气,好几回都甩手不干,要请假下山,晁宋两位头
领大力挽留,说什么非他不可。

吴用愁得茶饭不思,坐卧不宁,胡须都捻断好几根,最后想出妙计,门楣上挂两块牌子,十字相叠,横读:聚义厅,竖读:忠
义堂

众兄弟无不佩服的五体投地!军师果然不是一般人能当得,牛逼啊!
\section{}

今天去赌场玩骰子,通常,兄弟们要赌到月上梢头才罢休,有时兴起,还要挑灯夜战,今天邪了门了,太阳尚未落山,就早
早散了,就连嗜赌如命的王矮虎,也借口上茅房,尿遁了

我没啥事,去后山看了看老黄,这厮脸还肿着,忍着没煽它。

下山时遇到王矮虎,拎着个篮子,上面盖块红布,贼头贼脑。问他干啥去,这厮结结巴巴的说扈三娘养了条哈巴狗,他去
给割点草,我当时也没多想,好心告诉他山南边草特密,我经常去给老黄割,不过回头越想越不对劲,他娘的,拿我当猴
耍,狗不吃草!!

又陆续碰到几个兄弟,都拎着东西,鬼鬼祟祟,问他们干啥去也不说,我心中咯噔一下,坏了,一溜烟跑去翻老黄历,农历
初八,旁画一大猪,肚子下竖一棍子,我草!今天是宋老爹生日。

当头领不容易,大小节日你得记着,到时少不了表示表示。

领导生日,他爹妈生日,他老婆孩子的生日,你也得记着,到时也得意思意思

他祖宗的忌日你也得记着,当天一定不能大声说笑,更不能没事放鞭炮\dldots

节日多了,就容易混,传统节日还好说,到时自然能想起来,满大街都是卖月饼的,八成是中秋节要到了,要是满街都是
汤圆,那肯定是元宵节,要是兄弟们有事没事去你家闲扯淡,扯到半夜还不肯走,那保准是春节快到了,你丫该还钱了!

但别的节日,像宋大哥的生辰、晁老爹忌日,花二郎百日等乱七八糟的就记不准,又不好问别人,问了也白问,现在拍马
屁这行竞争这么激烈,好不容易打听到,都掖在心底当秘密,巴不得全山寨人都忘了就他自己记着,哪会告诉你?

我记性不好,又不会写字,好几次都搞错了,上次花荣他妈过八十大寿,我当他自己过生日,提了根虎鞭去,弄的老太太
脸耷拉的跟驴似的

为了马屁拍的准,我下狠心,拜朱武为师学识字,这厮开口给我讲什么四书五经,娘的,我又不打算科举,学那些有个鸟
用

我跟他说,只学头领们名字就成,没成想,写字跟砍人不同,我拿起刀就会砍人,想砍哪就砍哪,说砍你眉毛,绝砍不到你
眼睛,说砍你牙齿,绝碰不到你嘴唇。

写字就奇了怪了,握着笔,怎么都不听使唤,歪歪拉拉,光自己名字就学了三月,还总写错,他娘的,我那大字不识一个的
死鬼老爹怎么给我起了这么个鸟名字

这厮天天骂我笨,一气之下不学了。

笨人自然有笨办法,我想出了个主意,画杠子,一道杠,代表晁天王,两道杠,宋大哥,三道杠,吴军师\dldots 也不难,我
头上的领导总共就十来个。

可是他爹妈哪?

我冥思苦想了三天三夜,灵光一现,想出个绝妙主意,跑去拜金大坚为师学画画,只学画猪,学了大半个月,倒也有模有
样

够用了,我在老黄历上重要节日旁都做了记号:竖杠,代表头领,他爹妈,用大猪表示,肚子底下加根棍子的是爹,没有的
就是妈;生日忌日也好分,头上插把刀的,就是忌日,没刀的,就是生日;孩子用小猪表示。

老婆和小妾也好分,肚子下一个圈的是正房娘子,两个圈的是二房,三个圈的是三房,他娘的,晁盖那头猪下面最多的画
了二十几个圈,整个黄历都占满了\dldots

\section{}

人活着,真他妈的累!

人也是贱,没事过啥节嘛!平常喝喝酒,赌赌博,稀里糊涂活到死多好!

我本以为,过节这馊主意肯定当官的想出的,不过朱武这厮说,应是老百姓。

我就纳了闷了,老百姓过节又没人给送礼,没事折腾自己干啥。

这厮发表了一番长篇大论,说什么历朝历代的老百姓,不是在过日子,而是在熬日子,平时节衣缩食,饥一顿饱一顿,见
天干活、交租、还债,天天累的喘不过气来,不停的奔波,从没有一天消停,只有在过节的时候,才能停下来歇歇,女的
扯身新衣裳,出去显摆显摆,互相夸两句,乐呵乐呵,男的割两斤猪肉,弄两壶小酒,聚在一起骂骂娘\dldots 只有那一
刻,他们才是真正为自己而活

有了节,老百姓活着才有奔头,今年盼明年,明年盼后年,一年盼一年,直盼到临死那一刻,看着周围的世界,感慨一句,
生活,真他妈扯淡!
\section{}

今年的重阳节,恰逢山寨开创五周年,晁天王建议开庙会,普天同庆,宋大哥欣然同意。

这事由吴用操持,他是梁山大管家,大事小事插一杠子,母猪配种的屁事都管管。

这厮自诩诸葛亮再世,但临阵破敌那几招真没啥稀罕,官军来攻,他龟缩山中,让几个兄弟驾船在江中骂人祖宗,诱人下
水,再趁人不注意从水底捅两刀\dldots 官军来了几次,干着急,刀砍不着,弓箭又射不到,骂又骂不过,气得半死

要是攻城拔寨,就派时迁溜进城里放火,里外夹攻,实在攻不破,就用绝招,忽悠宋大哥装病,他再跳出来说什么哥哥身
体欠安,暂时班师回山来春再战之类的鸟话\dldots

这厮虽然行军打仗不行,但办事靠谱,能让各方满意,这次庙会,宴席交给宋氏酒楼操办,伙什购置由晁盖妹妹夫负责,
银两发放由公孙胜侄子经手\dldots 反正谁也不得罪,谁都说他会来事

\section{}

九月初九,重阳节

今年庆典,要检阅三军将士,为方便检阅,建点将台一座,高八尺,方圆十余丈,上面香花灯烛,幢幡宝盖

晁天王身披银铠,外罩绯色花袍,容光焕发;宋大哥身披黄金锁子甲,腰悬碧玉七宝带,黑脸泛着油光;吴军师一身新布
衣,素袍素带,一副书生模样

巳时,吉时已到

吴军师鹅毛扇一举,顿时锣鼓喧天,鞭炮齐鸣

晁天王矜持着,踏着鼓点,从左侧缓缓迈步,宋大哥微笑着,踩着锣声,从右侧拾阶而上

本来安排的是众头领都走左侧,但宋大哥宁死不走在晁大楞后面,吴军师只好把头领分成两批,从两侧同时登台

晁天王一边走一边向兄弟们招手,宋大哥一边走一边拍巴掌

晁天王招手很有特色,手臂在空中保持一个姿势,纹丝不动,手掌不停的抖,跟风中的旗似地

宋大哥拍手也很独特,左手摊开不动,右手拇指微蜷,四指伸出,轻轻的拍着左手掌心\dldots

最近晁宋两头领内讧的流言满天飞,山寨人心惶惶,吴军师特意安排了一个情节,让两位头领在台上握个手说两句话,
也好稳定军心!

两人很配合,像久别重逢的老朋友一样,脸上洋溢着热情的笑容,老远伸出双手,快速朝对方走去,四只手紧紧握在一起

晁盖低声说:你他娘的跟我后面走两步会死啊!

宋大哥说道:我日你大爷。

两人微笑,转身,面向台下的数万兄弟,携手举过头顶

众兄弟欢呼万岁! 

\section{}

山寨里最得意的,莫过于时迁了。

为方便指挥,点将台前特竖一木桩,高三余丈,需一人在上面向众人发号施令。

这可是件巨牛逼的事,你想想,全寨兄弟都仰着头,盯着你的一举一动,唯你马首是瞻,你说东他绝不敢西,你说南他绝
不敢北\dldots 就连晁天王和宋大哥,都得听你指挥\dldots

这是装逼的绝好机会,为这差事,吴用家门槛都被踏烂了,送银子的,拍马屁的,拉关系的\dldots

你若问,为这一天的装逼,值得吗?

值!绝对值!

牛逼虽然只能装一时,却可以吹一世!

你看武松,打虎也仅用了半个时辰,这都吹了多少年了?喝醉了就一拍桌子:老子当年打虎的时候\dldots

还有鲁智深,见谁都痛说逃难史:当年洒家三拳打死郑关西\dldots 对,三拳\dldots 只用三拳\dldots

人活着的乐趣,不就是年轻时干两件牛逼的事,年老时天天给人吹牛逼吗?

吴用最终选了时迁,虽然论关系,论相貌,论银子多少,怎么都轮不到他,但人有项特殊本领:会爬杆。

你送银子再多,到时爬不上去,那不闹笑话?毕竟山寨庆典非同儿戏。

时迁以前猥猥琐琐,跟人说话时满脸堆笑,目光游移不定。今天,容光焕发,戴一顶遮阳帽,外披一大红袍,背着手,昂着
头,东转转,西逛逛,见谁都板着脸,矜持的点点头,吩咐两句:一会注点意,看我手势,别误了大事\dldots 草!

这厮三下五除二爬到杆顶,坐上面,腰杆绷直,人模狗样,很是得瑟,前半生尽他妈的昂头看人了,今天,他终于可以低下
头看别人,那心情,无法形容\dldots

晁天王讲完话,下面鸣放礼炮。

放礼炮的是曹正,这家伙是个丧气鬼,好端端的事都能办砸了,谁也不让他插手。

曹正找到他师傅林冲,让给咂摸个差事,毕竟大家都在装逼时就他没逼可装是很痛苦的

林冲找到老朋友凌振,凌震虽然不乐意,却又摸不开林冲的情面,想了想,放礼炮跟打仗不一样,不用瞄准,随便往天上
放就行,没啥风险,就答应了。

晁天王讲话完毕,该鸣放礼炮,时迁掏出炮旗,猛力一挥

曹正为能参加装逼庆典很是高兴,把炮身东挪挪西动动,摆到最佳位置,看到炮旗后,点火\dldots

``轰''的一声巨响,众人大声喊好

抬头看时,却不见了时迁,只见那顶遮阳帽孤零零的挂在木桩上\dldots
\section{}

时迁被一炮轰出足足有二三里地,脸先着地,蹭的血淋淋的,袖子也磨没了,大红袍撕成了布条\dldots 躺在地上半死不
活\dldots

吴用赶去问他还行不行,这厮摆谱,问了三四遍都不应,眯着眼,直哼哼\dldots

吴用不耐烦了,说不行就换别人,这厮蹭的一下坐起来,肿成面包的嘴唇一张,咬牙挤出一个字:行。

这厮简单包扎了一下,呲牙咧嘴,手脚并用,勉强爬到上面,还不忘把帽子戴上!都这鸟样了还不忘装逼那!

放完礼炮,宋大哥上马检阅三军将士,

宋大哥骑匹千里嘶风马,戴顶如意紫金冠,左带一张弓,右悬一壶箭,雄赳赳,气昂昂,由南往北,依次从队伍前走过,一
边走一边喊口号:兄弟们辛苦了。

兄弟们报以热烈回应:替天行道!

走到点将台前时,嘶风马内急,马粪一地

宋大哥涨红脸,晁天王笑的跟花似地。

宋大哥想挽回点颜面,即兴发表了一番讲话,慷慨激昂,唾沫横飞,兄弟们都被感动的热泪盈眶,人群中都有人开始啜
泣,最后,宋大哥宣布:兄弟们,最后的胜利一定属于我们

表情悲壮,同时,右手握拳猛力一挥,撕拉一声,肩窝开了,露出白色内衣\dldots

宋大哥灰溜溜的结束了下面的检阅!
\section{}

下面是队伍游行

最前面是晁天王塑像,金大坚在一侧扶着,刘唐等十几个兄弟抬着,很吃力,弓着腰,绷着脸,不停喘粗气

塑像由金大坚日夜赶制而成,纯金打造,高两米,据说耗费近两千金

塑像通体金黄,阳光一照,金光灿灿。

金大坚技法高超,把晁天王刻画的面容冷峻,犹如天神,不怒自威。

晁天王很是满意,咧着大嘴,不停的朝自己金像招手。

这时,一只乌鸦赶来凑热闹,落在晁天王的金头上,不停的啄

金大坚急了,上蹿下跳的撵,乌鸦诚心跟他过不去,他在左边撵,乌鸦跳右边,他跑右边,乌鸦又跳左边

塑像底座大,金大坚来回跑了几圈,累的气喘吁吁

乌鸦自顾自的,啄完眼睛啄耳朵,啄完耳朵啄鼻子\dldots

华荣一时技痒,他想,反正是金的,肯定射不坏,拈弓搭箭,一箭射去,弓箭穿透乌鸦脖子直刺金像眼睛

意外,发生了

只听哗啦一声,晁天王脑袋碎了半边,瓷的,里面空空如也

金大坚愣了,刘唐等人也愣了,弓着腰走了两步,觉得再装下去没意思,挺起腰抬着塑像匆匆溜过去。

\section{}

接下来,九九八十一位兄弟,一人举块牌子,拼成十二个鎏金大字:祝晁天王和宋大哥寿与天齐。

这鸟主意是乐和出的,读书人花花肠子就是多,拍马屁也与众不同,他原本计划的是放鸽子,让上百只鸽子在空中摆个
``寿''字,可惜难度太大,只好放弃;后来又弄了只老虎,到时跑晁宋两头领前跪下磕头,本已经让武松训练的服服帖
帖,后又考虑一下,难保老虎到时候翻脸不认人,把宋大哥咬一口也说不定,拍马屁把马拍死的先例也不是没有,思前想
后,还是人最靠谱。

乐呵为了让这个马屁拍出最佳效果,让兄弟们把牌子抱胸前,等走到点将台前时,再突然举起来,给两位头领一个惊喜。

为达到整齐划一的程度,前后都排练了好几天,结果,举``天''字头上那一杠的石秀,走到点将台前时,为了躲嘶风马的
马粪,往里靠了靠,牌子一举,``寿与犬齐''

晁天王不识字,拍着巴掌,连声叫好

宋大哥气得直冷笑
\section{}

晚上开庙会,放焰火,与民同乐。

与民同乐,当然得有老百姓。老百姓虽然平常没啥用处,踩过来扁过去,想怎么欺负就怎么欺负,但到时候喊万岁还真
少不了他们

可是任乐和磨破嘴皮子,都没人愿意上山,给银子也不来,打死也不来,这也怪不得他们,平常兄弟们有事没事就去敲一
竹杠,都给吓怕了,在家都大气不敢出,更甭说主动往狼窝里跳

这可咋办?别的山上兄弟可以替,这个不好替啊,自己给自己喊好,忒磕碜!

关键时刻,葛老爷子领着一大帮男男女女涌到聚义厅前,高呼万岁,晁天王和宋大哥喜笑颜开,忙走下来与民同乐。

晁天王腰扎翡翠白玉带,翡翠居中,晶莹剔透,白玉镶边,洁白无瑕,是无价之宝,据说是梁中书送给蔡太师的礼物,被他
在黄泥岗给半路打劫了,每当重要场合就穿出来显摆。

宋大哥穿着件黄金锁子甲,当初蒋敬提拔为头领时送给他的,由一百零八块黄金甲串起来,宋大哥走到哪穿到哪

两人刚下来,大姑娘小媳妇你推我挤,哗的一下围了上去,比见了爹妈都亲

我就纳闷了,两人啥时候这么受欢迎?

一会功夫,人流散去,晁天王和宋大哥孤零零的站在原地,晁天王双手提着裤子,腰带不知所踪,宋大哥的黄金锁子甲,
只剩白色串线,风一吹,来回晃动\dldots

\section{}

梁山最近很兴旺,举着保境安民的大旗,四处征战,吞并了周边三州十二县。

那些官老爷,城破前,个个义正言辞,忠义凛然,当着全军将士的面,誓与城池共存亡;城破后,手持官印,趟过无数将士
的尸体,匍匐前行,磕头如捣蒜。

也有拒不归顺的,清风县的张县令,力尽被俘,被捆到宋大哥面前,犹骂不绝口,开口闭口都是杀不绝的梁山贼寇,宋大
哥本想凭三寸不烂之舌,说他归降,尚未开口,就被啐一脸唾沫\dldots 推出去咔嚓一刀了事,女儿妻子被兄弟们轮流糟
蹋,其余老少十八口,一刀一个,无一幸免,家资全部充公。

寿张县的李县令,被擒后冥顽不化,发誓忠于朝廷,说什么决不把清白之躯玷污了。吴军师用计,到处宣扬他已归顺梁
山,京师东路节度使把他全家老小问成死罪,打入死囚牢,他万念俱灰,嚎泣终日,目中流血,撞墙而死\dldots

屈膝投降的,成了功臣,官复原职,一家老小照样作威作福;

舍生取义的,尸体挂在城墙,全家遭殃。

张县令还好,朝廷为他罢朝三日,收养其子嗣,史书上也会留下浓重一笔。

李县令哪?

他忠于朝廷,却被朝廷所弃,他至死仍抱着宁肯朝廷负我我绝不负朝廷的决心,含恨而去,值得吗?\dldots 若梁山泊招
安了,成了朝廷忠臣,那他哪?史书会怎么写?忠臣?还是贼寇? <到那时,庆功宴上,是否还会有人记得,曾有一个县令,
为大宋朝尽忠而死?全家老小,却被朝廷所杀\dldots

他的孤魂,仍游荡在野外,忠烈祠中,也无他一席之地,那些杀死他的人,转而成为国家臣子,受万人敬仰,而他,若有若
无间,被人遗忘\dldots
\section{}

县令死了,位置腾出来了,兄弟们都很兴奋,别看平常一个个提起当官的来就恨得咬牙切齿,大骂当官的都是王八蛋,一
旦有当王八蛋的机会,谁也不肯错过

何况这是美差,不用天天窝山上,可以到处逛逛景,溜溜马,打打秋风,反正天高皇帝远,谁也管不着,爱怎么折腾就怎么
折腾,高兴了,开堂断断案,不高兴了,蒙头睡他娘的!

兄弟们上蹿下跳,有跑晁天王面前表忠心的,有给宋大哥送重礼的,有去找吴军师攀亲的\dldots

我也想当两天县令,过过瘾,毕竟我家祖上还没出过县令,也算是了了我那死鬼老爹的一桩心愿,就去找宋大哥

宋大哥听明来意后,断然拒绝,说我不是当官的料,我有些急,我虽然没文化,但也是在衙门混过的人,断案诉讼,多少懂
点门道

宋大哥不急不躁,端起茶杯,呷口茶,问道:你会扯淡吗?

我摇摇头,这个不怎么会

宋大哥又问:你会吹牛逼吗?

我又摇摇头,这个也不太擅长

宋大哥再问:你会睁着眼说瞎话吗?

我摇摇头,这个真不会

宋大哥大手一挥,把我给轰出来了,你丫哪里凉快哪里呆着去,连当官最基本的都不会,还来凑热闹!

\section{}

彤云密布,暴雨如瀑,一连数日

闲着无聊,去找宋大哥,这厮站门口,背手而立,一边摇头一边喃喃自语,说什么邦无道,天罚之\dldots 不懂啥鸟意思

去找鲁智深,这憨货还没醒,棉被蒙头,大腚外露,地下吐得一塌糊涂,脚丫子奇臭,刚进门就被顶出来\dldots

去找刘彦,这厮像上了炕的小媳妇,坐卧不宁,说雨再下,老百姓的房子就该塌了,真他娘的吃饱了撑的没事干,你吃好
喝好就行了,房子塌了怨他盖的不结实,管你鸟事\dldots

晁天王因身上有脂粉味,晁夫人跟他闹腾开了,晁天王说晁夫人无理取闹,晁夫人说晁天王拈花惹草,说着说着撕把开
了,晁天王经不住晁夫人的九阴白骨爪,脖子山黑一道红一道,一气之下,下山巡视,出门前不忘脱下锦袍,换上破棉袄

宋大哥不愿晁天王抢了风头,也换上宋老爹的旧羊皮袄,紧跟着下山,这样的事,他总不忘拉上我!

真被刘彦这乌鸦嘴说中了,山下房屋倒塌无数,灾民们挤在雨中,扶老携幼,瑟瑟发抖,不停的咒骂老天爷\dldots 看到
两位头领,哭声震天,盛赞两头领心系百姓,爱民如子\dldots

晁天王和宋大哥一脸沉重,顺手把棉袄脱下来罩到灾民身上,灾民们无不感动的热泪盈眶、泣涕横流,跪倒在地,大喊
青天大老爷\dldots

哎!我又得挨冻了! 

\section{}

出了王家道口,前后无人,我把外套脱下来给了宋大哥,棉衣给了晁天王,两人也不再装,坦然穿上

我只剩一件单衣,被雨淋透了,风一吹,一身鸡皮疙瘩,忒冷

没有灾民被困,两人很是失望,正准备打道回府,这时,探马来报,李家庄一处民房倒塌,村民王小二被压,两人两眼放
光,一溜烟跑去

戴宗已经安排人手,把现场围了起来,专侯两位头领到来。

王娘子蹲在水涡里,哭的撕心揪肺,三岁孩子浑身是泥,哭喊着叫爸爸,王老爹几次欲冲进去,都被戴宗拦住,闻讯赶来
的村民拿着铁锹站在外围,急的直跳脚

晁天王摩挲着王娘子的小手,不停的安慰,别着急,梁山一定会把人救出来的

宋大哥拍着王老爹的肩膀说,老人家你别上火,老百姓的利益高于一切,我们一定不惜任何代价救出王小二\dldots

宋大哥安慰完王老爹,又走到王小二身边,蹲下说道,兄弟,咱们虽然不是亲兄弟,但胜似亲兄弟,泥墙压在你身,却急在
我心,你的感觉,我们感同身受,我们一定急你所急,想你所想\dldots

宋大哥絮絮叨叨说了一炷香功夫\dldots

晁天王建议把泥墙掀开,直接快当,宋大哥说从侧面挖个洞,简单安全,两人又吵了起来,从中午吵到黄昏,都没吵出个
一二三四五来

王小二自己不耐烦了,从旁边找根棍子,插到泥墙下,一只胳膊撬着,另一只胳膊扒着地,自己爬了出来,站起身,拍拍屁
股走了

留下晁天王和宋大哥在一旁目瞪口呆\dldots

\section{}

前两天,跟时迁等人去济州府看花会,少不得逢客打尖,遇店吃饭。

兄弟们在一起,高兴的是饭前,个个吹能道会,满口兄弟义气,唾沫星子满天飞,恨不得剖开胸膛肝胆相照。尴尬的是饭
后,个个眼神漂移,目光深邃,揣着腰包,瞪着屋顶,犹如灵魂出窍。

第一顿饭,酒足饭饱,时迁调戏着碗里最后一根面条,一小口一小口的刺溜,怎么都吸不到嘴里;王矮虎上茅房,一头扎
进去,怎么等都不出来;乐呵醉了,趴桌子上呼噜震天响\dldots 哎,没辙,我一拍桌子,我请了!

刚付完帐,乐和醒了,一脸口水,大声嚷嚷,小二,结账\dldots

王矮虎从茅房窜出来了,一脸惊讶,啊?谁付的?不是说我付吗?

时迁嗖的一下把面条吸嘴里,把手伸口袋里,一脸实诚,黑哥,别,别,我来付,掏了七八下都没掏出毛银子来\dldots

连着几顿饭,顿顿如此!

吃一堑长一智,我也放聪明了,故意喝了很多酒,也想装醉,结果不小心真他妈的喝醉了,醒来时,店小二站面前,满脸堆
笑,说你兄弟去前面的小树林等你,先把帐结了吧!

又一顿饭,快吃完时,借口上茅房,跑里面蹲着,捂着鼻子呆了足足一炷香时间,只听店小二在外面喊,客官,出来吧,他
们没付账就走了\dldots 玛勒个逼的

\end{document}