\documentclass[12pt,twocolumn]{article}
\title{张飞日记}
\author{}
\date{}

\usepackage{config}

\linespread{1.2}

\begin{document}

% \pagenumbering{gobble}
\twocolumn[
\begin{@twocolumnfalse}
\maketitle\thispagestyle{empty}
% \tableofcontents
\end{@twocolumnfalse}
]

\clearpage

\pageaofbfoot{1}

\section{}

我写这个流水帐的时候,大哥和二哥都在睡觉,军师也在睡觉。

赤兔马站在我窗外,也在睡觉。

小时侯我就研究马为什么会站着睡觉,研究了很长一段时间后,我发现没有答案。而苦恼的是我的童年唯一能记起的
事就是这个了。

\section{}

长大以后有段时间我开始研究大哥和二哥为什么要睡在一张床上,同样也没有答案。

这个世界有太多的事是没有答案的,军师对我说过。

在我睁大眼睛思考问题的时候,我养成了睁眼睡觉的习惯,不知道以后有没有人研究我这个问题。

很多人都说我长的黑,魏延说我掉进煤堆里绝对找不着,其实我觉得他长的跟只绿头蝇一样,有什么资格来说我。

我最好的朋友是子龙(大哥和二哥是我的亲人),他说我长的很男人,这让我从此改变了小白脸没好心眼的观点。

我喜欢喝酒,因为喝酒会让我忘掉很多事。我最喜欢和二哥一起喝酒,虽然他不能喝。二哥喝酒有个特点,怎么喝都面
不改色,因为他的脸一直是那么红。

二哥其实是个很腼腆的人,有次他喝多了,和我唠唠叨叨地说了好多,他说他小时侯和女生说话会脸红,而偏偏坐在他
前后左右的都是女生,于是他的脸就变成了现在这个样子。

二哥不能喝酒,二两的小杯只喝一杯就可以睡在马棚里了。因此在温酒斩华雄的时候,我就知道他不会喝曹操那杯酒,否
则我又要到处去找他了。

我的酒量还可以,是二哥的两倍。

我喝多的时候觉得浑身充满了力量,胡子都极力地向外张着,再多一点的时候,我会想明白很多事情,虽然酒醒以后我
会全忘掉,但我还是喜欢那种感觉,军师说酒精可以刺激我的神经,让它们变得非常的敏锐。军师的话一向是正确的。

大哥不喜欢我喝酒,他说酒会乱性。我很听大哥的话,但这件事我没有听。我说不清楚我对大哥的感受,但这世界上有
些事情是没的选择的,你选对了专业,可能会跟错了导师,选对了行业,可能又选错了BOSS,没有任何事是十全十美的。
况且有些事情并非是机缘巧合,比如大哥之所以是我大哥,绝非因为我是他三弟这么简单。大哥有两个女人,死了一
个,后来又找了一个。二哥也有个女人。不过在有了阿斗后,大哥更喜欢和二哥睡在一起。不知道他们在一起聊的是
什么。

军师的女人很特别,很丑,不过军师好象很怕她。他们在一起不吵架的时候很少。怕她是因为爱她,军师说。

\section{}

我也曾有过一个女人,有段时间我们在一起很开心。可是过了没几年,情况就发生了变化,我们开始没有话说,我曾经
以为这就是传说中的默契,然而,后来我知道这不是。有一天她说她无法再忍受我的呼噜声了,过了没几天,她就收拾
东西走了,除了她的首饰,还带走了我的马夫。其实有句话我从没有告诉她,我一直觉得她的脚有点大。

后来他们又给我找了个女人,但过了两天就被我休了。

举个例子,就好象我喜欢吃煎饼卷大葱,我今天吃的是煎饼卷大葱,明天吃的是煎饼卷大葱,第三天我说我吃腻了,于是
你给我上了份大葱卷煎饼。外表看起来是有区别的,但吃起来的感觉却是一样的。

我的女人离开我的时候,子龙正在谈恋爱。那个女孩胖乎乎的,鼻子上有很多雀斑。

有一次我忍不住问子龙,你到底喜欢她什么?子龙偷偷地告诉我,你有没有发现她的胸部很大?我想了想说,是挺大。那
就够了,子龙眯成缝的眼睛里闪着光。

那天晚上我喝多了,轻飘飘的我觉得很舒服。我知道大哥和二哥在隔壁的房间,我看到子龙和那个女孩的影子印在窗
棱上,我听到军师夫人尖锐的喊叫声。突然间我明白了两件事情,鸡蛋永远也不知道狗的乐趣;袜子破没破只有自己的
脚知道。

\section{}

很多年前,当时我还在杀猪,一天有个算命先生说要给我算一卦,开始我不同意,后来他说只要一挂大肠就可
以了,于是我就同意了。他让我随便写个字,我只会写自己的名字,于是我就写了个``翼''字,那人沉思片刻,问我叫什
么名字,我说叫张飞,他又问我中午吃的什么,我说吃了三大碗米饭。于是他摇了摇头,把``翼''字上面的``羽''划掉,加
了个``米''上去,嘴里说道,酒囊饭袋,酒囊饭袋。然后转身走了,连大肠都没拿。

这个人后来我又见过他,他叫杨修,被曹操杀了。据说是因为他太聪明了。

原来太聪明也是一种错,这让我想起了阿斗。

阿斗是个很奇怪的孩子,9岁时才能用双手加双脚数到18,平日里总是呆呆地看着一个地方发笑,很少说话。他们都说
这孩子脑子有问题,比如你给他一块点心,他总是拿到屁股上蹭两下再吃,为此大哥打过他好多次也没用,于是大家总
趁大哥不在的时候用点心逗他。有段时间我一度以为他是让子龙在长坂坡那次给蒙在怀里憋坏的,觉得他怪可怜的。

后来我才知道我错了。应该可怜的人是你我和那些给他点心的人,军师说,你有没有发现阿斗几乎每天都有点心吃?我
恍然大悟,从此不再用点心逗他,而阿斗从此看军师的眼神也变得沉沉的。

我不知道阿斗是不是个聪明人,但有一点我敢肯定,至少他不是个傻子。

一个人装聪明不容易,装傻则更难,而一辈子装傻则更是难上加难。

一个真正聪明的人往往不怕他的对手装聪明,而害怕他的对手装傻。这使像我这种低智商的人也钻了不少空子。比如
那次在长坂坡。

那次的情况真的很危急,子龙一个人冲进曹营里救阿斗,大哥和军师脱了鞋狂奔了60多里地,醒过神来后让我回去接应
子龙。我单枪匹马的杀了回去,在桥头看见子龙,他已经累的口吐白沫了,见到我后说的第一句话就是:这倒霉孩子真
他妈的重!这是我第一次听到子龙说脏话,我说兄弟你先闪吧我挡着。

子龙走后不久曹军就追上来了,黑压压的好几万人,为首的那个白胖子我不认识,不过我猜他就是曹操,果然那人问道,来
者何人?我想反正今天是没戏了,索性大喊一声,俺是燕人张翼德,俺就一个人,来吧,拿钱砸死俺吧!曹操愣了一下,与
左右嘀咕了半天,然后下马在地上画了好几个美女图,再然后竟然一声呼啸撤了。

到现在我仍然不知道曹操画那几个美女图的意思,但有一点我知道,以后我在更多场合会变得更傻。

\section{}

那一年夏天,我还在涿郡,一日遇见一个道士,尖下巴,三角眼,手持拂尘,看起来有点猥琐。他拦住我上下打量着,我正
因为赌钱输了窝了一肚子火呢,上去就给了他一个大嘴巴,骂道,臭牛鼻子,你看大爷做甚?那老道端的好涵养,眼瞅着
五根手指印从他的脸上慢慢地凸现出来,他竟然咧嘴朝我笑了一下,说道,果然好力气。

他这么一笑,我倒有些不好意思了,我张飞向来是吃软不吃硬的,俗话说,恶狗不咬笑脸人,于是我便将底下踢出的一腿
生生地收了回来。伸手摸了摸衣兜,一文钱都没有了,于是把昨天卖剩的一个猪腰子掏出来给了老道,对他说,回去补
补身子吧,看你瘦的。

老道接了腰子,对我又笑了一下,然后噗噗从嘴里吐出两颗牙,说道,我这两颗牙折磨我一年了,始终没有勇气拔下
来,今蒙义士相助,又赠腰子一个,贫道感激涕泠,这里有三个锦囊,请义士收好。

说罢递给我三个绣花锦囊,我正疑惑中,却听老道继续说道,切记,不到万分危急时不要开启。说罢扬长而去。

我拿着锦囊愣了半天,难道这就是传说中的锦囊妙计?

回家后,我蹲在厕所里大便,心里想着刚才的事,越发觉得蹊跷,拿出锦囊来左右看着,一狠心,便拆开了一个,却见一张
纸条,上写:用此纸擦屁股吧。落款:左慈。

我靠,摆明了在耍老子嘛!我大怒,又拆了第二个锦囊,又是一张纸条,上写:一张不够?那这张也接着擦屁股吧,笨蛋。
落款依然是左慈。

人在盛怒之下脑子往往会比平时清醒一些,我当时就突然平静了,突然觉得这个锦囊真的很灵验啊,于是后悔自己莽撞
地拆开了那两个锦囊,但过了一会我就开心了,因为我庆幸自己没有打开第三个锦囊。其实有些时候开心很简单,只要
你换一个角度来看问题。

自从有了这个锦囊,我觉得做什么事都信心百倍,怕什么?我有锦囊妙计,大不了到了危机时刻我把它拆开,肯定万事大
吉的。于是我自此勇往直前,行军打仗,莫不身先士卒,敌人因此也闻风丧胆。我也格外地珍惜那个仅存的锦囊,有几
次算是危急的时刻我都没舍得用,比如徐州曹豹告密那次,比如芒砀山与大哥二哥失散那次,比如长坂坡。而每次也都
逢凶化吉,有惊无险。

直到有一天,我突然发现锦囊不见了。而且是早已不见。又似乎那个锦囊从来就没存在过。

\section{}

我平生有两个爱好:喝酒、赌钱。

喝酒我前面已经说过了,它让我可以得到暂时的清醒。而赌,是一种很奇怪的东西。我说不太清楚为什么会喜欢它。

大哥不喜欢赌钱,他更多的时候在想一些事情。二哥也不喜欢赌,通常没事时他会看书。军师就不用说了,假如有一天
他老婆没有和他吵架的话,他可能会搬个梯子到屋顶看星星,不过这种时候一般很少见。

而我喜欢赌钱,我喜欢那种屏神静气的气氛,也喜欢那种眼红脖子粗的场面,在那个青瓷大碗被揭开的一瞬间,血脉赍
张的那种感觉真好。

其实更多的时候我觉得赌博和打仗一样,一个是赌钱,一个是赌命。在赌场,如果你遇到一个高手,那么你的赢面会小
很多,同样,在战场上,如果你遇见吕布,那你活着回来的机会也很小。不同的是,我在赌场上是输的多赢的少,而战场
上却相反。这证明了你的武功和智商成反比,魏延说。这我得承认。

说起吕布来,我不得不伸出我的大拇指喊一声好汉子。在虎牢关那次,我、大哥、二哥三人都没从他身上讨到便宜,实
在是让我佩服得很。当天晚上回来,大哥在那里紧锁眉头长吁短叹,我以为他牙疼,就掏出块狗皮膏药来要给他贴
上,谁知大哥把手一挥,叹道,此人不除,我焉能得天下!我才明白原来大哥说的是吕布。于是我自告奋勇要去提他的人
头回来,虽然我知道他提我的人头的可能性要大一些,但大哥的心病就是我的心病,大哥的天下也是我的天下,当然最
重要的是大哥肯定不会让我去的。

虽然那次我没去,但我明白了大哥的意思。于是在接下来的很多场合我和大哥都在唱戏,他唱白脸,我唱黑脸,都不用
化妆。直到吕布在白门楼被曹操所擒,曹操让大哥来决定吕布的生死,我知道吕布死定了,虽然他口口声声提起辕门射
戟,但其实他不知道,那只戟原本就在底下系了透明的细线,他的弓一响,埋伏的士兵便将戟拉倒,否则就算他的射术再
精,又怎么可能将我们哥仨的性命压在他的稳定发挥上呢?

吕布死了,大哥那天破例请我和二哥喝酒,两杯酒下肚,我又感到了那种飘的感觉,我突然觉得其实我这一生也在赌博,我
把宝押在了大哥身上,他赢我才能赢,而大哥呢,他把宝押在谁身上呢?是我?是二哥?还是阿斗?亦或是他根本就没有参
与?酒劲上涌,我又糊涂了。

这几天阴雨绵绵的,没什么心情,看着身边来来去去的这几个人,觉得有必要把他们记下来,因为不知道哪一天可能就
见不到他们了。

那个坐在那里仰面看天的人叫魏延。当年魏延在长沙杀了韩玄救了黄忠,前来投奔大哥,大哥大喜,军师却大怒,命人
把他拖下去给砍了,我和二哥当时都很纳闷。当年大哥三顾茅庐请军师的时候,我当时倒是想把这个大冷天摇把鸟毛
扇子的家伙给砍了。

\section{}

大哥问军师为何要杀魏延,军师解释说魏延的后脑长了块反骨,日后必反。其实军师的这句话我根本不信,魏延当时带
着头盔,军师又没有过去摸一把,他怎么知道魏延脑后有反骨呢?按说军师跟魏延以前也不认识,没理由一上来就要杀
他呀?

后来子龙悄悄地告诉我,当时魏延跟黄忠一起投奔大哥的,当军师从外面走进来的时候,你有没有发现魏延没有跟军师
打招呼?我摇头说没注意,子龙说,这就是军师要杀他的原因。那天晚上我喝了两碗酒也没想明白是怎么回事,我也经
常不跟军师打招呼呀,军师为什么不杀我?子龙看着我笑,你没杀他就不错了。

其实魏延来了以后我挺开心的,虽然他经常地取笑我,说我黑,说我笨,但我也给他起了个外号叫绿头蝇。

魏延其实也不是个聪明人,估计比我高明不到哪去,但他却经常有意无意地装深沉。比如我俩一起去攻打一座城池,我
说冲过去厮杀一场吧,他偏要看过来看过去的说不行,咱要智取,他还在那里研究的时候,我已经拎着敌将的首级回来
了。当然也有反过来的时候,比如那次在葭萌关,我还在考察地形的时候,他已经冲了出去,等我追过去的时候,他已经
被马岱射了一箭坐在地上骂娘呢。

魏延经常喜欢跟我开玩笑,虽然他的玩笑并不怎么可笑,比如我俩一起骑着马走路,他会对我说,你那只驴是吃草还是
吃面呀?我说我骑的是马不是驴,他会一本正经地对我说,我知道,我就是在跟你的马说话呢。

军师一直不喜欢魏延,大哥对魏延还可以,其实大哥对谁都不错。二哥子龙他们对魏延总是爱搭不理的,其实魏延做事
很努力的,我没有看出他有一丝想造反的意思,不过军师既然说他有反骨,那自然比我说一千一万句都要有用。

有很多事情都是先入为主,也有很多事情是无中生有弄假成真。

但最主要的是有很多事情的起因往往是如此可笑。

\section{}

子龙这个人从严格意义上讲,是一个完美的人。

这世界上应该是没有任何完美的东西的,但如果说非要我找出一个无限接近于完美的人来,那我就会想到子龙。

子龙长的文质彬彬,谈吐风雅,满腹经纶,平日里抚琴执棋,舞文弄墨,乍一看,似一书生,但眉宇间掩不住一股英气,一
瞥之下双目中隐隐显出一种霸气。我最初见到子龙时,完全没有把他当回事,虽然他那时已经胜了文丑,但我总觉得这
样一个文弱书生,既便胜了也是巧合,也许是文丑那天正巧拉肚子或者长了痔疮。而军师最早见到子龙时说过一句话:子
龙,深藏不露者也!

直到长坂坡一战,我才真正认识了子龙。那次我得知大嫂与幼主掉队了,于是马上回去寻找,在桥头遇到子龙,我一肚
子的怒气都朝他发了过去,我骂他投靠了曹操,是个卑鄙小人,在那一刻我看到子龙的眼中闪过一道光芒,我竟然突然
觉得有一丝恐惧。子龙没有反驳我,而是掉转马头杀了回去,那可是千军万马呀,我只道我从此再也见不到子龙了呢,
正懊悔时,却见子龙又杀了回来,血染征袍,怒目圆睁,他对我说没有找到大嫂,说罢又转身杀了进去。如此七进七出,
最后一次我几乎都没认出他来。这一战,子龙名扬四海,两军阵前但凡提起常山赵子龙来,无不胆战心惊。我亦自此对
子龙刮目相看。

子龙平时话不是很多,但却经常给我讲一些道理。我有什么想不明白的事,也最喜欢去找他。比如我不知道我们整天
打来打去的是为了什么,子龙就会对我说,你看这天下百姓就好象是在一只沸腾的鼎里翻滚的肉,有好多人都在争来争
去的想独吞这锅肉,我们现在做的就是把这些人都打败,然后把火弄灭。

有时我忍不住想,上天为何对子龙如此眷顾?他身上几乎集中了所有男人的优点,世上又有哪个女子没有想过嫁给子龙
这样的人?

然而子龙并不快乐,我知道。

当年在桂阳的时候,子龙曾与一个叫赵范的人结拜为兄弟,后来二人饮酒时,赵范命其亡兄之嫂樊氏出来倒酒,子龙一
见之下,惊为天人,赵范见二人眉来眼去之间似有万条情丝,于是便要成全他们俩,子龙当时是大怒而起,痛斥赵范。再
后来大哥为其做媒,子龙亦不肯答应,说为了一个女子而败坏了名声,好男儿何患无妻!于是世人皆称子龙为真丈夫也。

但只有我知道子龙这么多年来一直郁郁不乐,他表面上很潇洒快乐,身边也一直不乏女人,但他对我说过他一直在想念
那个女人。有的时候你没有选择,真的没有,子龙喝多酒时红着眼睛说。

其实要是换作我的话,我要是喜欢一个人,天塌下来老子也要把她抢回来。但我不是子龙,所以我无福领略他的完美,
同样也无法体会他的痛苦。

老天爷其实还是比较公平的,不信你去做几天老天爷试试。

\section{}

今天来说说二哥。

二哥在世人眼中一直是神一样的人物。丹凤眼,卧蚕眉,飘三绺美髯;杀颜良,诛文丑,温酒斩华雄;过五关,斩六将,千
里走单骑;挂候印,封赏金,忠义冲宇宙。世上倘若当真有神,见到二哥也应该拜服。

其实论起武艺来,二哥未必胜得过子龙,论起力气来,他又稍逊我一筹。但真正跨上战马,两军对垒时,我和子龙却远不
如二哥杀伤力强。其原因之一是二哥面生神相,不怒自威,往那里一站端的威风八面,如天神降世。原因之二是二哥刀
重马快,一柄青龙偃月刀重约八十二斤,跨下赤兔马追风逐电,所以二哥杀敌往往是一打照面,敌将还没等看清楚的时
候就已经身首异处了。子龙当年与文丑战了五十多个回合不分胜负,但文丑至死都没看清二哥的长相。

二哥生性寡言,平日里总喜欢捧本历史书研究,你问他十句话,他或许能回答一句,但一般不超过三个字。由此很多人
都说二哥孤傲,但没有人责怪他,因为如果这世界上只有一个人有资格孤傲的话,那么这个人非二哥莫属。

其实我知道二哥绝非孤傲。二哥幼时家境尚可,父母老来得子,对其溺爱有加。然二哥打小就性格羞涩如处女,见到陌
生人就脸红,经常被同龄顽童欺侮。后来家中突遭变故,父母双亡,二哥当时只觉得日月无光,真想就此随二老而去。
后有亲戚收留,但亲情冷淡,受尽白眼,幸好二哥日渐长大,且性格亦变得十分的孤僻。在二哥二十多岁时,一日在集市
遭一混混戏弄,二哥自始至终一言未发,待那人言语侮辱其父母时,二哥怒目圆睁,随手抄起一把杀猪刀来,只一刀,便
结果了那混混的性命。自此二哥在外逃亡了五六年,后来遇到了大哥。

大哥和二哥之间的感情非常的微妙。不仅仅我弄不明白,甚至军师子龙也弄不明白。被我问的烦了,军师会用一句话
来敷衍我:子非鱼,安知鱼之乐?

有时候我看着二哥看大哥的那种眼神,突然觉得二哥好可怜。强极则辱,一个人外表看越是强大,软弱起来越是如此的
不可思议。而有的时候我看着子龙,又觉得其实二哥远比子龙幸福,虽说相见不如怀念,但又有几个人知道真正的刻骨
铭心的怀念的滋味?

一个过于自傲的人实际上是极端的自卑。而一个自卑过了头的人你最好不要去惹他。

\section{}

那天在荆州,恰是八月十五,大哥邀我们到后花园赏月饮酒,军师突然把一杯酒倒在地上,长叹一声道:此杯薄酒聊敬许
子远也。我们皆默然无语,大哥半晌突然长笑一声说:曹贼心胸狭窄,杀了许攸,实乃他的不幸。军师嘿然曰:可惜呀,
子远之才不在吾之下也。

许攸这个人我听说过,但是没有见过。他怎么死的我都不知道,于是回去后我就央求子龙讲给我听,子龙最近新泡了个
小妞,心情不错,便一五一十地给我讲了一遍。

许攸早先在袁绍那里做谋士,当时袁绍兵强马壮,中原各路诸侯无人能敌。许攸虽然足智多谋,但是袁绍手下人才济
济,也不太重视他。后来曹操挟天子以令诸侯,袁绍遂起兵讨伐曹操。两军对垒了数日,在官渡展开生死决战,曹军日
久粮草已绝,遂令人火速赶往许昌催粮,不料使者竟被许攸截获,许攸上报袁绍,反被怀疑,因为许攸乃是曹操少时的好
友。于是许攸一气之下索性降了曹操。后来正因为许攸的计谋,曹操才大败袁绍,占领了翼州。

却说当日曹操听说许攸前来投降,正在洗澡,来不及穿衣服,围着块布光着脚就跑出来了,后来曹操跣足迎许攸的典故
被世人盛传。

但许攸被曹操如此重用,死的却很冤,是被曹操手下的许楮给杀的。当日攻下翼州后,许攸遇见许楮后吹捧了一下自
己,却被许楮一刀砍死,后来曹操深责许楮,厚葬了许攸。

许楮是个火暴脾气,跟我差不多,估计他当时也是一时恼怒才杀了许攸。子龙摇摇头说:非也,倘若不是曹操想杀许攸,便
是借给许楮十个火暴脾气,他也不敢杀死许攸的。

曹操对许攸如此厚待,况且破袁绍许攸立了大功,又为何要杀他呢?我想不明白。

子龙告诉我,原因有三:一是许攸当日投靠曹操时第一句话就是问曹操粮草如何,曹操连着三次虚报,皆被识破,许攸当
时说了一句话:世人皆言孟德奸雄,今果然也。二是破了翼州后,许攸进了城后用鞭子指着城门直呼曹操的小名:阿瞒,你
若不得我,安得入此门?三是袁绍已经被打败了。

前两个原因我明白,曹操生性心胸狭窄,虽然表面不动声色,但内心却极为谨慎,一点小事他能记一辈子,城府颇深。但
第三个原因我就不懂了,袁绍败了跟许攸有何关系?

子龙笑了笑说:狡兔死,走狗烹;飞鸟尽,良弓藏。

子龙走了后,我呆呆地想了很久,鸟都死光了,要弓做什么呢?可是如此说来,如果大哥得了天下,那我呢?

\section{}

今天出门看见一个乞丐,长相甚是奇特,圆圆的身子上举着一个小脑袋,如同一个鼓槌插在一个西瓜上,让我忍不住想
起了祢衡。

祢衡长的比这个乞丐有过之而无不及,细长的身子上顶着个大脑袋,如同一个牙签上挑着一个八两的馒头,说话时还喜
欢摇头晃脑,让人看得胆战心惊,生怕他那个细脖子承受不住,万一那斗大的脑袋掉下来可不是闹着玩的。

那年我偶尔见到祢衡,吓了我一跳,这种人怎么不去死?居然还满大街溜达来吓唬人,我真想上前暴打他一顿,二哥制止
了我,对我说,有奇相必有奇能,此人有经天纬地之才,是出了名的大才子,孔融曾用四个字形容此人:不可多得。

原来这个人还挺厉害的,怪不得那么大的脑袋呢,可我看他的样子明显是个闲汉,怎么没人起用他呢?二哥摇了摇头,没
有说话。

后来我们在徐州的时候,曹操已经是汉丞相了,孔融终于向曹操推荐了祢衡,曹操也是久闻其名,于是就召见了祢衡。
至于后来发生的事,都是二哥对我讲的,我觉得非常的有趣。

祢衡见了曹操后,一上来就把曹操手底下的一群人给骂了一顿,骂夏侯惇的那一句最有意思:夏侯惇可以称得上是完体
将军。夏侯惇的眼睛瞎了一只,最忌讳别人提到自己的残疾,听祢衡这么一说,差点没背过气去。然而曹操却没有生
气,他反而让祢衡为之击鼓贺宴,命他做一个小鼓吏。祢衡欣然答应。

第二天,祢衡穿着一身破衣服来到厅上击鼓唱歌,歌声悲壮莫名,旁边的宾客哭倒了一大片。曹操的手下喝令祢衡更
衣,于是祢衡竟然当众脱光了衣服,一丝不挂,歌声不停。奇怪的是祢衡如此无礼而曹操居然没有杀他。只是派祢衡去
劝降刘表。

祢衡到了刘表那里,仍然是一顿讽刺挖苦,刘表的手下都要杀他,但刘表却也没有杀他,反而把他派到黄祖那里。

到了黄祖那儿,没说上几句话,黄祖就把祢衡给杀了。

听到这里我觉得很奇怪,按说曹操心胸狭窄,祢衡三番两次地羞辱他,早就应该把他给杀了,为什么反而派他到刘表那
里呢?二哥笑着说,这就是曹操狡猾之处了,祢衡是天下名士,曹操如果杀了他,必将遭天下人所不齿。同样刘表也是这
么想的,所以把他转派到了黄祖那里,而黄祖是个粗人,一句话不合便杀了他,平白无故地惹了千古骂名。

哦,原来如此,于是我明白了为什么黄祖杀了祢衡后,把祢衡的脑袋又送回到刘表那里,而刘表则连夜又送还给了曹操
的原因。

祢衡就好比一个烫山芋,谁捧在手里都觉得烫,但扔掉却又被人指责为浪费粮食,于是聪明的曹操转给了刘表,刘表则
转给了更傻的黄祖。

但还有一点我弄不明白,祢衡为什么走到哪里都会被人讨厌,每个人都想杀他呢?

在一天夜里我突然想明白了:是不是因为祢衡的脑袋太大了,他的脖子已经支持不住脑袋的重量了,而他自己又下不了
手,所以他假手曹操,曹操假手刘表,最终黄祖完成了使命。

想通了以后我觉得祢衡真是个聪明人,而我自己其实也不笨。今天无事,我吃了饭后出门晒太阳,正碰到大哥,他正眯
着眼睛看着天空,见到我,他忍不住对我说,三弟,你看那云。我抬头一看,顿时欢呼道:好大的一团棉花糖呀!大哥白了
我一眼,兴趣索然地说,什么东西都能让你想到吃的,倒也难得。

其实也不能怪我,我小时侯家里很穷,每当饿的时候,他们总是随手抓块东西给我,我吃不下去的时候,他们就会在我旁
边描绘,比如给我根玉米棒子,他们就会说那是一根鸡腿,黄黄的,泛着油光,于是我就在想象中把玉米连同棒子一起嚼
进肚中。再后来也不用他们说了,我自己也学会了想象,凡是看到的东西,我一律能想象成吃的。

我吃的其实也不能算多,最多一次也就吃了3斤包子2斤牛肉外加8张大饼,饭后又喝了点面汤,吃了两个10斤左右的西
瓜而已,每次跟他们说起来的时候,他们总是用很惊讶的眼光看着我,然后总用一种动物来比喻我,后来我忍不住对他
们说,其实猪一顿也吃不了这么多的。但是猪能连着三天不吃饭吗?我能。打起仗来,未必全能取胜,有时候是边打边
逃,几天不吃饭很正常,士兵们往往连腰带都煮了,但我不用,我滴米不进仍然勇往直前,请问猪能吗?于是通常第二次
再叫我猪的人都会受点教训。

年轻的时候我经常做梦娶媳妇,每次醒来的时候总是流着哈喇子。现在我也偶尔做梦娶媳妇,但每次醒来的时候总是
一身冷汗。我不象二哥那样喜欢读书,也不象军师那样喜欢思考,更不象大哥那样喜欢做皇帝,除了打仗和赌,我最大
的乐趣就是吃了。

我吃东西不太讲究,逮着什么吃什么。而且喜欢一样东西总要想方设法吃个够。比如我去兖州出公差的那次,军师临
行前封了三道锦囊,说里面写着我一天三顿所吃的食物,我自然一百个不信。去了兖州后,我见街上卖的全是煎饼卷大
葱,于是忍不住买了几个尝了尝,果然味道香甜,待到中午时,又忍不住买了几张,刚想吃,忽然想到军师那狡黠的笑,于
是我心生一计,我把葱从大饼里抽出来,包在饼的外面,然后开始吃了起来,真香呀,晚上我忍不住又买了几张,揣在怀
里,骑马开始往回赶,等回到城里,天已经全黑了。我进门就让军师打开锦囊,军师微微一笑,把锦囊递给我,我打开后
见上面写着:早晨饼卷葱,中午葱卷饼,晚上将军没吃饭,饿着肚子回城来。我顿时惊叹不已,军师果然如神一样,厉害
呀厉害!我一边伸着大拇指,一边从怀里掏出大饼大吃了起来。

有一天半夜我突然觉得饥饿难耐,于是爬起来到厨房找吃的,摸着黑我找到了一晚黑米粥,咕嘟咕嘟喝下去后意犹未
尽,咋叭了两下嘴,觉得有点苦,于是又摸了两个生土豆吃了。第二天早上听到军师在窗外大呼小叫:谁把我夫人的安
胎药给喝了?

\section{}

子龙有一天来找我,坐在那里东张西望的好象有什么事,我故意不去问,年轻人的肚子里是存不住东西的,果然他忍不
住对我说,三哥,我刚听了几个笑话,非常好笑,你要不要听?

我知道子龙前几天跟那个姑娘闹别扭了,他肯定想求我帮忙,于是便说,你想讲就讲,不想讲也罢。其实我并非故意与
他为难,我向来并不喜欢听什么笑话。

子龙本来兴致勃勃的,听我一说,登时索然无味,而话又说出来了,只好讲了一个:说有这么一个人家,家里很穷,有一天
父亲出门买了两条咸鱼,回来后悬挂于饭厅的梁上,从此每日里全家人吃饭时饭桌上便只有饭没有菜,每吃一口饭,便
抬头看一眼咸鱼,权当吃了一口鱼。这一日吃饭时,儿子吃了一口饭,实难下咽,忍不住多看了两眼咸鱼,父亲大怒,拍
桌子骂道:小畜生,也不怕咸死你!

讲到最后一句子龙已经开始哈哈大笑了,而我则面无表情,直直地盯着他,子龙笑了一阵觉得无趣,于是又讲了一个:说
有一个贼,半夜潜入一户人家里偷米,主人恰巧醒来如厕,见到贼后也没有声张,贼径直前往米缸处,脱下外衣铺在地
上,然后从缸中往衣服上捧米,主人跟在其后,悄悄地把衣服抽走,那贼捧了数把,正想用衣服包好后离开,在地上摸索
了半天却不见了衣服,于是大声呼叫:有贼!主人答曰:无贼。贼大怒曰:无贼?无贼我的衣服哪里去了?

子龙这次讲完后没有笑,他直直地盯着我看,我反而忍不住哈哈大笑,子龙大喜,也随着我大笑起来。其实我在听第二
个笑话时,心里一直在想着第一个笑话,开始我觉得没什么好笑,因为我小时侯家里也很穷,但后来想着想着我突然领
悟到:是不是我们每个人心中都有一条咸鱼挂在哪里?大哥、曹操、孙权的咸鱼是那皇帝的位子,而我和二哥、子龙的
咸鱼便是大哥坐在那个位子上的样子。那咸鱼挂在那里纵使每天看上千眼万眼也不解谗,但倘若真的拿下来吃上数口
的话,会不会真的咸死呢?想到这里我便忍不住大笑。

而第二个笑话,我没怎么听,隐隐约约感觉好象是个贼喊捉贼的故事,那个贼有点笨,而我在别人眼里也是个笨人,所以
我不觉得可笑,反而有点同情他,偷米是贼,偷衣服为何不是贼呢?

\section{}

二哥镇守荆州的时候,我正跟着大哥打成都,占了成都以后,我与子龙便请了个假去看二哥。进了二哥的府上,却见一
人如木鸡般立在客厅门口一动不动,仔细一看,原来是周仓。

周仓长的跟我是一个类型的,都属于掉到煤堆里找不到的那种。当年我们桃园三结义的时候,虽说是散兵游勇,但好歹
打着正规军的旗号。而周仓那时却在跟着黄巾军打游击战争,后来黄巾军被灭了,他便拉大旗扯虎皮的做了山贼,按说
这应该是个很有前途的职业,可这小子心气挺高,一直不满意,后来终于碰到了二哥,恰巧二哥当时正护送两个嫂子去
找大哥,见他块头挺大,便收他做了跟班。这家伙倒也卖力,平日里二哥走到哪,他便扛着大刀跟到哪,弄得有段时间我
也想收个跟班。

周仓有个毛病,就是嘴有点碎。甭管什么场合,甭管什么话题,他总要插上几嘴才过瘾。军师曾经当着周仓的面说:你
呀,就是骡子卖了个驴价钱,坏就坏在那张嘴上。二哥也曾无数次训斥过他,但江山易改,本性难移,这小子依然死性不
改。

当日我见他立在客厅门口,心里也猜了个大概,故意走上前去问:周仓,大热天的你杵这儿干什么啊?周仓挤了挤眼,努
了努嘴,面色很尴尬,可就是不说一个字,我忍不住哈哈大笑。

二哥听到声音后出来,把我们让到客厅,落座以后,我便问二哥是怎么回事。二哥长叹一声,道:我早晚要死在这小子的
嘴上。

原来前日鲁肃邀二哥到陆口临江厅赴宴,当时情况复杂,敌我不明了,很显然这顿饭不是那么好吃的,但二哥久在荆州,嘴
里都淡出个鸟来了,于是横下心便去了。酒过三巡时,便开始谈到正事了,鲁肃拐着弯的想把荆州要回去,二哥也兜着
圈子的就是不给,正在双方打着哈哈较劲的时候,周仓在旁边扯着嗓子喊了一句:天下土地,惟有德者居之。岂独是汝
东吴当有耶!这句话一出,双方都是一惊,气氛马上变得不融洽了。鲁肃挥了挥手,侍应把刚上的那盘大闸蟹给撤了,二
哥气的脸都绿了,回来后罚周仓站6个时辰,并且警告他,倘若再多嘴就把他扔到江里喂王八。

二哥说完后咂了咂嘴,唉,可惜那盘大闸蟹啊,我连条腿都没吃着。子龙笑了,说到,二哥,当日那情况很凶险啊,你能完
整地回来就已经不错了。二哥不以为然地说,比这凶险的事我经历的多了。于是我们三个便开始讨论天下最险的事。
子龙说天下最险的事莫过于火上了房,我知道他又想起赤壁之战了。但我想起小时侯家乡发大水的场景,一望无际的
大水,遍地都是浮肿的尸体,于是认为水上了墙才是最险。二哥沉吟了半天,说道:小孩趴在井沿旁。我和子龙想了一
下,齐声赞叹二哥有创意,这个果然是险中之险,真不亏是读书人啊!

正在这时,却听门外周仓大声喊道:喂王八就喂王八,天下最险之事就是流氓骑在媳妇身上!

\section{}

今天阳光明媚,我站在门口对着太阳剔牙。其实早上就喝了一碗稀得能数出米粒的稀粥,真没什么东西可以塞牙缝的。
但剔牙是一种姿态,如果你大清早看见一个人眯着眼睛很悠闲地剔着牙,你一定会觉得他生活得很有质量。

最近正是青黄不接的季节,加上连日作战,我们这些将领每天也只能领到一小把大米,底下的兵士们就更不消说了,个
个饿得面黄肌瘦的,站岗的拄着枪,巡逻的爬着走,真正的惨不忍睹。而我自己其实也饿得两眼发花,但我必须要挺住,这
样子才能稳定军心。

魏延弯着腰从旁边走过来,见到我愣了一下,上下打量着我,我被他看得莫名其妙。而且这小子不仅是看,还把大鼻子
凑过来不停地嗅,我猛然醒悟了,我靠,不会吧?这小子不会饿到如此地步吧?看着他白森森的牙齿我有些恐怖,连着往
后退了好几步。

魏延诡秘地一笑,又凑了上来,我大叫道:你,你离我远点!魏延依旧保持着笑容低声说:三哥,有什么好吃的啊?别自个
独吞啊,也让兄弟打打牙祭呀。我低头看了看手里的牙签,又想了想,突然开心起来,于是笑着对他说:嘿嘿,小点声,别
让别人知道哦,晚上来找我吧。

看着魏延屁颠屁颠的背影我在心里狂笑,可不大一会儿,子龙来了。子龙依旧保持着潇洒的身姿,虽然他的眼眶有点深
陷,但笑容依旧优雅迷人。他就那么笑着对我说:三哥,不够意思了吧?我愣了一下,疑惑地说:什么呀?子龙的脸一下拉
的比驴还长,转身便走,边走边说:得,以后甭说认识我,咱哥俩到此为止。

我用了一柱香的时间才想明白到底是怎么回事,没想到魏延也是个大嘴巴啊,正懊恼间,见一副将扶着墙进来了,有气
无力地对我说:将军,老大找你。

一进大哥屋里就发现气氛不对劲,人很多,军师,二哥,子龙,还有魏延,都在。个个虽说站的不是那么笔直,但表情绝对
严肃。我看了看大哥,说道:大哥,找我来什么事啊?大哥咳嗽了两声说:咳咳,这个\dldots 军师在一边接了茬:翼德啊,是
这样的,今天军士发现主公的卢马少了一只耳朵,不知道是被谁割掉了。我大怒:是谁这么大的胆子?说完后忽然发现
众人眼神有异,忍不住张口:你\dldots 你们\dldots 难道是怀疑我?

大哥挥了挥手:三弟,别胡思乱想,大哥是绝对不怀疑你的,别说区区一个马耳朵,便是整座城池你也不会要的。大哥虽
是这么说,可别人看我的眼神依旧没有变,当时把我气得须发皆张,刚想发作,忽然门外进来一人,扑通一声双膝跪到在
地:主公,臣罪该万死,是臣偷割了马耳朵。大家定睛一看,原来是马超。

一时间都面面相觑,很多时候当事情出现了你意想不到的转折时,大多数人通常都保持沉默。当然事情的结果还是不
了了之的,毕竟只是一只马耳朵嘛,况且大哥又是如此仁爱之人,但我总隐隐觉得过程中有点不对头,可怎么也想不明
白。

直到很多天以后的一次酒宴上,马超举着杯朝我走过来,当时我已经喝得看着他的头有两个大的程度了,他低声对我说
了一句:还记得马耳朵的事吗?我愕然,他微微一笑:那天早上我偶然看到主公在后山不知道埋什么东西。

在喝醉的时候我脑子总是特别灵光,于是我一下子全明白了。

背黑锅是谁都不愿意的,但关键要看背黑锅的场合,当然更关键的是你给谁背的黑锅。

后来马超与我们一起被封为五虎将的时候,虽然二哥老大的不高兴,但我却一点意见也没有。

\section{}

我的女人离开我的时候,给我留下了两个孩子,一男一女,儿子出生的时候我正在吃包子,于是便取名为包子,后来军师
给改为张苞。女儿就叫丫头,叫着倒也朗朗上口。(至于张绍是我手下一个偏将的儿子,偏将战死以后,我见他可怜,便
收为义子。)

当时大哥已经有了阿斗,二哥已经有了关兴。自从我知道阿斗这孩子深藏不露以后,便天天叫包子跟着阿斗混,俗话
说,近朱者赤嘛,我也想让包子多跟着阿斗学点心计。可还有句俗话叫做:龙生龙,凤生凤,老鼠的儿子会打洞,包子虽
然长的比我白一些,但那笨劲儿比我还略胜一筹。跟着阿斗不但没变聪明,反而越来越笨,后来我才知道,人家阿斗是
装傻,我儿子那是真傻。

有一天傍晚,包子从外面回来,坐在门槛上双手托着下巴望着天空发呆,我见状很奇怪,就问他在干什么,他说在看日出。
我吓了一跳,就听他继续说,你不是让我跟阿斗哥学习嘛,我早上去找阿斗哥,见他就是这个样子看日出的。

还有一次,军师来我家,我对包子说,去给军师沏杯茶。过了良久,包子端着一个大茶盘出来了,上面放了七杯茶。我大
怒,包子却得意洋洋地说,你不是让我给军师七杯茶嘛,你看,一、二、三、四、五、六、七,正好七杯,我数了好几遍
呢。军师摇着鸟毛扇子抿着嘴说,翼德啊,照我看来,阿斗这孩子是大智若愚型的,而你这包子却是典型的大愚若智啊。
谁知第二天我去包子卧室发现墙上贴了一副字,上面歪歪扭扭地写了四个大字:大愚若智,落款:张苞手黑。 看着那个
``苞''字我突然明白了军师的意思,``苞''不就是草包嘛。

眼瞅着儿子是完了,我便把心思放在了女儿身上。别看我长成这样,可我那丫头却一点都不象我,随着年龄的增长,出
落得如花似玉,越来越水灵,而且这孩子比她哥哥强一万倍,除了针织女红外,琴棋书画样样精通,谁见着谁夸,魏延那
次跟我说,看不出你这黑炭头生儿子不行,生女儿倒挺拿手。

有段时间包子每天回来都兴高采烈的,还经常带回些小东西,比如水果啊点心啊小扇子啊等等,说是阿斗哥给的。再后
来我发现阿斗来我家的次数越来越多,而且俩人关上门一聊就是一上午,我心想这小子行啊,几天没留神,居然跟阿斗
走的这么近了。可又过了一段时间,我发觉有点不对头,有一天丫头从我身边低头走过,我突然发现有点异样,她的腰
怎么那么粗?天那!我恍然大悟!

晚上我很郁闷,于是找子龙来喝酒,越喝越窝囊,唉,儿子不成器倒也罢了,那么好的女儿却也被人搞大了肚子,我活得
真失败。想着想着眼圈便红了。子龙劝我说:三哥,你别那么想,包子虽然不怎么聪明,可也不是没有优点啊,前阵子我
看他耍了一会枪,有模有样的。至于丫头,早晚都是人家的,退一万步来说,你想让你女儿一辈子待在家里守着你啊?

晚上躺在床上,看着银子般的月光透过窗子落在地上,一格一格的,我忽然想通了,人啊,怎么都是一辈子,健健康康的
快快乐乐的就最好了,事情虽然没朝着最好的方向发展,但至少也没朝着最坏的方向发展嘛。想到这里我特欣慰地睁
大眼睛睡去。我生命里有一个女人不得不提,说起来这事有些荒谬,但又有谁一生中没做过几件荒谬的事呢?

这个女人叫孙尚香,她哥哥叫孙权。她本来是大哥的女人,也就是我的大嫂。

孙尚香其实长的不好看,五大三粗的,黄头发蓝眼睛,有人说她和她哥哥都不是汉人,是没开化的胡人的种,但这话只能
背地里说说,因为他们的父亲孙坚是个地道的汉人。

当初大哥的这桩婚事本是周瑜的一个计策,结果弄假成真,赔了夫人又折兵这句话被当作童谣唱了好多年。对于很多
人来说这是一段佳话,但对孙尚香来说这是一个噩梦。

最早的时候哥哥对她说:刘备一表人才,二十年前,率兵攻打黄巾军势如破竹,威名显赫,才三十三岁。孙权把那个二十
年前说得很快,可怜的孙尚香只听到了最后的三十三岁,结果洞房之夜才发现是个老头子,由此可见说话的轻重缓急绝
对是门学问。

而反过来说呢,大哥却也只把这门亲事当作霸占荆州的一个棋子而已,说实话,自从有了阿斗以后,他似乎再没跟女人
睡过觉。于是这桩名存实亡的婚姻便造就了一个寂寞的女人。

但为什么是我?为什么不是军师、子龙或者大哥的马夫?很长时间我一直弄不明白这个问题。我不停地回忆那个晚上,可
惜很多细节都已经记不起来了,只记得那天晚上月亮很圆,我喝了很多酒。月圆之夜会有很多怪异的事情发生的,军师
曾经这么说过。而大哥则不止一次地对我说,酒不是好东西。

当倘若仅仅是月圆和喝酒那次倒也罢了,可后来\dldots 我得承认,人是会很多次掉进同一个坑里的,开始是偶然,后来就
是习惯了。我得承认我迷恋她那空洞而痴迷的眼神。

我努力地为自己找借口,事实上我们每个人做任何事情都在为自己找借口。但我发现随着事态的发展我越来越无法自
拔,我经常会在黑暗中大叫一声醒过来,浑身都是冷汗。我曾经拐弯抹角地咨询过子龙,子龙给了我一句话:有些事情
即便是如何的天经地义也会让有些人寝食难安,而有些事即便是如何的罪大恶极也会令人心安理得,因为我们看到的
只是事情的表面。

子龙的话让我想了好多天,最终我做了个决定:从坑里跳出来。也许很多年后我会为这个决定而后悔,但我做了决定以
后舒舒服服地睡了一觉。

她一点也没有惊讶,其实女人真的很可怕,在好多地方她们都显得远比男人理性而坚强。她就那么静静地坐着如同一
块石碑,不知道她在想什么,很多年以后马超对我说,永远也别企图知道一个女人在想什么。马超是个走一步踩一个脚
印的人,他的话应该有道理的。

没几天孙尚香就走了,走的时候还抱走了阿斗。大哥命我和子龙去追,我到了江边发现她就立在船头,我跳上船问她,
为什么抱走幼主?她表情恍惚地说,倘若我不这样做你会来见我最后一面吗?我愕然,想了半天,摇头说,不会。于是我
看见泪水顺着她的脸庞汩汩地流,不知道为什么我忽然觉得心里好乱,这时候船舱里上来一个人探头探脑的,我随手一
剑把他劈成两截,抱着阿斗上岸头也不回的走了。

自此我再也没见过她,也没有关于她的任何消息。在你的生命里,有一些人跟你的关系象两条平行线,保持着固定的距
离却永远也不可能相遇;还有一些人跟你的关系则如同两条交叉线,在经过一个交叉点以后便愈来愈远。

\section{}

我以前是个杀猪的,大哥就差一些,他是个卖鞋的,而二哥更凄惨,是个逃犯。我说这些的意思是我们的出身都很低下。

当然我们当中也有出身好的,比如马超,世袭王侯,虽然比袁绍的四世三公要差很多,但在西凉那种鸟不拉屎的地方却
相当于一个土皇帝。至于军师嘛,也不是个干体力活的人,虽然当年住了个破草房子,但他小小年纪便有书童伺候,想
必家里条件也不差。说起来还有件有意思的事,军师有个哥哥叫诸葛瑾,在孙权那里做大将军。军师还有个族弟叫诸
葛诞,在曹操手下做官。有一次大哥开玩笑地说,你们姓诸葛的一门三方为冠盖啊,真有一套。军师正色道:良鸟择木
而栖,乱世之间,各为其主,虽天下荣之,然难免手足相残,实乃迫不得已之下策啊。

开始皇帝老儿还在的时候,打仗时都要互通一下姓名、官位和出身背景,大哥还好一些,他不知道从哪里弄来一本破家
谱,非说自己的老子的老子的老子跟皇帝老儿的老子的老子的老子是一个人,这样一来虽然说起来比较拗口,但也能唬
人一跳。而我和二哥相比之下就悲惨了,我通常也只能大叫一声俺是燕人张翼德,而二哥在更多时候喜欢默不做声地
上去就是一刀,颜良和文丑错就错在话太多了。

到后来打仗打的乱套了,各种封号也就多了,象二哥被人称为汉寿亭候,这个官还是当年曹操给封的呢,而我最大的官
是大哥给的,叫什么西乡侯,其实管他什么东乡西乡的,也就是随便那么叫着而已,对我来说都无所谓。当初跟大哥出
来混的时候从没想过要做什么侯,大哥在安喜县做县官的时候,我和二哥一个打锣的一个叫堂的也做得很开心,如果不
是那个督邮过于仗势欺人的话,也许我就做一辈子衙役了。

大哥能有今天他自己也没想到,我不知道他以前的目标是什么,但我知道他现在想做皇帝。这就跟爬山一样,上了一个
山头,发现前面还有个更高的,于是便继续往前爬。我很奇怪为什么在我的前面就没有山头让我爬呢?子龙给我说了个
故事,说有一只驴子,主人在它鼻子前面拴了根胡萝卜,于是它就不停地走下去,但他永远都吃不到那根胡萝卜。我想
了半天,我是那只驴子,但胡萝卜呢?我的面前也没有胡萝卜啊。子龙笑着说,那你比驴子还蠢,没胡萝卜你都照样卖力
地干活。虽然我不想承认我比驴子还蠢,但事实就是这样,我弄不清楚只得接受。

其实有时想一想,倘若当初不是黄巾做乱的话,我也许会成为一个出色的肉贩子,或许还能开好多个分店,没准今天你
吃的肉上面就有我的商标呢。这么看起来,现在我骑着高头大马一人之下万人之上威风凛凛的样子也自豪不到哪去。

如果说人走每一步都是上天注定的话,那么在每一个交叉路口我们都没有必要停下来瞻前顾后的,没有哪条路是正确
的,同样也没有哪条路是错误的。

\section{}

军师今天又和夫人吵架了,和以往稍微不同的是这次吵的比较厉害,夫人甚至把他的鸟毛扇子也撕了,并把他关在门
外,看着军师那无奈的背影我觉得他的脾气实在是好极了。

我的脾气不好,跟军师没法比,甚至连魏延都不如。我有时喜欢打士兵,在这件事上所有人都说过我。我知道这解释起
来很困难,但我还是试图让你们明白。

比如一天早晨我起来去后山锻炼身体,突然发现漫山遍野绿油油的,哇,小草发芽了!

回来的路上遇到军师,他说:翼德,小草发芽了。我说:真的吗,小草发芽了?

遇到子龙,他说:三哥,小草发芽了。我说:哦,小草发芽了!

遇到大哥,他说:三弟,小草发芽了。我说:是,小草发芽了。

遇到一个手下,他说:将军,小草发芽了。我说:滚你妈的。于是我就把他吊起来用鞭子打了一顿。

也许我说的还是不够清楚,但很多时候除了打人我想不出更好的法子来发泄自己的郁闷。也许那个士兵很倒霉,但你
要知道,有些人生下来就是被人打的,反过来说,我们俩互换一下位置,挨打的那个肯定就是我了,社会就是这样,有些
东西根本无法改变,或者不打仗了会好一些,可谁知道呢?

二哥跟我不一样,他虽然孤傲,但对手下人很好,他几乎能叫出他下面所有士兵的名字,这真让人难以置信。而我则连
自己马夫的名字都不清楚。所以每次征兵的时候,倘若一个新兵被划到了二哥队里,他脸上的表情兴奋的如同中了大
奖,相反分到我队里的,则垂头丧气的如同死了娘。

当然这不代表我的部队打起仗来就不行,虽说战乱时代当兵就是为了填饱肚子,但士兵的天职就是服从命令,我平时虽
然对他们不是太友善,但打仗的时候我身先士卒,这让他们敬畏,因此我的部队的军纪和士气要比二哥的还要好一些,
这也从另一个方面稍稍补救了我有勇无谋的缺点。

军师说,行为决定习惯,习惯决定性格,性格决定命运。我的习惯已经养成了,因此我的命运也已经注定了。改变是一
件很难的事情,尤其是我这么大岁数的人了,当然最重要的是我从来就没想过要去改变。

天天在马背上驰骋,耳边是战鼓声、喊杀声和惨叫声,眼前是成堆的尸体和鲜红的血河,这种生活足以让任何一个正常
人变得不正常。死在我矛下的有名的无名的都数不过来了,我不是一个宿命的人,但我知道,有一天我也会死在别人的
手上,这很公平,也符合我的性格。

不过在这之前我得好好活着,其他的都去他妈的。

\section{}

今天早上大哥兴高采烈的样子,象吃了喜鹊屎似的。我们都有点纳闷,但却都憋着没问。后来还是魏延忍不住了,他凑
过去低三下四地问道:主公,何事如此开心啊?大哥先仰天哈哈了两声,然后眉飞色舞地说:昨夜我做了个梦,梦到曹操
死了。你们猜他是怎么死的?

我们几个面面相觑,还没等说话呢,大哥接着说:他是吃鸡蛋噎死的,哈哈哈\dldots 。我们都愣了,半天马超才发出雷鸣般
的笑声,恰好是大哥笑声刚结束的时候,因此显得特别突兀,而且过于生硬,笑了两声他也觉得无趣,于是噶然而止。大
哥见我们的样子有些奇怪,就转头问我:怎么?三弟,你不觉得好笑吗?我吱呜了半天说:这个,曹操被鸡蛋噎死,这个,也
太荒谬了吧?这时就听门外一个声音传来:哈哈,如此说来当给鸡蛋记一大功,封个讨贼将军什么才好。于是大家一起
哄堂大笑,原来是军师来了。

要不说这有学问的人说话办事就是不一样嘛,象军师这种人在任何场合都如鱼得水,天大的事到他那里都会应刃而解。
所有接触过军师的人,无论是朋友还是敌人,都会肃然起敬,当然除了一个人,军师的夫人。

军师的夫人似乎生下来就是跟军师作对的,所有人都不明白军师怎么会娶了她。军师的夫人小名叫阿丑,长的不能叫
丑了,简直就是惨不忍睹,一头黄发跟枯草似的,柿饼子脸,绿豆眼,鼻孔朝天,血盆大口,五短身材。平日里见了人总是
昂着头,用俩鼻孔看人。军师见了她如同耗子见了猫,让他往东他不敢往西,让他打狗他不敢撵鸡。平日里万人景仰的
军师居然怕老婆怕成这样,着实是一件非常有趣的事情。

我曾经私底下问过子龙,子龙说,动物里有种现象叫做天敌,两种动物没有任何利害关系,但生下来它们就是死对头,见
面就掐,没有任何原因,比如猫和狗。我想了半天,哦,这样看起来军师和他夫人就是一对天敌了?子龙笑道:也不能完
全这样说,人的感情很复杂的,不能跟动物相提并论。不过话说回来,军师上知天文下晓地理,神一般的人物,放眼天下
无人能出其右,也确实应该有个人管他的。这叫卤水点豆腐,一物降一物。

子龙这么一解释,我虽然似懂非懂,但也明白了个大概。后来再看到军师那狼狈的样子觉得其实很有意思,那样子远比
他坐在中军帐上镇定自若的样子可爱得多。

子龙中午郁郁不乐地来找我,进了门也不说话,端着茶杯一杯接一杯地喝。

这世界上倘若能有事情让子龙犯愁的话,那么这件事一定是和女人有关。于是我便问他:怎么了?被女朋友甩了?子龙
仰天长叹一声道:甩了还好了呢,这次是甩不掉了。

子龙最近找的这个女人姓范,名字我不清楚,只知道子龙平日里叫她二姐。这个女人姿色平平,却非常的有心计,否则
的话又怎么能让子龙在我这里长吁短叹呢。看起来这次子龙是遇到克星了。

我笑着对子龙说:你年纪也不小了,差不多也该找一个合适的人管着你了。要不你这次就从了她吧。子龙瞪大眼睛看
着我说:三哥,你什么意思?你这不明摆着坑兄弟吗?你的意思是让我跟她结婚?怎么可能!

可是是个人都要结婚的呀。我觉得子龙的反应有点不对头。

子龙放下茶杯,面色沉重地对我说:三哥,今儿我得给你上一课。就结婚这件事我给你举个例子,就比方说你饿了好几
天,然后有人把你领到一个饭店,最早给你上的是馒头,你吃不吃?

我毫不犹豫地说:吃呀,饿成那样了不吃还等什么。

子龙接着说:好,你咬了一口以后发现又上了包子,相对于馒头来说你更喜欢吃包子,但馒头你已经咬了,所以你必须要
把它吃完。于是你努力地把馒头吃完,开始吃包子,可等你咬了包子以后,又上来了烧鸡,然后后面还有燕窝啊鱼翅啊
等等,可惜你吃完了包子已经饱了,你只能眼睁睁地看着那些好东西被别人一一吃掉,你说你后悔不?

我琢磨了半天,点头道:后悔,但也没法子,能吃饱已经不错了啊。

子龙哈哈大笑着说:三哥,这就是咱俩的不同之处啊。这世界上有两种人,一种人安于现状小富既安。另一种人则永不
知足不断进取。你属于前者,而我则属于后者。不过两种人各有各的长处,前者不论生活环境的好坏都活得很开心,后
者则活的累一些,但生活的更有质量。

听到这里我有些纳闷了:子龙,可我不明白你举的这个例子跟结婚有什么关系啊?

子龙看着我象看见了一头怪兽:三哥,你还没明白呀?这个馒头啊包子啊燕窝啊鱼翅啊都指的是女人,你结婚了就表示
你吃了它了,就无法再吃别的了,懂了吗?

我点了点头:哦,现在有点明白了。可是你自己知道你最喜欢吃什么吗?你知道你喜欢吃的那东西一定能上来吗?你这
样一直等下去会不会饿死啊?

这下轮到子龙沉默了,他坐在那里托着腮想了半天,嘴里嘀咕着:有道理,问的好,问的好。一直到黄昏他还在那里叨叨
唠唠的象发了癔症。

过了几天,子龙又来找我,这次他眉飞色舞精神抖擞的,进门就喊:三哥,我想通了,不管怎么说我都会一直等下去,直到
我喜欢的那种食物出现!

我愣了半天,问:那范二姐呢?

子龙飞快地回答:甩了。

我又问:怎么甩的?

子龙道:我把我给你举的那个例子讲给她听,她问我她是馒头还是鱼翅,我说大概接近于熊掌那个级别,于是她很满意
地走了。

\section{}

人的一生中总有感到无奈感到恐慌的时候,即使是象我这样粗枝大叶的人。

现在我就陷入了这种境界,最近我得了一种病,一种很奇怪的病。开始的时候是腰部的皮肤有点麻木,我根本没当回
事,后来慢慢地生出一些小红疙瘩,一簇一簇的,从腰两边慢慢向中间扩散,奇痛,如同好多针尖刺到肉中的感觉。晚上
睡觉还好一些,白天顶盔贯甲,然后战马再那么一颠一颠的,简直就是在受刑。

到后来那一圈红疙瘩越来越多,我实在受不了了,于是叫子龙来看是怎么回事,子龙当年学过两年的兽医,现在也算半
个军医。子龙见了以后大惊失色,连声叫道:三哥,大事不好啊大事不好!

我当时双手提着裤子转着圈给他看本来就觉得很难堪,现在又听他大呼小叫的,心下有些慌乱,忙问:怎么回事?好治吗?

子龙的样子如同看到了外星人,惊讶中还有些猎奇的意思,连声说道:三哥,你真了不起,这种病很少有人能得,得也没
你这么严重的。

我气不打一处来,怒道:快说是什么病!

子龙围着我又转了一圈,然后慢悠悠地说:此病唤作腰带疮,长在腰间如同一条腰带,倘若首尾相连的话,也就是得病之
人寿尽之日。你看,你这个已经快连起来了,估计用不了几天了,三哥,准备后事吧。

我低头看了看,的确快连成一个圈了,不过看子龙那气定神闲的样子,又觉得有些半信半疑。我不怕死,但死我也要死
在战场上啊,这样挂了算是那门子事啊。

正在这时,黄忠来了,一进门看到我的样子吓了他一跳,待他看到那些疙瘩时面色凝重起来,说道:翼德啊,这真是腰带
疮啊,千万别让它们连起来呀!

黄忠这么一说,我顿时如同身陷冰窟,心想这回可错不了了,唉,可怜我那两个没娘的孩子啊。

正在我唉声叹气的时候,子龙却笑得象朵菊花似的凑过来说:三哥,你命大啊,幸亏遇到我了,你跟我来。说罢转身便走。

我半信半疑地跟着子龙出去,转了几个弯儿到了子龙的住处,一进门看见一个花白胡子的老头,弯腰驼背的,长相挺猥
琐。只见子龙对那人恭恭敬敬地施了一个礼,对我说:三哥,快来见过华佗先生。

原来此人就是神医华佗?我不由得大惊失色。早就听说过华佗这个名字,据说此人的医术已经达到了起死回生神乎其
技的地步,却没想到他竟然是子龙的朋友。我连忙过去施礼,华佗却也不回礼,面无表情地摆摆手。

子龙将我的腰带解开,华佗只看了一眼,回手拿出一个小药箱,从里面的瓶瓶罐罐中挑了两包药末,递给我说:黄色的外
敷,白色的内服,一日三次。我大喜过望,连忙拜谢。却见华佗把药箱背在背上,朝子龙拱拱手说了一句:吾去也。转身
便走了,子龙却也不留。

不大工夫门帘一掀他又回来了,似笑非笑地对我说道:张将军,切记一个月内不许饮酒,否则药效尽失。转身又走了。

从子龙那里回来后我就开始服药,不愧是神医,当天疙瘩便消了很多,并且不疼了。可还是有个问题,那就是不让我饮
酒,这简直比杀了我还难受,头两天还能熬过去,到第三天实在忍不住,死就死了,端着大碗我又喝了个酩酊大醉。早晨
起来的时候发现疙瘩全没了,周身一点异状也没有,我的病居然完全好了!

我拍着脑袋也没想明白是怎么回事,于是去找子龙。子龙听罢哈哈大笑,说道:其实你的病跟喝酒一点关系也没有,他
是在整蛊你呀。这老家伙是越老越顽皮了。

我听完以后是哭笑不得,原来这世上还真有这种人,有一技之长却玩世不恭,读书人称之为‘狂傲不羁,恃才傲物’。

后来听说华佗要去给曹操看病,我隐隐有些担心,果然不出所料,华佗去了以后胡言乱语地吓唬曹操,让曹操一怒之下
给杀了。一代神医连个徒弟都没留下,可惜啊!

开玩笑要分场合,更要分人。有些人可以任意开玩笑,有些人的玩笑却是万万开不得的。

\section{}

有个人我一直没提,就是同样身为五虎将的黄忠黄老头,没提他不是因为没什么可提的,而是这老家伙值得说的事太多
了,好比猴子吃螃蟹,不知从哪儿下口。

黄忠是跟魏延一起来的,别看年纪大了,却是一身的好武艺,一口大刀片子耍起来是虎虎生风,这还不算,最要命的是他
还射得一手好箭,百步穿杨,例无虚发。

黄忠的饭量惊人,我算是能吃的了,老家伙能吃我一个半。早年家里穷,全家半年的口粮还不够他一个星期吃的。没办
法,只好把他放出去自谋食物,四周也没别的,山上的动物不少,不过这也练就了他的神箭。自从他投奔大哥以后,很多
军士都抱怨自己吃不饱。每到宴席的时候,就看他先把眼前的东西风卷残云一扫而光,然后开始咂着嘴寻觅临座的。
后来大家不再叫他黄忠,而叫他蝗虫。

不过黄忠还有一手绝技,那就是烤野味。每次如果在树林里安营扎寨的话,那我们几个可都有口福了。他拎着弓出去
转一圈后,腰里挂的肩上扛的,大的如狍子、鹿之类,小的如野兔、山鸡之类,也有叫不上名的,品种繁多,应有尽有。
生一堆火,这时候我的丈八蛇矛便派上用场了。不知道为什么,用我的矛烤出来的东西跟用别人的枪烤出来的味道相
差很远,连黄忠也觉得稀奇。他说他有机会一定找人照着我的矛再打一把,专门用来做烧烤用。不过军师告诉我不要
答应他,我问为什么,军师说这样一来他每次烤东西你都有机会吃到了。我忍不住哈哈大笑,军师真是个聪明人。

黄忠虽然年纪比我们都大,但生性好胜,不管什么事情都不服输,最怕别人说他老了不中用了。有一次我去他房间找
他,发现他和魏延两人面对面地坐着不动,跟他们说话也没人搭理我,甚至连眼睛都不眨一下。我觉得很有趣,在屋里
转了一圈后发现桌子上有盘烤羊腿,于是拿起来三下五除二地吃掉了,吃完以后觉得口渴就找水喝,魏延猛地站起身来
说:妈的,不玩了不玩了,羊腿都被吃了!然后就听黄忠拍手哈哈大笑道:你输了你输了!原来两人在打赌看谁先说话,赌
注就是那条烤羊腿。

有时候看着这老家伙跟一群年轻人在一起嘻嘻哈哈的觉得有些纳闷,按说我比他小好多岁,可是我自己都觉得自己老
了很多。年轻时也喜欢事事争强,觉得做什么都有兴趣,很多事情想也不想就去做了。可随着年龄的增长,渐渐的变得
沉默,变得优柔寡断。年轻时特别喜欢笑,随便听一个笑话便能开怀大笑好长时间,可现在除非是见别人骑马摔断腿才
能笑出声来。大哥说:三弟,你成熟了很多。我不知道这成熟是不是好事,但我知道我丧失了很多做人的乐趣。

黄忠的出现让我明白了一个道理:一个人是否年轻不在于他的真实年龄有多大,而在于他的心态。一个八十岁的人如
果保持二十岁的心态,那么他便是一个二十岁的年轻人。我们生活在这世界上,最重要的是要活得开心,而是否开心,
与贫富无关,与贵贱无关,也与年龄无关。

现在这老家伙正坐在那里无所事事,我决定过去跟他打赌,看谁在一柱香内打死的苍蝇多,谁输了被罚用羽毛挠脚心。

\section{}

今天兵士捉了一个人带到我面前,说此人在帐外鬼鬼祟祟地窥视了好半天,并且身上还藏有利器。

这个人是个年轻人,看样子也就十六七岁,蓬头垢面的,穿的也很破,看起来是个流浪儿。我问他:小伙子,你到这里来
做什么呀?

他抬起头直盯着我,他的眼睛很黑,一刹那间我感到了一种前所未有的寒冷,我从没见过一个孩子有如此的眼神,他一
字一顿地说道:环—眼—贼,我—是—来—杀—你—的!

我气极反笑:哦?为什么要杀我呀?

那孩子双眼喷射着怒火说道:你杀了我的父亲!我一定要亲手杀了你!

我一愣,问道:我杀的人无记其数,你父亲是谁?

那孩子的脸扭曲着,尽力把身子往前探,恨恨地说道:我姓纪,我父亲叫纪灵,我叫纪同。你记住了,今天落在你手里,任
杀任剐,但我就是做鬼也饶不了你!

听他这么一说,我想起来了,当日在徐州拦截袁术的时候,我是杀了一个叫纪灵的,好象是个先锋,印象不是很深,似乎
也没什么本领。但眼前这个口口声声要为父报仇的孩子却有点意思,于是我起身笑道:我一生杀人无数,你却是第一个
找我来报仇的。也罢,我便成全你,今日我不杀你,你回去练习本领吧,等你长大了再来找我,我项上的人头就在这里,
等你来拿。

说罢我挥手让军士把他松绑,他愣愣地站在那里,咬牙切齿地说:好,我这就走,我要天天拜佛保佑你活着,总有一天我
要亲手杀了你!说完转身头也不回地走了。

这孩子走了以后我呆呆地坐在那里想了很久,人类的感情分好多种,仇恨是其中的一种,也是最奇怪的一种。比如这个
叫纪同的孩子,他可能在很小的时候就埋下了仇恨的种子,这么多年来他一直是为了仇恨而活着,他无时无刻地想着要
复仇,而可悲的是他想杀的人——我——居然毫不知情。

我知道他终有一天还会来找我的,只是不知道我能不能等到那一天。不过转念想一下,假如真有那么一天他来了,并且
杀了我,那么他会开心吗?他活下去的理由就是为了杀我,而一旦实现了,他活着还有什么意思?想到这里,我希望他永
远不要来,无论是爱还是仇恨,一个人有某种信念支撑着总比什么也没有要好,相对于大多数人来说,他们活得更单纯,走
的路也更直一些。

我把这些想法告诉军师,军师沉吟了片刻说:仇恨是平息不了仇恨的,错误也永远纠正不了错误,只可惜我们永远也不
能从其中解脱出来。

\section{}

今天我做了一个梦,梦到一片一望无际的草原,蓝蓝的天,白白的云,我在草地上跑来跑去,忽然看见大哥二哥他们朝我
走来,我大声地叫他们,他们却自顾自地走了,我一回头,猛地看到自己长着一条尾巴,再仔细一看,自己竟然变成了一
匹马,我大惊失色,拼命的大叫,发出来的却是嘶鸣声,一急之下于是醒了。

正巧军师来了,我就把梦说给军师听,军师饶有兴趣地听着,然后说道:翼德啊,你这个梦在很多年前一个叫庄周的人也
梦到过,不过他梦到的是自己变成了一只蝴蝶,在花丛中飞来飞去,醒了以后,庄周提出了一个问题:究竟是刚才庄周梦
见自己变成了蝴蝶呢,还是现在蝴蝶梦见自己变成了庄周呢?用在你身上的话就是:究竟是刚才张飞梦见自己变成了马
呢,还是现在马梦见自己变成了张飞呢?

军师这一番话把我说得云里雾里的,什么呀?我只是做了个梦而已,怎么可能是马做梦变成张飞呢?这个叫庄周的人是
不是脑子有毛病啊?

军师笑了:庄周可是个了不起的人物啊,别的不说,就庄周梦蝶这个典故就够后人分析几千年的了。你自个再好好琢磨
琢磨吧。

军师走了以后我越想越糊涂,你别说,这个姓庄的有点意思,我做梦梦到自己是马,说不定我本来就是一匹马而做了一
个变成张飞的梦呢。照这样想下去,我现在所做的一切都不过是一场梦而已,人生就是一场梦?

整个儿一个下午我都在思考这个问题,我在院子里踱来踱去,嘴里念叨着:我是张飞还是马?忽然墙外传来一声厉
喝:咄!你是张飞时自是张飞,是马时自是马,张飞既是马,马既是张飞,多想无益!

我如梦方醒,慌忙出门,转了一圈却没发现一个人,于是对空拜了一拜,说:燕人愚钝,谢高人指点!

饭也没吃径直去找军师,军师正在给夫人梳头,见我进门对我摆了摆手,我见夫人双目微闭,一脸陶醉的样子,于是屏住
呼吸立在边上。

好容易等军师把夫人安顿躺下,把我拉到院子里问:翼德,找我有事?我把下午那人的话对军师讲了一遍。军师听罢长
叹一声:果然是高人啊!翼德,这种问题纯属兜圈子的问题,你既然已经解脱出来就不要再陷进去了。

从军师家里出来后我很得意,因为临走时我问了他一句:你说刚才是你做梦梦到给夫人梳头呢还是夫人做梦梦到你在
给她梳头?我看到军师的脸色变了,变得很难看。

往回走的时候天已经很黑了,隐隐地有奇异的歌声传来:一年老一年,一日没一日,一秋又一秋,一辈催一辈,一聚一离
别,一喜一伤悲。一榻一身卧,一生一梦里\dldots

不由得痴了。

\section{}

军师派我到剑阁出差,其实也没什么大不了的事,但还必须要去个有头有脸的人物,而当时二哥子龙他们都有别的任
务,于是便让我去。

本来有一个向导兼随从,但出发前的晚上我恰好喝高了,而他又恰好在我身边唠唠叨叨的,于是第二天早上他满头绷带
的去不成了,我只得单枪匹马的出发了。

走了不远我就发现问题不对了,因为我不认识路,只知道剑阁在成都的东北方向,但具体走哪条路却一无所知。回去再
让军师给我找个向导?不行,他又该拉长脸问:不是已经给你派了一个吗?我可不想再跟他辩论喝酒的好处和坏处了。
管他呢,鼻子下不是有张嘴嘛,我一路走一路问,不信还走不到剑阁!

顺着官道走了大概有半天的路,前面是小路了,而且是好多条,我便随便拣了一条走下去,没多远看到路边有个小茅草
屋,心中大喜,便上前敲门。开门的是一个矮个儿,一脸麻子,见到我满脸堆笑地问:将军有何事?我开门见山地问他:知
道往剑阁怎么走吗?

麻脸矮个儿愣了一下:这个嘛\dldots 我是不太清楚,不过我家那头驴是从剑阁贩运过来的,估计它认识路。

我心想,这叫什么事儿啊,人还不如一头驴知道得多。也罢,你的驴多少钱?我买下来让它带路。

麻脸一脸的苦大愁深:将军,这驴我多少钱也不能卖,不是我为难你,实在是有苦衷的,我家里就指望这头驴拉磨做点豆
腐。这方圆几十里又没有牲口可以买,你要买走了我一家老小可就没法活了。

我气不打一处来,奶奶个熊,不就一头驴嘛,老子又不是不给钱。可人家不卖又不能真的动粗,想了半天,嘿,有了。我
对那麻脸说:小子,这样吧,我用我的马换你的驴,这样行了吧?

麻脸大喜:如此甚好如此甚好。

出了茅屋,我便上了驴背,你别说这驴虽然身材矮小,但走起来还挺稳当,最重要的是它认识路,不用我指挥,自个儿顺
着小道屁颠屁颠地走了下去。

眼瞅着天色变暗,我骑着小驴上了一个小山头,四下看了一下,发现山脚左边有一处炊烟,于是便一拍驴屁股朝那边走
去。

走近了看见一户人家,开门的是一老头儿,我掏出一块碎银子说:我今晚在这儿打个尖,你去准备点饭菜。老头儿接过
银子连声说:可以可以。老头儿手脚挺麻利,一会儿功夫就把饭菜弄好了。

吃饭的时候老头儿问我:将军这是往哪儿去啊?我说我去剑阁。老头儿奇怪地看着我说:去剑阁你怎么走到这里来了?
将军你走错路了。我说:不会吧?我可是跟着驴走的。老头儿一个劲地摆手说:错了,你这样走就是走一个月也到不了。
你相信我,我年轻时没少去剑阁,前几年身子骨好的时候还去过一次呢。

我一听大怒,这死驴子,居然领我走错路,随手抄起跟棍子我就要出去打它,老头儿连忙拦住我说:你跟一个畜生计较什
么呀。你别急,我倒有个法子能让你去剑阁。我大喜:快说。老头儿对着里屋喝了一声``阿黄'',一条大狗从里面蹿了
出来,老头儿摸着狗头对我说:这条狗跟我相依为命十几年了,它肯定能领你去剑阁,不过条件是你得把驴留下,因为我
有时要往山那边运点木炭什么的,以前都是阿黄帮我。

我想也没想就答应了,驴子既然不认识路要它也没用。

一夜无话。第二天一早我便起身赶路,老头临走前搂着阿黄耳语了几句,那狗竟然跟听懂似的不住地点头。我把干粮
行李什么的挂在矛上,扛着矛跟着狗一路走下去。

阿黄看起来还真是认识路,一溜小跑几乎没有停顿,幸亏我体力好,否则还真跟不上它。快到中午的时候,来到了一个
小村庄,阿黄走到一家门口,竟推开虚掩的门径直进去了,我觉得有些奇怪,就跟了进去,进门一看,一个妇人立在那里,阿
黄竟然围着蹭来蹭去的显得很亲热。

妇人见到我居然也不奇怪,低头对那狗说:阿黄,这个人是你领来的吗?我更加纳闷,就问道:你认识这条狗?那妇人说:
是呀,这是我公公的狗啊,它怎么会跟你在一起呢?我恍然大悟:哦,原来这是那老头儿的儿媳妇。于是便把经过一五一
十地跟她说了,她听罢笑道:原来如此,此处离剑阁不远了,将军赶紧赶路吧。

可阿黄居然不肯走,我怎么撵它都没用,总是围着那妇人转圈,这下连那妇人也为难了,她说:平日里都是公公调教的,
我也没办法指使它。

正在我束手无策的时候那妇人又开口了:将军莫慌,奴家保证能让将军顺利到达剑阁。说罢进了柴房,不大功夫抱出一
只鸭子来,对我说:将军要是信得过奴家,就让这鸭子给你带路好了,慢是慢一些,但跟它走绝对没错。鸭子居然也能带
路?我长这么大没经历过这么荒谬的事,但眼下实在没招,心一横,也罢,总比我一个人摸黑走强。于是便赶着鸭子上路
了。

那鸭子摇摇摆摆地前面走着,我满脸无奈地在后面跟着,越走越觉得窝囊,正懊恼呢,忽然从旁边树林里窜出一只红毛
狐狸,叼起鸭子就跑,我愣了一下,大喝一声拎着矛就追。

那狐狸本来行动快捷,但叼着鸭子就跑不了那么快,时不时地还得放下鸭子歇会儿,一直跟我保持着一段距离。也不知
追出去多远,前面有个小树林,狐狸噌的钻进去了,我跟进去找了半天也没找到,别提多上火了,鸭子没了我怎么去剑阁
啊?

从树林里出来,我眼前一亮,前面居然是一条官道,路边还有个界石,上面写着两个大字:剑阁。当时我那个心花怒放
啊,这真叫踏破铁鞋无觅处得来全不费功夫啊。

在剑阁办完事,他们派人把我送回来。回来后我跟军师他们说起这一路上的经历,把他们笑得前仰后合,魏延捂着肚子
说:马换驴,驴换狗,狗换鸭子,你个笨蛋被人耍了还不知道。我摸着脑袋想了半天说:不管怎么样,我最后还是到了剑
阁啊。

军师晃着扇子说了一句:只要目的达到,手段就是正确的。

\section{}

二哥最近在看《战国策》,而且是古装本,边上还要有个士兵替他捧着那厚厚的一摞竹简。二哥看得津津有味,不时的
击节赞叹。我觉得纳闷,就过去问他有什么好看的。

二哥今天看起来心情不错,就放下书给我讲荆柯刺秦王的故事。这个故事我以前大略也听过,但没二哥讲得这么生动
详细,我也听得如痴如醉。当讲到风萧萧易水寒时我血脉赍张,当讲到荆柯掷剑不中时我们俩一起拍着大腿惋惜。

听完了以后我隐隐觉得某处不妥,想了半天终于想到了,于是便问二哥:那秦武阳不是一个勇士吗?十二岁杀人于市,世
人莫敢直视,为何却在殿前尿了裤子?

二哥摸着那一把视为珍宝的胡子,良久也没说出个所以然来。恰好魏延来了,于是我便问魏延,魏延想了半天憋出一
句:他是不是尿急啊?

我们三人一起去问子龙,子龙笑着说:三哥这问题问得好啊,一般人还真不会往那方面去想。要说这秦武阳也不应该是
怕死啊,既然决定去了,自然是抱着必死的心态去的,无论成败都不免一死。但为何不学荆柯大义凛然留个千古侠名
呢?实在是越想越想不通啊。

最后我们还是去了军师那里,军师沉吟了片刻,说了一句:武阳并非畏死,而是畏势。

这话一出口我们几个人都面面相觑,何为畏势?军师接着说下去:秦武阳只是一个小混混,浪迹于下流社会里,以勇力欺
人,在那个阶层里他是天不怕地不怕的勇士。但到了秦王殿上,面对文武百官以及君临天下的始皇,在一种强大的
``势''的压迫下,他的精神垮掉了。

我猛然想起了曹孟德,曹操当年曾怀宝刀去刺杀董卓,却迟迟未敢下手,终于还是跑掉了,是不是也是畏势呢?

军师长笑一声说道:曹孟德当年若真是下了手对咱来说却是好事,不过对他自己来说,却也只能落得个乱刃分尸,何来
如今的挟天子以令诸侯的威风啊。

停了片刻,军师又说:当年孟子去见齐宣王,宣王说:``寡人有疾,寡人好勇'',孟子说:``王请无好小勇。夫抚剑疾视
曰,‘彼恶敢当我哉’!此匹夫之勇,敌一人者也。王请大之!'',其实说起来,无论是荆柯刺秦王还是曹操刺董卓,都只
是匹夫之勇而已。

听完军师的话后我们都默然无语,晚上喝了点酒后我忍不住想:二哥、我、子龙、魏延,我们几个整日驰骋于疆场,又
何尝不是逞匹夫之勇呢?都说命运掌握在自己手中,但天下众生又有几人不是被人摆布的棋子呢?

\section{}

魏延在跟他的手下们喝酒聊天,我也过去凑热闹。酒过三旬,魏延出了个题目,答对的奖励一个鸡翅膀。题目是这样
的:大象有几条腿?

由于在座的士兵大都是北方人,而且既然出来当兵也肯定不是富裕的主儿,因此他们的回答都很可笑,竟然异口同声地
回答是两条腿,有一个士兵居然回答是三条腿,我差点没笑背过气去。可没想到魏延居然把鸡翅膀给了那个回答三条
腿的士兵,在我疑惑的时候,魏延解释到:他的答案虽然也不正确,但却最接近于正确答案。

原来有些时候你的回答未必正确,只要比你的敌人更正确一点也就相当于正确,这让我忍不住想起了周公瑾。周瑜妙
计火烧赤壁,让曹操八十三万人马片甲不留,天下闻名。后来军师曾专门点评过赤壁之战,他说周瑜此战一共用了三条
计策,都不是上策,但却都成功了。

首先是蒋干盗书这一计,太多的破绽。试想周瑜如此谨慎小心之人焉能大意到将重要文件放在明处?江东大营把守森
严蒋干盗书后又如何能来去自如?最不可思议的是曹操见书后居然连审问都没审问就将蔡张给斩了,按说蔡张二人乃
是水军都督,掌握兵权之人,上来就杀也不符合曹操的性格。但这么多破绽居然也成功了,并非是周瑜计策高明,只能
说蒋干和曹操配合得好,一个太愚蠢,一个太糊涂。

第二计是黄盖的苦肉计,两军对垒带兵反戈者古往今来有很多,但大都是或斩主将或献城池或作内应,如黄盖这种在月
黑风高之时带兵投诚者摆明了是突袭嘛。

第三计最荒谬,是庞士元的连环计。把船首尾相连,拿大铁环拴上,稳是稳了,但那还叫做船吗?移动都不方便,如何打
仗?曹操手下那么多名人智士,难道只徐庶一人看出来了吗?非也,曹操当时正是踌躇满志,有点得意忘形的意思,从错
斩蔡张也看得出来。聪明人在这个时候都顺着他的意思来,象刘馥那种傻鸟也只能落得个被一槊刺死的结局。曹操本
非心躁气浮之人,但或许是因为双方实力相差过于悬殊,又或许是冥冥中自有天意?

军师到最后说了一句,周瑜这小子运气真好。子龙后来偷偷跟我说,这话里透着好浓的酸味啊。什么酸的甜的我却品
不出来。

周瑜死了以后军师很开心,破例喝了点酒,给我们讲了一个笑话,跟开头魏延的那个有异曲同工之妙。说有两个人去打
猎,不巧遇到一只老虎,其中一人便将身上所有的东西都扔掉,把鞋子也脱掉了。另一人说:没用的兄弟,你就是光着膀
子也跑不过老虎啊。前一人答道:我没想跑过老虎,只要跑得过你就可以了。

跑得过你就不会死,三条腿就有鸡翅膀吃,在特定的环境中,你不用做的很完美,只要比你周围的人好一点点就足够了。

\section{}

身为一员武将,我可以算得上是身经百战了。要说这两军对垒,甭管谁的兵多谁的兵少,只说单挑的话我跨下马掌中枪
还没服过谁,只除了一个人——吕布。

吕布外表上看起来跟子龙有些相像,略微比子龙高大一些,但绝不是我和许楮这种凶神恶煞型的。乍一看,相貌堂堂,
仔细一瞅眉目之间的那股杀气却不由得让人吸一口凉气。

吕布自幼父母双亡,在很小的时候被一位世外高人收养,并传授了他一身武艺,再加上他天赋异秉,出道以后很快便名
震天下。虎牢关那次,我们哥仨没占到便宜,回来以后二哥曾经紧锁眉头说了一句:今日始知人外有人天外有天啊。其
实这又何尝不是我想说的呢。

但我佩服吕布的只是他的武艺,说到做人,他却是我这辈子最不服气的一个人。

在这乱世之中,只要你有能力想出人头地很容易。吕布最早找的一棵大树是丁原,做贴身保镖。丁原当时是荆州刺史,对
吕布非常器重,认他为义子,二人以父子相称。谁知好景不长,董卓当权的时候,与丁原不合,于是用一匹赤兔马收买了
吕布,后来吕布一刀砍下丁原的脑袋,反身投靠了董卓这棵更大的大树,好笑的是二人仍以父子相称。

董卓为人骄横傲慢,众诸侯暗藏反心,形式岌岌可危的时候,吕布又一次挺身而出,跟杀丁原一样,轻车熟路,依旧是一
刀拿下。不同的是这次是为了一个女人而不是一匹马。

再后来,吕布四处乱窜,之后又投靠了袁绍,最终在白门楼被曹操给杀了。杀之前曹操曾经问过大哥的意见,大哥让他
想想丁原和董卓,于是曹操便不再犹豫。

军师经常说,尊重天地君亲师是人和禽兽最基本的区别。我是个粗人,不懂得那么多礼节,但天地君倒也罢了,尊重亲
和师却连我都丝毫不敢马虎。而吕布连杀两位义父,当真连禽兽也不如。

有时候我真的无法理解吕布的做法,因为我无法想象一个人竟然可以坏到如此地步。子龙对我说,其实人生下来跟其
它动物一样,都是自私残暴的,这跟水往低处流是一个道理。但人之所以为人,主要是后天的教育和环境的影响,一般
人做任何一件事潜意识里都有一个对和错的概念,人一般都朝着对的方向努力,但有一点,就是他认为是对的事情未必
是世人所接受的。因此,这个世界上可以有很多奇怪的人,做一些奇怪的事。

子龙的话说得我呆呆的,我忍不住想我以前做过的事,似乎每一件我都认为是正确的,但事实上呢?有些问题想着想着
会让人脊背发凉,还是去喝酒好了。

\section{}

今天赌钱的时候,边上的两个士兵在讨论女人。男人在没事的时候总喜欢讨论女人,如同商人没事喜欢数钱一样。他
俩说着说着就提到了貂禅,眼里放着光,嘴里嗬嗬地笑,猛然间让我也想起了这个女人。

貂禅不是她的名字,她以前叫什么没有人知道。貂禅只是一个称号,类似于巫师或者祭司。

一切都是一个偶然,从董卓踏进王府的那一刻开始,貂禅这个名字为世人所津津乐道。军师说过,凡事有因就有果,有
果就有因,一切的偶然都是必然,一切的必然也是偶然。

我见过这个女人,当年在白门楼的时候她坐在囚车上从我面前经过。那时候她已经名震天下,她的故事被演化成很多
版本,不同的版本有着不同的观点,有人说她是个烈女,有人说她是个荡妇,但只有一点是共同的:她是个美女,是个倾
国倾城的大美女。

这点我得承认,我不是个会欣赏女人的人,但当时她虽然衣衫不整头发凌乱,却依然掩盖不住她那绝世的容貌,体态婀
娜,肌肤雪白,真乃天生尤物。而给我印象最深的是她的眼神,清澈而平静,如一湾幽幽的潭水。跟身边其他女眷或慌
乱或悲切的表情相比,她平静得有些可怕。很多年后一个女人在登船离去时我看到了同样的眼神。现在我或许懂了,
但当时我却不明白。

白门楼上吕布向大哥求情,我清楚地看到二哥在拉大哥的衣角,忽然恍惚想起囚车经过的时候二哥的眼睛一眨也没眨
过,于是吕布死了,一切都是偶然中的必然。

二哥向曹操索要貂禅,这么多年来我第一次看到二哥主动向别人要过东西。而曹操似乎很痛快地答应了,大哥则意味
深长地叹了一口气。

自古英雄配美女,二哥与貂禅似乎是天设地造的一对,我想除了地下的董卓和吕布,没有人会反对这个说法。但那天二
哥是欣喜若狂地去迎接貂禅,却独自一人垂头丧气地回来了。没有人敢问到底发生了什么,但自此二哥郁郁不乐不近
女色,而貂禅则象是消失了一样,有人说她出家了,有人说她疯了,更有甚者说她死了。

我曾经借着酒劲问过二哥,为什么那天没有把貂禅接回来?二哥愣了一下,好一会他反问我一句:三弟,你说我跟董卓和
吕布做何比较?我也愣了一下,说:那两个宵小之辈如何跟二哥你相提并论呢?二哥却似自言自语地说道:在她眼中我却
跟他们没什么区别。良久,他又说了一句:自古红颜多祸水,知己有几人?

后来子龙曾经跟我讨论过这个话题,他那时正在和一个小女孩热恋之中,心情好得很,他笑着对我说:三哥,你养过猫没
有?我摇摇头,他接着说:我小的时候家里养过一只猫,在开始的时候我对它特别好,每次都是我喂它吃东西,它也特别
依赖我,睡觉的时候总偎依在我身边。但后来我有事出远门,回来的时候它却象是不认识我一样,睡觉时也去找最近喂
养它的老妈子了。到后来,我们家几乎所有人都喂过它,开始的时候它跟谁都很亲热的样子,最后它则对谁都爱搭不理。

我隐约听人说过这句话,不是所有的猫都象女人,但所有的女人都象猫。或许子龙的说法是对的,但我至今还记得貂禅
在囚车上的眼神,联想到离我而去的那个女人的眼神,我似乎明白了一些东西,但却又好像什么也不明白,而对于一些
永远无法理解的东西最好的方法是忘记。

因此我准备把貂禅连同那个女人一起从我的记忆中删掉。

\section{}

早上起来照镜子时猛的发现双鬓已有了些许白发,于是知道自己确实是老了。

军师说人变老的标志之一是开始唠唠叨叨,之二是开始怀旧。我虽然还没怎么唠唠叨叨,但有时无所事事的时候我却
忍不住回忆一些以前的事,往事无论是喜是悲,想到最后总有一丝淡淡的惆怅。

我承认我一直是个笨笨的人,很多读书人的道理我都想不明白,而且我也没打算去弄明白。

我的前半生是在昏昏噩噩中度过的,能记起的事少得可怜,但我那时却很快乐。我的父亲是个酒鬼,他有时喝多了会把
我抓过来饱揍一顿,在很多人眼中或许他不算是个好父亲,但在他死后的很多年里我竟然经常会怀念他的拳头。我的
母亲和其他所有母亲一样都是那么善良伟大,我现在经常会想起她,但却记不清她的模样。有时在路边偶然看到一个
老妇便会把她的面容安到母亲身上。他们说记不住母亲的长相是件很可耻的事情,或许他们说的对,但我想我的母亲
会原谅她的儿子,因为这世界上倘若只有一个人了解我的话,那就是她。

我说过我童年能记起的事很少,除了父亲的拳头之外,就是母亲的话了。母亲虽然没读过书,但她总能用一些浅显的话
让我明白很多道理。比如有次她买了十只蛋放在炕上孵小鸡,我非常兴奋,经常翻开棉被的一角偷偷的看,希望能看到
小鸡破壳而出的样子。我对母亲说,过几天我们就会有十只小鸡了。母亲却淡淡地回了我一句:在没有孵出之前,不要
计算小鸡的数量。

事实上最后我们一共只孵出了六只小鸡,于是母亲的这句话让我记了一辈子。在后来带兵打仗的时候,或者我们兵少
将寡占尽劣势,或者我们兵精将广处于绝对优势,但我都丝毫不敢气馁或者骄傲,因为我知道不是每只蛋在二十一天后
都会孵出小鸡来,有很多事情光看开头是猜不到结尾的。

母亲还有一句话让我记忆犹新,她说:拳头大不一定有理,但拳头小一定没理。我小时侯由于脑袋不灵活,经常被人取
笑,气极了我便冲过去狂打一顿,有时候是我打赢了,但更多时候是他们一拥而上把我打得鼻青脸肿,母亲对此一直视
而不见,在我被打得最惨的一次的时候她说了这句话。从那天起我象牛一样的锻炼身体,直到有一天我发现我的身体
也象牛一样的强壮,而那些以前欺负我的人却好象突然消失了,反而我身后经常跟着一群半大小子,整日里飞哥长飞哥
短的叫着,比叫他爹都亲。

后来我慢慢的长大,经历过很多事,接触过很多人,我越来越发现,其实有很多道理并非只有圣人才说得出来,每个人对
于生存都有他自己的哲理,只是他们或者不说,或者说了你也没在意而已。或者可以这样说,对于某个或某些个人来
说,其实每个人都是圣人。

\section{}

魏延新得了一匹马,样子很雄伟,他很得意的牵来向我显摆。我一直对马这种动物有好感,于是便借来溜溜。

这马的脚力的确可以,我骑得起劲,不知不觉已经出了成都城,沿着官道跑了一会我顺势插到了一条小路上,往前跑了
大概一柱香的时间,我勒住了缰绳,翻鞍下马,见那马呼吸均匀神态自若,不由得暗暗赞叹。

牵着马往前走了几步,忽见前面树林之间露出一个屋角,于是便朝那儿走了过去。走近时发现是一个小道观。

推门进去,真的是一间小道观,里面除了一张供桌之外几乎没有别的东西,甚至连个神像都没有,只一个牌位,上书``太
上老君混元上德皇帝''几个大字,牌位前有个小香炉,里面连点香灰都没有,更别说香了。墙角到处都挂满了蜘蛛网,
如果不是地上蹲着一个道士的话我还以为这是一座废弃了的道观呢。

说到这个道士,着实有点奇怪,我自进门来他始终背对着我竟然没有回头,我忍不住走过去看他到底在做什么。

走到他正面,发现他面前摆着一个小火炉,里面有几块红红的木炭,道士双手各持一串东西在火上面烤着,你猜他在烤
什么东西?反正当时是吓了我一跳,他居然在烤大蒜!我见过烤羊的,烤鸡翅膀的,烤馒头片的,却从未见过烤大蒜的,今
儿是开了眼界了。

眼瞅着两串大蒜已经变成金黄色,除了散发出一股浓重的蒜头味以外,还有一股奇异的香气让我的食指蠢蠢欲动。就
在这时,那道士突然抬头看了我一眼。

道士长得很普通,瘦,个儿不高,站起来不会超过五尺,稍微有些驼背,年龄应该在六十左右,长得其貌不扬,很多人喜欢
把这种人的相貌比喻成风干的核桃,而他看起来更象个被砸了一锤子,不,是砸了两锤子的核桃,因为他的两颊深深地
陷了进去。我注意到他的眼睛,很浑浊,几乎分不清黑眼球和白眼球,然而他看我那一眼却精光暴露,让我浑身一震。

道士的脚边有个罐子,里面有把小刷子,他拿起来往蒜上抹了点什么东西,随后递了一串给我,我愣了一下伸手接过来,咬
了一口,清香扑鼻,真没想到大蒜居然也能烤出如此味道!我连声赞叹好吃好吃!

道士眯着眼吃另外一串,突然冒出一句话:将军,你的眼睛很大。

我又愣了一下,不明白他到底要说什么。

道士又接了一句:眼睛大只有一个好处。

我含着一口大蒜没咽下去,等着他继续往下说,他却住口不说了,自顾自的收拾火炉。

我又等了一会,见他依然没有说的意思,于是忍不住开口问道:道长,你刚才说的眼睛大只有一个好处,到底是什么好处?

道士似乎盼望这句话很久了,脸上露出得意的笑容,继续卖着关子:你真的想知道吗?

我的火噌的一下就上来了,过去一把薅住他的领子,喝道:你他妈的要说就快说,少在这儿给老子卖弄!

道士显然对我的举动没有预料,吓得脸色苍白语无伦次:我说我说,您先把我放下来\dldots 是这样的,眼睛大的好处呢,是
我经过几十年的观察得出来的,藏在我心中很久了,我从没对任何人说过,今天在这里与壮士幸会,乃是莫大的缘分,因
此我决定把这个秘密告诉你,可是\dldots 您不会告诉别人吧?

我真想朝他那张核桃脸上打两拳让他变成杏仁脸,瞪了他一眼骂道:少啰嗦,快点说!

道士环顾了一下左右,把嘴凑上来,在一股浓郁的蒜臭味中我听他一字一顿地说道:记住,眼睛大只有一个好处,那就
是\dldots 眼皮也大。

\section{}

二哥的右手曾经受过伤,在很长时间里他都用左手吃饭。后来他的右手好了,他笑称从此可以左右
开弓了。过了很久,我偶尔跟一个手下吹嘘过,说我二哥可以同时用左右手吃饭,手下将信将疑,我一来劲就拖着他去
找二哥,让二哥当场表演一个给他看,谁知二哥举着筷子面有难色,我问他怎么了,他说自从右手好了以后便不再用左
手吃饭,时间一长左手便又恢复到以前那样子了。

我大叫郁闷,却忍不住想起了小时侯的一件事。

我小时侯,邻居有个小孩子,他的左眼是瞎的。外表看起来跟右眼一模一样,但捂住右眼的话他便什么都看不到。后来
有一天,来了个游方道士,号称能医百病,于是邻居请他给小孩子看眼睛,他看过以后大惊失色,说你儿子的左眼一点毛
病也没有啊,怎么可能看不到呢?他这么一说,大家也很奇怪,他确确实实是看不到啊。道士想了好久也没想通,摇着头
走了。后来又来了一个和尚,说是道士介绍来的,专门来给小孩看眼睛,他也研究了半天,最后得出的结论跟道士一样,就
是说,小孩的左眼是一点毛病也没有。

但小孩的左眼看不到东西却也是事实,和尚挠着光头问了一句,这孩子小时侯眼睛没受过伤吧?他母亲猛然想了起来,
说在孩子出生的第二天,左眼的眼角曾经有点红肿,于是就擦了点药油,包了几天。和尚一拍大腿连声说是了是了,众
人忙问是怎么回事,和尚解释道:由于孩子当初正处于发育期,你把这只眼睛给包起来了,他就误认为这只眼睛是没用
的,而所有应该为这只眼睛服务的器官都退化消失了,因此他就瞎了,成了一个摆设。

说实话,以前我对和尚的话依然是半信半疑,眼睛就包上那么几天就瞎了?但今天听二哥这么一说,我突然有点相信了。
为了证实,我跑去找军师。

军师听完了我的故事,也连声说稀奇,之后他说,眼睛不用便会瞎掉我没听说过,不过却知道脑子不用会荒废掉。于是
军师便给我也讲了一个故事:

小时侯,我有幸曾经跟着水镜先生读书,一同读书的有好几个孩子,有庞统、徐庶等。其中有一个叫张正的孩子,聪慧
过人,水镜先生曾私下跟我说过,就资质而言,你和庞士元都是人中龙凤,但你二人加起来却也比不过张正。这句话让
我一度很沮丧,但过了没多久,张正家中突然出事,家道败落,他母亲带他去投靠他乡下的舅舅,从此杳无音讯。很多年
以后,徐庶曾经去探望过他,回来以后跟我们说,他已经是个非常普通的乡下人,问起他以前学过的东西,早就忘得烟消
云散,甚至连一本普通的诗集也读得期期艾艾,更别说什么治国带兵的雄韬大略了。水镜先生知道后也很遗憾,他对我
们说了四个字:用进废退。也就是说,任何器官,你用得多了便会进步,反之,则会退化乃至荒废掉。

军师最后说,这四个字同样可以用来说明关羽的左手和那个小孩的眼睛。

用进废退,从军师家里出来我很高兴,因为我又学到了一个词,这证明我又朝聪明人的方向靠拢了一点。

\section{}

最近军中突然流行一种说话方式,就是把任何事情都分为两种:好消息和坏消息。

起初是探子在汇报军情的时候经常用到,比如:报!好消息,黄忠在雒城大败敌军!再比如:报!坏消息,庞军师死于落凤
坡!后来一些兵士没事便在平日里也常用这种方式说话聊天。比如俩人一块吃饭,甲说:好消息,我刚在菜里吃到了一
块肉。隔了一会乙说:坏消息,我在饭里吃出了一粒沙子。

要说一种东西流行起来可真是城墙都挡不住,在很短的时间内,全军上下几乎开口便是好消息坏消息,一时间弄得如果
有了不好不坏的消息都没法开口的地步。甚至连一向沉稳严谨的军师都跟上了潮流。那天演练阵法的时候,军师总结
发言如下:

好消息,我们今天这套阵法大家演练得不错!坏消息,有个别军士的动作不够整齐。好消息,今天主公说他要亲自来观
阵。坏消息,由于主公身体不适改为卧床休息。好消息,明天我们将继续演练第四套阵法。坏消息,昨夜我夜观天象发
现今天可能有暴雨\dldots 好消息,我带了伞。坏消息,怎么\dldots 下的是冰雹?

要说这流行的东西未必是好的,反正我是越来越不适应这种说话方式了,累不累啊。

有一天,魏延愁眉苦脸的来找我,进门就开始长吁短叹。我挺纳闷,魏延最近经子龙介绍认识了一个女人,好象叫如月,俩
人情投意合如胶似漆,把魏延给美的,那张蔫巴脸经常笑得跟菊花似的,今儿这是咋的了?

魏延叹了一口气说道:一个好消息,一个坏消息,你先听哪个?

先来好的吧。

好消息,如月怀孕了!

哇,我叫了一声:恭喜啊!

却听魏延继续说道:坏消息,孩子他妈的不是我的!

隔了几天,子龙兴冲冲的来找我,他最近泡的马子叫如霜,我还跟他们吃过两次饭呢,俩人在一起也有三个月了,按子龙
以往的记录来看,差不多也是到分手的时候了。不过按子龙的个性来说,一般都是泡上容易分手难,不晓得他来找我是
不是想告诉我这个。

谁知子龙一进门就喊道:一个好消息,一个坏消息,你先听哪个?

我犹豫了一下,有了魏延的那次,我还是先听坏消息吧。

坏消息,如霜怀孕了!

啊?我差点从椅子上摔下来,这算坏消息啊?那好消息呢?

好消息,孩子不是我的!

我靠,我差点没背过气去,同样的事在子龙和魏延身上发生,他俩的反应却有天壤之别,你说同样是生活在一起的俩兄
弟,做人的差距咋就这么大涅!

送走了子龙,我心里又开始嘀咕一件事,这俩野孩子到底是谁的呢? 

\section{}

大哥近日心血来潮,见什么好什么,自己经常说什么人生苦短啊时不我待啊之类的话。

先是跟着军师练了一阵子书法,说是可以修身养性,一时间满院子里都贴着他的作品,跟小孩儿的尿布一样。谁知不到
半个月他便放弃了,理由是书法这东西过于沉闷,容易消磨人的斗志。

后来他又找到二哥,准备练练大关刀,这次的理由充分得很,全民健身强身健体嘛。二哥一向是个认真的人,见大哥要
学,把看家底儿的招式都拿出来了,大哥穿一身短打,手持一把木制的大刀,俨然一副看家武师的造型,在演武厅内拉开
架式,一招一式的倒颇有些大家风范。这次坚持的时间比较长,将近有一个多月,后来那股子劲头过去了,便不再去找
二哥练武了。二哥却也实在,两天没见大哥来练刀,主动登门去找他,大哥支支吾吾地推说自己身体不适近日犯了痔疮
才搪塞过去。后来子龙把这个故事的结尾又演义了一句,说二哥出得门来,仰天长叹了一句:唉,可惜了如此一个练武
奇才啊!

最近这阵子大哥开始养鸟,要说起来这事还是魏延给惹的。有一天魏延骑马去山上溜圈,回来以后便带了一只鸟,这鸟
我们都不认识,蓝脊背红翅膀,头顶一撮白毛,尾巴却又是黄的,叫起来千折百回,煞是好听。大哥见了喜欢得紧,命人
精心打造了一个笼子,每日里亲自喂食喂水,连上朝都随身携带,简直比亲生儿子还要宝贝。

我对大哥养鸟没什么看法,我一直比较担心大哥会来找我学丈八蛇矛,后来他终于也没有来,看现在他养鸟养得过瘾,
想是他习武的兴致已经过去了吧,私下还有点庆幸。

忽有一日,都快半夜了,马超气喘吁吁地来到我房间里,面色苍白。进门后他见左右无人,从怀里小心翼翼地拿出一个
东西放在桌子上,我一看不由得大惊失色,原来竟然是大哥的那只鸟,而且浑身脏兮兮的已经死了!我忙问这是怎么回
事。马超紧张得声音发颤,说了好半天我才明白。原来马超家里养了一只猫,晚上马超要睡觉的时候发现猫在玩一个
东西,过去一看,玩的居然是大哥的鸟。你说现在怎么办,你说现在怎么办?马超在我屋子里走过来走过去。

怎么办呢?我挠着头想了半天,如果是我呢,我就去对大哥说实话,猫干的又不是你干的嘛,大哥又不能对一个畜生怎么
样。不行,绝对不行!马超断然回绝了这种方案。主公即使嘴上不说什么,但难免会对我有所记恨。

那你说怎么办?我是想不出再好的方法来了。我双手一摊。

有了!马超忽然面上一喜,七手八脚的开始忙活,先弄来一盆水,把鸟放进去洗刷了半天,然后用毛巾仔细地擦干,又用
梳子把羽毛梳理得整整齐齐,然后对我说,三哥,我趁天黑把鸟偷偷放到大哥的窗下,这样大哥只会想到或许是鸟从笼
里逃掉不小心撞到窗户上死掉了,你说这个主意怎么样?

好主意,我怎么就没想到呢?看起来马超的确比我聪明得多啊。

第二天,天还没亮,忽然听到外面大呼小叫的,我连忙披上衣服出去,却见一群人围着大哥,大哥手里拿着那只鸟,嘴上
喊道:见鬼了!见鬼了!昨天晚上它死了,我亲手把它埋在后山上,今天它竟然整整齐齐的躺在这里\dldots 

的确是见鬼了,不过我却知道那个鬼是谁,我强忍着笑回头看了一眼马超,只见他张大了嘴巴,伸着舌头,活象个吊死鬼。

还是军师的那句话正确,用错误来掩盖错误的话,得到的或许只是荒谬。军师说,人生就象一头拉磨的驴子,只能蒙着
眼睛不停的向前跑,否则就会挨鞭子。

这让我想起了我的理想。我小时侯的理想是做一名非常出色的泥瓦匠,长大以后我成了一名非常出色的屠夫,而现在
我基本上算一名非常出色的刽子手。

这种变化一是说明了世事难预料,二是说明了理想和现实之间的差距是很大的,三呢,本来没有三了,但魏延经常说我
什么事都说不出个一二三来,因此我要加上个三凑个数,显得整齐一些。

我小时侯之所以立志做一个泥瓦匠是有原因的。有次下雨把我们家的院墙淋倒了,我父亲当时已经不在了,因此砌墙
的任务便落到了我头上。我在砌墙的时候找到了久违的乐趣,要把那些大大小小千形百状的石头垒成厚度均匀两边光
滑的一堵墙,这不仅仅是力气活,也不仅仅是技术活,这分明是一门高深莫测的艺术啊!于是当墙砌完以后我基本上已
经把我的伟大理想定下来了。

可等我大到应该养家糊口的年龄时,我并没有成为一名泥瓦匠,主要的原因是当时泥瓦匠的数量过大,而盖房子砌墙等
工作又太少,为了生存,我只能放弃。

选择杀猪最主要的原因是可以有肉吃,而且我去学杀猪的第一天就有人说,瞧瞧你长的这模样这块头,不杀猪真可惜了。
后来我自己对着镜子看了半天,发现他们说得很对,既然上天把我长成杀猪的模样,那我就得去杀猪,上天的安排最大
嘛。

其实我杀猪绝对是有天赋的,在很短的时间内我就掌握了所有的技巧,并且从没失过手。当时杀猪都是有讲究的,一般
差不多的屠夫都要求一刀拿下,倘若第一刀没杀死,那就非常掉架。我见过这么一次,当时那个屠夫也算赫赫有名,外
号叫作王一刀。在众人把猪绑好了以后,王一刀神态自若地放下茶杯拿起屠刀,到了猪跟前,摆了一个很象白鹤展翅的
造型,然后出手,收刀,一气呵成面不改色,那猪连一声都没出,众人齐声喝彩。王一刀得意洋洋地端起茶杯继续喝茶,
几个人围过去给猪松绑准备忙活开膛剥皮,忽然听到一声尖叫,见一庞然大物从人群中直冲出来,噼里啪啦的不知道撞
翻了什么东西,箭一般地往西跑了。众人愣了一会,发了一声喊,开始追。这时有人发现王一刀倒在地上,扶起来一看,被
猪撞断了两根肋骨,而那头猪最后在数里外找到了,口吐白沫是活活累死的。于是后来众人给王一刀改了个外号叫王
一累。

我现在改行杀人,但经常在杀人的时候想起当年杀猪的情形,丈八蛇矛刺进对方的身体内和杀猪刀割断猪的动脉的那
种感觉其实差不多,同样也有一大群人等着给你喝彩或者看你的笑话。

就我从一个泥瓦匠变成了一个将军这件事来说,在很多人看起来我是很幸运的,或许会有一些人把我当成一个正面的
教材来教育那些不知上进的孩子。但我自己却没觉得很自豪,相反还有些失落,因为我经常做梦梦到我是一名快乐的
泥瓦匠,想象着自己亲手砌的一栋栋房屋,其兴奋程度远远超过了于百万军中取敌将首级。

子龙给我讲了个故事:有一次我们行军路过乡下,看到一位老农把喂牛的草料铲到一间小茅屋的屋檐上,感到奇怪,于
是就问道:``老伯,你为什么不把喂牛的草放在地上让它吃?''老农说:``这种草草质不好,我要是放在地上它就不屑一
顾;但是我放到让它勉强可够得着的屋檐上,它会努力去吃,直到把全部草料吃个精光。''

子龙说,我们很多人都象那头牛,都很在意那些遥不可及的东西,而对于已经到手的却不屑一顾。

如此说来,假设我当初真的当了泥瓦匠 ,看到一个耀武扬威的将军路过时,必定也是羡慕不已了。这么想着,我觉得心
里舒服多了。

\section{}

蜀中的这个夏季热得要命,不穿衣服坐在那里汗都不停的流,一天到晚后背总是湿漉漉的。树叶子都是蔫巴的,门前的
石板上居然可以煎鸡蛋。

晚上能稍好一些,但也热得睡不着觉,我拿一把蒲扇出门乘凉,看见魏延蹲在门口,舌头伸出老长,吓了我一跳,我说魏
延你伸着舌头干嘛呢?魏延说我在散热呢。我忽然想起当年我养的阿黄一到夏天也是这个样子,于是我也张嘴伸出舌
头哈哈了几下,发现还真的管用哦,于是我和魏延并排蹲着,张着嘴哈哈的吐着舌头,月光照在我俩身上,留下两个很奇
怪的影子。

天一热,人的脾气就变得暴躁。我最近三天已经摔了七个茶杯五个饭碗两个酒坛子,弄坏了两张椅子三把蒲扇外加一
个马鞍子,打了四个士兵共计一百零八鞭子。弄得几乎没有人敢靠近我,我感觉我无时无刻不在发火,我咒骂着这该死
的天气,魏延说我气急败坏的样子象只黑色的母鸡。其实他的脾气最近也好不到哪儿去,有次他乘凉时有只蚂蚁趴到
他腿上,他跳将起来踩了三十多脚,末了还亲自往蚂蚁洞里撒了一泡尿。

就连平日里稳文尔雅的子龙都受不了了,他最恼火的是这种鬼天气让他无法泡妞。说的也是,你说俩人要是往一起这
么一抱,浑身湿漉漉的象两条粘稠的蛇搅在一起,确实让人意兴阑珊。没有了女人滋润的子龙整日里垂头丧气如同门
前那棵老柳树。

最可笑的是二哥,一把大胡子象围了条毛围巾,他又舍不得剃,于是你会看到他每每用手撸一下胡须,拧下一把一把的
汗水。不过惟一的好处是再热他依旧是面不改色。

奇怪的是我们热成这样,军师却好象一点都不热,他依旧穿着他那一身棉布道袍,不紧不慢地摇着鹅毛扇子,迈着四方
步在烈日底下溜达着,额头上一滴汗都没有。这让我们羡慕得要命,有一次我忍不住问他,为什么你不出汗呢?军师微
微一笑,说道:心静自然凉。

切,我才不相信呢,我晚上睡着的时候心够静了吧?可醒来的时候下面的草席都跟被水浸过一样。我觉得军师肯定是在
敷衍我,不跟我说实话。对了,会不会是他穿的那件道袍有什么玄机呢?想到这里我便大大咧咧地跟军师借道袍穿,军
师愣了一下,哈哈大笑了几声把道袍脱了给我,我穿上虽然小了点,但也对付着能穿。

一天下来,道袍不知道湿了多少次,我只觉得比以前还要热,并没有觉得凉快。原来这道袍也不过就是普通的袍子而已。

有一天半夜我从梦中醒来,忽然觉得浑身发冷,我翻箱倒柜地把冬天的被子找出来盖上,依然觉得冷,上下牙都得得的
响,我知道我是感冒了,不过我却很开心。第二天一早,我挣扎着爬起来,把最厚的衣服找出来,披着斗篷,带着帽子,脸
色苍白却又得意洋洋地在人多的地方走来走去。

\section{}

当年在新野,大哥收了一个叫单福的人,初到便破了曹仁的八门金锁阵,大哥如获珍宝。后来才知道,原来此人就是许
庶。

要说起来,没有许庶,也就没有军师。要不是当年许庶走马荐的军师,估计军师现在还在南阳睡觉呢。

军师至今还经常提起许庶,他们俩以前就是好朋友。大哥更是隔三差五的就提起许庶,唏嘘一番,当年他送许庶走的时
候哭得跟泪人似的,为此二哥郁闷了好几个月。许庶现在虽然在曹操手下,但他是被程昱骗过去的,之后他立过誓,绝
不给曹操出一谋一策。于是有了``许庶进曹营,一言不发''这句话。

我曾经很纳闷,以曹操那种性格,如何能忍受养一白吃白喝不干活的闲人呢?军师笑了,他说这正是许庶的高明之处。
曹操忍受许庶的原因有二,其一是虽然许庶不干活,但他也没有为别人干活,象许庶这样的人,帮谁都将对曹操构成威
胁。留住许庶意思就是说,我用不上,你们也甭想用。其二呢,就比较有意思了,曹操为人心胸狭窄脾气暴躁,当年行刺
董卓失败逃跑的路上,误杀了吕伯奢一家九口之后,说了一句名言:宁教我负天下人,休教天下人负我。在这样的人手
底下工作,的确是一件提心吊胆的事。有人曾经说过,曹孟德一生杀人无数,但十有八九杀的是自己人,这八九中又有
七八是误杀。比较出名的是横槊赋诗杀刘馥、鸡肋杀杨修、梦中斩近侍、多疑杀华佗等等,另外有中计杀蔡瑁张允、
借刀杀弥衡、羞怒杀许攸等。总之在曹操身边做事相当于判了个死缓,具体缓多长时间得看你的表现了。但曹操所杀
之人中,总结起来基本符合一条规律:有才,狂傲,最重要的一条就是多嘴多舌。所以许庶的一言不发倒恰好是保全自
己的最佳姿态,不求有功,但求无过。

听军师这么一分析我觉得很有道理,不过军师又叹了一口气道:可惜了元直那满腹经纶啊!这句话让我想了良久,愈想
愈觉出许庶的可悲之处。

\section{}

二哥当年为了保全嫂嫂曾降过曹操,那也仅是权宜之策,他绝对没想真为曹操卖命,但还是为曹操斩了颜良诛了文丑,
设身处地的想一下,如果换作我,当时也会那样做。当时颜良文丑气焰嚣张,满朝武将竟无人敢应敌,究竟何方神圣有
如此的本领?作为一名习武之人自然心痒难耐,忍不住想去会会。想那学文之人跟学武之人应该有相同之处,想那许庶
每每跟着众人一起开会研究,听着众人或者高明或者荒谬的言论,居然能忍住不发一言,实在是一件很痛苦的事。

就好象一个高明的厨子整日对着满堂的佐料却不让他操刀;一个好色之徒每天对着一群美女却无能为力一样,人世间
最痛苦的事情莫过于此了。

这么想起来,我现在每天还能够喝上一碗酒实在是一件很幸福的事情了。提起许庶来,忍不住想说说杨修。

杨修算是曹操杀的人之中比较有意思的一个了。我忍不住央求军师把他的故事讲来听。

杨修官不大,是一个主簿,也就是一个文书。平日里也就负责写个字整理个文件什么的。按说他一没有造反乱上,二没
有违法乱纪,三没有损兵折将,也就是说因为他位低言轻,所以他连犯大错误的机会都没有。他的死完全是这小子咎由
自取的。

杨修祖上世代为官,他父亲是太尉,可谓位极人臣,当时的太尉跟丞相是平起平坐的,一个管行政一个管军事。杨修自
幼聪慧,远近闻名,加上身份显赫,所以免不了一个``骄''字。

一个从小在别人的赞扬声中长大的人,修为好的话那叫自信,修为不好的话就叫虚荣了。

杨修的虚荣不是一天两天的事了,屡次在人多的场合中抢曹操的风头。其实要说论起猜谜对对机敏反应之类的,杨修
那是远胜于曹操,可关键的问题是你和曹操又不是同班同学,位置没摆对。抢曹操的风头无疑是抢自己的饭碗,当然弄
到最后抢了自己的脑袋更是意料之外的事了。

还有一个原因是杨修不会说话,在很多时候有些话是不能随便乱说的,或许说者无心,但听者有意。为此军师讲了一个
笑话:旧时年关,有人在家设宴招待帮助过他的人,一共请了四位客人。时近中午,还有一人未到。于是自言自语:``该
来的怎么还不来?'',听到这话,其中一位客人心想:``该来的还不来,那么我是不该来了?'',于是起身告辞而去。其人
很后悔自己说错了话,说:``不该走的又走了'',另一位客人心想:``不该走的走了,看来我是该走的!'',也告辞而去。
主人见因自己言语不慎,把客人气走了,十分懊悔。妻子也埋怨他不会说话,于是辩解道:``我说的不是他们''。最后
一位客人一听这话,心想``不是他们?那只有是我了!'',于是叹口气,也走了。杨修便跟笑话里的主人似的,经常说一
些自以为一点错误没有的话,但没想到曹操却也同笑话中的客人似的,总觉得他的话刺耳。说到这里军师忍不住叹了
一口气,最近国人猜疑的风气愈来愈重,这其实是一个民族的悲哀啊。

杨修最后是死在所谓的``鸡肋''上,其实在曹操眼中,杨修还不如一块``鸡肋''。或许弥衡算得上一块,食之无味,弃
之可惜。至于许攸则接近于一条鱼,味道不错,丢了也可惜,但就是刺多了一些。杨修嘛,可以说是食之无味,弃之亦不
可惜。要说起来,以他做的那些事说的那些话,曹操能忍他到现在,的确也不容易了。

但是杨修聪明啊!我忍不住插了一句。军师白了我一眼,他那叫聪明?他那只能叫小聪明!离真正的聪明差十万八千里
呢。

小聪明?我小声嘀咕着,小聪明也总比我这种笨人强啊,至少不吃亏。

军师怔怔的看着我,说了一句:吃亏才是真正的大聪明。

\section{}

我们取了西川之后,基本上可以喘口气了。在此之前我们东奔西跑的,今天借你一块地儿,明天再借他一块地儿,当然
借了的一般都要还,也有不还的,比如荆州。

目前魏、蜀、吴天下三分的轮廓已经出来了,一切都朝着军师早年的预言发展。这其中魏国的形式最好,兵精粮足,曹
操又是挟天子以令诸侯。而吴则主要靠长江这道天然的城墙保护,至于蜀国目前则是最不稳定的,由于我们刚进来,民
心还未定,况且东有张鲁雄居汉中,北有曹操虎视耽耽,惟一能依仗的就是蜀道难难于上青天。

前几天得到消息说曹操西征,收了张鲁,取了汉中。当时整个蜀中上上下下一片喧哗,因为汉中与西川唇齿相依,曹操
取了东川,势必挥鞭西下。一时间兵士们都自动的开始磨枪擦剑,老百姓则议论纷纷,心细者已经把家中贵重物品打
包,随时准备撤退。

大哥眉头紧锁,就连军师也面色沉重,下令各关口将士做好备战准备,一直以来无论情形多么危急军师似乎都是谈笑自
若的样子,可这次看起来事态的确有点严重。

又过了几天,探马来报,曹操按兵不动,似乎没有进军西川的想法。军师有点纳闷,挥手让探马再去细查,然后自语道:
莫非这是曹贼的欲擒故纵之计?

三日后,探马又来报,说曹操确实不想进攻西川,事情经过是这样的:曹操取了汉中之后,司马懿、刘晔等都力谏曹操发
兵攻打西川,但曹操却说了一句:``人苦不知足,既得陇,复望蜀耶?''

这句话里有个典故,军师讲给我们听,当年光武帝平定陇西后给部下的信中写到:我们不应该得到陇西就满足,还要进
一步收取蜀地。于是流传下来叫做得陇望蜀。而其实曹操取的汉中比陇西距离蜀中的距离更近一些,他却不``复望
蜀'',当真出人意料。

军师忽然嘿然一笑说道:曹阿瞒当真把自己比作光武帝了。说罢摇着羽扇哼着小曲走了,留下我们一帮人大眼瞪小眼。

时隔多日之后,军师在一次宴席上又提起了这件事,他分析了一下曹操为什么不发兵西川,原因有二,一是他对我们还
是有顾虑的,他担心万一攻不下来,损兵折将不说,更有孙权在江东等着坐收渔翁之利。二是他说的那句人苦不知足
乎?也就是说曹操或许真的有点知足了。

曹操会知足?谁相信呢?

很多年以后,我们一次次的攻打魏国,收复了汉中大片土地,曹操当真一次也没主动进攻我们,一直到他病死。

军师说:曹孟德做了很多让我瞧不起的事,但他却也有很多让我十分佩服的地方。小时侯偶尔有一次吃了一支冰糖葫
芦,感觉好吃得不得了,忍不住想,以后有了钱,一定把满屋子都堆满糖葫芦,一天吃一百支。

现在我的确有了钱,可偶尔买支糖葫芦回来,却怎么也吃不出小时侯的味道。东西其实没变,只不过物是人非事过境迁
罢了。

\section{}

当年我兵败小沛,月黑风高,四面受困,只得自行杀出一条血路前往芒砀山。乱军之中我以矛开路,杀得兴起,待冲出包
围才觉得大腿湿乎乎的,用手一摸全是血,不知道被谁捅了一枪。当时也顾不了那么多,从战袍上撕下一条布绑在伤口
上,依然打马前行。

待走到第二日的中午,我又累又饥又渴,伤口开始发麻,眼前金光灿烂,远远看到前面似乎人影晃动,我大叫一声伸手去
摘矛,眼前一黑,就此人事不知。

醒来后见一大婶,慈眉善目的,见我醒了,她起身端了一碗东西过来,说道:将军,起来喝点东西吧。我挣扎着坐起来,发
现我的伤口已经被很细心的包扎好了,接过碗来一看,是一碗莲子粥,一仰脖,咕嘟咕嘟喝了下去,入口嫩滑香甜,只觉
得精神一震,忍不住叫道:再来一碗!大婶似乎面有难色,拿着碗转身出去,一会儿工夫回来了,我接过碗来没有一饮而
尽,这么好的东西,我得慢慢品尝,于是我一小口一小口地咂着,却发现味道虽然还是刚才那味道,但却稀了很多。这时
大婶说了一句话让我受用一辈子的话:好东西就那么多,要想再多,便只能稀释了。

很多年来,我把这句话说给很多人听,每个人都深有感触,但每个人的感触却都不一样。

我说给魏延听,魏延正在吃饭,他听完以后没吱声,夹起一块腊肉放嘴里慢慢地嚼,然后说,最早来蜀中的时候,觉得腊
肉是天底下最好吃的东西,后来我们住在这里不走了,上顿下顿的吃,现在便有味同嚼蜡的感觉了。东西虽然没稀释,
但自己的感觉却稀释了。

我说给子龙听,子龙刚甩了三十七号,正在寻觅三十八号。听完我的话,他神情黯淡地说,其实我早就知道,我一直都知
道,稀释到最后就跟白水一样,没有甜蜜,没有激情。但最起码这样让我看起来不孤单,三哥,也许你永远都无法理解,
一个人不孤单,想一个人才孤单\dldots 子龙的眼圈有点红,我知道他又想起遥远的地方的那个遥远的人来了,于是起身悄
然离去。

说给军师听的时候他正在房顶上看星星,我顺着梯子爬上去跟他并排坐着。军师说:翼德,你看天上那些星星。我抬头
看了半天也没看出个所以然,就那么一堆亮点有什么看头啊?军师笑了,所以说,好东西都是相对的,你所谓的好东西别
人也许不以为然,但别人的好东西你或许也觉得不值一钱。我们大多数人穷其一生都在追寻着别人眼中的垃圾,想想
的确有点可悲。但这个过程让你感到愉悦便是好的。说到最后,军师忽然幽幽的叹了一口气:唉,这个月我已经是第八
次上来看星星了,她已经二十多天没跟我吵架了\dldots

大哥散步的时候我说给他听,他忽然停下来,指着前面的一棵桃树对我说:三弟,还记得当年的桃园结义吗?那时我们三
人意气风发,一心想上报国家,下安黎民。上马杀贼,下马吃肉,何等的快意!可事到如今,我们的地盘大了,势力强了,
却没了以前的那种理直气壮。有时候我经常不明白自己现在到底在做什么,国不为国,君不为君,百姓依然在煎熬,离
我们的初衷愈来愈远。你得到一些东西的时候必然要失去一些东西,这或者也就是所谓的稀释后的好东西吧。

只有二哥听完我的话没有说话,什么也没有说,他就那么坐着,看着窗外,无欲无求的象一尊佛像。

\section{}

今天阳光明媚,太阳挂在天上象一个硕大的烧饼。大家都聚在大哥的大厅里喝茶聊天,大哥看起来心情非常好,说起话
来手舞足蹈的。他说自己昨晚梦到煮茶,味道那个香啊,刚想喝却醒了,不过那香味隐约还在,最后他总结道:那定是天
上才有的神物,能在梦中闻一次也算是三生有幸啊!

大家都随声附和,惟独军师似乎稍稍挑了下眉角,但很快也就恢复了原样。

眼瞅着将近中午,大哥忽然提议大家到后花园去娱乐一下,都亮亮绝活,所谓各显其能,命人捧出一盘子的夜明珠,也就
是``彩头'',表演好的除了有彩头以外,还将放假一个月,不用上朝。大家轰然叫好,拥着大哥一起奔向后花园。

我虽然不想要什么彩头,不过放一个月的假还是很有诱惑力的,那样我就可以天天从早上喝到晚上了。后花园里有块
空地,十几丈见方,边上有一排大树,下边是一排石凳,早有下人摆上茶果,大哥等人就坐在那里,五虎将个个摩拳擦掌,跟
唱戏似的。

第一个出场的是老将黄忠,这家伙性子急,早就按捺不住了。只见他先命人爬到树上用细线系了一个铜板,风吹过摆来
摆去的,然后他退到一百来步,大家明白他又要表演他的百步穿杨了。老黄忠摆了个马步,双膀较力,嘿的一声拉开了
雕花弓,只听得弓弦响,再看那铜钱,却已经被箭定在树上了,于是众人齐声喝彩,彩声未落,却见大哥端着茶杯脸色不
对,仔细一看,原来那一箭恰巧震落了树上的一块鸟粪,又恰巧掉到大哥的茶杯里,下人慌忙给大哥换了一杯茶,而黄忠
站在当场脸上青一块红一块的。

不过大哥终究还是很大度,谈笑自若地问道:下一个该谁了?魏延应声而出,虽说五虎将中没有他,但他总屁颠屁颠地跟
着我们,象是个超级替补。却见魏延手里拎着双刀,他是用刀的,但他一直用的都是大砍刀,从没见他使过双刀。魏延
提刀在手解释说,他自幼曾经练过一阵双刀,但大都只是花架子,不太实用,今天拿出来给诸位演示一下。说完后他一
摆手,上来一个士兵,手里拎着一桶水,魏延先摆了个起手势,然后开始舞了起来,只见双刀上下翻飞,舞到极处竟只看
到一团白光飞转,那士兵舀了一瓢水泼将过去,哗的一声空中水气弥漫,当真是泼水不进啊!于是众人又是一次满堂彩。
大哥叫的最响,而且还起身走过去,从士兵手里拿起水瓢,也舀了一瓢水泼了过去,只听哗的一声,大哥满头满脸都是
水,跟落汤鸡似的站在那里,众人都张大嘴巴发不出声音。魏延渐渐地慢下来,最后还摆了收刀的造型,好象完全不知
道发生了什么似的。原来他这一路刀法下来,自己是完全看不到的,大哥过去泼水的时候正是他要准备收势的时候。

大哥一言不发地回到座位上,头上还滴着水,最可气的是那士兵为了省事是就近从马槽里弄的水。现场的气氛稍微有
些尴尬,这时马超站了出来,白盔白甲白脸蛋,手里提一杆亮银枪,真如同画中的人物似的,不愧为``锦马超''。马超对
大哥施了一礼道:难得主公今日高兴,我与子龙来个单枪对单枪,有个名称叫作双龙出海,给大哥助兴。话音未落,见子
龙出场了,也是白盔白甲白脸蛋,手里也提一杆亮银枪,但看起来有些仓促,似乎没有马超准备的妥善。二人这么一来,就
连黄忠和魏延都眼里放光,我们虽然身经百战,但除了二哥与黄忠在战场上对阵过以外,我们几个却真的没有交过手。
众人还没来得及叫好的时候,场上二人已经动了手,两人都是灵活型,以快见长,只见双枪舞动,如梨花纷飞,身影晃动,似
双蝶飞舞,煞是好看。大哥早已忘了先前的不爽,伸长脖子看得津津有味。忽听二人同时喊了一声,双枪相错,喀嚓一
声,一物直冲大哥飞去,二哥手快,掀起面前的茶几一挡,砰的一声,一个断枪头扎在上面,仍在轻微的颤动。再看大哥,面
如土色,往后便倒,众人慌忙七手八脚地把大哥扶起来,探手一试,却没了呼吸。

待御医匆忙感到后,几经折腾,大哥大叫一声缓了过来。原来断枪飞来的时候,大哥正在喝茶,一个茶叶梗恰好噎在喉
里。

直到这时,军师才开了口:臣幼时学过七经八卦,略微懂得一些解梦,周公曰,梦到煮茶,必将倒霉。不过一是臣不太确
信,二是见主公如此兴致,于是忍住没说,却没想到\dldots 唉\dldots

我站在那儿心里七上八下的,侥幸的是我还没上场,否则还不一定出什么事呢。震惊的是原来以前人们说的倒霉起来
喝凉水都塞牙真的是千真万确啊! 

\section{}

我儿子张苞,二哥的儿子关兴,提起来很多人都知道,但却很少有人知道魏延的儿子。其实魏延也有一子,名字叫魏
猛。

魏延当初给儿子起名字的时候可能希望他将来能继承父业勇冠三军,但谁成想这个魏猛却天生一副小胳膊小腿,打小
娇生惯养,细皮嫩肉的,根本不是习武的料。后来请来私塾先生想教他学文,意思是说当不成将军好歹也弄个参军之类
的,可一年之内请了十六个师傅,没一个能教得了。军师曾经说过,魏猛是典型的朽木不可雕也。

后来魏延也就索性不管了,由他去吧。这下可好,眼瞅着这孩子一天天长大,成都城里多了一个小霸王。你说他要是作
奸犯科吧,倒也好说,抓起来就完了嘛。可他还真是不做大坏事,杀人放火之类的他不干。可诸如偷个梨摸个枣烧人家
衣服往别人门上涂大粪之类的事他没少干。当然他最喜欢的还是调戏妇女,甭管是年轻的年老的有点姿色的还是丑八
怪,就是头母猪他碰到了都得过去摸一把。

后来有一次出了点事,这小子摸来摸去有一天摸到了子龙当时的马子身上,子龙虽说脾气比我强,但也绝不是善茬,当
街把他给饱揍了一顿。魏延也觉得这样下去不行,万一哪天再遇到一个吃生肉的主儿把儿子给宰了就麻烦了。于是便
狠心把他锁在屋里,派人严加看管,不让他出门。可没出三天这小子就寻死觅活地闹着要绝食,把魏延弄得一筹莫展。
后来不知道谁给他出了一个主意,让他花钱雇几个妓女回家,让魏猛摸着玩呗,又安全又省心。嘿,要不说人类的智慧
是无穷的嘛,这招还真灵,足足半年多魏猛就没出过门。

可这终究不是长久之策,养妓女的花费倒是小事,儿子的终身幸福可是大事啊,魏延整日里还是长吁短叹。有一天我喝
了点酒,脑子灵光一闪,我忽然想起了点什么,于是就问魏延:你儿子是不是一种病啊?你找个大夫来给他看看啊。魏延
把脸一拉,说了一句,你儿子才有病呢。然后转身走了,好几天没搭理我。

几个月后的一天,魏延欢天喜地奔走相告,说他儿子现在好多了。原来魏猛虽然坏,但他毕竟不是个傻子,他也知道自
己这样下去终究不行,可还真是控制不住。于是有一天就跟魏延提议找个大夫来给他看看。魏延虽然奇怪,但还真找
了个神神道道的郎中来,那郎中问了半天,又看了半天,然后留下一本书,让魏猛早晚起来读三遍,如能坚持半年其病自
愈。抱着死马权当活马医的想法,魏猛就按大夫吩咐的去做。还别说,不到一个月就有了效果,魏猛现在几天不见女人
都不想了。

那本书是什么书啊?我忍不住问。魏延说没名字,上面的话也很奇怪,呜哩哇啦的,还有些很奇怪的图,刀山油锅之类
的,看着挺吓人。

魏延走的时候还挺得意地说了一句:还是我儿子聪明,自己都发现这是一种病。送走了魏延我好一阵郁闷,明明是我先
说的嘛,当初他还骂我呢。

军师后来知道了,他给我讲了一个智子疑邻的故事。说宋国有个富人的墙被雨淋倒了,邻居老头说不修的话会有人来
偷东西的,这个人的儿子也这么说。后来果然失窃了,于是这个人夸奖自己的儿子聪明,有先见之明,而怀疑是不是邻
居老头偷的东西。

讲完了以后军师笑着对我说:翼德啊,你就是那个邻居老头啊。

我摸着头嘿嘿笑了半天,忍不住想,如此说来,是不是我们好多人都经常在扮演着宋国富人的那个角色,只是没那么明
显或者自己压根就没想到而已呢?

自己的东西永远是好的,也许只有男人有一样例外,老婆总是别人的好。

\section{}

今天士兵捉到一个魏国的奸细,大哥对此事非常重视,派我和魏延亲自去审。

战争期间,捉到一个奸细很正常,为何此人还惊动了大哥呢?原来这个间细非同寻常,他自大哥进川就隐姓埋名藏进内
务府,乔装成一个哑巴,每日里打扫卫生,干些杂活,因为见他是一个哑巴,所以我们有很多机密都当着他的面商议,因
此此人知道我们的内情颇多,要不是有天夜里有打更的恰巧路过听到他在说梦话,估计我们至今也不会怀疑他。

我和魏延得了命令,便去牢房提审,到了一看,见此人绑在柱子上奄奄一息,浑身上下已经没一个好地方了,各种刑具摆
了一地,可这人端的是条硬汉子,至今只字未吐。我平生最欣赏硬骨头,要不是他的身份特殊,我还真想跟他交个朋友。

边上的士兵过来低声道:将军,不能再打了,再打下去估计会死人的。

我们当然不是来审死人的,这可如何是好呢?我和魏延大眼瞪小眼地看了半天。最后还是魏延想了一个主意,魏延的意
思是他不是不怕死,他是知道他一说出来肯定是个死,不说的话可能还能有转机,咱们吓唬他一下试试。

商量完了以后,魏延便去死牢里提了个犯人砍了,然后拖着半截血淋淋的大腿走进来,命士兵将那奸细用凉水泼醒,然
后我故意问他:魏延,你拖的什么东西?魏延道:还不是上次吴国的那个奸细,上老虎凳时弄断了条胳膊,他央求说身体
发肤受之父母,求我们帮他将断臂送回家乡,我见他可怜就照办了,谁知没几天他又断了一条腿,还让我给他捎回去,我
回去想了一下这事不对头,他分明是有计划地想分批逃跑,你说对吧?于是我一生气就把他给杀了,喏,砍头去尾也就剩
这么点东西了,幸亏我发现的及时,否则还真让他跑了。

魏延边说边在那奸细周围晃来晃去,可说了半天,唾沫星子溅了那人一脸,那人居然眼皮都没眨一下,更别说开口了。
这下我和魏延都傻眼了,还是去找军师帮忙吧。

军师想了半天,要不你们用软的试试?试试用金钱美女等东西诱惑他一下。不行的话就再用硬的,然后再用软的,这叫
打一巴掌给一个甜枣。抡圆了给一个大嘴巴,让他眼冒金星的时候给他嘴里塞一个甜枣,他一咂摸嘴,哎,还真甜,生活
还真美好,鲜明的对比说不定他一下就招了呢。

要不说军师有学问呢,这有学问的人就是不一样,平日里治国安邦的就不说了,连出个馊点子都这么专业。我和魏延连
连点头称谢,然后照着军师的指示去忙活了。

没成想到最后还是失败了,首先金钱这招没用,都拿黄金把他给埋起来了,他无动于衷。至于美女呢,更别提了,派了一
个妓女去试探了一下,此人竟是一个天阉,曹操可真会选人,所谓知人善用啊。

眼看就快到大哥规定的时间了,把我和魏延急得跟热锅上的蚂蚁似的,正好马超路过,问清了状况以后,马超自告奋勇
地说他去试试。死马权当活马医,让他去试一下也好。谁料马超进去以后只在那人耳边耳语了几句,那人面色大变,竟
连连作揖讨饶,随后马超命人拿来纸笔,那人便一五一十地开始招供。

马超摇头晃脑地走出来,我和魏延连忙迎上去,你在那人耳边说了些什么呀?马超哈哈一笑,原来他过去说:你要是再不
招的话,我们就把你洗的白白净净的,穿的整整齐齐的,选一个阳光明媚视线良好的天,把你从大门送出去,临了儿还跟
你挥手道别。

说这个他怎么会害怕呢?他高兴还来不及呢。我越听越糊涂。马超解释道:你想啊,他宁死不招为了什么?还不就图一
个名声吗?我们这样做,别人都会以为此人已经变节了,曹操眼线众多,肯定会知道的,而且以曹操的性格,定会派人追
杀此人,其手段估计比我们的还要残忍。为了名节宁死不屈还值得,死了还落一个叛徒的名声,便是傻子也不愿意的。
于是他权衡利弊就招了。

我和魏延听到最后恍然大悟,齐齐冲马超伸出大拇指:高,实在是高!

\section{}

今天又去听二哥讲列国故事,二哥说,列国时代有很多人都跟禽兽一样,他举了两个例子。一个是吴起杀妻求将,说吴
起当年在鲁国当差,当时齐国讨伐鲁国,吴起想做大将军带兵迎战,可鲁国国君嫌他妻子是齐国人,怕吴起到时候变节,吴
起居然回家将自己的妻子杀死,拿着人头去见国君。另一个例子是易牙煮婴,易牙是齐桓公的一个厨子,做菜非常好
吃,有一次齐桓公说:易牙做的饭太好吃了,只是还没有吃过易牙做的蒸婴儿肉。第二天,易牙就把自己的儿子蒸了端
来给齐桓公吃。

我听着听着突然想起一个人,这个人叫刘安。

当日芒殇山兵败的时候,大哥也东奔西跑的,有一日到了一个村庄,投宿在一个猎户家中,这个猎户就是刘安。刘安见
大哥来了,慌忙准备饭菜,端上一盆热气腾腾的肉,大哥当时已经饿了好几顿了,狼吞虎咽之后,随口问了一句:这是什
么肉啊?刘安说是狼肉。等第二天大哥要走的时候,去后院牵马,忽然发现一个妇人躺在厨房里,已经死了,两个胳膊上
的肉都被割了,于是惊问这是什么人,刘安老实交代说这是他妻子,昨晚吃的就是她的肉。

大哥事后说他当时感动得哭了,可我觉得要是我的话首先得吐了。

要说吴起杀妻是为了求将,易牙煮婴是为了讨好君主,按常理来看,刘安杀妻也是为了讨好大哥,以便取得点功名利益,但
事实并非如此。

这事或许只有我和孙乾知道。当日大哥终于遇到了曹操,把此事一说,曹操也很震惊,于是派孙乾拿了一百两黄金给刘
安送去。

很多年以后,有一天我忽然想起这个人,于是跑去找孙乾问起当时的情景。孙乾听我提起这个人,忽然有些紧张,我觉
得不太对头,于是便连哄带吓,孙乾对我说了实话。

原来当日孙乾来到那个村庄,恰巧刘安不在,据说上山打猎去了,跟村民随便聊起来,才知道刘安根本不是为了大哥而
杀妻的。此人是个烂酒鬼,喝点酒就对妻子非打即骂,后来有次下手重了点,失手将妻子打死了,正不知所措的时候,大
哥恰好来了,于是一不做二不休,索性割了妻子的肉来孝敬刘豫州。

原来是这么回事啊,其实说起来失手打死妻子总比故意杀死要好一些,后来割肉就权当废物利用了。

那后来呢?你见到这个人没有?

后来?孙乾脸上又显示出扭捏的样子,后来\dldots 我就离开了,那一百两黄金我也自己留下了。

啊?你贪污了?

三将军,你有所不知,我虽然不才,可却生了八个孩子,加上老母妻妾,当时的那点俸禄根本不够用啊。

八个孩子!他还真能折腾,瞧他那小干巴样还真有两下子。我临走的时候嘱咐孙乾:好好干,多赚点银子,否则小心哪天
你家揭不开锅,就是把你煮了都不够你那八个孩子吃一顿的。

\section{}

那年我们在新野,大哥命我去征兵。

战乱年代,征兵其实比较容易,有饭吃不说每个月还能领点俸禄,战死总比饿死强。

但我征了三天却只征了几百个人,觉得很纳闷。新野虽然不大,按后来许庶的话来说,新野也就是个屁大的地方。当然
他的话有些夸张,谁的屁股也不可能有那么大,但新野的确是个小地方。

但再小也不应该就征这么点兵啊,当年我和大哥讨伐黄巾军时没钱没粮没名没望,还一天征了五百呢。我越想越奇怪,于
是便走到街上看看到底是怎么回事。

正走着,忽然一个乞丐引起了我的注意,大街上乞丐很多,但这个乞丐却和别的乞丐不同,一般的乞丐都是蓬头垢面衣
衫褴褛的,他却白白胖胖的,穿的也很整洁,要命的是他的气质还非常好,往那一坐有种君临天下的气势,要不是他面前
摆着一只破碗和身边的那根打狗棒,没有人会认为他是一个乞丐。

我越看越纳闷,忍不住走过去,他见我来了却也不抬头,我掏出几文钱丢在碗里,他也不道谢,我越发觉得有意思。

低下身来,问他:我给你钱你怎么不谢我呢?

那人这才抬头看了我一眼,说道:钱是你的,你想给便给,我为什么要谢你呢?

嘿,我忍不住有点恼火,于是伸手去碗里拿钱,嘴上说:那我现在不给你了,我再拿回来。

那人一抬手将我拦住,说道:慢着,现在这碗里的钱是我的了,你若想要得经过我同意。

我怒极反笑,哈哈,你这乞丐还真赖皮。算了,我也不跟你计较了。我索性也坐了下来。

你好胳膊好腿的为什么不去当兵而做乞丐呢?坐下以后我问他。

那人反问我:好胳膊好腿为什么就不能做乞丐而非要去当兵呢?

他这么一问我倒愣了,当兵驰骋沙场,好男儿应当为社稷立汗马功劳,做乞丐低三下四的有何出息?

那人接着问我:那你说这大街上为何有如此多的乞丐呢?

我四顾了一下,向阳的墙角里坐满了形形色色的乞丐,想了一下答到:因为他们吃不上饭了呗。

那他们为什么吃不上饭呢?

我被问住了,本来我就不喜欢想问题,被他这么问来问去的我有点烦躁,反问他:那你说来看。

那人忽然长叹一声:连年战乱,民不聊生,青壮年劳力都被征走了,大片的土地荒芜,剩下一些老弱病残只能去做乞丐。
因为吃不上饭所以去当兵,而当兵的越多,就有越多的人吃不上饭。刘豫州也好,曹孟德也罢,袁绍、袁术、孙策、刘
表,他们谁得了天下我都没意见,只希望你们越早越好。

我愕然:照你这么说来,我们匡复汉室却跟曹贼谋权篡位没什么区别?

那人说道:匡复汉室?当年高祖斩白蛇揭杆而起反的是秦,你若匡复为何不匡复秦室呢?

我哑口无言,此言虽然大逆不道,却让我无从反驳。

曹孟德挟天子以令诸侯,世人都骂他为国贼,我们打着匡复汉室的旗号讨伐他,却不知我们捍卫的汉室也是一个反贼建
立的,那我们却真的是师出无名了。

大哥真的跟曹孟德没什么区别吗?匡复汉室真的无足轻重吗?我坐在人来人往的大街上却听不到任何声音。

\section{}

二哥最引人注目一是他的红脸,再一个就是他的胡子。

二哥的胡子足足有二尺长,而且特别顺,不跟我似的,乱蓬蓬的都找不到嘴。当年二哥在曹营的时候,上朝的时候皇帝
看见了,忍不住赞了一句:真乃美髯公也!虽说当时的皇帝只不过是一个摆设,但毕竟也是皇帝,说的话都是金口玉言,
于是美髯公这个称号不径而走,天下皆知。

二哥对自己的胡子异常的珍爱,每日都要细心地梳洗理顺,一般人洗脸很快就完事了,二哥洗一次要半个时辰。每到冬
天,他还要给胡子戴一个特制的口袋,两边有绳系在脖子后。因为冬天气候干冷,胡子特别容易掉。

不过二哥由于名气大,走到哪里都有百姓夹道观看,更有顽童冲上来撕扯胡须,按说胡子掉几根没什么,可到了二哥这
个年龄,当真是掉一根少一根了,于是二哥走到哪里一般都左有周仓右有关平,他们不是保护二哥,而是保护二哥的胡
子。

说起来也好笑,有一天吃饭,我看着二哥的胡子忽然想到一个问题,于是就问二哥:你睡觉的时候是把胡子放在被子里
面还是外面啊?

二哥愣了一下,歪着头想了半天,然后说:我还真没注意呢,你等我今晚回去留意一下明天告诉你。

第二天一早,我发现二哥的眼睛红红的,我吓了一跳,连忙问他:怎么了?是不是嫂子死了?二哥白了我一眼:死了还省心
呢,还不是怪你,昨天问的那个事,我晚上睡觉的时候就上了心,居然发现我把胡子放在被子里面也不舒服放在被子外
面也不舒服,折腾了一晚上也没睡着。

竟然有这种事?我忍不住哈哈大笑,正这时,军师来了,他听完以后也忍不住笑了,然后对二哥说:云长啊,我给你讲个故
事,说有一个老和尚和一个小和尚一起过河,恰巧遇到一个女人也要过河,女人都是三寸小脚,过河不方便,于是小和尚
就犹豫了,因为出家人是不近女色的,但出家人又要以慈悲为怀,正犹豫呢,忽见老和尚挽起裤腿背起女人就走。等过
了河,一路上小和尚都在想着这件事,最后终于忍不住问道:师傅,刚才那个女人\dldots 老和尚微微一笑说了一句:我都已
经放下了,你还放不下?

军师讲完了以后我和二哥都有点纳闷,这跟胡子有什么关系?军师接着对二哥说:你睡不着是因为心里想着胡子的事,
其实放在里面放在外面都是个习惯的问题,你这一追究,反而觉得不舒服了。你索性把它放下,今晚回去倒头就睡,什
么都不要想,早晨起床时你再看胡子在外面还是里面。

二哥连声称是,眉开眼笑地走了。

晚上我抱着酒坛子即将入睡的那一刻,想着军师白天讲的那个故事,忽然脑子里灵光一闪明白了一件事:一个人最严重
的病不是绝症,而是心病;一个人最大的敌人不是别人,而是自己。

\section{}

在我成年了以后,周围开始有很多人信一个人,家中供奉着他的名字:大贤良师张角。传说中张角得到神仙亲授的一本
《太平要术》,能呼风唤雨撒豆成兵,每到一处便散施符水,可以治百病。在很短的时间里手底下聚集了十几万信徒,
于是便开始造反,口号是``苍天已死,黄天当立''。个个头戴黄巾,称为黄巾军。打起仗来,个个身上贴满画符,口中念
念有词,当真是勇猛无比。后来大哥和我们便是因破黄巾军而扬名天下的,这是我始料未及的。

后来汉中出了个张鲁,他父亲据说也是个世外高人,平日里画点符,然后到处传教,凡入教者须交五斗米,于是朝廷称之
为米贼。汉中号称鱼米之乡,果然富足,很快他的手下也有十几万人,不过他没造反,只是在汉中一带称王,朝廷也嫌路
途艰辛没有派兵讨伐。等大哥进川以后,曹操出兵把张鲁给灭了。

张角和张鲁在很多人眼中都是神一般的人,而朝廷却说他们是装神弄鬼。后来证明他们的确不是神,但为什么会有那
么多的人追随他们呢?

子龙说,要想成事必须找个借口,喝了一口水以后接着对我说:这么说吧三哥,你走在大街上肯定不是见到一个人就打
吧?至少他踩了你一脚或者他瞪了你一眼。这跟造反的道理一样,老百姓吃不上饭必然要反,但一定要有个借口,比如
张角的天平要术,比如张鲁的五斗米。

听子龙这么一说,我忽然想起小时侯听大人说书,最常说的就是高祖斩白蛇起义的故事。说书的说高祖本来就不是凡
人,而是一条龙,还是红颜色的龙。说他喝多酒的时候身上有龙显现出来,我怀疑那可能是青筋,不过人家说的煞有其
事栩栩如生,还说什么他斩了白蛇以后,一个老太太当街而哭道:我儿白帝子,被赤帝子杀了。这摆明了真龙天子的意
思嘛,不过后来他真的当了皇帝,便不是真龙也是天子了。

按子龙的意思,这高祖当年斩白蛇也是个幌子了?忍不住又接着想了下去,如此说来,曹操的挟天子以令诸侯,孙权的世
居江东,大哥的汉室宗亲,都只不过是他们拉拢人心的幌子?

想到这里我隐隐有些不安。当年高祖起事成了,于是他被称为高祖,张角则是乱党。这便是军师经常说的成者为王败
者为寇吧?

那大哥会是王还是寇呢?这大概要等很多年以后才有人做结论吧?反正我是看不到了。很多人的一生都可以盖棺定论,但
有些人则盖上去又被挖出来然后再盖上去。

由此说来,其实做个普通人挺好的,至少死后很安宁。我总喜欢跟别人讲那个关于小草发芽的故事,因为总有人问我为
什么脾气如此暴躁。可几乎没有人听完以后明白我的意思,或许是我的表达能力太差了。

\section{}

有时候我搞不懂人活在这世上的意义,更搞不懂人和人之间的关系。比如大哥和二哥。军师说,子非鱼,安之鱼之乐?
可大哥又说,子非我,安知我不知鱼之乐?

我既不是鱼,也不是大哥,因此我什么乐都不知道,我只知道当第三碗酒还剩三分之一的时候我仿佛成了仙。

人活在这个世界上的目的就是寻找乐趣的。子龙对我说。

可乐趣在哪里呢?除了喝酒,我到哪里去找乐趣?

我看着子龙不辞辛苦地去山上采野花准备送给他新泡的妞;我看着魏延跟黄忠永不疲倦地斗嘴;我看着大哥和二哥相
视而笑;我看着军师衣衫凌乱地被夫人推出门外;我看着马超面带微笑地与士兵聊天;我看着阿斗趴在地上观察蚂蚁;
我看着张苞咧着嘴斗着蛐蛐。我突然发现我很寂寞。

他们说寂寞是高手的一种境界,是那种天人合一举世皆浊我独清的境界。可我不是高手,但同样寂寞。一个人独处时
的寂寞不可怕,可怕的是我在熙熙攘攘的人群中感到了寂寞。

我信步来到街上,漫无目的地走着,不知不觉出了城,又走了一会儿,看到前面有座独木桥,桥中间站了两个人,一个背
着一捆柴,腰里别着把斧子,看起来是个樵夫。另一个则挑着一副担子,看起来象是个挑夫。两个人就那么面对面站
着,一动不动,谁也不让谁。

我觉得有点意思,便找了块石头坐下来,看看到底是谁先认输。一柱香的时间过去了,两个人的脸上都见汗了。又过了
一柱香,身子都有点摇晃了。正在这时,突然一个人一路小跑地赶过来,冲那个樵夫喊道:二郎,快回家,你媳妇生了!那
樵夫听完以后身子没动,嘴上说道:不行啊,爹,我眼看就要赢了啊。却见后来的那个人走过去说:来,把柴给我,我替你
背着继续,你赶紧回家看孩子去。这时那挑夫发话了:慢着,这不公平,你等着,我也回家叫我爹去。

后来他们到底谁赢了我也没看,但着实让我的心情变得愉快了很多,快中午了,我得回去吃饭了。回到城里,听说大哥
中午请客,连忙赶过去,见众人已经坐好了等着开饭了,于是我也找了个座位坐下。吃饭的时候,魏延伸手夹了一个鸡
翅膀,不巧没夹住,掉在地上了,子龙在边上开口了:我说魏延,你喜欢吃鸡翅膀也用不着藏一块吧?你以为你藏在桌子
下面我们就不跟你抢了?魏延愣了一下居然反应奇快:没看我用脚轻轻踩着呢?你们抢不去的!嘿嘿\dldots 于是满桌的人
一起哈哈大笑。

从饭桌上下来,我突然发现我的心情好得要命,于是明白,生活中总有一些乐趣等你去发现。你不可能每时每刻都快
乐,但你可以努力把自己的心情调节到最接近快乐的那种状态。

蜀中气候潮湿,一年内难得见到几次太阳,来之前听人说蜀中的狗见到太阳都会感到很奇怪,以为是什么怪物,不停地
朝太阳狂叫。乍一听象是夸张,不过来了以后才知道确有其事。

连着下了几天的雨,好容易盼到雨停,却只能隔着灰色的云看到一个模模糊糊的太阳,即使这样也很难得了。子龙来约
我出去饮酒,据说城东新开了一个馆子,那里有道鱼头做得不错,于是我俩都换了便装,没骑马也没带随从,说着话溜溜
达达地步行过去。

快到了的时候,忽见一家门口晾了一床褥子,中间有一大片黄色的痕迹,想来是家中小孩尿床所致。我和子龙不禁相视
一笑,走过去后子龙突然又返回去,站在那里又端详了一会,我觉得有点奇怪,却见子龙笑道:三哥,你过来看,这象不象
西蜀地形图?我走近了仔细看了一下,忍不住哈哈大笑,果然很象!

说起西蜀地形图来,忍不住要说起一个人,此人姓张名松,是个土生土长的成都人,当初在刘彰手下官居别驾。提到这
个人总让我想起弥横,弥横是大脑袋细脖子长得挺吓人,张松是五短身材,尖嘴猴腮,獐头鼠目,不仔细看的话还以为是
哪个马戏团里跑出来的猴子。当年我见弥横的时候想下马揍他一顿,而我第一次见到张松时真想朝他脸上踹一脚。

就是这个张松,当年揣着一张西蜀地形图东奔西走,先到曹操那里准备把西川推销给曹操,结果差点让曹操给杀了,后
来遇到了大哥,于是西川四十一州都归了大哥。

说起张松见曹操跟弥横有点相似之处,弥横是裸衣击鼓骂曹操,张松没那么大的胆,但同样没给曹操好脸色。先是出言
顶撞,后来曹操领他去看兵马演习,想借此震一震张松,没想到张松不以为然,整个演习过程都是斜着眼看下来的(他眼
睛本来就不正,想不斜眼的话需要把脖子转好大的一个角度)。曹操有点恼火,吓唬张松道:我的大军所到之处,战无不
胜,攻无不取;顺我者生,逆我者死。张松连连点头说:是啊,曹丞相战必胜,攻必取,我早就听说了。比如濮阳攻吕布之
时,宛城战张绣之日;赤壁遇周郎,华容逢关羽;割须弃袍于潼关,夺船避箭于渭水。这都是无敌于天下的事啊。这下可
把曹操给气坏了,因为曹孟德一生打过很多胜仗,但也有几次惨败,差点儿连命也丢了。张松列举的,恰恰是曹操一生
处境最狼狈的几次。当下就要把张松给砍了,幸亏杨修拦阻才暂时把张松的脑袋留在他的脖子上。

弥横当年是持才傲物,张松虽说也有才,一目十行过目不忘,惊得曹操连他的孟德新书都烧了。但张松这么张狂却不仅
仅是自持有才,更重要的是他怀里有张西蜀地形图啊,西川四十一州画得清清楚楚,有了它取西蜀不过是囊中取物,所
谓奇货可居,因此张松目空一切。谁想到曹操也是个吃生肉的主儿,虽说当时曹操刚在赤壁被烧了个须眉皆无,但曹操
自命丞相坐镇许都,视天下皆为囊中之物,想来是不会为了区区四十一州而低三下四地去巴结张松的。其实张松的许
都之行很失败,表面上的任务是想让曹操帮忙攻打张鲁,暗地里的目的是想把西川献给曹操弄个官做做,到头来却是一
无所获。要不是后来遇到了大哥,他还得揣着图灰溜溜地回去做他的别驾。

按说现在我们坐在成都城里喝酒吃菜有张松的功劳,但实际上当年张松一出成都大哥就派人盯着他呢,他的一举一动
都在大哥的掌握之中,所以很多看起来偶然的事情其实都是必然的。

很多人说当年曹操冷落张松的主要原因是厌恶他的长相,这或许只是推测,但不能不说人的仪表有时候的确很重要,比
如刚才向我们推荐鱼头的那个店小二,如果他能把牙缝里的菜叶子和指甲里的黑泥清理一下,或许我们就不会换饭店
了

\section{}

曹操前不久刚刚病死,这个人一生的故事太多,我不想一一诉说,反倒想提一提曹操的儿子们。

曹操生性风流,大小老婆无数,因此儿子也颇多,有很多都默默无闻,不为人知,但出来混的几个却都天下闻名。

曹操的长子叫曹昂,乃曹操的原配刘夫人所生,后由二房丁夫人养大成人。长得一表人才,虽无什么过人之处,却也中
规中矩。由于是长子,所以理所当然地应该成为曹家王朝的继承人。可惜死的太早,他的死也比较冤。当年曹操南征
张绣,绣不战而降,本来是件挺好的事,没想到曹操竟然看中了张绣的婶婶,强行拉回营中与之作乐,张绣大怒,在一个
月黑风高的晚上偷袭曹营,如果不是曹昂把自己的马给了曹操的话,曹操早已是个死人了,而曹昂也被乱箭射死。当然
说起来当时还有一个比曹昂更冤的,那就是曹操的贴身保膘典韦。主子在里面行乐,他在寒风中守夜,喝了点小酒,吃
饭的家伙双铁戟竟然被人偷走了,弄了把单刀使不惯,最后没办法抓了两个尸体当双铁戟来用,最后被射得跟一只大刺
猬似的。

曹昂的死使得一人很高兴,这便是曹操的另一个儿子曹丕。因为曹昂死了曹丕便是长子。曹丕是个人才,聪明绝顶,见
识过人,其文治武功十倍于曹昂,按说合理地成为继承人曹操应该很开心,但事实上却不是这样。因为曹操还有一个更
加优秀的儿子,这便是赫赫有名的曹植。曹子建的文章名满天下,当年火烧赤壁之前,军师曾经拿着曹植的一篇文章去
戏弄周瑜,里面有一句``揽二乔于东南兮,乐朝夕之与共'',军师说这里的二乔便指周瑜和孙策的老婆,把周瑜差点给
气死。其实后来军师说这不是曹植的原文,原文是什么``连二桥于东西兮,若长空之锁殊'',不过军师也顺便说了一
句,说曹子建的文章天马行空,有着空前绝后的想象力。

由于曹丕和曹植都这么优秀,曹操欢喜之余还有点犯愁,因为只能选一个作为继承人,所谓鱼和熊掌不可兼得也。其实
这个问题本来不是什么问题,倘若曹操最小的那个儿子不夭折的话。那个夭折的天才儿童叫曹冲,曹冲七岁称象,满朝
皆惊!曹操一生中最喜欢的就是这个儿子,可惜天妒英才,曹冲十三岁便生一场重病死了。当时曹操痛不欲生,曹丕在
旁边劝父亲节哀,曹操悲痛之余竟然说了这么一句:此吾之不幸,而汝之大幸也!意思也就是说,如果曹冲不死的话你的
一切都是他的。

上面说的几个基本都是文人才子,而曹操却还有一个学武的儿子曹彰。一脸黄须,气力惊人,人称``黄须儿''。这个家
伙的确有俩下子,据说当年曾经手搏猛虎,最后拖着老虎的尾巴倒着走,老虎一点反抗的能力都没有。曹操对这个儿子
也是很喜爱,当年在渭河遇到马超的时候,马超勇冠三军无人能敌,曹操忍不住想起了曹彰,说了一句:吾儿若在此,倒
可以跟马超斗上几回。

曹操在我眼中一直以来是一个坏人的形象,不过听说他的死讯,竟然忍不住有一些失落。之所以提到他的这些儿子,是
想从一个侧面来描述一下曹操。人们常说,龙生龙凤生凤,老鼠的儿子会打洞。倘若只有一个儿子出类拔萃,或许是偶
然,但曹操的儿子个个都如此优秀,仅从家教这方面来说,不由得让人对曹操肃然起敬。

听手下人来报,说曹丕已经自立魏王,改建安二十五年为延康元年。忍不住有些感慨,曹操一生挟天子以令诸侯,却终
未篡权,现下他尸骨未寒,他的后代已经称王称帝了。而子龙最近几天则一直在念叨曹操生前的一句话:设使天下无有
孤,不知当几人称帝,几人称王?

\section{}

连着几天阴雨,道路泥泞,蜀道本来就难走,这下更不好走了。有一天我看到一个探子,四处找工匠做一副高跷,我觉得
很奇怪,就过去问他,那探子愁眉苦脸地对我说:将军有所不知,现在那路上一脚下去能带起五斤泥,根本没法走,我估
计踩个高跷能快一些。

连轻装步行都这么难,更别说那些负责运输的了,粮草啊武器啊各种军需是进不来出不去。

大哥拉长了脸摆弄着他那两只大耳朵,他郁闷的时候总是这个样子。连军师似乎也束手无策。平时只要大哥脸一长,
军师便凑过去慢吞吞地来一句微臣有一计之类的话,然后大哥便眉开眼笑,而军师也一副踌躇满志的样子。但现在不
行了,军师便是有天大的本领也无法把西蜀的山路都变成平路,把西蜀的泥道都铺上石板啊。

不过军师就是军师,这世界上没有什么人什么事可以难倒军师,除了他老婆。军师找了一批工匠,画了一些图纸,命他
们各自照样去做,几天以后,组装起来,大概是一头木牛的样子,用手一掰耳朵,便启动里面的机关,木头牛竟然能迈步
走路!真是神奇啊!

木牛做出来以后大伙儿纷纷来看,除了张大嘴巴赞叹之外没有什么别的表情。其中一个老木匠饭也不吃觉也不睡研究
了三天三夜,最后说了一句:这简直比鲁板发明的锯子还要伟大啊!丞相真乃神人下凡呀!

军师很得意,他这次得意的表情甚至比气死周瑜的那次都要明显,不过他的确值得得意,因为他发明了如此一件了不起
的东西。

晚上魏延陪我喝酒的时候突然冒出一句:三哥,你说咱也弄出点东西来给大伙瞧瞧,也在青史上留个名行不?我当时正
晕晕忽忽的,听他这么一说,嘿,听起来似乎不错嘛。于是我们哥俩各自去忙活了。

我本来就是个不愿意动脑子的人,最近这几天为了搞发明我把一辈子的脑子都用了,结果还是什么也没想出来。不搞
不知道,做起来我才发现,能发明的东西几乎都已经被人发明了,没有我看不到的,只有我想不到的。这下把我给愁坏
了,张苞见我如此伤脑筋,就也坐下来跟我一起想,唉,有其父必有其子啊,我也没指望他能想出点儿什么来,不过他这
份孝心让我很安慰。

这世上的事都没有绝对的,隔了几日,张苞满脸兴奋地来找我,对我说:爹,我发明出东西来了!你快来看!我将信将疑地
被他拉到后花园,见张苞手里拿着一个形状奇怪的东西,有点象月牙,一头是把柄,一头很锋利。张苞给我解释说这是
一种暗器,是人类历史上最伟大的暗器!它飞出去以后还能绕回来!真的假的?我越发地怀疑了,张苞说爹你退后我给你
演示一下。说完他拉开架式,对准前面的一片野花扔了出去,但见那东西呼啸着飞过去斩落一朵野花之后真的转了一
圈往回飞,我正惊讶之间,却听张苞一声惨叫,定睛一看,见那东西直直地扎在张苞的右肩膀上,血流如注。

有了张苞的这次,我更加对发明东西灰了心,没想到魏延又颠颠地来找我,一进门没说话先喜笑颜开,双手放在背后神
秘兮兮地对我说:三哥,我成功了!说完从背后取出一物朝我得意地晃着,仔细一看,一把黑色的雨伞,切,这也算发明?
我一脸的不屑,却听魏延说道:三哥,这不是一把普通的雨伞!它能自动打开!你跟我出来试一下。我跟着魏延来到院子
里,天气不错,阳光明媚的,魏延把伞举起来对着太阳,只听咯的一声果然自动打开了!我连声赞叹,真的不错啊,魏延你
怎么弄的?魏延此时的表情象一只骄傲的公鸡,卖了半天关子才给我解释说是利用太阳的能量让伞自动打开的。

送走了魏延,我有点郁闷,魏延的发明成功虽然是件高兴事,但也从一个方面证明了我的确比他蠢,坐在那里情绪有些
低落,忽然脑子灵光一闪,想起一件事,魏延的伞只能在有太阳的时候用,可是有太阳的时候谁打雨伞啊?我拔腿想去告
诉魏延,转念又一想,还是不告诉他了,让他美一阵子吧,反正早晚他也会发现的,他发明的不过是一件废物,想到这里
忍不住哈哈大笑。

晚上临睡觉前我想明白了,发明这种事是聪明人做的,这世界上有好多好多的聪明人,他们的脑子里有好多好多希奇古
怪的想法,而象我们这种蠢人能做的就是好好享用他们的成果,这也算是对他们最大的肯定了吧。

\section{}

今天子龙讲了一个故事,说有人把一个人关在一间不透光的屋子里,在墙上挖一个小洞,把那人的胳膊拽出来,然后用
刀子在他的手指头上划一小口,上面挂一盛满水的木桶,钻一小眼,让水缓慢地一滴一滴地落到下面的一个盆里。过了
一夜,屋里的那个人便死掉了,面色苍白象是死于失血过多,但其实他手上的伤口只流了一滴血而已。

我听完以后觉得有些毛骨悚然,试想一下,在一个黑咕隆咚的屋子里,听着外面滴答滴答的声音,感觉自己的血正在源
源不断地流失,的确是件恐怖的事情。

子龙说这是一种最阴险的杀人方法,因为你在尸体上几乎找不到任何痕迹。杀人者利用的是一种巧妙的心理暗示。

说起心理暗示,不由得又想起曹操。当年曹操带兵去讨张绣时,正值夏天,骄阳当空,军士们个个都汗流浃背。走了几
十里的山路,途中遍寻水源不见,所有人都口干舌燥,军士们怨声载道,有几个胆大的索性一屁股坐在地上不走了。眼
瞅着军心涣散,曹操忽然心生一计,命人传令下去:当年我来过此地,前方不远处有一梅林。众军士一听,不由得精神大
震,心里想着酸酸的梅子,哈喇子便流了出来。你要知道在你嗓子冒烟的时候咽一口唾沫是非常爽的。最终曹操的军
队顺利地前进并找到了水源。

后来人们把这件事叫做望梅止渴,这也是一个心理暗示的现象。虽然没有放血的那个精致巧妙,但却是信手拈来,实际
上远比处心积虑杀人的那个高明得多。

有一天,我说好了请魏延来家里喝酒,一大早我在客厅的茶几上放了一盘绿豆糕,然后坐在椅子上等客人上门。吃饭喝
酒的事魏延总是很积极的,来了以后我安排魏延坐下,然后说,你先坐会儿,我到后面方便一下。我故意在后堂磨蹭了
一会,出来时果然发现魏延正端着茶杯吃绿豆糕呢。于是连忙抢上前去作大惊失色状:哎呀,这绿豆糕是我命医师特制
的,最近我有点便秘,于是他放了一些巴豆在里面。话说完不大工夫,就见魏延捂着肚子龇牙咧嘴地往后面跑,不到一
盏茶的时间他去了六次茅厕。到后来我实在憋不住了,哈哈大笑地对他说了实话:那绿豆糕只是普通的绿豆糕,其实里
面什么东西都没放。魏延傻愣愣地盯着我,脑袋摇得跟拨浪鼓似的:你就甭蒙我了,没放巴豆?没放巴豆我能拉那么稀?

军师说,心理暗示其实是一把双刃剑,用好了便是动力,用得不好也就成了压力。在很多时候,我们都会受环境影响自
己给自己一些心理暗示,也就是说,有很多事其实是自己想出来的。

\section{}

有天早晨我起的比较早,于是便独自去巡营。此时天色微明,军士们有的已经起床了,有的则还在酣睡。转了一圈,没
发现什么异状,正想回去,忽然听左前方有人在大声地唱歌:小小姑娘,清早起床,迎着阳光上茅房~~~~那歌声嘹亮雄
厚,余音袅袅,我不禁暗暗称奇,于是便顺着声音走过去。

前面是一个简易的茅厕,声音就是从里面传出来的,我站在外面等了一会儿,见一人提着裤子从里面出来,不由得大吃
一惊,见此人五短身材,面黄肌瘦,与刚才那歌声全然无法联系在一起,于是我就冷不防地问了一句:刚才那歌儿是你唱
的吗?那人吃了一惊,转头一见是我,连忙立正,说道:是的,将军!他这一张嘴又把我吓一跳,我嗓门本身就够大的了,当
年在长阪坡我差点把肺给喊出来。可眼前的这个小矮子比我嗓门还要大还有洪亮,我觉得有点意思,于是把他带回帐
中。

回到帐中,经过询问得知此人姓刘,家中排行老三,人称刘三。我问他:从小嗓门就这么大吗?刘三答道:本来也不怎么
大,我小时侯赶上我们那里闹饥荒,家中没什么吃的,后来实在饿得不行了,我就偷偷地跑到田里去抓蛤蟆吃,吃了一个
夏天,从那以后说话嗓门就大了,想小都小不了。

我忍不住哈哈大笑,原来是吃蛤蟆吃的啊?怪不得呢,我以前见过一种小蛤蟆,叫起来跟牛似的,再看看面前的这个矮
子,粗脖子,大嘴巴,趴鼻梁,两只眼睛向外突出,确实有点象蛤蟆。

不管怎么说,嗓门大也是人才啊,我便把他留在身边做了个传令兵。不服不行,有时候我下一个十万火急的命令,他索
性就站在军营中间仰脖那么一喊,方圆几里地的人都听得一清二楚。有一次军师来视察,见识了一下,也不禁惊为天人。
回去跟大哥一说,大哥也想见识一下,便下旨把刘三调到城里去,做大哥的传令官,想不到这小子凭着一副大嗓门居然
平步青云,周围的兵士们也都羡慕不已。

本以为从此刘三就在城里吃香的喝辣的了,没成想不几天他便又被大哥派回来了。原来进城以后,大哥让刘三在早朝
晚朝的时候喊一喊,早朝倒罢了,赶上晚朝,刘三站在殿前,双手叉腰这么一叫,就听得满城的犬吠鸡鸣,小孩啼哭不止,大
人怨声载道。于是大哥赶紧把他给弄回来了。

那年张郃来犯,我带兵与之相拒在瓦口隘,张郃凭据山险,闭门不出。一连几天攻不下来,我有点着急,晚上喝点酒后突
然想了一个主意。第二天便把刘三找来,让他站在山下对着敌营叫骂,这刘三甫一开口,便震的遍山的鸟儿齐飞,冬眠
的蛇都被惊醒了,纷纷爬出洞口张望。骂的词也是他自己编的,蛮有意思的。什么新一代的乌龟新一代的人新一代的
王八不敢出门,还有什么小小张郃躲在茅厕迎着阳光流鼻血\dldots

如此骂了没俩时辰,张郃受不了了,命军士一起鼓噪,试图把刘三的声音给盖住,可他不知道刘三是谁啊?是吃蛤蟆长大
的主儿啊,千军万马中刘三的声音直入云霄,清晰可闻。最后张郃终于坚持不住了,引一队人马出来厮杀,结果中了我
的埋伏,大败而归。

大哥后来论功封赏,我给刘三请了一大功。军师笑着说,所谓知人善用,翼德此番做得不错啊!

其实我也不知道什么叫知人善用,我只知道尺有所长寸有所短,其实每个人都有自己的长处,关键看你怎么用和用到哪
方面了。

\section{}

今天演习阵法的时候,有个军士的头盔不小心掉在地上,竟露出一头白发,而看他的相貌不过三十左右,于是众人都哄
然而笑。

军师也忍不住莞而,说道:莫非阁下乃伍子胥转世?

我听着伍子胥这个名字好耳熟,但却想不起他是干什么的,于是演习完了便去找子龙,却没想到竟从子龙那里听到了一
种前所未闻的见解。

伍子胥,楚国人,在历史上赫赫有名,是个顶天立地的角色。书中写他身长一丈,腰大十围,眉广一尺,目光如电,有扛鼎
拔山之勇,经文纬武之才。在春秋战国时期,端的是一号人物,几乎所有的书中提起他来都是一片赞誉之词。

按算命的说法,两眉之间的距离代表着一个人的气量,眉距越宽,心胸越宽广,反之则越狭窄。照此说法,伍子胥眉广一
尺算是宰相肚里能撑船的主儿了。说到这里子龙话锋一转,然而就我分析,很多事实证明,伍子胥乃是一个不折不扣的
小人,尤其是其心胸狭窄,简直令人发指。

我忙问为什么,子龙给我讲道:

伍子胥的父亲被当时的楚平王囚禁,其父手书一封令伍子胥哥俩前来面君,否则将被平王斩首。伍子胥认为去了必死,于
是坚决不肯去,他的哥哥说,如果咱们俩不去的话,父亲肯定会死,这是父亲的亲笔书信,我一定要去,否则就是不孝。
于是伍子胥与其兄断绝兄弟关系,自己一个人逃难去了。在离家之前,他觉得自己的妻子是个累赘,于是他把他的妻子
给勒死了。

军师今天所提到的是一个流传甚广的典故:伍子胥过韶关,一夜白头。说当年的伍子胥逃到韶关,城门处都悬挂着画
像,过往行人一一盘查,伍子胥自认插翅难飞,于是一夜之间愁白了头。伍子胥堂堂一介丈夫,号称智勇双全,你可以想
方设法用计谋过关,实在过不去,便是拼得一死,也要豪气凛然,居然会一夜白头?着实可笑。

再往下走,伍子胥做的事就开始越发难以自圆了。出了韶关,前面有条大河,后面追兵已经赶到,此时有个渔翁用小船
救了子胥一命,子胥过河后,怕渔翁泄露他的行踪,于是拔剑杀了渔翁。

行至溧阳,伍子胥饥饿难耐,见一女子在河边洗衣服,于是上前讨食,那女子便给他吃了个饱,子胥这次也没有例外,依
然将其抛尸河中,防止泄露踪迹。

至于后来伍子胥终于利用两个刺客报了仇,这两个刺客在历史上也很有名,一个叫专诸,一个叫要离,其实不过是两个
亡命徒,相当于他养的两条狗而已。专诸倒罢了,要离行刺的过程简直就是变态。他自知不是庆忌的对手,于是自己出
了一个主意,要吴王把自己的妻子给杀了,然后斩断了自己的右手,以此换得庆忌的信任,最终在船上杀了庆忌,自己也
不免一死。

当然书上记载的都不是这样说的,他的妻子、渔翁、洗衣女子都是自杀的,他的妻子倒是有可能自杀,但后两个却纯属
胡说八道,因为无论从哪个方面来讲都讲不过去。当年曹操与陈宫结伴出逃时,也曾在路上杀了两个人,于是我们都骂
曹操心胸狭窄,而伍子胥则是正面人物,于是坏人都是他杀的,好人则都是为了他而自杀的。

子龙越说越愤愤不平,都说历史是人民写的,但实际上大多数人民是不识字的!

而我听着却有些糊涂起来,照子龙的说法,那很多前人记载下来的东西都不可信?

于是子龙给我举了一个浅显易懂的例子,我觉得非常有趣。

子龙是用射箭来打比方的。

拿着弓箭比划一下,然后把箭给手下人,让他跑步到靶前插上。\myrule 这是历朝历代的皇帝。

跑过去把箭插在靶心上。\myrule 这是历朝历代的功臣。

跑错了方向或者跑过去插歪了。\myrule 这是历朝历代的庸臣。

见人家快要把箭插上去的时候,在背后突施冷箭将其放倒。\myrule 这是历朝历代的奸臣。

在自己亲戚射出的箭周围画一个圈,标明:靶心。\myrule 这是历朝历代的历史学家。

在自己喜欢的人射出的箭上挂一个死兔子或者去了毛的鸡。\myrule 这是历朝历代的评论家。

把评论家挂上去的兔子或鸡换成烤牛肉或者酱猪蹄。\myrule 这是历朝历代的文学家。

那你是什么家呢?我忍不住笑着问子龙。

我?子龙想了一下,说道:把所有的箭都拔了,然后让当事人当着我的面再射一次。\myrule 这就是我,一个梦想家。

\section{}

二哥死了。

他的尸体躺在麦城的荒郊,而他的头则埋在洛阳城的南门。

他的赤兔马被一个叫马忠的人骑着,他的青龙偃月刀被一个叫潘璋的人拿着。

我最近一次见他是三个月以前,他一个人在荆州待了很久,我很想念他,于是星夜跑去见他,他表面上虽然不动声色,但
我知道他见到我很开心。我走的时候他送我送了很远,我记得他说,三弟,咱们都老了。这世界已经不再是咱们的世
界,这天下也不再是咱们的天下了。

他说这话的时候,风吹着他的胡须,有些凌乱。

大哥哭得晕过去好几次,我没有哭,我静坐了好几天,脑子里一片空白,什么都没想。周围的人都不敢靠近我,可能是我
的脸色太可怕。后来我饿了,于是找来东西吃,却发现连豆腐都咬不动了,原来这几日我竟然一直咬着牙。

他们说二哥死后成了神,我不知道这世界上有没有神,我也不指望二哥的在天之灵能保佑我什么,倘若他真的活在另一
个世界上的话,我只希望他能开心。

晚上我一个人坐在帐外,抬头看着南方的星星,正值冬天,星星看起来很遥远,模模糊糊想起一句话,遥远的地方真的一
无所有。寒风吹过来,四周的山有黑色的轮廓,隐约有狼的嚎声,我扯开衣襟,仰天长啸了一声,隔了很久,却没有回音。

酒是好东西,他可以让我忘掉很多无法忘掉的事情。所有的悲伤和喜悦都被酒精所稀释,在半醉半醒之间我仿佛到了
另一个世界。二哥在那里,坐着看书,见到我只是微微一笑,我喜极而泣,他轻轻地对我说,三弟,好想回去再看一眼家
乡那桃花。

我知道这是梦,但我希望我永远不要醒来。

我看着大哥红肿的双眼以及两鬓那苍苍白发,突然觉得他很可怜,他比我更了解二哥,也比我更加悲痛。我不是鱼,因
此我不知道鱼的快乐也不知道鱼的悲伤。

大哥哭够了以后拍着桌子要去报仇,相反我却表现得很冷静。突然之间我对生死有了另一种看法,很多年前,我在锦屏
山上遇到一个异人,道号紫虚上人,据说他能知人生死贵贱,于是我便去见识了一下,老道却只送了我一句话:生有何
欢?死有何苦?直到今日我才领悟到这句话的含义,可惜已经晚了。

回到军中,我把平日里打的最多的两员末将范疆、张达找来,命他们三日内备齐白旗白甲,否则满门抄斩,见二人面有
难色,我便叫军士把他们绑在树上痛打了一顿。临走时我用眼角的余光清楚地看到他们的恨意。

仇恨也是个好东西,它能促使人做出很多意想不到的事情。猛然间我想起了那个眼神跟锥子似的少年纪同,不知道为
什么,他再也没来找过我,但我知道,只要我不死他不死,总有一天他会找上来的,忽然之间我很渴望他现在来。

然而他终究没有来,来的是范疆、张达,我睁大了眼睛,据说如果刀快的话人临死时可以看到自己的心。

刀不是很快,但很锋利,我清楚地感觉到了冰冷的刀锋没在骨肉里,象一条凉凉的蛇。血飞溅出来,在半空中竟似凝固
了,在陷入黑色空间的一刹那,我清楚地看到了一树桃花,我知道,那就是家乡的那树桃花。

\clearpage

\section{后记}

{\sf \small

很多朋友看完我的流水帐后总要发一些感慨,有人说,真的好搞笑,笑得我肚子疼。还有一些人说,好悲伤,看得我心里
酸酸的。

其实我不知道是前一部分朋友的观点正确还是后一部分朋友的理解深刻,因为我写的时候并没有在一种特定的什么基
调下进行。其实我一直把写字做为一种消遣,有东西憋着的时候,不吐出来是很难受的。因为没有条件写得华丽,尽量
通顺便好了,前提是想法要真实,倘若能稍微的加点有趣,那我自己就很满意了。

张飞流水帐在性质上其实应该归入无厘头一类,但又不是单纯的无厘头,里面夹杂了我对人生的一些思考。有人说,这
里有王家卫的影子,又有人说,似乎有王小波的痕迹,但实际上我不姓王,我比他们少一横。由于当初它是一种连载的
性质,又有历史背景的局限性,所以它其实比一般的随笔要难写一些。开始的几篇我有些信马由缰,后来我慢慢开始变
得郑重起来。当然每一篇我都是很严肃地写出来的,写东西不是一件很轻松的事情,在此对所有在网上或在纸上从事
认真创作的人们表示敬意和尊重。

这世界上任何一样东西都会有人喜欢有人骂,每个人的审美观与出发点失之毫厘,对一件事物的看法可能会谬以千里。
我们不会指望所有人的看法一致,世界大同那是共产主义的事,可望不可及。但我们需要接受所有不同的声音,汇百流
方成江河。在此对看了我的作品感到反胃的朋友说一声抱歉。

在写这篇东西的时候,我经历了事业以及家庭上的种种挫折,我虽然一直是笑着面对这个世界的,但终于不能做到坦然。
杜甫说,文章憎命达。马克吐温说,幽默的内在根源不是欢乐,而是悲哀;天堂里是没有幽默的。

于是我用这两位中外名人的话来给自己戴一顶高帽,这看起来似乎有点阿Q。

有一类电影注定不是用来娱乐大众的,就像有一些国家注定没有面目,有一些河流注定没有名字,有一些人注定只能张
大嘴巴却发不出声音。

这是一篇影评中的几句话,给我的触动很大。我总认为我们之中的大多数都属于张大嘴巴却发不出声音的人。

我比较幸运,在朋友的不懈鼓励下写完了这个东西,发出了自己的一点声音。我知道这声音微不足道,但倘若能引起你
的一点点共鸣,那便已经超出了我写这个东西的初衷。

我总认为,世界上只有三种人,一种人开心,一种人不开心,另一种人不知道自己开心不开心。我希望所有看到这本书
的读者朋友成为第一种人,不要成为第二种人,更不要成为第三种,跟我似的。

最后的最后,送朋友们一句话,当你睁大眼睛却发现自己什么都看不到的时候,不要以为是自己瞎了,或许,前方真的是
一无所有。
}

\end{document}