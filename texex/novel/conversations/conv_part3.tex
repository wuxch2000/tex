%%% Local Variables:
%%% mode: latex 
%%% TeX-master: "conversations" 
%%% End: 
\section{Abstract Factory, Template Style}

\textit{``Do we know what it is?'' I gestured at the device lying on the cold metal table before us. It had been one of the
  first to have been retrieved by the recon parties exploring the immense building buried below us in the ice.}

\textit{Jeannine shook her head. ``Nyet, tovarisch. It could be a toaster, a hyperdrive motor, a child's toy, or just junk.
  Last week we thought we had the power system worked out, for this and the other devices, but when we ran the juice through
  them nothing happened. The techs are back to the drawing board as far as the power goes, but they're hopeful that they'll
  get it right soon.''}

\textit{``And what makes them think they will?''}

\textit{``Hope.''}

\textit{I glanced at the clock on the wall of the chamber. ``Well, it's nearly time. I'm beat, and I'm freezing. Every time
  we ask the recon teams to give us something new to work on, they produce something new and different. I just wish we could
  understand even one of these.''}

\textit{``I wish we knew more about what was coming out of our little historical 'artifact factory' here,'' Jeannine
  agreed.}

\textit{``That reminds me of something that happened to me back when--''}

\textit{Jeannine rolled her eyes. ``Tell me when we get someplace warm.''}

\wuxchcenterline{}

``Bad news,'' I called glumly over the cubicle wall to Wendy.

``Wazzup?'' she gophered up.

``You know that class I worked on last week?''

``Not intimately, but do go on.''

I ignored that. ``Well,'' I continued, ``I have to create a class factory for it. The Guru suggested I look at the class
factory the client group wrote.''

``Yeah, so?'' Then a sudden look of dismay crossed her face: ``Oh. Bob wrote that, didn't he?''

I nodded glumly. ``Well, one thing I'll give him credit for -- every time I crack open his code, I learn how not to do
something.'' Wendy giggled and sank back into her chair.

I sighed and checked out the source code. It wasn't quite as bad as I had anticipated. There was only one humongous series
of cascading if statements -- I had expected a lot worse. Still, it was pretty hairy code, so with the Guru's blessing I set
about implementing an Abstract Class Factory \footnote{Gamma, Helm, Johnson, Vlissides. Design Patterns: Elements of Reusable Object-Oriented Software (Addison Wesley, 1995)}. Absent any requirements for multithreading or other concurrency, I decided
to implement the factory as a singleton:

\begin{footnotesize}
\begin{lstlisting}
class BaseFactory
{
    typedef std::auto_ptr<Base> (*BaseCreateFn)();
    typedef std::map<std::string, BaseCreateFn> FnRegistry;
    FnRegistry registry;

    BaseFactory() {}
    BaseFactory(const BaseFactory &); // Not implemented
    BaseFactory & operator=(const BaseFactory &); // Not implemented

public:
    static BaseFactory & instance() { static BaseFactory bf; return bf; }

    bool RegCreateFn(const std::string &, BaseCreateFn);

    std::auto_ptr<Base> Create(const std::string &) const;
};

bool BaseFactory::RegCreateFn(const std::string & className, BaseCreateFn fn)
{
    registry[className] = fn;
    return true;
}

std::auto_ptr<Base> BaseFactory::Create(const std::string &className) const
{
    std::auto_ptr<Base> theObject(0);
    FnRegistry::const_iterator regEntry = registry.find(className);
    if (regEntry != registry.end()) 
    {
        theObject = regEntry->second();
    }
    return theObject;
}
\end{lstlisting}
\end{footnotesize}

In the Base implementation file, I added:

\begin{footnotesize}
\begin{lstlisting}
namespace 
{
    std::auto_ptr<Base> CreateBase()
    {
        return std::auto_ptr<Base>(new Base);
    }

    bool dummy = BaseFactory::instance().RegCreateFn("Base", CreateBase);
}
\end{lstlisting}
\end{footnotesize}

``OK, that's cool,'' I thought. ``You register a function with the factory, and creating an instance is as simple as one, two,
three:''

\begin{footnotesize}
\begin{lstlisting}
int main()
{
    std::auto_ptr<Base> anObject = BaseFactory::instance().Create("Base");
}
\end{lstlisting}
\end{footnotesize}

I then proceeded to create a derived class, to test creating it via the factory. In the derived class's implementation file,
I added:

\begin{footnotesize}
\begin{lstlisting}

namespace 
{
    std::auto_ptr<Derived> CreateDerived()
    {
        return std::auto_ptr<Derived>(new Derived);
    }

    bool dummy = BaseFactory::instance().RegCreateFn("Derived", CreateDerived);
}
\end{lstlisting}
\end{footnotesize}

But the compiler stopped me cold -- it complained that it couldn't convert CreateDerived to the correct type. After puzzling
over it for a moment, I remembered the problem I had experienced previously with implicit conversions to pointers-to-Base
\footnote{Jim Hyslop and Herb Sutter. ``Conversations: Roots,'' C/C++ Users Journal C++ Experts Forum, May 2001,
  http://www.cuj.com/experts/1905/hyslop.htm}. I realized I had just run into another situation where the compiler couldn't
implicitly convert the pointers, so I had to rewrite the create function slightly:

\begin{footnotesize}
\begin{lstlisting}
namespace 
{
    std::auto_ptr<Base> CreateDerived()
    {
        return std::auto_ptr<Base>(new Derived);
    }

    bool dummy = BaseFactory::instance().RegCreateFn("Derived", CreateDerived);
}
\end{lstlisting}
\end{footnotesize}

Looking back at the base class, I realized that the registration code was almost identical, so I created a macro to wrap it
up:

\begin{footnotesize}
\begin{lstlisting}
#define REGISTER_CLASS(BASE_CLASS, DERIVED_CLASS) \
namespace \
{ \
    std::auto_ptr<BASE_CLASS> Create##DERIVED_CLASS() \
    { \
        return std::auto_ptr<BASE_CLASS>(new DERIVED_CLASS); \
    } \
    bool dummy=BaseFactory::instance().RegCreateFn( \
        #DERIVED_CLASS, Create##DERIVED_CLASS); \
}
\end{lstlisting}
\end{footnotesize}

Using the macro was quite simple:

\begin{footnotesize}
\begin{lstlisting}
REGISTER_CLASS(Base, Base)
\end{lstlisting}
\end{footnotesize}

``Not bad for a few hours' work,'' I muttered when I was done.

``Yes, indeed, my child. But you can do better.''

I jumped yet again at the Guru's soft voice behind me. ``Wha--?''

``Macro expansions like that,'' she explained, ``are difficult to read and understand, and macros are not type safe. Also,
what is the purpose of the bool dummy variable?''

``Well,'' I said, slightly defensively, ``it was the only way I could think of to ensure that the registration function got
called automatically. I wanted to get away from the typical factory code that requires a modification each time you add a
new class.''

She inclined her head. ``Wise thinking, my child. However, your factory is not very generic. We are dealing with more and
more classes that require abstract factories. I want you to create a factory that will handle, not require, modification for
each class hierarchy it will be used on.''

``A generic abstract factory? That's a pretty tall order, isn't it?''

``My young apprentice, you are not afraid of a challenge, are you? Experience is by industry achieved\ldots and perfected by
the swift course of time,'' I finished \footnote{William Shakespeare. Two Gentlemen of Verona, I iii 22}. ``No, I'm not
afraid of the challenge, but I don't want the swift course of time to turn the schedule aside.''

``You are most of the way there already. All you need to do is think generically,'' she observed.

``Think generically\ldots as in, templates!'' I exclaimed and turned back to my monitor. A few minutes later, I glanced
behind me and saw that the Guru had disappeared again as silently as she had appeared. I shivered, but only slightly, and
went back to work.

After a short while, I came up with a factory template:

\begin{footnotesize}
\begin{lstlisting}
template <class ManufacturedType, typename ClassIDKey=std::string>
class GenericFactory
{
    typedef std::auto_ptr<ManufacturedType> (*BaseCreateFn)();
    typedef std::map<ClassIDKey, BaseCreateFn> FnRegistry;
    FnRegistry registry;

    GenericFactory();
    GenericFactory(const GenericFactory&); // Not implemented
    GenericFactory &operator=(const GenericFactory&); // Not implemented

public:
    static GenericFactory &instance();

    void RegCreateFn(const ClassIDKey &, BaseCreateFn);

    std::auto_ptr<ManufacturedType> Create(const ClassIDKey &className) const;
};
\end{lstlisting}
\end{footnotesize}

I realized that not all classes would necessarily need a std::string as the key, so I made the key type a parameter to the
template. The implementation of each function was the same as I had used for BaseFactory, except for the registration
function. I came up with a more elegant template for that:

\begin{footnotesize}
\begin{lstlisting}
template <class AncestorType, class ManufacturedType, typename ClassIDKey=std::string>
class RegisterInFactory
{
public:
    static std::auto_ptr<AncestorType> CreateInstance()
    {
        return std::auto_ptr<AncestorType>(new ManufacturedType);
    }

    RegisterInFactory(const ClassIDKey &id)

    {
        GenericFactory<AncestorType>::instance().RegCreateFn(id, CreateInstance);
    }
};
\end{lstlisting}
\end{footnotesize}

Now, each class derived from the base class simply had to add one line to get a type-safe registration of the creation
function:

\begin{footnotesize}
\begin{lstlisting}
RegisterInFactory<Base, Base> registerMe("Base");
\end{lstlisting}
\end{footnotesize}

The RegisterInFactory template's constructor registers the name of the class in the creation registry \footnote{The complete factory file, along with a small driver program, is available on the CUJ website at hyslop.zip}.

I glanced behind me, just in time to see the Guru approaching. I smiled to myself; for once I was slightly ahead of her
game. ``Very good, my apprentice,'' she said, coming up behind me. ``Your factory is generic, portable, and does not rely on
registration tables, DLL sentries, or other cumbersome techniques.''

``There is one drawback, though,'' I interrupted. ``Because the registration relies on a static object being initialized,
there is no guarantee that all the creation functions will be registered if you use the factory before main begins
execution.''

``Exactly, my apprentice, your factory is subject to the Static Initialization Order Fiasco, as explained by the prophet
Cline \footnote{Marshall Cline. C++ FAQ-Lite, http://www.parashift.com/c++-faq-lite/}.'' She turned to leave.

``Wait a second,'' I called. The Guru turned back. I was trying to figure out a graceful way of asking the question that was
on my mind. The Guru patiently tucked a lock of her hair behind her ear while she waited. ``About Bob. If he's such a bad
programmer,'' I managed to say, ``how come\ldots''

``Why is he still here?'' the Guru supplied. I nodded. The Guru looked down in thought. ``Have you noticed that Bob hangs
out with the execs? Senior management thinks he's some sort of golden boy and can do no wrong. Because of our\ldots um\ldots
past, Bob has them convinced that any of my complaints about his job performance are just the petty grievances of a bitter
ex-wife. And, according to him, anyone who agrees with me has been drawn into my little plots against him -- his words, not
mine.''

``Man, that must be tough. Why do you put up with it? You could find a new job like that.'' I snapped my fingers.

She shrugged. ``Other than Bob, I like it here. Besides, if I left, I wouldn't get the satisfaction of bugging the heck out
of him with this act.'' We both laughed. ``Seriously, though,'' she continued, ``this company has a lot of opportunities for
career growth. For example, our new biomedical device division is going to need a head of software development -- I've
applied for the position.''

``Cool! Good luck,'' I said. The Guru nodded her thanks. I turned back to my workstation, to put the finishing touches on
the abstract factory template.

It was late at night a few days later when the news came, but I didn't find out about it until I got up the next morning.
Still bleary-eyed, I was walking into the mess when the unusual level of noise made me put my hands over my ears and wince.
``What's with all the chatter?'' I muttered grumpily at the dozen people who were already there, talking excitedly. ``It's
too early for you guys to be this perky.''

It was Major Gilb who answered, smiling. ``We think we've got it, my boy, we think we do.''

``What?'' I grumped through mental cobwebs.

``Looks like the power's on,'' Jeannine said simply.

\section{How to Persist an Object}

\textit{It was 00:10 local Europa Base time when I slumped down the corridor. I was exhausted; three days of afternoon watch
  were getting to me.}

\textit{A small beam of soft light spilled out across the cool metal corridor. It was coincidence, I think, that Major
  Gilb's door was slightly ajar, although the fact that the ventilation system had been suffering lately probably also had
  something to do with that. It was more coincidental that the Colonel was visiting the Major in the latter's office,
  instead of the reverse.}

\textit{It was, on the other hand, certainly no coincidence that, when I heard low voices, I slowed down and listened. The
  sigh and hum of the ventilation made the listening more difficult, but it also concealed my presence.}

\textit{``--me but that they're persistent mother's sons,'' the other voice, which I later found out to be the Colonel's,
  was saying.}

\textit{There was a silence, and then Gilb's voice: ``They're en route, then? How many? What's the ETA?''}

\textit{``At least six ships.'' Pause. ``Give 'em a month. It's imperative that the UN retain control of this installation
  until we can determine what technologies we're sitting on here. However long it takes.''}

\textit{There was what had to be Gilb's laughter. ``The UN? You mean the Union.''}

\textit{``It's all the same. The Asians are persistent. We have to discourage them.''}

\wuxchcenterline{}

``Y'know, pardner, you've seemed a little distracted lately,'' Wendy said as she mixed her wasabi into the soy sauce,
getting ready to dig into her sushi. ``Everything all right?''

I concentrated on my California rolls for a minute. ``It's Anna,'' I confessed. ``Ever since I found out that she's Bob's
daughter, I've been avoiding her. She calls me every once in a while, but I haven't been returning her calls. I deleted her
phone number from my PDA. I like her, but I don't know if I can handle\ldots''

``Anna was at my baby shower last weekend,'' Wendy nodded understandingly. ``She really likes you.'' When I didn't answer,
Wendy continued: ``I'm going to tell you what I told her. When I first met my husband, I was working a lot of overtime. It
was one of those companies that always seems to be in deadline mode, you know? Anyway, Tom kept asking me out, and I wanted
to, but I always had to work. Finally, I decided to make time. And the rest, as they say, is history. Tom kept trying, and
so I told Anna to keep trying. Persistence pays off.''

``Agh! Don't talk to me about persistence!'' I grumbled.

Wendy blinked. ``Huh?''

``Oh,'' I shook my head, ``I'm trying to implement some simple object persistence in my class hierarchy. I haven't been able
to figure out how to do it. I asked the Guru for some help, but she said that she had taught me everything I need to know. I
just can't figure it out.''

Wendy pursed her lips in thought. ``This one time I'll waive our agreement about no shop talk over lunch. Fill me in a
little more.''

``I know it's not that difficult,'' I said around a mouthful of rice. ``Once I get going, I know I'll be fine. The main
application can instantiate a derived class. I need to be able to write that class to an \textbf{ostream}. No problem. But
I'm stuck on how to reconstitute the correct class when I read it back in.''

``The Guru's right -- you do know everything you need to know. How do you create an object, when you don't know the exact
type of the object until run time?''

``Use a class factory,'' I said.

``Right, just like the one you implemented last month \footnote{Jim Hyslop and Herb Sutter. ``Conversations: A Factory,
  Template Style,'' C/C++ Users Journal C++ Experts Forum, June 2001}. Now, what do you pass to the factory to get the
appropriate type?''

``Some sort of token. I used a \textbf{std::string}.''

``OK, now where do you get that token?''

``Ummm\ldots well, it isn't going to appear out of thin air, so I guess I read it from the file.'' At that moment, the penny
dropped. ``Oh, of course. When I write out the object, first I write out the token indicating what type of object it is, and
then I write the object's information. Reconstituting it is a simple matter of reading the token, passing it to the factory,
and then streaming it in from the file.''

``Right,'' Wendy replied. ``And if you want to get really fancy, you can\ldots.''

``Hold it!'' I signaled a time-out. ``Thanks for helping me clear my mental log jam. Now that it's clear, I'm re-invoking
the 'no-shop-talk' rule.'' I dug into the main course.

Later, when Wendy and I got back to the office, I sat at my computer and hammered out the base class fairly quickly:

\begin{footnotesize}
\begin{lstlisting}
class Base
{

private:
    // ... whatever data ...

    virtual std::string classID() const { return "Base"; }

protected:
    // Derived classes should call their parent's
    // implementation of this function after they 
    // load themselves.
    virtual void do_read( std::istream& );

    // Derived classes should call their parent's
    // implementation of this function after they 
    // save themselves.
    virtual void do_write( std::ostream& ) const;

public:
    // ... whatever virtual functions are required ... 

    // Streaming functions.
    void read( std::istream& );
    void write( std::ostream& ) const;

    virtual ~Base();
};

// Streaming helpers. Note that they are not friends!
std::ostream& operator <<
    ( std::ostream& o, const Base& b) { b.write(o); }
std::istream& operator >>
    ( std::istream& o, Base& b) { b.read(o); }
// \footnote{A full, working example program is available to download from hyslop.zip.}

\end{lstlisting}
\end{footnotesize}

``Very good, my child,'' I heard the Guru's voice behind me. As I jumped, I cursed at myself for not predicting her arrival.
She continued: ``Give me a quick walk-through of your writings.''

``Simple enough,'' I said, as the Guru sat in the guest chair. ``Base::write is called via the overloaded insertion
operator. Here's its implementation:''

\begin{footnotesize}
\begin{lstlisting}
void Base::write( std::ostream& o ) const
{
    o << classID() << std::endl;
    do_write(o);
}
\end{lstlisting}
\end{footnotesize}

``Ah, the Template Method pattern,'' the Guru nodded. ``Very wise, my child \footnote{E. Gamma, R. Helm, R. Johnson, and J.
  Vlissides. Design Patterns: Elements of Reusable Object-Oriented Software (Addison-Wesley, 1995).}.''

``The implementation is fairly simple,'' I continued. ``It simply writes out the class identifier and then invokes the
\textbf{do\_write} virtual function. \textbf{do\_write} handles the gory details of writing the object's information to the
stream. Each derived class must call its parent's \textbf{do\_write} function. Reading is slightly more complicated,
though.'' I called up the function I had written to encapsulate the reading:

\begin{footnotesize}
\begin{lstlisting}
std::auto_ptr<Base> loadBase( std::istream& inFile )
{
    std::string className;
    std::getline( inFile, className );

    std::auto_ptr<Base> newBase =
        genericFactory<Base>::instance().create(className);

    if( newBase.get() != 0 ) {
        inFile >> *newBase;
    }

    return newBase;
}
\end{lstlisting}
\end{footnotesize}

``It's a three-step process. First, I retrieve the class ID from the file. Then I use that identifier to get a new object
from the class factory, and finally I reconstitute the class's data, via the \textbf{read} function:''

\begin{footnotesize}
\begin{lstlisting}
void Base::read( std::istream& i )
{
    do_read(i);
}
\end{lstlisting}
\end{footnotesize}

``I'm using the Template Method pattern here, too, but that's probably overkill for a simple function like this.
\textbf{do\_read}, of course, handles the gory details of reading in the information, validating it, and so on. Like the
\textbf{do\_write} function, a derived class's \textbf{do\_read} should call its parent's \textbf{do\_read} function.''

``Well done, my child,'' the Guru said as she stood up.

``Before you go,'' I blurted out. The Guru paused. ``Tell Anna I've lost her phone number, but I'll be home tonight if she
wants to call me.'' The Guru nodded and glided away.

\wuxchcenterline{}

\textit{``'Discourage.' Ah.'' It was Gilb's voice again. I blinked and shook my head to clear away the tired haze. ``What do
  they know?''}

\textit{``Not much, we think. The Asians suspect something.''}

\textit{``They don't believe our cover story of communications problems that happen to cut off all of their nationals from
  talking with them?'' Gilb's voice was openly sarcastic.}

\textit{``They suspect,'' the other voice repeated. ``And they're coming. We need to deal with it. What have your tech teams
  discovered so far?''}

\textit{I was feeling increasingly uncomfortable standing in the corridor and didn't wait to hear more. There would be time
  enough after a good night's sleep.}

\section{Hungarian wartHogs}

\textit{``Breakthrough!'' one of the officers called, poking her head into the mess hall.}

\textit{Several people, Jeannine and I among them, looked up. ``What?'' ``Whose breakthrough?'' ``What happened?''}

\textit{The sudden excitement was palpable, and no wonder -- things were worse than ever since the senior officers had made
  known we'd lost contact with the orbiting station and the main surface base. Likely, they were overrun by the incoming
  forces, although our officers had not yet officially admitted that we were in a hostile situation. When the news came, we
  had been sealed into the local base near the excavation site for two days already. And progress on the alien artifacts had
  continued to be maddeningly slow, knowing how near we must be to a breakthrough. Jeannine seemed closer than ever to
  figuring out the power requirements.}

\textit{``J\"oger says he's figured out part of the language, that's what,'' she informed us. The obscure writing on the
  ancient, and mostly defunct, equipment had been a major obstacle to progress. Until now, attempts to decipher it had been
  largely unsuccessful. ``Something about a built-in redundancy -- he's doping it out. Don't all rush in to bug him; he
  needs to concentrate. Thought you'd like to know.'' And then the officer's head disappeared as she continued elsewhere to
  spread the news.}

\textit{``Redundant information,'' I sat back and mused aloud. ``Does that ever sound familiar. Why, back on my first
  job\ldots''}

\wuxchcenterline{}

``Warts!''

``Beg pardon?'' I turned at the sound of the Guru's voice.

``You have warts, my child.''

Suddenly feeling rather self-conscious, I asked: ``Uh, where? My medicated creams usually contr--''

``Your latest code is covered in warts!'' she interrupted. ``Otherwise known as Hungarian notation. Your variables are
beginning to look like this.'' She picked up the whiteboard marker and wrote: wartHog. ``In this case, the variable hog's
wart has nothing to do with the pottery school \footnote{The first reader to correctly identify the source of this oblique
  reference, as well as why it's appropriate for this November 2001 column, will receive an autographed copy of Sutter's
  More Exceptional C++ when it is available in November 2001. All submissions must be sent by email to
  \texttt{hsutter@acm.org} with the subject line "I know! I know!" and must include a valid (non-munged) return email
  address and snail-mail postal address. Contest closes at midnight on October 15, 2001.}.''

``Oh, that,'' I breathed a sigh of relief. ``Hungarian notation? Is that all? Sure, yeah, I read a cool article about
Hungarian notation, and it sounded like a good idea. Apparently it was quite the trendy thing for a while. Almost kind of a
type calculus-ish direction, sort of in a way, you know. So I--''

``Gibberish!'' the Guru exclaimed.

``Ah, well,'' I faltered. ``I thought I said that rather clearly.''

``Not you, my child,'' the Guru corrected. ``The naming convention you chose to experiment with results in code that looks
like gibberish. Even the prophet Petzold had his off days.''

``Hmmmph,'' I hmmmphed, only slightly mollified. ``Well, it's supposed to make code easier to write and read. For example, I
can catch some type errors just by reading the code. If I write something like
 \lstinline$strcpy( szDestination, pachSource);$ 
I can see a problem immediately -- the second parameter is a pointer to an array of characters, but not necessarily
null-terminated as strcpy requires. Or if I write \lstinline$printf("%s", ulValue );$ ,
I can see that I'm passing an unsigned long where a null-terminated string ought to go. The code tells me that I'm doing
thewrong thing.''

``Of dubious practical advantage even for type-unsafe calls in C, or in environments when nearly everything is a
type-challenged int or void* handle,'' she shook her head sadly, ``and of no relevance at all in a type-safe object-oriented
language like C++. This, you must unlearn. Warts are not information; they are disinformation.

``Consider your own parables. In your first example, in C++ you would use strings, which are objects, and problems like the
first cannot arise because a string's semantics are well defined and encapsulated. One who writes string destination =
source; just cannot get the buffer-copying semantics wrong, for they are never exposed and the calling programmer never
required to assist them; the implementation details are well and truly encapsulated and always managed for you. As for your
second example, in C++ you would normally use streams or other type-safe methods to write output, and type checking along
with overload resolution guarantee a type-correct result even if a conversion is needed. 
One who writes \lstinline$cout << value;$

just cannot get the type wrong: if value is an unsigned long, the \lstinline$operator<<$ for unsigned longs will be invoked;
or, if value has a type for which no \lstinline$operator<<$ is defined but value's type can be converted unambiguously to a
type for which \lstinline$operator<<$ is defined, that \lstinline$operator<<$ will be invoked; otherwise, a compile-time
error occurs. Not only does the C++ language and standard library detect what would be run-time type errors with other
type-unsafe calls, it turns them into compile-time errors.

``In sum, the compiler already knows much more than you do about an object's type. Changing the variable's name to embed
type information adds little value and is in fact brittle. And if there ever was reason to use some Hungarian notation in
C-style languages, which is debatable, there certainly remains no value when using type-safe languages.''

``Maybe so,'' I acquiesced. If nothing else, she had out-soliloquized me. ``But you have to admit that, once you know the
rules, using Hungarian makes it easier to create variable names.''

``So certain are you?'' the Guru arched her eyebrows. She placed herself in the guest chair. I knew what that meant: she was
warming up for a debate. ``I had that in mind when I said 'brittle.' How is it easier, say you?''

``Because the types tell you what to call the variables. It's almost mechanical,'' I said. ``The variable name is pretty
much generated from its type. Once you know the type, you can generate the name,'' I rambled, then realized I was rambling,
and stopped. When I had begun, I had somehow had the sense there would be something much more profound to say; now that I
heard the words coming out of my mouth, it all sounded a bit superficial. I remember wondering why, when it had seemed so
much deeper the first time.

``And if the type changes\ldots?'' she prompted me just then.

``Well, you'd have to change the variable name, I guess. But!'' I exclaimed, ``That then forces you to examine each usage of
the variable in your program, to ensure that it is still being used properly.''

``Not a bit of it,'' she riposted. ``It 'forces' you to do no such thing. In many cases, the programmers will simply forget
-- or worse, not bother -- to consistently and globally change the variable name, never mind check the usage. And once they
do not change the name, the code is lying to you, thus violating the commandment that says: 'Speak truth each one of you
with his cubicle neighbor.' This,'' she shook her head quietly, ``is reprehensible. Such is the disinformational evil that
must of necessity follow, sooner or later but probably sooner, from the deceptive wartHog style.''

``Ah,'' I smiled, ``that's what global search-and-replace is for!''

``Most assuredly not,'' she shook her head again. ``Never mind that you could inadvertently change other similar names of
objects whose types have not changed! But even if the global replacement were done correctly, what value has it added? For
it leaves the programmer no better off for his troubles than he would have been otherwise. If an error has been introduced
by the change in type, such as because of implicit conversions, the error remains the same regardless of the object's
superficial name, and you have merely added the menial and meaningless documentation work of changing the name. This too is
vanity and a striving after wind.''

It was time for me to fall back and regroup. ``But you'd have the error no matter what naming system you used. If I have int
count; and I change it to short count;, then many programmers might not bother to check the usage at all and just hope the
compiler catches any range problems.''

``That,'' the Guru acknowledged, ``is what I just said. The problem is the same whether you uglify the variable's name or
not, and by uglifying it you have merely added useless maintenance work because then you must additionally maintain the
warty name. If you are not yet convinced, my child, I have one small question for you now: How would you apply Hungarian
notation to templates?''

That stumped me. ``Touch, I guess,'' I acknowledged. ``A template doesn't really have a type of its own, because the
template generates an unknown number of types, one for each set of parameters it's instantiated with. There's no type until
it's instantiated, so you can't really create a variable name that encodes the type of the template itself.''

``Well spoken, well said. Even more,'' the Guru added, ``inside the template definition itself, how would you wartify the
names of objects of a template parameter type? You do not know what they are.''

``Oh, I see,'' I said. ``You mean like this.'' On the whiteboard, I scribbled an offhand example:

\begin{footnotesize}
\begin{lstlisting}
template<typename T>
T AddOne( T wartT ) // what wart should wart be?
                    // papuch? lpsz? huh? 
                    // (handle to unbounded harm)
{
 return wartT + T(1);
}
\end{lstlisting}
\end{footnotesize}

``I believe source code is a form of communication,'' the Guru pressed on. ``The question is, who is that communication
aimed at? The compiler? No. Source code is a medium of communication from one programmer to another. It is an expression of
intent, of what is desired to happen. We must strive to keep that communication as simple and clear as possible. In order to
do that, variable names should reflect the roles that those variables play. The exact type is secondary to the role. A
variable name such as sz tells you only that you are looking at a C-style string. It conveys no information as to the role
that string is to play.''

``Well,'' I put in, ``I don't think anyone would use just sz for a variable name.''

``No? If we are to discuss Hungarian notation, we should discuss the canonical version presented by Dr.~Simonyi
\footnote{harles Simonyi. "Hungarian notation." Reprinted at
  \url{http://msdn.microsoft.com/library/en-us/dnvsgen/html/hunganotat.asp}. Readers are encouraged to study the naming
  convention and decide for themselves, rather than rely on the opinions of the authors.}. His examples use variables named
sz, pch, and so on. Such names, alas, present no useful information. If the variables were instead called xyzzy and yeti,
respectively, or even merely x and y, I would still know simply by looking at their declarations that the first indicates an
array of characters, and the second is a pointer to a character. Calling them sz and pch adds no useful information not
already present in the code, and in particular, it adds no information not already well known to the compiler. Worse, it
could be a lie if the type has changed since the wart was chosen. In any event, even if the wart lies not, the questions are
still: What are the variables? What are they for? What do they do? How are they used? The wart helps not at all. I have to
study the code to understand the roles they play. The code fails to communicate the intent of the programmer; therefore the
names sz and pch are poor choices.''

``In your opinion,'' I ventured.

``In my opinion,'' the Guru nodded. ``Yes, this is still somewhat an area of opinion, rather than hard fact alone, although
fact it is that names such as sz are next to useless and that Hungarian warts are brittle in the face of change. When we
drafted the coding style guidelines several years ago, Hungarian notation was but one of many areas of lively debate. Unlike
some discussions, this one was actually reasonably civilized. We examined the naming convention, listened to the experiences
of those who had used it, and eventually came to the consensus that we did not like the convention. Although we did not
dislike it enough to prohibit its use outright, we did consider it brittle enough to actively discourage it. Hence, to quote
this team's standards:

``Avoid Hungarian notation. It will make a liar out of you. 

Warts are not information, but disinformation.''

``Hungarian is not only mendacious, but it is high-cholesterol; it is suspected of being fattening, and it is in all
probability a flagstone on the road leading to a wasted and dissolute life. Indeed, I recall only one time when Hungarian
notation was useful on a project.''

This hook intrigued me. ``What was that?''

``One of the programmers on the project was named Paul,'' the Guru explained. ``Several months into the project, while still
struggling to grow a ponytail and build his report-writing module, he pointed out that Hungarian notation had helped him
find a sense of identity, for he now knew what he was\ldots'' She paused.

I blinked. It took me about ten seconds, and then I shut my eyes and grimaced painfully. ``Pointer to array of unsigned
long,'' I groaned.

She smiled, enjoying my pain. ``True story,'' she said \footnote{Indeed a true story that happened to one of the authors.}.

It was then that I thought I had found a way to corner her in an inconsistency. ``Well,'' I asked innocently, having quickly
recovered from the awful pun, ``what about our convention of using a trailing underscore suffix for member variables? Isn't
that kind of a watered-down version of Hungarian?''

The Guru smiled pleasantly. ``So it may appear to the uninitiated, but appearances are in this case deceiving. The
underscore has nothing to do with type -- it has to do with flagging scope and privacy. It also has a small practical
benefit: we realized that passing a parameter to a member function -- particularly an initialization function or a
constructor -- we would often want to use the same name for both the member variable, and for the passed parameter. For
example:

\begin{footnotesize}
\begin{lstlisting}
class T
{
 int count_;
public:
 T() : count_(0) {}
 void init(int count) { count_=count; }
};
\end{lstlisting}
\end{footnotesize}

``We are giving the init function a count. Why come up with a different name for the parameter, when both the parameter and
the member play the same role? We actually discussed several options for such cases: prefixing member variable names with my
(or with our for static data), prefixing the parameter name with given, and so on. In the end, we decided to adopt a
trailing underscore for the member, and no underscore for the provided parameter.''

I frowned. ``Isn't this convention a little debatable?''

The Guru grinned wryly. ``Wendy's personal preference is the other way around. But then no one is perfect, and she did just
produce a most exceptionally beautiful child in Jeannine \footnote{Not the same Jeannine as in the framing story. See
  \textit{Conversations: Back To Base-ics,} C/C++ Users Journal C++ Experts Forum, September 2001,
  \url{www.cuj.com/experts/1909/hyslop.htm}.}, so one must make allowances.''



\section{New Bases, Part 1}

Things were very quiet in the Ballroom.

That wasn't entirely unexpected; after all, we had no idea how we would ever leave it. It was the second day after the
attacking forces had invaded our position inside the alien city beneath the Europan ice. With no help within reach from our
own political faction on Earth, Gilb ordered retreat into the Ballroom and we barricaded the doors. The room was immense,
but fortunately had few entrances. A few of our people had left us and surrendered in the disorder of the retreat; Jeannine
and I didn't get the chance, for we were watched too closely by Gilb and his officers.

It was cold in the Ballroom, and getting colder. Gilb authorized turning on the two portable heaters, since we had more
energy cells with us than our group was likely to use in a year. It helped.

We also had with us several of the alien artifacts, now powered but still mysterious, for we had only just managed to
activate them and had had no time to experiment. J\"ager was with us, which was something; he knew the mysterious city's
language better than anyone.

The three 10-foot-diameter balls still hung suspended above the floor. Each still showed a different scene: the field of
yellowish grasses with something like trees nearby; the dusk beach scene; and the metallic room much like our own, with the
cabinets, where we had seen the gray-clad human who had started at seeing us and then had run away. He had not returned, nor
had anyone else. We knew that throwing small objects into the balls put them into the scenes; a spent energy cell thrown
through the meadow gate -- we were beginning to think of the balls as ``gates'' now -- pushed through the growths and
bounced on the grassy ground before coming to rest. The energy cell was still visible, and nothing had jumped on it or
otherwise reacted to the experiment.

I turned to Jeannine, and we spoke of other cold and quiet times. It helped to distract us, while J\"ager tried to make
himself more useful at the other end of the room by continuing to experiment carefully with one of the artifacts.

\wuxchcenterline{}


Things were very quiet around the office.

That wasn't entirely unexpected; after all, it was January 2, and many people had taken the day off. I was puttering around
with some small bits of programming on my to-do list, because my mind wasn't ready for anything big.

I had recently started getting ``internal library'' development requests, which was a real thrill for me even though the tasks
were small. You see, our company didn't have an officially separate team whose job was to build software library
infrastructure for the whole company, which I thought was a good thing. That organization style never made sense to me, nor
I think to the Guru. But we did have a set of internal libraries for common things that were shared by several projects. The
internal libraries had no single ``owner,'' but items to be added or changed were reviewed by one of the more experienced
developers. If we wrote something that was a likely candidate for reuse, we'd submit it to the shared libraries. Less often,
someone would request a feature that wasn't in the shared area yet but probably should be, and someone capable would pick up
the request and write it.

So I was thrilled because being asked to write an addition to the internal library branded me as ``possibly capable.'' I was
also thrilled because Bob wasn't ever asked, and this knowledge gave me a little ego boost. Ah, the vanity of youth.

The problem seemed easy: ``Write a \textbf{ConvertBase} function that takes a \textbf{string} representing a number in base
N and converts it to a \textbf{string} representing the same number in base M.''

My first reaction was to write a hard-coded solution, something like this:

\begin{footnotesize}
\begin{lstlisting}
string ConvertBase( size_t base1, size_t base2,
                    const string& src )
{
  long value = 0;

  // Read a base base1 number from src
  for( int i = 0; i < src.size(); ++i )
  {
    // if src[i] is a valid base1 character,
    //  multiply value by base1 and add the next digit
    // else if src[i] is whitespace
    //  break
    // else
    //  throw logic_error
  }

  // Write a base base2 number to dest
  string dest;
  while( value > 0 )
  {
    // find highest base2 digit
    // append to dest
    // reduce value appropriately
  }
}
\end{lstlisting}
\end{footnotesize}

I had roughed in about that much and was still typing the last comment when a gentle throat-clearing behind me alerted me to
the Guru's presence.

I froze. There was silence.

Then I heard a brief rustle of pages that I knew without looking was the Guru putting a marker into her book and closing it,
and a calm low voice: ``Goodeve, my child.''

``Ah, goodeve, master,'' I replied. If she was in her Guru shtick, then Bob had to be nearby, and appearances were important
to successfully bugging Bob.

``Your code,'' she gestured, walking around me into view and tapping on my display with her pen, ``bears a striking
resemblance to library request 247.''

``Ah, yes, I hope it does,'' I acknowledged. ``That's what I was working on. It looks like it's going to get kind of
longish.''

She nodded thoughtfully. There was another pause. After a moment she added: ``Did you happen to notice library request
314?''

``Ah, er\ldots no,'' I admitted.

``No harm done. Request 314 asks for a facility to read and write a number of arbitrary base from a text stream. I presume
that whoever codes 314 will end up writing much the same code as you were about to write there. May I make a suggestion, my
apprentice?''

``Sure. You want me to do 314 too?'' I asked, hope rising. If it was an honor to be asked to write one shared library
facility, imagine the honor of being asked to write two after such a short time on the team!

``Yes, but not immediately,'' said the Guru, arranging a wisp of graying hair behind her ear. ``A principle of programming
is that we often gain useful insights into our code by writing the test cases first. Beck, Fowler, Martin, and others preach
this gospel to anyone who will listen and to many who will not. In this case, why not write the code for request 247
assuming that you already had the code for 314? Then write a simple test case for 247 and hard code just enough of a stub
for 314 to make that test case compile. Finally, when you see in your 247 code the form that the solution to 314 should
take, confirm that that form is acceptable to the people requesting 314, and do 314 too.''

I blinked. ``I think I got that. So I'll take a first cut at 247 that uses 314's not-yet-written facility, which will let me
see what 314's facility ought to look like. Right?''

``Indeed.''

``Okay.'' I thought for a moment and then started typing. I could assume that I had ``a facility to read and write a number
of arbitrary base from a text stream.'' All right. If I had that, I could use streams to translate between bases. After
further thought, I eventually came up with this code:

\begin{footnotesize}
\begin{lstlisting}
string ConvertBase( size_t base1, size_t base2,
                    const string& src )
{
  stringstream s1( src );
  long value;

  if( !( s1 >> Num( base1, value ) ) || 
      !( s1 >> std::ws ).eof() )
    throw logic_error
      ( "src is not a valid number" );

  stringstream s2;
  if( !( s2 << Num( base2, value ) ) )
    throw logic_error
      ( "unexpected error emitting converted number" );

  return s2.str();
}
\end{lstlisting}
\end{footnotesize}

Then I wrote a simple test case, with just enough of a \textbf{Num} stub hardwired to make the case compile:

\begin{footnotesize}
\begin{lstlisting}
class Num
{
public:
  Num( size_t base, long& value ) :
    base_(base), value_(value) { }

  size_t Base() const { return base_;  }
  long& Value()       { return value_; }

private:
  size_t base_;
  long& value_;
};

istream& operator>>( istream& i, Num& n )
{
  string s;
  i >> s;
  // todo: really convert input to n's base_
  n.Value() = 255; 
  return i;
}

ostream& operator<<( ostream& o, const Num& /* n */ )
{
  // todo: really output n.value_ in n.base_
  return o << "FF"; 
}

int main()
{
  string result = ConvertBase( 10, 16, "255" );
  cout << result << endl;
  return result == "FF" ? 0 : 1;
}
\end{lstlisting}
\end{footnotesize}

After the compiler caught a few typos, I was satisfied that this skeletal attempt worked. Next, I knew I would have to more
fully address request 314 by actually implementing \lstinline$<<$ and \lstinline$>>$ for \textbf{Nums}.

I looked at my watch; it was well past quitting time, but I could still make my train and therefore my date with Anna.
Request 314 could wait till the morning.

\wuxchcenterline{}

\begin{itshape}
I looked at my watch; it was well past time for our turn to sleep, so we took the opportunity. There was periodic banging on
the other side of one of the doors, presumably from the invading force, but that was all -- so far. Unless something more
happened, the strange balls, or gates, could wait till the morning. Although Gilb's people were still trying to use our
portable comms gear to get a message to Earth, or to ships in transit I didn't know about, I had a feeling that the gates
would be our only way out.
\end{itshape}

\section{New Bases, Part2}

\begin{itshape}
The change in control came suddenly.

Gilb had been watching the main Ballroom door, where we had heard occasional pounding. The attack, when it came, was through
the two side doors and caught us while most of us were asleep.

I woke to the sound of loud banging and the glare from the intense points of light moving across the two opposing doors. It
was immediately obvious that the enemy were cutting their way in, and quickly. Gilb's orders had been explicit -- in case of
attack, grab our packs and alien artifacts and jump through the forest gate, and that's what we tried to do.

Gilb and several of his officers made it through and quickly disappeared out of sight in the other world. Most of us were
still racing for the ball-gate as the doors crumpled and armor-clad soldiers started pouring in toward us from two sides,
shouting orders to stop. I was near the gate, ahead of Jeannine, and hesitated \ldots then shots were fired, and I jumped
forward.

I felt nothing unusual as I passed through the gate, except that my suit's external temperature reading jumped to 28C. I
took a quick glance backward -- still in the Ballroom, Jeannine was dropping her pack and holding out her arms.~I didn't
wait to watch her get captured; I hurried after Gilb's party into the new world's underbrush.
\end{itshape}

\wuxchcenterline{}

The change in management came suddenly.

Because neither our company nor our acquirer were publicly traded, the boards were able to negotiate the deal quietly and
keep it under wraps, even from the employees, until just a couple of weeks before the actual closing date. So it was that
shortly after New Year's Day we first heard the announcement of the merger (such a polite term for ``acquisition'' or
``buyout'' given that we were much smaller and ailing than our buyer) and within two weeks found ourselves already working
for the new boss.

Pete Williams seemed like a reasonable fellow at first, even if he was a bit too given to sports metaphors for a manager of
a technical team. I would have more of a chance to refine that opinion as time went on, and I gained experience working
under him. I did approve of Pete's first action, or perhaps inaction: he kept our team together and didn't monkey much with
the way we worked, at first. That was mostly a good thing. Except for Bob, who had somehow thought he would get a promotion
out of this, we were all pleased with this turn of events.

A few urgent assignments had taken me away from my lower-priority internal library-building work (not that the internal
library wasn't important, but none of the requests assigned to me were, for I was only just starting out). When I next had a
breather, I returned to the half-written library code. I had been filling two requests: Request 247 asked for ``a
ConvertBase function that takes a string representing a number in base N and converts it to a string representing the same
number in base M.'' The Guru had pointed out that Request 314 was for ``a facility to read and write a number of arbitrary
base from a text stream.'' Thus guided, I had written the following to try to implement the former in terms of the latter
and avoid redundant work:

\begin{footnotesize}
\begin{lstlisting}
string ConvertBase( size_t base1, size_t base2,
                    const string& src )
{
  stringstream s1( src );
  long value;

  if( !( s1 >> Num( base1, value )).eof() || 
      !( s1 >> std::ws ).eof() )
    throw logic_error( "src is not a valid number" );

  stringstream s2;
  if( !( s2 << Num( base2, value ) ) )
    throw logic_error( "unexpected error emitting "
                       "converted number" );

  return s2.str();
}
\end{lstlisting}
\end{footnotesize}

That seemed to do it for 247, though as it relied on 314's \textbf{Num}, it wouldn't work until I finished that. So far I
had just a simple test case, with a \textbf{Num} stub hardwired to make the case compile, and while I was looking at it, I
added a check to limit the base to a reasonable range:

\begin{footnotesize}
\begin{lstlisting}
class Num
{
public:
  Num( size_t base, long& value )
  : base_(base), value_(value)
  {
    if( base < 1 || base > 36 )
      throw logic_error( "base must be from "
                         "1 to 36" );
  }

  size_t Base() const { return base_;  }
  long& Value()       { return value_; }
  long  Value() const { return value_; } 

private:
  size_t base_;
  long& value_;
};

istream& operator>>( istream& i, Num& n )
{
  string s;
  i >> s;
  n.Value() = 255; // todo: really convert input
                   //       to n's base_
  return i;
}

ostream& operator<<( ostream& o, const Num& /* n */ )
{
  return o << "FF"; // todo: really output
                    //       n.value_ in n.base_
}

int main()
{
  string result = ConvertBase( 10, 16, "255" );
  cout << result << endl;
  return result == "FF" ? 0 : 1;
}
\end{lstlisting}
\end{footnotesize}

Next, I knew I would have to finish 314 by actually implementing \lstinline$<<$ and \lstinline$>>$ for Nums. As I was about
to start coding, I noticed a serious deficiency in the signatures of my existing operators \lstinline$<<$ and
\lstinline$>>$. I leaned closer to the screen and was nearly ready to start making changes when the telltale closing of a
book behind me alerted me to the Guru's presence.

``Is Bob around?'' I whispered.

``No, we can just talk,'' she said and sat in my visitor's chair. ``What on earth possessed you to do that?'' she continued, pointing at the screen.

``Yeah, I know, I was just looking at that. I've been away from this code for weeks, and now the obvious thing jumps out at
me.''

``Ah. Well then, counsel yourself.''

I pointed at the offending operator \lstinline$<<$ and \lstinline$>>$ signatures. ``Aw, I accidentally hardwired them for
basic char and \lstinline$char\_traits<char>$ streams only.''

She smiled and stood. ``Very good. Carry on, apprentice,'' she added, as we both heard the sound of Bob's sloshing latte
approaching.

``Uh, that's it?'' I asked. ``No longer speech, no morality play, no programming lesson?''

She arched an amused eyebrow. ``And why would there be? You spotted the wicked flaw in your writings without my help, and
now you need time to fix it. What more can I teach until that is done?'' And without waiting for a further reply, she smiled
quietly, reopened her tome, and glided silently away.

I guessed that I must be improving if she was willing to let me get away without a three-page lesson. So I went back to the
code. The first thing was to fix the signatures:

\begin{footnotesize}
\begin{lstlisting}
template<class CharT, class Traits>
basic_istream<CharT, Traits>& operator>>
  ( basic_istream<CharT, Traits>& i, Num& n );

template<class CharT, class Traits>
basic_ostream<CharT, Traits>& operator<<
  ( basic_ostream<CharT, Traits>& o, const Num& n );
\end{lstlisting}
\end{footnotesize}

That was better. I checked it by eye, and it looked right this time. Next, I decided that input would be easiest. I began
adroitly:

\begin{footnotesize}
\begin{lstlisting}
template<class CharT, class Traits>
basic_istream<CharT, Traits>& operator>>
  ( basic_istream<CharT, Traits>& input, Num& n )
{
  locale loc = input.getloc();
  CharT c;
\end{lstlisting}
\end{footnotesize}

Here I congratulated myself on my own presence of mind, as I avoided the twin pitfalls of reading simply \textbf{chars} and
of forgetting about locales. I then remembered that leading whitespace can be optionally skipped, and so I pressed on:

\begin{footnotesize}
\begin{lstlisting}
  // If we should, then skip leading
  // whitespace, if any.
  if( input.flags() & ios_base::skipws )
  {
    do
    {
      input >> c;
    }
    while( isspace( c, loc ) );
  }
   
  // Handle sign, if any.
  int sign = ( c == '-' ? ��1 : 1 );
  if( c == '-' || c == '+' )
    input >> c;
\end{lstlisting}
\end{footnotesize}

So far, so good. When it came to actually handling the digits, the logic was simple, and it was mostly a matter of deciding
on a reasonable way to decipher the digits. After all, green though I was, I was knowledgeable enough to realize that I
couldn't just assume that the digit and/or letter characters would be contiguous and in order in the character set (even
though each group was contiguous and in order in ASCII). For the time being, I decided on a helper structure that was
suitable for searching with \lstinline$std::find()$:

\begin{footnotesize}
\begin{lstlisting}
 CharT digits[]
    = "0123456789ABCDEFGHIJKLMNOPQRSTUVWXYZ";
  size_t numDigits 
    = sizeof( digits ) / sizeof( digits[0] );

  n.Value() = 0;
  while( input.good() )
  {
    int index = find(digits, digits + numDigits,
                     toupper (c, loc) )
                �� digits;
    if( index < 0 || index > n.Base() )
    {
      break;
    }

    n.Value() = n.Value() * n.Base() + index;
    input >> c;
  }

  if ( !input.eof() )
    input.putback( c );

  n.Value() *= sign;
  return input;
}
\end{lstlisting}
\end{footnotesize}

That seemed reasonable enough for input, leaving only output. Output was easier, in part because I made a deliberate
decision not to support the output formatting flags like \textbf{showpos} and \textbf{uppercase}; that could wait until a
later request, if it was needed at all--

I glanced furtively over my shoulder, in case the Guru had glided up and would pounce on me, remonstrating about supporting
all the standard stream flags. All was dark (it being that time of year) and quiet. There was no Guru there. I waited a few
moments, but there was no sound or movement. Satisfied, I turned back to my code.

--so yes, as I was saying, I ignored the issue of formatting flags. If the Guru didn't like it, she could remonstrate after
I checked it in, and I'd do a more complete job in version 2.0 of the code. If Wendy didn't like it, I would try to argue my
way out of it by saying we didn't yet know if that was a requirement, so why complicate the code unnecessarily up front? If
Bob didn't like it, he could go jump in the lake, particularly at this time of year.

So my first cut at a real insertion operator looked like this:

\begin{footnotesize}
\begin{lstlisting}
template<class CharT, class Traits>
basic_ostream<CharT, Traits>& operator<<
  ( basic_ostream<CharT, Traits>& o, const Num& n )
{
  long value = n.Value();
  basic_string<CharT, Traits> s;

  CharT digits []
    = "0123456789ABCDEFGHIJKLMNOPQRSTUVWXYZ";

  basic_string<CharT, Traits>::size_type start=0;
  do
  {
    s.insert( start, 1, 
              digits[ value % n.Base() ] );
    value /= n.Base();
  }
  while( value > 0 );

  if( n.Value() < 0 )
    s.insert( start, 1, '-' );

  return o << s;
}
\end{lstlisting}
\end{footnotesize}

\wuxchcenterline{}

\begin{itshape}
I started running the code through a more extensive set of test cases and got ready to tackle the next item on my normal
project's workload. Our new manager was big on ``time management,'' which seemed to consist of micromanaging our to-do
lists. I wondered how that would work out.

Fortunately, I was ahead of my list. Perhaps today I could leave a little early for a change.

Fortunately, it was a mostly hospitable world we now found ourselves in -- for it was obvious, if only from the slightly
heavier gravity, that it was a different world from any we had known among Sol's children. Yes, mostly hospitable: as the
few rations gave out, we found local flora that was edible, and we did not immediately come across any indigenous humans, or
more to the point they did not come across us.

Gilb found out about the local fauna the hard way. We buried him a few days later and started posting sentries against the
unfriendly five-legged carnivores we dubbed ``Goofies'' because of their passing resemblance to the cartoon character. That
was when Da Rosa took over command of our small base and instituted a more rigorous testing of some of the artifacts we had
determined were, or could be used as, weapons.

I worried about Jeannine.
\end{itshape}

\section{Template Specializations}

\begin{itshape}
We watched the gate from a short distance, alert that the invaders might follow us through from the Europan base. It
appeared, however, that the enemy was quite content to abandon us, or at least wait us out. The gate remained quiet.

Despite constant alertness, Goofies had got a couple of our sentries. We had killed many Goofies -- sometimes with our
conventional weapons, sometimes as tests of the alien artifacts, at least one of which spewed huge energy bolts that made it
dangerous to aim at anything too close nearby -- but the Goofies seemed to be in unlimited supply.

So it was that we were only partly surprised by the mutiny.
\end{itshape}
\wuxchcenterline{}

``Hey, Junior, how's it going?'' I tensed at Bob's voice. But then it struck me -- Bob was not his usual condescending self.
He was almost\ldots dare I say\ldots civil.

``Not bad, Bob,'' I replied, turning to face him. We stared at each other for a moment.

``Uh, say, Junior, I was wondering if I could get a little help.''

I was stunned. Speechless. Bob, the worst programmer in the department; the man who delighted in writing obscure, difficult
to maintain code; the man whose arrogance was so thick you could smell it a block away; the man who stopped by my cubicle
only long enough to dump his problems in my lap (and latte on my desk) was standing at my cubicle, asking me for help?

``Aaah, sure, Bob,'' I finally answered. ``What's up?''

Bob sat himself down in the guest chair. ``OK, here's the scoop. I've managed to condense it down to this source code.'' He
took over on the keyboard and called up a file on my workstation. I looked closely at it:

\begin{footnotesize}
\begin{lstlisting}
#include <iostream>

template<typename T, std::size_t size = 10>
class c
{
  T m[size];
public:
  void print_size() 
  {
    std::cout << size << std::endl;
  }
};

template<> class c<char>
{
  char m[100];
public:
  void print_size()
  {
    std::cout << 100 << std::endl;
  }
};

int main()
{
  c<char>().print_size();
  c<char, 10>().print_size();
}
\end{lstlisting}
\end{footnotesize}

``First, I don't see how it can compile. I mean, I thought the template argument has to be a class or type name, not a\ldots
well, an ordinary parameter.''

``You mean the size\_t parameter?'' I quizzed. Bob grunted, which I took to mean an affirmative. ``Yeah, the first time I
saw a non-type parameter, my brain stopped for a moment too. I was used to seeing class or typename declarations, not
regular parameters. The Guru assured me that it was allowed, but she made me look it up in the Standard for myself. You can
only use integral or enumeration types, pointers or references to objects or functions, or pointers to member $[$1$]$. In
fact, with integral types, the parameter becomes an integral constant expression\ldots.''

``A whozer whatzit?'' Bob interrupted.

``A compile-time constant,'' I translated from Standardese to English. ``Anyway, since it's an integr-- er, that is, since
it's a compile-time constant, that allows you to use size to declare the array of T objects. Pretty nifty.

``Once I learned this, I was toying with the idea of using a template to provide a more flexible switch statement, something
like,'' I opened a new text file and typed in:

\begin{footnotesize}
\begin{lstlisting}
template<int value1, int value2, int value3>
void f(int currentValue)
{
    switch(currentValue)
    {
    case value1:
        ...
    case value2:
        ...
    case value3:
        ...
    }
}
void g()
{
    f<1, 2, 3>(1); // executes value1 case
    f<2, 3, 1>(1); // executes value3 case
}
\end{lstlisting}
\end{footnotesize}

``I couldn't come up with a practical application for it, so I filed it in the 'nifty but no practical application for it'
category\ldots''

``Alongside other useless information that gets trotted out at parties, no doubt,'' Bob interrupted, glancing at his watch.
``That's, uh, real interesting, Junior, but getting back to my problem,'' he reached over, closed my text window, and
restored his original program. ``See, when I compiled and ran this, the first template prints out 100, just like I thought
it would, 'cause it um\ldots instantiates the\ldots uh\ldots. specialization \lstinline$c<char>$. I expected the second
template to print out 10, because I figured it'd match up the 10 with the default parameter of 10, but it prints out 100 as
well.''

I studied the code. Bob sipped his latte.

``Well,'' I welled to fill in the silence, ``My first reaction would be that I'd expect both templates to print out 10.''

``Y'know, that's what I told Her Holine-- uh, I mean, that's what I thought at first.'' Ah, that was a little more like the
Bob I knew -- he was about to call the Guru a nasty name.

``I try to think of default parameters as 'lazy shortcuts','' I continued. ``Whenever I write something that takes advantage
of default parameters, I have to remember that the compiler treats it as if I had actually written the parameter.''

``I know that,'' Bob said impatiently. He took over the keyboard and modified the program slightly:

\begin{footnotesize}
\begin{lstlisting}
int main()
{
  c<char>().print_size();
  // compiler acts as if I'd written
  // c<char, 10>().print_size();

  c<char, 10>().print_size();
}
\end{lstlisting}
\end{footnotesize}

``This is an exact match for the base template. In that case, though, the program would print out 10 for each size, but it
prints out 100, so what's going on?'' Bob asked impatiently, glancing again at his watch.

I savored this moment. Not only was Bob aware that I knew something he didn't, but he was actually acknowledging it!

``That's not the only place the default parameter is used,'' I finally said. Bob gave me a quizzical look. ``The
specialization uses it, too.'' I took over the keyboard, after carefully moving Bob's latte out of the way. ``The
specialization that you wrote as\ldots''

\begin{footnotesize}
\begin{lstlisting}
template<> class c<char>
\end{lstlisting}
\end{footnotesize}

``\ldots acts as if you wrote:''

\begin{footnotesize}
\begin{lstlisting}
template<> class c<char, 10>
\end{lstlisting}
\end{footnotesize}


``So,'' I continued, ``you have not specialized on \lstinline$c<char>$, but on \lstinline$c<char, 10>$, where the size
template parameter is an integral constant-expression with the value 10. In other words, your program acts as if you had
written:''

\begin{footnotesize}
\begin{lstlisting}
#include <iostream>
template<typename T, std::size_t size = 10>
class c
{
  T m[size];
public:
  void print_size() 
  {
    std::cout << size << std::endl;
  }
};
 
template<> class c<char, 10>
{
  char m[100];
public:
  void print_size()
  {
    std::cout << 100 << std::endl;
  }
};

int main()
{
  c<char, 10>().print_size();
  c<char, 10>().print_size();
}
\end{lstlisting}
\end{footnotesize}

``I get it,'' Bob smiled as he leaned back in his chair. ``In main(), both instantiations of c instantiate the
specialization.'' I didn't like the look of that smile on his face. He stood up, then in a louder tone of voice, he
continued, ``And remember, Junior, that default parameters apply to template specializations as well.'' I was puzzled by his
sudden change. I spotted the Guru approaching, an open tome in her hand as usual.

She stopped in front of Bob. ``Why are you interrupting my apprentice's devotions?'' the Guru demanded.

``Hello, doll-face,'' Bob beamed at the Guru. ``I was just explaining to Junior here, why the correct output of that program
we were talking about is 100 100.'' He glanced at his watch. ``Yep, coming up on two o'clock -- I better get to Pete's
office and go over this with him.''

The Guru and I watched him disappear in the direction of Pete Williams's office. Pete was the new manager brought in when
our company was acquired last month.

``Figured it out, did he?'' the Guru murmured as she turned to me. ``Now matters are worse.'' My stomach did a flip-flop.

``Um, what was that all about?'' I asked.

The Guru sat down wearily in the guest chair. With Bob gone, she was back in ``normal'' mode. ``You asked me once why Bob
was still around, even though he's an\ldots'' I could see her struggling for words.

``Incompetent, arrogant, misogynistic\ldots '' I supplied, which the Guru capped with a word I would not care to repeat to
my mother, even if she is stone deaf.

``Yes, exactly,'' the Guru continued, calmer now. ``Bob was in quite tight with the previous upper management. He somehow
had them hoodwinked into believing he's a\ldots well, a golden boy. All my attempts to expose him for what he is were swept
aside because he had the upper management convinced that I was still bitter about our divorce. As if!

``Anyway,'' she continued, ``when we had our change in management, I saw a perfect opportunity. I convinced Pete Williams to
pose him a challenging question, which I knew Bob would never figure out. Or so I thought. Pete agreed that the question I
posed would be simple enough even for you to answer.'' I bristled at that, but the Guru, absorbed in her explanation,
ignored me. ``If Bob could not figure out the answer by two o'clock, Pete was going to keep a close eye on him, perhaps even
put him on probation. Eventually, I'm sure I could have got Bob what he deserved -- a pink slip. Now, though, it's going to
be much harder.''

She sighed and silently stood and glided away.

My stomach rose in my throat. ``What have I done?'' I muttered to myself.

\wuxchcenterline{}

\begin{itshape}
It was just before daybreak that the conspirators acted. Several officers went armed into Da Rosa's tent, and we quickly
discovered that only the mutineers' sergeants had functioning weapons. Da Rosa put up no real opposition; I think he was
tired and already resigned to the fact that holding out further would be futile. He gave up his sidearm, Hinckel took
command, and we prepared to surrender.

A man was sent to throw a surrender note through the gate. We could see a brief glimpse of movement through the gate as the
note was retrieved and soon came the reply. Following instructions, we put down our weapons and filed through the gate one
by one. When my turn came, I stepped through into the dimly lit Ballroom and turned to find myself staring at the business
end of weapons carried by grim-faced men. I sighed and was bound like the others.
\end{itshape}

\pagebreak
%%% Local Variables:
%%% TeX-master: "conversations"
%%% End:
