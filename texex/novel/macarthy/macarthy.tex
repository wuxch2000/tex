\documentclass[10pt]{article}
\usepackage{kindle}

\begin{document}

% \setcounter{section}{0}
% \thispagestyle{empty}
\pagenumbering{roman}
\pagestyle{empty}

\begin{center}
\makebox[0.5\textwidth][s]{\textsf{麦卡锡:一个焦虑的时代}}
\end{center}

{\footnotesize\textsf{\tableofcontents}}
\pagebreak
\setcounter{page}{1}
\pagenumbering{arabic}

% \chead{\scriptsize{\textsf{麦卡锡:一个焦虑的时代}}}
% \cfoot{\scriptsize{\textsf{第 \thepage ~页,共 \pageref*{LastPage} 页}}}
% \pagestyle{fancy}

{\small

\section{关于麦卡锡时代的又一本书}

最近告密话题在引起大家关注,也看到有人提到``民族反省'',我想,那不是一个单纯讨论个人品质的道德问题,
根源也不是文化问题,而是社会制度使然。虽然,当一个扭曲的制度长期笼罩,久而久之,也会塑造群体行为模
式,也就塑造出特定的``民族文化''来了。但究其根源,仍然是一个制度问题。

否则我们无法解释,同为德意志民族德意志文化,何以东、西一分家,人的品性就很快变得南辕北辙;也无法解
释,美国这个出了名在立国当日就建立了一以贯之制度的国家,何以也出现一个短短的麦卡锡时期,在那个时期,
美国的不少文化人也突然进入告密行列,一定的文化圈内,也人人自危。而在那个时期结束之后,就在同一个美
国,好像恶梦一场醒来,一切又烟消云散。

这场恶梦让美国吓出一身冷汗。所以参议员麦卡锡演变成为``麦卡锡主义''这样一个很``重''的词,负面得一塌糊
涂。有关的书可以说是出了一把又一把。2005年还出了一本新书《焦虑的时代》,马上被评为《时代》最佳图书。

此书作者是大名鼎鼎的海涅斯\textperiodcentered 约翰逊(Haynes Johnson)。他和父亲麦考姆(Malcolm
Johnson),是迄今美国历史上唯一一对都获得普利策新闻报道奖的父子,真可谓``Johnson \& Johnson''父子兵。

老约翰逊当年在美国家喻户晓。1948年,老约翰逊受编辑派遣,例行公事地调查一件码头工人谋杀案,结果,他的
调查报道《水边罪恶》(Crime on the Waterfront)在纽约老《太阳报》连续刊出,震动全美。这些报道最后被改编
为电影《水边》,获得奥斯卡金像奖。老约翰逊出名在于他``创造了历史''。他让大家看到,假如不严格监督规
范,工会非常容易演变为有组织的犯罪集团。他揭露的码头工会如黑手党一般,拉帮结伙,谋财害命,整车整车地
大批量盗窃货物。他的报道不仅引发一系列司法调查和审理;罪证确凿的系列谋杀者被法庭定罪,还导致整个码头
工会被美国劳联\myrule 产联开除;美国从此成立管理码头的港务局,也启动美国国会对有组织犯罪集团的著名公
开听证,当时电视转播,轰动整个美国。他自己因此获得了1949年的普利策新闻报道奖。

老约翰逊的儿子海涅斯\textperiodcentered 约翰逊不仅是《华盛顿邮报》出色的新闻记者,还写了十五本书,是
多本畅销书的作者。他写了一系列的美国总统,《焦虑的时代》和前一本《克林顿时代的美国》(America in the
Clinton Years),都被《时代》杂志评为最佳图书。他还是公共电视台的美国评论主持人、美国问题的专栏作家。

\pagebreak
\section{小约翰逊的心结}

海涅斯\textperiodcentered 约翰逊在1966年获得普利策新闻报道奖,源于他对发生在阿拉巴马州塞尔马(Selma)的
民权运动,作出了一系列报道。他曾在接受采访时说,他之所以一直会关心种族关系和民权运动,因为他认定自己
是个``南方人''。虽然他出生在纽约,可是父亲的家乡就是我们现在住的典型南方州佐治亚。在种族隔离时期,美
国的南北方是两个世界,以北方为代表的美国大部分地区,并不清楚南方深腹地的种族隔离是怎样的具体体验。二
十年代,佐治亚的KKK还很猖獗,老约翰逊就是因为在报纸上揭露了KKK的罪行被追杀,报纸也被捣毁,还是州长掩
护了他,这才逃到纽约落脚。父亲的这个经历成为小约翰逊的一个心结。

海涅斯\textperiodcentered 约翰逊还有个心结,就是他的父母都是美国左翼,父亲曾经帮助过美国共产党。因
此,他对所谓``惧怕红色''的时代,特别关注和敏感,麦卡锡主义就是那个时代的象征。麦卡锡主义是上世纪五十
年代的事情,距他写《焦虑的时代》,已经过去了五十年。他为何在2005年的七十多岁高龄著书重提旧事,一看书
的副标题就知道:``从麦卡锡主义到恐怖主义''(The Age Of Anxinety:McCarthyism to Terrorism),他是受了新近
的恐怖战争时代美国社会面临种种新问题的刺激。他不仅在书名副标题暗示了两个时代的关联,也在``致读者''中
清楚解释了自己写这本书的原因,是因为在今天的时代,``麦卡锡主义是站在那里的一个参照标准。它告诉我们,
当恐惧和焦虑相结合的时候,会在公众和政治生活中制造歇斯底里''。

``社会性歇斯底里'',这是作者和许许多多美国人在今天回忆那个时代的定位,要举出实例讲出故事可以说是不胜
枚举。大家今天谈到麦卡锡主义,非常自然地和这本书的作者一样,会从麦卡锡这个人讲起,讲起他作为政治家,
一开始就是一个品质非常可疑的人,他很早就被怀疑接受百事可乐以及其他游说集团的好处。他本人酗酒,精神状
态不稳定。也就是说,一个歇斯底里的人,利用社会对红色的无端恐惧和焦虑,推动了社会疯狂。

今天大家都知道,麦卡锡主义是冷战的产物。人们通常理解,在冷战时期,由于社会制度和意识形态的差异,造成
了两个不同阵营的对立。很多人认为,发生在美国的``惧红'',那只是意识形态歧视或思想迫害的结果。最后,柏
林墙倒塌,冷战以一种戏剧性的方式结束,安然度过,似乎更凸现冷战期间发生的各种事件的``荒诞性''。人们也
因此不再去认真清理那个时代何以就会走向这一步。这实际上是对历史的一个简化叙述。在简化中,许多重要的历
史情节都被当作无意义的枝桠省略。这样,历史的经验教训也被简化了,好像我们只要不被一个疯子愚弄,我们只
要克服自己的无故焦虑,一切就没事了。

有眼光的历史学家常常说:麦卡锡时代其实开始在麦卡锡之前。

\pagebreak
\section{一个小人物成为冷战开端}

冷战这个词一直挂在大家嘴上,但知道冷战是怎么开始的人大概并不多。

就在大家关注伊拉克战争的2003年6月,几乎没有人注意到,加拿大渥太华市政府在该市的督唐纳德公园,安放了一
块纪念铭牌。将近一年后的2004年4月,加拿大联邦政府,在同一地点也安放了一块纪念铭牌,纪念的是同一个人,
他的名字叫伊戈\textperiodcentered 哥萨柯(Igor Sergeyevich Gouzenko)。他是前苏联的一个叛逃者,他的叛逃
成为冷战历史上第一个最重要的国际事件,一些历史学家们认为,正是哥萨柯叛逃事件启动了冷战。最后大家同
意,怎么也该为他,也就是为一个历史转折点,立个碑。

伊戈\textperiodcentered 哥萨柯1919年1月13日出生在前苏联,1943年开始,他在加拿大渥太华使馆工作,具体工
作是为苏军总参谋部的情报总局(GRU)担任密码员。这个位置使他能接触苏联针对西方国家的间谍网。二战结束之
际,他得知自己即将调离回国,他不愿意回苏联,随即作出了一个危险的决定,他决定带着家人叛逃。

1945年9月5日夜晚,哥萨柯趁黑带着一个公文包,内藏密码本和一百零九份涉及前苏联在西方国家间谍资料的秘密
文件,步出了苏联使馆大门。下面的遭遇是他做梦也没有想到的。

哥萨柯先是直奔加拿大皇家骑警队(RCMP),警官却根本不肯相信他的故事。官方警察部门不接纳,他还能去哪里?
他决定诉诸媒体,把事情闹大,他就有安全。于是他转而投奔《渥太华时报》,谁知夜报编辑对他的故事毫无兴趣。
编辑告诉他,你应该去司法部的。他又夜奔加拿大司法部,那里没人值夜班。这个时候,他一定是真的害怕了。他
不会料到是这种结局,这是在苏联绝不会发生的事情,加拿大真是个无可救药的天真国度。他只能赶到住处,一栋
公寓大楼,就是这栋公寓对面的公园,今天安放着关于他的纪念铭牌。哥萨柯把家人藏在邻居家,然后通过邻居家
大门的钥匙孔,眼看着苏联使馆来人破门进入他的住处,直到警察赶来他们才匆匆离去。

第二天,哥萨柯带着文件再次来到司法部,这次他被接纳了。因加拿大属英联邦,也由于文件涉及面广泛,所以,
牵动了英国军事情报部门MI5 (Military Intelligence,Section 5),也通知了美国的联邦调查局(FBI)和安全局
(CIA)。在各方审看文件期间,哥萨柯被转移到距离渥太华不远的一个秘密营地保护起来。他做梦也没有想到,自己
仍然是不安全的。

\myphoto{image001.jpg}{1945年伊戈\textperiodcentered 哥萨柯在加拿大避难时的居所}

当时日本宣布投降还不到一个月。二次大战中,西方民主国家和苏联组成盟军对抗共同敌人的侵略。在加拿大总理
金(William Lyon Mackenzik King)眼中,哥萨柯假如是德国叛逃者,事情就简单,现在发生在``盟友''之间,就是
个外交麻烦。虽然哥萨柯带来的资料已经充分说明对方的危险性,可是,金总理的外交原则仍然令他感觉无从应对。
他下令加拿大政府不要介入此事,也就是说,不要给叛逃者提供保护。幸亏他的下属没有执行此令,哥萨柯才得以
和妻子一起在加拿大生活,养大了他们的八个孩子。他写过两本书,一本是得奖小说,另一本就是讲自己叛逃经历
的非虚构作品《这是我的选择》(This Was My Choice)。


\pagebreak
\section{冷战第一颗``炸弹''}

哥萨柯事件首先给加拿大带来了深邃影响。如金总理在当天所写:``如同在所有的一切之上,当头扔下一颗炸弹。''根
据哥萨柯提供的证据,加拿大立即逮捕了三十九名间谍嫌疑,其中十八名后来被法庭定罪,其中包括加拿大共产党
在国会唯一的一个代表。后来加拿大组成皇家调查委员会,专门对哥萨柯事件以及苏联在加拿大的间谍网进行调查。
历史学家甚至认为,正是这一事件,把加拿大的安全情报系统引入了``现代''运作。加拿大从此不再有古典天真。

这一事件在英国和美国引发了更强烈``地震''。根据哥萨柯提供的情报,这两个国家无疑也被苏联情报系统严重渗
透。英国最著名的案件,就是德裔英国公民,出色的理论物理学家克劳斯\textperiodcentered 福柯(Klaus
Fuchs),为苏联窃取核机密而在英国被定罪。

克劳斯\textperiodcentered 福柯出生在德国,1932年在德国加入了后来转为共产党的社会民主党。一年以后,他
遭遇纳粹暴力,逃往法国,继而移民英国,在那里完成了他的博士论文。二战期间,作为敌国公民他被在一个岛上
隔离过一段,然后,他获得信任,参加了英国的原子弹研究计划。1942年他加入英国国籍,也签署了``国家安全保
密协议''。根据GRU的文件,他在此之前已经通过一个原来德共朋友的牵线,成为苏联间谍,此后与他接头的也是一
个德国人,是为苏军情报局工作的德国女共产党员。克劳斯\textperiodcentered 福柯也辗转美国的几个州,参加
了著名的试制原子弹的``曼哈顿计划'',之后还参加了研制氢弹的工作。由于他的存在,苏联对美国的核计划以及
核打击能力,一直了如指掌。

克劳斯\textperiodcentered 福柯在1946年回到英国就开始接受英国安全部门的调查,可是,在英国的司法制度
下,调查旷日持久,他的认罪和定罪,都已经是1950年的事情了。他被判处十四年徒刑,只坐了九年多牢就被释放。
此后他移居东德,据一些记录,他还在东德为中国的一些核科学家教过课。

冷战是从西方国家被苏联间谍大规模渗透的暴露而开始的。

\pagebreak
\section{美国的``伊戈\textperiodcentered 哥萨柯事件''}

虽然伊戈\textperiodcentered 哥萨柯事件深度涉及美国,毕竟是个加拿大案件。可是,就在加拿大司法部接收哥
萨柯文件的整整两个月后,类似的事件也在美国发生了。1945年11月6日,一个为苏联情报部门工作的美国女间谍领
导人伊丽莎白\textperiodcentered 柏特丽(Elizabeth Bentley)向美国联邦调查局自首。哥萨柯事件尚在调查,又
在美国本土炸响这颗炸弹,确实非同小可。她的自首,牵出了一大群为苏联工作的美国间谍。

非常不同的是,柏特丽是个土生土长的美国人,1908年元旦出生。家族移民来自英国,1933年她在哥伦比亚大学获
得学习英语、意大利语和法语等三种语言的奖学金。她在意大利住了一段,先是参加了法西斯组织,又在男朋友的
影响下跳到政治上的另一个端头。1935年,她加入了美国共产党(CPUSA)。

1938年,柏特丽在纽约找到一个和意大利语有关的工作,这个工作单位涉及在美意大利人法西斯组织的活动。她向
自己的共产党组织汇报,主动要求做一个卧底间谍,组织上派给她一个间谍领导人格罗斯(Jake Golos),他是苏联
内务部(NKVD)在美国最重要的间谍之一,格罗斯移民美国后,在1915年加入美国国籍。柏特丽和格罗斯很快有了恋
情。直到两年之后,她才知道格罗斯是为苏联工作,换句话说,这时她才明白,自己早已成了``苏联间谍''。

美国是一个容许外国集团合法政治游说的国家,也就有一大批人,名正言顺地在为外国政府工作。为了把间谍行为
从合法行为中严格区分出来,1938年美国通过了《外国代理人登记法》(Foreign Agents Regisration Act)。1940
年,负责执法的司法部,迫使格罗斯登记了自己作为外国代理人的身份。如此一来,格罗斯就在更为严格的法律监
督下。要再像以前一样掌控庞大间谍网,显然就不``安全''了。于是,他的身份和工作就逐渐转给了柏特丽。就这
样,柏特丽成了代号为``聪明女孩''的苏联在美最重要的间谍网领导人。情报内容也大量是美国的军事情报。

凡事离不开``时间、条件、地点''。那是一个非常困扰的年代。

柏特丽后来自己也写了书,名为《解除束缚》(Out of Bondage)。另外有一个作家Kathry S.~Omsted为她写过传
记:《红色间谍女王》(Red Spy Queen)。在这本书里,这位传记作家称这一大批间谍为``浪漫理想主义
者''(romantic idealists),不能说没有道理。

他们大多为一个绝对平等的浪漫共产主义理想所吸引,这个理想上世纪初在美国也曾随着工人运动而声势强大,也
因理想在苏联取胜而振奋。但共产主义在美国仍然并非主流,有着对它的各种批评和反对的声音。二次大战开始,
美苏成为盟国盟军并肩作战,这样的反对声音就几乎自动消失了。

\myphoto{image003.jpg}{1948年8月柏特丽在听证会上}

\pagebreak
\section{``浪漫理想主义''和二战后新的``左右''关系}

不论从理论还是现实,我们直到现在好像常常还是只能用两分法\myrule 左和右\myrule 来描述复杂的政治关系。
这是一个定义含混的划分。尤其是在上世纪三四十年代,当时好像苏联就是``左''的代表,而希特勒这样的就是
``右''的代表。那么英美这样的民主国家算什么呢?即便把前二者送入``极左''、``极右''两端,难道英美和纳粹当
归为一类,只是``温和''与``极端''的差别吗?

二战结束之后,德国的纳粹以及意大利法西斯主义至少暂时消退了,不再成为一个威胁世界的国家力量。却立即出
现冷战对立双方,也就是对立的苏联和西方民主国家。至少从理论上、在大家含含混混的概念中,接着二战似乎续
上了一个新的``左、右''对立关系,我总觉得是``something wrong'',哪里肯定不对了,至少是简化了历史。但
是,这至少对于苏联阵营的宣传是非常有利的。好像是苏联阵营原地不动,顺理成章保持原来``反法西斯阵营''名
号,而资本主义的西方国家理所当然地因与苏联对立,而被送到原来法西斯的``右''的位置上。相信共产主义学说
的人们,逻辑很容易就顺过来:一个在法西斯倒台后就接着和我们对立的,不是又一个法西斯还能是什么?更何况,
列宁的资本主义\myrule 帝国主义学说,不仅是理论,已经上升到政治信仰,信的就是西方资本主义国家``必定''
要走向``具有侵略本性的帝国主义''。

仔细想想,这个定位概念,对二战之前的西班牙内战以及佛朗哥政权的复杂性,起了重要的现实范例的作用。西班
牙在内战前是一个很特别的情况,在结束君主制之后,开始民主选举,可是却没有一个成熟的社会和政治理念在背
后支撑,结果一方选上推行的措施,是另一方绝对无法接受的,国家就在两极跳跃中推向极端,最终在野一方感觉
连安全可能都无法保证。1936年左翼选上之后局面大乱,最终在议会的一名右翼议员被杀之后,导致在野一方起兵。

西班牙内战其实是二战前发生在一国土地上的国际大战。共产国际动员了全世界(包括中国)的成员组成国际纵队支
持西班牙共产党,美国支队成为主力,许多美国人血洒西班牙;而希特勒的德国和墨索里尼的意大利都支持了佛朗
哥。西方民主国家的政府,却因为交战双方都是它们所不赞同的极端,因而按兵不动没有介入。西班牙内战对美国
左翼带来的强烈刺激,久久不散。

最难以解释清楚的是,假如不了解当时的西班牙国内政局,仅从表面上看,包括共产党的左翼是通过民主选举执政
的,而佛朗哥是一个得到法西斯国家军事支援的右翼政变。所以,共产国际举的是保卫民主的大旗,共产主义的浪
漫理想和民主口号在这一点怦然结合,因而,才能够在全世界吸引如此众多的理想主义者,愿意死在和他们毫不相
干的战场上。而内战后的西班牙更有另一层复杂:事实上佛朗哥和希特勒、墨索里尼却并不能画等号,佛朗哥的西班
牙不仅在二战中保持了中立,尤其在后期,还实际帮助了盟军,甚至为德国犹太人的逃生大开方便之门。在二战
后,西班牙是对外开放的。美国政府看到这一点差别,二战后最早和佛朗哥西班牙开始关系逐步正常化。可同时佛
朗哥在国内仍然镇压左派。在美国共产党眼中,美国政府战后和佛朗哥西班牙的交往,是坐实了美国将走向帝国主
义的列宁理论,暴露了资本主义和法西斯为一路的真面目。苏联人有这样的信仰,一心支持自己的政府,心理上就
比较顺理成章。问题是一大批美国左翼怎么办?他们不信任自己国家的理念,认为站在爱好和平的苏联一边是正义
的,阴差阳错,就有一批人成了苏联的卧底间谍。虽然美国苏联一度是盟国,可他们当然一开始就知道为外国政府
盗取自己国家的情报,这是违法的。在冷战开始后,这就更明确是与自己国家为敌的行为。

类似的逻辑也曾是上世纪初大量中国知识分子的心路历程。他们更是长期被国家的内忧外患所刺激,信仰使得他们
在国民政府卧底,收集政治、军事情报,甚至成为颠覆政府的武装力量的朋友,或者成为他们的``地下工作者''。

\pagebreak
\section{美国的恐惧和焦虑}

对于二战后的主流美国来说,苏联和纳粹德国在内政制度上,有着许多相通的地方;而苏联阵营``解放全人类,赤
化全球''的美好理想,在对方看来,就是有武力输出革命打上门来的潜在战争危险。冷战双方思维方式完全错位,
双方的感觉是鸡鸭互不相通的。而这个时候,一场导致使用核武器的世界大战刚刚结束。核武器的威力使得所有人
都笼罩在世界末日可能来临的恐怖中。因此,如何保持核机密成为二战后最焦虑的事情。在苏联拥有核武器之
后,1962年苏联暗自在古巴部署导弹,最大射程为四千五百公里,射程几乎涵盖美国所有重要城市,包括首都华盛
顿。预警时间只有五分钟。苏联还同时在古巴部署了轰炸机。虽然危机最终在十三天内通过外交手段解决,但是,
在当时的美国哪个敢保证,核武冲突,卷进双方阵营的各个国家引发第三次世界大战,就一定是绝对不可能的事情
呢?尤其在朝鲜战争之后,威胁就更为现实了。美国当时的中小学,常规的突发灾难训练之一,就是核弹爆炸了怎么
办。

与此相对应的是,苏联间谍的一大目标,就是包括核机密的军事机密。

柏特丽的自首只是一个偶然。她自己酗酒,也有精神方面问题,苏联一方面对她不放心,一方面也盼望跳过柏特
丽,直接掌握她手中的间谍网。她和苏方冲突日剧,开始担心自己的生命安全,她的联系人也确实在建议``摆脱
她''。就在情绪不稳定的时候,她发现自己的一个间谍,美国共产党党报的编辑Louis Francis Budenz,因为信仰
改变已经自首。这终于推动柏特丽也走出向联邦调查局(FBI)自首的那一步。

FBI决定隐下柏特丽的自首,让她转而为美国情报部门工作,以获得更多苏联在美间谍的情况。这个决定自然是一个
最高机密。可是,FBI还是通知了盟友英国情报部门,谁知英国新建的对苏情报部门负责人费尔比(Kim Philby),恰
是为苏联服务的双面间谍。他向苏联通报,苏联立即切断了和柏特丽有关的一切联系。直到1963年费尔比逃苏,柏
特丽自首被暴露的谜底才算揭晓。在当时,这个最高机密如此轻而易举就立即泄了密,对FBI带来的恐惧和刺激可想
而知。

与此同时,美国发现自己的司法制度,要处理核时代的间谍危机,显得十分软弱。核武时代,外部威胁千万倍暴
长,但是,要给间谍定罪却难上加难。

柏特丽交出的间谍名单有一百五十人,其中有三十七名联邦政府的雇员。这个名单和FBI掌握的一些间谍,以及伊戈
\textperiodcentered 哥萨柯提供的名单有一部分重合,所以,FBI从一开始就判断这是一个真实名单。这个判断在
苏联解体之后、在一些前苏联情报官员的回忆中,也再次得到证实。

可以想象,假如是在苏联阵营,事情一出,可以大逮捕大清洗。苏联政治大清洗在上世纪初就大规模演出过,对于
清洗制造的大批冤案,它在制度上没有什么约束和顾忌。立国之初,美国就建立在一个非常理想化的治国理念上,
打算让民众具有最大限度的自由,尽最大可能避免冤案是公民自由的重要保障措施。因此,作为宪法修正案的权利
法案,十条中就有五条涉及被告权利,其中第五条有一句是:被告``不得被强迫在任何刑事案中自证其罪''。因此,
在美国给间谍定罪是一件非常困难的事情。除非是你在交接情报的时候当场拿获,否则很难说服陪审团,不可能按
照一个人的揭发举报就定间谍罪。根据``第五条'',你也不能强制他坦白,所谓``自证其罪''。美国法庭上,被告、
证人说``take the Fifth''(采用第五条),是经常可以听到的一句话。

所以,柏特丽间谍名单上的一百五十人,绝大多数在审前的大陪审团面前、在国会听证会上,不置可否地宣布自己
``take the Fifth'',有少数人则宣称自己无罪。其结果是一样的,就是无法定罪。也就是说,这些人不仅不能惩
罚他们,还必须照常让他们在原单位上班,包括那三十七名联邦政府雇员。假如单位把他们调离原来能接触敏感资
料的位置,他们反过来可以告工作单位迫害。

Lauren Kesster也写过柏特丽故事,书名是《聪明女孩:伊丽莎白\textperiodcentered 柏特丽》(Clever Girl:
Elizabeth Bentley),书的副标题直接点出柏特丽事件的要害,这是``一个引向麦卡锡时代的间谍''(the spy who
ushered in McCarthy)。为什么这么说?

因为在一筹莫展的焦虑中,麦卡锡似乎是指出了一个折中的解决方案。

\pagebreak
\section{麦卡锡主义和一个漫画艺术家}

``麦卡锡''究竟是什么呢?当然,约瑟夫\textperiodcentered 麦卡锡(Joseph McCarthy)曾经是美国的一个参议员。
从1950年至1954年间,主持了一系列对美国政府机构是否被亲苏联势力渗透的调查。但是今天人们说``麦卡锡'',
当然已经超出了一个``人物''的概念。很多美国人的印象中,麦卡锡就是指``麦卡锡主义''。

很难想象,麦卡锡主义这个词,竟然是一个漫画艺术家``发明''的,他的真名叫布洛克(Herbert Lawrence
Block),大家都只熟悉他的``画名''赫布洛克(Herblock)。他专为报纸画政治讽刺漫画,当时已经为《华盛顿邮报》
工作了四年。这个专业需要极强的政治敏锐度,捕捉政治事件中的新闻性,再概括转换为一个变形夸张的艺术表达。
要能够一下子抓住视线,让大家认同、得到启发,又要是准确的,有思想含量。不是只在事情发生当下去迎合大众
去煽情,而是判断要经得起时间考验,哪怕多年后回头去看,要让读者有诠释历史的感觉,觉得还``真是这么回事
儿''。这行要干好,不容易。

赫布洛克算是这一行的奇人,他生于1909年,一共得过三次普利策新闻漫画奖,分别在1942年、1954年和1979年。
1994年,作为一个漫画家,他还获得了很难得到的总统自由奖章。1946年,他从二战战场上退役,进入《华盛顿邮
报》,从此一直不断工作,直到``9.11''事件后的2001年10月去世,他就没退过休。活九十一岁不稀奇,工作到九
十一岁实属罕见。他去世前六个星期,报纸还刊出了他画的最后一张漫画。他的漫画是耐看的,因为有艺术家本人
的独特思考在里面。赫布洛克一直在出漫画集,陆陆续续出了十三本。2000年的最后一本:《赫布洛克的历史:来自
千禧年迸裂的政治漫画》(Herblock's history: political cartoons from the crash to the millennium),还是
由美国国会图书馆为他出的。直到2008年的去年,华盛顿斯密松宁博物馆的美国肖像画廊,还开了他的专题画展,
题目是:``赫布洛克笔下的总统们。''

那么,赫布洛克在他的艺术表达中,究竟什么是``麦卡锡主义''呢?在画面上,几个人推着一只愁眉苦脸、万分不情
愿、挣扎后退的大象,大象是共和党的象征,它被推向一个摞一个、直堆到高处的一摞油漆桶,在摇摇晃晃的最上
端那个桶上,写着``麦卡锡主义''。漫画题目是大象的话:``那意思是要我站到这上面去吗?''

\myphoto{image004.jpg}{赫布洛克关于麦卡锡主义的漫画}

\pagebreak
\section{麦卡锡参议员至今是个争议人物}

虽然麦卡锡被抽象出来成为一个象征,但他当然首先是个人物。今天美国人对麦卡锡参议员的评价不说是一边倒,
也是绝大半边倒。这来自学校的历史课本,他在美国是个基本被盖棺论定的人物。但是,也是美国的好处,还是有
相反意见的表达。同时,虽然对麦卡锡的描绘有大量两极化倾向,却也有一些人决意保持自己的理性。

例如,三年前的2006年,北卡罗莱纳州的汤姆\textperiodcentered 维克(Tom Wicker)还写了麦卡锡的简要传记。
1957年,他曾被家乡的一家报纸派往首都成为驻华盛顿记者,年轻的他突然在夜晚的走廊里和麦卡锡邂逅。一个曾
经如此有名的风云人物向他伸出手来,令维克深深难忘。也许是麦卡锡人生的变幻莫测,让他给自己的书起了这样
的书名《流星》(Shooting Star:The Brief Arc of Joseph McCarthy),副标题是``麦卡锡的主要轨迹'',这本书
的轨迹线索抓得很紧,作者虽然基本上对麦卡锡持批评态度,但还是可以看到他尽量维持立场中立的努力。

我在一开始就提到过,有不少麦卡锡传记是以``歇斯底里''来描绘这个参议员的。一些批评参议员麦卡锡的书,通
常让人感觉他从一开始就是一个道德和政治品质有问题的人。

可也有唱反调的。不仅有当事人的辩解,如戈登瓦特(Barry Goldwater)参议员在1979年出版了《无歉意:参议员
Barry M.~Goldwater的个人及政治记忆》(With no apologies: The personal and political memoirs of United
States Senator Barry M.~Goldwater),就很具体地为自己参与的麦卡锡调查辩护。其实直到今天,还源源不断有
为麦卡锡辩护的新一代,例如,保守派女作家科尔特(Ann Coulter)。她写过七本书,本本进入《纽约时报》畅销书
榜,其中就包括在2003年写的《背叛:从冷战到反恐战争中自由派的背叛行径》(Treason: Liberal Treachery
from the Cold War to the War on Terrorism)。在书中,她从对麦卡锡的看法,直到对今天反恐战争引发问题的
看法,都和海涅斯\textperiodcentered 约翰逊在《焦虑的时代》中的观点,全面针锋相对。在这本书里,她指责
美国自由派在诋毁和妖魔化参议员麦卡锡,她心里的冲冲怒气几乎从书名就可以看出来。

这种长达六十年的不同观点的表述,至少让回顾历史的人有了更丰满的历史观。

\pagebreak
\section{关于一部纪录片的争论}

``麦卡锡主义''的提出,起于一张漫画,而大多历史学家认为,麦卡锡时代的结束,是来自于莫罗(Edward R.
Murrow)主持的系列电视片See It Now。这是一个CBS的系列节目,曾经获得四次艾美奖和各种奖项。有关麦卡锡的
只是其中一集,在1954年3月9日播放。这一集的片段后来反复在讨论麦卡锡时代的电视里播放,在今天看都能产生
很强的印象。最具打击性的是,这部电视片集中了麦卡锡党派性、情绪化,甚至作过度政治指控的发言摘要,例如
指责民主党``背叛美国二十年'',指责``美国公民自由联盟''是``为共产党服务'',尤其是对Zwicker 将军言词激
烈的指责,让人无法接受。这部影片给了人们最感性的麦卡锡印象。

也有人认为这部影片不公平。历史学家赫尔蒙(Arthur L.~Herman)写过麦卡锡传记《约瑟夫\textperiodcentered
麦卡锡:重新检视美国历史上一个最遭恨参议员的一生和遗产》(Joseph McCarthy: Reexamining the Life and
Legacy of America's Most Hated Senator)。这本书认为,拍摄影片的莫罗小组花了两个月时间精心编辑,是有意
编排采用麦卡锡不雅观的``形象镜头'',例如挖鼻孔之类,以毁坏他的形象。而沒有在影片中呈现他一个正常工作
的状态。他认为,那些发言片段也是精心剪裁的结果。直至今天,反对者仍然认为,这``根本不是一个报道'',而
是精心策划断章取义的``全方位攻击''。

对这部影片的批评也来自一些反对麦卡锡的人,例如,后来当过肯尼迪总统助手的记者、编辑柯戈雷 (John
Cogley),他当时为一个天主教政治、文化杂志《公益》(Commonweal)工作,这本杂志一直反对麦卡锡的调查方式,
柯戈雷也以多年坚持反麦卡锡著称,可是,他也``毫不留情地批评莫罗和制片人不顾事实,筛选镜头''。柯戈雷指
出,要是选另一些镜头,就很容易把麦卡锡打造成一个很正面的形象,他声称,``这可不是令电视新闻业自豪的时
刻。''并且警告说,要警惕``不当利用电视手段''。

上世纪五十年代,电视还刚刚开始在美国流行,世界上很多人还只是隐隐约约知道,甚至根本不知道什么是电视。
到现在,全世界都已经在娴熟使用影视的筛选、编辑、剪接手段,来表达和宣传自己的政治观点。例如在美国也引
发激烈争辩的《华氏911》(Fahrenheit 9/11)。也是赞成者极为推崇,反对者强烈抗议的影片,后者认为这只是利
用影视手段编造的欺骗宣传。此类争辩往往发生在已经具有固定看法的观者中间,对于普通观众来说,一般很难摆
脱精巧的影视陷阱,因为影视的视觉冲击几乎是所向披靡无可抵挡。

就这部有关麦卡锡的纪录片本身,究竟其表现是``当''还是``不当'',到现在还是两边各自表述。但是,这部纪录
片在今天对一代代美国人的历史教育中确实起着很大作用,作为视觉印象,这一题材没有等量齐观的替代品。和书
籍出版不一样,六十年过去,事实上并没有出现一部与它抗衡的同类影片。

\pagebreak
\section{争论背后的党派分歧}

如此延续六十年的争论,实际上反映出了美国政治党派的历史差别。在麦卡锡时代,不仅麦卡锡本人属共和党,基
本上也是共和党人在主推麦卡锡调查。站在被调查的对立一方,大多为共产党人,以及思想左倾的民主党人。而最
后推动结束麦卡锡调查,民主党是一个主力。这段历史深刻影响了后来的美国,历史情结也加剧了党派情绪,延续
至今。

大家都知道,今天美国主要是两大派别,民主党与共和党,分别代表民众中的自由派(liberal)和保守派
(conservative) 的思想倾向。由于麦卡锡时代在美国是个很深的历史烙印,而今天美国的主流价值中,共产主义又
是一个公认很负面的词。看今天的历史讨论有时就会隐隐感受到弦外之音,感受到两边对党派政治比较投入的那部
分人,在强烈涌动着的情绪,六十年不能平静。

这么说吧,在党派情结比较深的人心里,既然这一事件有党派关联,那么,``麦卡锡''假如是一场毫无道理的对思
想的政治迫害,它就成为共和党,或者说保守派的``党派耻辱'',是始终可以被对方用来指责的一个武器,例如在
今天反恐战争发生分歧的时候,就可以以此指责共和党政府采取的应对政策,只是简单延续了麦卡锡主义。只一句
话,对方就可以闭嘴了。因此,共和党和保守派中那部分气愤难平的人,更认为一切争论必须追溯到麦卡锡,应该
据理力争,让另一面事实说话,通过``指出真相''来为麦卡锡平反,以扭转这个被动局面。更何况,他们感到自己
看到的那部分事实,没有被``好好说说''。所以,对一些党派意识比较强的人来说,对麦卡锡的评价,是给两党作
``政治正确''孰是孰非论断的关键所在。

发明``麦卡锡主义''一词的赫布洛克,今天被大家认为是个有自由派倾向的人,例如他支持小罗斯福的经济政策等
等。可是他与那些在六十年前对共产主义持有单纯幻想的许多自由派人士并不相同,他在1954年获得普利策新闻奖
的那张漫画,是赫布洛克在斯大林去世后表达他对斯大林统治的判断:画面上是套着袍子的死神,对刚来到面前的约
瑟夫\textperiodcentered 斯大林说:``约瑟夫,你一向是我最了不起的朋友。''

对一张漫画,最笨的事情大概就是要用语言去解释它是什么意思。不解释兴许还有点意思,一解释可能就彻底没意
思了。可赫布洛克是第一个``发明''麦卡锡主义这个词的人,就躲不掉被要求出来解释这张著名的``麦卡锡''漫画。
他解释得还是很得体,``我对此并没有特殊灵感,这画只是简单用来表达一个全国范围内的苦恼,除此之外难以用
任何其他方式表达。''

赫布洛克发表这张漫画,是在麦卡锡被卷入麦卡锡主义之后刚两个多月时画的(我知道这话听上去有点语病,可既然
``麦卡锡主义''一词的定义已经超出了他的个人局限,这么讲应该也说得过去)。麦卡锡如日中天,事件还在高潮
中,赫布洛克却是冷静的。我看到这张漫画、读到作者诠释,已经过去了将近六十年,我已经看过无数版本对麦卡
锡主义的谴责和辩护。可是,我回看赫布洛克,仍然能感受到他的与众不同。他超越了过度敏感的党派对抗,他看
到事件背后的复杂性。在他眼中,主导麦卡锡调查的共和党是无奈的,也是正在``苦恼''中的美国一员。

\pagebreak
\section{两次``惧怕红色''}

人们常把美国民主党代表的自由派,和中国知识界经常提到的自由主义者相提并论,其实他们并不是一回事。一些
在美国生活过的作者,想对读者解释这个差别,就根据自己的体验,概括地作出一个对应:就是美国的自由派对应中
国的左派,而中国的所谓自由主义知识分子,对应美国的保守派。

这个说法也并不准确。例如,今日美国两大派对核心价值观理解和认同的一致性,远远超过了中国所谓``左、右''
两派的共性。但是,从历史发展的角度去看,这个说法又有它一定的道理。今天的美国老自由派人士,在历史上,
一些人可能加入过共产党,不少人曾经至少是同情过共产主义的主张。``红色''、``惧怕红色''都是美国历史上对
立的政治指责。这只是一百年前开始的世界思潮冲撞,在美国的必然折射。

一些历史学家把麦卡锡时代称为美国的``第二次惧怕红色''(The Second Red Scare)。所以,有些人想强调,它和
外部环境的压力无关,早在冷战中一系列苏俄红色间谍案发生之前,美国不是已经有过第一次``惧怕红色''(The
First Red Scare)了吗?麦卡锡时代只是``第一次惧红''的延续,只是对一种不同社会制度理想、对不同思想的无端
恐惧和不宽容,原因只是因为它和``我们''不一样。所谓``我们'',就是``美国主义'',简单说就是美国不能容忍
异端思想,是异端思想的迫害。

前面提到的女作家Ann Coulter一直坚持为麦卡锡个人辩护,在一次采访中,Ann Coulter要求福克斯新闻台(Fox
News)的著名主持人Bill O'Reilly举出一个被麦卡锡调查所伤害的无辜者例子来,O'Reilly举出了一个剧作家杜鲁
勃(Dalton Trumbo)。Ann Coulter马上反驳说,这是和HUAC有关的调查,发生在1947年,和1950年才开始推动参院
调查的麦卡锡并没有关系。这个例子被用于为麦卡锡参议员个人辩护当然有用,可是大家在讨论的``麦卡锡''已经
超越了``个人''。它不能用来为``麦卡锡主义''辩护。这里提到的HUAC,就是``众议院涉非美活动委员会''(House
Committee on Un-American Activities)。这个HUAC,也早就被划在麦卡锡主义之内了。

两次``惧红'',确实都和共产党有关,都和红色根据地苏联关系密切,也都和世界大战挨得很近。可是假如说麦卡
锡时代人们眼中的温和左翼是民主党、极端左翼是共产党的话,那么在第一次``惧红''中,共产党相比之下还算是
温和的粉红角色,更为激进深红的当是无政府主义。

\pagebreak
\section{第一次``惧怕红色''}

对``惧红''简单定义为无名恐慌和异端迫害,在今天很容易被美国年轻人接受。他们环顾四周,看不到美国共产党
的任何影响,在国际上也很少再有和美国彻底对立、形成严重威胁的红色外部压力。人们会根据今天的经验判断过
去,这是很自然的反应。他们很难真正体会,近百年前的世界和美国是如此不同。

第一次惧红时期也被称为``红色大恐惧''(The Great Red Scare),开始于1917年。不但是美国本身社会状况复杂、
内外交困,放大到世界范围,说世界大乱也并不为过。

那几年,世界第一次遭遇进入热兵器的现代世界大战。世界第一次遭遇一个共产主义政党成功暴力推翻了一个大国
政府。共产主义思潮如日中天。

对美国来说,美国长期奉行立国时定下的孤立主义原则:不要参与欧洲人为利益争夺的战争,好好过自己的日子。而
威尔逊总统根据他对世界新形势的判断,却决定以参战方式,把这场将世界拖入灾难的世界大战``打停板''。于是
美国第一次遭遇重大的参战决定。结果是,一次大战中的美国社会远不像在二次大战中那样同仇敌忾,相反出现了
大量民间反战反政府宣传。

一切都搅在一起:比如说劳工权益意识的觉醒,很快和各种说法的共产主义、无政府主义理论和信仰搅在一起。苏俄
反战对比了美国的参战,使得左翼民众对美国政府是否``恶变'',也形成思想混乱。反战的不仅有产业工人组织、
有美国社会主义政党、有革命组织,还有和平主义团体和德裔美国人团体等等(美军是在和德军打仗)。而在苏俄榜
样下,不少美国人,认为解决问题的办法就是推翻美国制度,追随苏俄,实现建立一个共产党国家的目标。更不要
说无政府主义者了,他们出生出世的目标,就是消灭政府。

而在美国政府一头,一边参与着世界大战,另一边,处理社会危机与处理战时紧急状态也搅在一起。美国邮局是联
邦机构,当时邮局设监察员,决定凡是认定是对美国参战取胜有破坏作用的宣传材料,就不给递送,结果一些极端
的、无政府主义的出版机构因此无法生存。这简直是一场猫捉老鼠的游戏。比如说一个意大利裔无政府主义组织,
广泛寄发一份名为《给你带来健康》的小册子,其实内容是教你怎样做炸弹。面对巨大社会紧张,美国在1917年通
过了``间谍法''(the Espionage Act),这个立法有战时法的意味,例如界定``凡是干扰军队行动和取胜''就是一种
罪行,除了针对间谍,显然也针对了当时号召抵制兵役等等的反战运动。

1918年国会更是通过了``煽动暴乱法''(the Sedition Act),变成了``间谍法''的扩张。这一法案界定``对美国政
府、国旗或战时的美国军队'',使用``不忠诚、亵渎、粗鄙恶劣和辱骂的语言''都是一种罪行。同时也界定:书写、
印刷、出版任何反政府文字,都是一种罪行。这在美国是很不寻常的立法,因为美国从立国开始,就明确在宪法中
维护公民的言论表达自由。这个立法重新界定了``罪与非罪''的界限,把一部分原本受宪法保护的辱骂政府和国家
象征的言论表达,从言论自由的范围内给划出去了。虽然它实际上想防止的,是对民众的激进的大规模暴乱的煽动。

\myphoto{image006.jpg}{``闭嘴!''\myrule 美国上世纪初关于``间谍法''的政治宣传画}

\pagebreak
\section{对执行``煽动暴乱法''的迟疑}

今天生活在一个成熟社会,美国人自然觉得这样的立法匪夷所思,可是,只要看过电影《列宁在十月》、《列宁在
1918》的人,就会理解这些立法是在怎样的刺激下产生。今天没有哪个人看到列宁站到台上,对劳工抛出手去大声
疾呼:``我们无产阶级、革命的、同志们……''会担心社会大暴动。但在那个时候,这样的鼓动,在每个刚刚结束传
统农业社会、飞速开始工业革命的国家,都可能引发``冲向冬宫''的暴动。因为当时新兴工业国遭遇的第一次,还
有产业工人的大规模集中和贫困、劳资关系的紧张、还不知如何处理这些爆发的新问题。演讲中列宁的手,有力地
抛出了俄国国界。今天我们谈论非常彻底的思想和言论自由,和社会成熟给出了很大空间有关。可在当时情况下,
新思想的力量、语言鼓动的力量,对聚集的民众来说,等同于对着一堆浇了汽油的柴堆,扔出一个火把去。

社会尚不成熟。各种极端思潮和追随者的左翼组织,正风起云涌建立起来。这一期间,美国共产主义劳动党党员大
约在一万到三万之间;美国共产党在三万到六万之间;社会党大约三万九千人。自认是社会主义共产主义者的美国
人占了工作人口的约百分之二十。1919年,美国共产党在它的成立宣言中说:``共产主义不是打算要'夺取'资产阶级
议会的国家,而是要征服并且摧毁它。''而这些激进组织,和无政府主义马上卷起袖子干的劲头相比,还是小巫见
大巫。

现在看来,当时立法有点迫不得已快刀斩乱麻的意思,好像先压住阵脚再说。实际上,在立法后,``煽动''事件还
是不断发生,可是美国政府的行政分支却迟迟没有真下决心动用这些条文去惩治,好像只是准备在最坏情况下迫不
得已再用。一个典型就是第五十一任司法部长帕尔默(Alexander Mitchell Palmer)。在第一次世界大战期间,威尔
逊总统曾任命帕尔默为战争部长,被他谢绝,理由是帕尔默自认是个和平主义者。他后来出任司法部长,做的第一
件事情,就是把战争期间作为``敌国公民''集中居住的一万德国人立即全部释放。他手里有``煽动暴乱法''这个武
器,可以轻易用来起诉各路极端分子和煽动者,可是他却迟迟没有动作。为此,帕尔默饱受各方责难,当时的国会
领袖们以及主流媒体,都强烈要求联邦政府依新法逮捕起诉极端活动分子,甚至要求将其中大量非美国公民驱逐出
境。当时的《纽约时报》就称那些参与无政府运动的移民是``反对美国政府的煽动者、无政府主义者和阴谋者''。

\pagebreak
\section{一个和平主义者面前的现实}

我想,作为司法部长,帕尔默非常清楚这个``煽动暴乱法''是一个焦虑时代的紧急应对,此法本身和宪法是有一定
冲突的。他一定也明白,立法是一回事,真正可能在历史上留下骂名的是那个执行者。

所以,帕尔默看上去是尽量不过度反应,不去利用现成法律条文给``言论出格''的人定罪。这样的处理方式在美国
很普遍,在当时一些州也有相应的有关``煽动罪''的州立法,有的立法保留了几十年也没有废除,其原因是,在特
定的社会紧张时过境迁之后,检察官并不去用它起诉和定罪,就像一件历史物件被大家遗留那里。

可惜,极端思潮形成的危机确实存在,你要忽略它,它还不肯放过你,而这些危机引发的暴力完全可能来自一个美
好理想。1996年,普林斯顿大学出版社还出过一本《无政府主义者的声音:美国无政府主义的口述历史》
(Anarchist Voices: An Oral History of Anarchism in America)。作了一百八十人的口述采访。这些美国无政府
主义者都表示,把他们集合在一起的是一个乐观信仰:他们坚信所有问题都来自政府,当政府消失之后,人民都将生
活在一个和谐社会之中。问题是,现有的``政府''又如何让它消失呢?比如说,美国政府。无政府主义的答案之一
是:炸掉它。

于是在1919年4月底,无政府主义者组织发起了一波邮包炸弹攻击。他们准备了一批邮包,还计划让这些邮包在五一
国际劳动节那天在接收者手里炸响。这个节日被无政府主义、社会革命主义团体等认为是他们共同的节日。这些邮
包都包着鲜艳的绿色包装纸、盖着商家样品的印戳,里面却是一个要命的炸弹。

西雅图市长在五一之前就收到了邮包,幸亏他的办公室人员误拆了炸弹的另一头,所以并没有爆炸,邮包当然立即
被送进了警察局。佐治亚州的参议员哈得维奇(Thomas W.~Hardwich)就没那么幸运,他的女仆在拆邮包的时候被炸
飞了两只手,哈得维奇夫人脸部严重受伤。

有的炸弹邮包因邮资不足被截拦。而已经被发现的炸弹邮包引起警觉,也幸亏炸弹邮包的包装相同,邮局迅速通知
检查和截拦。纽约市的一名邮局工作人员,一下子截下了十六个炸弹邮包,也就救了纽约市的市长和警察局长,因
为也各有一个准备寄给他们两位的炸弹。纽约市之外的其余邮包也被陆续截拦。虽然这一批只有三十个邮包炸弹,
可是只要细看收件人名单,就可以看出策划者要全面摧毁美国的决心。

这个名单有美国国会来自各州的参议员、众议员,有行政分支的部长、行政官员,有法官、检察官,甚至最高法院
大法官;有包括洛克菲勒的金融家企业家、有报纸编辑,还有警察局长、联邦调查局探员,有州长、市长。

另外值得一提,有一个被截拦而没有完成使命的绿色邮包,它的收件人正是这位和平主义者的联邦司法部长,帕尔
默。

\pagebreak
\section{从邮包炸弹到华尔街大爆炸}

司法部长帕尔默侥幸躲过了无政府主义的邮包炸弹,却不知更大危险还在后头。攻击者已经不满足小炸弹,正在策
划下一波更大威力的袭击,准备在八个不同城市同时炸响。

这一批中型炸弹计划袭击的目标全部是政府中人。帕尔默再次被列入名单。距离上次的邮包炸弹才不过一个
月,1919年6月2日,炸弹如期在八个城市爆炸,包括首都华盛顿,但是,没有一个他们打算炸死的人送命,却造成
误伤死亡:一个警察,还有一个过路妇女。前总统弗兰克林\textperiodcentered 罗斯福夫妇都差点被炸进去,他们
住在对街,正好在步行回家的路上。

随所有炸弹,炸弹手还送出粉红色传单:``战争,阶级战争,这是在你们所谓秩序的强大制度下,在你们法律的阴影
下,率先发动。必将流血,我们不会躲藏;必将谋杀,我们杀人是因为必须如此;必将毁灭,我们破坏是因为必须
去除被你们残暴制度控制的世界。''

最恐怖的还是发生在司法部长家门口的爆炸。被派到帕尔默家去扔炸弹的叫凡蒂诺奇(Carlo Valdinoci),是一个无
政府主义出版机构的编辑,当时好像还没有自杀炸弹这一说,他是打算完成任务后活着回去的。可是,不知是发生
了什么意外,有人猜测他是在台阶上绊了一跤,正好摔在炸弹上,总之就在帕尔默家门口,他把自己炸成了碎片。

这次设计的每个炸弹有二十公斤炸药,威力巨大。帕尔默的几家邻居都被爆炸气浪从床上掀到地上,而且炸弹是被
装在厚重的金属容器中,一旦炸开,金属碎片极具杀伤力。警察在两个街区之外都还找到有凡蒂诺奇的肢体残余,
他的头发头皮被甩上了对面房子的屋顶上。帕尔默一家都没有受伤,可是受到的震撼确实无法形容。

无政府主义者的炸弹袭击不仅针对官员住所,攻击目标还包括法院、警察局和教堂等等。大大小小炸弹还在源源不
断而来。高潮是一年多以后的纽约市华尔街大爆炸, 1920年9月16日中午十二点,一辆满载的马车,行驶到华尔街
金融中心的中心地带,马车夫在摩根总部大楼对面停下,车夫马上就溜走了。谁也没有想到,车上装着一百磅炸药
和五百磅铸铁小部件。几分钟后,定时炸弹引爆,那匹不幸被选中的马随马车被炸得粉碎,铁件如子弹喷射出去,
当场炸死三十多人,炸伤四百多人。死者只是一些普通人,书记员、交易员等等。直到今天,路人还能看到附近几
栋楼墙上留下的炸痕。所以,前不久美国历史书俱乐部还在提醒大家:``9.11''并不是纽约市遭遇的第一次恐怖袭
击,八十九年前的华尔街大爆炸才是。

\pagebreak
\section{躲不过的宿命}

一直顶着各方指责、迟疑着不肯动用``煽动暴乱法''来逮捕起诉激进分子的司法部长,终于被炸得逼到这条路上。
一些人认为,他后来的行动是因为个人受到的刺激,那天爆炸,他和妻子孩子刚刚睡下,假如不是杀手出意外,被
炸成碎片的可能就不是凡蒂诺奇而是帕尔默和家人了。更明显的事实是,他是被盯上了,司法部长已经被人家认定
是必须干掉的阶级敌人。我想,在一个民主国家,在历史事件中揣测个人动机,有时没多大意义,那不是一个``个
人说了算''的国家,而探讨事情的大局发展和推动,才是有意义的,因为可以看到整个国家面对历史难题的反应、
思考和发展步履。就这个案子来说,作为一个``个人''遭遇这样的刺激,当然会改变他的看法。可是,也必须看到
帕尔默开始采取行动,绝非是一个``个人报复''事件。帕尔默被追着炸,不是因为他是帕尔默,而是因为他是美国
司法部长,是正在被一些极端组织打算消灭的美国政府的一个部分,他家的炸弹是对美国攻击的一部分。他采取行
动的反应不是个人反应,是一个国家政府的必然反应。哪怕这一系列炸弹中没有对他个人的袭击,作为司法部长,
他也一样坐不住的。

帕尔默和平主义的来源是他的基督教教友派信仰,教友派在美国是出了名的``和平''。相对来说,帕尔默应该比其
他官员更温和。可是,在这样一个年代,在这样的位置上,作为联邦司法部最高官员,他的名字要和这个在历史上
声名不佳的立法连在一起,已经是躲不掉的宿命:虽然他当时是在威尔逊总统的直接命令下才开始行动,后人最终还
是把此后的执法行动称为``帕尔默袭击''(Palmer Raids)。执法行动普遍针对左翼团体,包括无政府主义、共产主
义和社会主义团体的极端组织。帕尔默自己也终于相信,共产主义思潮正在``啃出一条路直通美国劳动者的家''。

这个称为``帕尔默袭击''的行动此后饱受争议。它主要涉及对激进组织成员的大规模逮捕,以及对一些激进分子的
窃听。后世争议在于,一是这些窃听没有取得传统上美国法律规定的法庭许可,虽然在``煽动暴乱法''之下,这样
的窃听是``合法''的。所以,争议主要还是来自立法本身;争议之二是关于``移民递解''。这又是怎么回事?

\pagebreak
\section{一个妥协方案}

道理很简单,逮捕并不等于定罪,逮捕的只是犯罪嫌疑人而不是罪犯。司法部是属于总统一路的行政分支,和法院
没有关系。司法部长手下有警察和作为公诉人的律师,权限只到逮捕、起诉为止。而法院是三权分立政府中的司法
分支,严格独立于其他分支。法庭定罪仍然要求极为严谨的证据。所以,帕尔默只能尽量收集证据提起起诉,在证
据不足的情况下,抓了还是必须放掉。2005年,Charles H.~McCormick就根据对大量当时联邦调查局文件和其他资
料的研究,出了一本书,叫做《没有希望的案子:对惧红时期恐怖炸弹手的追猎》(Hopeless Cases: The Hunt for
the Red Scare Terrorist Bombers),道出了当时要给激进分子定罪的困难现实。例如``华尔街大爆炸'',抓了三
十个嫌疑人,却因为证据不足,一个都没能定罪。

问题是,证据不足不能定罪,并不等同于这名被告就没有犯罪、放出去对社会就不危险。

所以,急于遏制激进分子扩张消灭美国,却又在严格的司法程序面前屡屡碰壁、无可奈何的行政分支,就只能求助
于和移民递解有关的法律了。

现在的人回顾历史,提到1918年``煽动暴乱法''的更多一些,因为它涉及和宪法权利法案相关的言论自由,最为触
目,事后受到抨击和司法挑战也多。而注意``1918年移民法''(the Immigration Act)的就比较少,其实这个移民法
也是因同样原因被推出来,其中有这样的条款:``凡无政府主义者的外国人,凡相信或提倡以武力推翻美国政府或推
翻各种法律的外国人,凡不相信或反对各级政府的外国人,凡提倡或教育暗杀公共官员的外国人,凡提倡或教授非
法破坏财产的外国人'',另外,凡``参加或者接近''某个符合上述情况之组织的外国人,``都将被拒绝进入美国''。

政府在无法给激进分子定罪的时候,就从移民法找出路。因为整个事件本身很美国特色:美国是个移民国家,大家都
是移民,就连激进组织、极端分子,也大量是移民。比如说那些打定主意要炸掉美国的炸弹手们,都被称为``加利
阿尼分子''(Galleanists),因为他们都是追随一个叫加利阿尼(Luigi Galleani)的意大利人,就像俄国人追随列宁
一样,这些追随者也有很多是来自意大利的移民。

假如留在家乡,他们也会同样和意大利政府作对,他们只是理想主义者而已。可是从美国的角度来说,你要追求推
翻资产阶级政府的理想,你去和你们自己的政府作对吧,不要革命到我们这里来。所以从道理上来说,一个国家不
容许那些致力于消灭这个国家、专来扔炸弹杀人的外国人移民进来或已经进来的要把递解回他们自己国家,是很天
经地义的事情。美国司法定罪太困难,妥协方案就是:我定不了你的罪,我送你走总可以了吧。

\pagebreak
\section{加利阿尼其人}

之所以诉诸移民法,是因为这一部分运作,即移民和公民归化这一摊,是归在行政分支管。在当时也和美国劳动部
有关。这样一来,就避开了严格的司法程序,和司法分支的法院不沾边了。作为行政分支的总统和司法部来说,当
务之急只是要摆脱威胁国家的危险,而不是一定非要把这些人关进监狱。所以,只要能送走就是上上大吉。于是在
这几年时间里,共有五百五十六个外国人,依据1918年移民法被递解出境,离开美国回家。其中就有那份宣布``阶
级战争''传单的印刷参与者,还有加利阿尼本人。

加利阿尼的一生是国际职业革命家的一生。他本来是个读法律的意大利大学生,被无政府主义吸引,弃法律而从革
命。他在意大利被追捕就逃到法国,在法国被驱逐就去瑞士,瑞士发现他是个危险的煽动者,再次驱逐了他,他回
到意大利,最终因为和警察冲突,被关五年。五年后他从西西里出逃,到达埃及,因为怕被引渡回意大利,就通过
英国逃往美国,1901年,四十一岁的他下了船,成为美国无数一贫如洗新移民中的一个。

对加利阿尼来说,他只是换了个地方,他的志向没有任何改变。他是一个天生鼓动者,也以自己是一个革命鼓动家
为荣。他能够广泛宣扬一个号召行动的政治哲学,激起人们的斗志,他的一个追随者说:``你只要听了加利阿尼的演
讲,没出演讲厅的门就想去干掉第一个碰见的警察了。''此言不虚,1919年2月,就在邮包炸弹前几个月,四个麻省
年轻人听了他的演讲,第二天就去纺织厂放炸弹,因为是生手,结果四个人全被自己的炸弹炸死。他的追随者在美
国各地都有,他们安置的各种炸弹不胜枚举。不仅是炸弹,1915年,一些加利阿尼追随者还在芝加哥一个两百人聚
会的汤里下砒霜,幸亏下得太多,受害者立即吐翻天,吐光了毒药也就没有人死亡。下毒者成功逃亡,并没有受到
法律制裁。

在1919年6月2日大爆炸之后的三个星期,加利阿尼被遣返意大利。意大利政府也吓坏了,有很长一段时期只准他住
在意大利外岛,而不准他住在内陆。最后他在自己的母国活到七十岁寿终正寝,那是1931年了。

非常有意思的是,今天美国文科教授自由派居多,如作者方纳(Eric Foner)写了那本很有名的书《美国自由的故事》
(The Story of American Freedom),已被翻成中文出版。他是哥伦比亚大学的历史学讲座教授,是美国很有影响、
得过史学界几个大奖的历史学家。他的书中认为``数千名激进分子遭到逮捕,数百人被递解出境''的原因,只是
``镇压不同政见的运动'',他引用当时一家意大利裔美国人报纸的话说``自由这个词已经变成了一个不可思议的
谜''。但他并不介绍这些``不同政见''推动了怎样的恐怖袭击,不提诱发递解的一波波暴力攻击。这样,阅读历史
的人也就一直停留在天真烂漫的``思维清纯''时代,不知道不论是历史还是今天,要面对和处理的现实都比他们想
象中的要复杂得多。

\pagebreak
\section{普遍的冲突背景}

无政府主义者认为革命的恐怖袭击背后,是当时美国非常普遍的劳资冲突。其中一个例子是大家都熟悉的美国慈善
事业的鼻祖之一,卡内基基金会奠基人\myrule 安德鲁\textperiodcentered 卡内基(Andrew Carnegie)。我们今天
一般只记得他是著名慈善家。可是在卡内基的创业过程中,也有激烈的劳资冲突。1882年,卡内基的霍姆斯丹工厂
发生过轰动全美的悲剧。

卡内基出生于苏格兰,十三岁随家人移民美国,白天做童工,晚上读夜校。十四岁就在电报公司做信差。凭勤奋机
灵和运气,更移民而逢其时其地,步步晋升,参与投资,最后创办了供产销一体化的现代钢铁公司,可谓勤劳致富
模范。

当时美国钢铁工人联合会已经实力很强,积累雄厚运作资金,事件发生几年前,就成功地以罢工迫使卡内基在霍姆
斯丹的工厂签下附带五十八页附件的制约企业条约,使得企业成本压力大增,在竞争中处于不利地位。在工厂和工
会的条约期满、签署新约之际,卡内基和工厂主管福立克都希望利用这个机会在工会面前争回自己的利益。因此工
厂备足库存以备罢工,然后提出减薪方案,并宣布,若在一定期限内谈不成新契约,工厂停工。谈判破裂,工厂停
工和工人罢工交织。当时这样的冲突政府都不会介入,还没有发展出仲裁机构。劳资双方旗鼓相当,冲突逐步升
级,对抗长达一百四十三天,工会领导工人占领了霍姆斯丹镇,劳资双方互相封锁工厂。冲突达到高潮之前,卡内
基离开美国去了苏格兰老家,他想给自己留下最后调解双方的中立位置,而倒霉的福立克留在了第一线。

厂方封锁工厂,是不让工会占领工厂,工会在厂外再包围封锁,是不让替工进去上班。都小半年过去了,福立克决
定打破僵局,从纽约和芝加哥聘用三百名著名的平克顿私人保安侦探公司的雇员,武装护卫新工人进厂工作。这个
主意听听不错,可事实上不仅平克顿保安有武器,工会的工人们也有武器。在混乱之中,没人知道是谁先开枪,总
之最后造成十人(七名工人、三名保安人员)死亡,几百人受伤,是美国历史上最血腥的劳资冲突之一。平克顿保安
人员马上要求投降和撤退,在得到工会的安全承诺之后,保安人员投降了。投降之后,工会却没有保证他们的安
全,数名保安被工人殴打昏迷。工会占领这个镇之后,就给记者发证,无证不得入内。凡持批评工会立场的报纸都
不给发证,所以,最初新闻媒体的表现几乎是一边倒地站在工会一边。到这个时候,连这些完全同情工会的报纸也
开始转而批评工会。最终,州长派出两个旅的州民兵接管并戒严,保护工厂由非工会的新工人开工。

事件也震动全美,看到州长令国民兵干预,使得罢工失败,一名与事件毫不相干的外地无政府主义者,突然横插一
杠,冲进福立克办公室,对他开了两枪还捅了两刀。他没有料到的是,史家后来断定,正是这一突发事件反而导致
工会人心溃散,草草收场。福立克则侥幸没有死于非命。

\pagebreak
\section{卡内基事件绝非孤例}

卡内基事件只是当时美国大量劳资冲突和大大小小悲剧中的一个。这种早期冲突对突然诞生膨胀的工业社会而言,
几乎是与生俱来。因为它具有天然的对抗性,只有无数悲剧带来教训,才逐渐使得社会学会以冷静、成熟和智慧去
对待和处理。所以我一直认为,要发明``阶级斗争''理论真的不难,因为它来自非常直观和简单的逻辑,真正的智
慧要有能够看懂绕着弯的逻辑,首先是看明白劳资双方是拴在一条绳子上的蚂蚱,没有单赢的可能。

著名的工人领袖戴博斯领导的工人运动,同样典型地传达了当时美国社会的动荡,以及所谓``第一次惧怕红色''的
社会背景。其中最有名的就是1894年的``普尔曼罢工''(Pullman Strike),追下去,那还是1893年经济恐慌惹的祸。

戴博斯于十九世纪中叶出生在美国,他的父母都是法国移民,家道殷实,可以算资本家。可戴博斯自己十四岁就辍
学开始打工。十九世纪末美国第一批全国性产业工人工会开始建立起来,戴博斯参与其中,在1893年创建了``美国
铁路工会''(American Railway Union),简称``ARU''。ARU又成为国际组织``世界产业工人''(简称``IWW'')的创建
成员。戴博斯从此成为一个重要政治人物。他曾当选为印第安纳州的议员,五度成为美国不同左翼政党的总统候选
人。

今天的金融风暴使得一些习惯计划经济的人认为,政府干预是美国要改变自由经济本质的预兆,其实自由经济一路
走来,已经经历多次循环:经济扩张总会导致一些人投机,出现泡沫、危机、恐慌,然后调整、加强监管,又开始新
一轮的复苏和经济扩展。只是出现问题的形式在变化,全球化导致牵涉面扩大,而社会应付的能力也在增强。十九
世纪末,美国就出现过一次``铁路泡沫''。想想也对,什么叫作``看不见的手'',就是市场需求带来生产旺盛,旺
过头的时候,它是靠市场本身来调节的,大家收手不买了。从宏观来说,调节是合理的,可是在转折点上,岂不是
有很多生产者和商家要出问题,牵涉贷款,所以金融业总会被卷进去,尤其在缺乏监管的发展初期。1893年经济恐
慌的重要因素之一是``铁路泡沫''。投资者认为铁路业利润不错,就大修铁路,最后修过头,连带银行业的铁路贷
款出问题。而当时的所谓先发国家,还不知道如何应付经济恐慌,也不知道政府能不能干预、怎么做好。以资方自
然应对,牵涉到劳工只有两条路,不是解雇就是减薪,否则关闭企业结果可能更糟。事实上1893年经济恐慌期间有
一万五千家公司和五百家银行倒闭。

\pagebreak
\section{戴博斯反抗}

那是卡内基事件十二年后的1894年。普尔曼是一家铁路车厢公司(Pullman Palace Car Company),它在经济恐慌下
决定对三千雇员大幅减薪。减薪、解雇当然有幅度如何、是否合理的问题,哪怕是有合理原因,也必然引发利益受
损的劳工反弹。普尔曼公司劳资纠纷本来是局部问题,可是,正因为有了全国性工会组织ARU,ARU就必须判断自己
应该如何反应。这些大工会组织都有一呼百应的能力,事态会迅速扩大,所谓``星星之火,可以燎原''。所以,劳
工反弹同样涉及是否过度的问题。它可以是对一个局部弱势工人团体的有力支持,抵御不合理伤害;也可以是过度
反应,引发灾难性后果。例如,本来涉及三千工人减薪,假如罢工蔓延无度,可能引发更大经济恐慌甚至经济崩
溃,可能殃及更多企业生存、导致更大规模失业,那就不是三千人减薪,而是更多工人连饭碗都保不住的灭顶之灾
了。在各方冲击下,1893年美国失业率是11.7\%,这一事件发生的1894年失业率高达18.4\%。所以,在最近的金融
危机中,我看到法国工会的最初反应是号召全国大罢工,真正是服了他们。

戴博斯当时只是个工人领袖,还不是一个社会主义者。普尔曼公司罢工,戴博斯作为ARU主席,本来反对发起全国性
罢工声援,可他无法说服众人,也就随了大家。一旦卷入,因他的领导地位和号召力,这场最后引发悲剧的事件,
也就以他的名字留载史册,史称``戴博斯反叛''(Debs' Rebellion)。这个反叛,从铁路员工拒挂普尔曼公司和另一
公司生产的车厢,到最后卷入十二万五千铁路工人大罢工,涉及二十九条铁路,造成美国交通大动脉瘫痪,进一步
打击了恐慌中的经济,假如无法收场,经济面临崩溃。

当时美国作为新兴工业国家还很不成熟,矛盾非常容易激化。戴博斯领导和平抗议,却实际控制不了自己手下的群
体。他带领的和平抗议演化为焚烧建筑物和制造火车出轨事件;公司打算雇用黑人替代罢工者,又引发白人工人的
种族攻击;全国范围的各种罢工同情者更是无法控制,他们自行其是,到处堵塞铁路,以此作为支援手段,还扬言
要攻击替工。当时,罢工的相关法律还没有完善,而联邦政府必须依法才能介入执法。好不容易找到一条理由,就
是罢工拒载的货车影响了联邦邮件传递,因此申请得到了联邦法院对罢工的禁止令。美国邮局是联邦政府机构。假
如不是这样,当时的联邦政府还真很难要求法院依据州际贸易的相关法律禁止ARU的全国性铁路罢工。按当时的道理
说起来,劳资两方都和你政府没什么关系。发出禁止令的另一个原因是罢工伴随了失控的暴力行为。

\myphoto{image008.jpg}{1894年的 ``普尔曼罢工''}

\pagebreak
\section{悲剧带来教训}

戴博斯和其他工会领袖决定对法院的罢工禁止令不予理睬,而执法是联邦政府行政分支的责任。面对十二万五千工
人的对抗,当时的总统克里夫兰(Grover Cleveland)下令一万两千名陆军士兵前往芝加哥等地执法。执法目标是护
送替工们进入工会防线,保证重新开工。军队的到来却刺激了民众暴力反抗,最后引发的冲突导致十三名罢工工人
死亡,五十七人受伤,酿成大悲剧。整个事件造成的经济损失更是难以计数。戴博斯作为ARU主席,因违抗法庭禁止
令遭到起诉,替他辩护的律师戴洛(Clarence Darrow),就是后来在关于``进化论''的``猴子审判''中以咄咄逼人风
格出名的那位。尽管有名律师辩护,戴博斯最终还是以藐视法庭罪(违抗罢工禁止令)被判六个月监禁。

此案最后上诉到最高法院,争执的议题是非常技术性的,就是挑战联邦政府发出禁止令的权利。戴博斯的律师戴洛
辩解的落点是,禁止令要求复工的工人,可能面临解雇。1895年最高法院最后作出了意见一致的裁决,根据州际贸
易的相关法律,法庭判定联邦政府有权管制保障州际贸易以及邮政的运作,以及有责任``确保公众的普遍利
益''(ensure the general welfare of the public)。

1996年芝加哥大学出版社出版了一本四百页的专著《城市失序与信仰的形成》(Urban Disorder and the Shape of
Belief: The Great Chicago Fire, the Haymarket Bomb, and the Model Town of Pullman),作者史密斯(Carl
Smith)以工业城市芝加哥为中心,涵盖了当时和芝加哥有关的几个主要事件也评述了那个动荡的时代。

戴博斯裁决(In re Debs)以及美国在第一次红色浪潮冲击下发生的种种悲剧,深刻影响了此后美国对劳资冲突种种
的反省,也影响了美国各种相应法律和制度的建立,例如对劳资纠纷的独立仲裁机构。在今天的美国,和法国非常
不同的一点,就是涉及公共利益的领域,如交通,是不能随意罢工的,法院有权裁决罢工是否合理合法。劳资博弈
必须在合理范围内运行。例如2004年12月纽约地铁工人在工会领导下为薪水罢工,第一天就造成四亿美元的经济损
失。罢工本身并不能得到民众的广泛同情,一方面造成纽约交通瘫痪,同时他们的工资在工人兄弟中间都还算是高
的。美国最大的工会组织劳联产联决定不予支持。法院最终判决该次罢工必须中止,判工会罢工延续一天罚款一百
万美元。假如工会领袖不服从法庭裁决、拒绝下令停止罢工的话,也可能以藐视法庭罪坐牢。也就是说,在美国理
念中,政府权力必须受到制约,个人和民众权利也要受到具体规范和制约。就像你有拥枪的权利,但和任意开枪是
两回事。

然而,这些反省和制度建立,都经历了非常漫长的岁月。正因为有悲剧有流血,在当时,最直接的民众反应必然有
利于一部分的社会思潮向左翼推进,就像戴博斯的思想走向一样,在联邦监狱六个月坐牢期间,他用功阅读马克思
著作,终于把自己读成了美国历史上最有名的一个社会主义者。

\pagebreak
\section{过河的定位仪}

我有时候很纳闷,觉得美国是不是运气比较好。看看十九世纪末、哪怕二十世纪初,甚至是二十世纪三十年代的美
国劳工照片,童工的童工,贫困的贫困,劳工生存条件还非常糟糕。这就是所谓``先发国家'':问题的发生发展、带
来的冲击,无以预料。国父们在建国、制宪的时候,美国不但是农业国,还是个很落后的农业国。谁也没有料到,
工业革命会以突飞猛进的势头冲击社会,突然大城市化、突然大批农民涌入城市转入工业。这也是自由经济的特
点,它给人的创造力留下了很大发挥想象的空间,而创造力爆发的同时也可能引出负面效果,社会应对措手不及。
法律制约和社会调整是跟在后面补救,跌跌撞撞、磕磕碰碰走来,却给所谓的``后发国家''提供了前车之鉴。

现在常听到一些后发国家要为自己犯的错误辩护,就会说:``你们以前不是也……''其实是不一样的,因为第一次遇
到问题手足无措是很正常的,而后发国家可以有所防备,这条路别人已经走过,已经有了他人试过的种种药方。

这么看上去,这些在自由经济中航行、包括美国在内的先发社会,看上去好像也颇有点``摸石头过河''的意思。可
是它怎么就没翻到河里呢?它虽然也有红色浪潮袭来,却逃过了被淹没。从美国来说,它有非常明确的定位方向,这
是一个有稳定立国理念的国家,理念以宪法来表达,以法律制度来逐渐完善。

在突发的浪潮冲击下,匆忙应对的法律可能是过度却是应急有效的。在一个缺乏理念的国家,会迷恋强势控制,怎
么有效怎么做,在不容置疑的前提下,可能会失去转折机会,走向另一个极端。美国始终没有丢弃宪法比对,挑战
和质疑是时时发生的。宪法是它的明确目标,方向偏离太远,一定会停下来、纠偏,再继续前行。所以,美国在第
一次世界大战加上极端左翼冲击下制定的法律,首先是间谍法(这里指``1917年间谍法''及其扩张法案``1918年煽动
暴乱法'') ,几乎从立法第一天开始,就在一系列的司法挑战之下。

前面说过,这是威尔逊总统在一次大战腹背两面夹击的情况下,推动国会制定的法律。除了针对间谍、更针对了反
战宣传。这很好理解:一个国家确实不可能一面国内在大规模抵制兵役,而同时却要打赢世界大战。这个法律就把反
战鼓动拒服兵役,归在了该法律界定的``干扰军队行动和取胜''罪行中。当时的美国邮局有权不送反战小册子之
类,也是依据这条法律而来。

\pagebreak
\section{对间谍法的司法挑战}

有关间谍法的几个著名案例,又有戴博斯案。只是此案非彼案,虽然是同一主角,已经不是上次提到的``戴博斯反
叛''一案,而是戴博斯二十多年后因鼓动反战而引发的另一个案子了。

上次提到1894年``戴博斯反叛'',我想说说中国``体制内''概念和美国``制度内''概念的比较,听上去好像差不
多,其实有很大差别。在中国``体制内''概念里,说起``体制内'',都知道是指政府机构这一大圈,民间的就是
``体制外''了。美国不说什么``体制内''。民间不论是与官方互动还是对立,你只要以公开方式、光明正大地表
达,哪怕公开挑战某条法律,都算是``制度内''运作。

比如说``戴博斯反叛''事件,不论组织工会、领导罢工、公开抵制挑战法院禁止令,事后接受法庭审理、上诉、接
受判决,这些都没有越出制度之外。工会方的司法挑战,政府的处理方式,整个过程和细节当时和此后都一直是公
开的,容许质疑。媒体可以报道、学者可以分析、全民可以讨论,这一切都是公开的。可以公开谴责任何一方,分
析反省悲剧的发生究竟是哪一方哪个环节处理不当。深陷其中的人可能当时还是不认错,可是,通过这样的检讨反
省,社会在吸取教训。每个事件发生,都在清理大家的认识,也修正或者推出新的法规来。这样的对抗都被看作是
``制度内''的。而躲避法律制约、乃至要推翻制度的地下恐怖活动,才被认为是``制度外''运作。它除了引发社会
恐慌,并不能带来任何制度上的进步。所谓``惧怕红色'',实质是美国人惧怕毁灭立国根基的制度颠覆。

美国的``制度内''是一个非常宽泛的互动运作,它包含了看上去是非常对立的各方各面。

从1894年案的整个过程可以看到,对戴博斯违法的惩罚是量刑适度的,也并不影响他的政治生涯,也就不影响他代
表他的那部分民众作出表达,所以就有了下面的案子。

1918年6月16日,戴博斯在俄亥俄州公开演说,诉诸民众抵制美国政府决定参与第一次世界大战的决定,公开挑战间
谍法。

他宣扬的是当时非常经典的社会主义者思路。一是阶级划分,否定美国的代议制民主,认为议员们大多是富人,所
以对美国制度的理解应该是划分为:统治阶级、劳工被统治阶级,他们之间是阶级对抗。看戴博斯的照片,这个法国
殷实人家出来的左翼政治家,绅士派头十足。他的讲话也反对知识分子,认为他们也是富人,是被统治阶级控制的
帮凶,他们被美国民主、共和两大党控制,而两大党在他眼里既然都是资产阶级政党,就是一路货。他怀疑整个司
法制度,认为陪审团是被统治阶级收买了。

在宣扬社会主义理念的同时,他的演讲主题是号召反战。他对民众宣传:美国介入第一次世界大战,如同中世纪欧洲
贵族把自己的农奴送上前线打仗。他同时公开支持和声援已经被``间谍法''定罪的行为,其中主要是对斯康克案、
洛思\textperiodcentered 哈立特案的声援。他们因号召反战、抵制兵役被定罪。

\pagebreak
\section{``斯康克对美国''案}

``斯康克对美国''(Schenck v. United States)一案非常出名,可以说是``地标案件''。因为它上诉到最高法院,
由著名的霍姆斯大法官(Justice Oliver Wendell Holmes,Jr.)作出了对言论自由的重要解释。

霍姆斯大法官是美国司法制度的一个杰出代表。在一个劳工激烈冲突的年代,他以最大努力开启推动以立法来根本
保障劳工权益。使得美国与欧洲相比,能以事半功倍在劳工权益方面取得同样效果,却减少了许多暴力。这种对比
対后发国家其实很有现实意义:当时的欧洲或许没有选择,而今天的后发国家是可以有选择的。

斯康克是另一个叫``社会党''的左翼政党书记。在一战期间,左翼政党们基本都取反战立场,他领导自己的党印发
了一万五千份反征兵传单,辛辛苦苦寄给那些征兵对象,号召他们抵制。这个行动正对正地撞在了间谍法的枪口上。
待他的案子上诉到最高法院,已经是1919年。

左翼社会主义者挑战间谍法,死死盯住的是宪法所保障的言论自由。作为言论自由案件的一个标志性判决,今天很
多人对霍姆斯大法官提出的``清楚与现实危险''的测定原则,并不陌生。也都听说过霍姆斯大法官在整个案子中举
的例子:``在拥挤的剧院谎叫失火'',不在受保护的``自由言论''之列。这个测定原则,至今被认为是対言论自由的
保护,因为只要不会带来``清楚与现实危险''的言论,政府就必须容忍。

正因为它是一个言论自由的``正面''裁决,很多人也就没有去追问:那么以``言论自由''为诉求、引出这个``霍姆斯
测定''的斯康克案,以及后来援引这个案例的一批左翼反战领袖们,他们的案子是赢了还是输了呢?

他们输了。因为``测定下来'',他们的言论恰恰在当时具有``清楚与现实危险''。

``斯康克对美国''一案在最高法院作出了9:0对斯康克不利的一致裁决,由霍姆斯大法官写出判词。大法官们认
为:``许多言论在和平时期可以表达,可是,当国家处于战争状态,只要士兵们还在战斗中,这些言论就变得难以接
受。没有一个法庭会把这些言论放在宪法保护之下。''那么战时言论受到怎样的限制呢? ``对每个案子提出疑
问,质问这些在特定环境的用词,是否会带来清楚和现实的危险,这些言论是否会带来国会有权阻止它的真实危
害。''换句话说,这些在巨大战争危险之下受到限制的反战言论,若回到和平环境下,它们仍然受宪法保障。

\myphoto{image010.jpg}{判决斯康克败诉的美国最高法院大法官}

\pagebreak
\section{间谍法的另一挑战者是位传奇女士}

戴博斯演讲中声援的洛思\textperiodcentered 哈立特(Rose Harriet Pastor Stokes),也是美国历史上出名的左
翼女性活动家。当时左翼政党多如牛毛,洛思\textperiodcentered 哈立特最初是一个叫作``犹太裔美国人社会
党''(Jewish-American Socialist Party)的领导人,后来成了美国共产党的创始成员之一。她的出名和她的传奇婚
姻大有关系。

美国真是很适合共产党这样的国际性政党,人口组成本来就国际化。洛思\textperiodcentered 哈立特是个犹太
人,出生在波兰俄占区的犹太人村庄,三岁移居伦敦,十二岁移民美国。很多人认为犹太人是右翼保守的,这个印
象至少在美国是错的。知识分子最容易接受左翼思潮,而犹太人传统重读书,进入这样知识阶层的比例也高,这是
一个重要原因。我看到过一份对犹太人的调查,即使在今天,美国犹太人还是左翼偏多。

洛思\textperiodcentered 哈立特能够用意第绪语和英语双语写作,就给犹太人和美国人的报纸同时撰稿。她是信
仰``阶级两分法''的坚定社会主义者,却在一次采访百万富翁的任务中,爱上了她的采访对象史多可先生(James
Graham Phelps Stokes)。本来是很好理解的事情,更何况这位``阶级敌人资本家''也是一个社会主义者。不过还是
引起同志们不少非议。此后他们双双积极投身社会主义运动,在如火如荼的劳资冲突中,史多可先生积极支持工人
罢工,很是能够经受考验。他的身份使得洛思\textperiodcentered 哈立特也格外引人注目。他们在1905年结婚。
洛思\textperiodcentered 哈立特在名字后面冠上了夫姓,成了史多可夫人。她一直留在第一线,而史多可则逐渐
退到写作生涯之中。

第一次世界大战爆发,他们夫妇逐渐产生分歧。史多可先生是威尔逊总统的私人朋友,也支持总统对参战的判断。
1917年美国社会主义者抨击美国参战计划,史多可先生因而退党,而且身先士卒,立即参军上了前线。洛思
\textperiodcentered 哈立特一开始也赞同丈夫,认为德国是民主国家的威胁,她也离开了社会主义者的阵营,可
是,她后来又开始转变,认为威尔逊总统推进的民主,与社会主义者所追求的国际社会主义的民主,是两股道上跑
的车。她不仅返回原来阵营,而且走向这个阵营的最左翼,加入``左翼社会党''。这一萌芽后来逐渐成长,长成了
美国共产党的大树。

丈夫在前线浴血战斗,作为妻子的洛思\textperiodcentered 哈立特在家反战,她在全国旅行、演讲。后来,她给
编辑的一封信里,强烈指责美国参战,认定政府是和奸商结盟。这封信的公开,引发广泛的公众讨论,也招致联邦
检察官以违反间谍法对她提起诉讼,和戴博斯一样,她也在初审中被判十年监禁。但是她的律师马上成功进入上诉
程序。她也就没有马上去坐牢。而且一刻也没有停止她的社会活动。

只是,无法想象一个自觉自愿、义无反顾投入战火的士兵,回来后和同一场战争的反战妻子如何再共同生活下去。
政治理念分道扬镳,这对夫妇坚持了几年,最终还是劳燕分飞。

洛思\textperiodcentered 哈立特写了几本有关劳动妇女的书,他人研究她的书非常多,1992年,我家附近的佐治
亚大学出版社,还出了一本介绍她的书《``我属于工人阶级'':Rose Pastor Stokes未完成的自传》。

\pagebreak
\section{1919年``五一暴乱''}

从戴博斯1894年违抗法庭罢工禁令一案的过程,看得出对戴博斯的量刑是非常``节制''或者说适度的,判决是双方
博弈的一个合理步骤,而不是政治扼杀。并不影响他的政治生涯,不影响他代表的那部分民众作出表达。只要不扔
炸弹,左翼政治家还是有充分的运作空间。1895年出狱后的戴博斯,堂堂正正创建了社会民主党(Social
Democratic Party)。而且在他因反战再次入狱之前,戴博斯已经有四次被不同的左翼党派推选出来,成为总统候选
人。若不是他的声望节节上升,鼓动能力越来越强,也不会第二次上法庭了。

戴博斯反战演讲一案是跟在斯康克案之后,性质相同,上诉到最高法院,被认定只需要援引前判例即可。刑期本身
是初审法院判的,最高法院并不重新审理下级法院的判决,而只是审查法庭程序是否有``技术错误''、援引的法律
是否违宪。如没有这些问题,初审判决就开始生效。斯康克和戴博斯各自在最高法院败诉,就必须分别开始服刑。
但是,这两人的刑期相差很大。在不同的法院,斯康克被判刑半年,而戴博斯被判了整整十年。

刑期的长短是由不同法官根据情节轻重裁量,只要在法律规定之内,就是法官的权限。法官当然可以说,以情节轻
重论,四次成为总统候选人的戴博斯,号召反战和抵制征兵,效果当然和别人不能同日而语。而支持他的人认为,
定罪本身就已经是违宪的``冤狱'',刑期长必定更是来自法官的偏见和阶级报复。所以,此后立即引发波澜壮阔的
群众抗议一点不奇怪。另一个社会主义者的领袖查尔斯\textperiodcentered 卢森堡(Charles Ruthenberg),决定
发起各工会、各社会主义者、共产主义者、无政府主政者,共同在克里夫兰市举行左翼联合五一大游行。卢森堡发
起这个游行非常顺理成章,他自己和另外几个革命同志,也与戴博斯因同样罪名被定罪、坐了将近一年牢,刚刚出
狱还不到半年,正憋着一肚子火呢。

可他或许根本没想过,这样形形色色的左翼、包括激进的民众群体一起涌上街头是何其危险,游行演变为暴乱几乎
可以预料。至于具体怎样开始暴乱,我看到各种版本的说法,也不知到底信谁的好。总之,最后变成群殴和暴力事
件,卢森堡自己的党总部也被暴民洗劫。暴力斗殴造成两人死亡,几十人受伤。市政府动用警察都压不住,最后坦
克开路,军人参与驱散暴民,维持秩序。逮捕了包括卢森堡在内的一百一十六人。卢森堡是这次聚众的组织者,被
法庭指控``有谋杀倾向的攻击'',说白了就是他组织的游行演变成``斗殴群架'',``有往死里打的架势''。虽如此
起诉,法庭却并没能给卢森堡定罪。

要知道,这个五一节就是1919年5月1日,就是前面讲到的无政府主义者寄出三十个邮包炸弹的那个五一节。不仅邮
包炸弹之后又连连发生更大规模的炸弹袭击,这个著名的1919年``五一暴乱''也引发此后全国各地大小暴乱,令
``惧怕红色''升温。克里夫兰市报纸很快披露,说是查下来在斗殴现场被捕的这一百一十六人,居然只有八个是美
国公民,举国哗然。这也是后来美国行政分支试图借助移民法,对激进分子以移民递解出境来解决危机的原因之一。

\pagebreak
\section{没有公民权的``总统候选人'',被国会拒之门外的议员}

和洛思\textperiodcentered 哈立特一样,此后卢森堡也加入了作为美国共产党前身的``左翼社会党'',最后参与
创建美国共产党,并担任了多年总书记。但在这个过程中,为争夺共产国际的承认与资助,他经历了左翼政党们争
斗、联合、又分裂的万般痛苦。记得Warren Beatty集编导演于一身,获得三项奥斯卡奖的电影《赤党》(Reds),就
生动再现了这个过程。这也是后话了。

接下来的1920年是美国大选年,戴博斯已经当了四次总统候选人。可是这次他坐在亚特兰大的联邦监狱里,不但是
个刑事罪犯,判决还包括了剥夺公民权。他连选举权都没有,更不要说被选举权了。

这是美国的``制度内运作'',你可以连续挑战司法,胜了继续走,输了接受法律制裁。制裁结束,继续你的政治活
动。大家习惯这样的运作,习以为常。即使在制裁中,在支持你的民众眼里,你不但不是罪犯,而且是英雄。戴博
斯就是这样,有的书说他是四次成为总统候选人,有的说是五次。原来在1920年大选中,他的名字虽然没有印在选
票上,却有近百万支持者在自己的选票上写上了戴博斯的名字,他总共获得913,664张``自选票''(write-in
votes),占选票总数的3.4\%。作为一个没有任何竞选活动、连公民权都没有的刑事囚犯,可谓惊人成就。要知道
他第一次作为正式候选人竞选总统,只得到七万张选票。当然,从这个数字的变化也可以看到左翼运动的成长速度。

当时还有个``走议会道路''的左翼政治家维克多\textperiodcentered 博格(Victor Berger),1910年,他代表威斯
康星第五选区,第一个以社会主义者身份被选为国会议员。在一次大战中,他也因为以挑战间谍法被判刑,而且一
审被判了二十年。他也一路上诉到最高法院。历史学家们普遍认为,维克多\textperiodcentered 博格反战有他更
复杂的背景,除了社会主义者的政治立场,还掺杂着他的文化背景。按今天的说法,他出生在罗马尼亚,当时那里
是``奥匈帝国''。移民美国后,他家里讲德语,他在美国生活的族群环境是德裔环境。大战中美、德的敌对战事,
使得德裔移民普遍不能接受。

当时美国在民众思潮上对立的程度,可以从这些事情上看出来。维克多\textperiodcentered 博格和戴博斯一样,
他的刑事重罪身份,并没有阻挡威斯康星的选民继续把他选为联邦众议员。他当时还在上诉途中,并不像戴博斯那
样蹲在牢里,就如期到华盛顿国会大厦上任去了,着实给国会出了一道难题。国会当时专门成立一个委员会,以判
断是否能够接受他。最后的判断是``不能''。于是,国会留着空缺,而威斯康星的这个选区举行了一次特殊补选,
结果,民众第二次坚持选上了维克多\textperiodcentered 博格。国会再次拒绝接受。直到下一次选举,维克多
\textperiodcentered 博格自己竞选落败。

\myphoto{image012.jpg}{维克多\textperiodcentered 博格的演讲有列宁的风采}

\pagebreak
\section{你到底要干什么}

看到美国发生一系列对``间谍法''的司法挑战,以及对这些挑战的处理过程,我常常会想到有关法律在文本背后的
政治处理。

记得在最初写介绍美国的几本书的时候,我想出一个说法,叫``真诚的法律'',听上去很可笑也很荒唐。我的意思
是:不同的现代国家,会有近似、甚至同样的法律文本,究其来源,却可能有很大差别。一个立法文本,可能``真诚
地''来自要维护一个理念,或者说,来自要应对的现实社会危机。可是,它也可能来自外部世界的压力、来自和它
的表述完全不同的政治诉求。比如说,一个有关公民自由的立法,如果是一个``真诚立法'',它背后就会有坚实的
社会文化支撑,它相应的细则、判例会不断正面推进公民自由的目标。假如它是迫于压力被动推出、以应付民众或
者外部压力,那么它的细则和司法执法过程,会反向出现更多限制,法律保障的自由会更多停留在纸面上,这样的
立法,就是``不真诚''的。

以应对危机的法律来说,也是一样。假如是一个``真诚的法律'',会局限在``应对危机''本身,行政分支不会在法
律掩护下``外沿扩大''、借以打击政治对手。当时美国行政分支如何对待间谍法的司法挑战,就有这个问题。它也
是一个政治检验。

虽然只要在间谍法规定之下,判决的刑期都是合法的。但是,对刑期是否适度,人们还是会有个普遍感觉。间谍法
的案件更为特殊,因为依据的是对国家战争危机的考量,可牢还没坐完,战争结束了。虽然二者从逻辑上并不矛盾
\myrule 你坐的牢是针对你的战时行为判的。正因为司法独立,对刑期有异议的时候,一般是怀疑法官是否有偏
见,不可能怀疑是行政分支``使的坏''。在对待间谍法的案件中,能够看到美国各分支在尽力纠正危机处理可能发
生的偏差。

维克多\textperiodcentered 博格被判了二十年,却并没有坐牢。这个被普遍感觉是刑期过分的一审判决,在上诉
到最高法院的时候,被最高法院通过对一审法庭的``技术''审查,判定初审法官审理过程持有偏见而推翻。再看另
外两个著名的``十年刑期案'',洛思\textperiodcentered 哈立特案和戴博斯案。联邦政府在战后的1921年撤销了
对洛思\textperiodcentered 哈立特的起诉,当时她已经是共产党员,所以虽然初审判了十年,因在上诉过程中,
她一天牢也没坐。联邦政府撤诉的第二年,1922年,她代表美国共产党前往莫斯科,参加了共产国际的第四次代表
大会,回来风风火火地又公开创建了另一个左翼政党,美国工人党。

戴博斯一案在当时就穷尽了司法程序,看上去没有转机了。行政分支无法干预司法,可是总统有最后一个权力,就
是赦免权。在洛思\textperiodcentered 哈立特案被联邦政府撤销的同一年,哈丁总统利用赦免权为戴博斯减刑,
立即释放,还在白宫热情款待了这位美国政府的反对领袖。一战已经结束,国家危机过去,联邦政府显然不希望应
对战时危机的法律,在和平时期对政治反对派领袖作过度惩罚。

\pagebreak
\section{司法判决推进历史}

我看到今天一些描述这段历史的书,会提到在第一次``惧怕红色''时期,国会甚至拒绝接受社会主义立场的当选议
员,说的就是维克多\textperiodcentered 博格。实际上,当时困扰国会的并不是持红色政治观点的议员是否可以
合法上任,而是国会来了一个被法庭初审定罪的刑事重罪犯,我们怎么办。事实上,此前,以及维克多
\textperiodcentered 博格被最高法院推翻初审判决之后,在他没有罪犯身份的时候,著名社会主义者维克多
\textperiodcentered 博格只要当选,都正常进入国会工作,他因此后来还参与推动了美国和苏联的建交。

间谍法的扩展法案,即1918年的``煽动暴乱法''(the Sedition Act),存在的时间非常短,1920年12月,国会认为
它的条款太可能被滥用,就废除了这个法案。而间谍法在当时没有被废止,最高法院却在日后案例的一次次裁决
中,对它作出了越来越严格的限制。

最高法院依据间谍法对斯康克案的裁决,是基于一次世界大战的特定环境和大规模反战特定鼓动的危险碰撞。就社
会大环境来说,美国和美国不一样,战争和战争也不一样。美国并没有因此扼杀一切战时的反战言论。也就是说,
在第一次世界大战的美国社会大环境这个特定的时间、条件、地点下,最高法院判定,反战宣传会给美国的战况带
来``清楚与现实危险'',当时氛围下掀起拒服兵役大潮完全可能,假如真的出现这个局面,仗是无论如何打不下去
的。

可是,不要说和平时期,哪怕在另一次战争中,哪怕是同样言论,还是要另作判断。比如说,时过境迁,现在的美
国并没有当时的兵役制,现在当不当兵全在自己,对年轻人来说,就像考不考大学的决定一样,政府对你没有任何
强制。今天的美国,在阿富汗战争、伊拉克战争严重伤亡的情况下,反战派呼声天天不断,美国仍然不需要恢复兵
役制,也从来没有缺过兵源,更不要说惩罚反战言论了。

\pagebreak
\section{遣送俄国革命者回故乡}

煽动了一大批意大利裔炸弹手的加利阿尼,在1919年6月大爆炸的三个星期后就被送出去了。被遣送的基本上都是无
政府主义极端分子,所以,1918年移民法干脆被叫成了``无政府主义者排斥法''。除了意大利移民,还有其他族
裔,例如革命故乡来的俄国移民,其中最著名的有亚历山大\textperiodcentered 伯克曼(Alexander Berkman)和他
的女友爱玛\textperiodcentered 高德曼(Emma Goldman)。

前面讲到,卡内基工厂劳资冲突悲剧的结束,是一名外地``从天而降''的无政府主义者突然刺杀工厂主管福立克,
造成重伤。这名杀手就是亚历山大\textperiodcentered 伯克曼。他因此被判了十四年徒刑。1921年,他出版了
《一个无政府主义者囚徒文集》(Prison Memoirs of an Anarchist)。

出狱之后,伯克曼办了杂志,还参与在纽约办了叫做``范利中心''(Ferrer Center)的学校,范利是个西班牙革命者。
作家杰克\textperiodcentered 伦敦还去这个学校作过演讲。不久,洛克菲勒家族拥有的矿产公司下属科罗拉多州
矿因安全措施差等等原因,又引发全美矿工联合工会(UMWA)号召的大罢工。结果比卡内基冲突事件更为血腥,演成
真正的``阶级战争''。

煤矿的罢工对抗本身延续一年多,罢工工人与被工会认为是``工贼''的替工工人发生冲突,造成死亡。也发生替工
被谋杀找不出凶手的。资方和卡内基的厂家一样雇用私人保安公司围堵工人营地,这家保安公司比平克顿公司凶多
了。最后导致枪战。比卡内基事件更糟糕的是,在枪战中保安公司放火烧了工人营帐,不仅有枪战死伤,还有十几
名作为工人家属的妇女和小孩因此窒息死亡。事后工人又拉出一支游击队,袭击州国民兵以及公司保安队,最终无
法收场,以联邦军队逮捕双方人员平定。

可想而知,对伯克曼等人这是多大的刺激。1914年夏天,伯克曼在学校的三个合伙人制造了炸弹,准备进入小洛克
菲勒在纽约的公寓引爆,可是混不进去。他们回到一个同伴在莱克辛顿大道租的住所,在制作新炸弹时却意外引
爆,三个炸弹手全部死亡,还炸死另外一名无辜女人,炸塌了几层楼,多人受伤。伯克曼被参与者指认是策划合谋
者,他自己否认知情。最后缺乏证据不了了之。1916年伯克曼去了旧金山,第二年旧金山总统节大爆炸,两名炸弹
手炸死十人炸伤四十人。伯克曼安排国际宣传,俄国无政府主义者在彼得格勒的美国大使馆前抗议,最终加州州长
赦免了炸弹手的死刑。

1917年伯克曼和他同样出名的女友爱玛\textperiodcentered 高德曼一起创办``拒服兵役联盟'',也就是公然违反
了1917年间谍法,他们被判刑两年。在服刑释放之后,他们和其他俄裔无政府主义者,总共两百四十九人,被一船
递解回了革命后的苏联。这次递解专船后来被称为``苏联方舟''。恰在临走前一天,在伯克曼告别宴会上,传来了
二十五年前他刺杀未遂的卡内基工厂主管福立克去世的消息,伯克曼说,``福立克是被上帝递解了。''

\pagebreak
\section{移民递解也受到挑战}

司法部长帕尔默绕开司法程序,以移民递解来解除恐怖袭击危机的做法,也同样面临挑战。

恐怖袭击的特点,都是地下策划的阴谋,扔炸弹都是扔了就走,要抓住本来不容易,在严谨的司法程序、对证据的
严格要求下,抓住了也难定罪。司法部长能够做的,就是利用极端分子多为移民的特点,利用行政分支权限之内的
移民递解程序,把他们送出国境了事。可是没有料到,事情刚开头就被顶住,原因是劳动部来了个七十一岁的新部
长路易斯\textperiodcentered 波斯特(Louis F.~Post)。

美国曾经是个落后的农业国,建国差不多一百年后的1888年,才在内政部设立了劳动局。1913年威尔逊总统才建立
劳动部。1920年初,司法部决定递解极端分子不久,波斯特出任劳动部长。他一来就坚决反对按照司法部属下的
BOI(联邦调查局前身)提供的名单做移民递解。比较下来,他们的根本分歧在于:是否认同1918年移民法。

按照这个法案,``凡相信或提倡以武力推翻美国政府或推翻各种法律、凡提倡或教育暗杀公共官员的外国人''就应
被``拒绝进入美国'',已经进来当然可以递解,无政府主义者是移民法点了名``不得进入''的。而律师出身的波斯
特出于自己的职业习惯,认为这个标准太过分,他认为``相信或提倡''是不够的,必须有证据证明是``有犯罪行
为''才能够递解。他总共接到司法部给劳动部的两千五百个递解案,根据自己的标准审查,波斯特扣下了其中的绝
大部分没有执行。所以,在整个``第一次惧红时期'',被递解的是五百五十六人。他坚持认为,哪怕是外国人,在
这里也必须得到公正的听证机会。当时负责BOI的胡佛只有二十四岁,他急于缓解危机,他的观点是外国人并不受美
国宪法保护。

劳动部只是一个递解的执行部门。执法官竟然不执法,于是,来自堪萨斯的一个众议员,以不当使用权力为由,也
就是渎职罪,要求弹劾劳动部长。国会的规则委员会因此专门举行听证会,调查此事。在听证会上,波斯特成功地
为自己作出辩解。以一个个递解案例作辩解,委员会最终决定不对他启动弹劾程序,也就是肯定了劳动部有权对递
解作出审核。通过这个弹劾案,作为立法机构的国会实际上承认了``无政府主义者排除法''过于苛严,也实际认可
了劳动部在执法中的调整。虽然这个法案本身直到1952年才正式废除。

波斯特以七十九岁在1928年去世,在那个时代已经算是长寿老人了。他身后给自己的妻子留下一本自传《再长久地
活一回》(Living a Long Life Over Again)。

\pagebreak
\section{危机面前的争议}

就在劳动部和司法部的做法发生冲突的时候,今天在美国著名的民间组织美国公民自由联盟(ACLU)也开始从民间发
起挑战。它原先就是1917年在左翼反战高潮时成立,1920年刚改了名字。开始转向一直持续到今天的行动目标,就
是维护公民宪法权利。1920年5月28日,ACLU出版了一份调查报告《美国司法部非法行为报告》,质疑``帕尔默袭
击''的合法性。报告脱离了当时左翼组织惯用的激昂夸张的宣传模式,做得非常谨慎,列举了司法部在这一阶段执
法过度的案例。报告由一些著名律师和法律教授签署。这种竭力专业中性的立场和做法,使得这个组织从极端左翼
中脱胎出来,变得``美国化'',也就是反对派合法抗争、在制度内抗争。有理好好说。

为此,部长帕尔默受到国会规则委员会的听证传讯。在听证会上,帕尔默为司法部强烈辩护,他对议员们说:``在这
件事情上,我不为司法部作任何道歉。我为此自豪。''帕尔默坚持认为自己在做一件``对的事情''。当然,他也有
他的理由。

我读过戴博斯演说,当时的美国革命信仰者对夺取美国政权、改变国家颜色充满信心。戴博斯演说中就有这样的判
断:

``是的,现在正是我们席卷而进入这个国家和世界权力的大好时机。我们打算毁灭所有奴役的、可耻的资本主义制
度以及所谓自由、有人性的制度。世界日日在我们眼前变化。资本主义日薄西山,而社会主义如旭日东升……社会
主义是在生长中的理想,一个正在扩展的哲学。它正在覆盖整个地球,要抵御它,就如同要拖住明天的日出一样徒
劳。它来了,来了,来自四面八方。你看不到吗?假如没看见,我建议你去看眼科医生,一定是你的眼睛出问题了。
这是人类历史上最强有力的运动。能为它效劳是何等荣耀!''

除却那些炸弹和劳资冲突可能引发的``阶级战争'',还有世界革命的支援,就在1919年到1920年间,共产国际分四
次非法送入美国的左翼组织二百七十二万八千卢布,在1922年以前,卢布在国际货币市场上下波动,所以很难精确
算出美元价值。可按照历史记录,整个二十年代,共产国际非法运入美国、支援革命的金额,高达当时的几百万美
元。对帕尔默来说,假如移民递解要根据证据确凿的犯罪行为,加利阿尼本人就不符合条件,他没有直接参与爆
炸,可是他鼓动的影响难以估量。到今天,历史学家们还是普遍认定,在加利阿尼被递解之后,那些受了他鼓动、
自认是加利阿尼分子的炸弹手,还继续了十二年的炸弹攻击。帕尔默的逻辑很简单:对一个打算防备暴力革命的政府
来说,是端着枪冲向冬宫的人危险,还是列宁危险呢?我也没把美国列宁怎么样,我就是把外国列宁送回家去啊。

\pagebreak
\section{红色冲击很久,``惧怕''时间很短}

非常有意思的是,``红色冲击''的时间非常长,而美国所谓``第一次惧红''时期却非常短。历史学家一般公认是从
1919年6月大爆炸之后的``帕尔默袭击''开始,直至1920年6月结束,大约是一年不到点。也有的算法是把威尔逊总
统对一次世界大战反战反兵役者的起诉定罪也算进去,就有将近三年。我认为前一种算法比较靠谱,反战冲突只是
``惧红时期''的一个背景。

当时的主要媒体也大多支持帕尔默的看法,例如当时的《华盛顿邮报》就警告说:``在我们的自由面临被摧毁的时
候,我们已经没有时间拘泥细节。''反对的声音也从来没有停止过:假如我们现在不坚持美国民主法治的严格标准,
那么,我们和我们要反对的专制制度又有什么区别呢?

就在1920年的五一节之前,帕尔默根据前一年五一节邮包炸弹的经验,警告说,可能在今年的这一天发生革命大暴
动,并且要求纽约警察局的全体警察在那天二十四小时当班。结果,那天什么事情也没有发生。后来才知道,当天
本来左翼组织是有大游行的计划,可是领导者知道警力严阵以待,担心手下人失控造成大的冲突流血,就下令取消
了行动。即便如此,可能失控的大游行,也不等同于推翻政府的革命。

因此,帕尔默的``五一预言''被当时的美国左翼、后来的自由派,一直嘲笑到今天,称帕尔默是喊``狼来了''喊得
太多的那个小孩。可是只要经历过``9.11''之后的美国,就知道帕尔默警告还是很正常的事情。在``9.11''之后,
美国在一些重大节日几乎都会提高警报级别,因为对手在暗处,你在明处。除了极少数情况下你掌握确切情报,其
他时候你只能困难地作出大致判断。而官方和民间的责任是不一样的。只要存在危险的可能,你不敢不发警告。对
于民众来说,真是听多了也就只当它是``狼来了''的故事了。

最后,终于有二十个移民的案子被送往联邦第一巡回法院上诉法庭,1920年6月,该法庭的安德森法官(George
Weston Anderson)不但宣布全部撤销这些案子,还在法庭上抨击了司法部大规模逮捕和移民递解的做法。他在法庭
上宣布,他不认为共产党就马上会在美国推翻政府。历史学家认为,这是美国``第一次惧怕红色''的``官方终结''。

这意思就是说,不论再发生什么事情,美国决定还是由原来严格的司法程序来应对。随着一次世界大战的结束,应
对反兵役的间谍法已经不起作用,它的延伸部分,即对宣传鼓动推翻政府言论治罪的``煽动暴乱法'',在1920年底
被国会撤销,大规模递解移民的行动停止下来。美国回到原来的轨道,扔炸弹的要按照严格的证据要求和司法程序
在法庭审理。

当然,这样的坚持是有代价的,红色浪潮甚至恐怖攻击的问题,继续在美国存在了很多年,前面提到的造成三十八
人死亡的华尔街大爆炸,就在安德森法官宣判的三个月后。可是,以放宽制度要求为标志的``惧怕''时期,美国人
还是坚持给它画上了句号。

\pagebreak
\section{美国左翼在国际接轨中的困扰}

聊了不少美国``第一次惧红''时期的故事,因为它的红色背景,是所谓``第二次惧红''时期,也就是麦卡锡时代的
渊源。

从``第一次惧红''结束的1920年,到麦卡锡时代,期间有大约三十年时间。在这段时间里,世界发生了翻天覆地的
变化。而美国共产党在这段时间经历了大起大落、迂回曲折的过程。这和共产党的国际化有关,也和美国左翼理想
主义者所生活的大环境和它所追随的苏联,实际存在巨大差别有关。

1919年1月,列宁邀请美国社会党左翼参加共产国际,美国只有一大堆形形色色的左翼,但还没有一个组织就叫``共
产党''。这就带来一堆麻烦。在大家都意识到应该``和国际接轨''成为正宗的时候,就引起激烈竞争。更何
况,``接轨''、``正宗''意味着国际经费花落谁家。这是理念、荣誉和利益的鸡尾酒。著名的美国共产党创始人之
一约翰\textperiodcentered 李德,为了争这个正宗,偷渡苏联,九死一生,最后甚至归国无望,客死他乡,酿成
悲剧。

美国共产党是苏联政治的一个美国变异版本,它正式成立于1920年1月。当时共产国际命令美国最大两个左翼大党合
并,终于结束了旷日持久的争执,有了正儿八经叫做``美国共产党''的政党,党员一万两千名。当时正值美国政府
和左翼冲突最严重时期,再加上共产党颠覆本国制度和政府的公开目标,使得它一开始就是``秘密地下党''。比较
特别的是,当时的共产党员多半是下船不久的新移民,换句话说,很多美国共产党员并不是美国人。例如1921年又
一个左翼政党并入共产党时,新成员中只有5\%是讲英语的本土美国人。而共产党中的犹太人比例很高。渐渐地,
苏共的分裂也影响了美国共产党。

1925年,共产国际召开第五次大会,认定革命高潮已经过去,最近几年的任务是养精蓄锐。当时的美国共产党总书
记罗弗斯通(Jay Lovestone)已经相当``修正主义'',支持``美国例外论'',认为美国可以``和平演变为社会主
义'',而用不着暴力革命。斯大林一怒之下令他辞职,并且警告他不要昏了头:你不要以为有大多数支持,可那只不
过是因为共产国际当了你的后台。果然,罗弗斯通一回到美国就被清除出去,令他十分郁闷,他后来追随布哈林的
``共产国际反对派'',拉出另一个``美国共产党反对派'',但追随反对派的人却不多。斯大林在1928年决定和西方
国家的共产党脱钩,也使得美国共产党人数迅速下降,一度只剩下六千人。

在追随苏联中,最大的困扰还是两个国家本质上的差别。例如约翰\textperiodcentered 李德在苏联的最后几年,
他虽然不愿意否定自己的理想,但越是深入,李德却发现苏联远非自己想象中的工人阶级的天堂。

一个典型例子是被遣送回苏俄的无政府主义者亚历山大\textperiodcentered 伯克曼。虽然无政府主义和共产党的
目标相去甚远,可是在资本主义面前,他们感觉共同之处远远大于分歧。当十月革命消息传来,伯克曼兴奋不已,
称这是他``最幸福时刻''。1919年他被美国的递解船送抵苏联的那一天,他宣称那是他``一生中最庄严的一天''。

在苏联第二年的1920年,伯克曼和一起被递解的女友爱玛\textperiodcentered 高德曼一起,几乎走遍全苏联,为
筹建革命博物馆收集材料。可是,这趟旅行使他们幻想破灭,他们看到镇压、腐败,看到工人在极为苛刻的条件下
工作,任何质疑都会被当作``反革命''。他们面见列宁陈词,列宁却告诉他们,政府的``专政特权是公正的'',还
说:``假如革命摆脱了危险,也许会容许言论自由。''

苏联很快爆发反抗,1921年3月,亚历山大\textperiodcentered 伯克曼和高德曼支持了苏联工人的罢工抗议,接下
来是著名的克郎斯塔德军港反叛,反抗者的十五点要求包括了民主选举、言论自由、农夫使用自己土地牲畜的权利
等等。高德曼向法国媒体透露了事件,国外的报道使得苏联将事件定性为帝国主义阴谋。托洛斯基动用军队镇压,
水兵抵抗最终变成战役,六百水兵被杀,也有上千赶来镇压的军人在战役中死亡,两千人被捕。伯克曼在日记中写
道:``过去的这些日子是灰色的,希望的余烬一个个熄灭。恐怖和专制碾碎了十月后的新生命……独裁将民众踏在脚
下……我决定离开俄国。''那年12月,伯克曼和高德曼离开苏联。

\pagebreak
\section{反向接轨的必要}

伯克曼和高德曼在欧洲出版了一系列小册子:《俄国悲剧》(The Russian Tragedy),《俄国革命和共产党》(The
Russian Revolution and the Communist Party),《克郎斯塔德反叛》(The Kronstadt Rebellion)。他们二人还
各自出版了他们在苏联的经历,伯克曼的那本是1925年1月出版的,书名是《布尔什维克之谜》(The Bolshevik
Myth)。

凡是深入苏联的西方左翼,这样的困扰很普遍。约翰\textperiodcentered 李德曾经写出赞扬苏俄革命的《震撼世
界的十天》,在革命之后,他和一批深入苏联的左翼美国人,最难以接受的,就是苏联经验和他们的母国经验最直
接冲突的、和司法相关的那部分:革命后的苏联可以``未经审判就判刑'',甚至是判死刑。

不过,苏联很快明白,要继续领导世界革命,也必须和那个资产阶级的国际社会表面接轨,就是哪怕不认为审判是
必要的,也必须有一个``不真诚审判'',有比没有好,否则如何让那些习惯了法治社会的西方左翼们,比如美国共
产党,死心塌地跟你走呢? ``法治''是他们最大的心理障碍。结果,苏联开始出现大量这样的特殊审判,以至于英语
出现一个新词:``show trial'',这个词在字典上的解释是:集权国家为了达到某种目的而举行的摆样子的公开审判。

登峰造极的show trial发生在1938年1月至3月著名的``二十一人审判'',苏联最高法院军事法庭,判决包括布哈林
在内的二十一名苏联高级官员为``右派托洛茨基反苏集团''。罪行不仅有背叛国家,还有暗杀列宁、基洛夫、高尔
基等等。最后宣判,十八人被判处死刑,三人被长期剥夺自由。直到五十年后的1988年2月,苏联最高法院全体会议
才一致通过取消判决,为他们恢复名誉。

这场审判对于苏联来说,就是和国际接轨的行动。按照它的本意,这本来就是没打算让被告讲什么道理的政治清
洗,根本没打算以审理过程寻求公正。不过,如此接轨,我觉得还是证明了``人之初,性本善'',就是人类天性里
其实有一个善恶标准。绝大多数人在做一件错事的时候,有把它辩白为``对''的冲动,辩白就在证明内心有善恶标
准,否则不用辩白。对于苏俄来说,如1936年卡廷森林事件,集体屠杀了四千波兰军官,假如真的不自知其为恶,
就没有必要把它栽到德国人头上。审判也一样,因为对公正的要求仍然是人类最起码的良知,是一个国家的立国之
本。你无法长期用某个借口来覆盖和否定:国家必须有寻求公正的正当程序。

\myphoto{image014.jpg}{在列宁陵墓上观礼,1927年。前排左起:里科夫、布哈林、加里宁、斯大林}

\pagebreak
\section{二十一人审判对美国左翼的冲击}

对苏俄来说,审判事实上并不是``真诚''寻求公正。苏俄理念确实和西方国家泾渭分明。这只是表面接轨。五十年
后,二十一人案推翻,苏联人亚历山大\textperiodcentered 雅科夫列夫看了1938年出版的法院判决和其他文件,
他写道,被告``毫无例外地一致而谐调,陈述相同。被告都指责自己犯有最严重罪行,都自我揭露、谴责自己、忏
悔。谁也不与起诉人和证人争辩,不推翻任何对自己不利的证据,反而对证据进行补充。彻头彻尾地自我认罪''。
``在整个诉讼程序中,检察长占统治地位,主持和左右整个程序。起诉人揭发谴责,其他人附和。听不到法官声音。
辩护人只请求宽恕。没有法庭辩论。没有双方辩论,没有论证,也没有反论证,没有对证据的调查。没有专家参加
诉讼程序,没有任何鉴定……被告人详细描述自己是怎样的坏蛋。'' 他于是得出结论:``这令我产生如此印象:这是
一幕精心策划、预先排练好的闹剧。''

对于当时美国共产党中信仰坚定者,show trial还是起作用的。审判刚刚结束,1938年4月在纽约出版的《青年共产
党人观察》就立即跟进发出一篇文章《莫斯科审判:它的意义及其重要性》(The Moscow Trial:Its Meaning and
Importance)。这篇文章就为二十一人审判辩护:``被告之所以都认罪,那是因为他们有罪;因为他们已经无计可
施;因为事先调查已经摆出对他们不利的事实无可争辩;程序上的问题是第二位的。苏联的司法程序并没有什么特
别。''更有意思的是,作者把二十一人审判和美国宪法作比对。在美国宪法中,为了防止滥用叛国罪,有这样的条
款:``只有对合众国发动战争、或投向它的敌人,于敌人以协助及方便者,方构成叛国罪。无论何人,如非经过由两
个证人证明他的公然的叛国行为,或经由本人在公开法庭认罪者,均不得被判叛国罪。''所以,作者就辩护说,美
国宪法不是也认可``两个证人''、``本人在公开法庭认罪''为定罪依据的吗?

这是法治社会常识:维护公正并不是``两个证人,当庭认罪''那么简单,它必须有一整套社会制度的配合,司法独立、
新闻自由、民间监督机制,等等等等。而在这些全都缺席的情况下,不要说两个伪证,找两千个伪证也是轻而易
举,而``认罪''就更简单了。

二十一人审判对美国左翼政党带来强烈冲击,例如美国``社会主义工人党''对审判的看法就完全是负面的。在一篇
《二十一人审判》(The Trial of the 21)的文章中,他们和斯大林主义明确划清界限,文章认为,这个审判只是一
系列同类审判中最露骨的一个,审判是一连串谎言,而事实真相就在这些谎言背后。文章还说,在这一类审判中很
少考虑人的生命和尊严,它只为特殊政治目的服务。苏俄的show trial在这些左翼政党中起了相反作用,而主流社
会更认定苏联宣传中的光明大道我们万万走不得。

所以,在法治相对健全的西方国家的共产党中间,最后纷纷出现要和资本主义``和平共处''、要走议会道路的``修
正主义'',是很自然的事情,要坚持以极端方式推动革命反而不容易。回看美国左翼对``二十一人审判''的讨论,
不论是为其辩护、还是持反对态度的,他们依据的标准最终都是他们所习惯的法治。

\pagebreak
\section{国内局势在变化之中}

在左翼大派合并成立美国共产党之后,``党内有党''的派别斗争一天也没有停止过,很大程度上影响了革命工作。
如1926年共产国际女工联盟的纽约组织,领导制造业大罢工,主要就是因内部斗争而失败。这些派别常常和共产国
际或者苏联党内派别一一对号。比如某派只是因为偶然在加拿大接触了苏联托洛茨基派的宣传,就被说动从而成了
美国托派,最后也随着``布哈林\myrule 托洛茨基阴谋集团''一起陨落。读完线索繁多的派系,想想不要在这里拿
繁复的美国谱系折磨读者了,还是让专家们去伤脑筋吧。可是有一点不得不说,在第一次惧红时期之后不久,美国
共产党大致上可以说开始分为``一道两股''。说``一道'',指大方向是一致的,他们都要以苏联式社会主义取美国
制度而代之,认共产主义为最终目标;而``两股'',指的是具体分为两拨人。一拨开始脱离``地下''状态,给自己
的定位是一个``合法反对党'',扎根工人运动而不打算寻求立即推翻美国政权的革命,关键是他们浮出地面,不再
是地下状态。而另一股始终保持在地下状态,他们始终是苏联共产党事实上的美国支部。可是,不少人却具有地上
地下两种身份。

在1938年的``二十一人审判''前,美国和世界在三十年代都已经经历一些大变故。给美国共产党也带来如何应对和
判断的问题。

首先是美国经济大萧条。和当时的冲击相比,今天所谓金融风暴根本就是毛毛雨。美国很出名的大烟山国家公园,
当时组织年轻人参加半兵营,劳动艰苦纪律森严,基本上只有一口饱饭吃,报名者趋之若鹜。小罗斯福选上总统,
在1933年3月宣誓就职前去了一次迈阿密,一个人大叫着``大家都快饿死了!''对着未来总统就是几枪,要不是旁边
一个女人眼明手快,一把朝天托起枪口,就没有小罗斯福总统这回事了。在这个当口,大工会反而衰落。资本家都
在破产,罢工不是自杀吗?有历史学家说,工会力量突然倒退了二十五年。

1932年小罗斯福当选总统,对美国是件大事。他在竞选时就提出要实行``新政'',当时大家还不知是怎么回事,第
二天漫画家就给他画了幅漫画:一个``荷锄者''正迷茫地抬头仰望,一架写着``新政''的飞机正从他头顶飞过。大家
都知道小罗斯福提出``免于匮乏的恐惧'',其实他还提出``每个人都有享受舒适生活的权利''。第二年美国就通过
了国家工业复兴法案(National Industrial Recovery Act),由国家出钱投资的一系列价值三十亿美元的大工程刺
激经济。在今天遇到经济危机,人人都知道有政府投资这一招,在当时还是创新。

人们不大注意到,当时美国左翼走议会道路也相当成功,例如在大萧条下,议会通过的决议还相当``左'',小罗斯
福刚刚上任,参院就通过一项议案:凡是每周工作五天、每天工作六小时以上的人生产出来的东西,不准进入州际贸
易。所谓左,有时其实是一种``时机提前''。主张或者理念本身特别美好。正因为美好,你无法与之争论。难道劳
工不应享有更短的工作时间?可``不合时宜''又可能带来问题。也许未来我们一天只需要工作三小时,可哪怕是今
天,美国不能立法每天工作六小时。不管怎么说,小罗斯福当选,尤其是对组织工会权利、罢工权利、劳工谈判权
利的立法,促使许多左翼政党,看到了和美国制度合作,或者说至少是局部合作的可能。而这些立法在当时也一样
存在危险,就是它实际在刺激全国性工会振兴并且大大增加了引发大工潮的机会。

\pagebreak
\section{对希特勒态度的一波三折}

1932年对于美国共产党,有两件值得一提的事情,一是当时党员都知道,刚刚因心脏病退下的总书记威廉
\textperiodcentered 福斯特(William Z.~Foster)出了一本书:《向苏维埃美国进发》(Toward Soviet America)。
还有,就是坚定的斯大林主义者艾尔\textperiodcentered 布洛德(Earl Browder),继任总书记。他的上任对我们
的麦卡锡议题是个不小的事件。

上次提到美国左翼对苏联1938年``二十一人审判''有不同的看法和判断。独立文章却都把苏联``二十一人审判''和
希特勒挂钩,只是挂的方式不同。赞同莫斯科审判的文章配用了很苏联的大幅政治宣传画。一张画上,代表着革命
阶级的壮壮实实的工人,一手摁住一条正在挣扎中的、戴着布哈林眼镜的人头蟒蛇,另一只手高举锤子,正在往下
砸,而蟒蛇身上画满了纳粹标志。象征着``二十一人审判'',是除去了希特勒打入苏联内部的敌人。而反对莫斯科
审判的美国左翼在文章中认为,斯大林主义的党内大清洗,令亲者痛仇者快。尤其是清洗了一批军队高级将领,大
大折损了社会主义苏联战胜希特勒帝国主义的战斗力。

希特勒上台,是世界局势的一个大变故。美国从一开始就考虑应对。对主流社会来说,凡有暴力倾向、反民主制度、
地下操作等等政治势力,都是一些危险的``非美国式''行为。所以,在麦卡锡时期扮演了重要角色的``美国众议院
非美活动调查委员会''(House Committee on Un-American Activities),简称HUAC,就是在1934年成立的。它的所
谓``非美''并不是后来人们想象的指非同美国的``思想异见'',而指的是暴力暴动和间谍等等这样的``非法政治活
动''。所以,在HUAC成立之初,首先调查的是KKK组织,以及纳粹宣传、组织在美国的渗透和扩张。美国共产党以它
的暴力革命宗旨,以及始终维持和苏联情报部门密切联系的特征,在后来进入HUAC的调查范畴,是一件很自然的事
情。

美国共产党对希特勒的态度,其实是有波折的,而它的波折就是苏联态度的波折。在1936年至1939的西班牙内战
中,两个对抗极端的一方是追随苏联的西班牙共产党和各左翼极端,另一方是受到希特勒军事支持的右翼军人佛朗
哥,共产国际立即号召各国共产党派出``国际纵队''进入西班牙打内战。苏联以预支军事援助为借口,顺手拿走了
西班牙几乎全部的黄金储备。美国共产党把这一事件看作是自我发展的绝佳机会,``反法西斯''和美国主流的口号
吻合。如此在美国动员起大量民众支持,也有不少人因此加入共产党。美国``林肯支队''在西班牙战场上是最出名
的一支,人数超过三千。这是美国共产党在整个党史中最牛气的一段,党员人数突然上升到七万五千人。可是,苏
联很快在大清洗中枪毙了那些从西班牙战场回来的将军们,并在1939年与德国签订所谓``互不侵犯''条约,瓜分了
波兰。每个对这段历史感兴趣的人真是应该看看波兰电影《卡廷森林》(Katyn),苏德条约以河为界,可怜的波兰人
眼看着苏、德两国侵略军从两边压来,在桥上走投无路。问题是美国共产党也因此换了口气,开始在宣传品上倾向
德国。直到德国胃口太大,还想侵吞苏联,苏联被逼无奈才掉转枪口。而美国共产党也随之又转回到``反法西斯''
的路上。可是,这样一个波折,虽不影响党内中坚分子,却对前一阵进来的新党员是个很大打击,这些人呼呼地又
退潮般离美国共产党而去。

\myphoto{image016.jpg}{1938年,美共总书记布洛德在纽约州共产党大会上讲话。}

\pagebreak
\section{双重身份的党总书记}

美国共产党总书记布洛德或许在中国革命史中也是应该被记一笔的。1928年,他和女友海丽思(Kitty Harris)一起
远渡重洋,身负共产国际重任,被派到上海工作了一年。1929年才双双回到美国。

1930年布洛德已是高层领导,他之前就一直是个双重角色。一方面他带领美国共产党公开支持了小罗斯福总统的新
政,同时,他第一个实质推动美共配合苏联利益,建立了苏联在美国的地下间谍网,直接为苏联内务部(NKVD),也
就是后来的克格勃,招募情报人员。布洛德的妹妹就曾是苏联在欧洲的间谍。这套地下机构还时刻准备做一些违法
操作,例如政治暗杀和绑架之类。对美国共产党来说,小罗斯福政府虽然在为改善工人状况努力,也只是恶性比较
低的魔鬼。所以,1939年10月共产国际通过秘密短波命令布洛德不再支持小罗斯福,几天之后,他就领着美国共产
党开始了对罗斯福的攻击。

布洛德象征了美国共产党地上地下两面并行的状态。他是公开合法政党的领导人,甚至在1936年代表美国共产党竞
选美国总统,获得八万多张选票。他同时又是非法外国间谍机构的头目。在1938年莫斯科``二十一人审判''被指控
的托洛茨基,当时正在墨西哥策划成立第四国际。一个被苏联内务部招募的间谍后来回忆说,他当时就和布洛德一
起,面见苏联在美情报官员,一起讨论如何暗杀托洛茨基。布洛德后来持假护照去过苏联。

后面的事情出人意料,在1944年,布洛德在理论上成了修正主义先行者,他提出共产主义和资本主义可以``和平共
存''(peacefully co-exist),按照他的理论,不再是所有资本主义国家,都一定要被革命改变为共产主义国家。布
洛德是总书记,他的原则一度成为美国共产党的指导理论,这个``和平共存说''也就被叫做``布洛德主义''了。一
些历史学家认为,他的做法实际使得美国共产党一度拉开了和苏联的距离。布洛德因此受到各国共产党的猛烈批判。
其实,布洛德可能只是一个理论思考,并不是要和共产主义离心离德的意思,他的基本信仰并没有改变,所以,他
一边受着批判,一边还是继续为苏联情报部门组织在美间谍网。

在1945年11月6日向联邦调查局自首的女间谍领导人柏特丽,她手下的间谍网都是间接或直接来自布洛德。1943年她
接替男友成为间谍头目,也是布洛德的决定。布洛德把自己掌握的美国政府最重要内线,都逐渐交给了她。当时的
模式是,美国共产党为苏联招募情报网,由美国共产党掌控,苏联是间接指挥,并不掌握大多数间谍的名单。这始
终是克格勃的心病,只要关键人物有个三长两短,他们就可能断了线索而失去全部间谍。柏特丽是在苏联和布洛德
的双重领导下。此后克格勃不断要求柏特丽交出间谍网,她后来供称,自己是在布洛德的支持下,才一再坚持不交。
担心一旦交出去,手下的间谍们会更危险,也会失去他们自己在苏方的重要地位。1944年布洛德的修正主义化,使
得莫斯科对美国共产党和间谍的控制,出现双重危机。他们最终迫使他同意向柏特丽下令,向克格勃交出部分美国
间谍网。没有料到,柏特丽自首后愤愤地说,美国共产党总书记``只是一个傀儡,在背后扯线操纵的是莫斯科''。
她说正是因为布洛德的这个命令,成为她理念转变的关键,好像她的美国国民意识突然醒来。而她的传记作者却暗
示说,交出间谍网,她就``什么也不是了'',恐怕这才是她愤而自首的原因。

对莫斯科来说,好在那个一心要建立``苏维埃美国''的前任总书记威廉\textperiodcentered 福斯特还在,他在苏
联指示下,领导了美国共产党内的反修斗争。布洛德先是被夺去了实际领导权,继而总书记的职位被人取代。就在
柏特丽向联邦自首的几个月后,1946年,这位前总书记终于被开除出党。

\pagebreak
\section{HUAC来自一个混乱年代的危机}

上次提到,在1934年成立的``美国众议院非美活动调查委员会''(HUAC),在麦卡锡事件中扮演了重要角色。有人为
它大叫冤枉,说这事根本不能这么算:和麦卡锡有关的调查,是参院调查委员会。而HUAC是众院的,和麦卡锡根本没
关系。是不错,可是他们和``美国第二次惧红时期''有关系,而麦卡锡是这个时期的代名词。所以HUAC和``麦卡锡
时代''的关联是跑也跑不掉的。

开端来自混乱年代的一场危机。说混乱,是突发经济大萧条。胡佛总统是变成历史笑料最狼狈的一个,因为他刚宣
称大好形势,就崩盘了。混乱还在于很多处理社会大问题应有的具体制度还未建立。回看历史,美国一个危机连着
一个危机,内忧外困重重。它充斥着``刚下船''的新移民,全世界的问题都在这里集中,危机都乘积放大。一般国
家都在长期发展中形成维持中央政府财力的一套办法,这钱跟贵族要也好,跟百姓要也好,反正中央政府、国王不
会是穷人。而美国就是这样一个例外。它是一个州的联合体,联邦政府在很长时期里根本没什么钱。可是,第一次
世界大战这样的事情,显然必须联邦出面。美国从立国开始,长期奉行所谓孤立主义,几乎没什么军费预算。独立
战争开始,美国没有军队,就招募民众承诺发债券作为士兵军饷。独立战争除债券还有土地,每个士兵一百英亩什
么的,反正当时土地又多又便宜。兵役债券是政府承诺一个面值,按计算复利,限期兑现。期间根据物价上涨指
数,可以立法调整面值。

是否认真处理兵役债券是非常敏感危险的事情。美国立国之初,打完仗,独立的新政府根本没钱兑现债券,要不是
华盛顿将军利用威信化解危机,一批军官就兵变推翻国会了。这个故事很有名,可是当时还是出事的,1783年有几
百个宾夕法尼亚州的老兵``讨兵饷大游行''到国会,当时国会在费城。议员们还在大厅议事,外面老兵就把国会给
包围了。议员们抓个机会溜走,都逃到新泽西的普林斯顿小镇去躲债,也就是普林斯顿大学的所在地。当时美国政
府是个空架子,连总统都还没有,没有治安力量,也没有钱。只能找人数寥寥的军队把老兵们赶回家。那时的老兵
也老实,也就回家了。还是在美国立宪后,由于第一任总统华盛顿的财政部长汉密尔顿坚持,一定要建立国家财政
信用,才兑现了独立战争的兵役债券。

因为没有足够军费,美国政府在第一次世界大战沿用了这个兵役债券老传统。1924年国会有个《调整服役债券法》
(Adjusted Service Certificate Law),就是调整服役债券的金额、兑现日期等具体规定。按此立法,这批一战债
券当在1945年兑现。但期间老兵可先借支债券面值的22.5\%。一战1918年就结束了,到1932年,1924年立法也已经
八年,大家都依法耐心等待。可是,不是经济大萧条了吗?很多老兵从大萧条开始就失去工作,撑不下去了。兵役债
券危机突然爆发。

\pagebreak
\section{兵饷军和《地方武装法》}

1932年春夏之交,有一万七千名一战退伍老兵,连带家属和各种支持者,差不多五万人,他们直接以``立即提现兵
役债券''的诉求命名自己,自称``兵饷远征军'',后来简称``兵饷军''(Bonus Army)。在退休的布特勒将军
(Smedley Butler)支持下,开始了他们徒步进军华盛顿的``兵饷大游行''。布特勒将军虽已经退休,可在军人中德
高望重。他的鼓励是这次行动的一个重要原因。

其实截至1932年4月,政府已超额发掉了债券全额三分之一的借支,如今面对危机,国会出现分歧。当年6月15日,
众院通过全部提前兑现议案,6月17日,参院却在外面五万人包围抗议之下,还是投票否决了兑现议案,原因是政府
预算只有这点钱,这边发掉,大萧条的其他救济等计划就要断粮。但为了缓解危机,国会同意为回家的老兵支付火
车票。有大约六千人拿了火车票就回去了。可是剩余的人拒绝离开。

这样几万兵饷军就在华盛顿中心地带的河两岸安营扎寨,一点没有要走的意思。7月28日,司法部长米切尔命令警察
驱离。谁知发生冲突,在老兵抵抗中,不知什么情况下,警察开枪打死了两个人。胡佛总统一看,怕再下去更加恶
化,就找来十年后在二战中著名的马歇尔将军,令他指挥军队用催泪瓦斯驱散兵饷军。当然,胡佛总统决不会希望
看到流血冲突。

事实上,美国在1878年6月18日,已经通过《地方武装法》(Posse Comitatus Act)。该法严禁联邦政府行政分支动
用所谓``有制服部队''执行针对各州民众的治安任务。它的理念是,联邦军队是对外打仗的,不是镇压老百姓的。
该法充分限制联邦政府动用军队参与执法的权力,细则中还规定了联邦军队的各兵种,其中包括暂时应招、为联邦
服务的州国民兵。该法的出台是在南北战争之后的重建时期,南北战争第一次让各州看到,联邦召集一支强大军队
的可能,虽然南北战争后,联邦的所谓``北军'',和被打败的``南军''一样,立即作鸟兽散,回家去了。可各州还
是担心未来联邦是否会以军力侵犯州权。着眼点是落在``州权''上,所以该法限定决不容许联邦军队在美国国土之
内的``非联邦产权''(non-federal property)上参与``民用执法''。美国是州联合体,原本根本没有联邦立足之
地,没有什么``联邦房产地产'',它只是慢慢向各州购买到一些房产地产,一开始在大家印象中都小得可以忽略不
计;而且,也许立法者还想,联邦财产也就是一些联邦政府机构,联邦总不会用军队来对付自己吧。

\pagebreak
\section{《地方武装法》的漏洞}

美国立法者一向是小心抠字眼的,做梦也没想到,偏偏这个立法有漏洞,一是根据宪法,美国首都华盛顿特区直属
国会管理,也就是联邦财产,因此不在该法管辖范围。二是这里又偏偏是各州百姓最喜欢去抗议示威的联邦政府所
在地。这么说吧,根据《地方武装法》,联邦军队确实无权去各州执法,可是各州的人跑来占领联邦土地,此法并
不能禁止联邦军队作为执法力量来驱赶外人。所以,胡佛总统也就有这个合法权力。

当时美国对民间持枪还没有今天那么多限制,而兵饷军说起来又是战场上下来的,虽说只打算用催泪弹驱赶,可前
面和警察冲突已有伤亡,谁也吃不准对方会不会有暴力应对。所以,麦克阿瑟的军队是坦克殿后、端着刺刀上去
的,其中还包括二战中很有名的巴顿将军,虽然当时他只还是个少校。

这一幕真是太悲剧:是联邦军队去面对自己的退伍老兵,而且就在首都中心地带。军队就是从白宫前面最著名的宾夕
法尼亚大道列队过去,许多民众、包括政府工作人员,闻讯丢下工作跑去看,不断有人大叫``可耻''。有一种说
法,就是麦克阿瑟将军之所以会执行驱赶任务,有他主观上对兵饷军的反感,他认为这只是个乌合之众,最多只有
十分之一是真的老兵,其余都是些在一战中反战捣乱的人,乘机煽动利用了少数不明真相的老兵,来挑起事端。至
今有很多人以为,战争是士兵冲在第一线,所以一提反战,应该首先是士兵会支持,例如今天很多人以为伊拉克战
争的士兵必定会支持反战,我在军营住过之后,就知道事实并非如此。士兵反应还是要看士兵对自己正在打的战争
如何理解。

兵饷军有几个营地。军队先接近靠近中心地带的营地,营中人一见这个阵势,纷纷逃过河、进入对岸最大一个营地。
一种说法是,胡佛总统认为过河就算了,立即下令军队中止行动。是麦克阿瑟将军决定不顾总统命令,继续过河行
动,结果导致冲突,两名老兵死亡;另一种说法是传令有误,麦克阿瑟并不清楚后面的停止命令。当时可能确实无
法搞清到底老兵、家属以及有反战等政治诉求的民众团体,各为多少比例。现在不同的书还是数字不同。不管怎么
说,在这次冲突死亡的两人,有名有姓,确确实实是退伍老兵。

\pagebreak
\section{法西斯试图利用兵饷军事件,引出HUAC调查}

美国的``制度内''运作是这样的:在宪法框架下,一个行为假如没有法律细则限定,是合法的。随着发展,遇到问题、
甚至发生悲剧,发现有法律漏洞,就产生新立法弥补漏洞。一些立法在新时代被认为过时、不妥、与宪法实质精神
有违背,就再通过法定程序撤销。但是,并不是说,在法律漏洞存在的时候,你就可以理所当然地去钻漏洞制造恶
性事件。胡佛总统的决定有他的理由,而且驱散并不是一个动武镇压的命令,但有人认为,他应该考虑到可能发生
的冲突和伤亡。整个事件始终都在各大媒体、全国民众的观察、批评之下。

有历史学家认为,在处理兵饷军事件中出现伤亡,是胡佛总统后来在选举中输给小罗斯福的原因之一。但是小罗斯
福当选总统之后,也认为不应该提前兑现兵役债券。结果,1933年,兵饷军再次进发华盛顿。小罗斯福于是派出夫
人去和他们沟通,也说服他们参与政府工程提供的工作。他们很多人因此参与了修建美国最南端一号公路的工程,
可是这些老兵真是命运多舛。就在1935年美国劳动节那天,一场特大暴风袭击,随之引发洪水,参与修路的二百五
十八名一战老兵死于天灾。在噩耗刺激下,国会以压倒性多数越过小罗斯福总统的否决,通过了提前在1936年兑现
应该1945年到期的战争债券。

这一事件最大的效应,就是美国军队走出了现代化的一个关键步骤,有了完善的服役退役兵员优待制度的概念,促
成了1944年《军人再调整法》(Servicemen's Readjustment Act),也被大家叫做《大兵权利法案》(G.I Bill of
Rights)或《大兵法》(G.I Bill)。1944年是二战结束前一年,``大兵法''提前确保了对二战退伍兵的一系列优惠项
目。不仅有经济优惠,还有提供教育、职业培训的一系列人道措施,使得退伍兵能够有个过渡,重新回到社会。这
也是后话了。

就1932年兵饷军事件来说,直接效果是引发了大批一战退伍兵和政府的对立情绪。包括那个支持兵饷军的最重要人
物,就是退役将军布特勒。此人几乎是个传奇。他是美国海军陆战队的先驱人物,可以说没有什么战事他是不在场
的。有1900年义和团运动导致在中国的战事,当然也有第一次世界大战。作为士兵他以勇敢著称,作为军官他身先
士卒,凭着几十年军旅资历,他在军人中极有号召力。他有个外号,叫做``教友派战士'',这是个非常矛盾的称
呼,因为教友派在美国是出名地信奉和平主义。退役后,这个十六岁就参军的将军突然又回归和平主义,成为美国
的一个反战标志。在兵饷军事件的1932年,他刚刚退休,而且之前差点被任命为美国海军陆战队的司令官。可以想
见他的支持所产生的影响。反过来,事件结果对他产生极大刺激。所以,也就有人认为,假如要推翻美国政府,他
是最佳的参与者和新总统人选。

1934年的某一天,发生了一件后来震惊全美的事情,公信力极佳的布特勒将军步入美国国会,报告说,有美国财团
在联合欧洲的法西斯力量,试图阴谋推翻小罗斯福总统的政府,并且选中他作为政变首领。国会受到的震动非同小
可,决定要成立专门委员会展开调查,这个机构,就是麦卡锡时期著名的美国众议院非美活动调查委员会(HUAC)。

\pagebreak
\section{布特勒将军的法西斯政变故事}

根据布特勒将军作证,在1933年兵饷军事件中,就有一个华尔街的雇员迈可奎尔(Gerald MacGuire)来找他,希望他
出来竞选全国退伍兵协会主席。此后又到旅馆单独会面,告诉他,一些大企业家、金融家在策划一场政变,要布特
勒将军出来``领军''。期间迈可奎尔还去了一次欧洲,据说是考察各种形式的法西斯。

假如一个美国媒体发了一个错误消息,被起诉诽谤,法庭在判断媒体是否可原谅的各标准中,有一条就是它的消息
来源是不是具有公信力。公信力真是一件万分重要的事情。媒体失去公信力,就断送了自己的活路,政府更是如此。
当1934年布特勒将军走进美国国会,作证有一个非法的法西斯阴谋正在酝酿的时候,将军的良好声誉是一个背书。
可是,对于众议院非美活动调查委员会(HUAC)来说,这是不够的。调查必须招大量证人作证,希望得出可信结论,
至少结论要能维护国会公信力。可是,这并不容易。

事件的关键人物迈可奎尔,在誓言之下,坚决否认布特勒将军也是在誓言之下的指控。事件一开始,《纽约时报》
就断言这是个``天大笑话''。也有大量媒体追踪报道细节。结果,证人队伍越来越长,证词记录越积越厚,整整调
查了一年多。直到1935年2月,HUAC才宣布结果。调查显示,确有阴谋在策划之中,有迈可奎尔的欧洲报告在证实,
也有其他证人的证词确认。只是,这个密谋并没有付诸实际行动,资金没有到位,说是打算政变组织的五十万``队
伍''也还八字都没有一撇。但是,确实有过这样一个疯狂念头和策划。所以,国会宣布结果后,对这个基本上属于
纸上谈兵的密谋案,就算了结了。这其实很有美国特色,国会认为阴谋对国家还没有伤害,阴谋败露,估计也不会
继续下去,有了结论,搞清楚,就可以了。同时,也不认为有必要再移交司法机关。一是在美国嚷嚷着要革命要推
翻政府的组织很多,还在层出不穷。只要不具有``现实的、迫在眉睫''的危险,也就由他们去了;二是司法定罪要
求的证据不知要严格多少倍。众院并不惩罚,目的只是排除国家的危险。

但是,在民间,政府说法永远有人不买账,争论留存到今天,可也有历史学家指出,在经济大萧条冲击下,当时确
有一些美国人对自由主义的民主制度产生怀疑,一些人开始倾向共产主义,一些人认为,是不是希特勒鼓吹的``国
家社会主义''的管理更为有效。当然,那个时候还没有多少人知道,``国家社会主义''等后来导致灾难的各色主
义,会显露出怎样的面貌。也有很多人对法西斯,就是意大利的``工团主义''感兴趣:不管怎么说,除了墨索里尼,
有谁能让意大利火车准点呢?

通过这个调查,HUAC听证会的方式被确立和承认,从此开始了它的一系列调查。

\pagebreak
\section{政府和艺术}

大家知道,HUAC后来在麦卡锡时期饱受诟病,是因为盯上了好莱坞。他们和好莱坞怎么会过不去?事情的开端,是因
为小罗斯福政府和文化艺术界开始有了牵连。

HUAC的前一个调查是针对要推翻小罗斯福政府的阴谋,后面的调查反而和小罗斯福政府本身有关了。前面讲起过,
美国共产党一度是公开支持小罗斯福政府的。而当时所谓``新政''的一个重要部分,就是政府拿出税金,投资大量
政府工程,为大萧条失业者创造工作机会,主要是修桥铺路。美国直到今天,很多电线杆都是上百年历史、节节疤
疤东倒西歪的细木杆,因为电力公司是私营的,能对付就对付了。现在也一样,金融危机下,奥巴马政府就决定投
资铺设地下电缆。可是政府投资解决就业,有了一个问题,艺术家算不算救援对象?

艺术家有点不一样,他们更像是在为自己工作。他们的作品也卖给公司,可是并不固定卖给哪家。艺术和思想是自
由的,艺术家是个体游荡的。可在大萧条时期,第一个断粮的就应该是他们,很多顾客不吃面包不行,花闲钱享受
艺术的心思是断断没有了。

最后,小罗斯福政府还是决定在新政``工作项目委员会''(WPA)下面设一个支持艺术家的``一号计划''(Federal
Project Number One),是所谓``第二新政''中最大的计划,包括联邦作家、历史记录调查、联邦音乐、联邦艺术等
计划,记得离我们家不算太远的大学艺术博物馆,就陈列过新政时期联邦艺术计划的画家作品。``一号计划''中最
大的一个,就是``联邦剧院计划''(Federal Theatre Project),简称``FTP''。

可是,在美国政府资助艺术是存在``问题''的,就是艺术其实牵涉意识形态,可以是一种说不出道不白的政治表
达,因为人是一种无法描绘的奇怪动物。摇滚音乐可以一句歌词都没有,凭节奏就可能强烈冲击社会观念以至动摇
一个制度,否则现代艺术也不会在一些国家变得敏感甚至要禁止了。所以,政府要出钱投资文化艺术,美国人条件
反射般的第一反应,很可能不是欢欣鼓舞而是引起警惕:政府是不是想利用金钱控制思想表达呢?所以,立项的时
候,政府先想到的,也是要摆脱思想控制的干系。

在联邦剧院计划的主管人选上就很费脑筋,最后选中一位女士,海莉\textperiodcentered 弗兰纳甘(Hallie
Flanagan),她曾经留学欧洲,跟斯坦尼斯拉夫斯基等等一流大师学戏剧,自己教戏剧、写剧本,也活跃在实验剧
场,她向参与FTP,也就是接受资助的戏剧家艺术家定下原则:他们的作品将可以是``自由的、成人的、政府不审查
内容的''。也就是说,创作几乎没有任何政府限制。

\pagebreak
\section{美国大兴活报剧}

如此一来,立即聚集了一大批左翼作家、剧作家、演员和失业记者,立即引入了由俄国十月革命所开创的活报剧。
后来活报剧也流传到中国,抗日战争很出名的《放下你的鞭子》就是。活报剧在内战中的共产党占领区以及1949年
后的中国,一度作为宣传手段非常流行。苏联活报剧在1923年至1928年达到高潮,而负责FTP的弗兰纳甘也是个左
翼,1926年去过苏联,她在那里深深被活报剧吸引。活报剧进入英语世界,1935年的美国FTP计划算是首创,其中
``纽约活报剧组''是最活跃的一个剧组。德国和欧洲的工人剧院也有过活报剧演出。美国活报剧一度红红火火,确
实是FTP起了作用。

从政府干预的角度去看,也出过一点问题。其中有个1936年准备推出的活报剧,叫作《埃塞俄比亚》,被外交部通
知不准上演,原因是直接出现了墨索里尼和一批外国政治家的漫画化形象。当时的国际形势紧张而吊诡,外交部坚
持认为,美国人自己的政治家你要怎么丑化都可以,现在战云密布紧要关头,不见闪都要炸雷,冲突一触即发,舞
台就暂时委屈一下,还是不要惹是生非。气得写剧本的艾尔默\textperiodcentered 赖斯(Elmer Rice)当下退出。
接替他带领活报剧组的,是刚刚从莫斯科回来、大家都知道是共产党在支持的约瑟夫\textperiodcentered 罗西
(Joseph Losey)。

活报剧会吸收一批记者参加,因为在剧组理解中,这只是报纸的另类,是活人演示的时事政治批评。接下来,他们
在主管弗兰纳甘``三原则''之下,包括罗西自己也参与执导,编写演出了一批活报剧。从批评政府,``点名道姓''
批评某个立法,支持工人运动攻击资本家,讽刺大报业老板赫斯特,直到攻击美国最高法院,公开在剧中呼吁大家
参加``产业工人组织议会''(CIO),那是个出了名相当``好斗''的全国性左翼组织。这一来,必然引发了大量批评。
也许是为了保住FTP,他们于是掉转船头,又上演了一些为小罗斯福新政工程唱赞歌的活报剧,又有点``新政宣传
队''的味道。

所以,从美国主流社会观念来看,这个新政援助计划,怎么说也有点``不对''了。因为以前美国没有政府和艺术的
瓜葛,一开始担心援助变为管制,后来看上去好像出现了相反问题。FTP计划聚集大量左翼人士共产党人,演出内容
有明显的政治倾向,是不是共产党利用纳税人的钱来宣传推翻美国制度呢?因此HUAC在1938年对这个计划作了听证调
查,FTP主管弗兰纳甘也被传到国会听证,想搞清楚参与计划的到底有多少共产党,调查中一个有名的问题是:FTP是
不是把诗神缪斯已经悄悄演化成阶级斗争的战神了。

\myphoto{image018.jpg}{海莉\textperiodcentered 弗兰纳甘主管的``美国联邦剧院计划''资助排演艾略特的话剧
《大教堂凶杀案》(Murder in the Cathedral)}

\pagebreak
\section{HUAC调查艺术家}

今天的历史学家往往简化说,众院非美活动调查委员会,在三十年代就举行听证会,调查左翼艺术家的戏剧活动,
这是典型的思想迫害和意识形态控制。但其实是很复杂、很绕弯、很不容易说清楚的事情。

说到底,最根本原因,还是政府出钱惹的祸。这些左翼艺术家又不是刚刚冒出来,他们一直在那里公开活动,很多
人多年来在致力于利用艺术为鼓吹阶级斗争服务。

这是美国制度内反对派的言论自由。可是,现在不一样,你们是用纳税人的钱作``政治表达'',当然要得到纳税人
的同意。民众会说,我们就连利用税金来宣扬大多数人信奉的宗教都不可以的,怎么现在容许政府出钱宣传某个极
端政治观点。说得再极端点,就是假如有个剧团只是在民间宣扬法西斯主义,美国政府是不来管的。可是,假如国
会听说政府的行政分支,拿了老百姓税金,支持一帮剧作家和演员宣传法西斯主义,国会当然要调查,当然首先要
怀疑,政府行政分支的官员是不是被什么人收买操纵了。很有意思的是,反过来也不可以,行政分支也不能操纵组
织一群艺术家来宣传新政。

说它复杂、绕弯,因为FTP是一条奇怪路径:是大萧条失业救援,使得政府和艺术家发生独特联系;又因为艺术是
``表达'',政府就规定不作审查限制;而由剧作家自由发挥,在那个年代,很自然就是左倾占绝对优势,甚至有极
端左翼宣传。事实造成纳税人的血汗可能通过政府转手、支持了反对美国制度的宣传甚至外国操纵的政党。它的背
景又是世界革命浪潮上升期。离开当时的历史氛围,人们在今天判断这些事情就很难把握。

今天一些美国历史学家,对HUAC这次调查的判断是``迫害艺术家''。可是从一个事实就可以看出当时情况的复杂,
不要说FTP主管就是莫斯科仰慕者,就连HUAC的副主席迪克斯坦(Samuel Dickstein),后来都被发现,他的名字赫然
列在苏联内务部的间谍名单上。就和HUAC对布特勒将军牵涉法西斯政变案的调查一样,它只是一个调查。最后,国
会在1939年决定撤销FTP项目,应该说还是个很正常的决定。在它存在的四年期间,它始终是``一号计划''中,最庞
大、花钱最多的一个项目。

问题是,小罗斯福政府和文化艺术界的牵连不止于此,它还指示好莱坞拍了个赞扬苏联的电影,这在美国绝对是犯
忌的事情。于是,也就很自然把HUAC的调查进一步引向了好莱坞。

\pagebreak
\section{1945年``站住''的HUAC}

众议员非美活动调查委员会(HUAC)在第二次世界大战结束之前,是一个``有案招之即来,无案挥之即去''的临时调
查机构。可是在1945年``站住''(standing),就是变成了众院一个由九名委员组成的常设机构。1969年改过名字,
到1975年取消,实际上是把它的功能转给了众院司法委员会,就是管弹劾总统的那个,到现在还在那里。

HUAC转为常设机构的立法,来自第七十九届国会,由民主党占多数主导地位。它通过了一个《公共法案(Public
Law)601》,规定了HUAC的工作就是调查对``宪法所规定的美国政府形式''的颠覆威胁,以及调查为颠覆作宣传
(propaganda)的嫌疑。引号内的法律语言,说白了就是``民主制度''四个字。不过,从前面对``法西斯颠覆计划''
一案的调查,以及对``以税金援助戏剧''的调查,都可以看到,这种调查只不过是调查。前者虽确定有阴谋计划,
但评估判断它没有``即时危险'',也就算了,好像查过了大家可以放心,并不引发司法行动。被调查者回家原来该
干什么还干什么。对政府资助有反美倾向戏剧的调查也一样。国会也就是在一年后中止了政府资助计划,并没有对
左翼戏剧家进行政治迫害。

大家一定很奇怪,就是这个法案中的``宣传''的重要位置。话说宣传,真是不知如何说它是好。说到这里,文化差
异可谓天差地别。在二次大战以后,由于有了戈培尔声名狼藉的纳粹宣传部在前,在西方社会有人若是再提宣传
(propaganda)一词,人们会条件反射的很是奇怪:你不是在开自己玩笑吧?因为查百科全书,propaganda正儿八经的
解释就是:``它是出于某种动机要去影响民众的一种传达信息方式。为了影响它的宣传对象,'不公平地提供信息'是
它的最基本概念和方法。它提供有选择性的信息,旨在导致情绪化的反应、而不是理性的分析,以此期待和引导民
众转向政治宣传所指向的立场。''也就是故意误导民众。所以,有时我在美国听到哪个国家的英语节目,正儿八经
发布消息,竟然宣称来源是``the Office of Propaganda'',真是有心惊肉跳的感觉。

美国人对此极为敏感。威尔逊总统在第一次世界大战汹涌澎湃的反战浪潮面前,百般无奈,曾在1917年4月建立了政
府行政分支下的``公共信息委员会''(CPI),它的主持人乔治\textperiodcentered 克里尔(George Creel)出来说,
它的目标是为了向美国人民和世界解释``美国为何要拿起枪来,保卫自己的自由和自由体制''。CPI召唤了大量学者、
记者、艺术家、广告制作商,制作出一批宣传品来。至今在美国被不断反省和饱受批评,认为CPI有propaganda的性
质。

\pagebreak
\section{戈培尔的宣传曾经是一个公平竞争}

HUAC在1945年成为常设机构,会异乎寻常地把``宣传''列为和颠覆同等位置来对待,被今天的历史学家评论为惧怕
红色的``歇斯底里'',看上去无可辩驳。可是回到那个时代,它显然来自二战的强烈刺激:大家第一次看着一个现代
集权制度,有能力把一个国家甚至世界推到怎样的一个深渊;也是第一次看到,在一个民主国家,主张集权的政
党,如何以``宣传''一步步改变了国民,逐渐把它转化为一个集权国家,这居然是可能的。再说这是歌德、巴赫和
贝多芬的国家,并非蛮荒之地,国民受过良好教育,应该不是傻瓜,照美国人的想法,比我们的文明底子厚多了。
而追根溯源,无法否认戈培尔的宣传策略起了重要作用。

今天有很多人基于对集权制度的批判,反省批评第三帝国的宣传,指出它建立在一个控制信息、对异议遏制的政权
能力之上,也就是在1933年希特勒上台以后,由政府设立国民启蒙宣传部和全国作家协会,都是戈培尔当的头,以
国家力量掌控媒体。可是,对于美国这样的国家,他们更为震惊的还是1933年之前发生的事情。当时的纳粹,只是
一个言论自由国家的在野党,戈培尔只是在野党宣传部长(1928年开始),一开始他并没有国家权力的支持。戈培尔
从1927年开始办一张普普通通的民间报纸\myrule 《攻击日报》(Der Angriff),他手中没有任何权力去堵住其他报
纸的嘴,他只是运用自己的能力去宣扬自己的政治观点、攻击对手(你也可以说是辩论)、对读者煽情。他的煽情宣
传显然大有成效,可是你不能说他不公平(你也可以``对煽'')。人们直到今天还是只能承认,他运用的只是``新闻
工作者的高度技巧''。

今天人们提起戈培尔,总好像出世就是恶魔一般,其实他一开始也是个正常好学的德国小伙子,刻苦学着文学历史
和古典文献学,读出博士学位来。在种族问题上也曾并不偏激,不仅师从犹太导师,还有过犹太血统的女友,似乎
为人也厚道,对妻子与前夫生的孩子一直视如己出。他和千千万万受过良好教育的正常德国人一样,只是比他们早
一步信仰了纳粹思想,并且向他的同胞们推广。我想说的是,把这样的思想看作是``坏人''才接受的,是小看了这
些思想和信仰的力量了。

在经济大萧条的危难时刻,自尊心极强的戈培尔却和他的同胞们一样,饱受失业之苦,他因此寻求纳粹即``国家社
会主义党''(其实也可以译作``民族社会主义'')道路,有它极大的合理性和必然性。所以,他的话对处于同样困境
的同胞们深具感染力。他很明白,他的思想和左翼其实不是敌人,和自由经济的资本主义制度才是敌人。

\pagebreak
\section{天时地利人和}

当时柏林被称作``红色柏林'',就是社会主义、共产主义党派占上风,戈培尔认为,要考量的只是:``国家(民族)和
社会主义!何者为后何者为先?''``毫无疑问,先是社会主义,再是国家(民族)的解放(liberation)。''他认为希特
勒的立场是站在两者之间。1925年,戈培尔出版了著名的公开信,致``我的左翼朋友'',强烈呼吁社会主义政党和
纳粹结盟,一起反对资本主义。``我们虽然看上去针锋相对,可是我们并不是敌人。''戈培尔认为,他们和社会主
义政党之间只是内部争斗。在1928年5月,戈培尔通过民主选举被选为国会议员,当时纳粹党才占十二个议席。就是
那年,他当上政党的宣传部长,并且在他的努力下,两年后的1930年,纳粹党在民主选举中获得一百零七席,成为
德国第二大党。

今天人们对于纳粹通过民主选举的方式扩展觉得非常奇怪。实际上这是很简单的事情。当时现代工业暴长,社会问
题还根本来不及消化,社会救济和福利都还没有好好发展出来,更不要说完善。又突然遭遇前所未有的经济大萧
条,大家的第一反应当然是要换一条道路试试。所谓民主选举,只是一人一票。在当时,现代资本主义自由经济还
在发展初期,并没有表现出它自我调整和修正的能力,反倒是暴露出来的问题触目惊心,深刻影响家家户户和每一
个人的生存。而社会主义、共产主义的理想,在当时是画在天空彩虹般美好的理想。它只在苏联开始实践,一是它
没有如资本主义在全球大面积实践和暴露出问题,二是苏联恰是远在1933年的德国之前,第一个成功推行国家
propaganda的地方。也就是说,它最大限度地向世界掩盖了它的负面信息,而告诉世界,那里的人们生活幸福无比。

可以说,那个时代是左翼宣传发展的最佳时机。不论什么主义都还没有成熟地发展和展现自己。民众眼界还小,也
就不成熟,在一个经济制度出问题的时候,当然掉转头去迎合另一种主义的宣传。在今天,左翼宣传的机会相对就
少。因为两种制度都有过充分的表演,民众也就成熟了。现在遇到危机,美国政府向私人金融机构和企业大量注资
的时候,反而是总统要声明政府无意拥有它们、在危机缓解的时候会随时撤出,以消除民众顾虑,奥巴马恨不能对
疑虑重重的民众说,你们以为我比你们更笨吗?

回到1933年以前的戈培尔,虽然当时他还没有能力假借国家力量,可是他的做法符合propaganda的定义。国家民族
顶在前面,他是煽情的,更重要的是,他所宣扬的、未来由希特勒实践的社会,这个前景的许多信息、尤其是负面
信息,是并不出现在他的propaganda之中的。

关键是,那是激进主义天时地利人和的年代,是戈培尔们的天堂。民众已经准备好,只需要加一点点propaganda。

\myphoto{image020.jpg}{戈培尔执掌的宣传部是受希特勒控制的propaganda}

\pagebreak
\section{HUAC从propaganda切入好莱坞}

现在想来,纸上谈兵的主义辩论永远辩不出名堂,任何由学者研究出来的主义,从理论去看,往往都是逻辑自恰
的,在一定的气候下,都有充分道理使得民众为之动心。反倒是顺其自然发展起来的老制度,一看就知道它是不完
美的、无法和书斋里精心编织构想出来的理想化东西匹敌。老制度只能在实践中慢慢修正。最终,社会和民众的成
熟都是来自眼见为实,来自血的教训。在二战以后,纳粹成为人人喊打的过街老鼠,美国不再有人担心民众会被纳
粹宣传所鼓动,也就至今并不禁止。可是在德国,社会被纳粹宣传的力量所惊怵,反倒认为必须宣布纳粹为非法,
禁止它的政党组织,也禁止它的宣传。可以想见,在我们统而论之、不论历史阶段和社会环境(时间、条件、地点)
地谈论一些话题,例如思想的控制,是没有意义的,因为历史遭遇的难题远为复杂。

所以,今天有人嘲笑美国在1776年并没有立即给予全国一人一票,即全面而彻底的民主,这样的嘲笑,只是提醒大
家去看到美国立国者的务实和谨慎,他们深知社会和民众的成熟需要时间。而在今天,一些国家,其民间社会的进
步已经明显走在集权政府前面,而政府继续以民智未开,来抵御民主制度的逐步推进,也是非常危险的。因为这是
另一个时间、条件和地点的故事。

国会《公共法案601》要求警惕``非美活动'',其实质就是把民主制度当作一端,而把集权制度归作威胁民主的另一
端。时过境迁,纳粹、法西斯式的集权制度,不论是政变还是宣传,对于1945年的HUAC来说,已经完全不像1934年
调查布特勒将军一案时那样,会去认真对待。其原因不仅是法西斯外部威胁消除,更在于美国民众已经有了免疫力。
可是,美国国会认为,另一种集权,即苏式制度仍然是一个难以忽视的威胁。由于二战初期苏德短暂合作的破裂,
苏联同样受到德国入侵,还因此被迫与英美法等结盟,而苏式集权的残酷,尚躲藏在国家宣传的铁幕之后,亮在前
面的是``反法西斯''的英武形象。所以,二战造成了一个奇特效果,它使得两个非常相近的集权制度,一个堕入地
狱,一个却如上升的明星在天空闪耀,而苏式propaganda可以说是宣传典范,很快鼓动了一大批国家的左翼民间武
装组织,追随其后,有样学样,夺取政权裹胁着整个国家,投入了它的阵营。

因此,集权制度的``propaganda''在二战后被当作是一个对民主制度可能的威胁,实在不算奇怪。HUAC成为常设机
构之后,并没有什么大动作。两年后,在非常紧张的冷战形势下,好莱坞电影公司的管理人员提醒HUAC,他们认为
好莱坞拍过一些电影,应该可以看作是亲苏的``propaganda''。于是,在1947年,HUAC开始介入对苏联势力、也就
是美国共产党渗透好莱坞状况的调查。

\pagebreak
\section{好莱坞的亲苏电影}

1947年,有好莱坞工作室管理人员,不满美国亲苏电影的propaganda倾向,向众议员非美活动调查委员会(HUAC)提
出,这是属于他们工作范围,也就是调查宣传集权制度。

这些电影其实是二战中拍的所谓战争片。美国当时拍了好几个和反侵略主题有关的``外国故事''。例如赛珍珠写的
中国题材小说《龙种》(Dragon Seed)就在1944年拍成电影,由著名女星凯瑟琳\textperiodcentered 赫本
(Katharine Heburn)主演,讲二战期间一个中国小村庄反日军侵略的故事。同样,美国也拍了一些类似战争片,场
景是在苏联。如好莱坞1943年艺术家联盟工作室拍的《三个俄国女孩》(Three Russian Girls)、哥伦比亚公司拍的
《斯大林格勒来的男孩》(Boy from Stalingrad)等等。其中有些成为1947年HUAC调查的``亲苏电影''。

例如有个故事片叫《俄国之歌》,是米高梅(MGM)在1944年拍的。讲的是一个美国乐队指挥(由罗伯特
\textperiodcentered 泰勒$[$Robert Taylor$]$主演),他和经理一起,在德国入侵前访问了斯大林苏联。他们穿
越乡村,走访了几十个城市。沿途所见,到处莺歌燕舞。自由的苏联人民健康、满面笑容、幸福地生活在共产主义
梦想中,德国入侵摧毁了他们的完美生活。电影一开头是放美国国歌,然后逐渐歌声消失在镜头推出的头顶飘扬镰
刀铁锤红旗的苏联群众中。还有一个是1943年RKO拍的《北方的星》(The North Star),也是战争片。故事发生在当
时属于苏联的乌克兰小村庄,电影前面的部分描写了苏联集体农庄的幸福生活,照英国外交部门一名历史学家的话
说,它比苏联电影院放的集体农庄宣传电影还要虚假夸张。

《俄国之歌》的听证会上,HUAC请来一位名叫安\textperiodcentered 兰德(Ayn Rand)的美国畅销书作家和哲学家
作证,她出生于俄国,二十一岁来美国探亲,再也没有回去。1935至1936年间,兰德也是好莱坞的一个剧作者。她
的小说后来也被好莱坞拍成电影,个人信誉相当好。她的作证,一是说明她见证的苏联和电影描画有天壤之别。二
是作为一个剧作家,她认为类似片头的处理,已经超越技术,是一种propaganda。她特地提到电影里拍摄的德军入
侵的边境区,不是苏联而是波兰。那里在德军入侵前,已经被苏联入侵奴役。她说自己接触很多俄国移民,了解到
她离开苏联的1926年,相比后来还算是比较好的短暂时期,此后更糟,她列举了1933年集体农庄政策导致的大饥
荒,历史学家确认至少造成苏联农村三百万人死亡。

\pagebreak
\section{《出使莫斯科》}

当然,兰德的证词后来是大家都熟悉的历史事实。可是,在今天,美国人一定会说,那又怎么样?不就是一部对他国
历史没认真考据就瞎编的故事片吗?好莱坞粗制滥造的故事太多了,值得那么认真吗?

HUAC的调查进行下去,事情似乎真的就变得认真起来。

追溯这些``亲苏片''的源头,是分量最重的一部:华纳兄弟电影公司拍的《出使莫斯科》(Mission to Moscow)。这
部电影和其他那些很不相同,它是根据小罗斯福1936年至1938年派往苏联的美国大使约瑟夫\textperiodcentered
戴维斯(Joseph E. Davies)的同名回忆录改编的。所以,照现在时髦的说法,它具有一定的``公信力'',它产生的
影响,不是那些容许胡编乱造的好莱坞商业片娱乐片所能够匹敌的。在上世纪三四十年代,苏联作为唯一以共产主
义为蓝图构造的新国家,所有美国人、甚至各国民众,都至少有强烈好奇心,渴望通过一个可信渠道,一个中性立
场的人物,深入现场,看看那条新道路到底走得怎样。那么,还有什么能比一个美国大使更合适向大家做这样的介
绍呢?所以这本书在美国卖了七十万册,还翻成十三种语言在世界各地发行。

电影改编认真、忠实原著,在作者和华纳兄弟公司签署的合同中,作者享有对内容的所有权力,一切修改都必须得
到大使认可。影片是那种所谓仿纪录片风格。电影一开始,是大使真人坐在椅子上说一段话:``在两次世界大战之间
的危难岁月中,没有一个国家像苏联那样,它的一批领导人被如此误传和误解。''接着,转向电影正文,它是以两
部分内容交替:画外音以大使身份介绍苏联政治和共产主义,同时穿插大使一家在苏联的生活,里面出现大量政治人
物。影片也提到1938年的``莫斯科二十一人审判'',戴维斯大使在给出他的权威结论,他确实比任何人都有这个资
格。一是他在斯大林安排下,是为数极少的进入法庭现场的外国人,旁听见证了审判全程;二是戴维斯大使是律师
出身。他说根据自己在美二十年的律师实践经验,可以得出结论:审判是公正的,布哈林等人,都是德国和日本安插
在红色心脏内的``第五纵队''。大使一家在苏联的大量旅行画面,让大家``看到一个真苏联''。

\pagebreak
\section{政府对好莱坞的介入}

在HUAC的1947年听证会上,华纳兄弟公司的创始人之一,杰克\textperiodcentered 华纳,才第一次提到,《出使
莫斯科》是作者拿着小罗斯福总统的许可信,要求他们公司拍摄的。所以我前面曾经提到过,调查其实涉及小罗斯
福的政府行政分支。

在调查中还逐渐显示,在电影拍摄期间,小罗斯福非常关心这部影片的拍摄,曾多次和戴维斯商讨电影进程。电影
受到政府战争信息办公室的关注,直到他们认为很满意,这片子可以端出去了。他们在给白宫的报告中写道:这部电
影``以最可信的方式使得美国人理解盟友俄国。每个细节都在展示美苏之间根本没什么差别。电影中俄国人居住之
舒服、食物之丰盛的程度将令美国人深感吃惊。最佳镜头表现之一,是展示了一批俄国领导人……是一些有远见、
诚挚、负责任的政治家''。甚至在政府官员的建议下,电影制作者在电影中粉饰和解释了瓜分波兰的苏德协议,以
及苏联红军在入侵波兰之后,又在1939年11月对芬兰的入侵。

事实上,正是几乎和德国同时进行的这两场苏军非法侵略,导致苏联立即被联合国前身的国联开除。人们一直说,
二战的开端是德国入侵波兰,准确地说,是苏德秘密协议的签订、苏德瓜分两国之间的中欧,才是二次世界大战的
开端。

作为行政分支来说,战争当然是这些propaganda影片制作的一个借口。罗斯福总统完全可以说,面对现实,你要我
怎么办呢?两个有侵略野心的集权国家先是密谋、对外侵略。可谁知接着他们自己就打起来了。待到美国参战,苏联
早已经加入了对抗德国的欧洲同盟。这个时候,美国除了和苏联结盟没有别的路可走。但是,你又如何向你的国民
和提着命上战场的士兵交待呢?说是我们联合作战的盟国是个比纳粹德国好不到哪里去的国家吗?也许,这是二次大
战美国行政分支的``战时宣传必要论''。

然而,再进一步深究下去,好像这个借口又是有疑问的。首先是电影《出使莫斯科》的基础是一本畅销回忆录。戴
维斯大使并不是为了给政府解决与苏联结盟作战的理由,才写的这本书。也就是说,为苏联作propaganda是这位重
要外交官员的本人意愿。

\pagebreak
\section{引申出来的故事}

无疑,苏联对外propaganda非常用心的首要对象,就是各国外交使节,当然是要求相关部门努力给他们留下最好印
象,包括精心安排的旅行、对指定专门接待外宾的集体农庄和工厂的参观,这种特殊安排,也有个专门的英语词,
叫作``清洁旅行''(sanitized tours)。在戴维斯之前,美国驻苏联大使是威廉\textperiodcentered 布立特
(William Christian Bullitt,Jr.),他刚到莫斯科,也是无法幸免地就在苏式propaganda面前中招。但是,他还
是很快就发现真相并且表示了对斯大林的厌恶。

令戴维斯大使的使馆部下们百思不得其解的是,继任的戴维斯大使似乎始终穿不破propaganda的金罩。所有的负面
消息,包括成千成千的俄国人、甚至有外国人的失踪,他都没有什么感觉。美国大使有责任向自己国家汇报出使国
的基本状况,他在发回美国的介绍中,只是简单抽象提到苏联的独裁主义,接下来是对苏联建设社会主义的衷心赞
扬,对斯大林及其政治却没有具体批评。看上去戴维斯确实是看好斯大林的苏联,对它持乐观态度。当然,最不可
思议的还是律师出身的大使对``莫斯科二十一人审判''的判断。当时,他手下有个三十岁刚出头的外交官叫查尔斯
\textperiodcentered 伯伦(Charles Bohlen),是个苏联专家,也是个头脑清楚的家伙。1939年苏德瓜分两国之间
中欧地区的秘密协议,就是他最早拿到的。他后来成为美国非常重要的一个外交家。记得2006年美国邮政部还为他
出了纪念邮票。

总的来说,我觉得伯伦不仅脑子清楚,而且务实。比如说,在二战后,他和当时的美国驻苏大使乔治
\textperiodcentered 肯南(George Kennan)是很好的朋友,却不赞同他在东欧围堵斯大林影响的主张,因为当时在
事实上不可能有效,他认为,不管你是否乐见此事,你只能让斯大林苏联有这个东欧影响圈。在肯南大使被斯大林
宣布为``不受欢迎的人''赶掉之后,美国在斯大林后期就没有驻苏大使。斯大林死后,立即恢复正常美苏外交,伯
伦由艾森豪威尔总统提名为驻苏大使,那是1953年,正是麦卡锡调查最热火朝天的时候,麦卡锡本人反对他的任
命,最终参院还是以七十四票赞成、十三票反对的记录,通过了他的任命。这样,伯伦作为驻苏大使再次派往莫斯
科。苏联领导人已经换成了批判斯大林主义的赫鲁晓夫等人,伯伦大使和他们的关系其实处得还不错。

伯伦曾经在回忆中评论过戴维斯大使在莫斯科的工作,他写道,``直到现在,想起那些拍回外交部的有关(莫斯科)
审判的电报,我都还觉得脸红……我只能猜测他(戴维斯大使)写报告的动机,他热心渴望促成亲苏路线的成功,可
能是反射了一些罗斯福的顾问们的看法,他们希望提高罗斯福在美国的政治声望。''

好了。深入下去,问题来了。它表现在两个层面:一是极权propaganda是否可以任意在美国推动,这涉及民众层面。
二是,冷战当前,极权国家组织对美国政府到底有怎样的影响甚至渗透。

第二个问题才是把麦卡锡参议员推上历史舞台的原因。

\pagebreak
\section{不反感才会误导}

众议员非美活动调查委员会(HUAC)在深入调查好莱坞亲苏片之后,影片有极权制度宣传,所谓propaganda,应该是
可以确认的。那么,如何看待和处理它呢?第一个问题就是:是否容许?

为亲苏电影《俄国之歌》作证的俄裔作家兰德阐述了一些理由,她认为社会不应该容忍。对以谎言引导政治立场的
propaganda作出定义之后,她说,《俄国之歌》结尾时,借剧中苏联人对美国主角说,你可以回自己国家,用语言
和音乐,把这里的真相告诉他们。也就是电影最终要灌输给美国人:你们在电影里看到的就是苏联真相。兰德说,她
相信制片人的话,他们并非要有意制作一部苏共宣传片,也看到他们剪掉了一些很过分的镜头,可事实上,在苏联
发生许多可怕事情的同时,电影却告诉大家,在那个制度下人民很幸福。兰德说,假如谁有疑问,只需回答一个问
题:设想一下,假如你明知纳粹德国的生活状况、明知犹太人灭绝营,是不是可以同时接受发生在纳粹德国浪漫恋爱
小故事的电影,在电影里演奏一点华尔兹,说人民生活很幸福。假如你无法接受,那么有关苏联的虚假电影就也不
能被接受,道理是一样的。那是二战刚刚结束的1947年,战争创伤犹在眼前,要是她的听众是中国人,大概就要举
日本的例子了。

对亲苏电影在感情上反应如此强烈的,主要是俄国移民。他们对苏联有切肤之痛。可是,一般美国人对那一段苏联
状况缺乏了解,看电影就是看电影,他们可能就没什么反感,而HUAC之所以要调查,正在于美国人的``不反感'',
也就是在电影手段精心编织的谎言下,可能就信了。国会认为,关键不是发生了某个民众感情无法接受的、甚至是
厌恶的``言论表达'',而是提供虚假信息的propaganda,会不会在民众赏心悦目之间就误导了他们的政治立场,起
到煽动崇拜极权制度的作用。所以,关键是社会风险度:是否有人先利用民主制度提供的言论自由,成功误导多数
人,再利用民主制度提供的合法选举,把美国``非美''成极权国家。假如有人怀疑这一点,HUAC当然可以说:看看
1933年前的德国,别告诉我这``绝对''没有可能。

说起钻制度漏洞,我不由想起以前看茨威格自传《昨日的世界,一个欧洲人的回忆录》,留下最深印象的是他描写
的一个细节:为保护作为精神圣殿的大学,奥地利早就进化出一条文明规则,就是警察绝对不得进入大学校门。结
果,纳粹在奥地利开始兴起之后,茨威格看到法西斯分子在维也纳的大学校园内殴打学生,警察却站在校门之外袖
手旁观。去年一个雨天的傍晚,我漫无目标地在维也纳街头闲逛,湿漉漉铺着小石块的街景和歌剧院,与茨威格年
轻时代一模一样,你会感觉文明是连续的,无论如何想象不出中间的一段断裂。这突然插入的群体疯狂就是有人误
导民众的结果。

\pagebreak
\section{言论的风险评估}

所以,HUAC的这次调查其实涉及对煽动性言论的社会风险评估。

前面提到了1919年由霍姆斯大法官提出的、是否带来``清楚与现实危险''的言论保护测定原则,今天大家都认为,
这个原则正面保障了言论自由。判断没有错,因为今天能够被美国法律划入``危险''的言论微乎其微。所以,就是
在美国,也很少有人注意到,这个``测定原则''同时反映了在某种历史氛围下,某些言论可能引发``现实危险''。
在提出这个原则的``斯康克对美国''一案的判词中,斯康克恰是没有通过``危险测定''而在最高法院九比零判决下
输了官司的。换句话说,这个著名的``保护言论自由''的里程碑判词,在当时是用来限制某个危险言论。

尤其是世界上还存在极权社会的今天,人们普遍关注保护言论自由,而往往忽略另一面,也就是言论代价、言论社
会风险评估在不同历史阶段的演变。在许多历史书中,一般倾向于对HUAC调查持完全否定的看法。在一个又一个社
会危机过去之后,人们很难回过头来重新体验那份真实存在过的历史焦虑。但是,在实际生活中,这是不可忽略的
两面,危险并不会因为忽略而消失,只会由于忽略而爆发原来可以避免的社会危险。所以深入探讨另一面,也是非
常有意思的话题。各个不同社会、在不同时期,如何面对不同的,或者类似的风险。二战后民主德国禁止纳粹宣传
是一个例子。另一个很有意思的例子是1965年由意大利导演Gillo Pontecorvo拍的一部电影《阿尔及尔之役》(The
Battle of Algiers)。它记述了阿尔及利亚独立过程,尤其是它冷峻中立的观察角度,堪称完美。正由于它强烈的
表现力,虽然获得金狮奖,却在电影所记录的冲突一方,也就是在拥有民主制度的法国,长期遭到禁演。也同样因
为它的反殖民冲击力,同时也在香港被禁演。这些地区权衡之后,认为他们冒不起这个风险。另外,上世纪末东欧
发生多米诺骨牌般的翻牌,事后各国共产党都顺利转为民主制度下参与竞争的合法在野党,包括前苏联的俄国。但
是罗马尼亚的转变过程却出现恶性冲突,原国家最高领导齐奥塞斯库夫妇被匆匆枪决,它引发的社会张力,使得在
东欧这一批转型国家中,罗马尼亚共产党和它的宣传是唯一遭到官方长期禁止的。

\myphoto{image021.jpg}{电影《阿尔及尔之役》在法国长期遭到禁演}

\pagebreak
\section{``言论作自由市场竞争''的风险}

霍姆斯大法官在1919年同时提出``让言论作自由市场竞争''的理论。和他的言论``现实危险''测定原则一样,在美
国也越来越得到正面评价,它基于一个前提,就是今天美国社会和民众的成熟。换句话说,同样的极权
propaganda,在1920年的美国有很大社会危险,对今日美国民众不会有任何影响。那么,在1947年,在1950年呢?
这是HUAC需要举行听证会、进行评估的原因。

说起言论的自由市场竞争,我想起一段上世纪初的中国故事。上世纪二十年代初,我父亲曾在上海浦东中学读书。
那是上海最早的完全中学,建于1907年,黄炎培任第一任校长。在当时的中国有``北南开,南浦东''的声名,听父
亲说,当时学校就已经有了``言论自由市场竞争''的概念,那时每个周六下午有全校集会,本意是让中学生听关于
孙中山思想之类的正统政治报告,可学校当局很开明,认为应该让当时各种活跃的思潮都进入学校作``言论的市场
竞争''。所以,学校轮换请来不同名人,这一周是无政府主义思想家,下周可能就是共产党,当然也有右翼和``正
统''。让大家自己``明辨是非''。可以料想,当时的俄国风潮、中国局势和十几岁中学生的激情单纯相结合,必然
是左翼思潮大获全胜。学校最终培养出一大批激进青年。去莫斯科参加共青团的蒋经国,就是这些学生中的一个。

国民政府退至台湾之后,至少在某种程度上,把失败的原因归于激进思潮的推动,因此,同一个国民政府,它的言
论尺度在台湾骤然收紧,几乎禁绝左翼书籍,使得在两岸同样自认民主派的双方,在终于有机会相遇的时候,才发
现他们之间可能有非常遥远的距离。大陆寻求民主一族,在改革开放后追寻西方自由民主思想,因为这曾是他们长
期无法得到的阅读和思想资源。而台湾寻求民主一族,在开禁后急于阅读马恩列斯等红色读物,这也是《毛主席语
录》小红书去年在台湾热销的背景。在最初的对立差异冲撞之后,多数人还是会有相当程度的趋同。这种趋同显示
了``言论的自由市场竞争''在现代成熟社会的潜力。

但是,并非这种风险在今天就不再存在,发展不平衡的世界仍然有不成熟地区。而在高速全球化、新闻传播技术手
段千百倍增强的时候,一个地区的言论可能引发全球范围意想不到的风险。例如2005年有反战人士伪造了关塔那摩
监狱亵渎《可兰经》的新闻,在美国《新闻周刊》刊发,引出全世界大规模抗议风潮至少十七人死亡,导致伊斯兰
世界反美仇恨对抗激化,它可能带来的局势危险度,难以估量。这些也都是新时代面临新问题的后话了。

\pagebreak
\section{对HUAC和国会的宪法约束}

国会焦虑的一个背景,就是美国政府没有宣传部,本身并不掌握任何媒体、娱乐、文化机构,也没有政府可以掌握
操纵的电视台、电台、电影公司、报纸杂志等等,哪怕真看到某个宣传有煽动的危险,也不能说下个行政命令,要
某电影公司现在按照政府口味,拍部主旋律片子什么的,作一个``反宣传''。这就是HUAC调查出来小罗斯福政府居
然暗中``指使''好莱坞拍了个亲苏电影,在美国成为重大政治事件的原因。

HUAC在1947年的风险评估是一个前后时期的衡量。原来因莫斯科``二十一人审判'',苏共信誉在美国锐减,却又因
美苏二战中的盟军关系,给打了一针强心针,在美国声望大涨。它们的正面形象又得到小罗斯福政府的大力推动,
使得美国的共产党员暴增到五万,达到一个新高峰。然而在战争结束的1945年,由于一批苏联间谍通过美共运作的
事实暴露,冷战局势日益清楚,又使得民众对苏联、对共产主义的热情急遽降温。HUAC调查是在这个高涨又回落的
动荡中,评估前后落差的强弱对比。后来促成中美建交的尼克松,当时只有三十四岁,很快成为HUAC主席的得力助
手。

然而,风险评估又若何?HUAC只是美国众院的一个机构,HUAC调查不是司法过程,它对调查对象没有任何起诉、干预、
约束的权力。不要说HUAC,就是国会也无权对电影公司说:评估下来这是煽动,有社会风险,你这电影要禁。因为这
种做法在美国是受到宪法彻底禁止的:宪法规定国会``不得立法''禁止言论自由。

同时,它不会引发任何司法动作。这时就看出``现实危险''测定原则对言论的保护功能来了。这个原则后来是越来
越严,言论非要有引发``迫在眉睫''的危险,法院才可能下令禁止。要禁一部电影?门也没有。电影怎么可能对社会
有``迫在眉睫''的危险?美国在这些方面特别和自己较劲。所以假如要回答一开始的问题:是否容许这样的煽动?答案
是:不管愿意不愿意、高兴不高兴,都必须容许。美国人更害怕的是无约束的权力导致法治倒退。既然如此,还要听
证会干什么?它的作用只是国会作为民众代表,搞清楚某件事``到底是什么状况''。而国会听证这种形式,其实是一
个延续至今的传统,它把民众最关心、最担心的问题,以听证会的方式公之于众。尤其是现在电视普及之后,任何
关心某个议题的人,都可以追踪听证会,由议员们替大家传唤证人和提问,所以这是民众间接参与政治的一个过程。

可是,就在HUAC的1947年调查过程中,半路杀出``好莱坞十好汉''(Hollywood Ten),在听证会上拒绝回答某些问题、
大有抵制HUAC调查的趋势。结果,导致事情的发展急转直下,走向了一个谁也没有料到的方向。

\pagebreak
\section{``好莱坞十好汉''}

众议院非美活动调查委员会(HUAC)对好莱坞亲苏片的调查中,总共传唤过四十一个为好莱坞工作的证人。结果,在
这四十一个证人中,有十人拒绝回答问题,被大家称为``好莱坞十好汉''。这个译名我好像是加了一点``褒义'',
准确地说,英语``Hollywood Ten''是一个中性词。可是,看今天人们解读这段历史,总是给他们赋予一些英雄主义
的悲剧色彩。

十人中有七个剧作家、两名导演、一个制片人,在不同时期参加过美国共产党,都较深地卷入左翼政治。例如阿瓦
\textperiodcentered 培西(Alvah Cecil Bessie),哥伦比亚大学毕业,1938年就响应共产国际号召,加入国际纵
队的美国林肯支队,是西班牙内战战场上的幸存者,回来就写了一本书《战斗的人们》(Men in Battle)。他们都才
华横溢,如剧作家阿尔伯特\textperiodcentered 马茨(Albert Maltz),在阿瓦\textperiodcentered 培西去西班
牙打仗那年,他的短篇小说获得了当年的欧\textperiodcentered 亨利奖。还有左翼剧作家拉德那(Ringgold
Wilmer Lardner Jr.),也在西班牙战争期间积极帮助筹集经费,组织游行。虽然老板对他卷入政治过深比较烦,却
也欣赏他的才华,在当时美国一般雇员每周只有几美元工资的年代,他为二十世纪福克斯公司写剧本,已经可以拿
到每周两千美元的报酬了。在他们中间,更有好几位在1947年调查之前或者之后,获得奥斯卡金像奖的提名。达顿
\textperiodcentered 特鲁伯(Dalton Trumbo)是他们中间比较典型、也是最出名的一个。

达顿\textperiodcentered 特鲁伯在读高中时,就已经是跑法庭的校园小记者。他在科罗拉多大学上了两年学,就
又开始当大学校园报刊的记者。毕业后在记者职位上开始发表小说,也开始关注现实题材。照现在时髦说法,他在
1937年``触电'',作为剧作者进入电影界。在上世纪四十年代,他已经是好莱坞报酬最高的合约剧作者之一了。他
1940年写的电影Kitty Foyle,使他被提名奥斯卡剧作奖。1939年,他以反战题材的小说《强尼拿到了他的枪》获得
美国国家图书奖。他很早就开始同情和支持美国共产党,也深受影响,最终在1943年入党。

在第二次世界大战爆发之后,由于苏德密约,战争初期苏联在德国一方,在1939年二战初期美国关于参战的争论
中,美共也就坚决反对美英结盟和德国打仗,因为这等于站到了苏联对面、与苏联为敌了。达顿
\textperiodcentered 特鲁伯也就在这样的大形势下,以文学表达自己的政治立场,写了一本《非凡的安德鲁》
(The Remarkable Andrew)。借美国前总统安德鲁\textperiodcentered 杰克逊的幽灵之口,警告美国不得参加二战。

\pagebreak
\section{是否可以藐视美国政府}

当时《时代》周刊有篇书评,说是这些年美国共产党的作品反正是已经把开国总统华盛顿和林肯总统都拉进去了,
再加个杰克逊将军也真算不了什么。达顿\textperiodcentered 特鲁伯不顾书评的冷嘲热讽,马上把小说改编为剧
本,他不愧是杰出作家,不论小说还是影片,都十分动人。五十多年前的电影,直到今天观众还在给它打满分的五
颗星。可是特鲁伯真是运气不好,小说是1940年出版的,1941年德国就对苏联翻脸发动进攻。电影周期更是慢一
拍,待到上映已是1942年,也就是说,大局已翻转,特鲁伯忽然发现,不论是苏联立场还是美国立场,他站的立场
都已经``政治不正确''了。

之所以今天十好汉有悲壮意味,是因为他们为自己在HUAC的调查中,个人都付出了巨大代价\myrule 先就坐了牢。
不了解美国的制度设置,想当然会认为``好莱坞十好汉''坐牢,是政府的迫害,实际上却不是这样的逻辑。

这十好汉都是好莱坞高手,不是等闲之辈。所以在HUAC调查中,他们有自己比较独立的思考和态度,例如,他们觉
得电影只是个言论表达,调查是干涉言论自由。这个看法越到后来,越有人持这样看法,因为事件的背景消失了。
把事情从时代中抽象剥离出来,就很难理解。例如在上世纪九十年代,达顿\textperiodcentered 特鲁伯的母校科
罗拉多大学,就把学校中心喷泉命名为``特鲁伯自由言论喷泉'',以纪念这位名人校友。

就具体做法来说,十好汉认为,他们可以引用美国宪法的权利法案第五条,就是不能强迫被告``自证其罪''的条
款,他们有权不回答问题,也就真的在听证会上拒绝回答提问了。

这其实涉及一些纠葛在一起的概念。首先,大家都知道,美国作为立法机构的国会、司法的法庭,以及行政的白宫
一摊,是独立并行的三大分支。但是有些规则就不是大家所熟悉的了。比如说,我们常常说,美国人可以随便骂政
府,好像可以不把它放在眼里。可是,在这么说的时候,我们多半在潜意识里,又把三大分支混在一起了。实际
上,最可以随意骂的,是总统、白宫、内阁等行政分支这一路,假如总统出来面见民众,有人在人群中当面骂了总
统,甚至扔皮鞋之类,照美国法律,最多是扰乱公共秩序;或者总统接见你,你毫不领情还当面骂了他,都没有法
律上特定的``藐视总统罪''。可是在法庭上你就不能辱骂法官。在美国并不是政府的任一分支你都可以``藐
视''(contempt),``藐视''是一个法律概念,明确规定范围。美国虽然没有``藐视总统罪'',但是确确实实有``藐
视法庭罪''和``藐视国会罪''。

\pagebreak
\section{十好汉触动了一个古老开关}

法治国家的法庭大概是最尊严的地方,阻碍法庭的司法职责,故意贬低、损害法庭和法官的权威和尊严,干扰其执
行权威和职责,这样的行为是属刑事罪的藐视法庭罪;若在庭外拒不执行法庭判决,则是属于民事罪的藐视法庭罪。
例如你付不起房租法庭判限期搬离,你不搬的话,算民事范围的藐视法庭罪。

我们的``十好汉''眼前的一个关键是,对司法、立法的两个``藐视罪''的行为范畴是不一样的,因为这两大分支的
性质太不相同。

在法庭的刑事审理中,被告或者证人可以有引用宪法权利法案``第五条''的权利,只消说``引用第五条'',就可以
合法地拒绝回答问题了。宪法如此设置,是为了在刑事案中保护被告的权利。因为刑事案件的起诉人是政府行政分
支司法部的检察官,判决涉及被告的自由或者生命,因此,宪法给予相对政府处于弱势的被告,一些保护自己的特
权。可是,国会不是刑事法庭,国会各个委员会的听证会,是在代表国民们向你做某些情况的调查,国会也并不对
你的作证内容定罪。所以,作为公民,你有义务和责任回答问题,无权拒绝回答。所以,美国法律对``藐视国会
罪''(contempt of Congress)的界定是,故意阻碍国会执行职权和权力,界定的最主要行为就是``证人拒绝回答国
会所属委员会的问题''。而且,藐视国会是刑事罪,可以被判监禁。据此,联邦上诉法庭驳回了十好汉的上诉,他
们就必须坐牢了。这么一来,真是很悲壮了,虽然国会藐视法庭罪的刑期都不长。这十个人中只有一个坐了一年
牢,其余都是几个月,例如阿瓦\textperiodcentered 培西是坐了十个月的牢。

所以,就``藐视国会罪必须坐牢''这件事本身来说,只是一个既定程序,它的起源是英国的``藐视议会罪''。它和
美国的各种制度设置一样,早早地,在两百多年前,就开始一步步设定好了。你应该知道那是违法的,千万别去触
碰古老的开关;你一旦碰了,程序就开始启动,就会自动往前走,谁也救不了你。但是,这个判罪有一个特点,就
是别人救不了你,你自己却可以救自己:任何时候你决定按照国会要求作证,刑期马上中止。``十好汉''中有个电影
导演,爱德华\textperiodcentered 德米奇克(Edward Dmytryk),他是很小就随乌克兰裔父母移民美国,在美国长
大,可是直到二战爆发的1939年,他才刚刚入美国国籍。可他在成为美国人之前,就已经是美国共产党党员了。就
在德米奇克被HUAC传到听证会的1947年,他执导的电影Crossfire获得奥斯卡最佳影片奖。他也拒绝与HUAC合作,因
藐视国会罪被判入狱。但是德米奇克在坐牢七个月后,决定回到听证会作证,也就中止刑期出狱了。

\myphoto{image024.jpg}{``好莱坞十好汉''与他们的律师在一起}

\pagebreak
\section{意外的形势急转直下}

按说,即便如此,本来整个事件也就涉及十个人的短期监禁,不会有更大的影响面。这样的处罚在既定程序中,并
不是什么了不起的事情,更谈不上是政治迫害事件。这十个人在自己的专业领域都出类拔萃,只不过就是中间离开
几个月、至多一年,回到好莱坞,他们应该还是十条好汉。

可是,谁也没有想到,就在他们因藐视国会被传讯的第二天,1947年11月24日,他们的老板们,包括美国电影协会
(MPAA)的美国电影工业的三大专业组织,也就是好莱坞各主要电影公司的四十八个CEO和制片人,一起汇集在纽约,
在华道夫\myrule 阿斯朵利亚(Waldorf-Astoria Hotel)开了几天会,作出了一个决定。正是这个决定,推动了整个
事件意外地急转直下。

他们决定解雇这十个受到起诉的雇员。解雇他们的理由,并不仅仅是因为这十个人对抗听证会,而是他们作出了一
个很不寻常的进一步决定:我们好莱坞将不再雇用任何颠覆集团的成员。1947年12月3日,美国电影协会主席艾略克
\textperiodcentered 约翰斯顿(Eric Johnston)代表这个会议发表了一项声明,这就是后来非常有名的``华道夫声
明''。

事实上,这些电影工业的掌门人在HUAC一开始调查的时候,是支持包括``十好汉''在内的听证会证人的。他们宣称
听证会调查美国共产党组织对美国电影工业的影响,只不过是子虚乌有的政治抹黑。所以,也有历史学家认为,这
个声明是他们立场的转变,而这么大的转变,是来自于公众主流观点的压力。可是,我们假如仔细研究他们的声
明,会发现并非如此。

``华道夫声明''的关键条款是:``我们将不会明知故犯地雇用共产党员或者任何其他组织或者政党的成员,他们提倡
以武力或其他非法的或者违宪的手段,来推翻美国政府。'' 而解雇这``十好汉''的依据是``我们将立即停止或者说
悬置对这十人的雇用,直到他们洗刷藐视罪,并且在誓言之下宣称他们不是共产党员''。

必须先说明的是,美国电影协会这样的组织,是和政府没有任何瓜葛的民间行会。从这个声明其实可以看出,他们
的态度其实始终和HUAC是一致的,也就是说,他们一开始支持自己的雇员,是因为在调查开始的时候,他们并不认
为好莱坞,也就是美国电影工业,受到了共产党的渗透和影响,所以才宣称这样的说法是对好莱坞的``政治抹黑''。
可是随着调查的推进,他们已经承认这是一个事实。他们认同美国民主制度通过选举轮换执政的制度内运作方式,
反对以制度外的``非美手段'',也就是``武力或其他非法的或者违宪手段,来推翻美国政府''。他们作为私营企业
的雇主,决定以选择雇员的方式,来抵制有``非美手段'',即对民主制度使用非法颠覆性手段的政党,对事实上可
能被利用作为政治宣传的一个特殊工业的渗透。

显然,这么一来,就不是局限于``十好汉''的单一事件了,它会引出一大批好莱坞雇员的雇用问题。开出这张解雇
名单,就是著名的麦卡锡时代``好莱坞黑名单''事件。当时,谁也没有看透,一个看上去好像很有逻辑的民间介
入,却使得``罪与非罪''的界限,被突然模糊了。

\pagebreak
\section{华道夫声明}

所谓``好莱坞黑名单''事件,虽然起于麦卡锡进入公众视线之前,却几乎是麦卡锡时代历史叙述和回顾的主体部分
之一。原因是它在今日历史叙述中,是一场针对普通平民的政治迫害。记得有朋友说起一个真实故事,说是一个中
国人对一个美国人解释文化大革命,可怎么解释对方都理解不了,最后就说,这就是放大了一千倍的麦卡锡时代,
对方似乎就明白了。可想而知,麦卡锡在不论中国人还是美国人心中是个什么样的概念。那四十八名签署``华道夫
声明''的好莱坞CEO和业主,现在也就基本上是迫害者,或者说是麦卡锡时代歇斯底里的象征。可是,仔细去探究历
史,你会发现麦卡锡和``文革'',其实是风马牛不相及的两回事。中美两个不同历史事件,假如要吸取教训的话,
完全是性质、来源不同的教训。

``华道夫声明''很特别,在于宣称好莱坞三大专业行会不再雇用``提倡以武力或其他非法或违宪手段推翻美国制
度''的组织成员。可是,在这样声明的同时,他们又在思考这个行为的合法性,所以,声明也表示:``国会对于私人
企业雇用共产党组织成员没有建立任何国家立法,这个缺失使得我们的行动变得格外困难。我们国家是一个法治国
家。我们要求国会立法帮助美国企业摆脱搞颠覆和背叛国家的人。''

与此同时,他们也在担心会给好莱坞这样一个文化事业带来伤害和负面效果,``华道夫声明''一面表示决心:``在执
行这项决定的时候,我们不打算对来自任何方向的歇斯底里和胁迫让步。''一面预先对自己作出警告:``我们坦率承
认,这样的决定有着危险和冒险的成分:有存在伤害无辜者的危险;冒着营造惧怕气氛的风险。作为创造性工作,不
应有任何惧怕顾虑的气氛。我们会预先提防伤害无辜的危险,预防这样的风险,预防这种惧怕。''

他们认为,通过努力可以做到两害相权取其轻:``我们将邀请好莱坞有才能的行会和我们一起工作,去除颠覆者、保
护无辜者,在任何受到威胁的地方保护自由言论和自由银幕。''

从好莱坞雇主们以往给这``十好汉''的高薪和他们的成就,可以看到,从盈利和事业发展的角度,雇主是希望和他
们继续合作的。细看``华道夫声明'',可以看到那个时代特殊的焦虑和困扰,而这种困扰来自冷战威胁的特殊紧
张,以及美国制度事实无法高效应对这种威胁的事实。于是,就出现了官方、民间的折中应对方案。

\pagebreak
\section{扑朔迷离的局面}

``好莱坞十好汉''的藐视国会罪,在众院是以三百四十六票比十七票通过的,但是他们一路上诉,也曾经上诉到联
邦最高法院,但最高法院并没有接这个案子。不接的案子是无需说明理由的,但是一般都认为不接很正常。因为最
高法院通常并不是接``轰动有名''的案子,而是接有极大争议的案子,就``藐视国会''来说,案情简单、法理上逻
辑清楚,论``判断''是很简单的案子。可是再三上诉,一拖,案子也就过了将近三年。穷尽了司法程序,十好汉只
能坐牢去了。那已经是1950年,也就在那一年,麦卡锡参议员出场了,是麦卡锡把``麦卡锡时代''推向了最高潮。

事情发生在1950年不算偶然,那是个很自然的内外局势推动过程。1949年中国大陆的易帜,也是外部局势刺激之
一,就是原先作为亲密盟友的中国绝大部分,由于国内革命,突然与美国拉开距离,加入了苏东阵营,苏联阵营在
迅速扩大。加上1950年的朝鲜战争,更使得这些变化显得目不暇接、触目惊心。这场战争虽然发生在亚洲,对美国
人来说,它的概念相当于东德突然进攻西德而且差点把它给灭了,也象征着两大阵营被拖入大战的现实可能。这时
距二战结束刚刚五年,对那些二战幸存下来的士兵和他们的家庭来说,和平的日子还没有过上多久,感觉几乎就是
二战的继续。冷战顿时不再是说说而已的对抗。间谍战也就更让人看到背后的杀伤力。

前面介绍过,1945年11月,为苏联情报部门工作的美国女间谍领导人伊丽莎白\textperiodcentered 柏特丽自首,
揭露了前美国共产党领袖布洛德从三十年代开始,就为苏联在美国建立情报网,并且持续下来的事实。而柏特丽交
出的一百五十人间谍名单中,有三十七名联邦政府的雇员,在法庭上,他们又大多引用刑事案被告可以``不自证其
罪''的宪法条款,拒绝回答问题。按照美国严格的司法程序,间谍除非在交接情报时当场抓获,很难定罪。前面提
到的畅销书《出使莫斯科》以及电影的作者,美国驻苏大使戴维斯也是一例。在众议员非美活动调查委员会(HUAC)
公布调查结果后,人们显然要问,这位大使究竟怎么回事,是糊涂透顶被苏联的宣传灌迷糊了呢,还是装着糊涂、
实际上被苏联收为``自己人''了呢?那么,力推他的亲苏作品的小罗斯福政府呢?力推斯大林政府高层领导、竭力要
推动战后苏美密切合作的政府战争信息办公室呢?这些问题搅得大家人心惶惶。

人心惶惶的最重要原因,是大家都明白,现在已经进入核武器时代,时代不同了。

所以,间谍也不是以前的概念了,而是直接牵涉到可能把毁灭性武器的秘密交到对方手中。这种紧张在1949年达到
高潮,苏联在那一年第一次成功进行了核试验。虽然所有人都知道,大家早早晚晚都会折腾出核武器来,可西方阵
营自然希望这样的威胁尽量晚一些到来。

\pagebreak
\section{历史在推出麦卡锡}

苏联在1949年核爆,比各国专家的估计日期远远超前。也就是在1950年,曾在美国核计划工作的德国科学家克劳斯
\textperiodcentered 福柯,供认自己是苏联间谍,并且指认了一些为苏联工作的重要美国间谍,其中就有戴维
\textperiodcentered 格林格拉斯(David Greenglass)。格林格拉斯参与了美国最早的研制核武器的曼哈顿计划,
他的姐夫和姐姐就是美国最著名的间谍卢森堡夫妇。格林格拉斯和姐姐埃塞尔\textperiodcentered 卢森堡(Ethel
Rosenberg),都是裘利\textperiodcentered 卢森堡(Julius Rosenberg)发展的一个间谍网的成员。

卢森堡夫妇是美国历史上平民以间谍罪被判死刑的唯一例子。在当时引起极大震动。虽然在赫鲁晓夫回忆录中,提
到曾经听斯大林和莫洛托夫说过,苏联从卢森堡夫妇间谍网得到极有价值的核武器资料,但这个案子始终被不断质
疑。直到去年,2008年9月11日,与卢森堡夫妇同案的莫顿\textperiodcentered 索贝尔(Morton Sobell)在九十一
岁高龄,终于在多年否认罪名之后,承认了自己是苏联间谍,同时也确认裘利\textperiodcentered 卢森堡``合谋
参与向苏联递送机密军事工业情报,也包括核武器情报''。虽然他还是为自己辩解,说他们传送的这些核武器资料
对苏联已经价值不高,因为他认为苏联已经从其他美国间谍那里得到了类似资料。卢森堡夫妇留下的两个孩子,多
年来一直在努力试着为父母洗刷罪名,在索贝尔去年的声明之后,放弃了这种努力。但是他们也认为,母亲并没有
像父亲那样,卷入间谍活动如此之深。而且不管怎么说,对卢森堡夫妇二人,死刑都是过重了。

我们现在已经很难体会当时的人对美苏核战争的恐惧了。对冷战中的美国人来说,二战前后世界有了本质差别,虽
说两大阵营主要是``冷对抗''而不是``热交战'',可是,局部冲突不断,军备竞赛的等级规模史无前例。万一有个
闪失打起来,大概只能是``一毁俱毁''。这可能也是当时艾森豪威尔总统拒绝特赦卢森堡夫妇的原因,他恰好是从
二战战场上刚下来不久的将军。

就在1950年的历史背景下,2月9日,参议员麦卡锡在西弗吉尼亚的一个饭店发表讲话,他认为冷战的对手利用美国
共产党,对美国外交部进行了渗透。他宣布说,自己手上掌握了五十七个官员的个例,他们或者是美国共产党员,
或者是亲共的人,而他们仍然在参与制定我们的外交政策。

在当时的局势下,麦卡锡的讲话无疑是扔下了一颗重磅炸弹。因此引发了美国国会参议院组成委员会,展开对共产
党渗透政府状况的听证会调查。

\myphoto{image026.jpg}{民众在美国白宫门前声援被判死刑的卢森堡夫妇}

\pagebreak
\section{罪与非罪}

回头看,美国只是再次置身于一个危局之中,在仓皇应对。它置身一个史无前例的危机:刚刚经历二战,使它看到世
界大战可能达到怎样的破坏;核武器刚刚诞生,使它看到世界可能因大战毁灭;军备竞赛成为应对冷战的重要步
骤,间谍战成为军备竞赛的关键,而美国严格的司法制度显然没有应对大规模间谍战的能力。怎么办?

所谓麦卡锡主义,实际上是一个对``敌情不明''威胁下的被动应对。它首先要解决的问题是,政府高官、涉及军事
机密的机构,必须忠诚于国家民主制度,而不是准备推翻这个制度的、冷战敌对阵营的间谍或者卧底者。

可是看上去,美国共产党``对外''像是个全球化政党,似乎有时它效忠于自己的国际体系远胜过效忠于自己的国
家;在美国``对内'',它似乎又是个时而制度内时而制度外、部分地上部分地下、捉摸不定的政党。本来,一个要
害部门,不论国家机关还是核研究机构的工作人员,都要签署忠诚国家保证书。美国人认为,你签了字就是认真的
誓言,可是既然现在暴露出来,美国共产党有为冷战敌对阵营组织间谍网的情况,既然都间谍了,当然什么保证都
可能是虚假承诺。那是外部核威胁下、内部由政党组织的间谍战,所以,不知如何是好。

麦卡锡的参院调查委员会的思路是,在美国司法制度无法应付间谍网、给他们定罪的时候,出来一个折中方案,和
司法定罪无关,只是找出所有政府要害部门中的美国共产党,不留他们在政府工作,以缓解间谍战的危机。

绕了两圈以后,一个美国历史上独特的、似乎很难理解的复杂情况,就这样出来了。

假如进入司法程序:间谍是刑事罪,但必须有万分确凿证据证明其犯有间谍行为才能受到刑事惩罚。而仅仅是共产党
员,信仰``反对或推翻美国制度'',不构成任何罪行,法院是不管的。

假如进入麦卡锡主义的实际操作程序:摆脱了司法的严格严谨程序,也摆脱了正常的要害部门要求的效忠宣誓程序,
而是凡有潜在的间谍可能或颠覆美国制度意向的,就给予特定的``工作惩罚''。假如查出来``是'',会失去原来在
政府要害部门的工作。而有``惩罚''就变成``类罪行''了。同时,麦卡锡的参院听证会是和政府工作有关的清查,
原本和民间的企业公司等等没有任何关系。可是,因好莱坞涉及亲苏政治宣传,受到众院调查,本来这并不涉及
``惩罚'',却由于``华道夫声明'',也就在好莱坞这个范围卷入了民间雇主的``雇用惩罚''。

这个局势一旦运转起来,它就有了自己的规律,``华道夫声明''中预料到的恐惧气氛和伤及无辜的事情就开始发
生,试图避免的事情却难以避免。事情开始顺着自己的逻辑不由自主地外延扩大:共产党员?前共产党员?共产党同情
者?参加过共产党的会议?参加过共产党员家里的派对?……

告密(name the names)突然变成听证会上的正当要求。一个焦虑时代的紧急应对,困扰了里里外外所有的人。

\pagebreak
\section{白宫那一头}

前面说的,不论是众院非美活动调查委员会(HUAC)还是麦卡锡的参院调查,都是立法分支的国会系统。实际上,间
谍案爆发使得民间最担心的是美国行政机构被渗透,压力当然首当其冲在行政分支,在迫使他们作出反应,也就是
白宫这一头。

这种压力是通过选举在起作用的。联邦行政机构的工作人员,大量是招工应聘来的。各部部长、外派大使、联邦法
官等高层官员是由总统提名,再到国会的听证会上``受审查'',例如最近奥巴马提名的中国大使,就必须面对国会
外交委员会的听证会,通过电视,当着美国民众的面,让大家知道你对中美关系的看法,打算如何履行使命。行政
分支人员绝大多数不是民选官员。可是,他们的头儿,美国总统,是民选的。他必须对自己的那一摊子负责:假如你
管的行政分支出问题,你就选不上了。

总统很复杂,他是全民的总统,不能以权为自己一党谋利。可是,之所以老百姓选了此党的此人,而没有去选彼党
的彼人,就是在选的一刻,大多数人倾向于他所提出的、有别于对方党的政策。绝大多数民众认同的两党共同核心
价值没有差别,简单说就是都认同宪法和民主制度。所以,对总统选择,没有选择``走社会主义道路还是走资本主
义道路''、``自由民主还是无产阶级专政人民民主专政''这样的大是大非,而只是选择一党所研究出来的具体施政
方略。这是总统选举的政党关联,因此,两次总统选举之间所谓国会中期选举的政党倾斜,就可能是下次总统选举
的一个风向标。1946年,杜鲁门总统所属的民主党在中期选举的参众两院分别大败。参院民主党丢了十二席,共和
党以51:45席占上风;众院大丢五十四席,共和党以246:188席占上风。在新上任的新科参议员中,就有刚刚从二战
战场上下来的海军陆战队麦卡锡上尉。

1946年共和党在国会选举得胜,历史学家认为,除了后来那句名言,``笨蛋,关键还是经济''之外,还有民众在间
谍案暴露之后,对政府如何防止间谍渗透措施不力的疑虑。国会的共和党领袖裘\textperiodcentered 马丁(Joe
Martin)就在竞选当口,誓言要彻底清除政府内的共产党。杜鲁门这个总统本来就当得不``神气'',他不是选总统选
上来的,而是小罗斯福意外死亡,他作为副总统``顺''上来的,``顺''的时候这个副总统才刚刚当了八十二天。有
差不多整四年的时间需要面对执政能力的怀疑。说句题外话,我对杜鲁门的意外提升一直深表同情,副总统几乎就
是个闲差,无需承担重任。现在是总统,一上来就必须作出骇世惊俗的抉择:是扔原子弹马上结束太平洋战争,还是
牺牲估计更多的双方军民、继续打旷日持久的常规战。只要想到这一刻,我就对所有竞选总统的人佩服不已。

正因为杜鲁门作为总统不是选上来的,所以,下面那次选举,才是他真正``证明自己''的那一次。

\pagebreak
\section{一纸行政命令}

中期选举风向标一出来,杜鲁门总统决心做一些事情解决间谍渗透问题。不仅是应对民众压力,局面严峻,也使得
行政分支无法再回避。1946年5月,FBI的头头胡佛拿了一张政府高阶官员名单找上门来,向杜鲁门总统报告``苏联
渗透之大阴谋'',名单头一个就是副国务卿艾奇逊(Dean Acheson)。杜鲁门总统继承了一个政府,也同时承继了小
罗斯福政府亲苏的大众质疑。艾奇逊就是一个很典型的复杂例子。至少在1945年,他被公认是倾向于和斯大林苏联
密切合作的高层官员,直到1946年,他仍然希望和斯大林有个调和关系,据说是两件事情彻底改变了他的看法,一
是清楚看到斯大林要控制东欧和东南亚,另一件事情对他刺激更大,就是发现苏联其实根本无意以正常外交处理国
际关系。他是个老派传统外交家,前面一段立场只是对斯大林认识估计都不足。也许就是这一段,使得FBI对他产生
怀疑。

杜鲁门总统认为胡佛的警告有夸大其词的成分,他还是坚持自己对艾奇逊的信任,在当选总统后的1949年,杜鲁门
升任艾奇逊为国务卿。艾奇逊作为大时代的大国外交家,一生完成大事无数,例如在1941年,他策划了美、英、荷
对日本的禁运,切断了日本百分之九十的燃油供应,给了中国实实在在的支援。历史学家一致认为,艾奇逊非常清
楚这一举动早晚会导致美日战争。他1949年一上任就组织外交部研究中美关系,发表了著名的《中美关系白皮
书》,诚实预言了中国在1949年的前景有它自身的逻辑,美国的影响极为有限。虽然杜鲁门对艾奇逊有信心,但也
并不能因此否认美国政府存在苏联间谍渗透的事实。在1946年,面对FBI名单,杜鲁门也不敢不作任何调查,就指其
为诬陷不实之词。

美国非军事政府机构一向是一派和平景象,过去的概念中,所谓间谍都是和打仗、军事部门有关,现在第一次提出
非军事政府机构安全问题。难怪杜鲁门总统迟迟推不出一个措施来:假如一向有安全门槛,就只需对新雇员有个忠诚
审核程序,相对简单得多。现在是首次提出,就先要面对联邦政府数量庞大的现存雇员,要对他们全部作一个清查。
这可怎么查?

杜鲁门左右为难,1946年底,他终于指定了一个委员会,要求他们研究和提出具体对策。委员会研究结果,最终推
出了历史性的9835行政命令,也被称为忠诚命令(The Loyalty Order)。总统在1947年3月签署此令。在美国,这还
是第一次。

\myphoto{image027.jpg}{杜鲁门(左)一直信任艾奇逊(右)}

\pagebreak
\section{间谍\myrule 安全\myrule 忠诚}

美国从来不搞政治运动,对大规模清查没有任何经验,更惧怕滥用职权、伤及无辜。在这方面,他们一直认为自己
是有历史教训的,那个教训就是1692年发生在英属殖民地的著名``追巫''(Witch Hunt)案。虽然当时还没有美国,
可事情发生在后来属于美国的新英格兰,美国人还是把它当作自己的污点,深深刻印在美国历史记录的账上,也从
此草木皆兵。只要提清查,大家第一个反应就是:会不会``Witch Hunt''?!

按说,行政分支要展开清查最简单,因为FBI就归在总统属下,本系统就有个现成调查机构可用。更何况,胡佛一直
摩拳擦掌,一副只待下令就要冲上去查个清楚的样子。可胡佛的积极性可能反而吓坏了杜鲁门总统。9835行政命令
撇开了FBI,要求在联邦行政机构内另建专门的雇员忠诚审查委员会。这个委员会有来自联邦政府六个部门的代表,
由司法部长主管。必须说明的是,美国司法部属行政分支,和法院系统的司法分支不是一回事。司法部由一大批检
察官和律师组成,司法部长应该是行政分支中最苛严法律、最不可能违法的人。这样,最重要的审查的``程序问
题''由委员会解决,以免FBI追查无度,造成``Witch Hunt''。

所谓``忠诚问题'',其实是由``间谍\myrule 安全\myrule 忠诚''这样一个顺序推下来的。它面对的困扰就是麦卡
锡困扰:疑问是间谍案引发的,定间谍罪是司法职责,司法没有这个能力,政府被渗透问题又要解决,只能变通转化
为行政分支出于安全考量的``忠诚''要求。所以,审查的所谓忠诚,不是``是否忠于总统'',关键是``是否终于宪
法''、有没有违宪颠覆政府的意向。说到底,司法管你是不是间谍;行政审查则希望清除潜在间谍。

所以,在9835行政命令的第五部分对``不忠诚''的定义是:``破坏、间谍、通敌、暴乱,或者拥护鼓吹这些做法;未
经授权故意泄露机密。鼓吹推翻美国政府,参加任何极权主义、法西斯主义、共产主义组织、颠覆组织或者同情支
持这些组织。'' 其实这些都和防止间谍有关,针对眼前急迫的苏共间谍危机,假如没有具体标准和程序控制,这样
泛泛的界定太容易导致清查失控、走向追查无度了。因此,审查委员会最后的做法是``和组织挂钩'',公布了``司
法部长界定之颠覆组织名单'',也就是麦卡锡时代著名的``AGLOSO''。这个名单主要是三大类组织:KKK组织、法西
斯组织以及共产主义阵营组织。因为这些组织自己宣布的宗旨明确就在清查界定的范围。

\pagebreak
\section{忠诚和宪法权利}

任何国家当然都有涉及安全的忠诚问题。美国作为大规模移民国家,更为特殊。移民和土生土长的国民有很大差别。
移民会有忠诚困扰,一般情况下可能感觉不到,可两国开战怎么办?一次大战的大量德裔、二次大战的大量日裔,所
谓战时敌国侨民,就产生间谍隐患困扰。即便和平时期,作为移民蜂拥而至的国家,还是有国外``颠覆''``间谍''
渗透的安全考量。所以,美国在移民入藉时,至今还要求在誓言下答复安全提问:你是否加入过纳粹党?你是否加入
过共产党?这些组织和美国制度对立的理论目标从未宣称改变,美国人当然也认为它们都是认真的、打算说到做到,
不敢不认真对待。归化的移民被要求有个``忠诚誓言''。这样两个提问的所谓安全审查,和9835行政命令颁布以
后,新雇员进入一般联邦机构的程序差不多。在外人看来,似乎对一般人无意义,对间谍根本没用。这种看上去很
书生气的程序,和它的宗教和传统文化有关,如法庭誓言。

美国在政府的非军事机构第一次引入``忠诚''概念,实际上是国家、政府从一个很松弛状态,突然由于敌方发起间
谍战而开始应对,也开始建立自己的现代反间谍系统。现在联邦雇员已经发展出一整套安全审查系统,克林顿总统
1995年曾发出12968行政命令,主要是接触保密资料雇员的安全审查规范。

正因为9835行政命令是``第一次'',从一开始就引出一系列思考、质疑和公开讨论。要通俗解释这个忠诚要求,好
像也在常理之内。这么说吧,一个老板雇人,要求雇员不偷商业情报出去,不阴谋推翻老板,不是专事推翻本公司
为宗旨的组织成员,应该算说得过去;有上述问题,说本公司不敢留你,另谋高就,好像也不算过分。可是,现在
这个老板是联邦政府,问题就不那么简单。质疑主要涉及宪法和法律。大家马上要问,9835行政命令对联邦雇员的
忠诚要求,其实是缩减了他们的部分公民权利,主要涉及言论自由、结社自由。此后,``忠诚审查''的做法不断面
对司法挑战,也就由一系列法庭判决来逐步规范相关概念和做法。后来著名的``国旗能不能烧''之类的司法界定,
起诉都归属在``忠诚案''名下。

杜鲁门总统把审查交给司法部长主持,只容许联邦调查局(FBI)查看联邦雇员的名单,FBI只能对一些已经发现有
``负面资料''的人,展开全面背景调查。可是,FBI掌握的资料必须交给委员会处理;不让敌情观念过重的FBI主持
审查程序。现在回头去看,这些措施还是有利于清查不无端扩大。在1948年至1958年的十年中,这个``忠诚审查计
划''对四百五十万联邦雇员展开了调查,以安全理由,十年来总共解雇了被认为符合界定的三百七十八人。一万个
雇员中解雇了不到一人,也没有引发联邦雇员的普遍不安和恐慌。作为一个危机应对,能做到这样并不容易。

杜鲁门在忐忑不安地、很不情愿发出9835行政命令的时候,他并不知道,军方有一个在二战后期偶然开始的庞大破
译计划,正在进行之中,而杜鲁门一辈子都被瞒在鼓里。假如他知道这些破译内容,他签署9835行政命令的时候,
可能就稍微心安理得一点了。

\pagebreak
\section{当然,政府是有秘密的}

什么是机密,什么不是机密,怎么对待机密的时效问题?在一些国家,这似乎是政府事务,民众不容置疑。其实它可
能对一个公民是非常现实而性命攸关的。比如中国``文革''时期有草木皆兵倾向,政府下发的样样红头文件都加保
密或机密二字。很少有人想过,现在四十年过去,研究这段历史的学者可能仍然踩在一个陷阱边缘。虽然即便在当
时,泄露这些文件也未见得会危害国家,更不要说时过境迁的今天了;可既然没有解密,那么在没人较真的时候,
历史学者也都在阅读、复印、拥有和引用这些历史文件,很少有人想到,假如哪天突然要认真了,你可能就掉下
``未解密''的陷阱去了。我还真见过为这点事掉下去的。

现在大家都很熟悉美国在1976年通过的有关公开政府会议的《政府阳光法案》(Government in the Sunshine
Act),更早的是1966年约翰逊总统签署的《自由信息法》(Freedom of Information Act,简称FOIA)。顺便说一
句,这不是美国的发明,此前两百年的1766年的瑞典《新闻自由法》中,已经有了类似的精神和立法遗产。美国这
两个强调政府透明公开的法案,同时也都界定了``豁免公开'',也就是保密的范围,其中最主要的是``已被法律规
定为保密的事项''。对保密的界定涉及公众的现实知情权,对``阳光照不透''的保密历史文件如何处理?就牵涉历史
学者的研究权利和公众对历史的知情权。

1994年克林顿政府指定了一个``保护和减少政府密件委员会''(Commission on Protecting and Reducing
Government Secrecy)。原来的规定是,绝密文件(Top Secret)必须由针对这个文件的解密立法才能够解密。而这个
委员会发现,截至1994年,联邦政府已经有了一百五十亿页达到甚至超过二十五年历史的绝密文件,要针对这些文
件一个个立法解密,根本不可能。因此克林顿总统在1995年签署了12958号行政命令,反向规定,所有政府绝密文件
达到二十五年限期全部自动解密,对公众开放,只有针对某些文件再立法``不得解密''的,才可以继续留在绝密状
态。历史学家们自然大喜。十几年执行下来了,切实可行。一般和国家安危有关的政府机密,超过二十五年也就过
了危机了。极少数需要继续保密的再立法保护,而其余的就可以交给历史学者和大众,去回顾和总结历史经验教训
了。

委员会的研究报告称,美国联邦政府在当时(1994年),大约每年产生四十万份机密文件。他们因此也指出,保密已
经成为现今政府工作的一部分常态。在2001年反恐之后,需要保密的政府文件一定暴增。因此,在发出12958号行政
命令之后,克林顿总统在1995年立即又发出12968号行政命令,非常细致地规定接触保密资料雇员的安全审查规范。

\myphoto{image029.jpg}{政府是有秘密的,《阳光法案》也不能保证公众的知情权。}

\pagebreak
\section{主题切换带来的问题}

在今天,一般国家的行政机构对涉及机密部分,需要有对雇员的安全忠诚审查,几乎是一个不假思索的措施,好像
既然需要,做就是了。美国没有什么事情是政府红头文件指派下来,大家就令行禁止照做的,总是一大堆疑问,经
久不息的讨论,甚至司法挑战。最后可能归于妥协。这种妥协,可能是左右两端意见的折中,也可能是理想主义状
态对现实危机有限的让步。关键在于细节,在于让步的细节幅度。

忠诚本非新议题。美国宪法还详细规定了总统忠诚宣誓的誓词,只是第一任总统华盛顿宣誓结束时不由自主冒出
``愿上帝保佑我''(So help me God),后面总统也就纷纷跟进,比宪法规定多了一句。同样,州长、国会、州议员、
法官等司法分支官员、军人、保密部门人员、归化入籍的公民等,都要求有忠诚誓言。另外,也有特定历史时期的
忠诚要求,例如最初各英属殖民地独立而组成联邦制美国,各州民众都被要求忠诚于这个新国家,其实是要大家习
惯自己开始了州和联邦的双重国民新身份。

冷战开始,对间谍的安全防范落实到行政分支,就必须切换到忠诚主题,因为是``防范''而不是执法部门``抓间
谍''。在1946年开始第一次对政府雇员进行安全、忠诚考察,就出现许多质疑。前面说过,杜鲁门的联邦政府处理
时还是相当谨慎,可当时焦虑也在扩散到州政府一级,结果,虽然是同一个``忠诚''问题,实际上开始分流:在联邦
层面,实质关注点始终聚焦在间谍和安全问题,而州的层面在切换主题之后,很快抛开``间谍危机'',开始把浮在
表面的``忠诚''做成了单独文章。

州政府在这个问题上的``异化''其实很自然,州的层面并不面对国家安全防卫,它们本不该群起``接棒'',有时还
接错了跑道。于是引发一大堆忠诚问题, ``忠诚''可以是一个文学化甚至情绪化的概念,那么,要求宣誓的``忠
诚''具体内容是什么?联邦制下,只要不违反联邦宪法和法律,联邦政府无权要求州政府做什么,同时也管不了州政
府``要做什么''。于是各州自行其是,大多州政府也在杜鲁门被称为``忠诚命令''的9835号行政命令之后,仿而效
之,对雇员开始有``忠诚要求'',延伸出来的做法五花八门,甚至在一些地方有泛滥趋势。应对这类泛滥,美国人
通常的办法就是司法挑战解决。反正司法分支是最后的支撑。

\pagebreak
\section{对忠诚誓言的司法挑战经久不息}

例如在麦卡锡时期,加州就在1954年立了个地方法,让二战老兵享受房地产免税待遇,但是之前必须要签一个``忠
诚誓言''。当地有个名叫劳伦斯\textperiodcentered 斯宾塞的二战退伍兵,恰巧也是美国公民自由联盟(ACLU)的
律师。打言论自由官司是ACLU的特长,这个法可以说是恰好撞在了他枪口上。于是就有了著名的``斯宾塞对兰待尔
案''(Speiser v. Randall)。案子最后一路上诉到最高法院,最高法院判定该地方法违宪。最高法院并在其他案子
中明确,忠诚誓言的要求只和政府雇员有关,和普通人无关。

可是,有一个忠诚议题确实和州以下政府密切相关,那就是教育。联邦政府不拥有学校,教育是各州在管。这样,
公立学校教师就也是州政府雇员。州政府也特别担心一些反对宪法和民主制度的教师,可能借讲台便利,轻易就用
极端思想煽动了青少年,也不能说州政府的焦虑就完全没有道理。于是,当时大多数州都要求公立学校教师,包括
大学教授签署忠诚誓言作为雇用条件。一些学校认定,若教师``不忠诚'',或者向学生教授``不忠诚'',就有可能
被解雇。这些忠诚誓言,多半只是要求他们支持联邦宪法和州宪法。但是,也有一些地方要求教师宣扬爱国主义、
发誓自己不属于共产党组织以及``赞同以暴力推翻政府''的组织。

类似忠诚誓言陆陆续续在司法挑战中失败而取消。例如在佛罗里达州的橘县,在1950年规定公职人员必须签一个不
支持帮助共产党的誓言。一个公立学校教师克来穆(Cramp)在规定出来后教了九年书,学校却忘了让他签誓言,发现
的时候已经是1959年了。学校要他补签,却被他拒绝了。克来穆去了州巡回法院,以学校规定违宪为理由,申请法
院禁止令,被法院驳回。1961年,此案(Cramp v. Orange County)最后上诉到联邦最高法院,法院判定这个忠诚誓
言侵犯了正当程序,也侵犯了教师的言论自由。

1967年联邦最高法院又以5:4裁定纽约州的行政条例违宪。该条例规定凡``颠覆分子''和拥护鼓吹暴力推翻政府的,
不符合教师资格。教育董事会也指定了一个组织名单,认定凡属这些组织的成员就不符合教师资格,其中就包括共
产党。最高法院认为,``颠覆''、``拥护''这些用词过于含糊。一个大学教师的某个学术观点很可能被认为是违反
规定,学术和教学内容就会受到极大限制。最高法院认为不能对学术研究作任何预设的限制,在判词中明确提出
``教室尤其是思想的市场'',学术自由``是第一修正案所特别关注的'',有关忠诚的条例,会``在教室中投射对正
统思想的厌恶''。这个关键案子使得公立学校教师从定义含混不清的忠诚困扰中解脱出来。

回过头来,教育问题当然并不是简单的是非黑白。在学校里,学术和教育相交相融。学生都很年轻,相对单纯冲
动,会有一些特定倾向。他们很难在一开始就理解事物的复杂两难,只有随阅历增加才会慢慢成熟。好的教师是以
自己的阅历,尽早帮助学生早一些成熟,而不是利用学生的单纯热情来支持偏激的政治倾向。这是社会无法通过法
律作出规定的。法律能够做的只是两害取其轻。

在1964年的``巴格特对布立特''(Baggett v. Bullitt)一案中,最高法院认定华盛顿州对政府雇员、公立学校教师
作忠诚誓言的要求无效。这个誓言除了要求忠于宪法,还要求增进对国旗的尊重,以及尊重法律与秩序。法院认
为,誓言的词义含糊得不恰当、不确定,也过于宽泛,因此违宪。

行政系统的安全、忠诚要求,关键在于,它是从间谍问题切换过来的。可以说,间谍是对国家``不忠诚行为'',但
它是一个刑事罪。可是一个美国人表示他不忠诚国家,不仅不是刑事罪行,还是他自由言论的个人权利。换句话
说,要求一个普通政府雇员作忠诚誓言,就是要求他放弃部分权利。这本身是否违宪?1954年有一个``巴伦布赖特对
合众国案''(Barenblatt v. United States),在1959年上诉到联邦最高法院。这也是一个拒绝作证引出``藐视国会
罪''的案子,它本身和``忠诚誓言案''无关,可是,在这个案子中由最高法院大法官提出的概念,却和``忠诚誓言
疑问''有关。大法官认为,言论自由概念从一开始就不是绝对的。它牵涉``公共利益和个人利益之间的权衡''。所
以,并非所有忠诚誓言都违宪,它需要具体权衡。最高法院判决的基本原则是:假如誓言对忠诚的解释是宽泛的、含
糊不清、没有明确界定,那么它基本上是违宪的。正因为需要对具体案例的具体权衡,所以``忠诚案''层出不穷,
也就是在一系列判决中,最高法院把忠诚的法律适用范围,逐步细节地界定出来。

``忠诚案''是麦卡锡时代的重要议案,因此,可不可以``不忠诚''、具体可以怎样``不忠诚''、到什么地步,成为
很多人的难解心结。挑战``忠诚''的案件至今不休不眠,国旗烧得烧不得,迄今未有最终定论,还在一轮轮司法挑战
中。上法庭挑战的理由有的出人意外,例如1994年的``贝沙德对加州社区大学''(Bessard v. California
Community College)一案,提出的是宗教理由,贝沙德以1993年的《宗教自由重建法》(the Religious Freedom
Restoration Act,简称RFRA)为依据,提出``不起誓''是自己宗教信仰的一部分,故政府不能要求他签署忠诚誓言。
法庭在一审中支持了他的观点。但是在1997年上诉到最高法院时,判定RFRA因超越了宪法第十四修正案支持的国会
履行安全措施权力,因此,RFRA违宪。贝沙德败诉。

麦卡锡时代,最高法院对忠诚案最重要的判定,当属和``结社自由''有关的判决。待下回分解。

\pagebreak
\section{法官和中尉的麦卡锡}

虽说``麦卡锡时代''不是麦卡锡参议员一人能够顶下全部是非与责任,但他是一个象征,一个最后的重推手。而麦
卡锡时代形成的政治派别倾向、至今未消的恩怨争论,自然也统统落到他的头上。所以,除却历史学家之外,大众
对一个实实在在的人,一个客观存在,也争执不休,判断迥异。

我看了各种说法,感觉麦卡锡其实是个比较简单的人。他于1908年底出生在美国中部一个农庄家庭。农庄经营不
顺,麦卡锡必须先工作后读书,取得大学法律学位已经二十七岁。他显然是个努力想``做好''的年轻人,1939年,
三十一岁的麦卡锡成为威斯康新州最年轻的巡回法院法官,并且一上任就超时拼命工作,审清了以前长期积压的全
部两百多个案件。就在他当上法官的那一年,二次大战在欧洲爆发,但美国还没有参战。有当地报纸为证,麦卡锡
三年法官的工作颇受好评。1941年底珍珠港事件爆发,美国参战。法官依法可以免服兵役,可三十三岁的麦卡锡还
是志愿加入海军陆战队,上了战场。此举对美国人来说倒也不算很稀奇,那些一开始来中国战场的飞虎队就根本不
是服兵役,只不过一激动就自己跑来帮忙。记得《华盛顿邮报》总裁格雷厄姆的《个人回忆》中提到,美国参战后
她丈夫去白宫,白宫顾问见了他刚打个招呼就对他``咆哮'':``怎么没去当兵!?''格雷厄姆周围的富家子弟很多都上
了战场,战死的也不在少数。她丈夫后来也上了战场,而且他还是联邦最高法院大法官助理。所以,麦卡锡此举不
算很特别,至少是正常的爱国和勇敢。

参战期间麦卡锡去的是太平洋战场,他本是轰炸机中队的中尉情报官,相对比较安全,却志愿要求担任轰炸机的尾
部机枪手。现在对麦卡锡时代耿耿于怀的人还有在争论他出任务究竟多少次,是不是在战后和人家吹牛的时候虚报
了次数,我想这都是没见过战场的后辈在斤斤计较,能主动要求提着脑袋去哪怕一次,也算是堂堂正正的军人了。

进一步的证据是:麦卡锡在1952年获得十字飞行荣誉勋章(Distinguished Flying Cross),海军陆战队的空军副司令
哈利斯少將(Field Harris)给麦卡锡写信说:``尤其是战场表现,你得到太平洋舰队司令尼米兹(Admiral Chester
Nimitz)的赞扬。你服役期间所有指挥官都毫无例外地对你的表现给予最高评价。''司令的评价是:作为情报官和俯
冲轰炸机尾部机枪手``表现出色。他参加大量战斗任务,还在正常任务外承担航空摄影师,冒着敌人猛烈的防空火
力拍摄火力布局''。麦卡锡的直接上司穆少校(E.E. Munn)在推荐信中特別提到他参加的轰炸包括了那些遇到日军猛
烈抵抗的地区。

\pagebreak
\section{一个有雄心的人}

还没有退伍脱下军装,麦卡锡已经开始竞选威斯康新州的联邦参议员。但是论麦卡锡的资历来说,这一步迈得够大
胆的,很有点``敢想敢干''的味道。参众两院差别非常大,众议员有点大区人民代表的意思,众议员提案非常荒唐
离谱的都有,因为他只需要迎合一小块地区的民众意愿。而参议员每州只有两名,更被认为是精英代表。对``德高
望重''的要求也高得多。

要作为共和党代表出来竞选,麦卡锡先要过党内竞选这一关。麦卡锡来自底层平民,假如竞选众议员,兴许还是个
优势,竞选参议员就难说。而他挑战的对手小拉弗拉特(Robert M.~La Follette, Jr.)来自政治世家,父亲老拉弗
拉特就是老牌参议员,后来在1982年一次由美国历史学家投票的调查中,被评为``有史以来最佳参议员''的头一号。
这当然无形之中会给儿子从政加分。1925年老拉弗拉特在参院任上去世,威斯康新百姓们就选小拉弗拉特补了他爹
的缺。所以,他已经是个有经验的参议员。而论从政经验,严格地说,麦卡锡只有三年州巡回法院法官的经验,其
余只是在当兵。从竞选过程本身来说,麦卡锡也很不利,他没有那么多操作时间,一开始来竞选还是向部队请了假
的。可是,二战不仅以它的残酷、更以其严峻局面使得``兵''的分量变得非常重\myrule 没有兵去拼命,世界落在
谁手里都难讲。而法官和兵的结合引人注目。麦卡锡手里主要也就是这张牌。

在竞选过程中,麦卡锡采用负面竞选的方式,攻击了自己对手的弱点。这在当时还是被相当普遍使用的竞选方式,
现在虽然政治家都表示自己文明进化了,负面竞选大大减少,可是它充满实际获利的诱惑,政治家们经常还是未能
免俗。麦卡锡当时指责对手没有去申请当兵,小拉弗拉特其实够冤的,不要说珍珠港被袭击那年,他已经四十六岁
了,而且还身体有病,为此年轻时读大学都辍了学的。麦卡锡还攻击对手:在战时不去打仗保家卫国,反倒在家炒股
牟取暴利,听上去像是在发国难财,杀伤力不小。其实这是美国常态,后来大家验证下来,麦卡锡虽然确实在战场
殊死战斗,可也没耽误投资股市,他一边打仗一边在股市获利高达四万两千美元,而小拉弗拉特的两年收益是四万
七千美元。总之,最后两人各方面优劣上下扯平,旗鼓相当。投票结果是,在两人各得二十多万张选票的情况下,
麦卡锡只多了五千票,实属险胜。然而,负面竞选不仅容易造成不公平,而且对政治品质显然是一种毒害。

\pagebreak
\section{共产党帮了麦卡锡的忙}

大家一定没有想到,选举结果出来,小拉弗拉特当时就宣称,他是被共产党打败的。此话还不能说完全没有道理。
后世历史学家分析起来,麦卡锡的初选胜利,确实还得益于几个共产党控制的工会,他们鼓励自己的工会成员去投
票给麦卡锡。因为小拉弗拉特是充分表现过自己的政治家,立场明确。他们判断要两害取其轻的话,麦卡锡可能对
共产党还更有利一点。当然,后来他们一定悔断肠子了。

进入共和党和民主党对决的正式选举,麦卡锡反而很轻松,实际上这不是麦卡锡取胜,而是1946年共和党大胜。麦
卡锡就这样脱下军装,在1947年直接进了参议院。在麦卡锡推动所谓``麦卡锡主义''之前的三年中,他作为一个立
法者,支持了推动放开战时限定的物价,糖价是其中很重要一项。这个推动本身并没有问题,问题是他曾经获得百
事可乐公司的两万美元贷款,因此他推动放开糖价也就饱受诟病。

在最初这三年期间,麦卡锡参与的重要立法之一,是支持了美国重要的《劳资关系法》(The Labor-Management
Relations Act),这个立法使得美国的劳资互动和一些欧洲国家,如法国等,有了根本差别。它的目标是期待在劳
资之间取得平衡,使得商品生产、商业发展有一个最佳流程。因为到了上世纪四十年代末,美国工会已经非常强
大,也需要立法限制工会权力的过度扩张,比如工会不得禁止雇主雇用不参加工会的工人,最重要的是提出禁止危
害``普遍社会安全福利''的概念,以及设立相应的仲裁机制。从此,美国不会出现长期瘫痪社会交通的大罢工。在
立法之时,它自然引起工会强烈反弹,逼迫杜鲁门总统对该法案签署``否决''。之后国会又依照宪法规定,以三比
二以上的大比数赞成票推翻了总统否决,完成立法,这条法案至今有效。对麦卡锡这个投票立场最为愤怒的,当然
就是当年号召自己的会员在初选中支持麦卡锡的威斯康新工会。

从麦卡锡这位参院新人在最初三年的政治立场和活动看来,竞选时宣称``国会需要尾机枪手''的麦卡锡,并没有什
么很强的攻击性。有这样一个例子。当时正在审理二战达豪(集中营)案,案子的一部分是著名的曼眉地屠杀案
(Malmedy Msaascre),德国冲锋队的一个部队屠杀了美军战俘,还犯下其他类似罪行,几案并审。麦卡锡却不知从
哪里听说,以前军方在审理中有非法逼供,因此他认为这些德国冲锋队员没有受到公平待遇,要求对他们的死刑减
刑。据说他没有拿出扎实证据来支撑自己的说法。这件事情使得他声誉大降,招致普遍批评,却同时也可以说,麦
卡锡并不是一个轻易就为公众舆论和民愤所左右的人。

\myphoto{image031.jpg}{麦卡锡主义挑战的是``免于恐惧的自由''}

\pagebreak
\section{麦卡锡对``麦卡锡主义''的第一次推动}

就在1950年,麦卡锡在西弗吉尼亚州的维灵市讲话,突然宣布他手上掌握了外交部五十七个有安全隐患者的具体名
单,引发公众舆论强烈反应。民主党参议员对麦卡锡的这次讲话当然反感,这等于是在直接攻击民主党杜鲁门总统
的内阁。1950年2月20日,麦卡锡为此在参院作了一场长达数小时的演说,民主党一些议员不断以质问打断演讲,达
一百二十三次。一个要求是非常起码和正当的,就是``拿出证据来''。麦卡锡当时承认,他没有掌握这些人的全部
信息,因此不适合在参院公布名单,但是他会向参院的调查委员会提供必需的资料。

虽然民主党议员非常愤怒,但是事至如此,参院迫于压力,还是立即在参院外交关系委员会下面成立了一个下属调
查委员会,准备对外交部作彻查。可以说是麦卡锡第一次参与推动``麦卡锡主义''。虽然参院是共和党占多数,为
公正起见,委员会由民主党参议员米拉德\textperiodcentered 泰丁斯(Milard Tydings)主持,后来就被叫做泰丁
斯委员会。麦卡锡向委员会提出了一个调查名单,听证调查持续了四个多月,到当年7月工作结束。泰丁斯主持写出
的总结报告,对麦卡锡的指控完全持否定意见,指责麦卡锡给美国带来分裂,起的作用比共产党更糟。而共和党指
责泰丁斯是要把隐藏在内部的敌人``洗白''。参院几次表决是否接受这个报告,投票完全以党派为界,赞成和反对
两分。

在表面的党派之争背后,反映出整个事情的矛盾和困扰。现在读历史,反对麦卡锡的,当然赞成泰丁斯参议员对麦
卡锡的指责,也认为共和党对麦卡锡的支持只是党派性所致。可是,一方面,麦卡锡的名单并非完全是空穴来风,
他的依据之一,是二战中美国各军事机构在战后有数千人转入外交部,审查过程中认为有接近三百人有安全隐患,
其中有七十九人因此没有被雇用,其余还是进入外交部工作。另一个很重的砝码,是在听证过程中,前苏联间谍布
丹茨(Louis Francis Budenz)出来作证,他是苏联在美国间谍网最主要的自首者之一,他的证词支持了美国外交部
有苏联潜伏间谍的说法。但是,从另一面看,泰丁斯参议员的报告也显然有理:证据仍然不足。

困扰就在于,第一,参院调查不是一个司法过程,就不可能像法庭那样证据``钉是钉,铆是铆'',能够铁板钉钉
``定案''。而它又是在调查非同小可的外国间谍渗透政府机构案。第二,间谍案确实还有它非常特别的地方,就是
有证据可能也不会公开。就在1950年2月7日,麦卡锡在维灵市讲话的前三天,联邦调查局长胡佛就在众议员拨款委
员会作证:反间谍``和处理一般刑事犯罪的目标不同。反间谍重要的是找出间谍联系人,搞清间谍的目标意图、情报
來源、联系方法'',``拘捕、公开信息是最后迫不得已才做的事情''。他说,只有彻底了解整个间谍网,才有安
全,才能使他们失去作用。当时美国的反间谍还刚刚开始,现在大家都知道,这些话只不过是行内常识。后来证
实,只有胡佛和极少数人,当时确实掌握了一批在美国政府之内的苏联间谍,而他们被迫对所有人保密,总统也毫
不知情。司法无能为力,国会追查眼前也是一片泥沼。

所以,麦卡锡的第一次推动。并没有什么实质结果。

\pagebreak
\section{两次``惧红'',惧怕对象其实并不同}

麦卡锡在1950年的第一次推动并不成功。但是出现了``麦卡锡主义''这个词,也就把他推到了整个事件的舞台中心。
泰丁斯委员会调查不算成功,原因是调查过程虽然出现有力支持``外交部有间谍渗透''的证人证词,但是又没有具
体能给某人定间谍罪的证据\myrule 当然是没有的,假如有,只需要直接诉诸司法了。这就是国会调查与生俱来的
矛盾之处。

所以,倾向``无切实证据就证明没问题''一派,就是泰丁斯的结论和绝大多数民主党议员的基本立场;可结论无法
被参院接受。共和党的基本立场倾向于``相信证词''判断外交部存在被间谍渗透的问题。

可是,民主党参议员并不是完全反对调查。就在泰丁斯委员会结束工作的两个月后,1950年9月,民主党的麦凯伦参
议员(Patrick Anthony McCarran)就带头提出一个《国内安全法》(Internal Security Act),后来就被叫作《麦凯
伦法》。麦凯伦参议员一向以坚持自己的主见出名,例如他看到佛朗哥的西班牙和德、意的本质差别,很早就赞同
资助西班牙,被骂得不轻,他也不在乎。他和蒋介石的国民政府的关系也相当好。

从这个法案却可以看出国会多少有点黔驴技穷。他们首先想到的,还是1917年以《移民法》解决无政府主义者恐怖
活动的招术,就是先把外来威胁堵住。所以《麦凯伦法》的一个重要条款,就是将极权国家的颠覆组织成员阻挡在
国门之外。政府有权阻止其组织成员加入美国籍,在必要情况下,可以递解出境。假如在入籍五年之后发现问题,
政府也有权取消其国籍。

立法本身由苏联间谍问题引发,共产党当然是主要对象,另外排列在该法案上的,就是法西斯组织。可是,这个立
法其实根本不可能像1917年《移民法》那样,成为一个应对、解除危机的法案。道理很简单,两次所谓``惧怕红
色'',面对的危机是不一样的,也就是``惧怕''的对象其实不同。

 ``第一次惧红''时期,是国家面对工业革命初期劳资关系极度紧张、劳资调节的立法跟不上,无产阶级革命的理想
却刚刚诞生,苏联在美国人面前如同解决所有问题的一剂良方;加上一次世界大战,美国民众普遍处在困扰中。当
时的美国共产党,尤其是无政府主义,所做的事情一部分是公开宣传鼓动反战和革命,哪怕是另外那部分扔炸弹的
地下恐怖活动,毕竟是活跃的活动。虽然司法定罪困难,但是顺藤牵瓜抓住嫌疑人的可能很大,而且参与者多是新
移民。这时建立一个针对无政府主义组织的严苛的《移民法》应急,以``移民递解''减轻压力,虽然从程序上来说
不完美,但确实行之有效。待危机解除,应急措施也相应进入历史。

现在的所谓麦卡锡时代的``第二次惧红'',尤其在它的高潮,也就是麦卡锡自己站出来的1950年,局面却大不相同。
红色阵营已经在全球占领半壁江山,不仅冷战持续了将近五年,而且苏联有了核武器,朝鲜战争还在进行中。这次
``惧红'',惧的不是红色思潮煽动革命,惧的是核时代的苏联间谍。他们不声张也不公开活动,他们要做的就是不
声不响地在政府要害部门潜伏下来。他们人数不多,甚至是在美国生长、受过良好教育的人。可是一旦哪天冷战变
热战、核大战,他们的情报却可能是致命的。

\pagebreak
\section{外交部的苏联间谍}

可是,在一个大规模移民国家,以这样的移民法对付间谍,根本没用。例如,在冷战早已结束的今天,这个1950年
的《麦凯伦法》几经最高法院规范和修改,已经大部分失效;对入籍移民是否加入过纳粹和共产党这样的问题,也
已经变得像例行公事。我问过一个曾经``在组织''的朋友,他告诉我他在入籍时没有说实话,只是为了避免``不必
要的节外生枝'',相信这样的情况很普遍。可以想象,在冷战期间的苏联间谍,当然就更不愿意节外生枝了。所以
这个立法条款简直就成了一个国会表态,或者说是对民众的一个安慰了。

《麦凯伦法》的另一个重要内容,就是成立参院下属的``颠覆活动控制委员会'',授权它合法调查苏联间谍网对政
府的渗透。立法过程再次凸现了白宫与国会的冲突,杜鲁门总统也否决了《麦凯伦法》。非常引人注目的是,此时
已经是民主党在国会占多数,可是,这个国会依据宪法规定,再以大比数强行通过了《麦凯伦法》。这样,就像众
议院有非美活动调查委员会(HUAC)一样,参议院也在1950年底有了具有同样目标的调查机构。它的第一任主席就是
麦凯伦,所以也被称为``麦凯伦委员会''。这样一来,两院齐发,真的就有点很紧张的样子了。

麦卡锡第一次出来攻击外交部有间谍,背后有一个重要推动。HUAC在1948年作了一次最重要的调查。在1948年8月3
日的听证会上,《时代》杂志资深编辑、前美国共产党员维坦可\textperiodcentered 强博斯(Whittaker
Chambers),指认前外交部高官艾尔格\textperiodcentered 西斯(Alger Hiss)在为联邦政府工作期间,一直是个秘
密共产党员。西斯从1933年开始就进入联邦政府工作,当时还是小罗斯福时代。1936年他和他兄弟都进入了外交部。
西斯是二战之后雅尔塔会议的美国代表团成员,也参与了联合国工作。此时他已经离开外交部,在卡内基国际和平
基金会任职。得知自己被指控,西斯主动来到听证会,在誓言之下断然否认自己是秘密党员。可是强博斯其后在接
受电台采访时,再次重复了自己对西斯的指认。结果,西斯就干脆以诽谤罪把强博斯告上了法庭。

其实西斯一直在FBI的监控下,因为根据冷战时期最早在加拿大反水的苏联使馆间谍伊戈\textperiodcentered 哥萨
柯对美国外交部间谍的描述,以及美国最主要的苏联间谍柏特丽的揭发,目标都指向西斯。在法庭上,强博斯为了
证明自己所言不虚,进一步指控西斯是间谍。他拿出了当时西斯交给他的外交部文件副本、有西斯手迹的文件打印
稿,指出这是在西斯家的打字机打出来的。专家核对下来打字机特征吻合,这成为最关键的证据。在HUAC听证会
上,强博斯和西斯都曾经在誓言之下否认自己是苏联间谍。看上去,前者显然是害怕这样的检举把自己也卷入司法
起诉之中。可是西斯现在反正已经倒打一耙告他诽谤,他也就干脆一不做二不休全部说出来了。

这就是走司法程序困难的地方,司法是一板一眼的。按照法律,不论证据多么确凿,都已经无法再对西斯提起间谍
罪诉讼,因为强博斯提供的证据,显示的是上世纪三十年代的活动,已经过了间谍罪起诉的时效。于是,检察官决
定对西斯作两项伪证罪的起诉,因为他在誓言之下撒了谎。鉴于强博斯的检举,就决定不再追究强博斯的伪证罪。
这彻底解脱了强博斯的顾虑,他接着提供了新证据\myrule 五卷胶卷,其中两卷是外交部文件。1952年,在刚成立
不久的参院麦凯伦委员会上,美国前驻法大使作证说,法国总理曾经在1939年对他说,根据法国情报,美国外交部
两个姓西斯的都是苏联间谍。

\myphoto{image033.jpg}{艾尔格\textperiodcentered 西斯1954年出狱时依然受到媒体的广泛关注}

\pagebreak
\section{ ``西斯是否有罪''已经和西斯无关}

西斯是个联邦政府高官,在法庭作证的证人也就``高''得异乎寻常,其中有两名证人当过民主党总统候选人,一名
是联邦最高法院大法官。在1949年第一次审判的时候,陪审员无法达成一致而``流审''。第二次审判找到新证人,
在1950年1月25日,西斯被判伪证罪有罪。他一路上诉到联邦最高法院,最后还是败诉,被判处五年徒刑,实际坐牢
四十四个月。西斯就是个接受过良好教育、土生土长的美国精英。这一点深深震动了美国民众。

1957年,西斯出版了自己的书《在民意法庭上》(In the Court of Public Opinion)。他原来就当过律师,在书中
细细为自己辩解。意思当然是,是麦卡锡时代歇斯底里的民众焦虑,使他蒙受冤屈。西斯在有生之年一直试图洗刷
自己。1976年,根据《自由信息法》,西斯要求FBI交出那五卷作为证据的胶卷,两卷确实是外交部文件,可是另外
三卷没有任何意义。西斯据此要求律师协会恢复了他的律师证书。意思是当年错判了他。他是唯一一个因刑事罪被
撤销律师执照、却又重新恢复执照的律师。

西斯案审判后十五天,麦卡锡宣称他掌握了五十七个外交部安全隐患名单,显然麦卡锡此举也是受到这个案子的刺
激。正因为如此,西斯案件也无形中成为对麦卡锡时代定位的一个关键,成为今天的政治派别、政党派别纠纷的一
个历史焦点。好像假如西斯是间谍,麦卡锡时代的种种,就获得一定理由;假如西斯无辜,就更证明``麦卡锡主
义''整体为一场毫无道理的歇斯底里。同样,在一些纠缠在党派性中的人看来,今日左翼派别是当年叛国分子声气
相通的``后裔'',还是力挽政治歇斯底里狂澜的好汉传人?今日保守党派究竟是迫害狂的承继者,还是爱国者的精神
延续?这么大而空的议题,有时就会荒唐地聚焦到``西斯究竟是不是间谍''之上。西斯至今没有走远。

从此以后,西斯的一点一滴都没有被人放过,上穷碧落下黄泉,从美国到苏联到东欧。随着苏联解体,文件逐步解
密,从硬实的解密证据,到捕风捉影的小道消息,统统被汇集在一起,一头反过来一头又倒过去。1998年,两党合
作的政府保密委员会的民主党主席莫尼汉(Daniel Patrick ``Pat'' Moynihan)出版了自己的新书《保密:美国经
验》,在书中他也提到:``相信艾尔格\textperiodcentered 西斯是有罪还是清白,已经成为美国知识分子生活中的
一个分界了。''此前的1997年,莫尼汉根据他对解密资料的研判,声明说,``对外交部的艾尔格
\textperiodcentered 西斯的看法似乎可以达到共识了。他的情况和财政部的亨利\textperiodcentered 瓦特
(Harry Dexter White,另一个被确认的间谍)其实是一样的。''可是,我看到最近的证据分析是2007年的,相信还
会继续争论下去。

我曾阅读这些证据,试图从中得出自己的判断。后来发现,这样的证据推敲和争论,或许对西斯本人的身后声名是
有意义的,而对当时讨论的外交部间谍案却失去了意义。因为双方都认同,根据解密监听资料,美国外交部有一名
苏联间谍,有时被称为艾尔格,有时被称为西斯,``姓''和``名''从来没有一起用过。这个艾尔格参加了雅尔塔会
议,之后又去了莫斯科,这一点和艾尔格\textperiodcentered 西斯的行程吻合。但是反对者又争论说,监听资料
有些细节和西斯情况有出入。

所以,假如讨论的题目是外交部是否存在间谍,那么,答案是肯定的。而是不是这个``西斯'',并没有多大关系。
今天撇开历史背景的情绪化``对抗'',对理解那个时代和吸取历史教训也并无意义。而西斯伪证罪的定罪,在当时
产生的效果,确实复杂而危险。它加剧了焦虑的情绪。它虽然肯定了国会两院调查的``成就'',可是,这种肯定和
鼓励蕴藏着巨大危险。

这一事件也在``协助''把麦卡锡参议员推进了``麦卡锡主义'',他变得身不由己。自从他宣布自己掌握了外交部安
全隐患的名单,得到本来在焦虑中找不到出路的大众的热烈支持,也受到反对一派的强烈质疑。他急于为自己辩
护,却一时拿不出关于这些名单进一步强有力的证据来,所以他需要更强劲地推动调查,以调查结果来证实自己。
而麦卡锡又是一个性格弱点明显的人。这使他本人和``麦卡锡主义'',都逐渐雪上加霜。

\pagebreak
\section{间谍的``大义''}

其实卷入检举、告密、间谍、卧底等非常行为,引出的民间评判可能是截然相反的。对大众来说,常常只是基于一
个判断,就是判断行为背后是否有他们所认同的``大义''在。一般来说,这些行为有两个方向,一种是反对自己国
家政府的秘密活动,如潜伏卧底,甚至为外国政府服务的间谍;另一种就是政府鼓励的检举告密、为自己政府服务
的卧底甚至间谍。称``非常'',是因为它们大多需要利用和背叛信任信托,有悖常情。

对一个现代国家,以非法手段甚至暴力来颠覆国家,总是违法的。可是,在引进类似革命概念以后,民间的道德判
断、历史判断显然会不同,对同一事件就会出现``起义''和``叛乱''这样的反向描述。就美国共产党和左翼来说,
追溯它颠覆宗旨的来源,他们每个人都会告诉你,他们的理念最初来自对劳工弱势群体的深刻同情,苏式社会主义
理想国家是大义,而资本主义丑恶没落是不义的,不以暴力手段推翻不能实现正义,所以宣称要革命的理直气壮。
在上世纪末,他们背后有大量同情支持的左翼民众,这是因为很多人的判断和他们接近,而现在失去民众支持,也
只是民众改变了政治判断。

判断并不总是简单。例如《潜伏》这样的电视剧,假如背景是发生在抗日战争,潜于侵略阵营,``大义''前提就轻
松解决,就像二战盟军间谍片一样,潜下来的都是孤胆英雄。假如背景是国共之争,复杂性马上出来:在共一方,主
角当然至今是大家的英雄,虚构的陈年故事仍然看得津津乐道,这很好理解。可是,假如从支持国民政府的一方
看,``大义''前提立即消失,同一主角立马成为阴险无比的敌人。

由于非常行为多少是在利用和背叛某种信任,所以,大义之下,其实掩盖着个人极大的道德伦理压抑和挣扎。而人
们在探讨的时候,常常不由自主略去难以解决的内在冲突,把事件放在一个比较容易的盘子里托出来,这样作道德
判断就比较简单。

现在我们谴责告密,通常看的是单向场景,就是设定告密者不义,被告密者是无辜受害,如此推演不义告密带来的
个人伤害,格外惊心动魄。或者为大义卧底者,背叛的都是坏人,就没有这个问题。实际上,就个人遭遇的伤害来
说,是一摸一样的,这种伤害在脱离``义和不义''的前提下孤立存在。这也是在人的世界里,文学永远有其一席之
地的原因。

所以,卷入这些非常行为的人,他迫切需要倚仗``大义''来突破自己的心理障碍。而所谓大义,当时可能只是一种
判断或者说是政治信仰。提到信仰高度,对个人就没有什么道理可讲,信就可以了。所以,美共中大多做了苏联间
谍的人,感觉绝非做了什么坏事,相反自我感觉良好。这和现代间谍很不相同,在那个时候,他们很多人不把间谍
和钱联系在一起,而是和政治理想联系在一起。所以,一点不奇怪,历史上很多人理所当然都以``理想主义''来解
释自己的行为甚至罪行。

间谍、卧底、告密,由于它和常情相悖的特点,在正常社会中总是罕见的非常行为,假如一个社会制造``大义''借
口,把非常行为广泛推广到社会,必然毒化人际关系,令整个社会处于非常状态,正常人不断异化,也给卑劣提供
借口,陷入这种状况的社会必定要出大问题。

\pagebreak
\section{政府方大义的基础是法治}

很多人认为,美国社会是绝对不鼓励检举告密的,其实不然。应该这么说,正常社会中,政府的``大义''依据就是
法律。现代一个非常典型的例子就是校园炸弹手西奥多\textperiodcentered 卡辛斯基(Theodore John
Kaczynski)。他是个科学家,有他的大义在,就是反对工业化对自然环境的毁坏,他认为呼吁不起作用,只能以惊
世骇俗的办法解决。他因此在1978年至1995年的十七年里,寄出了十六个炸弹邮包。寄给那些他认为必须为破坏自
然承担罪责的人。炸弹共杀死三人,杀伤二十三人。联邦调查局重金悬赏要求民众举报,最后他兄弟认出卡辛斯基
``宣言''的文字风格,和母亲商量之后,决定``大义灭亲'',向联邦调查局举报。卡辛斯基的兄弟没有要举报奖
金,也努力为他争取免除死刑的交换条件。尽管如此,作为亲属他们当然还是很困扰。

政府的``大义''很容易变异。法治概念可以偷换为专政,``罪与非罪''的严格法律界定可能变异成非常含糊的``敌
我''分野。然后,可无限扩大的``敌人''概念,与一些负面形容词同构,使得法律概念``文学化'',``民愤''可以
是量刑依据,而民愤又可以用权力调动。``敌人''概念的宽泛,决定了``罪行''的宽泛,与权力正统有别的一切都
可能被宣判为罪行,音乐舞蹈言论文学思想等等,可以无所不包,甚至把整个族群划为敌人,例如纳粹德国的犹太
人群体。不仅有大规模司法冤狱,还有社会惩罚。这是二战后美国人对极权国家的扩张感到恐惧,想竭力逃脱同样
命运的原因。可是,实际上事情并不是``是、非''那样简单。民众希望维持自由的愿望,很可能就是一个失去自由
的陷阱:自由受到威胁,你在焦虑下怎么做?

纪录片《盲点》(Blind Spot)是对希特勒女秘书Traudl Junge的晚年采访。Traudl Junge提到,希特勒面对疑问,
一再对德国人解释,你们不知道,假如我们失败、苏俄得胜,我们国家会遭遇可怕灾难。Traudl Junge后来被苏军
抓获,苏军请了一名美军士兵作翻译。翻译官同情Traudl Junge只是个年轻秘书,设法让她进入西德,她因此在美
军审查后释放,得以生活在西德。她看着家乡在盟军协助下逐步建立民主体制,她在影片中说,这时她想起希特勒
``假如失败''的话,觉得自己受了骗,很是生气。我看的时候就想,她这段话其实有逻辑错位:假如她没被领到西德
呢?可是,看完电影《窃听风暴》展现的东德,希特勒就是对的吗?当然不是。极权国家通常在渲染一些或真或假的
外在威胁,如希特勒,利用民众恐惧而使他们彻底放弃权利并听任摆布。使得国家变得比政府宣称要预防的糟糕情
况更糟。希特勒说是要防止苏式专制,而他已在德国推行了专制的极端。而在另一端,后来苏联掌控下的东德,又
在反帝口号下,把自己国家置于《窃听风暴》之中。

\pagebreak
\section{好莱坞黑名单}

美国人很早意识到,他们也有同样的危险。应对安全威胁是国家大义,如果对间谍和颠覆组织的清查推过头,法治
社会就可能变异,陷入``不义''之中。如何应对制度和国家面临的威胁,美国的左右翼两极之争,从第一次惧红时
期,到麦卡锡时代,直到今天,争论不休。不认真对待外部威胁,可能有灭顶之灾;反应过度,自己就先毁了自己。
这也是今天美国争论恐怖战争威胁,会一路追溯到几十年前麦卡锡之争的原因。

这一警觉始终存在。因此,在美国,政府鼓励举报杀人放火,更容易被大众接受。虽然卡辛斯基也是有``大义''的
理想主义者,很像是一个人的革命党。可是他毕竟成了恐怖分子杀了人。类似的情况还有国会有过对KKK和黑手党的
调查,要求交待同党,民众也容易接受。而国会调查美国共产党,在听证会上要求告发同党,民众的感觉很不同。
国会调查不是司法审判,它有宽的一面:只要回答问题,它不涉及惩罚;但是有了好莱坞黑名单,就有了是否存在
``社会惩罚''、``类罪行''等灰色地带的疑问。虽然民间``雇用黑名单'',只存在于因涉及宣传而被国会调查的好
莱坞。

哪怕在麦卡锡时代,还是争辩、抗议的``两军对决''局面,而不是压倒一切的一边倒局势,法律仍然是被动方时时
可以动用的武器。例如``好莱坞十好汉''被判藐视国会罪,好莱坞三大专业行会发表``华道夫声明'',宣称不雇用
颠覆组织成员,十好汉马上提起司法诉讼,要求天文数字的赔偿。吓得行会马上宣称,``对判断谁是共产党、对现
在和未来雇用的决定权,完全保留在行业各制片人和各公司手中。''并且坚决否认他们有拒雇名单。因为他们在声
明中呼吁国会对``私营企业排斥颠覆组织成员''立法,是不可能的立法期待。既然没有立法,他们就必须非常小心
对待,赶紧退下来。这也是所谓``黑名单''称呼的来源,因为不能公开拒聘。虽然各公司雇主仍然按照自己的意
愿,悄悄列出拒聘名单。总共有过三百二十个姓名被好莱坞拒之门外,包括优秀的剧作者、演员和其他工作人员。

说是``姓名''而不是``人'',是因为黑名单上的一些剧作家用假名继续在为好莱坞写作,作品照拿大奖甚至拿奥斯
卡奖。当然他们也为欧洲工作,剧本写作在那个时候早已经全球化。但是,并非每个人都如此优秀、有高超生存技
巧和好运气。有很多人再也没有回好莱坞工作,影响了他们的家庭和人生轨迹,甚至有人离开美国。所以,提到麦
卡锡时代,人们更多提到的不是麦卡锡本人参与的参院调查,而是好莱坞。

\pagebreak
\section{伦理冲突使得民众积累着不安}

好莱坞黑名单是私营公司的民间雇用惩罚行为,也正因为如此,它没有严格标准,对有些人变成毫无道理的雇用迫
害。也正因为是私人雇用,又是``黑''名单,说起来雇主总有选择雇员的权力,上了黑名单的人完全是哑巴吃黄
连,无处申诉。个别案例还有联邦调查局暗中插手,最著名的例子是卓别林。

卓别林的思维轨迹非常典型,从同情工业化兴起后处于弱势的工人开始,形成自己的左翼观点,他反对希特勒独
裁,却赞扬和推崇苏联。卓别林本人并没有加入美国共产党。但他公开表示,二战以后共产主义将代表人类进步而
席卷全球。正因为他太出名,这些演说产生很大影响。他二十五岁移民美国,在好莱坞成名,很多人以为卓别林是
美国人,其实他始终保留英国国籍。1952年卓别林回英国,联邦调查局局长胡佛乘机要求移民局取消他的回程许
可,使得卓别林后期只能留在欧洲发展。这成为麦卡锡时代好莱坞的最大丑闻。直到1972年,好莱坞才向他道歉,
邀请卓别林夫妇参加奥斯卡颁奖仪式,授予他奥斯卡荣誉奖。

美国共产党非常糟糕的境遇是,他们怀着政治信仰的``大义''投入``革命''组织,现在又面临国会的反颠覆、反间
谍调查,合法地把他们卷入检举告发自己同志的困境中。更何况,他们大多数人以前并不知道组织里有外国间谍这
一说,突然发现自己怎么和阴谋挂了钩。所以,一些人认为,他们是应该与国会调查配合的,这种思想反复甚至
``好莱坞十好汉''也不例外。

例如十好汉之一的达顿\textperiodcentered 特鲁伯,他虽然在第一次众院听证会上拒绝回答问题,却在1951年的
另一次听证会上作证,内容包括了他所知道的共产党员名单。我印象最深的是1999年好莱坞给卡赞(Elia Kazan)颁
发奥斯卡荣誉奖。一批好莱坞演员,包括年轻人,拒不起立鼓掌,以谴责卡赞曾在配合国会调查时,提供了八个同
党姓名。五十年后人们仍然不肯原谅,他自己始终认为没有做错什么。可是不论有没有``大义''在,告发本身都是
背叛信任,在道德上有困扰。

对民众来说,虽然这个组织有颠覆宗旨、其中有苏联间谍,可是很多党员尚未有任何颠覆行动。即便理解调查的必
要,民众仍然会对其中每个人在道德冲突下的煎熬感到心理紧张和极其不安。

这是麦卡锡时代的另一种焦虑,紧张在不断的积累中。

\myphoto{image034.jpg}{面对台下拒不起立鼓掌的尴尬场面,卡赞在举起奥斯卡小金人的时候,是否有点心虚。}

\pagebreak
\section{麦卡锡安全隐患名单}

需要强调的是,众院非美活动调查和麦卡锡的参院调查,有着很大差别,前者从调查``宣传''切入,所以对民间企
业好莱坞的调查,占了一大块;而参院调查基本是针对联邦政府最上层:所谓政府被渗透问题。1950年2月9日,麦卡
锡的``维灵讲话''提到了他掌握一批外交部安全隐患名单,引起轰动。麦卡锡名单自然成为后来专家们研究争论的
一个重要内容。

所谓麦卡锡名单,一开始只是麦卡锡转述他人调查名单的一个``人数'',没有具体姓名。他的一个依据是``李名
单'',是众院由罗伯特\textperiodcentered 李(Robert E.~Lee)领导的一次委员会调查,李原来在联邦调查局工作
过,调查结束后列出一个安全隐患名单。麦卡锡后来还提到一个二百零五人名单。这个名单来自二战后的国务卿詹
姆斯\textperiodcentered 拜伦(James Byrnes)给一国会议员的信。信中称,安全调查发现二百八十四名雇员有安
全隐患,当时建议不要作为长期雇员,可结果有二百零五人留用。顺便提一下,拜伦部长是个奇人,他在联邦政府
三大分支都有经历,是美国历史上的极罕见的现象。除了行政分支内阁部长等等职位,他还在国会参众两院都当过
议员,更奇特的是,还当了一年半的联邦最高法院大法官。他还是自学成才,连中学都没有读完。当时他眼看美国
马上要参与二战,觉得这种时候待在最高法院太乏味,居然辞了出来。他在七十二岁离开联邦政府之后,还回家竞
选州长成功。拜伦个人声誉相当好,所以他的这封信也应该是相当有力的证据。可是,仔细查下来,拜伦的信写于
1946年,在麦卡锡拿这封信作证的1950年,这二百零五人只剩六十多个还留在外交部。

麦卡锡在参院调查中,陆续提供了麦卡锡名单的具体姓名,主体是上述两个名单的结合。由于麦卡锡的争议性,多
年来不同专家对``麦卡锡名单''作了详尽分析,有名有姓的一百九十五人中,后来被非常确凿的证据证实为苏联间
谍的有九人。按照冷战时期的标准,可以说名单中大多数人可能有``安全隐患'',就是他们通过长期参加一些组织、
活动等等,表现出他们在政治信仰上曾长期认同倾向于冷战敌手苏联。而具体考量每一个人,又有很大差别。一些
人后来或许已经改变信仰,例如斯蒂芬\textperiodcentered 布鲁纳(Stephen Brunauer),他在二十年代加入美国
共青团,十几年后似乎抛弃了原来信仰,在1946年曾经回到母国匈牙利,是受美国政府派遣,为帮助那里的科学家
离开红色匈牙利去美国。虽然这些人今天已经被人反复研究,可是在当时只是一个粗粗的``隐患''归类。

今天的专家结论认为,麦卡锡名单中,有少数人是没有安全隐患问题的,其中极少数在今天被大家确认是绝对没有
问题,其中就包括前面提到过的国务卿艾奇逊,以及杜鲁门在1947年指定的国务卿马歇尔将军。而他们被麦卡锡指
责为背叛国家的怀疑对象,都和中国有关。

\pagebreak
\section{ ``安全隐患''相比间谍是更模糊的概念}

我一开始很纳闷,艾奇逊和马歇尔将军,他们和其他出现在麦卡锡名单上的人很不一样。那些人多半因为加入过共
产党或者外围组织,或者多少有点什么蛛丝马迹,而这两位好像确实``清清白白'',而且身居高位,为什么会被麦
卡锡公开谴责为``安全隐患''?看下来,关键就在于从警惕``间谍'',到警惕政府决策上层``安全隐患''概念的扩展。

 ``隐患''的意思是,哪怕并非间谍、没有在递送情报,可是,假如他有着忠诚共产主义政治信仰、亲苏倾向,甚至
内心同情共产主义,而他偏偏又在外交部等要害部门工作,他就可能在工作中、在参与决策时,出于信仰而有意无
意中助对方一臂之力,使得自己阵营在冷战中由于一系列决策错误而失败。可是怎么判断谁是``隐患''?看上去这已
经到了诛心的地步。所以``是否加入过组织''是一个大致的判断方式,另一个是反向判断,就是假如发现决策有明
显问题,来反推决策者是不是居心叵测。当然,这都是我今天的总结,但是实际上,艾奇逊、马歇尔,甚至杜鲁门
自己,就是这样被麦卡锡反推出来、发动攻击的。

艾奇逊、马歇尔都对中国二战后的局面有自己的一套看法。前者上任后组织专家研究,结论是中国当时趋势很难改
变。而马歇尔将军更是在1945年曾被杜鲁门派往中国,试图对国共双方作和平调停,但是他很快发现所谓``国共和
谈''只是双方缓兵之计,他说服不了任何一方。最后,马歇尔将军和驻华大使司徒雷登一起宣布``调停失败'',马
歇尔将军留下一句名言:``我看到一个前所未有的时代在等待中国。''然后婉辞蒋介石希望他担任的``最高顾问''一
职,离华回国。确实,杜鲁门的两任国务卿对中国命运前景作出的判断都是:外力无可逆转。但是,美国也有一些政
治家认为,冷战中两大阵营对决,中国的选择会对美国有生死攸关的影响,在此关键时刻,应不惜再大幅增加军援
甚至出兵协助中国国军,也要守住不让一个大国``翻红牌''。马歇尔确实不赞同这样的看法,当时杜鲁门政府表现
出对中国国民政府和中国前景都相当悲观。这里确有许多具体决策,例如,是否继续援助国民政府,具体援助多少
等等。最后经过惨烈内战,1949年的中国投入苏联为首的共产主义阵营。这也是美国外交部首当其冲受到责难、被
怀疑有严重安全隐患的主要原因。推出隐患概念,怀疑面就应声扩大。

\pagebreak
\section{政治决策的吊诡规律}

政治决策其实常常有一个吊诡的规律,就是你总是只有一个选择。你选择这样做,那么其他没有选择的方式就没有
试过,它的结果也就没有发生,你就无法用那个``结果''来证明自己是作了最好选择,决策者无法用没有发生的事
实来为自己辩护。与此同时,批评者倒是可以做到永远不错,用的也是那个``没发生的事实''。当时的美国对华政
策就是如此。在杜鲁门受到质疑,称其外交政策错误导致中国变色的时候,他拿不出``事实''来证明,假如换一种
政策,中国是否就会避免那个``前所未有时代''的宿命。这个决策规律使得美国两党的政治辩论、政治博弈永远可
以热热闹闹地进行下去。在1949年以后,立即发生朝鲜战争,引发对民主党的杜鲁门外交政策的强烈质疑,这是
1952年大选艾森豪威尔成为二十年来第一个共和党总统的原因之一,同时,参众两院也都是由共和党占多数。麦卡
锡则更是一口咬定,假如换一种政策,冷战中的亚洲局势一定大不相同,虽然他也不可能拿出``事实''来印证自己
的推断。麦卡锡不仅指责对华政策错误,同时还倒推:艾奇逊、马歇尔这样作出``错误决策''的人,可能就是暗中推
出红色中国的``安全隐患''。

所有的人都说,麦卡锡最后的失败是在调查军方的时候``踢到铁板''。其实在此之前,麦卡锡和两届总统、和军队
首领们的关系一直是紧张的。其中一个重要原因就是麦卡锡对马歇尔这样的前军队将领发动猛烈攻击,出言不逊,
甚至语言无礼、粗鲁。从各种回忆录看,麦卡锡是个在性格上有缺陷的人。在1950年``出名''之前,麦卡锡在朋友
和助手眼中显得友好、容易亲近。但议员同事们也感觉他在工作中性急、脾气容易失控。人的优点和缺点其实就是
一个分币的两面,全在分寸。对一个普通人来说,影响面很小,对政治家来说,特定机遇下,可能后果非同小可。
麦卡锡是从战争勇士转为国会议员。国会代表民意,监督行政分支是它的职责。直到今天,议员对总统及其行政分
支,还是十分严厉,批起来毫不留情。麦卡锡的轰炸机尾翼机枪手经历,需要的就是英勇无畏。转到国会这样一个
鼓励批判、以批评督察为主要职责的位置,更是勇猛得口无遮拦。而从麦卡锡站出来的第一分钟开始,他本人也面
对众多尖锐尖刻的批评,这也在刺激他``越战越勇''。

\pagebreak
\section{双方紧绷的弦}

从当时围绕麦卡锡发生的一些冲突,可以感受在麦卡锡时代,卷入其中的人,都处在极为矛盾和紧张的状态,也同
时可以看出其中的复杂。

例如,最早站出来反对麦卡锡主义的记者很少,其中就有德鲁\textperiodcentered 皮尔森(Drew Pearson),他是
当时最知名的专栏作家之一,作过欧亚旅行报道,到过中国,是很早出来批评中国蒋介石政府的。他站出来反对麦
卡锡主义,或许有点为自己的工作小团体自卫的意思。他的两个属下被国会调查。一个是戴维
\textperiodcentered 卡尔(David Katz),卡尔很早参加共产党,在联邦政府的战争信息办公室(OWI)工作,1943年
受到众院非美活动委员会的调查,卡尔在誓言之下编造了个反向的故事,宣称自己不是苏联间谍,其实是美国联邦
调查局线民,居然大家也拿他没办法。国会调查断定他``不忠诚可靠'',可是认为这并不能作为解雇他的理由。卡
尔后来从OWI辞职,皮尔森马上雇了他。麦卡锡提出卡尔有安全隐患问题的时候,他已经在皮尔森手下工作了。同
时,皮尔森的另一个雇员安德鲁\textperiodcentered 欧德(Andrew Older),他们夫妇二人都受到国会调查,欧德
的姐姐朱莉亚\textperiodcentered 欧德(Julia Older)也受到国会调查。皮尔森再三为卡尔辩护。1950年皮尔森和
麦卡锡在华盛顿俱乐部的衣帽间相遇,发生冲突,麦卡锡事后承认自己打了他一个耳光。皮尔森咽不下这口气,收
集了一大堆虚假八卦,提供给他人写成文章,称麦卡锡是同性恋,这在当时的美国社会可以算是一个``丑闻''。他
的这些``材料''都没有被以后麦卡锡的传记作家们采纳。多年后根据解密资料,卡尔和朱莉亚
\textperiodcentered 欧德都被确认为苏联间谍。

另一事件可以说是一个悲剧。前面提到过的小拉弗拉特,1946年在威斯康星州参议员竞选中被麦卡锡击败之后,成
为杜鲁门政府的对外援助顾问。1953年2月8日,他在一家很有名的杂志《克里尔周刊》上发表了一篇文章,谈到以
前在参院主持的民权委员会的经历,他认为那里被共产党渗透。两个星期之后,他在华盛顿开枪自杀。一些历史学
家认为,他自杀的一个重要原因是承受不了心理压力:文章发表之后,他一定意识到自己可能会被麦卡锡主持的委员
会传去作证。确实可能:七个月后,就有人在麦卡锡的调查中作证说,他知道该委员会有共产党员在里面工作。假如
小拉弗拉特还活着,他很可能会接着收到作证传票。听上去这很过分,可是,多年后确认,当时在这个受调查的委
员会中,就有四名苏联间谍。

在这样的对立氛围中,麦卡锡的情绪不稳定成为他自己的致命伤,也成为他主导的那部分``麦卡锡推动''的致命弱
点。到一定年龄,人的自控能力本来就开始减弱。加上在整个事件中承受巨大压力,麦卡锡开始酗酒,越来越严
重,也越来越难以控制情绪,而来自底层、军人生涯、战争经历等等,这许多本来就说不清对一个人的影响是正面
大于负面,还是负面大于正面的因素,在一定条件下可能就是``很负面''。甚至在整个麦卡锡事件高潮的参院听证
会上,他也屡屡言语失控。我常常想,麦卡锡本人也是麦卡锡时代的一个牺牲者。

\myphoto{image036.jpg}{麦卡锡(中坐者)在听证会上喜形于色}

\pagebreak
\section{调查触及陆军}

艾森豪威尔总统曾经非常明确地表达说,他支持麦卡锡参议员推动调查,但是他反对``麦卡锡方式''。究竟什么是
麦卡锡方式?批评他的人认为,这就是麦卡锡主义,就是捕风捉影、煽动、追查隐私,麦卡锡在调查中出了名咄咄逼
人的风格,也可算上一份。麦卡锡和他的支持者自然认为,这些做法都是调查所必需的,否则怎么查得出。
1952年,麦卡锡因此还出了本书为自己辩护:《麦卡锡主义:为美国战斗》(McCarthyism:The Fight For America)。
回看美国国会在上世纪调查黑手党和黑帮化工会等等的听证会,其实场面都很火爆,议员们都很有穷追猛打的劲头。
为什么麦卡锡的劲头大家就特别难以承受?除了大家担心是否会走向``猎巫''感受紧张,更有一层原因是调查政治组
织,对象不同感觉总是不同。而麦卡锡平日出言不逊地攻击行政官员,其中就有德高望重的前二战将军,更不要说
后来直接调查军队了。

今天人们很难体会,二战后人们对军队的感觉是很不一样的。杜鲁门虽是行政分支总统,可是根据宪法,美国总统
也是三军统帅。平时看不出来,可在世界大战中,这个三军统帅还是货真价实的,不说别的,两个原子弹就是杜鲁
门拍板给扔出去的。战后,军队将领从战场退役转向行政,也比平时多得多,例如马歇尔将军先后任国务卿、国防
部长,艾森豪威尔当选总统,等等。就连麦卡锡自己,一个来自底层只有短暂州司法经验的军人,能在竞选中战胜
政治世家出身的前参议员,不能说和民众对二战英雄的敬重信任无关。

这本来就是非常矛盾的局面:追查不紧,等于不追查,过了分寸,大家心理上其实难以承受,但这只是一种感觉,民
众其实并不清楚,难以承受的究竟是什么。所以,民众对麦卡锡的支持度一开始是观望为多,1951年8月的时候,民
调显示有百分之六十三的人还不知道自己应该支持还是反对麦卡锡。此后大多数时间,麦卡锡的反对者都多于支持
者。1954年初,有过短暂的支持率达到一半,但是,这时几乎所有观望的人都已经被卷了进来,也就是社会的矛盾
对峙、心理张力,都已经到达了临界点。

虽然麦卡锡作为议员质疑行政分支是职责。可是,看到一个前低级军官,对前将军没有充足根据就肆无忌惮横加指
责,言词粗鲁、失去起码的礼貌,二战后的美国民众会感觉特别不舒服,军官们就更难接受。更何况,不论中国变
色还是朝鲜战争,麦卡锡指责的是``安全隐患'',就是指不是政策失误而是一个通敌可能了。当他质疑这些经历二
战的将领们的忠诚,很多军人和民众都觉得非常过分。在参院听证会上,最可能反弹的就是和麦卡锡一样从战场上
下来的军人、尤其是将军了。所以,形势突然逆转。这个转变就出现在麦卡锡对军队的调查中,真是一点不奇怪。

\pagebreak
\section{``非常状态''}

麦卡锡是在1953年底开始调查陆军的蒙莫思堡(Fort Monmouth),那里有机密研究机构,被指称有间谍。蒙莫思堡一
案到底如何至今还在各执一词,一方宣称那里根本没有问题,而当时一个参议员后来著书说,蒙莫思堡安全防线被
间谍突破,在内部早已不是秘密。里面的机密部门被迁出是个事实。总之当时调查并没有结果。麦卡锡又转到另一
个军队案件,一个叫欧文\textperiodcentered 佩雷斯(Irving Peress)的军医参加过左翼政治组织,却拒绝填写例
行相关表格,被陆军限令三个月内退伍,就在这个时候,他被传到参院作证,又拒绝回答问题。作为藐视国会罪他
应该接受审判,可他又是个军人,就必须去军事法庭。麦卡锡通知陆军,欧文\textperiodcentered 佩雷斯立即按
照原来的军队要求申请退伍。美军退伍有``荣誉退伍''和``不荣誉退伍''两种,一般有过错的会被要求不荣誉退伍。
可是他的上司,维克准将(Ralph W. Zwicker)准许了他荣誉退伍。将军显然在自己权限内、以自己的方式表示对调
查的不满。麦卡锡一怒之下又用传票传来这位二战英雄将军,维克将军在军队法律顾问建议下,也拒绝回答麦卡锡
的提问。麦卡锡当场大发雷霆,辱骂这位将军处理佩雷斯事件的智力还不及一个五岁小孩、``根本不配这套军装''。
此等无礼终于激怒了所有的人,从总统、国会两党议员到媒体,当然,还有军队和所有民间的二战退伍老兵。陆军
部长罗伯特\textperiodcentered 史蒂文斯(Robert Stevens)下令,维克将军不得再去麦卡锡听证会。但是,事后
陆军部长还是向麦卡锡妥协,答应军队尽量配合调查,这个妥协却在五角大楼官员中引起大哗。欧洲报纸也嘲笑
说,这是美国陆军向麦卡锡参议员投降。实际上,这是制度限制:军队属于人民,军人没有权利违抗国会调查。

国会代表民众,负责立法和监督行政,它是民主制度中``人民是主人''的象征,本来在制度设计中就是偏强的。行
政对它的制约十分有限,因为行政更是管家的意思。当然,国会议员不能违法、不能强行制定违宪的法律,司法分
支是对它最有效的制约。但是,司法的制约只是把可能的偏差约束在``合法''的限度内。在它没有违法、在它的权
限之内,它也非常可能出现合法偏差。所以,国会也隐含着某种危险,那就是多数意愿的偏差。现在,国会突然在
某个点上显得突兀了。

冷战开始以后,有关间谍和安全隐患引发的国会调查,虽然只是断断续续有听证会,但也持续了很长时间。麦卡锡
本人卷入已经四年,过去国会对黑帮等组织的听证会,都非常短暂,从来没有一个主题,相对密集的调查,延续时
间如此之长。虽然拿到传票进入听证会的只是少数人,可是对``供出同党''的``非常''要求,尤其严词逼迫的都是
些谦谦君子,又不是杀人越货的黑手党。对一个文明社会来说,心理感觉是一种``非常状态'',已经接近极限。再
要攻击二战英雄,实在忍无可忍,而对于刚刚走出战争的军队来说,出生入死都过来了,哪里肯受这种鸟气?军队有
军队的特殊荣誉感、等级尊重传统和逻辑,提起麦卡锡,可以料想他们会说,不就是个中尉吗?怎么都指着将军鼻子
骂了,我们还要低声下气!

\pagebreak
\section{陆军反击}

很快,显然是被调查的陆军组织了一场反击。他们找了个听上去很不相干的事情,要求参院调查麦卡锡和他的首席
法律顾问罗伊\textperiodcentered 柯恩(Roy Cohn)。他们投诉这两人行为不端、越过自己职权向军队施压,要求
给一个叫作薛恩(G. David Schine)的大兵予以额外照顾。

柯恩的形象后来进入许多电影,其中最著名的是他的传记片《公民柯恩》(Citizen Cohn)。好莱坞对麦卡锡的怨气
几乎都倾泻在这部电影里,电影从柯恩得艾滋病濒临死亡倒叙,表现一个个曾在法庭上由他参与起诉被定罪者的幻
影,来到他的床头找他算账。可见柯恩是多么招人怨恨。在听证会上,他和麦卡锡一搭一挡,是出了名有逼迫感的
提问人。虽然作为曼哈顿的联邦检察官办公室助理和检方律师,成功参与了一系列大案起诉,包括美国共产党领导
人案、西斯间谍案、卢森堡夫妇间谍案,之后又转到参院麦卡锡委员会担任了几年首席法律顾问。可是在被军队投
诉遭到调查时,柯恩其实还只是个二十七岁的青年人。

柯恩进麦卡锡委员会工作的时候不过二十五岁,虽然雄心勃勃其实还是玩心很重的年龄。他结识了和自己同岁的旅
馆业富家子弟薛恩,两人政治倾向一致,在各方面都很合得来,很快成为好朋友玩在一起。他把薛恩也拉进他的顾
问团队,薛恩就有了个不支薪的主顾问头衔。他们经常一起在公众场合露面。以至痛恨柯恩的人,在薛恩结婚生了
六个孩子,还坚持说他们两人是同性恋伴侣。直到最近,历史学家才基本确认,虽然柯恩是同性恋,但是他和薛恩
只是朋友而非恋人关系。不久,薛恩因当时的兵役制被征兵入伍。一帆风顺的柯恩年轻气盛,得意洋洋不知天高地
厚,就想利用自己在政界位置,向部队打招呼给朋友一些照顾,例如不要吃那么多苦、额外的假期、不要派驻海外
等等。他没有想到这些事情假如被揭露出来,这是非常严重的官员违规。对美国人来说,兵就是兵,穿上军装一律
平等,否则还叫什么军队。现在还了得,居然连当兵都想利用职权享受特权,那军队还有什么信誉?

从此案调查,可以看到美国国会委员会听证的厉害:就这样一件事情,听证会从1954年4月22日开始,持续了整整三
十六天,共三十二个证人出庭作证,广播和电视现场转播。这么看,当初国会为间谍案查得翻天覆地也就不那么奇
怪了。应该说,结论相当公正。被投诉调查的是两个人,查下来最后结论是:柯恩有错,麦卡锡被洗清,在这件事情
上他没有过错。

这显然是陆军受到调查后的反击,可是,在一个法治社会,你避免反击的唯一办法是自己不犯错误,而柯恩确实授
人以柄。在听证会上,他们当然也全力抗辩,指称陆军醉翁之意不在酒,实际是要阻挡国会调查。当然,他们也必
须提供证据。最后听证会的结论是,陆军部长罗伯特\textperiodcentered 史蒂文斯和一名法律顾问约翰
\textperiodcentered 亚当斯(John Adams),确实做过努力,试图阻挠麦卡锡委员会对蒙莫思堡的调查。

\pagebreak
\section{逻辑之中 意料之外}

可是,调查结果都已经不那么重要了,重要的是这三十六天中民众的直观感受。这次麦卡锡和柯恩是被调查对象,
他们就再也不是坐在讯问者的位置,而是转到了证人席上。再不是那个逼问者,而是要接受别人提问。军队的律师
同时挑战他们的诚信。结果,麦卡锡所有个人弱点暴露无遗。他过去站在主动攻击的位置上,一些夸张说法难以当
场被质疑,此刻不同,他如同站在法庭证人席上,任何一点不实之词都会被律师穷追猛打,当作撒谎、没有诚信、
个人品质恶劣的证据。这样的挑战必然涉及麦卡锡调查中的许多问题。三十六天的听证会,在通过电视观看听证的
两千万美国人面前,恼怒的麦卡锡情绪完全失控。媒体对他的报道常常是负面的,反过来一定也在给他更多刺激。
在第三十天的时候,局势终于崩盘。

1954年6月9日,陆军法律代表韦尔奇(Joseph Nve Welch)对柯恩挑战:麦卡锡曾说过国防工厂有一百三十个共产党和
同情者,那么,``太阳下山前'',你们向联邦司法部长交出名单。麦卡锡被激怒,突然插进来指责韦尔奇:你还是先
把手下的共产党费雪(Fred Fisher)给解雇了再说。

律师并非军人,只是陆军聘用了波士顿的一个律师事务所,费雪也是其中一个。他1942年毕业后就上了二战战场,
战争结束后读了哈佛法学院,然后加入了这个律师事务所。他在校时曾经加入了一个以左翼出名的律师组织``国家
律师协会''(National Lawyers Guild),这个组织因其强烈政治倾向,不断受到质疑。费雪后来离开了这个组织。
他本来要参加此案律师团,却因为这个经历被质疑,临时换了人。麦卡锡在这个时候再次以夸大其词的方式提到费
雪,韦尔奇几乎出于本能作出回应,而这段回应在美国成为历史经典。

他指责麦卡锡``无所顾忌地冷酷无情'',他说:``参议员,让我们不要再毁掉这个小伙子,你干得够了,你难道不懂
什么是人之常情?你终于到了不剩一点常情常理的地步了吗?''他打断麦卡锡辩解的企图,指示说:传下一个证人。

突然,会场爆发一阵欢呼。所有人在瞬间心灵点触,只希望结束一切``非常'',让我们回到常情之中。

麦卡锡时代,就在这一刻结束了。

\pagebreak
\section{维诺那计划}

每一次美国社会在危机面前,焦虑都是对立双向的。除却对危机的焦虑,有应对危机引发``非常态''带来的反向焦
虑。每一次,后者会渐渐压倒一切,最后,一根稻草压倒骆驼。人们便不顾一切,要求回归正常。据说军队律师代
表韦尔奇后来奇怪:我也没说什么特别的啊。在英语中,他击中大家那个词是``decency'',简单却含义丰富。
Decency是正派,是庄重得体,是待人有礼举止得当,是行为符合道德感受、适度有分寸。在这一事件中我把它译为
``人之常情''、``常理常情''。一个时代的负面声名全压到麦卡锡头上,就是人们感觉他是悖常情之道,走得最远
的那个。这是美国现实:危机压力下,整个社会正常观念从未改变,没有被一个非常思潮洗脑而认``非常''为``正
常'',只是在危机压力下,暂时在极小范围让步,容忍却极其有限、很容易反弹。

军队投诉的国会听证结束,麦卡锡大势已去。不久,参院通过对麦卡锡的一个行为不当谴责议案,主要指他对维克
准将态度``过分''。1954年共和党失去在参院多数席位,麦卡锡退出历史舞台。三年后,年仅四十八岁的麦卡锡患
急病去世。时过境迁,麦卡锡在历史书里成为一个迫害狂。一个重要原因是美国共产党、外围相关组织、激进左翼、
温和好心左翼等等,当年没头没脑都被麦卡锡扫作一堆,一起质疑进去。这一支在美国变化,最后,激进部分变得
很小,温和左翼则很大,在今天被称为自由派。他们是批判麦卡锡的主力,不仅为前辈抱屈,也会想,假如我们在
那个时代,不是也……可是,简单痛批、归罪一人并不等同于吸取历史教训,麦卡锡穷追猛打追查的威胁,并非子
虚乌有。只是以麦卡锡方式查不出也无法确认,``麦卡锡''只是美国突遇现代间谍战的手足无措,现在就知道,交
给FBI、CIA就好。当时麦卡锡就像在街上总是凶巴巴揪住几个人说有贼,可揪半天总是找不出被偷钱包,不是自己
找骂吗?

事情的开端是自首间谍的揭露,而苏联对美国政府渗透的真正谜底,对公众揭晓很晚。1995年维诺那计划(Venona
Project)文本解密,都已经是克林顿时代。计划从二战的1943年开始,当时美军情报部门主管官员卡特
\textperiodcentered 克拉克(Carter W.~Clarke),担心斯大林再次和德国密约、单方媾和,那会对盟军很不利,
就决定截取苏联外交密电破译,意外发现苏联在美布下间谍网。这个截获破译计划,就是``维诺那''。

计划和英国情报部门合作,美国只有极少人知道,既然罗斯福特別助理柯里(Lauchlin Currie)也赫然在破译的间谍
名单上,联邦调查局长胡佛决定对总统也守口如瓶。实际上,计划有效期只是1945年前短短两年。计划失效的原
因,也在印证危机的可怕:一个叫危斯邦(Bill Weisband)的美国人参与破译工作,他立即向苏联通报,待美国发现
他的间谍身份,已是1950年。

\pagebreak
\section{《史密斯法》的反颠覆条款}

1945年后的间谍发展,当时就断了线。直到苏联解体,公布大量文件,才有进一步了解。可是,仅此两年截获的文
件,已经非常惊人。维诺那文件给出了三百四十九个间谍,他们为苏联情报部门工作。涵盖美国政府高层,其中还
有1994年在中国去世的前美国财政部官员所罗门\textperiodcentered 阿德勒(Solomon Adler)。文件公布使得一系
列历史上争论不休的案件有了定论。

1995年由两党共组``政府机密议题委员会'',主席是来自纽约州的民主党参议员莫尼汉(Daniel Patrick
Moynihan),他写道,平衡叙述这段历史的机会开始展开:``维诺那信息无疑是提供事实的巨大密库'',他解释原先
叙述不平衡,是因为:``保密需要曾制度性地否决了美国历史学家接触文件的可能。''他写道:``最近,根据莫斯科
的苏联档案,回答了下面疑问:本世纪中叶在华盛顿究竟发生了什么……维诺那破译含有苏联间谍网在美国活动无可
辩驳的证据,包括姓名、日期、地点和行动。''中央情报局负责历史情报收集的负责人匹克(Hayden Peake)则指
出,``还没有过一个现代政府像美国政府那样被全面渗透。''我看了心想,这家伙一定没有用功读读中国国民政府
史。

大家长期盯着麦卡锡,实际上,更值得关注的是司法审判,就是依据《史密斯法》(Smith Act)的系列审判。它是美
国历史上不多的几个有反颠覆条款的立法之一。这些法律通常是应对危机或战争的应急立法。如1861年《煽动暴乱
法》,源于南方准备脱离美国,内战就在眼前,立法是试图阻挡南方分离,立法后没有起诉任何案件。直到八十年
后首次起诉也没有成功。另有一战中应对煽动拒服兵役的《1917年间谍法》。二战也一样,移民国家参加世界大战
总是紧张的,怕间谍、怕后院起火,1940年的《史密斯法》也是危机法律,由小罗斯福总统签署,正式名称是《外
国人登记法》(Alien Registration Act),它要求外国人四个月内向政府登记,当时登记的超过四百七十四万人。

最有名的是它的反煽动颠覆条款:``明知而有意鼓吹、煽动、劝导或者教唆以暴力推翻美国政府和州政府的必要性、
可取性和正当性;或建立组织教唆、劝导、鼓励颠覆,加入或附属上述组织'',都属刑事罪的颠覆罪。这和《1917
年间谍法》很像。但两次战争差别很大。一战遇到国内宣传拒服兵役,起诉就多。美国参加二战,国内并无强大反
战力量。《史密斯法》在二战期间也就很少动用。可二战后涉及间谍的冷战立即开始。此法成为麦卡锡时代的重要
组成部分,眼看法庭无力给绝大部分间谍定罪,行政分支的司法部想到以《史密斯法》起诉共产党领袖,作为一个
应急措施。

\myphoto{image038.jpg}{1952年,Eagle Sgnal公司的女员工聚会抗议FBI滥用《史密斯法》}

\pagebreak
\section{特殊的``有法不依''}

回看这段历史和《史密斯法》,你会发现麦卡锡时代是多么``荒唐''。原来他们没有必要这样麻烦,弄得自己灰头
土脸,弄得举国上下痛苦不堪,他们只需要认真执行《史密斯法》就可以了。既然间谍和安全隐患都和美国共产党
有关,而《史密斯法》有成员条款:``加入或附属颠覆组织的成员'',都在违法之列,为什么不名正言顺依法一网打
尽所有危险组织成员、立即解除危机呢?这是理解这段历史的复杂部分,因为要论``罪与非罪'',本来大规模扫荡颠
覆组织成员、起诉定罪的条件是现成的。可所有的人,包括国会和麦卡锡本人,包括负责起诉的司法部,自始至终
非常明白,决不可能轻易动用战时应急备用的立法,虽然这是现成的法律惩治工具。

来到美国,都会对这里有法必依、执法严格,留下深刻印象。可是,深入阅读历史,发现美国有些法律条款是非常
谨慎对待、能不用就不用的、一旦动用最高法院必定出来进一步规范,如《史密斯法》涉及煽动颠覆国家,就和政
治反对紧密关联。这些是美国人认为非常``危险''的法律。假如真正有人``使用暴力'',可以非常简单归到刑事罪。
而``煽动暴力''则非常不同,它涉及言论自由的冲突,涉及对政治反对的容忍,涉及政治表达的自由,涉及对民主
制度、自由国家的根本理解。假若轻率动用,可能引发广泛迫害政治反对的恶果、可能滥用司法、动摇法治根基带
来更深刻的社会颠覆。这种认识,源于美国立国理念。历史上类似法案的产生、使用、最高法院判例限制等等,从
托马斯\textperiodcentered 杰弗逊开始,就引发社会警惕。条款门槛越低,越没人敢轻率动用。

似乎心照不宣,都知道这苛严的条文是为战时最危急情况准备的,如大战中,国内真有组织准备助敌暴动、危及国
家安全,政府可能紧急动用该法起诉其全部成员,解燃眉之急。而即使冷战面临外部威胁,甚至有局部热战,但要
把颠覆罪大规模``推广落实''到危险组织所有成员,绝不可能被社会接受,这是美国法律中一个非常特别的``有法
不依''现象。在危机下,司法部决定走一小步,有限动用这条法律,起诉一些组织头目,打击间谍网扩展。因
此,1948年7月,以《史密斯法》起诉了十一名美国共产党高层领导(最后判处三至五年监禁),至1957年,全美陆续
起诉了一百四十多个共产党领导人。

共产党领导人案在1951年进入最高法院,没有被否决。直到1957年另有十四个共产党领导人被定罪,案子再次进入
最高法院,大法官裁定:对共产党暴力推翻政府信条和鼓吹暴动,到底是理念,还是真要教唆党员实施武装暴
动,《史密斯法》若对此不加区分,是违宪的。最高法院认定《史密斯法》不能禁止``把暴力推翻政府作为抽象概
念来宣传鼓吹''。因此推翻了十四人的定罪。至此,行政分支借助《史密斯法》的司法起诉,全部停止。即使定罪
的共产党领袖,出狱后照样从事原来政治活动,例如第一批被定罪十一名共产党领袖中的嘎斯
\textperiodcentered 霍尔(Gus Hall),他出狱后继续公开组织领导美国共产党,支持苏联政府,并且四次竞选美
国总统。根据《华盛顿邮报》引用的秘密文件,他接受苏联经费超过两百万美元。

\pagebreak
\section{尾声}

长期坚持为麦卡锡鸣不平的,是保守派中的一部分人。维诺那文件如同给了他们一剂强心针。他们宣称麦卡锡积极
寻求危害国家的真相,本没有什么过错。和他们的对手一样,在潜意识中,为历史辩护也是为自己正名,因为每当
他们观点和自由派相左,常常直接被对方指责为邪恶麦卡锡的继承者,令他们十分郁闷。事实上,维诺那文件公布
后,期待中的``平衡的历史叙述''并没有出现。不论美国内外,知道麦卡锡和知道维诺那计划的人,一定远远不成
比例。美国民众已经不能体会当年的危机,而同情当年受到伤害的无辜,却是人之常情。那么,麦卡锡时代究竟伤
害面有多大?

绝大部分间谍都因缺乏证据没有起诉,其中不少人公开引用``不得强迫自证其罪''的宪法修正案第五条,拒答法庭
问题,避开惩罚。参众两院听证会调查,确使被传来的证人非常难堪,也使他们的隐私权受到侵害,可是调查只是
调查,除极少数人拒答问题以藐视国会罪入狱(一年以下),不论是否承认自己是共产党,都没有被起诉。真正有后
果的是:联邦政府通过内部安全隐患调查,解雇了三百七十八名雇员,在州的层面,各州政府机构含公立学校也有类
似解雇。民间反应比较集中是好莱坞,各电影公司老板私自决定不雇共产党,前后有三百多雇员失去续聘合约,一
些人专业前程被毁。也有一些雇员因为在国会作证被老板解雇。零散事件难以统计。一般估计,美国有一万人左右
在这一时期失去原来的工作。当然,还有《史密斯法》起诉的约一百四十名美共领导人,他们大多坐了牢。虽然也
有争执说,一个听命于外国政府、以暴力推翻美国政府为信条的政党、为外国政府建立了反对自己国家的庞大间谍
网、送出情报涉及核机密,在核战阴影下,动用《史密斯法》对该组织领导人定罪,还是在正常危机反应之内。动
用《史密斯法》有过``扩大化'',有一个非领袖普通党员被定罪,大家为他鸣不平,最后由总统大赦出狱。这些个
人伤害的发生,导致麦卡锡在美国永远会是一个负面概念,大家会说,我们不就是为了建立一个更decency的社会吗?

不过,走出美国,看看其他地区的大清洗、白色恐怖、红色恐怖,麦卡锡时代还是远不能与其比肩而立。(全文完)■

}
\end{document}

