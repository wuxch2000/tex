\documentclass[10pt]{article}
\author{}
\date{}

\usepackage{kindle}
\title{我在美国当大陪审员}

\pagestyle{empty}

\begin{document}
% \pagestyle{fancy}
% \date{}
% \maketitle
\begin{minipage}[t]{\textwidth}
\vspace{1.5in}
\begin{center}
\textit{我在美国当大陪审员}
\end{center}
\end{minipage}
% \pagebreak

% \pagebreak
% \section{我在美国当大陪审员}
\pagebreak
\setcounter{page}{1}
{\small

大家都知道``绿卡持有者''与``美国公民''有一个重要区别:后者有选举权,前者没有。其实,还有另一个重要区别:
公民有但任陪审员的义务。凡年满18岁,无犯罪记录的美国公民,都有义务当陪审员。当陪审员与选举不同的是: 选
举是自愿的,愿不愿投票悉听尊便。但当陪审员是义务,一旦抽中,除非特殊情况(需提出证明),没跑。你休想``蒙混
过关'',否则以违法论处,违者将受到从罚款到被起诉的惩罚。

由于陪审员是由电脑从一个由符合条件的人组成的``后备军''里随机抽中的,跟彩票一样,我的同事们便将被抽中当
陪审员戏称为``中彩''。今年春天,我就``中彩''了。``中''的还是``大彩'':充任大陪审团的陪审员。


收到区最高法院的通知时,我左右为难: 孩子正在上学,每天早上送她去学校后再赶到法庭,无论如何也没法在规定时
间内赶到。我仔细读了一遍通知,发现通知上说明,可以推迟一次。推迟方法很简单,也无需提出推迟的理由,只需拨
指定的电话号码,然后按照录音的指示,将自己可以参加陪审团的日期由电话键盘输入就行了。于是,我将服务的日期
推迟至7月。7月中旬,法院第二次通知寄到。这回可别想再赖了。在指定的那天,我一大早就赶到区最高法院报到。


对于``大陪审团''我一无所知。以前从报纸上看到一些轰动案件审判前检方组成``大陪审团''听证云云,还以为大陪
审团是专为这类大案组织的。 又记得O. J. Simpson案光是挑选陪审员就花了好长时间,以为大陪审团也是如此办理。
有趣的是, 原来被抽中当大陪审员的机率比审判陪审员低, 我的同事们\myrule 全是土生土长的老美\myrule 居然
没有一个当过大陪审员。他们对它的了解跟我一样,一无所知。几天前,同事们还纷纷给我出各种``馊主意'',传授给
我一大堆如何``落选''的经验,从``假装不懂英语''到``声称自己不可能做到100\%客观''等等。到了``报到''那天
才知道,这些全不管用。法警把我们上百口子男女老少带进一间气象恢弘的大厅,也就是法庭审判厅,陪审团召集人
\myrule 法庭书记官把每个人轮流叫到他面前,只问一句话:你当大陪审员的四周内雇主付不付工资?回答YES者当场
``中选'',回答NO者当场打发回家。退休人士和失业者每天由政府付\$40。 幸亏我没听同事的``馊主意'':声称不懂
英语者必须当着全体候选陪审员的面接受法官提问。 要是发现有诈,罚洋\$1,000,外加一个轻罪记录!

``中选''之后,法官宣读有关大陪审员的法律条文,纪律等等, 大家宣誓服从。然后,法官宣布陪审团的正副``头
人'' 。 这也是电脑随机抽的,不想当的可以当场说明。接着,我们就由法警带到各自的陪审团室,当天就开始听案子
了。我们这个地区大概是``犯罪高发区''吧,每月有一期大陪审团,每期3组。每组有一位法警负责日常的行政安排,
记录每人每天是否到场等等。

原来,大陪审团是专听刑事案的。所有刑事案件在进入审判之前得先到大陪审团听证。 某人控告某人涉嫌刑事犯罪,
检察官认为可以立案,即收集各种证据。 但是,检察官不能决定该案件证据是否足够进入审判,这个决定必须由大陪
审团做出。大陪审团由23人组成,其中包括正负``头人''和两位自愿担任的书记。 听证时必须有至少16人在场,听完
后必须12人投票通过。如果通过,该人即被正式起诉, 但这并不意味着他/她有罪。如果案件进入审判,则由另一个陪
审团\myrule 12人组成的审判陪审团\myrule 在听完双方律师的辩论后,决定该人被控的罪名是否成立。也就是说,
一个犯罪嫌疑人要经过两个陪审团, 共35人的听证才能被定罪。而民事案件则不必经过大陪审团这道程序。

我觉得陪审团成员的组成还是挺公平的。我们这组年龄从28岁到70多岁;职业有警察,政府公务员,工程师,家庭主
妇,清洁工,会计师,教师,退休人士等。种族方面,白人是绝对的``少数民族'',23人里只有3位白人,其中还有一位是
第一代爱尔兰移民。其他包括亚裔,西裔,非洲裔,加勒比海地区等。总之,大家都属``芸芸众生''中极普通的人。而
且,没有人有任何正规的法律训练,换句话说,大家全是``法盲''。

一个月内,我们听了30多个案子,范围很广,包括贩毒,拥有毒品,家庭暴力,盗窃,抢劫,贩卖盗版CD/DVD,企图谋杀
(Attempted murder),攻击等。 有些案子,我们不得不看被害人受伤后血淋淋的照片,或者可怕的伤疤(设想一下:一
道伤疤从头顶到下巴, 把受害者的脸划成了两半, 够不够可怕?),让人心里直哆嗦。

令我印象最深的是:

首先, 整个听证过程中对程序一丝不苟,严格得几乎到了``宗教''程度。从检察官进入陪审团室,开口说话之后,整个
听证过程的每句话,包括陪审员提出的问题,都由速记员记录下来,经整理成正式记录后送交法官审查。如有不符合程
序之处,即使陪审团已经通过的案子也将被自动撤销。同一个案子撤销之后,依法不能再起诉。

因此,每个案子,除了检察官对证人的提问内容不同外,程序是一模一样的。有时,不够老练的检察官忘了问出纯属程
序上的问题,后来想起来了,赶紧把证人再叫进来补上。有位年轻的检察官把一个倒霉的证人叫进来3次。一周后,我
们也老练了,检察官要是忘了问某些程序上必须问的问题,连我们这些外行陪审员也能听出来, 直替他/她干着急。

每个案子的听证都是如此这般:

检察官进来,先把写者被告被起诉的罪名的纸条交给陪审团秘书和``头儿'' 然后转身 面对我们:``大陪审团的女士
们先生们,早上/下午好。我是副检察官某某某某. 今天我向你们提交一个新案件,题目是某某州的人民对某某某,大
陪审团号码是......现在,你们将听取证人某某某的证词。''

然后,检查官开门,招证人入内。证人(原告方面)进入证人席,面对``头人'',举手宣誓。顺便说一句,誓词共有7种之
多,其中``标准誓词''有两种,一种最后有 ``愿上帝帮助你'', 另一种则取消了这句,无神论者或者佛教徒可以要求
以这条誓词宣誓。

副检察官开始提问。问题非常简短,回答必须简短,明白,具体,不可以转述第三者的话 ,也不可以猜想。 还得``就事
论事'', 不可东拉西扯与事发之时无关的, 过去的事件。 比方说,当丈夫的暴打妻子, 然后企图说出过去他们``曾
有多少次争吵, 每次都是她挑起''之类,检察官就会立刻打断他的话, 叫他只谈事发时的情况。检察官觉得证人的陈
诉已经清楚了,便宣布不再提问,然后问我们是否有问题。有的案子简单, 大家也就不问了。有的挺糊涂,大家就举手
提问。题问者必须走到副检察官和速记员旁边小声发问,由副检察官决定这个问题是否应该由证人回答。如果他认为
有必要回答的话,就会对证人提出。有时候,副检察官还会问得更深入。证人回答后, 副检察官要问提问者对证人的
回答是否满意。

被告也可以要求向陪审团陈述。但是,被告陈述时必须由他的律师陪同。同时,被告作证也就意味着放弃赦免权,他的
证词可以被检方利用。因此,被告必须宣誓两次,第一次是宣誓放弃赦免权, 第二次才是``标准誓词''。被告陈诉时,他
的律师不得发言,只能小声给被告建议。有一个案子,被告被检察官逼问得``体无完肤'',他的律师连叫``抗议!'',
副检察官不理不睬,那律师也无可奈何。

所有证词听完后,副检察官向我们宣布起诉罪名。宣读法律条文之前还得先宣读我们的判断标准。 到这会儿,我才知
道为什么美国法律条文为何如此之多。且不说50个州的法律都不同, 首先,每项罪状都有不同程度/等级。 比方说,
攻击罪从一级到四级,每级程度不同,惩罚的程度当然也不同。

其次,每个关键性的词都得给出定义。比方说,攻击罪,先是``攻击罪''本身的定义。然后,定义里的主要词如``攻击
性武器'',``伤害''等又另有定义。比如, ``企图攻击罪''这``企图''一词就是一大段定义;每个不同的等级又是一
大段定义。这些定义每次都得读出来,尽管我们已经听得耳朵快起茧了。这些定义都留了一定的``伸缩性'',以便可
以根据不同的情景解释。比方说,在特定的情况下,汽车,尺子,铅笔也可以被定义为``攻击性武器''。

各项罪名的定义宣布完了以后,副检察官问我们对这些条文是否有疑问。没有问题的话,副检察官与速记员退出,隔音
很好的门关紧,我们开始投票。

当然,投票之前我们通常要吵上一阵子。有的人某个细节没听清,有的人对整个事情觉得不明不白。。。,有时整个吵
成一锅粥,要劳``头人''敲着桌子让大家静下来,举手投票。 如果通过,秘书 用红笔在起诉的各项罪名上打上一个
钩,那哥儿们就算被正式起诉了。投票之后,``头人''按铃,副检察官进来拿走那张纸头,一个案子就算听完了。

最麻烦的案子是,很明显两方都没说真话,又没有实在的证据,双方各执一词,而我们得决定哪方的证词可信度较高。
被告当然使尽浑身解数来影响我们的看法,我们呢,有时真的不知该信哪方。到这时我们就真是吵得天翻地复。最后,当
然影响我们决定的因素每个人都不同,大家只能根据自己个人的判断来决定了。 我个人觉得,原告不说真话是不明智
的,理由是,当我们无法决定哪方的可信度较高时,好象不知不觉会偏向被告。因为,一旦被我们投票起诉,那家伙就只
有两个选择:认罪以换取较轻的罪名,或者不认罪接受审判。怎么着他都算倒了大楣。不过,要被大陪审团投票否决并
不是容易的事,除非证据实在不足,或者原告和被告的证词相差太大,原告的可信度受到怀疑,绝大多数案子都能通过。
我们听了30多个案子,才否决了3个。

其次, 我的感觉是, 辩方证人最好还是说真话。一上了证人席, 面对训练有素的检查官,谎言很难滑过去。辩方证人
说谎被检查官抽丝剥茧一般盘问出破绽\myrule 他若有``前科''的话也得给抖落出来,那他想维护的被告就``死''定
了。是否听取辩方证人的证词, 检察官不能决定,必须由我们投票决定。因此,辩方证人前来法庭并不一定就能帮上
忙,极有可能白跑一趟。要是陪审员们决定听取他们的证词,他们不知道的就说不知道也罢了,如果非要编个不同版本
的故事出来,往往适得其反。

有一个案子,我们给``卡''住了,没有确实的证据,只有警方和原告的证词。被告提出一个完全不同的版本,我们无法
决定,便决定听辩方证人的证词。谁知那辩方证人又拿出一个不同的版本。检察官慢条斯理,不慌不忙地反复问他,同
一个问题从不同的角度问上好几次;有时还把同样问题在不同的上下文中重复问上几遍。那证人完全给绕晕了,自相
矛盾得让我们直乐。结果,被告被我们起诉。事后大家一致同意,辩方证人大大帮了个倒忙。当然,如果审判的话,被
告不一定就能被定罪,可他也得破上一笔财请律师,出庭得误工,还搭上不知多少脑细胞。

第三,陪审员与证人或副检察官如果认识,哪怕有过一面之交,或与其亲属认识,都得回避。有一天午饭时间里,我坐在
外边的长椅上看书,有个中国人跟我搭话,请我帮他读一份英文文件。我接过来一看, 是一位副检察官给他的信,要他
于当日出庭。午饭后开始听证,恰好那人就是我们听的那个案子的被告。我立刻向副检察官说明我与此人的谈话,她
马上宣布我必须回避。我们全部的讨论只限于陪审团室内,讨论时绝对不得有任何``外人''在场, 连法警都不得进入。
出了陪审团室,绝不允许谈论案件。这点大家全有共识,出了门就不提案子了。回来上班后,同事们也绝不问我听的是
什么案子。

最后,观察到一个有趣的现象:原来``窝里斗''是``放之四海而皆准''的``普遍真理''。我们听的这些案子,全是``窝
里斗'':白人揍白人,华人劫华人,西裔抢西裔,黑人砍黑人,只有一个专偷名牌货的``蒙面大盗''是``种族平等''的,
他谁的东西都偷我的美国老板说,有一次,他被抽中担任一个案子的陪审员,那案子拖个没完没了,足足折腾了好几个
月,12个人天天见面,大家都成了朋友。期间,陪审员们开了一个派对,庆祝一位陪审员``喜添贵子'',还有一哥们被日
复一日的配审烦透了,于是自己给自己休假,结果,有一天他一进陪审员室,来俩警察,喀嚓一下把他拷走了:陪审员擅
自离开属违法行为。这哥们最终倒是不用来了,但是代价惨重:惨遭罚款不说,还给自己的记录上弄了个``轻罪''。有
了这个记录,7年之内别想找政府工作,对借贷还有影响。

我们这个陪审团没有如此戏剧性的插曲。陪审员生涯的最后一天,一进门,每张椅子上放者一朵玫瑰花,是我们中的一
位买的。我带来照相机,给大家拍照。大家纷纷交换地址电话电子信箱。然后,区检察官亲自到每间陪审团室来,向大
家道谢,感谢大家为美国司法制度做出的努力。我至今记得他说的话:``世界上没有一个司法制度能做到绝对公正,但
是,美国司法制度尽可能做到公正。美国的司法制度之所以能够做到这点,是由于你们的参与和努力。''

担任陪审员的经历,不仅使我对美国的司法制度有了直接的了解,也给了我很好的教育。从那以后,遇到问题,我学会
从法律角度来思考,而不是凭着冲动去做。当然,也真正理解了,什么是``法制'',它的具体内容是什么,以及它的操作
过程。

}
\end{document}

