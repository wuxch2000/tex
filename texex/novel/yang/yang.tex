\documentclass[11pt]{article}
\title{过程改进和专业素质问与答}
\author{杨万强}

\usepackage{config}

\begin{document}

\begin{titlepage}

\begin{minipage}{\textwidth}
% \centering
\vspace{130pt}
\hspace{33pt}{\large \itshape 杨万强的}
\end{minipage}

\begin{center}
\begin{minipage}{.8\textwidth}
\centering
% \vspace{10pt}
\rule{\linewidth}{0.5mm}\\[.4cm]
\makebox[.7\textwidth][s]{{\Huge \sffamily 过程改进~}{\huge \itshape 和}{\Huge \sffamily ~专业素
    质}}\\[.2cm]
\rule{\linewidth}{0.5mm}
\end{minipage}
\end{center}

\begin{minipage}{\textwidth}
\centering
\vspace{50pt}
\makebox[.2\textwidth][s]{\huge \itshape 问与答}
\end{minipage}
\vfill
\end{titlepage}


\thispagestyle{empty}
\begin{minipage}{\textwidth}
\vspace{200pt}

{\centering \itshape \Large {\fontspec[Ligatures=Common]{Adobe Garamond Pro}%
Get busy living or get busy dying. That is goddamn right.} }
\begin{flushright}
---  {\fontspec[Letters=SmallCaps]{Adobe Garamond Pro}%
The Shawshank Redemption
}
\end{flushright}
\end{minipage}

\begin{minipage}{\textwidth}
\vspace{50pt}
{\centering \Large \textit{你们应该扪心自问,到底自己要改变什么程度?}}
\begin{flushright}
---  {\fontspec[Letters=SmallCaps]{Adobe Garamond Pro}%
Man Keung Yeung}
\end{flushright}
\end{minipage}

\pagebreak
\setcounter{page}{1}
\pagenumbering{Roman}
\pagestyle{plain}
{\footnotesize\textsf{\tableofcontents}}

\pagebreak
\setcounter{page}{1}
\pagenumbering{arabic}
% \pagestyle{plain}
\pagestyle{fancy}
\section{问答\footnote{\textsf{Q:员工的问题~~~Y:杨万强的答复}}}
\subsection{需求与设计的继承关系}\label{link1}

\qlogo \rmfamily 在指导项目组编写用户需求说明书、产品需求说明书、概要设计、详细设计过程中,针对这些文
档的相互继承关系和内容的粒度上有困惑。通常来说,模板要求非常细致和详细,也有相关示例,但是实际上如果
完全按照模板的方式写,对于我们目前小规模项目来说,负担太重,几乎沉没于文档编写工作中,重要的市场突破、
战略工作无法有效的开展。

那么,我们应该如何对各方面取得好的平衡,既能保证质量又不牺牲效率呢?


\ylogo \itshape 我们很多事情,都是自己把它变得非常复杂。其实过程改进不是这么复杂的。我最近跟深圳这边
一个单位交流,就是谈你现在提的问题。我们的项目很多是规模满小的。都有同样的疑问。

这个疑问,反映了我们的思维。我们一开始就把CMMI当成一个步骤。因为我们的思维,是``流程''的思维,是表面
的,是形式的。所以是额外的负担。我已经非常强调这个是一个错误的态度。错误,因为这是低效的。

我们应该用``过程''的概念。这包括了每一个步骤后面的原因(不同事务之间的因素)。``流程''只是一系列的步
骤,他限制了,规范了我们的行为。``过程''呢,是一个框架,有优化过程本身的机制,也有帮助员工自己提高的
机制。知道了这些,才能自主地符合优质过程的要求。

但是我们的EPG不明白这个,而且EPG的技术知识,比不上项目的人员,所以他们定义的规程与模板,很多都是多余
的,提供的价值不大。

答案其实在你的问题上:``针对这些文档的相互继承关系和内容的粒度上有困惑。''解决这个问题不在于模板,是
吧?我们要解决了这个困惑,了解了这些继承关系之后,制定出来的规程与模板才可能有意义。否则一定没有意义。

所以我不能直接回答你的问题。我不清楚你们的项目,你们的规程,你们的模板。但是我看过这里的规程、模板,也跟
这里的项目经理谈过。我正在安排基于他们给我的需求文档,来做一个工作坊之类的活动,让大家了解如
何``描述''需求,来体现你说的``文档之间的继承关系''。只有项目经理和系统工程师了解这一点,能够做到这一点,制
定出来的规程与模板才可以有效,才不会如你所言的,``负担太重''。

如论是你的,或是这个单位的,当我们觉得对粒度有困惑,是因为我们没有实践过使用需求来规范产品,和规范下游的
工作。这里是规范,因为下游的工作,是需要与需求一致的。了解了这个要求,通过实践,才能体会需求的最佳粒度。
呢一个粒度,才能规范产品,才能规范开发活动?答案是在系统工程师自己。他们要通过承担责任的态度,追求达到规
范产品与开发活动的目标,这样的思考来培养自己的分析能力,才可以解决你提到的困惑。

但是过程管理是不能不做的,因为过程管理要求考虑的因素,是系统工程师需要考虑的。所以这里其实是很有学问的。
请留意,``过程''里要求的,不一定是员工现在知道的, ``过程''要求的,是这个岗位需要能够做到的,所以是一个提
高的方向。

建议你的步骤如下:
\begin{enumerate}
 \item 规程定义与模板,还是要开始的。

 \item 同时,我们要知道规程的意义,在于提高项目效率。不是每一个项目的每一次效率,是项目的总体效率,和
组织里总体的项目的效率。所以我们要知道,开始的遵从度,只是建立纪律而已。过程定义本身,还是需要优化的。
所以,我们不能拿过程来考核!否则这个优化过程的机会就没有了。

 \item 项目经理,系统工程师,高层经理,是过程改进第一批受影响最大的人。这些岗位上的人,都需要有改变。
你的问题,是系统工程师的改变。

 \item 系统工程师要提高对各个文档之间的继承关系进行了解。这个是知识与技能水平。提高需要一个过程,一段
时间。所以过程改进本身,需要提供这些学习的机会。

 \item 领导或是高层需要要求每一位员工能够符合岗位的要求。(这是 OT 里说明的。)我们不能姑息。
\end{enumerate}

谢谢你的努力,希望你能够成功。


\subsection{需求开发技术的挑战}
\qlogo \rmfamily 需求工程师相关技能的提升,重点是组织需求系列培训,包括需求工程、需求收集、需求分析、
沟通技巧,以及用例技术等培训,这些培训公司都有,最大的问题是如何结合产品的实际来开开展培训?

\ylogo \itshape 我认为我们以前的培训,效率低的原因很多,其中包括:

\begin{enumerate}
\item 急功近利。我们往往希望在很短的时间,提供很大的利益,反而连一定点的好处都达不到。期待越高,失望
  越大。
\item 没有做好策划。目标是什么,成功因素是什么?如何建立、强化这些成功因素,降低失败因素?
\item 成功因素之一是高层的持久支持。请留意``持久''的意义。要看远的,不单单是看近的。
\item 没有配套。如果培训出来的后果,得不到团队其他的人,或是高层的鼓励和支持,很难成功。
\item 没有学习的动力。我们经常激励错误的、消极的因素,而不是激励正面的因素。一个案例,就是要求黑带找项目,而
  不是要求总经理确保过程的效率。另一个案例就是无能的人,常常因为做了些表面的东西就得到奖赏,而有意改进的
  人往往受到压制。
\end{enumerate}

我们要问的是:如果需求做好以后,有人能够判断么?所以有明确的质量准则,直到什么是好的,什么是不好
的,非常重要。然后,得到激励么?我们很多时候是激励了错误的东西。如果这个做不到,培训是没有用的。

如何提高成功的机会?
\begin{enumerate}
\item 我觉得在组织层面,领导要真正地了解需求的重要性:需求是主宰项目一切活动的源头,产品的所有特
  征,功能,统统都必须是由需求来主导的。
\item 需要有一个系统工程师的团队。不单单是形式的。每一个系统工程师都需要明确这个岗位的意义。我遇到过
  这样的问答(我们不需要这样的系统工程师!):
  \begin{verse}
    问:``你的岗位是?''

    系统工程师:``系统工程师!''

    问:``你的职责是什么?''

    系统工程师:``我写需求文档!''

    问:``很好。那么,你们项目的困难在哪里?''

    系统工程师:``没有人告诉我有什么需求!''
  \end{verse}
\item 明确让系统工程师知道对他们的期待,有哪些职责和权限,提供他们成功的条件。
\item 要求项目按需求开发产品。如果需求内容不好,就先把需求弄好,再开发。就是说,要知道开发什么,才开
  始开发(CMMI RM SG1)!只有这样,才可以提升系统工程师的水平。
\item 提供有效的培训。并且激励实施培训的内容。
\item 提供了解市场、客户、产品技术、产品应用等领域的方便,并要求他们成为这方面的专家。
\end{enumerate}


\qlogo \rmfamily 如何加强需求的复用,各个项目,包括不同的制式的产品,个人觉得需求不应该有很大的差异,因为
基本功能模块相同,基本需求就应该相同,差异有,但不应该占很大的比重,只是需求功能点数量和UI的区别,最好能有
一个标准需求库,项目组在库中DIY?

\ylogo \itshape 需求可以复用。其实不一定是``复''用,能够参考好的需求,裁剪全面的需求已经是很有价值的。 需
求和UI,在软件需求(适合开发的需求)来讲,是相互相承的。如果我们开始的时候,只有功能需求,我们就需要明确操作
场景和步骤。这是CMMI的要求。如果做了这个,UI的需求就明确了。功能和UI都是软件需求的一部分。

请留意,软件的复用,就期待是完全不需要修改的!(为什么?这个满有趣的。)

\qlogo \rmfamily 如何加强产品质量属性需求的开发和在设计中分配,这部分需求需要从事业部层面,以平台的方式
开展,因为项目组关注短期利益,周期短,人手紧,不能深入开展此项工作?

\ylogo \itshape 我希望大家理解``周期短,人手紧'',不是``不能深入开展此项工作''的原因。如果我们这样考
虑,我们是永远都不会进步的。周期短,人手紧,都是资源分配的问题。我们需要学会如何这样做。比如先一边外判,一边
训练。还有很多很多其他的方法。但是我们偏偏要说这个方法不成,那个不可以。那么我们肯定是没有进步的。我们需要
拿出决心来。

大家还记得我要求大家拿最低限度5\%的时间来提升自己么?不能这样做,是``分配''做的不好,不是不能够。现在我们
的管理水平,就是没有到有效分配的程度。

我的儿子是学钢琴的。有一次,你非常骄傲的告诉我:``我可以所有的手指都接触着琴键,同时还可以弹出调子来!''儿
子的能力,就是在于力量的分配和控制。只有有了这个能力的,才能够弹得出好音乐。

我们在管理上也需要能够把资源有效分配和控制,就是说,``短期利益,周期短,人手紧'',就把100\%的资源放到赶周
期,是低效的。我们需要只用最多90\%(比方)的资源在赶周期上。其他10\%作长远工作,只有非常非常紧迫的情况才调
用。这样调用的情况,一年也不应该发生多与一、两次。而且也只能是短期的。否则就是滥用了。

在这里再提一个例子。美国经常有罢工。罢工的时候,研发人员也得分派到生产线。这就是调动``100\%''来处理紧急事
件。企业不能接受太多的罢工。平常研发人员是不会做生产线的作业的。生产线不能说,市场紧,研发调人过来吧!研发
在企业,就是我们的5\%自修、10\%长远、等等,是轻易不能动的。

需求里的功能、属性和性能,如何分配到模块,道理也是类似的。这里我希望强调两点:
\begin{enumerate}
\item 分配是有原则的。这些原则包括内聚、耦合(功能方面)、历史数据(性能方面)等等。
\item 最佳的分配,几乎一定是通过几个改动更新的。越牛的人,不是越可能一次到位,反而越会做两三个方案。
  这就是CMMI TS 里的要求。
\end{enumerate}
希望大家留意。

\qlogo \rmfamily 性能需求的开发,性能指标在设计中的分配和测试手段的提升也急待改进?

\ylogo \itshape 提升需要时间。决心是否具备?才是主要的问题。有了决心,就开始考虑如何支持这个能力。这
  些考虑周边的因素的,才是成功的关键。就好像是一个飞机场,如果机场建设本身,达到一个质量水平之后,成
  功与否,从市区到机场的交通配套,远远比跑道和机楼更重要。

\qlogo \rmfamily 理顺需求、UI设计和测试用例的关系,如何减少重复性工作,增强需求在项目中的一致性,做
  好需求跟踪和追踪,当前最大的问题,不论是借助工具还是手工完成,需求追踪的工作量都很大,似乎可执行性
  不强,如何处理?

\ylogo \itshape 需求跟踪的方法,我跟你们技术部提过了。是非常简单的,但是需要开发一个工具。当时的部
  长就觉得非常好,要无论如何都要把工具开发出来。但是现在又旧事重提了。

我们真的是非常``可重复''的!

\subsection{谁的目标驱动 与 总体的项目管理}

\qlogo \rmfamily ``需要目标驱动,不能够只关注步骤''具体指的是什么含义?我们现在的项目应当在各个阶段
  目标还是比较明确的,大家工作就是为了该目标,这是否可以认为是 目标驱动?我的理解是:我们现在太过于
  关注目标,为了达到此目标,好多流程步骤都忽略了;我自己感觉是:步骤关注不够,不知道理解是否正确?

\ylogo \itshape ``需要目标驱动,不能够单单关注步骤''指得是员工个人。每一位员工,都需要知道事情的``为
  什么要这样做?''而不是单单的,``领导要我做什么,我就做什么!''这就是说,领导要解释,要指导,要关
  心,而不是单单的发命令(指派任务)。

员工呢,就接到任务的时候,不单单要考虑怎样做,还需要考虑为什么这个任务重要?

如果能够这样,项目的效率会提高很多。

\qlogo \rmfamily ``项目需要从监控个别任务事情转化为项目总体的控制和目标管理''指得是什么?我们的项目
  目前每月的项目月报,包括项目周报 对 项目的关键任务,质量状况,风险情况,资源投入 等都有一个整体性的
  描述,您的意思是我们目前关注的范围不够 还是 关注的深度不够? 

\ylogo \itshape ``项目需要从监控个别任务事情转化为项目总体的控制和目标管理。''是指现在的项目管
理,基本上是任务监控。而且,每一个任务,没有在整个项目的比重、分量等有什么定位。这样的监控,不能对
比现在实际情况跟策划的真正差异。有时候,每周的任务,是否和项目的总体计划符合也没有关联起来,甚至不
可以确定。而且项目经理的工作量很大,很累。我看到的报告,都是一大堆表,里面满是``信息'',都是很难连
起来,提供一个总体项目状态的。应该多用一些恰当的图,显示趋势,前因后果,能够有预测能力的图。一目了
然,如果有恰当的工具,工作量也不大。

我们应该跟踪的是项目的总体状况,由几个主要的项目参数代表。而不是一系列的任务。这些我看已经在PMBOK里是
说的清楚的,只是我们没有用起来。

我真的非常希望有一两个项目,自愿试一下这些,才可以知道它们之间的分别。

\subsection{不清楚的需求}\label{link6}

\begin{staff}
\qlogo 心中仍有些未能想通的问题:我们经常遇到前期需求就不清楚,但是非要开发,我们称之为``酌情开发'',这
  样的情况已经不是一次两次 了,作为开发方面,想要拒绝又无能为力;比如:
\begin{enumerate}
\item 局方也不清楚需求,我要``一个机柜,长10米,宽5米,高5米'',至于颜色、是否有门,门的长宽高等
  等,局方也不清楚,你们先做一个出来,最后的结果往往是临到上线或者割接了,才告诉我们,匆忙修改;

\item 现场需求调研不清楚,我要``一个机柜,长10米,宽5米,高5米,天蓝
  色,有门,门长1米,高0。5米,厚0。2米'',颜色是涂还是喷,门的位置,钢板的厚度等等,现在搞不清楚,你们
  先做一个出来,最后还是临到上线或者割接匆忙修改;

\item 需求调研清楚了,2天就要,摆出项目种种厉害关系,这种情况我们只有加班加点匆忙开发;
\end{enumerate}

在时间很紧的情况下,匆忙开发往往会导致BUG泄漏,我们也采取了一些措施(比如现有数据做故障泄漏分析采取相应
措施),但是收效甚微,而且对士气影响很大;

就现有情况而言,我们只能从自己做起,如何能够提高开发质量,特别是在时间紧迫的情况下!
\end{staff}

\begin{yang}
\ylogo 非常感谢你的心得。从你的心得里,我可以知道你的确是留心的,听到一些有用的东西。
这个很好,我非常高兴。

你也有一些问题。希望在这里可以对你找答案有一点帮助。我是不能给你答案的。答案是要你自己找的。

你的问题,主要是面临这么多的实际困难,看不出能够有机会来实施听到的一些东西。
如果是这样,我是非常理解的,很多同志都有这个情况。问题是,在这样的情况,如何处理。

我们最大的问题,也的确是如何从实际的情况下,打破这个恶性循环。比如:

\begin{staff}
  ``局方也不清楚需求,我要``一个机柜,长10米,宽5米,高5米'',至于颜色、是否有门,门的长宽高等等,局方也不清
  楚,你们先做一个出来,最后的结果往往是临到上线或者割接了,才告诉我们,匆忙修改;''
\end{staff}

提议:我会采取两个举措,第一,我会尽可能去了解他们现在的情况:他们的机柜回访那里,如何放,旁边有什么其
它的东西,谁负责这个机柜的管理和维护工作,更他们交流,等等。其中我会告诉他们,最后可能在他们要来我们
场地验收,或是在他们的场地喷漆。这些信息,可以帮助我们接近他们的真实需要。

同时,我们要知道我们的商业利益和权利。

不要告诉我他们不愿意见你或是跟你谈。你需要用种种方法接触他们,了解他们。让后利用你对他们的最大最清楚
的了解程度,心目中提供两、三个他们最可能需要的机柜特征组合,来解决这个问题(制定他们的需求)。

另一方面,我同时会设计一个最简单的,长10米,宽5米,高5米的机柜,只有底漆,没有门。其中布线、空调的安
装方法,要找不会跟张来的门有冲突的种种布线、设计方法。

然后邀请他们来我们的场地检查开发的进度和验收。

请留意,我们只应该做我们能够保护自己的权益的交易。但是我们不能害怕在追求自己的权益过程之中,尽自己最
大的努力。这个努力,通常都是我们以前没有做的,或是不习惯的。我们会有一个不愿意去做的倾向。我们会
说,``这个工作量很大!''在竞争激烈的市场,这是会失败的,是很可惜的。

\begin{staff}
``需求调研清楚了,2天就要,摆出项目种种厉害关系,这种情况我们只有加班加点匆忙开发;''
\end{staff}

石头是榨不出汁的。领导的话,无论对错,也得听从的。所以我们需要加班,而且按时交付的机会也不大。质量可
能也有问题。这些我们都要告诉领导,让他做决定。

我们不能有意气。比如,如果我们加班,就可以做到。但是我们其实是不愿加班而已。这样不愿意做,就不应该。
心中有气,也于事无补。

我们还是专注利用历史数据,清楚了解自己的能力,再用证据来跟领导沟通。请留意,我们的历史数据,可能证明
领导的不合理,但可能反过来证明我们低估自己的能力也说不定!

最后一个因素,就是在不断加班的情况下,我们的确可以完工。但我们不高兴,因为这样我们没有喘息的机会。我
的意见就是,请不要烦恼。自己一定要为自己留一定的休息、修养的机会。这个你要自己坚持的。如果这样,你的
历史数据就会反映这个现实。否则就反映另一个现实。

我们不能够做出自己不满意的产品,就是领导要你这样做。我们不能做对不起客户的事。就是这样做会有所延误。因
为这样会有长远的损害。我们要做到能够掌握这个判断。看美国的电影知道,美国军方有一个规定,在某种情况地下,士
兵可以不服从军官的命令。我们的心底,也应该有这么一个底线。

所以,在实践紧迫的时候,我要用历史数据说服自己,最有效率的方法,就是第一次把事情做好、做对!千万不要
因为领导无能,或是客户无能,你就做一个无能的员工!

你还需要把听到的,用到实际工作中。因为你问题的答案,恰恰就是我当天告诉你们的。答案你们已经知道了。
我们还是倾向于诉苦多于找出路。请小心思考这个问题。

谢谢你的支持努力。
\end{yang}

\subsection{需求开发}

\qlogo \begin{staff}
如何评价一个项目工作量,我们是根据用户需求的数量还是根据产品需求的数量?
\end{staff}

\begin{yang}

  \ylogo 从一个开发项目的角度来看,项目的工作量,是产品需求的数量。这个数量,跟项目团队的成熟度有关。 成熟
  的团队,产品需求粒度可能比较一致。所以不同的团队,同一个用户需求数,可能开发出不同的产品需求数。所以同一
  个用户需求,开发出来的产品本身,可能也不一致。大家都可能具备基本的功能,但可靠性、易用性、可维护性、兼
  容性,等等就会非常不同。产品给用户的感觉就非常不一样。他们的满意度,就会不同。这就是为什么成熟的团队更有
  竞争力的原因。

  当我们的水平提高到一定的程度之后,就会有兴趣研究这些关系与比例。现在言之过早。现在我们需要建立我们的纪
  律性,接受高效的管理理念。通过有效的管理,才能在上面的技术领域有效提高。我们现在缺的,是``按计划办
  事'',``按需求开发''这些基本得要命的纪律!
\end{yang}


\qlogo \begin{staff}
  如何诱导项目重视需求开发过程?
\end{staff}

\begin{yang}

  \ylogo 很难。我现在正在尝试着做。我只能想到,用实例来说明。但实例也很难找,需要统计分析。我没有足够可信的
  数据。

  在现实世界里,改进成功,都是领导``相信''这条道路,用行政手段,加上领导的知识、才干,才可以成功地。 中兴的
  最高领导,和一大部分的中层领导,其实是不明白如何改进的。我们还不断地说不改变中兴的文化! 不改很多中兴的
  文化,我不知道怎样可以有所改进?

  实际上,我们很多同志都反映,听课的时候什么都清楚,回到做的时候就处处碰壁。我们的领导,以为向下属要求一个
  结果,比如:需求要稳定!需求就会稳定。问题不是在下属。问题是在领导,需要考虑需求稳定的因素是哪些。领导自己
  要知道如何控制这些因素。这样才会有效。我们现在领导不单单不知道这些因素,反而不断地保护那些防碍进步的因
  素,不让员工克服它们。比如:基线的概念;原始需求的概念。在这个情况底下,员工只能自己争取机会提高自己。这
  样就比较难。

  我们还是相信运动形式的方式。这样效率很低,改进的时间很长,而且不知道改成什么样子。
\end{yang}


\qlogo \begin{staff}
  外部需求(不是项目组进行分析处理的需求,如产品线对项目的需求)是否可以不做项目需求分析开发过程就可以进
  入方案及实现阶段?
\end{staff}

\begin{yang}

  \ylogo 这个问题的``任务''与``形式''成分高的要命。我假设``外部需求''是可以开发层面的需求。从任务的层
  面,我很容易可以回答说,``对''! 但是在项目里,这样做有用么?不是项目自己开发的需求,大家的了解是绝对美满
  的么?所以我们还需要有一个了解的过程。然后要和需求提供者确认这个了解是否正确。这个过程可能,甚至应该,包
  括技术方案的概念。 然后用这个概念,和客户或是需求提供者确认。你叫这个``需求分析''么?

  需求分析是存在于很多很多层次上的。是在每一个步骤都需要做一点点的。我们不要单单图节省工作量。作为一个系
  统工程师,我们是不会想到``我们是否不需要做分析''?我们只会想到如何有效地``按需求开发'',其中一定需要了
  解,需要分析,我们就分析。这样才是目标驱动。我说的清楚么?
\end{yang}

\subsection{GUI需求}

\begin{staff}

\slogo 对于GUI程序
\begin{enumerate}
  \item 软件需求描述中是否需要 描述到具体的界面?

  \item 如果需要描述,怎么描述? 界面是最易变的,如果界面变了,如何更新软件需求中的描述? 我想如果在
需求描述中界面写的很细的话,后面要有变更的话,那可就累了。

  \item 界面原型的开发 是属于需求阶段的工作,还是设计阶段

  \item 涉及到界面,肯定就有易用性。易用性也是一种需求。易用性需求该如何描述,是否需要涉及到具体的界
面这一层次? 如果不涉及,该如何描述。
\end{enumerate}
\end{staff}

\begin{yang}

\ylogo 谢谢您的问题。耽误了很久,不好意思。你的答案,简单地,就是:
\begin{enumerate}
  \item 界面有两个意义:功能性的,风格性的。大致上,我们知道功能性的,如有哪些字段,
   每一个字段的内容,等等,这是需求,一定要在需求文档里描述。这个没有异议。

   风格方面,经过多年的争论,也已经有定论,也是需求。所以是需求的一部分。

   但是需求文档里,是否需要描述具体的界面?其实是需要的。我以前使用图片描述的。
   我们要按自己的需要,比如,我们是否有特殊做界面的人,(如果有,通常就有一个
   独立的界面描述文档。)或是,是否有一个完整的界面风格规范标准。如果有,就可以
   在需求里连接,如果没有,就可以在需求文档里把界面的图片贴上。等等。

  \item 如何描述,上面也有谈到。是否需要经常改动,这个是水平问题。我们没有写需求的习惯
   之前,也是怕需求会经常变。我们现在改了代码,还是不愿意改需求和设计文档的。因为
   我们还没有成熟到那个程度。(所以我很害怕听见领导说我们是四级、五级的企业!
   其实我们很多时候连二级都不如!)

   是的,我们的水平高了,改动的机会会减少。开始的时候是需要改的多一点。这是一个成熟
   的过程,而不是一个是否需要描述的因素。描述是一定需要的。

  \item 原形可以在很多个阶段使用。

   这个问题,在我们的情况问的很多。因为我们是任务驱动的。我们只是看一个任务的工作量
   的大小,不看任务的作用,意义,目的。所以我们看不到同一个活动(原形),其实可以
   有好几种应用的地方。所以要问这个问题。但是如果知道原形的用处,就明白原形可以用于交流讨论,
   这样,如果跟客户交流,就是需求的抽取,或是需求的确认。如果在系统工程师和开发人员
   之间进行,就是需求的评审。如果在设计过程中,工程师自己建立原形,这样通常就是可行
   性研究,或是对不同的方案进行比较。

   我非常期待大家可以考虑深一层的意义,不要单单看表面的东西。我们要问事情的目的,后果,
   意义,不要单单问怎样做。要目标驱动,不要任务驱动。这样才能变得能干!

  \item 通过界面和菜单结构这两个方面,就是很好的易用性需求表达形式了。对于易用性的原则,
   以前也有讨论过。你可以参考其他的资料。

\end{enumerate}
谢谢您的支持与努力。
\end{yang}

\subsection{控制不良成本,QA的任务}
\qlogo \begin{staff}
您认为要使得一个研发项目完全按照CMMI模式运行,至少需要有多少专职人员来进行?每个过程的执行是
  否可以项目组成员兼职?是否可以由项目管理人员负责组织执行?
\end{staff}

\begin{yang}
  \ylogo 这个很难说。如果是一个项目,那么,就是项目本来的项目成员就够了。在项目里做CMMI,重要的,是了
  解CMMI这个模型的要求。更重要的,是这个模型的目标是提高效率。它的要求,都是为提高效率而作的。所以项
  解高效的因素。CMMI里提了,我们不相信。就是相信了,也要一段时间来养成这些高校的习惯。要我们该习惯很
  目成员需要知道模型的要求,和项目效率的关联性。这不是简单的事情。我们的项目效率不高,是因为我们不了
  难。这是困难的地方。

  我们的误区,就是认为执行过程改进是项目之外的人做的。这个是非常错误的。规程改进,就是要改进项目的效
  率。所以每一个项目的人都要``执行''CMMI里要求的东西!

  如果您的问题,是在组织里,要多少人推广CMMI。那么,答案是要一个EPG小组。最低是两、三个有经验的人,人
  数少,推广的时间也可能长一点。但是,项目也需要时间吸收,所以EPG组的人数可以小一些。
\end{yang}

\qlogo \begin{staff}
CMMI过程中谈到了成本控制,您以前的经验是如何监控不良成本呢?有没有什么好办法,或者是
否需要专人来专门统计各种过程中的不良成本?
\end{staff}

\begin{yang}
  \ylogo 研发的成本控制,主要是减少返工。这个主要是一个坚持以``第一次就把事情做好''的态度做事。在研发
  过程的角度看,就是尽量早发现问题。这个需要远见。我们的文化不利于这样做。我们也需要有监控预防举措的有
  效性的能力。我们现在没有这个能力。所以觉得很难,或是这样做没有效果。但是如果我们可以看得远一点,同时
  提高我们的监控能力,直到如何判断预防措施的有效性,我们就可以减少返工,提高效率,控制成本。

  CMMI模型里提到的,都是有效的预防措施。我们需要看到这一点,改变我们的习惯,才可以有效提高效率,减低
  成本。

  如果是按CMMI办事,那么,项目里很多项目经理和其他决策,都要用CMMI的价值观来做。在没有接受CMMI的价值
  之前,我们的决策是低效的。比如,我们认为评审是无用的。认为自己已经充分了解情况,不需要用数据了。等
  等。这些都会防碍进步,破坏效率的。这样就不可能降低成本了。
\end{yang}

\qlogo \begin{staff}
您认为过程优化是否需要引入人因工程? 如果不引入人因工程,那么如何进行过程优化?
\end{staff}

\begin{yang}
  \ylogo 人因是一个工程的概念。但是人是一个过程的非常重要的因素。推过程改进,需要被大家接受。从这个角
  度来看,人的因素是重要的。如何有效推广,就是一个广义的人因工程。一般来讲,我们不会这样提。

  但是这个人因工程,和产品的人因工程不一样。产品的人因工程,是如何让产品直观,容易使用。两者都有人的
  行为学和心理学的因素。但是应用起来有非常不同的地方。
\end{yang}

\qlogo\begin{staff}
您认为QA的职责是什么?
\end{staff}

\begin{yang}
  \ylogo 首先,QA是质量保证人员,是过程管理的一个部分,是过程体系里的警察。过程管理还有EPG的成员,负
  责定义和推广过程。EPG里面应该有度量人员,监控过程的效能。

  我们有一个很大的误区,就是要求QA做项目的事情。这是不对的。其实项目成员,尤其是项目经理,需要自己了
  解如何最佳地开展项目的活动,包括分析项目的数据,以监控项目的状态和调整项目的进度。这些都是项目本分
  的工作。项目经理也不能完全依赖QA的指导。QA对项目提供的``指导''只能是偶尔的``提醒''作用而已。总的来
  说,项目经理要接受CMMI,过程改进才可以有效。不这样做,是``姑息'',这是我们的管理文化的一个很大很大
  的弱点。

  QA的工作包括:
  \begin{enumerate}
    \item 按项目计划,制定QA计划。
    \item 根据质量保证计划,实施项目审核和评审活动,输出项目审核、不合格单、项目情况汇总等相关报告。
    \item 充分和项目解释不符合项,并跟踪直到关闭。
    \item 向高层汇报项目的过程状况,包括不符合项的关闭情况。
    \item 向EPG反映过程实施的问题、情况、和强项,以利过程有效优化。
    \item 充分了解、掌握当时有效的标准规程。协助EPG推广标准规程。
  \end{enumerate}
\end{yang}

\subsection{项目的基本测量}

\qlogo \begin{staff}
我们当前衡量研发效率的指标有哪些呢?
\end{staff}

\begin{yang}
  \ylogo 你的问题是指公司的考核指标呢,还是过程有效性的指标呢? 我不清楚考核的指标。我就按过程效率的指
  标来谈一下吧。

  项目的效率,主要是三个方面:周期、生产率和产品质量。我们需要三个一起分析,因为它们是有关联的。否则如
  果单单关注其中一个,在我们的情况来说,主要是``周期'',那么,我们是不可能知道追求进度追过头是如何影
  响效率和成本的,就不可以知道那一点是最佳的平衡点。能够有效做出平衡,才是成熟的特征。

  我们对测量生产率和产品质量有困难。这些指标,需要有产品的规模为基础度量。这个在软件来说,最容易的就
  是代码行。这个是受语言影响的。硬件的项目就很难说了。本来可以用功能点。但是这个在业界的使用效果也不
  是十分理想。我现在提议的就是需求项。如果我们把每一个需要验证的因素,都写成一个需求项的话,我们的产
  品就有很大量的需求项,需求粒度就很小,同时平均的需求相遇实际规模的差异就变得相对稳定。那么,需求项
  的数目,就可以用来作为产品规模来用。这个应该准确并且容易统计,但是我们需要有规范的需求管理,包括按
  需求开发的纪律。

  产品质量,如果要有意义,需要有现场的数据作为参考基准(reference),因为现场才是真正体现质量的情况。在
  开发过程中,我们只能用发现的缺陷密度,但是这个受工作产品本身的潜在质量和验证人员的效率影响,所以需
  要现场的参考基准,作为一个真实的最终缺陷数的依据。

  在项目、部门这样的团队层面,生产率应该是项目(产品、版本)总体的,不是各个任务独立的绝对值。总体
  的,就是投入的,可以使用的工作量,可以是人-日,或是人-月。绝对的,就是实实在在地在工作的时间。这个
  只能在统计个别任务的工作量上使用。但还是需要了解它的限制,因为我们在处理多个任务的过程中,一定会有
  空闲或是形式的活动,所以绝对生产率往往是大于用总体工作量算出来的实际生产率的。(我把这个总体工作
  量,称为``成本工作量'')

  部门的指标,就麻烦多了,尤其是一个员工从事好几个项目的任务,那么规模,工作量等等,都变得非常复杂。

  在低(细)一个层面,在过程的层面,测量指标就多得多了,比如缺陷的发现分布,需求稳定性,缺陷泄漏率,等
  等一大堆。如果有兴趣,可以参考:软件工程技术丛书,之 软件度量系列 的 《实用软件度量》由吴超英、廖彬
  山翻译。或是可以再来邮件详细再谈。

  对于考核指标和改进指标的分别,是在于使用的方法。考核用的数据,是一个体现值(a single instantiated
  value)所以是受当时的不同情况影响的,它的值,不足以正确代表实际业绩,所以不需要比较两个数据的大
  小,因为当时的环境因素不一样,是不应该比较的。用来改进时考虑的是从统计的角度分析这个数据是否合理。
  绝对值不重要,只要把它管理、控制在在合理的范围(过程的声音)内,就可以了。

  谢谢你的支持。如果有任何需要,请随时通知我。
\end{yang}

\subsection{估算的意义和难处}
\qlogo \begin{staff}
请杨老师给我提几个估算的要求或者说注意事项,谢谢!
\end{staff}

\begin{yang}
  \ylogo 估算的目的,在于提供一个合理的策划基础。我们希望他是准确的。但是估算是没有发生的事情,不是一
  个自然规律,是一个指导,所以估算不能``准确'',只能``合理''。合理,就是可以按照他来开展工作,在我们
  的条件不变的情况,按估算完成的机会会比较高,如此而已。我们在思维上一部登天,要求估算准确,这是不可
  能的。

  但是,要估算有效,需要``尊重''估算。这个我们觉得很难。其实这个不是很难的,我们经常需要这样做。比如
  计划,我们有了计划,是否就全心全意地按计划去做呢?有些人是,结果没有延误,就很自豪。这个很好。但是我
  们知道这个计划是一开始就制定的好么?如果计划太紧,或是太松,都按时完成,都一样好么?我们通常以为其
  实``尊重''计划,就是尽力地让他按计划完成。太早、太晚都不好。这个概念,在TL9000说的很清楚,但是我们
  的价值观跟她还有一段距离。

  说了这么多,就要表达一个意念,我们虽然力求估算的准确,但是这个没有``尊重估算'',立意要用它来做计
  划,来得重要。请留意,估算是用来做计划的。计划才是用来指导活动的。我们经常不理解这一点,把追求估算
  准确作为目标,所以觉得很难。其实这个也是任务(把估算做准)驱动和目标(为什么估算要准)驱动的分别。

  估算的方法,的确是帮助我们力求准确。但是这个准确度,不是很精细的,做出来的计划有指导意义就好了。那
  么,方法就不需要太关注了,反而对估算的尊重和承诺更重要。

  还得谈一下估算方法:无论哪一个方法,估算都是依靠经验。历史数据是经验,宽带Delphi也是靠经验,就是功能
  点还是靠经验。问题是,如何尽量除掉主观因素、个人特征,让估算的结果,可以用来制定计划,适合整个项目
  团队,提供一个合理的开展工作的指导?以下是不同方法的不同途径:

\begin{enumerate}
\item 宽带Delphi是通过三四个人,用2到4次的迭代,提高对任务的了解,让大家逐步把主观因素除掉,达到一个
  合理的估算值。
\item 使用历史数据,需要把以前的参数值,按任务的类型、规模、整理好。使用的时候,对比当前的任务,找一
  个最接近的参数值,作一些调整,以适合当前任务的规模和特征。
\item 三点法,是利用三个机会(最乐观、最可能、最悲观),调整不同的倾向,强制估算能够达到一定的合理性。
  这个方法,使效率比较低的,因为它还是以来一个人的意见,没有机会排除主观误解。如果几个人一起做,就接
  近宽带Delphi了。但是没有宽带Delphi方法那样,可以中和不同的人,对任务的不同的认识。
\end{enumerate}

再多加一句,让领导接受他们的估算,让员工愿意承诺。让大家都知道,估算需要合理,要力求准确,但不一定需
要正确。目的是项目的效益,所以不能拿来考核员工个人。
\end{yang}

\qlogo \begin{staff}
``源千行代码''这个数据很难估准确,所以我就担心这个估算的质量。
\end{staff}

\begin{yang}
  \ylogo ``难''不是一个理由。难就可以不做,就是``姑息''。姑息会妨碍进步。开始的时候一定难。难就要
  学,就要把难变成不难。其实他们是怕不准确,在考核的时候受害。你看,这样考核的贻害,就是妨碍进步。

  业界大部分都用源代码行。这个不是``准确''的规模度量,但是是最广泛使用的。人家可以用,为什么我们不可
  以用?

  我以前提到的``需求项数'',是另外一个好方法。但是需要把需求水平提高。我们还是应该开始使用它。这样会
  有利提高需求描述的水平,只是现在收集的规模数据,可能不可以直接跟将来需求项粒度稳定之后的规模数据比
  较。

  如果需求的描述,可以做到每个需求项,都只有一个需要测试的因素,我们就得到稳定的需求粒度。那么,使用
  需求项数,就有望建立同类项目之间的横向可比性了。
\end{yang}

\subsection{产品的配置管理}
\qlogo \begin{staff}
  CM大家都对其有较为深刻的认识和理解,但是目前的CM基本上都是建立于项目级别的配置管理,尤其目前我们实
  际按产品+主版本立项目进行开发的情况下,完善和规范产品级的配置管理意义非常大,不知杨老师那里有无最新
  的关于按产品进行配置管理的方法和理论的最新资料和建议,给我们参考。
\end{staff}

\begin{yang}
  \ylogo 方法都是一样的。只是配置项不一样,因为管理的,已经不是产品的组成部件,而是产品系列的``组成部
  件''。其实我是不应该用``组成部件''来谈产品系列的概念。要管理一系列的产品,我们关注的,是由哪几个版
  本,每一个版本的功能、性能的分配,和他们的兼容程度。

  另外一个不同,就是生命周期,比产品研发长好多。一些版本、需求等等,它们的状态可以是:可行性、研发、确
  认、发布、应市、终止生产、终止维护、退市,等等。

  我提议你好好地跟产总们谈一下,他们面临的问题是什么。他们到底要管理什么? 然后再考虑EPG的工作在这方面
  如何开展。
\end{yang}

\qlogo \begin{staff}
  现在小组在进行流程讨论时有关需求变更关闭有不同的意见,一部分人员认为需求变更修改了相应的需求项并且
  到最后代码实现验证后才关闭,这样变更请求的关闭率和时间将比较长,但部分人员认为需求变更修改了相应的
  需求项评审完成基线化后就应该关闭。这点导致我们和故障的变更的流程的思考存在不同。

  希望杨老师给出指导,谢谢!

\end{staff}

\begin{yang}

  \ylogo 这个问题你们应该有能力去解决的。我这样说,不是想把你吓跑了,以后不再问这个问题了。我还是欢迎
  你们问。但是我们也要好好地学。要不然,是你自己损失了提高的机会。(我鼓励你分析一下,为什么你不明白这
  个问题的解决方法!)

  为什么我们要控制变更?其中一个原因是因为我们要让项目对目前如何开展工作的依据有一致的理解。无论什
  么,改了,就需要立刻``关闭'',否则员工到CC检出文档、资料时,拿到的,就不是新的内容了。这就会有很多
  混乱了,是吧?所以需求改了,评审完了,就要尽快``基线'',让它可以生效。我们万万不能等到代码验证了
  才``基线''。

  请留意,一个文档其实是需要在``被评审''之前``基线''的,因为,评审的版本,也是需要受控的! 这个我们很
  不了解。(你分析一下为什么?)

   那么,你还会问以下的两个问题:
\begin{enumerate}
    \item 如何知道需求已经实施了?

      这个很容易,我们有追踪表,上面每一个需求都有对应的代码。需求可以由状态。对应的代码也可以用状态
      来描述,如``自测通过'',``集成成功'',``系统验证''等等。
    \item 如果一个设计的困难,导致需求与代码都需要变更,因为这都是由不同的人负责的,如何保证(监控)他
      们的处理过程和状态? 

      这个就是``父子''变更请求单的作用。

      当一个设计的变更请求提交之后,CCB就发现需要该需求和代码,就应该把这个变更请求分成三个子变更单:
      一个是需求的,一个是设计的,一个是代码的。他们都是用一个``父''变更请求单的``后代''。三个子变更
      单都独立处理,独立检入,独立基线,独立关闭。但是那个``父''变更单,就要等到三个都关闭了,才关闭。
      这个功能,有些工具是支持的。如果工具不支持,也可以在CCB的评语字段里,这样注明,就可以了。
      查起来麻烦一点,但也是可行的。

      请你把这几个概念弄清楚:检入、基线、关闭。不同的工具,可能有不同的意义。我们要的,是管理好变更,
      这些概念的应用,需要清楚,才能够达到管理的目标。(你回答以下,管理的目标是啥?)
\end{enumerate}

谢谢你的支持和努力。希望你能更加目标驱动。
\end{yang}

\subsection{基线之前的缺陷是否需要统计}
\qlogo \begin{staff}
你说:我的习惯是,重点不是``未基线''而是``配置项''(这个概念要抓紧!)。不是配置项,它的变更是不受控
的。但是未基线的配置项,如果需要修改,还是需要记录在CQ单里。

这好像提到另一种处理思路,工作产品从无到有也需要有变更触发,即哪怕该工作产品完成的过程中其输入并没
有受到变更的影响,该工作产品包含的完全是规划的内容,在做工作产品第一次基线的时候也需要有变更触发。我
们目前没有这样操作,还是认为规划的内容以及工作产品基线之前的内容不算作对工作产品的变更。但您的意见也
有道理,哪怕后端的工作产品还未基线,但是前端进行了变更,后端会产生更多的工作量,可能是返工工作量(前端
发现缺陷或修改内容),也可能是新增加工作量(前端新增内容),根据您的说法,还是需要分解一条子CQ单给未基线
的工作产品,统计的时候把它也算做是工作产品的缺陷(虽然它确实产生了更多的工作量),合适吗?如果这种情况不
算作工作产品缺陷,如何剔除呢?

\end{staff}

\begin{yang}
  \ylogo 一件事情是否合理,要看目标。所以我不断地强调我们要目标驱动。在配置管理的领域里,我们可以有好多
  管理模式。有些严谨一些,有些粗一些。粗、细,其实本身没有价值含义的,没有什么好与坏的。主要要看目标。我
  们需要统计返工量么? 我们需要统计缺陷数么? 这就决定了我们的粗、细的程度。

  当我们还没有利用配置管理作为测量数据的来源的时候,我们很难想象某些做法的好处。我不断提要关注返工
  量,整个中兴可能还没有一个做的起来,是因为大家没有想到这些问题。

  你的统计,在于``缺陷''的定义:在员工没有提交之前,算不算是``缺陷''?这个对员工很重要。他会说, ``这个明
  明不是我的缺陷''!重点在哪里?因为这个影响他的绩效!你们习惯这个目标,要知道员工的绩效! 可能这个不是你
  的明确目标,其实在潜意识里,你们是这样想的。所以你会问``合适么?''

  原则上,我同意你的关注。我自己也支持和落实在基线受控(是工作产品的基线,不是里程碑基线!)之前犯的错
  误,自己发现,不应该作为缺陷纪录。但是我们有了另外一个因素,另外一个考虑:有一个外在(来自需求,而不是写
  代码的人)的影响。我们应该如何处理呢?

  我提议的做法,是我以前的习惯。在我来说是理所当然的。因为这样做,代表了我可以得到一个正确的变更的历史
  与波及。比如,这个需求变更,影响到设计。那么,在需求变更的时候,无论这个设计文档,是否已经完成,我还是希
  望在配置管理的纪录里,看到这个事实。我觉得这个很有用,所以是合适的。(注:这样做,可能会有副作用,产生一
  些问题需要你处理。)

  在实际上,CCB 应该是不断地作这些判断的,每一个变更的波及。某个还没有提交的工作产品,是否需要提一个
  子CQ单,所以这个处理是通过评审和判断的,也不是每一次都一定是一个方法处理的。(你看,这个我们就做不到了!)

  另外一个议题:

  把一个新的工作产品检入的时候,是否需要CQ单的问题,我在一开始(2001年)的配置管理培训已经不断地提过
  了。Rational的培训师也强调了新需求和需求变更都应该用 CQ 单。但是当年我们还是非常初步的水平,不能理解
  这些概念。我们不单单不理解同一个``变更管理系统''可以管理新增的,和变更的,我们也不理解同一个``变更管
  理系统''可以用来管理需求的(开始的时候,好几个单位都是用不同的管理系统分别管理需求和缺陷的)。而且,我
  们应该一起管理需求和缺陷,因为需求变更几乎一定会波及其他工作产品的。举一个例,就是用
  了RequiPro 或是 DOORS,需求的变更,还是应该在CQ里有记录的。

  你们负责配置管理的,不应该多年来都处于同一个水平。你们应该不断地研究、探讨,发现最佳的原则,处理次要的
  问题(见上面的注),尽量地发挥配置管理的威力。这个是我对配置管理在中兴的发展的期待。

  但是,请你千万留意。我鼓励你们做你们能够理解的事情。我说的,是我自己的最佳实践。如果你们没有整理好一
  套``端到端''的流程,你不应该盲目地听我的。因为这样,你们会犯错的,而且不知道如何整改,出了问题,不知道如
  何收拾残局。

  谢谢你的关注和努力。
\end{yang}

\subsection{项目计划调整}

\qlogo \begin{staff}
如何调整项目计划?其根本原则是什么?
\end{staff}

\begin{yang}

  \ylogo 项目计划的目的,是作为开展项目活动的基础。就是说,项目员工应该可以看着计划决定下一步的工作的。
  如此,就是计划的内容如果不能够有意义的,就有问题了。

  所以,里程碑,重要的依赖关系,等等的准确信息,就变得重要了。

  重大子系统、其他主要的干系人,接口人,等等,对员工有主导作用的,也需要更新。需求变更导致增加(减
  少)了引起延误的新功能,新子系统,新的顶层WBS任务,等等就需要更新计划。但是一般的任务指派,在浮动范
  围内的延误,就不需要关注了。
\end{yang}

\qlogo \begin{staff}
项目目标一定,其它人、财、物发生变化,如何进行项目计划的调整?
\end{staff}

\ylogo \begin{yang}
提变更请求。通常如果不是配置项文档,如项目计划,是可以处理的很快的,如,不需要CCB,就直
  接改了算了。任何变更的处理都是一样的,都通过同一个CQ,同一组人审批。只是有一些变更分了严肃轻度而已。
\end{yang}

\qlogo \begin{staff}
项目目标发生一定的变化,是否该项目就不是原来的项目,需要重新立项进行新的项目计划吗?
\end{staff}

\begin{yang}

\ylogo 通常一个项目,是一个产品。或是一套(几个同时的)、或是一系列(几个时间上串连的)的产品。
   如果这个产品不再需要开发了,就应该终止这个项目。
   但我们可以用种种方法,虽然产品变了,还把它当为只不过是需求变了,就避免了终止项目。这个是好是坏,
   就得看公司的经营策略了:投资是如何核算的。如果公司要问责到一个产品概念,这个概念成功与否,就
   对公司有不同的意义。那么,按这个公司的策略,在里程碑的评审的时候,就参与,并按政策作决定。
   这个不是项目的决定。

   其他的目标,如质量、进度、成本,等等,都是使命以外的目标,通常都是可以变的。比如,交付期延误了。
   或是资金减少了,导致交付期推迟了,等等。
\end{yang}


\subsection{原意承担的风险就不需要应对措施?}\label{link2}
\begin{staff}
  \qlogo 今天遇到一个项目经理提出的问题, 风险处理策略是 承担的风险,他认为不需要给出风险应急计
  划? 理由是无论风险发生了什么影响他都承担。但我认为 承担的风险也应该有应急计划,只是我目前
  从PMBOK和CMMI中都没有找到能够强有力的支持我的观点的证据。所以请教您,您能否告诉我我的观点是否正确
  呢?

\end{staff}

\begin{yang}
  \ylogo 我也在考虑这个问题。他其实是要逃避做策划,所以说可以承担这个风险。他实在是在应付你。

你可以在几个方面考虑这个问题:
\begin{enumerate}
  \item 如果有决策,要承担某一个风险,其实也需要是一个分析的结果。
   就是说,我们已经决定了风险的概率、影响。他有这个信息么?

  \item 如果这个风险是很微小的,那,他为什么要提?我们只应该提有意义的风险。
   要考虑的,有意义的风险,应该是有影响的。那么,为什么他要承担?

  \item 如果有影响,就不能无条件承担。所以它需要考虑多个应对计划,对比
   每一个的投入,都比风险的影响更大。所以才可以决定承担。
\end{enumerate}

无论如何,都可以在它的风险策划、管理方面,找到这样做的不符合项,因为
这样做是不负责任的。我们需要控制风险,不是虚应一下故事!

看看这样的解释是否可以?
\end{yang}

\subsection{一些零散的规程定义原则}
\qlogo \begin{staff}
在定义规程的活动中需要注意哪些事情?(比如有些做法是错误的或者低效的)
\end{staff}

\begin{yang}
\ylogo 这个问题很大。一时三刻回答不完。

\underline{规程的作用,是规范大事情,不是规范小活动}

  如:自测试、单元测试、集成测试、系统测试等等是否有需要。而不是:自测试的报告一定要15个小时之后完
  成。(这些是项目计划规范的,不是规程规定的。)

  总的来说,规程是用来共享经验,提高效能的,不是用来规范行为的。

  所以,规范的,是重要的因素,是实践证明有价值的。如:同行评审如果有预审,效果就会好一点。那么,我们就要求
  同行评审的方法里面,包含预审的要求,并在会议之前,确认了预审结果满意,才开始评审会议。又:同行评审的主持
  者,如果有协调会议的能力,效果会明显提高。那么,规程应该要求``同行评审员''这个岗位有技能要求。项目策划
  的时候,要看项目里面是否缺乏这个技能,让后安排培训,等等。

  就是说,我们的每一个规定,都应该对提高项目有效性提供一些价值。

\underline{规程用文件、工作产品来定义比较明确}

  规程包括活动和工作产品。规程是可以用活动或是artifact用定义的。比如,估算这个子过程,可以用一系列的活
  动来描述。也可以说,估算的输入是什么,内容和质量标准是什么。输出是什么,内容和质量要求是什么。活动可以
  在``指导书''里说明。

  一个完整的规程定义,一定会包含活动和工作产品的规范(要求和标准)。我们一向强调活动,如何去做(任务)而轻
  视了工作产品(活动的目标)。所以我们的定义是流程,不是过程。我鼓励大家多考虑工作产品,作为过程定义的一
  个非常重要的环节。

\underline{职责单元的概念}

  当有了高层规程之后,我们要调配角色,我们应该考虑职责的范围是什么。一个角色,或是岗位,都应该承担一个职
  责。比如项目经理,就是应该由一个人负责,不应该有什么计划经理,技术经理这些并行的职责。如果一定要这样定
  义,那么,就需要有一个在这几位``经理''上面的人,承担这个项目成败的责任。这个才是真正的项目经理!奇怪的
  是,他们的领导又不回这样想。我们认为除了盯任务、进度之外,没有其他的因素。所以除了进度和技术之外,就没
  有其他的项目管理事情。这个是非常低效的想法。

  当我提到这个问题的时候,有些领导就说:一个人做不了这么多事情。这个想法是不对的。如果其他的企业可以,为
  什么我们不可以?事实是:如果我们相信员工,授权给他们,他们就很有可能把事情做好,而且会越做越好。这样才是
  真正的改进!所以我们这个想法(文化的一部分)防碍了我们的进步!

  反之,如果一个岗位,有了授权,员工就需要履行这个岗位的要求。领导需要帮助他提高水平,满足岗位的要求。否
  则这些定义就没有意义了,项目就不会有效能了。弱化岗位要求,导致团队的低效。容许这样的行为,是一种``姑
  息'',也会妨碍进步。

  又如QA是负责提供过程遵从度的可视性。所以不应该说QA做过程审核,项目管理员做工作产品的审核。如果这
  样,就没有一个岗位、角色,可以提供一个完整的可视性。就弱化了这个岗位了。

  所以,角色、岗位,需要按职责来定义,不是按活动的方便来定义的。(我们需要培养承担责的态度)。

   总的来说,过程管理是一门博大精深的学问。还有很多很多因素要理解,要讨论。新东西
   越来越多,我自己也在手忙脚乱地不停地学习呢!
\end{yang}

\subsection{单元测试与自测}
\begin{staff}
\qlogo 目前有几个问题:
\begin{enumerate}
  \item 在什么情况下,可以考虑编写代码和单元测试人员的分离?
  \item  开发和测试并行还是串行?如果并行的话,祥设需要达到什么样的要求?
\end{enumerate}
\end{staff}

\begin{yang}
\ylogo 您的两个问题,其实都和成熟程度有关。
\begin{enumerate}
  \item 单元测试如果要和编码人员分开,一定要很成熟。我们现在,老实讲,还没有达到这个水平。
  \item 串行、并行,对于详设来讲,都有困难。我不知道你们的详设定义是什么。一般的开发,软件的,详设都不能
    做好。这个也是成熟的问题。如果我们详设做不好,无论如何要求,都没有用。我们只能量力而为。

    千万不要把并行作为不把详设做好的借口。我觉得水平不够,已经是一个很好的借口了。如果详设的定义是合理
    的,无论串行、并行,都要做详设。

    串行和并行,都需要有模块(类、函数等)与模块之间的接口定义。模块的功能、性能也应该定义好。所以串行和
    并行开发都可以。但是我们可能没有到这个水平。
\end{enumerate}

我提议,不要把单元测试,做的这么正规。这样会很难做好。我们有自测的习惯么?
如果有,就把自测规范起来。如果没有,就建立自测的习惯。条件就是自测需要规范,
需要QA检查。

自测的目标有三:
\begin{enumerate}
\item 保证接口没有不健康的输出,同时保证可以应付不健康的输入。
  \item 实现基本功能
  \item 开发的VIEW和集成的VIEW一致,保证将来集成的成功率。
\end{enumerate}

这样做,一来目标明确,操作比较简单,二来保证下游活动顺利,将来的构建成功率高。
对于什么,学术上的定义,标准,各种各样的覆盖,我觉得现在是没有意义的。先把
习惯建立起来,同时要提升设计水平。到一定程度,再讨论覆盖的问题。

当然,这是个``标准规程''。牛人可以裁剪成很先进的。但是这是个最低的标准,懒人不能够把他裁剪掉!

如有其他问题,欢迎来信。谢谢您的努力。
\end{yang}

\subsection{平台项目和产品项目的配合}
\begin{staff}
  \qlogo 基于产品很多功能是基本功能,为了充分、高效利用资源,产品部成立了一个纯软件的平台项目,其目标是
  为各个产品提供软件解决方案。各型号机产品的软件方面的需求是该项目需求的输入(含进度要求),各型号机的软
  件就是从平台项目软件拉出来的不同的分支。目前,配合各型号机产品的进度,该软件项目规划了几个阶段,定义了
  每个阶段先后实现的功能,各阶段分别发布V1.0,V2.0,V3.0\ldots 的基线版本

  前期提出的验收方式是对应阶段的所有型号机项目都完成验收(设计定型)是纯软件项目的必要条件。但这个方式
  有个缺陷:进度绑定。也就是说,纯软件项目本身已经实现了阶段目标(对应该阶段各型号产品需求的全集),却可能
  因为某款型号产品进度的延误导致该纯软件项目迟迟不能结束。也正因为如此,对该项目的验收方式还意见不一
  致,该项目认为绑定会影响其项目按期完成(公司会考核一级里程碑),而型号机项目认为作为客户之一,客户还没有
  验收,该项目就不能算结束。

  所以想听听您的建议,业界的类似项目又是如何验收的呢?
\end{staff}

\begin{yang}
\ylogo 您的困难,其实就是平台开发项目,希望开发完就立刻验收;整机承受项目希望自己方便的时间验收。这
  个困难其实不难解决。但是我们有两个条款,造成这个困难。

  \begin{enumerate}
  \item 把不同的承受项目的各自不同的要求,绑定、绑死平台项目的进度。这个是一个过程管理的原则,不知道为
    什么我们要这样定义。

  \item 一刀切的管理理念。就只看一个指标的一个条件:项目终结是否按时。这个管理理念非常不成熟。不成熟的
    管理造成很多效率和执行力的流失。为什么这个不好呢?如果我们参考 TL9000的数据定义,什么算、什么不
    算,总不是一刀切的。比如说General Availability (广泛供应)就不是一个定义,而是:运送出去的,或是人家来
    拿得到的。又如:组织需要提供的故障数,包括组织产品造成的故障。但如果客户不让组织验证这个故障,那就不
    算是组织的故障,等等。
\end{enumerate}

有了这个一刀切的管理,加上用这些一刀切的指标考核,两个利益重叠的团队不争得头崩额裂才怪!我们又如何建立团
队精神呢?又如何提高执行力?

这个解释根据不合理的标准进行考核造成的困难。

您现在可以做到的,就只能是把问题提出来:
\begin{enumerate}
\item 平台开发项目需要按版本管理。每一个软件,有一点点不同的,就是一个不同的版本。每一个版本是一个管理
  单元。项目要有这样的能力。

\item 版本是考核单位,但需要多个指标。如:系统测试通过时间,研发过程的缺陷泄漏率,等等。不应该用不可以控
  制的因素考核任何人。领导考核的时候需要平衡这些指标。千万不应该给平台项目对整机项目一些考核权来补救。
  这样是补救不了的。同时又是一个领导推卸责任的机会。
\end{enumerate}

我的回答,就是业界成熟的企业,用最有经验的团队定义规程,定义度量指标。然后,领导的考核准则,是达成业务目标
的全面因素。 因为全面,比如完成自己的业绩以外,是否有效沟通?是否对兄弟团队提供支持等。考核也不会是每月,甚
至每周一次。大概是每年一次,才能全面。我们呢?只看少数指标,而且量化的要命(结果是领导不需用脑,而不是减少
主观性)每月每周的考核,又烦又片面。

这样的管理,我们可以有成熟的度量体系、效率、共同目标才怪!

如果领导的管理理念不改,我们还是尽量地提高自己吧!鼓励您千万不要以为无能为力。您要做自己的主人。利用好
时间。
\end{yang}

\pagebreak
\section{点评}

\subsection{项目例会的作用}

\begin{statement}
  说明:收到一位同志对项目例会的意见,觉得他是能够有条理地思考的人。同时,我总体上感觉到他的意见大都是
  先有了结论,再找理由的那种,而不是先有需要解决问题的开放心态,然后讨论可能的方案。如果我们不觉得现
  在的管理理念有什么问题,这样的意见是理所当然的。

  在中兴的过程之中,其实问题是项目对最终产品属性的控制力不够。可能的问题在于我们单单控制任务的进
  度,没有控制任务之间的关系。我们的控制,在于任务的延误,这是一个结果参数。更有效的控制,在于控制他
  的前瞻参数,如风险。我们现在依赖一个人(项目经理)来担心任务的延误,我们可以利用一群人(所有承担任务责
  任的项目成员)分担一部分的依赖。凡此种种,都是例会可能解决的。

  所以我在赞赏这位同志的思维的同时,也希望表达我的意见。下面是同志的文章,我在其中用红字加插了我的意
  见。

  谢谢大家的关注与努力。
\end{statement}

\begin{myquote}
项目为什么要开周例会?归纳起来目的如下:
\begin{enumerate}
    \item 及时发现项目风险
    \item 通报项目组成员工作情况
    \item 项目团队建设,增强项目凝聚力
\end{enumerate}

1、项目的风险来源大约包括:由环境、进度、范围、技术和人员等因素与预期不符而产生的风险。项目发现和控制
风险的手段非常多:方案评审、代码走查、员工交流(正式、非正式)、对外交流、技术研究、测试活动、技术讨
论(会议或邮件)、方案验证、员工培训等。通过例会(受时间周期限制,过于频繁则效率太低,反之则发现问题
滞后)由口头或书面汇报工作状态而发现风险的概率非常低。应用型项目与模块型项目不同,主要的风险来自于项
目周边环境,如:局方突发的需求、协作部门实现方案的变动、外部重大测试故障和调试环境的可用与否及是否与
预期相符等,这些都不是简单的项目例会内部交流或通报就可以解决的,而应用型项目很少存在理论上无法实现或
有重大障碍的未预见技术风险,所以通过会议汇报发现项目风险的概率很低。
\end{myquote}

\begin{comment}
  管理是一个灵活的活动。如果目标是风险管理,我们就要关注风险,尤其是风险的蛛丝蚂迹。这个如何做
  呢?CMMI没有说的明白。但是``警觉''是必须的。一个警觉的人,会经常留意,利用机会。所有提到的,方案评
  审、代码走查、员工交流(正式、非正式)、对外交流、技术研究、测试活动、技术讨论(会议或邮件)、方案
  验证、员工培训等,都是机会,周例会,也是一个机会。还会有其他类型的机会。我们要决定的,是周例会是否
  值得。这个谈起来很有争议性。正反双方都可以提很多道理。但归根结底,是我们是否有效利用好这些机会?如
  果我们说这个形式不好,我们是否用其他方式把事情做好?所以问题不单单在于周例会是否发现风险的效率
  低,而是在于,我们是否用其他方法处理好风险管理了?

  如果我们通过实践,发现在项目里,可以系统地,有效地,发现、识别风险,远远比在周例会作风险管理有
  效,所以不用周例会,那么这个是一个很好的成就,可以作为最佳实践,向其他项目推广。但是如果没有尝试
  过,就有这个结论,这就不很好。

  另一个角度,到底周例会是否能够有效发现风险。这个最终是由员工(包括项目经理)的投入决定的。CMMI提到
  很多很多对项目效率很重要的事情,我们都觉得不重要,所以不想试,或是试一下就说没有效。CMMI只能告诉我
  们什么重要。我们如果要有效,不但需要接受这个因素重要,还要找到有效实施这个因素的方法。但我们真正接
  受某因素是重要的时候,我们就会投入,找方法,并且关注他的有效性。

  在项目的风险的体现,最大的机会,在于任务的延误。延误的原因可以很多,技术,资源、沟通、质量等等问题。
  最终反映的,都是延误。一部分任务,可以在其他的地方发现。但是周例会是检查任务状况的,所以是一个系统
  的发现风险的渠道。是项目里最密切的干系人一起在场,警觉地在了解项目的进展情况,是充满了发现风险的机
  会的。

  下一个问题是如何有效掌握这个机会?答案也是投入,经验,警觉,等等。就是说员工的水平。水平如何建立
  呢?只能通过实践!实践机会那里来呢?在于相信CMMI里说到的重要因素,所以努力找最有效的方法处理这些因
  素。还有其他方法么? 
\end{comment}

\begin{myquote}
2、应用型项目的项目组成员之间有紧密的协作关系,各模块之间常常存在接口联系,所以关联模块间的相互交流是
非常频繁和琐碎的,通过项目例会而达到工作情况沟通的效果很难理想:工作汇报太详细则浪费会议时间,工作汇
报太粗略则容易落入形式化的俗套。而事实上项目经理在布置项目任务时对任务责任人的能力和工作的大体难度已
经有所评估,同时在任务执行过程中常常会以多种形式的沟通(询问、讨论、协作人评价等)了解特定任务的完成
情况,依赖个人汇报的形式存在片面和失真的风险,并不能做为项目经理判定任务进展情况的唯一或有力判据。
\end{myquote}

\begin{comment}
这个问题和刚才的非常类似。紧密协作有很多方法。每一个都有它的特殊用途。周例会是项目的主要骨干
  一起在场,对于传递项目信息,是最有效的。可惜我们的文化,比如不爱如实表达情况,让例会变得形式化。有
  些人认为如果项目经理非常清楚每一位员工的任务进展情况,就不需要几个骨干一起在谈一遍。其实我们不了解
  人的组合的特征。可能我们现在利用的不好,例会暂时不能发挥它应有的作用。如果要有效发挥例会的作用,我
  们需要有主导会议的能力。这个我们很缺,理会低效我一点都不出奇。问题还是在于,我们是否需要提高自己的
  管理能力。如果我们觉得我们是足够的,那么就当然不会想到要提高。

  要知道自己是否足够,只能通过效能指标。但是这个指标的X很多,我们都很容易找到不是自己的不足做成的低效。
  其实我们如果关心团队的效率,我们就不会试图证明我们不是低效的原因!我们就会积极保证我们是高效的。这
  是两个非常不同的态度。

  到底例会开的最好的时候,是什么样的境界呢?例会开完之后,大家都有一种新的理解,自己的任务在项目
  里,对比其他的同事任务,得相对情况,从而知道如何相互配合。这是一种团队的感觉。这跟其他的团队建设活
  动不一样。这个根植于项目的共同工作。而且是单对单或是小组讨论所不能提供的。我要说明一下,理会通常都
  是小组的,人数不会太多,否则效果很难体现。

  到底这些效果,是否需要,是否值得,当然要看项目的实际情况,包括项目经理是否能够把例会控制到可以达到
  一些例会的目标。但是不能达到,跟不可能达到是不一样的。过程改进就是有一个比现在强的愿境,然后努力地
  去改进、提高,去达到它。
\end{comment}

\begin{myquote}
3、通过项目会的形式可以达到团队建设的部分效果,但不是唯一手段,且没有必要以固定周期的形式体现。部门经
常组织的项目间竞赛活动就是一种项目团队建设的有效手段;此外同一个项目组的人员为同一项目目标而共同协作
工作、经常性开展技术讨论,这些工作行为本身其实也就是团队建设;时常通报项目相关市场振奋人心的消息更是
团队建设的强心剂。集中开发、发放物质奖励等等,项目组可以使用的团队建设手段实际上非常多。靠会议做团队
建设反而显得简单、从上对下、不和谐。
\end{myquote}

\begin{comment}

  我上面提到的例会的境界,其实是说明了这些活动的分别。他们不能相互替代。当然,不是每一个活动都一定要
  有,最低限度不易定要经常发生。但是如果选择例会,就需要是``例''会,效果才可以体现。从这个方面,大家
  也可以考虑考虑为什么例会可能有效。

  上面的论调,我的感觉是自己已经不愿意去做,就找到一些理由。而不是我们有哪些问题,然后考虑这个是否一
  个解决问题的方法。
\end{comment}

\begin{myquote}
文字和会议可以促进沟通、增进交流,但是``文山会海''往往会分散员工的精力,影响工作效率和士气,所以文字
和会议的强度需要控制。图像和语音项目工作任务零散、人员工作地域分散,项目(兼科室)周例会可以达到很好
的效果。而应用型项目工作任务统一,人员集中,但是科室分散,项目周例会与科室周例会存在功能、时间和人员
上的冲突,所以周例会的必要性和可行性,以及应该如何操作,建议应该先多充分讨论,而后再制定规定。
\end{myquote}

\begin{comment}

  这个我比较同意。异地开发的例会效果的确会打节扣,但不能说一定没有。比如,在当地组织当地的例会,收集
  进展情况和风险,然后再汇总,虽然大家只能清楚当地的情况,只能建立当地团队的感情,还是比完全没有的好。
  老实讲,我觉得如果能够主持的好,例会的效果是明显的。正如上面所说的,必正例会就是小组的。

  规程的定义,也应该是按项目实际情况灵活裁剪。
\end{comment}

\subsection{我们过程改进的``可重复性''}

\begin{statement}

  收到一封邮件,是一些有关过程改进的问题的。看过问题之后,一种无可奈何的感觉油然而生。这是一堆重复地
  发生着的问题。我不清楚大家是否要作改进。如果是,这些都是容易解决的问题。如果这些问题都解决不了,我
  也不知道如何是好。大家总得要拿出决心来。这个是关键。我们不能在没有改变的情况下,收到改进的效果。

  以下是邮件里的问题,我的评语用红字插在要回答的地方。
\end{statement}

\begin{myquote}
根据公司总经理的指示,今年作为公司的过程改进年,所以,从无到有,我们正在做一些工作,比如明确了公司的
组织结构、MSG/EPG/TWG的组成,并正在梳理需要重点改进的过程(包括了市场、物流、研发等所有过程)。您提供
的教材非常好,前几天的TWG的学习有些教材已经提供给大家自学了。

改进过程中还是有很多困难,比如:在产品经理或其他领导眼里,过程改进工作永远是优先级低于产品开发工作,
\end{myquote}

\begin{comment}

  过程改进的确没有业务重要。所以,分配到过程改进的工作量资源,应该是远远比业务工作少的。这个是正常的。
  解决的方法是分配:时间的分配,工作量的分配,人员的分配。能有效分配,就可以平衡,就可以有效。但是我们
  的管理理念,只能管注一件事情,要么就全部资源都分配到业务,否则就害怕业务完成不了。这是单维思维,结
  果就是气球效应,防碍了长远的任务。就是分配到过程改进的资源,一有机会,又夺回来用在业务上了。比如要
  求QA做项目的活动。

  要强调一下,分配的概念,是分配好的资源,是要有保障的,不可轻易挪用的。
\end{comment}

\begin{myquote}
所以专职的过程改进人员非常累,是要盯着催着大家作改进的;比如:合适的过程工具很难找,为了能够进行度量
同时减少项目组的工作量,一些合适的工具最好是集成的工具需要考虑进来;度量工作的重要性大家都知道,在改
进前期没有好的工具支撑情况下,度量工作如何开展呢?
\end{myquote}

\begin{comment}
我说过很多次,在我的工作生涯之中,很多事情是用手做的。因为这样,我学习到很多道理。就好像我们
  没有计算器,就只能背九因歌,没有MS Project就用手画。在明白了过程的其中奥妙之后,才可能选择正确有用的
  工具,否则工具永远都会是一个发散的讨论议题,让人不能专心于业务,开发出来的工具,也不会很好用,好像
  我们的情况一样。


  其实真正的困难,在于我们还不知道如何应用度量的概念,不是在于没有工具。如果我们知道度量的重要性,我
  们就有决心的解决度量的应用问题。从每个人的效率开始,从会议的效率问题开始,这些都不需要很多工具。这
  样可以建立度量意识。需要时间。但没有这个意识,工具是一定没用的。

  问题是,我们要坚持一段时间,让大家可以知道项目的部分活动的效率,先尝尝一些甜头。但是大家其实说这知
  道度量的重要性,但往往拒绝能够建立度量体系的成功必由之路。

  任何改进,开始一定会有一点效率下降的。这个是要决心的。我们的单维思维,就不容许这个短暂的损失。没有
  投入,就不能有回报。
\end{comment}

\begin{myquote}
一个小的公司,市场和效率是第一的,如何很好的平衡市场/效率/质量的关系呢?
\end{myquote}

\begin{comment}
简单到不能再简单。要有效,需要大概8-10\%的资源分配到长远发展的活动,如过程改进等。分配好的资
源,就要好好地利用在改进,不能临时用于业务,如要求QA做项目的事情。8-10\%的资源,包括项目大概4-6\%的
资源,用于改进。就是说,大概4\%的组织资源,是实实在在的EPG组织的资源。

进度要求要有足够耐心。效果是慢慢出现的。分配少于这个,效果不会好,因为都只能做形式的东西。

我的经验是,我们的领导,我们的管理理念,是不认同这样的理念的。他们希望不用改变思维,只拿一些招数,如
每日构建、代码走查、等等。

这个理念认为这不单单是可能的,而且是正当的方法。但是要持续地发展这些招数,也是需要管理理念的改变的。
同时进度也不见得快。反而如果过程管理理念成熟,能够得到的效率(生产力)提高,是成倍增长的。

这里授人以渔是不被重视的,因为大家都只是关注于鱼。这个就好像买东西,又要不付钱,又要拿到好东西,这个
只能是偷而已。但是``知识、经验''是偷不来的。可惜,可惜!
\end{comment}

\begin{myquote}
之前我也给杨老师发过一封邮件,是关于需求粒度方面咨询的,在ZTE的经验是,原始需求-市场需求-产品需求,每
个需求的模板都是复杂的,内容上是交叉的。
\end{myquote}

\begin{comment}
这个情况非常不好。其实在这里开始改进非常有必要。请留意,我从来没有支持过``原始需求库''这个
  概念,因为这里``守株待兔''的味道太重了。

  如果大家都有了一定的抽取需求的水平,拿个原始需求库来存放,是可以的,但是意义不大,是一个次要的工具。
  我们在水平不够的情况,这个库就有如一个借口,一个依赖。这样就变成了毒草。
\end{comment}

\begin{myquote}
我们在考虑如何保证这些模板的意义,但是又要简化使用。大家还是有争议的。我的理解是:原始需求是对所有需
求来源作记录,包括了各种需求的原始描述/来源/提出日期/要求急迫度等,甚至于包含了部分客户的信息;而市场
需求是对原始需求的提炼,需要判断哪些是我们要规划到产品里面去的,并且这些需求需要得到市场/技术等各方面
的确认,并且最终得到高层的认可;产品需求则是把市场需求的语言转化为技术的语言,便于指导后续的设计开发
和测试。不知这样理解是否正确?
\end{myquote}

\begin{comment}
  需求,是跟着时间变得非常快的,是需要即时抽取的,是需要有前瞻性的,是需要有市场和产品根据的。放到库
  里的东西,不能存多久,否则没有意义。所以管理``原始需求''是浪费时间,没有意义的。对需求理念的错误理
  解,会导致发展系统工程的困难。这些模板如果复杂,就要改呀。还能做什么呢?

  客户(市场)需求,和产品需求的作用也讨论了很多。最近和这里的单位作了一、两个工作坊。我希望他们现在
  对这两种需求的分别的认识应该是清楚了很多吧。

  我愿意帮助大家理解这些事情,但我不明白为什么这些简单的东西这么困难。我猜是因为大家没有尝试过。如果
  尝试过,也可能是一遇到困难就停了,没有探讨下去。这样就使进步的障碍。我也没啥可说的。
\end{comment}

\begin{myquote}
 很多很多的疑问,希望能够得到您的指引。谢谢。
\end{myquote}

\begin{comment}
祝愿你成功。
\end{comment}

\subsection{如何对待不尊重他人的领导?}\label{link3}

\begin{myquote}
收到邮件:

``如果遇到不尊重他人的领导,我们怎么办?''

1)当你勤勤恳恳、加班加点的做完所负责的工作,而领导告诉你,你这部分工作很简单,一个大专生都可以
做。(听了这些话,很打击人的工作积极性。个人认为,抛开工作本身的难易程度,任何一个项目的完成,哪一个
组成部分都不可或缺。)

2)当你非常真诚的请求这位领导安排难度高的、重要的工作,将现在承担的工作找个大专生来做的时候。该领导
说``大专生能做的工作,你就应该可以做的更好。没有人看不起你的工作。是你自己心态不正确。有这样的心
态,所以你就产生了消极抵触情绪,不好好工作。''实际情况并不是这位领导说的这个样子。实际是你一直也都很
努力的工作,只是这位领导他从一开始就认为你做的是大专生做的工作。从来不重视你的工作。

3)当领导让你在某个时间之内完成某个任务的时候,如果可以完成,你就直接答应下来。如果不能完成,你会告诉
他可能会有哪些困难会影响进度,大约什么时间可以完成。也就是,答应的事情一定会按时完成,不开空头支票。
另外有一种人,他们当领导交代下任务的时候,好像很给领导面子,很爽快的答应了。但是在期限到的时候,完不
成,他们这时候会告诉领导有哪些原因。导致工作一拖再拖。领导就喜欢后边的这种人。

4)当考核的时候,别人考核优秀,而你考核成绩明显不如别人。你想问领导究竟是自己哪部分出了问题,以便于以
后改进。该领导给你列出了很多的考核指标,用那些条条框框告诉你,``你看,你有不少地方做的不够好。总有个
第一第二之分,你就非要当个第一?''实际情况是,他的这些指标即使是那些评为优秀的员工也没有能够达
到。(领导不能得罪呀!一定要和领导处理好关系。免得他总用尺子来度量你)

你自己是一个很奋发图强,你总是想尽办法想着怎么样把自己所承担的工作做的更好一些。而不把精力专注于研究
领导是一个怎么样的人而投其所好。事实证明,这样的人很吃亏,出力不讨好。

碰到这样的领导,我们应该怎么办?
\end{myquote}

\begin{statement}
谢谢你的问题。答案在问题后面。

这些问题反映了你对现实的不满。这在年轻人身上,是非常普遍的,因为我们最能看到、感觉到的,就是与自己有
关的事情。我们知道自己工作努力,我们感觉到自己很累。但是我们感觉不到他人可能也是这样的。我们只能从观
察形成一些印象。这可能是主观的。

同时,我留意到我们有比较强大(相对老年人)的意愿(这些意愿有时甚至变成要求)现实按自己的意愿开展。加上基
于缺乏经验,又不知道如何达到自己的意愿,也没有充足的经验留意观察现实的限制。所以觉得很困惑。

这些都是你问题的原因之一。我会用一些如何处理情况的方法回答你的问题。这是最初步的。第二步就是要看自己
的问题,要看自己有什么(尤其是态度方面)需要改善的。这就是修养。

所以,正确的态度,不能是要证明人家是不对的,自己是对的。这就好像我们不能说``客户自己不知道自己要什
么。''一样,事实是否如此不重要,我们就不能这样想。成熟的人,有修养的人,是不考虑人家的不对,只考虑自
己的不对,只关注自己如何做的更好。

请你在这个心态底下,来看以下的答案吧。
\end{statement}

\begin{myquote}
  1)当你勤勤恳恳、加班加点的做完所负责的工作,而领导告诉你,你这部分工作很简单,一个大专生都可以
  做。(听了这些话,很打击人的工作积极性。个人认为,抛开工作本身的难易程度,任何一个项目的完成,哪一
  个组成部分都不可或缺。)
\end{myquote}

\begin{comment}
  这个不是一个成熟、高效的领导行为。成熟、高效的领导,需要鼓励员工,不能伤害员工的感情。到底这个工作
  是否大专学生可以做不重要,就是不能说出来。但是这个你是不可以改变他的。我们的领导,的确有很多错误的
  管理理念,其中包括什么事都认为不需要判断、不需要创新,所以他们会说这些工作是大专生可以做。其实做起
  来就没有这么高效。

  问题是,你是否真的使用了你的判断力,你是否更好地掌握了重点?你是否用了更有创新的理念来完成你的任
  务?你能够做的,就只有尽可能地证明、表达你工作的成绩。如果你可以证明一个大专生不能做的比你好,就更
  有说服力。他可以看到组织里的所有任务,你不能看到,所以你的比较不会比他的全面,在这方面,你要说服他
  是不容易的。虽然你的领导可能不成熟,他也有一定的经验与技能,也有他的目标。如果你真的能够让他认识到
  你是能抓重点的,能作判断的,有创新的,积极主动地,支持他的目标的,他对你的成见可能就可以解决了。
\end{comment}

\begin{myquote}
2)当你非常真诚的请求这位领导安排难度高的、重要的工作,将现在承担的工作找个大专生来做的时候。该领导
说``大专生能做的工作,你就应该可以做的更好。没有人看不起你的工作。是你自己心态不正确。有这样的心态,
所以你就产生了消极抵触情绪,不好好工作。''实际情况并不是这位领导说的这个样子。实际是你一直也都很努力
的工作,只是这位领导他从一开始就认为你做的是大专生做的工作。从来不重视你的工作。
\end{myquote}

\begin{comment}
  如果他说大专生都可以做的,他已经看不起你的工作了。这样不对。虽然他是领导,也不应该这样对待员
  工。(中国的下一个思想解放,我希望就是领导们知道自己没有特权!权力,只是实施更大更重的任务的工具而
  已!)

  但是他的要求是对的。如果你真的能够承担更复杂的任务,你就需要能够把简单的任务作的更好。好的因素,我
  已经在上面表达过了。能够抓重点,能作判断,有创新意志,同时积极主动,承担责任,重视承诺,让领导有一
  个安全感,可以信赖你。

  你要这样要求自己。可能应付这类的领导,你也需要有坚强的意志,不让领导的评语伤害。这样会增加成功的机
  会。

  这个很重要,我不鼓励关注人家的不对;只关注如何解决问题。
\end{comment}

\begin{myquote}
3)当领导让你在某个时间之内完成某个任务的时候,如果可以完成,你就直接答应下来。如果不能完成,你会告
诉他可能会有哪些困难会影响进度,大约什么时间可以完成。也就是,答应的事情一定会按时完成,不开空头支票。
另外有一种人,他们当领导交代下任务的时候,好像很给领导面子,很爽快的答应了。但是在期限到的时候,完不
成,他们这时候会告诉领导有哪些原因。导致工作一拖再拖。领导就喜欢后边的这种人。
\end{myquote}

\begin{comment}
你的问题,让我不能不再说一遍,这样的管理理念,是不成熟和低效的。但是你听了之后,你可能觉得好过一点,
但对你还是没有帮助的。

每一个人,都有一定的道德观念。如果你知道领导是这样的,你可以按这个模式操作么?我是可以的,因为
我需要尊重``传统的权威''。

这个问题,其中的一个因素,就是领导心中觉得任务是简单的,而你觉得比较复杂。我不会相信他任务简单的话,
因为我相信自己的估算。我会告诉他我的估算。

你能够有这样的信心么?你如果没有这个信心,人家是很难相信你的。
现在的问题,就是你的领导还没有对你建立起信心。所以你的问题,是如何建立他对你的信心。
这个只能通过你的绩效。那么,你需要做的,是知道如何提高你的绩效,高到一个让每个人对你都有信心的程度。

你现在达到这个境界没有?一般的员工对你的印象如何?
很大部分的解决方法,其实就是提高自己的能力。你就开始去做吧。不要太关注领导如何不成熟,如何对你不好。
你要老老实实地把自己变得大家一定对你另眼相看的。

好了,如果领导要求一些不可实现的任务,我会首先告诉他自己的想法,如果他还是坚持,我是会尊重的,
我就会真心地尽可能在他要求的条件完成。如果能够做到,我会问自己,为什么我开始的时候说不能够?
我会接受领导比我英明的结论。如果做不到,我会好像其他员工一样,告诉领导,做不到,为什么,等等。
我一定要想办法让领导知道我是已经尽力的。(这个做法有点像你投诉的员工。可能他们不是拍马屁的。可能
他们的方法是可以理解的,可能是正确的方法?)
\end{comment}

\begin{myquote}
4)当考核的时候,别人考核优秀,而你考核成绩明显不如别人。你想问领导究竟是自己哪部分出了问题,以便于
以后改进。该领导给你列出了很多的考核指标,用那些条条框框告诉你,``你看,你有不少地方做的不够好。总有
个第一第二之分,你就非要当个第一?''实际情况是,他的这些指标即使是那些评为优秀的员工也没有能够达到。
(领导不能得罪呀!一定要和领导处理好关系。免得他总用尺子来度量你)
\end{myquote}

\begin{comment}
成熟的领导需要能够帮助员工成长。所以``你有不少地方做的不够好''的回答是不够的。领导需要告诉你如何
不好,如何改进,而且这些改进是合情合理的。这样才是成熟的领导。

另一方面,你有跟人家比较的习惯。这样一定做成不满。这就是明证。很简单,你一定是跟考核结果比你好的人比,
你不会跟比自己差的人比。是吧?如果你是客观的,最低限度,你会告诉我,你相对80\%人差,20\%的人好。
我就开始跟你讨论这个问题。但是老实说,如果一个员工能够做到这个水平,他一般已经不会再斤斤计较
这些问题了。因为,要么他的考核结果不会太差,要么他的修养就已经是比较成熟了。

对,修养跟技术水平是有关联的。你要相信这一点。
\end{comment}

\begin{myquote}
  你自己是一个很奋发图强,你总是想尽办法想着怎么样把自己所承担的工作做的更好一些。而不把精力专注于研
  究领导是一个怎么样的人而投其所好。事实证明,这样的人很吃亏,出力不讨好。
\end{myquote}

\begin{comment}
  你这样说,证明你的奋发图强到现在还没有什么效果。否则,你会觉得,你的奋发图强,已经把自己磨练得非常
  能干。这样那里有什么吃亏的?可能你奋发图强的方向、方法可以改善一下?多注意自己的效能水平。在完成任
  务的过程中,做到提高自己。这样才是为自己做事,不是为他人做事。也是在课程里的。

  你自己要好好地问问自己,到底自己的能力水平如何?这个可能的确是你的问题所在。
\end{comment}

\begin{myquote}
杨老师,想请教您,碰到这样的领导,我们应该怎么办?

另外,工作两年多了,深刻感觉到,工作真的不像在学校,只要好好学习就可以了。需要处理很多的关系。哪个关
系处理不好都会非常影响自己发展。会碰到很多人为的障碍。想请杨老师指点,怎样才能在变幻莫测的职场游刃有
余的去做自己想做的事情?还想请杨老师给我推荐一些书籍或者资料学习学习。
\end{myquote}

\begin{comment}
我也遇到过这样的领导。你要么就服从、要么就斗争、要么就离开。如果斗争,就一定会有一个离开。
大部分的情况,我会服从,因为我尊重传统的权威。在日常的工作中,我会尽量通过解决技术的,人事的困难,
通过解决问题,提高自己。我会关注、考虑任务之中自己喜欢的议素,每完成一个任务,都得到一些提高。

注意,每次遇到困难,就换岗位,离职,就是逃避,不是解决问题,对自己没有帮助。换岗位,离职,有
它主动的价值和需要,不是为逃避而为。否则,你是放弃了一个磨练的机会。有困难,要发展处理的能力。否则,
跑到另外一个岗位,你还会遇到很多不同的困难情况不能解决,你的职业生涯,就会是一连串的挫折。

找办法处理好领导的关系。可能自己的期待和能力,还不匹配。心态需要平衡,面对困难,分析自己的问题,
不断找机会磨练。如此积累下来,才有争取达到自己意愿的机会。

你需要明确目标。要有毅力。要刻苦。缺一不可。

谢谢你的努力。希望你的努力带来成果。
\end{comment}

\subsection{如何对待不尊重他人的同事?}\label{link4}

\begin{myquote}
``遇到不尊重他人的人,我们怎么办?''

这些人:
\begin{enumerate}
  \item 在领导面前表现自己。
  \item 遇到好事总喜欢把功劳都贴到自己身上。
  \item 但是遇到什么问题的时候,他总是会推得一干二净。
  \item 而且会将他自己应该做的工作往别人身上推。
\end{enumerate}
\end{myquote}

\begin{comment}
这些都是非常不专业的行为。他们可能会一时成功,但是他们的成功不会是太长远的。这就是
专业态度是可持续的成功,否则是可能成功,但未必是可持续的。

我肯定不会鼓励你如他们一样,变得不专业。我的提议是:

1)学会表达你自己。你自己做的任务,要有报告的习惯。每周,写一个简单的报告,主动提交给领导。这样,你就
保护了你自己的努力。也不用争功,也不是拍马屁。只是老老实实、简简单单地作报告。报告要能抓重点。这个就
是一个磨练。要磨练,就是要写了报告之后,自己回答这些问题: ``这个报告精简么?'' ``这个报告里,有哪些是
可以不说的?'' 好的报告包括:过去的实施如何对比计划的。将来的计划是什么?有什么风险?如果你能够做到这一
点,任务的功劳,人家是抢不了的。这是保护自己。当然,不是每一件事都能保护。我自己也是被人抢了功劳。这
些情形,我们只能让人家知道你是有自我保护意识的,来让他们知难而退而已。只能这样。

2)要看开一点。``本来无一物,何处染尘埃?''人家的不良习惯,我们不需要太关心人家是否邀功。就算不是他的
权利(因为这样不专业),他也有这个自由。让他们找他们的道路。有机会,恰当的时候可以谈一下,否则自己关
注把事情做好就是了。

3)要保护自己。要学会拒绝过分的任务。这个很容易,你要练习说一些礼貌地拒绝的话。比如:``这个你做更适
合。'' ``这个你更有能力做的好。'',甚至简单地,``我很忙,对不起。''其中有一个方法,老外叫做``broken
record'',就像一个破唱片一样,不断地重复同一小段的句子。无论人家说什么,你就是安安静静地一句``我知道
你要我做这个。但是我很忙,对不起。''或是``我很忙,这次不能帮你。这个你更有能力做的更好。''不要轻视这
个。我自己就是这样锻炼自己的。很有效果。

我不赞同说:``对不起,我很忙,你找(另一位同事)吧!''因为这个是你与他之间的事,不要把问题无必要地扩
大,让其他的人受干扰。如果你拒绝了人家,他自己去找其他的人,是他的责任。如果你提议人家找某某人,是你
把某某人拉到水里去。这个你不应该做。

如果再有问题,请给我邮件。

谢谢你的参与、兴趣和努力。
\end{comment}

\subsection{处理和统计还未基线的工作产品}
\begin{myquote}
  您提到的:``如果我们有两个CQ单层次,我们的缺陷数,就是低层CQ单有关。高层故障数的意义在研发的角度看来,不
  是很大。''缺陷数用于统计工作产品的缺陷比统计发现了多少故障更有价值,主要是考虑按低层CQ单统计各工作产
  品的返工工作量,是吗?
\end{myquote}

\begin{comment}
  是的。所有度量体系需要的,如缺陷数、返工量,都只需要低层(工作产品层面)的就够了。要不然数值就会重复了。
  可以说,在项目的观点,我们关注缺陷数,而不是失效的次数,或是故障数。TL9000要求的,如客户问题单数,就是故障
  数。这个数值,应该是从客户支持中心的体系得到的。要不然项目就不能专注在研发活动了。这个在IPD的组织方
  面,跨领域产品研发团队要负责故障(客户问题单)统计,开发的功能团队(就是现在的项目)就负责缺陷统计。

  请留意,Rational也有两个层面的,但他们不叫``故障''(我也没叫高层CQ单人和名称,只是说``故障''比工作产品
  的CQ单高一个层次。),叫``关联任务''或是什么的。我忘了。

\end{comment}

\begin{myquote}
  再有,高层CQ单经过分析分解成低层CQ单的时候:

  1、是否按人员分解低层CQ单,比如一条高层CQ单需要修改的代码涉及两个开发人员,分解两条低层CQ单给二人,带
  有一点分配任务的性质?

\end{myquote}

\begin{comment}
  是的。按人员分CQ单是最自然、有效的。这个不是配置管理的原则,而是实际的效果。
\end{comment}

\begin{myquote}
  2、是否只需要对已基线的工作产品分解低层CQ单,还未基线的工作产品通过其它渠道通知相应人员做相应更新即
  可,这样操作的话要求项目CCB清楚工作产品的基线情况?或是只要高层CQ单影响到的工作产品,不论其是否已基
  线都分解低层CQ单,这样操作的话,这部分分解给未基线工作产品的低层CQ单不能算作是工作产品缺陷,统计的
  时候去要剔除出去,对吗?

\end{myquote}

\begin{comment}
  我的习惯是,重点不是``未基线''而是``配置项''(这个概念要抓紧!)。不是配置项,它的变更是不受控的。
  但是未基线的配置项,如果需要修改,还是需要记录在CQ单里。

  这个当然是需求变更导致的。举一个例,第一个版本的需求,需要加一个功能,导致要改一些模块的接口。其中
  有一个模块还没有开发完成,还没有``基线''。

  我还是会提一个子CQ单,到这个没有基线的模块名义上,和这个需求的父CQ单关联起来。

  处理这个模块的人,看到这个CQ之后,肯定不会先基线旧接口,再用这个子CQ单基线新接口。我鼓励第一次基线
  一个工作产品,都有一个CQ单。你们是这样做么?如果是,处理上面的例子,通常都是,工作产品第一次基
  线,已经包含了那个子CQ单。然后,他就会在那个子CQ单里记录了``已经包括在某某号CQ单里处理。'',并且在
  子CQ单里,填报做这个新接口的额外工作量,作为返工工作量。然后就可以两个CQ单一齐关闭了。

  填报额外工作量时,是比较从现在开始,如果没有这个子CQ单的工作量(估算的),和有了这个子CQ单的工作
  量(实际的)。
\end{comment}

\subsection{全能战士和制度约束}
\begin{statement}

如果看到以下的心得,大家会有什么想法?比较熟识吧。大家先自己找找问题在那里?

\begin{quote}
  这一次的需求培训,感觉最大的收获是,需要将自己的能力、耐力、智力和忍受力不断提高,也就是强调自我素
  质的提高,将自己锤炼成一名全能战士,这样就可以解决研发过程中的所有问题。多位同事多次向杨老师请求答
  案时,最终似乎都让我有这种感觉。

  道理似乎确实如此。不过如果深究的话,可能也未必如此。个体的能力本身是会有差异的,这样可以推演出
  来,个体的能力达到某一层次时,将很难再有大的突破。提高项目的整体研发能力,还是应该有一套比较切实可
  行的制度去约束个体。让每一位项目成员可以不全力以赴,也能达到整体较优甚至是长远的最优。而这一点对于
  多数没有受过专业培训的同事,特别是研发经理、产品总工和管理干部,也不大可能有先天的感知,需要外力来
  强化意识、不断教练才可得。而此种外力,在我们公司,应该有一些象杨老师这类专职人士组成的团体,以目标
  为导向,从实践中取经,理论上固化,其团队输出,便是我们可赖以方便高效简单运作的研发流程和研发模式。
\end{quote}

\end{statement}

\begin{comment}

  这是一个很典型的心得,因为我遇到过很多类似的想法。 让我们分析一下:
\end{comment}

\begin{myquote}

  \begin{itemize}
  \item 将自己锤炼成一名全能战士
  \item 就可以解决研发过程中的所有问题
  \end{itemize}
\end{myquote}

\begin{comment}

  说这个不对,我非常同意。将自己锤炼成一名全能战士还不一定可以解决研发过程中的所有问题。我也不记得说过这
  样的话。在每一个问题提出来的时候,都有一些我们可以控制的因素,有一些我们不可以控制的因素。我只是说,我们
  只能够把我们可以控制的因素做好。这个包括影响自己不可以控制的因素,但后果我们不可以控制。

  这样做,自己是否可以变成一个全能战士,不是任何人可能期待的。``全能''是很难定义的,而且是更难达到的。

  但是我们要不断自我提高。到怎么程度,要看每个人自己了。但是我们不能说:``因为有自己不可控制的因素(人家
  的,制度的问题),我们自己就可以不努力了,不提高了。不断自我提高就是解决问题的最基本的态度。 也是代表人的
  智慧和上进心。

\end{comment}

\begin{myquote}

  \begin{itemize}
  \item 个体的能力达到某一层次时,将很难再有大的突破。 
  \end{itemize}
\end{myquote}

\begin{comment}

  这个很对。但是我们达到了这个很难再有大突破的境界么?

  这个境界,因人而异,但同一个人,在不同的环境,也会有不同的层次。 环境的因素是什么?就是如你所说的:``切实
  可行的制度'',``研发经理、产品总工和管理干部的感知、意识、接受教练''等等。

  个人因素是怎么?就是自己的天聪、毅力、积极态度、上进心、等等。其中,我希望你跟大家清楚了解的是,天聪是
  最不重要的个人因素。我看着有一些非常聪明的人,一事无成,反而是不特别聪明的人,切而不懈,反而成就很大。

  就是因为世界上解决各种各样的问题的能力,有无穷的层次,而且我们就是非常努力,还只是努力了几十年而已。在
  解决现在的问题上,肯定有更高的层次。所以我们,包括我与世界上所有的人,在任何时间,任何情况,都有提升自己
  能力的空间。

  你为什么会觉得``很难再有大突破''?你不是期待着人家去改变吧? 你希望领导会给你更多资源?更多的鼓励你?你
  可以做两件事:
\begin{enumerate}
\item 改变领导,要他听老杨的课,老杨告诉他要给员工提供充分的资源!要多鼓励员工!
\item 改变自己,要能够用证据说服领导你的资源不够,同时要证明,你如果拿到足够的资源之后,的确会对领导提供
  很大的价值!
\end{enumerate}

我希望你从这里可以看得出来,问题的解决,永远都是从自己开始的。
如果你可以看到这一点,就是你的一个大突破。你的进步是会快得多,有意义得多。
\end{comment}

\begin{myquote}

  \begin{itemize}
  \item 提高项目的整体研发能力,还是应该有一套比较切实可行的制度去约束个体。
  \end{itemize}
\end{myquote}

\begin{comment}

  制度的确起到一定的约束作用。但我在中兴从来不强调这个看法,而且反对它。如果我们把制度看成约束个体行为
  的,团队的效率只能发展到一个地步,就不可以再发展了。就好像我们中兴现在一样。我们在自己文化底下,已经发展
  到最成熟的可能层次了。如果要再提高,就好像你说的一样,需要有突破。这个突破之一,就是不再用约束行为作为制
  度的主要作用!那么,制度是怎么呢?

  我们应该看得到,制度是保证效率的。

  拿交通规矩来讲。红灯停车是一个限制。后果是罚款。在中国,把这个看成一个限制的人均比例很高,所以我们有抵触
  情绪,有很多冲红灯的事故。在国外,红灯也要停车。但是冲红灯的人很少。甚至在深夜里,四野无人、无车、无警察,红
  灯时,车还是停在那里。为什么?是怕被罚款么?不是!他们有能力支付罚款。 其中的道理,是因为他们已经不把红灯
  停车当成一个对他们的限制。他们了解到,这是对社会效率的一个保证。 他是社会的一分子,所以他知道要保持自己
  的利益、价值,就要遵守这个``约束''。

  在中国,红灯是一个约束,后果也是罚款。所以我出得起罚款的,我就冲红灯!甚至有人先把钱拿出来,然后去冲那个
  红灯。这是一个低效的制度。中国有世界上1。9\%的车,却有15\%的死亡事故。我们大概每一万辆车之中,每年有9个死
  亡;欧美是不到3个!

  这个看法,把制度看成``约束'',跟我们的考核很相像。考核已经不是一个真正的激励,而是一个行为的``约束''。
  所以我也反对中兴现在的考核。因为它贬低了人的品格和价值。这样会防碍我们的发展,因为这样不能真正地激励员
  工的热情!

\end{comment}

\begin{myquote}

  \begin{itemize}
  \item 应该有一些象杨老师这类专职人士组成的团体,\ldots 其团队输出,便是我们可赖以方便高效简单运作的研
    发流程和研发模式。
\end{itemize}
\end{myquote}

\begin{comment}

  还是要等!这是守株待兔!不要这样恭维我。我承担不起。我更希望你看得到我要说的话!

  你的期待,是很难很难会发生的。其实团队是有的。公司也请过不知多少老外、本土专家!我也还在这里。我提供培训、
  咨询,到底又有几多内容是被接受的呢?使用的呢?同志们也总觉得要等领导改变。领导们更自以为自己是对的,问题
  都是在基层员工!没完没了!如此,就蹉跎了快六年了。

  不要再等了,出路只有是自己行动起来!

  不要以为我偏护领导。我这句话,是对所有人说的,包括领导。领导也要改进。但是我感觉到,很多员工对这个想法就
  害怕。领导在他们心中是神。要不然大家就进步的更快了。员工自己也需要改变这一点。

  归根结底,出路还是在改变自己(能力、思维、品格),提高自己。
\end{comment}

\subsection{从一个心得的疑惑看执行力}
\begin{statement}
  我收到的心得之中,有很多是表达``疑惑''呀,或是``这个不是最好的方法呀''等等,里面的意思,就是不接受
  一些有用,有效的概念。这个很可惜,因为在过程改进的过程中,其实是想我们提供了一个非常好的机会来改善
  我们的习惯,提高自己的能力的。

  这里就是一个很好的案例。我在这里,提供了我回复:
\end{statement}

\begin{myquote}
\begin{enumerate}
  \item IPD是什么?

    从字面看,很简单。就是集成产品开发,包括流程重整和市场驱动,强调是市场导向和投资驱动。在培训的时
    候,老师讲课的时候说只包括全业务流程的市场营销和产品开发两个部分?对此我很感疑惑,实施IPD的目标应
    该是整体最优,只改进这两部分能达到整体最优吗?我理解的IPD应该至少是全业务流程,支持流程也应该包括。
    不知我的理解是否正确?如果像教材上的只包括市场营销和产品开发两个部分,是否有必要实施IPD?

  \item 我们要改进什么

    这么多年,我们公司轰轰烈烈推行过很多思想,CMMI,IPD等等,感觉是为了改进而改进,一直不能把改进与公
    司的目标联系起来。都寄希望一个万金油,一旦使用能解决所有问题。目前我感觉IPD也有这个趋势,说得完美
    无缺,但它真的适合我们公司的环境和结构吗?当然作为我们来说能有时间学习IPD的理念,是一件很好的事
    情,但希望这些培训将来都能有所收益。
\end{enumerate}

其他由于我所学甚少,也就不谈及了。
\end{myquote}

\begin{comment}
谢谢你的心得。我看过之后,不得不提一下我的意见。
\begin{enumerate}
  \item 你说:``临时被叫去开会,培训并没有听完,。。。''

    这代表了,你在接受培训和交流的过程之中,已经损失了其中一部分的作用和效果了。 我们现在还不知道,或是还
    不承认、接受,这样同时进行做好几件事的情况,是低效的行为。 你的学习受到打击。其他的事情也未必可以专心
    做好。 这是我们不能提高执行力的很主要的原因。

  \item 你的疑惑:``只改进这两部分能达到整体最优吗?''

    IPD的作用,当然是要``整体最优''。这个观察非常对。所以我对公司的文化、氛围、方法、重点,都有很多的批
    评,因为这些批评,是我作为过程改进顾问的职责份内的事。 但是我不会觉得现在单单关注其中的两个部分,是让
    我疑惑的事。因为,这样做我们虽然未必可以做到``整体最优'',但是我们可以提高研发和产品管理的能力,可以
    做到这个已经非常非常不错了。将来的改进需要,努力,困难,都会因此而减少了。你说是么?

    所以``只包括市场营销和产品开发两个部分,还是有必要实施IPD的。'' 千万不要给自己任何借口,不去努力改进!

  \item 你的问题:``我们要改进什么?''你说的:``其他由于我所学甚少,也就不谈及了。''

   这个的确是我们问题的核心。

   我可以非常严肃地说:``我们需要改的,通过所有这些TL9000, CMMI, 6sigma, IPD, 等等,都是要改变我们自
   己,和自己份内的行为!''我们只能考虑如何把研发和产品做的更好。 我们不能判断领导强调研发和产品,是否最
   佳的方向。你现在可以考虑这个问题,但不能因此而放弃你现在的职责,万万不能,``其他由于我所学甚少,也就不
   谈及了。''参加培训就是要学呀!现在需要你把目前的任务做好,做的有效,其中包括学习好你岗位份内的事,你的
   努力,有了成就,才可能把你推举成为领导,之后,你就需要解决这个怎样推动过程改进的问题了。
\end{enumerate}

我读了你的心得,也有一个疑惑,你为什么不为``所学甚少''而苦恼呢? 你不觉得花了时间来听了培训,没有什么收
获,只是产生了一些疑惑,时间是花的不值得么?

你是否认识到一心二用是低效的呢?要专心把眼前的每一件事做好,才是提高效力的基本基础。 只有这样,才能建立执
行力。

谢谢你的支持和努力。
\end{comment}

\pagebreak
\section{讨论\footnote{\textsf{S:员工~~~Y:杨万强}}}
\subsection{CMMI与Agile,XP的比较}\label{link5}
\begin{staff}

\slogo CMMI是重量型的开发方法,内容繁多,覆盖软件开发的方方面面。培训并不能让我
  对CMMI整体有一个很清晰的认识,但对CMMI强调的一些思想和方法,我觉得还是有收获。

这里谈几点:
\begin{enumerate}
\item CMMI强调过程标准化及最佳实践,认为最佳实践就是经验教训的积累,而我们经常想当然不加实践的认为某
  个最佳实践不适合我们,就随意修改,显示是违背了CMMI思想的。
\item 杨老师在课上经常强调的一点:我们不是为了学习而学习,需要通过学习改变我们的行为方式,这样的培训才
  是有效的。我很认同。那么我想,即使我对CMMI整体认识不清晰,对于我所能认可的一些流程和方法,也需要去
  应用在工作中来尝试提高我们的效率。这样才不是白参加培训了。
\end{enumerate}

稍微跑题一点,我最近也看了一些敏捷软件开发和XP的书和资料,里面很多思想和方法与CMMI所强调的完全是相反
的。比如它强调个体和交互远胜过过程,另外有些评论也会认为CMMI这种重型方法适合于大型开发项目,对于大多
数20人以下团队开发XP更适用。这导致我在学习CMMI时有一种矛盾的心理。
\end{staff}

\begin{yang}

\ylogo 谢谢你的心得。写的非常好。好的原因在于你是思考过的。有些听不进去的,你没有提,对你有用的,你就急着
  了。心得就是这个意思,希望大家可以过滤一下听到的。

  CMMI和敏捷是两个非常不同的思路。但是他们其实也有相同的地方。

  不同之处,在于CMMI是以工程纪律为基础,要求事情时有依据地开展活动。Agile 的理论是以个人的创意为
  主,依赖员工自己的经验、积极性和灵活性来体现效能。

  两种方法的对象不一样。CMMI是考虑如何管理一般(不单单是规模大的)的团队。Agile不关注一般的团队,它假设
  员工们都是成熟的。这个假设非常重要。在某些情况下,员工可以成熟的快一点,但不是每一个员工都能够是成
  熟的。所以Agile在某种条件下,非常成功。但是没有成功推广到一般的团队的层面。

  两种方法一致的地方,在于大家都关注质量。Agile 也要求纪律,但这个纪律是远远比CMMI灵活的。后果就
  是,如果员工成熟,能力强大,积极主动,Agile 的效果非常好。但是如果员工的能力稍有不及,效果就非常混
  乱。

  Agile 是高水平的开发人员,为保持开发人员的主动性,而找到的一个符合他们个性的开发方法。但从过程管理
  的角度出发,Agile 不是一个过程,最低限度,不是一个完整的过程,他有点像 6 sigma,是一组方法。同
  时,因为它的成功,依赖某些不普遍的员工技能,它的效果不能保证。在使用Agile的时候,员工会开心好多,但
  是他们不能如CMMI一样推广积累的经验。

  我从70年代开始从事开发,对于开发的过程与方法,都有一个实际的经历和体会。Agile有其作用,尤其是在满足
  员工的积极性方面。但我认为不是一个可以在现在的中国推广的方法或是过程。

  谢谢你的支持与兴趣。
\end{yang}

\begin{yang}

\ylogo 非常感谢您的耐心解释,关于CMMI和Agile的问题,我还想再和您讨论几句,希望不会耽误您的时间。

CMMI高度强调过程,过程如果能建立完善,那么我们每个人只需要做好自己的事情:需求分析、设计、评审、走
查。。。。而我们每个人更多是一个过程的遵守者,而不是一个创造者,这一点我觉得也是妨碍我理解CMMI的一个原
因:缺乏动力(当然CMMI比较复杂也是原因)。而Agile高度强调个体和交互;比如Test Driving Development,
Pair Programming等,这些重点活动都是我们可以亲身实践的而且可以亲自主导的,因此我大概读一两本好书,我
就能大概把握Agile的思想了,而且急切想在工作中体验。

我这里并非主张Agile比CMMI好,因为无论Agile还是CMMI的倡导者都是软件工程界的大师,而且都宣称有非常成功
的案例。作为我们这些开发经验不到10年的小兵,在学习这些方法时,遭遇这两种比较对立的思想时,有时就比较
迷惑了。您提到``Agile不关注一般的团队,它假设员工们都是成熟的。'',这点我倒不很认
同,因为Agile里面Pair Programming的方法就是一个以老带新的方式,并不要求大家都成熟。

我现在的想法倒是不管CMMI/Agile强调过程或个体哪个是对的,我们先去试着做一下我所能理解的一些实践活动,
如果能真正提高我们的研发效率,就行了。
\end{yang}

\begin{yang} 

\ylogo Agile 是一系列的方法,比较实在,没有CMMI那么抽象。它离工程活动比较近,开发人员容易理解。
而且它的理念,的确是非常符合开发人员所喜欢的。

CMMI是一个过程概念,有很多团队之间协作有关的管理理念,对我们中国的情况,的确是比较困难的。我们有一个
自然的重技术、轻管理的倾向。所以国外的人觉得我们的管理落后很多。要讨论这个问题的确需要比较长的时间。
因为我们``轻管理''所以我们对于了解管理有关的议题会比较困难。

你说:``CMMI高度强调过程,过程如果能建立完善,那么我们每个人只需要做好自己的事情:需求分析、设计、评
审、走查。。。。而我们每个人更多是一个过程的遵守者,而不是一个创造者,''对于这句话,我有两个觉得不舒服的
地方: 
\begin{quote}
1)你认为CMMI因为强调过程,就会把人变成了一个遵守者。

2)每个人``只需要做好自己的事\ldots ''
\end{quote}

1)这里我们可能把CMMI看成一套规范、限制着我们的行动。希望你留意到我从来不这样看CMMI。CMMI是一系列的最
佳实践和这些实践之间的依赖关系。我们要如何利用它,实施他,是我们自己决定的。但是CMMI作为最佳实践,的
确提到很多``限制''我们的东西:要有估算、要跟踪需求、要追踪问题直到关闭。等等,很多很多。我只把它们当
成有效的习惯。如果我们有这些习惯,我们成功的机会就会大一些。在我们养成这些习惯的过程中,尤其是这些习
惯我们不很认同,我们的确觉得它是框框,可能破坏了我们的创造性。但当我们养成了这些习惯的时候,我希望你
也可以看得出来,这些习惯不但没有破坏我们的创造性,反而帮助我们在创造的过程中,可以减少混乱,可以更有
序,从而变得更有效。

2)这里我害怕你变得``个人自扫门前雪''那就可怕了。我希望你不是这个意思。``职责明确''这个概念,我们还掌
握的不好。我们有主要的任务、职责,我们一定要把他们做好,但也有一些职责之外的任务和义务,我们需要履行
的。这是一个过程的概念,也是一个态度。

Agile里的一些东西,可以在CMMI的范围之内应用。比如 Test-Driven Development, Pair Programming,都可以
在CMMI里实施。CMMI只是一个框架,让大家可以把最佳实践作为过程单元放进去。反而是CMMI的要求, Agile 的人
觉得太过分了。比如需求的变更,太频繁,就不要求文档记录。这些事情,在小团队是可以的。但是团队大了,就
会发生混乱了。这些都是发生过的。Agile 从来没有大规模地成功过。

其实Agile也要求文档记录的。但是那个记录的方法比较灵活而已。但是上面的情况还大概是对的。有些开发方法从
需求分析到编码,都用同一套图标,只是内容不断按阶段变动。这样比较方便。我们要考虑是否需要更详细的记录。
如果不需要,就不需要有好几份文档。我的个人意见是,很难完全抛开一些阶段性的文档的。

又比如,需求的了解。CMMI要求项目和需求提供者有一致的了解。在项目里也是。Agile 是否也是如此呢?它可能
不这样么?但是Agile没有一个规范的活动来达到这个目标。这里,也有如何最佳的沟通的问题。不同水平的员工,
就需要不同的沟通方法。CMMI没有限制如何得到一致的了解,只是要求达到一致的了解。

对于 Pair Programming 的目标,重点不是以老带新。他是一个任务备份和排错的方法。以老带新,是次要的,也
可能破坏它的任务备份和排错的目标。请你留意这一点。

总之,我们看到有用的东西,就要利用。第一,不要过分被``品牌''欺骗了。第二,我们要清楚``品牌''里面卖的
是什么。不要被假``品牌''糊弄了。

Agile里面有好东西,这是可以肯定地。但是从过去十多年来的发展, CMMI的确是软件成熟的一个很大的动
力。Agile可能比CMMI出现的更早,它的效果如何,也是可以查看的。

谢谢你的兴趣与努力。
\end{yang}

\subsection{理念的转变和疑惑}

\begin{staff}
  \slogo 杨老师的CMMI培训对我来说也是启蒙的。我对于CMMI的理解更是少到不知了,一些书籍凭空读起也是味如
  嚼腊。经过这一段时间的培训,开始还不觉得,但到后期感觉收获越来越大,一直到12月的最后一个周五,我自
  己在看同行评审的一篇文章时,突然觉得自己在思想意识上好象有了很大的变化,呵呵,虽然可笑,但是真的。
     
  做项目管理这个岗位,我都认为这个工作好虚。我对为什么要这样做并没有太大的认同,我的意识中实用性的观
  点比较强。CMMI等书籍中所讲述的内容,有关成熟度这个东东我也是一直不理解,我无法说服自己接受CMMI的可
  信度和对业界数据值。
    
  同样,对于成熟度的理解,我真正明白这一句话也是在两周以前。什么是能力,就具有这种级别的团队可以有更
  高的战斗力,并且可以不断的自我提高和优化,在响应和达到目标等事件中,会更有效率(前提是与普通团队的
  人员素质是一样的)。这些依据的是整个团队的运作机制,而非项目经理个体。打个比方吧,在美国社会,如果
  总统某日突然遇刺,对新闻界可能会有轰动,但是不会引起社会政治经济的动荡。在中国封建社会,如果皇帝一
  驾崩,那整个社会都会处于动荡和危险中。比较先进的机制就是成熟度能力强的机制,他能够减少个人对于整个
  组织的影响。它的优势在于它的机制,而不是在于领导。中国人比较难于接受这一点,一般说和文化有关,可能
  也就是这个道理。

  另一个问题,我们的改进是不是应该和CMMI的级别保持一致性呢?有本书中曾经说过,要按级别改进。在半个月
  前,我的态度可能和大多数开发人员比较象,认为只要有用的,用就好了,哪有那么多的说法呢?能做的有用的
  都做。现在知道,这种想法很可笑。CMMI级别还真是特别的,因为没有二级的框架,也就是不做好二级中的内
  容,其它的级别中的内容也是无法开展的(除同行评审和EPG组的建立两个工作外),CMMI的每个过程都和级别有
  着关系。其实我们从目前的混乱,到CMMI2级,也就是搭框架,搭好了我们才可以进行改进。这个在CMMI成熟度推
  行过程中的最基本的要求。这个框架我认为就是项目管理。如果项目管理的框架没有搭起来,说起改进,都不会
  有真正意义上的效果。而这其中的关键人物就是项目经理。项目经理管理着项目,肯定有着自己的管理方法,但
  为什么不愿意按规范进行工作呢?有句话说:好的过程不用推就能自己用起来。那项目管理是不是好的过程
  呢?为什么推都推不动?

  项目管理从长远看,从整个项目的的作战性看都会提高,部门的研发能力会提高,公司的研发水平会提高。项目
  管理是一个科学的管理方法。项目管理的过程应该是项目经理主动要求做的事情就对了,可是现在为什么项目经
  理却不愿意主动去做这件事情呢?我觉得关键还是源于公司对于项目的成本是没有控制要求的,对项目也是没有
  要求的,这样就导致了目前的现状。因为成本是不控制的,杨老师说的汽球效应原理中,有一个汽球的空间可以
  无限长大,所以其它两个气球都可以任意澎胀。在此情况下要想让大家按CMMI工作,本身无法在现有的机制上让
  项目经理感受到好处,反而受到约束,所以如果项目经理的只在乎近期项目进度,不在乎成本,不在乎人员的培
  养,不在乎自身管理水平的提高,再加上不想让别人可视并监控到自己的实际情况的情况下,他就会抵制。

  还有一点,整个过程做好了,过程的质量有保证了,这时是不是我们生产出的产品的质量就会高,效率就会
  好?我现在对业界的数据及其统计办法是有一些疑惑的。我觉得整个过程和这个过程中参与的人员素质也是有关
  系的。尤其是技术人员。我们只能说,在相同的人员情况下,过程好的,就会有更好的收益。同时我也相
  信,CMMI的过程对人员的培养是有帮助的,一方面是研发素质的培养,另一方面,(这个观点我说得不一定正
  确,供大家参考)我觉得如果CMMI做得比较好,其中的人员工作重点、分工等是能做得更为专业的。所以当项目
  计划明确,人员安排和占用就会明朗,此时,人员空闲时间和人员的培养计划就可以同时产生效力,同时达到比
  较好的人员培养效果。
\end{staff}

\begin{yang}
\ylogo 非常感谢你的心得。写得非常真挚。这个很重要。因为说些表面的话,对自己是没有帮助的。
你说了这些,我可能可以试图回答一下,这样就有沟通,才可能有进步的。

一个好奇的人才可以提高的快。我们需要多沟通,才可以自己提升,同时工作顺利,因为这样
减少了误会和混乱。

我可以看得出来你有转变。这个很好。同时也有困惑的地方,这个很自然,也是必然的。

1)CMMI的级别,是高层的实践,需要底层的实践作基础。事实上,没有底层的支持,也可以做高层的实践,但是效
果不能很好。

你小心想一想这个问题。我们的进步慢,为什么?其实就是因为我们要的进步,是科技上的进步。科技上的进步(三
级),恰恰需要建筑在管理上(二级)。我们的管理就是不利科技的发展的。所以科技的进步,有,但是比成熟的企业
慢。

哪些管理因素妨碍了科技的进步?我说了很多次,我们的考核,是监控行为的。所以员工变得任务驱动了。就不敢创
新,或是没有时间尝试新东西了。所以进步就慢了。

成熟的团队,是激励员工的(让我帮助你提高),不是考核(我是官,你被我管)员工的。

我希望你开始了解过程的作用了。过程远远不是流程!

2)虚与实在的概念,非常值得推敲清楚。我们的问题在于看不到长远一点的效果。所以长远的,我们就觉得虚。我
们看得到活动,步骤,所以这些是实在的。

要解决这个问题,你需要经验,否则无从判断。经验要尝试才会有的。如果我们工作,只看到活动,我们是不能积
累经验的。所以我们看不到影响深远的事情。

这个不单单是你这样想,很多高层也是这样的。这样做大事的时候,错误百出。

3)CM 是一个很有用的过程域。明显地是实在的,立竿见影的。

但是项目管理的效果其实是更广泛的。我们有项目经
理,有关木管理员,这个制度就奇怪。好像就是说,项目经理不用管项目!我们还不明白有效的管理是超越了项目的
活动的。但是我们的项目管理员,恰恰就是要来保证项目的活动的进行。

真正的项目管理包括提升员工的效能。是从员工的心态、士气来激励员工,来提高效能。在CMMI方面,管理项目要
求考虑很多我们没有考虑得因素。所以我们的项目经理觉得麻烦。

同时,因为还没有建立这个技能,就是做起来,也不会见效。所以很困难。所以过程改进个是一个长远的工作。我
们要一点一点地做。比如,估算。我们要通过一起监控周期和延误率来帮助项目经理。这个要领导的支持。

有些组织要过级,一下子什么都要做,造成很大的破坏。我们现在的考核体系,又不鼓励项目经理做过程改进的事
情。所以他们不愿意做。如果组织希望项目经理做过程改进,领导就需要订一些过程效能的指标,每年都要提高。
这样项目经理就要解决过程效能提升的问题。

好了,说的太多,也不知道是否对你有用。

谢谢你的支持和努力。
\end{yang}

\begin{staff}
  \slogo 谢谢杨老师的答复,您的回答对我帮助很大,尤其是对项目管理工作的推进方面,又给了我一些想法:以
  前我更多的是关注于具体的操作性工作,对于流程的关注超过过程,我总是力图在寻找与项目经理之间对管理认
  识的共同点,希望从他们的认同来帮助推进工作的开展,让项目经理从这些具体的工作中体会到项目管理工作的
  作用,呵呵,从这两个月以来事实证明这种做法的效果在项目管理推进方面不是很有效。如果想要项目经理能够
  自自觉自愿的去思考自己项目管理工作的改进与提高(我指的是长远意义上的改进和提高),优化我们的考核机
  制,建立指在提高项目管理过程改进的考核体系的效果可能会更好。

  希望以后还能持续的获得您的帮助和指导,非常感谢!
\end{staff}

\subsection{做一个高效的领导}

\begin{staff}
\slogo 刚刚开始负责做一个项目,项目规模比较小,项目成员两个人,目前做了1个多月,有两点体会比较深。

1.开始做这个项目的时候会觉得有些流程很麻烦,一共就两个人,本来直接沟通一下就可以交代清楚的事情,偏
偏还要走一堆的流程。尤其是项目紧急有时候项目管理工具又出问题的时候,确实觉得很烦。而且我自为我对项目
的计划安排、风险等等都有比较充分的考虑,有时候觉得流程有些多余。但是有一次开发人员跟我建议说,能不能
让相关的模块项目提前把他们的版本计划告诉我们,这样我们才好有针对性的准备测试用例等,否则的话每次模块
更新都要浪费比较多时间,而且也会造成我们自身任务的一些调整。这件事情对我的触动比较大,让我认识到项目
规模再小,也并不是一两个人的事情,涉及到方方面面。即使有些事情只有我一个人心里清楚也是不够的,所以就
开始非常认真地来对待项目计划、版本计划等。这样就逼着自己不得不做好计划、不得不提前把很多问题考虑周全。
真正投入去做之后,发现对自己的思路的整理是非常有好处的,也有了一些新的体会。比如说以前我并不觉得项目
计划和系统设计有什么特别大的联系,但是我现在觉得项目计划不能细化的原因之一是因为系统设计、子系统设计
做得不够好。

所以说,有时候考虑问题的角度不一样,会得出不一样的体会。面对一个新事物的时候,尽量抱着学习的态度逐步
了解,从多个角度思考问题,不要太武断。有些事情不投入不去用心体会,永远都只会浮于表面肤浅的认识。

另外我的感觉是我之前对管理本身的成本估算的不足。其实制定项目计划、跟踪、评审、走查、协调等等都是需要
比较大的工作量的,这些在做计划的时候都要充分的考虑到。

现在还比较困惑的是,对于目前项目的现状来说,我不得不肩负起方案设计甚至很多技术细节的研究上。感觉很难
做到管理和技术发展的平衡。我很希望项目成员的专业技术水平能够快速的提高,可以分担一些这方面的压力。大
家的水平都提高了,整个项目才能有比较良性的发展,但是目前每个新项目基本上都需要从头开始培养新人,所以
感觉这个愿望很难达到。


2.目标很重要,做每一件事情都要明确自己的目标是什么。比如说我觉得做风险和估算是为了更好的制定有效的
项目计划,而不是为了完成风险和估算这个流程;代码走查的目标是为了尽量多的发现缺陷,而不是为了追求代码
走查覆盖率。如果没有明确的目标,那么很容易在过程中流于形式。所以在项目管理的实践过程中,我很希望有人
来告诉我做这个是为了什么?而不是因为这个流程存在,公司要求用这个工具,所以你就必须用。而目前的培训大
多是关于工具的使用培训。希望能有象这次CMMI的深层次的培训,可以在更多理念上、理论联系实际的培训。

说到目标,我觉得我们现在项目管理的目标并不很明确。为了提高效率还是保证质量?如果是提高效率,用什么来
评估工作效率?人均代码行?我觉得目前公司和部门并不追求这个,反而是尽量多的估算工作量,才能不断地扩大
项目规模,因为没有度量,或者有度量,但度量数据没有得到正确的分析使用,也不好衡量到底效率如何?仍然是
根据感觉上来评估,只要看起来大家都在忙都在加班就是工作量比较饱满,而且也并不知道哪些是在做无用功。相
比之下,保证质量的目标就清晰的多,大家都不希望作出来的东西有问题。但是有的时候,保证质量是可以通过降
低效率来保证的,举个极端的例子,1个人写100行代码,通过10个人来走查,再通过10个人来做单元测试,代码质
量的确可以保证,但是效率显然是比较低的。这个就像我们作路径搜索一样,连现在的目标是找最短的还是找最快
的都不明确,那么每一步的决策依据是什么呢?我觉得这个可能是造成项目管理执行比较大阻力的原因之一吧。


可能我现在做得这个项目比较特殊,在公司内不一定具有代表性,不过我还是希望能够通过过程中的实践,能够对
项目管理有更深层次的理解,养成好的习惯,积累一些项目管理经验。到时候也可以有一些度量数据来分析一些问
题。

\end{staff}

\begin{yang} 

\ylogo 非常感谢你的心得。写的非常好。

对于你发现过程的必要性,是非常重要的一个体会。我们大部分的经理,都不能好像你这样,了解到各个方面的因
素之间的关系。所以过程改进非常困难,因为我们看不见问题。

这样单单考虑一个因素,不理解多个因素的关系的管理理念,我把这个称为``单维思维''。在中国,这个比比皆是。
因为我们太关注自己,太关注短期。结果是项目可以完成,因为我们能干;但效率就相对低下,因为我们管理不善。

你心得提到的第二个因素是项目目标。我们有考核指标,但是你还提到这个,就代表考核指标未能明确项目的方向。
最近流行的``平衡记分卡''(balanced scorecard)就代表评判一个对象,是复杂的。我们的考核指标,因为我们
自己的单维思维,往往是不全面的,造成``气球效应''。

解决的办法,在流程的角度来说,就是``平衡记分卡''这类的方法。而从过程的角度,项目的目标,永远都是生产
力。也就是效率。有了效率,我们才能有效地满足任何在周期,或是产品质量之间的选择。

第三个因素是技术水平。你的项目只有两个人,每一个成员,一定都得做多个角色。所以实施和理论的分别是比较
大的。就是说,结合理论与实际是比较难的。

如果项目比较大,比如有8-10个成员。那么,作为一个项目经理,就要承担管理的任务,而不是技术的任务。员工
的技术水平,不能一朝一夕可以提高。但是符合培育技术水平的管理理念是让员工承担责任,并且提供指导。要员
工承担责任,就需要不断地告诉他们你对他们的期待,比如他们要具备什么技能,承担什么的责任,进行那些判断
与决策。这就是授权。有了授权,员工才有提升的动力。

让员工承担责任,就是让员工参与,让他们接触高层设计的思考方式。也让他们作判断,但是要辅导他们,要提供
质量指标,什么是好的,什么是不对的。我在贝尔实验室的一个项目,就预早设定了方案的质量准则。不单没有延
误,反而因为大家都清楚要求,整个项目的效率得到很大的提高。产品质量也变得更可维护、更兼容、更易用。

最后一个因素是质量指标。你用目标来引入这个非常重要的过程概念:质量指标。目标驱动,就是需要这道我们要
达到怎么效果。这个也是质量指标之一。成熟的管理,是重要的工作产品,都有明确的质量指标的。这个我们需要
交流,或是自己积累经验。

你提到的问题非常好。只要你努力去找答案,你一定会成功,变成一位出色的领导。

谢谢你的职持、参与、兴趣、和努力。
\end{yang}

\subsection{关于度量}

\begin{staff}
\slogo 心得:
\begin{enumerate}
\item 身为EPG成员,对流程的改进应该从自身做起,前期只是顾着编写文档,并没有意识到在规程编写过程中如果
  体会到有需要改进的地方就应该着手改进,延迟了改进的时间,现在开始做起应该不晚
\item 度量是一个长期的过程,不能因为初期某些数据不准确或不真实就放弃收集的工作,需要持之以恒,慢慢积
  累达到最终的度量目标
\end{enumerate}

问题:
\begin{enumerate}
 \item 如何使公司目标与度量目标保持一致,就是说我们到底要度量什么,由谁判断,怎么判断
 \item 度量数据初期可能不准确,而且初期数据收集的样本肯定不会太多,如果这些不准确的数据融合在一起进行分
析那么得到的度量区间肯定是不准确的,如何用这样的数据支持项目估算
\end{enumerate}

\end{staff}

\begin{yang}
\ylogo 谢谢你的心得。非常高兴你听到料两个非常重要的观点。你要在实际情况中应用这些观念,就会对你有帮助。

回答你的问题:

1)每一个人都需要能够从高层目标引申出来底层的,或是实际层面上的目标。这个包括你自己。

   你要多试一下,就会学会的。比如,公司的目标是通过快速响应,提高客户满意度。
   那么,项目的目标就是缩短周期,缩短变更请求的处理时间,降低通过系统测试的版本数,等等。
   为什么会是这些呢?因为这些都是与我工作有关,并且与业务目标有关的。

   是不是这些事最好的?我们尽可能找最好的,最关键的。但我们不用证明他们是最好的。这个不是很重要。
   但是需要是有关的、重要的。

   所以,项目就需要度量周期、变更请求的开始与完成时间,与中间版本数。

   能够有这个能力,是你的责任。你要勇敢地争取,建立,这个能力。不要第一个想到的就是``很难''或是
   ``最好不是自己的责任''。

2)你还是不愿意放弃``不完美的度量是不可用''这个错误的观念而已。我们要接受``不完美的度量还是很有用的'',
   只要他们是真实的。还记得什么是``真实''的数据么?真实的数据就是那些按定义收集,没有其他居心的数据。
   这样的数据,就可以用了。

   当你能理解第一个问题的时候,知到目标的重要性,如何制定目标,如何制定度量目标的指标之后,你就自然地
   知道如何回答这个问题了,因为你就知道如何使用这些数据了。

   当那些数据的确是不能用的时候,也是那些数据自己告诉我们的。小心思考这个问题。

如果不明白,你把你的思路告诉我,我看可不可以解释清楚。但我希望你先思考一下:举一些实际的案例,为什么
你觉得一些数据不可用?为什么数据不多有不可用?你是否不想去开始分析数据?等等。

谢谢你的支持,参与和努力。
\end{yang}

\subsection{严厉惩罚没有改变习惯}

\begin{staff}

  \slogo 问题的提出:项目强调的一些研发流程注意事项,执行一段时间之后,项目成员经常会有意无意的忽视,与项
  目期望培养项目组成员逐渐养成好的习惯的理想背离。

  问题分析:人是有惰性的,项目管理者应该通过抓几个典型严厉惩罚还是应该通过经常提醒来实现目标呢?项目以前
  的做法是对于严重违反流程的项目成员直接从经济上处罚,而对于轻微违反者给予警告。但是过一段时间之后,大家
  比较容易善忘,工作没有做到位。可能是大家研发工作很忙,为了赶进度往往会忽视了一些工作的流程规范工作。或
  者认为工作流程规范工作并不是那么重要。

  解决措施:抓住最佳时机项目成员的不同态度和存在的问题,进行教育,转变认识,提高觉悟。一是晓之以理,对项目
  成员的片面看法和不正确的想法,要通过摆事实,讲道理,予以剖析问题的根源,并针对性进行教育批评。可以在项目
  总结会议上问违反工作流程规范的人,请他解释为什么不按照公司规定的工作流程来工作,听听他们的道理。如果他
  们说不出理由,我想他们会改进的。如果再不改变,继续问他。我想他们不会有不改进的理由。二是动之以情,要以满
  腔的热情来讲明按照流程工作的好处,消除误解。让他们明白按照流程工作不是束缚研发工作的枷锁。三是导之以
  行,作为领导者要以坦荡的胸襟、高尚的情操和规范的言行,在具体的工作中作出表率,使其感到愧疚。同时,应该广
  开言路,变被动为主动,化消极因素为积极因素。
\end{staff}

\begin{yang}

\ylogo 非常感谢您的心得。写的非常非常好。有一个很小的问题,希望讨论一下。您说:

\begin{quote}
  一是晓之以理,对项目成员的片面看法和不正确的想法,要通过摆事实,讲道理,予以剖析问题的根源,并针对性进行
  教育批评。
\end{quote}

这是一个很大的进步,比罚款好得多了。因为款是罚了,但是员工还不知道这个行为的目的,也不知道领导的意图,是形
式权威呢,还是希望规程有效。如果目的是要规程有效,领导一定很关注。自然会以身作则。所以其实员工是会被这些感
动,而不是罚款!

好,我们知道罚款无效这个道理了,那么,改进的,就拿人家的``不正确的想法''啦,自己的``事实、道理''啦出来,很
合理吧?哈哈,有问题!我们还是把重点放在对立的层面上。领导应该考虑的,不是员工是否``不正确!''否则,人家感觉
到就不爽。那怎办呢?

您是领导,您的职责就是要决定事情需要达到什么后果,因为作为领导,您需要承担这个责任!所以,领导应该建立目
标,并且让每一个成员都接受,这个就变成了``共同目标!''领导不是骑在员工背上,不是批评员工不正确。领导只是履
行自己的任务,制定目标而已。这样才是上乘的管理。

希望您好好思考这个问题。但不要急。可能需要好几年才可以领会,然后再过好几年才能有效利用。

谢谢您的支持与努力。
\end{yang}



\subsection{如何对待破坏性的考核}
\begin{staff}

\slogo 个人认为所谓的专业,就是要负起责任,为自己的行动负责,其他的都是对此的注脚,仅此而已。

做一份工作,就要对得起这个职位。我们不需要以当前社会环境的恶化、生活成本的增加而怨天尤人。我们需要做的
是,关注自己能够控制的因素,增大自己影响圈。具体来说就是增加工作需要的专业知识储备,主动积极的表现。余世维
的讲座中曾经提到,世界上有两种人,一种人不停的找借口,另外一种人就是不停的表现。表现,不是平常意义上的出风
头,只是一贯的为自己的行动负责,是做事让人放心,让人们觉得自己就是这行的人。
	
另:我有一个困惑,考核基本上是针对个体考核,故障出来了还是个人负责,总要找个人为团队的错误``背黑锅'',导致
了现在的各人``自扫门前雪''的现象,为了赶自己的任务,对评审等团队事项草草了事\ldots ,不知如何改善这种局面?
\end{staff}

\begin{yang}

\ylogo 我不十分同意你的``其它的都是对此的注脚,仅此而已''这个说法。
承担责任是第一步。其它的是高一个层次的关注点而已。

回答你的问题,对于考核的问题,你需要:
\begin{enumerate}
  \item 了解它的问题与局限性。你已经做到了。
  \item 接受这个现实。
  \item 防止它对你产生的坏影响和破坏性。
  \item 在有机会的情况下,提供影响,推动它的改进。
  \item 当你是领导的时候,千万不要这样考核人家!要把重点放在培育员工上。
\end{enumerate}

谢谢你的关心与努力。
\end{yang}

\subsection{培训低效的原因}

\begin{staff}

\slogo CMMI培训几年来已经组织过多次,但是效果并不如人意,个人有下述原因,不知是否正确:
\begin{enumerate}
  \item 公司各事业部(研究所)各自为战,步调不统一,造成执行混乱,跨事业部项目很难操作,跨事业部资料
管理很难执行

  \item CMMI的实施应该是分阶段的,阶段怎么划分,各阶段应该做哪些工作应该有一个详细的计划。大家都知道
项目运作的基础是做计划,可是CMMI推动执行本身就是一个影响很大的项目,这个项目的计划在哪里?这个项目的
管理很粗犷,令人很难信服。

  \item 培训目的不明确,受培训人员在现阶段应该做什么没人能说清楚,培训时什么都知道,培训后还是不会做
也不能做,培训总是不自觉地落入理论上可行的怪圈。

  \item 好的项目管理需要存在于好的组织环境下,现在组织过程还没有定型,项目外部环境不规范,所有经过项
目外部的文档都需要项目组自己去臆造,这样的``自说自划''严重违背了项目管理的精神,要想搞好项目文档就自
然而然地变成搞好``形象工程'',劳民伤财。
\end{enumerate}

\end{staff}

\begin{yang}

  \ylogo 非常感谢你的心得。你的积极性是不可置疑的。可是我得指出一个特点,你的心得里没有课程里的东西。你在
  听过这些交流之后,没有去想``我听到了什么?'',就去解决问题了。你要解决的问题很大,你想到的原因可能都对。但
  是这个心态的确不能保证交流是有效的。因为,我可以确信,你的原因,不需要这个培训就可以得到。所以其实这个不
  是这个培训的心得,而是你平常(可能早已形成)的心得。

  你的原因之中,只有``培训目的不明确''可能是与这个培训有关的。我知道你谈的培训,可能不是这个培训。但是用
  案例说问题,比较清楚,我就用我们的培训吧。

  我们的培训,包含:CMMI概念;项目管理;度量。在每一个课程里面,都有理论的方面,也有实质的方面。比如,CMMI模
  型里,有很多成熟的习惯,承诺、主动、目标驱动。也有政策上的问题,比如,如何激励员工,质量和进度的平衡,先把
  项目管理好,着重过程,而不是员工的过失,领导的责任等等。你一点都听不进去么?没有联想到日常工作了么?

  其他的两个就更多指导性的内容了:要做估算、策划是要考虑哪些因素?如何监控?要用S曲线,要有问题单,等等。

  度量里有如何按业务目标制定项目的度量目标。

  你觉得那些不会做,那些不能做,就是一个好心得呀!我们从中可以得到一些启发的。

  你觉得培训的目标不明确么?目标是改变学员的行为!这个需要你的支持的。如果你有支持的态度,就会发现其中相
  关的内容。如何应用在日常工作里。应用的时候有什么困难。这才是心得。

  学习,要把重点放在自己身上。这不是说自己以外的事情都完美了。只是说,这样做是低效的。如果我们把重点放在自
  己身上,就知道自己如何可以把事情做得更有效了。

  培训的效率不高,你说的原因都对。但是我鼓励你不因为要关注这些,而忽略了一个非常重要的原因,而且是你可以
  控制的原因,就是:培训只是一个刺激,效果需要学员的主动性去吸收。

  我在你的心得里,感觉不到你吸收到任何东西。

  谢谢你的参与。希望你继续关心培训的效率。

\end{yang}

\subsection{培养兴趣和有效的心态}
\begin{staff}
  \slogo 杨老师,我在听你讲《专业态度》的过程中开头的一个问题还是引起了我的一些想法。可能和培训的主题有
  些偏差。就是在开始您,提到专业的态度的几个方面中,有对工作的内容,也有对同事的内容。对于对自己的内容您在
  培训教材中放在了首位。我想,对于外在的因素的了解和运用,应该还是相对来讲更加容易些。我讲容易些主要是指
  可以通过看其它的同事做事的方法学习,来做到这些要求。

  而在对自己的要求中,你提出来了,要做自己喜欢事。在这一点上,我一直有些疑问。如何区别自己喜欢做的事和现在
  的工作之间的差异呢?我认为工作是一种生存的手段,因此在做工作的时候,在定义喜欢与否的时候是要加一些限制
  条件的。有很多时候,对于工作和生存是不得不做的事。而自己喜欢做的事有时候还是和工作有差别的。有时候有人
  评价我是把自己喜欢的事当成了工作,所以就做了现在的工作。所以我就有些困惑。您是走过很多大公司的人了,肯
  定会有很多经验,所以我还想了解下,对于我这样的问题,您是怎么看的呢?
\end{staff}

\begin{yang}

  \ylogo 我把对自己的要求提在第一位,这个跟修身、齐家、治国、平天下的路程,从自己做起是一致的。 就是
  说,我们要从自己做起。作的方法的确是观察:观察外部的客观世界,这其实就是观察``自己''的周围。 我们太关
  注``事实、真理、技术'',我在这里希望大家了解的,是``观点、态度、方向。'' 希望大家可以尝试一下这个新的
  体会。这样做,进步会快一点。

  做自己喜欢的事,也是一个心态。把工作作为生存的手段,也是一种心态。但是前一种心态,有助提高,甚至有利延年
  益寿。因为这样我们会开心一点,内心压力小一点,效率会高一点。所以我们要培养这个心态。因为这样对自己有利。

  如何知道自己喜欢的是什么东西,的确是一个困难的问题。了解自己、培养心态,都不是一朝一夕可以达到的。 我知
  道,因为我走过这个过程,的确是漫长而需要努力的(但不一定是辛苦)。开始时自己以为喜欢科学。 后来因为工作的
  改变,变成做软件工程了。我才发现,``软件''和``工程''都是我的最爱。现在我做过程改进,这个也是我十分喜欢
  的。我还在发现自己,这条路我还没有走完。

  同样,您也要自己决定如何处理自己的前途。

  要了解自己,在自己开心、苦恼、等等情绪底下,要问自己为什么会这样。慢慢就会了解自己了。 当然还有其他方
  法。您自己要努力去发现,要踏开第一步。

  同时,要记着,兴趣、喜爱,在一定程度下,是可以培养的。现在不喜欢的,未必将来也不喜欢。 我们需要有一个开放
  的心态,能够包容。如果能够把现在不特别喜欢的工作,改变成为自己喜欢的,的确是一个正面的发展。所以有正面的
  态度、方向,才能培养有效的心态和兴趣。

  再多加一个如何培养对工作的兴趣的方法:要知道自己工作的重要性。然后要有成就,兴趣就会来了。 要有成就,就
  要学会如何把事情做好。这个听起来好像是在循环逻辑。事实上的确是循环的,从底层开始,越循环越提高。我们知道
  需要有钱,才可以挣钱。只是开始的时候只有一丁点点钱,后来越来越多钱而已。 能否挣钱,与开始的时候有多少钱
  没有太大的关系。

  我是一个非常幸运的人。我在 BellLabs 的经验非常宝贵。我可以自己做很多自己喜欢的工作。我现在在中兴,也能
  够在这个环境里,做我爱做的事情,说我爱说的话。大家不能有这样的自由。所以我知道我是非常幸运的。但是大家也
  不是完全没有自由的。不很多,就好像开始的时候只有一丁点钱,所以更需要尽量利用。

  谢谢您的努力,希望您进步。
\end{yang}


\subsection{气球效应的感想和研发周期}
\begin{staff}
\qlogo 
\begin{enumerate}
  \item 关于气球效应:
    我个人比较深刻的体会到这个效应在我工作经历中的历史变迁和教训。从我最初加入EPG时候,工作方向先被
    领导定位为测试。可是测试改进什么呢?发现效率比较低,于是就加强自动化测试;后来发现系统测试缺陷较
    多,前期开发自测较少,就开始推动单元测试。实践中发现单元测试不好做,原因是设计可测试性不好,就进
    而抓设计...... 无论做哪一种改进,都是气球上的某些部分,但是总体效率是否得到提升,并没有得到足够
    信服的分析;

    后来有一些思路成熟了,转移到从整体的目标和过程来分析和考虑改进,但是又陷入了一个怪圈。影响目的的
    改进方面较多,从每个方面都进行改进。工作比较繁重,项目组做的很累,最终分散精力,其实每个方面改进
    都和理想目标差距较大。如果找几个关键点改进,就又回到了从前的方式。一个主要原因还是我们没有能力把
    影响最终目标的因素之间的关系搞清,无法把最有效的改进方案筛选出来。有的时候,可能是我们太关注细节,
    关注精细化和量化,遮蔽了宏观的规律。

    认识到了气球效应会给我们带来什么呢?假设我们的目标是改进生产率P, P=f(x,y,x...),这里的x,y,z是影
    响生产率的各个因素。其实每个因素都可以看出和P有促进或阻碍的关系,或许我们还能模拟出相关曲线;但
    是实际的因素之间还有关系,并不是独立的。很多非常关键的因素没有量化的方法,比如员工的积极性,态度,
    经验等。更不要说很多重要因素并不在我们的过程改进范围内,比如考核,薪资,文化。如果能够找到一个最
    大值(或者极大值)整体的改进因素方案呢?是一直困扰我的一个问题。

    这个让我想起一个数学界的例子:在金融数学专业,有一个学科专门研究利率的预测;数学家们采用一些数学
    统计分析方法,对这些``随机''的利率时间系列进行分析,来预测未来的利率值。统计中当然就会有一些假设
    模型,这些模型多么体现实际情况,如果没有特殊原因,预测准确率实践下来的确很好。但是国家的法规变化,
    宏观调控行为一旦发生,远远就跳出了预测的范围。数学家的精心计算,远远比不上一个政府内偶然听到动向
    的清洁工的预测;

    同样的情况也在我们的研发过程中,独特性的因素往往压倒了普适性的因素,非过程的因素对项目的影响更大。
    比如代码走查在两个部门进行推广,人员经验、流程和方法一致,效果相差10倍以上。  

    另外一个方面,气球效应也给我另外一个启示: 研发过程改进, 仅仅是研发这个气球中的一只耳朵。我们再怎
    么挤它,其他方面没有协作一致的配合,还是不能把它挤大;怎么配合起来挤,就不是研发过程改进能够全局
    考虑的了。我们经常埋怨某些东西从自己的角度看明显不合理,为什么领导会犯这么大的错误?其实我们并不
    能全局平衡,就像一个盲人是不可能搞清楚大象到底怎么样的。   


  \item 关于缩短研发周期:

    单纯的强调周期指标并不可取,我们很多同行,拿出一些最原始的数据,告诉我们他们是缩短了百分之几十的
    周期;但是从来没有提到规模多大,确实可笑。我们提出了几年把周期缩短一半的目标,也一样经不起推敲。
    生产率(规模/工作量)也不是一个好的指标。太简单和便利计算的指标,只能给出一个大体的趋势,难以给出
    指导和评测作用。 我觉得一个稍好一些的指标是:同类产品或相似功能上市,比最强的竞争对手快多少比例,
    当然还要除以投入人力和成本;
\end{enumerate}
\end{staff}

\begin{yang}
\ylogo 高手就是高手。非常好的心得。看了之后,我有两、三点的体会吧:
\begin{enumerate}
  \item 在您谈到的改进的困难,不知道是否有总体提升,后来谈到对最终目标影响的因素不明确。
   这类经验我在国外也有,但是在中兴体会更多。在国外,的确成功的机会比较大。
   在中兴,相对来说,困难得多。

   其实,改进的确是一个一个活动地关注的。比如测试,自动化测试,单元测试,等等。任何一个都是可行的。这个几
   乎不需要分别那个75\%,那个69\%。我们的问题不是没有抓对因素,而是没有贯彻改进。我不是说这个是您的过
   错,我只是支持您的观点,我们不清楚某些因素。或是我们没有有效地面对这些因素。比如:这些改进是否形式的?员
   工是否自愿的? 其中最重要的,是否是领导支持的?这些因素,很少在模型了提到的,因为这个在成熟的团队里,是
   比较明确而稳定的。

   如果我们坚持其中的一个改进活动,我们自然应该可以在控制图里看得见趋势的。就是有些人做的很成功有效,有
   些人效果不大,但也会反映在控制图里。有一些点特别好,就是好员工的,有些点特别差,就是不用心的员工的。但
   是趋势应该是可以看得到的。

   我们的价值观的确是有问题。太短视和急功近利了。反映在激励体系里,导致形式化。导致改进活动不能贯彻好。
   个人因素对这个坏影响有一点点的缓解作用,效果就立刻上来的。这个问题不解决,就会有您观察到的特殊因素破
   坏了随机因素的改进。就好像,政策因素,往往破坏了改进的努力。
   
   但是我们不理解这点,连正式SCAMPI评估也不小心观察通用目标和通用实践的效果,所以连评估结果也不反映这个因素。

  \item 你提到``我们经常埋怨某些东西从自己的角度看明显不合理,为什么领导会犯这么大的错误?'' 我们是否觉得
   批评领导非常不自在?其实我们可以看见的领导的错误,从企业总体来看,也不是那么大。朗讯的领导的错误,把朗
   讯搞垮了,才真正大呢!但是在种种政策上,的确是做成了一个低效的团队。困惑您的以上的问题,其实就是一个典
   型的案例。您将来也会看得很清楚的。这个错误造成员工的被动心态,限制了效能提升,但保持了企业里一定的稳
   定性。这是我们需要分析和面对的。

  \item 周期的确是响应能力的度量,未必一定要在规模可比的情况下才有意义。如果有这样的要求,那么我们用生产力
   就够了。周期代表一个企业研制版本的能力。有些企业长于大规模的产品,有些方便开发小产品。但是他们的周
   期,作为内部的指标,独立来考虑,意义还是重要的。如果要把周期横向比较,就需要考虑其他问题。比您提到的因
   素更复杂的多了。但是这些不是同一个``维''了。需要用几个指标(指示器)一起来分析。
\end{enumerate}
谢谢您的支持和努力。
\end{yang}

\subsection{改进文化的移植}

\begin{staff}
\slogo 我负责的工作是内部组织绩效考核,针对各部门和产品分别设置了考核指标,06年下半年,每个月都对这些
  指标提取了数据。数据是有了,但是如何分析?如何使用?要让这些数据能够起到考核导向作用,这是我们考核的目
  的,不然就成为一堆死的数字了。这次项总希望借您过来的机会,对如何分析和使用这些考核数据,想跟您请教和探
  讨,让我去参加您的培训。结果是,因为我们的需求和您的培训内容有较大出入,对培训的专业内容我基本上没有听
  得懂,有种腾云驾雾的感觉。

  但是,有幸亲耳聆听大师的培训,感受杨老师严谨认真的治学做事风范,获得大师在思想方法上的开启,实在感觉所
  得收获已在培训内容之外了。因为第一次培训您希望我不要听,第二次我没敢再去。

  有一个问题,我听懂了,并且也跟着杨老师着急,您说CMMI在中兴已经有5年了,但是项目管理仍然是按原有的方
  式,几乎没什么成效,这实在让人摇头叹息,为什么在国外很成功的管理方法,引入中国本土后却步履维艰?叹息过
  后,我又觉得也能理解,因为是这个管理方法所根植的环境发生了变化,它从西方搬到中国,面对的人不一样了,而东
  西方人不同的文化背景带来的价值观、行为习惯,直接影响着管理方法实行的效果。

  打个植物的比方,我们不能期待相同的植物能在差异很大的土壤环境下,开出一样的花朵,结出相同的果实。怎么办
  呢,我觉得要么改变土壤环境,消灭环境差异,但是要彻底改变环境又几乎不可能,因为这从根本上说,需要改变的是
  东西方文化的差异,这个难度太大了,努力使环境加以改善也许还有可能,要么就是对这个植物进行稼接,让它在新
  的土壤环境下,开出与原来不一样的,但是同样美丽的花朵。

  但是我们现在的问题是还没有种下去,或者说可能是种下去了,可是没长出东西来,还谈不上稼接改良呢。所以,我
  觉得要先消化再慢慢吸收营养发挥作用,这第一步的工作,要种下去,要让它长出个东西来,这一步的工作最最重要
  的是需要领导强有力的推动。可是领导也未必懂这个东东,要让他下决心发挥这么大的作用,他需要先认识才行,就
  要看杨老师您怎么推动领导了。

  呵呵,我一个CMMI的外行,斗胆冒昧向杨老师发表这些怪论狂言,是因为知道杨老师您在主导这项工作,每个人都渴
  望工作能给自己带来成就感,还有是因为热爱这个企业,希望企业的表现能尽如人意,这就是培训后的感受。我个人
  比较认同西方文化,信仰耶稣基督,凡事为荣耀主而行,并从主那里支取力量,这是我工作的动力所在。我觉得任何
  一项大的变革,都是需要对这项事业有宗教信仰般的热崇才能有效果的。

  有机会希望听到杨老师对考核工作的指导意见或者建议,今后要是有杨老师您做的CMMI初级的培训,尽管可能与工
  作不相关,我还是希望能够参加。
\end{staff}

\begin{yang}
  \ylogo 文化的适应性是非常重要的一个议题。不同的土壤,种出来的结果可以非常不一样。这个问题经常有人提
  起,在改进的过程中,也确实是必须面对的一个关键问题。在转到改进的主题之前,让我们看一下另外一个非常重
  要,非常大,但是非常有启发性的一个案例:中国近二十多年的成就。

  记得在二十世纪初期,我们是一个次殖民地的状况。我们谈德先生(democracy)、赛(science)先生救国。最后是
  社(socialism)先生初步成功了。但是社先生就一帆风顺么?当然不是。最后有人知道要美丽的花,社先生需要让路。
  我们要市场经济,虽然我们说这是有社会主义特色的市场经济,才能开出美丽的发展花朵。如果这个花朵,不能在计
  划经济的土壤生长,只能在市场经济的土壤生长,我们是种其它的花呢,还是改变土壤呢?

  好,有了市场经济,人家就说,市场经济需要要民主体系这个土壤,才可支持发展。但是,到现在为止,我们没有完美
  的民主体系,但还在发展。老外就是模不着头脑。是这个花朵其实是需要这个土壤呢,还是我们有什么方法改变了
  这个花朵的品种,需要其它的土壤了?

  这个案例,这是可以观察到的现实。我没有投放太多的分析。这个请大家自己去做。但是事情有客观的规律,也有
  主观的意愿。

  回到过程改进,我们的花朵是过程效率。也就是企业的效能。现在我们有一定的文化(土壤),我们要问的,就是我们
  要其它的花朵么?比如``人治''而不是``以人为本?''从严谨达到效能呢,还是通过马虎达到效能? 第一个问题就
  是:到底我们需要什么样的花朵?还是种那一种花朵不重要,我们的土壤生出什么花朵,就什么花朵都行?如果不知
  道,这个讨论就变得没有意义。``如果我们不知道要到哪里,哪一条道路都可以!''

  第二个问题是:我们知道要什么花朵了,但我们是否知道我们的土壤是否适合这个花朵?如果不知道,这个讨论就变
  得没有意义。``如果我们不知道现在在哪里,有了地图也没有用!''

  第三个问题是:如果我们明确知道我们要什么样的花朵,又知道这个花朵需要什么土壤。我们是否愿意改变我们的
  土壤?还是我们不要这个花朵了?

  我们是知道我们不需要这个花朵,找到另一个更好的花朵呢? 还是我们不愿意改变自己的土壤,因为我们不以为这
  个土壤是关键的?
\end{yang}

\subsection{幼稚的管理模式}

\qlogo \begin{staff}在部长的安排下,12月30日我们集中学习了CSQE(美国注册软件质量工程师)的知识体系模
  型,大家感觉这一知识体系对于过程改进人员很有必要,非常希望能参加相关的培训和认证。但由于CSQE培训和认
  证在国内尚未全面推广,所以今年我们计划依据CSQE知识体系模型,以自学为主、以内外部培训交流为辅全面开展
  个人能力提升活动。近期我们将建立一份过程改进人员的技能评估表,并建立年度的个人能力提升计划,请杨老师
  多多指导,也非常希望杨老师能在公司层面推动CSQE培训和认证。
\end{staff}

\begin{yang}
\ylogo 过程改进最终是通过人来实施和体现的。所以如果员工的能力提高不了,或是员工的积极性不高,过程效
能是很难提高的。

可怜我们的政策,如监控吃饭时间,如何穿衣服,数有多少个电灯没有关,更严格地刷卡,等等,受益是微乎其微的,但是
对员工的技能没有提高,同时打击积极性。把人当成工具,这是非常低级、幼稚的管理理念。当然,员工自身需要专业、
守时、端庄、节俭。但是需要用对待专业人员的角度来谈这些问题。

反而,我们应该关注员工能力、素质的提高。在这方面,CSQE是其中一个非常值得参考的方法。
\end{yang}


\subsection{运动与体系}
\begin{staff}
\qlogo 下面是一些我找不到答案的问题,听听你的意见。
\begin{enumerate}
  \item 最佳实践为何难以推广?
  \item  原来专题形式的资源共享杂志为何中止了?按专题组织文章,恰当时组织专题研讨会,应该算个不坏的想法吧。
  \item 在本部工作时,有个技术积累平台,文章分部门级/产品级/事业部级,发表文章有奖励,一度还列入部门和个
    人的考核指标,也是难以为继。这是为什么?
\end{enumerate}

从以前的工作看,专题和定期组织并重,有激励手段,有考核措施,还是解决不了问题。感觉需要一些新的思路,也想听
听你的看法。
\end{staff}

\begin{yang}
  \ylogo 其实最佳实践的推广、专题讨论以共享经验、资产库的利用、以至度量体系的建设等等,都是长远的活动,都
  需要一个体系支持。粗看起来,推广和研讨会都是运动。我们应该可以做得好。

  其实也是。我们的推广和研讨会都可以做得好。甚至建立资产库和度量体系。我们的确实能把他们建立起来。

  但是我们没有恒常的执行力。代码走查推了一下,专题研讨会,开了一下,都没有下文了。我们在组织这些运动活动
  的时候,就没有把他们当成为一个长远的功能。所以我们没有``维护''的安排。用另一个说法,就是没有制度化。
  体系可以建立,但是没有维护,所以不可以生存。

  EPG以为只是制定规程而已。其实EPG包括恒常不断地去推广过程优化。EPG需要不断地监控组织里的过程性能,不
  断地收集并审核改进建议,推广改进。但是大部分的EPG没有在这个方面做好。资产库,好像本部的那个,需要一个
  正式的维护岗位。需要有管理的制度,进行日常的管理任务,也需要培训接班人。度量也是。但是我们都没有意识
  到这个需要。所有这些,我们都是靠个人的努力。在这个方面,我们是一级而已,还没有到二级。

  究其原因,就是我们不觉得这个维护是重要的。所以不会安排一个岗位去做。更不能建立制度。想想看,高层可能
  认为,什么事都要有岗位,人数就失控了。很多时候,高层的指示就是在不能有体系变动的情况下完成任务。这是一
  个判断力和价值观的问题。

  为什么我们不能做恒常的事? 我们的体系有缺陷,如太多的保密,做成共享的困难。也可能是员工没有共享的意愿。
  共享人家的东西,自己是要努力的,要学习人家的东西。这个需要一些动力。我们没有提供这类动力的管理理念。
  我们的管理理念,是少设些岗位,尤其是资产库管理员,或是度量负责人,等等。你看,连系统工程师的设立都这么困
  难。

  其他同志给您的意见是有关问题的技术方面。其实更重要的是管理理念。有什么样的管理理念,就有什么样的员
  工,就有什么样的环境,就有什么样的效率。

  要改变这个情况,我们需要有一套不同的管理理念。这个是不可接受的。我是无能为力的。所以我的重点,只有放
  在提高员工的个人能力方面。

  谢谢您的关注。
\end{yang}

\pagebreak
\section{文章\footnote{\textsf{from : http://mk6yeung.blog.51cto.com/}}}
\subsection{智慧永远填补不了道德的空白}
\it

最近国家好像开始讨论高校学生、官员干部、以及社会人员的素质与品格问题。我们了解到品格对和谐社会的必要
性。这是一个好现象。我们的发展,好像到了一个台阶,如果要继续增进,我们就需要有新的体会,建立新的价值
观,改变我们部分的行为。否则很难再进步。 

我们可能还没有看见,品格的各样因素,是可以影响技术与质量的发展的。比如:我们有短视的倾向,不利于长远
的发展。我们的CMMI运动,大部分是拿了证书,过了级,但总没有效果。这个就是原因之一。又比如:在前一个帖
后面,有一位同志就写下“这个很难”的心情。你看,这位同志看了之后,用了多少时间考虑这个问题?半天?半
个小时?我看更像是立刻的反应。这个反应的必然后果,就是这位同志不会去尝试解决这个问题。那么我们就很难
有进步了。
 
我甚至觉得,我们的进步不大,不是因为我们没有能力,而是因为我们没有这个态度与决心。
 
最近家里我需要与两个装修公司打交道。我们的房子已经二十多年,屋顶的沥青片已经有几个地方断裂崩离,虽然
还没有漏雨,还是要修的。这个公司的人,用了三天,把事情做得很妥当,事后屋顶焕然一新,非常好看。花了八
千多美元。但是整个过程都没有意外,顺利,愉快。另一个是墙壁要油漆,地板要翻新。要六千多块钱。但是不断
有意想不到的事情发生。把要做得东西写好一个字条就不要,让后经常忘记要做的事情,又说我们没有告诉他。工
艺质量非常不好。大家非常不愉快。大家可以考虑一下,到底这两类的技工的心态有什么分别。如果大家有兴趣,
可以参考一下我的《专业态度、行为、与道德》这个讲义。

 
我一向都认为我们的道德、品格、个性,对个人的技术水平与职业生涯又很巨大的关系。其实我们的技术水平提高
的不够快,部分原因在于我们的价值观。短视、怕难,走捷径等等,都是进步的障碍。
 
在贝尔实验室的时候,有一位中国来的同事,非常能干,很有创意。有一次谈话,他告诉我如何如何用假证件拿到
一个职位。我告诉他这样不够诚信,他就非常不以为然。其实我们知道诚信是道德,但我们没有把缺乏诚信的行为,
与“不够诚信”这个不道德的词关联起来。
 
如果要成材,我鼓励大家要留意培养我们的品格。以下是两个有趣的故事。我不知道是否真实。但是一个故事谈的
是诚信,另一个谈积极参与,不埋怨领导、工作氛围,只努力争取识别根本原因与解决问题。这些故事里面的道德
与价值观,却是发达的西方社会比较认同的。我也相信,一个强大的社会,它大部分的公民一定会有高尚品格。这
里的两个故事,如果没有看过,就请好好欣赏;如果已经看过,还是可以重新体会其中的道理。请小心欣赏:
 
\subsubsection{智慧永远填补不了道德的空白}
 
为什么欧洲许多先进国家还是很富强?
 
若你到奥地利自助旅行,坐车、坐船都是没人验票、检票的!

道德是国力提升的基础!

在某个电视访谈节目中,嘉宾是一位当今颇具知名的青年企业家。 节目渐近尾声时,按惯例,主持人提出了最后一
个问题。

请问:你认为事业成功的最关键质量是什么?

沉思片刻之后,他并没有直接回答,而是平静地叙述了这样一段故事:

十二年前,有一个小伙子刚毕业就去了法国,开始了半工半读的留学生活。 渐渐地,他发现当地的的公共交通系统
的售票处是自助的,也就是你想到哪个地方,根据目的地自行买票,车站几乎都是开放式的,不设检票口,也没有
检票员。甚至连随机性的抽查都非常少。

他发现了这个管理上的漏洞,或者说以他的思维方式看来是漏洞。凭着自己的聪明劲,他精确地估算了这样一个概
率: 逃票而被查到的比例大约仅为万分之三。他为自己的这个发现而沾沾自喜,从此之后,他便经常逃票上
车。 他还找到了一个宽慰自己的理由:自己还是穷学生嘛,能省一点是一点。

四年过去了,名牌大学的金字招牌和优秀的学业成绩让他充满自信,他开始频频地进入巴黎一些跨国公司的大门,
踌躇满志地推销自己,因为他知道这些公司都在积极地开发亚太市场。但这些公司都是先热情有加,然而数日之后,
却又都是婉言相拒。

一次次的失败,使他愤怒。他认为一定是这些公司有种族歧视的倾向,排斥中国人。

最后一次,他冲进了某公司人力资源部经理的办公室,要求经理对于不予录用他给出一个合理的理由。

然而,结局却是他始料不及的。下面的一段对话很令人玩味。

``先生,我们并不是歧视你,相反,我们很重视你。 因为我们公司一直在开发中国市场,我们需要一些优秀的本土人才
来协助我们完成这个工作,所以你一来求职的时候,我们对你的教育背景和学术水平很感兴趣,老实说,从工作能力上,你
就是我们所要找的人。''

``那为什么不收天下英才为贵公司所用? '' 

``因为我们查了你的信用记录,发现你有三次乘公交车逃票被处罚的记录。'' 

``我不否认这个。但为了这点小事,你们就放弃了一个多次在学报上发表过论文的人才?'' 

``小事?我们并不认为这是小事。 我们注意到,第一次逃票是在你来我们国家后的第一个星期,检查人员相信了你的
解释,因为你说自己还不熟悉自助售票系统,只是给你补了票。但在这之后,你又两次逃票。''

``那时刚好我口袋中没有零钱。'' 

``不、不,先生。我不同意你这种解释,你在怀疑我的智商。我相信在被查获前,你可能有数百次逃票的经历。'' 

``那也罪不至死吧?干吗那么认真?以后改还不行吗?'' 

``不、不,先生。此事证明了两点: 

\begin{enumerate}
 \item 你不尊重规则。不仅如此,你擅于发现规则中的漏洞并恶意使用。 

 \item 你不值得信任。而我们公司的许多工作的进行是必须依靠信任进行的,因为如果你负责了某个地区的市场开
   发,公司将赋予你许多职权。为了节约成本,我们没有办法设置复杂的监督机构,正如我们的公共交通系统一样。
   所以我们没有办法雇用你,可以确切地说,在这个国家甚至整个欧盟,你可能找不到雇用你的公司。''
\end{enumerate}

直到此时,他才如梦方醒、懊悔难当。 然而,真正让他产生一语惊心之感的,却还是对方最后提到一句话: 

道德常常能弥补智慧的缺陷,然而,智慧却永远填补不了道德的空白。

\begin{quote}
行正直路的, 步步安稳;
走弯曲道的, 必致败露。
\end{quote}
 
\subsubsection{你无法把香蕉皮骂进垃圾桶}
 
大学阶梯教室里,一场演讲会即将开始。主讲人是蜚声海内外的知名教授,海报两天前就贴出去了, 反应异常热烈,
同学们纷纷赶到现场,要一睹教授的风采。
 
离开讲还有十分钟,学生们纷纷进入到会场中, 在他们跨进会场的一瞬,不约而同地发现脚下有一块香蕉皮, 在
抬腿避开时,都不忘埋怨两句:是谁这么缺德?一点公共意识都没有!组织者是怎么搞的?现在的人,什么素质?
 
大家叽哩咕噜抱怨着跨过那块香蕉皮,坐到自己的位置上,静等着教授的光临。几分钟后,教授准时到达。
 
他也发现地上的香蕉皮,扶扶眼镜上前仔细端详。教室里顿时静了下来,大家都伸长脖子,看教授的一举一动。
 
教授看清楚脚下是一块香蕉皮,勃然大怒, 指着它大声说道:

``你怎么可以呆在这个地方呢?你应该是在垃圾桶里睡觉! 怎么这么没有公德心、没有环保意识,要是有人踩到你
摔伤怎么办? 太不象话了!''
 
愤怒让他的眼镜在鼻梁上跳动着,让人一下子想起被小事激怒的唐老鸭, 听众席上顿时传来一阵阵笑声。 教授没
理会,继续愤怒,对着香蕉皮继续发火。
 
听众席上,有学生不耐烦了, 大声说:``算了吧!教授,别费力气了,你不可能把香蕉皮骂进垃圾桶的!''教授听
了,突然,转过头来,满脸红光地笑了, 并伸手把香蕉皮捡起来,放进讲台旁的垃圾桶里, 用纸巾擦擦手说:
``刚才那位同学说什么?能再说说吗?''
 
教室顿时静了下来,没人说话。
 
教授说:``我听见了,你不能把香蕉皮骂进垃圾桶!这就是我今天晚上演讲的题目!''这时,墙上的大屏幕上开始
播放同学们刚才入场时的镜头, 同学们千姿百态地跨越香蕉皮 和 版本各异的埋怨声清晰地传了出来。 大家最初
哄笑着,慢慢变得雅雀无声。
 
教授说:``这是我特意安排的一个环节, 我想给大家讲的道理,其实你们已明白并喊了出来。但对你们来说,明白
道理是一回事,而用道理指导自己的行为,却又是另外一回事!''我相信,在坐的几百名同学,没有一个人不懂得
香蕉皮是骂不进垃圾桶的, 但大家缺乏动一动手,以举手之劳去改变现状的行为。
 
这就如同许多人感觉社会冷漠,而又吝于付出一个笑脸; 埋怨环境污染,却又不愿意捡一片垃圾; 咒骂腐败和贪
污,遇事却本能地想去 托关系 走后门; 感叹道德水平下降,却又不愿意身体力行地去做任何一件善事\ldots  几乎
所有的人都在埋怨和咒骂。 几乎所有人都不愿意身体力行去做事。
 
责任永远在别人身上,而自己永远都是受害者! 这些做法与心态,无限放大了消极面,而使人看见的都是绝望。事
实上,并非如我们所想的那样, 社会的每一分进步,都是需要人们用行动去构建, 如果我不乱扔垃圾,这个世界
就少了一个污染源; 如果我再将身边的垃圾清理掉,世界就干净了一分; 如果我的行为感化并带动了一个人,那
么世界上又多了一份干净的原因。
 
地球上只有五十多亿人,这不是一个望不到边的数字, 因而,我们应该为自己的五十亿分之一,抱有信心。记住,
垃圾不会被骂进垃圾桶,你得行动!从现在开始!教授的演讲结束了,会场里响起声音宏大但情绪极其复杂的掌
声。

\subsection{向过程管理人员提三个建议}

在这个春节期间,恭祝大家虎年兴旺、精神爽利、开心快乐、步步高升、如意吉祥。

回想过去的过程管理工作,一般来讲形式上增长了很多认识。可是对提高过程效率方面就成就不很明显。当然原因
很多。上一次提到的``领导是关键''确实是一个我们不能控制的因素。我总觉得无论在什么情况、氛围底下,自己
总可以提供一丁点的改变,对过程管理、项目效能、以及自己的理论认识与技能方面提供有用的价值。

首先让我们从过程管理人员的工作模式开始。我们从领导接受了任务去落实一种体系。比如,企业要落实CMMI、
TL9000等认证,实施 6 Sigma,或是要让项目经理拿到PMP证书,推广IPD、JIT 等等的方法。

当然领导要求在太短的时间完成。

这个很奇怪。因为我们的任务往往是能够按时完成的。

就是说,我们能够没有太多的延误,拿到认证或证书,建立了6 Sigma黑带,甚至做了几个黑带项目。通常的过程就
是接受有关新体系的培训,因为我们需要从新体系找到可以操作的灵感。按自己对新体系的理解制定一些规程。这
个过程一般非常短促,根本没有机会体会新体系的目标与精神所在。就是因为这样,我们经常觉得这些不同的体系
都是不同的事情,因为我们看见它们的操作层面要求,而看不到它们的目标与精神其实是一致的。因为需要确保员
工按我们制定的规程操作,(要大家``听话''),也制定了一些监控机制,我们能想到的,就只有借助一些惩奖标准。
然后就是推广,包括向员工提供培训,让大家知道如何操作,观察大家的操作是否符合规程定义。

所以在领导的心中,和在我们的心中,大家都认为我们已经完成任务。总是等到不远的将来,我们发现我们常见的
问题还是经常出现,然后困难的事情还是一样困难,来来去去还是老样子,我们才突然发现,原来这些体系、法宝、
都是``不符合''我们的,或是都是``骗人''的。

其实在以上的场景,我们非常清楚,大家的执行力是非常强大的。就是指导方向有点偏了。就是因为我们的执行力
强大,就是方向偏了点,只要不与正确方向成正交,还会有部分力量往正确方向投送,只是效果就比应有的弱少好
多。

这是专业态度、敬业精神所不容许的,因为过程管理人员的使命,就是改进过程的效能。

归根结底,以上简单的描述,就牵涉到好几个主要的管理理念与它们的必然后果。如果大家有兴趣,可以自己尝试
识别它们。很多时候,我们觉得在我们这个氛围底下,很难有所作为。其实这个也是不完全对的观念。无论在任何
不良、不利的情况,我们总应能够争取到一定的提高的。我希望在这里向过程管理人员提供三个态度与操作方面的
建议。这些建议,都是自己可以从自己的工作中得到提升的。希望大家参考。

\subsubsection{第一:竭力追求提高项目的效率}

这是一个态度问题。无论氛围如何,每一个人都可以决定自己的态度。这个态度,如果能够得到环境的许可或是支
持与鼓励,当然可以发展得很好。但是如果情况不利,比如上面说的时间压力,关注效率的机会就比较难掌控,但
是也不是没有机会。一般的情况,当我们面临压力的时候,我们自然地找通往完成任务的``直线''上走。就是只考
虑一个因素的单维思维。因为这个``直线'',往往让我们容易犯错误,让下游需要额外的时间精力整改,要么就造
成将来的问题,让将来的工作更复杂与困难。记得听过一句话:``两点之间最短的途径,往往不是那条直线''。

有部分同志可能会说这是理论,希望知道如何操作。这是守株待兔的态度。

这就是缺乏积极主动的态度。我当然可以列举很多的案例。我期待大家做到的,就是在工作之中心怀``效率'',观
察事情的发展。如果有这个心,这样的观察花不了很多时间,却是增长自己指示的必要手段。如果观察了,对比了,
琢磨了,很快自己就可以掌握很多很多的例子了。这是没有人可以阻止的。自己发现的案例,远远比人家告诉你的
更深入。``知识不能传递,只能自己体会''。

我们为什么要这样做?因为过程的效率是过程的灵魂。

所有的过程目标,无论是质量的或是响应能力,都在于提高效率。效率越高,越能够在越短的时间,把任务完成得
更好,更高质量。

作为过程管理者,需要非常明白制定的规程为什么可以提高效率。要做到这点,我们需要这里提到的``追求效率''
的态度,因为只有这样,自己才能具备规程如何提高效率的知识,并且愿意为项目的效率而制定规程。这样的过程
管理人员,不单能解释规程的定义精神,还应该能够显示与示范如何使用规程以提高项目效率。

\subsubsection{第二:从项目里来,到项目里去}

前面提到推广新体系需要制定规程。我们往往很快就把新规程制定好。很多时候,制定规程的人都在自己的办公桌
上制定的。大部分同志不清楚规程的后果,甚至不能说明白新体系要解决的问题与关键因素。举一个例,如果我们
制定IPD新规程的人员,都没有能力提供有关IPD的演讲,你觉得制定出来的规程能够收到预期的效果么?

这就是一个任务的``直线''实施方法。将来要么就没有效果,要么还要从头做起。而且将来还要努力克服这样造成
的习惯。因为这个案例范围非常广泛,如果我们具备追求效率的强烈欲望,还是可以看得非常清楚的。可惜这是不
断地在重复的做法。

那么规程应该如何制定呢?就是从``项目来,到项目去''。

从项目来,就是要观察项目,从项目中了解项目操作的效果,哪里做的好,哪里做的不好。这是对项目的观察与了
解。要知道项目最关注的问题是什么。如果这些项目关注的事情与企业目标一致,就需要在规程方面助长它们的做
法。如果项目关注的事情与企业不一致,做需要在规程上面帮助项目建立与企业一致的目标与关注点。

``从项目来''还要在项目里找到最佳实践。``最佳''在这里,就是``做的有效''的含义。不需要计较这些实践是否
``最''佳,凡是明显有效的就可以考虑。同时也留意项目那些是做得不很有效的,(无论符合与否)。对于低效的操
作,不能让它们出现在规程定义里面,还可以作为反面教材添加到培训教材里。目的是让项目加强有效的操作,避
免低效的操作。

这样定义的规程,项目是可以接受的,因为它们已经在项目实施过,所以是适合项目的。这样的规程就非常有说服
力。项目也容易认可与接受。

规程在项目的实施情况,是需要监控的,就是说,需要收集数据,观察,分析,然后总结出来提高项目的关键因素。
这是持续优化规程的程序。所以过程管理不单是制定规程,也需要负责规程对项目产生的作用。我们需要检查项目
操作的符合程度,否则我们不能分析规程的有效性。项目需要符合规程,是因为规程可以帮助项目高效地满足企业
期待的目标。这样才是规程需要``到项目去'',因为规程的效率需要``到项目去''。就是说,过程管理的目的,是
提高项目的效率。所以说:

规程应该从项目来,所以适合项目;到项目去,让项目提高效率。

要实际做到这一点,需要比较长的时间。在我们的进度至上的氛围下的确是很难实现的。我们还是一定需要满足领
导的要求,在紧张的进度下完成交付给我们的任务。在完成任务的过程中,同时可以做到的,就是一些观察与分析。
因为时间资源的限制,我们可能不能做的完美,或是做到彻底,做到有成果。但是在我们自己的观察中,我们也可
以得到很多自己的知识。这是可以做到的。很多同志也有这样做。当然,动力不会很大。请留意,自己争取到的知
识,将来一定会非常有用的。

总之,需要让项目感觉到,过程管理的工作就是帮助项目提高效率。

\subsubsection{第三:建立一个持续改进的机制}

企业现在不是没有改进规程的机制。当前的模式大概就是高层接触到一些高层次的、主流的管理理念(如TL9000、
CMMI、6Sigma 、IPD等),就开展一个指令式的改进运动。这些运动一般都按一个生命周期模型发展,从立项、策划、
推广、以至维护。如果在这个过程中,发现有效用,将会推行的长久一点,如果效果不明显,就不会再被关注,甚
至被取消。这个周期也可能有好几年的光景。

可以想象,这样的运动模式,也有它的价值与需要。但也有非常明显的不足。第一,在发现需要改进的机会,过程
的问题已经存在很久了,依靠这个长周期运动方法的延误比较长。第二,改进的要求相对来说是比较外加的而不是
内在的需要,导致企业吸收过程管理理念比较困难,过程改进的效果也有很大的风险。这也是运动形式造成形式化
的道理。这样的回报很难是理想的。

出路在于持续改进这个概念,不能单靠运动形式。

持续改进就是不断地改进。我们要生活在改进中。我们要做到:改进就是生活。这需要大家有一个``竭力的追求效
率''态度,这也是一个质量的态度。我希望大家真正的了解这个态度的重要性,因为它是``改进''的唯一可靠的动
力。

持续改进需要经常监控过程效能的表现来提高过程的效能。控制图在持续改进中非常重要。首先需要稳定过程,找
到任何在过程操作中可以识别的困难与问题,然后防止这些问题重复发生,过程效能就可以被稳定下来。通过不断
地逐步优化提高过程效能。这种情况,6 Sigma 是非常好的方法。所以大家知道,CMMI与6 Sigma其实是一致的,不
冲突的。过程管理人员基本上需要了解6 Sigma 这个方法。

就是如果当前的规程已经是最佳的,我们仍然必须持续改进的另一个原因就是市场与技术都在不断改变,所以过程
也需要不断调整以用最适合市场情况的高效率来保持企业在当时市场情况下的竞争力。

我建议每一个过程管理的团队,都要有一个持续改进的机制。最合适的机制,就是类似项目组的CCB的过程修改监管
委员会(Process Change Control Board)。组织结构可以参考项目的CCB。关键在于不能只有一个人,一定需要包括
几个有代表性的成员。需要过程管理人员牵头,可以容纳领导代表、企业其它领域(财务、产品策划、市场)、项目
成员等。

我们可能觉得``委员会''这个词洋气大了点。这个可能真是没有在中国传统普遍使用。可是这个对过程得到认同与
接受有很大的帮助。在讨论共同目标、团队精神等议题的时候,这些广纳众议的手法都是关键。

过程持续改进,需要有一个反馈的机制,可以能够吸收经验,优化标准规程定义。就是说,过程管理人员与他们的
领导需要了解标准规程是可以优化的。这是一个文化的问题。

过程管理组自己需要分析项目的过程数据,从中提出改进建议,提交过程修改委员会。这应该是持续改进的主要输
入。但是我们不应该单单依靠过程管理组的分析结果。我们需要能够接受任何人员的过程改进建议。第二个重要的
改进意见的来源,应该是QA们。他们最接近实际操作层面,最能了解过程的成功因素与问题所在。当然,他们需要
具备过程有效性的知识与判断能力,才能提供这方面的价值。所以培养QA人员成为有效性QA是非常重要的。总的来
说,我们需要鼓励越多人关注与参与过程改进越好。

有一个我们经常忽略的,对持续改进非常重要的概念就是过程的监控与定义应该有一定的独立性。

表面上看来,这个概念不是十分明显。原因在于定义者监控的时候比较不善于发现规程可以优化的地方。但是如果
规程定义能够成功地被第三者理解,并且通过第三者观察实施的情况,就更能证明规程定义是可以理解的,可以实
施的,并且真正的效果大概会是怎么样的。这样就更容易识别规程优化的可能性。这就是一个很好的过程优化的动
力。

最后,让我总结一下:

过程管理是长远的职能,是企业分配资源的结果。过程管理,也需要能够分配资源,并且作长远打算,把自己的资
源在各个任务之间分配好。

我们通常都是按任务的紧急程度分配资源。最紧急的任务,通常得到全部的人员、时间与资源。这样非常短视。资
源分配就是知道一些长远任务的重要性,正确地制定长远任务的优先级(重要性),然后按优先级分配资源。资源一
经分配,就不应该改动。比如,组织决定培训的重要程度,觉得培训应该占员工的10\%的时间。那么,这个分配就
不能改变。在策划的时间段里,一定要能够让员工的10\%时间,花费在接受培训方面。这就是说,领导需要在没有
紧急事情的情况下,尽早完成这个培训需要。千万不要把这10\%分配好做培训的资源到用在其它任务。这个也是一
个新的管理理念吧?

同时,过程管理人员也需要 能够勇于学习与改变行为。他们需要具备寻根问底的精神,与识别、处理关键因素的能力。

这样,过程管理人员才有可能为企业带来过程效能,提高企业竞争力。希望大家共勉。

\pagebreak
\section{支言片语\footnote{\textsf{源至杨万强培训课的课堂笔记}}}
\subsection{2005}

``你们真的很聪明,一听就知道会怎样影响自己的工作量,这不是一个好的态度''

``老板可以占用你的工作时间,但是他不可以占用你的大脑''

``你有权力要求人家做好''

``要树立一种质量意识,对于任何小的质量问题都不能容忍''

``估算根本上需要经验,我可以教给你方法,不可以教给你经验''

``Well,我们总是说别人做得不好,大家都这么想,所以我们是一团散沙,为什么我们这么自私呢?''

``很多问题我们没有看到,那是因为我们不愿去看''

(大意)每个人都可以把自己的事情做好,``要做得像个人样'',即使领导不采纳你的意见,你也可以做足够多的考
虑,之后验证自己的想法,这样个人才可以提高。

(大意)对故障单多的员工进行经济处罚,那时把这个员工当小孩子的行为,(这样这个小孩子就可以不负责任了)。国
外的做法是领导谈话,好处是……。

``项目经理必须对项目的每一件事情负责,我说的是负责,不是去做!''

``不了了之是一个通病,通常我们解决的重大问题之后没有考虑规程上的原因。解决的方式应该是投入资源,有专人
研究,通过制定规程避免这样的问题。否则过程改进会成为运动时的活动''

``QA需要保持客观性和独立性,所以QA不应该与项目组的业务方面有联系。项目组的度量数据QA可以收集,但是仅限
于自己使用''

``我不相信没有时间,只是他认为这件事情的优先级不够高而已''

``我们要相信这个过程,这是我们项目开发的基础。同样我们要相信每一个步骤都有它的价值。''

``配置管理项准备好之后就需要尽快入库,入库表示同时项目这个配置管理项可以被使用了,不要等待一切都完美了
才入库,那样会推迟项目开发时间''

``领导永远都有权这样做(不经过流程向外发版本),但是你不要你承担不了的责任''

``尊重规程,即使你认为这个规程不好,你也需要认真地按照这个规程去做(首先不要去讨论这个规程行不行,应该先
去做)。就像上游提供一个很烂的需求,你也需要把这个需求实现,这样上游可以现实的看到需求的问题。规程只有所
有人尊重它才能起作用''

``美国的``术''很厉害,他们完全不考虑``道''''

``又说项目经理不支持,我想问的是你做了多少,我想知道你自己对度量和分析了解多少,首先你要知道这些。不需要
你现在就做,需要你至少现在知道''

``我希望大家不要把注意力集中在估算的结果,估算永远是不准确的,首先我们要做第一步,先进行估算,再不断的验
证、改正和优化,这个态度才是重要的''

``我们估算不准,为什么我们还要做估算?因为我们需要估算的结果作计划,没有计划我们就不知道改怎么做。刚开始
可能不准,慢慢的估算会越来越接近实际值''

``Project计划不用做得非常精确,越精确就越不准确''

\subsection{2007}

\subsubsection{估算}

``方法不是出路,越复杂的方法对经验的要求越高,要先积累经验,要学习。''

``估算不会准确,但是可以作为指导。''

``项目的估算应该分解到WBS的责任人估算。''

\subsubsection{文档}

``我们习惯的文档水平,可以接受的文档水平是不够的,否则 \ldots ''

文档模板的作用:
\begin{enumerate}
 \item 提醒不要遗漏
 \item 他人接受方便
\end{enumerate}

\subsubsection{进度}

``进度永远都没有那么重要,不要把所有的压力都传递到最底层。''

\subsubsection{项目管理}

``通常开发人员的有效工作时间不大于70\%。''

``不要用个例的现象作为管理的基础。''

``要转变个体的管理为基于统计数据的管理。''

\subsubsection{其他}

``有什么样的态度,就有什么样的结果。不是:有什么样的结果,就有什么样的态度。因为:下层会专门做一个结果
给上层看。''

``目前上层之所以管理理念差劲(而不改变),是因为他们认为现在是最有效的。''

回答``时间不够'' --- 这时间最公平的就是时间,重要的是如何有效的工作。

``无论怎样忙,一人任务延误5\%是可以在绝大多数情况下可以接受的。如何有效利用这个5\%去提高时间是非常重要
的。不要认为你是普通员工,你们是中国的TOP 1\%!''

``There's no fear except fear itself.''

需求的两个方面:
\begin{enumerate}
 \item 需求要达成一致
 \item 要尊重需求
\end{enumerate}

``你们应该扪心自问,到底自己要改变什么程度?''

\pagebreak
\section{条目精选}
\begin{table}[htbp]
\caption{}
\centering
\begin{tabular}{lc}
\toprule
条目 & 页码 \\
\midrule
需求与设计的继承关系 & \pageref{link1} \\
不清楚的需求 & \pageref{link6} \\
原意承担的风险就不需要应对措施 & \pageref{link2} \\
如何对待不尊重他人的领导? & \pageref{link3} \\
如何对待不尊重他人的同事? & \pageref{link4} \\
CMMI与Agile,XP的比较 & \pageref{link5} \\
\bottomrule
\end{tabular}
\end{table}

\end{document}
