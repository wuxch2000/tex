\documentclass[11pt,a4paper,onecolumn]{article}
\usepackage{fontspec,xunicode,xltxtra}
% \setmainfont[Mapping=tex-text]{Times New Roman}
\setmainfont[Mapping=tex-text]{Arial}
\setsansfont[Mapping=tex-text]{Arial}
% \setmonofont[Mapping=tex-text]{Courier New}
\setmonofont[Mapping=tex-text]{Times New Roman}

\usepackage{xeCJK}
% \setCJKmainfont[ItalicFont={Adobe Kaiti Std}]{Adobe Song Std}
% \setCJKmainfont[ItalicFont={Adobe Kaiti Std}]{Adobe Kaiti Std}
\setCJKmainfont[ItalicFont={Adobe Kaiti Std}]{Adobe Heiti Std}
\setCJKsansfont{Adobe Heiti Std}
% \setCJKsansfont{Microsoft YaHei}
\setCJKmonofont{Adobe Heiti Std}
\punctstyle{banjiao}

\usepackage{calc}
\usepackage[]{geometry}
% \geometry{paperwidth=221mm,paperheight=148.5mm}
% \geometry{paperwidth=9.309in,paperheight=6.982in}
\geometry{paperwidth=7.2cm,paperheight=10.8cm}
% \geometry{twocolumn}
\geometry{left=5mm,right=5mm}
\geometry{top=5mm,bottom=5mm,foot=5mm}
% \geometry{columnsep=10mm}
\setlength{\emergencystretch}{3em}


\usepackage{indentfirst}

%生成PDF的链接
\usepackage{hyperref}
\hypersetup{
    % bookmarks=true,         % show bookmarks bar?
    bookmarksopen=true,
    pdfpagemode=UseNone,    % options: UseNode, UseThumbs, UseOutlines, FullScreen
    pdfstartview=FitB,
    pdfborder=1,
    pdfhighlight=/P,
    pdfauthor={wuxch},
    unicode=true,           % non-Latin characters in Acrobat’s bookmarks
    colorlinks,             % false: boxed links; true: colored links
    linkcolor=blue,         % color of internal links
    citecolor=blue,        % color of links to bibliography
    filecolor=magenta,      % color of file links
    urlcolor=cyan           % color of external links
}
\makeindex

\usepackage[dvips,dvipsnames,svgnames]{xcolor}
\definecolor{light-gray}{gray}{0.95}

\usepackage{graphicx}
\usepackage{wrapfig}
\usepackage{picinpar}

\renewcommand\contentsname{目录}
\renewcommand\listfigurename{插图}
\renewcommand\listtablename{表格}
\renewcommand\indexname{索引}
\renewcommand\figurename{图}
\renewcommand\tablename{表}

\usepackage{caption}
\renewcommand{\captionfont}{\scriptsize \sffamily}
\setlength{\abovecaptionskip}{0pt}
\setlength{\belowcaptionskip}{0pt}

\graphicspath{{fig/}}

\usepackage{fancyhdr}

% \usepackage{lastpage}
% \cfoot{\thepage\ of \pageref{LastPage}}

% 嵌入的代码显示
% \usepackage{listings}
% \lstset{language=C++, breaklines, extendedchars=false}
% \lstset{basicstyle=\ttfamily,
%         frame=single,
%         keywordstyle=\color{blue},
%         commentstyle=\color{SeaGreen},
%         stringstyle=\ttfamily,
%         showstringspaces=false,
%         tabsize=4,
%         backgroundcolor=\color{light-gray}}

\usepackage[sf]{titlesec}
\titleformat{\section}{\normalsize\sffamily\bf\color{blue}}{\textsection~\thesection}{.1em}{}
\titleformat{\subsection}{\normalsize\sffamily}{\thesubsection}{.1em}{}
\titlespacing*{\section}{0pt}{1ex}{1ex}
\titlespacing*{\subsection}{0pt}{0.2ex}{0.2ex}

\usepackage{fancyhdr}
\usepackage{lastpage}
\fancyhf{}
\lhead{}
\rhead{}
\chead{\scriptsize{\textsf{蜗居}}}
\cfoot{\scriptsize{\textsf{第 \thepage ~页,共 \pageref*{LastPage} 页}}}


% \usepackage{enumitem}
% \setitemize{label=$\bullet$,leftmargin=3em,noitemsep,topsep=0pt,parsep=0pt}
% \setenumerate{leftmargin=3em,noitemsep,topsep=0pt,parsep=0pt}

% \setlength{\parskip}{1.5ex plus 0.5ex minus 0.2ex}
\setlength{\parskip}{2.0ex plus 0.5ex minus 0.2ex}

% \setlength{\parindent}{5ex}
\setlength{\parindent}{0ex}

% \usepackage{setspace}
\linespread{1.25}

% 英文的破折号--不明显,使用自己画的线。
\newcommand{\myrule}{\hspace{0.5em}\rule[3pt]{1.6em}{0.3mm}\hspace{0.5em}}

\begin{document}
\setcounter{section}{544}

\pagestyle{fancy}

\section[\thesection]{}

朱祁镇最终做成了他的先辈们没有做的事情,这并不是偶然的,他没有他的先辈们有名,也没有他们那么伟大的成
就,但朱祁镇有一种他的先辈们所不具备(或不愿意具备)的能力 --- 理解别人的痛苦。

自古以来,皇帝们一直很少去理解那些所谓草民的生存环境,只要这些人不起来造反,别的问题似乎都是可以忽略
的,更不要说什么悲欢离合,阴晴圆缺。

但朱祁镇做到了,至少在废除殉葬这件事情上,他理解了后宫那些无辜者的痛苦。八年前,他从一个作威作福的皇
帝变成了俘虏,之后又成为囚犯,从衣来伸手,饭来张口到衣食不继,相拥取暖,这一惨痛的经历让他深刻地了解
了身处困境,寄人篱下的悲哀,也知道了身为弱者要生存下去有多么的艰难。

所以在生命的最后一刻,他决定违背祖制,去解救那些无辜的人。

应该承认,这是一个勇敢而伟大的行为。

在这个世界上,任何人都没有无故去夺取别人的生命和尊严的权利。

虽然他一生中干过很多蠢事、错事,但在我看来,他比那些雄才伟略的帝王们更像一个``人''。

我们可以用一句话来评价朱祁镇的一生:

他是一个好人,却不是个好皇帝。

天顺八年(1464)正月,明英宗朱祁镇结束了他传奇的一生,终年三十八岁,太子朱见深继位,一个让人哭笑不得的
朝代就此拉开序幕。

明宪宗朱见深

曾经有一个朋友让我帮他解决一个难题:他和他的女友关系很好,但是由于他的女友比他大两岁,家里人反对,他
拿不定主意,想问问我的意见。

我想了一下,给他讲了一个故事,朱见深的故事。

悲惨的童年

一般说来,皇帝的童年或许不会快乐,却绝不会悲惨,明代皇帝也是如此,当然了,首任创业者朱重八同志例外。

但朱见深先生的童年似乎可以用这个词来形容,客观地讲,这位仁兄确实受尽了累,吃够了苦,虽然他后来终于成
功继位,当上了皇帝,但如果你研究过他的发展史,相信你也会由衷地说一句:

兄弟你实在不容易啊!

正统十二年(1447),朱见深出生了,他是皇位未来的继承者,用今天的话说,他是含着金钥匙出生的,可是没有人
会想到,仅仅两年之后,他的人生悲剧就将开始。

\section[\thesection]{}

正统十四年(1449),父亲朱祁镇带兵出征,却成了肉包子打狗 --- 一去不回。在大明王朝的最关键时刻,朱见深毛
遂他荐,被挺而出,在牙还没长全的情况下被光荣任命为皇太子,时年两岁。

两岁的朱见深自然不会知道,他之所以在这个时候被立为皇太子,有着极为复杂的政治背景。

当时,朱祁镇战败被俘,朱祁钰即将顶替他哥哥的位置,老谋深算的孙太后早已料到这个弟弟是不会就此罢手的,
为防止皇位旁落,她急忙拥立朱见深为太子,并以此作为支持朱祁钰登基的交换条件。

虽然孙太后成功地将朱见深立为太子,但她深知深宫之中,人心险恶,保不准朱祁钰先生什么时候来一个斩草除根
之类的把戏,而她自己也不可能时刻与宝贝孙子在一起,为确保安全,她做出了一个决定:派出自己的一个亲信去
保护朱见深。

她做梦也不会想到,正是这个不经意的决定,改变了朱见深的一生。

她派出的亲信是一个姓万的宫女,从此这位宫女开始无微不至地照料幼童朱见深。

那一年,她十九岁,他两岁。

事实证明,孙太后的政治感觉是很准确的,朱祁钰坐稳皇位之后,丝毫没有归还的意思,不但自己追求连任,还想
让自己的儿子也能连任。于是在景泰三年(1452),他买通了大臣,废除了朱见深的太子地位,改立自己的儿子朱见
济为太子。对于这一变动,孙太后虽然极不服气,却也无可奈何。

这些政治人物为了自己的利益争来斗去,却没有人意识到,他们的举动,已经为一场悲剧拉开了序幕。

此时已经五岁的朱见深自然不知道大人们的事情,他每日只是在深宫中闲逛,由于他身处险境,且地位不稳,大家
都认为他被废掉是迟早的事情,所以没有多少人愿意接近这位所谓的皇太子,对他十分冷淡。

从两岁时起,孤独和寂寞就不断缠绕着这个幼童,对他而言,童年是灰暗色的。而在这灰暗的生活中,唯一可以给
他带来安慰的就是那位万姑姑。

无论周围的人对他如何冷淡,也无论人们如何排斥他,不陪他玩耍,这位万姑姑却总是一直陪伴着他,安慰着他,
照料他的生活,虽然他的母亲周贵妃也常常来探望他,但宫中到处都是朱祁钰的耳目,为了不惹麻烦,每次总是来
去匆匆,在他那幼小的心灵中,这个日夜守候在他身边的人才是他可信赖的依靠。

\section[\thesection]{}

就这样,朱见深和他的万姑姑相依为命,过着这种冷清而又平静的生活,可有一天,这种生活被打破了,一群人突
然闯进了朱见深的宫殿,气势汹汹地对他说,你不可以再用太子的称谓,从此以后,你的称呼是沂王。

然后这些人告诉他,沂王是没有权利继续住在这里的,你要马上滚出宫去,因为你的堂兄朱见济将很快搬进来,成
为这里的主人,新的太子。

接下来要处理的就是原任太子,现任沂王身边太监宫女的下岗分流遣散问题,而从使用价值方面来说,废太子还不
如废旧轮胎。这是因为废旧轮胎还能回收利用,而根据历史经验,废太子往往会一废到底,永久报废。

人们很早就知道这个道理,所以这种时刻经常出现的景象就是树倒猢狲散,身边的人纷纷收拾行李,离开朱见深,
另寻光明的前途。

面对着这一突变,那位姓万的宫女的表现却异于常人,她没有说话,只是默默地看着那些离去的人,默默地为朱见
深准备着出宫的行装。

五岁的朱见深并不清楚到底发生了什么事,他只知道他很快就要搬出这里,而那些熟悉的面孔也即将离他而去,在
他的脑海中没有答案,只有疑惑和忧虑。

``你也会走吗?''

``不会的,我会一直在你身边陪伴着你。''

这句话,她最终做到了。

景泰三年(1452),朱见深被废为沂王,搬出宫外。

这一年,她二十二岁,他五岁。

朱见深的沂王生活开始了,事实证明,这是他人生中最为黑暗的一个时期,虽然他的父亲已经从蒙古载誉归来,但
又立刻被委以囚犯的重任,关进南宫努力工作,由于事务繁忙,无法与他见面,而由于他已经搬出了宫外,他的母
亲周贵妃也无法出宫来看他。此外,他身边布满了朱祁钰的手下,无时无刻都在监视着他的举动,如果被人抓住把
柄,没准就要从废太子更进一步,变成童年早逝的废太子。

五岁的朱见深,没有父母的照料和宠爱,没有老师的耐心教导,身处不测之地,过着今日不知明日事的生活,他随
时都可能被拉出去砍掉脑袋,或者在某一次用餐之后突然食物中毒,暴病不治而亡。对他而言,每一天都可能是生
命的终点,每一天都是痛苦的挣扎,而这样的生活持续了整整五年。

\section[\thesection]{}

在这让人绝望的环境中,只有她始终守在他的身边,照顾他,安慰他,无论遇到什么困难,也从未动摇过。

对朱见深而言,这个人已经成为了他的母亲,他的朋友,他的依靠,是他不可分离的一部分。在那黑暗的日子里,
这个人支撑着他,和他一起熬过了最困难的时刻。

五年后(1457),朱见深的父亲又一次得到了皇位,他的苦日子终于熬到了头,风水轮流转,他又一次搬回了宫中,
恢复了太子的身份。自然,她仍陪伴在他的身边。

这一年,她二十七岁,他十岁。

在担任东宫太子的日子里,日渐成熟的朱见深逐渐对这位大他十七岁的女人产生了微妙的感情,相信就在这段时间
之内,他们的关系发生了特殊的变化。

对于这些情况,他的父亲朱祁镇和母亲周贵妃都有所察觉,但他们并没有阻止,而是为朱见深挑选了三个女子作为
皇后的候选人,等待他登基之后挑选册封。因为他们相信,这个姓万的宫女绝不可能成为皇后,等到朱见深长大懂
事后,自然会离开她的。

天顺八年(1464),朱祁镇病死,朱见深继位,从此这位万宫女正式成为了皇帝的妃子。

这一年 ,她三十五岁,他十八岁。

皇后又如何!

虽然明代的宫廷政治十分复杂,王公贵族、文臣武将个个粉墨登场,卷起袖子你来我往,斗得不亦乐乎,不过在我
看来,要论斗争水平,后宫的诸位佳丽们也层次甚高,顾盼一笑,举手投足之间,足以致人死命,可谓巾帼不让须
眉。

对于这个问题,其实很早以前,亲爱的花木兰同志就曾经教导过我们:

谁说女子不如男!

太子朱见深成了皇帝,万宫女也变成了万妃,大致可以算是功德圆满,此时的万妃历经风波,已经年近不惑之年,
但让众人惊异的是,这个女人竟然得到了皇帝朱见深的大部分宠爱,很多人都不理解。

而这一情况的出现,对后宫那些正值妙龄的女子们来说就不仅仅是一个理解的问题了,她们十分愤怒,也很不服
气:这样的一个女人凭什么得到专宠?

在那些不服气的女人中,级别最高的是皇后吴氏

\section[\thesection]{}

要说这位吴小姐,那可是大大的有来头,有背景,想当初竞选皇后的时候,评委(朱祁镇)最先定的是一位王小姐,
可是这位吴小姐凭着自己家出身官宦,而且交际甚广,竟然找人搞定了评委,搞了暗箱操作,把王小姐挤了下去。
最终当上了皇后。

要知道,皇后的人选是朱祁镇亲自定的,那这位吴小姐到底有什么神通,能够改变朱祁镇的决定呢?

这是因为她认识一个十分厉害的人 --- 牛玉。

关于这个人我们不用介绍太多,只用说两点就够了:1、他是朱祁镇的亲信太监。

2、 朱祁镇临死前召见了两个人,一个是朱见深,另一个就是他。

有这样的一个人关照,吴小姐当上皇后自然不在话下,实在不用搞什么潜规则。

有这样的后台和关系网,年轻貌美的吴小姐自然不把三十五岁的万阿姨放在眼里,她绝对无法忍受自己被朱见深冷
落,于是她想了一个办法去整治万阿姨。

可是不幸的是,事实证明,这不是一个好的方法。

可能毕竟是太年轻了,吴小姐丝毫不考虑后果,竟然直接找到万阿姨,把她拉回来打了一顿板子。

这个方法可以用四个字来形容:简单粗暴。当然了,她打这顿板子还是有理论基础的,她到底是皇后,所以对此美
其名曰:整顿后宫纪律。这一顿板子打得万阿姨差点丢了命,也帮很多后宫的妃子出了一口气,此时的吴小姐可谓
是威风凛凛,风头甚猛。

据说最猛的风是十二级的暴风,这位吴小姐的举动也真可谓是暴风骤雨,但事实证明在历史中,最猛烈的风不是暴
风,而是枕头风。

万妃挨了打,回去就向朱见深告了状,在这场争斗中,吴皇后靠的是家世和身份,而万妃靠的是宠信,那么结果如
何呢?

自然是万妃赢了。(还是皇帝说了算)

朱见深听说万妃被打之后,十分生气,当即作出了处理。

他废掉了吴小姐的皇后名分,而此时她刚当皇后一个月。除此之外,吴小姐的父亲也被免官充军,而吴家的老朋友
牛玉也被牵连在内,这位原来的司礼监竟然被发配去孝陵种菜,做了菜农。

更让人难以置信的是,那位曾挺身而出,平定叛乱的孙镗也被免了职,原因竟然是据说他和牛玉有亲戚关系。

\section[\thesection]{}

皇后只干了一个月就被废掉,这可谓是前所未闻,而且此事竟然牵涉进去那么多无关的人,影响实在太坏,内阁成
员李贤、彭时向朱见深进言,希望皇帝能够三思,收回成命。

朱见深只是笑了笑,没有回答,也没有解释,只是一如既往地宠爱万妃。

一年后(成化二年 1466),万妃迎来了她人生的转折点,这一年正月,她为朱见深生下了一个儿子,朱见深闻讯大喜
过望,立刻封她为贵妃,还为此去宗社祭天,感谢祖宗保佑。

如无意外,万贵妃的这个儿子必定会成为将来帝国的继承者,可是遗憾的是,这一幕最终并没有出现。

第二年,这位皇子就患病夭折了,而这一年万贵妃已经三十八岁,她几乎不可能再生育儿女了。

这一事件严重地打击了朱见深,却并没有影响到朱见深对万贵妃的喜爱,此时的朱见深年仅二十一岁,正是少年风
流的时候,可他却一反常态,日夜守在这个大龄女人的身边,似乎永远也不会厌倦。

朱见深不急,下面的大臣们可急了,内阁成员彭时估计是分管妇联工作的,眼看朱见深如此专宠万贵妃,而这位中
年妇女很明显已经过了生育年龄,担忧皇帝无后,于是便发挥了文官集团以天下为己任、无论大事小事都要管的居
委会工作精神,给皇帝上了一封十分特别的奏折。

这封奏折堪称奇文,具体内容就不写了,大致意思是:

皇帝陛下,您的后宫有很多妃子,可是到现在却还没有儿子,臣想这应该是陛下过于宠爱某一个人所致吧,所以希
望陛下能够将宠爱分给其他的妃子,这是国家大计啊。

真是不看不知道,一看吓一跳,这位彭时先生竟然干涉起皇帝的私生活来,公然上书劝皇帝平时多找其他老婆联络
感情(文言''雨露均沾''),按说一般的皇帝看到这样的文书早就跳起来骂了:''我睡老婆,还要你管吗?''

可这位朱见深先生的反应更加出人意料,他一点也不生气,只是淡淡地说道:

``这是我的私事,你让我自己做主吧。''

然后他依然故我。

大臣们的疑惑已经到了极点,他们不明白,这个万贵妃容貌并不突出,年龄也大了,为什么皇帝陛下竟然可以忽略
那么多年轻貌美的女子,专宠她一个人呢?

\section[\thesection]{}

朱见深明白大臣们的疑虑,但他并不想解释什么,因为他知道,这些人是不会理解的。

在那孤独无助的岁月里,只有她守护在我的身边,陪伴着我,走过无数的风雨,始终如一,不离不弃。

是的,你们永远也不会明白。

在这世上,爱一个人不需要理由,从来都不需要。

意外的收获

对于朱见深而言,万贵妃是他的妻子,是这个世界上最善良、最可信的人,但可惜他不知道,这位万贵妃还有另外
一幅隐藏的面孔。

要知道,虽然朱见深是一个很专情的人,可他毕竟是皇帝,绝不可能只宠信万贵妃一个人,他也会时常找后宫的其
他妃子或是宫女,万贵妃也从未反对过,双方似乎相安无事,但朱见深似乎一直以来都忽略了一个重要疑点:为什
么这么久过去了,他还没有任何子女呢?

朱见深万万想不到,之所以出现这种情况,是因为所有怀上他孩子的妃子或宫女都被人逼迫堕胎了!而干这件缺德事
正是那位集万千宠爱于一身的万贵妃。

但从来就没有人告诉过朱见深这些事情,原因很简单,他们不敢。

如果就这样搞下去,也许下一任皇帝朱祐樘先生就得另找地方投胎了。但也就在此时,万贵妃真正的敌人出现了,
正是这个人彻底打破了万贵妃的如意算盘。

说来滑稽,万贵妃的这位敌手并不是选出来的,而是打出来的。

成化初年(1465) 广西 大藤峡

都察院都御史、远征军指挥官韩雍正站在峡谷的入口,仰望着上方的悬崖绝壁,为了平定两广土官叛乱,他带兵千
里行军赶到这里,却发现山势险恶,方向难寻,常年的带兵经验告诉他,这里是最好的伏击地点。

正当他为找不到一条安全的出山之路发愁的时候,手下兴奋地向他报告,他们在前方找到了十几个当地的儒生和里
长,熟悉附近地形,愿意为大军带路。

韩雍说:带我去看看。

他缓步走到那些当地人的面前,并没有迎上前去和他们热情握手,感谢他们即将为祖国做出的贡献,却出人意料地
大笑起来:

``就凭你们这几个人,也敢来行刺!都给我抓起来!''

\section[\thesection]{}

儒生里长们大惊失色,左右人却都是莫名其妙,士兵们随即上前搜身,果然在他们身上发现了行刺的利器。

部下们十分惊奇:你怎么就知道这些人是叛军派来的呢?

韩雍笑着说道:''你们还不明白吗,此地荒郊野岭,道路难行,鬼才来闲逛,而且附近都是叛军,怎么会有儒生里
长四处活动?不是奸细刺客还能是谁?''

这件事情传到了叛军那里,没文化的土官们十分惊讶,以为韩雍有特异功能,惊为天神,士气受到了严重打击,不
久之后韩雍分兵五路进攻大藤峡叛军营地,叛军不堪一击,被全部歼灭。

得胜功成的韩雍站在山顶之上,俯视着山间的那条大藤,所谓大藤峡即因此藤而得名,历来被土官们视为圣物,顶
礼膜拜。

韩雍笑着问被俘土官:''这藤是干什么用的?''

土官对他的调侃态度十分不满,一脸严肃地回答:''此藤横跨山崖,白天不见踪影,夜晚方现,是此地天赐神物。''

韩雍的脸上闪过一丝坏笑,对身边士兵说道:''拿斧头来!''

没等土官们反应过来,韩雍突然举起大斧,朝那藤全力砍去,于是神物就此一斧两断,成了废物。

这下子土官们一下子炸了锅,个个目瞪口呆,惊慌失措看着韩雍,而韩雍却只是轻松地笑了笑:

``诸位不要激动,藤断了也没什么,改个名字就行,我拿主意,今后此地就叫断藤峡吧。''

这就是明代历史上著名的成化两广叛乱和断藤峡之战,要说这事也算是个大事,但因为如果和由此事引发的后续事
件比起来,那可就只能算是小巫见大巫了。

说来让人难以相信,后来那惊心动魄的一幕幕活剧竟是由这样一件小事引起的:

平定了叛乱后,韩雍准备班师回朝,这时他的一个部下向他请示了一件事:

``我们俘获了很多当地土民,如何处理?''

韩雍漫不经心地回答道:

``这还不简单,交当地官衙放归乡里严加管束就是了。''

说到这里,他突然想起了什么,便补充了一句:

``去挑一些年轻的,男女都要,我要带回京去。''

这里有必要说明一下,韩雍的举动也算是老习惯了,明朝每逢边界打仗抓到俘虏,总会挑一些男男女女到京城,送
进王府或是宫里各有不同用场。

\section[\thesection]{}

一般说来,女的会被安排做宫女,而男的就比较惨,他们的新职业比较统一 --- 太监。伟大的郑和同志就是这样进
入宫廷的。

韩雍做梦也想不到,他的这一举动将给大明帝国带来深远的影响,并导致了两个截然不同的结果 --- 八年心惊肉跳
的乱世,十八年国泰民安的盛世。

因为在那批进宫的人中,有这样一男一女,男的叫汪直,女的姓纪,名字不详。

男的还没到出场的时候,让他先在后台等等吧,而那个姓纪的女孩,将成为风光无限的万贵妃最为可怕的敌人。

最强大的武器

吴小姐的下场让所有的人都知道了一个常识:这个不起眼的中年妇女是皇帝最为宠信的人,如果要得罪了她,只有
死路一条。

接替皇后位置的王小姐也是胆战心惊,经常串门,主动问安,就怕这位无冕之后什么时候心血来潮,闲来无事整她
一下,那可就大大的不妙了。这也难怪,吴皇后有容貌有权势有名分,来势汹汹,万贵妃却只用一个小报告就结束
了她的皇后任期,杀人于无形之中,着实厉害得紧。

此时的万贵妃俨然已经成为了后宫真正的统治者,呼东喝西,指南骂北,但凡有后宫妃嫔宫女怀孕,她便立刻指使
手下的人去逼迫堕胎,好不威风,自己生不出来就不让别人生,真可谓是断子绝孙、一统江湖。

也就在这个时候,广西来的纪姑娘进入了深宫,此时的她背井离乡,孤苦一人,怯生生地注视着周围陌生的一切,
没有人会想到(包括她自己),就在不久之后,这个羞涩胆怯的小姑娘将会撼动万贵妃那看似稳如泰山的权势与地位。

纪姑娘被分配入宫,做了一名普通的宫女,可是出人意料的是,这位宫女一进宫就得到了宫中几乎所有人的喜爱,
因为很快人们就发现,她是一个十分容易相处的人,她原先是广西土官的女儿,养尊处优,还能够识文断字,却从
不因由官宦之家的小姐沦为宫女而怨天尤人,即使人家欺负她,交给她很多脏活累活,她也并不在意,只是一个人
默默地做完。

她虽然没有权势、没有背景、甚至于没有过人的容貌,却有着一样女人最为强大的武器 --- 善良。

\section[\thesection]{}

她真心诚意地对待每一个人,从不去计较什么,只是一心一意地完成分派给自己的工作,由于她的出色表现,上级
派给了她一个重要的职务 --- 仓库管理员。

一般来说,这管仓库实在不能算是个体面的差事,但纪姑娘这个仓管员当得却是十分风光,这是因为她管的那个仓
库比较特别 --- 钱库。

更为重要的是,她管的这个钱库并非国库,而是内藏库,这里有必要解释一下,国库里存放的就是国家的钱,是由
户部管的,而所谓内藏库里存的是皇帝的私房钱,由他自己掌管,并不用交给后宫的老婆们(不容易啊)。这也为后
来发生的一切打下了伏笔。

成化五年(1469)的一天,纪姑娘正如往常一样认真清点着仓库,一个人走了进来。

这位仁兄就是朱见深同志,不知他是不是闲来无事,想去自己的钱库数钱玩,便一路进了仓库,正遇上仓库管理员
纪姑娘。

这是他们之间的第一次相遇。

朱见深对这个管仓库的小姑娘起初并不在意,他关心的只是仓库里的钱,四处巡视之后,他开始询问仓库的收支情
况。

可是问着问着,朱见深突然发现了一件很有趣的事情。

后宫中女子众多,许多人几年也难得见皇帝一面,所以每当真正见面时,往往都是''激动地心,颤抖的手,一句话
也说不出口''。对这一场景朱见深已经是司空见惯了,可这一次,通常的那一幕却并没有发生。

眼前的这个小姑娘十分特别,虽然初次见面,却应答如流,而且神情自然,不卑不亢,回答问题条理清楚,井然有
序,毫不紧张,好像并没有意识到眼前的这个人就是众多妃嫔争夺的对象,君临天下的皇帝。

后宫的那些你争我夺,勾心斗角的是是非非似乎与她毫不相干,回答完朱见深的问题,她便退后静立一旁,不说一
句多余的话;,不问一个多余的问题。在她的眼中,管理仓库才是自己唯一的工作。她不想去获取什么,也不想去
争夺什么。

不自是,故彰;不自伐,故有功;不自矜,故长;

夫唯不争,故天下莫能与之争! --- 道德经

朱见深被深深地打动了,这个看仓库的小姑娘没有矫揉造作的仪态,也没有心思机敏的试探,她的身上只有如清风
流水一般平淡的随和与友善,但这已经足够了。

\section[\thesection]{}

他喜欢上了这个小姑娘,当然了,由于他是皇帝,自然不用经过加深了解、互致问候、拜见双方父母之类的复杂过
程,直接就``临幸''了。

这以后的事情出乎意料地平淡,仓库管理员纪姑娘并没有如诸多后宫小说中描述的那样飞黄腾达,这并不奇怪,因
为以她的性格,是不会主动向朱见深要求些什么的。

此后,她依然如往常一样管理着她的仓库,也从未对人谈论过这件事情,对她而言,这件事情似乎从来都没有发生
过。

可是上天偏偏要给她一个不平凡的命运,就在不久之后,她发现自己竟然怀孕了。

按照常理,在古代,要是哪位女子怀上了皇帝的孩子,那可是了不得的大事,地方政府要到该女子的家中敲锣打
鼓,燃放鞭炮,洽谈将来的合作事宜,家中父母要一把鼻涕一把泪地给祖宗上柱香,而那些风水先生们也会跑到这
家的祖坟上去搞理论研究,总而言之两个字 --- 风光。

可当时纪姑娘面临的环境则应该用另外两个字来形容 --- 危险。

因为当时的后宫正处于万贵妃的管辖之下,而这位万贵妃最不能忍受的声音就是婴儿的啼哭,因为对于她而言,这
无异于丧钟的轰鸣。为了她的地位,她必须除掉所有可能对她造成威胁的新生命 --- 包括那些即将诞生的。

出于母亲的天性,纪姑娘很想保住她即将出生的孩子,所以她多方隐瞒,可是很不幸,她怀孕的事情最终还是被万
贵妃知道了,于是这位后宫的统治者决定派她身边的一位亲信宫女去处理此事 --- 堕掉那个即将出生的孩子。

夺走她孩子的人就要来了,纪姑娘却没有任何对策,她身处后宫,无处可逃,更无处伸冤,她很清楚,之前很多妃
嫔的孩子都是这样被处理掉的,而她作为一个小小的仓库管理员,又能够做些什么呢?

上天无路,遁地无门。

万贵妃的亲信终于还是来了,她走进纪姑娘那所简陋的住所,面无表情地看着她挺起的肚子和惊慌的眼神,没有说
一句话,转身走了。

然后她回到万贵妃的寝宫,回复了她的答案:

``她的身体有病,但并未怀孕。''

``你肯定吗?''

``我肯定。''

\section[\thesection]{}

我没有能够在史书中找到这个宫女的名字,这并不奇怪,因为在后世史家的眼中,她不过是个无足轻重的小人物,
不过在我看来,在王侯将相的历史中,她也有着属于自己的称呼 --- 一个有良心的人。

万贵妃被瞒了过去,而纪姑娘肚子里的孩子终于保住了性命,后宫又恢复了往日的平静,但在这平静的外表下,事
情才刚刚开始。

成化六年(1470) 七月 己卯

伴随着一声响亮的啼哭,经历了痛苦分娩的纪姑娘终于生下了一个男孩,和所有的母亲一样,她欣喜地看着自己的
孩子,看着这个刚刚诞生的生命,紧紧地将他拥入怀中。她已经没有了父母,没有了兄弟姐妹,因为即使他们没有
在战乱中死去,也注定永远不能再见面。

现在她终于有了在这个世界上唯一的亲人 --- 儿子。

这是幸福的一刻,她孤独的生命终于有了寄托,有了希望。

可是她的幸福并没有延续多久,因为这一声啼哭也惊动了后宫中的另一个人,一个满怀失落和仇恨的女人。

她终归还是知道了这个孩子的诞生,嫉妒的火焰在她的心中燃烧起来,为什么她有孩子,而我没有?!我才是后宫的
统治者,是皇帝最为宠信的女人,任何人都不能将这一切从我身边夺走!

她下达了命令:

``溺死那个孩子!''

接受命令的人叫张敏,他只是一个普通的宦官,但希望大家能够记住这个名字。

他奉命来到纪姑娘的住所,推开房门,看见了纪姑娘和她怀中正在吃奶的孩子。

这一次,纪姑娘不再惊慌了,历经这么多的风风雨雨,她很清楚即将发生些什么。

她从容地说道:

``做你该做的事情吧。''

张敏站在门口,静静地看着这对母子,一动也不动,过了很久,他走了进去,从纪姑娘手中小心翼翼接过了孩子。

``孩子在这里不安全,还是交给我吧,过段时候你再来看他。''

他没有再看纪姑娘那惊愕的表情,抱着孩子径自走了出去。

张敏抱走了孩子,找了宫中一间空置的房子,安顿了这个孩子,他还和宫中的其他太监商议,从他们那少得可怜的
收入中挤出一些钱,买来乳糕裹着蜜糖喂养这个没奶吃的孩子。在没人注意的时候,纪姑娘也会经常来看望她的孩
子。

\section[\thesection]{}

从此,这个孩子就成为了后宫中宫女太监们那枯燥生活的最大乐趣。他们都很喜欢这个孩子,原因很简单,作为这
座冷酷的后宫中的普通一员,他们永远也不可能有自己的孩子。

可是随着这个孩子一天天长大,张敏等人逐渐发现了一个新的问题:他们养不活这个孩子。

张敏是一个普通的宦官,并非司礼监,而他的同事和那些知情的宫女们都只是这座金碧辉煌的后宫中的最底层,没
有额外的收入,除了自己花销外,每月根本剩不下什么钱,虽然这个孩子不用上托儿所,也不用交什么择所费,更
不用上那些各种各样的辅导班,但即使如此,他们还是无法承担养育他的费用。

对于这个问题,纪姑娘也没有更多的办法,她只是一个小小的仓库管理员,也没有额外收入,养不起自己的孩子。

大家都养不起,难道要拿去送给万贵妃?,正当他们一筹莫展的时候,另一个人说话了。

``那就交给我来养吧。''

讲这句话的正是前任皇后吴小姐。

虽然是前任皇后,但毕竟瘦死的骆驼比马大,吴小姐家有钱有势,养一个孩子自然不在话下,当然了,她的动机估
计没有那么单纯,打倒万阿姨仍然是她的最终目的,无论如何,这个孩子能够活下来了。

这之后的五年,纪姑娘的这个孩子一直在宫中生活,虽然他不能出去玩,但在她母亲、吴阿姨、张叔叔以及无数叫
不出名字的内监宫女的照料下,他一直幸福地成长着 --- 至少比他的父亲幸福。

日子一天天的过去,孩子一天天地长大,而这些生活在后宫最底层的人们却没有发现,他们已经创造了一个奇迹。

从成化六年(1470)到成化十一年(1475),整整五年时间,紧密森严的后宫中多了一个孩子,这一点,几乎所有的宦
官、宫女、妃嫔们都知道,但他们却无一例外地保持了沉默,守住了这个秘密。

只有一个人不知道 --- 万贵妃。

这不是一个故事,而是真实的史实,是发生在以争宠夺名、勾心斗角闻名于世的后宫中的史实。在这里,人们放弃
了私欲和阴谋,保守了这个秘密,证明了善良的力量。

读史多年,唯一的发现是:几千年来我们似乎在重复着同一种游戏 --- 权力与利益的游戏,整日都是永远也上演不
完的权力斗争、阴谋诡计,令人厌倦到了极点。但这件事似乎是个例外,它真正地打动了我。

我们这个古老国度有着漫长的历史,长得似乎看不到尽头,但我却始终保持着对这些故纸堆的热情。

因为我始终相信,在那些充斥着流血、屠杀、成王败寇,尔虞我诈的文字后面,人性的光辉与伟大将永远存在。

\section[\thesection]{}

最后的抉择

这个吃百家饭长大的孩子就这样在后宫中快乐地生活着,对他而言,有母亲的陪伴,还有那么多叔叔阿姨宠爱着
他,每一天的生活都是幸福的,但纪姑娘明白,这种日子是不会长久的,她和她的孩子最终还是要面对命运的最后
裁决。

这一天终于来临了。

成化十一年(1475) 五月 丁卯

朱见深坐在镜子面前,一个宦官正站在他的身后为他梳头,端详着镜中自己那憔悴的容貌,他深深地叹了一口气,
虽然他还不到三十岁,却已未老先衰,这倒也罢了,他真正担心的是另外一件事。

``我还没有儿子啊!''

当朱见深为自己的不育问题而烦恼时,站在他身后的那个人也正在痛苦中思索着自己的抉择 --- 说,还是不说?

这个梳头的宦官正是张敏。

六年前的那个夏天,他奉命去除掉一个孩子,面对着那对孤苦的母子,他最终违背了冷酷的命令,选择了自己的良
知。五年之中,他和这个孩子朝夕相处,看着他一天天地长大,度过了很多快乐的日子,可他很清楚,这件事情总
会有一个了结。这个孩子必须获得他父亲的承认,才能活下去,并成为这个帝国的继承者。

现在时机到了。

但他也很明白,自己不过是一个普普通通的宦官,无权无势,如果说出真相,以万贵妃的权势,他将必死无疑。

真相大白之日,即是死期来临之时。

这是张敏一生中最为痛苦的时刻,要让这个孩子活下去,他就必须舍弃自己的生命。

除此之外,别无选择。

一生低声下气,地位卑微,终日带着讨好笑容的张敏终于作出了他人生最后的抉择 --- 一个伟大的抉择。

``陛下,你已经有儿子了。''

离别

朱见深惊诧地回过头,第一次认真地打量着这个为他梳头的宦官。

``你刚才说什么?''

``陛下,你已经有儿子了。''

朱见深一动不动地盯着跪在地上的张敏,确定他并非精神错乱之后,方才半信半疑地问道:

``在哪里?''

但这一次,张敏没有立刻回答他的问题,而是选择了沉默。

朱见深疑心顿起,厉声追问道:

``为什么不答话?!''

跪在地上,半辈子卑躬屈膝的张敏抬起了头,无畏地看着朱见深,提出了一个条件:

``我自知说出此事必死无疑,但只要皇上能为皇子做主,死亦无憾。''

\section[\thesection]{}

就这样吧,我相信我做出了正确的决定。

朱见深被眼前的这个小人物震慑住了,他知道,一个有胆量说出这句话的人是不会说谎的。

``我答应你,告诉我在哪里吧。''

然后他得知自己有一个已经五、六岁的儿子,正在后宫的安乐堂内玩耍。

此时的朱见深什么也顾不上了,他喜形于色地奔向了后宫,并立刻派人去安乐堂接他的儿子,大明皇位未来的继承
者。

此时的后宫已经乱成一团,大家都已知道皇帝派人来接孩子的消息,宦官宫女们都十分高兴,而妃嫔们也纷纷来到
纪姑娘的住处,向她道贺。

这也是一件十分自然的事情,自古以来母以子贵,纪姑娘保住了孩子,很快就能成为纪贵妃甚至纪皇后,甚至有可
能取代万贵妃成为后宫的统治者。

纪姑娘微笑着送走了前来祝贺的人们,然后她关上了房门,向她的儿子做了最后的道别。

她在战争中永别了自己的亲人,被俘获进宫,在孤苦中延续着自己的生命,直到这个孩子的出现。六年的含辛茹
苦,九死一生,她和自己的孩子最终熬到了出头的这一天。

但此刻的纪姑娘并没有丝毫的喜悦,因为她十分清楚,虽然皇位正向她的儿子招手,但死亡却离她自己越来越近。

万贵妃会毫不犹豫地杀死所有与她为敌的人,在这座皇宫中,没有任何人可以保护她的安全,即使她是皇子的母亲。
而孩子的父亲,软弱的朱见深对此无能为力。

她看着自己的孩子,这个她在世上唯一的亲人,最后一次亲手为他穿上了衣服,最后一次紧紧地将他拥入怀中,哭
泣着向他告别:

``孩子,你走后,我也活不了多久了,你去到那里,看见一个穿着黄色衣服,有胡子的人,那就是你的父亲啊,今
后一切千万小心,母亲再也不能陪伴你了。''

年幼的皇子并不知道发生了什么事情,为什么周围的人今天表现得如此奇怪,为什么母亲会痛哭失声。他只知道,
自己就要离开这里,到另外一个地方去,去找一个有胡子的人。

离开了哭泣的母亲,这个孩子在他出生六年后第一次走出了自己居住的地方,离开了母亲,坐上了迎接他的小轿,
踏上了未知的道路。

\section[\thesection]{}

很快,他到达了这次旅行的终点,他的父亲正在那里等待着他。

由于深居简出,这位皇子直到六岁还未理发,头发一直垂到了地上,他就这样跌跌撞撞地向那个穿着黄色衣服,坐
在椅子上正凝视着他的人走去。

朱见深看着这个向自己走来的孩子,激动的心情再也无法抑制,他立刻迎上前去,抱住了这个孩子,放在自己的膝
上,仔细地端详着他。

很快,他哭了,他一边流着眼泪,一边紧紧地抱着孩子大声说道:

``这是我的儿子,这是我的儿子啊,他像我!''

不用亲子鉴定,不用指认,不用证据,这就是我的儿子,毫无疑问。

他牵着这个孩子回到了自己的寝宫,并告知母亲周太后和所有的大臣们,自己有儿子了。

所有的人都欢呼雀跃,周太后更是兴奋异常,抱着她这个来之不易的孙子丝毫不肯撒手,大家都在为大明帝国后继
有人而高兴,只有一个人例外。

后宫中的那个女人已经愤怒地几乎丧失了理智,派去堕胎的人敷衍了她,派去谋杀的人了隐瞒了她,所有的人都知
道这个孩子的存在,却没有一个人告诉她。

``你们都欺骗了我!''

复仇的意愿在她心中猛烈地膨胀。

让那个孩子和她的母亲消失,让一切都回到事情的起点,敢于欺瞒我的人,一个也不能放过!

那个在宫中躲藏了多年的孩子终于可以正大光明地生活下去了,他有了自己的寝宫,自己的宫女宦官,自己的从
属,也有了自己的名字 --- 朱祐樘。

纪姑娘也变成了纪妃,正式成为了朱见深的合法妻子,这个广西来的小姑娘似乎已经迎来了人生的转折。但事实证
明,她对自己命运的判断十分准确。

朱祐樘进宫一个月后(成化十一年六月),纪妃死于后宫住所,死因不详。

关于她的死亡方式,最终并没有一个定论,有的说她是被逼自尽,有的说是突发重病身亡。但她的死因却似乎并没
有引起什么争论,后世那些特别热衷于挖人隐私的历史学家们,出人意料地对这件事情也没有产生太大的兴趣。

因为所有的人都知道凶手的名字以及行凶的动机。

\section[\thesection]{}

这位从广西来的小姑娘就此结束了她的一生,直到现在,我们仍然不知道她的名字,她的家庭成员,甚至于她的准
确年龄。因为她不善言谈,入宫之后大多数时间,她只是静静地干着自己的工作,接受着别人交给她的任务,从未
向人谈起她的故乡和亲人。

十二年后,她的儿子,已经成为皇帝的朱祐樘曾发动无数人去寻找她母亲的家世和亲人,广西各级官员自发动员起
来,从布政史到县令,甚至包括当年曾经出征广西的韩雍手下的将领们,纷纷赤膊上阵,改行当了户口查缉员,他
们挖地三尺,历时近十年,把广西全境翻了个底朝天,闹得四处鸡犬不宁,最终却只找到几个想借机发财的骗子。
无奈之下,朱祐樘唯有在当地树立祠堂,册立封号,以缅怀对这位伟大母亲的哀思。

在历史上,她最终也只是一个昙花一现,连名字也未能够留下的女子。

但我仍然记下了她的名字 --- 一个尽力保护自己孩子的母亲,一个善良的女人。

听到纪妃去世的消息,宦官张敏苦笑着叹了一口气:

`` 这一天迟早是会来的。''

几天之后,他在后宫中吞金自尽。

当一个人不得不走向死亡时,自杀代表着尊严和抗争。

就在给朱见深梳头的那一天,张敏对天许下了一个承诺,用他的死亡去换取这个孩子的生存。上天在这个问题上表
现得很公平,他履行了义务,给了这个孩子快乐的生活,也行使了权利,把张敏送上了不归之路。

我查了一下才发现,从仕途上讲,这位叫张敏的宦官混得实在很失败,从头到尾,他只是一个门监,在今天这一职
务又被称为``门卫''或是``看大门的''。

可就是这个普通得不能再普通的看大门的宦官,却做出了无数名臣名相也未必能够做到的事情。面对死亡的威胁,
他选择了良知。

舍弃生命,坚持信念,去履行自己的承诺。这种行为,我们称为舍生取义。

张敏,是一个舍生取义的人。

\section[\thesection]{}

幸存者

纪妃和张敏都死了,短短一个月间,朱祐樘就失去了他最为亲近的两个人,此时的他还不懂得什么是哀伤,只是偶
尔会奇怪为什么母亲再也不来看他。

而与此同时,死亡的阴影也正悄悄地笼罩着这个孩子,对于后宫的万贵妃来说,这个孩子是个极为危险的人物,他
会夺走朱见深的宠爱。于是另一场谋杀的阴谋即将实施。

可能有人会奇怪,如此恶行,难道没有人管吗?

要知道,万阿姨虽然年纪大了,却并不是傻瓜,她之所以敢如此肆无忌惮地除掉每一个他厌恶的人,其中可是大有
奥妙。

她看着朱见深长大,十分了解这位皇帝,如果用两个字来概括朱见深的性格,那就是懦弱。公正地讲,朱见深并不
糊涂,智商也不低,算是一正常孩子,可童年的阴影使他的性格十分软弱,并且有极强的恋母情结(关于这个问题,
可以参照四百年后弗洛伊德先生的理论),因而极度依赖万贵妃。

这样的一个家伙,有啥好怕?

眼看朱祐樘就要英年早逝,另一个女人站出来挽救了一切。万贵妃虽然统领后宫,但这个女人,她无论如何也是惹
不起的。

此人就是朱见深的母亲周太后,按照辈分,万贵妃还要叫她一声娘亲。要说这位周太后,那可是见过大世面的,想
当年,正统土木之变,景泰金刀疑案,刀光剑影,你来我往,周太后都挺住了,万贵妃搞的这点名堂,只能算是和
风细雨的小场面。

``把孩子交给我,看谁敢动他一指头!''

一声令下,朱祐樘住进了太后的仁寿宫,这下万贵妃彻底没戏了。

可是历史告诉我们,阶级敌人是不会甘心失败的,不久之后,朱祐樘就接到了万贵妃的热情邀请,希望皇太子(此时
已册立)殿下大驾光临。

朱祐樘也没想太多,松一松腰带就准备上路,此时周太后却站了出来,郑重其事地告诉他:

``去到那里,什么也不能吃!千万记住了!''

``要是一定让我吃呢?''

``就说你吃饱了!''

到了地方,万贵妃果然拿出了很多好吃的东西,和颜悦色地对朱祐樘说:

``吃点吧。''

朱祐樘收住了口水,说出了违心的答案:

``吃饱了。''

按说事情到这里就算结束了,可是朱祐樘小朋友,世事难料啊。

``那就喝点汤吧。''

完了,这句没教过啊!

\section[\thesection]{}

他低下头开始思考标准答案,一旁的万贵妃却仍在不停地催促着,要说这孩子心眼还真是实在,憋半天憋得脸通
红,终于蹦出了一句惊世骇俗的话:

``我怕有毒!''

万贵妃目瞪口呆,看着一脸无辜的朱祐樘,几乎当场晕倒在地:你小子也太直接了吧。

阴谋被搞成了阳谋,这下彻底没戏唱了,那汤里到底有没有毒也不重要了,太子殿下过了一回眼瘾,就此打道回府。

万贵妃晕倒前最后留言:

``这小子现在就敢这么干,将来还不得吃了我!''

自此之后,万贵妃就如同被斗败的公鸡,彻底失去了往日的威风,不敢再堕掉别人的孩子,而朱见深同志也趁开放
的大好形势,越发神勇,又生下了他的第四个儿子,(前两个夭折了,朱祐樘是第三个),此后他又接连生了十余个
儿子,一举彻底洗刷了不育的恶名。可他怎么都不会想到,除了太子之外,那位第四个出生的皇子在经历了无数风
波之后,最终竟然也成了皇帝。

这些事情得等到四五十年后了,还是先安排成化年间的诸位大人们出场吧,他们已经等不及了。

武林大会

要说这成化年间的朝政,用一个词就可以完美地概括和形容 --- 一塌糊涂。

这一点也不奇怪,朱见深同志的领导水平实在对不起人,他连自己的老婆都管不住,怎么管得住身边的秘书们?

在这种情况下,成化年间的政治顿时变得异彩纷呈,黑暗无比,而涌现出的各个政治流派更是多姿多彩,百花齐
放,聚集在这个混乱的江湖中,召开了一场花招层出不穷,犯规屡禁不止的武林大会。

下面我们开始介绍参加武林大会的各大门派(排名不分先后)

春派

全称:春药研究派。

掌门:梁芳

门下弟子构成:术士、番僧

独门绝技:化学物品研究(春药,现俗称伟哥),生理卫生知识研究

仙派

全称:修道成仙派。

掌门:李孜省

门下弟子构成:和尚、道士

独门绝技:炼丹(属化学门类)、修道

监派

全称:内监宦官派

掌门:汪直、尚铭

门下弟子构成:太监

独门绝技:地下工作(特务)、打小报告

后派

全称:后宫老婆派

掌门:万贵妃

门下弟子构成:宫女、太监

独门绝技:一哭二闹三上吊(此绝技经过长期演变,现已普及使用)

混派

全称:混日子派

掌门:万安

门下弟子构成:文官集团

独门绝技:混日子、弹劾(告状)

这就是当时纵横江湖的五大门派,要诉说他们的来历瓜葛,您且上坐,听我慢慢道来。

\section[\thesection]{}

什么是江湖?有人的地方就有江湖。

话说两千年多前绝世高手赢政一统武林,荣任第一任武林盟主之后,江湖便陷入了众派林立,腥风血雨的光辉岁月。

在众多的门派中,资格最老、水平最高的是两大门派 --- 监派和后派。

这两派的地位大致相当于少林和武当。其中后派的历史学名叫做外戚,监派的历史学名叫权阉。

两派虽然都服从武林盟主(皇帝)的调遣,但从挂牌子成立那天起,就是不共戴天的死敌,此消彼长,你死我活,几
千年来就没消停过,而两派门中也都是高手辈出。

比如监派的赵高、单超、李辅国、鱼朝恩以及后派的吕后、杨坚、韦后等人,全部都是纵横一时的高人,为本派争
得了极大的荣誉。两派在斗争之余,偶尔也会携手合作,一旦这种情况出现,武林盟主便会趁机混水摸鱼,不断在
两派间挑起是非,以维护自己的盟主地位。

当然了,有时候如果盟主武功不高,也有可能被这两派的高手取而代之,如杨坚就成功地脱离后派,成为新的武林
盟主。

到了成化年间,这一情况并没有改变,后派和监派仍然水火不容,而其他门派也趁此机会,开张的开张,壮大的壮
大,这就是我们之前介绍过的另外三派。

春派是后派的附属门派,春派掌门人梁芳原先是后派掌门万贵妃的物品采购员,由于胆大心黑,敢于中饱私囊,贪
污公款,工作干得十分出色,被提拔为春派掌门,自立门户。

这里还要表扬一下梁芳同志的刻苦认真态度,大家知道他是研究春药的,但他干这行也真不容易,因为他本人是个
宦官,在看得见吃不着且理论脱离实际的情况下,能够如此卖力工作,着实体现了卓越的钻研精神和职业素养。

这是春派,下面我们说仙派。

\section[\thesection]{}

仙派也是一个历史悠久的派别,该门派最出名的人物应该就是秦朝那个据说去了日本留学的徐福,而到了成化朝,
仙派也出人头地了,该派掌门李孜省原先在江西衙门里当小公务员,后来改行去京城北飘,顺便也干点诈骗的活。

后来他在行骗过程中遇见了春派掌门梁芳,就当了梁掌门的随从,而梁掌门对他也甚是欣赏,支持他另立门户,发
挥特长,为盟主朱见深炼丹修道,从而一举打响了仙派的威名。

接着是鼎鼎大名的监派,此派在明代极为兴盛,前有郑和、王振,后有刘谨、魏忠贤,可谓人才济济,而在成化
朝,这一派却出现了分裂。

如同华山派有气宗和剑宗一样,监派也分裂成了东监派和西监派,两大掌门各行其是,彼此之间斗争激烈,东监派
掌门尚铭根基深厚,秉承传统,不断壮大本派的传统附属企业 --- 东厂,脚踏实地做好刺探情报、诬陷忠良的特务
工作。

而西监派掌门汪直,自从被韩雍大军带到京城,挨了一刀变成宦官之后,奋发图强,打破传统发展模式,积极进取
(拍马屁),努力争取盟主朱见深的信任,并以人无我有,人有我优的创新精神在西安门开办了新型企业 --- 西厂,
其企业口号是``没有最坏,只有更坏'',在掌门汪直的带领下,全厂职工正全心全意加班加点坑人害人,力图效益
早日超过东厂。

后派就不用多介绍了,成化年间的万贵妃可谓一女当关,万夫莫敌,她不但是后派掌门,还是武林盟主朱见深的老
婆兼保姆,独门招式枕头风和枕头状横扫武林,无人能挡。

最后是混派,此派原叫臣派,本是与监派、后派齐名的大派,门下出过无数如李斯、霍光、房玄龄、王安石、三杨
之类的绝顶高手,可是到了此任掌门万安的手中,门庭冷清,万掌门武艺稀松,除了坚持练习磕头功和拍马功之
外,没有什么其它的本事,逐渐成为了后派和监派的附庸,直到十几年后,这种情况才得到了改观。

综上所述,成化年间的武林形势是这样的,后派和春派、仙派是同盟关系,可称之为泛后阵营。监派内部存在矛
盾,对外则与后派同盟敌对,最窝囊的是混派,无论监派后派它都不敢得罪,派如其名,只能乖乖地混日子。

以上就是武林五大门派的情况,相信你已看得出,这些都是所谓的邪派,如果你还在等待着名门正派的出现,恐怕
就只能失望而归了,因为此时江湖的情形完全可以用一句话来概括:

这年头,没有好人了。

\section[\thesection]{}

五派风云录

各派都到齐了,好戏也就该上演了。

春派掌门梁芳,卓越的药品批发商,物品采购商,他的发家之路主要有两条,其一是送礼给万贵妃,此外就是制造
春药送给皇帝,两面讨好,大家都喜欢他,所以在一段时间里他十分得势。

他虽身为宦官,却并非监派成员,当时的宦官首领司礼太监尚铭和怀恩都曾试图收编他,梁芳的回答却是:你算老
几?一边凉快去吧。

他仗着有人撑腰,大肆侵吞财物,朱见深同志的内藏原本有很多私房钱,可没过几年,就被这位仁兄用得干干净
净,气得盟主大人几天吃不下饭。

但梁掌门也有一个好处,由于他本人读书少,没什么见识,和王振、魏忠贤等人比起来,档次差得太远,除了捞钱
之外,也就是帮万贵妃去后宫堕个胎,更大的坏事他也干不出来(不是不想,实在是水平不高),他万万没有料到,
自己做过的最有影响的事情竟然是招募了一个人。

这个人就是后来的仙派掌门李孜省。

如果要问五派中谁最受朱见深的宠信,估计很多人会回答是后派或者监派,但实际上,朱见深最看重的恰恰是这个
不起眼的仙派掌门李孜省。

对这一点,实在不必吃惊,朱见深的心声可以明确地告诉我们原因:

我真的还想再活五百年!

春药也好,耳目也好,老婆也好,只要有这条命在,随时都可以再找。

生命是最宝贵的,朱见深明智地认识到了这一点。

所以,号称可以长生不老的李孜省自然成了朱见深的宠臣,而他本人也可谓再接再厉,不满足于用修道成仙糊弄盟
主,在炼丹的同时还在生产线上加入了副产品 --- 春药,开始抢自己老领导梁掌门的生意。

这样一来,多面手李孜省就成了炙手可热的人物,混派的掌门万安和大弟子刘吉、二弟子彭华都是靠他的关系才进
入内阁,做大官的。

可这位掌门并不满足,他还打算跨行业发展,竟然把手伸到了特务工作上,自己组织人员为盟主大人探听消息,这
下子可算是捅了马蜂窝,东厂西厂的众多特务们都眼巴巴地靠着这行吃饭呢,你李孜省算是个什么东西?!竟然敢打
破垄断,搞竞争!

监派掌门尚铭、汪直卷起裤腿,抄起家伙,准备向这个无名小卒发动进攻。

可是斗争的结果是他们意想不到的。

李孜省和太监的斗争就放到后面吧,先说其他两个门派。

\section[\thesection]{}

后派就没有什么可说的了,万贵妃仍然过着她的日子,三天两头巡视后宫,然后心又不甘地凝视着太子东宫的方
向,仅此而已。

下面轮到混派出场了,我个人认为,这是最有趣的一个门派。

在成化五年(1469)之前,内阁是一个庄严神圣的地方,那时的内阁成员是商辂和彭时。

商辂也算是老熟人了,早在北京保卫战时,他就露了一次脸,站出来支持于谦的主张,但他更出名的还是他的考试
成绩 --- 连中三元。想当初乡试发榜的时候,榜刚刚贴出来,人家还在瞪大眼睛找名字,他随便看了一眼,就打道
回府睡觉去了。同乡问他怎么不找自己的名字,他若无其事地指着榜单说道:

``费那功夫干啥,排最上面那个不就是我嘛!''

除去靖难时被朱棣打击报复,删去名字的黄观,他是明代唯一一个完成这一高难度动作的人,事实证明,他的为官
之道确实十分出色,而另一位内阁成员彭时也是状元出身,为官清正,在他们的带领下,大明帝国有条不紊地向前
行进。

就在这个时候,万安进入了内阁。

万安,四川眉州人,正统十三年(1448)进士,这位仁兄书读得很好,当年高考全国第四名,位居二甲第一,可惜从
他后来的表现看,他实在是应试教育的牺牲品,高分低能的典型代表。

他入阁后,不理政务,只是一门心思地干成了一件事 --- 拉关系。他充分地使用了自己的姓氏资源,竟然和万贵妃
拉上了亲戚。

什么亲戚呢?

据万安同志自己讲,万贵妃的弟弟的老婆的母亲的妹妹是他的妾,这可是了不得的近亲啊!

于是他跑到万贵妃的弟弟家,声泪俱下地认了这门亲事,并光荣地宣布,我万安终于找到亲人了啊!

无论亲戚是真是假,万安确实获得了提升的机会,成化十四年(1478),商辂退休回家,万安成为了内阁首辅。

从此,在他的``英明''领导下,文官团体的历史进入了一个新的时代 --- 混派时代。

话说这混派虽然以混日子为第一宗旨,却也并非毫无作为,承蒙江湖各位人物看得起,混派的许多精英都被赋予了
外号。叫起来甚是响亮,不可不仔细谈谈。

\section[\thesection]{}

外号党

混派掌门万安,江湖人送外号``万岁阁老''

成化七年(1471),万安和内阁其他两名成员商辂、彭时前去拜见朱见深,商讨国家大事,彭时开口刚谈了几件事,
正说到兴头上,突然听见旁边大呼一声:

``万岁!''

回头一看,万掌门已经跪在地上磕头了。

商辂、彭时瞠目结舌,呆了一会儿,无奈地叹了口气,也跪了下来,磕头叫道:

``万岁!''

这奇怪的一幕之所以会发生,完全是因为万安的那一声万岁,这关系到一个严肃的礼仪问题。

在清代,官员之间商谈事情,若端起茶杯,就意味着本人不想再谈,请你走人,即所谓端茶送客。

而明代面圣也有着一套礼仪,朝见完毕,口呼万岁,这意思就是皇上再见,俺们下次再来。

万掌门不知是不是急着上茅房,没等谈几句,匆匆忙忙地喊了再见,搞得内阁极为尴尬,成为了满朝文武的笑柄,
故而有了这个光荣的称号``万岁阁老''。

混派大弟子刘吉 江湖人送外号``刘棉花''

刘吉,河北人,正统十三年(1448)进士,是万掌门的同期同学,成化十一年(1475)成为内阁成员,这人品行和万安
差不多,但还有一点要强于万安 --- 脸皮更厚。

明代弹劾成风,言官也喜欢管闲事,刘吉这种人自然成为了言官们的主要攻击对象,可这位仁兄心理承受力好,言
官说了什么权当没有听见,所以江湖朋友送他一个雅号``刘棉花''。

何意?

棉花者,不怕弹也!

混派跟班小弟倪进贤 江湖人送外号``洗鸟御史''

倪进贤,安徽人,半文盲,拜入万掌门门下,系关门弟子,身无长物,却有着一个祖传秘方,据说配成药粉融于水
后,可以治疗ED(学名),万掌门估计亲身试验过,所以一喜之下,让这位兄台干了个御史。

要是换在今天,他大可不必去干什么御史,投身医药界,必定能兴旺同类行业,胜过辉瑞公司,为国争光。

考虑到他对万掌门的巨大贡献,江湖朋友十分尊敬地送给他一个外号``洗鸟御史''。

内阁中硕果仅存的刘翊,基本上也是每天混日子,至于下面的六部尚书,着实不愧为混派的优秀弟子,秉承门派章
程,每日坐在衙门里喝茶聊天,啥事也不干,严格遵守门规。

由于成化内阁及各部官员的优异表现,人民大众特别授予他们集体荣誉称号:

内阁三成员集体获得``纸糊三阁老''光荣称号

六部尚书集体获得``泥塑六尚书''光荣称号

这是群众给予他们的肯定。

叹服,叹服,都是些什么玩意儿!

\section[\thesection]{}

下面我们将最后一个门派 --- 监派,之所以把它留在最后讲,是因为成化年间最大的黑幕、最狠毒的人物都由此派
而起,却也由此派而灭。

汪直的奋斗史

在韩雍带回来那一大群俘虏中,汪直并不是一个显眼的人,也没什么特长,咔嚓之后老老实实地做了宦官,不过他
的运气很好,在宦官培训完毕分配时,他有幸被分到了后宫侍候皇帝的一位妃嫔 --- 万贵妃。

事实证明,虽然汪直没有啥才艺技术,但他的服务态度是十分端正的,服务水平也很高,哄得万贵妃十分开心,一
来二去,万贵妃就推荐汪直到朱见深那里继续培养深造,而汪直也着实不负众望,步步高升,最终成为了御马监的
太监。

我们曾经介绍过,御马监是仅次于司礼监的重要部门,能爬到这个位置,可以说已经是宦官中的成功人士了,可是
汪直并不满足,他又把手伸向了皇宫内最为神秘的太监管理机构 --- 东厂。

汪直自发组织人外出打探消息,汇报京城及各地的一举一动,表现自己的情报收集能力,就是希望朱见深能够把东
厂的控制权交给他。一时之间,京城内外四处都是汪直的便衣密探,没日没夜的打探消息,抓人关人,势头非常之
猛。

有了这些``政绩'',汪直便得意洋洋地去向朱见深汇报,准备接手东厂这个明朝最大的特务组织,干一把地下工作。

盟主大人听取了他的报告,给予了高度的评价,并表示希望他继续努力,可盟主似乎讲上了瘾,在上面长篇大论,
讲得头头是道,就是不说关键问题,汪直跪得腿发麻,终于忍不住插话:

``皇上,东厂的事情应如何办理?''

盟主被打断了发言,却并不生气,只是笑着摆摆手说道:

``那个人干得还不错,就这样吧。''

汪直的东厂梦想就此破灭。

盟主口中的``那个人''就是现任东厂掌印太监尚铭,这可不是一个简单的人。

\section[\thesection]{}

尚铭入宫很早,办事十分利落,性格极其谨慎(注意这个特点),东厂在他的手下搞得有声有色,为了扩大财源,他
还干起了副业 --- 绑票敲诈。

尚掌门有一个公认的闪光点 --- 对待工作认真负责,对他的副业也是如此,他一上任,就搞了一个花名册,上面一
五一十的记载了京城各大富户的地址、家庭环境,并就财富多少列出了排行榜。

同时他还有着扎实的哲学功底,始终坚信世界是一个联系的整体,所有的事情都是有联系的,每当东厂有了案件,
他都会把这些富户和案件联系起来,并且逐个上门抓人,关进大牢,让家人拿赎金来才放人。

这实在是一件十分缺德的事情,但出人意料的是,虽然他一直这样干,名声却还不错,许多人谈到他还时有夸奖,
着实是一件十分奇怪的事情。

这是因为尚掌门还有一个很大的优点 --- 讲究诚信。他虽然绑票,却从不虐待人质,而且钱到放人,从不撕票,和
他打过交道的人质家属也不禁如此感叹:收钱就办事,是个实诚人啊。

此外他虽然劫富不济贫,却也不害贫,从来都只在富户身上动手,不惹普通百姓。在中下层群众中间很有口碑。他
资历很高,却从不欺负后辈,人缘很好,还经常给盟主大人和后宫万掌门送礼,群众关系也不错。

这样的一个人,汪直自然是扳不动的。

可是汪直实在是一个很执著的人,他下定决心要打破尚铭的垄断,开创特务工作的新局面。禁不住他的反复要求,
成化十三年(1477)朱见深终于特批汪直开办新型企业 --- 西厂。

新官上任的汪直对此倾注了全部的心力,他立刻颁布了厂规和指导方针,大致可以概括为:

东厂害不了的,我们害,东厂整不死的,我们整,东厂做不到的,我们做!

他是这样说的,也是这样做的。

此后,西厂特务就成为了死亡的代名词,他们比东厂手段更为狠毒,一般百姓进了西厂几乎就等同于进了鬼门关,
压根就别想活着出来。京城上下人心惶惶,谈虎色变。

西厂日以继夜辛勤工作,可不久之后,汪直却郁闷地发现,无论业绩还是名声,他的西厂始终赶不上东厂。这是很
自然的,毕竟东厂有着悠久的历史和特务文化积淀,短时间内西厂确实望尘莫及。

\section[\thesection]{}

汪直是一个不服输的人,他不愿意屈居在尚铭之下,也不愿意等待,为改变这一局面,他发动下属提合理化建议,
并虚心采纳意见。

很快,一个下属给他出了一个主意,要想快点压过东厂,就得解决几个重量级的人物,这样才能短时间内打出威
信,打响西厂品牌。

事后证明,这是个馊主意。

可是汪直却觉得这个建议十分好,立刻准备付诸实施。

方针已经确定,那么拿谁开刀呢?

汪直冥思苦想,终于找到了一个当时谁也不敢惹的人物,他决定首开先例,用来树立自己的威信。

这位即将倒霉的仁兄叫覃力鹏,也是个太监,他虽然不在京城,却是除汪直外,地位仅次于司礼太监怀恩和东厂太
监尚铭的第三号人物,时任南京镇守太监。

明代虽然迁都北京,但南京依然是明朝都城,南京镇守太监向来就是一个十分重要的职位,而且覃力鹏背景深厚,
和许多皇亲国戚都有私人关系,虽然经常违法,却从来没有人敢找他的麻烦。

可是这次汪直决定麻烦一下他,虽然同是太监,但为了西厂的品牌,只好牺牲老兄你了。

他打定主意,马上动起手来,收集了很多覃力鹏的罪证(那是相当容易),东扯西拉的,竟然搞出一个罪当斩首的结
论。

覃力鹏万没想到,汪直竟敢拿他开刀,可这位仁兄也实在不是好欺负的,他连夜派人入京,做了一番工作,结果大
事化小,被批评了两句也就算了。

汪直没有打垮覃力鹏,却也得到了朱见深的表扬,被授予敢于办事,公正无私的称号,受到领导称赞的汪直顿时精
神焕发,接连搞出了几件莫名其妙的事情。

首先是几个刑部官员,刚刚从外地出差回来,一进京城就被西厂的人逮捕,放进牢里猛打了一顿,也不说他们犯了
什么法,就又被释放出狱。搞得这几个人稀里糊涂,还以为是在做梦。

之后是一个外地的布政史进京办事,还没等找地方住下,也被西厂的人拉去打了一顿,吃了几天牢饭。

这当然都是汪直指使的,他的行为看似很难理解,其实只是想证明一点:

他能够在任何时间,以任何理由,解决任何人。

此时的汪直内有皇帝的宠信,外有西厂的爪牙,在很多人看来,他已经是一位不折不扣的成功太监。

可是汪直并不这样认为:

成功?我才刚上路哎。

\section[\thesection]{}

他没有满足于目前的业绩,谦虚地认为还需要不断地进步,为了更好地确定自己的权威,他决定寻找第二个重点打
击的目标。不久后,他找到了。

这次被盯上的人叫做杨晔。他本人虽然只是个小官,名气不大,却也不是等闲之辈,他的曾祖父就是大名鼎鼎的
``三杨''中的杨荣。由于在家惹了麻烦,他和他的父亲杨泰一同来到京城暂住。

对汪直来说,这是一个绝好的机会,这一次,他准备彻底解决问题。

当然,他不会想到,这件事情中最终也解决了他自己。

汪直派人逮捕了杨晔和他的父亲杨泰,关进了大牢。

在牢里,汪直耍起了流氓。他下达命令,给杨晔表演了东厂乐队的拿手节目``弹琵琶''。

所谓``弹琵琶'',并不是演奏音乐,而是一种独特的行为艺术。具体说来,是用利刃去剃人的肋骨,据说行刑之时
痛苦万分,足可以让你后悔生出来。这一招当年开国时老朱也没想出来,是东厂的独立发明创造。

可怜 杨晔先生,足足被弹了三次,体力不支,竟然就死在了监狱里。

汪直却并不肯善罢甘休,一定要把事情做绝,他接着安插罪名,判处杨晔的父亲杨泰死刑,斩首。

此时的西厂也已经嚣张到了顶点,比如杨晔的叔父杨仕伟,时任兵部主事(正处级),西厂没有办理任何法律手续,
逮捕证也没一张,就跑到他家里去抓人,半夜三更,搞得鸡飞狗跳,住在旁边的翰林侍讲陈音听见动静,十分恼
火,拿出官老爷的派头,隔着墙大喝一声:

``你们这样胡作非为,不怕王法吗?!''

可对面的西厂特务倒颇有点幽默感,也隔墙答了一句:

``你又是什么人,不怕西厂吗?!''

事情闹大了,汪直却满不在乎,毕竟杨晔本人也不是什么了不得的人物,可后来事情的发展彻底打破了他的幻想。

他没有想到,虽然杨荣已经死去多年,但威信很高,是文官集团的楷模,他的子孙出了事,大臣们怎肯甘休!

\section[\thesection]{}

第一个作出反应的是内阁首辅商辂,他派人查明了杨晔的冤情,召集内阁开会,痛斥汪直的罪行,并写了一封奏折
给朱见深,要求废除西厂,罢免汪直,其中有一句非常厉害的话:

``不驱逐汪直,天下迟早大乱!''

朱见深发怒了,他虽然脾气温和,看到这句话也气得不行,大叫道:

``用一个太监,也会天下大乱吗?!''

他十分激动,立刻叫来身边的人,传达了他的口谕:

``让商辂明白回话,到底是谁指使他的!主谋是谁!''

朱见深很少发火,但发起火来绝不善罢甘休,按照常理,商辂要吃大苦头了。

但他这次的运气实在不错,因为奉命传旨的人,是司礼太监怀恩。

怀恩,山东人,本姓戴,宣德年间,因父亲涉罪抄家,他被逼入宫成为宦官,改名怀恩,历经三朝,最终成为了手
握重权的司礼太监。

这是一个十分关键的人物,正是他多次挽救了时局,并在最后时刻力挽狂澜,将朱祐樘送上了皇位。

怀恩奉旨出发了,他刚刚领教了朱见深的怒火,却没有想到,在内阁等待着他的,是另一个更为愤怒的人。

怀恩来到内阁,刚好商辂、刘吉、万安等人都在,他便二话不说,传达了朱见深的口谕:

``奏折是谁写的,何人指使?!''

这是两句十分严厉的问话,说明皇帝生气了,后果很严重,可商辂却出乎所有人的意料,他不但没有丝毫畏惧,反
而拍案而起,大声说道:

``奏折是我写的,也是我主使的,那又如何!你就这样回复皇上好了!''

``汪直不过是个太监,竟然敢私自关押处死朝廷官员,擅自调动边关将领和内宫人员,让他这样放肆下去,天下必
定大乱!不除汪直,王法何在!''

商辂这一激动,内阁的全体成员也跟着激动起来,你一言我一语大有闹事的苗头。

关键时刻,怀恩保持了镇定,他安抚了商辂等人,即刻紧急回复朱见深,转述了商辂的回复,希望朱见深认真考虑。

听完了怀恩的汇报,朱见深感到了一丝恐惧,他意识到,商辂是对的,汪直已经成为了一个有威胁的人,必须采取
行动了。

不久之后,朱见深下谕,罢免了西厂,将汪直逐回御马监。

对于内阁来说,这是一次了不起的胜利,商辂等人弹冠相庆,高兴万分。

但与此同时,御马监太监汪直却并不沮丧,因为他十分清楚,软弱朱见深不会坚持多久,他仍然需要自己,不久之
后,他就能回到原来的位置。

\section[\thesection]{}

汪直的疏忽

汪直是对的。

对于朱见深而言,正确还是错误、忠臣或是奸臣,都并不是那么重要,童年时候的经历给朱见深打下了深刻的烙印
--- 过得舒舒服服就好。

所以他需要的并不是在背上刻字的武将,也不是在朝廷上骂人的文官,他只喜欢一种人 --- 听话的人。

汪直是一个听话的人,不但老老实实地伺候朱见深,还能够提供各种娱乐服务,这样的人上级自然不会让他闲太久。

于是不久之后,西厂重新开张,汪直也成为了新任厂长。

汪直又一次达到了他太监生涯的顶峰。

然而不久之后,他就犯了一个错误,一个他的先辈曾经犯过的错误。

和王振一样,汪直也有着一个横刀立马的梦想。

既然是个太监,就应该踏踏实实地干好这份有前途的工作,可汪直先生偏偏要出风头,但问题是当时边界比较平
静,为了达到自己的目的,汪直贯彻了新的边防方针:人不犯我,我也犯人。

事实证明,汪直确实是一个不折不扣的孬种,他所谓的进攻不过是杀掉人家进贡使者,或是趁人家大人不在家的时
候去骚扰一下老少妇孺。等人家来报复了,他又成了和平主义者,一溜烟地就逃了,可经过他这么三下两下胡搞,
鞑靼和辽东各部落真的被惹火了,不断地到明朝边界找麻烦。

朱见深纳闷了,原本平安无事的边境突然四处传来战报,他没有相信汪直的鬼话,而是自己派人出去打听,这才发
现原来所有的事情都是汪直惹出来的,这下他火大了。

朱见深同志要求不高,只想老婆孩子热炕头,过两天安逸日子,没事研究一下金丹春药之类的化学制造,可是汪直
偏偏不让他消停,他开始对汪直不满了。

这种情绪很快被两个人察觉到了,他们决定利用这个机会把汪直彻底打垮。这两个人一个是李孜省,另一个人是尚
铭。

他们两个人决定抛弃以往的成见,精诚合作,尚铭寻找汪直的罪证,而李孜省则串通万安上书告状,双方各司其
职,准备着最后的攻击。

成化十七年(1481),机会来了。

\section[\thesection]{}

这一年,鞑靼部落开始进攻边境,朱见深接到消息十分不满,立刻找汪直进见,直截了当地对他说:

``你自己惹出的麻烦,自己去解决!''

汪直大气也不敢喘就连夜去了宣府,可当他到达那里的时候,人家已经抢完东西走了。汪直便急忙向皇帝打报告,
说这边已经完事了,我准备回去。

朱见深同志回复:

那里非常需要你,多呆几天吧。

尚铭和李孜省敏锐地感觉到,汪直快要完了,他们立刻按照计划发动了最后攻势。一时之间,弹劾满天飞,原本优
秀太监,先进模范突然变成了卑鄙小人,后进典型。朱见深立刻下令,关闭西厂,将汪直贬为南京御马监。

出来时还风光无限的汪直灰溜溜地去了南京,沿途风餐露宿,以往笑脸相迎的地方官们此时早已不见了踪影,汪直
已经没有别的野心,只希望能够安心到南京做个太监。

可是我国向来都有痛打落水狗的习惯,尚铭还嫌他不够惨,又告了一状,这下子汪直的南京御马监也做不成了,只
能当一个小小的奉御,他又操起了当年刚进宫时候打扫卫生的工具,在上级太监的欺压下,干起了杂务。

成化初年进京成为奉御,成化十九年又被免为奉御,十余年从默默无闻到权倾天下再到打回原型,一切如同梦幻一
般。

明史没有记载汪直这位风云人物的死亡年份,这充分说明,此人已经不值一提。

汪直的离去,最为高兴的自然是尚铭了,东西监派终于可以统一了。可他没有想到,下一个倒霉的人就轮到自己了。

要说仙派掌门李孜省也实在不够朋友,当年弹劾汪直的时候,他就给尚铭准备了另外一份备用本,没等过河,他已
经准备拆桥了。

很快言官们就把矛头对准了尚铭,纷纷上书弹劾他的罪行,于是尚铭掌门终于也被盟主大人废了武功 --- 去明孝陵
扫地。

仙派和后派打倒了显赫一时的监派,成为了武林的主宰,当然了,这两派也不是啥好东西,江湖还是那个江湖,但
就在一片黑暗之中,光明的种子开始萌芽。

说来可笑,亲自播下这种子的居然是李孜省,因为正是拜他所赐,尚铭和汪直才被赶走,从而使得另一个人登上了
掌门之位,这个人就是司礼监怀恩。

怀恩敏锐地抓住了时机,安排自己的亲信陈准登上了东厂厂公的位置,全面掌握了监派的大权,小心地保护着光明
的火种,等待着时机的到来。

\section[\thesection]{}

坚持到底

我一直认为,好人和坏人是不能用职业以及读书多少来概括的,饱读诗书的大臣有很多坏人,而以文盲居多的太监
里也有很多好人,郑和自不必说,而成化年间的怀恩也是其中的优秀代表。

他本来出生于官宦之家,衣食无忧,却飞来横祸,父亲罢官,家被抄,他自己被送进宫内,强行安排做了宦官,最
缺德的是,皇帝陛下竟然还要他感激涕零,赐了个叫``怀恩''的名字。

在这样的境遇下成长起来的怀恩,如果尽干坏事,那实在是不稀奇的,可怪就怪在,这位仁兄却是个不折不扣的好
人。

在鬼哭狼嚎、妖风阵阵的成化年间,他和商辂努力支撑着大局。但怀恩要比商辂聪明得多,他早就看出了这黑暗时
局的真正始作俑者不是梁芳,不是李孜省,甚至也不是万贵妃,而是软弱的朱见深。

因为这乱七八糟的五派都是为皇帝服务的,春派给他提供化学药品,仙派为他求神拜佛,监派为他打探消息,后派
照顾他的生活,混派拍他的马屁。只要朱见深还活着,这出丑剧将一直演下去。

所以当商辂心灰意冷,退休回家时,怀恩依然坚持了下来,因为此时的他已经找到了破解这片黑慕的唯一方法 ---
朱祐樘。

他曾与后宫的人们一起保守过那个秘密,也经常去看望这个可怜的孩子,在张敏说出实情的时候,他主动站了出
来,为此作证,他见证了朱祐樘的成长,并且坚信这个饱经苦难的少年一定能够成为他心目中的明君英主。

他最终没有失望。

但此时,上天似乎认为朱祐樘受的磨难还不够,于是,它为这个孩子安排了最后一次,也是最为致命的一次考验。

事情是由一次谈话开始的:

成化二十一年(1585) 三月

朱见深又一次来到后宫的内藏库查看他的私房钱。由于忙于炼丹等重要工作,他已经很久没有来过了,可当他打开
库门时,眼前的景象让他大吃一惊。

他立刻下令:

``把梁芳叫来!''

梁芳来了,朱见深没有说话,只是让他自己往库门里看。

里面空空如也。

十余年之前,这里还曾堆满金银财宝,一个质朴的小姑娘在这里默默地工作。如今已经是人去楼空。

朱见深指着库房,冷冷地说道:这些都是你花的吧。

\section[\thesection]{}

按说盟主发怒了,梁掌门就应该低头认罪了,可这位仁兄竟然回了一句:

``这些钱我可是拿去修宫殿祠堂,给皇上祁福了。''

花了钱还不认账,把皇帝当冤大头!

这下盟主大人火大了,气得满脸通红,可他憋了半天,却冒出了一句匪夷所思的话:

``我不管你,将来自然有人跟你算账!''

这句话大概类似现在小学生打架时候的常用语:你等着,我回家叫人来打你!

盟主混到这个份上,也真算是窝囊到了极点。

朱见深愤愤不平地走了,可是在梁芳的耳中,这句话的意思发生了变化:

``我管不了你,将来我的儿子会来对付你!''

好吧,既然这样,就先解决你的儿子。

梁芳明白,要想达到这个目的,必须得到一个人的帮助,于是他跑到后宫,找到了万贵妃。

自从十年前的那次失败之后,万贵妃已沉默了很久,但她对朱祐樘的仇恨却一点也没有消散,梁芳的建议又一次点
燃了她复仇的火焰。更重要的是,她杀死了朱祐樘的母亲,一旦朱祐樘登基,她是不会有好下场的。

不能再等了,趁这个机会彻底打倒他吧,否则将来我们必定死无葬身之地!

这一年,她五十五岁,他三十八岁,朱祐樘十五岁。

虽然已经年过半百,万贵妃的枕头风依然风力强劲,在她的反复鼓吹下,朱见深终于下定了决心。

在做出决定的前夕,朱见深作出了一个关键的决定,他找到了怀恩,想找他商量一下执行问题。

``我想废掉太子,你看怎么做才好。''

跪在地上的怀恩听见了这句话,却没有说话,只是脱下了自己的帽子,向朱见深叩首。

朱见深等了很久,也没有回音。

``为什么不说话?''

``请陛下杀了我吧。''一个低沉的声音这样回复。

``为什么?''朱见深惊讶了。

``因为陛下的这道谕令,我不会遵从。''

``你不要命了吗?''朱见深愤怒了。

怀恩抬起头,大声说道:

``今日我若不为,陛下杀我,但我若为之,将来天下人皆要杀我!''

``是以虽死,亦不为。''

\section[\thesection]{}

朱见深惊呆了,这个平日恭恭敬敬的老太监竟然来了这么一手,他以更为凶狠的眼神盯着怀恩,却发现毫无效果。
怀恩那平静的眼神没有丝毫的慌乱。

朱见深突然发现,虽然他是皇帝,主宰着千万人的生死,却战胜不了眼前的这个人。

一个人要是不怕死,也就没有什么可怕的了。

他万般无奈之下,只好对怀恩说:

``这里不用你了,回中都守灵吧!''

所谓中都,就是老朱的老家凤阳,当时已经比较荒凉了。

怀恩丝毫不动声色,也没有求饶,只是磕了个头,谢恩之后飘然而去,只留下了无计可施的朱见深。

但是怀恩的执著并没有能够打动朱见深,在万贵妃的不断鼓吹下,他仍然决定废掉太子。

事情到了这个地步,也真算是无计可施了,朱祐樘先生唯一能做的也只能是对天大呼一句:

``天要亡我!''

没准他还真的喊过,因为不久之后,老天爷也看不下去了,近来掺和了一把。

成化二十一年(1485) 四月

泰山地震

古代虽然没有地震局普及科学知识,但地震也算是司空见惯的常事了,没有啥希奇的,可这次地震实在不一般。

要知道,这次地震的可是泰山,那是古代帝王封禅的地方,秦皇汉武才够资格上去,光武帝同志斗胆上去了一次,
还被人骂了几句。朱元璋一穷二白打天下,天不怕地不怕的人,也没敢去干这项工作。用现在的话来说,这座山有
着重要的政治意义。

朱见深有点慌,他立刻派人去算卦,看看到底哪里出了问题,结果那位算卦的鼓捣了半天,得出了一个结论:

``应在东宫。''

这意思就是,泰山之所以地震,是因为东宫不稳,老天爷发怒了。

朱见深一听这话,马上停止了他的行动,他还打算长生不老呢,老婆可以得罪,老天爷不能得罪。

就这样,朱祐樘在上天的帮助下,迈过了最后一道难关。

但此时的朝政之黑暗,已经伸手不见五指。朱见深虽不废太子,也不怎么管理朝政了,梁芳肆无忌惮地贪污受贿,
李孜省肆无忌惮地安插亲信,混乱朝纲,万安则是肆无忌惮地混日子。

五大派失去了所有的管制,开始了任意妄为的疯狂,但这一切不过是黎明前的最后黑暗,因为光明即将到来。

\section[\thesection]{}

成化二十三年(1487)春,朱见深终于遭受了他一生中的最大打击,万贵妃在后宫去世了。

这个陪伴了他三十八年的女人终于离开了,无论风吹雨打,她始终守护在这个人的旁边,看着他从两岁的孩童成长
为四十岁的中年人,从未间断,也从未背叛。

``我会一直在你身边陪伴着你。''

整整三十八年,她履行了自己的诺言,。

她并不是什么十恶不赦的坏人,只是嫉妒的火焰彻底地毁灭了她的理智,对她而言,朱见深已成为她生命中不可或
缺的一部分,她不能容忍任何人把他抢走。

卑劣、残忍、恶毒不是她的本性,却是她必须付出的代价 --- 为了她的爱情。

朱见深彻底崩溃了,几十年过去了,春药、仙丹早已毁坏了他的身体,万贵妃的死却更为致命地摧毁了他的精神,
他登上了皇位,成为了统治帝国的皇帝,但他的心灵仍然属于三十多年前的那个孤独无助的孩子,需要她的照顾。

谢幕的时候终于到了,你虽然先走一步,但你不会寂寞太久的,很快我就会来陪伴你。几十年后宫的你争我夺,其
实你并不明白,即使你没有孩子,也没有任何人可以取代你在我心中的地位。皇位和权势对我而言并不重要,我也
不感兴趣,我所要的只是你的陪伴,仅此而已。

结束吧,让一切都回到事情的起点。在那个时候,那个地方,只有你和我。

成化二十三年(1487)八月,朱见深病倒,十日后,不治而亡,年四十一。

朱见深是一个奇特的皇帝,在他统治下的帝国妖邪横行,昏暗无比,但他本人却并不残忍,也不昏庸,恰恰相反,
他性格温和,能够明白事理,辨别忠奸,出现如此怪状,只因为他有着一个致命的缺点:软弱。

他不处罚贪污他钱财的小人,也不责骂痛斥他的大臣,因为他畏惧权力,畏惧惩罚,畏惧所有的一切,归根结底,
他只是一个想安安静静过日子的人。

他应该做一个老老实实的农夫,或者是本分的小生意人,被迫选择皇帝这个职业,对他来说,实在是一个不折不扣
的悲剧。

朱见深不是一个好皇帝,也不是一个好人,他是一个懦弱的人,仅此而已。

\end{document}

