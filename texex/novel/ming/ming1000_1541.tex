\documentclass[11pt,a4paper,onecolumn]{article}
\usepackage{fontspec,xunicode,xltxtra}
% \setmainfont[Mapping=tex-text]{Times New Roman}
\setmainfont[Mapping=tex-text]{Arial}
\setsansfont[Mapping=tex-text]{Arial}
% \setmonofont[Mapping=tex-text]{Courier New}
\setmonofont[Mapping=tex-text]{Times New Roman}

\usepackage{xeCJK}
% \setCJKmainfont[ItalicFont={Adobe Kaiti Std}]{Adobe Song Std}
% \setCJKmainfont[ItalicFont={Adobe Kaiti Std}]{Adobe Kaiti Std}
\setCJKmainfont[ItalicFont={Adobe Kaiti Std}]{Adobe Heiti Std}
\setCJKsansfont{Adobe Heiti Std}
% \setCJKsansfont{Microsoft YaHei}
\setCJKmonofont{Adobe Heiti Std}
\punctstyle{banjiao}

\usepackage{calc}
\usepackage[]{geometry}
% \geometry{paperwidth=221mm,paperheight=148.5mm}
% \geometry{paperwidth=9.309in,paperheight=6.982in}
\geometry{paperwidth=7.2cm,paperheight=10.8cm}
% \geometry{twocolumn}
\geometry{left=5mm,right=5mm}
\geometry{top=5mm,bottom=5mm,foot=5mm}
% \geometry{columnsep=10mm}
\setlength{\emergencystretch}{3em}


\usepackage{indentfirst}

%生成PDF的链接
\usepackage{hyperref}
\hypersetup{
    % bookmarks=true,         % show bookmarks bar?
    bookmarksopen=true,
    pdfpagemode=UseNone,    % options: UseNode, UseThumbs, UseOutlines, FullScreen
    pdfstartview=FitB,
    pdfborder=1,
    pdfhighlight=/P,
    pdfauthor={wuxch},
    unicode=true,           % non-Latin characters in Acrobat’s bookmarks
    colorlinks,             % false: boxed links; true: colored links
    linkcolor=blue,         % color of internal links
    citecolor=blue,        % color of links to bibliography
    filecolor=magenta,      % color of file links
    urlcolor=cyan           % color of external links
}
\makeindex

\usepackage[dvips,dvipsnames,svgnames]{xcolor}
\definecolor{light-gray}{gray}{0.95}

\usepackage{graphicx}
\usepackage{wrapfig}
\usepackage{picinpar}

\renewcommand\contentsname{目录}
\renewcommand\listfigurename{插图}
\renewcommand\listtablename{表格}
\renewcommand\indexname{索引}
\renewcommand\figurename{图}
\renewcommand\tablename{表}

\usepackage{caption}
\renewcommand{\captionfont}{\scriptsize \sffamily}
\setlength{\abovecaptionskip}{0pt}
\setlength{\belowcaptionskip}{0pt}

\graphicspath{{fig/}}

\usepackage{fancyhdr}

% \usepackage{lastpage}
% \cfoot{\thepage\ of \pageref{LastPage}}

% 嵌入的代码显示
% \usepackage{listings}
% \lstset{language=C++, breaklines, extendedchars=false}
% \lstset{basicstyle=\ttfamily,
%         frame=single,
%         keywordstyle=\color{blue},
%         commentstyle=\color{SeaGreen},
%         stringstyle=\ttfamily,
%         showstringspaces=false,
%         tabsize=4,
%         backgroundcolor=\color{light-gray}}

\usepackage[sf]{titlesec}
\titleformat{\section}{\normalsize\sffamily\bf\color{blue}}{\textsection~\thesection}{.1em}{}
\titleformat{\subsection}{\normalsize\sffamily}{\thesubsection}{.1em}{}
\titlespacing*{\section}{0pt}{1ex}{1ex}
\titlespacing*{\subsection}{0pt}{0.2ex}{0.2ex}

\usepackage{fancyhdr}
\usepackage{lastpage}
\fancyhf{}
\lhead{}
\rhead{}
\chead{\scriptsize{\textsf{蜗居}}}
\cfoot{\scriptsize{\textsf{第 \thepage ~页,共 \pageref*{LastPage} 页}}}


% \usepackage{enumitem}
% \setitemize{label=$\bullet$,leftmargin=3em,noitemsep,topsep=0pt,parsep=0pt}
% \setenumerate{leftmargin=3em,noitemsep,topsep=0pt,parsep=0pt}

% \setlength{\parskip}{1.5ex plus 0.5ex minus 0.2ex}
\setlength{\parskip}{2.0ex plus 0.5ex minus 0.2ex}

% \setlength{\parindent}{5ex}
\setlength{\parindent}{0ex}

% \usepackage{setspace}
\linespread{1.25}

% 英文的破折号--不明显,使用自己画的线。
\newcommand{\myrule}{\hspace{0.5em}\rule[3pt]{1.6em}{0.3mm}\hspace{0.5em}}

\begin{document}
\setcounter{section}{999}
\pagestyle{fancy}


\section[\thesection]{}

徐阶的预料一点没错,就在谕令颁布的当天,严世蕃的儿子,锦衣卫严绍庭便连夜出发赶去报信。但当他上气不接
下气到达江西时,看到的却只是一片狼藉。因为两天之前,林润已经到此一游,抓走了正在砌砖头的严世蕃。

这还不算,林御史送佛送上天,连小兄弟罗龙文也一起抓了,并上了第二份弹章,历数严世蕃的罪恶,连人带奏疏
一并送到了京城,

严世蕃再次成为了囚犯,再次来到了京城,这一次,所有的人都认定他将彻底完蛋,包括徐阶在内。

然而当这位严大少爷进入京城之后,让人意想不到的事情再度发生了。

严世蕃和罗龙文刚刚到京,便解掉了身上的镣铐,堂而皇之地接受朝中官员的宴请,吃好喝好后连监狱大门都没
去,就直接住进了早已为他们准备好的豪宅。

总而言之,这二位仁兄并非囚犯,反倒像是到前来视察的领导。

目睹这一奇观的徐阶再次被震惊了,两个朝廷钦犯在光天化日之下竟然如此嚣张,而朝廷百官却视若罔闻,无一例
外地保持了沉默。大理寺不管、刑部不管、都察院也不管。

难道只有我看到了这一切?!徐阶禁不住颤抖起来,他感到了一种前所未有的恐惧。

严嵩倒台了,严世蕃也二进宫了,事情已到了这个地步,严党竟然还有这么强大的力量,还能如此肆无忌惮、无法
无天!

从震惊中恢复过来的徐阶开始了新的思索,他终于确定一定以及肯定:这是一股极其强悍的势力,是一个无比坚固
的利益共同体,而要彻底毁灭它,单靠常规手段,是绝对办不到的。

要击破它,必须找到一个突破口,而严世蕃是最为合适的人选,既然弹劾没有用,逮捕没有用,甚至关进牢房也没
有用,那么我只剩下了一个选择----杀了他。

要让所有胁从者都知道谁才是朝廷的真正统领者,要用最严厉的手段告诉他们,依附严党,死路一条!

就在徐阶下定决心的时候,严世蕃正颇为轻松地与罗龙文饮酒作乐,但同为囚犯,罗龙文却没有严世蕃那样的心理
素质。虽说严党关系广势力大,不用蹲黑牢,也不用吃剩饭,但毕竟自己是来受审的,如果到时把干过的那些破事
都摊出来,不是死刑立即执行,至少也是个死缓。

然而严世蕃笑着对他说:

``我等定然无恙,不必担心。''

\section[\thesection]{}

罗龙文松了一口气,他以为严世蕃已经搞定了审案的法官。

严世蕃却告诉他,负责审理此案的三法司长官,刑部尚书黄光升以及都察院左都御史、大理寺卿全都不是严党,而
且素来与他有仇,隐忍不发只是时机未到,到时一定会把他往死里审。

还没等罗龙文消化完这个噩耗,严世蕃又接着说了一件让他匪夷所思的事情:

``我已派人四处散播消息,为杨继盛和沈链申冤,说他们之所以会死,全都拜我等所为。相信这件事很快就会传到
三法司那里。''

罗小弟就此陷入了极度的恐慌,他大声向严世蕃吼道:

``你疯了不成?这不是自寻死路吗?!''

``不要慌'',严世蕃依旧镇定自若,``这些罪名不但杀不掉我们,还能够救我们的命。''

他平静地看着一脸疑惑的罗龙文,自信地说道:

``杀我的罪名自然有,却不是三法司的那些书呆子能够想出来的,在这世上,能杀我者,唯两人而已。''

``一个是陆炳,他已经死了,另一个是杨博,我已打探过,他前不久刚刚犯事,现大权旁落,在皇帝面前已说不上
话,不足为惧。''

于是严世蕃自信地发出了最后的预言:

``任他燎原火,自有倒海水!''

我的计划万无一失,是绝不会落空的,陆炳死了,杨博废了,世间已无对手,举世之才唯我一人而已!谁能杀我?!

徐阶能。

在十多年前夏言被杀之时,他还只是个未经磨砺的副部级愣头青,无论是权谋水平还是政治水平都还差得太远。但
经过多年的血雨腥风,他已习惯并掌握了所有的规则和技巧。到了今天,他已具备了参加这场死亡竞赛的能力。

事实上,从严世蕃进京的那天起,他的一举一动就已在徐阶的严密监控之下,从花天酒地到散布消息,徐阶都了如
指掌,与三法司的官员们不同,经过短暂的思考,他就明白了严世蕃的企图,并了解了他的全盘计划。

这是嘉靖年间两个最高智慧者的对决,胜负在此一举。

这是最后的考验,十余年的折磨与修炼,历经山穷水尽,柳暗花明,终于走到了这一步,优势已尽在我手。在我的
面前,只剩下最后一个敌人。

杀了此人,天下将无人能胜我。

\section[\thesection]{}

徐阶的正义

正如严世蕃所料,三法司采纳了街头巷尾路边社的意见,将杀害杨继盛、沈链的罪名套在了严世蕃的头上,所谓冤
杀忠臣,天下公愤之类,写得慷慨激昂。

完稿之后,他们依例将罪状送交内阁首辅徐阶审阅。

徐阶似乎已经等待他们多时了,他接过稿件,仔细看完,然后微笑着夸奖道:

``这件事情你们做得很好,文辞犀利,罪名清楚。''

``不过我有个问题想请教各位'',徐阶突然收敛了笑容,用冷峻的口气说道:

``你们是想杀严世蕃呢,还是想要救他?

这是一个侮辱智商的问题,几位司法干部当即涨红了脸,大声叫道:

``那还用说,自然是要杀了他!''

看着激动的同志们,徐阶笑出了声:

``此奏疏一旦送上,严世蕃必定逍遥法外,诸位只能白忙一场了。''

这又是个什么说法?众人目瞪口呆,愣愣地看着徐阶,等待着他的解释。

``你们并不明白其中奥妙,虽说杨继盛之事天下已有公愤,却绝不可上奏皇帝,要知道,杨继盛虽是为严氏父子所
害,斩首的旨意却是皇上下达的。''

``当今皇上是英察之主,从不肯自认有错,你们如果把这条罪状放上去,岂不是要皇上好看?如此受人欺瞒,皇帝的
颜面何存?到时皇上发怒,严世蕃自然无罪开释。''

徐阶说得没有错,严世蕃的如意算盘正是如此,为了实现自己的企图,他先放出风声,说自己最害怕杨继盛事件,
然后诱使三法司的人将此罪状上达,因为嘉靖皇帝的性格他十分了解,这位仁兄过于自负,认定自己天下第一,没
人能骗得了他,也从不肯认错。

现在你要告诉他,兄弟你错了,人家借你的手杀掉了杨继盛,你还在上面签了字,你是个白痴冤大头,他自然要发
火,否定你的说法,于是严世蕃同志刚好可以借机脱身。

这招十分狠毒,即所谓拖皇帝下水,不是一般人能想得出,用得上的,比如后来的上海滩第一老流氓杜月笙,也曾
用过这手,当时正值解放前夕,蒋介石之子蒋经国奉命到上海整顿金融秩序,打击投机,干得热火朝天,结果搞到
了杜月笙的头上,不但毫不留情,还明确表示整的就是你。

\section[\thesection]{}

杜月笙也不争辩,乖乖受罚,暗中却指使他人检举孔祥熙儿子投机倒把,把事情直接闹到了蒋经国那里:如果你不
处理他,凭什么处理我?

于是轰轰烈烈的上海金融保卫战就此草草收场,蒋氏家族和孔氏家族都是一家人,有事好商量,杜流氓也得以解脱。

但严世蕃却没有杜月笙的运气,因为他的对手是徐阶,是一个足以与他匹敌的人。

书呆子们头晕眼花了,他们的脑袋还没回过神来,只是傻傻地问徐阶,既然如此,那就请您出个主意,定个罪名,
我们马上去办。

然而徐阶接下来的举动却让他们更为吃惊,这位深不可测的首辅大人只是微微一笑,从袖子里拿出了一份早已预备
好的奏疏:

``我已经写好了,你们送上去就是了。''

怎么着?难道您还能未卜先知?

怀着对徐大人的无限景仰和崇敬,三法司的官员们打开了那份奏疏,杀气扑面而来。

简单说来,严世蕃的罪名有以下几点,首先他和罗龙文是哥们,而罗龙文勾结倭寇,严世蕃也与倭寇挂上了钩,他
们聚集海匪,并企图里通外国,逃往日本。

其次,他勾结江洋大盗,训练私人武装,图谋不轨。

最后,他还占据土地修房子,根据现场勘查,这是一块有王气的土地,严世蕃狗胆包天,竟然在上面盖楼,实在是
罪大恶极(这条罪名当年胡惟庸也挨过)。

看完了这封奏疏,连三法司的书呆子们也已断定了严世蕃的结局----必死无疑,因为嘉靖最为反感的两个词语,正
是``犯上''与``通倭''。

法司的官员们揣上这份致人死命的奏疏,哆哆嗦嗦地走了,临走时,他们以无比敬畏的眼神向徐大人告别,而徐阶
依旧礼貌的回礼,面色平静,似乎之前的那一切从未发生过。

在近三百年的明代历史中,这是让我感触极深的一幕,每念及此,不禁有毛骨悚然之感。因为在这场平淡的言谈分
析中,虽然没有刀光剑影,却蕴含着一种更为可怕的智慧。

作为当时世间最为精明的两个人,严世蕃和徐阶都敏锐地抓住了这场斗争的最关键要素----嘉靖。事实上,严世蕃
死不死,放不放,并不取决于他有没有罪,有多大罪,别说内通日本人,就算他勾结外星人,只要嘉靖不开口,严
世蕃就死不了。

问题的关键就在这里.

\section[\thesection]{}

打了这么多年的交道,严世蕃简直比嘉靖还要了解嘉靖,他知道这位皇帝是死要面子的人,才想出了这一绝招,如
无例外,安全过关应该不成问题,可惜他偏偏碰上了徐阶。

只要分析一下前面的那段对话,你就能明白,徐阶的城府只能用两个字来形容----恐怖。他破解了严世蕃的计划,
还提前写好了奏疏,定好了罪名,而要做到这些,他必须了解以下三点,缺一不可:

首先,他十分清楚嘉靖的习性,知道他打死也不认错,所以他明白哪些能说,哪些不能说。

其次,他知道三法司的那帮蠢人的想法,也料到他们会定哪些罪名。

能够掌握皇帝和群臣的心理,已经极为不易,但我们可以肯定地是,对于这两点,严世蕃也了如指掌,因为他的诡
计正是建立在此之上。

但徐阶之所以能够成为最后的胜利者,是因为他还掌握了最重要的第三点----严世蕃的心理。

他不但知道皇帝是怎么想的,法官是怎么想的,还知道严世蕃的想法,甚至连他用的阴谋手段也一清二楚,自负天
下才智第一的严世蕃机巧狡猾、机关算尽,却始终在徐阶的手心里打转,最后被人卖了还在帮着数钞票。

但是,这绝不能怪严世蕃同志,套用一句电影台词:不是国军无能,只是共军太狡猾。

对人心的准确揣摩,对事情的精确预测,还有深不可测的心机谋划,这是极致的智慧,在我看来,它已经超越了人
类的极限。

在这场暗战中,严世蕃输了,却输得并不冤枉,因为他输给了一个比他更聪明的人,而真正可悲的人,是嘉靖。

这位天资聪慧,刚愎自用的皇帝,终于为他的自以为是付出了代价,一生都致力于耍心计,控制人心的他,最终却
沦为了两个大臣的斗争工具,他的脾气和个性被两位大臣信手拈来,想用就用,想耍就耍。

就这样,木偶的操控者最终变成了木偶,也算是报应吧。

还要特别提醒大家一句,具体问题要具体分析,徐阶和严世蕃之所以能把皇帝捏着玩,归根结底还是因为嘉靖同志
爱面子,要换了朱元璋,估计不但严世蕃活不成,连办案的那几个书呆子也跑不掉,大家携手并肩一起见阎王。所
以千万不要乱用此招,教条主义害死人啊。

\section[\thesection]{}

不出徐阶所料,奏疏送上去之后,嘉靖勃然大怒,当即下令复核之后,立斩严世蕃、罗龙文,真是比他儿子还听话。

和许多人想象中不同,明代的死刑制度是十分严格的,草菅人命,那是谣传,地方官是没有权利杀人的,死刑的复
核权归属于中央,确切地说,是皇帝。

每次处决名单送上来,皇帝大人都会亲自批阅,也不是全杀,看谁不顺眼,就在上面划个勾,这人就算没了,等到
秋决之时砍头了事,这才能死。要是这次没轮上,那还得委屈您在牢里再蹲一年,明年还有机会。

而按照严世蕃的情况,最多也就是个秋决,可是在徐阶同志的大力帮助下,嘉靖极为少有地做了特别关照----立斩。

死到临头的严世蕃,却依然被蒙在鼓里,他毫不知情,还在自鸣得意地对着罗龙文吹牛:

``外面有很多人想杀我,为杨继盛报仇,你知道不?''

罗龙文已经不起折腾了,他毕竟心里没底,看着眼前的这个二百五,气不打一处来,又不好翻脸,只好保持沉默。

似乎是觉得玩笑开过头了,严世蕃这才恢复常态,拍着罗小弟的肩膀,给他打了保票:

``你就放心喝酒吧,不出十天,我们就能回家了,说不定我父亲还能复起(别有恩命未可知),到时再收拾徐阶、林
润,报此一箭之仇!''

罗龙文这才高兴起来,但说到具体问题,严世蕃却又只字不吐,看来他十分喜欢这种逗人玩的游戏。

严世蕃同志,既然喜欢玩,那就接着玩吧,趁你还玩得动。

很快,满怀希望的严世蕃等到了他企盼已久的结果----大批锦衣卫和立斩的好消息。

正是所谓希望越大,失望越大,好吃好住好玩的严世蕃突闻噩耗,当即晕倒在地,经泼凉水抢救成功后,虽然神智
恢复了清醒,却留下了后遗症----不停打哆嗦。一直哆嗦到严老爹派人来看他,让他写遗书,他都写不出一个字。

罗龙文自不必说,相信老大哥这么久,最终还是被忽悠了,怎一个惨字了得,整日抱头痛哭,早知如此,还不如当
年死在抗倭战场上,好歹还能追认个名份。

嘉靖四十四年(1565)三月辛酉,严世蕃和罗龙文被验明正身,押赴刑场,执行斩决。

这位才学出众,聪慧过人,却又无恶不作,残忍狠毒的天才就此结束了他罪恶的一生。

恶贯至此,终于满盈。

\section[\thesection]{}

在严世蕃被处决的那一天,京城民众们奔走相告,纷纷前往刑场观刑,并随身携带酒水、饮料、副食品等,欢声笑
语,边吃边看,胜似郊游。

人缘坏到这个份上,倒也真是难得了。

也就在这一天,一位在京城就读的太学生不顾一切地挤进人群之中,占据了最佳的观刑地点,他的手中还高举着一
块布帛,上面只有七个醒目的大字----锦衣卫经历沈链。

在亲眼目睹严世蕃的头颅被砍下之后,他痛哭失声,对天大呼:

``沈公,你终于可以瞑目了!''

言罢,他一路嚎哭而去,十几年前,当沈链因为弹劾严嵩被贬到保安时,曾不计报酬,免费教当地的贫困学生读书
写字,直到他被严嵩父子害死为止,而这个人,正是当年那些穷苦孩子中的一员。

为了这一天的到来,他已经等待了太久,而他终究看到了公道。徐阶终于实现了他的正义,用他自己的方式,不是
礼仪廉耻,不是道德说教,而是阴谋诡计,权术厚黑。

严世蕃死得冤不冤?冤,实在是冤。

罗龙文勾结倭寇,不过是想混口饭吃,他又不是汪直,凭他那点出息,就算要找内通的汉奸,也找不到他的头上。

严世蕃就更别说了,这位仁兄贪了那么多年,家里有的是钱,当年的日本从上到下,那是一穷二白(不穷谁出来当倭
寇),严财主在家盖别墅吃香喝辣不亦乐乎,干汉奸?别逗了,当天皇老子都不干。

至于占据有王气的土地,那就真没个准了,当年没有土地法,凭严世蕃的身份,随意占块地是小意思,但你硬要说
这块地有王气,那谁也没辙。关于这个问题,当时徐阶曾信誓旦旦地表示,他曾派人实地勘察,确系王气无疑。

可这事儿哪有个谱,又没有权威认证机构,但徐大人当政,他说有那就算有吧。

唯一确有实据的,是纠集亡命之徒,收买江洋大盗,但严世蕃同志本就不读书,是个彻头彻尾的混混,平时打交道
的也多是流氓地痞,发展个把黑社会组织,那是他的本分,况且他似乎也还没干出什么惊天大案,图谋犯上更不靠
谱。

所以结论是:严世蕃是被冤杀的。

那又如何?

杨继盛、沈链、还有那些被严党所害的人,哪一个不冤枉?还是那句老话:对付流氓,要用流氓的方法。

\section[\thesection]{}

关于这个问题,我将再次引用无厘头的周星驰先生在他的《九品芝麻官》里,说出的那句比无数所谓正直人士、道
学先生更有水平的台词:

``贪官奸,清官要比贪官更奸!''

我想,这正是最为合适的注解。

事情的发展证实,徐阶对严党的判断完全正确,严世蕃一死,严党立刻作鸟兽散,纷纷改换门庭,希望能躲过一劫。
但徐阶并不是一个慈悲为怀的人,在短短一个多月时间里,他就连续罢免调离了二十多名严党成员,可谓是雷厉风
行,把持朝政十余年的第一奸党就此被连根拔起。

但这件事尚未结束,还剩下最后一位老朋友,需要我们去料理。

严嵩的家终于被抄了,事实证明,他这么多年来,虽说国家大事没怎么管,捞钱却是不遗余力,据统计,从他家抄
出了黄金三万余两,白银三百万两,名人书画奇珍异宝不计其数,光抄家就抄了一个多月,连抄家的财物清单都被
整理成书,后来还公开刊印出版,取名《天水冰山录》,成为了清代的畅销书。

严嵩至此才彻底绝望,儿子死了,爪牙散了,嘉靖也不管了,他终于走到了人生的末路,而面对着忙碌的抄家工作
人员,这位仁兄在沮丧之余,竟然又提出了一个要求。

严嵩表示,因为家里的仆人多,所以希望能够留点钱给自己,作遣散费发放。

看着这个一脸可怜的老头,抄家官员于心不忍,便把这个要求上报给了徐阶,建议满足他的要求。

徐阶想了一下,便一字一句地说出了他的回答:

``我记得,杨继盛的家里没有仆人。''

现在是祈求慈悲的时候了吧,那么夏言被杀之时,慈悲在哪里?杨继盛、沈链被杀之时,慈悲在哪里?不出一兵一
卒,任由蒙古骑兵在城外烧杀抢掠,无恶不作之时,慈悲又去了什么地方?!

严嵩就此净身出户,孤身一人回到了老家,这里曾是他成功的起点,现在又成为了失败的终点。所谓兴衰荣辱,不
过一念之间。

胜利再次到来,而这一次,是如假包换、童叟无欺的胜利,没有续集。

十几年的潜心修炼,十几年的忍耐,在愤怒与仇恨,诡计与公道中挣扎求生的徐阶赢了,从奸党满朝到一网打尽,
他凭借自己的毅力和智慧,逐渐扳回了劣势,并将其引向了这个最后的结局,一切的一切都如同预先排演一般,逐
一兑现。

除了一个例外。

\section[\thesection]{}

在此前的十几年中,徐阶曾无数次在心中彩排:反击成功后,应该如何把严嵩千刀万剐,但当这一天真的到来时,
他却改变了之前的打算。

看着黯然离去的严嵩,徐阶的心中萌发了新的想法,不要杀他,也不能杀他。

自嘉靖初年得罪张璁被贬时起,三十多年来,徐阶从一个刚正不屈、直言上谏的愤青,变成了圆滑出世,工于心计
的政治家,但在他的个性特点中,有一点却从未变过----有仇必报。

十几年来,他对严嵩的仇恨已经深入骨髓,现在是报仇的时候了,面对这个罪行累累的敌人,他决心用另一种方式
讨还血债,一种更为残忍的方式。

罢官抄家,妻死子亡,但这还不够,还远远不足以补偿那些被你陷害、残杀,以致家破人亡的无辜者。

我不会杀你,虽然这很容易,我要你眼睁睁地看着身边的亲人一个个地死去,就如同当年杨继盛的妻子那样,我要
你亲眼看着你曾经得到的所有一切,在你眼前不断地消失,而你却无能为力。

继续活下去,活着受苦,严嵩,这是你应得的。

嘉靖四十四年(1565)四月,严嵩被剥夺全部财产,赶回老家,没有人理会他,于是这位原先的朝廷首辅转行当了乞
丐,靠沿街乞讨维持生计,受尽白眼,两年后于荒野中悲惨死去,年八十八。

正义终于得以伸张,以徐阶的方式。

奇人再现

严嵩倒了,徐阶接替了他的位置,成为了朝廷首辅,朝政的管理者,此时的内阁除他之外,只剩下了一个人----袁
炜。而这位袁炜,偏偏还是徐阶的学生。

于是徐阶的时代来到了,继严嵩之后,他成为了帝国的实际管理者。

其实后世很多人会质疑这样一个问题,徐阶和严嵩有什么不同?严嵩贪污,徐阶也不干净,严嵩的儿子受贿,徐阶的
儿子占地,严嵩独揽大权,徐阶也是。

表面上是一样的,实际上是不同的。

如果用一句简单的话来说明,那就是:严嵩怠工,徐阶干活。

\section[\thesection]{}

如果考察一下明朝的历代首辅,就会发现这帮人大都不穷(说他们穷也没人信),要单靠死工资,估计早就饿死了,
所以多多少少都有点经济问题,什么火耗、冰敬、碳敬等等,千里做官只为钱,不必奇怪。但徐阶是干实事的,与
严嵩不同,他刚一上任,就在自己的办公室挂上了这样一块匾:以威福还主上,以政务还诸司,以用舍刑赏还公
论,而他确确实实做到了。

在严嵩的时代,大部分的官职分配,都只取决于一个原则----钱,由严世蕃坐镇,什么职位收多少钱,按位取酬,
诚信经营,恕不还价。徐阶废除了这一切,虽然他也任用自己的亲信,但总的来说,还是做到了人尽其用,正是在
他的努力下,李春芳、张居正、殷正茂等第一流的人才得以大展拳脚。

在严嵩的时代,除了个别胆大的,言官们已经不敢多提意见了,杨继盛固然是一个光荣的榜样,但他毕竟也是个死
人。于是大家一同保持沉默,徐阶改变了这一切,他对嘉靖说:作为一个圣明的君主,你应该听取臣下的意见,即
使他们有时不太礼貌,你也应该宽容,这样言路才能放宽,人们才敢于说真话。

嘉靖听从了他的劝告,于是唾沫再次开始横飞,连徐阶本人也未能幸免,但与此同时,贪污腐化得以揭发,弊政得
以纠正,帝国又一次恢复了生机与活力。

徐阶是有原则的,与严嵩不同,严大人为了个人利益,可以不顾天下人的死活,可以抛弃一切廉耻去迎合皇帝,这
种事情徐阶也做过,但那是为了斗争的需要,现在是让一切恢复正常的时候了。

嘉靖想修新宫殿,徐阶告诉他,现在国库没有钱给你修。

嘉靖想继续修道服丹,徐阶告诉他,那些丹药都是假的,道士也不可信,您还是歇着吧。

甚至连嘉靖的儿子(景王)死了,徐阶的第一个反应都不是哀悼,而是婉转地表示,我虽然悲痛,却更为惦记这位殿
下的那片封地,既然他已经挂掉了,那就麻烦您下令,把他的地还给老百姓。反正空着也是空着,多浪费。

对于这种过河拆桥的行为,嘉靖虽然不高兴,却也无可奈何,他看着眼前的徐阶,这个人曾为他修好了新宫殿,曾
亲自为他炼丹,曾无条件地服从于他,但现在他才发现,这个性格温和的小个子并不是绵羊,却是一只披着羊皮的
狼。

\section[\thesection]{}

嘉靖感到上了当,却没有办法,严嵩已经走了,所有的朝政都要靠这个人来管理,想退个货都不行,只好任他随意
折腾。

绝对的权力产生的不仅仅是绝对的腐败,还有绝对的欲望,也是永远无法满足的欲望,这才是一切祸患的起始,严
嵩所以屹立数十年不倒,贪污腐败,横行无忌,正是因为嘉靖有着无尽的欲望,而严嵩恰好是一个无条件的迎合者。

于是徐阶出现了,他虽然也曾迎合过,但那不过是伪装而已,他真正的身份,是制衡者。他隐忍奋斗的最终目标,
并不是严嵩,而是嘉靖

很多人并不清楚,在漫长的明代历史中,徐阶是一个极为重要的人物,重要到几乎超出了所有人的想象,他最伟大
的成就,并不是打倒了严嵩,而是他所代表的那股势力。

自朱元璋废除丞相后,随着时代的变迁,明朝逐渐形成了一个极为特别的权力体系,皇帝、太监和大臣,构成了一
个奇特的铁三角,皇帝有时候信任太监(比如明武宗),有时候信任大臣(比如明孝宗)。

而在政治学中,这个铁三角的三方有着另外一个称呼:君权、宦权和相权。这就是帝国的权力架构,他们互相制
衡,互相维持,在此三权之中,只要有两者联合起来,就能控制整个帝国。

在过去的两百年中,前两种组合都已出现,皇帝曾经联合太监,也曾联合大臣,而无论是哪一种联盟,第三方总是
孤立无助的。

只有一种情况,从来都没有出现过,事实上,也没有人曾期待过那种局面的出现,因为在那个君临天下的时代,它
似乎永远不可能实现。

但它的确成为了现实,而这个奇迹联盟的开创者,正是徐阶。

具有讽刺意味的是,最早打破三角平衡,为这一奇迹出现创造条件的人,竟然是嘉靖。作为明代历史上最为聪明的
皇帝,他有着前任难以比拟的天赋。

凭借着绝顶的智慧和权谋,他十六岁就解决了三朝老臣杨廷和,然后是张璁、郭勋、夏言,而在打击大臣的同时,
他还把矛头对准了太监,严厉打压,使投身这个光荣职业,立志建功立业的无数自宫青年,统统只能去当洗马桶,
倒垃圾。纵观整个嘉靖朝,四十余年,竟然没有出过一位名太监,可谓绝无仅有。

他不想和任何人联盟,也不信任任何人,他相信凭借自己,就能控制整个帝国,而他所需要的,只是几个木偶而已。

\section[\thesection]{}

一切都如此地顺利,帝国尽在掌握之中,直到他遇上了严嵩和徐阶。

经过二十几年的试探,严嵩摸透了他的脾气和个性,并在某种程度上成功地影响并利用了他。

而徐阶则更进一步,在打垮了严嵩之后,他成为了一个足以制衡嘉靖的人,嘉靖要修房子,他说不修就不修,嘉靖
儿子的地,他说分就分。这是一个不太起眼,却极为重要的转折点,它意味着一股强大势力的出现,强大到足以超
越至高无上的皇权。

这才是徐阶所代表的真正意义,绝非个人,而是相权,是整个文官集团的力量。

当年的朱元璋废除了丞相,因为他希望能够控制所有的权力,现在的嘉靖也是如此,他们都相信,不需要任何人的
帮助,仅凭自己的天赋与能力,就能打破权力的平衡,操控一切,而事实证明,他们都错了。

一个人的力量再强,也是无法对抗社会规律的,它就如同弹簧一般,受到的压力越大,反弹的力度就越大。

作为超级牛人,朱元璋把劳模精神进行到底,既干皇帝,又兼职丞相,终究还是把弹簧压到了生命的最后一刻。嘉
靖就没有那么幸运了,和老朱比起来,他还有相当差距,所以在他尚未成仙之前,就感受到了那股强大的反扑力。
他的欲望已被抑制,他的权力将被夺走。

所有敢于挑战规则的人,都将受到规则的惩罚,无人例外。

当三十多年前,嘉靖在柱子上刻下``徐阶小人,永不叙用''字样的时候,绝不会想到,这个所谓的``小人''将会变
成``大人''。他以及他所代表的势力将压倒世间的所有强权----包括皇帝本人在内。

伟大的转变已经来临,皇帝的时代即将结束,名臣的时代即将到来,他们将取代至高无上的帝王,成为帝国的真正
统治者。

但徐阶只是这一切的构筑者与开创者,那个将其变为现实,并创下不朽功业的人,还在静静地等待着。

总而言之,嘉靖的好日子是一去不复返了,无论他想干什么,徐阶总要插一脚,说两句,不听还不行,因为这位仁
兄不但老谋深算,而且门生故吏遍布朝中,威望极高一呼百应,要是惹火了他,没准就得当光杆司令。

\section[\thesection]{}

那就这样吧,反正也管不了,眼不见心不烦,专心修道炼丹,争取多活两年才是正经事。

徐阶就这样接管了帝国的几乎全部政务,他日夜操劳,努力工作,在他的卓越政治领导之下,国库收入开始增加,
懈怠已久的军备重新振作,江浙一带的工商业有了长足的发展,万历年间所谓资本主义萌芽,正是起源于此。

你成你的仙,我干我的活,大家互不干扰。历史证明,只要中国人自己不折腾自己,什么事都好办。在一片沉寂之
中,明朝又一次走上了正轨。

徐阶着实松了一口气,闹了那么多年,终于可以消停了。但老天爷还真是不甘寂寞,在严党垮台后不到一年,他又
送来了一位奇人,打破了这短暂的平静。

但请不要误会,这位所谓的奇人并不是像严世蕃那样身负奇才的人,而是一个奇怪的人,一个奇怪的小人物。

嘉靖四十五年(1566)二月,嘉靖皇帝收到了一份奏疏,自从徐阶开放言论自由后,他收到的奏疏比以前多了很多,
有喊冤的,有投诉的,有拍马屁的,有互相攻击的,只有一种题材无人涉及----骂他修道的。

要知道,嘉靖同志虽然老了,也不能再随心所欲了,但他也是有底线的:你们搞你们的,我搞我的,你们治国,我
炼丹修道,互不干扰。什么都行,别惹我就好,我这人要面子,谁要敢扒我的脸,我就要他的命!

大家都知道这是个老虎屁股,都不去摸,即使徐阶劝他,也要绕七八个弯才好开口,所以这一项目一直以来都是空
白。

但这封奏疏的出现,彻底地填补了这一空白,并使嘉靖同志的愤怒指数成功地达到了一个新的水平高度。

奇文共享,摘录如下:

``陛下您修道炼丹,不就是为了长生不老吗?但您听说过哪位古代圣贤说过这套东西?又有哪个道士没死?之前有个陶
仲文,您不是很信任他吗?他不是教您长生不老术吗?他不也死了吗?''

这是骂修道,还有:

``陛下您以为自己总是不会犯错吗?只是大臣们都阿谀奉承,刻意逢迎而已,不要以为没人说您错您就没错了,您犯
过的错误,那是数不胜数!''

\section[\thesection]{}

具体是哪些呢,接着来:

``您奢侈淫逸,大兴土木,滥用民力,二十多年不上朝,也不办事(说句公道话,他虽不上朝,还是办事的),导致
朝政懈怠,法纪松弛,民不聊生!''

这是公事,还有私生活:

``您听信谗言不见自己的儿子(即陶仲文所说的``二龙不可相见''理论),不顾父子的情分,您天天在西苑炼丹修
道,不回后宫,不理夫妻的情谊(真奇了怪了,关你屁事),这样做是不对的。''

此外,文中还有两句点睛之笔,可谓是千古名句,当与诸位重温:

其一,嘉者,家也,靖者,净也,嘉靖,家家净也。

其二,盖天下之人,不值陛下久矣。

这就不用翻译了,说粗一点就是:在您的英明领导之下,老百姓们都成为了穷光蛋,他们早就不鸟你了。

综观此文,要点明确,思路清晰,既有理论,又有生动的实例,且工作生活面面俱骂,其水平实在是超凡入圣,高
山仰止。

文章作者即伟大的海瑞同志,时任户部正处级主事。此文名《治安疏》,又称直言天下第一事疏,当然,也有个别
缺心眼的人称其为天下第一骂书。

一位著名学者曾经说过,骂人不难,骂好很难,而骂得能出书,且还是畅销书,那就是难上加难了。整个中国一百
多年来,能达到这个高度的只有两个人,一个是鲁迅,另一个是李敖。

而在我看来,如果把时间跨度增加四百年,那么海瑞先生必定能加入这个光荣行列。

嘉靖愤怒了,自打生出来他还没有这么愤怒过,自己当了四十多年皇帝,竭尽心智控制群臣,我容易吗我。平时又
没啥不良习性,就好修个道炼个丹,怎么就惹着你了?

再说工作问题,你光看我这二十多年白天不上朝光修道,那你又知不知道,每天晚上你睡觉的时候,老子还在西苑
加班批改奏章,不然你以为国家大事都是谁定的。

还有老子看不看儿子,过不过夫妻生活,你又不是我爹,和你甚相干?

所以在嘉靖看来,这不是一封奏疏,而是挑战书,是赤裸裸的挑衅,于是他把文书扔到了地上,大吼道:

``快派人去把他抓起来,别让这人给跑了!''

说话也不想想,您要抓的人,除非出了国,能跑到哪里去?

\section[\thesection]{}

眼看皇帝大人就要动手,关键时刻,一个厚道人出场了。

这个人叫黄锦,是嘉靖的侍从太监,为人十分机灵,只说了一句话,就扑灭了皇帝大人的熊熊怒火:

``我听说这个人的脑筋有点问题,此前已经买好了棺材,估计是不会跑的。''

黄锦的话一点也没错,海瑞先生早就洗好澡,换好衣服,端正地坐在自己的棺材旁边,就等着那一刀了。

他根本就没打算跑,如果要跑,那他就不是海瑞了。

青天在上

作为一位有着极高知名度的历史人物,海瑞先生有一个非同寻常的荣誉称号----明代第一清官。

但在我看来,另一个称呼更适合他----明代第一奇人。

在考试成绩决定一切的明朝,要想功成名就,青史留芳,一般说来都是要有点本钱的,如果不是特别聪明(张居
正),就是运气特别好(张璁),除此之外,别无他途。

而海瑞大概是唯一的例外,他既不聪明,连进士都没中,运气也不怎么好,每到一个工作岗位,总是被上级整得死
去活来,最终却升到了正部级,还成为了万人景仰的传奇人物。

正德九年(1514),海瑞出生在海南琼山的一个干部家庭,说来这位兄台的身世倒也不差,他的几个叔叔不是进士就
是举人,还算混得不错,可偏偏他爹海翰脑袋不开窍,到死也只了个秀才,而且死得还挺早。

父亲死的时候,海瑞只有四岁,家里再没有其他人,只能与母亲相依为命。

虽然史料上没有明确记载,但根据现有资料分析,海瑞的那几位叔叔伯伯实在不怎么厚道,明明家里有人当官,海
瑞却没沾过一点光,童年的生活十分困苦,以至于母亲每天都要做针线活贴补家用。

很明显,在海氏家族中,海瑞家大概是很没地位的,大家都看死这对母子闹不出什么名堂,实际情况似乎也差不
多,海瑞同学从小既不会作诗,也不会作文,没有一点神童的征兆,看情形,将来顶了天也就能混个秀才。

虽说境况不太乐观,但海瑞的母亲认准了一条死理:再穷不能穷教育,再苦不能苦孩子。不管家里多穷多苦,她都
保证海瑞吃好喝好,并日夜督促他用心学习。

这就是海瑞的童年生活,每天不是学堂,就是他娘,周围的小朋友们也不找他玩,当然海瑞同学也不在乎,他的唯
一志向就是好好学习,天天向上。

\section[\thesection]{}

很多史料都对海瑞的这段经历津津乐道,不是夸他刻苦用功,就是表扬他妈教子有方。而在我看来,这全是扯淡,
一挺好的孩子就是这样被毁掉的。

孤僻,没人和他玩,天天只读那些上千年前的老古董,加上脑袋也不太好使,于是在学业进步的同时,海瑞的性格
开始滑向一个危险的极端----偏激,从此以后,在他的世界里,不是对,就是错,不是黑,就是白,没有第三种选
择。

此外,小时候的艰苦生活还培养了他的顽强个性,以及无论何时何地都不轻易认输的精神,但同时也产生了一个副
作用:虽然在他此后的一生中曾经历过无数风波,遇到过许多人,他却始终信任,并只信任一个人----母亲。

在困苦的岁月里,是母亲陪伴他、抚养他,并教育他,所以之后虽然他娶过老婆,有过孩子,却都只是他生命中的
过客,说句寒心的话,他压根就不在乎。

孤僻而偏激的海瑞就这样成长起来,他努力读书,刻苦学习,希望有一天能金榜题名,至少能超越自己的父亲。

然而他的智商实在有限,水平就摆在那里,屡考屡不中,考到二十多岁,连个秀才都混不上,没办法,人和人不一
样

但海瑞先生是顽强的,反正闲着也是闲着,继续考!就这么一直磨下去,终于在二十八岁那年,他光荣地考入了县
学,成为了生员。

说来惭愧,和我们之前提到的杨廷和、徐阶相比,海瑞先生的业绩实在太差,人家在他这个年纪都进翰林院抄了几
年文件了。就目前看来,将来海瑞能混个县令就已经是奇迹了,说他能干部长,那真是鬼才信。

当然海瑞自己从没有任何幻想,对他而言,目前的最大理想是考中举人。

那就接着考吧,不出例外,依然是屡考不中,一直到他三十六岁,终于柳暗花明了,他光荣地考中举人。

下一步自然是再接再厉,去京城考进士,海瑞同学,奋斗!努力!

进京,考试,落榜,回家,再进京,再考试,再落榜,再回家。一眨眼六年过去了。

奋斗过了,努力过了,自己最清楚自己的实力,不考了,啥也不说了,去吏部报到吧。

之前我们曾经讲过,在明朝,举人也是可以做官的,不过要等,等现任官死得多了,空缺多了,机会就来了,但许
多举人宁可屡考不中,考到胡子一大把,也不愿意去吏部报到。有官做偏不去,绝不是吃饱了撑的,要知道,人家
是有苦衷的。

\section[\thesection]{}

首先这官要等,从几年到几十年,就看你运气如何,寿命长短,如果任职命令下来的时候,正赶上你的追悼会,那
也不能说你倒霉。

其次这官不好,但凡分给举人的官,大都是些清水衙门的闲差,小官,什么主簿、典史、教授(从九品,不是今天的
教授)之类的,最多也就是个八九品,要能混到个七品县令,那就是祖坟起了火,记得一定回去拜拜。

再次这官要挑,别以为官小就委屈了你,想要还不给你呢!你还得去吏部面试,大家排好队站成一排,让考官去挑,
文章才学都不考,也没时间考,这里讲究的是以貌取人,长得帅的晋级,一般的待定,歪瓜裂枣的直接淘汰。顺便
说一句,相貌考核有统一规范,国字脸最上等,宽脸第二,尖嘴猴腮者,赶回家种红薯

最后这官窝囊,在明代最重视出身,进士是合格品,庶吉士是精品,至于举人,自然不是次品,而是废品。

有一位明代官僚曾经总结过,但凡进士出身,立了功有人记,出了事有人保,从七品官做起,几十年下来,哪怕灾
荒水旱全碰上,也能混个从五品副厅级。

但要是举人,功劳总是别人领,黑锅总是自己背,就算你不惹事,上级都要时不时找你麻烦。从九品干起,年年丰
收安泰,能混到七品退休,就算你小子命好。

海瑞面对的就是这么一个局面,好在他运气还不错,只等了五年,就等来了一个职位--福建南平县的教谕。

所谓教谕,是教育系统的官员,通俗地说,就是福建南平县的教育局长,这么看起来,海瑞的这个官还不错。

如果这么想,那就错了,当年的教育系统可没什么油水,没有扩招,也没有择校费,更不用采购教材,四书五经就
那么几本,习题集、模拟题、密卷之类的可以拿去当手纸,什么重点大学,重点中学,重点小学,重点幼儿园,考
不中科举全他娘白费。

而县学教谕的上级,是府学的教授,前面说过,教授是从九品,教谕比教授还低,那该怎么定级别呢?这个不用你
急,朝廷早就想好了,这种职务有一个统一的称呼----不入流。

也就是说你还算是政府公务员,但级别上没你这一级,不要牢骚,不要埋怨,毕竟朝廷每月还是发工资给你的嘛,

\section[\thesection]{}

就这样,海瑞带着老母去了南平,当上了这个不入流的官,这年他四十一岁。

已经四十多岁了,官场的青春期已过,就算要造反也过了黄金年龄,海瑞却踌躇满志,蓄势待发,换句话说,那是
相当有战斗力,把这个不入流的官做得相当入流。

县学嘛,就是个读书的地方,只要你能考上举人,上多久课,上不上课其实都无所谓,所以一直以来,学生想来就
来,想走就走。但现在不同了,既然海瑞来了,大家就都别走了。

他规范了考勤制度,规定但凡不来,就要请假,有敢擅自缺课者,必定严惩,而且他说到做到,每天都第一个到,
最后一个走,一个都不能少。

这下学生们惨了,本来每天早退旷课都是家常便饭,现在突然被抓得死死的,这位局长大人脸上又总是一副你欠他钱
的表情,于是不久后,海瑞先生就得到了人生中的第一个绰号----海阎王。

难熬归难熬,但学生们很快也发现,这位海阎王倒有个好处----从不收礼金。

所谓礼金,就是学生家长送给老师的东西,不一定是钱,什么鸡鸭鱼肉海鲜特产,一应俱全。说实话,这玩意谁也
不想送,但如果不送,难保老师不会特意关照你的儿女:置之不理,罚搞清洁,罚坐后排等等,那都是手到擒来。

但海阎王不收,不但不收礼金,也不为难学生,他平等地对待每一个人,虽然他很严厉,却从不因个人好恶惩罚学
生。所以在恐惧之余,学生们也很尊敬他。

其实总体说来,这个职业是很适合海瑞的,就凭他那个脾气,哪个上级也受不了,干个小教谕,也没什么应酬,可
谓是得心应手。

但人在江湖漂,总要见领导,该来的还是要来。

一天,延平知府下南平县视察,按例要看看学堂,海瑞便带着助手和学生出外迎接,等人一到,两个助手立马下跪
行礼,知府同志却还是很不高兴,因为海瑞没跪。

不但不跪,他还正面直视上级,眼睛都不眨。

知府五品,海瑞没品,没品的和五品较劲,这个反差太大,心理实在接受不了,但在这么多人面前,发火又成何体
统,于是知府大人郁闷地走了,走前还咕嘟了一句:

``这是哪里来的笔架山!''

两个人跪在两边,中间的海瑞屹立不倒,确实很像个笔架,也真算是恰如其分。

\section[\thesection]{}

虽然他说话声音不大,但大家都听到了,由于这个比喻实在太过形象,所以自此以后,海瑞先生就有了第二外
号----海笔架,两个外号排名不分先后,可随意使用。

大家都慌了,海瑞却若无其事,他还有自己的理论依据:教育官员不下跪,那是圣贤规定的(哪个圣贤待查),我听
圣贤的话,有什么错?

知府大人不爽了,但让他更不爽的还在后面,不久之后,一位巡按御史前来拜访了,前面提过,所谓巡按御史,虽
说才六七品,却能量极大,能干涉巡抚总督的职权,何况是小小的知府。

知府诚惶诚恐,鞍前马后地服侍,御史大人摸着撑饱的肚皮,边打嗝边说:下去看看吧。

这一去,就去了南平,消息传下来,知县也紧张了,御史说到底是中央干部,说几句话写几个字就能要人命,于是
他带领县城的全部官员,早早地迎候在门口,等着御史大人光临。

御史来了,知县一声令下,大家听从指挥,整齐划一、动作规范地跪了下来,除了海瑞以外。

这回知县麻烦大了,上次不过是三个人,笔架就笔架,也没啥,这次有几百个人,大家都跪了,你一个人鹤立鸡
群,想要老子的命啊!

御史大人也吃了一惊,心里琢磨着,这南平县应该没有比自己官大的,好像也没有退休高干,这位哥们是哪根葱?

等他弄清情况,顿时火冒三丈,但当着这么多人也不好发火,只好当没看见,随便转了转,连饭都没吃就走人了。

知县擦干了冷汗,就去找海瑞算账,破口大骂他故意捣乱,可海瑞同志脸不红气不喘,听着他骂也不顶嘴,等知县
大人骂得没力气了,便行了个礼,回家吃饭去了。

软硬不吃,既不图升官,也不图发财,你能拿他怎么样?

海纳百川,有容乃大。壁立千仞,无欲则刚。

因为无欲,所以刚强。

海瑞确实没有什么欲望,他唯一的工作动力就是工作,在他看来,自己既然拿朝廷的工钱,就要给朝廷干活,升官
发财与他毫无关系。

这样的一个人,要想升迁自然是天方夜谭,但老天爷就是喜欢开玩笑,最不想升官的,偏偏就升了,还是破格提升。

\section[\thesection]{}

嘉靖三十七年(1558),海瑞意外地接到吏部公文,调他去浙江淳安担任知县。

这是一件让人匪夷所思的事情,在此之前,海瑞不过是个不入流的小官,花名册上能不能找到名字都难说,现在竟
然连升六个品级,成为了七品知县!

无数举人拼命钻营送礼拍马屁,几十年如一日,无非是想捞个知县退休,海瑞干了四年,别说礼物,苍蝇都没送一
只,上级对他恨得咬牙切齿,这么一个人,怎么就升官了?

原因比较复杂,据说是福建的学政十分欣赏海瑞,向上着力推荐了他,但更重要的是,作为一个教谕,他的工作十
分认真,而且干出了成效,这已经充分证明了他的能力,对于帝国而言,马屁精固然需要,但那些人是拿来消遣
的,该干活的时候还得找有能力的人。

关于这个问题,朝廷大员们心里都有数。

于是海瑞揣着这份任命状,离开了福建,前往浙江淳安,在那里,他将开始新的传奇。

在城门口,海瑞见到了迎接他的县里主要官员,包括县丞、主簿、典史,当然,也有教谕。个个笑容可掬,如同见
到久别的亲人一样,并纷纷捶胸顿足,叹息海县令怎么没早点来。

这些仁兄心里到底怎么想的不好说,但可以肯定的是,如果他们知道这里即将发生的事情,一定会叹息当初为啥没
有向朝廷请愿,把这人早点赶走。

俗话说,新官上任三把火,海县令似乎也不例外,他一到地方,便公开宣布,从今以后,所有衙门的陋规一概废
除,大家要加深认识,下定决心,坚决执行。

所谓陋规,也就是灰色收入,美其名曰计划外收入,历史最悠久,使用最频繁的有两招,一个是银两火耗,另一个
是淋尖踢斛,具体方法之前已经介绍过,这里就不多讲了,但随着时代的发展,陋规也不断推陈出新,到了海瑞的
时候,已经形成了一个上瞒朝廷、下宰百姓、方法灵活、形式多样的完美体系。

我们说过,明代的官员工资是很低的,虽说勉强能够过日子,但辛辛苦苦混个官,不是为了过日子的,明代的官
嘛,出门要有轿子,家里要有仆人,没准还要多娶几个老婆,你突然要他勤俭节约,那就是要他的命。

海瑞就打算要他们的命。

\section[\thesection]{}

海大人发布了规定,火耗不准收了,余粮不准收了,总而言之,所有朝廷俸禄之外的钱都不准收。

开始大家都不以为然,反正类似的口号喊得多了,我们不收你也不收吗?他们相信等到这三把火烧完,海县令会恢复
理智的。

但日子一天天过去,海瑞先生却迟迟没有恢复的迹象,他始终没有松口,而且也确实做到了,他自己从不坐轿,步
行上下班,从不领火耗,每天吃青菜豆腐,穿着几件破衣服穿堂入室。

完了,看起来这兄弟是玩真的,不但是火把,还是个油库,打算用熊熊火焰燃烧你我。

一定要反击,要把这股``歪风''打压在萌芽之中!

不久后,淳安县衙出现了一幕前所未有的景象,县丞请假了,主簿请假了,典史请假了,连县公安局长都头也请假
了。总而言之,大家都罢工了,县衙完全瘫痪。

这既是所谓``非暴力不合作'',你要是不上道,就看你一个人能不能玩得转。

他们端起了茶,翘起了腿,准备等看好戏,最终却看到了奇迹的发生。

没有师爷,不要紧,主意自己拿,没有文书,不要紧,文件自己写,没有人管治安,不要紧,每天多走一圈,就当
是巡街。审案的时候没有助手,不要紧,自己查,自己审,自己判!判下来没人打板子,不要紧,家里还有几个老下
人,凑合着也能用。

而海县令的私人生活也让他们大开眼界,自从搬入县衙,海瑞同志就把自己的家人动员了起来,每天老婆下厨做
饭,这就省了厨子的钱,每天老仆上山砍柴,这就省了柴钱。海瑞自己也没闲着,工作之余在自己家后院开辟了一
片菜地,浇水施肥,连菜钱也给省了。

就这么七省八省,海县令还是过得很艰苦,全家人都穿得破破烂烂,灰头土脸,与叫花子颇有几分神似,说他是县
太爷,估计丐帮长老都不信。

情况就是如此了,看着海兄弟每天上堂审案,下地种菜,大家的心里越来越慌,这位大爷看来是准备长期抗战了,
无奈之下,只好各归其位,灰色收入还是小事,要被政府开除,那就只能喝风了。

\section[\thesection]{}

于是众人纷纷回归工作岗位,继续干活,不干也不行,话说回来,你还能造反不成?

久而久之,大家逐渐习惯了艰苦的生活方式,而对海大人的敬仰,也渐如滔滔江水,连绵不绝,因为他们发现,海
县令可谓是全方面发展,不但约束下级,刻薄自己,连上级领导,他也一视同仁。

在明代,地方官有火耗,能征税,所以油水多,而京官就差得远了,只能等下面的人进京的时候,才能大大方方地
捞点好处。所以每次地方官到京城报到,都要准备很多钱,方便应酬。

淳安虽然比较穷困,财政紧张,但这笔钱生死攸关,是绝对省不得的,历任知县去京城出差,至少都要用到近千
两,这还算是比较节省的。

海瑞也进京了,去了一趟回来,支出交给县衙报销,财务一看数字,当时就呆了,空前绝后,绝无仅有----五十五
两。

此数字包括来回路费、车费、住宿费、吃饭费、应酬费以及所有可能出现的费用,是一个绝对破纪录的数字。

这个纪录是怎样创造出来的呢?我来告诉你:上路时,要能走路,绝不坐车,随身带着几张大饼,能凑合,绝不上饭
馆。赶得上驿站就住驿站(驿站凭县衙介绍信不要钱),赶不上绝不住私人旅馆,找一草堆也能凑合一宿。

到了京城,能不应酬就不应酬,要非吃不可,随便找个面摊大排档就打发了,要做到即使对方的脸红得像猪肝,你
也不要在意,要使用联想法增加食欲,边看边吃,就当下饭菜了。争取多吃点,回去的路上还能多顶一阵,顺便把
下顿的饭也省了。

遗憾的是,即使你能做到,也未必可以打破这个纪录,因为海瑞先生瘦,还是精瘦(可以参考画像),吃得不多不
说,衣服用的布料也少,想要超越他,那是非常困难的。

与得罪京官相比,之前冒犯下属实在是件小事,但要和后来他得罪的那两位大人物比较起来,这几个京城里的小官
实在是不值一提。而由一个小人物变成大人物,由无名小卒到闻名遐迩,也正是由此开始。

第一个大人物是胡宗宪,当时他已经是东南第一号人物了,其实说来滑稽,以海瑞的背景和官衔,别说得罪,想见
胡总督一面,起码也得等上半个月,还要准备许多给门房的红包。

但小人物有小人物的方法,海瑞兄不但让胡宗宪牢记住了他的名字,且一分钱没花,还从胡总督那里额外挣了好几
千两银子。

\section[\thesection]{}

说到底,这事还得怪胡宗宪没有管好自己的亲属,虽说他本人也贪,但还不至于和海瑞这种级别的人打交道。可惜
他的儿子没有他的觉悟。

话说胡公子有一个习惯----旅游,当然他旅游自己不用花钱,反正老子的老子是总督,一路走过来就一路吃,一路
拿,顺便挣点零花钱,这还不算,他还喜欢反复游览同一景区,走回头路,拿回头钱。

即使如此,还是有很多知府知县盼着他去,毕竟是总督的儿子,能美言两句也是好的,反正招待费不用自己出,何
乐而不为。

但是海瑞不愿意,在他看来,国家的钱也是钱,绝对不能乱花,对此很不感冒。可是不感冒也好,不愿意也罢,该
来的还是要来。

在一次游览途中,胡公子恰好经过淳安,便大摇大摆地住进了当地招待所,等着县太爷来请安,事情就此开始。

这个消息很快就传到了海瑞的耳朵里,尽管下属反复强调这是胡宗宪的儿子,海瑞的回答却只有一句:

``胡宗宪的儿子,又不是胡宗宪,管他做甚?''

招待所的工作人员接到指示,就按打发一般客人的标准请胡公子用饭,海瑞先生自己吃糙米饭,喝咸菜汤,他招待
客人的水平自然也高不到哪里去。于是很快第二个消息传来,胡公子大发脾气,把厨子连同招待所管理员吊起来狠
打了一顿。

大家都急了,正想着如何收这个场,让总督的儿子消消气,海瑞却把桌子一拍,大喊一声:

``还反了他了,马上派人过去,把他也吊起来打!''

这个天才的创意超出了所有人的思维范畴,所有的人都目瞪口呆,包括打人的衙役在内。看见没人动,海瑞又拍了
一次桌子,加了一把油:

``去打就是了,有什么事情我负责!''

本来就不待见你,竟然还敢逞威风,打不死你个兔崽子!

好,这可是你说的,反正有人背黑锅,不打白不打,于是众人赶过去一阵火拼,虽说胡公子身边有几个流氓地痞,
到底打不过衙门里的职业打手,被海扁了一顿,这还不算,海县令做完了打手还要干抢劫,连这位胡公子身边带着
的几千两银子也充了公。

\section[\thesection]{}

人打完了,瘾过足了,鼻青脸肿的胡公子被送走了,海大人也差不多该完蛋了。这就是当时众人对时局的一致看法。
打了人家的儿子,抢了人家的钱,还不收拾你,那就真是没有天理了。

海瑞却不这么看,他告诉惊慌失措的下属们,无须害怕,这件事情他能搞定。

怎么搞定?去磕头请安送钱人家都未必理你!

不用,不用,既不用送钱,也不用赔礼,只需要一封信而已。

事实确实如此,万事如意,天下太平,一封信足矣。

奇迹啊,现将此信主要内容介绍如下,以供大家学习参考:

胡大人,我记得你以前出外巡视的时候曾经说过,各州县都要节约,过路官员不准铺张浪费,但今天我县接待一个
过往人员的时候,他认为招待过于简单,竟然毒打了服务员,还敢自称是您的儿子,我一直听说您对儿女的教育很
严格,怎么会有这样的儿子呢?这个人一定是假冒的,败坏您的名声,如此恶劣,令人发指,为示惩戒,他的全部财
产已被我没收,充入国库,并把此人送到你那里去,让你发落。

胡宗宪看到之后哭笑不得,此事就此不了了之,海瑞依然当他的县令,胡宗宪依然抗他的倭,倒是那位胡公子,据
说回去后又挨了老爹一顿臭骂,从此旅游兴致大减。

这是一段为许多史书转载的记录,用以描绘海瑞先生的光辉形象,但事实上,在它的背后,还隐藏着两个不为人见
的重要信息:

首先,这个故事告诉我们,海瑞先生虽然吃糙米饭,穿破衣烂衫,处事坚决不留余地,却并不是个笨人,蠢人做不
了清官,只能当蠢官。

而隐藏得更深的一点是:胡宗宪是一个品格比较高尚的人,虽说海瑞动了脑筋,做了篇文章,但胡宗宪要收拾他,
也不过是分分钟的事情,总督要整知县,随便找个由头就行了,儿子被打了,脸也丢了,胡总督却没有秋后算账。
所以他虽然不是个好父亲,却实在是个好总督。

这一次,海瑞安全过关,但说到底,还是因为遇见了好人,下一次,他就没这么幸运了。

说来惭愧,明代人物众多,但能上兄弟这篇文章的,毕竟是少数,因为篇幅有限,好人也好,坏人也罢,只有名人
才能露脸。

就以严党为例,其实严嵩的手下很多,我算了一下,光尚书侍郎这样的部级官员就有二十多个(包括南京及都察院同
级别官员在内),当年虽然耀武扬威,现在却啥也不是,所以本着本人的``写作三突出''原则(名词解释:在坏人中
突出主要坏人,在主要坏人中突出极品坏人,在极品中突出坏得掉渣的坏人),在其中只选取了严世蕃、赵文华和鄢
懋卿出场,其中赵文华是配角,鄢懋卿龙套。

但事情就这么巧,鄢龙套虽说已经退场,却又获得了一次上镜的机会,全拜海瑞所赐。

\section[\thesection]{}

真是机缘巧合,在当年像海瑞这样的小人物,竟然和朝中的几位大哥级红人都有过联系,得罪完胡总督,又惹了鄢
御史。

嘉靖三十九年(1560),鄢懋卿受皇帝委派,到全国各地视察盐政,鄢兄的为人我们已经介绍过了,那真是打着电筒
也找不出闪光点,每到一处吃喝嫖赌无不涉猎,还要地方报销,这也就罢了,偏偏他既要做婊子,又要立牌坊,还
四处发公文,说自己素来俭朴,地方的接待工作就不要太铺张,要厉行节约。

就这么吃吃喝喝,一路晃悠,鄢大人来到了浙江,准备由淳安路过,海瑞不想接待,也没钱接待,希望他能绕道
走,但鄢大人毕竟是钦差,你要设置路障不让他过,似乎也说不过去。

于是海大人开动脑筋,又用一封信解决了问题。

这封信十分奇特,开头先用了鄢懋卿自己的告示,大大地捧了他一番,说您不愧是清廉官员的典范,景仰之情如滔
滔江水等等,然后突然笔锋一转,开始诉苦:

不过我也听到过一些谣言,说您每到一地接待都非常奢华,我们这里是个穷县,如果按那个标准,我们实在接待不
起,况且还违背您的本意。可万一……,那我们不就得罪大人您了嘛。

卑职想来想去,不知如何是好,只好向您请教,给我个出路吧。

这就算是捅了马蜂窝了,鄢懋卿的鼻子都气歪了,但毕竟是老江湖,他派人去摸了海瑞的底,发现这哥们软硬不
吃,胡宗宪也吃过亏,于是钦差大人一咬牙,绕道走!

海瑞再次赢得了胜利,却也埋下了祸根,因为不是每个人都有胡宗宪那样的风格。

当然,海大人除了工作认真、生活俭朴之外,有时也会奢侈一下,比如有一次,他的母亲生日,海县令无以为贺,
便决定上街买两斤肉,当他走进菜市场,在一个肉摊面前停下来的时候,现场出现了死一般的寂静,大家都目不转
睛地看着这惊人的一幕。

人人都知道,海县官是自然经济的忠实拥护者,自己砍柴,自己种菜,完全实现了自给自足,别说买菜,他不把自
己种的菜拿出来卖,搞市场竞争,就算积德了。

然而他买肉了,竟然还买了两斤,等他付完钱,接过肉一声不吭地扬长而去时,在场的人这才确信,他们刚才看到
了一幕真实的场景。

肉贩子激动了,他压抑不住自己内心的冲动,壮怀激烈,仰天长啸:

``想不到我这辈子还能做上海县令的生意啊!''

海县令竟然买肉了!

\section[\thesection]{}

在那个没有电话、送封信要好几天的年代,海县令的这一壮举以惊人的速度被传播到了大江南北,知府知道了,巡
抚知道了,很快,胡宗宪也知道了。

于是,在之后召开的一次政务会议上,胡总督高谈阔论一番抗倭形势之后,突然神色一变,以一副极为神秘的表情
向大家通报了这个消息。

所有的人都被震惊了,海县令竟然买肉了!

似乎很可笑,不是吗?

我不觉得。

一晃三年过去了,在海瑞的治理之下,淳安人民生活水平不断提高,官吏们的生活水平却在不断下降,可他们又惹
不起这位活阎王,只能埋头干活。但临近年终,唉声叹气的官员们却突然变了模样,往日愁云密布的脸孔,开始绽
放憧憬的笑容。

这和发年终奖无关,要知道,在海阎王手下干活,这类型的玩意基本上不要指望,真正让他们欣喜若狂的,是一个
小道消息----海阎王就要高升了。

明代的官员制度规定,但凡地方官,每三年由上级部门考核一次,对照吏部的标准打分,如果是劣等,就要被记过
警告,没准就要回家种红薯,而要能评个优等,就能升官。

海瑞无疑是优等,不管别人对他有何等看法,他的工作是无可挑剔的,而这对淳安县的官员们来说无异于一场及时
雨,他们开始积极准备送行仪式:永别了,海大人,无论您去哪里,只要不在这里就好,祝您一路顺风。

就在众人带着对未来的无限向往埋头准备时,确切的消息下来了,不是消暑的大雨,却是平地的惊雷。经过吏部考
核,认定海瑞为优等,应予晋升,为方便工作开展,决定就地提拔为嘉兴府通判,即刻上任。

完了,彻底地完了,这下整个嘉兴地区都轰动了:你们淳安县城自己倒霉不算,竟然还要闹腾上来?

淳安的例子就在眼前,必须采取行动,否则后果不堪设想!

嘉兴的官员们随即开始了紧急总动员,大家纷纷回家查家谱,无论是三姑六婆、七姐八姨,吃过饭的,见过面的,
点过头的,只要是个人,有关系,统统都去找,务必要把海瑞赶走。

很快,海瑞就受到了人生中的第一次弹劾,弹劾者是都察院监察御史,联系到鄢懋卿同志的职务和他的为人(都察院
左副都御史),我们不难猜出其中奥妙,至于弹劾的罪状,那实在是一件无关紧要的事情。

应该说,这是一个不错的开始,因为它意味着海瑞已经具有了相当的影响力,要是名声不大,鬼才骂你。

\section[\thesection]{}

但后果仍然是极其严重的,海瑞失去了通判的职位,并接到了吏部的第二道调令--改任江西兴国知县。

兴国是个穷地方,调去那里似乎也算一种发配,所以看上去,这是个合乎情理的结果,然而事实并非如此。

根据鄢懋卿之前的预计,在他的授意弹劾下,像海瑞这样毫无背景和关系的人,不但无法升官,还会被革职查办。
但他万没想到,此人虽然未能晋升,却也保住了官位。多年的政治经验告诉他,其中必有名堂,所以吃惊之余,他
也没敢再找海瑞的麻烦。

鄢懋卿的直觉没有错,在看似孤立无援的海瑞背后,确实隐藏着另一个人,而且还是个大人物,他就是当年的那位
福建学政,现在的吏部侍郎朱衡。

在这个世界上,有正直的人,自然就有欣赏正直的人,朱衡就是一个,别人厌恶海瑞,他却赞赏有加,所以之前他
力排众议,向上级推荐了海瑞,破格提拔了他。

而三年之后,他再次挺身而出,保住了海瑞,真是人算不如天算,朱大人偏偏就去了吏部,还偏偏是个副部长。

就这样,海瑞去了江西兴国,继续当他的县令,因为朱衡的保护,他安然度过了人生中的第一个危机,此时他四十
九岁,依然是个七品芝麻官,再混几任就光荣退休,这似乎已是他的宿命。

如果此时有人告诉他,短短几年之后,他这个小人物将闻名天下,并成为中央的高级官员,重权在握,恐怕连海先
生自己都不会相信。

然而事实正是如此。命运之神实在很照顾海先生,他虽然性格不对,天赋不高,运气却出奇地好,虽然他后来惹出
了更大的麻烦,却依然涉险过关,安然无恙----因为另一位大人物的帮助。

在海瑞看来,兴国和淳安除了名字不同,没有什么两样,该怎么干还怎么干,这下又轮到兴国的衙役们受苦了,但
出人意料的是,在兴国的这几年,海县令竟然没惹过事,想来还是因为地方太穷,没人从这儿过,自然也就没有是
非了。

就在海县令专心致志干活的时候,却突然接到一道出人意料的调令,命他即刻进京,就任户部云南司主事。

此时是嘉靖四十三年(1564),还没到三年考核期,而户部云南司主事,是一个正六品官,从地方官到京官,从七品
到六品,一切都莫名其妙。

虽然海瑞不知道,但我们知道,这自然又是那位朱副部长帮忙的结果。就这样,海县令成了海主事,职务变了,地
方变了,人却是不会变的。

\section[\thesection]{}

在地方当县令就敢和总督对着干,按照这个标准,到了京城,如果不找皇帝的麻烦,那简直就没有天理了。

在亲眼见识了真正的政治黑幕和贪污腐化后,海瑞终于忍无可忍,写下了那封天下第一名疏,用他的正直痛斥这一
切的罪魁祸首----皇帝。

在明代,骂皇帝的人并不少,却只有海瑞先生脱颖而出,名垂千古,对此我只能说,不是侥幸,绝不是侥幸。

因为骂人固然轻松,却还要看你骂得是谁,在明代的十几位皇帝中,要论难伺候,嘉靖同志绝对可以排在前三名,
这个人极其难搞,不但疑心重,还好面子,但凡骂过他的人,比如之前的杨最、杨爵、高金等人,只是提了点不同
意见,就被拉了出去,不打死,也得打个半死。

好汉不吃眼前亏,事实证明,言官之中还是好汉居多,许多人本来就是为骂而骂,纯粹过过嘴瘾,将来退休回家还
能跟邻居老太太吹吹牛:想当年,老子可是骂过皇帝的咧。

基于这种动机,在骂人的时候,诸位言官是要考虑成本问题的,而嘉靖同志太过生猛,不是打就是关,亏本的生意
还是不做的好。

海瑞偏偏就做了这笔亏本的生意,因为在他的思维里,根本没有成本这个概念。他只知道,他是朝廷的官员,吃着
朝廷的俸禄,就该干活,就该做事,就该为民做主!

他不是不清楚呈上奏疏的后果,所以他提前买好了棺材,据说是他亲自去挑的,好棺材还买不起,只能买口薄皮
的,好歹躺得进去,凑合能用就行。

他的老婆在家等他下班,却看到了这口棺材,顿时惊得目瞪口呆,随即痛哭失声,海瑞却只是平静地对她说:

``记得到时把我放进去就是了。''

如果说杨继盛是死劾,那么海瑞大致就是死谏了,虽不是当场死亡,也等不了多久。要知道,脑袋一团浆糊,盲人
瞎马地掉下山崖,那叫失足,为了一个崇高的目标,昂首阔步踏入深渊,才叫勇敢。而这口棺材,正是他勇气的证
明。

不知死而死,是为无知,知死而死,是为无畏。

海瑞,你是一个无畏的男人。

\section[\thesection]{}

不听话的下属

一切正如海瑞预料的那样,皇帝震怒,满朝轰动,关入监牢,等待处斩。但让他感到纳闷的是,自己的情节应属于
极其恶劣,罪大恶极,斩立决都嫌慢的那一类,可左等右等,挂在头上的那把刀却迟迟不落下来。

因为皇帝还不打算杀他,在听完黄锦的话后,他愣了一下,捡起了那份奏疏,看了第二遍。

嘉靖不是个笨人,他知道,一个人既然已买了棺材,自然是有备而来,而在对这份奏疏的再次审视中,他看到了攻
击、斥责之外的东西----

忠诚、尽责和正直。

于是他发出了自己的感叹:

``这个人大概算是比干吧,可惜我不是纣王。''

能讲出这种水平的话,说他是昏君,那也实在太不靠谱了。

海瑞就这样被关了起来,既不是有期,也不是无期,既不杀,也不放,连个说法都没有,他自己倒是很自在,每天
照吃照睡,一点心理负担都没有。

看起来命是保住了,实际上没有。

你要明白,嘉靖同志可是个很要面子的人,就算他懂得道理,知道好歹,你用这种方式对待他,似乎也有点太过
了,一个千里之外的杨慎他都能记几十年,何况是眼皮底下的海瑞?

终于有一天,他又想起了这件事,便发火了,火得受不了,就开始骂,骂了不解恨,就决定杀。

眼看海瑞就要上法场,第二个保他的人出现了----徐阶。

徐阶与严嵩有很多不同,其中之一就是别人倒霉,严嵩会上去踩两脚,而徐阶会扶他起来。

徐大人实在是个好人,不收钱也办事,他认定海瑞是一个难得的人才,便决定拉他一把。

但是这事很难办,因为嘉靖这号人,平时从不喊打喊杀,但一旦决定干掉谁,大象都拉不回来,之前也曾有人上书
劝他放人,结果被狠打了一顿,差点没咽气。

\section[\thesection]{}

但徐阶再次用行动证明,嘉靖这辈子的能耐算是到头了,因为这位内阁首辅只用了一段对话,就把海瑞从死亡线上
拉了回来:

``皇上你上了海瑞的当了!''

嘉靖带着疑惑的神情,目不转睛地看着发出惊呼的徐阶。

``我听说海瑞在上书之前,已经买好了棺材,他明知会触怒皇上,还敢如此大逆不道,用心何其歹毒!''

歹毒在什么地方呢,听徐老师继续忽悠:

``此人的目的十分明确,只求激怒陛下,然后以死求名而已,皇上你如果杀了他,就会正中他的圈套!''

嘉靖一边全神贯注地听,一边连连点头,是的,无比英明的皇帝陛下,怎么能受一个小小六品主事的骗呢?就算上
当,也得找个有档次的高级干部嘛----比如徐阶同志。

就这样,海瑞的命保住了,他继续在监狱住了下来,对他而言,蹲牢房也算不上是啥坏事,反正家里和牢里伙食差
不多,还能省点饭钱。

事实上,在徐阶看来,海主事闹出的这点麻烦实在是小儿科,他现在急于解决的,是另一个极为棘手的问题。

在严嵩当权那几年,内阁里只有徐阶给他跑腿,后来徐阶当权,就找来自己的门生袁炜入阁跑腿,可是这位袁先生
似乎不打算当狗腿子,压根没把老师放在眼里,时不时还要和徐阶吵一架。徐大人当然不会生气,但自然免不了给
袁炜穿穿小鞋,偏偏这位袁先生心理承受能力不强,郁闷之下竟然病了,嘉靖四十四年(1565)告病回了家。

不听话的走了,就找两个听话的来,这两个人,一个叫严讷,一个叫李春芳。

严讷兄就不多说了,他于嘉靖四十四年(1565)入阁,只干了八个月就病倒了,回了老家,内阁中只剩下了李春芳。

这位李春芳同志,那就不能不说了,他的为人可以用一句话概括:厚道、太厚道了。

在几百年后看来,作为嘉靖二十六年的状元,李春芳是不幸的,因为与同科同学相比,他的名声成就实在有限,别
说张居正,连杨继盛、王世贞他也望尘莫及。但在当时,这位仁兄的进步还是很快的,当张居正还是个从五品翰林
院学士的时候,他已经是正二品礼部尚书了。

他能升得这么快,只是因为两点:一、擅长写青词。二、老实。自入朝以来,外面斗得你死我活,他却不闻不问,
每天关在家里写青词,遇到严嵩就鞠躬,碰见徐阶也敬礼,算是个老好人。

所以徐阶挑中了他,让他进内阁打下手。

事情到了这里,可以说是圆满解决了,但接下来,徐阶却作出了一个错误的判断,正是这个判断,给他种下了致命
的祸根。

\section[\thesection]{}

嘉靖四十五年( 1566)三月,经内阁首辅徐阶力荐,皇帝批准,礼部尚书高拱入阁,任文渊阁大学士,与其同时入阁
的还有吏部尚书郭朴。

在这个任命的背后,是一个精得不能再精的打算。

高拱不喜欢徐阶,徐阶知道。

自打嘉靖二十年(1541)高拱以高分考入朝廷,他就明确了这样一个认识----要当,就当最大的官,要做,就做最大
的事。

高翰林就这样踌躇满志地迈进了帝国的官场,准备找到那个属于自己的位置,然而现实对他说----一边凉快去。

在长达十一年的时间里,翰林院新人,七品编修高拱唯一的工作是整理文件,以及旁观。

他看到了郭勋在监牢里被人整死,看到了夏言被拉出去斩首,看到了严嵩的跋扈,徐阶的隐忍,他很聪明,他知道
如果现在去凑这个热闹,那就是找死。

直到嘉靖三十一年(1552),他才第一次看到了自己的希望,在这一年,他成为了裕王府的讲官。

对于寂寂无名,丢进人堆就没影的高翰林而言,这是一个千载难逢的机会,而高拱牢牢地抓住了它。

自从嘉靖二十八年(1549)太子去世以后,嘉靖就没有立过接班人,不但不立,口风还非常之紧,对剩下的两个儿子
裕王、景王若即若离,时远时近。

这件事干得相当缺德,特别是对裕王而言。按年龄,他早生一个月,所以太子应该非他莫属,但嘉靖同志偏偏坚信
``二龙不相见''理论,皇帝是老龙,太子就是青年龙,为了老子封建迷信的需要,儿子你就再委屈个几十年吧。

不立太子也就罢了,可让裕王想不通的是,按照规定,自己的弟弟早该滚出京城去他的封地了,可这位仁兄仗着没
有太子,死赖着就是不走,肚子里打什么算盘地球人都知道。

于是一时之间群魔乱舞,风雨欲来,景王同志还经常搞点小动作,整得裕王不得安生,唯恐到嘴的鸭子又飞了,整
日提心吊胆,活在恐惧之中。

在这最困难的时刻,高拱来到了他的身边,在之后的日子里,这位讲官除了耐心教授知识之外,还经常开导裕王,
保护他不受侵扰,日夜不离,这十几年的时间里,高拱不求升官,也不图发财,像哄小孩一样地哄着这位软弱的王
爷,并用自己的行动对他阐述了这样一个事实:面包会有的,烧饼会有的,皇位也会有的,就算什么都没有,也还
有我。

所以在那些年,虽然外面腥风血雨,裕王这里却是风平浪静,安然无恙,有高门卫守着,无论严嵩、徐阶还是景
王,一个也进不来,比门神好用得多。

\section[\thesection]{}

裕王很感激高拱。

关于这一点,严嵩清楚,徐阶也清楚。

于是高拱就成了抢手货,双方都想把他拉到自己这边,严嵩当政的时候,高拱从一个讲官被提拔为太常寺卿(三品)
兼国子监祭酒,成为了高级官员。

高拱没有推辞,他慨然就任,却不去严嵩家拜码头:朝廷给我的官嘛,与你严嵩何干?

等到嘉靖四十一年(1562),严嵩退休了,徐阶当政,高拱再次升官,成为了礼部副部长,没过多久他再进一步,任
正部级礼部尚书。

傻子也知道,这都是徐阶提拔的结果,然而高拱却依然故我,官照做,门不进,对徐大人的一片苦心全然无视。

说句实诚话,徐阶对高拱是相当不错的,还曾经救过他一次:原先高拱曾经当过会试的主考官,不知是那根神经出
了岔子,出了个惹事的题目,激怒了嘉靖。皇帝大人本打算打发他回家种地,好在徐阶出面,帮高拱说了很多好
话,这才把事情解决。

现在徐阶又一次提拔了高拱,把他抬进了内阁,然而高拱的反应却大大地出乎了徐阶的意料。

他非但不感激徐阶,还跟徐阶捣乱,自打他进内阁的那天起,就没消停过。而闹得最大的,无疑是值班员事件。

当时的内阁有自己的办公楼,按规定内阁成员应该在该处办公,但问题是,嘉靖同志并不住在寝宫,总是呆在西苑。
当大臣的,第一要务就要把握皇帝的心思,对这么个难伺候的主,要是不时时刻刻跟着,没准明天就被人给灭了。
所以但凡内阁大臣,都不去内阁,总是呆在西苑的值班房,坐下就不走。

终于有一天,嘉靖没事散步的时候去了值班房,一看内阁的人全在,本来还挺高兴,结果一盘算,人都在这呆着,
内阁出了事情谁管?

嘉靖不高兴了,他当即下令,你们住这可以,但要每天派一个人去内阁值班,派谁我不管,总之那边要人盯着。

于是内阁的大臣们开始商量谁去,当然了,谁都不想去,等了很久也没有人自动请缨,于是徐阶发话了:

``我是首辅,责任重大,不能离开陛下,我不能去。''

话音还没落,高拱就发言了:

``没错,您的资历老,应该陪着皇上,我和李春芳、郭朴都刚入阁不久,值班的事情您就交给我们就是了。''

徐阶当时就发火了。

\section[\thesection]{}

从字面上看,高拱的话似乎没错,还很得体,但在官场混了这么多年,徐阶自然明白这位下属的真正意思,估计高
拱先生说话时候的语气也有点阴阳怪气,所以二十多年不动声色的徐首辅也生气了:严嵩老子都解决了,你小子算
怎么回事?

虽然发火,但是涵养还是有的,徐阶同志涨红了脸,一言不发,扬长而去。

看起来,高拱似乎有点不识好歹,然而事实并非如此。

但凡混朝廷的人,都有这样一个共识----不欠人情,欠了要还。

这才是高拱与徐阶两个人的根本矛盾所在,徐大人认为高拱欠了他的人情,高拱认为没有。

徐阶不是开慈善机构的,他之所以提拔高拱,自然是看中了他的裕王背景,虽说自己现在大权在握,但毕竟总有下
岗的一天,要是现在不搞好关系,到时高拱上台,想混个夕阳无限好自然死亡就难了。

可惜高拱也很清楚这一点,要知道,在斗争激烈的嘉靖年间生存下来,官还越做越大,绝不是等闲之辈能做到的,
他早就看透了徐阶的算盘。按照皇帝现在的身体,估计熬个几年就能升天了,到时候裕王必定登基,我高拱自然就
是朝廷的首辅,连你徐阶都要老老实实听我的话,哪要你做顺水人情?

加上高拱此人身负奇才,性格高傲,当年不买严嵩的帐,现在的徐阶当然也不放在眼里。

精明了一辈子的徐阶终于糊涂了一回,他没想到提拔高拱不但没能拉拢他,反而使矛盾提前激化,一场新的斗争已
迫在眉睫。

更为麻烦的是,徐首辅在摸底的时候看走了眼,与高拱同期入阁的郭朴也不地道,他不但是高拱的同乡,而且在私
底下早就结成了政治同盟,两人同气连枝,开始跟徐阶作对,而李春芳一向都是老好人,见谁都笑嘻嘻的,即使徐
阶被人当街砍死,估计他连眼都不会眨一下。

在近四十年的政治生涯中,徐阶曾两次用错了人,正是这两个错误的任命,让他差点死无葬身之地。这是第一次。

当然,现在还不是收场的时候,对于高拱和徐阶来说,这场戏才刚刚开始。

丰富的政治经验及时提醒了徐阶,他终于发现高拱并不是一个能够随意操控的人,而此人入阁的唯一目的,就是取
自己而代之。

虽然走错了一步,在内阁中成为了少数派,但不要紧,事情还有挽回的余地,只要再拉一个人进来,就能再次战胜
对手。

\section[\thesection]{}

天才,就是天才

当何心隐帮助徐阶除掉严嵩,在京城晃悠了大半年,飘然离京之时,曾对人说过这样一番话:

``天下之能士尽在京城,而在我看来,能兴我学者并非华亭,亡我学者也非分宜,兴亡只在江陵。''

这是一句不太好懂却又很关键的话,必须要逐字解释:

所谓我学,就是指王学,这段话的中心意思是描述王学的生死存亡与三个人的关系。而这三个人,分别是``华亭''、
``分宜''与``江陵''。

能兴起王学的,不是``华亭'',能灭亡王学的,不是``分宜'',只有``江陵'',才能决定王学的命运。

在明清乃至民国的官场中,经常会用籍贯来代称某人,比如袁世凯被称为袁项城(河南项城),黎元洪被称为黎黄陂
(湖北黄陂)。套用这个规矩,此段话大意如下:

兴我王学者,不是徐阶,亡我王学者,不是严嵩,兴亡之所定者,只在张居正!

何心隐说出这句话的时候,张居正的职务是从五品翰林院侍讲学士。

张居正,字叔大,号太岳,湖广江陵人,明代最杰出的政治家,最优秀的内阁首辅。

请注意,在这两个称呼的后面,没有之一。

嘉靖四年(1525),湖广荆州府江陵县的穷秀才张文明,终于在焦急中等来了儿子的啼哭。

作为一个不得志的读书人,儿子的诞生给张文明带来了极大的喜悦,而在商议取名字的时候,平日不怎么说话的祖
父张诚却突然开口,说出了自己不久之前的一个梦:

``几天之前,我曾梦见一只白龟,就以此为名吧。''

于是这个孩子被命名为张白圭(龟)。

虽说在今天,说人是乌龟一般都会引来类似斗殴之类的体育活动,但在当年,乌龟那可是吉利的玩意,特别是白
龟,绝对是稀有品种,胡宗宪总督就是凭着白鹿和白乌龟才获得了皇帝的宠信,所以这名也还不错。

此时的张白圭,就是后来的张居正,但关于他的籍贯,却必须再提一下,因为用现在的话说,张家是个外来户,他
们真正的出处,是凤阳。

\section[\thesection]{}

两百年前,当朱元璋率军在老家征战的时候,一个叫张关保的老乡加入了他的队伍,虽然这位仁兄能力有限,没有
干出什么丰功伟绩,但毕竟混了个脸熟,起义成功后被封为千户,去了湖广。

这是一个相当诡异的巧合,所以也有很多讲风水的人认为,这还是朱重八太过生猛,死前就埋下了伏笔,二百年后
让这个人的后代拯救明朝于水火之中,这种说法似乎不太靠谱,而事实的确如此。

当然,和朱重八的父亲朱五四比起来,张文明的生活要强得多,起码不愁吃穿,有份正经工作,但要总拿穷人朱五
四开涮,也实在没啥意思,毕竟和他的同龄人比起来,张文明这一辈子算是相当的失败,他虽然发奋读书,二十岁
就考中了秀才,此后却不太走运,连续考了七次举人都没有中,二十多年过去了,还是个秀才。

父亲实现不了的梦想,只能寄托在子女身上,据说张白圭才几个月,张文明就拿着唐诗在他面前读,虽说他也没指
望这孩子能突然停止吃奶,念出一条``锄禾日当午''之类的名句来,但奇迹还是发生了。

不知是不是唐诗教育起了作用,张白圭一岁多就会说话了,应该说比爱因斯坦要强得多,邻居们就此称其为神童。

一晃张神童就五岁了,进了私塾,而他在读书方面的天赋也显现了出来,过目不忘,下笔成文,过了几年,先生叫
来了他的父亲,郑重地对他说:

``这孩子我教不了了,你带他去考试吧。''

所谓考试,是考县学,也就是所谓的考秀才,张文明领着儿子随即去了考场,那一年,张白圭十二岁。

张白圭的运气很好,那一年的秀才考官是荆州知府李士翱,这位兄弟是个比较正直爱才的人,看到张白圭的卷子
后,大为赞赏,当即不顾众人反对,把这个才十二岁的孩子排到了第一。

这是个比较轰动的事情,整个荆州都议论纷纷,可李士翱却只是反复翻阅着张白圭的答卷,感叹着同一个词:

``国器!国器!''

他约见了张文明和他的儿子张白圭,在几番交谈和极度称赞之后,李知府有了这样一个念头:

在他看来,乌龟虽然吉利,但对于眼前的这位神童而言,顶着乌龟的名字过一辈子似乎也不太妥当,于是他对张文
明说道:

``你的儿子前途不可限量,但白圭之名似不大妥当,我看就改名叫居正吧。''

此后,他的名字便叫做张居正。

\section[\thesection]{}

秀才考上了,下一步自然就是举人了,和考进士不同,举人不是隔年就能去的,按照规定,您得在学校再熬个两三
年,过了资格考试才能考,但那是一般性规定,张秀才不是一般人,所以他第二年就去了。

所谓赶得早不如赶得巧,正是这次破格的考试中,张居正遇上了那个影响他一生的人。

在考试开始之前,考官照例要向领导介绍一下这一科的考生情况,于是湖广第一号人物顾璘得知,有一个十三岁的
孩子也来考试了。

六十五年前,一个十三岁的少年曾应考举人并一举中第,他就是闹腾三朝,权倾天下的杨廷和,所以对于这位后来
者,顾璘不敢怠慢,他决定亲自去见此人一面。

两人见面之后的情节就比较俗套了,顾巡抚先看相貌,要知道,张居正同志是明代著名的帅哥,后来做了首辅,跟
李太后还经常扯不清,道不明,传得风言风语,年轻的时候自然也差不到哪去。这是面试关,满意通过。

然后就是考文化了,据说顾巡抚问了张居正几个问题,还出了几个对联,张居正对答如流,眼睛都不眨一下。顾璘
十分惊讶,赞赏有加。

两人越说越高兴,越说越投机,于是在这次谈话的结束阶段,巡抚大人估计是过于兴奋了,一边说话,一边作出了
一个惊人的举动----解腰带。

当然,顾巡抚绝对没有耍流氓的意思,他的那条腰带也比今天的皮带贵得多----犀带。

在将腰带交给张居正的时候,顾璘还说了这样一句话:

``你将来是要系玉带的,我的这一条配不上你,只能暂时委屈你了。''

事实上,这绝不仅仅是一个关于裤腰带的问题,而是一个极具寓意的场景,是一个非同小可的政治预言。

在明代,衣服是不能随便穿的,多大的官系多高级的裤腰带,那也是有规定的,乱系是要杀头的。而像顾璘这样的
高级官员,系一条犀带招摇过市已经算很牛了。

但他认为,眼前的这个少年可以系玉带,而玉带,只属于一品官员。

懵懵懂懂的张居正接过了这份珍贵的礼物,他看着顾璘的肚子,随即作出了一个准确的判断----自己多了一条用不
了的腰带。

\section[\thesection]{}

张秀才捧着腰带回去备考了,顾璘也收起了原先满面欣赏的表情,跑去找到了主考官,下了这样一道命令:

``这科无论张居正答卷如何,都绝不能让他中第!''

这是一个让在场的所有人都目瞪口呆的决定,顾巡抚翻脸的速度似乎也太快了点,但巡抚的命令自然是要听的,于
是张秀才费尽心机写出的一张答卷成了废纸,打破杨廷和先生纪录的机会也就此失去。

郁闷到了极点的张居正回到了家乡,开始苦读诗书,准备三年后的那次考试,蒙在鼓里的他想破脑袋也想不通,到
底是哪里出了问题?

多年以后,张居正再次遇见顾璘时,才终于得知原来罪魁祸首正是这位巡抚大人,但他没有丝毫的埋怨,反而感动
得痛哭流涕。

顾璘实在是一个难得的好人,他曾亲眼见过无数像张居正这样的年轻人,身负绝学才华横溢,却因为年少成名而得
意忘形,最终成为了一个四处游荡以风流才子自居的平庸官僚。所以当他看见张居正的时候,便决定不让这一悲剧
再次上演。

只有经历过磨难的人,才能够走得更远,张居正,你的未来很远大。

嘉靖十九年(1540),带着不甘与期望,张居正再次进入了考场,这一次他考中了举人。

正如顾璘所料,张居正还是太年轻了,十六岁的他在一片赞赏声中开始迷失,认定自己中进士不过是个时间问题,
书也不读了,开始搞起了兴趣小组之类的玩意,每天和一群所谓名士文人聚会,吃吃喝喝吟诗作对,转眼到了第二
年,张才子两手一摊----不考了。

反正考上进士易如反掌,那还不如在家多玩几年,这大致就是少年张居正的想法。

玩是一件幸福的事情,但不干正事,每天只玩就比较无聊了,就在张居正逐渐厌倦这种所谓的``幸福''时,真正的
痛苦降临了。

在这次痛苦的经历中,张居正受到了人生的第一次打击,确立了第一个志向,也找到了自己的第一个敌人。

事情是这样的,虽然张居正的父亲张文明只是一个穷秀才,但他的祖父张镇却是有体面工作的,具体说来,他是辽
王府的护卫。

\section[\thesection]{}

荆州这个地方虽然不大,却正好住着一位王爷----辽王,说起这个爵位,那可是有年头了,当初朱重八革命成功后
分封儿子,其中一个去了辽东,被称为辽王,到了他的儿子朱老四二次革命成功,觉得自己的诸多兄弟在周围碍
眼,便把北京附近的王爷统统赶到了南方。辽王就这样收拾行李去了荆州。

根据明代规定,只要家里不死绝,王位就一直有,于是爷爷传给儿子,儿子传给孙子,铁打的爵位,流水的孙子,
两百年后,这位孙子的名字叫做朱宪火节。

这里顺便说一句,有明一代,出现过许多怪字奇字,可谓是前无古人,后无来者,不要说新华字典、康熙字典,火
星字典里都找不到,原因很简单,这些字压根就不存在。

说到底,这还要怪朱重八,这位仁兄实在太过劳模,连子孙的名字都搞了一套规范,具体如下:自他以后,所有的
儿子孙子名字中的第三个字的偏旁必须为金木水火土,依次排列,另一半是啥可以自便。

可是以金木水火土为偏旁的字实在有限,根本满足不了大家的需要,什么``照''、``棣''、``基''之类的现成字要
先保证皇帝那一家子,取重名又是个大忌讳,于是每一代各地藩王为取名字都是绞尽脑汁,抓破头皮,万般无奈之
下,只好自己造字,确定偏旁后,在右边随便安个字就算凑合了。

这是一个极为害人的规定,其中一个受害者就是我,每次看到那些鬼字就头疼,什么输入法都打不出来,只能也照
样拼一个。

而这位辽王朱宪火节(为省事,以下称辽王)除了名字让人难受外,为人也不咋地,自打他继承辽王爵位后,就把仇
恨的眼光投向了张居正。

这说起来是个比较奇怪的事情,张居正从来没有见过辽王,而他的祖父,所谓的王府护卫张镇,其实也就是个门
卫,门卫家的孩子怎么会惹上辽王呢?

归根结底,这还要怪辽王他妈,这位辽王兄年纪与张居正相仿,同期吃奶同期入学,所以每次当张居正写诗作文轰
动全境的时候,辽王他妈总要说上这么一句:

``你看人家张白圭多有出息,你再看你……''

被念叨了十多年,不仇恨一下那才有鬼。

但恨归恨,长大后的辽王发现,他还真不能把张居正怎么样。

\section[\thesection]{}

在很多电视剧里,王爷都是超级牛人,想干啥就干啥,抢个民女,鱼肉下百姓,那都是家常便饭。但在明代,这大
致就是做梦了。

自从朱棣造反成功后,藩王就成了朝廷防备的重点对象,不但收回了所有兵权,连他们的日常生活,都有地方政府
严密监视控制,比如辽王,他的活动范围仅限于荆州府,如果未经允许擅自外出,就有掉脑袋的危险。

说到底,这也就是个高级囚犯,想整张居正,谈何容易?

但仇恨的力量是强大的,当张居正洋洋得意,招摇过市的消息传到辽王耳朵里时,一个恶毒的计划形成了。

不久之后的一天夜里,护卫张镇被莫名其妙地叫进王府,然后又被莫名其妙地放了出来。中间发生过什么事情实在
无法考证,但结果十分清楚----回家不久就死去了。

这是一个疑点重重的死亡事件,种种迹象表明,张镇的死和辽王有着很大的关系,对此,张文明和张居正自然也清
楚,但问题在于,他们能怎样呢?

虽说藩王不受朝廷待见,但人家毕竟也姓朱,是皇亲国戚,别说你张神童、张秀才、张举人,哪怕你成了张进士,
张尚书,你还能整治王爷不成?

这就是辽王的如意算盘,我整死了你爷爷,你也只能干瞪眼,虽说手中无兵无权,但普天之下,能治我的只有皇
帝,你能奈我何?

张居正亲眼目睹了爷爷的悲惨离世,却只能号啕大哭悲痛欲绝,也就在此时,年轻的他第一次看到了一样东西----
特权。

所谓特权,就是当你在家酒足饭饱准备洗脚睡觉的时候,有人闯进来,拿走你的全部财产,放火烧了你的房子,把
洗脚水泼在你的头上,然后告诉你,这是他的权力,

这就是特权,在特权的面前,张居正才终于感觉到,他之前所得到的鲜花与赞扬是如此的毫无用处,那些游山玩水
附庸风雅的所谓名士,除了吟诵几首春花秋月外,屁用都没有。

荆州知府也好,湖广巡抚也罢,在辽王的面前,也就是一堆摆设,拥有特权的人,可以践踏一切道德规范,藐视所
有的法律法规,想干什么就干什么。而弱者,只能任人宰割。

\section[\thesection]{}

辽王不会想到,他的这次示威举动,却彻底地改变了张居正的一生,并把这个年轻人从睡梦中惊醒。正是在这次事
件中,张居正明白了特权的可怕与威势,他厌恶这种力量,却也向往它。

站在祖父的坟前,陷入沉思的张居正终于找到了唯一能够战胜辽王,战胜特权的方法----更大的特权。

我会回来的,总有一天,我会回来向你讨要所有的一切,让你承受比我更大的痛苦。

向金碧辉煌的辽王府投去了最后一瞥,紧握拳头的张居正踏上了赴京赶考的路,此时是嘉靖二十三年(1544),张居
正二十岁。

不管情绪上有多大变化,但对于自己的天赋,张举人还是很有信心的,他相信自己能够中第,然而现实再次给他上
了一课----名落孙山。

这是一个张居正无法接受却不能不接受的事实,他的所有骄傲与虚荣都已彻底失去,只能狼狈地回到家乡,苦读不
辍,等待下次机会。

嘉靖二十六年(1547),张居正再次赴京赶考,此时他的心中只剩下一个念头:考中就好,考中就好。

赵丽蓉大妈曾经说过:狂没有好处。这句话是有道理的,张居正不狂了,于是就中了,而且名次还不低,是二甲前
几名,考试之后便被选为庶吉士,进入了翰林院庶吉士培训班。

庶吉士培训班每三年开一次,并不稀奇,但嘉靖二十六年的这个班,却实在是个猛班,班主任是吏部侍郎兼翰林院
掌院学士徐阶,学员中除了张居正外,还有后来的内阁成员李春芳、殷士儋等一干猛人,可谓是豪华阵容。

正是在这个培训班里,张居正第一次认识了徐阶,虽然此时的徐阶已看准了张居正,并打算把他拉到自己门下,但
对于这位似乎过于热情的班主任,张居正却保持了相当的警惕,除了日常来往外,并无私交。

十分滑稽的是,张居正虽对徐阶不感冒,却比较喜欢严嵩,在当时的他看来,严大人六十高龄还奋战在第一线,且
精力充沛,神采奕奕,实在让人佩服得紧。

所以在此后的两年中,纵使夏言被杀,可怜的班主任徐阶被恶整,他也从未发出一言一语,表示同情。恰恰相反,
他倒是写了不少赞扬严嵩的文章,每逢生日还要搞点贺词送上去。

对此,徐阶也无可奈何,但他相信总有一天,这个年轻人能够体谅到他的一片苦心。

上天没有让他等得太久,嘉靖二十九年(1550),张居正与严嵩决裂。

\section[\thesection]{}

在这一年,``庚戍之变''爆发了,张居正眼看着蒙古兵来了又走,走了又来,放火又抢劫。严大人吃了又睡,睡了
又吃,就是不办事。

人不能无耻到这个地步,张居正愤怒了,对严嵩的幻想也随着城外的大火化为灰烬,他终于转向了徐阶。

此时徐阶的职务是礼部尚书兼内阁大学士,已经成为了朝廷的高级官员,在张居正看来,他是可以和严嵩干一仗
的,可几次进言,这位徐大人却只是笑而不言,对严嵩也百般依从,毫无反抗的行动。

难道你竟如此怯弱吗?张居正没有想到,自己寄以重望的老师,竟然是个和稀泥的货色,只顾权势地位,不敢挺身而
出。当然了,愤怒归愤怒,张居正自己也没有站出来,毕竟他此时只是一个七品翰林院编修,况且他也没有杨继盛
那样的胆子。

严嵩日复一日地乱来,徐阶日复一日地退让,张居正日复一日地郁闷,终于有一天,他无法忍受了,便作出了一个
改变他一生的决定----请病假。

在临走的时候,他给徐老师留下了一封信,痛斥了对方的和稀泥行径,其中有这样一段极为醒目的话:

古之匹夫尚有高论于天子之前者,今之宰相,竞不敢出一言,何则?!

从字面上理解,大致意思是:徐阶老师,你还不如匹夫!

看到信的徐阶却仍只是笑了笑:

小子,你还太嫩了。

天下,己任

嘉靖三十三年(1554),带着一腔愤懑,三十岁张愤青回到了家,说句实话,他选择这个时候回家,实在是再合适不
过了,因为此时朝廷正斗得你死我活,杨继盛拼死上书,严嵩大施淫威,徐阶左右逢源,一片腥风血雨,按照张居
正的那个性格,想不卷进去都难。

不搞政治,又没有其他娱乐方式,只好游山玩水了,于是在那三年之中,张居正游览了许多名胜古迹,从西子湖畔
到武当之巅,处处都留下了他的足迹,然而这一轮全国三年游不但没有舒缓他的心情,却使他发现了另一个问题。

原来人生可以如同地狱一般。在看过了无数百姓沿街乞讨,卖儿卖女,只求能够多吃一顿,多活一天的惨象后,张
居正发出了这样的长叹。

\section[\thesection]{}

从神童到秀才,再到举人、进士、翰林,纵使有着这样那样的不快,但张居正的一生还是比较顺利的,他不缺衣
食,有学上,有官当。

而直到他游历各地,亲眼目睹之后,才明白了这样几个真理,比如:一个人如果没有土地,就没有收入,没有收入
就没有食物,没有食物,就会开始变卖家产,从家具、房子到老婆,孩子,到了卖无可卖,就会去扒树皮,树皮扒
完了,就去吃观音土,而观音土无法消化,吃到最后,人就会死,死的时候肚子会胀得很高。

同时他还发现,原来这个世界上还有很多人不喜欢诗词书画,也没有那么多的忧伤哀愁,他们想要的只是一碗掺着
沙子的米饭,对那些骨瘦如柴、眼凹深陷的饥民而言,一幅字画是王羲之的还是怀素的,一点也不重要,重要的是
那张字画纸够不够厚,方不方便消化。

在看到那些倒毙在街头,无人理会也无人收拾的尸体时,他有时也会想,这些人生前是不是也有过妻子、丈夫、孩
子,是不是也曾有过一个欢笑的生活,一个幸福的家。

就在张居正为此痛心疾首之时,一个冤家却再次找上了门来。

这个人就是辽王,说起来,这实在是个缺心眼的家伙,听说张居正回来了,竟然主动找来,只为了一个目的----玩。

作为一个藩王,呆在荆州这么个小地方,平时又不能走远,只能搞点吃喝嫖赌,真是大大的没趣,所以在他看来,
张居正可谓是供消遣的最好人选。

这位仁兄还很健忘,他似乎不记得眼前这个玩伴的祖父曾被自己活活害死,而张居正则成为了玩具,被叫到王府,
陪这位公子哥每天饮酒做诗,强颜欢笑。

在那些屈辱的日子里,张居正默默忍受着这一切,与此同时,他又发现了这个世界的另一面:原来人生也可以如同
天堂一般。

比如这位辽王,含着金钥匙出生,丰衣足食却依然不知足,鱼肉着属地的百姓,想用就用,想拿就拿,他要做人,
百姓就得做牲口,他要潇洒地去活,百姓就要痛苦地去死。

每当张居正结束应酬,离开丰盛的酒席,走出金碧辉煌的王府门口时,总能看到饿得奄奄一息的饥民和无家可归只
能睡大街的流浪者。

原来天堂和地狱只有一墙之隔。

这就是大明天下的真相,当无数的贫民受到压榨,失去土地四处流浪的时候,高贵的大人们却正思考着明天去何处
游玩,该作一首什么样的诗。

这些在官员们看来并不稀奇的场景却深深地打动了张居正,因为他和大多数官员不同,他还有良心。

\section[\thesection]{}

面对着那些乞求和无助的眼神,面对着路旁冻饿而死的尸骨,张居正再次确立了他的志向,一个最终坚持到底的志
向----以天下为已任。

所谓以天下为已任,通俗点说就是把别人的事情当作自己的事情来办。地球人都知道,却似乎只有外星人办得到。

几百年前,一位叫亚当斯密的人在自己的家中写下了一本书,名叫《国富论》,在这本被誉为经济学史上最为伟大
的著作中,亚当同志为我们指出了这样一个真理----人天生,并将永远,是自私的动物。

只要回家照照镜子,你就会发现这个法则十分靠谱,试问有谁愿意为了一个素不相识的人去拼搏、奋斗,付出自己
的一切努力、心血乃至生命?顺便说一句,没准人家还不领你的情。

不是个傻子,也是精神病。相信这就是大多数人的回答,但问题在于,这样的人确实是存在的,他们甘愿牺牲自己
的一切,只是为了别人的利益。

而这个特殊的群体,我们通常称之为伟人,所以说伟人不是那么容易干的。

孔子应该算是众多伟人中的一位,他的一生都致力于寻求真理,普及教育,当然,他并不是一个所谓``不可救药的
乐观主义者'',他的言行自然也不是``心灵鸡汤''或``励志经典'',在我看来,他倒像是个``不可救药的悲观主义
者''。

他流浪数十年,周游四方,目睹了最为残酷的屠杀与破坏,但他依然选择了传道,把希望与知识传递给更多的人,
这无疑是一个伟大的行为,而他这样做的真正原因决不是乐观,而是----悲悯。

了解世界的黑暗与绝望,却从不放弃,并以悲天悯人之心去关怀所有不幸的人。

这才是伟人之所以成为伟人的真正原因,这才是人类最为崇高的道德与情感。

张居正就是这样一个伟人,他锦衣玉食,前途远大,不会受冻,更不会挨饿,他可以选择作一个安分守己的官僚,
熬资历混前途,最终名利双收。

然而和那位骑着摩托车横跨南美洲的格瓦拉医生一样,在见识了世上的不公与丑陋后,他选择了另一条道路,一条
无比艰苦,却无比光辉的道路。

\section[\thesection]{}

在黑暗之中,张居正接过了前人的火把,成为了又一个以天下为己任的人。

所以我相信,即使这个世界十分阴晦,十分邪恶,即使它让你痛不欲生,生不如死,但依然应该鼓起勇气,勇敢地
活下去。

所以我相信,希望是不会死去的。

天赋,无与伦比

嘉靖三十六年(1557),张居正回到了北京,此时的他已经脱胎换骨,他知道自己要做什么,也知道该如何去做。

如果单以智商而论,嘉靖年间的第一聪明人应该还轮不上徐阶,因为从实际表现上看,张居正比他还要厉害得多。

在那年头,想在朝廷混碗饭吃实在不易,为了生存,徐阶装了二十多年孙子,还要多方讨好妥协,而张居正的表现
却让所有的人大吃一惊。

这位年轻人虽然刚刚三十出头,且在不久之前还是个标准愤青,但在短短几年之间,他已经变成了一个喜怒不形于
色,城府深不可测的政坛高手。当时徐阶已与严嵩公开对立,除了个把胆子大的,没人敢与徐阶公开接触,唯恐被
严党当作敌人干掉。即使像吴时来、邹应龙这样的死党,每次找徐阶都是趁着夜里,悄悄地进府,打枪的不要。

唯一的例外就是张居正,他总是白天来,还喜欢坐官轿,高声通报,似乎唯恐人家不知道他和徐阶的关系,甚至在
朝堂上,他也敢公开和徐阶交头接耳。

而更为奇怪的是,对于这一幕幕景象,严嵩及其党羽却不感到丝毫奇怪,也不把他当作对手,因为张居正和他们这
边的关系也不错,虽然没有深交,却也经常走动。

即使在我们普通人看来,张居正的行为也无疑是典型的两面派,但在当时,连精得脑袋冒烟的严嵩都认为,这位张
翰林是一个光明磊落的人,从不结党,坦坦荡荡。

明明是徐党,明明是耍手段,那么多人都看着,就是看不穿。在长达四十余年的嘉靖朝中,这是最让人莫名其妙的
一幕。

而对此怪象,唯一合理的解释是:张居正是个超级能人。在他的身上,有着一种可怕的政治天赋。即使在最为险恶
的政治环境中,他也能够进退自如,在交战双方的枪林弹雨中游刃有余,如此绝技,估计连国际红十字会也望尘莫
及。

所以在那几年里,虽然外面你死我活,血流成河,张居正却稳如泰山,安然无恙。

可你要是由此认为他安分守己,那就错了.

\section[\thesection]{}

在徐党中,张居正大概是最为激进的一个,经常在徐阶面前喊打喊杀,大有与严嵩不共戴天的气势。

然而徐阶只是微笑,他安排吴时来、董传策、张翀试探严嵩,命令邹应龙弹劾严世藩,但张居正这颗棋子,他却从
未动过。因为他很清楚,这是一个非同寻常的人,而现在,还不是让他上场的时候。

事实上,张居正不但没有出场的机会,连官都升得慢,嘉靖二十六年的进士,一转眼都十多年了,还是个正七品编
修,连杨继盛都不如。

对此张居正也想不通,怎么说自己跟的也是朝廷的第二号人物,进步得如此之慢,实在有点说不过去。

但当两年之后,他听到那道任职命令之时,所有的抱怨顿时烟消云散,他终于知道了徐阶的良苦用心。

嘉靖三十九年(1560),翰林院编修张居正因工作勤奋努力,考核优异,升任右春坊右中允,兼管国子监司业。

右春坊右中允和国子监司业都是六品官,看上去无足轻重,也不起眼,但事实绝非如此:

右春坊右中允的主要职责是管理太子的来往公文,以及为太子提供文书帮助,而国子监司业大致相当于中央大学的
副校长,仅次于校长(祭酒)。

现在明白了吧,成了右中允,就能整理太子的文件,就能和太子拉上关系,这叫找背景。当上中央大学的副校长,
所有的国子监学员都成了你的门生,这叫拉帮派。要知道,蒋介石就最喜欢别人叫他校长,那不是没有道理滴。

况且这两个职务品级不高,也不惹人注意,没有成为靶子的危险,还能锻炼才干,对于暂时不宜暴露的指定接班人
来说,实在是再合适不过了。

算盘精到这个份上,徐阶兄,我服了你!

但天衣无缝总是不可能的,顺便说一句,当时的国子监校长恰好就是高拱,而这一巧合将在不久之后,给徐阶带来
极大的麻烦。

徐阶对张居正实在是太好了,好得没了谱,嘉靖三十九年,徐阶与严嵩的斗争已经到了生死关头,双方各出奇招,
只要是个人,还能用,基本都拉出去了,但无论局势多么紧张,作为徐阶最得意的门生,张居正却始终没有上阵,
只是安心整理公文,教他的学生,

照这个势头看,即使要去炸碉堡,徐大人也会自己扛炸药包。

而这一切,张居正都牢牢地记在心里,他知道徐阶对自己的期望。

\section[\thesection]{}

严嵩终究还是倒了,倒在比他更聪明的徐阶脚下,于是张居正的前途更加光明了,嘉靖四十三年(1564),他被提升为
右春坊右谕德。

右谕德是从五品,也就是说张居正在四年之间,只提了半级,然而当他听到这个任命的时候,高兴得差点跳起来,
因为这个右谕德的唯一工作,就是担任裕王的讲官。

裕王跟徐阶从来就不是一条线,能把张居正安插进去,那实在是费了九牛二虎之力。

就这样,张居正进入了裕王府,成为了裕王的四大讲官之一,说来有趣,其他三位都是他的老熟人,他们分别是:
当年的老同事高拱,当年的老同学殷士儋,还有当年的老师陈以勤(高考时是他批的卷)。

这四位讲官就此开始了朝夕相处的教学生活,在不久的将来,他们将成为帝国政坛的风云人物。

徐阶本打算让张居正再多磨砺几年,到时再入阁接班,但现在情况发生了变化,由于自己的错误判断,高拱已然占
据了优势,必须提前开始行动了。

但当徐阶准备收获自己栽培了十几年的庄稼时,意外发生了。

他惊奇地发现,在张居正这块自留地上,竟然长出了杂草。

杂草的名字叫做高拱。

高拱这个人人如其名,性格高傲且极其难拱,与他同朝为官的人很少能成为他的朋友,因为他不但自负才高,且常
常藐视同事和上级,动不动就是一句:你们这帮蠢……

或许你会奇怪,这人自己不蠢吗,群众基础如此之差,怎么还能升官?我告诉你,高先生可不蠢,你要知道,他虽然
瞧不起上级同事,却很尊重老板(皇帝)。经常写青词送给嘉靖,且文辞优美,当时的大臣们公认,他写这种马屁文
章的水平可排第二(第一名是状元李春芳),徐阶都要靠边站。

更何况,他手里还捏着一个裕王,有如此雄厚的资源,鄙视也罢,骂也罢,你能怎样?

所以他的朋友很少,郭朴算一个,张居正也算一个。

郭朴是他的同乡兼战友,就不多说了,而张居正之所以能成为他的朋友,完全是靠实力。

\section[\thesection]{}

高拱曾经对人说过,满朝文武,除叔大(张居正字叔大)外,尽为无能之辈。

刚到国子监的时候,高拱对自己的这位副手十分不以为然,把张居正当下人使唤,呼来喝去,人家到底是个副校
长,这要换了个人,估计早就闹起来了。

然而张居正一声不吭,只是埋头做事,短短几个月,就把原先无人问津的国子监搞得有声有色。高拱就此对他刮目
相看。

几年之后,当两人以裕王讲官身份重逢的时候,高拱已经彻底了解了这个人的学识和器量,于是他第一次放下了架
子,每次见到张居正,居然会主动行礼,而且经常找他聊天,交流思想。

久而久之,两人成了要好的朋友,还经常一起相约出去游玩。正是在那次郊游之中,高拱向张居正袒露了自己内心
的秘密。

在那个阳光明媚的清晨,屹立在晨风之中的高拱面对着眼前的江山秀色,感慨万千,对站在身边的张居正说出了这
样一句话:

``以君之材,必成大器,我愿与君共勉,将来入阁为相,匡扶社稷,建立千秋不朽之功业!''

张居正目不转睛地盯着眼前这个意气风发的人,然后他走上前去,面对这位志同道合的战友,坚定地点了点头。

是的,这也正是我的目标。

在那一刻,五十二岁的高拱与三十九岁的张居正结成了联盟,一个雄心万丈,于危难中力挽狂澜、建功立业的志向
就此立下。

天下英雄,尽出于我辈!

老谋深算的徐阶很快就发觉了两人之间的关系,他知道,要指望张居正一边倒,帮他打击高拱,已经不可能了。但
高拱在内阁中气焰日渐嚣张,一时之间他也想不出更好的方法。

就在他苦苦思索对策的时候,一个意外事件发生了,遗憾的是,对徐阶而言,这实在不是一件好事。

事情是这样的,在当时的朝廷里,有一个叫胡应嘉的言官,话说这位仁兄有一天闲来无事,便干起了本职工作--弹
劾,这次他选中的目标是工部副部长李登云。

\section[\thesection]{}

他的本意其实只是骂骂人而已,可问题是他的弹章写得实在太好,没过几天,消息传来,李登云被勒令退休了。

这下子胡应嘉懵了,虽说一篇文章搞倒了一个副部长,也算颇有成就,但问题在于,这位李登云有个亲家,名叫高
拱。

完喽,胡应嘉同志这下麻烦了,得罪了高拱,迟早吃不了兜着走,而且他还由路边社得知,高拱大人对此事极为恼
火,准备收拾他。

无奈之下,胡应嘉决定铤而走险,与其坐以待毙,不如奋力一博。他开始打探消息,准备先下手为强。

很快,他就得知了这样三个消息:首先,嘉靖最近得了重病,身体很不好。

其次,高拱搬了家,住到了西安门。

最后,高拱曾把自己西苑值班房的一些私人物品搬回了家,还经常回家住。

这三个情况看上去毫无关系,也无异常,但杀人的血刀却正隐藏其中,胡应嘉灵机一动,想出了一个极为毒辣的计
策,并随即挥毫泼墨,写下了一封弹章。

我曾整理过明代言官的奏疏,看过不下百封的弹章,骂法各异,精彩纷呈,但要论阴险毒辣之最,那还要算是胡应
嘉的这封大作,百年后读来仍让人毛骨悚然,冷风刺骨。

``臣吏科给事中胡应嘉上奏,礼部尚书,文渊阁大学士高拱身受陛下大恩,却于皇上病重之时脱离职守,擅自回
家,并将其值庐(即值班房)内的物品尽数搬回家中,臣实不知其有何用心?!''

毒,实在太毒了,要知道,嘉靖这一辈子最怕的就是大臣另有所图,当年徐阶提议立太子,都差点被他给废了,现
在正值病重之时,高拱就开始收拾行李了,这不摆明了是要另起炉灶吗? 

按照嘉靖的性格,如无意外,他看到这封弹章之日,即是高拱毙命之时。而这条毒计更为阴险的地方在于,胡应嘉
已经看透了高拱与徐阶的矛盾,他知道,一旦此文上传内阁,挑起战火,高拱必定认为是徐阶所主使,到时全面开
战,这个黑锅就可以转嫁给徐阶,没准还能得到他的赏识。

顺便提一下,胡应嘉是徐阶的老乡。

这是一个几近完美的一石三鸟之计,胡应嘉布置完毕,便得意洋洋地等待着高拱的死讯,却没有想到他疏忽了一个
最为重要的问题----病人是容易被激怒的,但要是病到一定程度,想怒也怒不了了。

\section[\thesection]{}

此时的嘉靖同志已经病入膏肓了,躺在床上奄奄一息,就等着去阎王那里报到,哪里还有精力去看胡应嘉的弹章?
于是胡言官这份饱含杀人热情的文书就落入了高拱的手中。

当高拱看完这份奏疏之后,顿如五雷轰顶,冷汗直冒,他大为恼火,当即认定这是徐阶的阴谋,公开表示与首辅大
人势不两立,并连夜找到郭朴,商量反击的对策。

内阁里被人排挤,张居正被人插足,现在又多了个胡应嘉,徐首辅恨不得去撞南墙,就在他焦头烂额之际,另一个
惊天动地的消息传来:

嘉靖死了

终于还是死了,死并不奇怪,这么晚才死,那才是怪事。

要知道,这位仁兄几十年如一日,把有限的精力投入到无限的修道中去,并以大无畏的精神亲身品尝了据说吃了能
长生不老的新型药品----金丹,据分析,其主要成分包括金(Au)、银(Ag)、汞(Hg)以及多种重金属,矿物质。

嘉靖是个好同志,就这么些玩意,他一吃就是四十年,且毫无怨言,而他竟然还是坚强地活到了六十岁,奇迹,真
是奇迹。

说实话,对于这位仁兄,我并不感冒,但没有办法,他当政四十余年,手下能人辈出,怪事频发,不写也实在说不
过去,而回过头来,看看这位天才皇帝的一生,实在令人感慨。

嘉靖是个聪明人,十六岁就能控制朝政,操纵群臣,而他的下属大都能力超强,文臣夏言、徐阶、胡宗宪全都权谋
老到,武将戚继光、俞大猷、谭纶个个凶狠强悍,可谓是人才济济。

然而国家却变成了这样一幅样子,正如海瑞所说,百姓穷困潦倒,家家干净,官场腐败横行贪诈成性,国家入不敷
出,年年闹赤字,大明帝国逐渐滑向崩溃的边缘。

出现如此之怪象,只是因为两个字----自私。

嘉靖很自私,他认为做皇帝就是来享福的,没有义务,只有权利,而为了享受,就必须分裂群臣,让他们斗来斗
去,自己的地位才能稳固。为了享受,就必须修道,这样才能活得更长。至于国计民生,鬼才去管。

总之一句话,在我死后,哪怕洪水滔天!

太上大罗天仙紫极长生圣智统三元证应玉虚总掌五雷大真人玄都境万寿帝君朱厚熜----还是死了。

不过如此

所以对他的死,也只有一个字可形容:

该

\section[\thesection]{}

不朽

在嘉靖崩掉的那一夜,第一个接到死讯的人,是徐阶。

当然,你要指望他号啕大哭,痛不欲生,那是不太现实的,但听到这个消息后,徐阶确实沉默了,并非默哀,只是
因为几十年的政治经验告诉他,一个千载难逢的反败为胜之机已经出现,就在这个死人的身上。

他立刻下达了命令:

``把张居正叫来!''

此时的张居正只是一个翰林院学士,还不是内阁成员,自然也没有值班的义务,所以当他从热被窝里被人叫出来,
顶着北京十二月的寒风跑进宫时,还是一头雾水。

徐阶告诉他,皇帝死了。张居正却极为平静,不置可否。

死就死了吧,又不是我爹,有啥好激动的。

但他还是激动了,因为徐阶又说了一句话:

``要写一道遗诏,我来拟,你来写。''

张居正的手发抖了,因为兴奋而发抖。

在明代,皇帝活着的时候可能发布过无数文件,但最重要的一份却是他死后的遗诏,因为这是他一生的总结,而国
家的大政方针也将在这封文书中被确定。

而遗诏最关键的秘密在于,它根本就不是皇帝本人的遗嘱,却是由大臣代写的,所以大多数遗诏都被写成了检讨
书,把自己骂得狗血淋头,连街头混混都不如的也不在少数。反正您已经死了,还能爬起来算账不成?

遗诏在手,天下在握。

所以能参与这份历史性文件的草拟,张居正极为兴奋,他知道按照规定,自己这个五品翰林院学士根本没有动笔的
资格,但现在,他坐在桌前,手握着毛笔,和千千万万天下人的命运。

他抬起头,向站在身边忙着沉思造句的徐阶投去了感激的一瞥。

但他并不知道,当他埋头写作之时,徐阶也曾反复审视着他,眼光中充满了得意。

太好了,一切都在掌握之中。

这是徐阶政治生涯中最为精彩的一招,也是他政治智慧最为辉煌的闪光。

在这个夜里,他露出了自己的真面目,将积蓄了二十多年的怒火全部发泄,彻底否定了几十年胡搞乱搞的嘉靖,痛
斥他的乱政怠政,当然,从程序上看,这些话都是嘉靖同志自己说的,怪不得别人。

这就是明代历史上著名的《嘉靖遗诏》,据说全文刊出后,举国欢腾,许多文人纷纷写诗讴歌此文,个别不地道
的,竟敢在大丧期间放鞭炮庆祝,皇帝干到这个份上,失败,太失败了。

凭借着这封遗诏(作者大家心里有数),徐阶的威望达到了顶点,权势也如日中天,高拱的气焰被打压了下去。但事
实上,在那个夜晚,这封遗诏并不是徐阶最为得意的成就。

\section[\thesection]{}

真正的收获是张居正。

天真的张居正并不知道,当他提起笔,写下第一个字的那一刻,他与高拱已经彻底决裂。

正是因为遗诏极为重要,所以根据惯例,其拟定必须由内阁大臣共同商议决定,但在那天夜晚,到达现场的人,却
只有一个徐阶,高拱、郭朴、李春芳都不知道,统统被放了鸽子,这是大忌中的大忌。

李春芳是个老实人,也就算了,高拱和郭朴却不是好打发的。竟敢背着我们吃独食?饶不了你!

不久高拱得知,与徐阶一同草拟文件的还有一个人,而此人竟然就是张居正。张居正是什么级别?凭什么拟遗诏!

他大吃一惊,又怒不可遏,一颗仇恨的种子就此埋下,从此以后,张居正不再是他的朋友和伙伴。而对于张居正而
言,在老师和朋友之间,他只剩下了一个选择。

姜还是老的辣,狐狸还是老的精。

一天之后,京城监狱的看守得知了嘉靖的死讯,他们商议了一下,便开始分配工作,买菜的买菜,买肉的买肉,做
了一顿丰盛的饭菜,然后请牢里的一位犯人吃饭。

这个犯人名叫做海瑞。

自从骂完皇帝,海瑞先生的名气是一天大过一天了,无数官员把他当作榜样,有些老百姓甚至把他的相挂在家里,
早请示晚汇报,成了不折不扣的偶像级人物。

现在皇帝死了,以海瑞的名头,自然是无罪开释,加官进爵,看守们也想求个进步,便打算投个机,请海大人吃一
顿,将来也好有个照应。

饭菜送到牢房里,海瑞一看,有鱼有肉,再一算,太上老君的生日还差得远,自己的生日更不靠谱,明白了,这是
断头饭。

所谓断头饭,就是杀头前吃的饭,一般说来都还不错,咱中国人仁义,坚决不给阎王增加负担,保证不让一个饿死
鬼去报到。当然了,这顿饭一般人都吃不下去----心理压力太大。

可要搁到海瑞身上,那就是两说了,海猛人二话不说,提起筷子就刨,狼吞虎咽,吃完了还要添,等到盘子能够照
出人影,他终于吃完了。

然后他坐了下来,看着看守,那意思是我吃完了,你们怎么还不动手。

看守被他那种找死的眼神看得发毛,便小心翼翼地对他说:

``海先生,你还不知道吧,皇上已经驾崩了,您很快就能出去了。''

接下来发生的事情被写进了大大小小的史书,堪称史上之奇观。

\section[\thesection]{}

在听到这句话后,海瑞呆了一会,然后突然大哭起来,哭得撕心裂肺,哭到喘不过去,然后就开始吐,先吐这顿
的,再吐上顿的,最后是黄胆水。

看守呆住了,他不知道这是怎么回事,吓得魂不附体,紧紧贴着墙壁,一动也不敢动。

海瑞是真哭,嘉靖死了,他很悲伤,说来真有点讽刺意味,嘉靖信任严嵩、信任徐阶,给了他们高官厚禄,结果一
个把他当工具,一个把他当傀儡,唯一为他的死而感到悲哀的人,竟然是那个痛骂过他,又被他关进监狱的海瑞。

嘉靖,原来你竟如此的孤独。

而对于海瑞的这一表现,大致有两种不同的评价,捧他的人刻意回避,压根不提,骂他的人说这是他愚昧与盲从的
集中体现。

记得在我小时候,曾经看过一套连环画《说岳全传》,算是我的历史启蒙教材,在每本连环画的前言部分,会介绍
本集故事梗概,但无论是哪一集,下面总会有这样一句话:请读者注意,岳飞的行为是封建忠君思想的体现,应该
予以批判。

我个人觉得,这是一句相当无耻的话。

封建社会嘛,又没有民主推荐、差额选举,除非你自立门户,不然除了忠君还能忠谁,难不成去信上帝?

在封建时代,就做封建时代的事,说封建时代的话,别指望人家有多高的觉悟,这就叫历史唯物主义。

海瑞没有看过孟德斯鸠和卢梭的书,嘉靖活着的时候,海瑞骂他,是尽本分,嘉靖死的时候,海瑞哭他,也是尽本
分。

本分,本分而已。

但哭是哭不死的,哭完了还得活,不出看守们所料,海瑞很快就被释放了,几年之后,他将再次出山,并闹出更大
的事情。

十天之后,全天下的人都知道了皇帝的死讯,这其中也包括湖广蕲州(今湖北蕲春)的一个平民,对于这个消息,他
表现得十分平静,因为十几年前,当他见到尚在壮年的嘉靖时,就已经料定,这位嗑药的皇帝是撑不了多久的。

他无奈地摇了摇头,然后回到简陋的小屋里,继续写他的那本书。

三十年多前,作为一个想要考取功名的秀才,他曾三次参加乡试,不过运气不太好,总是考不上,于是一气之下,
便干起了父亲的老本行。

虽说名落孙山总是一件悲痛的事,但这个人的落榜实在值得全人类放鞭炮庆祝,因为他的名字叫李时珍。

\section[\thesection]{}

事实上,李时珍原本不想做医生,因为他的父亲虽是当地的名医,家里也有点钱,但在那年头,四书五经才是正
道,医学算是杂学,那么医生就是杂人了。

杂人自然是不受待见的,有钱又如何,就是瞧不起你!所以李时珍的老爹千叮咛万嘱咐,将来千万不能从医。

李时珍是听话的,但就是考不上,你有什么办法?更为麻烦的是,二十岁的时候,他还染上了一种极为难治的肺病,
百般折腾,死去活来,才算保住了一条命。

于是不久之后,久病初愈的他找到了父亲,只说了一句话:

``我不考了,请将医术传给我。''

父亲想了一下,点头同意了。

我所经历的痛苦与折磨,不想再让别人承受。

在我看来,这大致就是李时珍的行医动机。

虽说读书不在行,但摆弄药材,李时珍还是很有点天赋的,而随着年龄的增长,他见过的怪病越来越多,经验越来
越丰富,医术也越来越高。

这么看来,现在医院里五六十岁的老头老太太坐镇门诊,二三十岁的医生只能坐在一旁打苍蝇,也实在不是没有来
由的,医术如何暂且不说,人家毕竟多活了几十年,没有功劳也有苦劳。

但李时珍明显不是一个具备现代观念的医生,一点潮流意识都没有,他给穷人看病,竟敢不收上百万的医疗费,竟
敢热情问诊嘘寒问暖,竟敢免除所有的检验费、治疗费,实在是``罪大恶极''!

行医十几年,不计成本,只求救人,李时珍就这么坚持了下来,他的积蓄越来越少,名声却越来越大。

于是到了嘉靖三十年(1551),他迎来了人生的一场大变,在这一年,几个人找到他,十分客气地把他请到了楚王
府,希望他担任楚王的私人医生。

能吃饱饭,还有无数的医书和药材资源,李时珍不是傻瓜,他答应了。

在楚王府,李时珍干得很不错,治好了很多人,被称为神医,名震天下。

好东西人人都想要,尤其是嘉靖这样的人,所以在听说李时珍的大名后,他便告诉楚王,你另外找一个医生,把这
个给我送过来。

就这样,李时珍进入了太医院,并见到了大明帝国最高级的病人嘉靖。

其实能进入太医院,李时珍是很高兴的,能做到太医,也算是医生中的成功人士了,不得意一下,实在也说不过去。

但没过多久,他就想走了。

\section[\thesection]{}

具体原因并不像许多书上所说的那样,什么嫉恶如仇、厌恶庸医等等,李时珍不是海瑞,走南闯北混了那么多年,
场面上的事情还是过得去的,他之所以要走,实在是因为力不从心。

李时珍是神医,在那个年头,只要不是天花、肺结核之类的绝症,他基本上都能搞定,可问题在于,他那位唯一的
病人是没病找病。

嘉靖其实身体很好,只要能够坚持锻炼,每天早上跑跑步打打太极拳,活个七八十岁应该不成问题,可他的目标过
于远大,七八十?至少也要活个七八百才够本。

于是他开始没事找抽,日复一日地吃重金属和水银,还美其名曰金丹,李时珍倒是劝过他,也想帮他,却毫无用处。

这实在怪不得李时珍,因为要从科学门类来分,嘉靖同志弄的这一套应该算是有机化学,隔行如隔山,李医生当年
也没搞过化学,只能爱莫能助了。

太医院别的没有,医书和药材是不缺的,于是嘉靖接着磕他的药,李时珍接着搞他的研究,直到有一天,他认为自
己已经学不到更多东西了,便打起背包,收拾资料,离开了这个他曾无限向往的地方。

嘉靖三十一年(1552),李时珍回到了民间,这一年他三十五岁,见过最穷的贫民,也看过最富的天子,到过寒酸的
茅舍,也走过金銮大殿,人世间的富贵、疾苦他已了然于胸。

鲁迅先生曾经说过,他爹就是被一堆奇形怪状的药材给治死的,在表示哀悼的同时,我们有理由相信,在李时珍的
那个年代,患了感冒开给你几剂砒霜应该也不是什么稀奇的事。

没办法,咱中国地大物博,药材植物也多,到底哪种东西治什么病,谁都搞不清楚,被乱治胡吃搞死的人,也只有
阎王才能数得清。

忆往昔,他此起彼伏,于是他决定写一本书,写一本囊括所有植物药材以及正确用法的书。

这本书的名字叫做《本草纲目》。

从嘉靖三十一年(1552)起,李时珍开始写这本书,要知道,医书不是小说,你不但要写出药用植物的形状、外貌,
还要详细描述它的特点、疗效。坐在家里胡编乱造,是一个字也写不出来的。

所以从决定写书的那一天起,李时珍便开始了另一种生活----奇特而艰苦的生活。

\section[\thesection]{}

作为曾经在太医院干过的医生,此时的李时珍已经成为了传奇人物,来找他看病的人络绎不绝,医术且不说,想想
当初这人给皇帝都号过脉,那就是御医,说起来咱这辈子还看过御医,也够吹个三五十年的。

名声大了,收入自然也高了,李时珍就算闭着眼睛号脉,混个百万富翁也绝不成问题。然而他默默地收拾行囊,开
始远行,足迹踏遍了全国十三省,无论是名山大川,还是悬崖峭壁,凡是有药材的地方,就有他的踪影,为了弄清
药物的疗效,他曾亲自品尝过许多药材植物,好几次差点植物中毒,一命呜呼。

为了写这本书,李时珍从一个名医变成了流浪汉,他居无定所,风餐露宿,他放弃了舒适的生活,放弃了宽敞的诊
所,也放弃了自己唾手可得的幸福生活。

但他依然执著地写了下去----为了更多人的幸福。

从嘉靖三十一年(1552)开始,历经二十六年,李时珍走遍了全国各地,尝遍了无数植物药材,查遍了世上的所有医
书,最终完成了这部中国历史上最为伟大的医学著作。

《本草纲目》共计十六部,五十余卷,全书记载药物一千九百余种,还详细记载了这些药物的采集、制作、特性、
治疗病症,并全部附有手绘插图(佩服),此外书中还收入经检验有效的方剂一万一千多则。

李时珍于万历二十一年(1593)去世,他没有能够看到此书的出版。

三年后,《本草纲目》正式印刷发行,很快脱销,并迅速传入日本、朝鲜以及东南亚一带,几十年后又传入欧洲、
北美,并被翻译成几十种文字,成为世界医学史上的权威书籍,而李时珍也得以超越嘉靖、徐阶、张居正,成为被
世界公认的伟大科学家。

而对于《本草纲目》的意义,其实不需要用它的传播范围以及受到的夸赞加以肯定,我们只要知道,它的出现已经
拯救了无数人的生命,直到现在仍然继续,这就够了。

鲁迅先生除了痛斥庸医外,自己也当过医生,当然,之后他又不干了,原因大家在课本里都学过,他觉得医人无
用,``启发民智''才是正道。

对于这个判断,自然不能说错,但凑巧的是,我看过一个类似的故事。

\section[\thesection]{}

在很久以前(具体多久我也不知道),有一个医生,这位医生的医术很高明,很多人来找他看病。

当时恰逢战乱,打得你死我活,敌对双方的受伤士兵都来找他治疗,他来者不拒,悉心照料使他们很快康复。

很快,他就惊奇地发现,原先治好的人竟然又负伤了,还是来找他,没办法,战争年代刀剑无眼,其实我们也不想
光荣负伤,您受累了。

看起来这场仗时间很长,不断有新伤员来找他,但让人高兴的是,老伤员似乎越来越少----战死了就不用治伤了。

如此周而复始,他终于崩溃了,我治好了他们,他们又去打,然后又负伤,我再去医治,这样做有什么意义!

于是他丢掉了药箱,远离了诊所,跑到山区隐居起来。

但没过多久,人们惊奇地发现,他又回到了诊所,照旧开始医治那些负伤的士兵。

于是有人问他:

``为什么你会回来医治这些人?''

他笑着回答:

``因为我本就是个医生啊!''

这就是最终的答案。

无论徐阶是否斗倒了严嵩,无论张居正是不是一个杰出的改革家,都不关李时珍的事,他只是一个医生,他知道,
生命很珍贵,也很柔弱,作为一个医生,有责任和义务去维护生命的存在。

这就是明代医生李时珍的觉悟,以及他抛弃荣华富贵,历经困苦三十年著书救人的唯一动机与目的。

在我被吸收为医学事业中的一员时,我严肃地保证将我的一生奉献于为人类服务。

我将用我的良心和尊严来行使我的职业。我的病人的健康将是我首先考虑的。我将尊重病人所交给我的秘密。我将
极尽所能来保持医学职业的荣誉和可贵的传统。我的同道均是我的兄弟。

我不允许宗教、国籍、政治派别或地位来干扰我的职责和我与病人之间的关系。

我对人的生命,从其孕育之始,就保持最高的尊重,即使在威胁下,我决不将我的医学知识用于违反人道主义规范
的事情。

我出自内心和以我的荣誉,庄严地作此保证。

----1948年医学日内瓦宣言

我知道,李时珍没有读过这一段宣言,但他做到了。

他告诉我们,最伟大的人是没有派系的,最伟大的爱是没有分别的。

所以,在我国漫长的你死我活斗争史中,我写下了这一节,并以不朽命名,以纪念这个医生,这个超越信仰与差
别,以一己之力挽救无数人生命的伟大人物。

伟大的李时珍医生永垂不朽!

\section[\thesection]{}

禁书

与上一节不同,这一节我考虑了很久才落笔,按说嘉靖都死了,追悼会也办完了,事情就完了,该他儿子出场了。

如果还要接着搞总结,相信会有人说我罗嗦,天地良心,我从来不管小事,问题不闹得天翻地覆,鬼哭神嚎,是断
然不会被写下来的,而这嘉靖年间的最终问题,如果不写,实在是对不起那几位光辉人物,于是我毅然决定,把这
个最后的问题写完。

嘉靖年间是个多事的时代,嘉靖本人复杂,连带着他的大臣、子民跟着一起复杂,什么事都有,什么人都出,忠臣、
奸臣、骂臣、海盗、汉奸、英雄、还有日本、葡萄牙、西班牙等多国友人进来掺和,不热闹是不可能了。

对了,还漏了一个,文人。

嘉靖这四十五年是一个争议很大的时期,有人说是嘉靖中兴,也有人说是亡于嘉靖,但有一点是大家都不否认
的----灿烂的文化。

除了杨慎、王世贞、徐渭等人的诗词书画外,更值得人们骄傲的是,在这个时期前后,伟大的明代四大名著已经全
部诞生,并得以发扬光大,它们分别是《水浒传》、《三国演义》、《西游记》以及《金瓶梅》。

由于《水浒传》和《三国演义》的作者是老熟人,所以成书年代也差不多(明初),而到嘉靖年间,由于市民文化普
及,这两本书已经家喻户晓,得到了广泛的流传。

至于《西游记》,我们目前得知的是,其作者为吴承恩,江苏淮安人,其它情况不是不详,就是存在争议,吴先生
就如同孙猴子一样,神出鬼没,难以捉摸。

而《金瓶梅》,应该是争议最多的一本书了,连成书时间都存在争议,不过大抵也就是嘉靖后期到万历之前的这一
段,跑不了多远。

但有一点是可以肯定的:《金瓶梅》是一本具有伟大意义的杰出著作,它应该被堂而皇之地与其他三本书摆在一
起,被后人顶礼膜拜。

\section[\thesection]{}

《金瓶梅》的作者以其精湛的笔法,深刻的思想,勾勒出了西门庆、潘金莲等知名人物(拜水浒所赐)的形象,并以
这些鲜活的人物描述了明代中期的市民生活、被冲击得千疮百孔的封建礼教,以及不可遏制的思想解放与性解放潮
流(拜王守仁心学所赐)。

即使从文学体裁上讲,它也是杰出的,连一些红学家也认为,《红楼梦》关于人物日常生活的写作,是承继自《金
瓶梅》的。

疑问最多的,大概就是此文的作者了,那个所谓的``兰陵笑笑生''如果要列出嫌疑名单,是可以另写一本书的,其
实作者不留名倒也可以理解,毕竟这书里还有些不堪入目的东西(专用名词:糟粕),咱们到底是礼仪之邦,有些事
情上不得台面,写了这么个玩意,总还是有点不良影响,要顾及脸面。

而王世贞之所以被确定为重点作者嫌疑人,说起来还和严世藩先生有着莫大的关系。因为很多人认为,金瓶梅中的
这位西门庆是有原型的,而原型就是严世藩。

其实就生活腐化而论,西门庆和严世藩压根就不是一个档次的,西门庆的老婆说起来也就潘金莲那么几位,严世藩
那就多了去了,基本都是两位数起算,要谈贪污的钱财数目,更是无从说起,西门大官人才什么级别,严侍郎可不
是吃素的。

当然,说他们两人有关系,那也不是凭空讲白话,人家还是有证据的,比如严世藩同志又叫东楼,东楼和西门似乎
还对得上,再比如严世藩同志有个小名,叫做庆儿,这种类似猜谜类的玩意数不胜数,就不多说了。

而王世贞之所以被扣上这个帽子,实在是因为他和严嵩有仇,且名声太大,文章写得太好,大家觉得如此精彩的一
本小说,不是寻常村夫或是文学青年能写出来的,思来想去,就是他了。

当然现在也有许多人说王世贞不是作者,并列举了很多证据,我不搞考证,也就不写了。

不管有多少争议,但至少我们知道,明代曾有过怎样辉煌的文化,伟大的四大名著自诞生之日起,便已成为了经
典,此后的五百年中,除了一部《红楼梦》,无人可望其项背,不知道后面那帮人都干嘛去了。

但还有一点必须说明,那就是在当时,四大名著之中,有一本是禁书,如果藏有此书,是要惹麻烦的。

我大致知道许多人的答案,但我要告诉你们,不是那一本。

真正被禁的,是《西游记》。

\section[\thesection]{}

如果你还记得,书中有这样一个情节,唐僧师徒四人曾经到过一个叫车迟国的地方,那个地方的皇帝推崇道教,迷
信成仙,还搞出了几个虎力大师之类的邪门道士,最后被孙猴子一顿收拾,见阎王去了。

说到这里,你应该明白为什么它会被禁了,这种骂人不吐脏字的把戏历来就不少见。还有那句著名的``皇帝轮流
坐,明年到我家'',除了孙猴子外,估计也没人敢说。

总而言之,那是一个痛并快乐着的时代,至少我认为如此。

明穆宗朱载垕

公元1566年,朱载垕继位了,年号隆庆,他等了二十多年,终于等死了自己的老子,等到了皇位。

这位仁兄能混到这个位置实在不易,因为他是奉遗诏登基的,遗诏是怎么回事前面已经说过了,嘉靖忽悠了儿子那
么多年,临死也没说句接班的话。

不管怎样,毕竟已经是皇上了,隆庆开始召集大臣们上朝。

被嘉靖冷落了那么多年,终于有了发言的机会,大家都十分激动,滔滔不绝,唾沫横飞,甚至在朝堂上公开对骂,
然而从第一天起,大臣们就惊奇地发现,这位皇帝似乎有点不对劲。因为无论下面吵得多热闹,上面的这位兄弟却
一句话都不说,始终保持沉默。

沉默的隆庆是个很可怜的人。

他是嘉靖的第三个儿子,皇位本没有他的份,安心做个藩王,好好过日子就行,可偏偏老天爷开眼,前面两个都没
能熬过去,于是老三就变成了老大。

但这对于他而言,实在算不上一件好事,因为嘉靖同志不但命硬,还极难伺候,能和他打交道的,也都是徐阶、严
嵩这类老滑头,以朱载垕的智商水平,只能是重在参与了。

而现在看着下面这帮杀气腾腾,脸红脖子粗的陌生人,他经常会发出点感叹:我怎么会呆在这种地方,和这些人打
交道?

他知道,如果自己开口说话,不管好坏,按照言官们的光荣传统,一定会被骂,既然如此,那我就不说话了,看你
们还能怎么样?

不久之后,隆庆终于明白,原来不说话也有不说话的骂法。

很快就有人找上门来了,这个人叫郑履淳,他慷慨陈词,严厉指责皇帝继位以来,放任大臣发言,自己却不说话,
长此以往国家怎么得了?

\section[\thesection]{}

说来有点搞笑,因为这位郑先生时任尚宝丞,是管机要文件的,并不是言官,就算要骂,怎么着也轮不上他,不知
是不是穷极无聊,想找点事情干。

于是皇帝愤怒了,老子都不说话了,让你们去骂街,竟然还是闹到了我的头上,说话也骂,不说话也骂,你要造反
不成?!

恨得牙痒痒的皇帝终于没能忍住,随即命令把郑先生拖出去打屁股,然而终究还是放了他。

隆庆兄终于雄起了一次,这实在是不容易的,因为在他执政的大多数时间内,他是比较窝囊的。

除了说话的问题外,皇帝大人还惊奇地发现,原来做皇帝,也是可以很穷的。

一般说来,新官上任都有三把火,作为大明帝国的统治者,刚刚登基自然也想摆摆场面,于是隆庆下令,由户部拨
款,为后宫购买一些珠宝首饰,算是送给诸位老婆的礼物,其实也花不了多少钱,所以在他看来,这件事情并不过
分。

然而结果是,户部尚书马森上书表示:你买可以,我不出钱。

这句话看似耸人听闻,却也不是没有来由的,要知道,在明代,财政制度是很严格的,户部相当于财政部,而财政
部的钱,就是国家的钱,皇帝是无权动用的,即使要用,也要经过财政部部长(户部尚书)、内阁分管财政部的大学
士(一般是首辅)层层审批,还要详细说明你把钱用到什么地方去了,准备用多久,打不打算还,什么时候还。

要不说清楚,一个子都甭想动。

所以历代皇帝要用钱的时候,大都会动用内库,也就是他们自己每年的收入,除非是穷得没办法,一般都不会去找
户部打秋风。

既然明知,为什么还要去触这个霉头呢,因为他就是穷得没办法了。

原先内库还有点钱,但到他爹手上,都拿去修道和给道士发工资了,等传到他这里,已经是一穷二白,干干净净。

现在马森不给,他也没办法,本打算再下一道谕令,希望这位部长大人手下留情,多少施舍点,但就在此时,大麻
烦来了。

言官们不知从哪里知道了这个消息,于是大家兴奋了,这回有事干了。

首先是给事中魏时亮上书,严厉批驳皇帝的浪费行为,很快御史贺一桂跟进,分析了买珠宝的本质错误所在,还没
等皇帝大人回过神来,另一个重量级的人物出场了。

这个人的名字叫做詹仰庇,人送外号詹三本,很快你就会知道这个外号是怎么来的。

\section[\thesection]{}

这位詹兄是嘉靖四十四年(1565)的进士,换句话说,他刚当官才两三年,虽说资历浅,可谓是人混胆子大,看见大
家上书,他也上了一本:

``陛下你要知道,历史上的贤君都不喜欢珠宝,比如某某某某(此处略去),现在您刚刚登基,就开始喜欢这类东
西,一旦放纵后果不堪设想,我听说两广还在打仗,您怎么能够本末倒置呢?''

皇帝又愤怒了,户部又不给钱,我也没追究,你们还一拨一拨地上,老子不还没买吗,你们到底想干什么?!

然而这一次,他忍了下来,没有发作,继续保持沉默,珠宝的事情也不提了,就当没这回事。

然而他万万没有想到的是,詹三本又行动了。

不久之后,这位仁兄在宫里闲逛,偶然看见了太医,就上前打招呼,一问,是进宫给皇后看病的,换了别人,这事
也就完了,但詹三本不是别人,他就开始琢磨了,这皇后怎么就生了病呢,再一打听,原来是夫妻双方闹矛盾,皇
后搬到别处去住了。

好了,好了,用功的时候又到了,詹三本琢磨来琢磨去,又上了第二本:

``臣最近听说皇后已经搬到别处居住,而且已经住了近一年,最近身体还不好,臣觉得这件事情陛下不应该不理
啊,要知道皇后是先皇选定的,而且一向贤淑,现在您不去看望皇后,万一有个什么三长两短,那可怎么得了?''

``所以希望皇上听我的话,前去看望皇后,臣就算死,也好过活着了(虽死贤于生)。''

这就是无理取闹了,人家夫妻俩吵架,与你何干,还要你寻死觅活?

隆庆收到奏疏,大为恼火却不便发作,不回答又不行,只好回了个话:

``皇后生了病,所以才住到别处去养病,我的家事你怎么知道,今后不要乱讲话!''

就这样,詹仰庇出名了,他本来预计这次投机是要挨板子的,而现在居然毫发无伤,这笔生意做得太值了,正是所
谓----中外惊喜过望,仰庇益感奋(史料原文)。

于是感奋不已的詹仰庇再次感奋了,他决定再接再厉,把弹劾进行到底,很快,他就上了第三本,这一次他把矛头
对准了宫内的宦官,说他们多占田产,收取赋税,希望皇帝陛下驱逐他们。

\section[\thesection]{}

事实证明,詹仰庇先生的弹劾,欺负欺负隆庆皇帝这样的老实人还是可以的,但对付真的坏人,那就不灵了,宦官
们立刻找了个由头,坑了他一把,把他赶出了京城。

起于弹劾,终于弹劾,詹三本到此终于功德圆满,十几年后他还曾经复起,担任过都察院左副都御史,为了巴结当
时的大学士王锡爵,甘当打手四处骂人,后又被人骂走,事实证明这位仁兄是典型的没事找抽型人格。

隆庆皇帝面对的就是这么一群人,说得好听是读过书的大臣,说得不好听就是有牌照的骂街流氓,他的心理承受能
力又不如内阁的那几头老狐狸,实在是疲于招架。

所以从登上皇位的那天起,他就意识到了这样一点:皇帝是不好干的,国家是不好管的,而我是不行的,国家大事
就交给信得过的人去干,自己能过好小日子就行了。

事实证明,正是这个判断使大明王朝获得了重生的机会。

那么谁是信得过的人呢,对于隆庆而言,自然就是身边的那几位讲官了,除殷士儋外(原因很复杂,后面再讲),高
拱、张居正、陈以勤都是最合适的人选。

于是在隆庆初年(1567),礼部尚书陈以勤与吏部左侍郎张居正同时入阁,至此内阁已有六人,他们分别是首辅徐阶、
次辅李春芳、郭朴、高拱、陈以勤、张居正。

请注意上面的六人名单排序,它的顺序排列实在非同寻常。

在明代,内阁是讲究论资排辈的,先入阁的是前辈当首辅,后来的只能做小弟当跟班,那小弟怎么才能做首辅呢?
很简单,等前辈都死光了,你就能当前辈了。

这里特别说明,早你一天入阁就是你的前辈,你就得排在后面,规矩是不能乱的。可能有人要问,要是两人同一天
入阁怎么办呢?

那也简单,大家就比资历吧,你是嘉靖二十年的进士,我是嘉靖二十六年的,那你就是前辈,如果连资历也相同,
就比入阁时候的官级,你是正部,我是副部,你还是前辈,如果官衔也相同,那就比年龄,反正不分出个先后不算
完。

所以张居正虽然与陈以勤同时入阁,但论资历和官级,他都要差点,只能委屈点,排在第六了。

其实这种排序本也说不准,要说起来,排第二的李春芳还是陈以勤的学生,谁让人家进步快呢?这种事情,不能怨天
尤人。

\section[\thesection]{}

这就是隆庆初年的内阁顺序表,考虑到排序,再看看前面几位生龙活虎的状态,如果按自然死亡计算,张居正要想
接班,至少也得等到七八十,这还是保底价。

不过幸好,除了论资排辈外,明朝也不缺乏其他的优秀传统,比如不斗到死不罢休的斗争哲学。

就在张居正刚刚入阁之后不久,一场猛烈无比的风暴来临了。

正所谓十处打锣,九处有他,这次挑事的又是一位老熟人----胡应嘉。

虽说上次投机不成,没有搞掉高拱,反而结了仇,但胡应嘉没有辞职,更不退休,这位仁兄注定是闲不下来的,很
快,一个偶然事件的发生,为他提供了新的发挥途径----京察。

明代的官员制度是很严格的,每三年考核一次,每六年京察一次,顾名思义,京察就是中央检察,对象是全国五品
以下官员(含五品),按此范围,全国所有的地方知府及下属都是考察对象(知府正五品)。当然,也包括京城的京官。

这么一算起来,那些整天叫嚷的言官也都是考察对象,全国十三道监察御史统统是正七品,六部六科都给事中是正
七品,给事中才从七品,算是包了饺子。

我查了一下,这个条例是明宪宗朱见深时开始实施的,很怀疑这是不是朱同志受不了骂,故意这么干的。

如果这真是他的本意,那他就要失望了,因为一百多年来,每次京察的结果总是地方官倒霉,言官安然无恙。想想
也是,管京察的是吏部尚书和都察院左都御史,并不是内阁大学士,连皇帝都怕言官,两位部长大人怎么敢干得罪
人的事情呢?

但这次似乎有点不同了,除了地方官外,许多原先威风凛凛的御史、给事中都下了课,乖乖地回了家,朝野一片哗
然,敢闹事的却不多。

因为人和人不一样,此时的吏部尚书是一个超级猛人,他虽然没有入阁,却比大学士还狠----杨博。

说来惭愧,这位当年严世藩口中的天下三杰竟然还活着,而且老而弥坚,这次京察是由他主导的,那就真算是一锤
定音了。

想当年我二十多岁的时候就陪大学士巡边,之后镇守蒙古边疆,杀了二十多年人,又干了十几年政务,严嵩在时都
要让老子三分,你们这些小瘪三,也只能去欺负皇帝,免了就免了,辞了就辞了,你敢怎样?

\section[\thesection]{}

那倒也是,现在的内阁成员中,除了徐阶外,其余五人见到他都得恭恭敬敬的行礼,谁还敢动他?

但这世上从不缺胆大的,胡应嘉估计是得罪了高拱,反正豁出去了,就摸了这个老虎屁股,他上书弹劾了杨博。

当然,弹劾也是有理由的,虽说这次从中央到地方,撤掉了很多的官员,但唯独有一类人却丝毫未动----山西人。
而``凑巧''的是,杨博就是山西人。

狭隘的老乡观念是要不得的,是一定要摒弃的,这就是胡应嘉弹劾的主要内容。但文书送上去后,杨博还没作出反
应,内阁就先动手了。

具体说来,是高拱要解决胡应嘉,他握着胡言官的那封奏疏,大声疾呼应该让胡应嘉趁早滚蛋,回家当老百姓。

之所以会出现这一幕,只是因为胡应嘉先生过于激动,却忽视了一个基本程序问题。

京察的主办单位是吏部和都察院,而作为给事中,也是要参与其中的,胡应嘉全程办理了此事,却一言不发,现在
京察结束了才来告状,你早干嘛去了?

高拱等这个机会已经很久了,他辞严厉色,一边骂胡应嘉还一边斜眼瞟徐阶,那意思是你能拿我怎样,而郭朴也趁
机凑了回热闹,跟着嚷起来,要严惩胡应嘉。

像徐阶这种老江湖,自然是不吃眼前亏的,如果再闹下去,就要骂到自己头上来了,所以他腰一弯,就势打了个滚:

``那好吧,我也同意。''

高拱,这可是你自找的,不用我出手,自然有人收拾你。

事实证明,高拱兄还是天真了点,他万万想不到,处罚令下达之日,就是他倒霉之时。

自打胡应嘉要贬官的传言由路边社传出之后,高拱就没消停过,京城里大大小小的言官已经动员起来:胡应嘉替我
们说话,既然高大人要他下课,我们就要高大人下台!

最先跳出来的是给事中辛自修,御史陈联芳,他们分别弹劾高拱滥用职权、压制言论等罪名,但高拱不愧为老牌政
治家,轻而易举便一一化解。

然而当听说另一位言官准备出场弹劾时,高拱却顿时感到了末日的来临,这个人的名字叫欧阳一敬。

欧阳一敬,嘉靖三十八年进士,给事中,从七品。江湖人送外号----骂神。

\section[\thesection]{}

这是一份并不起眼的履历,但只要看看他的弹劾成绩,你就会发现他的可怕。

嘉靖年间,他弹劾太常少卿晋应槐,晋应槐罢官。

接着,他弹劾礼部尚书董份,董份罢官。

后他调任兵科给事中,弹劾广西总兵(军区司令员)恭顺侯吴继爵,吴继爵罢官。也正是因为这位仁兄的一状,饱经
沧桑的俞大猷大侠才得以接替此位,光荣退休。

三个月后,弹劾陕西总督陈其学、巡抚戴才,陈其学、戴才罢官。

如果你觉得他已经很有胆,很敢弹的话,那我建议你还是接着往下看,因为他还曾经弹劾以下这些人(排名不分先
后):

英国公张溶,山西总兵董一奎、浙江总兵刘显、锦衣卫都督李隆等等等等。

所谓英国公,就是跟随永乐皇帝朱棣打天下的那位张玉的后代,最高公爵,世袭罔替。山西总兵和浙江总兵都是省
军区司令员,而李隆都督是特务头子。

弹劾结果:以上官员中,除英国公张溶外,全部罢官。

总而言之,在欧阳一敬不到十年的弹劾生涯中,倒在他脚下的三品以上部级文武官员合计超过二十人,并附侯爵一
人,伯爵两人。

当我看到这份成绩单时,总会不禁感叹,原来骂人也是有天赋的。

骂神出马,自然不同凡响,欧阳一敬实在是骠悍得紧,不但弹劾高拱,还捎带了杨博,并大大夸赞了高拱的奸恶水
平,说他比历史上的著名奸臣蔡京同志还要奸。

在弹章的最后,他还体现了有难同当的高尚品质:

``胡应嘉弹劾的事情,我事前就知道了,你们要处罚胡应嘉,就先处罚我吧!''

这种江湖义气,实在颇有几分黑社会的神韵。

这回高拱扛不住了,可还没等他开始反击,另一个人却蹦了出来,此人就是他的学生齐康。

齐康也是御史,但老师吃了亏,同行也就顾不上了,他立马站出来,先骂欧阳一敬,再骂徐阶,但是事实证明,骂
架和打架的道理大致相同,人多打人少才能打赢。

齐御史刚出头,就被欧阳一敬方面的口水彻底淹没,而徐阶兄也不甘示弱,趁你病要你命,还找来了几个六部官
员,大家一起去踩高拱。

这下再也扛不住了,隆庆元年(1567),屁股还没坐热的高学士主动提出辞职回家,一个月后,他的同乡好友郭朴也
退了休。

徐阶,算你狠,我们走着瞧!

\section[\thesection]{}

就这样,徐阶轻而易举地获得了胜利,这也只能怪高拱兄不自量力,徐首辅久经考验,当年孤身一人,尚且敢跟严
嵩对干,如今天下在握,皇帝都不好使,何况高学士,内阁里你排老几?

高拱走了,最伤心的人是皇帝,但他也无能为力,因为他说了不算。

此时的徐阶已经比皇帝还皇帝了,隆庆被他抓在手里,动弹不得,皇帝说:中秋节到了,咱们摆个宴席,庆祝一下。

徐阶说:铺张浪费,你就不要办了。

皇帝说:那好,我听你的。

不久之后,皇帝又说:我这么多年一直呆在北京,想要出去转转。

徐阶真是个直爽人,说了一大堆话,概括起来两个字:不行。

隆庆终于出离愤怒了,我爹还不敢这么管我呢!你凭什么!?一气之下,他毅然收拾行李,还是去了。

虽然这次英雄的举动为他赢得了一次自助游的机会,但长此以往,怎么得了?高拱又走了,身边连个出主意的人都没
有,就在皇帝大人苦苦思索对策的时候,一件出乎他意料的事情发生了。

徐阶致仕了,他放弃了首辅的位置,打好包裹,准备回松江老家。

这在当年,算是一件奇闻,要知道,以徐首辅的地位和威望,想干多久就干多久,想灭谁就灭谁,完全是天下无敌
的状态,所谓金盆洗手,急流勇退,那只是一个遥远的童话。

然而童话确实成为了现实,而原因也十分简单----疲惫,以及欣慰。

隆庆二年(1568),徐阶六十六岁,暂住北京,即将退休。

四十八年前,他十八岁,家住松江华亭县,在那里他遇见了一个叫聂豹的七品知县,听从了他的教诲:

``我将致良知之学传授于你。''

四十五年前,他二十一岁,来到北京考中了进士,在大明门前,他见到了首辅杨廷和,听到了他高声的预言:

``此子之功名,必不在我辈之下!''

三十八年前,他二十八岁,面对首辅张璁的怒吼,他从容不迫地这样回答:

``我从未曾依附于你!''

然后他前途尽毁,家破人亡,被发配蛮荒之地,在那里,他第一次见识了这个世界的黑暗与残忍。

二十年前,他四十六岁,看着自己的老师夏言被人杀死,不发一言。

因为他已经了解了这个世界的规则,报仇雪恨也好,伸张正义也罢,冲动解决不了任何问题。

四年前,他六十二岁,经过十余年的忍耐与经营,他除掉了严嵩,杀死了他的儿子,成为了一个工于心计,城府深
不可测的政治家,世间的一切都在他的掌控之中。

现在,一切又回到了起点。

\section[\thesection]{}

当年的青年才俊,现在的老年首辅,当年的热血激情,现在的老到深沉。从黑发到白发,从幼稚到成熟,一切都变
了,唯一不变的,是志向。

徐阶这一辈子,被人整过,也整过人,干过好事,也干过坏事,但无论何时何地,他始终没有背弃自己当年的誓
言,在他几十年的从政生涯中,许多正直的官员得以任用,无数普通百姓的生活得到保障,高拱与张居正的伟大新
政由他而起,我想,这已经足够了。

在为国效力的同时,他的一生都献给了斗争事业,这么多年来,他一直在第一线勤勤恳恳地斗,奋发图强地斗,干
了一辈子斗争工作,也该歇歇了。

虽然皇帝陛下第一时间就批了他的致仕申请,且唯恐他反悔,当即公布天下,发退休金让他走人,明显有点不够意
思,但徐阶却并不在意,因为他已欣慰地看到,自己为之奋斗终身的那个报国救民的理想,将由一个更为优秀的人
去实现。

张居正,我相信,你会比我做得更好。

除了张居正外,对另一个人的提拔与关照也让他倍感安心,他认为,这个人将成为张居正的得力帮手。

这个走运的人,就是我们的老相识海瑞先生,自打从牢里放出来,那可真叫一发不可收拾,先是官复原职,很快就
升了官,当了大理寺丞(正五品),专管审案,也算发挥特长。

不久之后,这位当年的小教谕竟然当上了都察院佥都御史(正四品),成为了名副其实的高级官员。

海瑞能够飞黄腾达,全靠徐阶,在徐首辅看来,海瑞是个靠得住的清官,是应该重用的,临退休前把他提拔起来,
将来还有个指望。

然而事实证明,这正是他人生中第二次错误的任命,很快,一次致命的打击就将向他袭来。

但此时的徐阶依然是幸福的,他看着自己亲手创造的一切,微笑着离开了这里,离开了这个带给他痛苦、仇恨、喜
悦和宽慰的地方。

隆庆二年(1568)十一月,徐阶回到了松江府华亭县,他又看到了熟悉的风景,和他离弃多年的家。

四十多年前,他从这里出发前往北京,一切就此开始,而现在,是结束的时候了。

他推开了家中的那扇门。

柴门闻犬吠,风雪夜归人。

我回家了,终于。

\section[\thesection]{}

你的命运,在我的手中

世界上的事情实在是说不准的,短短两年,高拱和郭朴走了,徐阶也走了,原本甩尾巴的张居正一下子排到了第
三,当然,这只是看上去很美,因为甩尾巴的依旧是他。

所谓老实人不吃亏,李春芳现在有了充分的心得,像他这样的好好先生,从来不争不闹,居然也成了首辅,而陈以
勤则当上了次辅,这两位老好人脾气不大,才能不高,以一团和气为指导思想,整天就忙着和稀泥,劝架,从不惹
事,看起来,和平终于来临了。

不过终究只是看起来而已,很快,一场新的狂风巨浪就将掀起,而这一切的始作俑者,是一个极为神秘的人物。

隆庆三年(1569),赋闲在家的徐阶突然接到了仆人的通告,说有人来拜会他。作为朝廷前任首辅,地方上那些小芝
麻官自然要经常上门拜码头,为省事起见,但凡遇到这种情况,仆人会直接打发他们走人。

但这一次,是个例外,仆人告诉他,来访的这位虽不是官,却比官还牛,口口声声说有紧急机密的事情要找徐阶,
且口气极大,极其嚣张。

于是徐阶也好奇了,他把这个人叫了进来。

这是一个其貌不扬的人,自称姓邵,别号``大侠'',没有官职,没有身份。然而他进来之后,只说了一句话,就让
久经沙场的徐阶目瞪口呆。

他说的这句话是:我能帮助你再当上首辅,你愿意吗?

等徐阶确定自己的耳朵没有问题后,便大笑了起来,他没有说话,只是不停地笑,在他四十多年的执政生涯中,遇
到过无数怪事、怪人,但眼前此情此景,实在是闻所未闻,见所未见。

我在内阁混了十几年,九死一生才当上首辅,天下到处都是我的门生亲信,皇帝都要服我管,你既无官职,也无名
望,也就算个二流子,竟然要扶持我当首辅!

差点笑岔气的徐阶挥了挥手,让人把眼前这个不知天高地厚的家伙赶了出去,在他看来,这是退休生活中一次有趣
的娱乐插曲。

但他并没有注意到,在他放声大笑之时,这位邵大侠并没有丝毫惊慌与尴尬,在他的眼中,只有两种情绪在闪动:
失望、以及仇恨。

\section[\thesection]{}

于是被赶出徐家之后,他立刻调转了方向,前往另一个地方----河南,在那里,他将会见第二个人,并兑现自己的
诺言。

十几天后,高拱在自己的家中见到了这位邵大侠,也听到了他的承诺,但与徐阶不同的是,他相信了眼前的这位神
秘访客。而一个传奇也就此开始。

我最早是从一些杂谈笔记中看到这一记载的,当时只是一笑了之,从古至今,像邵大侠这样的政治骗子一向不缺,
拿着几份文件,村长就敢认部长的,也不在少数。

一个无权无势的无名小卒,怎么可能把高拱扶上首辅的宝座?打死我也不信。

然而打不死,所以我信了。

因为在后来的查阅中,我发现,有许多可信度很高的史料也记载了这件事,而种种蛛丝马迹同时证明:这位邵大侠
虽然是个骗子,却是骗子中的极品。

邵大侠,真名不详(一说名邵方)、具体情况不详,但可以肯定的是,他是一个混混。

这位仁兄自小就不读书,喜欢混社会,一般说来,年轻人混到二十多岁,就该去找工作娶老婆了,但他却是个例
外,对他而言,混混已经成为了一种事业,从南混到北,从东混到西,最后混到了京城。

正是在京城,他圆满完成了转型,成功地由一个小混混变成了巨混混。因为在这里,他认识了一个人,这个人虽不
起眼,品级不高,也不是内阁成员、六部部长,却有着不亚于内阁首辅的权势。

他的名字叫做陈洪,时任御用监掌事太监。

前面曾经说过,在太监的部门中,司礼监权力最大,因为他们负责批红,任何命令没有他们打勾都不能算数。而这
位陈洪兄虽也干过司礼监,此时却只是个管日用品的御用监。

但事实上,这位陈兄是当年最牛的太监之一,究其原因,那还要感谢嘉靖同志。

因为嘉靖不信任太监,加上当时的内阁过于强悍,都是夏言、严嵩、徐阶之流老奸巨滑的人物,所以司礼监的诸位
仁兄早就被废了武功,又练不成葵花宝典,每天除了在公文上打勾外,屁都不敢放一个。

\section[\thesection]{}

于是御用监脱颖而出了,你再威风再嚣张,吃喝拉撒总得有人管吧,日常用品总得有人送吧,这就是关系,这就是
机会。所以不起眼的陈洪,却有着极为惊人的能量。

但太监是不能自己随意出宫的,有钱没处花,有劲没处使,于是邵大侠就成为了陈太监的联络员,而高拱,就是陈
洪的第一个同盟者。

绝顶聪明的徐阶赶走了高拱,安插了张居正,在他看来,高拱已经永无天日,事情已经万无一失,却没有想到,还
是留下了这唯一的破绽。

于是隆庆三年(1569)十二月,经过无数说不清道不明的内幕交易与协商,高拱又回来了,此时距他离去仅仅过了一
年。

得意了,翻身了,凭借着一个太监的帮助,高拱以十倍于胡汉三的精神状态回到了京城,在他看来,天下已尽在掌
握。

但他万万想不到的是,三年后,他将沿原路返回老家,而赶他回家的,是另一个太监。

所谓人走茶凉,有时候也不靠谱,听说高拱回来了,隆庆十分高兴,亲自接见他,并刻意叮嘱好好工作,天天向上。

说是这样说,但毕竟人走了一年,原先在内阁排老四,现在也只能去甩尾巴了。朝廷的规矩,就算天王老子,也不
能插队!

但皇帝大人实在很够意思,为保证高老师不至于被排在前面的几位熬死,他玩了一个小小的花招,而正是这个花招
成就了高拱。

在下令高拱为大学士进入内阁的同时,隆庆兄还悄悄地送给他的老师一个职务----吏部尚书。

这是一个非同小可的任命,根据历朝的惯例,为保证皇帝大权在握,内阁大学士不能兼管吏部,因为吏部是人事
部,是中央六部中权力最大的部门,如果把人事权和政务处理权都交到一个人的手中,不出鬼才怪。

但咱们谁跟谁啊,战火中结交,斗争中成长,是铁得不能再铁的兄弟,不信你高老师还能信谁?

于是大权在手的高拱准备行动了,为了得到那最高权力的宝座,为了实现自己报国救民的抱负,必须先铲除几个敌
人。

高拱黑名单上的第一个目标,不是一个,而是一群。

那群叽叽喳喳的言官们终于要吃苦头了,高学士不是隆庆皇帝,说整你就整你,绝不打折扣,于是短短几个月中,
二十多名言官不是撤职,就是调任,反正当年只要朝高先生吐过口水的,基本都被罚了款。

这些小鱼小虾都在其次,高先生最惦记的,还是欧阳一敬。

\section[\thesection]{}

为了对付这位传说中的骂神,高拱做好了充足的准备,但正当他要下手的时候,一个出人意料的消息传来----欧阳
兄主动辞职了。

骂神不愧为骂神,骂人厉害,闪人也快,见势不妙立刻就溜号了,但不知是不是骂人太多,过于缺德,或是高老师
玩了什么把戏,这位兄弟在回家的路上竟然不明不白的死了,对他而言,没有死在骂人的工作岗位上,实在是一种
遗憾。

现在只剩下胡应嘉了,欧阳一敬好歹还是个帮凶,胡先生可是真正的罪魁祸首,那是怎么也跑不掉的,但让高拱想
不到的是,他竟然还是没能整治这位仁兄。

因为胡应嘉的避祸方法更有创意,他直接就死掉了。

在得到高拱上台的消息后,胡应嘉由于心理压力过大,几天后就不幸死亡了,对一个死了的人,还能怎么整治呢?
也就这样吧。

言官们完蛋了,高拱快刀斩乱麻,准备对付下一个对手,和那些只会骂人的家伙比起来,这个敌人才是真正的威胁。

高拱王者归来之时,在欣喜之余,他也惊奇地发现,自己只能排在第五了,而多出来的那个第四内阁学士,就是赵
贞吉。

说起这位赵兄,那也算是老熟人了,之前他曾多次出场,骂过严嵩,支持过王学,时任礼部尚书,现在入阁,可谓
功德圆满了。

但自打这位声名显赫的尚书大人来后,内阁的其他四位同志就没过上一天舒坦日子,因为赵兄弟一反常态,热衷于
惹麻烦,一天到晚都要没事找事,从李春芳到陈以勤,都挨过他的骂,最惨的是张居正,每天都被横眉冷对,心理
压力极大。

为什么呢?说到底,还是一个心态问题。

要知道,李春芳和张居正都是嘉靖二十六年的进士,陈以勤是嘉靖二十三年的,而赵学士,是嘉靖二十年。

论资历,他是内阁里最老的,他当官的时候,其他的内阁同事们还在家啃书本,现在他虽然也入了阁,却排在最
后,连张居正都不如,咱中国就讲究个论资排辈,你要他倚老而不卖老,那实在是要求太高。

但好在李春芳和陈以勤都是老实人,张居正翅膀没硬,也不怎么吭声,所以内阁里每天都能听见赵学士大发感慨,
叹息``老子当年''之类的话,也没人敢管。

现在高拱回来了,排在了最后,赵学士终于找到了心理安慰,开始找高拱的麻烦。

\section[\thesection]{}

可实在不巧,高学士也是嘉靖二十年的进士,论资历旗鼓相当,而他也不把赵贞吉放在眼里:混那么多年才入阁,
只能说你无能!

更为重要的是,他的目标是首辅,就算赵贞吉不找他,他也要去解决赵贞吉,不把你解决掉,我怎么当老四?

很快,他就纠集手下的言官弹劾赵贞吉,加上他还是吏部尚书,各级官员一起上,不搞掉你誓不罢休!

可赵学士也不是省油的灯,事实上,在当时的内阁里,唯一能与高拱对抗的人就是他,因为十分凑巧,在内阁里他
恰好分管打手机关----都察院。

从某种程度上讲,当时的都察院可算是疯人院,里面许多人都是穷极无聊,一放出来就咬,咬住了就不放,一时之
间又是口水满天飞。

然而赵贞吉没有高兴多久,就惊奇地发现,那些言官突然安静了下来,也不再卖力骂人了,不管他好说歹说,就是
不动。

对于此中奥妙,我们还是请高拱同志来解释一下:

``别忘了,老子是吏部尚书,还管京察!''

要明白,言官骂人那是要计算成本的,赔本的买卖没人做,海瑞那种赔钱赚吆喝的也着实少见。

于是赵贞吉绝望了,高拱已经胜券在握,但就在此时,一件出乎双方意料的事情发生了,高学士排到了第四,而赵
学士也排到了第三。

因为陈以勤辞职了。

陈以勤实在受不了了,他本就是个老实人,准备干几年就回家养老,偏偏这二位不让他休息,整天闹来闹去,高拱
是他当年的同事,而赵贞吉是他的老乡,帮谁也不好,于是他心一横----不干了,回家!

但辞职的归辞职,该斗的还得斗,很快赵学士就败下阵来,收拾包袱回去了,而高拱则再接再厉,直接超越了张居
正,排到了李春芳的后面,成为了次辅。

全国人民都知道,李春芳是热爱和平的,于是大权就落在了高拱的手中。几乎所有的人都认为他应该收手了,然而
直到此时,他才终于亮出了自己名单上的最后一个敌人----徐阶。

斗争形势是复杂的,斗争路线是曲折的,而敌人是狡猾狡猾的,所以要想一劳永逸地解决问题,必须做好充足的准
备,找好突破口,才能一举搞定。

而现在,这个突破口已经出现了,他的名字叫海瑞。

\section[\thesection]{}

隆庆三年(1569),海瑞终于得到了他人生中最肥的一个职位----请注意,不是最大,是最肥。

大家同样在朝廷里混,有的穷,有的富,说到底是个位置问题,要分到一个鸟不生蛋的地方,十天半月不见人,穷
死也没法,而某些职位,由于油水丰厚,自然让人趋之若鹜。

而在当时,朝廷中公认的四大肥差,更是闻名遐迩,万众所向,它们分别是吏部文选司、吏部考功司、兵部武选司、
兵部武库司。

文选司管文官人事调动,要你升就升,考功司管每年的官员考核,要你死就死,这是文官。

武选司管武将人事任命,战场上拼不拼命是一回事,升不升官又是另一回事,而武库司从名字就能看出来,是管军
事后勤装备的,不肥简直就没天理了。

这就是传说中的四大肥差,也是众人日夜期盼的地方。然而和海瑞先生比起来,那简直不值一提,因为他要担任的
职务,是应天巡抚。

所谓应天,大致包括今天的上海、苏州、常州、镇江、松江、无锡以及安徽一部,光从地名就能看出来,这是一块
富得流油的地方,光是赋税就占了全国的一半。

而海瑞之所以能得到这个职务,自然也是徐阶暗中支持的结果,对此海瑞也心知肚明,他虽然直,却不傻。

但如果徐阶知道接下来即将发生的事情,估计他能立马跑去给海先生三跪九叩,求他赶紧退休回家养老。

``海阎王就要来了!''

随着几声凄厉的惨叫,中国历史上一场前无古人,相信也后无来者的壮观景象出现了:

政府机构没人办公了,从知府到知县全部如临大敌,惶惶不可终日,平常贪污受贿的官员更是不在话下,没等海巡
抚到,竟然自动离职逃跑,

而那些平时挤满了富商的高级娱乐场所此时也已空无一人,活像刚被劫过的,大户人家也纷纷关门闭户,听见别人
说自己家有钱,比人家骂他祖宗还难受。高级时装都不敢穿了,出门就套上一件打满补丁的破衣烂衫,浑似乞丐。
恰巧当时南京镇守太监路过应天,地方上没人管他,本来还想发点脾气,再一问,是海瑞要来了。于是他当机立
断----不住了,赶紧走! 

走到一半又觉得不对,便下了第二道命令----换轿子!(按照规定,以他的级别只能坐四人小轿)就这样连走带跑离开
了应天。

\section[\thesection]{}

于是等海巡抚到来之时,他看到的,已经是一片狼藉,恶霸不见了,地主也不见了,街上的人都穿得破破烂烂,似
乎一夜之间就回到了原始社会。

但这一切似乎并未改变海瑞的心情,他是个始终如一的人,该怎么干还怎么干,到任之后第一件事就是张榜公布,
欢迎大家来告状,此外还特别注明免诉讼费,并告知下属,谁敢借机收钱,我就收拾谁。

告状不要钱!那就不告白不告了,于是司法史上的一个奇迹发生了,每天巡抚衙门被挤得像菜市场一样,人潮汹涌,
人声鼎沸,最多一天竟收到了三千多张诉状,而海阎王以他无比旺盛的精力和斗志,居然全部接了下来,且全部断
完,而结果大多是富人败诉。

这是海瑞为后人津津乐道的一段事迹,然而事实上,它所代表的并非全是光明和正义,因为在这个世界上,还有一
种人叫做刁民。

所谓刁民,又称流氓无产者,主要工作就是没事找事,赖上就不走,不弄点好处绝不罢休,而在当时的告状者中,
这种人也不在少数,而海瑞照单全收,许多人借机占了富人的家产,自己变成了富人,也算是脱贫致富了。

但总体说来,海巡抚还是干得不错的,毕竟老百姓是弱势群体,能帮就帮一把,委屈个把地主,也是难免的。

可是与以往不同的是,这次海瑞大张旗鼓地干,却没有人提出反对,也不搞非暴力不合作,极其听话。说到底,大
家怕的并不是他,而是他背后的那个人--徐阶。

得罪海瑞无所谓,但徐阶岂是好惹的,所以谁也不触这个霉头。

然而随着追究恶霸地主工作的进一步深入,平静被彻底打破了,因为海瑞终于发现了应天地区最大的地主,而这个
人正是徐阶。

其实徐阶本人也还好,关键是他的两个儿子,仗着老爹权大势大,在地方上肆意横行,特别喜欢收集土地,很是捞
了一把。而徐阶兄不知是不是整天忙着搞斗争,忽略了对子女的教育,也没怎么管他们,所以搞到现在这个样子,
所以徐阶同志的深刻教训再次告诉我们,管好自己身边的亲属子女,那是十分重要滴。

不过海瑞倒是不怎么在乎徐阶的教育问题,他只知道你多占了地,就要退,不退我就跟你玩命!

\section[\thesection]{}

不过看在徐阶的面子上,海瑞还是收敛了点,给徐大人写了封信,要他退地。

徐阶还是很有风度的,他承认了部分错误,也退了一部分地,在他看来,自己救了海瑞的命,还提拔了海瑞,现在
又带头退地,应该算是够意思了。

可海瑞却不太够意思,他拿到了徐阶的退地,却进一步表示,既然你有这个觉悟,那就全都退了吧,就留一些自耕
田,没事耕耕地,还能图个清静,我是替你着想啊!

徐阶当时就懵了,我辛辛苦苦干了一辈子,还是内阁首辅主动退休,准备回家享享福,你要我六十多岁重新创业,
莫非拿我开涮不成?

于是他又写信给海瑞,表示自己不再退田,希望他念在往日情谊,高抬贵手,就当还我的人情吧。

可是事实证明,海瑞兄的脑袋里大致没有这个概念,这位兄弟几十年粗茶淡饭,近乎不食人间烟火,什么是人情?
什么是欠?什么是还?

到此徐阶终于明白,自己混迹江湖几十年,竟然还是看走了眼,这位海瑞非但油盐不进,连砖头都不进。

他下定了决心,要顽抗到底,并摆明了态度--不退。

海瑞也摆明了态度----一定要退。

双方开始僵持不下,就在这时,高拱来了。

最好的工具

活了这么大年纪,高拱从来没相信过天上会掉馅饼,但现在他信了。

虽然已经身居高位,但他从不敢对徐阶动手,这并非因为他宅心仁厚,只是徐阶地位太高,且在朝廷混了那么多
年,群众基础好,如果贸然行动,没准就被闹下台了,所以一直以来,他都是冷眼旁观。

等他知道海瑞正在逼徐阶退田的事情后,立即大喜过望,反攻倒算的时候终于到了!

原因很简单,如果用自己的人,大臣们一望即知,必定会去帮徐阶,现在大家都知道,海瑞是徐阶的人,你自己提
拔的人去整你,我不过是帮帮忙,总不能怪我吧。

海瑞,是一件最合适的利用工具。

高拱很快对海瑞的举动表示了支持,并且严厉斥责了徐阶的行为,海瑞得到了鼓励,更加抖擞精神,逼得徐阶退无
可退。

于是徐阶准备妥协投降了,他表示,愿意退出全部的田地,在海瑞看来,问题已经得到了圆满解决,然而就在此
时,事情又出现了意想不到的变化。

\section[\thesection]{}

朝廷里的言官突然发难,攻击徐阶教子不严,而一个叫蔡国熙的人被任命为苏州兵备使,专职处理此案,很巧的
是,这位蔡先生恰好是高拱的学生,还恰好和徐阶有点矛盾。

事情闹大了,徐阶的两个儿子被抓去充军,家里的所有田产都被没收,连他的家也被一群来历不明的人烧掉了,徐
大人只能连夜逃往外地。

看起来,海瑞赢了,然而事实证明,最后的胜利者只有高拱。

隆庆四年(1570年),海瑞接到了朝廷的命令----收拾东西走人。

于是仅仅当了半年多巡抚的海瑞走了,他本着改造一切的精神跑来,却发现被改造的只有他而已。

海瑞先生岂是好惹的,这么走算怎么回事?他一气之下写就了另一封骂人的奏疏。

在海瑞的一生中,论知名度和闹事程度,这封奏疏大概可以排第二,仅次于骂嘉靖的那封。

要知道,骂人想要骂出新意是不容易的,既然骂过了皇帝,骂其他人也就没啥意思了,但海瑞先生再次用行动证明
了他的骂人天赋,这一次他找到了新的对象----所有的大臣(除他以外)。

而他在奏疏中,也创造了新的经典骂语----``举朝之士,皆妇人也''

这句话可谓是惊天地泣鬼神,在古代骂对方是妇人,比骂尽祖宗十八代还狠,于是满朝哗然一片,然而奇怪的是,
却没有人出面反击。

究其原因,还是海瑞先生太过生猛,大家都知道,这位兄台是个不要命的主,要是和他对骂,后果不堪设想,于是
所有人都原地不动,愣愣地看着海瑞大发神威。

只有两个人说话了

第一个是李春芳,作为朝廷的首辅,他不表态也说不过去,然而出人意料的是,他既没有攻击海瑞,也没有处分
他,却拿着海瑞的奏疏,说了一句让人哭笑不得的话:

``照海瑞的这个说法(举朝之士,皆妇人也),我应该算是个老太婆吧!''

还真是个老实人啊。

另一个人是高拱,其实事情闹到这个份上,也算拜他所赐,在这最后摊牌的时刻,他终于揭示了其中的奥妙:

``海瑞所做的事情,如果说都是坏事,那是不对的,如果说都是好事,那也是不对的,应该说,他是一个不太能做
事的人。''

这是一个十分中肯的评价。

面对这个污浊的世界,海瑞以为只有自己看到了黑暗,他认为,自己是唯一的清醒者。

然而他错了。

\section[\thesection]{}

海瑞是糊涂的,事实证明,徐阶看到了,高拱看到了,张居正也看到了,他们不但看到了问题,还有解决问题的方
法。而海瑞唯一能做的,只是痛骂而已。

所以从始至终,他只是一个传奇的榜样,和一件好用的工具。

隆庆五年(1571),海瑞回到了海南老家,但这位主角的戏份还没完,十多年后,他将再次出山,把这个传奇故事演
绎到底。

在海瑞的帮助下,高拱终于料理了徐阶,新仇旧怨都已解决,大展拳脚的时候到了。

其实从根本上说,高拱和徐阶并没有区别,可谓是一脉相承,他们都是实干家,都想做事,都想报效国家,但根据
中国的传统美德,凡事都得论个资历,排个辈分,搞清楚谁说了算,大家才好办事。

现在敢争敢抢的都收拾了,高拱当老大了,也就该办事了。

于是历时三年,闻名于世的高拱改革就此开始,史称``隆庆新政''。

说实话,这个所谓新政,实在是有点名不副实,因为即使你翻遍史书,也找不出高先生搞过什么新鲜玩意,他除了
努力干活外,即不宣誓改革,也不乱喊口号,但他执政的这几年,说是国泰民安、蒸蒸日上,也并不夸张,可见有
时候不瞎折腾,就是最好的折腾。

但要说高先生一点创新进步都没有,那也是不对的,徐阶是明代公认的顶级政治家,他的权谋手段和政务能力除张
居正外,可谓无人匹敌,但这位高兄在历史上却能与之齐名,是因为他虽在很多地方不如徐阶,却在一点上远远超
越了这位前辈----用人。

具体说来,他用了三个人。

第一个,叫做潘季驯。

一般说来,要是你没有听过这个名字,并不需要惭愧,但如果你的专业是水利,那我只能劝你回去再读几年书。

几年前,我曾看到过这样一条新闻,大意是水利工作者们开动脑筋,调集水库积水统一开闸,冲击泥沙,缓解了黄
河的淤积情况,意义重大云云。

虽说搞水利我是门外汉,但如果没有记错,早在四百多年前,潘季驯先生曾经这样做过,而它的名字,叫做``束水
冲沙法''。

潘季驯,嘉靖二十九年(1550)进士,浙江吴兴人,明清两代最伟大的水利学家。

这位兄台算是个奇人,高考成功后被分配到江西九江当推官,管理司法,他的官运也不错,十几年就升到了监察院
右佥都御史,成为了一名高级言官。

\section[\thesection]{}

恰好当时黄河决堤泛滥,灾民无数,高拱刚刚上台,急得没办法,四处找人去收拾残局,恰好有一次和都察院的一
帮言官吵架,潘季驯也在场,高拱看这人比较老实,也不乱喷口水,当即拍板:就是你了,你去吧!

张居正是个比较谨慎的人,觉得这样太儿戏,就去查了潘季驯的底,急忙跑来告诉高拱:这人原来是个推官,法律
和水利八杆子打不着,他怎么懂得治水?

高拱却告诉他:只管让他去,他要不会治水,你只管来找我。

事实证明,高学士的眼光确实很毒,虽说没学过水利专业,潘季驯却实在是个水利天才,他刚一到任,堵塞缺口之
后,便下令把河道收窄。

这是一个让人匪夷所思的命令,大凡治河都是扩宽河道,这样才有利于排水,收缩河道不是找死吗?

施工的人不敢干,跑来找潘季驯。

潘季驯说你只管干,出了事我负责。

于是奇迹出现了,收缩河道之后,黄河不但没有泛滥,决堤的出现也大大减少,大家都惊叹不已。

看上去很神奇,实际上很简单,在长期的观察中,潘季驯发现了这样一个问题----黄河之所以泛滥,是因为河道逐
年升高,形成了岸上河,于是河堤也越来越高,稍有不慎一旦决堤,后果就会极其严重。(住在黄河边上的人应该深
有体会)

而要降低河道,就必须除掉河里的泥沙,好了,关键就在这里,怎么除沙呢?

找人去挖,估计没人肯干,也没法干,找挖掘机,那还得再等个几百年,用什么才能把这些泥沙除去呢?潘季驯苦思
冥想,终于醒悟,原来那件制胜的武器就在他的眼前----水。

收紧河道,加大水的冲力,就可以把河底的泥沙冲走,所谓``水流沙中,沙随水去'',就此大功告成。

除此之外,他还想出了一种独特的治水方法,名叫滚水坝,具体说来,是事先选择一个低洼地区,当洪水过大之
时,即打开该处堤坝,放水进入,以减轻洪峰压力。

看起来很眼熟是吧,没错,这就是流传至今,众人皆知的治水绝招----分洪。

有这么一位水利天才坐镇,泛滥多年的黄河得到了平息,在之后的数十年内没有发生过大的水患。

这是第一位,算是个干技术的,相比而言,下面的这位就麻烦得多了。

\section[\thesection]{}

黄河泛滥,多少还有个期限,等汛期洪峰过了,该埋的埋,该重建的重建,也就消停了,但是暴动就不一样了,要
闹起来你不管,指望他们突然放下屠刀,皈依我佛,那种事西游记里才有。

隆庆四年(1570),永不落幕的两广叛乱再次开演了,在当年,这个地方算是蛮荒之地,文盲普及率较高,不读书自
然不服管,不服管自然不纳税,不纳税自然是不行的。于是来来往往,双方都喜欢用拳头刀枪讲话,每到逢年过
节,不闹腾一下,那就不正常了。

但这次闹腾的动静很大,两广全境都有叛乱,且叛军有一定的战斗经验,派了几个人去都被打了回来,于是高拱一
拍脑门:

``没办法了,派殷正茂去吧!''

殷正茂,嘉靖二十六年进士,是当年传奇科举班的一员,和诸位名人同学相比,他没有张居正的政务能力,王世贞
的文采,更没有杨继盛的胆量,但他也有着属于自己的专长----军事。

他虽是文官出身,却极具军事才能,多次领兵出战,从无败绩,被认为是一代名将,按说他应该是最理想的人选,
可为什么直到没办法才找他呢?

原因很简单,他太贪。

这位兄弟虽说很有才能,却是个不折不扣的贪污犯,原先当地方官就吃农民赋税,到军队后就吃士兵的军饷,明代
贪污不算什么大事,但殷先生却贪得天下皆知,贪得名闻全国,着实不易。

果然,任用殷正茂的消息一传出,就如同往厕所里丢了颗炸弹,分量十足,在大贪污犯殷正茂的面前,大臣们第一
次消除了分歧和派系,异口同声地表示绝对不行。

高拱却是吃了秤砣铁了心,表示一定要用,每天朝廷里都吵得天翻地覆,最后还是高学士水平高,只用一句话,就
让所有的人都闭上了嘴:

``谁再反对殷正茂去两广,我就派谁去!''

这就不好玩了,殷正茂即刻光荣上任。

但他的亲信,给事中陆树德站了出来,劝告高拱,人你可以派去,但军饷你要看紧,最好在户部找个人随从前去,
搞好财务审核制度,要内防家贼。

然而高拱说:

``不用派人,所有军饷直接拨给殷正茂就是了。''

陆树德急了:

``殷正茂必定贪污军饷!''

``我知道。''高拱却笑了笑,``那又如何?''

\section[\thesection]{}

``我拨一百万两军饷给殷正茂,他至少贪污一半,但以他的才能,足以平定叛乱,如果我派一个清廉的人去,或许
他一两也不贪,但是办不成事,朝廷就要多加军饷,这么拖下去,几百万两也解决不了问题。''

``所以殷正茂不去,谁去?''

事实确实如此,殷正茂去后,仅仅几个月就平息了叛乱,班师凯旋,当然了,军饷他也没少拿,如果不贪,那就不
是殷正茂了。

但高拱还是赚了,说到底,这是个成本核算问题。

在高拱的正确指导下,潘季驯和殷正茂成为了名噪一时的风云人物,但和第三个人比起来,前面这二位就只能算是
小儿科了。因为这位最后出场的压轴主角解决了一个问题,一个连朱元璋都没能解决的问题。

这个人的名字叫王崇古,时任都察院右副都御史。

其实之前他曾经露过一面,在浙江时,他作为俞大猷的副将出击倭寇,获得大胜。这之后他官运亨通,一直升到了
现在的位置。

在当时的朝廷中,有三个人是言官们不怎么敢惹的:杨博、谭纶以及这个王崇古。

所谓不敢惹,绝不是因为官衔问题,越大的官骂得越起劲,此三人之所以能幸免,是因为他们有一个共同的特殊身
份----军事文官。

在明代武将出身的人是很受歧视的,经常被人看作大老粗,比如戚继光、俞大猷等人也不能幸免,而进士出身改行
当武将的,就不同了,这类人既有文化,又会打仗,且由于长期在边界砍人,性情比较彪悍,不守游戏规则,你要
是敢骂他,他没准就敢拿刀砍你,看谁吃亏。

而这位王崇古除了喜欢领兵打仗外,还有后台,作为嘉靖二十年的新科进士,他和高拱同学的关系很好。

于是他被委派了一个极为重要的职务----宣大总督。

伟大的军事家、政治家、哲学家王守仁曾在他的著作中说过这样一句话:

``大明虽大,最为紧要之地只有四处,若此四地失守,大明必亡。''

王守仁所讲的四个地方,是指宣府、大同、蓟州、辽东,它们是明代边界最让人头疼,也最难防守的重要据点。

所以自明代中期后,它们被分为两个独立军区(宣大、蓟辽),由朝廷直接管理,其指挥官为总督,超越各级总兵,
是明朝国防部长(兵部尚书)以下最高级别的军事长官,只有最富军事经验的将领才能担当此任。

顺便说一句,当时的蓟辽总督是谭纶,而他手下的两位总兵分别是蓟州总兵戚继光,以及辽东总兵李成梁。

\section[\thesection]{}

看到这个豪华阵容,你就应该明白,王崇古同志找了个多么光荣的工作。

踌躇满志的王崇古前去赴任了,他做梦也想不到,一个天大的金元宝即将砸到他的头上。

飞来横财

就在王崇古上任的几乎同一时刻,一个人从蒙古鞑靼的帐篷中走出,在黑夜中向故乡投去了最后仇恨的一瞥,朝着
另一个方向走去,那里是敌人的营垒。

于是天明之时,边关的明军突然开始紧急戒备,并派出快马,告知新上任的王崇古总督:横财来了。

这个人的名字叫做把汉那吉,是俺答的孙子,说起这位俺答兄,也算是老朋友了,当年闯到北京城下,杀人放火好
不威风,然而现在他的孙子竟然跑到敌人那边,当了叛徒,归根结底,这是一个恋爱问题。

事情的起因是这样的,不久前,把汉那吉准备要娶媳妇了,而且这位未婚妻很漂亮,所以小伙子一天到晚都乐呵呵
的。

可事情坏就坏在这个漂亮上,有一天,爷爷看见了这位孙媳妇,便当机立断:把汉那吉你再娶一个吧,这个我就带
回家了。

顺便讲一下,据某些史料记载,这位孙媳妇也是俺答的外孙女,要这么算起来,那俺答应该算是乱伦了,不过从这
位仁兄以往干过的种种``光辉事迹''来看,搞这么一出倒也不出奇。

虽说当时没有什么婚姻法,鞑靼部落也不讲究什么三纲五常,但把汉那吉依然愤怒了:好不容易找了个老婆,竟然
被老头抢走了,真是岂有此理!

可这位老头偏偏是他的爷爷,还是部落首领,自己一无兵,二无权,又能怎样呢?

思前想后,他找到了一个报复的方法----投奔明朝。就算不能带兵打回去,至少也能出一口恶气。

于是事情就闹到了这个份上,边关守将捞到这么个重量级人物,十分高兴,马上派快马去向王崇古报喜。

可他等到的不是王崇古的夸奖,却是一番严厉的训话:自今日起,全军收缩,准备迎战!

此外还有一条特别的声明:副将(副总兵级别)以上军官一律不得外出作战!

这是一条让人莫名其妙的命令,军官不去打仗,难道让小兵指挥?

事实证明,王崇古同志作出了一个无比英明的决定。

三天之后,俺答就来了,带着他的全部家当----十几万蒙古骑兵。

\section[\thesection]{}

但这一次他们似乎不是来抢东西的,在大同宣府附近转悠了好几天,不断挑衅闹事,但边防军牢记王崇古的教诲,
打死也不出头,偶尔只派小股部队出去转转,就这么折腾了几天,蒙古军粮食吃光了,才抓了几个小兵,只能打道
回府。

身为一名长期从事抢劫工作的专业人士,俺答有着充分的绑票经验,抓人、谈判、收赎金一整套流程了如指掌,而
现在自己的孙子成了敌人的人质,作为该行业的资深从业人员,他没有去谈判,筹集款项,而是直接选择了最为简
单的方式----绑票。

只要能够抓到对方的高级将领,拿人去交换,既方便操作,又节省成本。可惜的是,王崇古是狡猾狡猾地,不吃这
一套。

俺答失望地走了,王崇古却犯了愁,该怎么处理这位把汉那吉呢?你把他留在这里,俺答自然会来找麻烦,而这位仁
兄除了身份特殊外,也没啥特殊才能,每天你还要管饭,实在是个累赘。

大多数人建议:好歹也是个蒙古贵族,养在这里费粮食,咱们把他剁了吧,也算是立个功。

也有人说,还是放了吧,省得他爷爷来闹事。

面对激动的群众,王崇古保持了冷静,长期的官场经验告诉他,如果不知道该怎么办,就去请示领导,领导总是英
明的,即使不英明,至少也能负责任。

于是他上报了高拱,请领导批示处理意见。

高拱接到了报告,即刻找来了张居正,两位老狐狸凭借多年朝廷打滚的经验,在第一时间作出了判断:既不能杀,
也不能放。

那该怎么办呢?在长时间思考之后,高拱眼睛一亮:

``我要用他,去交换一个人。''

高拱所说的那个人,叫做赵全。

明代是一个不缺汉奸的朝代,而在吴三桂之前,最为可恶的汉奸非赵全兄莫属。

在逃到鞑靼之前,赵全是明军中的一员,估计是由于福利待遇之类的问题,他义无反顾地投奔了俺答,成为了一名
臭名昭著的汉奸。

历史证明,汉奸往往比外敌更为可恶,高拱之所以如此看重赵全,是因为这位汉奸实在坏得离了谱,坏出了国际影
响。俺答虽说喜欢抢劫杀人,但总体而言,人品还是不错的,也比较耿直,抢完了就走,不在当地留宿过夜。

但赵全的到来改变了这一切。

\section[\thesection]{}

赵全熟悉明军的布防情况,经常带领蒙古军进攻边界,此外他还劝说俺答当皇帝,组织政权和明朝对着干,破坏能
量非常之大。

因为他为祖国做出的``巨大贡献'',赵全极其光荣地成为了明朝头号通缉要犯,上到皇帝,下到小兵,个个都知道
他的大名,而这位仁兄也极其狡猾,朝廷重金悬赏,但凡抓到他的,升官赏钱不说,还能分房子,但十几年过去,
连根毛都没逮到。

现在机会终于到了。

在高拱的命令下,王崇古派出了一名使者,前往俺答军营谈判,这名使者的名字叫做鲍崇德。

在很多人看来,这是一个看上去并不复杂的任务,但实际上非常复杂。

使者踏入了俺答的营帐,等待他却不是谈判的诚意和酒宴,而是冰冷的刀剑和这样一句话:

``你知不知道,之前来过的两个明朝使者,已经被我杀掉了。''

死亡的威胁扑面而来,因为这位俺答似乎根本没有谈判的打算。

万幸的是,那个看上去并不起眼的使者鲍崇德,实际上非常起眼。

鲍崇德,当地人,原本是翻译,之后不断进步,兼职干起了外交,这一次,他将用自己出众的能力去完成这次凶险
无比的任务。

``我知道。''鲍崇德从容不迫地回答。

``那你知不知道,之前与我对阵的明军将领,也大都被我杀掉了。''----嚣张是可以升级的。

``我知道。''

``那你为什么还敢来?!''

然而嚣张的俺答最终沉默了,因为鲍崇德的一句话:

``如果我不来,你的孙子就没命了。''

虽然俺答摆出了一幅坚决不谈判的架势,但鲍崇德却十分肯定,他不过是在虚张声势,虽说他抢了孙子的老婆,和
孙子的感情也不好,但无论如何,他绝不会放弃这个孙子。

因为在此之前,鲍使者曾得到了一个十分准确的情报:俺答是一个怕老婆的人。

虽然俺答有好几个老婆,且生性野蛮,也没受过什么教育,但他依然是怕老婆的,特别是那个叫伊克哈屯的女人。

这位伊克哈屯大概算是俺答资历最老的老婆,也是最厉害的一个,虽说当时的蒙古部落娶几个老婆很正常,是不是
孙女,算不算乱伦也没人管,可偏偏那位跑掉的把汉那吉,就是伊克哈屯养大的。

你娶几个老婆我不管,但你赶走了我养大的孙子,我就废了你!

\section[\thesection]{}

于是在那之后的一段时间内,俺答的宿营地经常会出现这样一幕:满面怒气的伊克哈屯追着俺答跑,并且一边追一
边挥舞着手中的木棍,发出了大声的怒吼:

``老东西,快把我的孙子要回来,要不就打死你!''

虽然在过去的几十年中,俺答杀了很多人,抢了很多东西,但他毕竟也是人,这么个闹腾法,每天都不得安生,实
在受不了,可要他拉下面子求人,也确实干不出来,不得已才出此绝招,希望给对方一个下马威。

可惜鲍崇德并非等闲之辈,这位仁兄也是在官场打滚的,要论玩阴谋手段,俺答还得叫他一声爷爷。

于是大家都不忽悠了,开始摆事实讲道理,俺答开门见山:

``我的孙子现在哪里,情况如何?''

``他的近况很好,我们给他安排了住处,你不用担心。''

情况摸清楚了,下面谈条件:

``你们何时才肯放回我的孙子?''

``随时都可以。''鲍崇德笑着回答道。

``其实我们只需要一个人而已。''

然后他说出了那个人的名字。

俺答想了一下,只想了一下。

于是他也笑了。对他而言,那个人实在无足轻重。

几天之后,穿着新衣服的把汉那吉回到了蒙古,还带来了许多礼物,而俺答也终于得以从每日的追逐中解脱出来,
不用担心棍棒会随时落到自己的头上。

唯一的失败者是赵全,这位仁兄毫无廉耻地当了十几年走狗,最终却得到了这样的下场。

历史又一次证明,所有背叛自己祖国的人,终将被所有人背叛,因为奴才终究只是奴才。

赵全抓回来了,被凌迟处死,据说他身体还不错,割了上千刀才死,把汉那吉回家去了,继续过他的日子,毕竟老
婆是不难找的。

按说事情到了这里,已然结束了,明朝白捞了一个汉奸,王崇古的横财也该到此为止,但事实上,发财的机会才刚
刚开始。

在这次外逃风波之前,明朝和鞑靼之间除了刀光剑影,没有任何共同语言,明朝看鞑靼是土匪,鞑靼看明朝是恶
霸,经过这件事双方发现,原来对方并非洪水猛兽,虽说有代沟,但还是可以沟通的。

于是接下来,他们开始谈论一个全新的问题----封贡互市。

\section[\thesection]{}

所谓封贡互市,具体讲来是这么个过程,明朝封鞑靼,发给俺答等人新衣服(官服)、公章(官印)等官僚主义用品,
承认他们的土财主地位。而鞑靼要听从明朝大哥的教诲,不得随意捣乱抢劫,这叫封。

当然了,俺答虽说读书少,也绝不是白痴,给几枚公章,发几件衣服就想忽悠他,那还是有难度的,要我听话,你
就得给钱。实际操作方法为,每年俺答向明朝进贡土特产(马匹牛羊不限,有什么送什么),而明朝则回赠一些金银
珠宝,生活用品等,这叫贡。

但封贡毕竟是小买卖,蒙古部落上百万人,对日用品市场需求极大,又没有手工业,要想彻底解决问题,最好的方
法就是搞边境贸易。大家找一个地方,弄个集贸市场,来往商贩把摊一摆,你买我卖,这就叫互市。

其实自从元朝取消国号后,混吃等死就成了大多数蒙古人的心愿,所谓回中原当大地主,梦里时常也能见到。

可是没办法,蒙古的经济结构实在太单一,骑马放牧人人都会,可你要他造个锅碗瓢盆出来,那真是比登天还难。
如果要几十年不用这些玩意,似乎又说不过去,找人要,人家又不给,没办法,只有抢了。

现在既然能靠做生意挣回来,那自然更好,毕竟为抢个脸盆把命丢掉,实在也是太不划算。

体育就是和平----顾拜旦说的。

贸易也是和平----这是我说的。

有一点必须说明,只有在实力对等的前提下,贸易才能带来和平,边境有王崇古、谭纶、戚继光这帮狠人守着,谁
抢就收拾谁,人家才肯老老实实做生意,否则还是抢劫划算。

对于封贡互市制度,蒙古是一呼百应,极其欢迎,但他们的热脸却贴上了冷屁股----明朝的屁股。

虽然王崇古极力推动这一制度,但朝廷的许多大臣却对此极不感冒,因为在许多人看来,蒙古鞑靼那一帮子都是野
蛮人,给点好处让他们消停点就行了,做生意?做梦!

当时的朝廷已经是一片混乱,反对派气势汹汹,其主要观点是:东西我大明多的是,但即使送给要饭的,也不能给
蒙古人!

这一派带头的,就是骂神欧阳一敬手下唯一的幸存者英国公张溶,而海瑞的那位后台老板朱衡也是反对派的干将,
真可谓是一脉相承。

\section[\thesection]{}

而赞成的自然是高拱、张居正一帮人,但高拱毕竟是内阁大学士,算是皇帝的秘书,不便公开表态,他是个聪明
人,一看朝廷里反对一片,强行批准定被口水淹死,便见风使舵,想出了一个办法。

在我看来,正是因为想出了这个方法,高拱才得到了明代杰出政治家的光荣称号。而这个办法,也充分地体现出了
中国人几千年来的卓越才能,包括:钻空子、绕道走、打擦边球、以及民主精神。

他找到了反对派首领张溶,可还没等他说话,张溶就叫嚣起来:

``无论你说什么,我们都绝不同意!''

``没问题'',高拱笑着说道,``如果你们不同意,那我们来表决。''

张溶目瞪口呆,因为事实证明,高拱并没有开玩笑。

于是中国历史上最为奇特的``封贡票决''事件发生了,大家不闹腾了,开始投票,据史料记载,参与此次投票的共
有四十四人,在会议上,赞成反对双方坚持了各自的观点,陆续发言,而最后的结果却更让人哭笑不得。

经皇帝公证,验票统计如下:赞成封贡互市者二十二人,反对封贡互市者二十二人。

这下白闹了,事情又被踢给了皇帝。

这大概算是中国政治史上少有的一幕,皇帝说了不算,内阁说了也不算,在万恶的封建社会,竟然要靠投票解决问
题,实在有负``黑暗专制''的恶名。

当然,高拱兄不是什么自由斗士,对搞民主也没啥兴趣,他之所以来这么一出,实在是另有企图。

根据我的估计,在此之前,他一定曾算过票数,知道会有这样的结果,所以才提议投票,因为一旦投票不成,事情
就会推给皇帝,可是皇帝不会管事,自然就会推给内阁,而内阁,是高拱说了算。

于是一圈绕回来,还是绕到了高拱的手上,这就是传说中的乾坤挪移大法。

既然大臣解决不了,封贡互市的决定权便回到了内阁,李春芳可以忽略不计,高拱和张居正本来就是幕后主谋,于
是事情就这么定了。

隆庆五年(1571),边境市场正式开放,各地客商陆续赶到这里,开展贸易活动,一个伟大的奇迹就此出现,自朱元璋
起,折腾了两百多年的明蒙战争终于落下帷幕,此后近百年中,双方再未爆发大规模的战争。

和平终究还是实现了,这是高拱立下的不朽功勋。

\section[\thesection]{}

决裂

潘季驯、殷正茂和王崇古的任用,证明了高拱是一个无比卓越的优秀政治家,在他的统领下,大明王朝开始重新焕
发生机活力,而他的声名也随之达到了最高峰。

然而就在那光辉灿烂的顶点,一个阴影却已悄然出现,出现在他的背后。

张居正并不是个老实人,他或许是个好人,却绝不老实,对于高拱同志,他一直都是有看法的:

论资历,高拱比他早来三年,论职务,高拱从翰林院的科员干起直到副部长、部长、大学士,几十年辛辛苦苦熬出
来的,劳苦功高,而他却是从一个从五品副厅级干部被直接提拔为大学士,属于走后门的关系户,论能力,高拱可
谓是不世出之奇才,能够善断,相对而言,他还只是个愣头青。

所以无论从哪一方面看,张居正都只能乖乖当小弟,而一直以来他也是这样做的,凡事唯高拱是从,遇到大事总是
请示再请示,十分尊重领导。

可问题在于,高拱并不满足于当老大,他还要当爹,他要所有的人都听命于他,服从他的指挥,谁要不听话,是要
被打屁股的。

刚开始的时候,张居正也没啥意见,毕竟高拱是老同志,耍耍威风似乎也没什么,但很快他就改变了自己的看法--
当他亲眼看到那个被打屁股的人时。

这位倒了霉的仁兄就是殷士儋,关于此人,那真是说来话长。

嘉靖二十六年(1547),殷士儋和张居正同期毕业,由于成绩优秀,被选为庶吉士,之后又被调入裕王府,担任裕王
的讲官。

既有翰林的背景,又是太子的班底,官运也不错,隆庆二年(1568)还当上了礼部尚书,但奇怪的是,他偏偏就是入
不了阁。

在明代,这实在是个要命的问题,记得我当年小学时曾被任命为卫生委员,现在想来,那是我担任过的最高职务,
虽说唯一的好处就是每天多扫一次地,却实在让人心潮澎湃,激动不已,为什么呢?

因为卫生委员是班委成员。

要知道,各科科代表虽说平时管收作业,实在是威风八面(特别是对我这种不爱交作业的人),但他们不是班委成
员,老师召集开会的时候,他们是没有资格去的,也得不到老师的最高指示。

\section[\thesection]{}

卫生委员就不同了,虽然每日灰头土脸,但每当听到老师召唤时,将手中的扫把一挥,高傲地看一眼收作业的课代
表,开会去也!

那是相当的牛。

相信你已经明白了,课代表就是各部部长,班委就是内阁,老师就是……

扫地的强过收本子的,就是这个道理。

殷士儋讨厌收作业,他想去扫地,但他始终没有得到这个机会。

而根正苗红的殷部长入不了阁,说到底,还得怪他的那张嘴。

在这个世界上,同样一件事,不同的说法有截然不同的效果,比如一个胖子,体重一百公斤,如果你硬要说人家体
重0.1吨,被人打残了我也不同情你。

殷士儋大致就是这么一个人,他是历城(今山东济南)人,算是个地道的山东大汉,平时说话总是直来直去,当年给
裕王当讲官时经常严辞厉色,搞得大家都坐立不安,所以后来裕王登基,对这位前老师也没什么好感。

其实皇帝怎么想还无所谓,关键是高拱不喜欢他。

这很正常,高拱要听话的人,而殷士儋明显不符合此条件。

所以入阁的事情拖了好几年,人员进进出出,就是没他的份,这不奇怪,奇怪的是,到了隆庆四年(1570)十一月,
这位收作业的仁兄竟然拿到了扫把----入阁了。

这自然不是高拱偶发善心,实在是殷部长个人奋斗的结果,既然高拱不靠谱,皇帝也不能指望,那就只剩下了一条
路----太监。

殷士儋一咬牙,走了太监的门路,终于得偿所愿,对此高拱也只能望洋兴叹,毕竟他也是靠太监起家的。

但老奸巨滑的高学士自然不会就此了结:不能挡你进来,那就赶你出去!为了及早解决这个不听话的下属,他找来了
自己的心腹,都给事中韩楫。

几天之后,在韩楫的指示下,言官们开始发动攻击,殷士儋同志的老底被翻了个遍,从上学到找老婆,但凡能找到
的都拿来骂,搞得他十分狼狈。

高拱得意了,这样下去没多久,殷士儋只能一走了之,事实证明他是对的,但他也忽略了十分重要的一点----殷士
儋的脾气。

于是一场意外就此发生。

事情从一次会议开始,本来内阁开会只有大学士参加,但有时也邀请言官们到场,偏偏这一次,来的正是韩楫。

\section[\thesection]{}

殷士儋不喜欢高拱,本打算打声招呼就走人,一看韩楫来了,顿时精神焕发,快步走上前去,说了这样一句话:

``听说韩科长(韩楫是六科都给事中,明代称为科长)对我有意见,有意见不要紧,不要被小人利用就好!''

高拱就在现场。

殷学士的这句话只要不是火星人,想必都明白是什么意思,加上在场的人又多,于是高拱的脸面也盖不住了。

``成何体统!''

好!你肯蹦出来就好!

孙子当够了,殷士儋终于忍无可忍,发出了一声惊天动地的怒吼:

``高拱!陈大人(指陈以勤)是你赶走的,赵大人(指赵贞吉)是你赶走的,李大人(指李春芳)也是你赶走的,现在你看
我不顺眼,又想赶我走!首辅的位置是你家的不成!?''

高拱当时就懵了,他万万没想到,像殷士儋这种档次的高级干部,竟然会当众发飚,一时反应不过来,但更让他想
不到的还在后头。

殷士儋真是个实诚人,实诚得有点过了头,这位仁兄骂完了人,竟然还不解恨,意犹未尽,卷起袖子奔着高拱就去
了。

反正骂也骂了,索性打他一顿,就算要走,也够本了!

到底是多年的老政治家,高拱兄也不是吃素的,看见殷同志来真格的,撒腿就跑,殷士儋也穷追不舍:脸已经撕破
了,今天不打你个半死不算完!

关键时刻,张居正站了出来,他拉住了殷士儋,开始和稀泥:

``万事好商量,你这又何必呢?''

然而殷士儋明显不是稀泥,而是水泥,一点不给面子,对着张居正又是一通怒吼:

``张太岳(张居正号太岳),你少多管闲事,走远点!''

老子今天豁出去了,谁敢挡我就灭了谁!

所幸在场的人多,大家缓过劲来,一拥而上,这才把殷大学士按住,好歹没出事。

我算了一下,闹事的时候,殷士儋五十六岁,高拱六十岁,张居正最年轻,也四十七岁,三位中老年人竟然还有精
力闹腾,实在让人钦佩。

殷士儋不愧是山东人,颇有点梁山好汉的意思,敢作敢当,回家后没等高拱发作,就主动提出辞职,回家养老去了。

在高拱看来,这个结果还不错,虽说差点被人打,但自己还是赢了,可以继续在内阁当老大。

\section[\thesection]{}

但他绝对想不到的是,这场风波正是他覆亡的起点,因为在那个纷乱的场景中,张居正牢牢地记住了那句被很多人
忽略的话:

现在你看我不顺眼,又想赶我走!首辅的位置是你家的不成!?

是啊,既然李大人可以被赶走,陈大人可以被赶走,那么我也会被赶走----当高大人看我不顺眼的时候。

况且,我也喜欢首辅的那个位置。

于是,从那一天开始,张居正就确定了这样一个认识----两个人之中,只能留一个。

而那个人,只能是我。

为了实现我的梦想和抱负,高拱,你必须被毁灭。

张居正打定了主意,准备对他的老朋友、老同事动手了,然而出人意料的是,先出招的人,竟然是高拱。

其实一直以来,高拱虽说对张居正抱有戒心,却还是把他当朋友的,直到有一天,他听到了那个传闻。

对高拱而言,赵贞吉是可恶的,殷士儋是可恶的,但只要他们滚蛋,倒也没必要赶尽杀绝,只有一个人除外----徐
阶。

对徐大人,高拱可谓是关怀备至,对方家破人亡之后,他还是不依不饶,经常过问徐阶的近况,唯恐他死得太轻松。

就在这个时候,有人突然跑来告诉他,张居正和徐阶有秘密来往,答应拉他一把,帮他儿子免罪,当然了,张居正
也没白干,他收了三万两白银。

高拱平静地点了点头,他准备用自己的方法,去解决这个问题。

不久之后的一天,他找到张居正闲聊,突然仰天长叹:

``老天爷真不公平啊!''

张居正没有说话,他知道后面的话才是正题。

``为什么你有那么多儿子,而我一个也没有?''

张居正这才松了一口气,高拱确实运气不好,六十多岁的人了,无儿无女,将来也只能断子绝孙了。

为缓和气氛,张居正发挥了他和稀泥的专长,笑着说了这么一句:

``儿子多,但也不好养活啊!''

好了,要的就是这句话。

``你有徐阶送你的三万两白银,养活几个儿子不成问题。''高拱微笑着,露出了狰狞的面目。

张居正慌了,他这才发现对方来者不善,无奈之下,他只得赌神罚咒,说些如果收钱,出门让车撞死,生儿子没屁
眼之类的话,最后搞得声泪俱下,高拱才作了个样子,表示这是有人造谣,我绝对不信,然后双方握手言和,重归
于好。

给他一个教训,今后他就会老实听话----这是高拱的想法。

必须尽快解决他,再也不能迟疑!----这是张居正的决心。

\section[\thesection]{}

一个过于优秀的太监

决心下了,可该怎么动手呢?扫把不到,灰尘不会自己跑掉,张居正明白这个道理。

但现在的高拱已经今非昔比,连无比狡猾的徐老师都败在他的手下,单凭自己,实在没有胜算。而且这位六十高龄
的高老头身体很好,每天早起锻炼身体,精神十足,等他自然死亡太不靠谱。

就在山穷水尽之际,一个人进入了张居正的视野,他的名字叫冯保。

和明代的同行们比起来,冯保是个非常奇特的太监----奇特得不像个太监。

一般说来,太监由于出身不好,且家庭贫困,能认识几个字,写自己的名字就算知识分子了,按照这个标准,冯保
绝对可以评上教授,因为他不但精通经史,而且还是著名的音乐家,擅长演奏多种乐器,此外他还喜欢绘画,时常
也搞点收藏。

比如后来有一次,他在宫里闲逛,``无意''地走进了宫内的收藏库,``无意''地信手翻阅皇帝的各种收藏品,然后
``无意''中喜欢上了其中一幅画,最后便``无意''地``顺''(学名叫偷)走了这幅画。

事实证明,冯保先生的艺术鉴赏眼光是相当高的,因为那幅被他收归己有的画,叫做《清明上河图》。

像这种事情,一般都是天知地知,而我这样的小人物之所以也能凑个热闹,是因为冯太监在偷走这幅画后,还光明
正大地在画上盖上了自己的收藏章----以示纪念(类似某某到此一游)。

捅出冯太监的这段隐私,只是为了让你知道,他虽然有文化,搞艺术,却绝非善类,做坏事敢留名,偷来的锣还使
劲敲,这充分说明他具备了以下几种优良品质:胆大、心细、脸皮厚。

然而历史告诉我们,只有这样的人,才最适合搞阴谋。

而更让张居正喜出望外的是,这位冯保最恨的人,恰恰就是高拱。

我们之前曾经介绍过,明代的太监机关中,权力最大的是司礼监,因为这个部门负责帮皇帝批改奏章,具体说来是
用红笔打勾,然后盖上公章,上到军国大事,小到鸡皮蒜毛,都得过他们这关。

从嘉靖年间开始,冯保就是司礼监中的一员,隆庆登基后,他也官运亨通,成为了东厂提督太监兼御马监管事太监。

\section[\thesection]{}

这是一个了不得的职务,要知道,东厂是特务机关,而御马监手握兵权,是十二监中仅次于司礼监的第二号实力机
关。既管特务,又管部队,一个太监能混到这个份上,就算成功人士了。

但冯保并不满足,他要做太监中的霸主,就必须回到司礼监,得到另一个位置----掌印太监。

司礼监的工作是打勾和盖章,打勾的人数不等,叫秉笔太监,有资格盖章的却只有掌印太监----有且仅有一位。

天下大事,都要从我的公章下过,你不服都不行。

恰好此时前任掌印太监下课,太监也要论资排辈,按照职务资历,应该是冯保接任,但他却没有得到这个位置,因
为高拱插手了。

高拱横空出世,把御用监管事太监陈洪扶上了宝座,原因很简单,当年陈洪帮他上台,现在是还人情时间。

你陈洪不过是个管仓库的御用监,凭什么插队?!然而可怜的冯保只能干瞪眼,高拱实在太过强悍,是招惹不得的。

那就等吧,总有一天等到你。似乎是冯保的痴心感动了上天,陈洪兄上台没多久,也下课了。这下应该轮到冯太监
了。

然而高拱又出手了,他推荐了孟冲来接替陈洪的位置。

冯保出离愤怒了,愤怒之情如滔滔江水连绵不绝,据说在家里连骂了三天,余音绕梁不绝于耳。

如此激动,倒不全是有人抢了他的职位,而是这位孟冲兄的身份实在有点太过特殊。

按照规定,要当司礼监掌印太监,必须在基层单位或重要岗位锻炼过,这样才能当好领导太监,可是孟冲先生原先
的职务却是尚膳监,这就有点耸人听闻了,因为尚膳监的主要职责,是管做饭。

也就是说,尚膳监的头头孟冲先生,是一名光荣的伙食管理员。

太欺负人了!上次你找来一个管仓库的,我也就忍了,这回你又找个做饭的,下次莫不是要找倒马桶的?

冯保终于明白,不搞倒高拱,他永远都没有出头之日,于是在经过短时间观察后,不需要介绍人介绍,也未经过试
探、牵手、见家长之类的复杂程序,冯保与张居正便一拍即合,结成了最为亲密的联盟。

\section[\thesection]{}

但双方一合计,才发现高拱兄实在很难拱,他的威望已经如日中天,皇帝也对他言听计从,朝中爪牙更是数不胜
数,一句话,他就是当年的徐阶,却比徐阶难对付得多,因为看起来,这位仁兄似乎打算革命到底,丝毫并没有提
前退休的打算。

于是两人很快达成了共识,目前只能等--等高拱死。

但这种事情哪有个准,正当这对难兄难弟准备打持久战时,局势却出现了进一步的恶化。

为保存实力,张居正与冯保商定,遇到事情由冯保出面,张居正躲在暗处打黑枪,两人不公开联系,总是私下交流
感情。

但意外仍然发生了,一天,张居正突然得到消息,说隆庆皇帝病情加重,这是一个极为重要的情报,但此时天色已
晚,为了给冯保报信,张居正便写了一封密信,连夜派人交给冯保。

安全抵达,安全返回,张居正松了一口气。

然而第二天,当他刚刚步入内阁办公室的时候,一声大喝镇住了他:

``昨天晚上,你为什么送密信给冯保?信上写了什么?如果有事情,为什么不与我商量?!''

这回高拱也不兜圈子了,反正内阁里只有我们两人,既然是破事,咱们就往破了说。他死死地盯着张居正,等待着
对方的回答。

张居正没有准备,一时间手足无措,但老狐狸就是老狐狸,片刻之间,他就换上了一幅招牌式的笑容,笑嘻嘻地看
着高拱,也不说话。

所谓伸手不打笑脸人,老子死活不表态,看你怎么办?

这大概算是耍无赖的一种,于是在对峙一段时间后,高拱撤退了,他警告张居正不要乱来,便气鼓鼓地扬长而去。

事情闹大了,一听说联系暴露了,冯保就炸了锅:

还搞什么地下工作,高拱都知道了,索性摊牌吧!我们两个一齐上,鱼死网破,看看谁完蛋!

张居正明白,冯保是对的,现在情况紧急,高拱可能已经有所察觉,所谓先下手为强,如果现在动手,还能抢占先
机,再晚就麻烦了。

最关键的时候到了,动手还有一丝胜算,等待似乎毫无生机。

面对着极端不利的局面,张居正却做出了一个出人意料的抉择:

``再等等。''

\section[\thesection]{}

无以伦比的天赋,以及二十多年朝廷打滚的政治经验,最终拯救了张居正,让他做出了一个极为准确的判断:

``高拱依然是信任我的。''

继续隐藏下去,等待时机的到来。

隆庆六年(1572)五月二十六日,机会来临。

隆庆皇帝终于不行了,这位太平天子做了二十多年的替补,却只当了六年的皇帝,估计是当年压力太大,他的身体
一直不好,加上一大群言官口水乱飞,他又没有他爹那种心理素质,一来二去就一病不起。

这位循规蹈矩的皇帝知道自己不能干,所以把工作交给能干的人,在他统治期间,经济得到发展,百姓安居乐业,
连蒙古人都消停了,也算是相当不错了。

一句话,他是个老实人。

就在这一天,这位老实人感觉自己快要不行了,便紧急下令,召见三个人,他们分别是高拱、张居正,以及刚刚入
阁不久的高仪。

这里说一下这位高仪,虽说他姓高,却绝非高拱的亲戚,这位兄台当年是高拱的同班同学,几十年勤勤恳恳,小心
谨慎,是个不折不扣的老实人,老实到了令人发指的地步:

比如当年他做礼部尚书的时候,家里的房子失了火,烧得一干二净,好歹是个正部级干部,重新盖一座就是了。

可是高仪却极为另类,他自己没钱,也不向组织开口,竟然找了个朋友家借住,而且一直到死,也没买过房子,就
这么凑合了十几年。

所以很明显,高拱拉这个人入阁,就是用来凑数的,在他看来,高仪不过是个老实本分,反应迟钝的人,然而此后
的事情发展告诉我们,他或许老实,却绝不迟钝。

在接到入宫的命令后,高拱立刻意识到皇帝可能不行了,为了不耽误事,他撒腿就跑,据史料记载,这位仁兄连轿
子都没坐,六十多岁的老头,一溜烟从东安门跑进东华门,终于在皇帝咽气之前抵达目的地,实在让人叹为观止。

顺便说一句,这条路线今天还在,有兴趣的朋友可以试着跑跑,从东安门起始,跑进故宫乾清宫(记得带钱买票),
体验古迹之余也可以缅怀一下先人。

当高拱到达寝宫时,才发现有五个人已经先他而来,他们分别是皇后、太子朱翊钧、太子生母李贵妃、张居正,以
及那个他最为讨厌的人----冯保。

\section[\thesection]{}

这是一个看似平常的人员组合,前三个人先到场是正常的,他们住得近,张居正比自己先到,也还情有可原,毕竟
这小子年轻跑得快,冯保是司礼监秉笔,是皇帝的秘书,过来凑凑热闹,似乎也说得过去。

所以紧要关头,高拱也没多想,奔着半死不活的皇帝去了。

然而他万没想到,张居正之所以早到,是因为他早就从冯保那里得到了消息,而冯保之所以在场,是因为他策划已
久的阴谋即将在此实现。

看见高拱来了,已经在阎王登记本上签了名的皇帝,似乎又撤了回来,他用尽全身的力气,对这位陪伴他三十余
年,历经坎坷共赴患难的朋友、老师,说出了最后的话:

``太子年纪还小,天下大事,就麻烦先生你了。''

讲完,走人。

隆庆六年(1572)五月二十六日,隆庆皇帝朱载垕驾崩,年三十六。

皇帝死了,按照惯例,大家都得哭一场,无论真心假意,该走的程序还是得走,同理,按照惯例,哭完了就该商量
遗产、权力方面的问题。

此时,最自信的人是高拱,皇帝死前都说了,太子交付给我,还有谁能取代我不成?

从法律的角度上讲,皇帝大人对高拱提出要求,这叫口头要约,而高拱答应了这个要求,这叫口头承诺,然而事实
证明,无论是要约还是承诺,都比不上合同。

高拱同志就是吃了不懂法的亏,因为就在他最得意的时候,原先站在一旁死不吭气的冯保行动了----他拿出了合同。

这份所谓的合同,就是遗诏。

关于这份合同的内容,就不多介绍了,大体也就是些我干过什么错事,对不起国家人民,对不起劳苦大众,现在我
死了,请诸位多多照顾我儿子之类,但当高拱看到那句关键的话时,当即暴跳如雷:

``着令司礼监掌印太监与内阁大学士共同辅政!''

这回算是反了天了。

在明代两百多年的历史中,太监即使再猖獗,哪怕是王振、刘瑾这样的超级大腕,担任辅政也是痴心妄想,这是有
道理的,毕竟大家都是明白人,跟着个太监能学到啥呢?

然而这个例竟然在自己手上给破了,高拱气得七孔冒烟。

更何况,按规定,遗诏应该是我来拟的,皇帝死得急,没来得及写,大家也都理解,现在你冯保竟然搞出一份遗
诏,天上掉下来的?!

\section[\thesection]{}

但是激动归激动,毕竟人刚死不久,孤儿寡母在眼前,闹起来也不好看,况且遗诏也没指明冯保辅政,司礼监掌印
太监还是自己的人,有帐慢慢算,咱们走着瞧。

只过了一天,高拱就知道自己错了。

第二天,另一条遗旨颁布:原司礼监掌印太监孟冲退休,由秉笔太监冯保接任。

原来如此!

瞧不起太监,偏偏就被太监给耍了,高拱终于发现,他已经陷入了一个圈套,局势十分不利。

但老滑头毕竟是老滑头,在短暂惊慌之后,高拱恢复了镇定,叫来了自己的心腹大臣雒遒、程文,整夜商议之后,
他们订下了一个几近完美的攻击计划。

这一天是隆庆六年(1572)六月八日,高拱相信,胜券已经在握。

唯一的漏洞

隆庆六年(1572)六月十日,第一波攻击开始。

这一天,司礼监掌印太监冯保刚刚上班,便收到了一封呈交皇帝的奏疏,作者是高拱,他立即打开阅览,却被惊得
目瞪口呆。

奏疏的大致内容是说:太监不过是下人,却一直参与政治,我高拱实在看不过去,特向皇帝陛下建议,收回司礼监
的权力,并对敢于乱凑热闹的有关人等进行严惩。

冯保懵了,却并非因为恐惧,而是他怎么也想不通,高拱为何会犯如此低级的错误!

对这封奏疏中的建议,冯保早有心理准备,高拱兄每日磨刀霍霍,动手是迟早的事情,但用这种方式直接上奏,却
着实让人匪夷所思。

因为虽说大臣的奏疏是直接呈送皇帝的,但那已是朱元璋时代的事情了,随着皇帝越来越懒,许多文书都是由太监
转呈,皇帝往往看也不看,就丢给内阁,让内阁票拟处理意见,然后再转给司礼监批红盖章,事情就算结了。

这就奇怪了,你高拱明明知道皇帝小,不管事,文件都是我盖章,怎么还会上这样的东西,难道你指望我精神失
常,打自己耳光不成?

冯保把脑袋想破,也没明白怎么回事,但这个事总得解决,于是他扣住了奏疏,没有转交内阁,而是自己代替皇
帝,在上面批了六个字,然后批红盖章,还给了高拱。

这六个字是:``知道了,遵祖制''。

这又是一句传说中的废话,什么祖制,怎么遵守?

然而高拱却并不生气,因为这早在他的意料之中。

\section[\thesection]{}

高拱明知这六个字出自冯保的笔下,却只是冷笑了一声,对同在内阁的张居正与高仪说了这样一句话:十岁太子,
如何治天下?

高仪摇了摇头,张居正笑了。

冯保,你尽管闹吧,很快你就会知道我的厉害。

高拱没有就此罢手,而是再次送上奏疏,并特地说明,皇帝公务繁忙,就不劳烦您亲自批阅了,把我的奏疏送到内
阁就行,内阁有人管。

谁管?不就是高拱嘛。

高先生的意思很简单,翻译过来就是:冯保同志,我知道上次你当了一回皇帝,签了我的奏疏,这次就不劳烦你
了,把我的奏疏交给内阁,当然,也就是交给我,我自己来签。

一见这家伙又开始闹,冯保就头大,要私留文件可能要出麻烦,反正这封奏疏只是要个名分,那就给了你吧!

一念之差,他把奏疏交给了内阁。

这是一个差点让他送命的决定。

高拱就是高拱,比冯保有文化得多,轮到他当皇上,大笔一挥唰唰唰,在自己的奏疏上批了十九个字,其大体意思
是:

``我看了你的奏疏,对时政非常有用,显示了你的忠诚,就按你说的办吧!''

高拱表扬高拱,也算有性格。

文件又送回了冯保那里,看了高拱的批复,他哭笑不得:自己跟自己玩有意思吗?但无奈之下,他还是盖了章。

不就要个名分吗,你还能翻天不成?给你就是了。

我要的就是一个名分,高拱得意地笑了,冯保,你还太嫩。

这一天是隆庆六年(1572)六月十二日,计划圆满完成,第二波攻击即将开始。

隆庆六年(1572)六月十三日,冯保最黑暗的日子来到了。

一大早,工部都给事中程文上书,弹劾司礼监掌印太监冯保罪大恶极,应予惩办,主要罪恶摘录如下:

身为太监,竟然曾向先帝(隆庆皇帝)进送邪燥之药(春药),导致先帝因此而死。此外他还假传圣旨,以实现自己掌
权的野心,总之一句话,奸恶之徒,罪不可赦!

照程文兄的说法,不但冯保的官位是改圣旨得来的,连皇帝的死都要由他负责,这是把人往死里整。

同日,礼部都给事中陆树德,吏部都给事中雒遒上书,弹劾冯保窃权矫诏,应予逮捕审问。

\section[\thesection]{}

这还是明的,要知道,程文、陆树德、雒遒都是都给事中,也就是所谓科长,手下都有一大批给事中科员,科长出
马,科员自然也不会闲着,四处串联,拉关系闹事,京城里人声鼎沸,杀气冲天,不把冯保千刀万剐不算完事。

冯保崩溃了,他这才知道高拱的厉害,但他已然束手无策,而且高拱手上还有那封批准免除司礼监权力的奏疏,找
皇帝说理也没戏,冯太监彻底绝望了。

事情十分顺利,现在只剩下最后的一步,天下将尽在我手!

隆庆六年(1572)六月十四日 最后的准备

高拱去拜访了两个人----张居正、高仪。虽说他一直以来都把这两个人当摆设,但毕竟是内阁同僚,要想彻底解决
冯保,必须争取他们的支持。

但高仪的态度让高拱很失望,无论高拱说什么,这位老同学兼老实人都只是点头,也不讲话,于是寒暄几句之后,
高拱便离开了。

张居正就截然不同了,他十分热情地招呼高拱,并尊为上宾,高拱感受到了同志般的温暖,随即将自己解决冯保的
全盘计划告知了张居正,当然,最后他还是问了一句:

``高仪那边已经没有问题,你怎么样?''

张居正毫不迟疑地回答:

``自当听从差遣!''

为表示决心,他还加上了一句:

``除掉冯保,易如反掌!''

高拱满意地走了,他还要忙着去联络其他人。

张居正也很忙,他要忙着去找冯保。

至此,冯保终于知道了高拱的全部计划,然而在极度恐慌与愤怒之后,他才发现自己毫无办法,满朝都是高拱的
人,骂人的言官都是对头,唯一的盟友张居正,也不过是个次辅,无济于事。

冯保急了,张居正却丝毫不乱,他镇定地告诉冯保:有一个人可以除掉高拱。

``谁?''

``皇帝''。

冯保恍然大悟,这段时间忙里忙外,圣旨都是自己写的,竟然把这位大哥给忘了,虽说他才十岁,但毕竟是皇帝,
只要他下令解决高拱,那就没问题了。

但是皇帝和高拱又没矛盾,他凭什么支持我们呢?

面对着冯保的疑问,张居正陷入了沉思,很快,他就想起了一件事:

``除掉高拱,只需要一句话而已。''

张居正的脸上洋溢着灿烂的笑容:

``不过,这句话还需要改一改。''

\section[\thesection]{}

隆庆六年(1572)六月十五日  

冯保一早就找到了皇帝,向他报告一个极为重要的情况:经过自己的缜密侦查,发现了高拱图谋不轨的阴谋。

既然是阴谋,既然是图谋不轨,那自然要听听的,于是十岁的万历皇帝好奇地抬起头准备听故事,旁边站着紧张到
极点的李贵妃。

当然了,冯保是有犯罪证据的,且证据确凿,具体说来是一句话:

``十岁孩童,如何做天子!''

从``十岁太子,如何治天下''到``十岁孩童,如何做天子'',只改了几个字,就从牢骚变成了谋反,中国文化之博
大精深,实在让人叹为观止。

虽然张居正搞文字狱,耍两面派,狡诈阴险到了极点,但他还是说错了一点----真正能够解决高拱的,不是皇帝,
而是皇帝他妈。

皇帝他妈,就是李贵妃,通俗叫法是李寡妇。

用这个称呼,绝无不敬之意,只是她确实是个寡妇,而且是非多。

我在外地讲学的时候,曾几次谈到张居正,讲完后下面递条子上来提问,总有这样一个问题:据说李太后(即李贵
妃)和张居正有一腿,不知是否属实。

遇到这种情况,我总是十分认真地回答那位认真的求知者:不知道。

我确实不知道,因为即使他们俩之间有什么冬瓜豆腐,史书也不会写,至于野史,张大人和李寡妇连孩子都有了,
这种事情,乱讲小心被雷劈死。

但这些传言充分说明,李贵妃是一个不一般的女人。她并不是什么名门闺秀,只是一个宫女出身,但据说人长得很
漂亮,是宫里面的头号美女,而且工于心计,城府很深,是一块搞政治的材料。

所以在当时,真正拿主意的并不是穿衣服都不利索的万历,而是这位李寡妇。

于是李寡妇愤怒了,皇帝刚刚去世,你高拱竟然来这么一下,欺负我们孤儿寡母!

为了把戏做全,做大,据说张居正也出场演了一回,还和冯保唱了双簧,说高拱准备废了万历,另立藩王,讲得有
鼻子有眼。

这下子连十岁的万历都憋不住了,张大人和冯太监的谎言深深地伤害了他幼小的心灵,直到后来高拱死了,他连个
葬礼仪式都不批,可见受毒害之深厚。

李贵妃就更不用说了,高拱那个干瘦老头,一看就不是好人,张居正自然不同了,不但有才能,而且长得帅,不信
他还信谁?

就这么定了!

\section[\thesection]{}

隆庆六年(1572)六月十六日 成败就在今日

高拱十分兴奋,因为一大早,宫里就传来了消息,命令六部内阁等机关领导进宫开会,在他看来,这必定是弹劾起
了作用,皇帝要表态了。

想到多日的筹划即将实现,高拱按捺不住心中的喜悦,一反常态,派人去找张居正与高仪一起走,他要所有的人都
亲眼目睹他的胜利。

然而让他想不到的是,前几天还活蹦乱跳的高仪竟然病了,而且病得很重,什么病不知道,反正是不能走路。

可见老实人虽然老实,却未必不聪明。

张居正就更搞笑了,他的回答很干脆:

``我前几天中暑,就不去了。''

这个谎话明显没编好,不说中风瘫痪,至少也说你瘸了才好办,中暑又死不了人,大不了抬你去嘛。

于是高拱再三催促,还说了一句之后看来很可笑的话,以鼓励张居正:

``今天进宫理论,如果触怒皇上,我就辞职不干了,你来当首辅!''

张居正连忙摆手,大声说道:

``哪里,哪里,不要开这样的玩笑!''

首辅嘛,我是要当的,不过,无须你让。

禁不住高拱的一片热情,张居正还是上路了,不过他说自己不太舒服,要慢点走,高大人你先去,我随后就到。

这么看来,张居正还算个厚道人--至少不愿看人倒霉。

高拱兴冲冲地朝早朝地点无极殿走去,却意外地发现,一个手持圣旨的人已经站在了道路中间,于是他跪了下去,
准备接受喜报:

``先帝宾天(即挂)之日,曾召集内阁辅臣,说太子年幼,要你们辅政,但大学士高拱却专权跋扈,藐视皇帝,不知
你到底想干什么?''

骂完了,下面说处理结果:

``高拱回籍闲住,不许停留!''

从听到专权跋扈四个字开始,高拱就陷入了半昏迷状态:明明是自己找人黑了冯保,怎么会被人反攻倒算?这位几十
年的老江湖彻底崩溃了,从精神,到肉体。

据史料记载,这位兄台当时的表现是面如死灰,汗如雨下,趴在地上半天不动窝。

但这里毕竟是宫里的御道,你总这么占着也不是个事,高先生还没有悲痛完,就感觉一双有力的手把自己扶了起
来,所谓雪中送炭,高拱用感激的眼神向身后投去了深情地一瞥,却看见了张居正。

\section[\thesection]{}

张居正没有食言,他还是来了,时间刚刚好,圣旨念完,人还没走。看起来,他刚知道这个消息,脸上布满了痛苦
的表情。

刚看到张居正时,高拱险些产生了错觉,明明是自己被罢了官,这位仁兄怎么比我还难受,活像死了亲爹?

但张居正没有让他想太久,当即叫来了两个随从,把高学士扶了出去。

高拱的命运就此终结,他聪明绝顶,历经三朝,审时度势,在狂风暴雨中屹然不倒,熬过了严嵩、赶走了赵贞吉、
殷士儋以及一切敢于挡路的人,甚至连徐阶也被他一举拿下,最后却败在了这个人的手下,这个他曾经无比信任的
同志与战友。

啥也别说了,这就是命。

离开皇宫的高拱却没有心思去想这些,他必须马上就走。因为圣旨的命令是``不许停留'',说滚就滚,没有二话。

这是一个十分严厉的处理,一般官员被罢职,都能领到一张通行证,凭着证件,可以免费领取马匹,在路上还可以
住官方招待所(驿站),毕竟为朝廷干这么多年,没有功劳也有苦劳,给个人性化待遇不过分。

然而高拱却分毫没有,只等到了一群手持刀剑的大兵,催促他赶紧滚蛋,于是这位曾经权倾天下的大哥只好找了几
头骡子,将就着出了城,后面的人还不依不饶,一直把他赶出二十里外才回京,真是有够狠。

离开了京城,刚刚喘口气,却又遇上一个等候他们多时的人,与当兵的不同,这个人手上拿着一样高拱急需的东
西----驿站使用通行证。

然而高拱却没有接受,因为这位兄弟自报了家门:张大学士派我来的。

张居正实在很体贴,他一手导演了那道圣旨的诞生,自然也知道高拱的待遇,所以他派人等在这里,就当是送给高
拱的退休礼物,朝廷第一号善人非他莫属。

何谓善人?

做好事要不留名,做坏事要擦屁股,这就叫善人。

第一个独裁者

高拱愤怒了,他不是白痴,略加思考,就明白自己上当了,这个所谓的战友同志,竟是个不折不扣的叛徒败类,然
而为时已晚。

赶我走的是你,送我通行证的也是你,既上香又拆庙,你装什么孙子?

所以他用自己剩下唯一的方式表示了抗议----不收。

\section[\thesection]{}

气鼓鼓的高拱扭头就走,在此后的岁月中,他埋头于学术研究,偶尔也骂一骂张居正,为表示对此人的蔑视,他给
了这位昔日同事一个响亮的称呼----荆人(张居正是湖广荆州人)。

人走了,事情也该完了,这是高拱的想法。

然而事实证明,他实在是高估了张居正的道德水平,玩死人不偿命的把戏还在后头。

此时,最为得意的莫过于张居正了,他巧妙地利用了冯保与高拱的矛盾,只出了几个点子,就整倒了这位老到的政
治家,为这个延续了三十余年的死亡游戏画上了句号。

自嘉靖二十七年起,在嘉靖的英明怠工下,大明王朝最为优秀的六位天才开始了角逐,除了一边看热闹的杨博外,
大家都赤膊上阵,近身肉搏,徐阶等死了陆炳,除掉了严世藩,把持了朝政,却被高拱一竿子打翻,家破人亡,之
后高调上台,风光无限。

然而胜利最终却属于一直低调的张居正,他等到了最后,也熬到了最后,在暗处中用一记黑枪干掉了高拱,成为了
游戏的终结者。

严嵩输给了徐阶,不是正义战胜邪恶,而是他不如徐阶狡猾,徐阶输给了高拱,不是高拱更正直,而是因为他更精
明,现在,我除掉了高拱。所以事实证明,我才是这个帝国最狡诈,最杰出的天才。

再见了,我曾经的朋友,再见了,我曾经的同僚,你的雄心壮志,将由我去实现。

其实我们本是同一类人,有着同样的志向与抱负,我也不想坑你,但是很可惜,那个位置实在太挤。

大臣是我的棋子,皇帝是我的傀儡,天下在我的手中,世间已无人是我的对手。

好吧,那么开始我的计划吧,现在是时候了。

一般说来,当官能混到张居正这个份上,也就算够本了。

高拱走了,内阁里只剩下他一个人,但凡有什么事情,都由他批示处理意见,批完后,去找死党冯保批红、盖章。
他想怎么办,就怎么办。

而皇帝同志基本上可以忽略不计,这位仁兄刚十岁,能看懂连环画就算不错了,加上皇帝他妈对他还挺暧昧,孤儿
寡母全指望他,朝中大臣也被他治得服服帖帖,一句话,从高拱走的那一刻起,大明王朝的皇帝就改姓张了。

而现在,张皇帝打算干一件朱皇帝干不了的事情。

\section[\thesection]{}

纵观中国历史,一个老百姓家的孩子,做文官能做到连皇帝都靠边站,可谓是登峰造极了,要换个人,作威作福,
前呼后拥,舒舒坦坦地过一辈子,顺便搞点政绩,身前享大福,身后出小名,这就算齐了。

然而事实告诉我们,张居正不是小名人,是大名人,大得没边,但凡有讲中国话的地方,只要不是文盲村,基本都
听过这人。

之所以有如此成就,是因为他干过一件事情--改革。

什么叫改革?通俗的解释就是,一台机器运行不畅,你琢磨琢磨,拿着扳手螺丝刀上去鼓捣鼓捣,东敲一把,西碰一
下,把这玩意整好了,这就叫改革。

看起来不错,但要真干,那就麻烦了,因为历史证明,但凡干这个的,基本都没什么好下场,其结局不外乎两种:
一种是改了之后,被人给革了,代表人物是王安石同志,辛辛苦苦几十年,什么不怕天变,不怕人怨,最后还是狼
狈下台,草草收场。

另一种则更为严重,是改了之后,被人革命了,代表人物是王莽,这位仁兄励精图治,想干点事情,可惜过于理想
主义,结果从改革变成了革命,命都给革没了。

由此可见,改革实在是一件大有风险的事情,归根结底,还是因为两个字----利益。你要明白,旧机器虽然破,可
大家都要靠它吃饭,你上去乱敲一气,敲掉哪个部件,没准就砸了谁的饭碗,性格好的,找你要饭吃,性格差的,
抱着炸药包就奔你家去了。总之是不闹你个七荤八素誓不罢休。

如果把天下比作一台机器,那就大了去了,您随便动一下,没准就是成千上万人的饭碗,要闹起来,剁了你全家那
都是正常的。

所以正常人都不动这玩意,动这玩意的人都不怎么正常。

然而张居正动了,明知有压力,明知有危险,还是动了。

因为他曾见过腐败的王爷,饿死的饥民,无耻的官员,因为他知道,从来就没有什么救世主,也不能靠神仙皇帝。
因为他相信,穷人也是人,也有生存下去的权利。

因为在三十余年的勾心斗角,官场沉浮之后,他还保持着一样东西----理想。

\section[\thesection]{}

在我小时候,一说起张居正,我就会立刻联想到拉板车的,拜多年的胡说八道教育所赐,这位仁兄在我的印象里,
是天字第一号苦人,清正廉明,努力干活,还特不讨好,整天被奸人整,搞了一个改革,还没成功,说得你都恨不
得上去扶他一把。

一直十几年后,我才知道自己被忽悠了,这位张兄弟既不清正,也不廉明,拉帮结派打击异己,那都是家常便饭,
要说奸人,那就是个笑话,所有的奸人都被他赶跑了,你说谁最奸。

更滑稽的是,不管我左看右看,也没觉得他那个改革失败了,要干的活都干了,要办的事都办了,怎么能算失败?

所以我下面要讲的,是一个既不悲惨,也不阴郁的故事,一个成功的故事。

在张居正之前,最著名的改革应该就是王安石变法,当然,大家都知道,他失败了。

为什么会失败呢?

对于这个可以写二十万字论文的题目,我就不凑热闹了,简单说来一句话:

王安石之所以失败,是因为他自以为聪明,而张居正之所以成功,是因为他自以为愚蠢。

在这个世界上,所有存在的东西,必有其合理性,否则它就绝不会诞生。而王安石不太懂得这个道理,他痛恨旧制
度,痛恨北宋那一大帮子吃闲饭的人,但他不知道的是,旧有的制度或许顽固,或许不合理,却也是无数前人伟大
智慧的结晶,制定制度和执行制度的人,都是无以伦比的聪明人,比所有自以为聪明的人要聪明得多,僵化也好,
繁琐也罢,但是,能用。

所以这位老兄雄心勃勃,什么青苗法搞得不亦乐乎,热火朝天,搞到最后却不能用,所以,白搭。

而张居正就不同了,他很实在。

要知道,王安石生在了好时候,当时的领导宋神宗是个极不安分的人,每天做梦都想打过黄河去,解放全中国,恨
不得一夜之间大宋国富民强,所以王安石一说变法,就要人有人,要钱有钱。

相比而言,嘉靖就懒得出奇了,反正全国统一,他也没有征服地球的欲望,最大的兴趣就是让下面的人斗来斗去。
张居正就在这样的环境中成长,从小翰林到大学士,他吃过苦头,见过世面,几十年夹缝中求生存,壮志凌云,那
是绝对谈不上了。

所以在改革的一开始,他就抱定了一个原则----让自己活,也让别人活,具体说来,就是我不砸大家的饭碗,大家
也不要造我的反,我去改革,大家少贪点,各吃各的饭,互不干扰。

改而不革,是为改革。

\section[\thesection]{}

似乎上天也想成全张居正,他刚接任首辅,大权在握不久,就获知了另一个好消息--高仪死了。

高仪同志不愧是天下第一老实人,自从高拱被赶走后,便开始寝食不安,唯恐张居正手狠心黑,连他一锅端了,日
复一日,心理压力越来越大,一个月后就吐血而死,去阎王那里接着做老实人了。

对高仪的死,张居正丝毫不感到悲痛,因为从根子上说,他和高拱是同一类人,却比高拱还要独裁,看见有人在眼
前晃悠就觉得不爽,管你老实不老实,死了拉倒。

其实这也怪不得张居正,因为在中国历史上,共同创业的人大都逃不过``四同''的结局----同舟共济----同床异
梦----同室操戈----同归于尽。

于是自嘉靖登基时起,经过五十余年的漫长斗争,张居正终于一统天下,上有皇帝他妈支持,下有无数大臣捧场,
外有亲信戚继光守边界,内有死党冯保管公章,皇帝可以完全无视,他想干什么就干什么,比真皇帝还皇帝,一呼
百应,真正实现了团结。

把所有不服你的人都打服,敢出声就灭了他,所有人都认你当老大,这就叫实现团结。

团结之后的张居正终于可以实现他的理想了,这就是后来被无数史书大书特书的``张居正改革''。

说起改革,总有一大堆的时间、地点、人物以及背景、意义等等等等,当年本人深受其害,本着我不入地狱,谁入
地狱的精神,就不罗嗦了,简单说来,张居正干了两件事情。

第一件事,叫做一条鞭法。这个名字很不起眼,但这件事情却极其重大,用今天的话说,那是具有跨时代的意义。

因为这个跨时代的一条鞭法,改变了自唐朝以来延续了八百余年的税制,是中国赋税史上的一个具有里程碑意义的
转变。

上面这段话是我在历史论文中用的,看了头晕也别见怪,毕竟这话不说也不行,把伟大意义阐述完了,下面说实在
的,保证大家都能看懂:

自古以来,国家收税,老百姓交税,那是天经地义的事情,毕竟朱重八等人不是慈善家,出生入死打江山,多少得
有个盼头。

\section[\thesection]{}

怎么收税,各朝各代都不同,但基本上税的种类还是比较固定的,主要分为三块:

一是田税,皇帝拼死拼活抢地盘,你种了皇帝的地,自然要交钱。

二是人头税,普天之下莫非王土,天下百姓都是皇帝的子民(都是他的资源),有几个人交几份钱,这是义务。

三是徭役,说穿了就是苦力税,所谓有钱出钱,有力出力,遇到修工程,搞接待的时候,国家不但要你出钱,有时
还要你出力。两手一起抓,一个都不能少。

有人可能会说,要是我那里都是山,没田怎么办呢?或者说我有田,但不种粮食,又怎么办呢?这个你不用担心,国
家早就替你想好了,权利可以不享受,义务绝对跑不掉。

简单说来是有什么交什么,山里产蘑菇,你就交蘑菇,山里产木材,你就交木材,田里要种苹果,你就交苹果,要
种棉花,你就交棉花,收起来放仓库,反正一时半会也坏不了。

个把``刁民''可能会问:那我要是捕鱼的渔民呢,你又没冰箱,鱼总不能放着发臭吧?

嘿嘿,放心,朝廷有办法,做成咸鱼不是照样交吗?跑不了你小子。

中国的老百姓上千年就背着这么三座大山,苦巴巴地熬日子。

实事求是地讲,在中国历史上,大一统王朝的统治者,除了某一些丧心病狂,或是急等用钱的人外,对百姓负担还
是很重视的,田赋的比例基本都是二十比一(百分之五),或是十比一(百分之十),能收到五比一(百分之二十),就
算是重税了。

从这个数字看,老百姓的生活在理论上,还是能够过下去的。

不过很可惜,仅仅是理论上。

说起来是那么回事,一操作起来就全乱套。

因为在实际执行中,各级官吏很快发现,能钻空子捞钱的漏洞实在是太多了:比如你交苹果,他可以挑三拣四,拿
起一个,说这个个头小,算半个,那个有虫眼,不能算。你交棉花,他可以说棉花的成色不好,抵一半,你也只能
回家再拉去。

这还是轻的,最大的麻烦是徭役。因为田赋和人头税多少还能见到东西,县太爷赖不掉,徭役可就不好说了,修河
堤、给驿站当差、整修道路,这都是徭役,完成了任务,就算完成了徭役。

那么谁来判定你是否完成任务呢?--县太爷。

\section[\thesection]{}

这就是所谓的黄鼠狼看鸡了,遇到良心好的,还能照实记载,遇到不地道的,就要捞点好处,你要没钱,他就大笔
一挥----没干,有意见?这事我说了算,说你没干就没干,你能咋地?

事实证明,在当时,除了一小部分品行较好的人外,大多数朝廷官员还是不地道的,是不值得信任的,有漏洞不
钻,有钱不捞,这个要求实在有点高。总之是一句话,玩你没商量。

无数的老百姓就是这样被玩残的,朝廷没有好处,全被地方包干了。

此外,这一收税制度还有很多麻烦,由于收上来的都是东西,且林林总总,花样繁多,又不方便调用。

比如江浙收上来一大堆粮食,京城里吃不了,本地人又不缺,听说西北缺粮食,那就往那边运吧?一算,粮价还不够
运输费。那就别折腾了,放在粮仓里喂老鼠吧。

更头疼的是,各地虽然上交了很多东西,除了粮食,还有各种土特产,中药药材等等,却没有多少银两,这些玩意
放在京城里又占地方,还要仓管费,遇上打仗,你总不能让当兵吃棉花,提几两药材当军饷吧。

而某些吃饱饭的大臣无聊之中,想了个馊主意,说既然有这么多东西,闲着也是闲着,不如拿去给京城的官员们发
工资,比如你是户部正六品主事,按规定你该拿多少工资,但到发钱那天告诉你,国家现金不够,我们现在只能发
一部分钱和粮食给你,剩下的用棉花抵,不过你放心,我们到市场上估算过,如果等价交换,拿这些棉花绝不吃亏。

奶奶的,老子辛辛苦苦干到头,就拿着这几袋棉花回家?老婆孩子吃什么?

必须说明,这绝对不是搞笑,自朱元璋以来,明代官员都是这么领工资的,有时是粮食,有时是药材,个别缺了大
德的皇帝还给纸币(胡乱印刷的不值钱),早上领工资,下去就去集贸市场兼职小商贩叫卖的,也绝不在少数。

国家吃了亏,百姓受了苦,全便宜中间那帮龟孙了。

于是张居正决定,改变这一局面,他吸取地方经验,推出了一条鞭法。

一条鞭法的内容很多,但最主要的,是颁布统一规定,全国税收由实物税变为货币税,明白点说就是以后不收东西
了,统一改收钱。

\section[\thesection]{}

这是一个看上去很简单的命令,却有着绝不简单的历史意义。

因为从此以后,不管是田赋、徭役还是人头税,都有了统一的标准,不是当官的说了算,交上来真金白银,有就是
有,没有就是没有,不再任由官员忽悠。

当然了,根据官员必贪定律,张居正也给大家留下了后路,因为各种物品如粮食、水果、药材、丝绸,都按照规定
折算成银两上缴,而折算比率虽是由朝廷掌握,但地方上自然有特殊情况,适当照顾照顾,从中捞一笔,似乎也是
很正常的。

于是皆大欢喜,朝廷拿到的,是白花花的银子,老百姓也不用听凭官员糊弄,贪也好,抢也好,说好了宰一刀就宰
一刀,至少日子好过点。官员们好处少了,但也还过得不错,就这么着了。

所以事实证明,越复杂的政策,空子就越多,越难以执行,王安石就大体如此,一条鞭法虽然看似简单,却是最高
智慧的结晶,正如那句老话所说:

把复杂的问题搞简单,那是能耐。

张居正和他的一条鞭法就此名留青史,并长期使用,而那三座大山也一直没动窝,雍正时期实行摊丁入亩,将人头
税归入田赋,才算化三为二(实际上一点都没减,换了个说法而已),徭役直到解放后才正式废除,而历史最为悠久
的田赋,也就是所谓的农业税,前几年也正式得以停征。

社会主义好,这是个实在话。

张居正干的第二件事情,其实是由一封信引起的。

万历元年(1573),张居正上书皇帝,当然了,其实就是上书给他自己,在这封自己给自己的信中,他写下了这样一
句话:

``月有考,岁有稽,使声必中实,事可责成。''

一个历史上鼎鼎大名的政策就此诞生,而它的名字,就是此句的头尾两字----考成。

这就是张居正改革的第二大举措----考成法。

如果你不知道考成法,那很正常,但如果你没有被考成法整过,那就不正常了,因为从某种意义上讲,这个考成大
致就相当于今天的考勤。

张居正搞出了一整套制度,但他很清楚,制度是次要的,执行是主要的,指望自己手下这群懒汉突然良心发作,辛
勤工作,那是天方夜谭。

所以经过反复思索,张大学士想出了这个绝妙的办法。

\section[\thesection]{}

张居正的办法,就是记账。比如一个知府,每年开初就把要完成的工作一一列明,抄录成册,自己留一份,张居正
那里留一份,到了年底一对,如果发现哪件事情你没做,那就恭喜你了,收拾东西准备去县城吧。

如果你到了县城依然如此,对你的处分也依然如此,直到捆被子滚蛋为止。

该法令适用范围近似于无穷大,从中央六部到边远山区,如不照办,一概都照章处理。

按照以往规律,新官上任三把火,雄心勃勃一回,烧完之后该干嘛就干嘛,所有有些官兄也不在意,以为咬牙挺一
挺就过去了,可他们把牙咬碎,也没等到完事的那一天。

张居正这次是动真格的,真格到了有点恶心人的地步,比如万历三年(1576),有人反映,赋税实在太难收,你说收
十万就十万,遇到欠收你让我去哪淘银子?

事实证明,张学士还是很民主的,很快,他就颁布规定,从今以后地方赋税,只要收到一定数量,就算没收全,也
可以不处分。

但指标下来了,大家都高兴不起来,因为这个``一定数量''是九成。

这明摆着是把大家涮着玩,我能收到九成,还用叫苦吗?然而张先生用行动告诉大家,收不收得到,那是你的事,处
不处分你,那是我的事。

第一个当火锅底料的,是山东的一群难兄难弟,运气实在不好,死收活收就是没收全,更可笑的是,其中有位仁
兄,赋税收到了八成八,还是被咔嚓一刀,全部集体降级。

于是从此以后,官员们一改往日作风认真干活,兢兢业业,只求年底弄个考核合格,那就菩萨保佑了,工作效率也
得以大幅度提高。

当然了,考成法能够实施,那还要靠张居正,要知道这位兄弟当年也是一路混过来的,朝廷里那些歪门邪道,贪污
伎俩,他都清清楚楚,想当初他老人家捞钱的时候,下面这帮小年轻还在啃烧饼。如今最滑的老滑头当权,谁敢跟
他玩花样。

以上就是考成法的主要内容,但并非全部内容,因为事实上,张居正相当狡猾,在那封信中,他还偷偷夹杂了一句
极为重要的话,以实现他的个人目的,这句话很不起眼,却是他死后被人清算的真正原因。

这事留到后面讲,因为光荣事迹还没说完。

\section[\thesection]{}

在张居正的严厉督促下,官员们勤勤恳恳,努力工作,国家财政收入不断上升,自正德以来走下坡路的明朝,又开
始爬坡了。

内政蒸蒸日上的同时,明军的实力也得到了进一步的加强----因为几位猛人的存在。

戚继光自然是头把交椅,虽说他只是个总兵,职务比谭纶和王崇古要低,但大家心里都清楚,这个人的后台太硬,
哪怕是兵部尚书,每次到蓟州视察,对戚总兵都是客客气气的。

而事实也是如此,张居正对戚继光实在是好得过了头,下属不听话了,换!副手不听话了,换!上司不听话了,换!

这么一搞,就把戚继光搞成了个无人敢碰的角色,大家都对他尊敬有加,偏偏这位戚大哥还很会来事,每次京城有
领导来参观,他都要亲自作陪,请吃请喝请娱乐,完事还要送土特产,据说都是用车拉回去的,如此猛料的人物,
谁惹?

在戚继光之前,十七年间,蓟州总兵换了十个人,平均任期1.7年,没办法,这个鬼地方,天天有蒙古人来转悠,守
这里不是被打跑,就是被打死,运气好的被抓回去追究责任,实在没法呆。

但戚继光就不同了,他到这里之后,只打过几个小仗,之后一直镇守边界十六年,竟然没人敢来。

究其原因,还是他守得太好,刚到边界不久,他就大力推广修建烽火台,把城墙连成一片,形成了稳固的防御体
系,此外,他还大力发展火器,基本上是人手一杆枪。原先在浙江打日本人,好歹还用个鸳鸯阵,现在索性就不搭
理人了,蒙古骑兵每次来,还没等挨着城墙,就被一阵乱枪扫射,等你在城外跑累了,再派兵出去打落水狗,这么
个折腾法,蒙古人实在受不了,长此以往,大家就都不来了。

由于戚继光这边密不透风,蒙古部落就跑到辽东去混饭吃,希望有条生路。

可惜的是,镇守辽东的,恰恰是李成梁,这位李总兵堪称当时第一号横人,他所管辖的地方,既不修城墙,也不搞
火器,防务看似十分松懈,所以很多蒙古人慕名而来,想抢一把,可是事实告诉他们,李总兵虽然不砌墙头,却擅
长扔砖头。

他之所以不守,只是因为他喜欢进攻。

\section[\thesection]{}

别人都怕骑兵,唯独李成梁不怕,因为他是当时明朝最为优秀的骑兵将领,手下有一支精锐的骑兵,人称``辽东铁
骑''。

这支部队战斗力极强,在他镇守期间,出战三十余次,战无不胜,经常追着蒙古人到处跑,让人闻风丧胆,是后来
天下第一强军``关宁铁骑''的前身。

当然,这位兄台因为打仗太多,杀人太狠,也有点混,还惹了个大祸,这些都是后来的事情,到时再讲。

蓟州和辽东有这两人守着,宣大那边也不打了,大家正忙着做生意,没有功夫打仗,于是困扰了明朝几百年的边界
问题终于得以缓解。

国库充裕,边界安宁,大明王朝已经建立了两百年,混到这时候竟然还有如此局面,不能不说是个奇迹,而这一切
的缔造者,正是张居正。

在国家陷入深重危机,财政入不敷出,流民四处闹事,政治腐败不堪的情况下,张居正以他深不可测的心计,阴险
无比之手段,夺取了最高领导权,并发挥其不世出之奇才,创造性地进行了伟大的政治运动----和稀泥,在尽量不
得罪人的情况下把事给办了,为明朝迎来了新的生机,无愧于最杰出的政治家的称号,堪称国家之栋梁,民族之骄
傲。

好话说完了,下面说坏的。

张居正这人,说他是老实人,那就是见鬼,老实人坐不到他这个位置,说他是好人,也不太靠谱,毕竟他干了很多
好人都干不出的事情,确切地说,他是个猛人。

关于这一点,王世贞同志是很有感慨的。

在嘉靖万历年间,第一才子的名头牢牢地挂在这位仁兄的脖子上,连徐渭都比不上他,因为他不但是著名的文学
家,还是戏剧家、诗人、画家、文艺评论家、史学评论家,极其有名,有名到他头天晚上喝醉了,说谁谁不错,是
个牛人,第二天无论这人是不是真牛,立马就能变成名人,明史说他``书过目,终身不忘'',有这种特异功能,实
在不是吹出来的。

但问题在于这位名人虽然身负大才,写了不少东西,这辈子也就干了两件事,第一是骂严嵩,第二就是骂张居正,
骂严嵩已经讲过了,那是个人恩怨,骂张居正就不同了。

\section[\thesection]{}

在这件事情上,王世贞投入了很大精力,说张先生贪污受贿玩女人,有严重的经济问题和生活作风问题,既然受
贿,那就得有人行贿,为了证明这一点,他连传统正面形象,民族大英雄戚继光也不放过,把他一把拉下了水,说
戚继光送了几个女人给张居正,搞得后来许多主旋律作家十分难堪,对此统统无视。

他的骂法也很特别,不是几天的事,一骂就是若干月,若干年,骂得实在太频繁,太上瘾,骂得我耳朵都起了茧,
其实在明代,朝廷官员捞点钱很普遍,工资太低,咱中国人又爱讲个排场,不捞钱咋活得下去?至于女人问题,那就
真是恶搞了,据我所知,王世贞的老婆也不少。

不过话说回来,王世贞被后世称为历史学家,还比较客观公正,虽说他有点愤青,但大致情况还是靠谱的,之所以
这么恨张居正,是因为张居正太猛,而他这一辈子最恨飞扬跋扈的人(比如严嵩),然而他是个文人,张居正是个猛
人,也只能是有心杀贼,无力回天了。

因为猛人可以整人,文人却只能骂人。

下面我们就来介绍一下猛人张居正的主要事迹,看完之后你就能发现,猛人这个称呼可谓名不虚传。

张猛人的第一大特征是打落水狗,在这一点上,他和他的老师徐阶有一拼,一旦动手,打残是不足的,打死是不够
的,要打到对手做鬼了都不敢来找你,这才叫高手。

徐阶是这么对付严嵩的,张居正是这么对付高拱的。

自打被张居正赶回家,高拱就心如死灰,在河南老家埋头做学问,但让他想不到的是,几百里外的京城,一场足以
让他人头落地的阴谋即将上演。

万历元年(1573)正月二十日晨 大雾

十岁的万历皇帝起得很早,坐上了轿子,准备去早朝,在浓雾之中,他接近了那个遭遇的地点----乾清门。

就在穿过大门之时,侍卫们忽然发现了一个形迹可疑的人,当即上前围住,并将此人送往侍卫部门处理。

这一切发生得相当突然,在这片灰蒙蒙的迷雾中,忽然开始,又忽然结束,加上那位被捕的兄弟没有反抗,所以并
没有引起太多人的注意,而皇帝还小,要他记住也难。

在这片神秘的雾中,事情似乎就这么过去了,然而事实证明,这只不过是那个致命阴谋的开始。

\section[\thesection]{}

三天之后,相关部门向内阁上交了一份审讯报告,一份莫名奇妙的报告:

擅自闯入者王大臣,常州武进县人,身带刀剑一把,何时入宫不详,如何入宫不详,入宫目的不详,其余待查。

这里说明一下,这位不速之客并不是大臣,他姓王,叫大臣(取了这么个名,那也真是个惹事的主)。

张居正一看就火了,这人难道是钢铁战士不成?你们问了三天,就问出这么个结果?

然而转瞬之间,他突然意识到,这是一个机会,一个千载难逢的良机。一丝笑容在他的嘴角绽放。

很好,就这么办。

一天后,王大臣被送到了新的审讯机关,张居正不再担心问不出口供,因为在这个地方,据说只有死人才不开
口----东厂。

据某些史料记载,东厂的酷刑多达三十余种,可以每天试一种,一个月不重样。有如此创意,着实不易。

但张居正的最终目的并不是让他开口说真话,他要的,只是一句台词而已。

然而王大臣同志似乎很不识相,东厂的朋友用刑具和他``热烈交谈''一阵后,他说出了自己的来历,很不巧,恰恰
是张居正最不想听到的:

``我是逃兵。''王大臣说道,``是从戚继光那里跑出来的。''

来头确实不小。

这下头大了,这位兵大哥竟然是还是戚继光的手下,带着刀进宫,还跑到皇帝身边,必定有阴谋,必定要追究到
底,既然有了线索,那就查吧,顺藤摸瓜,查社会关系,查后台背景,先查当兵的,再查戚继光,最后查……

小子,你想玩我是吧!

没关系,反正人归东厂管,东厂归冯保管,既然能让他开口,就必定能让他背台词。

于是在一阵紧张工作之后,王大臣又说出了新的供词:

``我是来行刺皇帝的,指使我的人是高阁老(高拱)的家人。''

不错,这才是最理想的供词,冯保笑了,张居正也笑了。

看着眼前低头求饶的王大臣,两人相信,高拱这次是完蛋了。

然而事实证明,这两位老奸巨猾的仁兄还是看错了,不但看错了形势,还看错了眼前的这个逃兵。

当审讯结果传出之后,反响空前激烈,以往为鸡皮蒜毛小事都能吵上一天的大臣们,竟然形成了空前一致的看
法----栽赃。

\section[\thesection]{}

这都是明摆着的,先把人搞倒,再把人搞臭,最后要人命,此套把戏大家很清楚,拿去糊弄鬼都没戏。

于是在供词公布后不久,许多人明里暗里找到张居正,希望他不要再闹,及早收手,张大人毕竟是老狐狸,一直装
聋作哑,啥也不说,直到另一个人找上门来。

别人来可以装傻,这个人就不行了,因为他不但是老资格,还曾是张居正的偶像----杨博。

杨老先生虽然年纪大了,战斗力却一点不减,关键时刻挺身而出,准备为高拱说情。

但对于他的这一举动,我还着实有点好奇,因为这位仁兄几十年来都是属于看客一族,徐阶也好,严嵩也罢,任谁
倒霉他都没伸过手,而根据史料记载,他和高拱并无关系,这次竟然良心发现,准备插一杠子,莫不是脑筋突然开
了窍?

于是怀着对他的崇敬,我找了许多资料,排了一下他的家谱,才终于找到了问题的答案。

杨博和高拱确实没有关系,但他有个儿子,名叫杨俊卿,而很巧的是,杨俊卿找了个老婆,岳父大人偏偏就是王崇
古。

王崇古和高拱就不必说了,同学兼死党,王总督的这份工作还是高拱介绍的,不说两句话实在不够意思。

没有无缘无故的爱,也没有无缘无故的恨,我信了。

杨大人开门见山,奔着张居正就去了:

``你何苦做这件事情?''

这句话就有点伤自尊了,张居正立刻反驳:

``事情闹到这个地步,你认为是我安排的吗?''

``我不是这个意思'',杨博终究还是说了句实诚话,``但只有你,才能解决这件事。''

张居正沉默了,他明白,杨博是对的,高拱的生死只在自己的手中。

于是在送走了杨博之后,他决定用一个特殊的方法做出抉择----求签。

良久跪拜之后,张居正在庙里拿到了属于他的那一支签,当他看到上面内容的那一刻,便当即下定了决心。

据说在那支签上,只刻着八个字----所求不善,何必祷神!

但事情已经出了,收手也不可能了,于是他决定不参与其中,让冯保自己去审,并特意指定锦衣卫都督朱希孝一同
会审。

事实证明,这个安排充分体现了张居正卓越的政治天才,却苦了他的朋友冯保,因为很快,这位冯太监就将成为中
国司法史上的著名笑柄。

\section[\thesection]{}

万历元年(1573)正月二十九日,对王大臣的审讯正式开始,一场笑话也即将揭幕。

案件的主审官,是东厂管事太监冯保和锦衣卫都督朱希孝,这二位应该算是大明王朝的两大邪恶特务头子,可不巧
的是,那位朱都督偏偏就是个好人。

这位朱兄来头很大,他的祖上,就是跟随永乐大帝朱棣打天下,几十个人就敢追几千人的超级名将朱能,到他这
辈,虽说打仗是不大行了,但这个人品行不错,也还算个好人,觉得冯保干得不地道,打算拉高拱一把。

所以在审问以前,他仔细看了讯问笔录,惊奇地发现,王大臣的第一次口供与第二次口供有很多细节不对,明显经
过涂改,但更让他惊奇的是,这样两份漏洞百出的笔录,卷尾处得出的结论竟然是证据确凿。

于是他当即找来了当场负责审问的两个千户,拿着笔录笑着对他们说:这样的笔录,你们竟然也敢写上证据确凿?

那两名千户却丝毫不慌,只说了一句话,就让朱大人笑不出来了:

``原文本是没有的,那几个字,是张阁老(张居正)加上去的。''

朱希孝当即大惊失色,因为根据惯例,东厂的案卷笔录非经皇帝许可,不得向外人泄露,如若自行篡改,就是必死
之罪!

张居正虽然牛,但牛到这么无法无天,也实在有点耸人听闻。

所以在正式审问之前,朱希孝十分紧张,冯保和他一起主审,张居正是后台,如此看来,高拱这条命十有八九要下
课了。

然而当审讯开始后,朱希孝才发现自己错了,错得十分搞笑。

明代的人审案,具体形式和今天差不多,原告被告往堂上一站(当年要跪),有钱请律师的,律师也要到场(当年叫讼
师),然后你来我往,展开辩论,基本上全国都一样。

只有两个地方不一样,一个是锦衣卫,另一个是东厂。因为他们是特务机关,为显示实力,开审前,无论犯人是
谁,全都有个特殊招待----打板子。

这顿板子,行话叫做杀威棍,历史十分悠久,管你贵族乞丐,有罪没罪,先打一顿再说,这叫规矩。

事情坏就坏在这个规矩上。

\section[\thesection]{}

案台上朱大臣还没想出对策,下面的王大臣却不干了,这人脑筋虽有点迟钝,但一看见衙役卷袖子抄家伙,也还明
白自己就要挨打了,于是说时迟那时快,他对着堂上突然大喊一声:

``说好了给我官做,怎么又要打我!''

这句话很有趣,朱希孝马上反应过来,知道好戏就要开场,也不说话,转头就看冯保。

冯太监明显是被喊懵了,但毕竟是多年的老油条,很快做出了回应,对着王大臣大吼道:

``是谁指使你来行刺的!?''

话讲到这里,识趣的应该开始说台词了,偏偏这位王大臣非但不识趣,还突然变成了王大胆,用同样的语调对着冯
保喝道:

``不就是你指使我的吗,你怎么不知道?干嘛还要问我?''

朱希孝十分辛苦,因为他用了很大的力气,才憋住自己,没有笑出声,而他现在唯一感兴趣的,是冯保大人怎么收
这个场。

自打从政以来,冯保还没有遇到过这么尴尬的事情,事已至此,演戏也得演到底了,于是他再次大吼:

``你昨天说是高阁老指使你来的,为什么今天不说!?''

王大臣却突然恢复了平静,用一句更狠的话让冯保又跳了起来:

``这都是你让我说的,我哪里认识什么高阁老?''

丢脸了,彻底丢脸了,这句话一出来,连堂上的衙役都憋不住了,审案竟然审到这个份上,冯保寻死的心都有了。

关键时刻,还是朱大臣够意思,眼看搞下去冯太监就得去跳河,他也大喝一声:

``混蛋,竟敢胡说八道,诬陷审官,给我拖下去!''

这位兄弟还真是个好人,回头又笑着对冯保说了一句:

``冯公公,你不用理他,我相信你。''

我相信,当冯公公听到这句话时,应该不会感到欣慰。

闹到这个份上,高拱是整不垮了,自己倒有被搞掉的可能,为免继续出丑,冯保下令处死了王大臣,此事就此不了
了之。

但这依然是一个扑朔迷离的事件,王大臣一直在东厂的控制之下,为什么会突然翻供呢?他到底又是什么人呢?

我来告诉你谜底:

冯保并不知道,在他和朱希孝审讯之前,有一人已经抢先一步,派人潜入了监狱,和王大臣取得了联系,这个人就
是杨博。

\section[\thesection]{}

高拱走后,智商水平唯一可与张居正相比的人,估计也就是这位仁兄了,取得张居正的中立后,杨博意识到,冯保
已是唯一的障碍,然而此人和高拱有深仇大恨,绝不可能手下留情,既要保全高拱,又不能指望冯保,这实在是一
个不可完成的任务。

然而杨博名不虚传,他看透了冯保的心理,暗中派人指使王大臣翻供,让冯太监在大庭广众之下,吃了个哑巴亏,
最后只能乖乖就范。以他的狡诈程度,被评为天下三才之一,可谓实至名归。

而根据某些史料反映,这位王大臣确实是戚继光手下的士兵,因为犯错逃离了军队,东跑西逛,结果把命给丢了。

但疑问仍然存在,要知道皇宫不是公共厕所,想来就来想走就走,哪怕今天,您想进去,也得买门票,这位仁兄大
字不识,也没有通行证,估计也没钱,这么个家伙,他到底是怎么进去的?

不好意思,关于这个问题,我也没有答案,就当他是飞进去的好了。

高拱算是涉险过关了,无论如何,他还算是张居正的朋友,对朋友尚且如此,仇人就更不用说了,因为张猛人的第
二大特征就是有仇必报,在这一点上,他简直就是徐阶2.0版。

第一个刀下鬼,是辽王。

说起这位兄弟,实在让人哭笑不得,几十年一点正事没干过,从四岁到四十岁,除了玩,什么追求都没有。

小时候,他喜欢玩,玩死了张居正的爷爷,现在一把年纪了,还是玩,反正家里有钱,爱怎么玩就怎么玩!

然而玩完的时候还是到了。

一直以来,张居正都没有忘记三十年前,祖父被人整死的那一幕,君子报仇,三十年也不晚。

当时还只是隆庆二年(1568),张居正在内阁里只排第三,不过要对付辽王,那是绰绰有余。

很快,湖广巡按御史突然一拥而上,共同弹劾辽王,王爷同志玩了这么多年,罪状自然是不难找的,一堆黑材料就
这么报到了皇帝那里。

皇帝大人虽对藩王一向也不待见,但怎么说也是自己的兄弟,听说这人不地道,便派了司法部副部长(刑部侍郎)洪
朝选去调查此事。

\section[\thesection]{}

其实说到底,皇帝也不会把辽王怎么样,毕竟大家都姓朱,张居正对此也没有太大指望,教训他一下,出口恶气,
也就到头了。

然而他们都高估了一点----辽王的智商。

人还没到,也没怎么着,辽王就急了,在房里转了几百个圈,感觉世界末日就要来了,于是灵机一动,在自己家里
树了一面旗帜,上书四个大字``讼冤之纛'',壮志飘扬,十分拉风。

这四个字的大致意思,是指自己受了冤枉,非常郁闷,可实际效果却大不相同,因为辽王同志估计是书读得太少,
他并不清楚,这种行为可以用一个成语描述----揭竿而起,而它只适用于某种目的或场合。

于是他很快迎来了新的客人----五百名全副武装的士兵,而原先拟定的警告处分,也一下子变成了开除----废除王
位。

玩了一辈子的辽王终于找到了自己的归宿,他的余生将在皇室专用监狱中度过,也算是玩得其所了。

张居正解决的第二个对象,不是他的仇人,而是徐阶的死敌。

在高拱上台之后,张居正本着向前辈虚心学习的精神,总结了高拱的成功经验,在整理工作中,他惊奇地察觉了那
个神秘的人物----邵大侠。

张居正万万没想到,这个姓邵的二流子竟然有如此大的能量,且不说徐老师被他整得要死要活,如果任他乱搞一
通,没准有一天又能搞出个王拱,陈拱,也是个说不准的事情。

所以他想出了一个最简单的方法----杀掉他。

邵大侠既然是大侠,自然行踪不定,但张居正是大人,大人要找大侠,也不太难,隆庆六年(1572),在解决高拱之
后一个月,张居正找人干掉了邵大侠,这位传奇混混将在阎王那里继续他的事业。

第三个被张居正除掉的人,是他的学生。

隆庆五年(1571),作为科举的考官,张居正录取了一个叫刘台的人,在拜完码头之后,两人确立了牢固的师生关
系----有效期四年。

刘台的成绩不太好,运气倒还不错,毕业分配去了辽东,成为了一名御史,之前讲过,在明代御史是一份极有前途
的工作,只要积极干活,几年之后混个正厅级干部,也不会太困难。

刘台就是一个积极的御史,可惜,太积极了。

\section[\thesection]{}

万历三年(1575),辽东第一号猛人,总兵李成梁一顿穷追猛打,大败蒙古骑兵,史称``辽东大捷''。消息传来,巡
抚张学颜十分高兴,连忙派人向朝廷报喜,顺便还能讨几个赏钱。

结果到了京城,报信的人才发现,人家早就知道了,白讨了没趣。

张学颜气得直抖,因为根据规定,但凡捷报,必须由他报告,连李成梁都没有资格抢,哪个孙子活得不耐烦了,竟
敢抢生意!

很快人就找到了,正是刘台。

作为辽东巡按御史,刘台只是个七品官,但是权力很大,所以这次他自作主张,抢了个头彩。但他想不到,自己将
为这个头彩付出极其惨重的代价。

最先发作的人,并不是张学颜,而是张居正,他得知此事后,严厉斥责了学生的行为,并多次当众批评他,把刘台
搞得灰头土脸。

这是一个极不寻常的举动,按说报了就报了,不过是个先后问题,也没捞到赏钱,至于这样吗?

如果你这样认为,那你就错了,张居正同志向来不干小事,他之所以整治刘台,不是因为他是刘台,而是因为他是
御史。

高拱之所以能够上台,全靠太监,但他之所以能够执政,全靠言官,要知道,想压住手下那帮不安分的大臣,不养
几个狗腿子是不行的,而这帮人能量也大,冯保都差点被他们骂死,所以一直以来,张居正对言官团体十分警惕,
唯恐有人跟他捣乱。

刘台就犯了这个忌讳,如果所有的御史言官都这么积极,什么事都要管,那我张居正还混不混了?

然而张居正没有想到,他的这位学生是个二愣子,被训了两顿后,居然发了飚,写了一封奏折弹劾张居正。

如果说抢功算小事的话,那么这次弹劾就真是大事了,是一件前无古人,后无来者的大事!

张居正震惊了,全天下的人都可以骂我,只有你刘台不行!

自从明朝开国以来,骂人就成了家常便饭,单挑、群骂、混骂,花样繁多,骂的内容也很丰富,生活作风问题,经
济问题,政治问题,只要能想得出的,基本全骂过了,想要骂出新意,是非常困难的。

然而刘台做到了,因为他破了一个先例,一个两百多年来都没人破的先例----骂自己的老师。

\section[\thesection]{}

在明朝,大臣和皇帝之间从来说不上有什么感情,你帮我打工,我给你俸禄,算是雇佣关系,但老师和学生就不同
了,江湖险恶,混饭吃不容易,我录取了你,你就要识相,要拜码头,将来才能混得下去。

所以一直以来,无数``正义人士''骂遍了上级权贵,也从不朝老师开刀。因为就算你骂皇帝,说到底,不过是个消
遣问题,要骂老师,那可就是饭碗问题了。

张居正这回算是彻底没面子了,其实骂的内容并不重要,连你的学生都骂你,你还有脸混下去?

于是张居正提出了辞职,当然,是假辞职。

张居正一说要走,皇帝那里就炸了锅,孤儿寡母全靠张先生了,你走了老朱家可怎么办?

之后的事情就是走程序了,刘台的奏折被驳回,免去官职,还要打一百棍充军。

这时张居正站了出来,他说不要打了,免了他的官,让他做老百姓就好。

大家听了张先生的话,都很感动,说张先生真是一个好人。

张先生确实是一个好人,因为现仇现报实在太没风度,秋后算账才是有素质的表现。

刘台安心回家了,事情都完了,做老百姓未必不好,然而五年后的一天,一群人突然来到他家,把他带走,因为前
任辽东巡抚,现任财政部长(户部尚书)张学颜经过五年的侦查,终于发现了他当年的贪污证据,为实现正义,特将
其逮捕归案,并依法充军。

张居正的做事风格大体如此,很艺术,确实很艺术。

而张先生干掉的最后一个有分量的对手,是他当年的盟友。

万历七年(1579),张居正下令,关闭天下书院,共计六十四处。

这是一个策划已久的计划的开端。

从当政的那天起,张居正就认定了一个理念----上天下地,唯我独尊,具体说来,是但凡敢挡路的,不服气的,提
意见的,都要统统地干掉。

折腾几年之后,皇帝听话了,大臣也老实了,就在张居正以为大功告成之际,一个新的敌人却又出现在他的眼前。

这个敌人不同于以往,因为它不是一个人,甚至于不能算是人,而是一个极为特别的团体势力,它的名字叫做书院。

\section[\thesection]{}

书院是中国传统的教育形式,明代许多书院历史十分悠久,流传五六百年的不在少数,今天说起外国的牛津、剑
桥,一算历史多少多少年,简直牛得不行,再一看国内某大某大,撑死了也就一百多年,都不好意思跟人家打招呼。

实际上大可不必自卑,因为古代书院就是现代意义上的大学,不过是大学这词更时髦而已,要知道,欧洲最老的巴
黎大学,也就是1261年才成立,而且基本上都是教些神学之类的鬼玩意,这也难怪,当时欧洲都是一帮职业文盲,
骑着马,提着长矛到处冲,能读懂拉丁语的人扳着指头都能数出来,鬼才有心思上什么大学,中国的书院倒是有始
有终,一直之乎者也了上千年,到清朝末年,基本都停的停,改的改,这一改,就把历史也改没了,年头从头算起。

但在书院上千年的历史中,明代书院是极为特别的,因为它除了教书外,还喜欢搞政治。

所谓搞政治,也就是一些下岗或上岗的官员,没事干的时候去书院讲课,谈人生谈理想,时不时还骂骂人,发发脾
气,大致如此而已,看上去好像也没啥,但到嘉靖年间,一个大麻烦来了。

麻烦是王守仁同志带来的,因为此时他的思想已然成为了一种潮流,在当时的书院里,如果讲课的时候不讲心学,
那是要被轰下台的,按说讲心学就讲心学,似乎也没什么,可问题在于,心学的内容有点不妥,用通俗的话说,是
比较反动。

在这段时间,心学的主流学派是泰州学派,偏偏这一派喜欢搞思想解放、性解放之类的玩意,还经常批评朝政,张
居正因为搞独裁,常被骂得狗血淋头,搞得朝廷也很头疼。

这要换在徐阶时代,估计也没啥,可张居正先生就不同了,他是一个眼里不揉沙子的角色,无论是天涯还是海角,
只要得罪了他,那是绝对跑不掉的。一个人惹我,就灭一个人,一千个人惹我,就灭一千人!

于是在一夜之间,几乎全国所有有影响的书院都被查封,学生都被赶回了家,老师都下了岗。

事情到这里,似乎该结束了,然而张居正同志实在不是个省油的灯,他不但要抓群体,还要抓典型。

\section[\thesection]{}

  所谓抓典型,就是从群众之中,挑选一个带头的,把他当众干掉,以达到警示后人的目的。

  而这次的典型,就是何心隐。

  这位明代第一神秘人物实在太爱管闲事,在批评张居正的群众队伍里,他经常走在第一线。平日也是来无影去
无踪,东一榔头西一棍,打了就走,绝不过夜,而且上到大学士,下到街头混混,都是他的朋友,可谓神通广大。

  事实证明,他看人的眼光也很准,十四年前,当他离开京城之时,就曾断言过,兴灭王学之人,只在张居正。

现在他的预言终于得到了实现,以最为不幸的方式。

  在万历七年(1579)的一天,优哉游哉了半辈子的何心隐走到了人生的尽头,当他在外地讲学之时,湖广巡抚王
之垣突然派兵前去缉拿,将他一举抓获,带回了衙门,还没等大家缓过神来,官方消息已传出:根据朝廷惯例,犯
人刚到,衙门的兄弟们都要意思意思,给他两棍,没想到何心隐体质太弱,竟然一打就死。遗憾之至,已妥善安排
其后事,并予安葬。

事情一出,天下哗然,王学门人一拥而上,痛骂王之垣,但人已经死了,王巡抚又十分配合,表示愿意背这个黑
锅,也不发火,大家骂足了几个月,就此收场。

  当然了,这事到底是谁干的,大家心里都有数。

这位泰州学派的领军人物虽然通晓黑白,张居正大人却是黑白通吃,虽然何心隐是他老师(徐阶)的同门,虽然何心
隐曾经与他并肩作战,共同解决了严嵩。

  但对张居正而言,朋友还是敌人,只有一个判断标准:顺我者昌,逆我者亡!

曾经的敌人除掉了,曾经的学生除掉了,曾经的盟友也除掉了,为了实现我的梦想,我坚信,这是值得的。

当然了,作为大明帝国的实际统治者,做了这么多工作,也受了这么多的苦,再过苦日子似乎也有点说不过去,而
在这一点上,张居正同志是个明白人。

  于是张先生的许多幸福生活方式,也随之流传千古,而其中最有名的,大概就是他的那顶轿子。

  在一般人的概念中,轿子无非是四个人抬着一个人,摇摇晃晃地往前走,轿子里的人跟坐牢似的,转个身也难。

  应该说这些都没错,但如果你看到了张居正先生的轿子,你就会感叹这个世界的神奇。

\section[\thesection]{}

张先生的交通工具不叫轿子,它有个专门名称----如意斋。一般人坐一般轿子,张大人不是一般人,轿子自然也不
一般,别人的轿子四个人抬,张大人的轿子嘛……

下面我们先详细介绍一下此轿的运行原理以及乘坐体验。

该轿子(?)由真定地方知府赶制,轿内空间广阔,据估算,面积大致不低于五十平方,共分为会客室和卧室两部分,
会客室用来会见各地来客,卧室则用于日常休息,为防止张大人出行途中内急找不到厕所,该轿特设有卫生间,体
现了人性化的设计理念。

此外,由于考虑到旅途辛苦,轿子的两旁还设有观景走廊,以保证张大人在工作之余可以凭栏远眺,如果有了兴
趣,还能做两首诗。

而且张大人公务繁忙,很多杂务自己不方便处理,所以在轿中还有两个仆人,负责张大人的饮食起居。

此外,全轿乘坐舒适,操作便利,并实现了全语音控制,让停就停,让走就走,决不含糊,也不会出现水箱缺水、
油箱缺油、更换轮胎、机械故障之类的烦人事情。

你说这么大的轿子,得多少人抬?

我看至少也要十几个人吧。

十几个人?那是垫脚的!三十二个人起,还不打折,少一个人你都抬不起来,张大人的原则是,不计成本,只要风头!

相信我,你没有看错,我也没有写错,关于这部分,我确定一定以及肯定。

顺便补充一句,这顶轿子除了在京城里面转转之外,还经常跑长途,张居正曾经坐着这东西回过荆州老家,其距离
大致是今天京广线从北京出发,到武汉的路程,全部共计一千多公里,想想当年那时候,坐着这么个大玩意招摇过
市,实在是拉风到了极点。

这段史料着实让我大开眼界,并彻底改变了我对祖国交通工具的看法,什么奔驰、宝马、劳斯莱斯,什么加长型、
豪华型,什么沙发、吧台,省省吧,也好意思拿出来说,丢人!

日子过得舒坦,工作也无比顺利,张居正的好日子似乎看不到尽头,然而事实告诉我们,只进不退的人生是没有
的,正如同只升不跌的股票绝不存在一样。

万历五年(1577),张居正一生中最为严峻的考验到来了,因为一件看似毫不相干的事。就在这一年,张居正得到了
一个不幸的消息--他爹死了。

\section[\thesection]{}

张文明一辈子没啥出息,却有了这么个有出息的孩子,虽说他没给儿子帮啥忙,反倒添了很多乱(此人在地方飞扬跋
扈,名声很差),但无论如何,生子如此,他也可以含笑九泉了。

但他死也想不到,自己的死,将会让儿子张居正生不如死。

张居正的爹死了!消息传来,满城轰动,因为表现忠心的机会到了。无数官员纷纷上门,哭的哭,拜的拜,然后一把
鼻涕一把泪的摸出门,最后再说两句``节哀顺变'',完事,收工。

这并不奇怪,自古以来,当官的如果死了爹妈,自然是空巷来拜,宾客盈门,上门的比自己全家死绝还难受,但你
要相信,如果你自己挂了,是没有几个人会上门的。

对此,张居正也十分清楚,虽说父亲死了他很难过,但此时此刻,他的脑海里思考的,却是另一个问题。

这个问题的名字,叫做丁忧。

在当时的中国,张居正已经是近似于无敌了,他不怕皇帝,不怕大臣,不怕读书人议论,骠悍无比。

但他仍然只是近似于,因为他还有一个不能跨越的障碍----祖制。

所谓祖制,就是祖宗的制度,规矩,虽然你很牛,比皇帝还牛,但总牛不过死皇帝吧,上百年前定下的规则,你再
牛也没辙。

丁忧就是祖制,具体说来,是朝廷官员的父母亲如若死去,无论此人任何官何职,从得知丧事的那一天起,必须回
到祖籍守制二十七个月,这叫丁忧。到期之后可以回朝为官,这叫起复。

这个制度看上去有点不近人情,官做得好好的,一下子就给扒得干干净净,负责的那摊事情也没人管,不但误事,
还误人心情。

但这个制度一直以来却都是雷打不动,无论有多麻烦,历任皇帝都对其推崇备至,极其支持,如果你认为这是他们
的脑筋一根筋,食古不化,那就错了,人家的算盘,那是精到了极点。

因为根据社会学常识,只有出孝子的地方,才会出忠臣,你想想,如果一个人连他爹都不忠,怎么能指望他忠于老
板(皇帝)呢?

但贪官们自然是不干的,死了爹,我本来就很悲痛了,正想化悲痛为贪欲,搞点钱来安慰我无助的心灵,你竟然还
要罢我的官,剥夺我的经济利益,太不人道!

\section[\thesection]{}

于是很多人开始钻空子,你不是规定由得知死讯的那天开始计算吗,那我就隐瞒死讯,就当人还活着,一直混到差
不多为止,就算最后被人揭穿,也是可以解释的嘛,人死了,我没有上报,那是因为老爹一直活在我的心中。

当然,一次两次是可以理解的,时间长了朝廷也不干了,自明英宗起,就开始正式立项,打击伪报瞒报的行为,规
定但凡老爹死了不上报的,全部免官为民。

如此一来,贪官们也没办法了,只好日夜祈祷,自己的老爹能多撑几年,至少等混到够本再含笑而逝,到时也能多
搞点纸钱给您送去。

但也有一个群体例外,那就是军队,领兵打仗,这就绝对没辙了,总不能上阵刚刚交锋,消息来了,您喊一声停:
大家别打了,等我回去给我爹守二十七个月,咱们再来,还是老地方见,不打不散。

张居正不是军人,自然无法享受这个优待,而他的改革刚刚才渐入佳境,要是自己走了,这一大摊子事情就没人管
了,心血付之东流且不说,没准回来的时候就得给人打下手了。

于是他只剩下了唯一的选择----夺情。

所谓夺情,是指事情实在太急,绝对走不开的人,经由皇帝的指示,在万般悲痛中恢复职务,开展工作。由于考虑
到在痛苦之中把人强行(一般不会反抗)拉回来,似乎很不人道,所以将其命名为``夺情''。

然而张居正并不愿意走这条路,当然,并不是因为它``很不人道。''

其实在他之前,已有一些人有过类似的经验,比如著名的``三杨''中的杨荣,还有那位帮于谦报了仇的李贤,都曾
经被这么``很不人道''过,除了个把人骂了两句外,倒也没啥问题,但到了嘉靖年间,夺情却真的成为了一件很不
人道的事情,不人道到想不人道都不行,如果有人提出夺情,就会被看作禽兽不如。

之所以会有如此大的变化,都要拜一位孝子所赐,这人的名字叫做杨廷和。

说起来,这位杨兄弟的能量实在是大,闹腾了三朝还不够,死了还要折腾别人。当初他在正德年间的时候,父亲死
了,皇帝说杨先生你别走,留下来帮我办事,他说不行,我非常悲痛,一定要回去。

\section[\thesection]{}

结果几番来回,他还是回去了,从正德九年(1514)到正德十二年(1517) ,这位仁兄结结实实地旷了三年工,才回来
上班。这要搁在现在,早就让他卷铺盖回家了。

由于他名声太大,加上又是正面典型,从此以后,朝廷高级官员死了爹妈,打死也不敢说夺情。就这么一路下来,
终于坑了张居正。

张居正没有选择,只能夺情,因为冯保不想他走,皇帝不想他走,皇帝他妈也不想他走,当然了,最重要的是,他
也不想走。

辛辛苦苦奋斗三十多年,才混到这个份上,鬼才想走。

虽说夺情比较麻烦,但只要略施小计,还是没问题的。

于是老把戏很快上场了,万历五年(1577)十月,痛苦不堪的张居正要求回家守制,两天后皇帝回复--不行。

一天后,张居正再次上书,表示一定要回去,而皇帝也再次回复--一定不行。

与此同时,许多大臣们也纷纷上书,表示张居正绝不能走,言辞激烈,好像张居正一走,地球就要完蛋,可谓用心
良苦。

行了,把戏演到这里,也差不多该打住了,再搞下去就是浪费纸张。

准备收场了,事情已经结束,一切风平浪静,擦干眼泪(如果有),再次出发!

我亲眼看着严嵩沦落,徐阶下台,我亲手解决了高拱、刘台、何心隐,天下已无人能动摇我的地位。

对于这一点,张居正始终很自信,然而事实证明,他错了,错得相当厉害,真正的挑战将从这里开始。

万历五年(1577)十月,翰林院编修吴中行,翰林院检讨赵用贤上书----弹劾张居正夺情。

编修是正七品,检讨是从七品,也就是说,这是两个基层干部,也就能干干抄写工作,平时连上朝的资格都没有,
而张居正以前的敌人,不是朝廷高官,就是黑道老大、学界首领,并且还特别不经打,一碰就垮,这么两个小角
色,按说张大人动根手指,就能把他们碾死。

然而就是这么两个小角色,差点把张大人给灭了。

因为这二位仁兄虽然官小,却有个特殊的身份:他们都是张居正的门生。

而且我查了一下,才惊奇地发现,原来吴兄弟和赵兄弟都是隆庆五年(1571)的进士,和之前开第一炮的刘台是同班
同学。

这就只能怪张大人自己了,左挑右挑,就挑了这么几个白眼狼,也算是自己跟自己过不去。

\section[\thesection]{}

这下好了,当年只有一个二愣子(刘台),已经搞得狼狈不堪,这回竟然出了两个,那就收拾不了了,因为一个二愣
子加另一个二愣子,并不等于二,而是二愣子的平方。

可还没等张居正反应过来,又出事了,就在二愣子们出击的第二天,刑部员外郎艾穆,主事沈思孝也上书弹劾张居
正,希望他早早滚蛋回家,去尽孝道。

当张居正看到这两封充满杀气的奏疏时,才终于意识到,真正的危机正向自己步步逼近。

经过长达三十余年的战斗,他用尽各种手段,除掉了几乎所有的敌人,坐上了最高的宝座,然而在此君临天下之
时,他才发现一个新的,更为强大的敌人已经出现。

那些原先乖乖听话的大臣们似乎一夜间突然改变了立场,成为了他的对手,不是一个,是一群,而他们攻击的理由
也多种多样,经济问题,作风问题,夺情问题,方式更是数不胜数,上书弹劾,私下议论,甚至还有人上街张贴反
动标语,直接攻击张居正。

对于眼前的这一切,张居正感到很吃惊,却并不意外,因为他很清楚,带来这些敌人的,正是他自己,具体说来,
是他五年前的那封奏疏。

五年前,当张居正将写有考成法的奏疏送给皇帝时,他在交出自己改革理想的同时,还附带了一个阴谋。

因为在那封奏疏中,有着这样几句话:

``抚案官有延误者,该部举之,各部院有容隐者,科臣举之,六科有容隐欺蔽者,臣等举之。''

这句话的意思是,地方官办事不利索的,中央各部来管,中央各部办事不利索的,由六科监察机关来管,六科监察
机关不利索,由我来管!

事情坏就坏在这句话上。

根据明代的体制,中央各部管理地方,正常,给事中以及御史监察各部,也正常,内阁大学士管理言官,这就不正
常了。

两百年前,朱元璋在创立国家机构的时候,考虑丞相权力太大,撤销了丞相,将权力交给六部,但这位仁兄连睡觉
都要睁只眼,后来一琢磨,觉得六部权力也大,为怕人搞鬼,又在六部设立了六科,这就是后来的六科给事中。

\section[\thesection]{}

六科的领导,叫做都给事中,俗称科长,下属人员也不多,除了兵部给事中有十二个人之外,其余的五个部都在十
人之内。而且这帮人品级也低,科长才七品,下面的人就不用说了。

但他们的权力却大到让人匪夷所思的地步,比如说部长下令要干什么事,科长不同意,二话不说,把命令退回给部
长,让他修改,如果改得不满意,就再退,直到满意为止。

别说部长,连皇帝的某些旨意,给事中也是可以指手划脚一番的,所以虽然这帮人品级低,地位却不低,每次部长
去见他们,还要给他们行个礼,吃饭的时候别人坐下座,他们可以跑去和部长平起平坐,且指名道姓,十分嚣张。

给事中大抵如此,都察院的御史就更不得了,这伙人一天到晚找茬,从谋反叛乱到占道经营、随地大小便,只要是
个事,就能管。

六部级别高,权力小,言官级别小,权力大,谁也压不倒谁,在这种天才的创意下,大明王朝搞了二百多年,一向
太平无事,而到了张居正,情况被改变了。

在张居正看来,六部也好,给事中也好,御史也好,都该归我管,我说什么,你们就干什么,不要瞎吵。

因为他很明白,互相限制、互相制约固然是一种民主的方式,但是民主是需要成本的。

一件事情交代下去,你讲一句他讲一句,争得天翻地覆,说得振振有词,其实一点业务都不懂,结果十天半个月,
什么都没办,而对于这些人,张居正一贯是深恶痛绝。

所以他认为其他人都应该靠边站,找一个最聪明的人(他自己)指挥,大家跟着办事就行,没有必要浪费口水。于是
在他统治期间,连平时监督他人的六科和御史,都要考核工作成绩。

然而遗憾的是,大臣们却不这么想,在他们看来,张居正是一个破坏规则的人,是一个前所未见的独裁者。自朱元
璋和朱棣死后,他们已经过了一百多年的民主生活,习惯了没事骂骂皇帝,喷喷口水,然而现在的这个人比以往的
任何皇帝都更为可怕,如果长此以往,后果实在不堪设想。

所以无论他要干什么,怎么干,是好事还是坏事,为了我们手中的权力,必须彻底解决他!

一个精心策划的阴谋就此浮出水面。

\section[\thesection]{}

耐人寻味的是,在攻击张居正的四人中,竟有两人是他的学生,而更让人难以理解的是,这四个人竟没有一个是言
官!

该说话的言官都不说话,却冒出来几个翰林院的抄写员和六部的小官,原因很简单----躲避嫌疑,而且第一天学生
开骂,第二天刑部的人就跟着来,说他们是心有灵犀,真是杀了我也不信。

所以还是那句老话,夺情问题也好,作风问题也罢,那都是假的,只有权力问题,才是真的。

张居正不能理解这些人的思维,无论如何,我不过是想做点事情而已,为什么就跟我过不去呢?

但在短暂的郁闷之后,张居正恢复了平静,他意识到,一股庞大的反对势力正暗中涌动,如不及时镇压,多年的改
革成果将毁之一旦,而要对付他们,摆事实、讲道理都是毫无用处的,因为这帮人本就不是什么实干家,他们的唯
一专长就是摆出一幅道貌岸然的面孔,满口仁义道德,唾沫横飞攻击别人,以达到自己的目的。

对这帮既要当婊子,又要立牌坊的人,就一个字----打!

张居正汇报此事后,皇帝随即下达命令,对敢于上书的四人执行廷杖,也就是打屁股。

张大人的本意,大抵也就是教训一下这帮人,但后果却大大出乎他的意料。

打屁股的命令下来后,原先不吭声的也坐不住了,纷纷跳了出来,搞签名请愿,集体上书,反正法不责众,不骂白
不骂,不请白不请。

但在一群凑热闹的人中,倒也还有两个比较认真的人,这两个人分别叫做王锡爵和申时行。

这二位仁兄就是后来的朝廷首辅,这里就不多说了,但在当时,王锡爵是翰林院掌院学士,申时行是人事部副部
长,只能算是小字辈。

辈分虽小,办事却是大手笔,人家都是签个名骂两句完事,他们却激情澎湃,竟然亲自跑到了张居正的府上,要当
面求情。

张大人哪里是说见就见的,碰巧得了重病,两位大人等了很久也不见人,只能从哪里来回哪里去。

申时行回去了,王锡爵却多了个心眼,趁人不备,竟然溜了进去,见到了张居正。

\section[\thesection]{}

眼看人都闯进来了,张居正无可奈何,只好带病工作。

王锡爵不说废话,开门见山:希望张居正大人海涵,不要打那四个人。

张居正唉声叹气:

``那是皇上生气要打的,你求我也没用啊!''

这话倒也不假,皇帝确实很生气,命令也确实是他下的。

这种话骗骗两三岁的小孩,相信还管用,但王锡爵先生……已经四十四了。

``皇上即使生气,那也是因为您!''这就是王锡爵的觉悟。

话说到这个份上,张居正无话可说了,现场顿时陷入了沉寂。

见此场景,王锡爵感到可能有戏,正想趁机再放一把火,然而接下来发生的事情却是他做梦也想不到的。

沉默不语的张居正突然站了起来,抽出了旁边的一把刀,王锡爵顿时魂飞魄散,估计对方是恼羞成怒,准备拿自己
开个刀,正当他不知所措之际,更不可思议的事情发生了:

九五至尊,高傲无比,比皇帝还牛的张大人扑通一声----给他跪下了。

没等王学士喘过气来,张学士就把刀架在了自己的脖子上,一边架一边喊:

``皇帝要留我,你们要赶我走,到底想要我怎么样啊!''

面对无数居心叵测的人,面对如此困难的局面,张居正一直在苦苦支撑着,他或许善于权谋,或许挖过坑,害过
人,但在这个污浊的地方,要想生存下去,要想实现救国济民的梦想,这是唯一的选择。

现在他的忍耐终于到达了顶点。

张居正跪在王锡爵的面前,发出了声嘶力竭的呐喊:

``你杀了我吧!你杀了我吧!''

王锡爵懵了,他没有想到,那个平日高不可攀的张大学士,竟然还有如此无奈的一面,情急之下手足无措,只好匆
匆行了个礼,退了出去。

张居正发泄了,王锡爵震惊了,但闹来闹去,大家好像把要被打屁股的那四位仁兄给忘了,于是该打的还得打,一
个都不能少。

万历五年(1577)十月二十三日,廷杖正式执行,吴中行、赵用贤廷杖六十,艾穆、沈思孝廷杖八十,这么看来,师
生关系还是很重要的,要知道,到关键时刻能顶二十大板!

事情前后经过大致如此,打屁股的过程似乎也无足轻重,但很多人都忽略了一个十分有趣的地方----打屁股的结果。

两个人在同一个地方,挨了同样的打,却有着截然不同的结局.

\section[\thesection]{}

在这次廷杖中,张居正的两位学生在抗击打能力上,表现出了完全相反的特质,吴中行被打之后,差点当场气绝,
经过奋力抢救,才得以生还,休养了大半年,还杵了一辈子拐杖。

但赵用贤就不同了,据说他被打之后虽然伤痕遍布,元气大伤,却明显能扛得多,回家后躺了一个多月,就能起床
跑步了。

这是一个奇迹,同样被打的两个人,差别怎么会这么大呢?要说明这个问题,我们必须以科学的态度,严谨的精神,
去详细分析一下这个明代特有的发明--打屁股。

关于打屁股问题的技术分析报告

廷杖,也就是打屁股,是明代的著名特产,大庭广众之下,扒光裤子,露出白花花的屁股,几棍下去,皮开肉绽,
这就是许多人对打屁股的印象。

然而我可以负责任的告诉各位,打屁股,并非如此简单,事实上,那是个技术工种。

根据人体工程学原理分析,明代的廷杖是一种极为严酷的刑罚,因为那跟你在家挨打不一样,你爹打你,无非是用
扫把,小棍子,惨无人道点的,最多也就是皮带。

但廷杖就不同了,它虽然也用棍子,却是大棍子,想想碗口粗的大棍以每秒N米的加速度向你的屁股着陆,实在让人
胆寒,所以连圣人也说过,遇到小棍子你就挨,遇到大棍子,你就要跑(小杖则受,大杖则走)。

而执行廷杖的人,基本上都是锦衣卫,这伙人平时经常锻炼身体,开展体育活动,随手一抡,不说开碑碎石,开个
屁股还是不难的。

所以经过综合分析,我们得出如下结论,如无意外,二十廷杖绝对足以将人打死。

但一直以来,意外始终在发生着,一百杖打不死的有,一杖就完蛋的也不缺,说到底,还要归功于我国人民的伟大
智慧。

纵观世界,单就智商而言,能和中国人比肩的群体,相信还没生出来,而我国高智商人群最为突出的表现,就在于
从没路的地方走出路来。

打不打屁股,那是上级的事,但怎么打,那就是我的事了,为了灵活掌握廷杖的精髓,确保一打就死,或者百打不
死,锦衣卫们进行了艰苦的训练,具体方法如下:(有兴趣者,可学习一二,但由此带来之后果本人概不负责)

\section[\thesection]{}

找到一块砖头(种类不限),在上面垫一张宣纸(一点就破那种),用棍子猛击宣纸,如宣纸破裂,则重新开始,如此
这般不断练习,以宣纸不破,而砖头尽碎为最高层次。

如果能打到这个级别,基本就可以出师了,给你送过钱的,就打宣纸,打得皮开肉绽,实际上都是软组织损伤,回
家涂了药,起来就能游泳。

要是既无关照,又有私仇的,那就打砖头,一棍下去表皮完整,内部大出血,就此丧了命那是绝不奇怪。

顺便说一句,在当时,另一个技术工种也有类似的练习,那就是砍头的郐子手,这也是门绝活,操作方法与打屁股
恰好相反,找一块平整的肉,然后在上面放上一块宣纸,用刀剁宣纸,把下面的肉剁碎,上面的宣纸不能破损,就
算是炉火纯青了。

练这一手,那也是深谋远虑,如果给钱的,一刀下去就结果,不会有痛苦,不给钱的,随手一刀,爱死不死,多久
才死,反正是你的事。

如果有给大钱的,那就有说头了,只要不是什么谋反大罪,不用验明首级,再买通验尸官,犯不着人头落地,就能
玩花样了:顺手一刀砍在脖子上,看上去血肉模糊,其实上大血管丝毫无损,抬回去治两天,除了可能留个歪脖子
后遗症外,基本上没啥缺陷。

这才是真正的技术含量,什么``庖丁解牛'',和砍头打屁股的比起来,实在是小儿科。拉到刑场上都杀不死,打得
皮开肉绽都没事,这就是技术。

技术决定效益,这是个真理

所以长久以来,打屁股的锦衣卫日夜操练技术,毕竟人家就靠这手本事混饭吃,不勤奋不行,但日久天长,朝廷也
不是傻瓜,慢慢地看出了门道,为保证廷杖的质量,也研发了相应的潜规则口令,分别是:打、着实打、用心打。

所谓打,就是意思意思,谁也别当真,糊弄两下就没事了。

而着实打,就是真打了,该怎么来怎么来,能不能挺得住,那得看个人体质。

最厉害的,是用心打,只要是这个口令,基本上都是往死里打,绝对不能手软。

这三道口令原本是潜规则,后来打得多了,就成了公开命令,不但要写明,而且打之前由监刑官当众宣布,以增加
被打者的心理压力。而赵用贤和吴中行的廷杖命令上,就明白地写着着实打。

\section[\thesection]{}

既然是着实打,那就没什么说的了,虽然有人给锦衣卫送了钱,也说了情,但毕竟命令很明确,如果过轻,没准下
次被打的就是自己,和钱比起来,还是自己的屁股更重要。

但问题依然没有解决,既然同样是着实打,同样是读书人,体质相同,为什么吴中行丢了半条命,赵用贤却如此从
容?

原因很简单,赵用贤是个胖子,而吴中行很瘦,用拳击术语讲,这二位不是一个公斤级的,抗击打能力不同,赵用
贤有脂肪保护,内伤较小,而吴中行没有这个防护层,自然只能用骨头来扛。

这一结果也生动地告诉了我们,虽说胖子在找老婆、体育活动方面不太好使,但某些时候,有一身好肥肉,还是派
得上用场的。

挨打之后还没完,吴中行和赵用贤因为官职已免,被人连夜用门板抬回老家(没资格坐轿子),这场学生骂老师的闹
剧就此划上句号。

当然,不管他们出于何种动机,是否有人主使,但这两位仁兄由始至终没有说过一句软话,坚持到底,单凭这一
点,就足以让人敬佩。

但在整个事件中,最让人胆寒的,却不是张居正,也不是这两位硬汉,而是一个女人。

在赵用贤与吴中行被打的时候,许多同情他们的官员在一旁议论纷纷,打完之后,王锡爵更是不顾一切地冲了上
去,抱住吴中行痛哭不已,但没有几个人注意到,与他同时冲上去的,还有一个女人----赵用贤的老婆。

但这位大嫂的举动却出人意料,她初步照料了自己的丈夫后,便开始在现场收集一样东西----赵用贤的肉。

由于打得太狠,赵用贤虽然是个胖子,腿上也还是被打掉了不少肉,赵夫人找到了最大的一块,带回了家,用特制
方法风干之后,做成腊肉,从此挂在了家里。

这位悍妇之所以干出如此耸人听闻之举,是因为在她看来,被打是一件无比光荣的事情,她要留下纪念品,以表示
对张居正的永不妥协,并利用这块特殊的肉,对后代子孙进行光荣传统教育----你爹虽然挨了打,但是打得光荣,
打得伟大!

打完了四个人的屁股,却打不完是非,此后攻击张居正的人有增无减,什么不回家奔丧,就禽兽不如之类的话也说
了出来,骂来骂去,终于把皇帝骂火了。

\section[\thesection]{}

虽然才十五岁,但皇帝大人已经是个明白人了,他看得很清楚,那些破口大骂的家伙除了拿大帽子压人外,什么也
没干过,而一直勤勤恳恳干活的张居正,却被群起而攻之,天理何在!?

敢跟我的张先生(皇帝的日常称呼)为难,废了你们!

万历皇帝随即颁布了自他继位以来,最为严厉的一道命令:

胆敢再攻击张居正夺情者,格杀勿论!

事实证明,在一拥而上的那群人中,好汉是少数,孬种是大多数,本来骂人就是为了个人利益,既然再骂要赔本(杀
头),那就消停了吧。

张居正又一次获得了胜利,反对者纷纷偃旗息鼓,这个世界清静了。

但他的心里很清楚,这不过是表象而已,为了改革,为了挽救岌岌可危的国家,他做了很多事,得罪了很多人,一
旦他略有不慎,就可能被人打倒在地,永不翻身,而那时他的下场将比之前的所有人更悲惨。

徐阶厌倦了可以退休,高拱下台了可以回家,但他没有选择,如果他失败了,既不能退休,也不能回家,唯一的结
局是身败名裂,甚至死无葬身之地。因为徐阶的敌人只是高拱,高拱的敌人只是他,而他的敌人,是所有的人,所
有因改革而利益受损的人。

是啊,张居正先生,你为什么要这么闹腾呢?你已经爬上了最高的宝座,你已经压倒了所有的人,你可以占据土地,
集聚财富,培养党羽,扶植手下,只要你不找大家的麻烦,没有人会反抗你,也没有人能反抗你。

但你偏偏要搞一条鞭法,我们不能再随意鱼肉百姓,你偏偏要丈量土地,我们不能随意逃避赋税,你偏偏要搞什么
考成法,我们不能再随意偷懒。

大家都是官员,都是既得利益者,百姓的死活与我们无关,你为什么要帮助他们,折腾我们呢?

因为你们不明白,我和你们不同。

我知道,贫苦的百姓也是人,也有父母妻儿,也想活下去。

我知道,我有极为坚强的意志,我的斗志不会衰竭,我的心志不会动摇,即使与全天下人为敌,我也决不妥协。

我知道,在几十年之后,你们已经丢弃了当年的激情壮志,除了官位和名利,你们已别无所求,但我不同。

因为在历经无数腥风血雨、宦海沉浮之后,我依然保存着我的理想。

我相信,在这个世界上,还有公理和正义。

我相信,在这个世界上,所有的人,无论贵贱,都有生存的权力。

这就是我的理想,几十年来,一天也不曾放弃。

这就是张居正,一个真正的张居正。

\section[\thesection]{}

在对他的描述中,我毫不避讳那些看上去似乎不太光彩的记载,他善于权谋,他对待政敌冷酷无情,他有经济问
题,有生活作风问题,这一切的一切,可能都是真的。

而我之所以如实记述这一切,只是想告诉你一个简单而重要的事实:张居正,是一个人,一个真实的人。

在这个世界上,最猛的人,应该是超人同志,据说他来自外星球,绕地球一圈只要几秒,捏石头就像玩泥巴,还会
飞,出门从不打车,也不坐地铁,总在电话亭里换衣服,老穿同一件制服,还特别喜欢把内裤穿在外面,平时最大
的业余爱好是拯救地球,每年至少都要救那么几次,地球人都知道。

然而没有人认为他很伟大,因为他是超人。

超人除了怕几块破石头外,没有任何弱点和缺点,是无所不能的,他压根就不是人。

张居正不是超人,他出生于一个普通的家庭,从小熟读四书五经,挑灯苦读,是为了混碗饭吃,进入官场,参与权
力斗争,拉帮结伙,是为了保住官位,无论从哪个角度看,他都是一个不折不扣的俗人。

然而正是这个真实的人,这个俗人,在权势、地位、财富尽皆到手的情况下,却将枪口对准了他当年的同伴,对准
了曾带给他巨大利益的阶层,他破坏了规则,损害了他们的利益,只是为了一个虚无缥缈的概念----国家,以及那
些和他毫不相干的平民百姓。

所以我没有详写张居正一生中那些为人津津乐道的情节,比如整顿官场,比如惩办贪官,比如他每天都工作到很
晚,再比如他也曾严辞拒收过贿赂,制止过亲属的腐化行为,在我看来,这些情节并不重要。

只有当你知道,他是一个正常人,有正常的欲望,有自己的小算盘,有过犹豫和挣扎,有过贪婪和污点,你才能明
白,那个不顾一切,顶住压力坚持改革的张居正,到底有多么的伟大。

所有的英雄,都是平凡的人。

千回百转,千锤百炼,矢志不改,如此而已。

爱与恨的边缘

万历五年(1577)的夺情事件结束了,张居正获得了彻底的胜利,事实证明,以眼前这些小喽罗的实力,是动不了张
大哥分毫的,自打严嵩、徐阶、高拱这批高水平选手退役后,江湖人才是一代不如一代了。

\section[\thesection]{}

张居正对此有着十分清醒的认识,所以他越发有恃无恐,推行自己的政令,谁不听话就灭了谁,自从赶走高拱后,
内阁中只剩他一人,为体现民主风格,他又陆续提拔几人入阁,先是吕调阳,然后是张四维,马自强,申时行,当
然了,这几位仁兄虽然籍贯不同,爱好不同,高矮胖瘦长相各异,但对于张居正而言,他们是同一类人----跑腿
的,有着共同的优点--听话。

但后来的事实发展证明,对于这四个人,他还是看走了眼,至少看错了一个。

除了工作上独断专行外,张居正还常常对人说这样一句话:我非相。

这句话看上去十分谦虚,表明我张居正不是宰相。但很不幸的是,这句谦虚的话还有下半句:乃摄也。

综合起来,这就是一句惊天地泣鬼神的话:

我不是宰相,而是摄政。

所谓摄政,就是代替皇帝行使职权的人,对张居正而言,宰相已经是小儿科了,只有摄政才够风光。一个平民竟然
如此风光,如果当年废除宰相的朱元璋泉下有知,恐怕会气得活过来。

但张居正明显是不怕诈尸的,他受之无愧,并在家里挂上了这样一副对联:

日月共明,万国仰大明天子,

丘山为岳,四方颂太岳相公。

这幅对联用黄金打造,十分气派,但要换在以前,这是个要人命的东西。因为所谓太岳,就是张居正的字,而众所
周知,对联的下半句要高于上半句,如此一来,张居正就比皇帝更牛了。

而牛人张居正非但没有拒收,还堂而皇之地裱起来,就差贴在门口当春联用了。

但一个人天下无敌太久,老天爷也会不满的,毕竟他老人家喜欢热闹,于是在冥冥之中,他给张居正找来了两个敌
人,一个是他的上级,一个是他的下属。

张居正的上级,就是皇帝。

说起这二位的关系,实在是错综复杂,你中有我,我中有你,但综合说来,这是一个由爱生恨的故事。

万历皇帝朱翊钧,嘉靖四十二年(1563)出生,是隆庆皇帝的第三个儿子,这位仁兄运气很好,六岁就立了太子,四
年后又死了爹,直接当了皇帝,比起他那位连个太子名分都没有,提心吊胆当了三十多年王爷的爹来,强得不是一
星半点。

而如果仔细分析他的履历,你就会发现,这位被誉为明代第一懒人的皇帝,实际上曾是一个无比聪明勤奋的人。

\section[\thesection]{}

万历是个很聪明的孩子,从小认字很早,而且很懂事,虽然不用他帮家里做饭,打洗脚水,但他也知道父亲死得
早,母亲一个人不容易,要想维持住这个家,就得靠张先生。

这是他的母亲告诉他的,在近十年的时间里,他对此深信不疑。

他和张先生的第一次亲密接触,是在父亲刚死的时候,他还清楚地记得,那是一个十分危急的时刻,万恶的高老头
(高拱同志)欺负他年纪小,他妈又是个寡妇,准备把他的皇位夺走,让他下岗走人,关键时刻,张先生出现了,这
位盖世英雄拯救了他们母子,并赶走了邪恶的高老头,在伟大的张先生的帮助下,好人战胜了坏人,世界再次恢复
了和平。

这大概就是万历对张居正的第一印象,而此后母亲的种种言行也加深了他对张先生的好感。

由于父亲死得早,他的小学教育基本上是由张居正完成的,这位首辅大人可谓多才多艺,除了处理政务外,对他的
学习也丝毫不放松,闲来无事还编了一本书,叫做《帝鉴图书》。

毫不夸张地说,如果今天搞一个优秀少儿图书评选,这本书绝对可以名列前茅,在此书中,张居正特意挑选了一百
一十七个历史事件,其中好事八十一件,坏事三十六件,每件事情都配有插图,类似于小人书,讲明白为什么好,
为什么坏,相信只要不是白痴,就一定能看得懂。

为了贯彻以人为本的教育理念,张居正确实下了很大功夫,他不但编了书,还每天跑来给小皇帝讲故事,指着书上
的插图,告诉万历,哪个是好人,哪个是坏人。

万历的童年就是这样度过的,对这个既帮自己干活,又给自己讲故事的张先生,他有着十分深厚的感情。甚至于每
次张居正上朝时站在他的面前,他都觉得过意不去:张先生站着,我怎么好意思坐着?

问题在于皇帝没法站着上朝,于是他给了张居正一个特殊待遇,每到夏天热时,张居正的身边就站着两人,专门给
他扇扇子;冬天冷时,张居正的脚底下总有一块铺好的毡布(当然,别人是没有的),当旁边的诸位同僚擦汗打哆嗦
时,张先生这里却是气定神闲,搞得大家总仰天长叹:人和人就是不一样啊。

在万历看来,张居正是一个类似父亲的人。

\section[\thesection]{}

而那位在一旁煽风点火,引导万历的李贵妃(现在是太后了),对张居正却有着完全不同的动机。

李太后是一个不寻常的女人,她籍贯山西,出身低微,家里原来虽做过小生意,也无非是混碗饭吃。幸好长得漂
亮,被皇帝选中,还生了个儿子,估计她从小经常逛集贸市场,讨价还价,社会经验丰富,所以在宫中很会来事,
人缘也好,这才开始发达起来。但后来的事情发展证明,她的本性始终未曾变过----生意人。

从看到张居正的第一眼起,李太后就意识到,这是一个极有利用价值的人,不但能谋善断,而且政务能力极强,加
上他的丈夫隆庆皇帝为人老实、胆小怕事不说,还是个老病号,哪天脚一蹬就咽了气,那都是说不准的事情。

虽说李太后精明强干,也有一定的政治野心,但她很清楚,中国很广阔,事情很复杂,像收税、打仗、城管、救灾
之类的事,自己是搞不定的,只能依靠大臣去办。换句话说,她知道自己有几斤几两,从这一点看,她比后来的那
位慈大妈(慈禧)不知要强多少倍。

关于后宫参政问题的调研

这是个十分有趣的问题,纵观整个明代,什么事情都有,太监专权,大臣独裁,可偏偏老婆(后宫)参政的问题并不
多见,什么女主当国,垂帘听政,压根就没有市场,看上去很让人费解,但只要略为分析,就会发现,其实原因十
分简单。

先介绍一下相关知识,要知道,在中国历史上,女性参政折腾事的并不少见,但折腾出好结果的却并不多见,像慈
禧这类的二杆子更是数不胜数,讲到这里,也请诸位女性同胞暂不要动手,容俺说完。

女性在从政方面之所以比男性困难,说到底是个生理结构问题,政治问题是世界上最复杂的问题,需要极大的理
性,但女性情感丰富,很多事情上往往会跟着感觉走,比如慈禧大妈,开始知道光绪改革,还比较支持,但一听说
改革要革自己,就把人给废了,这还在其次,关键在于她明明知道大清国快完蛋了,不改革不行,只为了吐口恶
气,把维新派的那一套也给废了,实在太不理智。

冲动是魔鬼,这话一点不错。

\section[\thesection]{}

当年秦孝公的儿子恨透了商鞅,等老爹一死就找来几匹马把他给分了(五马分尸,学名车裂),但分尸归分尸,商鞅
的那一套他还是照着用,一点不耽误,相比而言,慈大妈的档次实在差得太远。

到后来,慈大妈因为洋人不准她废掉光绪,且一直指手划脚,一怒之下,就去利用义和团,把那一帮大师兄、二师
兄都请到京城,估计是戏看多了,什么刀枪不入的鬼话都相信,还公然向全世界列强宣战(早干嘛去了),也不派兵
出国,唯一的军事行动就是攻打各国使馆,就那么高几层楼,对方撑死也就上百人,清兵围,义和团围,十天半个
月打不进去,等到人家一派兵又慌了,赶紧撤除包围,还往使馆里送西瓜,被人赶到西边,一路上吃尽了苦受尽了
累,回来却又十分大度,表示愿意以举国之力,结列强之欢心。

说起这位慈大妈,真是一声叹息,不知从何讲起,国家被她搞得一团浆糊,乱象丛生,归根结底,还是因为慈大妈
感情太丰富,不按常理出牌,虽说工于心计,也只能玩玩权谋,整死几个亲王,过过舒坦日子,让她治国安邦,那
是没有指望的。

当然了,成功的例子也是有的,比如伟大的武则天女士,那就真是身不能至,心向往之,一步一个脚印,从宫女到
皇后,再到皇帝,但凡敢挡路的,全部干掉,连儿子也不例外,看似和慈禧没什么区别,但她在历史上的名声比慈
禧实在好得太多。

因为当慈禧看戏的时候,武则天在看公文,慈禧在吃几百道菜的时候,武则天连晚饭都顾不上,自执政以来,她始
终兢兢业业,不敢有丝毫松懈,她很清楚,作为一个政治家,除了得到,还必须付出。

所以慈禧只是个阴谋家,而武则天是政治家,阴谋家只能整人,政治家除了整人外,还要整国家。

而李太后就不同了,她既不是阴谋家,更不是政治家,从某种意义上说,她是个维持家庭的家庭主妇。

历朝历代,之所以老婆干政频繁出现,说到底还是因为皇帝权力大,用历史术语讲,这叫后权源自皇权,一旦皇帝
死了,儿子又小,老婆想不掌权都不行。可在明代,皇帝本人就没什么权,隆庆皇帝干了五六年,有一多半时间在
挨骂,想买点珠宝首饰,户部还不给钱,过得非常之窝囊,面对这种局面,想把日子过下去,也就只能依靠张居正
了。

\section[\thesection]{}

而且张居正这个人除了工作出色外,长得也帅,当然这个帅的定义和今天不同,在明代,有一把大胡子是帅哥的第
一特征(络腮胡子不算,在当时那是土匪特征),最符合标准的,是关公的那一种,随风飘扬,不但美观,沾点墨水
就能写字,也很实用。张居正五官端正不说,还有一把这样的胡子,既有能力又有相貌,李太后要不喜欢他,那就
真没天理了。

所以虽然这对母子的阅历和动机不同,但有一点他们是一致的,那就是张先生是一个很重要的人,必须依靠他----
至少目前是这样。

对这对孤儿寡母的心思,张居正十分明白,对李太后,他礼敬有加,给足面子,毕竟这人也算自己的上级,但对万
历,态度就完全不同了,张先生似乎完全不把皇帝当干部,想怎么说就怎么说,想怎么训就怎么训,比爹还爹。

最骇人听闻的一件事情,是在万历读书的时候发生的,那时万历正在读论语,张居正站在一边听,读到其中一句
``色勃如也''的时候,小朋友一时大意,认了个白字,把勃读成了``背''音。

这实在不是个大事,可万历刚刚读完,就听得身旁一声大吼:

``这字应该读勃!''

如果你今天在学校里读错字,被人这么吼一句,也会不高兴,估计个把性格型的还会回一句:老子就爱读背,你怎
么着?

但当时的万历,至高无上的皇帝大人却没有回嘴,不但没有回嘴,还吓得发抖,赶紧修正,相信这句话他一辈子再
也不会读错了。

在封建社会,无论从哪个角度看,张居正的行为都是大逆不道,拉出去剐一千遍都不过分,连孩子他亲爹都没这么
训过,张先生竟敢如此放肆,真是欺负朱重八不在了。

但张居正之所以有如此举动,绝不是为了耍威风,只是因为在他的内心深处,隐藏着一个梦想。

三十年前,当他刚刚进入朝廷时,坐在皇位上的是嘉靖,这位极难伺候的仁兄让张先生吃尽了苦头,前后躲闪,左
右逢迎,历经千辛万苦才把他熬死。

接班的隆庆却是个完全相反的人,什么事情都没主意,也不管,大事小事都得自己干。

虽说这样也不错,但张居正知道,自己总有一天是要死的,摊上这么个皇帝,出了事谁来给他擦屁股?

所以他希望培养一个合格的接班人,他希望经他之手,成就一位千古明君。

万历,你就是我的目标,我将用毕生之心血去培养你,我已不再年轻,也终将死去,但我坚信,你的名字将和汉武
帝、唐太宗并列,千古传诵,青史流芳。

如此,则九泉之下,亦当含笑。

\section[\thesection]{}

事情似乎比想象得还要顺利,在很长一段时间内,所有人都在张居正的轨道上有条不紊地行进着,朝政很稳定,皇
帝很听话,皇帝他妈很配合。

然而正是因为太正常,正常到了不正常的地步,就出问题了。

我当年上高中的时候,有一个同学,简直嗜玩如命,每天最大的梦想就是不用上学,到处去玩耍,于是经常旷课终
于惹怒了老师,让他回家去了。开始这位兄弟还很高兴,可在家住了两个月,死乞白赖地又回来了。我问:何以不
玩?答:玩完,无趣。

万历皇帝的情况大致如此,刚即位时,他才不到十岁,什么事情有张居正管着,啥也不用干,高兴都来不及,可时
间一长,就没意思了,拿起一份奏疏,想写点批示,一看,上面张居正都给批好了,一二三四,照着办就行。这还
不算,连划勾盖章的权力他都没有,要知道,那是冯保的工作。

毕竟十六七岁了,没有事干,那就找人玩,但很明显,张居正没有陪他扔沙包的兴趣,于是万历只好找身边太监玩。

太监玩什么他就玩什么,太监斗蛐蛐,他就斗蛐蛐,太监喝酒,他就喝酒,太监喝醉后喜欢睡觉,他喝醉后喜欢闹
事(酒风不好)。

于是万历八年(1580),酒风不好的万历兄终于出事了,有一天,他又喝醉了,在宫里闲逛,遇上了一个太监,突然
意气风发,对那位仁兄说:你唱个歌给我听吧。

一般说来,在这种场合,遇上这种级别的领导,就算不会唱歌,也得哼哼两句过关。可这位太监不知是真不会唱
歌,还是过于害怕,站在原地半天没有出声。

皇帝大都没什么耐心,特别是喝醉的皇帝,看着眼前的这个木桩子,万历十分恼火,当即下令把这位缺乏音乐素养
的兄弟打了一顿,打完了还割了他一束头发,那意思是本来要砍你的头,而今只割你的头发,算是法外开恩。

换在其他朝代,这事也就过了,天子一言九鼎,天下最大,不会唱歌就人头落地也不新鲜,但万历不同,他虽是皇
帝,上面还是有人管的。

在万历刚刚发酒疯的时候,冯保就得到了消息,他即刻报告了李太后,于是当皇帝大人酒醒之后,便得到了消
息----李太后要见他。

\section[\thesection]{}

等他到地方的时候,才知道事情大了,李太后压根不跟他说话,一见面就让他跪,然后开始历数他的罪恶,万历也
不辩解,眼泪一直哗哗地,不断表示一定改过自新,绝不再犯。

好了,到目前为止,事情还不算太坏,人也骂了,错也认了,就这么收场吧。

然而李太后不肯干休,她拿出了一本书,翻到了其中一篇,交给了万历。

这似乎是个微不足道的举动,但事实上,张居正先生的悲惨结局正是源自于此。

当万历翻开那本书时,顿时如五雷轰顶,因为那本书叫《汉书》,而打开的那一篇,是《霍光传》。

霍光,是汉代人物,有个异母兄弟是名人,叫霍去病。但在历史上他比这位名人还有名,干过许多大事,就不多说
了,其中最大的一件事情,就是废过皇帝。

废了谁,怎么废的,前因后果那都是汉代问题,这里不多讲,但此时,此地,此景,读霍光先生的传记,万历很明
白其中的涵义:如果不听话,就废了你!

而更深一层的含义是:虽然你是皇帝,但在你的身边,也有一个可以废掉你的霍光。

万历十分清楚,这位明代的霍光到底是谁。

生死关头,万历兄表现了极强的求生欲望,他当即磕头道歉,希望得到原谅,并表示永不再犯。

毕竟是自己的儿子,看到惩罚已见成效,李太后收回了威胁,但提出了一个条件:皇帝大人既然犯错,必须写出检
讨。

所谓皇帝的检讨,有个专用术语,叫``罪己诏'',我记得后来的崇祯也曾写过,但这玩意通常都是政治手段,对
``净化心灵''毫无作用。

想当年我上初中时,为保证不请家长,经常要写检讨,其实写这东西无所谓,反正是避重就轻,习惯成自然,但问
题在于,总有那么几个缺心眼的仁兄逼你在全班公开朗诵,自己骂自己,实在不太好受。

而皇帝的``罪己诏''最让人难受的也就在此,不但要写自己的罪过,还要把它制成公文,在天下人面前公开散发,
实在太过丢人。

万历兄毕竟还是脸皮薄,磕完头流完泪,突然又反悔了,像大姑娘上轿一样,扭扭捏捏就是不肯动笔,关键时刻,
一位好心人出现了

``我来写!''

无私志愿者,张居正。

\section[\thesection]{}

要说还是张先生的效率高,挥毫泼墨,片刻即成,写完后直接找冯保盖章,丝毫不用皇上动手。

万历坐在一旁,呆呆地看着这一切,喝醉了酒,打了个人,怎么就落到这个地步?差点被人赶下岗?

在他十八岁的大脑里,一切都在飞快运转着,作为一个帝国的统治者,为什么会沦落到如此境地?是谁导致了这一
切?是谁压制了自己?

他抬起了头,看到了眼前这个正在文案前忙碌的人,没错,这个人就是答案,是他主导了所有的一切,这个人不是
张先生,不是张老师,也不是张大臣,他是霍光,是一个可以威胁到自己的人。

在张居正和李太后看来,这是一次良好的教育机会,万历兄将从中吸取经验,今后会好好待人,在成为明君的道路
上奋勇前进。

然而就在这一团和气之下,在痛哭与求饶声中,一颗仇恨的种子已经埋下,八年的感情就此划上句号,不是因为训
斥,不是因为难堪,更不是因为罪己诏,真正的原因只有一个----权力。

我已经十八岁了,我已经是皇帝了,凭什么指手划脚,凭什么威胁我?你何许人也?贵姓?贵庚?

这就是万历八年发生的醉酒打人事件,事情很简单,后果很严重,皇帝大人的朋友和老师消失了,取而代之的,是
敌人。

但整体看来,局势还不是太悲观,毕竟还有李太后,有她在中间调和,张居正与万历的关系也差不到哪去。

可问题在于,这位中年妇女并非缓冲剂,反倒像是加速剂,在日常生活中,她充分证明了自己的小生意人本色----
把占便宜进行到底。

自从有了张居正,李太后十分安心,这个男人不但能帮她看家,还能帮她教孩子,即当管家,又当家庭教师,还只
拿一份工钱,实在太过划算。

对于小生意人而言,有便宜不占,那就真是王八蛋了,于是慢慢地,她在其他领域也用上了张居正,比如……吓唬孩
子。

\section[\thesection]{}

小时候,我不听话的时候,我爹总是对我说,再闹,人贩子就把你带走了,于是我立刻停止动作,毛骨悚然地坐在
原地,警惕地看着周围,虽然我并不很清楚,人贩子到底是啥玩意,只知道他们喜欢拐小孩,拐回去之后会拿去清
炖,或是红烧。

万历也有淘气的时候,每到这时,顶替人贩子位置的,就是张居正,李太后会以七十岁老太太的口吻,神秘诡异的
语气,对闹腾小孩说道:

``你再闹!让张先生知道了,看你怎么办?'' (使张先生闻,奈何)

这句话对万历很管用,很明显,张先生的威慑力不亚于人贩子。

自古以来,用来吓唬小孩的人(或东西)很多,从最早泛指的老妖怪,魔鬼(西方专用),到后来的具体人物,比如三
国时期合肥大战后,战场之上彪悍无比的张辽同志,就曾暂时担任过这一角色(再哭,张文远来了!),再后来,抗日
战争时期,日本鬼子也客串过一段时间,到我那时候,全国拐卖成风,人贩子又成了主角。

总而言之,时代在变,吓人的内容也在变,但有一点是不变的,但凡当这类主角的,绝不是什么让人喜欢的角色。

所以从小时起,在万历的心中,张居正这个名字代表的不是敬爱,而是畏惧,而这在很大程度上,应该归功于他的
那位生意人母亲。

对不断恶化的局势,张居正倒也不是毫无察觉,在醉酒事件之后不久,这位老奸巨滑的仁兄曾提出过辞职,说自己
干了这么多年,头发也白了,脑袋也不好用了,希望能够早日回家种红薯,报告早晨打上去后,一顿饭工夫回复就
下来了----不行。

万历确实不同意,一方面是不适应,毕竟您都干了这么多年,突然交给我,怎么应付得了;另一方面是试探,毕竟
您都干了这么多年,突然交给我,怎么解释得了。

两天后,张居正再次上书,坚决要求走人,并且表示,我不是辞职,只是请假,如果您需要我,给我个信,我再来
也成。

张居正并不是虚情假意,夏言、严嵩、高拱的例子都摆在眼前,血淋淋的,还没干,唯一能够生还的人,是他的老
师徐阶,而徐阶唯一的秘诀,叫做见好就收。

现在是收的时候了。

这话一出来,万历终于放心了,不是挖坑,是真要走人。按照他的想法,自然是打算批准了,如果事情就这么发展
下去,大团圆结局是可以期待的,然而关键时刻,闹事的又出场了

\section[\thesection]{}

生意人和政治家是有区别的,最大的区别在于,政治家是养羊,生意人是养猪。养羊的,每天放养,等到羊毛长长
了,就剪一刀接着养,无论如何,绝不搞鱼死网破,羊死毛绝的事情,而生意人养猪,只求养得肥肥的,过年时一
刀下去,就彻底了事,没有做长期生意的打算。

李太后是生意人,她没有好聚好散、细水长流的觉悟,也无需替张居正打算,既然好用,那就用到用废为止,于是
她开了尊口:

``张先生不能走,现在你还年轻,等张先生辅佐你到三十岁,再说!''(待辅尔到三十岁,那时再做商量)

这可就缺了大德了。

想走的走不了,今年都五十六了,再干十年,不做鬼也成仙了。

想干的干不上,今年才十八岁,再玩十年,还能玩出朵花儿来?

但太后的意旨是无法违背的,所以无论虚情假意,该干的还得干,该玩的还得玩,张居正最后一个机会就此失去。

既然不能走,那就干吧,该来的总要来,躲也躲不掉,怀着这种觉悟,张居正开始了他最后的工作。

从万历八年(1580)到万历十年(1582),张居正进入了一种近乎癫狂的状态,他日以继夜地工作,贯彻一条鞭法,严
查借机欺压百姓的人员,惩办办事不利的官员,对有劣迹者一律革职查办,强化边境防守,俺答死了,就去拉拢他
的老婆三娘子(当年把汉那吉没娶过去的那位),只求对方不闹。里里外外,只要是他能干的,他都干了。

大明帝国再次焕发了平静与生机,边境除了李成梁先生时不时出去砍人外,已经消停了很多,国库收入极为丰厚,
存银达到几百万两,财政支出消除了赤字,地方粮仓储备充足,至少饿不死人,一切看上去都是那么的完美。

与蒸蒸日上的帝国相反的,是张居正蒸蒸日下的身体,在繁杂的工作中,他经常晕倒,有时还会吐血,然而事已至
此,又能如何?

这就是张居正的最后两年,每一天,他都相信国家的前途,相信平民百姓的生计,相信太平盛世的奇迹,相信那伟
大的抱负终会实现。

以他的生命为代价,他坚信这所有的一切。

在他的人生的每一刻,都洒满了理想与信念的光辉。

\section[\thesection]{}

失去、得到

万历十年(1582)六月二十日,帝国内阁首辅,上柱国,正一品太师兼太傅,中极殿大学士张居正卒,年五十八,谥
文忠。

张居正死了,皇帝十分之悲痛,这是真的,毕竟一个人陪伴了自己那么久,干了许多事,没有感情是不可能的,所
以他很是哭了几场,甚至有几天悲痛得上不了朝。悲痛之余,他还下令抚慰张居正的家人,并举办了隆重的悼念活
动,一时之间,全国处处都是哀悼之声。

但以他和张居正的关系,和从前那许许多多不堪回首的往事,太有感情也是不可能的,所谓十分之悲痛,其实也就
悲痛十分钟而已。

所以在短暂悼念之后,长期清算的时候就到了,六月份张居正死,十二月份就动手了,当然,对手还不是张居正。

事实上,在当时的朝廷里,最为人忌恨的人,是冯保,张先生好歹是翰林出身,一步一步熬上来的,冯太监这样一
步登天的人,要不是后台硬,早就被唾沫星子给淹死了。

现在张居正死了,但冯保似乎还是很镇定的,因为小时候冯保经常陪小皇帝玩,万历也对他很亲热,不叫他名字,
只叫他大伴,关系相当之铁,所以他认为,纵使风雨满天,天还塌不下来。

然而天就塌下来了,十二月有人告他十二大罪,几天之后当年的那位小皇帝就在告状信上大笔一挥,下了结论:冯
保欺君蠹国,罪恶深重。

冯保措手不及,当时就晕了过去。

冯保同志敬请节哀,蠹国虽是胡说,欺君却是事实,其实一直以来,他都是排在万历最讨厌人榜的第二名,仅次于
张居正,因为这位仁兄一直以来都在干一件万历最为讨厌的事情----打小报告。

自打掌权后,冯保就以二管家自居了,但凡万历有啥风吹草动,他都会在第一时间告诉李太后,什么斗蛐蛐、打弹
弓,包括喝醉酒闯祸的那一次,都是他去报告的。

在我小时候,这种人一般被叫做``特务'',是最受鄙视的。到了万历那里,就成了奸贼,年纪小没能量,也无可奈
何,长大以后那就是两说了,不废此人,更待何时?

冯保闯了这么大的祸,竟还如此盲目乐观,其实原因也很简单:一个人当官当久了,就会变傻,并产生一系列幻
觉,自我感觉过于良好,最后稀里糊涂完蛋去也。

不过看在小时候陪自己玩过的份上,万历还是留了一手,安排他去南京养老,也没要他的命。

这是冯保,张居正就没那么好对付了。

\section[\thesection]{}

张先生在朝中经营多年,许多大臣都是他的人,现在刚死不到一年,立刻翻案恐怕众怒难犯。更麻烦的是,现任内
阁首辅张四维也是张居正一手提起来的,自然不肯帮忙,要想整治张先生,谈何容易。

然而很快,万历就发现自己错了。种种蛛丝马迹表明,除自己外,张先生还有一个敌人,一个他曾无比信任的
人----张四维。

这是一个极为古老的复仇故事,在真相揭开前,张四维已隐忍了太久。

张四维,字子维,山西蒲州人,嘉靖三十二年进士,看起来,这不过是份普通的官僚记录,但实际上,他的背景要
比想象中复杂得多。

张四维的父亲,叫做张允龄,是一名普通的山西商人,不算什么人物,但他母亲王氏却不同凡响----王崇古的姐姐。

也就是说,张四维是王崇古的外甥。之前已经说过,朝廷实力派人物杨博也是山西人,而且他的儿子娶了王崇古的
女儿,也就是说,杨博的儿媳妇是张四维的表妹,看上去比较复杂是吧,后面还有。

后来张四维生了两个儿子,一个叫张甲徽,一个叫张定徽,他们两个几乎同时结婚,老婆却是亲姐妹----杨博的两
个孙女。

什么叫特殊利益集团,相信你已经明白了。

王崇古是宣大总督,杨博是兵部尚书(后改吏部尚书),位高权重,却并非张居正的人,还经常对他颇有微辞。舅舅
和亲家都这样,张四维的立场自然也差不多。

当然,张四维的这些路数张居正都很清楚,所以早在万历三年(1575),他就推荐张四维进入内阁,成为了大学士,
也算是先下手为强,卖个人情。

然而这一次,他终于犯了一个错误,一个他的老师曾经犯过的错误。十年前,徐阶推荐高拱入阁,认为能卖高拱一
个人情,十年后,张居正也这样想。

但事实上,张四维远没有他想得那么简单,在这个人的心中,还隐藏着一个更深的秘密。

五年之前的那一天,殷士儋大闹内阁,要和高拱单挑,张居正劝架,却也挨了骂,正是在这场闹剧中,张居正坚定
了除掉高拱的决心,但与此同时,他似乎也忽略了一个重要的问题----为什么殷士儋会在那一天突然发作?

\section[\thesection]{}

原因很简单,因为就在那一天之前,殷士儋得到了一个确切的消息:高拱准备赶走他,换一个人入阁。实在是忍无
可忍,殷学士鱼死网破,这才算雄起了一回。

而那个由高拱安排,入阁顶替殷士儋的人,正是张四维。

对于这份五年之后迟到的邀请,要他感恩戴德,实在比较困难。

好了,这起迷案就要水落石出了,我们现已掌握了如下四点:

1、 王崇古与高拱关系紧密,他的职务是由高拱推荐的。

2、 张居正准备解决高拱之时,杨博曾亲自上门,为高拱求情。

3、 张四维是王崇古的外甥,也是杨博的亲家。

4、 高拱曾推荐张四维入阁,以取代不听话的殷士儋。

于是我们可以得出这样一个结论:

张四维是高拱的亲信,一个由始至终,极为听话的亲信。

所以我们有理由相信,当张居正联合冯保赶走高拱的时候,一道阴冷的目光正投射在他的背后。

当然,自信的张居正是绝对不会在意的,在得意的巅峰,无人能撼动他的地位,于是当内阁缺少跑腿的人时,他毫
不犹豫地选择了张四维,那个看上去极其温顺听话的张四维。

之后的一切,就是顺理成章了,张居正活着,他无能为力,现在人死了,该是行动的时候了。

万历十一年(1583),陕西道御史杨四知突然发难,上书弹劾张居正十四大罪,就如同预先彩排过一样,原先忠心耿
耿、言听计从的诸位大臣一拥而上,把张居正从五、六岁到五十六岁的事情都翻了出来,天天骂日日吵,唯恐落后
于人。

眼见群众如此配合,万历自然也不客气,立刻剥夺了张居正的太师等一切职务,并撤销了他``文忠''的谥号。之后
不久他更进一步,抄了张先生的家。

之所以搞抄家,原因只有两个,愤怒,以及贪婪。

在万历小时候,张居正经常对他提出一个要求----勤俭。每年过年的时候,万历想多摆几桌酒席,张居正告诉他,
国家很困难,应该节俭,万历表示同意,皇帝进出场合多,万历想多搞点仪仗,显显威风,张居正告诉他,这些把
戏只会浪费国家资源,搞不得,万历表示同意。

在张居正死前,无论万历对他有何不满,也就是个工作问题,然而随着检举揭发的进一步进行,皇帝大人惊奇地发
现,原来张先生的日子过得很阔,不但好吃好喝,而且出门阔气无比,还有顶三十二个人抬的轿子。

让我省吃俭用,你自己过舒坦日子?还反了你了!

\section[\thesection]{}

而在愤怒之后,就是贪婪了,毕竟皇帝陛下也要用钱,被卡了这么多年,不发泄实在对不起自己,抄家既能出气,
又能顺便捞一把,何乐而不抄?

万历十一年(1583)四月,抄家正式开始。

其实说起来抄家也没啥,抄家的人家多了去了。倒霉了就抄家,抄完拉倒,今天你抄我,明天我抄你,世道无常,
习惯了就好。

但是张家的这次抄家,却并非一个简单的经济问题,而是一场不折不扣的惨剧,是惨无人道的人间地狱。

四月底,司法部副部长丘橓由北京出发,前往张居正老家荆州抄家。本来也没什么,人到了就抄好了,可是破鼓总
有万人捶,对广大官员们而言,看见人家落井,不丢一块石头下去,实在是件太难的事情。

原先毕恭毕敬的地方官听说张居正倒了台,为了在抄家中争取一个好的表现,竟然提前封住了张家的门,不准人转
移财物。

这么一搞,不但财物没能转移,连人也没转移,因为张家的几十口人还躲在家里,又没有粮食,但这似乎不关地方
官的事,于是等丘部长抵达,打开门的时候,他看见的,是十几个已经饿死的人和几十个即将饿死的人。

没关系,饿不死的,抄家也可以抄死你。

经过几天的抄家统计,从张居正家中共抄出黄金上万两,白银十多万两,如此看来,张居正在搞政治的同时,也没
少搞经济。但总的来说,还不算太过分,和他的前辈严嵩、徐阶比起来,也算是老实人了。

没办法,大仇未报,人家本来就是冲着人来的。很快就传出消息,说张居正家还隐藏了二百万两白银,不抄出来誓
不罢休。于是新一轮运动开始,先是审,审不出来就打,打得受不了了,就自杀。

自杀的人,是张居正的长子张敬修,但在死前,他终于发觉了那个潜伏幕后的仇人,并在自己的遗书中发出了血泪
的控诉:

``有便,告知山西蒲州相公张凤盘,今张家事已完结,愿他辅佐圣明天子于亿万年也!''

所谓张凤盘,就是张四维,所谓辅佐圣明天子于亿万年也,相信读过书的都能明白,这是一句骂人的话,还顺道拉
上了万历。

这就是张敬修临死前的最后一声呐喊。

\section[\thesection]{}

但张敬修不会想到,他这一死,不但解脱了自己,也彻底解脱了张居正,以及所有的一切。

张敬修一死,事情就闹大了,抄家竟然抄出了人命,而且还是张居正的儿子,实在太不像话。恰好张四维两个月前
死了爹,回家守制去了。他这一走,原先的内阁第二号人物申时行,就成为了朝廷首辅。

这位仁兄还比较正派,听说此事后勃然大怒,连夜上书要求严查此事。万历也感觉事情过了,随即下令不再追究此
事,并发放土地,供养张居正的母亲家人。

事情终于解决了,万历的仇报了,他终于摆脱了张居正的控制,开始行使自己的权力。张四维的心愿也已了结,他
在家乡守孝两年,即将期满回朝之际,却突然暴病身亡,厚道的人说他死得其所,不厚道的人说这是干了缺德事,
被张居正索了命。

无论如何,仇恨与痛苦,快乐与悲伤,都已结束。

在之前的文章中,我曾经写过无数个人物,有好人,也有坏人,而张居正,无疑是最为特殊的一个。

他是一个天才,生于纷繁复杂之乱世,身负绝学,以一介草民闯荡二十余年,终成大器。

他敢于改革,敢于创新,不惧风险,不怕威胁,是一个伟大的改革家,他也有缺点,他独断专行,待人不善,生活
奢侈,表里不一,是个道德并不高尚的人。

一句话,他不是好人,也不是坏人,而是一个复杂的人。

但在明代浩如烟海的人物中,最打动我的,却正是这个复杂的人。

十年前,当我即将踏入大学校园时,在一个极为特殊的场合,有一个人对我说过这样一番话:

你还很年轻,将来你会遇到很多人,经历很多事,得到很多,也会失去很多,但无论如何,有两样东西,你绝不能
丢弃,一个叫良心,另一个叫理想。

我记得,当时我碍于形势,连连点头,虽然我并不知道这句话的真实含义。

一晃十年过去了,如他所言,我得到了很多,也失去了很多。所幸,这两样东西我还带着,虽然不多,总算还有。

当然,我并不因此感到自豪,因为这并非是我的意志有多坚强,或是人格有多高尚。唯一的原因在于,我遇到的人
还不够坏,经历的事情还不够多,吃的苦头还不够大。我也曾经见到,许多道貌岸然的所谓道学家,整日把仁义道
德放在嘴边,所作所为却尽为男盗女娼之流。

我并不愤怒,恰恰相反,我理解他们,在生存的压力和生命的尊严之间,他们选择了前者,仅此而已,虽不合理,
却很合法。

我不知道,是否所有的人在历经沧桑苦难之后,都会变成和他们一样的人。

直到我真正读懂了张居正,读懂了他的经历,他的情感,以及他的选择。我才找到了一个答案,一个让人宽慰的答
案。

他用他的人生告诉我们,良知和理想是不会消失的,不因富贵而逝去,不因权势而凋亡。

不是好人,不是坏人,他是一个有理想,有良心的人。

\section[\thesection]{}

张居正,字叔大,嘉靖四年(1525)生,湖广江陵人。

少颖敏绝伦,嘉靖十八年(1539)中秀才,嘉靖十九年(1540)年中举人,人皆称道。

嘉靖二十六年(1547),成进士,改庶吉士,授翰林编修,徐阶辈皆器重之。

嘉靖四十一年(1562),徐阶代嵩首辅,倾心委于张居正,信任有加,草拟遗诏,引与共谋。

隆庆元年(1567),张居正四十三岁,任礼部尚书兼武英殿大学士,加少保兼太子太保,进入内阁。

隆庆六年(1572),隆庆驾崩,张居正引冯保为盟,密谋驱逐高拱,事成,遂代拱为内阁首辅。

万历元年(1573),张居正主政,推行考成法,整顿官吏,贪吏闻风丧胆,政令传出,虽万里外,朝下而夕奉行。

万历六年(1578),丈量天下土地,推行一条鞭法,百姓为之欢颜,天下丰饶,仓粟充盈,可支十年有余。

万历十年(1582)六月,张居正年五十八岁,去世,死后抄家。长子自尽,次子充军。

有的人活着,他已经死了。

有的人死了,他还活着。

世间已无张居正。

一个神秘的年份

张居正死了,但生活似乎并没有什么变化,特别是对万历而言。

刚满二十岁的他踌躇满志,虽然他不喜欢张居正,却继承了这位老师的志向。自从正式执政以来,一直勤奋工作,
日夜不息,他似乎要用行动证明,凭着自己的努力,也能够治理好这个国家,至少比那个人强。

所以从万历十一年(1583)起,他显现出了惊人的体力和精力,每天处理政务时间长达十余个小时,经常到半夜还要
召见大臣,而且今天的事情今天办,绝对不会消极怠工。

这并非夸张,事实上,他还干过一件更为夸张的事情。

万历十一年(1583),北京地区大旱,当年没有天气预报,也搞不了人工降雨,唯一的办法是求雨。

虽然这招不一定灵,但干总比不干好。一般说来,求雨的人级别越高,越虔诚,求到雨的机率就越大。因为当时的
人认为,龙王也有等级,也讲人际关系,降不降雨,降多少,什么时候降,马屁响不响,那是比较关键的。

而这一次,万历打算自己去。

他求雨的地点,在南郊天坛。

皇帝求雨也不新鲜,但这次求雨却十分不同,因为万历兄……是走着去的。

\section[\thesection]{}

我来解释一下这件事情的特别之处,当年皇帝住的地方,就是今天的故宫,而天坛----就是今天的天坛。

去过北京的人应该知道,这两个地方相隔比较远,具体说来,至少有五公里。上个月我坐出租车去,还花了二十分
钟,而万历是坐11路车去的----两条腿。

不但走着去,还走着回来,在场的人无不感佩于他的毅力,同时也无奈于他的执著----皇帝走,大家也得跟着走。

除了徒步拉练锻炼身体外,万历对百姓生活也很关注,比如当时山东、山西、湖广等地遭遇灾荒,地方官报告上来
说:按照考成法,无论如何我们也是收不齐了,麻烦您通融通融,把今年的任务降一降。

一天之后,他们等到了皇帝的回复,一个出人意料的回复:

``既然如此,那就不用收了,全都免了吧!''

这就是万历同志的觉悟,在张居正死后,他一直保持着激昂的斗志与热忱,直到那个神秘年份的来临。

人生很漫长,但关键处只有几步。相信这句话很多人都听过,但是许多人并不知道,其实历史也是如此。

公元755年,当唐朝文明处于巅峰之时,一个叫安禄山的矮胖子突然起兵闹事,揭开了安史之乱的序幕,繁荣的唐朝
从此陷入衰弱。

公元755年,这个年份就此成为了一个转折点,被载入史册。

八百年后,宿命的转折再次到来。没有原因,没有预兆,停留在这个神秘的年份----万历十五年。

简单说来,在这一年,发生了三件事情,两件不大的大事,一件不小的小事。

第一件大事:戚继光去世了。

在十余年的时间里,戚继光是个无人敢惹的角色,虽然偶尔也有几个不怕死的言官弹劾他吃空额搞钱,在军中培养
个人势力等等,却始终没有结果。究其原因,除了后台太硬外,还是由于水平太高,边界没他不行。

但事实证明,水平不如后台好使,张居正死后,戚继光就被调离了蓟州,去了广东,虽然职位没变,但戚继光明
白,自己的时代已经结束了。

于是他称病不出,不久后,便离职回了登州老家。

三十六年前,他从这里出发前往北京,开始了波澜壮阔的一生:先打蒙古人,再打日本人,练兵东南,横扫倭奴,
驱逐胡虏,无人可挡,

战功之显赫,四十年中无人可望其项背,盖世之威名,四百年后声震寰宇,万民皆知。

\section[\thesection]{}

万历十五年(1587)二月,这位传奇英雄在家乡病逝,年六十岁。去世前留言如下:三十年间,先后南北、水陆、大
小百余战,未尝一败!

我知道,他之一生,已无任何遗憾。

第二件大事:海瑞死了。

海先生终于还是死了。在被高拱罢官之后,他回到了老家,没人管他,三年之后,高拱下台了,张居正执政,依然
没有人管他。

这实在不是高拱和张居正不识货,恰恰相反,他们都很去清楚海先生的实力。无奈的是,海先生的能量就如同熊熊
烈火,和他呆久了,不被烧死,至少也是个残废。

现在张居正死了,用某位史学家的话说,朝廷里的明白人都死光了,于是海瑞先生得到了再次出山的机会。

万历十三年(1585),经万历皇帝亲自批示,海瑞被任命为南京都察院佥都御史,赶赴南京上任。这一年,海瑞七十
二岁。

海先生是天字第一号职业官僚,接到命令即刻上路,连东西都不怎么收拾(当然,他也没多少东西),就去了南京。

而当他来到南京郊外的时候,才发现原来进城是一件极其困难的事情----太挤

海先生要来了!南京城轰动了,官员们激动了,商人激动了,农民也激动了,于是大家集体放了假,不做生意,不种
地,凌晨就带着被子,跑到城外占地方,想抢一个靠前的位置,一睹海先生的风采。

由于人太多,导致海先生一直未能进城,被牢牢地堵在外面,直到南京兵部派出军队开路,这才把海大人迎了进来。

等到海瑞进了城,找到都察院住下来,才被告知,他不应该住在这里,倒不是人家欺负他(谁敢),只是因为他老人
家又升官了。

万历兄实在是大方,感觉给个佥都御史(四品)还不够意思,人还在路上,就下了第二道任命令,把海先生再提一
级,让他当了南京人事部副部长(吏部侍郎)。

据说这个消息公布后,南京都察院的御史们一片欢呼雀跃,兴高采烈,而吏部的官员们垂头丧气,比死了爹还难
受,但事实证明,他们还是悲观了点,实际上,此时的海瑞先生压根没空去收拾他们。

因为他连家门都出不去。

自从进入南京,海瑞的家就被众多闻名而来的粉丝围得水泄不通,那架势,比天皇巨星还要天皇巨星。

\section[\thesection]{}

更让人吃惊的是,在没有汽车火车的当时,有很多人是从远处走来的。最猛的当属一位福建的老兄,据说他走了上
千里路,穿坏了十多双鞋,一个多月才到南京。

海瑞听说此事,十分感动,以为他要伸冤,亲自接见了他。

可是这个人进来后,只是看着海瑞,行了个礼,然后扬长而去。

有人问:你干嘛来?又干嘛走?

答:我只想看看海青天,看完了,不走还等什么?

这就是清廉与正直的力量。

除了吸引大批拥护者外,海瑞还获得了一个荣誉,一个前有古人,后无来者的荣誉。

中国的老百姓历来都怕妖魔鬼怪,所以有贴门神的习惯,几乎家家都贴,款式也不一而同,但门神的主要人物是固
定的,也就是关羽、秦叔宝那一拨人。上千年来也就这么几个,毕竟要成为形象代言人要求太高,不但要能打,长
得还得有特点(想把鬼吓跑,没特点不行)。

而现在,海瑞先生终于加入了这个光辉的队伍,成为门神部队的最后一名成员(此后再无编制)。在当时的南京,作
为正义与公道的象征,海瑞先生的画像被贴得满街都是,除了门上,客厅、卧室里也有人挂。据说每天看一眼,可
以百病不侵,而且具有良好的避邪作用。

虽然经常忙于公共宣传事业,但海瑞先生没有怠慢工作,他没精力去整治吏部的那帮人,却也没闲着,百忙之中仍
向皇帝上了一封奏疏:

根据以往经验,海瑞先生的文书,一般都是惊天地泣鬼神的,这篇也不例外。在文章中,海瑞先生建议,考虑到目
前贪污情况严重,应该恢复太祖(朱元璋)时期的刑法,对贪污八十贯以上者一律处决,并将其剥皮,放在县衙门
口,警示后人。于是大家真的愤怒了,惹不起你,总躲得起你吧。可海先生却是躲都不让人躲,不搞出个玉石俱焚
誓不罢休。

客观地讲,海瑞的这封文书的确是过分了,且不说剥皮问题,都过了两百多年了,经济发展这么快,确定死刑标准
时总得考虑个通货膨胀问题吧,当年买一栋房,今天也就能买点粮,为几斤粮食就要剥人皮,兄弟你也太狠了点吧。

但在海瑞看来,他的做法是对的,当然,这只是他的个人想法。

\section[\thesection]{}

万历兄虽然年轻,但神智也很清醒,他好言抚慰了海先生一把,就把奏疏丢进了废纸堆。

而海先生在南京日盼夜盼,没有等到剥皮匠的出现,却等来了升官的命令,由于工作努力,他被任命为南京都察院
右都御史,那位四十多岁还不入流的教育局长,经过二十多年不可思议的经历,终于成为了正二品(相当于正部级)
的高级官员。

这回都察院的仁兄们完蛋了。

南京是明朝的第二首都,从六部到都察院,所有北京有的中央机构它都有,但毕竟皇帝大人住北京,所以除了南京
户部(管理南方户籍)和南京兵部(统领南京军队)外,大多数机构都是摆设。

一般说来,只有在朝廷混不下去的,才会被发配到南京,美其名曰:养老。

都察院就是一个闲人部门,大家都没事干,骂人的自然也没事干,然而仅一夜之间,一切都已改变----海先生上任
了。

由于上班没事可干,自然就没人去上班了,于是都察院的御史们总是自得其乐,逛街的逛街,看戏的看戏----工作
没前途,还不准偷偷懒?

海瑞先生的答案是不,他拿出了三十年前治理学生的方法来对付御史----记考勤。但凡敢于迟到早退的,必须到单
位,哪怕没事干,也得坐在这里。

虽然大家明显表示出不适应,但海先生的威胁是很明确的----养老不是最惨的结局,下岗才是。

而随着整顿工作的进一步深入,御史们才发现,原来一切才刚刚开始,海先生很快玩出了新花样。

一次,有位御史过生日,在家请了戏班子唱戏,这在当年,应该是最寻常不过的事情,老百姓家也经常干,但海先
生却勃然大怒,把这位御史抓了起来,打了一顿板子,理由是:根据明太祖时期律令(注意这个日期),官员请人唱
戏违法,所以是打你没商量。

因为这件事干得实在有点过,御史们的精神压力开始陡然增大,每日在海先生的恐怖阴影下,战战兢兢,终于有一
天,畏惧变成了愤怒。

在明代,御史专管骂人,从皇帝到扫地的,想骂谁就骂谁,除了一个例外----御史长官,要知道,那是顶头上司,
不到万不得已,没人愿意给自己惹事。

现在,万不得已的时候到了。

\section[\thesection]{}

万历十四年(1586),御史房寰率先发难,攻击海瑞``大奸极诈,欺世盗名''。奏疏一上来,朝廷就炸了锅。海瑞这种
传奇人物,恨的人多,喜欢的也不少,大家开始吵作一团。而海瑞兄还是那么有性格,啥也不说,上了个辞职报
告----不想干了。

吵到最后,报到了皇帝那里,但万历兄的态度却十分奇怪。他既不处理骂人的房寰,也不批准海瑞辞职。该干嘛还
干嘛,搞得两位当事人都非常纳闷。

万历是一个很聪明的人。至少在海瑞的问题上,他比张居正要聪明得多。从一开始,他就没打算真用海瑞,因为他
很明白,这是个偶像型的人物,可以贴在门上,挂在墙上,烧香拜佛地供起来。

但绝不能用。

说到底,海先生只是个撑门面的。然而他自己,并不知道。就这样,他稀里糊涂地在这个位置上干了下去,直到万
历十五年(1587)的那个冬天,死亡降临到他的头上。

他没有儿子,仅有的妻子女儿也已先他而去。在他生命的最后一刻,只有一个老仆人陪伴着他,在寒风呼啸之中,
海瑞对仆人说出了人生的最后遗言。

按照常理,像海瑞先生这样的奇人,遗言必定非同凡响,往往都带有深刻含义,比如什么人生短暂,努力工作之
类,或是喊两句口号,让大家热血沸腾一番。

然而海先生的遗言既不深刻,也不沸腾,只是让人瞠目结舌:

``明天,你送六钱银子到兵部。''

说完就去了。

这是一句看上去十分无厘头的话,也是威名赫赫,语不惊人死不休的海瑞先生的唯一遗嘱。

这句话的来由是这样的:由于当年没有暖气,每逢冬天,兵部就会给各部的高级官员送柴火钱,数量也不多。

而在他死之前的那天,兵部送来了柴火钱,而经其本人测量,多给了六钱银子。

这一次,我是彻底无语了。

在海瑞死后,他的好友佥都御史王用汲来为他收尸。遍寻海瑞的住处后,他只找到了几件打着补丁的破衣服,和几
口装着破衣服的破箱子。

为官三十年,二品正部级南京都察院右都御史海瑞,这就是他的全部财产。

在听说海瑞的死讯后,南京城出现了一幕前所未有的场景:男女老幼无论见过海瑞与否,都在家自发为他守孝,嚎
啕大哭。出殡的时候,据说为他送葬的人排了上百里,整整一日,无人离去。

人民,只有人民,能公正地评价一个人。

\section[\thesection]{}

如何评论这位传奇人物,实在是一个难题,对的说了,不对的也说了,现在要搞个总结,实在谈何容易。

在名著《围城》中,钱钟书先生借用别人之口,对那位命运多变的主人公方鸿渐做出了这样一个评价:

你是个好人,却并无用处。

我想,这句话也同样适用于海瑞。

在黑暗之中的海瑞,是一个无助的迷路者。

第三件事,才是一切的关键所在。

自万历十四年(1586)十一月起,一贯勤奋的万历皇帝突然变了。

他开始消极怠工,奏疏不及时批示,上朝也是有一天没一天,大臣询问,得到的答案是:最近头晕眼黑,力乏不兴。

既然身体不舒服,那就歇会吧,在当时的内阁首辅申时行看来,这不过是个生理问题。不久之后,没准还要陪这位
仁兄去天坛拉练,等一等就是了。

一直等到死,他也没能等到这个机会。

到万历十五年(1587),万历兄算是彻底不干了,不但不上朝,除了内阁大臣外,谁也不想见,每天闷在宫里,鬼知
道在干些什么,他的爷爷嘉靖皇帝怠工二十多年,看这个势头,这孙子打算打破这一纪录。

事实上,他确实做到了。

在明代历史中,有很多疑团,比如建文帝之谜,比如明武宗之死,对于这类问题,我一向极有兴趣,研究之后,多
少也能略得一二,只有这个迷题,我始终未能解开。

为什么那个热血青年会突然变成懒汉?为什么偏偏是这个时候?为什么偏偏是这种举动?

一般说来,人性的突然转变,往往是因为受了某种较大的刺激,那么到底是什么刺激?在万历十五年的深宫之中,到
底发生了什么?

以上问题,本人全然不知。

我唯一知道的是,自此之后,大明帝国进入了一个奇怪的状态,迷一般的万历王朝正式拉开了序幕,无数场精彩的
好戏即将上演。

闪电战

万历十五年(1587),万历皇帝消停了,但这对于老百姓而言,未尝不是一件好事----不动总比乱动好,只是大臣们
有点意见,毕竟每天都见不到领导(内阁大臣除外),伤心总是难免,不过到目前为止,也还没闹出什么大事。

平静,一切都显得那么平静。

四年之后,平静被打破,因为一封不起眼的奏疏。

\section[\thesection]{}

万历十九年(1591)八月,福建巡抚赵参鲁奏报:

根据琉球使节反映,近日突然出现上百来历不明者,前往琉球朝鲜一带收购海图以及船只草图,并大量收购木材火
药,用途不明。

在当时,每天送往朝廷里的奏疏多达几百封,基本上都由内阁批改(皇帝已经不干活了),和什么水灾民变比起来,
这件事情实在太小,于是它很快就被埋入了公文堆中。

两个月后,浙江巡抚奏报:

近日获报确知,倭酋平秀吉于北九州肥前国荒野之上修筑城池,规模甚大,余情待报。

上一封大家都看得懂,这一封就需要翻译了。

所谓倭,就是日本,所谓酋,就是头头,所谓平秀吉,就是丰臣秀吉。

具体说来,是日本的头头丰臣秀吉在北九州的荒野上修了一座城池。

这实在是一条太不起眼的新闻,所以很快它也被埋入了纸堆。

顺便说一句,丰臣秀吉修建的那座城池现在还在,而且还比较有名----名古屋。

今天的名古屋是日本的重要城市,关西地区的经济交通中心,但在当时,修建这座城池,只有一个缘由。

当这座城池建好的时候,站在城楼的最高点,就可以清晰地看到一个地点----朝鲜海峡。

这是两条看起来毫无关联的信息,所以无人关注,但当它们联系到一起的时候,事情已经不可挽回。

万历二十年(1592)五月二十四日,水落石出。

五月二十六日,辽东巡抚紧急奏报:

``急报!前日(二十四日),倭贼自釜山登陆,进攻朝鲜,陆军五万余人,指挥官小西行长,水军一万余人,指挥官九
鬼嘉隆,藤堂高虎,水陆并进,已攻克尚州,现向王京(汉城)挺进,余者待查。''

六月十三日,辽东巡抚急报:

``急报!已探明,倭军此次进犯,分九军,人数共计十五万八千七百余人,倾国而来,倭军第一军小西行长,第二军
加藤清正,第三军黑田长政已于昨日(十二日)分三路进逼王京,朝军望风而逃,王京失陷。朝鲜国王李昖逃亡平
壤,余者待查。''

七月五日,辽东巡抚急报:

``十万火急!七月三日,倭军继续挺进,抵近平壤,朝军守将畏敌贪生,打开城门后逃之夭夭,平壤已失陷,朝鲜国
王李昖逃往义州。''

七月十六日,兵部尚书石星奏报:

``自倭贼入侵之日起,至今仅两月,朝鲜全境八道已失七道,仅有全罗道幸保。朝军守将无能,士兵毫无战力,一
触即溃,四散而逃,现倭军已进抵江(鸭绿江)边,是否派军入朝作战,望尽早定夺。''

最危急的时刻到了。

• 更正声明

经本人检查,发现前日更新中有一地理错误,现诚挚检讨,并探究该错误之来由,以达到惩前毖后,治病救人之目
的。

经查,1591年丰臣秀吉所建的城,确实叫名古屋,俺查阅了很多日文史料,白纸黑字写着是名古屋,而且发音似乎
都一样,所以就写上了,但后来再一查,原来此名古屋非彼名古屋,当时的那个名古屋,具体位置是在九州的肥
前,也就是今天的日本佐贺县,和今天的日本名城名古屋是两个地方。

总而言之,是搞错了,对不住大家,日语不到家,实在抱歉,今后一定注意,特此更正。

那就这样吧,诸位保重,过两天见。

当年明月

2007年11月23日

\section[\thesection]{}

答案已经揭晓,原因却发生在七年之前。

万历十三年(1585),当万历兄步行拉练到天坛的时候,几千里外的日本正在闹腾一件大事。

丰臣秀吉在京都接受了日本天皇的册封,成为了日本的最高官员----关白(相当于丞相),长达二百余年的战国时代
终于结束了。

日本是一个比较喜欢折腾的国家,天皇是挂名的,说话算数的是幕府的将军,换句话说,是手里有兵的人。但自
1467年起,由于内部胡搞乱搞,将军失去了对全国的控制,这下子热闹了。

日本的管理体制,天皇下面是将军,将军下面是大名,也就是各地的诸侯,既然天皇没屁用,将军又过了期,就轮
到大名说话了。

所谓大名,也没个谱,在那年头,只要你有兵有地盘,就是大名,日本国家不大,闹事的人却多,转瞬之间冒出来
几十个大名,个个有名有姓,占山为王,什么羽前羽后,越前越后,土佐中国,上总下总(全都是日本地名),看起
来好似广阔,其实许多地方也就是个县城。

说句寒酸话,日本历史中大书特书的所谓战国时代,也就是几十个县长(个别还是乡长)打来打去的历史,更讽刺的
是,最后统一县长们的,竟然是个农民。丰臣秀吉,原名木下藤吉郎,本来在乡下种地,后来种不下去了,就出去
做小生意,正好到处打仗,他就去参了军,在县长大名织田信长的手下混碗饭吃。

偏巧这人种地做生意都不行,打仗谋略倒是一把好手,从小兵干起,步兵队长,步兵大队长,家老,部将,一级级
地升,最后成为了织田县长的第一亲信,由于这人长得很丑,和猿猴有几分神似,所以织田县长给他取了个外
号----猴子。

当时织田县长已经统一了大半个日本,如无意外,等到其他县长们被解决完,织田兄去当将军,猴子兄应该也能混
个县长干干。

可是猴子的运气实在太好,1582年,织田县长在寺庙休息的时候,被一个叫明智光秀的手下给干掉了,据说是因为
当晚织田县长嫌送上来的鱼臭,把明智乡长给骂了一顿,于是乡长一怒之下,把县长干掉了(就为这么个破事,心理
实在太过阴暗)。日本史称``本能寺之变''。

\section[\thesection]{}

此时木下藤吉郎已经改名了,他先改叫木下秀吉,现在叫羽柴秀吉(最后又改成丰臣秀吉),日本人的观念是有奶就
是娘,改个把名字那是家常便饭,不用奇怪。

这位羽柴乡长正在攻击中国(日本地名)地带的毛利县长,得到消息后十分镇定,密不发丧,连夜撤军回援,日本史
称``中国大回转''。

回去之后,羽柴乡长和明智乡长打了一仗,把明智乡长打败了,此后他又再接再厉,在贱岳(日本地名)击败了最强
的竞争对手柴田胜家,获得了织田县长的全部地盘,史称``贱岳之战''。

在和柴田乡长的战斗中,羽柴乡长的军队中涌现出了七名优秀的将领,他们作战勇敢,后来被统称为``贱岳七支
枪''。

顺便提一下,本人曾经考证过,这七个人中,有几位在战场上中使的是刀,如此说来叫贱岳X把刀似乎也可以,不过
人家说是枪那就叫枪吧。

之所以提到这件事,是因为这七支枪里的五支,和后来那场惊天动地的战争有着莫大的关系。

此后,羽柴乡长更是一发不可收拾,陆续打平其余县长,最终统一日本,搞定了天皇,改名为丰臣秀吉,并自称为
太阁。

丰臣秀吉这个人,内心相当相当之阴暗,自打成功当上乡长,他就一直对天感叹,俺怎么呆在了日本,在他看来,
像自己这样的天才,征服个把县城实在显不出威风,只有统一全世界,才能体现个人价值。

当然,猴子兄的目的只限于征服朝鲜,中国,印度及东南亚,这并非他太过谨慎,实在是因为他一天到晚呆在岛
上,地理知识有限,不知道什么法国德国,对他而言,世界就那么几个国家而已。

其实丰臣兄并非特例,事实证明,日本国一向盛产心理阴暗之变态者,后来的如近卫文磨、东条英机之流,都是一
路货,在他们的心中,从没有什么和平发展之类的概念,总觉得别人的比自己的好,抢劫的比生产的好,而他们的
世界观,也有着惊人的一致:

欲征服世界,必先征服亚洲,欲征服亚洲,必先征服中国。

从爷爷开始,到孙子,再到孙子的孙子,这帮孙子几百年来窝在岛上,做着同一个梦,却始终不醒,实在是难能可
贵。

而丰臣秀吉,就是这些孙子中的极品。

\section[\thesection]{}

丰臣秀吉在统一日本之后,嘴边开始念叨这样一句话:

``在我生存之年,誓将唐之领土纳入我之版图。''

这里的唐,就是指明朝。因为唐朝时候,日本向中国派遣了很多留学生,带走了很多技术、文化,甚至政治制度,
所以日本人一直称中国为唐。

几百年前,无私之援助,全力之支持,只换来今天的野心、杀戮和侵略,所以同志们务必要记住一个道理:

扶贫,是要看对象的。

但要占据中国,必须征服朝鲜,于是他开始和朝鲜国王李昖谈判,要求他们让路,帮助自己进攻明朝。

当时的朝鲜并不是独立国家,而是明朝的属国,国王要向大明皇帝称臣,称明朝为天朝,称明军为天兵。但凡国王
即位,册立世子,甚至娶老婆,都要事先向明朝报批,获得批准之后才能做。

所以虽然这位李昖国王是个比较糊涂的人,关键问题上还把握得住,严辞拒绝了日本使臣。

既然软的不行,那就来硬的,丰臣秀吉随即决定:先攻朝鲜,再占中国!

可他还没壮志凌云几分钟,就得知了一个消息,他的养子丰臣秀次反对进攻朝鲜,理由固然是世界和平,大众平安
之类的话,但丰臣秀吉明白,这位养子是不想去卖命。

于是他灵机一动,写了一张纸条,派人交给了丰臣秀次。

这张纸条充分地证明了一点:丰臣秀吉已彻底疯狂。

因为上面只写了这样一句话:

五年之内必定攻下明国,到时你就是明国的关白!

但事实证明,他的疯狂也是有理由的。

客观地讲,丰臣秀吉是一个奇才。他以庶民出身,苦熬几十年,最终一统日本,绝非寻常人物,而且此人在日本国
内,向来以谋略出名,从不打无把握之战,战国时期曾亲自指挥过几十次战役,除掩护撤退的必败之战外,他只输
过一次。

顺便提一下,他唯一战败的那一次,对手叫德川家康。

而在决心打这一仗之前,丰臣秀吉已经考虑了很久。

日本人的一个最大特点是做事认真,比如在后来中日甲午战争之前,他们向中国派出了大量间谍,拍摄了很多照
片,北洋舰队每条船的吨位,人员,指挥官,炮口直径,缺点,日军都有详细的记录。

\section[\thesection]{}

而在抗日战争开始前,其工作更是无以伦比,所有中国少将以上的军官,他们都有细致的档案留存,其个人特点、
作战方式甚至生活习惯都一清二楚,更为可怕的是,他们绘制的中国地图,比中国人自己绘制的还要准确,连一个
山丘,一口井都标得极为清晰。当年阎锡山的部队伏击日军后,既不抓俘虏,也不扛弹药,第一要务就是开始找日
军军用地图----拿回去自己用。

而一贯小心谨慎的丰臣秀吉之所以如此自信,是因为他想不自信都不行:

当时的日本,刚刚实现和平统一,在此之前,国内已经打了一百多年的仗,用今天的话说,打仗已经成了一种生活
时尚,有些武士家吃饭的时候,一手拿筷子,另一只手都握着刀,只要外面招呼一声,立马就抄家伙出去砍人。

而且这帮人打仗极其勇敢,每次作战都要争先锋(首先发起冲锋者),还经常为此发生纠纷,没有当上先锋愤然自杀
的,也不在少数。

总而言之,这是一帮亡命之徒。

相信出乎很多人的意料,当时的日本,军队装备已经十分先进,为了打赢对手,他们纷纷进口先进武器,大刀长矛
之类的玩意已不吃香了,大名们纷纷长枪换鸟枪,鸟枪换大炮,加上还有汪直这类的军火贩子一个劲地往日本倒腾
武器,到战国末期,日本已拥有了大规模的火枪部队。

在战术方面,日军也有相当的进步,公元1575年,织田信长在长筱发起了一场决定性的战役。对手是号称战国第一
诸侯武田信玄的儿子武田胜赖,其部队以骑兵为主,使用孙子兵法四如真言``风林火山''为军旗,战斗力极为强
劲,在骑兵对决无法取胜的情况下,织田信长冥思苦想,创造性地发明了三线战术(日语:三段击)

关于这一战术,之前已经介绍过了,由于火枪部队射程有限,且装弹药需要时间,故将部队分为三线,一线开枪,
二三线装子弹,形成持续火力,对骑兵有较大杀伤力。

虽说早在两百多年前,明军开国将领沐英就曾首创这一战术,但至少在日本,织田县长还是有专利权的,而且和后
来使用同一战术的普鲁士腓特烈二世相比,他也还早了一百多年。

整体看来,日军的战斗力、军事装备、战术水平已经达到了一个相当高的程度,高到丰臣秀吉足以为之而自豪。

相对而言,日本的对手就有点疲软了。

\section[\thesection]{}

朝鲜自李成桂光荣革命,成立李氏王朝后,基本就没打过什么仗,所谓``两百年平宁之世,民不知兵'',部队也就
是个混饭吃的地方,军事素质极差,连民兵都不如。

虽说在军事上朝鲜十分差劲,但搞起政治斗争来,那是一点也不消停。与明朝比起来,有过之而无不及,当时的朝
廷内部,分成两大派,分别叫做``东人党''和``西人党'',闹了一段之后,东人党又分裂成``南人派''和``北人
派'',东南西北都来齐了,足可以凑一桌麻将。

大体就是如此,反正朝鲜是乱得一塌糊涂,指南打北,不是东西。这么个状况,想让人家不动你,实在是有点难。

而日本的另一个对手,中国,就比较有趣了。

由于没有电报和照相机,加上当年日本穷,衣服也很土,想派间谍混入中国,很有可能被当成盲流遣返,所以关于
中国的情报,来源大都要靠倭寇。

而对丰臣秀吉影响最大的,无疑是这样一段对话。

那是在1585年,丰臣秀吉刚刚当上关白后不久,无意之中见到一个人,此人姓名不详,曾在汪直海盗有限公司工作
过,为了解明朝实力,他找这人谈了几次话,询问明军实力。

该仁兄是这样回答的:

``当年,我曾经跟着三百多人,到福建抢劫一年,所向披靡,无人可挡,最后平安而回。''(下福建过一年,全甲而
归)

吹完了,这位兄弟还搞了个评论:

``明朝很害怕日本,若日军进攻,就会如同大水崩沙,利刀破竹,无坚不摧。''(唐畏日如大水崩沙,利刀破竹,何
城不催)

除此之外,他还痛斥了明朝的政治腐败,官员贪污,老百姓流离失所,老百姓胆小怕事等等情况,总之,明朝就是
一软柿子,不捏都会烂。

丰臣秀吉大喜,于是他信了。

应该说,这位兄弟说的可能还是真话,一般说来,去当倭寇的,不太可能是良民,大都是些社会最底层的流氓无产
者,对政府不满,那是很自然的。

至于所谓打劫一年安然无恙,也可能是真的,倒不是他有多厉害,明军有多无能,而是倭寇这一行本来就是游击事
业,打一枪换一个地方,要真建立个根据地之类的玩意,几天也就没了。

唯一算得上有问题的,估计就是最后几句话了,所谓大水崩沙,利刀破竹,事实证明之后确实如此,不过就是换了
个主语而已。

\section[\thesection]{}

但必须承认,丰臣对中国形势的判断大致是正确的,当时的中国,已经没有开国时期的朝气,思想混乱,组织混
乱,吏治腐败,除了几支戚家军那样的模范军队,其余的所谓卫所部队,由于长官吃空额,且无人抓训练,基本都
变成了农民部队----除了种田,啥也不会。

用战斗经验丰富,基本不怕死的士兵,先进的武器装备和战术,去进攻政治腐败,喜欢内斗,且多年不打大仗的明
朝,无论从哪个角度看,都是稳赢不输。

所以丰臣秀吉很乐观----实在没有悲观的理由。

然而他错了,即使他运用经济学原理,把明朝的各种情况输入电脑,用模型公式证明自己必定能赢,他也一定会输。

因为他不懂得中国人。

几百年后的1937年,日本人决定开战,因为他们认为自己不可能输,当时的日本比中国有钱,士兵比中国精锐,武
器比中国先进,他们有三菱重工,有零式战斗机,有航空母舰,而中国内地四处是军阀混战,黑社会横行,老百姓
大多不认字,还怕死,重工业基本谈不上,飞机能数得出来,几条破船在长江里晃来晃去,且人心惶惶,一盘散沙。

所以他们告诉全世界,灭亡中国,三个月足矣。

于是他们打了进来,于是他们打了八年,于是他们输掉了战争。

因为他们不懂得中国人。

因为我们这个民族,是世界上最为坚韧的民族。

所谓的四大文明古国,其实大多名不副实,所谓埃及,所谓两河流域,所谓印度,在历史长河里,被人灭掉了N次,
雅利安人,犹太人,阿拉伯人,莫卧尔人,你来我往,早就不是原来那套人马了,文化更是谈不上。

只有中国做到了,虽然有变化,有冲突,但我们的文化和民族主体,一直延续了下来,几千年来,无论什么样的困
难,什么样的绝境,什么样的强敌,从没有人能真正地征服我们,历时千年,从来如此。

这是一个有着无数缺点,无数劣根性的民族,却也是一个有着无数优点,无数先进性的民族,它的潜力,统计学和
经济学计算不出,也无法计算。

日本人打进来之后才惊讶地发现,仅仅一夜之间,所有的一切都变了,军阀可以团结一致,黑社会也可以洁身自
好,文盲不识字,却也不做汉奸,怕死的老百姓,有时候也不怕死。

因为所有的一切,都已经牢牢地刻入了我们的骨髓----坚强、勇敢、无所畏惧。

日本人不懂得,所以他们失败了,以前如此,现在如此,将来依然如此。

从来不需要想起,也绝不会忘记,这是一个伟大民族的天赋。

\section[\thesection]{}

朝鲜的天才

万历二十年(1592)五月二十一日 名古屋

面对朝鲜海峡的方向,丰臣秀吉投下了他人生最大,也是最后的赌注。

十五万名日军士兵分别从福冈、名古屋、对马海峡出发,向着同一个目标挺进----为了同一个人的野心。

事实证明,这次行动的运输成本并不太高,因为在半年之后,一个可怕的对手将出现在对岸,为他们节省回程船票。

但既然是一生中最大的赌博,自然要押上全部的老本。

日本侵略军由日本国内最精锐的部队构成,总计十五万人,分为九军,由九个极有特点的人指挥,如下:

第一军:小西行长 一万八千人

第二军:加藤清正 二万二千人

第三军:黑田长政 一万二千人

第四军:岛津义弘 一万四千五百人

第五军:福岛正则 二万五千人

第六军:小早川隆景 一万五千人

第七军:毛利辉元 三万人

第八军:宇喜多秀家 一万一千人

第九军:羽柴秀胜 一万一千人

之所以列出这帮鬼子的姓名和军队人数,是因为其中大有奥妙。

以上九位鬼子军官的名字,中国人看了可能毫无感觉,但在日本国内,这帮人可谓是如雷贯耳,大有来头。

首先,人家有名字,就说明不是一般人了,因为在日本,姓名是奢侈品,只有贵族才有姓名,普通老百姓消费不
起,小孩生出来起个太郎、次郎之类的浑名(类似于阿猫阿狗),就这么凑合一辈子。

一直到后来明治维新,天皇感觉手下这一大帮子阿猫阿狗实在有损形象,便下令百姓申报姓名,当然了,具体姓什
么叫什么,都是自己说了算。

这下就热闹了。

在取名字(包括姓氏)的问题上,日本人充分发扬了能凑合就凑合的精神,不查字典,也不等不靠,就地取材,比如
你家住山上,就姓山上,你家住山下,就姓山下,家附近有口井,就叫井上,有亩田,就叫田中。

\section[\thesection]{}

而这九位仁兄自然不同,人家名字是有来历的,事实上,他们都是日本国内所谓的``名将''。

其中,第一军军长小西行长是丰臣秀吉的亲信,在九人之中,此人有一定文化,军事素养也较高。

而且他十分特别,虽说是个鬼子,却很有新潮意识,既不信佛教,也不信神道教(日本本土宗教),却是个基督徒。
每星期做礼拜,人家念阿弥陀佛,他说上帝保佑。

第二军军长加藤清正,和第五军军长福岛正则,是铁杆兄弟,他们就是之前提到的``贱岳七本枪''成员,分别排名
第二和第一。

这两个人在日本国内被誉为盖世名将,在战国时期立下了显赫战功,以勇猛善战著称,而且这两个人都是丰臣秀吉
的养子,对其十分忠心,但文化程度偏低,基本属于半文盲状态。

第三军军长黑田长政,在日本被称为``兵法大家'',据说精通兵法。他的父亲叫黑田官兵卫,是丰臣秀吉的两大军
师之一,号称日本智谋第一。

第六军军长小早川隆景,和第七军军长毛利辉元,是亲戚关系,具体说来,小早川隆景是毛利辉元的叔叔,为了混
家产,改了名字当了人家的养子,这也可以理解,那年头在日本,名字不值钱,一年改个十次八次的人也有。

这位小早川隆景,在日本也是个大名人,被称为``中国第一智将''(中国是日本地名),据说智商极高,和丰臣秀吉
有一拼。

最后一个拉出来评论的,是第四军军长岛津义弘。

之所以最后提到这个人,是因为他是个十分特殊的人物,特殊在哪里,很快你就会知道。

其余的几位就不提了,因为他们也就露这一次面,之后毫无出场机会,基本属于废物类型。虽然他们在日本国内也
被吹得神乎其神,但事实证明,废物就算吹一千遍,也还是废物。

而我提到的这几位,更是传奇级的人物,被吹得神乎其神,几乎个个都是智勇双全,成为了日本引以为豪的骄傲,
是日本战国时期的形象代言人,至于战场上的实际效果嘛……

但必须承认,这几位日本国内的战争精英到了朝鲜,确实表现出了精英的素质。

五月二十二日,日军先锋第一军小西行长发起进攻,仅用两个小时即攻破釜山,一路势如破竹,击破各路朝鲜军
队,仅半月之后就打到了汉城,第二军加藤清正,第三军黑田长政随即跟进,一路打到了平壤,把朝鲜国王赶到了
鸭绿江边。

之所以写得如此简略,不是我偷懒,真的是没办法,翻阅中日韩三国史料,这段时间可以用三个字来概括----一边
倒。

\section[\thesection]{}

总而言之,是朝鲜军不断地跑,日本军不断地追,甚至日军不追,朝鲜军也跑了,汉城不守,平壤也不守,仗打成
这个样子,要树立正面形象,那是相当的难。

但后来的事实充分说明,不是日军太坚强,只是朝军太软弱,建国二百多年,土匪都没怎么打过,除了自己折腾自
己,搞点政治斗争,闲来无事啃啃人参,估计也就差不离了。

而日军将领们的威名也就此树立起来,在无数日本史料,如《日本外史》,日军参谋本部所编的《日本战史》等一
系列记载中,日本将领们有如天神下凡,似乎谈笑风生之间,就运筹帷幄,破敌千里。

特别是第二军的军长加藤清正,此人极其残忍,战场对垒不知所谓,未见有何高明,却十分喜爱杀害平民,屠城放
火。史料上说他是威名远播,战绩丰厚,还取了个外号``虎加藤'',如此之精神,可谓无耻。

当然,根据日本人一条路走到黑的性格,这种无耻精神绝不会丢,那两位在南京大屠杀里,拿着武士刀,比赛杀害
手无寸铁平民的小军官,被日本国内称为``百人斩''的英雄,武士道精神的典范,还曾回到日本(战后又被拉回中国
毙了),给小学生宣讲``光辉事迹'',受到热烈欢迎,而无数新的无耻之辈就是这样炼成的。

所谓建威朝鲜,不过是欺负弱小,所谓战功显赫,不过是屠杀百姓。隐藏在这一切背后的,只有四个字----欺软怕
硬。

于是四个月后,当那个强敌出现之时,一切的光环都将卸去,一切的伪装都将暴露,所谓的日本名将们,将了解到
自己的真实水平,以及强大的真正意义。

此时,被追到鸭绿江边的朝鲜国王李昖却没有这个心思,他只知道,再被人追着打,就只能跳江了,于是他连夜派
出使者,向明朝提出了一个要求----渡江内附。

所谓渡江内附,说穿了就是避难,不过李昖同志的这次避难还是比较特殊的,因为但凡避难,总有个期限,过段时
间该回还得回,可这位兄弟似乎压根就没这个打算,面对前来拜见的明朝使者,他十分激动,用一句十分真诚的
话,表达了他的心声:

``与其死于贼手,毋宁死于父母之国!''

这觉悟,还真不是一般的高。

总之一句话:过去,就不回来了。

\section[\thesection]{}

当然,李昖绝不是缺心眼的人,好好的国王不做,要去当难民,实在是因为没办法了。两个月时间,全国八道就丢
了七道,追着屁股后面跑,再跑就只能跳江了,不找明朝大哥,还能咋办?

这也是没有办法的办法。

但事实证明,李昖过于悲观了,此时朝鲜虽然支离破碎,却并没有亡。

之所以没有亡,是因为一个人的存在。

两百年的太平岁月,麻痹了无数人的神经,将领不会打仗,士兵不会拼命,大家一拨接一拨地去搞政治,碰上职业
打手日本人,输得这么惨,这么快,实在很正常。

但就在最紧急的关头,上天帮助了朝鲜,给他们送来了唯一的希望----李舜臣。

李舜臣,字汝谐,德水人,在那场惨烈的战争中,被捧为名将的人非常多(主要是日本那一大帮),但在我看来,其
中名副其实者,不过四人而已。而李舜臣,正是其中的一员。

说起来,李舜臣的成分相当高,他出身于朝鲜王族,是王室宗亲,一般有这个背景,早早就去汉城搞斗争了,然而
李舜臣却是个例外,他武科毕业后,就去了边界,在那里,他遇到了一群十分厉害的对手----女真。

可是在对方的骑兵面前,李舜臣的表现却非常一般,经常打败仗,虽然在长期的战斗中,他积累了丰富的军事经
验,但至少在那时,瞧得起他的人实在不多,所谓``民族英雄''、``军事天才''这样的词语,跟他更是毫不沾边。

但时机终于到来,不久之后,在朝鲜丞相柳成龙的推举下,他升任佥事,并获得了一个新的职务----全罗道水军节
度使。

正是这个职务,改变了他的一生的命运。

在这个世界上,所谓名将,大都有自己的擅长的战法和兵种,攻击、防守、阵法、步兵、骑兵,不一而同,而在全
罗道,李舜臣终于找到了属于自己的天赋----水军。

他对于水军战法,有着常人难以企及的领悟力,研习了许多水军战法,在他的指挥督促下,水军日夜操练,所以虽
说陆军一塌糊涂,朝鲜水军还是摆得上台面的。

当然,这个所谓摆得上台面,那是和陆军比,相对而言,日本海军就威风得多了。

\section[\thesection]{}

日本是海岛国家,历来重视海军,三百年后,在太平洋上和财大气粗的美国还拼了好几年,让对方吃了不少苦头,
其实力确实非同一般。

而在战国时期,日本的海军也十分强悍,因为他们有群众基础----海盗。

靠山吃山,靠水吃水,内陆的兄弟打来打去,可以抢地皮,靠海的就只好当海盗了。朝鲜、东南亚、甚至是日本国
内的船队,只要打这过,就要抢,很有点国际主义精神,战国打了一百多年,他们就抢了一百多年。

这之间只有一个人例外,那就是汪直汪老板,要知道,这位仁兄是卖军火的,敢抢他,那就真是活腻了,二话不
说,拿大炮轰死你没商量。

在这一点上,日本兄弟们有着清晰的认识,因为一直以来,他们都保持着自己的传统性格----欺软怕硬,所以后来
美国打败日本,在日本领土上胡作非为横行霸道,日本人对美国依然是推崇备至,景仰万分。

千言万语,化成一句话:不打不服。

而在这些海盗中,最为强悍者,无疑是日本海军统帅九鬼嘉隆,此人在海上抢了几十年,水战经验十分丰富,后来
归依织田信长,在与战国时期日本最强海军诸侯毛利辉元(即第七军军长)作战时,表现十分出色,为其统一日本立
下了汗马功劳。

此后他被统一收编,成为日本政府海军的一员(还干抢劫老本行,名义不同而已),被称为日本海军第一名将。

而日本海军的装备也相当不错,虽然造大船的技术不如明朝,但在战船上,还是很有几把刷子的,日军战舰高度可
达三四丈,除了装备大量火炮外,在船的外部还装有铁壳,即所谓``铁甲船'',有相当强的防护能力,一般火枪和
弓箭对其毫无作用。

拥有这样的海军实力,日军自然不把对手放在眼里,战争刚一开始,日本海军主力两万余人,七百余艘战船便倾巢
而出,向朝鲜发动总攻。

日军的打算是这样的,总的来说分两步走:首先,由釜山出发,先击破朝鲜主力南海水军。其次,在歼灭朝军后,
转头西上进入黄海,与陆军会合,一举灭亡朝鲜,为进攻中国做好准备。

日本海军统帅除九鬼嘉隆外,还有藤堂高虎,加藤嘉明、胁坂安治三人,此三人皆身经百战,其中加藤嘉明、胁坂
安治是``贱岳七本枪''成员,有着丰富的战争经验。

\section[\thesection]{}

有如此之装备和指挥阵容,丰臣秀吉认为,朝军必一触即溃,数日之间即可荡平。

事情比想象的还要顺利,当日本海军出现之时,朝鲜水军根本未作抵抗,一枪都没放就望风而逃,水军主帅元均更
是带头溜号,所谓的主力部队,就是这么个水平。

战略目标已经实现,日军准备进行下一步,进入黄海,与陆军会合,水陆配合,歼灭朝鲜陆军。

之前的胜利让日军得意忘形,在他们看来,朝鲜水军已经覆灭,到达预定地点只是个时间问题。

然而他们错了,从釜山前往黄海的水路,绝不是一条坦途,因为在这两点之间,有个地名叫做全罗道。

当日军入侵的消息传来后,李舜臣十分愤怒,却也非常兴奋,作为一名军人,他的天职就是战争,而这个时机,他
已等待了很久。

正对着日军进犯的方向,李舜臣率领舰队出发了,他不知道将在哪里遇到他们,他只知道,两军相遇之际,即是他
名扬天下之时!

万历十九年(1591)六月十六日,李舜臣到达了他辉煌人生的起点----玉浦海。

停留在这里的,是日本海军主帅藤堂高虎的上百条战船,当李舜臣突然出现之时,他着实吓了一跳,但转瞬之间,
他就恢复了镇定。

因为这个对手看起来并不起眼。

由于被人排挤,未能成为水军统帅,李舜臣的兵力并不充足,手下战船加起来还不到一百艘,而此次出征,舰队规
模更是微不足道,放眼望去,只有几十艘板屋船(船上建有板屋),看起来很大,实际上也就是个摆设,和日军铁甲
战舰完全不是一个档次。

藤堂高虎笑了,主力尚且如此,何况这几条小鱼小虾?

李舜臣也笑了,他知道,胜利已在掌握之中。

因为在我的手中,有一件必胜的武器。

此后的事情发展将证明,李舜臣最厉害的才能并不是水军,而是工程设计。

乌龟的战斗力

藤堂高虎没有丝毫犹豫,他随即发布号令,几十艘铁甲战舰开始向李舜臣军发动攻击。

由于敌人船只实在太不起眼,日军战舰连炮都不开,直接向对方扑了过去,在他们看来,对付这种破船,用撞就
行,使用炮弹估计会赔本。

但当日舰靠近朝军之时,却意外地发现,那些板屋船突然散开,一种全新的战船就此登上历史舞台。

\section[\thesection]{}

站在舰队前列的日军将领掘内吉善,在第一时间看到了这种前所未见的怪物,当即发出了惊呼:

``龟!龟!''

应该说,这位仁兄还是很有悟性的,虽然他第一次见,却准确地叫出了这种秘密武器的名字。

龟船,又叫乌龟铁舰,该船只整体,从船身到船顶,都有铁甲覆盖,而船头形状极似龟首,故得名龟船。这船用今
天的话说,是封闭式结构,士兵进入船只,就如同进了保险箱,头上罩着铁甲,既能档对方的火枪炮弹,平时还能
挡雨,可谓是方便实用。虽然这船的长相和乌龟很有几分神似,但事实证明,真用起来,这玩意比乌龟要生猛得
多,那可是真要人命。

在龟船的四周,分布着七十多个火枪口,用来对外发射火枪,从远处打击敌人,而船只的前后,都装有锋利的撞
杆,用来撞击敌船,大致是打不死你,也撞死你。

此外,龟船的船首带有大口径火炮,威力强大。更为难得的是,李舜臣虚心地向章鱼们学习,还创造性地发明了烟
雾弹,追击敌船之时,龟首可以发射炮弹,如果形势不妙,龟首口中即释放浓烟,掩护部队撤退。

就这么个玩意,远轰近撞,打不赢还能跑,说它是超级乌龟,那是一点也不夸张。

不过事实上,这种全封闭式的战舰也是有弱点的:由于外部无人警戒,如果被人接近跳上船(学名:跳帮),砸砸敲
敲再放把火,那是相当麻烦。

当然,这个弱点只是理论上的,为防止有人跳帮,李舜臣十分体贴地在船身周围设置了无数铁钩、铁钉,确保敢于
跳船者在第一时间被彻底扎透,扎穿。

总而言之,这种乌龟能轰大炮,能放火枪,能撞,浑身上下带刺,见势不妙还能吐烟逃跑,除了不能咬人外,基本
上算是全能型乌龟。

后来的舰船学家们一致认定,在当时,龟船是世界上最为强大的战舰之一。

藤堂高虎当然不知道这个结论,他只知道自己人多船大,占据优势,在短暂观察之后,他下达了全军突击令。

然而仅仅半个时辰(一个小时)后,他就下达了第二道命令----弃船令。

因为战局只能用一个词来形容----惨不忍睹。

\section[\thesection]{}

就在藤堂高虎下令攻击的同一时刻,李舜臣也发布了攻击令,二十艘龟船同时发出怒吼,当即击沉五艘敌舰。

日将掘内吉善大惊失色,但毕竟人浑胆子大,他随即命令日军战舰继续前进攻击,逼退敌舰。

可是更让他想不到的事情发生了,这群乌龟船不但不退,反而越靠越近,日军这才发现情况不对,慌忙用火枪射击
龟船,却全无效果。

于是接下来的事情就顺利成章了,日舰不是被打沉,就是被撞穿,水军纷纷跳海逃生,个别亡命之徒想要跳帮,基
本上都成了人串,一些运气不好的还挂在了龟船上,被活活地拖回了朝鲜军港,结结实实地搞了次冲浪运动。

眼看即将完蛋,藤堂高虎船也不要了,直接靠岸逃跑,玉浦海战以朝军胜利结束。

在此次海战中,日军二十六条战舰被击沉,死伤上千人,朝军除一人轻伤外,毫无损失。

日本海军终于吃了败仗,九鬼嘉隆十分吃惊,但事实证明,这只是他一系列噩梦的开始。

六月十七日,在玉浦海战后的第二天,李舜臣率领船队来到赤珍浦,在这里,他遇到了加藤嘉明的附属舰队,共计
十三艘。

可刚开打,连李舜臣也吃了一惊,因为这帮日军很有觉悟,没等他开炮就纷纷逃窜,主动弃船登陆,狼狈撤退,其
所乘舰船均被击沉。

在沉没的日舰和狼狈逃窜的日军面前,李舜臣再也没有任何疑虑,他终于明白:他属于这个时代,在这里,他将所
向无敌。

李舜臣继续进发,向着日军出没的所有水域,敌人在哪里出现,就将在哪里被消灭!

七月八日,李舜臣到达泗水港,发现敌船十二艘,发起攻击,敌军全灭。

七月十日,李舜臣到达唐浦,发现敌船二十一艘,发起攻击,敌军全灭,舰队指挥官,九州大名龟井真钜被击毙。

七月十二日,李舜臣遭遇日军主将加藤嘉明主力舰队,双方开战,三十三艘日军战舰被击沉,加藤军主力覆灭。

七月十五日,李舜臣到达釜山水域,发现日军舰队,击沉四艘,俘获三艘后,扬长而去。

打完这次海战后,李舜臣就拍屁股走人了,在他看来,之前的五次海战中,就数七月十五日的这一次,规模最小,
战果最少,所以连战场都没打扫、战利品也没捡就溜了,事实上,他错了。

\section[\thesection]{}

李舜臣并不知道,当他打着呵欠催促返航的时候,一个人正站在岸上,绝望地看着他的背影,拔出腰刀,切腹自尽。

这个人的名字叫做来岛通久,如果说九鬼嘉隆是日本国内第一海军名将的话,他大概就是第二。

这位仁兄之前也是海盗,在中国(日本地名)地区盘踞多年,向来无人敢惹,连织田信长、毛利元就等超级诸侯都要
让他三分,然而在李舜臣的面前,他彻底崩溃了,除了他的舰队,还有他的尊严。

其实来岛兄还是太脆弱了,事实证明,被李舜臣打得自卑到自尽的人,绝不只他一个。

来岛通久的死,以及一连串的失败,终于让日本海军明白,这个叫李舜臣的人,是他们无法逾越的障碍。

日本人是很有组织性的,遇到问题不能解决,就逐级上报,一层报一层,最后报到丰臣秀吉那里,丰臣老板一看,
顿时大怒:一个人带着几十条船,就把你们打得到处跑,八嘎!

但是八嘎不能解决问题,于是他亲自制定了一个战略,命令集中所有舰队,寻找李舜臣水军,进行主力决战,具体
战略部署为:

胁板安治统帅第一队,共七十艘战舰,作为先锋。

加藤嘉明统帅第二队,共三十艘战舰,负责接应。

九鬼嘉隆统帅第三队,共四十艘战舰,负责策应。

以上三队以品字型布阵,向全罗道出击,限期一月,务必要将李舜臣主力彻底歼灭!

九鬼嘉隆(他掌握日本海军实际指挥权)接受了这个任务,并立即安排舰队出发,一百四十艘战舰浩浩荡荡地向着全
罗道开去,现在,他们的首要任务是找到李舜臣。

九鬼嘉隆认为,自己目前的战力,李舜臣是绝对无法抵挡的,他最担心的,是李舜臣闻风而逃,打游击战,那就很
头疼了。

事实证明,他的担心是多余的,十分多余。

联合舰队日夜兼程,抱着绝不打游击的觉悟,向全罗道赶去,然而就在半路上,他们的觉悟提前实现了,因为李舜
臣,就在他们的面前。

在得到日军总攻击的消息后,李舜臣十分兴奋,他已经厌倦了小打小闹,于是连夜带领海军主力,于八月三日到达
庆尚道闲山岛,找到了那些想找他的人。

虽然李舜臣实在有点过于积极,虽然日军的指挥官们个个目瞪口呆,但既然人都到了,咱们就开打吧。

\section[\thesection]{}

具体过程就不提了,我也没办法,实在是不值一提,在短短四个小时之内,战斗就已结束,日军舰队几乎全军覆
没,共有五十九艘战舰被击沉,九鬼嘉隆、加藤嘉明、胁板安治三员大将带头逃跑,两名日军将领由于受不了刺
激,切腹自杀,上千日军淹死。史称``闲山大捷''。

总而言之,在日本国内战史被吹得神乎其神的海军,以及所谓海军名将们,就是这么个表现,真是怎一个惨字了得。

在李舜臣的阻击下,日军水陆并进的企图被打破,海上攻击暂时处于停顿状态,李舜臣以他的天赋,完成了这一壮
举。

但毕竟只有一个李舜臣,朝鲜人民也不能都搬去海上住,所以该丢的地方还是丢了,该跑的人还是跑了。朝鲜亡国
在即,李舜臣回天无术。

日本国内史料对这段``光辉历史''一向是大书特书,特别对诸位武将的包装炒作,那是相当到位,在《日本战国史》
中,就有这样一句极为优美的话:

耀眼无比的日本名将之星照亮了朝鲜的夜空,如同白昼。

而相关的战国游戏,战国电影等等,对战国名将们的宣传更是不遗余力,入朝作战的这几位日军军长,也被吹得神
武无比。

我也曾被忽悠了很久,直到有一天,我放下游戏和电影,翻开日本和朝鲜的古史料,才终于证实了一句话的正确
性:实践是检验真理的唯一标准。

在战争初期,由于朝鲜的政府军实在太差,日本的诸位名将们可谓一打一个准,出尽了风头,但很快,他们就发
现,事情并非如此简单。

最先有此觉悟的,是小早川隆景,这位日本国内的著名智将率领第六军进军全州,此地已无朝军主力,此来正是所
谓``扫清残敌''。

结果出人意料,``残敌''竟然主动出现了----光州节度使权朴。

这位仁兄名不见经传,且是名副其实的``残敌''----部队被打散了,光州的节度使,带着两千残兵,跑到了全州打
起了游击。

著名智将对无名小卒,精锐对游击队,当面锣对面鼓躲都没法躲,无可奈何,那就打吧。

结果是这样的,经过几个钟头的战斗,日军大败,被阵斩五百余人,小早川隆景带头逃窜,权节度使也并未追
击----手中兵力太少。史称``梨峙大捷''。

这是打``残敌'',还没完,下面这位更惨,而他遇到的,是民兵。

这位更惨的仁兄,名叫福岛正则。

\section[\thesection]{}

万历二十年(1592)八月二十日,福岛正则率领大军向新宁方面进军,途中遇到权应铢带领的义军(老百姓自发组织的
武装),双方展开大战。

在鏖战中,由于福岛正则指挥不利(日方自承),优势日军竟被民兵击退,丢弃大量武器粮食,全军撤退。

由于福岛正则的失败,民兵们乘胜追击,一举收复永川、义城、安东等地,``名将''福岛正则连连败退,固守庆州。

和小早川叔叔比起来,毛利辉元侄子也不走运,他也输给了民兵。

万历二十年(1592)八月十四日,毛利辉元、安国寺惠琼率第七军,向全州进发,由于官兵都已逃走,民兵首领黄璞
率军与敌作战,激战一天,日军死伤惨重,被迫退走。

下一个倒霉的是黑田长政。

万历二十年(1592)九月六日,忠清道义军首领赵宪,率领民兵攻击黑田长政第三军,经过激战,黑田长政输了。

不但输了,而且他输得比上几位更彻底,不但被民兵打败,连老巢清州城(朝鲜地名)都丢了,连夜逃走。

这还没完,一个月后(十月三日),他又率三千余人进攻延安府(朝鲜地名),守城的只有不足千人的民兵(政府军早没
影了),经过三天的战斗,日军攻城不下,反而被城内突袭,大败而退。

总而言之,日军将领的水平呈现反比例,实践证明,吹得越厉害,打得越差。搞笑的是,那位而今在日本国内评价
一般的第一军军长小西行长,在战争中却表现得很不错,之所以没人捧,主要是因为他后来在日本关原之战中被人
打败,下场也惨,被泼了无数污水,成了反面典型。

所以说,鬼子的宣传要真信了,那是要过错年的。

很明显,丰臣秀吉不玩游戏,也不看电视,他很清醒,于是在初期的胜利与失败的乱象之中,他选定了那个最合适
的指挥官----小西行长,并将大部分作战指挥权交给了他。

而在此之后长达数年的战争中,这个名字成了史料中的明星人物,出镜率十分之高,其他的诸多所谓名将,都成了
跑龙套的,偶尔才出来转转。要知道,日本人并不傻,这一切的一切,都是有理由的。

打了一辈子仗的丰臣秀吉,是一个杰出的军事家,在以往的几十年里,他的眼光几乎从未错过,这次似乎也不例
外,种种迹象表明,他做出了一个极其正确的抉择----相对而言。

\section[\thesection]{}

时间,只需要时间

从战绩上看,小西行长是一个相当不错的指挥官,作为先锋,他击溃了朝鲜军队,并巩固了战果,虽然其他同行的
表现不如人意,李舜臣也过于强悍,但在他的掌控下,朝鲜大部已牢牢地控制在日军的手中。

很快,各地的叛乱将被平息,我们将向下一个目标挺进。

日本正在准备,朝鲜正在沦亡,明朝正在争论。

自打日军六月入侵以来,明朝的朝廷一刻也没消停过,每天都大吵大闹,从早到晚,连个中场休息都没有,兵部那
帮粗人十分想打,部长石星尤其激动,甚至主动请愿,表示不用别人,自己带兵收拾日本人。

但他刚提出来,就被骂了回去,特别是兵科给事中许弘纲,话说得极其难听,他认为,把敌人挡在门口就行了,不
用出门去挡(御倭当于门庭),此外他还批评了朝鲜同志,说他们是被人打就求援,抓几个俘虏就要封赏,自己打仗
却是望风而逃,土崩瓦解(望风逃窜,弃国于人),去救他们是白费劲。

朝廷大多数人都同意他的看法。

恰好此时,朝鲜国王又提出渡江避难,按说过来就过来吧,可是辽东巡抚又上了个奏疏,说我这里地方有限,资源
有限,只能接收一部分人,其余的切莫过江,本地无法接待。

末了还附上可接收难民名额--``名数莫过百人''

这下朝鲜国王也不干了,我好歹是个国王,只让带一百人过来,买菜做饭的都不够啊!

难民问题暂不考虑,到底出不出兵,几番讨论下来,朝中大臣几乎达成了共识----不去。

事情到此,眼看朝鲜就要亡国,一个人发话了:

``应该早日出兵救援!(宜速救援)''

听到这句话,所有人都沉默了,经过商讨,明朝确定了最后方针----出兵。

因为说这句话的人是万历。

很多人都知道万历皇帝很懒,知道他长期不上朝,知道他打破了消极怠工的最长时间纪录(之前这一纪录由嘉靖同志
保持),但有一点很多人并不知道:

他虽不上朝,却并非不管事。

因为一个不会管事,不会控制群臣的人,是绝不可能做四十八年皇帝的,四十八天都不行。

事实证明,由始至终,他都在沉默地注视着这个帝国的一举一动。而现在,是说话的时候了。

\section[\thesection]{}

应该说,这次万历皇帝做出了一个正确的判断:日本的野心绝不仅于朝鲜,一旦吞并成功,增强实力养精蓄锐,必
定变本加厉,到时更不好收拾。

打比不打好,早打比晚打好,在国外打比在国内打好,所谓``无贻他日疆患'',实在是万历同志的真知灼见。

万历二十年(1592)七月,明朝向朝鲜派出了第一支军队。

受命出击的人,是辽东副总兵祖承训。

祖承训,辽东宁远人,原先是李成梁的家丁,随同李成梁四处征战,有着丰富的军事经验,勇猛善战,是一个看上
去很合适的出征人选。

看上去很合适,实际上不合适,这倒不是他本人有何问题,只是因为在鸭绿江的那边,有十五万日军,而祖将军,
只带去了三千人。

更滑稽的是,他并非不知道这一点,在部队刚到朝鲜时,朝鲜重臣柳成龙出来迎接,顺便数了数队伍,觉得不对
劲,又不好明讲,便对祖承训说道:

``倭兵战斗力甚强,希望将军谨慎对敌。''

祖承训的回答简单明了:

``当年,我曾以三千骑兵攻破十万蒙古军,小小倭兵,有何可怕!''

首先我们有理由相信,祖承训先生吹了牛,因为虽然李成梁很猛,似乎也还没干过如此壮举,打下手的祖承训就更
不用说了。

其次,祖承训实在是自信得有点过了头,别说十五万名全副武装的日军,就算十五万个白痴,站在原地不动让他
砍,只怕也得十天半个月。

但就此言败似乎为时过早,祖承训所带的,是长期在边界作战的明军,战斗力较强,就算和日本人死磕,也还是有
一拼。

然而,事情似乎进展得比想象中更顺利,这一路上,祖承训压根就没碰上几个敌人,他更为自信,快马加鞭,日夜
兼程,向目标赶去。

平壤城,已在眼前。

看来日军确实吓破了胆,不但城墙上无人守卫,连城门都敞开着,里面只有几个零散日军,机不可失,祖承训随即
发动冲锋,三千人就此冲入了城内。

祖承训率军进入朝鲜那天,小西行长便得到了消息,对于这个不请自来的客人,他有着充分的心理准备,当加藤清
正等人表示要固守城池,出外迎敌之时,他却表示了反对。

因为他知道,还有一个更好的方法。

\section[\thesection]{}

要以最小的代价,获取最大的胜利,即使在占据优势的情况下,也不例外,事实证明,丰臣秀吉没有看错人。

当祖承训全军进城后,随着一声炮响,原先安静的街道突然喧哗起来,日军从隐藏地纷纷现身,并占据有利地形,
用火枪射击明军。

几轮齐射之后,明军损失惨重,祖承训也被打蒙了,他原以为,日军都是些没开化的粗人,谁知道人家不但懂兵
法,还会打埋伏。

慌乱之下,他率领残兵逃了出去,但损失已经极其惨重,死伤两千余人,几近全军覆没,副将史儒战死。

明军的第一次进攻就这样结束了。

当这个消息传到朝鲜国王那里时,李昖基本肯定,自己离跳江不远了。而丰臣秀吉更是欣喜若狂,他终于确定,明
军的实力正如他所了解的那样,根本不堪一击。

万历得知这个消息后,却并未激动,他只是沉默片刻,便叫来了兵部侍郎宋应昌,告诉他,正式开战的时候到了。

好吧,既然如此,我们就认真开始吧,很快,你们将为此付出沉重的代价。

宋应昌,字思文,嘉靖四十四(1565)年进士,时任兵部右侍郎。

和部长石星比起来,副部长宋应昌并不起眼,因为石部长不但个子高(长八尺),长得好(相貌过人),而且经常大发
感慨,抒发情怀。而宋应昌每天不是跑来跑去,就是研究地图兵书,一天说不了几句话,这么一个人,想引人注意
也难。

然而万历却接连两次拒绝了石星的请战,将入朝作战的任务交给了宋应昌,因为他是个明白人,能不能吹和能不能
打,那是两码事。

此后事情的发展证明,这是一个极为英明的选择。

宋应昌虽然为人沉默寡言,却深通韬略,熟知兵法,他虽然从未主动请战,却是一个坚定的主战派,且做事毫不拖
拉,在受命之后,他片刻不停,即刻开始制定进攻计划,调兵遣将。

然而没过多久,让万历想不到的事情发生了,一向办事极有效率的宋应昌竟然主动表示,虽然朝鲜局势极度危险,
但目前暂时还不能出兵。

万历问:为什么?

宋应昌答:我召集的将领之中,有一人尚在准备,我要等他,此人不到,不可开战。

\section[\thesection]{}

对于宋应昌所说的那个人,万历也十分欣赏,所以他表示同意,并问了第二个问题:需要多久?

宋应昌回答道:至少两个月。

事情就这样定了,派遣明军入朝作战,日期初定为两个月后,即万历二十年年底。

问题在于,明朝这边可以等,朝鲜人你可以告诉他兄弟挺住,可日本人那里怎么办呢?你总不能跟他说,我是要打你
的,无奈还没准备好,麻烦你等我两个月,先别打了,我一切齐备后就来收拾你。

对此,宋应昌也束手无策,他只会打仗,不会外交,于是几番踢足球后,这个光荣而艰巨的使命被交给了兵部尚书
石星。

然而石星也没办法,他是国防部部长,连老本行都不在行,搞外交更是抓瞎,但他是一把手,关键时刻是要背黑锅
的,这事他不干就没人干了,可又不能不干。

在抓耳挠腮、冥思苦想几天后,石大人终于想出了一个主意----招聘。当然,不是贴布告那种搞法,而是派人私下
四处寻访。

在石星看来,我大明人才济济,找个把人谈判混时间,应该还是靠谱的。

从政治学的角度讲,这是个馊主意,如此国家大事,竟然临时上外边找人,实在太不严肃。

但事实证明,馊主意执行起来,倒也未必一定就馊。因为很快,石星就找到了一个合适的人选,这个人的名字,叫
做沈惟敬。

大混混的看家本领

沈惟敬,嘉兴人,关于此人的来历,史料上众说纷纭,但有一点倒是相当一致----市中无赖也。

所谓市中无赖,用今天的话说,就是市井的混混。

对于这个评价,我一直有不同意见,因为在我看来,沈惟敬先生不是混混,而是大混混。

而之后的事情将告诉我们,混混和大混混是有区别的,至少有两个。

大混混沈惟敬受聘后,很快就出发了,他的第一个目的地是义州,任务是安抚朝鲜国王,在这里,他见到了避难的
朝鲜官员。

据朝鲜官员后来的回忆录记载,这位沈惟敬先生刚一露面,就让他们大吃了一惊--天朝怎么派了这么个人来?

因为据史料记载,沈惟敬长得很丑(貌寝),而外交人员代表国体,一般说来长得都还过得去,如此歪瓜劣枣,成什
么体统。

\section[\thesection]{}

但接下来,更让他们吃惊的事情发生了,这位仁兄虽然长得丑,且初见此大场面,却一点也不怯场,面对朝鲜诸位
官员,口若悬河,侃侃而谈,只要他开口,没人能插上话。

于是大家心里有了底,把他引见给了朝鲜国王李昖。

李昖已经穷极无奈了,天天在院子里转圈,听说天朝使者来了,十分高兴,竟然亲自出来迎接,并亲切接见了沈惟
敬。

接下来,他将体会到沈大混混的非凡之处。

一般说来,混混和大混混都有一项绝技----忽悠,但不同之处在于,他们忽悠的档次和内容差别很大,一般混混也
就骗个大婶大妈,糊弄两个买菜钱;大混混忽悠的,往往是王公贵族,高级干部,糊弄的也都是军国大事。

而沈惟敬很符合这个条件,他只用了几句话,就让准备去寻死的李昖恢复信心,容光焕发。

他主要讲了这样几件事:首先,他是代表大明皇帝来的(基本上是没错),其次,他很会用兵,深通兵法(基本上是胡
扯),希望朝鲜国王不要担心,大明的援兵很快就到(确实如此),有七十万人(……)。

在谈话的最后,他还极其神秘地表示,和平是大有希望的,因为他和平秀吉(即丰臣秀吉)的关系很好,是铁哥们(我
真没话说了),双方摊开来谈,没有解决不了的事。

每当我觉得人生过于现实时,经常会翻开这段史料,并感谢沈惟敬先生用他的实际行动,让我真正领略了忽悠与梦
想的最高境界。

综合分析沈兄的背景:嘉兴人,会说日语,还干过进出口贸易(走私)。当过混混,我们大致可以推断出,他可能和
倭寇有过来往,出过国,估计也到过日本,没准也有几个日本朋友。

当然,说他认识丰臣秀吉,那就是胡扯了,人家无论如何,也算是一代豪杰,日本的老大级人物,不是那么容易糊
弄的。

但是李昖信了,不但信了,而且还欣喜若狂,把沈惟敬看作救星,千恩万谢,临走了还送了不少礼品以示纪念。

话说回来,朝鲜也有精明人,大臣柳成龙就是一个,这位仁兄搞了几十年政治斗争,也是个老狐狸,觉得沈惟敬满
嘴跑火车,是个靠不住的人。

但这兄弟偏偏还就是明朝的外交使者,不服都不行,想到自己国家的前途,竟然要靠这个混混去忽悠,包括柳成龙
在内的很多明白人,都对前途充满了悲观。

十几天后,沈惟敬又来了,这次他的任务更加艰巨----和日本人谈判,让他们停止进攻。

\section[\thesection]{}

李昖没在社会上混过,自然好忽悠,可日本人就不同了,能出征朝鲜的,都是在国内摸爬滚打过来的,且手握重
兵,个个都不是省油的灯,所以在柳成龙等人看来,这根本就是不可能完成的任务。

但是事实证明,这是一个不太靠谱的世界,正如那句流行语所言:一切皆有可能。

万历二十年(1592)九月,沈惟敬再次抵达义州,准备完成这个任务。

作为国王指派的联络使者,柳成龙饶有兴趣地想知道,这位混混准备凭什么挡住日本人,忽悠?

事情似乎和柳成龙预想的一样,沈惟敬刚到就提出,要先和日军建立联系,而他已经写好了一封信,准备交给占据
平壤的小西行长,让小西行长停止进攻,开始和谈。

这是个看上去极为荒谬的主意,且不说人家愿不愿和谈,单说你怎么建立联系,谁去送这封信?你自己去?

沈惟敬道:当然,不是我去。

他派了一个家丁,背上他写的那封信,快马奔进了平壤城,所有的人都认为,这注定是肉包子打狗,一去不回,除
了沈惟敬外。

一天之后,结果揭晓,沈惟敬胜。

这位家丁不但平安返回,还带来了小西行长的口信,表示愿意和谈。

然而问题并没有就此解决,因为这位小西行长同时表示,他虽然愿意谈判,却不愿意出门,如要和平,请朝鲜和大
明派人上门面议。

想想也对,现在主动权在人家手里,说让你去你还就得去。

柳成龙这回高兴了,沈惟敬,你就吹吧,这次你怎么办?派谁去?

然而他又一次吃惊了,因为沈惟敬当即表示:

谁都不派,我自己去。

包括柳成龙在内的许多人都愣住了,虽说他们不喜欢这个大忽悠,但有如此胆量,还是值得佩服的。于是大家纷纷
进言,说这样太危险,你最好不要去,就算要去,也得带多几个人,好有个照应。

沈惟敬却哈哈一笑,说我带个随从去就行了,要那么多人干嘛?

大家想想,倒也是,带兵去也白搭,军队打得过人家,咱也不用躲在这儿,不过为了方便,您还是多带几个人上路
吧。

当然,这个所谓方便,真正的意思是如果出了事,多几个人好收尸。

于是,在众人的注视中,沈惟敬带着三个随从,向着平壤城走去。大家又一次达成了两点共识:第一,这人很勇
敢;第二,他回不来了。

\section[\thesection]{}

但沈惟敬却不这么想,作为一个混混,他没有多少爱国情怀。同理,他也不做赔本生意,之所以如此自信,是因为
在他的身上,有着大混混的另一个特性----随机应变,能屈能伸。

而关于这一点,还有个生动的范例。

曾盘踞山东多年的著名军阀张宗昌,就有着同样的特性。这位仁兄俗称``三不知''(不知兵有多少,不知钱有多少,
不知老婆有多少),当年由混混起家,后来混到了土匪张作霖的手下,变成了大混混。

有一次,张作霖派手下第一悍将郭松龄去张宗昌那里整顿军队,这位郭兄不但是张大帅的心腹,而且还到外国喝过
洋墨水,啃过黄油面包,一向瞧不起大混混张宗昌,总想找个机会收拾他,结果一到地方,不知张混混那根筋不
对,应对不利,竟然得罪了郭松龄。

这下就不用客气了,郭大哥虽然是个留学生,骂人的本事倒也没丢,手指着张大混混,张口就来:X你娘!

在很多人的印象中,军阀应该是脾气暴躁,杀人不眨眼,遇此侮辱,自当拍案而起,拔剑四顾。

然而关键时刻,张宗昌却体现出了一个大混混应有的素质,他当即回答道:

你X俺娘,你就是俺爹了!

说完还给郭松龄跪了下来,我记得,他比郭兄至少大一轮。

这就是大混混的本领,他后来在山东杀人如麻,作恶多端,那是伸,而跪郭松龄,认干爹,就是屈。

沈惟敬就是一个大混混,在兵部官员、朝鲜国王的面前,他屈了,而现在,正是他伸的时候。

小西行长之所以同意和谈,自然不是为了和平,他只是想借此机会摸摸底,顺便吓唬明朝使者,显显威风,用气势
压倒对手。

于是他特意派出大批军队,于平壤城外十里列阵,安排了许多全副武装的士兵,手持明晃晃的大刀和火枪,决定给
沈惟敬一个下马威。

柳成龙也算个厚道人,送走沈惟敬后,感觉就这么了事不太地道,但要他陪着一起去,他倒也不干。

于是他带人登上了平壤城附近的一座山,从这里眺望平壤城外的日军,除了平复心中的愧疚外,还能再看沈惟敬最
后一眼(虽然比较远)。

然而在那里,他看到的不是沈惟敬的人头,而是让他终身难忘的一幕。

\section[\thesection]{}

当沈惟敬骑着马,刚踏入日军大营的时候,日军队列突然变动,一拥而上,把沈惟敬围得严严实实,里三层外三层
水泄不通。然而沈惟敬却丝毫不见慌张,镇定自若地下马,在刀剑从中走入小西行长的营帐。

过了很久(日暮),沈惟敬终于又走出了营帐,毫发无伤。而柳成龙还惊奇地发现,那些飞扬跋扈的日军将领,包括
小西行长、加藤清正等人,竟然纷纷走出营帐,给沈惟敬送行,而且还特有礼貌(送之甚恭)。

数年之后,柳成龙在他的回忆录里详细记载了他所看到的这个奇迹,虽然他也不知道,在那一天,沈惟敬到底说了
些什么----或许永远也没人知道。

但有一点是肯定的,那就是沈惟敬确实干了一件很牛的事情,因为仅仅一天之后,日军最高指挥官小西行长就派人
来了----对沈惟敬表示慰问。

来人慰问之余,也带来了小西行长的钦佩:

``阁下在白刃之中颜色不变,如此胆色,日本国内亦未曾见识。''

日本人来拍马屁了,沈惟敬却只是微微一笑,讲了句牛到极点的话:

``你们没听说过唐朝的郭令公吗?当年回纥数万大军进犯,他单人匹马闯入敌阵,丝毫无畏。我怎么会怕你们这些人
(吾何畏尔)!''

郭令公就是郭子仪,曾把安禄山打得落荒而逃,是平定安史之乱的主要功臣,不世出之名将。

相比而言,沈惟敬实在是个小人物,但在我看来,此时的他足以与郭子仪相比,且毫不逊色。

因为他虽是个混混,却同样无所畏惧。

马屁拍到马腿上,望着眼前这位大义凛然的人,日本使者手足无措,正不知该说什么,却听见了沈惟敬的答复:

``多余的话不用再讲,我会将这里的情况回报圣皇(即万历),自然会有处置,但在此之前,你们必须约束自己的属
下。''

怎么约束呢?

``日军不得到平壤城外十里范围之内抢掠,与之相对应,所有朝鲜军队也不会进入平壤城内十里!''

很多人,包括柳成龙在内,都认为沈惟敬疯了。当时的日军,别说平壤城外十里,就算打到义州,也是轻而易举的
事情。让日军遵守你的规定,你当小西行长的脑袋进水了不成?

事实证明,确实有这个可能。

\section[\thesection]{}

日本使者回去后没多久,日军便派出专人,在沈惟敬划定的地域树立了地标,确定分界线。

柳成龙的嘴都合不上了,他想破脑袋也不明白,这到底是怎么一回事?

只有沈惟敬,知道这一切的答案。

一直以来,他不过是个冒险者,他的镇定,他的直言不讳,他的狮子大开口,其实全都建立在一个基础上----大明。
如果没有后面的那只老虎,他这头狐狸根本就没有威风的资本。

而作为一个清醒的指挥官,小西行长很清楚,大明是一台沉睡的战争机器,如果在目前的局势下,贸贸然与明朝开
战,后果不堪设想,必须稳固现有的战果,至于大明……,那是迟早的事。

万历二十年(1592)十一月二十八日,沈惟敬再次来到朝鲜,这一回,小西行长终于亮出了他的议和条件:

``以朝鲜大同江为界,平壤以西全部归还朝鲜。''

为表示自己和谈的诚意,他还补充道:

``平壤城亦交还朝鲜,我军只占据大同江以东足矣。''

最后,他又顺便拍了拍明朝的马屁:

``幸好天朝(指明朝)还没有派兵来,和平已经实现,我们不久之后就回去啦。''

跑到人家的家里,抢了人,放了火,抢了东西,然后从抢来的东西里挑一些不值钱的,还给原先的主人,再告诉
他:其实我要的并不多。

这是一个很不要脸的人,也是一个很不要脸的逻辑。

但沈惟敬似乎并没有这样的觉悟,他本来就是个混事的,又不能拍板,于是他连夜赶回去,通报了日军的和平条件。

照这位沈先生的想法,所谓谈判就是商量着办事,有商有量,和买菜差不多,你说一斤,我要八两,最后九两成交。
虽然日本人的条件过分了点,但只要谈,还是有成功的可能。

但当他见到宋应昌的时候,才知道自己错了。

因为还没等他开口,宋侍郎就说了这样一段话:

``你去告诉那些倭奴,如果全部撤出朝鲜,回到日本,讲和是可以的(不妨),但如果占据朝鲜土地,哪怕是一县、
一村,都绝不能和!''

完了,既不是半斤,也不是八两,原来人家压根就没想过要给钱。

虽然沈惟敬胆子大,敢忽悠,确有过人之处,但事实证明,和真正的政治家比起来,他仍然只是混混级别。

因为他不明白,在这个世界上,有些原则是不能谈判的,比如国家、主权、以及尊严。

\section[\thesection]{}

沈惟敬头大了,但让人惊讶的是,虽然他已知道了明朝的底线,却似乎不打算就此了解,根据多种史料分析,这位
仁兄已把和谈当成了自己的一种事业,并一直为此不懈努力。在不久之后,我们还将看到他的身影。

但在宋应昌看来,目的已经达到,因为他苦苦等待的那个人,已经做好了准备。

军阀

宋应昌等的人,叫做李如松。

李如松,是李成梁的儿子。

以往我介绍历史人物,大致都是从家世说起,爷爷、爹之类的一句带过,然后再说主角儿子,但对于这位李先生,
只能破例了,因为他爹比他还有名。

作为明朝万历年间第一名将(首辅申时行语),李成梁是一个非常出名的人----特别是蒙古人,一听到这名字就打哆
嗦。

李成梁,字汝器,号银城,辽东铁岭卫(即今铁岭)人。这位仁兄是个超级传奇人物,四十岁才混出头,还只是个小
军官,不到十年,就成为了边界第一号人物,风头压过了戚继光,不但当上了总兵,还成了伯爵。

当然,这一切都不是白给的,要知道,人家那是真刀真枪,踩着无数人的尸体(主要是蒙古人的),扎扎实实打出来
的。

据统计,自隆庆元年(1567)到万历十九年(1591),二十多年间,李成梁年年打仗,年年杀人,年年升官,从没消停
过,平均每年都要带上千个人头回来报功。杀得蒙古人魂飞魄散,搞得后来蒙古人出去抢劫,只要看到李成梁的旗
帜,基本上都是掉头就跑。

事实上这位仁兄不但故事多,还是一个影响大明王朝命运的人,关于他的事情,后面再讲。这里要说的,是他的儿
子李如松。

李如松,字子茂,李成梁长子,时任宣府总兵。

说起来,宋应昌是兵部的副部长,明军的第二把手,总兵都是他的下属。但作为高级领导,他却一定要等李如松,
之所以如此丢面子,绝不仅仅因为此人会打仗,实在是迫不得已。

说起来,那真是一肚子苦水。

两百年前,朱元璋用武力统一全国后,为保证今后爆发战争时有兵可用,设置了卫所制度,也就是所谓的常备军,
但他吸取了宋代的教训(吃大锅饭,养兵千日,用不了一时),实行军屯,并划给军队土地,也就是当兵的平时耕地
当农民,战时打仗当炮灰。

\section[\thesection]{}

事实证明,这个方法十分省钱,但时间久了,情况就变了,毕竟打仗的时间少,耕田的时间多,久而久之,当兵的
就真成了农民,有些地方更不像话,仗着天高皇帝远,军官趁机吞并了军屯的土地,当起了军事地主,把手下的兵
当佃农,有的还做起了买卖。

搞成这么个状况,战斗力实在是谈不上了。

这种部队要拉出去,也只能填个沟,挖个洞,而且明朝的军队制度也有问题,部队在地方将领手中,兵权却在兵部
手里,每次有麻烦都要临时找将领,再临时安排部队,这才能开打。

真打起来,就热闹了,说起打仗,很多电视剧上都这么演过:大家来自五湖四海,关键时刻指挥官大喝一声:为了
国家,为了民族,冲啊!然后大家一拥而上,战胜了敌人,取得了辉煌的胜利。

这都是胡扯。

兵不知将,将不知兵,平时谁也不认识谁,饭没吃过酒没喝过,啥感情基础都没有,关键时刻,谁肯为你卖命?你喊
一句就让我去冲锋?你怎么不冲?

总之,卖命是可以的,冲锋也是可行的,但你得给个理由先。

在很长的一段时间内,大明王朝都找不到这个理由,所以明军的战斗力是一天不如一天,仗也越打越差,但随着时
间的推移,一些优秀的将领终于找到了它,其中最为著名的一个人,就是戚继光。

而这个理由,也可以用一句经典电影台词来概括----跟着我,有肉吃。

很多人并不知道,戚继光的所谓``戚家军'',其实并不算明朝政府的军队,而是戚继光的私人武装,因为从征集到
训练,都是他本人负责,从军官到士兵,都是他的铁杆,除了戚继光外,谁也指挥不动这支部队。

而且在戚继光部队当兵的工资高,从不拖欠,也不打白条,因为戚将军和胡宗宪(后来是张居正)关系好,军费给得
足。加上他也会搞钱,时不时还让部队出去做点生意,待遇自然好。

长官靠得住,还能拿着高薪,这种部队,说什么人家也不走,打起仗来更是没话说,一个赛一个地往上冲。后来戚
继光调去北方,当地士兵懒散,戚继光二话不说,把戚家军调了过来,当着所有人的面进行操练。

那一天,天降大雨,整整一天。

戚家军就在雨里站了一天,鸦雀无声,丝毫不动。

在这个世界上,没有无缘无故的爱,也没有无缘无故的忠诚。

\section[\thesection]{}

但要论在这方面的成就,戚继光还只能排第二,因为有个人比他干得更为出色----李成梁。

戚继光的戚家军,有一流的装备,优厚的待遇,是明朝战斗力最强的步兵,但他们并不是唯一的精英,在当时,还
有一支能与之相匹敌的部队--辽东铁骑。

作为李成梁的精锐部队,辽东铁骑可谓是当时最强大的骑兵,作战勇猛,且行动迅速,来去如风,善于奔袭,是李
成梁赖以成名的根本。

拥有如此强大的战斗力,是因为辽东铁骑的士兵们,不但收入丰厚,装备精良,还有着一样连戚家军都没有的东
西----土地。

与戚继光不同,李成梁是一个有政治野心的人,他在辽东土生土长,是地头蛇,也没有``封侯非我意,但愿海波
平''的高尚道德,在与蒙古人作战的过程中,他不断地扩充着自己的实力。

为了让士兵更加忠于自己,他不但大把花钱,还干了一件胆大包天的事情。

在明代,驻军有自己的专用土地,以用于军屯,这些土地都是国家所有,耕种所得也要上缴国家。但随着时间的推
移,很多军屯土地都被个人占有,既当军官打仗,又当地主收租,兼职干得不亦乐乎。

当然,这种行为是违法的,如果被朝廷知道,是要惹麻烦的。

所以一般人也就用地种点东西,捞点小外快,就这样,还遮遮掩掩不敢声张,李成梁却大不相同,极为生猛,不但
大大方方地占地,还把地都给分了!但凡是辽东铁骑的成员,基本上是人手一份。

贪了国家的粮也就罢了,连国家的地,他都敢自己分配,按照大明律令,这和造反也差不太远了,掉脑袋,全家抄
斩,那都是板上钉钉的事。

但事实证明,李成梁不是木板,而是板砖,后台极硬,来头极大,还很会来事,张居正在的时候,他是张居正的嫡
系,张居正下去了,他又成了申时行的亲信,谁也动不了他一根指头。

如果按马克思主义阶级理论分析,李成梁的士兵应该全都算地主,他的部队就是地主集团,那真是平民的没有,良
民的不是。

有这么大的实惠,所以他的部下每逢上阵,都特别能玩命,特别能战斗,跟疯子似地向前跑,冲击力极强。

地盘是自己的,兵也是自己的,想干什么干什么,无法无天,对于这种人,今天我们有个通俗的称呼----军阀。

\section[\thesection]{}

对于这些,朝廷自然是知道的,可也没办法,那地方兵荒马乱,只有李成梁镇得住,把他撤掉或者干掉,谁帮你干
活?

所以自嘉靖以后,朝廷对这类人都非常客气,特别是辽东,虽然万历十九年(1591)李成梁退休了,但他的儿子还在。
要知道,军阀的儿子,那还是军阀。

而作为新一代的军阀武将,李如松更是个难伺候的人物。

在明代,武将是一个很尴尬的角色,建国之初待遇极高,开国六公爵全部都是武将(李善长是因军功受封的)。并形
成了一个惯例:如非武将、无军功,无论多大官,做了多少贡献,都绝对不能受封爵位。

所以张居正虽位极人臣,干到太师,连皇帝都被他捏着玩,却什么爵位都没混上。而王守仁能混到伯爵,只是因为
他平定了宁王叛乱,曾立下军功。

但这只是个特例,事实上,自宣德以来,武将的地位就大不如前了,这倒也不难理解,国家不打仗,丘八们自然也
就无用武之地了。

武将逐渐成为粗人的代名词,加上明代的体制是以文制武,高级武官往往都是文科进士出身,真正拿刀拼命的,往
往为人所鄙视。

被人鄙视久了,就会自己鄙视自己。许多武将为提高社会地位,纷纷努力学习文化,有事没事弄本书夹着走,以显
示自己的``儒将''风度。

但这帮人靠打仗起家,基本都是文盲或半文盲,文言中有一句十分刻薄的话,说这些人是``举笔如扛鼎'',虽说损
人,却也是事实。

所以折腾来折腾去,书没读几本,本事却丢光了,为了显示风度,军事训练、实战演习都没人搞了----怕人家说粗
俗,武将的军事指挥能力开始大幅滑坡,战斗力也远不如前。

比如明代著名文学家冯梦龙(三言的作者)就曾编过这么个段子,说有一位武将,上阵打仗,眼看就要被人击败,突
然间天降神兵,打垮了敌人。此人十分感激,便向天叩头,问神仙的来历和姓名。

神仙回答:我是垛子(注意这个称呼)。

武将再叩首,说我何德何能,竟然能让垛子神来救我。

垛子神却告诉他:你不用谢我,我只是来报恩的。

武将大惊:我何曾有恩于尊神?

垛子神答道:当然有恩,平日我在训练场,你从来没有射中过我一箭(从不曾一箭伤我)。

真是晕死。

\section[\thesection]{}

就是这么个吃力不讨好的工作,职业前景也不光明,干的人自然越来越少。像班超那样投笔从戎的人,基本上算是
绝迹了,具体说来,此后只有两种人干这行。

第一种是当兵的,明代当兵的,无非是为混口饭吃,平时给长官种田,战时为国家打仗,每月领点死工资,不知哪
天被打死。拿破仑说,不想当将军的士兵不是好士兵,明朝的士兵不想当将军,但在如此恶劣的环境下,混个百户、
千户还是要的----至少到时可以大喊一声:兄弟们上!

为了实现从冲锋到叫别人冲锋的转变,许多小兵都十分努力,开始了士兵突击,苦练杀敌保命本领。一般说来,这
种出身的武将都比较厉害,有上进心和战斗力,李成梁本人也是这么混出来的。

第二种就是身不由己了,一般都是世家子弟,打从爷爷辈起,就干这行。一家人吃饭的时候,经常讨论的也是上次
你杀多少,这次我干掉几个之类的话题,家教就是拳头棍棒,传统就是不喜读书,从小锦衣玉食,自然也不想拼
命,啥也干不了,基本属于废品。嘉靖年间的那位遇到蒙古人就签合同送钱的仇钺大将军,就是这类人的光荣代表。

总体看来,第一类人比第二类人要强,但特例也是有的,比如李如松。

用一帆风顺来形容李如松的前半生,那是极其贴切的,由于他爹年年杀人,年年提干,他还没到三十岁时,就被授
予都指挥同知的职务,这是一个从二品的高级官职,实在是有点耸人听闻。想当年,戚继光继承的,也就是个四品
官而已,而且还得熬到老爹退休,才能顺利接班。

李如松自然不同,他不是袭职,而是荫职。简单说来,是他不用把老爹等死或是等退休,直接就能干。

明代的武将升官有两种,一种是自己的职务,另一种是子孙后代的职务(荫职)。因为干武将这行,基本都是家族产
业,所谓人才难得,而且万一哪天你不行了,你的后代又不读书(很有可能),找不到出路,也还能混口饭吃,安置
好后路,你才能死心塌地去给国家卖命。

前面是老子的饭碗,后面是儿子的饭碗,所以更难升,也更难得。比如抗倭名将俞大猷,先辈也还混得不错,留下
的职务也只是百户(世袭),李如松的这个职务虽说不能世袭,也相当不错了。

\section[\thesection]{}

说到底,还是因为他老子李成梁太猛,万历三年的时候,就已经是左都督兼太子太保,朝廷的一品大员,说李如松
是高干子弟,那是一点也不过分。

而这位高干子弟后来的日子更是一帆风顺,并历任神机营副将等职,万历十一年(1583),他被任命为山西总兵。

山西总兵,大致相当于山西省军区司令员,握有重兵,位高权重。而这一年,李如松刚满三十四岁。

这是一个破纪录的任命,要知道,一般人三十多岁混到个千户,就已经算是很快了。所以不久之后,给事中黄道瞻
就向皇帝上书,说李如松年级轻轻,身居高位,而且和他爹都手握兵权,实在不应该。

客观地说,这是一个很有理的弹劾理由,但事实证明,有理比不上有后台。内阁首辅申时行立刻站了出来,保了李
如松,最后此事也不了了之。

李如松的好运似乎没有尽头,万历十五年(1587),他又被任命为宣府总兵,镇守明朝四大要地之一,成为了朝廷的
实权派。

一般说来,像李如松这类的高干子弟,表现不外乎两种,一种是特低调,特谦虚,比普通人还能装孙子;另一种是
特狂妄,特嚣张,好像天地之间都容不下,不幸的是,李如松正好是后一种。

根据各种史料记载,这人从小就狂得没边,很有点武将之风----打人从来不找借口,就没见他瞧得上谁,因为这人
太狂,还曾闹出过一件大事。

他在镇守宣府的时候,有一次外出参加操练,正碰上了巡抚许守谦,见面也不打招呼,二话不说,自发自觉地坐到
了许巡抚的身边。

大家都傻了眼。

因为李如松虽然是总兵,这位许巡抚却也是当地最高地方长官,而按照明朝的规矩,以文制武,文官的身份要高于
武将。李公子却仗势欺人,看巡抚大人不顺眼,非要搞特殊化。

许守谦脸色大变,青一阵白一阵,又不好太发作,他的下属,参政王学书却看不过去了,上前就劝,希望这位李总
兵给点面子,坐到一边去,让巡抚好下台。

李总兵估计是嚣张惯了,坐着不动窝,看着王学书也不说话,那意思是老子就不走,你能把我怎么样?

很巧,王参政也是个直人,于是他发火了。

\section[\thesection]{}

王参政二话不说,卷起袖子上前一步,就准备拉他起来。这下子可是惹了大祸,李如松岂肯吃亏,看着对方上来,
把凳子踢开就准备上去干仗,好歹是被人拉住了。

许巡抚是个老实人,受了侮辱倒也没说啥,御史王之栋却想走胡宗宪的老路,投机一把,便连夜上书,弹劾李如松
骄横无度,应予惩戒。

事实证明,干御史告状这行,除了胆大手黑,还得看后台。

奏疏上去之后,没多久命令就下来了----王之栋无事生非,罚俸一年。

但在这个世界上,大致就没有明代言官不敢干的事情,王之栋倒下来,千千万万个王之栋站起来,大家一拥而上,
纷纷弹劾李如松,说什么的都有,舆论压力甚大。

这么多人,这么多告状信,就不是内阁能保得住的了,但耐人寻味的是,李如松却还是安然无恙,毫发无伤。大家
就奇了怪了,内阁的人都是你家亲戚不成?

后来个把太监透风出来,你们的奏疏,皇帝都是看过的。大家这才恍然大悟,原来最大的后台在这里。

说来也怪,万历对戚继光、谭纶这种名将似乎兴趣不大,却单单喜欢李如松,把他看作帝国的武力支柱,对他十分
欣赏,且刻意提拔,有他老人家做后台,那自然是谁也告不动的了。

简单说来,李如松是一个身居高位,却不知谦逊,且嚣张至极,到哪里都讨人嫌,碰谁得罪谁的狂妄家伙。

但我们也不得不说,这是一个有狂妄资本的家伙。

李如松的实力

万历二十年(1592),宁夏发生叛乱,万历虽然已经修养五年,且一直是多一事不如少一事,然而叛乱逐渐扩大,眼
看不管是不行了,便下令出兵平叛。

戚继光已经死了,李成梁又退了休,指挥官自然是李如松,于是万历命令,任命李如松为提督陕西讨逆军务总兵
官,前去平叛。

这是一个非同小可的任命,所谓提督陕西讨逆军务总兵官,并非是陕西一省的军事长官,事实上,他带领的,是辽
东、宣府、大同、山西各省的援军,也就是说,只要是平叛的部队,统统都归他管,不受地域限制,权力极大,类
似于后来的督师,即所谓的平叛军总司令。

而在以往,这种大军团指挥官都由文官担任,以武将身份就任提督的,李如松是第一个。

\section[\thesection]{}

得到这一殊荣的李如松着实名不虚传,到地方后一分钟也不消停,就跟当地总督干了起来,不服管,合理化建议也
不听,想干什么就干什么,兵部尚书石星看不下去,先去信劝他收敛点,结果李如松连部长的面子也不给,理都不
理,石星气得不行,就告到了皇帝那里。

可是皇帝也没多大反应,下了个命令,让李如松注意影响,提督还是照做,跟没说没两样,石星丢尽了面子,索性
也不管了,只是放话出来:纨绔子弟,看他如何平叛!

然而石星大人明显忽略了一个问题:纨绔子弟,就一定没有能力吗?

纨绔子弟李如松去宁夏了,在那里,他遇到了叛军,还有麻贵。

麻贵,大同人,时任宁夏总兵,和李如松一样,他也是将门出身,但要论职业发展,这两人实在是一个天上,一个
地下。

早在嘉靖年间,这位仁兄就已经拿刀上阵拼命了,打了若干年,若干仗,到了隆庆时期,才混到个参将,然后又是
若干年,若干仗,到万历年间,终于当上了大同副总兵,万历十年(1582)修成正果,当上了宁夏总兵。这一路走
来,可谓是一步一个坑,吃尽了苦,受够了累。

人比人,那真是气死人,看人家李如松随便晃晃,三十四岁就当上了山西总兵,现在更是摇身一变,当了讨逆总司
令,跑来当了自己的上司,麻贵的心里很不服气。

可还没等他老人家发作,李如松就发火了,刚来没几天,就把他叫去骂了一顿,还送了他一个特定评价:无能。

这句话倒不是没有来由的,李如松到来的时候,叛军首领哱拜已收缩防线,退守坚城,麻贵也已将城团团围住,并
日夜不停攻打,但这帮叛军很有点硬气,小打小守,大打大守,明军在城下晃悠了半个多月,却毫无进展。

麻贵打了多年仗,是军队的老油条,且为人高傲,动辄问候人家父母,平时只有他骂人,没有人骂他。

但这次挨了骂,他却不敢出声,因为他清楚眼前这个人的背景,那是万万得罪不起的,而且他确实攻城不利,一口
恶气只能咽肚子里,苦着脸报告李司令员:敌军坚守不出,城池高大,十分坚固,实在很难打,最后还毕恭毕敬地
向新上司请教:我不行,您看怎么办?

\section[\thesection]{}

虽然麻贵识相,但李公子脾气却着实不小,一点不消停,接着往下骂,麻贵一咬牙,就当是狗叫吧,骂死也不出
声,等到李如松不骂了,这才行个礼准备往外走,却听到了李如松的最后一句话:

你马上去准备三万口布袋,装上土,过几天我要用。

攻城要布袋作甚?麻贵不知道为什么,也不敢问为什么,但有一点他是知道的,如果几天后没有这些布袋,他还要挨
第二次骂。

几天之后,李如松站在三万口土袋的面前,满意地点了点头,然后下达了简洁的命令----堆。

麻贵这才恍然大悟。

李如松的方法并不神秘,既然敌城高大,难以攻打,那就找土袋打底,就好比爬墙时找两块砖头垫脚,够得差不离
了就能翻墙,简单,却实在是个好办法。

就这么一路往高堆,眼看差不多了,当兵的就踩在布袋堆上往城头射箭,架云梯,准备登城。

但城内的叛军首领哱拜也不是吃素的,很有两下子,在城头架起火炮投石机,直接轰击布袋堆上的士兵,打退了明
军的进攻。

敌人如此顽强,实在出乎李如松的意料,于是他派出了自己的弟弟李如樟,在深夜发动进攻,李如樟也没给哥哥丢
脸,领导带头爬云梯,无奈叛军十分强悍,掀翻云梯,打退了明军,李如樟同志自由落体摔伤,好在并无大碍。

进攻再次受阻,李如松却毫不气馁,他叫来了游击将军龚子敬,给了他一个光荣的任务----组建敢死队。

所谓敢死队,就是关键时刻敢拼命的,龚子敬思虑再三,感觉一般士兵没有这个觉悟(客观事实),便召集了军中的
苗军,先请吃饭,再给重赏,要他们卖命打仗,攻击城池南关。

要说还是苗兵实在,吃了人家的,感觉过意不去,上级一声令下,个个奋勇当先,拼死登城,城内守军没见过这个
阵势,一时之间有点支持不住。

李如松见状,亲自带领主力部队前来支援,眼看就要一举拿下,可这伙叛军实在太过扎实,惊慌之后立刻判明形
势,并调集全城军队严防死守,硬是把攻城部队给打了回去。

明军攻城失败,麻贵却有些得意:说我不行,你也不怎样嘛。

但让他吃惊的是,李如松却不以为意,非但没有愁眉苦脸,反而开始骑着马围着城池转圈,颇有点郊游的意思。

几天后,他又找到了麻贵,让他召集三千士兵,开始干另一件事----挖沟。

\section[\thesection]{}

具体说来,是从城外的河川挖起,由高至低,往城池的方向推进,这种作业方式,在兵法上有一个专用称呼----水
攻。

李如松经过几天的围城观察,终于发现,叛军城池太过坚固,如果硬攻,损失惨重不说,攻不攻得下来也难说。

但同时他也发现,城池所处的位置很低,而附近正好也有河流,于是……

这回哱拜麻烦了,看着城外不断高涨的水位,以及墙根处不断出现的裂缝管涌,只能天天挖土堵漏,面对茫茫一片
大水,想打都没对手,手足无措。

此时,李如松正坐在城外高处,满意地看着眼前的这一幕,他知道敌人眼前的困境,也知道他们即将采取的行
动----因为这是他们唯一的选择。

三天之后的一个深夜,久闭的城门突然洞开,一群骑兵快速冲出,向远处奔去----那里有叛军的援军。

明军似乎毫无准备,这群人放马狂奔,竟未受阻挡,突围而去。

但自由的快乐是短暂的,高兴了一阵后,他们惊奇地发现,在自己的前方,突然出现了大队明军,而且看起来,这
帮人已经等了很久。

逃出包围已然是筋疲力尽,要再拼一次实在有点强人所难,所以明军刚刚发起进攻,脱逃叛军便土崩瓦解,死的
死,降的降。

由始至终,一切都在李如松的掌握之中。

他水攻城池,就料定敌军必然会出城求援,而城外叛军的方向他也早已探明,在敌军必经之路上设下埋伏,是一件
再简单不过的事情。

但有一件事情仍出乎了他的意料--叛军援军还是来了。

其实来也不奇怪,围城都围了那么久,天天枪打炮轰,保密是谈不上了,但这个时候叛军到来,如果内外夹攻,战
局将会非常麻烦。

麻贵一头乱麻,赶紧去找李如松,李司令员仍旧是一脸平静,只说了一句话:

管他城内城外,敌军若来,就地歼之!

对方援军很快就兵临城下了,且人数众多,有数万之众,城内的叛军欢欣鼓舞,明军即将败退,胜利触手可得!

然而不久之后,他们就亲眼看到了希望的破灭,破灭在李如松的手中。

\section[\thesection]{}

麻贵再次大开眼界,在这次战役中,他看到了另一个李如松。

面对人多势众的敌军,李如松不顾他人的劝阻,亲自上阵,更让麻贵吃惊的是,这位正二品的高级指挥官竟然亲自
挥舞马刀,冲锋在前!

和西方军队不同,中国军队打仗,除了单挑外,指挥官一般不在前列。这是很明智的,中国打仗规模大,人多,死
人也多,兵死了可以再招,将军死了没地方找,也没时间换,反正冲锋也不差你一个,所以一般说来,能不冲就不
冲。

明军也不例外,开国时那一班猛人中,除了常遇春出于个人爱好,喜欢当前锋外,别人基本都呆在中军,后来的朱
棣倒也有这个喜好,很是风光了几回,但自此之后,这一不正常现象基本绝迹,包括戚继光在内。

但李如松不同,他带头冲锋,那是家庭传统,他爹李成梁从小军官干起,白手起家组建辽东铁骑,一向是领导率先
垂范,带头砍人,老子英雄儿好汉,李如松对这项工作也甚感兴趣。

于是在李如松的带领下,明军向叛军发动了猛攻,但对方估计也是急了眼了,死命抵住明军的冲击后,竟然还能发
动反攻。

毕竟李如松这样的人还是少数,大多数明军都是按月拿工资的,被对方一冲,怕死的难免就往回跑。而此时,李如
松又表现出了患难与共的品质----谁也不许跑,但凡逃跑的,都被他的督战队干掉了。他也不甘寂寞,亲手杀了几
个退却的士兵(手斩士卒畏缩者),在凶神恶煞的李如松面前,士兵们终于认定,还是回去打仗的好。

在明军的顽强阻击下,援军败退而去,城内叛军失去了最后的希望。

正所谓屋漏偏逢连夜雨,哱拜又发现,经过多日水泡,城池北关部分城墙已经塌陷,防守极其薄弱。

现在无论是李如松还是哱拜,都已经认定,战争即将结束,只剩下最后的一幕。

在落幕之前,李如松召开了一次军事会议,讨论下一步的进攻计划。

在场的人终于达成了一致意见----进攻北关,因为瞎子也看得见,这里将是最好的突破口。

李如松点了点头,他命令部将萧如薰带兵攻击北关。

但是接下来,他却下了另一道让所有人大吃一惊的命令:

全军集合,于北关攻击开始后,总攻南关!

所有人都认定北关将是主攻地点,所以进攻南关,才是最好的选择。

兵者,诡道也。

\section[\thesection]{}

从那一刻起,麻贵才真正认识了眼前的这个人,这个被称为纨绔子弟的家伙,他知道,此人的能力深不可测,此人
的前途不可限量。

进攻开始了,当所有的叛军都集结在北关,准备玩最后一把命的时候,却听到了背后传来的呐喊声,李如松这次也
豁出去了,亲自登云梯爬墙,坚守了几个月的城池就此被攻陷。

紧跟在李如松身后的,正是麻贵,看着这个小自己一茬的身影,他已经心服口服,甘愿步其后尘,但他不会想到,
五年之后,他真步了李如松的后尘。

看见明军入城,叛军们慌不择路,要说这哱拜不愧是首领,比小兵反应快得多,一转手就干掉了自己的两个下属,
并召集其余叛军,找李如松谈判,大意是说我之所以反叛,是受了这两人的骗,现在看到你入城,已然悔过自新,
希望给我和我家人一条活路。

李如松想了一下,说:好,放下武器,就饶了你。

哱拜松了口气,投降了。

延续几个月的宁夏之乱就此划上句号,由于其规模巨大,影响深远,史称``万历三大征''宁夏之乱。当然,关于哱
拜的结局,还要交代一句。

史料上是这样记载的:尽灭拜(哱拜)族。

这正是李如松的风格。

投降?早干嘛去了?

无需谈判,干掉就好

对李如松而言,万历二十年(1592)实在是个多事的年份。刚刚解决完宁夏这摊子事,就接到了宋应昌的通知,于是
提督陕西就变成了提督辽东,凳子还没坐热,就掉头奔日本人去了。

其实说起来,李如松并不是故意耍大牌,一定要宋部长等,之所以拖了几个月,是因为他也要等。

事实上,所谓辽东铁骑,并非李如松一人指挥,而是分由八人统领,参与宁夏平乱的,只是其中一部分。

而这一次,李如松并没有匆忙出发,在仔细思虑之后,他决定召集所有的人。战争的直觉告诉他,在朝鲜等待着他
的,将是更为强大的敌人。

作为大明最为精锐的骑兵部队,辽东铁骑的人数并不多,加起来不过万人,分别由李成梁旧部、家将、儿子们统
管,除了李如松有三千人外,他的弟弟李如梅、李如桢、李如梧以及心腹家丁祖承训、查大受等都只有一千余人,
所谓浓缩的才是精华,应该就是这个意思。

而除了等这帮嫡系外,他还要等几支杂牌军。

\section[\thesection]{}

奉宋应昌命令,归李如松指挥的,包括全国各地的军队,自万历二十年(1592)八月起,蓟州、保定、山东、浙江、
山西、南直隶各军纷纷受命,向着同一个方向集结。

万历二十年(1592)十一月,各路部队辽东会师,援朝军队组建完成,总兵力四万余人,宋应昌为经略,李如松为提
督。

部队分为三军,中军指挥官为副总兵杨元,左军指挥官为副总兵李如柏,右军指挥官为副总兵张世爵,所到将领各
司其职。

简单说起来,大致是这么个关系,宋应昌是老大,代表朝廷管事,李如松是老二,掌握军队指挥具体战斗,杨元,
李如柏,张世爵是中层干部,其余都是干活的。

细细分析一下,就会发现,这个安排别有奥妙,李如柏是李如松的弟弟,自然是嫡系,杨元原任都督佥事,却是宋
应昌的人,张世爵虽也是李如松的手下,却算不上铁杆。

左中右三军统帅,实际上也是左中右三派,既要给李如松自由让他打仗,又要他听话不闹事,费劲心思搞平衡,宋
部长着实下了一番功夫。

但实际操作起来,宋部长才发现,全然不是那回事。

按明代的说法,李如松是军事主官,宋应昌是朝廷特派员,根据规定,李如松见宋应昌时,必须整装进见,并主动
行礼,但李如松性情不改,偏不干,第一次见宋应昌时故意穿了件便服,还主动坐到宋部长的旁边,全然不把自己
当外人。

宋应昌自然不高兴,但局势比人强,谁让人家会打仗呢,爱怎么着就怎么着吧。

对领导都这个态度,下面的那些将领就更不用说了,呼来喝去那是家常便饭,且对人总是爱理不理,连他爹的老部
下查大受找他聊天,也是有一句没一句,极其傲慢。

但他的傲慢终将收敛----在某个人的面前。

万历二十年(1592)十二月,如以往一样,在军营里骂骂咧咧的李如松,等来了最后一支报到的队伍。

这支部队之所以到得最晚,是因为他们的驻地离辽东最远。但像李如松这种人,没事也闹三分,只有别人等他,敢
让他老人家等的,那就是活得不耐烦了,按照以往惯例,迎接这支迟到队伍领兵官的,必定是李如松如疾风骤雨般
的口水和呵斥。有丰富被骂经验的诸位手下都屏息静气,准备看一场好戏。

\section[\thesection]{}

然而出乎所有人的意料,好戏并没有上演,充满找茬欲望,一脸兴奋的李如松竟然转性了,不但没有发火,还让人
收拾大营,准备迎接,看得属下们目瞪口呆。

这一切的变化,从他听到那位领兵官名字的一刻开始----吴惟忠。

吴惟忠,号云峰,浙江金华义乌人,时任浙江游击将军。

这个名字并不起眼,这份履历也不辉煌,但只要看看他的籍贯,再翻翻他的档案,你就能明白,这个面子,李如松
是不能不给的。

简单说来,二十多年前,李如松尚在四处游荡之际,这位仁兄就在浙江义乌参军打倭寇了,而招他入伍的人,就是
戚继光。

李如松不是不讲礼貌,而是只对他看得起的人讲礼貌,戚继光自然是其中之一,更何况他爹李成梁和戚继光的关系
很好,对这位偶像级的人物,李如松一向是奉若神明。

作为戚继光的部将,吴惟忠有极为丰富的战斗经验,而且他大半辈子都在打日本人,应该算是灭倭专家,对这种专
业型人才,李如松自然要捧。

而更重要的是,吴惟忠还带来了四千名特殊的步兵----戚家军。

虽然戚继光不在了,第一代戚家军要么退了休,要么升了官(比如吴惟忠),但他的练兵方法却作为光荣传统流传下
来,一代传一代,大致类似于今天的``钢刀连''、``英雄团''。

这四千人就是戚继光训练法的产物,时代不同了,练法还一样,摸爬滚打,吃尽了苦受尽了累,练完后就拉出去搞
社会实践----打倭寇。

虽说大规模的倭寇入侵已不存在,但毕竟当时日本太乱,国内工作不好找,所以时不时总有一群穷哥们跑过来抢一
把,而戚家军的练兵对象也就是这批人。

于是在经历了长期理论与实践相结合的锻炼后,作为大明帝国最精锐的军队,打了十几年倭寇的戚家军(二代),将
前往朝鲜,经历一场他们先辈曾苦苦追寻的战争,因为在那里,他们的敌人,正是倭寇的最终来源。

和吴惟忠一起来的,还有另一个人,他的名字叫骆尚志。

(长篇)明朝的那些事儿-历史应该可以写得好看$[$1199$]$

骆尚志,号云谷,浙江绍兴余姚人,时任神机营参将,这人用一个字来形容就是猛,两个字就是很猛。据说他臂力
惊人,能举千斤(这要在今天,就去参加奥运会了),号称``骆千斤''。

虽说夸张了点,但骆尚志确实相当厉害,他不但有力气,且武艺高强,擅长剑术,一个打七八个不成问题,而不久
之后,他将成为决定胜负的关键人物。

除了精兵强将外,这批戚家军的服装也相当有特点,据朝鲜史料记载,他们统一穿着红色外装,且身上携带多种兵
器(鸳鸯阵必备装备),放眼望去十分显眼。这也是个怪事,打仗的时候,显眼实在不是个好事,比如曹操同志,割
须断袍,表现如此低调,这才保了一条命。

但之后的战争过程为我们揭示了其中的深刻原理:低调,是属于弱者的专利,战场上的强者,从来都不需要掩饰。

至此,大明帝国的两大主力已集结完毕,最优秀的将领也已到齐,一切都已齐备,摊牌的时候,到了。

但在出发的前一刻,一个人却突然闯入了李如松的军营,告诉他不用大动干戈,仅凭自己只言片语,就能逼退倭兵。

这个人就是沈惟敬。

虽然宋应昌严辞警告过他,也明确告诉了他谈判的条件,这位大混混却像是混出了感觉,不但不回家,却开始变本
加厉,频繁奔走于日本与朝鲜之间,来回搞外交(也就是忽悠)。

当他听说李如松准备出兵时,便匆忙赶来,担心这位仁兄一开战,会坏了自己的``和平大业'',所以一见到李司令
员,便拿出了当初忽悠朝鲜国王的本领,描述和平的美妙前景,劝说李如松同意日方的条件。在他看来,这是有可
能的。

他唾沫横飞地讲了半天,李如松也不答话,聚精会神地听他讲,等他不言语了,就问他:说完了没有。

沈惟敬答道:说完了。

说完了就好,李如松一拍桌子,大喝一声:

抓起来,拉出去砍了!

沈惟敬懵了,他并不知道,李如松对于所谓和平使者,只有一个态度----拿板砖拍死他。

老子手里有兵,杀掉他们就好,谈判?笑话!

眼看沈大忽悠就要完蛋,一个人站出来说话了。

(长篇)明朝的那些事儿-历史应该可以写得好看$[$1200$]$

这个人的名字叫做李应试,时任参谋,虽说名字叫应试,倒不像是应试教育的产物,眼珠一转,拦住了李如松,对
他说了一句话。

随即,李如松改变了主意,于是吓得魂不附体的沈惟敬保住了自己的性命(暂时),被拖回了军营,软禁了起来。

李应试的那句话大致可概括为八个字:此人可用,将计就计。

具体说来,是借此人假意答应日军的条件,麻痹对方,然后发动突袭。

示之以动,利其静而有主,益动而巽,此云暗渡陈仓

三十六计之敌战计

万历二十年(1592)十二月二十六日,李如松率领大军,跨过鸭绿江。

朝鲜国王李昖站在对岸,亲自迎接援军的到来,被人追砍了几个月,又被忽悠了若干天,来来往往,就没见过实在
的,现在,他终于等来了真正的希望。

但柳成龙却不这么看,这位仁兄还是老习惯,来了就数人数,数完后就皱眉头,私下里找到李如松,问他:你们总
共多少人?

李如松回答:四万有余,五万不足。

柳成龙不以为然了:倭军近二十万,朝军已无战力,天军虽勇,但仅凭这四万余人,恐怕无济于事。

要换在以往,碰到敢这么讲话的,李如松早就抄家伙动手了,但毕竟这是国外,要注意政治影响,于是李大少强压
火气,冷冷地说出了他的回答:

阁下以为少,我却以为太多!

柳成龙一声叹息,在他看来,这又是第二个祖承训。

而接下来发生的事情,更让他认定,李如松是一个盲目自信,毫无经验的统帅。

作为李成梁的家丁,祖承训身经百战,一向是浑人胆大,但自从战败归来,他却一反常态,常常对人说日军厉害,
具体说来是``多以兽皮鸡尾为衣饰,以金银作傀儡,以表人面及马面,极为骇异'',类似的话还有很多,那意思大
致是,日本人外形奇特,行为诡异,很可能不正常,属于妖怪一类,没准还吃人肉。

应该说,这种观感还是可以理解的,战国时期的日本武将们都喜欢穿些稀奇古怪的玩意,比如黑田长政,每次打仗
都戴着一顶锅铲帽(形似锅铲),而福岛正则的帽子,是两只长牛角,类似的奇装异服还有很多,反正是自己设计,
要多新潮有多新潮。

第一次见这幅打扮,吓一跳是很正常的,就如后来志愿军入朝作战,头次见黑人团,竟然被吓得往回跑,那都是一
个道理。

(长篇)明朝的那些事儿-历史应该可以写得好看$[$1201$]$

但没过多久,祖承训这种妖魔化日军的行为就停止了,因为李如松收拾了他。虽然祖承训是他父亲的老部下,虽然
祖承训从小看他长大,虽然祖承训也算是高级军官,但对于李如松而言,这些似乎并不重要。

祖总兵被打了二十军棍,并被严厉警告,如再敢妖言惑众,动摇军心,就要掉脑袋。

这些倒也罢了,问题是李司令不但容不下``妖言'',连人言也不听,祖承训几次建言,说日军士兵勇猛,武器独
特,战法奇异,不可轻敌。李如松却丝毫不理。

看到这幕似曾相识的景象,柳成龙绝望了,他曾私下对大臣尹斗寿说:提督(指李如松)不知敌情,却如此自信轻
敌,此次是必败无疑了。

而拜祖承训的宣传所赐,许多明军将领也对日军畏惧有加(毕竟都没见过),李如松却又狂得冒烟,对日军不屑一
顾,很有点盲目自信的意思,总而言之,大家心里都没谱。

只有一个人,知道所有的真相。

虽然已过去了很久,李如松却仍清楚地记得,二十多年前,在一个又一个深夜,那个落魄的老人站立在他的身边,
耐心地告知他所有的一切:他们从哪里来,来干什么,他们的武器战术,他们的凶狠残忍,以及战胜他们的方法。

然后,他就离开了自己,很多年过去了,那个人的一切却始终牢牢地铭刻在脑海中,他的博学、教诲和那沧桑、期
望的眼神。

今日我所传授于你之一切,务必牢记于心。

是的,我记得所有的一切,二十多年之中,一日也不曾忘却。

这一刻,我已等待了太久。

误会

万历二十一年(1592)正月初四,在无数怀疑的眼光中,李如松带兵抵达了安定馆(明史为肃宁馆),在这里,他见到
了前来拜会的日军使者。

但这些人即不是来宣战,也不是来求和的,他们只有一个比较滑稽的目的----请赏。

李如松的计策成功了,在他的授意下,沈惟敬派人向小西行长报信,说明朝同意和谈条件,此来是封赏日军将领,
希望做好接待工作云云。

要说这日本人有时还是很实在的,听说给赏钱的来了,小西行长十分高兴,忙不迭地派人去找李如松。

一般说来,办这种事,去个把人也就够了,不知是小西行长讲礼貌,还是穷疯了,这次竟然派了二十三个人,组了
个团来拿封赏。

\section[\thesection]{}

顺便说一句,这里的数字,源自我所查到的兵部侍郎宋应昌的奏疏,但据明史记载,是二十个人,而且事后剩余人
数也不同,这也是没办法,明代史难度就在于史料太多,这本书这么说,那本书那么说,基本上就是一笔自相矛盾
的烂账,类似情况多如牛毛。

在本书中,但凡遇到此类头疼问题,一般根据顾颉刚先生的史料辨析原则,故此处采信宋应昌的奏疏。

这二十三人到的时候,李如松正在大营里,他即刻吩咐,把带头的几个人请到大营,他马上就到。

马上的意思,就是很快,当然,也是还要等一会儿。

出事,也就是一会儿的事。

李如松很懂得保密的重要性,所以沈惟敬的情况以及他的打算,只有少数几个人知晓,这中间不包括李宁。

李宁是李如松的部将,性格简单粗暴,天天喊打喊杀,这天正好呆在大营外,先听说来了日本人,又听说李提督要
处理这些人,当即二杆子精神大爆发,带着几个人,这就进了大营。

一进去,李宁二话不说,拔刀就砍,日本人当时就傻了眼,两国交战还讲究个不斩来使,来讨赏的竟然也砍?于是仓
皇之间,四散逃命。

由于李宁是自发行动,又没个全盘计划,一乱起来谁也不知怎么回事,一些日本人就趁机逃掉了,于是乱打乱杀之
后战果如下:生擒一人,杀十五人,七人逃走。

等李如松``马上''赶到的时候,看到的就是这么个一地鸡毛,狼狈不堪的场面,他当即暴跳如雷,因为这个傻大粗
不但未经命令擅自行动,还破坏了他的整体计划。

李提督自然不肯干休,当即命令,把李宁拉出去砍头。

但凡这个时候,总有一帮将领出场,求情的求情,告饶的告饶,总而言之,要把人保下来。

这次也不例外,李如松的弟弟李如柏亲自出马,且表演得十分卖力(哭告免死),碍于众人的面子,李如松没有杀李
宁,重责他十五军棍,让他戴罪立功。

但就在大家如释重负的时候,李如松却叫住了李如柏,平静地对他说了一句话:

今天你替人求情,我饶了他,但如果你敢违抗我的将令,我就杀了你(必枭首)。

李如柏发抖了,他知道,自己的哥哥从不开玩笑。

从那一刻起,无人再敢违抗李如松的命令。

\section[\thesection]{}

教训了李宁,又吓唬了弟弟,但事情依然于事无补,日军使者已经杀掉了,你总不能去找小西行长说,这是误会,
我们本打算出其不意,过两天才撕破脸打你,所以麻烦你再派人来,咱们再谈谈。

只要日本人精神还正常,估计这事是没指望的,所以李如松认定,自己的算盘已经落空。

然而最蹊跷的事情发生了,仅过了一天,小西行长就派来了第二批使者,而他的任务,并不是宣战,也不是复仇,
却是澄清误会。

误会?李如松目瞪口呆。

估计是沈惟敬的忽悠功底太强,小西行长对和谈信心十足,就等着明朝册封了,听说自己派去的人被杀了,先是吃
了一惊,然后就开始琢磨,想来想去,一拍脑袋,明白了:一定是误会。

由于担心上次那批人没文化,礼数不到,所以这次他派来了自己的亲信小西飞,让他务必找到李如松,摸清情况。

事情正如他所想的那样,在短暂的惊讶之后,李如松笑容满面地迎接了他,还请他吃了顿饭,并确认了小西行长的
疑问:没错,就是误会。

既然是误会,小西行长自然也就放心了,误会总是难免的,死了就死了吧,希望大明队伍早日到达平壤,他将热情
迎接。

李如松回复,十分感激,待到平壤再当面致谢。

万历二十一年(1593)正月初六 李如松到达平壤。

日本人办事确实认真,为了迎接大明队伍,在城门口张灯结彩不说,还找了一群人,穿得花枝招展在路旁迎接(花衣
夹道迎),据说事先还彩排过。

而当李如松远远看到这一切的时候,他简直不敢相信自己的眼睛,彩旗飘飘,夹道欢迎,这算是怎么回事?侮辱我?

但在短暂的诧异之后,李如松意识到,这是一个千载难逢的机会,如能一鼓作气冲入城去,攻占平壤,唾手可得!

他随即下达了全军总攻的命令。

然而意想不到的事情发生了。他的部队似乎中了邪,有的往前冲了,大部分却只是观望,几道命令下来,也只是在
原地跺脚,龟缩不前。

之所以出现如此怪象,说到底还是老问题----没见过,千里迢迢跑过来,没看见拿着刀剑的敌人,却看见一群衣着
怪异在路边又唱又跳,混似一群疯子,换了谁都心里没底。再加上祖承训的妖魔化宣传,大多数人都认定了一个原
则----不急,看看再说。

这一看,就耽误了。

\section[\thesection]{}

戚家军打日本人起家,自然不会少见多怪,二话不说撩起袖子就往前赶,可是他们是步兵,行进速度慢,而大多数
骑兵都在看稀奇,无人赶上。

这么一闹腾,傻子也明白是怎么回事了,小西行长如梦初醒,立刻关上城门,派兵严加防守(悉登城拒守),把明军
挡在了城外,虽说丢了个仪仗队,总算是保住了平壤。

李如松彻底发作了,城门大开,拱手相让,居然不要,你们都是瞎子不成?!

但恼怒之后,李如松仔细观察了眼前这座城池,很快,他意识到,这或许不是一次成功的进攻,却并非毫无价
值----只要采取适当的行动。

于是一幕让小西行长摸不着头脑的情景出现了,已经丧失战机的明军不但没有停下来,反而重新发起了攻击,而他
们的目标,是平壤的北城。

平壤的北城防守严密,且有牡丹峰高地,易守难攻,进攻很快被击退,明军并不恋战,撤兵而去。

站在城头的小西行长,看到了战斗的全过程,他十分不解,为何明军毫无胜算,却还要攻击此地。

不过无论如何,这次战斗结束了,自己并没有吃亏,于是在小西行长的脑海中,只剩下了这样一个印象----明军曾
经进攻过北城。

但对李如松而言,这已经足够了。

进攻结束了,但李如松的脾气却没有结束,回营之后,他一如既往地召集了所有将领,开始骂人。

这次骂人的规模极大,除了吴惟忠、骆尚志少数几人外,明军下属几十名将领无一幸免,都被暴跳如雷的李司令训
得狗血淋头。

但事已至此,人家已经关门了,靠忽悠已然不行,骂也骂不开,只有硬打了。

既然要硬打,就得有个攻城方案,怎么打,谁来打,但李司令员却似乎没有这个意识,骂完就走,只说了一句话:

``李如柏,今夜带兵巡夜,不得休息!''

作为李如松的弟弟和属下,李如柏认为,这个命令是对自己的惩罚,也是另一次杀鸡儆猴的把戏。

几个小时之后,他将意识到自己的错误。

\section[\thesection]{}

寅时,平壤紧闭的大西门突然洞开,三千余名日军在夜幕的掩护下,向明军大营扑去。

这是小西行长的安排,在他看来,明军立足未稳,且人生地不熟,摸黑去劫一把,应该万无一失。

据说小西行长平日最喜欢读的书,就是《三国演义》,所以对劫营这招情有独钟,但是很可惜,这一套有时并不管
用,特别是对李如松,因为他也是此书的忠实读者。

这三千多人还没摸进大营,刚到门口,就被巡逻的李如柏发现了,一顿乱打,日军丢下几十具尸体,败退回城。

日军的第一次试探就此结束。

正月初七 晨 大雾

小西行长十分紧张,他很清楚,这种天气有利于掩藏部队和突袭,便严厉部队加强防范,但让他意外的是,整整一
个上午,对面的明军却毫无动静。

想来想去却全无头绪,无奈之下,小西行长决定再玩个花招,去试探明军的虚实。

他派出使者去见李如松,表示愿意出城投降,希望明军先后退三十里。

李如松说:好,明天就这么办。

但双方心里都清楚,这种虚情假意的把戏已经玩不了多久了,真正的好戏即将开场。

正月初七 夜

不知是小西行长看《三国演义》上了瘾,还是一根筋精神作怪,继昨夜后,他再次派出近千名日军趁夜出城,结果
又被巡夜的明军打了个稀里哗啦。

小西行长毫不气馁,今天不行,明天再来,一直打到你走为止!

但他已经没有机会了,因为就在这天夜晚,李如松召开了第一次,也是惟一的一次军事会议。

会议刚开始,李如松便通报了他计划已久的进攻时间----明日(正月初八)。

当然,为何此时宣布作战计划,他也作出了解释:

``倭军所派奸细如金顺良等四十余人,已于近日被全部擒获,我军情报,毫无外泄。''

大家恍然大悟。

如果过早宣布计划,很可能泄露,不利作战,而明天打仗,今天才通报,除了保密外,还有另一层意思:就算有奸
细,现在去通报,也已经来不及了,而且开会的就这么些人,如果到时军情被泄,要查起来,那是一查一个准。

这明摆了就是不信任大家,实在让人有点不爽。

更不爽的还在后头。

``明日攻城,各位务必全力进攻,如有畏缩不前者,立斩不赦!''

末了还有一句:

``不准割取首级!违者严惩!''

\section[\thesection]{}

虽然李如松极不好惹,但当将领们听到这句话时,依然是一片哗然,议论纷纷。

关于这个问题,有必要专门解释一下,在明代,战争之后评定军功的标准,就是人头,这也容易理解,你说你杀了
几个人,那得有凭据,人头就是凭据,不然你一张口,说自己杀了成百上千,上那里去核实?

甚至明军大规模作战,向朝廷报战果的时候,都是用级(首级)来计算的,而且事后兵部还要一一核实,多少人头给
多少赏。

所以在当时,人头那是抢手货,每次打死敌人,许多明军都要争抢人头(那就是钱啊),有时候抢得厉害,冲锋的人
都没了,大家一起抢人头。

李如松很清楚,明天的战斗将十分激烈,人头自然不会少,但攻城之时战机转瞬即逝,要都去抢人头,谁去破城?

可是大家不干了,辛辛苦苦跟你来打仗,除了精忠报国,辛勤打仗外,总还有个按劳取酬吧,不让割人头,取证
据,怎么报销?我报多少你给多少?

事实证明,李司令是讲道理的,干活不给钱这种事还干不出来,歹话讲完,下面说实惠的:

``明日攻城,先登城者,赏银五千两!''

在听到这句话的那一瞬间,大家的眼睛放出了金色的光芒。

五千两白银,大致相当于今天的多少钱呢?这是一个比较复杂的问题,因为在明代近三百年历史中,通货膨胀及物价
上涨是始终存在的,且变化较大,很难确定,只能估算。

而根据我所查到的资料,套用购买力平价理论,可推出这样一个结论:在万历年间,一两白银可以购买两石米左右
(最低),即三百多斤。经查,一斤米的市价,大致在人民币两元左右。

如此推算,万历年间的一两银子大致相当于人民币六百元。五千两,也就是三百万元人民币。

谁说古人小气,人家还真肯下本钱啊。

几乎就在同一时刻,平壤城内的小西行长正进行他的最终军事部署,自明军到来后,他曾仔细观察明军动向,希望
找到对方主攻方向,由于大雾,且明军行动诡异,始终无法如愿,所以城中的布防也是一日三变,未能固定。

时间已经不多了,长期的军事经验告诉他,决战即将到来,而今夜,可能是他的最后一次机会。

于是在一段紧张的忙碌后,小西行长做出了最终的决定。

\section[\thesection]{}

守卫平壤部队,为日军第一军全部、第二军一部,共计一万八千余人,以及朝鲜军(朝奸部队),共计五千余人,合
计两万三千人。

根据种种蛛丝马迹判断,明军的主攻方向是西北方向,此地应放置主力防守,于是小西行长命令:第一军主力一万
两千人,驻守西北方三门:七星门,小西门,大西门,配备大量火枪,务必死守。

而在东面,明军并无大量军队,所以小西行长大胆做出判断:明军不会在东城发动猛攻。

现在只剩下南城和北城了。

短暂犹豫之后,小西行长作出了这样的决定:

``南城广阔,不利用兵,新军(朝鲜军)五千人,驻守南城含毯门。''

``余部主力防守北城!''

我相信,在这一瞬间,他脑海中闪过的,是一天前的那一幕。

``剩余部队为预备队,由我亲自统领!''

至此,小西行长部署完毕。

从明军的动向和驻扎看,东面应无敌军,南面必有佯攻,而主攻方向一定是西北两城,我相信,这个判断是正确的。
只要打退明军总攻,固守待援,胜利必定属于我们!

此时,在城外的明军大营,李如松终于说出了他隐藏已久的进攻计划:

``我军的主攻方向,是西城。''

攻城明军共计四万五千余人,具体部署如下:

``左军指挥杨元,率军一万人,攻击西城小西门。

中军指挥李如柏,率军一万人,攻击西城大西门。右军指挥张世爵,率军一万人,攻击西北七星门。''

``以上三万人,为我军攻击主力。''

第二个被部署的地区,是北城。

``南军(即戚家军)指挥吴惟忠,率军三千人,攻击北城牡丹台!''

平时开会时,李如松说话基本上是独角戏,他说,别人听,然而就在此时,一个人打断了他的话:

``此攻城部署,在下认为不妥。''

打断他的人,叫做查大受。

查大受,铁岭人,李成梁家丁出身,时任副总兵。

作为李成梁的得力部将,查大受身经百战,有丰富的战斗经验,且与李成梁感情深厚,凭着这层关系,他还是敢说
两句话的:

``我军驻扎于西城,已有两日,日军可能已判断出我军主攻方向,如在西城加强防守,我军恐难攻克。''

``此外,南军虽为我军主力,但北城地势太高,仰攻十分不利,难以破城。''

\section[\thesection]{}

要说还是查大受有面子,李如松竟然没吭声,听他把话说完了。

当然,面子也就到此为止,李司令把手一挥,大喝一声:

``这些事不用你理,只管听命!''

接下来是东城和南城:

``东城不必攻击!''

``为什么?''这次提出问题的,是祖承训。

虽然他很怕李如松,但实在是不明白,既然兵力有余,为何不进攻东城呢?

而回答也确实不出所料,言简意赅,简单粗暴:

``你没有读过兵法吗?围师必缺!''

所谓围师必缺,是一种心理战术,具体说来,是指在攻城之时,不可将城池围死,因为如果敌军深陷重围,无处可
跑,眼看没活路,必定会拼死抵抗,如果真把城围死了,城里这两万多玩命的冲出来,能不能挡得住,那实在很难
说。

最后一个,是南城。

``神机营参将骆尚志,率南军精锐两千,辽东副总兵祖承训,率军八千,攻击南城含毯门,由我亲自督战,务求必
克!''

直到这最后的一刻,李如松才摊出了所有的底牌。

在宁夏之战中,李如松亲眼看到了困兽的威力,在优势明军的围困下,城内叛军却顽固到了极点,土包堆不上,水
也淹不死,内无粮草,外援断绝,居然坚持了近半年,明军千方百计、死伤无数,才得以获胜。

在这场惨烈的战役中,李如松领悟了极其重要的两点秘诀:

一、要让对方绝望,必先给他希望,此所谓围师必缺。

二、要攻破城池,最好的攻击点,不是最弱的位置,而是对方想象不到的地方。

于是在两天前,他攻击了北城,并将主力驻扎在西城,放开东城,不理会南城。

西城是大军的集结地,这里必定是主攻的方向。

南城过于广阔,无法确定突破点,不利于攻城,绝不会有人攻击这里。

北城曾被进攻试探,这很可能是攻击的前奏。

所以,我真正的目标,是南城,含毯门。

当所有人终于恍然大悟的时候,李如松已经说出了最后的安排:

``副总兵佟养正,率军九千人,为预备队。''

应该说,这是一个不起眼的人,也是一个不起眼的安排,在之后的战役中也毫无作用。

但十分滑稽的是,这个不起眼的副总兵,却是一个影响了历史的人,所谓主将李如松,和他相比,实在是不值一提。

\section[\thesection]{}

顺便说一句,他的弟弟佟养性也还值得一提,这位仁兄投降后金之后,领兵与明军搞对抗。结果被一个无名小卒带
兵干掉,这个无名小卒因此飞黄腾达,当上了总兵,成为边塞名将,他的名字叫毛文龙。

后来这位毛文龙由于升了官,开始飞扬跋扈,不把上级放在眼里,结果被领导干掉了,这位领导叫袁崇焕。

再后来,袁崇焕又被皇帝杀掉了,罪名之一,就是杀掉了毛文龙。

想一想这笔烂帐,真不知该从何说起。

按常理,预备队宣布之后,就应该散会了,李如松也不说话了,大家陆陆续续离开军营,回去安排明日战备。

祖承训也是这样想的,然而就在他即将踏出大营的那一刻,却听见了李如松的声音:

``祖承训,你等一等,还有一件事情,要你去办。''

平壤 血战

万历二十一年(1593)正月初八,明军整队出营。

李如松一如既往地站在队伍前列,审视着眼前这座坚固的城池,他知道,一场伟大的战役即将开始。

\section[\thesection]{}

李如松,天赋异禀,骁果敢战,深通兵机,万历二十六年(1598)四月,土蛮寇犯辽东。率轻骑远出捣巢,身先士
卒,中伏,力战死。

此时距离他的死亡,还有五年。

李如松的人生并不漫长,但上天是厚待他的,因为他那无比耀眼的才华与天赋,都将在这光辉的一刻绽放。

拂晓,明军开始进攻。

此时,小西行长正在西城督战,如他所料,明军的主攻方向正是这里。面对城下的大批明军,他却并不慌乱。

之所以会如此自信,除了早有准备外,还因为他得到了一个十分可靠的情报。

在开战之前,日本曾试图调查明军的火器装备情况,但由于信息不畅,无法得到第一手资料,之后七弯八绕,才得
知明军也有许多火枪,但杀伤力比日本国内的要小,先进更是谈不上。

而日本国内使用的火枪,虽然都是单发,且装填子弹需要相当时间,射程为一百五十步至二百步,但用来对付武器
落后的明军,实在是太容易了。

此外,在两天前的那次进攻中,明军确实没有大规模使用火器,这也验证了小西行长的想法。

所以,小西行长认定,在拥有大量火枪部队守卫,且墙高沟深的平壤城面前,只会使用弓箭和低档火器的原始明
军,只能望城兴叹。

据《明会典》及《武备志》记载,自隆庆年间始,明军使用之火器,摘录其一如下:

火器名:五雷神机,隆庆初年装用,有枪管五个,各长一尺五寸,重五斤,枪口各有准星,柄上装总照门和铜管,
枪管可旋转,转瞬之间,可轮流发射。

如此看来,这玩意大致相当于今天的左轮手枪,还是连发的。

上面的只是小儿科,根据史料记载,明军装备的火枪种类有二十余种,且多为多管火器,打起来哗哗的,别说装
弹,连瞄准都不用。

鲁迅先生曾经说过:火药发明之后,西方人用来装子弹,中国人用来放鞭炮。

我可以说,至少在明朝,这句话是很不靠谱的。

以小西行长的知识水平,竟能如此自信,也实在是难能可贵。

然而滑稽的是,从某个角度来说,小西行长的判断是正确的,因为根据史料记载,虽然当时明朝的火枪相当先进,
援朝明军却并未大规模使用。

当然,这是有原因的。

很快,小西行长就将彻底了解这个原因。

\section[\thesection]{}

辰时,号炮声响,进攻正式开始。

西城先攻。

站在西城的小西行长严阵以待,等待着明军的突击,然而出乎他意料的是,炮声响过很久,明军却既不跑,也不架
云梯,反而以两人为一组,在原地架设一种两米多长,看似十分奇怪的装置。

正当他百思不得其解之时,却听见了惊天动地的雷声----天雷。

伴随着震耳欲聋的巨响,明军阵地上万炮齐鸣,无数石块、铅子从天而降,砸在西城的城头之上。

日军毫无提防,当即被打死打伤多人,小西行长本人也被击伤,在被扶下去包扎之前,他大声喊出了这种可怕武器
的名字:

``大筒!''

在日语中,火枪被称为铁炮,而被称为大筒的,是大炮。

谜底就此揭晓,明军之所以不用火枪,是因为他们用火炮。

跑了几百里路远道而来,自然要拿出最好的礼物招待客人,藏着掖着,那是不地道的。

不过确切地讲,明军刚刚使用的那玩意,不能称作大炮,按今天的军事分类,应该算是手炮或是火箭筒,它的真实
名字,叫做佛朗机。

嘉靖初年,一次海上遭遇战中,海道副使汪鋐击败了自己的敌人----葡萄牙船队,战后,他来到对方毁弃的战船
上,发现了一批从未见过的火器,经过演示,他发现这玩意威力很大,值得推广,于是他决定,将此物上交中央,
并建议仿照。

这是明代火器发展史上的一个转折点。

由于在明代,从外国来的人,大都被统称为佛郎机人,所以所有从外国进来的火器,无论是走私的,偷来的,还是
抢来的,统统被称为佛郎机。

而汪鋐所缴获的这批佛朗机(即船炮),是当时世界上较为先进的火炮,朝廷十分重视,立刻派人进行研究。

要知道,中国人一向善于研究,但凡世界上弄出个新东西,甭管是不是自己研制的,拿过来研究研究,几天就能造
个差不多的出来,仿制且不说,往往质量比原件还要好。

佛朗机就是如此,从葡萄牙人的船上卸下来,装上弹药射上两发,别说,还真好用,于是乎先用再改,先改再用,
再用再改,再改再用。原本放在船上用的大家伙,体积越改越小,种类越改越多。

\section[\thesection]{}

到嘉靖二十六(1547)年,明代佛朗机成功实现国产化,完全使用国产料件,自主研发,填补了国内空白,并能批量
生产,达到十六世纪国际先进水平。

明朝军事工作者们也用实际行动证明,国产货的品质是有保障的。

比如明军装备的大样佛郎机,全长仅两米,有准星供瞄准,炮身可左右旋转。具有极强大的杀伤力。

两米的大炮,一两个人就能用,按说是差不多了,但中国人的改造精神实在厉害,很快,明朝又研制出了小佛郎机。

小佛郎机,全长仅九十厘米,炮身附有钢环,可供随身携带,打仗的时候一个人就能揣着走,到地方把炮筒往地上
一架,瞄准了就能打,比火箭筒还火箭筒。

这玩意现在还有,实物存放于北京军事博物馆,本人曾去看过,个头确实不大,估计我也能扛着走,有兴趣的也可
以去看看。

除了这些步兵炮外,明朝还发明了骑兵炮----马上佛郎机,这种火炮的尺寸比小佛郎机更小,仅七十厘米长,可随
骑兵在快速移动中发炮,具有很强的威慑力。

总而言之,明代佛郎机极易携带,操作简便,实在是攻城拔寨,杀人砸墙的不二选择,有了这玩意,那真是鬼才用
火枪。于是几万明军就扛着这些要命的家伙来到了平壤城下,并让日军结结实实地过了一把瘾。

但小西行长不愧久经战阵,他很快镇定下来,并带伤上阵,召集被打懵了的日军,告诉他们不必惧怕,因为明军火
炮发射后必须重新装弹,可趁此时机,整顿队伍,加强防守。

根据小西行长的经验,大炮与火枪不同,每次发射后,都需要较长时间重新装弹,才能再次射击,所以他放心大胆
地集结部队,准备防御。

这个说法看上去,是对的,实际上,是错的。

正当日军刚刚回过点神,准备在城头上重新冒头整队的时候,却立刻遭到了第二轮炮击!石块、炮弹从天而降,日军
被打了个正着,损失极其惨重。

日军莫名其妙,可还没等人缓过劲来,第三轮炮击又到了,又被打得稀里哗啦,然后是第四轮,第五轮……

小西行长彻底糊涂了:这一打还不消停了,难不成你们的大炮都是连发的不成?!

没错,明军的大炮确实是连发的。

\section[\thesection]{}

应该说,小西行长的观点是对的,因为明朝时的大炮,所用的并不是后来的火药炮弹,一打炸一片,而是先塞入铁
砂,石块,然后再压入铅子,并装药(火药)点燃发射,其作用类似于现代的钢珠弹(将钢珠塞入炮弹,炸响时钢珠四
射,基本上碰着就完蛋,属于禁用武器),杀伤面极广,不死也要重伤,不重伤也要成麻子。

当然,相对而言,缺点也很明显,要往炮膛里塞那么多杂七杂八的东西,还要点火装药,这么一大套程序,等你准
备好了,人家估计都下班了。

可当年没有现成的炮弹,想快实在力不从心,但历史告诉我们,古人,那还是相当聪明的。

明朝的军事科研工作者们经过研究,想出了一个绝妙的方法----子母铳。

所谓子母铳,其原理大致类似于火箭炮,母铳就是大炮的炮筒,子铳就是炮弹,其口径要小于母铳,在出征前先装
好铁砂、石块、铅子、火药,封好,打包带走。

等到地方要打了,把子铳往母铳里一塞,火药一点,立马就能轰出去,放完了,把子铳拉出来,塞进去第二个,就
能连续发射,装填速度可比今日之榴弹炮。

所以明军的佛郎机,那是不鸣则以,一鸣不停,为保持持续火力,普通佛郎机都带有四个子铳,在几分钟内可以全
部发射出去,足以打得对手抬不起头。

而此次入朝作战,为了适应国际环境,明军还特意装备了新型产品----百出佛郎机,而它的特点也很明显----十个
子铳。

在明军几轮排炮的攻击下,日军损失极大,城头上黑烟密布,四处起火,尸体遍地。

此时明军的大规模炮击已经停止,西面三路大军开始整队,向各自的目标挺进。在这短暂的瞬间,喧嚣的战场如死
一般的宁静。

随着又一声炮响,平静再次被打破,三路明军在杨元、张世爵、李如柏的统领下,分别向小西门、七星门、大西门
发动猛攻。

炮弹可以飞,人就不行了,要想破城,还得老老实实地爬墙,明军士兵们开始架起云梯攻城。而此时的西城城头,
已看不到大群日军,接下来的事情似乎顺理成章:受到沉重打击的日军失去抵抗能力,已四散而逃,只要爬到城
头,就能攻占平壤!

然而,正当明军接近最后胜利之时,城头却忽然杀声震天,日军再次出现,向城下明军发射火枪,掀翻云梯,明军
受到突然打击,死伤多人,进攻被迫停止。

\section[\thesection]{}

在遭到明军连续炮击后,日军虽然伤亡惨重,却并未撤退。

经历了短暂的慌乱,日军逐渐恢复了秩序,在小西行长的统一调配下,他们以极强的纪律性,开始重新布阵。

著名抗日将领李宗仁曾评价说:日军训练之精,和战斗力之强,可说举世罕有其匹。用兵行阵时,俱按战术战斗原
则作战,一丝不乱,作事皆能脚踏实地,一丝不苟。

应该说,这是一个十分客观的评价,因为日本人最大的性格特点就是一根筋,还有点二杆子,认准了就干到底,且
有寻死光荣倾向,像剖腹之类的工作,还是武士专用的,普通人没这资格。说是亡命之徒,那是一点也不夸张。

而在平壤之战中,其二杆子精神更是发挥到了极致,在打退明军进攻后,日军士气大振,向城下倾倒煮沸的大锅热
水,投掷巨石、滚木,并不断用火枪弓箭射击明军。

面对日军的顽强抵抗,在职业道德(爱国情操)和物质奖励(五千两啊)的双重鼓励下,明军依然奋勇争先,爬梯攻城。

但日军的战斗意志十分坚定,明军进攻屡次受挫,个把爬上去的,也很快被日军乱刀砍死,战斗陷入焦灼。

七星门的情况最为严重,日军的顽固程度超出了许多人的想象,眼看这五千两不容易挣,没准还要丢命,一些人开
始调转方向,向后退却,明军阵脚开始随之动摇。右军指挥张世爵眼看形势不妙,急得破口大骂,但在混乱之中,
毫无用处。

就在右军即将败退之际,李如松到了。

战役打响后,李如松即披甲上阵,带领两百骑兵围城巡视,眼看张世爵压不住阵,便赶了过来。

但他没有理会张世爵,而是直接来到了城下,拦住了一个败退的明军,挥起了马刀。

手起刀落,人头也落。

败退的士兵们惊恐地看着这恐怖的一幕,看着这个挥舞着带血马刀的人,听见了他一字一字吐出的话:

``后退者,格杀勿论!''

败退的明军停下了脚步。

在这枪炮轰鸣,混乱不堪的吵闹中,他们无一例外地听见了李如松那音量不大,却极为清晰的声音。那一刻,他的
眼中充满了坚毅,以及激昂:

``杀尽倭奴,只在今日!''

\section[\thesection]{}

在西城激战的同时,北城明军发动了进攻。

北城,是平壤地势最高的地方,日军盘踞于牡丹峰高地,居高临下,并设置了大量火枪弓箭,等待着明军的进攻。

两天前,当吴惟忠第一眼看见北城的时候,他就认定,要想攻克这里,基本上,是不太可能的。

打了几十年的仗,这点军事判断,吴惟忠还是拿得准的。

但一天之后,李如松告诉他,你的任务,是攻击北城,而你的全部兵力,是三千人。

吴惟忠很清楚,这是一个不可能完成的任务,李如松的真正意图,是要他去牵制日军,所谓牺牲小我,成全大我,
往俗了说,就是当炮灰。

然而他回答:听从调遣。

没有丝毫的犹豫。

所以现在他面对的,是人数占优的日军,密密麻麻的枪口和坚固防御,还有必须抬头仰视,才能看见的日军城垒。

吴惟忠回过头,看着手下的士兵,只用一句话,就完成了所有的动员:

``倭寇,就在那里!''

对于这些在浙江土生土长的士兵而言,倭寇两个字,无异于兴奋剂,且不算什么父母被杀,家里被抢的帐,单是从
小耳闻目睹的传统教育,就足以让他们对其恨之入骨。所以打这仗,基本上是不需要动员的。

更何况,他们是戚家军!

四十年前,戚继光在义乌,组建了这支特别的军队,从那时起,他们就和这个光荣的名字紧紧地联系在一起,并在
他的光芒笼罩之下,奋战十余年,驱逐了那些无耻的强盗。

现在,他们在不同的地方,不同的时代,面对着同样的敌人。所以,他们也只需要同样的举动。

于是,在吴惟忠的亲自率领下,三千戚家军向北城牡丹台高地发动了冲锋。

事实证明,吴惟忠的判断是正确的,北城易守难攻,说实诚点,是根本没法攻,地势险要,日军还不断向下发射火
枪,虽说戚家军有丰富的作战经验,比较灵活且善于隐蔽躲闪,伤亡不大,但两次进攻,刚冲到一半,就被打了回
去。

吴惟忠没有放弃,他知道,自己的攻击越猛烈,敌军的的注意力就越集中,越容易被死死拖住,而真正的突破,将
在那时开始。

第三次冲锋开始了,这一次,吴惟忠站在队伍的最前列,挥刀,向着那个不可能攻克的目标冲去。

这是一个太过生猛的举动,很快,一颗子弹便击中了他的胸部(铅子伤胸),顿时血流不止。

\section[\thesection]{}

但吴惟忠没有停下脚步,他依然挥舞着军刀,指挥士兵继续冲锋,因为在他看来,自己的使命尚未达成。

直到攻克平壤,日军逃遁,北城才被攻陷。

但在战后,所有的人都认定,攻击北城的士兵们,已经圆满地完成了任务。

在历史的长河中,吴惟忠是一个极不起眼的名字,在之后的朝鲜史料中,这位将军也很少出场,撤回国内也好,朝
鲜养伤也好,似乎无人关心。这倒也正常,在这场大戏中,和李如松相比,他不过是个跑龙套的。

一位国民党的将军在战败后哀叹:国民党之所以战败,是因为都想吃肉,而共产党的军队之所以战胜,是因为有人
愿意啃骨头。

吴惟忠就是那个啃骨头的人。

所以在历史中,他是个跑龙套的,却是一个伟大的跑龙套的。

当西城和北城打得热火朝天的时候,南城的守军正在打瞌睡。

南城,即平壤的正阳门到含毯门一线,地形平坦宽广,不利于部队隐蔽和突袭,很难找到攻击重点,所以日军放心
大胆地将这里交给了五千名朝鲜军。

说起来,X奸这个词还真并非专利,而某些朝鲜人的觉悟也实在不高,平壤才失陷几个月,就组建出这么大一支朝奸
部队,也算不容易了。

当然,这五千人的战斗力,日军是不做指望的:一个连自己祖国都不保卫的人,还能指望他保卫什么?

不过,让这批朝军欣慰的是,西城北城打得震天响,这里却毫无动静。

但很快,朝军就发现,自己注定是不会寂寞的,一支军队正悄悄地向城池逼近。

朝军十分紧张,但片刻之后,当他们看清对方的衣着时,顿时如释重负,兴高采烈起来。

因为那批不速之客穿着的,是朝鲜军装。

事实证明,带着X奸名头的部队,有着如下共同特点:没战斗力,没胆,还特喜欢藐视同胞。

这帮朝奸部队也是如此,看见朝鲜军队来了,就喜笑颜开,因为他们知道朝军战斗意志十分薄弱,且一打就垮----
当年他们就是如此。

那支朝军攻城部队似乎也如他们所料,不紧不慢,慢悠悠地靠近城池,看那架势,比慢动作还慢动作。

但当这些同胞兄弟抬出云梯,开始登城时,朝奸们才发现,大事不好了。

\section[\thesection]{}

城下朝鲜同胞们的行动突然变得极为迅速,眨眼的功夫,几十个人就已经爬上了不设防的城头。

还没等朝奸们缓过劲来,这帮人又开始换衣服了,这也可以理解,外面套件朝鲜军装,实在有点不太适应。

很快,朝鲜军的惨叫就传遍了城头:``明军,明军攻上来了!''

坦白讲,要说他们算是攻上来的,我还真没看出来。

昨天夜里,当所有人都散去之后,李如松交给祖承训一个任务:给明军士兵换上朝军军服,不得有误。

祖承训自然不敢怠慢,就这样,第二天,城头上的朝军看见了自己的同胞。

攻上南城的,是明军的精锐主力,包括骆尚志统率的戚家军一部和祖承训的辽东铁骑,这帮粗人当然不会客气,上
去就抽刀砍人。朝奸部队也就能欺负欺负老百姓,刚刚交手就被打得落花流水,落荒而逃。

小西行长的机动部队倒是相当有种,看见朝军逃了,马上冲过来补漏,可惜已经来不及了。如狼似虎的明军一拥而
上,彻底攻占了含毯门。

战斗的过程大致如此,和西门、北门比起来,实在不甚精彩,当然伤亡还是有的,只不过有点滑稽:由于进展过于
顺利,又没有人射箭放枪,基本上是个人就能爬上城头,于是一万多人拼了命的往前挤,比冲锋还卖力。

不过这倒也正常,五千两白花花的银子,不费吹灰之力,挤上去就有份,换了谁都得去拼一把。

南城并不是防御的重点,城防本来就不坚固,加上大家又很激动,这一挤,竟然把城墙挤塌一块,恰好骆尚志打这
过,被砸个正着,负伤了。

当然,也有些史料说他是作战负伤,具体情况也搞不清,就这样吧。

无论如何,总算是打上来了,明军的大旗插上了平壤的城头,南城告破。

但这对于西城攻击部队而言,实在没什么太大的意义。

南城之所以很好打,是因为西城很难打,日军在城头顽强抵抗,放枪、扔石头、倒开水,导致明军死伤多人,而明
军也打红了眼,云梯掀翻了再架,摔下来没死的接着爬,爬上去的就举刀和日军死战。

\section[\thesection]{}

虽然南城被破,但平壤并不是个小城市,要从西城绕到南城,也不是一时半会的事儿,而且仗打到这个份上,对明
军而言,哪个门已经不重要了,砍死眼前这帮龟孙再说!

不过日本鬼子实在有两下子,战斗力非常之强悍,也不怕死,面对明军的猛攻毫不畏惧,无人逃跑,占据城头用火
枪射击明军,如明军靠近,则持刀与明军肉搏,宁可战死也不投降。就战斗意志而言,确实不是孬种。

由于日军的顽强抵抗,明军久攻不下,伤亡却越来越大,小西门主将杨元带头攻城,被日军击伤,部将丁景禄阵亡。
大西门主将李如柏更悬,脑袋上挨了日军一枪,好在头盔质量好,躲过了一劫(锦厚未至重伤)。

主将李如松也没逃过去,由于他带着二百骑兵四处晃悠督战,目标太大,结果被日军瞄上,一排枪过去,当场就被
掀翻在地。

在李如松倒地的那一刻,在场的人都傻眼了,主将要是被打死了,这仗还怎么打。

就在大家都不知所措的时候,李如松却突然从地上爬了起来,并再次诠释了彪悍这个词的含义。

虽然摔得灰头土脸,还负了伤----流鼻血(触冒毒火,鼻孔血流),形象十分狼狈,但李司令员毫不在意,拍拍土,
只对手下说了四个字:

``换马再战!''

领导都这么猛,小兵再不拼命就说不过去了,明军士气大振,不要命地往城头冲,但日军着实不含糊,死伤过半也
毫不退缩,拿刀与登城明军对砍,很有点武士道的意思。

战斗就这样进行了下去,虽然明军已经占据优势,但始终无法攻陷城池,进入南城的明军也遭到了日军的顽强阻
击,伤亡人数越来越多,如此拖下去,后果不堪设想。

然而站在七星门外的李如松并不慌张,因为眼前发生的这一切,早在他的预料之中:

``把那玩意儿拉上来!''

这是李如松最后的杀手锏。

所谓那玩意,是一种大炮,而当时的名字,叫做``大将军炮''。

大将军炮,炮身长三尺有余,重几百斤,前有照星,后有照门,装药一斤以上,铅子(炮弹)重三至五斤,射程可达
一里之外。

由于这玩意体积大,又重,没人愿意扛也扛不动,但李如松坚持一定要带。所以出征之时,是由骑兵装上车架拖着
走的。李如松不会想到,他已经无意中创造了一个记录--世界上最早的马拉炮车部队。

但李司令把这些大玩意拉到朝鲜,不破纪录,只为破城。

\section[\thesection]{}

不过话又说回来,这玩意儿虽然威力大,问题也很多,比如说容易误伤自己人,且准头不好,来个误炸那可不好
玩,加上由于技术含量不够,这种炮十分容易炸膛(该问题一直未解决),所以不到万不得已,是不用的。

现在就是万不得已的时候。

明军炮兵支炮、装弹、瞄准,一切就绪。

随着李如松一声令下,大炮发出了震天的轰鸣,没有炸膛,没有误伤,准确命中目标。

七星门被轰开了,平壤,被轰开了。

信用

七星门的失陷彻底打消了日军的士气,纷纷弃城逃窜,杨元和李如柏随即分别攻破了小西门和大西门,三万明军亮
出了屠刀,睁着发红的眼睛,杀进了城内。

一般说来,剧情发展到这儿,接下来就是追击残敌,打扫战场了,可是鬼子就是鬼子,偏偏就不消停。

在城门失守后,小西行长表现出了惊人的心理素质和军事素质,丝毫不乱地集合部队,占据了城内的险要位置,准
备打巷战。

这就有点无聊了,要说保卫自己的领土,激战一把倒还无所谓,赖在人家的地盘上,还这么死活都不走,鬼子们也
真干得出来。

日军盘踞的主要地点,分别是平壤城内的练光亭、风月楼和北城的牡丹台。这三个地方的共同特点是高,基本上算
是平壤城内的制高点,明军若仰攻,不但难于攻下,还会损兵折将,只要等到自己援兵到来,翻盘也说不定。

这就是小西行长的如意算盘。

李如松虽然不用算盘,但心算应该很厉害,到城内一看,就挥挥手,让士兵们不用打了,干一件事就行----找木头。

噼里啪啦找来一大堆,丢在日军据点附近,围成一圈,然后放话,也就一个字:烧!

这下子日军麻烦了,本来拿好了弓箭刀枪准备居高临下,再搞点肉搏,没想到人家根本就不过来,围着放起了火准
备烤活人。于是一时之间,火光冲天,浓烟滚滚,高温烘烤加上烟熏,日军叫苦不迭。

但李如松认为还不够苦,于是他派出五千人,携带大批火枪、火箭、佛郎机,也不主动攻击,只是站在火堆之外,
对准日军据点,把带来的这些东西射出去。

于是一时间火箭火炮满天飞,据点被点燃,烟火大作,要救火没处打水,日军被烧得鬼哭狼嚎,本来是高地,结果
变成了高炉。

\section[\thesection]{}

鉴于刚刚入城,还要营救平壤居民,救治伤员,事务繁杂,李如松司令员安排好围剿部队后,就去忙别的事了。

但值得称道的是,奉命围剿的部队很有责任心,虽然领导不在,还是尽职尽责地放火、射箭、放枪放炮。

整整一夜,他们加班加点,没有休息。

第二天(正月初九),查大受的家丁查应奎起得很早,来到了北城要地牡丹台,昨天,这里还是日军的坚固据点,然
而现在,展现在他眼前的,是一幕真正的人间地狱。

牡丹台以及其瓮城,已被烧得面目全非,昨天还枪声炮声不断的地方,现在已经寂静无声,他走入据点,看见了无
数倒毙的尸体,手脚都缠绕在一起,却没有一具能够辨认,因为他们已经被烧成了黑炭。

查应奎随意数了一下,发现在狭窄的瓮城里,竟有四五百具日军尸首,很明显,他们大多数是被烧死或活活熏死的。

当然,家丁查应奎没有感叹战争残酷的觉悟,他只是兴高采烈地跑了回去,向自己的领导查大受汇报,并就此被记
载下来,成为了那幕场景的见证。

事实上,查应奎看到的只是冰山的一角,在初八的那天夜里,平壤城内火光冲天,明军在外面放火,日军在里面叫
苦不迭,被烧死者不计其数,尸体的烤焦味道传遍全城,史料有云:焦臭冲天,秽闻十里。

干掉残暴的敌人,就必须比他更加残暴,在某种情况下,我认为,这句话是对的。

但日军的耐高温能力还是值得称道的,硬是挺了一夜,没有出来投降。

挺到了第二天,挺不住了。

盘踞在据点的敌人终于崩溃了,被枪打、炮轰不说,还被火烤了一夜,别说武士道,神仙道也不好使了。除小西行
长所部几千人,由于据点坚固,防御严密,尚在苦苦支撑外,城内日军全部逃散。

但逃散也得有个目标,平壤已是明军的天下,往哪里逃呢?

要说日军逃起来也很有悟性,一看,西城、南城、北城都有人守,只有东城,防御十分松懈。

于是日军大喜过望,纷纷向东城逃窜。

事情似乎十分顺利,败军一路往东逃,虽然明军在后紧紧追赶,但在求生的欲望驱使下,日军竟然成功地逃出了东
城的城门。

但很快他们就将发现,其实战死在城内,倒未必是一件坏事。

\section[\thesection]{}

当初李如松布阵之时,取兵法围师必缺之意,空出了东边。但是很多人可能忽略了这样一个问题:为何是东面?

而当日军蜂拥逃出东城城门的时候,我相信他们已经找到了答案。

东城城门外,是一条大河,波浪宽。

谁要选这里当攻击阵地,只怕真是脑袋进了水。

于是日军麻烦了,要绕着城墙跑,只怕是没个头,要回头跑进城,估计明军不让,前无去路,后有追兵,百般无奈
之下,只剩下一个选择----跳河。

我记得,那一天是正月初九,北风那个吹……

朝鲜的天气,大概和东北差不多,一般说来,这个时候是很冷的,估计起码是零下几度,然而日军依然勇敢地跳了
进去。

虽然气温到了零下,但我可以肯定,当时的江面还没有冻住,因为在朝鲜史料中有这样一句话:溺死者约有万余。

先被烤得要死不活,然后又跳进冰水冷冻。古语有云:冰火两重天,想来不禁胆寒。

在这种情况下,如果还不死的,只有超人了,很明显,日军缺乏这种特种人才。

逃出去的基本上都死了,不淹死也得冻死,而呆在城内的小西行长更不好过,他很清楚,自己已经完蛋了,现在他
要考虑的,不是封赏,不是守城,而是怎么活下去。

在生死的最后关头,日军爆发出了惊人的战斗力,在小西行长的指挥下,明军的数次进攻被打退,看那势头,不拼
个鱼死网破、同归于尽决不算完。

虽然明军占据优势,且人多势众,但毕竟打了一天一夜,就算不领加班费,喘口气总还是要的,何况胜局已定,赏
钱还没领,在这节骨眼上被打死,也实在有点亏。

日军虽然人少,却敢于拚命,生死关头,什么都豁得出去。用今天的话说,这叫双方心理状态不同,所谓穷寇莫
追,就是这个道理。

于是,一个奇怪的情景出现了,在经历了一天一夜的激战后,城内再次出现了短暂的平静。

接下来,一件十分神秘的事情发生了。

之所以说神秘,是因为直到今天,这件事情也没全搞清楚。

关于这件事,在史料中,大致有如下四个历史版本。

\section[\thesection]{}

按照明军监军及部将战后给皇帝的总结报告,事情的发展是这样的:日军残部由于抵挡不住明军的攻击,全军主动
撤退,李如松将军神机妙算(料贼计已穷,必遁),设下埋伏,并派兵追击,大败日军。

第二版本是朝鲜大臣柳成龙给国王的报告,说法也差不多,李如松料敌如神,在日军逃遁之后发动攻击,大败日军。

第三版本,是朝鲜国王给大明神宗皇帝的报告(他算是明神宗属臣),这份东西可作为上下级的规范文本,说到自己
的看法,都是``臣窃念'',说到明朝,都是天兵、天朝,大明皇帝英明神武,大明总兵神兵天降,从头拍到尾。

而开战后,明军是``天地为之摆裂,山渊为之反覆'';自己(朝军)是``小邦袖手骇缩,莫敢助力'',;日军则是
``螳臂据辙,无敢抵敌''。照他的意思,日军是碍于明军的神威,一触即溃了。

而讲得最详细,也最实在的,是第四个版本。

根据朝鲜《李朝实录》记载,事情是这样的:

在战斗陷入僵局后,李如松做出了一个出人意料的举动,他派出了使者,去找小西行长谈判。

对于这个决定,很多人并不理解,人都围住了,还要谈什么判?

但事实证明,这是一个很明智的决定,因为此时日军主力已被歼灭,平壤也已攻克,战略目的已经完全达到,目前
最需要的,是争取时间修整,以防敌军反扑,而城里面放着这么一群亡命之徒,硬攻不但耗费精力,伤亡也会很
大,时间一长还可能生变,所以还是谈判最划算。

李如松的谈判条件是这样的:

``以我兵力,足以一举歼灭,然不忍杀人命,姑为退却,放你生路。''

这意思是,我可以灭了你,但无奈心太软(其实是太费力),就放你们走了吧。

小西行长是这样回复的:

``俺等情愿退军,请无拦截后面。''

他的意思是,我认输了,麻烦逃走的时候高抬贵手别黑我。

如此看来,也算是皆大欢喜,双方达成协议,明军撤去包围,日军在万分警戒之下,手持武器逐步退却,撤出了平
壤城。

局势发展到此,看似平淡无奇,但怪也就怪在这里,既然事情圆满解决,为什么在官方报告中,却都没有提到这件
事呢?

这大致有两个原因,其一、跟敌人谈判,把敌人放走,无论出于什么目的,有什么样的结果,似乎都是不大好宣扬
的。

而第二个原因,应该算是人品问题。

\section[\thesection]{}

如果小西行长了解李如松,或者听说过半年前宁夏叛乱的经过,相信即便打死他,也绝不会和李如松谈判。

因为根据李如松的性格,以及宁夏叛军首领哱拜的最终结局,我们大致可以得出这样一个结论:

李如松,至少在这方面,是个不守信用的人。

几乎就在小西行长带领日军退出平壤的同一时刻,李如松叫来了查大受,交给他一个任务:领兵三千,赶赴江东小
路埋伏。

困兽是不好斗的,但只要把它放出来,就好斗了。

于是,当小西行长带队远离平壤,终于放松所有警惕,放心大胆逃命的时候,查大受出现了。

据史料分析,此时日军的兵力,大致在五千人左右,如果敢拼命,查大受手下这三千人应该还不够打,但经过李司
令员这么一忽悠,日军已经满心都是对和平的祈望,斗志全无,一见明军不用人家动手,撒腿就跑。

查大受随即命令追击,大败日军,击毙日军三百余名,但毕竟部队作战时间过长,十分疲劳,日军又跑得贼快(奔
命),明军追赶不及(不及穷追),只能到此为止了。

平壤战役就此结束,明军大胜,日军大败。

此战,明军阵亡七百九十六人,伤一千四百九十二人

而日军的伤亡数字,就有点意思了,据记载,此战中明军斩获日军一千六百四十七人,看起来似乎并不多。

应该说,这是个很准确的数字,但它并不是日军的伤亡人数,而是日军的人头数。

由于战前李如松命令不许抢人头,所以对于这一宝贵资源,明军并没有过于关注,也没有妥善保存,加上后来火攻
水淹,不是烧成黑炭,就是冻成冰,要提取人头,实在有点困难。于是挑来拣去,只捞出一千来个,已经很不容易
了。

至于日军的实际伤亡数,朝鲜和明朝史料都没有明确记载,只有几句``万余''、``千余''之类不靠谱的话,这就是
管杀不管埋导致的恶劣后果。

说到底,还是鬼子们最实在,既然没人帮着数,就自己数。在《日本战史》中,有这样一段记载:

万历二十一年(1593)三月二十日,日军在汉城集结残兵,统计结果摘录如下:第一军小西行长部,原有人数
18700人,现存6520人。

\section[\thesection]{}

虽然入朝的日军数量共计十余万,但很多都是来自于各地的军阀,并不是丰臣秀吉的人,用今天的话说,就是杂牌
军。而他真正信任的人,只有第一军小西行长和第二军加藤清正,也就是所谓的嫡系。

因此这两军,才是丰臣秀吉的精锐和主力部队,其中尤以第一军战斗力为最强,之前攻击朝鲜义军时表现十分出
色,打起来毫不费力。

但在朝鲜之战时,该军几乎被全歼,具体数字大家做个减法就知道了,基本上算是被打残废了。

这还只是第一军的损失人数,第二军共损失八千人,其中相当部分战死于平壤。

以上合计起来,朝鲜之战,日军的损失,至少在两万人以上。

当然,那五千朝鲜军不在统计内,我们有理由相信,他们应该还活着,因为李如松虽然不大守信用,但还不怎么杀
俘虏。

孤军之迷

攻陷平壤后,李如松没有丝毫迟疑,立即派遣军队,继续出击。

由于明军总共不过四五万人,很多部将都担心兵力不足,然而之后的情景却告诉了他们,什么叫做闻风丧胆。

小西行长被击溃之后,各地日军纷纷得到消息,并采取了整齐划一的行动----逃跑。

仅仅三天之内,黄州、平山、中和等地的日军就不战自溃,连明军的影子都没有看到,就跑得一干二净。军事重镇
开城,就此暴露在了明军的面前。

驻扎在开城的,是日军第三军和第六军,指挥官是黑田长政。

而攻击开城的,是李如松的弟弟李如柏,他统帅八千骑兵一路杀过来,声势震天,黑田长政还是很有点骨气的,开
始表示一定要抵抗到底,但随着逃到开城的日军越来越多,明军越吹越神,这位仁兄也坐不住了,还没等真人现
身,正月十八日,在城里放了把火,一溜烟就跑了。

李如柏本想好好打一仗,没想到是这么个结果,积极性受到了打击。便不依不饶,追着黑田长政不放,死赶活赶,
还是赶上了,一通乱打,黑田长政毫无招架之力,带头逃跑。日军后卫被重创,死亡达五六百人,明军仅阵亡六人。

自正月初九至正月二十,仅用十二天,平壤至开城朝鲜二十二府全部收复,日军全线崩溃,退往南方。

但李如松没有满足,因为在他的面前,还有一个最后的目标----王京。

\section[\thesection]{}

王京,就是今天的汉城。日军全线败退后,大部撤到了这里,至正月二十日,聚集于此地的日军已达五万,而且看
起来也不大想走。在这里,李如松即将迎来他人生中的最大考验。

虽然李如松一生打过无数恶仗硬仗,但这一次,他也没有十足的把握。

孙子先生告诉我们:上兵伐谋、其次伐交、其次伐兵、其下攻城。

此外,他还告诉我们:用兵之法,十则围之,五则攻之。

综合起来是这么个意思,打仗的时候,最次的打算,是攻城,而攻击时,如果人数十倍于敌人,就围他,五倍,就
攻他。

城里,有五万日军。

李如松的手上,也就五万人。

在守城战中,防守方是很占优势的,平壤战役中,李如松用四万打两万,耍了无数花招,费劲力气,才最终得以攻
克。

五万人攻五万人,任务是艰巨的,困难是突出的,胜利基本上是不可能的。

当所有人都把目光投向王京之时,一场意外却彻底搅乱了这个困局。

万历二十一年(1593)正月二十六日,李如松发布了一道命令:

总兵查大受、副总兵祖承训、游击李宁,率三千精兵,前往王京探路。

仅仅半天之后,他接到了明军送回的战报:

我军于半路遇敌,大受(查大受)纵兵急击,斩获六百余级。

自平壤之后,日军毫无战力,这种打落水狗的报告,李如松已经习惯了。

如果一个人长期听到同一类型的消息,他就有可能根据这类信息,做出自己的判断。

所以一贯谨慎的李如松,做出了一个决定----亲自前往侦察。

其实就李如松而言,这个行动并不算大胆,平壤激战时,他就敢骑马四处逛,现在自然更不在话下。

但他绝不会想到,一切都将因这个决定而改变。

万历二十一年(1593)正月二十七日,李如松率副将杨元、李如柏、张世爵,统领两千骑兵向王京前进。

部队的行进速度很快,没过多久,便到达了马山馆,这里距离王京,只有九十里。

李如松突然拉住了缰绳。

长期的战场感觉告诉他,前方可能不像自己想得那么简单。

于是他想了一会,下了一道命令:

``我带一千人先行,副将杨元率军一千,随后跟进。''

就是这道命令,挽救了他的性命。

\section[\thesection]{}

分兵之后,李如松继续出发,很快他就到达了另一个地方,这里据王京仅四十里,名叫碧蹄馆。

在这里,他终于看见了遍地的尸体和兵器,很明显,这里就是查大受所描述的战场,而震耳的厮杀声告诉他,这场
战斗还没有结束。于是他毫不犹豫地带兵冲了进去。

冲进去后,才发现事情坏了。

一天前,查大受得意样样地发出了捷报,事实上,他也确实打了胜仗,杀了人家几百口子,还不肯罢休,非要全歼
不可,结果追着追着,追出问题来了。

要知道,这是在王京附近,就算日军再怎么怕事,好歹也是大本营,有好几万人,你带三千多人过来闹事,还想赶
尽杀绝,实在是有点过分了。

于是缓过劲来的日军开始稳住阵脚,发动反击,据史料记载,此时聚集在碧蹄馆的日军来源复杂,除第一军外,还
有第四军、第六军、第八军若干,基本上在附近的,能来的,全都跑来了(悉众而来)。

由于之前日军表现过于疲软,查大受根本没把他们放在眼里,等到他砍过瘾,追够本,才惊奇地发现,自己已经被
包围了。

杀退一批,又来一批,到二十七日晨,外围日军人数已达两万,查大受这才明白大事不好,左冲右冲无法突围,派
人求援也没指望,于是心一横,抱定杀一个够本,杀两个赚一个的精神,带领士兵与日军殊死血战。

就在这时,李如松冲进来了。

这也算是``他乡遇故知''了,查大受却没有丝毫喜悦,因为眼下这种环境,在兵法中基本属于``死地'',而他是李
成梁的家丁,看着李如松长大,感情十分深厚,如果因为自己的疏忽,把李如松的命也搭了进来,别说活着回去,
就算到了阎王那里,也不好意思见李成梁。

日军的反应也相当迅速,很快发现冲进来的这支队伍人数并不多,于是在短暂混乱后,便开始堵塞缺口,重组包围
圈。

看着漫山遍野的日军,李如松明白,自己这次是冲错了地方,一般说来,在目前敌众我寡的情况下,他有两个选择:

其一是趁日军包围圈尚未围拢,突围出去,然后逃走。

其二,与查大受合兵,寻找有利地形防守,等待援军。

包围圈的缺口越来越小,四千人的生死,只在李如松的一念之间。

在片刻犹豫之后,李如松做出了抉择--第三种抉择。

\section[\thesection]{}

李如松手持长刀,面对全军,发出了怒吼:

``全军攻击!如敢畏缩不前者,斩!''

这种选择,叫做死战不退。

有一种人是无所畏惧的,纵使寡不敌众,纵使深陷重围。

当然,李如松之所以无所畏惧,除了胆大外,也还是有资本的。

他的资本,就是身边所带的一千人。

列宁同志说过:宁可少些,但要好些。这句话用在这一千人身上,实在是名副其实,因为这些人都是李如松直属的
辽东铁骑部队。

而辽东铁骑之所以战斗力强,除了敢拼命外,还有一个重要的原因----武器装备。

在日本战国时期,有一个特殊的兵种,曾作为日本战争史上的模范被大力宣传,它的日文汉字名,叫做骑铁。

所谓骑铁,是骑马铁炮的简称,具体说来就是骑兵装备火枪,在马上发射火器,其主要使者者,是日本东北部的诸
侯伊达政宗,由于兼具骑兵的突击性和火枪的攻击力,被誉为日本战国时期最强的兵种。

当然,这支队伍也有着致命的缺陷,由于火枪不能连发,要一边骑马一边装弹,技术含量也实在太高,所以在打完
一枪后,要换兵器才能接着干。

如果按照日本人的标准,那么辽东铁骑应该也算是骑铁兵种,只是他们的武器并非普通的火枪,还有个专业称
呼----三眼神铳。

三眼神铳,全长约120厘米,共有三个枪管,枪头突出,全枪由纯铁打造,射击时可以轮流发射,是辽东铁骑的标准
装备。发起冲锋时,辽东铁骑即冲入战阵,于战马上发动齐射,基本上三轮下来,就能冲垮敌军。

但问题似乎也未完全解决,三枪打完后怎么办呢?

一般说来,换兵器是免不了的了,但中国人的智慧在此得到了完美的验证,这把火铳之所以用纯铁打造,枪管突
出,是因为打完后,吹吹枪口的烟,换个握法,把它竖起来使,那就是把十分标准的铁榔头。

人骑着马冲进去,先放三枪,也不用装弹,放完抡起来就打,这么几路下来,估计神仙也扛不住,铁骑之名就此横
扫天下。

顺便说一句,这种三眼铳今天还有,就在军事博物馆里。每次当我看到那些铁榔头的时候,都会不禁感叹:科学技
术,那真是第一战斗力。

\section[\thesection]{}

有这样的装备,加上这一千多号人都是李如松的亲军,打起仗来十分彪悍,基本上属于亡命之徒。听到李如松的命
令后,二话不说,操起火铳,向日军发动了猛攻。

虽然李如松十分自信,但有一点他并不知道----这绝非遭遇战,而是一个精心设计的圈套。

在平壤战败后,日军对明军产生了极大的心理恐惧,各地纷纷不战而逃,且全无斗志,为防止全军彻底崩溃,挽回
军心,日军大本营经过详细策划,制定了一个周密的诱敌计划。

具体说来,是先派出小股部队,诱使明军大部队追击,并在王京附近的马山馆设下埋伏,待其到来发动总攻,一举
歼灭。

据日本史料记载,参与该计划的日军为第四军和第六军主力,以及其余各军一部,总兵力预计为一万五千人至两万
人,其中诱敌部队一千余人,战场指挥官为小西行长、黑田长政、小早川隆景、立花宗茂等人,反正只要没被打
残,还能动弹的,基本上都来了。

行动如期展开,在探听到查大受率军出发的消息后,诱敌的一千余名日军先行出发,前往马山馆,大军分为两路,
偃旗息鼓,悄悄的过去,打枪的不要。

日军的预期计划是,一千人遭遇明军后,且战且退,将明军引到预定地点,发起总攻。

但事情的发展告诉他们,理论和实际总是有差距的。

由于之前日军逃得太快,查大受一路都没捞到几个人,已经憋了一肚子劲,碰到这股日军后,顿时精神焕发,下了
重手穷追猛打,转瞬间日军灰飞烟灭,一千多人连个水漂都没打,眨眼就没有了。

这回日军指挥官们傻眼了,原本打算且战且退,现在成了有战无退,更为严重的是,查大受明显不过瘾,又跟着追
了过来,越过了马山馆,而此时日军的大部队还在碧蹄馆,尚未到位。

无可奈何之下,日军指挥官们决定,就在碧蹄馆设伏,攻击明军。

于是当查大受赶到之时,他遇到的,是两万余名全副武装,等待已久的日军。

已经退无可退了,横下一条心的日军作战十分勇猛,查大受率军冲击多次,没能冲垮敌军,反而逐渐陷入包围,战
斗进入僵持状态。

事已至此,所谓诱敌深入、全歼明军之类的宏伟壮志,那是谈不上了,能把眼皮底下这三千多人吃掉,已经算是老
天保佑了。

可计划总是赶不上变化,打得正热闹的时候,李如松来了。

\section[\thesection]{}

这下日军喜出望外了,原本想打个埋伏,挽回点面子,结果竟然捞到这么条大鱼,更让他们高兴的是,这位明军最
高指挥官竟然只带了这么点人。小西行长顿时兴奋起来,他立即下令,方圆四十里内的日军,只要还能动弹,立即
赶来会战,不得延误。

与此同时,他还命令,所有日军军官必亲临前线指挥,包括黑田长政、立花宗茂等人在内,总而言之,是豁出去了。

在小西行长的部署下,日军发动了自入朝以来最为猛烈的进攻,并充分发扬其敢死精神,哪里的明军最显眼,最突
出,就往哪里冲。

不巧的是,在战场上,最引人注目的人正是李如松。

这位仁兄实在过于强悍,虽被日军重重包围,却完全不当回事,带着铁骑左冲右突,如入无人之境,这也似乎有点
太欺负人了,于是日军集中兵力,对李如松实行合围。

事后,李如松在给皇帝的报告中,曾用一个词形容过此时自己的环境----围匝数重。

虽然说起来危险万分,但事实上,当时他倒很有几分闲庭信步的风度,据日本史料记载,李如松带领骑兵左右来
回,几进几出,铁骑所到之处,日军无法抵挡,只能保持一段距离跟着他。所谓的包围,其实就是尾随。

然而历史告诉我们,一个人太过嚣张,终究是要翻船的。

正当李如松率军进进出出,旁若无人之时,一位神秘的日军将领出现了。

这位日军将领出场就很不一般,史料上说他是金甲倭将,先不说是真金还是镀金,穿不穿得动,敢扛着这么一副招
风的行头上战场,一般都是有两下子的。

而之所以说这是个神秘的人,是因为他的身份一直未能确定。

参加碧蹄馆之战的主力,是日军第四军,该军以日本九州部队为主,九州是日本最穷困、民风最野蛮的地区,此地
士兵大都作战顽强,凶残成性,是实实在在的亡命之徒。所以很多史料推测,此人很有可能是隶属于第四军的将领。

虽说哪里来的讲不清,但敢拼命是肯定的,这人一上来,就抱定不要命的指导思想,带兵向李如松猛冲(博如松甚
急),突然冒出来这么一号人,李如松毫无准备,身边部队被逐渐冲散,日军逐渐围拢,形势十分危急。

\section[\thesection]{}

此时,李如柏和李宁正在李如松的两翼,发现事情不妙,便指挥部下拼死向李如松靠拢,但日军十分顽强,挡住了
他们的进攻。

紧急关头,还是兄弟靠得住,眼看李如松即将光荣殉职,弟弟李如梅出手了。

虽说在乱军之中,但李如梅依然轻易地瞄准了这位金甲倭将(所以说在战场上穿着不能太时髦),手起一箭,正中此
人面目,当即落马。

主将落马后,士兵们也一哄而散,李如松终于转危为安,但事实上,真正的危机才刚刚开始。

此刻,双方已鏖战多时,虽然明军勇猛,战局却已出现了微妙的变化,此时日军正陆续由四面八方赶来(接续愈添,
沿山遍野),人数优势越来越大,而明军势单力薄,这么打下去,全军覆没,那是迟早的事。

不过明军固然陷入苦战,日军的情况却也差不多,日军主将立花宗茂,性格顽固,在日本国内是出了名的硬骨头,
素以善战闻名,这回也打得撑不住了,竟然主动找到小早川隆景接替自己的位置,退出了战场。

仗打到这个份上,胜败死活,只差一口气。

关键时刻,杨元到了。

杨总兵实在是个守纪律的人,他遵照李如松的命令,延迟出发,到地方一看打得正热闹,二话不说,带着一千人也
冲了进去。

早不来,晚不来,来得刚刚好。日军正打得叫苦不迭,杨元的骑兵突然出现,阵型被完全冲垮,混乱之际也没细看
对方的人数,以为是明军大部队到了,纷纷掉头逃窜。

小西行长见大势已去,也只能率军撤退。李如松惊魂未定,装模作样地追了一阵,也就收兵回去了,毕竟手底下有
多少人,日军不知道,他还是清楚的。

碧蹄馆之战就此结束,此战明军阵亡二百六十四人,斩获日军首级一百六十七人,伤亡大抵相当。

对于这场战役,可以用一句话来概括:

撒网捕鱼,鱼网破了。

应该说,这并不是一场大的战役,但在历史上,此战争议却一直未断,其中最激烈的,是双方的伤亡问题。

在日本的许多战史书籍中(如《日本外史》、《日本战史》),碧蹄馆之战是日军的一场大胜,个别特别敢吹的,说
此战日军歼灭明军两万余人,要这么算,李如松除了全军死光外,还得再找一万五千个垫背的,着实不易。

\section[\thesection]{}

虽然事情不容易办,鬼子还是办了,而且一直在办,后来抗日战争里的台儿庄战役,日军矶谷师团(编制相当于一个
军)被打成了残废,死伤一万多人,几乎丧失战斗力,日本战报却说就损失两千人,脸不红心不跳,由此可见,其不
认账和乱记账,那是有悠久传统的。

说到底,碧蹄馆之战,不过是一场微不足道的小规模战斗而已。

但微不足道,并不代表不重要。事实上,这确实是一场改变了战争进程的战斗。

通过此战,死里逃生的李如松明白了两点:首先,敌人是很难打垮的。

虽然日军被击败,但战斗力尚存,以明军目前的兵力,如要硬攻,很难奏效。

其次,朋友是很难指望的。

在碧蹄馆之役发生前,李如松曾嘱托朝军随后跟进,人家确实也跟着来了,但仗一打起来,不是脚底抹油就是袖手
旁观,仗打完才及时出现,真可谓是反应敏捷。

而更让李如松气愤的,是某些混人。

此时正逢朝鲜阴雨连绵,火器难于使用,日军伏击失败后,全部龟缩于王京,打死不出来,还拼命修筑坚固堡垒,
准备死守。但凡稍微有点军事常识的人都明白,如果现在进攻,那就是寻死。

可柳成龙偏偏装糊涂,他多次上书,并公开表示李如松应尽早进攻王京,不得拖延。

出征之前泼凉水,不出头,现在却又跳出来指手划脚,反正打仗的都是明军,不死白不死,人混账到这个份上,真
能把死人气活了。

李如松没有理会柳成龙,他停下了进攻的脚步。

但停下来并不能解决问题,因为作为朝鲜的都城,王京是必须攻克的。

于是在经过缜密的思索后,李如松做出了如下部署:

总兵杨元率军镇守平壤,控制大同江;李如柏率军镇守宝山,查大受镇守临津,互为声援;李宁、祖承训镇守开城。

这是一个让人莫名其妙的安排,因为明军本就兵力不足,现在竟然分兵四路,要想打下王京,无异于是痴人说梦。

所以几乎所有的人都认为,李如松已经放弃了进攻计划。

事实证明,他们都错了。

因为要攻克一座城池,并不一定要靠武力。

命令下达了,进攻停止了,战场恢复了平静,日军也借此机会加强防守,整肃军队,等待着李如松的下一次进攻。
因为在被忽悠多次后,他们已经确定,眼前的这个对手,是绝对不会消停的。

这个判断十分正确,很快,他们就等到了李如松的问候,但并非攻城的枪炮,而是一把大火。

\section[\thesection]{}

李如松很清楚,凭借自己手中的兵力,是绝对无法攻下王京的,于是他索性分兵各处防守,加固后方,因为他已经
找到了一个更好的进攻目标----龙山。

龙山是日军的粮仓所在地,积粮数十万石,王京、釜山的日军伙食,大都要靠此处供应。

于是,在一个月黑风高的夜晚,李如松密令查大受,率敢死队(死士)连夜跑到龙山,放了一把火,彻底解决了鬼子
们的粮食问题。

这么一来,事情就算是结了,因为武士道再怎么牛,也不能当饭吃,在这一点上,鬼子们的意识是清楚的,认识是
明确的。

万历二十一年(1593)四月十八日,日军全军撤出王京,退往釜山。十九日,李如松入城,王京光复。

自万历二十年(1592)十二月明军入朝起,短短半年时间,日军全线溃败,死失合计三万五千余人,其军队主力,第
一军小西行长部几乎全军覆灭,日军的战斗力遭到致命打击,疲惫交加,斗志全无。

到了这份上,已经打不下去了。

四月下旬,日军继续撤退至蔚山、东莱等沿海地域,回到了一年前的登陆地点,全军八万余人渡海回国,仅留四万
人防守。

至此,抗倭援朝战争第一阶段结束,日军惨败而归。

日军退却了,但李如松并没有痛打落水狗,不是不想,而是不能。

事实上,此时明军的处境也好不了多少,由于朝军几乎是一盘散沙,许多地方都要依靠明军防守,李如松能够调动
的,仅有一万余人,靠这点本钱,想把日军赶下海去,几乎是不可能的。

但最严重的问题还不是缺人,而是缺钱。

要知道,刀枪马炮,天上掉不下来,那都是有价钱的,而所谓打仗,其实就是砸钱,敌人来了,有钱就对砸,没钱
就打游击,朝鲜战争也一样。

明军虽然是帮朝鲜打仗,但从粮食到军饷,都是自给自足,而在这一点上,朝鲜人也体现出了充分的市场意识,非
但不给军费,连明军在当地买军粮都要收现款,拒收信用卡,赊账免谈。

李如松在朝鲜呆了半年,已经花掉了上百万两白银,再这样打个几年,估计裤子都得当出去。

所以谈判,是唯一的选择。

\section[\thesection]{}

高档次的忽悠

第二次谈判就此开始。

所谓谈判,其实就是忽悠的升级版,双方你来我往,吹吹牛吃吃饭,实在的东西实在不多。

客观地讲,明朝在谈判上,一向都没什么诚意。相对而言,日本方面还是比较实诚的,他们曾满怀期望的期盼着明
朝的使者,等到的却是火枪大炮。

说到底,这是个认识问题,因为当时的明朝,管日本叫倭国,管日本人叫倭奴,而且这并非有意歧视,事实上,以
上称呼是一路叫过来的,且从无愧疚、不当之类的情感。

一句话,打心眼里,就从没瞧得上日本人。

第一次谈判,是因为准备不足,未能出兵,等到能够出兵,自然就不谈了。

现在,是第二次谈判。而谈判的最理想人选,是沈惟敬。

半年前,这位仁兄满怀激情地来到李如松的大营,结果差点被砍了头,关起来吃了半年的牢饭,到今天,终于又有
他的用武之地了。

万历二十一年(1593)三月,沈惟敬前往日军大营,开始了第二次谈判,在那里等待着他的,是他的老朋友小西行长。

虽然之前曾被无情地忽悠过一次,但毕竟出来抢一把不容易,死了这么多人,弄不到点实在东西也没法回去,日方
决定继续谈判,平分朝鲜是不指望了,能捞多少是多少。

日军的谈判底线大抵如此,而在他们看来,事到如今,明军多少也会让一两步。

会谈进行得十分顺利,双方互致问候完毕,经过讨价还价,达成了如下意见:

首先,明朝派遣使者,前往日本会见丰臣秀吉。其次,明军撤出朝鲜,日军撤出王京(当时尚未撤出)。最后,日本
交还朝鲜被俘王子官员。

沈惟敬带着谈判意见回来了,出乎他意料的是,这一次,李如松和宋应昌都毫不犹豫地表示同意。

沈惟敬感到了前所未有的喜悦,他认为,一切都将在自己安排下有条不紊地进行下去,建功立业的时候到了。

但他并不知道,所谓谈判和执行,那完全是两码事。

在第一次谈判时,明军只是为了争取时间,压根儿不打算要真谈判,而这一次…,似乎也没这个打算。

因为在战后,宋应昌曾在给皇帝的奏疏中写过这样一段话:

``夫倭酋前后虽有乞贡之称,臣实假贡取事,原无真许之意。''

这句话的大概意思是,日本人是想谈和的,但我是忽悠他们的,您别当真。

\section[\thesection]{}

这就是说,明军从上到下,是万众一心,排除万难,要把忽悠进行到底了。

但协议毕竟还是签了,签了就得执行,而接下来,李如松用行动证明了这样一点:他除了会打仗,搞政治也是把好
手。

根据协议,明军要撤出朝鲜,但李如松纹丝不动,反而烧掉了日军的粮仓,端掉了对方的饭碗。

日军是真没办法了,打不过又闹不起,明知李如松是个不守信用的家伙,偏偏还不敢得罪他,就当吃了个哑巴亏,
硬着头皮派出使者。那意思是,你不撤我认了,但互派使者的事,麻烦你还是给办了吧。

在这一点上,李如松还是很够意思的,他随即派出谢用梓与徐一贯两人,随同沈惟敬一起,前往日军大营。

小西行长十分高兴,因为自从谈判开始以来,他遇到的不是大混混(沈惟敬),就是大忽悠(李如松的使者),感情受
到了严重的伤害,现在对方终于派出了正式的使者,实在是可喜可贺。

但他不知道的是,明朝派来的这两位所谓使者,谢用梓是参将,徐一贯是游击,换句话说,这两人都是武将,别说
搞外交,识不识字那都是不一定的事。

之所以找这么两个丘八去谈判,不是明朝没人了,而是李如松根本就没往上报。这位仁兄接到日军要求后,想也没
想,就在军中随意找了两人,大笔一挥,你们俩就是使者了,去日本出差吧。

现在忽悠你们,那是不得已,老子手里要是有兵,早就打过去了,还谈什么判?!

李如松没当真,但日本人当真了,万历二十一年(1593)五月中旬,小西行长带领沈惟敬、谢用梓以及徐一贯前往日
本,会见丰臣秀吉,进行和谈。

对于明朝使臣的来临,丰臣秀吉非常高兴,不但热情接待,管吃管住,会谈时更是率领各地诸侯权贵到场,亲自参
加,张灯结彩,搞得和过节一样,仪式十分隆重。

当沈惟敬看到这一切的时候,他明白:这下算是忽悠大了。

虽然日本人糊里糊涂,但一路过来,他已经很清楚,身边的这两位使者到底是什么货色。

但事已至此,也只能挺下去了。

\section[\thesection]{}

沈惟敬就此开始了谈判,虽然从名义上讲,谢用梓和徐一贯才是正牌使者,但这两个大老粗连话都说不利索,每次
开会口都不敢开,只能指望沈惟敬忽悠了。

于是每次开会之时,大致都是这么一副场景:丰臣秀吉满怀激情,口若悬河,谢用梓、徐一贯呆若木鸡、一言不
发,沈惟敬随口附和,心不在焉。所谓的外交谈判,其实就是扯淡。

就这么个扯淡会,竟然还开了一个多月,直到六月底,才告结束。

在谈判终结的那一天,丰臣秀吉终于提出了日方的和平条件,该条件也再次证明了这样一点:

丰臣秀吉,是个贪婪无耻、不可救药的人渣。

其具体内容如下:

一、明朝将公主嫁为日本后妃。

二、明朝和日本进行贸易,自由通商。

三、明朝和日本交换誓词,永远通好。

四、割让朝鲜四道,让给日本。

五、朝鲜派出王子大臣各一人,作为人质,由日方管理。

六、返还朝鲜被俘的两位王子

七、朝鲜宣誓永不背叛日本。

在这份所谓的和平条款中,除交还朝鲜王子外,没有任何的友善、和睦,不但强占朝鲜土地,还把手伸到了明朝,
总而言之,除了贪婪,还是贪婪。

这样的条款,是任何一个大明使臣都无法接受的。

沈惟敬接受了。

这位仁兄似乎完全没有任何心理负担,当场拍板,表示自己认可这些条款,并将回禀明朝。丰臣秀吉十分高兴。

其实丰臣秀吉并不知道,他已失去了一个过把瘾的机会--即使他提出吞并中国,这位大明使者也会答应的。

因为沈惟敬同志压根就不算是明朝的使臣,说到底也就是个混混,胡话张口就来,反正不是自家的,也谈不上什么
政治责任,你想要哪里,我沈惟敬划给你就是了。反正也不是我买单。

日本和谈就此结束,简单概括起来,是一群稀里糊涂的人,在一个稀里糊涂的地方,开了一个稀里糊涂的会,得到
了一个稀里糊涂的结论。可怜一代枭雄丰臣秀吉,风光一辈子,快退休了,却被两个粗人、一个混混玩了一把,真
可算是晚节不保。

但在办事认真这点上,丰臣秀吉还是值得表扬的,为了把贪欲进行到底,他随即安排了善后事宜,遣送朝鲜王子回
国,并指派小西行长跟进此事。

小西行长高兴地接受了这个任务,不久之后,他就会悔青自己的肠子。

\section[\thesection]{}

和谈结束了,沈惟敬回国了,他在日本说了很多话,干了很多事,但在中国却无人知晓,连李如松、宋应昌也只知
道,这人去了趟日本,见了丰臣秀吉,仅此而已。

按说到这个时候,沈惟敬应该说实话了,在日本胡说八道也就罢了,但军国大事,不是能忽悠过去的,鬼子虽然脑
袋不好使,也不是白痴,想蒙混过关,那是不可能的。

但这位兄弟实在是人混胆大,没有丝毫政治敏感性,兵部尚书石星代表朝廷找他谈话时,竟对日方提出``和平条件
``只字不提,只顾吹牛,说自己已经搞定了日方,为国家做出了卓越贡献云云。

这话要换了宋应昌,估计是打死也不信的,可石星同志就不同了,从某个角度讲,他还是个比较单纯的人,一顿忽
悠之下,竟然信了,还按照沈惟敬的说法,上奏了皇帝。

明神宗倒不糊涂,觉得事情不会这么简单,但石星一口咬定,加上打仗实在费钱,半信半疑之下,他同意与日方议
和。

于是历史上最滑稽的一幕出现了,经过一轮又一轮的忽悠,中日双方终于停战。

万历二十一年(1593)七月,在日军大部撤出朝鲜后,明军也作出部署,仅留刘珽、骆尚志等人,率军一万五千余人
帮助镇守军事要地,其余部队撤回国内。

无论有多么莫名其妙,和平终究还是到来了,尽管是暂时的。

宋应昌升官了,因为在朝鲜战场的优异表现,他升任右都御史,兵部侍郎的职务,由顾养谦接替。

李如松也升官了,本就对他十分欣赏的明神宗给他加了工资(禄米),并授予他太子太保的头衔。

三年后,辽东总兵董一元离职,大臣推举多名候选者,明神宗却执意要任用李如松,虽然许多人极力反对,但他坚
持了自己的意见。

李如松走马上任,一年后他率军追击敌军,孤军深入,中伏,力战死。

在所有的战斗中,他始终是身先士卒,冲锋在前的,这次也不例外。

他不是一个与人为善的人,更谈不上知书达理,他桀骜不逊,待人粗鲁,但这些丝毫无损于他的成就与功勋,因为
他是一个军人,一个智勇双全、顽强无畏的军人。在短暂的一生中,他击败了敌人,保卫了国家,在我看来,他已
经尽到了自己的本分。

其实很多人并不知道,他虽是武将,却并非粗人,因为在整理关于他的史料时,我发现了他的诗句:

春来杀气心犹壮,此去妖氛骨已寒。

谈笑敢言非胜算,梦中常忆跨征鞍。

我认为,写得很不错。

四百年华已过,纵马驰骋之背影,依稀可见。

\section[\thesection]{}

烽火再起

沈惟敬是一个比较奇怪的人,作为一个局外人,他毅然决然搞起外交,且不怕坐牢,不怕杀头,义无反顾,实在让
人费解。

一个混混,不远千里,一不怕苦、二不怕死,专程跑来插足国家大事,在我看来,这就是最纯粹的掺和精神。

但既然是掺和,一般说来总是有动机的。因为就算是混混,也得挣钱吃饭。可由始至终,这位仁兄似乎除了混过几
顿饭外,还没有狮子大开口的记录,也没怎么趁机捞过钱,所以我们有理由相信,他是真想干点事的。

然而沈惟敬并不知道:虽然从某种意义上说,外交政治也是混,不过,绝不是他那个混法。如果胡混一气,是要掉
脑袋的。

万历二十二年(1594)十二月七日,一个人的到来让沈惟敬明白了一个道理:说过的话,签过的字,不是说赖就能赖
的。

小西飞来了,根据日本和谈的会议精神,他作为日本的使者,前来兑现之前明朝的承诺。

沈惟敬迎来了一生中最大的危机,因为小西飞并没有参与他的密谋,而日方使者到来,必定有明朝高级官员接待,
到时双方一对质,事情穿帮,杀头打屁股之类的把戏是逃不了了。

人已经到京城了,杀人灭口没胆,逃跑没条件,就算冲出国门也没处去----日本、朝鲜也被他忽悠了,要冲出亚
洲,估计还得再等个几百年。

在沈惟敬看来,他这辈子就算是活到头了,除非奇迹出现。

奇迹出现了。

万历二十二年(1594)十二月十九日,兵部尚书石星奉旨,与小西飞会谈。

在会谈中,石星提出了议和的三大条件--真正的条件:

一、 日本必须限期全部撤军回国。

二、 封丰臣秀吉为日本王,但不允许日本入贡。

三、 日本必须盟誓,永不侵犯朝鲜。

然后他告诉小西飞,如果同意,就有和平,如果拒绝,就接着打。

出发之前,小西飞被告知,明朝已经接受了日方提出的七大条件,他此来是拿走明朝承认割让朝鲜的文书,如果一
切顺利,还要带走明朝的公主。

而现在他才知道,公主是没影的,割让朝鲜是没谱的,通商是没指望的。日本唯一的选择,是从明朝皇帝那里领几
件衣服和公章,然后收拾行李,滚出朝鲜,发誓永不回来。

小西飞已经彻底懵了,他终于明白,之前的一切全是虚幻,自己又被忽悠了。

然而接下来,他却做出了一个出人意料的举动。

\section[\thesection]{}

面对石星,小西飞说出了他的答复:同意。

所谓同意,代表的意思就是日本愿意无条件撤出朝鲜,不要公主,不要通商,不再提出任何要求。

当然,这是不可能的。

所以结论是,小西飞撒了谎。

而只要分析一下,就会发现,他的确有撒谎的理由。

首先,他是小西行长的亲信,这件事又是小西行长负责,事情办到这个地步,消息传回日本,小西行长注定是没好
果子吃的。

其次,他毕竟是在明朝的地盘上,对方又是这个态度,如果再提出丰臣秀吉的``梦幻''七条,惹火了对方,来个
``两国交兵,先斩来使''也不是不可能的。

所以当务之急,把事情忽悠过去,回家再说。

听到小西飞的回答,石星十分高兴,他急忙向明神宗上奏疏,报告这一外交的巨大胜利。

可他万万没有想到,明神宗竟然不信!

要知道,这位皇帝虽然懒,却不笨,他得知此事后,当即表示叫来石星询问此事:如此之条件,日本人怎么会轻易
接受?

石星本来脑袋就不大好使,这么一问,算是彻底糊涂了,半天也不知怎么回答。

最后还是明神宗替他想出了办法:

``明日,你在兵部再次询问日使,不得有误。''

之后还跟上一句:

``赵志皋随你一同去!''

赵志皋,时任大学士,特意交代把他拉上,说明皇帝对石星的智商实在是缺乏信心。

万历二十二年(1594)十二月二十日,第二次询问开始。

这次询问,明朝方面来了很多人,除了石星和赵志皋外,六部的许多官员都到场旁听。

在众目睽睽之下,石星向小西飞提出了八个问题,而小西飞也一反常态,对答如流,说明日本的和平决心,听得在
场观众频频点头。

经过商议,石星和赵志皋联合作出了结论:小西飞,是可以相信的。

然而石星并不知道,小西飞之所以回答得如此顺畅,是因为他所说的每一句话,都是不折不扣的胡扯。

具体说来,是想到哪说到哪,捡好听顺耳的讲,动不动就是``天朝神威''之类的标志性口号,反正千穿万穿,马屁
不穿。

虽然在场的官员大都饱读诗书,且不乏赵志皋之类的政治老油条,但毕竟当时条件有限,也没有出国考察的名额,
日本到底是怎么回事,谁也不清楚。

于是,大家都相信了

\section[\thesection]{}

凭借着在明朝的优异表现,小西飞跻身成功外交家的行列,成为了堪与沈惟敬相比的大忽悠。

但正所谓长江后浪推前浪,虽然是后进之辈,在忽悠方面,小西飞却更进一步,将其发展到了一个新的境界----除
了忽悠别人,还忽悠自己。

事情是这样的,和谈结束后按照外交惯例,明朝官员准备送小西飞回国,然而这位仁兄却意犹未尽,拿出了一份名
单。

这份名单是丰臣秀吉授意,小西行长草拟的,上面列出了一些人名,大都是日军的将领,在出发之前,他交给了小
西飞,并嘱托他在时机成熟时交出去,作为明朝封官赏钱的依据。

事已至此,小西飞十分清楚,所谓和谈,纯粹就是胡说八道,能保住脑袋回去就不容易了,可这位仁兄实在是异常
执着,竟然还是把这份名单交给了明朝官员,并告诉他们:名单上的人都是日本的忠义之士,希望明朝全部册封,
不要遗漏。

明明知道是忽悠,竟然还要糊弄到底,可谓意志坚定,当然,也有某些现实理由----小西飞的名字,也在那份名单
上。

更为搞笑的是,在交出名单之前,根据小西行长之前的交代,小西飞还涂掉了两个名字,一个是加藤清正,另一个
是黑田长政。

之所以这么干,那是有深厚的历史渊源的,虽然同为丰臣秀吉的亲信,小西行长和加藤清正、黑田长政的关系却很
差,平时经常对骂,作战也不配合,小西行长对此二人恨之入骨。

据说后来这事捅出去之后,加藤清正气得跳脚:明知册封不了的名单,你都不列我的名字?跟你拼了!

等到后来回了日本,这几位也不消停,继续打继续闹,最后在日本关原打了一仗,才算彻底了结。这都是日本内
政,在此不予干涉。

综观整个谈判过程,从忽悠开始,以胡扯结束,经过开山祖师沈惟敬和后起之秀小西飞的不懈努力,丰臣秀吉、明
神宗一干人等都被绕了进去,并最终达成了协议,实在是可喜可贺。

而更值得夸奖的,是日本人的执着,特别是小西行长,明知和谈就是胡扯,册封就是做梦,仍然坚持从名单上划掉
了自己政敌的名字,其认真精神应予表扬。

虽然这是一件极其荒谬、极为可笑的事情,但至少到现在,并没有丝毫露馅的迹象,而且在双方共同努力忽悠下,
和平似乎已不再是个梦想。

\section[\thesection]{}

这关终于过去了,沈惟敬总算是松了一口气,不过,这口气也就松了一个月。

明朝的办事效率明显比日本高得多,万历二十三年(1595)正月,明神宗便根据谈判的条款,对日本下发了谕旨,并
命临淮侯李宗城为正使,都指挥杨方亨为副使,带沈惟敬一同前往日本宣旨。

沈惟敬无可奈何,只得上路,可还没等到日本,就出事了。

事情出在明朝正使李宗城的身上,应该说,这是一个有鲜明个性特点的人,具体说来,就是胆小。

此人虽然是世袭侯爵,但一向是大门不出,二门不入,每天只想在家混吃等死,突然摊上这么个出国的活,心里很
不情愿,但不去又不行,只好一步三回头地上了路。

就这么一路走,一路磨,到了朝鲜釜山,他才从一个知情人那里得知了谈判的内情,当即大惊失色,汗如雨下。

其实这也没什么,反正没到日本,回头就是了,浪费点差旅费而已。

可这位兄弟胆子实在太小,竟然丢下印玺和国书,连夜就逃了。消息传回北京,明神宗大怒,下令捉拿李宗城,并
命令杨方亨接替正使,沈惟敬为副使,继续出访日本。于是,什么都不知道的杨方亨和什么都知道的沈惟敬,在经
历这场风波后,终于在七月渡海,到达日本。

对于他们的来访,丰臣秀吉十分高兴,他安排了盛大的欢迎仪式,并决定,在日本最繁华的城市大阪招待明朝的使
者。

九月,双方第一次见面,气氛十分融洽,在这一天,杨方亨代表明神宗,将冠服、印玺等送给了丰臣秀吉。

丰臣秀吉异常兴奋,在他看来,明神宗送来这些东西,是表示对他的妥协,而他真正想要的东西,也即将到手。

因为第二天,明朝的使者,就将宣布大明皇帝的诏书,在那封诏书上,自己的所有愿望都将得到满足。

但沈惟敬很清楚,当明天来临,那封谕旨打开之时,一切都将结束。事情已经无可挽回,除非日本人全都变成文
盲,不识字(当时的日本官方文书,几乎全部使用汉字),或者……奇迹再次出现。

想来想去,毫无办法,沈惟敬在辗转反侧中,度过了这个绝望的夜晚,迎来了第二天的早晨。

然而他并不知道,在那个夜晚,他并不是唯一无法入睡的人。

\section[\thesection]{}

在获知明朝使者到来的消息后,小西行长慌了手脚。因为在此之前,他已经从小西飞那里知道了事情的真相,却没
有去报告丰臣秀吉。

不是不想说,而是不能说。

自和谈开始,丰臣秀吉就处于一种梦幻状态,总觉得人家欠他点什么,就该割地,就该和亲。如果这个时候把他摇
醒,告诉他:其实你被忽悠了,人家根本没把你放在眼里,也不打算跟你谈判。其后果,是不堪设想的。

更为严重的是,这件事情是小西行长负责的,一旦出了事,背黑锅的都找不到。

那就忽悠吧,过一天是一天。

可现在明朝的使者已经来了,冠服也送了,诏书明天就读,无论如何是混不下去了。

为了自己的脑袋和前途,小西行长经过整夜的冥思苦熬,终于想出了一个办法。

于是,在那个夜晚,他去找了一个人。确切地说,是个和尚。

根据丰臣秀吉的习惯,但凡宣读重要文书,都要找僧人代劳,除了日本信佛的人多,和尚地位高外,还有一个重要
原因--和尚有文化,一般不说白字。

小西行长的目的很明确,他找到那位僧人,告诉他,如果明天你宣读文件时,发现与之前会谈条件不同,或是会触
怒丰臣秀吉的地方,一律跳过,不要读出来。

当然某些嘱托,比如要是你读了,我就怎么怎么你,那也是免不了的。

安排好一切后,小西行长无奈地回了家,闹到这个地步,只能过一天是一天了。

无论如何,把明天忽悠过去就好。

第二天,会议开始。

从参加人数和规模上说,这是一次空前,团结的大会。因为除了丰臣秀吉和王公大臣,大小诸侯外,德川家康也来
了。

作为丰臣秀吉的老对头,这位仁兄竟然也能到场,充分说明会务工作是积极的,到位的。

更为破天荒的是,丰臣秀吉同志为了显示自己对明朝的尊重,竟然亲自穿上了明朝的服装,并强迫手下全部换装参
加会议(皆着明服相陪)。

然后他屏息静气,等待着那个激动人心时刻的到来。

依照程序,僧人缓慢地打开了那封诏书。

此刻,沈惟敬的神经已经绷到了顶点,他知道,奇迹不会再次发生。

小西行长也很慌张,虽然事先做过工作,心里有底,但难保丰臣秀吉兴奋之余,不会拿过来再读一遍。

总而言之,大家都很紧张。

\section[\thesection]{}

但最紧张的,却是那个和尚。

昨夜小西行长来找他,让他跳读的时候,他已经知道事情不妙----要没问题,鬼才找你。

而在浏览诏书之后,他已然确定,捧在自己手上的,是一个不折不扣的火药桶。

全读吧,要被收拾;不读,不知什么时候被收拾。激烈斗争之后,他终于做出了抉择,开始读这封诏书。

随着诵读声不断回荡在会场里,与会人员的表情也开始急剧变化。

小西行长死死地盯着和尚,他终于确信,忽悠这一行,是有报应的。

而德川家康那一拨人,表情却相当轻松,毕竟看敌人出丑,感觉是相当不错的。

沈惟敬倒是比较平静,因为这早在他的意料之中。

最失态的,是丰臣秀吉。

这位仁兄开始还一言不发地认真听,越听脸色越难看,等到和尚读到封日本王这段时,终于忍不住了。

他跳了起来,一把抢过诏书,摔在了地上,吐出了心中的怒火:

``我想当王就当王(吾欲王则王),还需要你们来封吗?!''

被人当傻子,忽悠了那么久,发泄一下,可以理解。

接下来的事情就顺理成章了,先算帐。

第一个是沈惟敬,毕竟是外国人,丰臣秀吉还算够意思,训了他一顿,赶走了事。

第二个是小西行长,对这位亲信,自然是没什么客气讲的,手一挥,立马拉出去砍头。

好在小西同志平时人缘比较好,大家纷纷替他求饶,碍于情面,打了一顿后,也就放了。

除此二人外,参与忽悠的日方人员也都受到了惩处。

然后是宣战。

窝囊了这么久,不打一仗实在是说不过去。所以这一次,他再次押上了重注。

万历二十四年(1596)九月,丰臣秀吉发布总动员令,组成八军:

第一军,指挥官加藤清正,一万人

第二军,指挥官小西行长,一万四千人。

第三军,指挥官黑田长政,一万人。

第四军,锅岛植茂,一万两千人。

第五军,岛津义弘,一万人。

第六军,长宗我部元津,一万三千人。

第七军,蜂须贺家政,一万一千人。

第八军,毛利秀元,四万人。

基本都是老相识,就不一一介绍了。

以上人数共十二万,加上驻守釜山预备队,日军总兵力约为十四万人。

相对而言,在朝的明军总数比较精确,合计六千四百五十三人。

\section[\thesection]{}

日军加紧准备之时,明朝正在搞清算。

杨方亨无疑是这次忽悠中最无辜的同志,本来是带兵的,被派去和谈,半路上领导竟然跑了,只好自己接班,临危
受命跑到日本,刚好吃好住了几天,还没回过味来,对方又突然翻了脸,把自己扫地出门,算是窝囊透了。

当然了,杨方亨同志虽然是个粗人,也还不算迟钝,莫名其妙被人赶出来,事情到底怎么回事,他还不大清楚,沈
惟敬也不开口,但回来的路上一路琢磨,加上四处找人谈话,他终于明白,原来罪魁祸首,就在自己身边。

水落石出,他刚想找人去抓沈惟敬,却得知这位兄弟已经借口另有任务,开溜了。

普天之下莫非王土,反正也跑不出地球。杨方亨一气之下,直接回了北京。并向明神宗上了奏疏,说明了事情的来
龙去脉。

这下皇帝也火了,立即下令捉拿沈惟敬,找来找去,才发现这兄弟跑到了朝鲜庆州,当年也没什么引渡手续,绳子
套上就拉了回来,关进了诏狱,三年后经过刑部审查定了死罪,杀了。

沈惟敬这一生,是笔糊涂帐,说他胆小,单身敢闯日军大营;说他混事吹牛,丰臣秀吉经常请他吃饭,说他误国,
一没割地,二没赔款,还停了战。

无论如何,还是砍了。

从他的死中,我们大致可以得到这样一个启示:

有些事不能随便混,有些事不能混。

倒霉的不只沈惟敬,作为此事的直接负责人,石星也未能幸免,明神宗同志深感被人忽悠得紧,气急败坏之余,写
就奇文,摘录如下:

``前兵部尚书石星,欺君误国,已至今日,好生可恶不忠,着锦衣卫拿去,法司从重拟罪来说!''

看这口气,那是真的急了眼了。

很快,石星就被逮捕入狱,老婆孩子也发配边疆,在监狱里呆了几个月后,不知是身体不好还是被人黑了,竟然死
在了里面。

所谓皇帝一发火,部长亦白搭,不服不行。

既然谈也谈不拢,就只有打了。

但具体怎么打,就不好说了。要知道帮朝鲜打仗,那是个赔本的买卖,钱也不出,粮也不出,要求又多,可谓是不
厌其烦,所以在此之前,兵部曾给朝鲜下了个文书,其中有这样一句话:

宜自防,不得专恃天朝

这句话通俗一点说,就是自己的事自己办,不要老烦别人。

\section[\thesection]{}

而且当时的明朝,并没有把日本放在眼里,觉得打死人家几万人,怎么说也该反思反思,懂点道理。谁知道这帮人
的传统就是冥顽不灵、屡教不改,直到今天,似乎也没啥改进。

但无论如何,不管似乎也说不过去,于是经过综合考虑,明朝还是派出了自己的援军,如下:

吴惟忠,三千七百人。

杨元,三千人。

完毕。

看这架势,是把日军当游击队了。

虽然兵不多,将领还是配齐了,几张新面孔就此闪亮登场。

第一个人,叫杨镐,时任山东布政司右参政,后改任都察院右佥都御史,负责管理朝鲜军务。

这是一个对明代历史有重大影响的人,当然,不是什么好的影响。

杨镐这个人,实在有点搞。所谓搞,放在北京话里,就是混;放在上海话里,叫拎不清;放在周星驰的电影里,叫
无厘头。

其实,杨镐是个不折不扣的好人,因为根据朝鲜史料记载,朝鲜人对他的印象极好,也留下了他的英勇事迹,相关
史料上,是这样说的:

所过地方,日食蔬菜,亦皆拔银留办。

这意思是,杨镐兄的军纪很好,且买东西从来都付现款,概不拖欠。这么大方的主,印象不好,才是怪事。但能不
能打仗,那就另说了。

作为万历八年的进士,杨镐先后当过知县、御史、参议、参政,从政经验十分丰富,仗他倒也打过,原先跟着辽东
总兵董一元,还曾立过功。不过这次到朝鲜,他的心情却并不怎么愉快。

因为就在不久前,他带着李如松的弟弟李如梅出击蒙古,结果打了败仗,死伤几百人,本来要处理他,结果正好朝
鲜打仗,上面顺水推舟,让他戴罪立功,就这么过来了。

戴罪,本来就说明这人不怎么行,竟然又送到朝鲜立功,看来真把日本人当土匪了。

客观地讲,杨镐还是有些军事才能的,而且品行不错,做事细致,但他的优点,恰好正是他的缺点。

清朝名臣鄂尔泰曾经说过一句话:大事不糊涂,小事必然糊涂。

这是一句至理名言,因为人的精力是有限的,而世界上的折腾是无限的,把有限的精力投入到无限的折腾中去,是
不可能的。

李如松是个明白人,他知道自己是军人,军人就该打仗,打赢了就是道德,其他的问题都是次要的。

杨镐是个搞人,而搞人,注定是要吃亏的。

幸好,明朝也派来了一个明白人。

\section[\thesection]{}

万历二十一年(1593),送别了李如松后,麻贵来到了延绥,担任总兵,继续他的战争事业。在这里,他多次击败蒙
古部落,立了无数大功,得了无数封赏。到了万历二十四年(1596),终于腻了。于是他向朝廷提出了退休。

考虑到他劳苦功高,兵部同意了他的申请,麻贵高兴地收拾包袱回家修养去了。

但工作注定是干不完的,万历二十五年(1597),第二次朝鲜战争爆发,麻贵起复。

而他被委任的职务,是备倭大将军总兵官,兼任朝鲜提督。

接到命令后,麻贵立即上路,没有丝毫推迟。他很清楚,几年前,那个无与伦比的人,曾担任过这个职务,并创建
了辉煌而伟大的成就。

四年前,我跟随着你,爬上了城楼,现在,你未竟的事业,将由我来完成。

麻贵的行动十分迅速,万历二十五年(1597)七月七日,他已抵达汉城,开始筹备作战。因为根据多年的军事经验,
他判定,日军很快就会发动进攻,时间已经不多了。

但事实上,他的判断是错误的,时间并非不多,而是根本没有。

万历二十五年(1597)七月二十五日,全面进攻开始。

日军十二万人,分为左右两路,左路军统帅小西行长,率四万九千人,进攻全罗道重镇南原。

右路军统帅加藤清正,统军六万五千人,进攻全州。

从军事计划看,日军的野心并不大,他们不再奢求占领全朝鲜,只求稳扎稳打,先占领全罗道,以此处为基地,逼
近王京。

而要说明军毫无准备,那也不对,因为在南原和全州,也有军队驻守。

比如南原,守将杨元,守军三千人。

比如全州,守将陈愚衷,守军两千五百人。

经过计算结果如下,攻击南原的日军,约为守军的16.3倍。而攻击全州的日军,约为守军的26倍。

大致就是这么回事。算起来,估计只有神仙,才能守住。

杨元不是神仙,但也不是孬种,所以南原虽然失守,却一点也不丢人。面对十几倍于自己的敌人,杨元拼死抵抗,
并亲自上阵与敌军厮杀,身负重伤,身中数枪率十余人突围而出,其余部队全部阵亡。

相对而言,全州的陈愚衷就灵活得多了,这位仁兄明显名不副实,一点也不愚忠,倒是相当灵活,听说日军进攻,
带着兵就溜了,所部一点也未损失。

\section[\thesection]{}

南原和全州失陷了,两路日军于全州会师,开始准备向汉城进军,四年之后,他们再次掌握了战场的主动权。

胜负之间

杨元逃回来了,麻贵亲自接见了他,并对他说了一句话:

``南原之败,非战之罪''。

想想倒也是,几千人打几万人,毕竟没有投降,也算不错了。对于领导的关心和理解,杨元感到异常地温暖。

但是,他并没有真正理解这句话的意思。

事实上,就在他倍感安慰的时候,麻贵在给兵部的上书中写下了这样几个字--``按军法,败军则诛''。

所谓``非战之罪'',并不代表``非你之罪''。虽然杨元很能打,也很能逃,但城池毕竟还是丢了,丢了就要负责任。
数月之后,他被押到辽阳,于众军之前被斩首示众。

麻贵很理解杨元,却仍然杀掉了他,因为他要用这个人的脑袋,去告诉所有人:这场战争,不胜,即死!

现在,摆在麻贵眼前的,是一个极端的危局。

攻陷全州后,日军主力会师,总兵力已达十余万,士气大振,正向王京进军。

此时,另一个坏消息传来,朝鲜水军于闲山大败,全军覆灭。

虽然朝鲜打仗不怎么样,但必须承认,搞起政治斗争来,他们还是很有点水平的。第一次战争刚刚结束,就马不停
蹄地干起了老本行。

这次遭殃的,是李舜臣,击退日军后,李舜臣被任命水军统制使,统帅忠清、全罗、庆尚三道水军,大权在握,十
分风光。

十分风光的结果,是十分倒霉。还没得意几天,就有人不高兴了,同为水军将领的元均看他不顺眼,便找了几个志
同道合的哥们,整了李舜臣一把。这位革命元勋随即被革职,只保住了一条命,发配至军中立功赎罪。而元均则得
偿所愿,官运亨通,接替了李舜臣的位置。

但可以肯定的是,元均同志的脑筋并不是很好使,因为他忽略了一个十分重要而明显的问题----在享受权力的同
时,还要承担义务。

万历二十五年(1597)六月,元均走马上任,七月七日,日军来袭。

从技术角度讲,打仗是个水平问题,能打就打得赢,不能打就打输。而元均,就属于不能打的那一类。

日军的水军指挥官是藤堂高虎,就其指挥水准而言,他比之前的九鬼嘉隆要低个档次,但很不幸的是,和李舜臣比
起来,元均基本算是无档次。

\section[\thesection]{}

双方交战没多久,不知是队形问题,还是指挥问题,朝军很快不支,死伤四百余人。元均随即率军撤退,并从此开
始了他的逃窜生活。

七月十五日,逃了一星期后,元均被日军追上了。双方在漆川岛展开大战,朝军再次大败,元均再次逃窜。

七月二十三日,又是一个星期,元均又被日军追上了。这次作战的地点是巨济岛,朝军又大败,但元均终于有了点
进步,他没有再逃下去--当场战死。

经过几次海战,日方不费吹灰之力,击沉船只一百五十余艘,朝鲜海军被彻底摧毁。

朝军完了,明朝水师人数很少,日军就此控制了制海权,十二万大军水陆并进,扑向那个看似唾手可得的目标----
王京。

镇守王京的将领,是麻贵,他已经调集了所有能够抽调的兵力,共计七千八百四十三人。

对于这个数字,麻贵是很有些想法的,所以他连夜派人找到了直属领导,兵部尚书兼蓟辽总督邢玠,请求放弃王京
后撤。

邢玠的答复很简单:不行。

既然领导说不行,那就只有死磕了。毕竟杨元的例子摆在前面,自己可以杀杨元,邢玠就能杀自己。

但手下就这么点人,全带出去死拼,拼未必有效果,死倒是肯定的。琢磨来琢磨去,麻贵决定:打埋伏。

经过仔细筹划,埋伏的地点设在王京附近的稷山,此地不但地势险要,而且丛林众多,藏个几千人不成问题。

九月六日夜,麻贵亲自选派两千精兵,深夜出城,前往稷山设伏。

他很清楚,这已是他的全部家底,如伏击不能成功,待日军前来,就只能成仁了。

生死成败,一切都在冥冥之中。

九月七日,日军先锋部队一万两千人到达稷山。

在日军指挥官看来,眼前形势很好,不是小好,是大好,十几万大军对几千人,无论如何是赢定了。

上级领导的乐观也感染了广大日军,他们纷纷表示,在进入王京时,要全心全意地烧杀抢掠,绝不辜负此行。在这
种情绪的指导下,日军各部队奋勇争先,力求先抢,军队的队列极其混乱。

这正是明军所期待的。

拂晓,日军进入伏击圈,明军指挥,副总兵解生发动了攻击。

没有思想准备的日军顿时大乱,明军又极狡猾,他们并没有立即冲出来肉搏,而是躲在丛林中发射火枪火炮,所以
虽然杀声震天,人却是一个皆无。挨了打又找不着主,日军越发慌乱。

\section[\thesection]{}

第三军军长黑田长政闻讯,当即带领三千人前来支援,可慌乱之间毫无作用,自己的军队反而被败退的前军冲乱,
只得落荒而逃。

眼看时机成熟,解生随即下令发动总攻,两千明军全线出击,奋勇追击日军。

这是日军的又一次崩溃,简单说来,是两千明军追击一万五千日军,且穷追不舍。这一景象给日军留下了深刻印
象,所以在相关的日本史料中,留下这样的记载:稷山之战,明军投入了四万大军,布满山林,不见首尾(遍山盈
野)。

只有鬼才知道,那多余的三万八千人,是从哪里寻来的。

就这样,日军大队被两千明军追着跑,损失极为惨重,追赶鸭子的游戏一直进行到下午四点,直到日军右路军主力
到达,才告结束。

此战,日军大败,阵亡八百余人,伤者不计其数,史称``稷山大捷''。

这是极为关键的一战,虽然日军仍占有绝对优势,但麻贵的冒险迷惑了对手。几乎所有的日军指挥官都认定,在王
京等待着他们的,是一个更大的陷阱。

于是他们停下了脚步。

这是一个极为错误的军事判断,此后,他们再也未能前进一步。

虚张声势的麻贵赢得了时间,而不许后退的邢玠也没有让他失望。在短短两个月时间内,他已完成了部署,并抽调
两万余人进入朝鲜作战,加上之前陆续赶到的部队,此时在朝明军的数量,已经达到五万。

错失良机的日军这才恍然大悟,但已于事无补,随即全军撤退,龟缩至南部沿海釜山一带,离下海只差一步。

战争的主动权再次回到明军的手中,麻贵知道,该轮到自己了。

为了让日军毫无顾虑,放心大胆地下海,麻贵制定了一个全新的作战计划。

四万明军随即分为如下三路:

左路军,统帅李如梅,杨镐,一万六千人,进军忠州

中路军,统帅高策,一万一千人,进军宜宁

右路军,由麻贵亲率,一万四千人,进军安东。

此外,朝军一万余人,进军全州。

这是一个很有趣的阵型,因为各路大军的进军方向,正是日军的集结地,而他们,将面对各自不同的敌人。

中路军的前方,是泗川,这里驻扎的,是日军岛津义弘部。

朝军的前方,是顺天,呆在此地的,是日军小西行长部。

两路大军气势汹汹地向着目标挺进,然而,他们是不会进攻的。

\section[\thesection]{}

派出这两支部队,只为一个缘由----迷惑敌人。

日军有十二万人,明军只有四万,所以分别击破,是明军的唯一选择。

而麻贵选中的最后目标,是蔚山。

蔚山,是釜山的最后屏障,战略位置极为重要,交通便利且可直达大海,是日军的重要据点。

麻贵据此判定,只要攻占蔚山,就能断绝日军的后勤,阻其退路,全歼日军。

驻守蔚山的,是加藤清正,兵力约为两万,就人数而言,并不算多,看上去,是一个再理想不过的下手对象。

但事情并不那么简单,日军明显吸取了四年前的教训,在布阵上很有一套。顺天、泗川、蔚山各部日军,摆出了品
字型阵型,形成了一个十分坚固,互相呼应的防御体系。

所以麻贵决定耍阴招,他先后派出两路部队进逼顺天、泗川,造成假象,使其无法判断进攻方向。此后,他将主力
明军三万余人分成左右两路,分别向不同的目的地挺进,以降低日军的警觉。

一切都按计划进行,万历二十五年(1597)十二月二十日,左右两军突然改变方向,在距离蔚山不到百里的庆州会
师,麻贵的最后一层面纱终于揭开。

明军即将亮出屠刀,敌人却还在摸黑。相对而言,日军的将领都是比较实诚的,接到敌情通报后,小西行长和岛津
义弘立刻加紧自己防区的戒备,严防死守,而没有敌情的加藤清正,由于没有任务,竟然离开了蔚山,跑到附近的
西生浦出差去了。

将领水平如此低下,当兵的还不挨打,那就没天理了。万历二十五年(1597)十二月二十二日夜,明军从庆州出发,
黎明到达蔚山,进攻正式开始。

先锋李如梅率先出击,带领三千骑兵直插日军城外大营,对于这群不速之客,日军毫无思想准备,当场被斩杀一千
余人,损失惨重。明军乘胜追击,彻底击溃了城外敌军,日军全线退守城内。

明军进攻之时,加藤清正正在西江浦扛砖头修工事,而他也用自己的实际行动,证明了这样一个道理----没有最
慢,只有更慢。

这位仁兄实在是迟钝到了极点,之前毫无准备不说,仗打了一天,日军快马来报,他竟然还不相信,等败退日军前
来亲身说法,他才大惊失色,直到晚上才赶回蔚山。

\section[\thesection]{}

二十三日夜,各路明军陆续到齐,除左路杨镐、右路麻贵外,中路军董策一部也已赶到,共四万余人,成功实现合
围。

对麻贵而言,一切都很顺利,三个月前,他仅凭七千余人,就吓退了十余万日军,两个月后,他得到了增援,并成
功地分割了日军,包围了敌城。现在,他相信,最终的胜利即将到来。

实在太顺利了,顺利得超出了想象。

古语云:反常者必不久。

第二天,事情出现了变化。

明军没有丝毫松懈,于凌晨再次发起了猛攻,而战局的发展与麻贵设想的一摸一样,日军虽顽强抵抗,但在明军的
火炮猛攻下,逐渐不支,而更出奇的是,就在双方僵持不下时,城内突起大火,乱上加乱的日军再也扛不住了,随
即撤往内城高地。

到目前为止,命运之神始终在对麻贵微笑,现在,他准备哭了。

日军盘踞的地方,叫做岛山营,此地建于陡坡上,城墙由石块筑成,极其坚固,是加藤清正的杰作。

虽然这位仁兄在日本国内被称为名将,但就其战场表现来看,实在是惨不忍睹,不过此人倒也并非一无是处,在某
些方面,他还是很有水准的,比如说----搞工程。

在修筑工事和城楼方面,加藤清正是个十分合格的包工头,工作认真细致,日本国内的许多坚固城池,都出自他的
手笔。而岛山营,正是他的得意之作。

在这个世界上,有些事情是不能勉强的,打仗就是其中之一。

明军士气旺盛,人多势众,火炮齐发,日军士气低落,人少势孤,枪炮很少,无论怎么分析,明军都是稳赢的。

但现实是残酷的,明军的攻击失败了,只有一个原因----地形。

日军城池依山而建,不但高,而且陡,云梯架不上,弓箭也射不到,火炮虽有效果,但面对石头城,杀伤力有限,
加上敌军防守严密,明军仰攻一天,毫无建树,只能收兵回营。

弓箭火炮都不顶用,云梯又太短,想来想去,也只有爬了。

于是自十二月二十五日开始,在炮火的掩护下,明军开始爬山。

二十六日,明军休息,朝军奉命爬山,被击退。

二十七日,明军继续爬山,未果。

二十七日夜,经过商议,明军决定改变策略,以炮火掩护,准备柴草,借火箭射入城,发动火攻。

二十八日,大雨。

\section[\thesection]{}

从天堂到地狱,这大概就是麻贵现在的感觉。攻击不利,好不容易想了个招,又被天气搅乱了。但事实上,一切才
刚开始,因为据说地狱有十八层,而他刚进门。

就在二十八日下午,麻贵得知了另一个消息----小西行长就要来了。

作为兵力最多,脑袋最好使的日军将领,小西行长轻易摆脱了朝军的纠缠,率领船队日夜兼程,向蔚山赶来。加藤
清正可以死,但蔚山不能丢,虽说平时势不两立,但现在同乘一条破船,只能拉兄弟一把了。

形势越来越严重了,目前久攻不下,士气不振,如果让敌军成功会师,明军就有被分割包围的危险。

敌人越来越多,没有预备队,没有援军。打到这个份上,如稍有不慎,后果将不堪设想。许多将领纷纷建议,应尽
早撤退。

经过慎重考虑,麻贵终于做出了决定----围城。

这是一个让所有人都吃惊不已的抉择,但麻贵坚信,自己是正确的。

他敏锐地意识到,如果就此撤退,敌军将趁势追击,大败不可避免,虽然日军援军已到,但决定战斗成败的,却是
城内的敌人。只要残敌覆灭,胜利仍将属于自己。

于是他调整了作战部署,派部将卢继忠率军三千堵住江口,组织火炮弓箭,加强防御。高策则带兵监视釜山及泗川
日军,其余部队集结于城下,断绝敌人的一切补给,总之一句话:打不死,就围死!

麻贵的决定是明智的。因为此时明军处境不佳,日军却更惨,基本上算是山穷水尽,城内没有水源,只能喝雨水,
粮食吃光了,石头又不能啃。打仗还能提提神,不打就真没办法了。

于是在明军围困两天后,加藤清正主动派人送信给杨镐,表示希望讲和,杨镐倒也实在,说你出来吧,出来我和你
谈判。

加藤清正回复,你们明朝人不守信,我不出来。

在我看来,这就是随意忽悠的恶果。

日军的境况持续恶化,之前日军有两万余人,战斗死伤已达四五千人,躲入城的,由于没有粮食衣被,许多都冻饿
而死,到万历二十六年(1598)正月初一,城内仅余四千余人。

麻贵十分肯定:敌人,只剩下最后一口气。

可这一口气,终究没能挺过去。

\section[\thesection]{}

到目前为止,麻贵的判断一直是正确且周密的,从假象、兵力部署、战略战术、计划变更,都无一失误。

综观整个战役,他只犯了两个错误,两个看似微不足道的错误。

然而成败,正是由细节决定的。

第一个错误的名字,叫做心态。

虽然麻贵准确地判断出了日军的现状,做出了继续围困的决定,但他却忽视了这样一点:城内的日军固然要比明军
艰苦,但双方的心态是不同的。日军如果丢失蔚山,就会失去退路,除了下海喂鱼,估计没有第二条路走。所以他
们唯一的选择,就是顽抗到底。

而明军作为进攻方,占据优势,就算战败,回家睡一觉再来还能打,毕竟是公家的事儿,犯不着玩命。而在战役的
最后阶段,这一看似微小的差别,将成为决定成败的关键。

正月初二,外海的日本援军发起了潮水般的进攻,明军拼死作战,终于遏制了日军,暂时。

正月初三,日军发动猛攻,明军在付出重大伤亡后,再次抵挡了进攻,但士气已极度低落,开始收缩阵地。

正月初四,麻贵做出决定,撤退。

事情已经很明显,敌人异常顽强,此战已无胜利可能,如不立即撤退,必将全军覆灭。在随后的军事会议上,麻贵
做出了具体的撤退部署----城北右路明军先行撤退,其他部队随后跟上,部将茅国器率军殿后。

而统领城北明军的任务,他交给了杨镐。

这是他犯的第二个错误。

在接到撤退命令后,杨镐带队先行,开始一切都很顺利,部队有条不紊地行进着,但随着部队的行进,越来越多的
明军得知了撤退的消息,特别是受伤及患病的士兵,唯恐被丢下,开始喧哗起来。

应该说,在撤退中,这种事情是难免的,如能及时控制,就能平息风波。退一步讲,就算杨镐没能力,控制不住,
毕竟有人殿后,也不至于出大事。

然而在蜂拥的士兵里,嘈杂的叫喊声中,杨镐慌乱了。

这个厚道的老好人,这个连买根白菜都要付现钱的统帅,终于在最关键的时刻,暴露出了他最致命的弱点。

面对眼前的乱局,惊慌失措的杨镐做出了毁灭性的决定----逃跑。

局势再也无法挽回。

\section[\thesection]{}

从某种意义上讲,撤退就是逃跑,但两者间是有区别的:撤退是慢慢地跑,有组织地跑,而逃跑的主要内容,只有
跑。

杨镐毫无顾忌地带头逃跑了,领导有跑的权力,下属自然没有不跑的义务。一个跟着一个,明军很快大乱,四散奔
逃。

沿海日军趁机登岸追击,明军大败,伤亡惨重,余部退回庆州。蔚山之战就此失败。

此战,明军伤亡共计两万余人,进攻受挫,战线收缩至王京,而日军损失也高达一万余人,无力发动反击,朝鲜战
局再度进入了僵持状态。

战争最残酷的地方,其实并不在于死了多少人,有多少财产损失,而是它一旦开始,就很难停止。

开打前可以随便嚷嚷,可要真打起来,那就痛苦了。双方各出奇谋,什么阴招狠招都用出来,全都往死里掐,如果
双方实力差距大,当场掐死了还好,赔款割地,该干嘛就干嘛。最恶心人的,就是死掐偏掐不死,你能打,我也不
差。

但凡遇到这种情况,双方都头疼,要不打吧,死了那么多人,花了那么多钱,这笔帐找谁算?更何况,还有一个面子
问题。

麻贵面临的,就是这种状况。

蔚山战役之后,明军开始收拾残局。

第一件事是整军队,麻贵亲自出马,把战败的士兵重新集结起来,并向朝廷打报告,要求增兵。

第二件事是整人,也就是追究责任,首当其冲的就是杨镐。这位仁兄自然没个跑,仗打成这样,作为主要责任人,
处罚是免不了的。被言官狠狠地参了一本,搞得皇帝也怒了,本打算劈他,大臣求情,这才罢官免职,没挨那一刀。
这位兄弟的事还没完,后面再说。

善后处理圆满结束,可是接下来就难办了。

日本方面力不从心,很想和谈。打到今天,独占朝鲜是不敢想了,可毕竟投入本钱太多,还是希望多少捞点好处,
挽回面子,才好走人。

然而明朝却是死硬派,根本就没想过谈判,别说割地赔款,连路费都不打算出,且毫无妥协退让的意思。

谈是谈不拢了,可要打也打不起来。日军虽然人多,但之前被打怕了,只是龟缩在沿海地区,不敢进犯。估计是学
精了,占多少是多少,死赖着不走。

\section[\thesection]{}

明军倒是很有进取精神,总想赶人下海,无奈兵力实在太少,有心而无力,只能在原地打转。

总而言之,谁也奈何不了谁,于是大家只能坐在原地,继续等待。

等着等着,日军开始吃不消了。因为他们部队太多,且长期出差在外,国内供养不起,又没人种田,只能陆续往回
拉人,在朝日军人数随即减至八万。

与此同时,明朝军队却源源不断地开入朝鲜,加上麻贵之前整顿的新军,总数已达七万。

明军从未如此强大,日军也从未如此弱小,于是麻贵认为,行动的时候到了。

万历二十六年(1598)七月,麻贵再次做出了部署:

东路军,由麻贵亲率,所部三万人,攻击蔚山。

中路军,统帅董一元,所部两万六千人,攻击泗川。

西路军,统帅刘綎,所部两万人,攻击顺天。

九月七日,三路明军正式出征,这一次,没有假象,不用转弯,所有的军队,都将直奔他们的对手。

在当时的麻贵看来,选择这个时候出征,实在是再好不过了。此时距上次出征已有半年,各部修整完毕。而在此期
间,锦衣卫也来凑了把热闹。事实证明,这帮人除了当特务,干间谍也有一套,探明了日军的虚实和实际兵力,并
提供了大量情报。

出于对特务同志们的信任,加上手里有了兵,麻贵相信,最后的胜利即将到来。

但是他又错了。

麻贵不知道的是,锦衣卫的工作虽然卓有成效,却绝非尽善尽美,因为有一条最为重要的情报,他们并未探知:

万历二十六年(1598)八月,丰臣秀吉病死于日本,年六十三。

这位日本历史上的一代枭雄终于死了,他的野心也随之逝去,归于梦幻,但他亲手挑起的这场战争,却还远未结束。

丰臣秀吉死后,日本方面封锁了消息,并指派专人前往朝鲜,传达了这样一道命令:

极力争取议和,如议和不成,即全线撤退。

撤军日期为万历二十六年(1598)十一月五日,此日之前,各军应严加布防,死守营垒,逃兵格杀勿论,并应誓死击
退明军之一切进攻。

为保证撤退成功,当时知道这一消息的,仅有小西行长、加藤清正等寥寥数人,连许多日军高级将领也不知道。

但世上没有不透风的墙,丰臣秀吉的死讯竟然还是传到了朝鲜,然而没有人相信,因为根据以往的传闻计算,丰臣
秀吉至少已经死掉了十多次。

于是,在前方等待着麻贵的,是日军最后的疯狂。

\section[\thesection]{}

第一个到达目的地的,是西路军,主帅刘綎

刘綎,字子绅,江西洪都(今南昌)人。应该说,他是一个名副其实的猛人。

刘珽的父亲叫做刘显,是明军的高级军官,而且经常领兵出战,基本上没怎么在家呆过。但值得夸奖的是,虽然他
长期不在家,刘珽的教育辅导工作却一点也没耽误----刘显打仗,是带着儿子去的。

自幼出入军营,吟诗作对基本上是不可能的,每天见惯砍砍杀杀。有这样优良的家庭教育打底,刘珽很早就体现出
了武将的天赋。他不但勇猛善战,而且力大无穷,用的兵器也很特别----镔铁大刀。

所谓镔铁,到底是啥成分,已经无人知晓,但它的重量,史料上是有记载的:一百二十斤。

当然了,一百二十斤的大刀也不算太重,只要身体还行,练一练也还举得起来。不过刘綎同志不光举,而且用,其
具体用法,史料上是这样形容的----轮转如飞。

每次我看到这四个字,都有不寒而栗之感。

在战场上用这种兵器,那真是想低调都不行,所以很快刘珽就出名了,而且还有一个响亮的外号--刘大刀。

刘大刀不但手里的家伙实在,人也很实在,说砍就砍,从不含糊。万历初年,刘显奉命去西南讨伐蛮族,大刀兄虽
然才二十多,也跟着去了,并且在战场上表现活跃,勇猛无畏,立下了战功。

从此他就再也没有消停过。

万历十年,他又跑到了缅甸,把当地人打了个落花流水,并被升为游击。之后他挥舞大刀,听从祖国召唤,哪里需
要就往哪里砍,全国各地都留下过他的身影。到朝鲜战役前夕,他已升任参将。

仗虽然打起来了,却没他什么事,也没人想用他,于是大刀兄坐不住了,自己提出申请,希望带兵去朝鲜打仗。朝
廷一想,反正这人闲着也是闲着,就派他去了。

刘綎的运气不错,刚到朝鲜没多久就升了官,当上了副总兵,但在这次战争中,他却并非主角,因为他资历太浅,
而且上面还有一个更猛的李如松,所以在朝鲜的这几年,他很少承担主战任务,基本上是配合吴惟忠、查大受等人
作战。

\section[\thesection]{}

到万历二十三年,明军撤军时,他奉命留守朝鲜,帮助朝军训练部队,当上了教官,直至再次开战。

现在,他的机会终于到了。

在当时的赴朝明军中,有三支公认战斗力最强的队伍,他们分别是李如松的辽东铁骑、吴惟忠的戚家军,以及刘綎
的车军。

作为武将世家子弟,刘綎也有一支隶属于自己的特殊部队----车军。它没有辽东铁骑的迅猛,也不如戚家军善战,
却被日军认为是最难应付的军队。

车军,共计五千余人,以川人为主。与辽东铁骑和戚家军不同,它是一支混合部队,除了步兵,还有骑兵,火枪
兵,当然,还有大车。

具体战法是这样的,每逢出战,骑兵先行,步兵和火枪兵推着大车前进,敌人出现时,即迅速将大车围成圆圈,组
成车阵,火枪兵以此为屏障,用火枪对敌发动齐射,完成第一波攻击。

待敌军锐气已尽时,便发动骑兵由车阵内冲出,击垮敌阵,然后步兵出击,追歼敌军。

很明显,这是一种攻守兼备的战法,守时滴水不漏,攻时锐不可挡,凭借这支部队,刘綎赢得了无数次战斗的胜利。

所以他一直坚信,在自己的大刀和车军面前,所有的敌人都将崩溃,小西行长也不例外。

自从入朝以来,小西行长的大部分时间都呆在顺天。与其他人不同,他的脑袋十分清醒,所谓侵朝灭明,不过是痴
人说梦,跟着混事就行。现在痴人已经死了,梦也结束了,就等着收拾包袱回家。

可这事八字还没一撇,就来了送行的,而且看架势,是要把自己直接送进海里。

万历二十六年(1598)九月十九日,刘綎部逼近顺天。

小西行长和刘綎交过手,也知道车军的厉害。但此时此刻,面对这个可怕的对手,他却并不慌张,因为他已经找到
了克制车军的方法。

其实这个方法并不神秘,简单说来就两个字:不打。

反正打不赢,索性不理你,看你还能怎么办?

敌人死不出头,这下刘綎也没招了,只得命令部队强攻,但大车毕竟不是坦克,又不能撞墙,而小西行长坚守营
垒,凭借有利地形,多次击退明军。刘綎进攻受挫,只得暂停攻击。

既然攻不下来,刘綎决定,与小西行长和谈。当然,和以往一样,这次也不是真谈

\section[\thesection]{}

如果评选被忽悠次数最多的将领,小西行长排第二,没有人敢排第一。这位仁兄不但多次被忽,还举一反三,加上
了忽人的行列。按说以他在这一行的资历,是不会再相信这类话了。

开始也确实如此,刘綎连续派出了三批使者,小西行长都不信。但刘大刀却是不依不饶,一定要把阴招进行到底,
又派出了第四批使者。

这次,小西行长终于相信了。他准备出城与刘綎谈判。

然而关键时刻,明军出了叛徒,泄露了刘綎的计划,小西行长又缩了回去。

从沈惟敬开始,再到李如松、刘綎,谈了无数次,被骗了无数次,我相信,即便打死他,下辈子再投胎,他也不会
搞谈判了。

刘綎正确地认识到了这一点,所以他改变了策略,全力监督部下攻城,但日军防守严密,多次进攻毫无进展,刘綎
毫不气馁,亲自上阵指挥战斗。

然而,十月三日,他却突然停止了攻击。

因为在这一天,他得到了中路军的战报。

董一元到达泗川的时间,是九月二十日。而他的对手,是岛津义弘。

三年前,当丰臣秀吉听到僧人宣读的诏书,明白自己已经上当,怒火中烧之时,曾对沈惟敬和杨方亨说过这样一句
话:

``且留石曼子兵于彼,候天朝处分!''

联系上下文,这句话的真实含义是,我把石曼子和他的兵留在那里(朝鲜),看你们(明朝)怎么办!

石曼子,就是岛津义弘。

作为日本九州地区的诸侯武将,岛津义弘绝非丰臣秀吉的嫡系,恰恰相反,在丰臣秀吉统一日本的过程中,他是一
个极其顽固的死硬派,硬到全国基本都被打服,他还硬挺着。

然而,丰臣秀吉却对其十分欣赏,多次重用,原因很简单----好用。

日本人的性格特点是一根筋,而九洲地区则将此传统精神发扬到了登峰造极的地步。无论是做买卖还是打仗,都很
实诚,绝不偷奸耍滑,作战时一定在前,撤退时必定垫背,其勇猛顽强连丰臣秀吉也望而生畏。

\section[\thesection]{}

更值得称道的是,直到今天,这里依然是民风犹存。比如说黑社会,经过多年改良,而今在东京干这行的,全都是
西装革履,讲究秩序,遵纪守法,连收保护费都讲纪律,从不随意捣乱。

九州萨摩地区的就没谱了,时代不同了,传统一点没丢,但凡遇上抢地盘、谈判之类的事,经常二话不说,拿着刀
赤膊上阵,往死里砍,在日本黑社会组织中极具威望,向来无人敢惹。

岛津义弘和他的第五军就属于这一类型,其作战特点是勇猛、凶残,不怕死,即使寡不敌众也敢打,是日军的战斗
主力。

而岛津义弘除陆上作战勇猛外,还精通水军指挥,也算是两栖人才。虽然脑筋不太灵活,但贵在敢玩命,而且他还
有一项独门绝技----突围。

所谓突围,其实就是逃跑。岛津义弘最绝的地方就是,他打仗不含糊,逃跑也很厉害,不但逃得准,而且逃得快,
专往敌军结合部跑,一眨眼就没影。在后来的日本关原之战中,他所隶属的西军全线溃败,剩下他带着一千多人,
面对德川家康几万大军的重重包围,竟然还是逃了出去,实在很有两把刷子。

总而言之,此人能攻善守,经验丰富,可算是朝鲜战场上的日军名将。

相对而言,中路军指挥董一元就低调得多了,此人名气一般,才能一般,连兵力都一般。日军有两万人,他也只有
两万六千。

但这位一般的人,有个不一般的先锋----李宁。

这位仁兄的脾气可谓是尽人皆知,每天喊打喊杀,见到日本人就拔刀,连使者都砍,差点坏了李如松的大事。

现在,他表现的机会到了。九月二十七日,明军刚刚到达泗川,他就等不及了,二十八日夜便率军一千,连夜冲入
了泗川城内。日军准备不足,被冲得七零八落,但毕竟人多势众,随即组织反击。李宁由于过于靠前,被日军围
攻,战死。

但他的死是值得的,董一元带领大军随后赶到,一顿猛砍猛杀,全歼守军,击毙日军大将相良丰赖,主将川上忠实
身负重伤,率领一百余人逃进内城。

内城的守备者,正是岛津义弘,他倒不怎么慌张,因为城内还驻扎着第五军主力一万余人,且地势险要,三面环
水,易守难攻。所以他打定算盘,在此坚守,等候援军到来。

话虽这么说,但当明军进攻之时,他才发现,自己的算盘估计是打错了。

\section[\thesection]{}

董一元虽然才能平平,却也不是善茬,他压根就没想过要派人去硬攻,地形如此险要,还是用炮合算。

十月初一,总攻击正式开始。

明军在离城百米处布下阵地,架设大量佛郎机炮,对准城内猛烈轰击。城内日军死伤甚多,且火光四起,顾此失
彼,一向镇定自若的岛津义弘也不镇定了,当即集合部队,准备发挥他的逃跑绝技。

事实上,他的判断是很正确的,明军的炮火已扫清了外围,城门也被攻破,大批明军已集结待命,只等一声令下冲
入城内,此时的日军已毫无斗志,即将完全崩溃。

俗话说:三分天注定,七分靠打拼。现在打拼已过,七分到手,接下来的是三分。

前方已经没有阻拦,董一元下达了总攻令。

正当他准备拿下最后三分的时候,一阵猛烈的巨响却轰鸣而起----在他的身后。

爆炸发生在明军部将彭信古的大营中,并引发了营中火药连锁效应,许多明军士兵被当场炸死,火光冲天而起,军
心顿时大乱。

事后调查证实,引发此事的,不是日军的伏兵,更不是什么忍者之类的玩意,而是安全工作疏漏--失火。

这就真没办法了,命苦不能怨政府。

混乱之中,明军不知所措,皆以为是被人抄了后路,纷纷逃窜,眼看到手的泗州城就此落空,原本打算溜号的岛津
义弘立即来了精神,出城发动攻击,明军大败。

泗川之战以失败告终,明军损失惨重退守晋州,日军侥幸取胜不敢追击,依旧固守原地。

因为此战,岛津义弘名声大振,在日本国内被捧上了天,称为``鬼石曼子'',其实说穿了,这位仁兄的胜利秘诀只
有一条----运气好。

但无论如何,赢了就是赢了,输了就是输了。而输了的结果,是很严重的。

因为除西路军刘綎外,此时的麻贵,也正处于进退两难之际。

他的东路军于九月底到达蔚山,却无事可干。因为自从上次吃了亏后,加藤清正每天都呆在蔚山,一动都不动,打
死也不出头。麻贵攻,他就守,麻贵不攻,他还是守。总而言之,不打,只拖。

就这么拖到十月份,泗川战败的消息传来,无论是麻贵,还是加藤清正,大家都松了一口气--解脱了。

在麻贵的统一调配下,东西两路军分别撤退,返回出发地,九月攻势宣告结束。

\section[\thesection]{}

在这次进攻中,明军立功心切,日军保命要紧,拼了半个多月,战局却无丝毫改变,大家都白忙活了。

最郁闷的人是麻贵,他尽心竭力策划的进攻方案,却无任何效果,实在是比较窝囊。但更让他绝望的是,经过此
役,他已经确定,凭借目前明军的实力,是不可能打破战场僵局的,绞尽脑汁也无济于事。

麻贵并不知道,此时距离日军撤退,仅剩一个月时间。如无意外,十一月五日,日军将带着抢掠的无数战利品从容
退回国内。而那时,明军只能望洋兴叹,目送日军安然撤退。

但一个人的到来,终究还是改变了这一切。

这个人的名字叫陈璘,字朝爵,广东翁源人。

说起来,这位兄弟也算是老油条了,嘉靖末年就当上了指挥佥事,此后又东征西讨,几十年下来,到万历年间,终
于当上了总兵。

但他的仕途并不顺利,破格提拔从来无分,领导赏识一直无缘。游击、参将、副将、副总兵一级级地升,做官做得
那真叫艰苦。据说是因为他是广东人,且只会讲粤语,官话(即当时的普通话)讲不好,也听不懂,总不招人待见,
所以进步很慢。

而且这人还有个缺点----贪,且不是一般的贪。方式是多种多样,层出不穷:派他去管兵,就放纵手下抢掠民财;
派他去镇守地方,就大兴土木,贪污工程款;派去打仗,竟然又克扣军饷。在贪污这行当里,可谓相当之牛。

可就是这么个人物,偏偏极会打仗,而且什么仗都打过。开头在山区打土匪地痞,后来到地方,又管过治安,抓过
强盗小偷,还曾跟着一代名将(兼贪污犯)殷正茂混过(物以类聚,人以群分),剿灭了许多叛乱军。

算起来,不听招呼的各类人等,只要在陆地上,他都灭过了。

更为难能可贵的是,连海上的品种,他也没有放过,海贼、海盗、乃至于倭寇,都在他的消灭范围之内。

可是这位水陆两用人才,实在是毛病太多,谁沾上谁倒霉,所以一直以来,既没人用,也没人举荐(朝士惜其才,不
敢荐)。

和平年代,大家不想惹事,这种人就不能用,但战争一来,自然就变成不能不用了。

\section[\thesection]{}

万历二十年(1592),陈璘出山,前往朝鲜。

按照朝廷的原意,把这个爱惹麻烦的家伙放出来,自然是要他卖命打仗,可不知为什么,这位兄弟去了一年,竟然
什么也没干,官却升得极快,刚去的时候只是个参将,很快就升为副将,万历二十一年,他已经当上了副总兵。

一仗没打就混到这个地步,几乎所有人都莫名其妙。

当然,陈璘除外,战争结束后,他怀揣着升官的秘密,高高兴兴地收拾行李去了福建,并就任总兵,凭借他多年累
积的捞钱经验,发财致富指日可待。

但纸包不住火,三年后,中日和谈失败,沈惟敬的忽悠被识破,石星被判下狱,而另一个秘密也就此曝光。

原来陈璘兄并非只进不出,他除了能贪外,还很能送,石星收了他的钱,自然要帮他办事,陈璘同志这才得以一路
春风,扶摇直上。

可是现在石星倒了,官自然是没法当了,去监狱找他退钱估计也不成,亏了本的陈璘只好再次回了老家。

但人只要有本事,就不怕没活干,(1597)万历二十五年,中日再次开战,朝鲜水军全军覆没,李舜臣还在军营里扛
木头,要夺回制海权,只能靠明朝水军了。

于是陈璘再次找到了工作,虽然兵部尚书邢玠极端厌恶这个老官僚,可他没有第二个选择。

万历二十六年(1598)六月,陈璘率五千广东水军到达朝鲜,与他一同到来的,还有邓子龙。

邓子龙,丰城人,时任钦差备倭副总兵,都督佥事。

要论年头,他的资格比陈璘还要老,嘉靖中期,他就已经从军打仗了,多年来,奔波于广东、云南、缅甸、福建,
东征西讨,战斗经验丰富,而论人品,那就更不用说了,几十年兢兢业业,从小兵干起,不走后门,不搞关系,是
个不折不扣的老实人。

正因为他过于老实,没有后台,到六十多岁,才混到副总兵,且平时沉默寡言,即使受了委屈,也不与人争辩。万
历二十年(1592),他奉命出征,本来打了胜仗,却背了黑锅,被言官参劾免职,他没有辩解,只是默默地回了家。

但当万历二十五年(1597),他接到朝廷调令时,依然毫不犹豫地动身出发,尽管此时他已年逾七十,尽管他的职务
只是副总兵,尽管他即将听从一个年纪比他小,品行比他差的人(陈璘是总兵)的指挥。

就这样,两个性格截然不同的人终于走到了一起,他们的出现,将彻底改变无数日军的命运。

\section[\thesection]{}

安置邓子龙后,故地重游的陈璘见到了他的另一个下属----李舜臣。

此时的李舜臣刚刚得到解脱,元均战死后,他奉命重新组建朝鲜水军,虽然朝中还有很多人看他不顺眼,但眼下局
势危急,这个烂摊子也只能指望他了。

李舜臣之所以不招人待见,和他本人的性格有关,此人虽才具甚高,为人处世却不行,不善与人相处,碰谁得罪
谁,作为下属,是十分难搞的。

但陈璘干净利落地搞定了他,虽然他在国内一口粤语,官话讲得鬼都听不懂,但到了国外,也就无所谓了,反正无
论官话、粤语,人家都分不出来,一概不懂。而陈璘也充分发挥了他搞关系的特长,用一种特殊的方式,与李舜臣
进行了良好的沟通。

这种方式就是写诗。

一到朝鲜,陈璘就写了这样一首诗给李舜臣:

不有将军在,谁扶国势危?

逆胡驱襄日,妖氛倦今时。

大节千人仰,高名万国知,

圣皇求如切,超去岂容辞!

就文学水平而言,这首诗大致可以划入打油体或是薛蟠体,还不是一般的差劲,但如果细细分析,就会发现,其中
的政治水平十分高超。

前四句是捧人,作为李舜臣的上级,对下属如此称赞,也真算是下了血本。

第五六句继承风范,大肆夸奖李舜臣同志众望所归,威名远扬,但这只是铺垫,核心部分在最后两句,所谓圣皇求
如切,隐含的意思就是劝人跳槽,建议李舜臣别在朝鲜干了,到明朝去另谋高就。

纵观全诗,捧人是为了挖墙角,挖墙角也就是捧人,浑然天成,前后呼应,足可作为关系学的指定教材,写入教科
书。

李舜臣被感动了,于是他连夜写了几首和诗回复陈璘,表达自己的感慨。并同时表示,愿意听从陈璘的指挥,齐心
协力,驱逐倭奴。

我一直认为,像陈璘这样的人,无论明朝兴衰与否,他都是饿不死的。

在成功实现团结后,经过麻贵鼓动,陈璘率军参加了顺天战役,然而由于战局不利,麻贵率陆军先行撤退,水军失
去支援,只得铩羽而归。

对麻贵的行径,陈璘十分愤怒,然而没过多久,麻贵再次找到了他,并交给他另一个任务。

\section[\thesection]{}

麻贵告诉陈璘:我军作战计划已定,自即日起,你所属之明军,应全部开赴海上。

陈璘问:所往何事?

麻贵答:无定事,来回巡视即可。

陈璘再问:那你准备干什么

麻贵回答:我哪里也不去,驻守原地。

看着一头雾水,满腔怒火的陈璘,麻贵终于说出了迷题的答案。

三路攻击失败之后,麻贵已经确定,强攻是不可行的。即使攻下,明军的损失也会极其惨重,而事情到了这步田
地,谈判也是不可信的。进退两难之际,他想到了陈璘,想到了一个不战而胜的方法。

麻贵下令,所有明军立即停火,中路军董一元、西路军刘綎派出使者,与对峙日军协商停战。总而言之,大家都不
要动了。

唯一活动的人,是陈璘。而他的任务,是率舰队沿朝鲜海岸巡航,并击沉所有敢于靠近海岸的日本船队。

这一军事部署,在今天的军事教科书里,叫做囚笼战术;在街头大婶的口中,叫关门打狗。

经过无数次试探与挫折,麻贵终于找到了日军的最大弱点----粮食。

无论日军多敢玩命,毕竟都是人,是人就要吃饭,而这些后勤补给必须由日本国内海运而来,所以只要封锁海岸
线,打击日本船队,敌军必定不战而溃。

事实证明,麻贵的判断是正确的。自十月中旬起,陈璘开始改行,干起了海盗。率军多次扫荡,见船就抢,抢完就
烧,把朝鲜沿海搞成了无人区。他干得相当彻底,以至于某些朝鲜船队由此经过,也被抢了。

无奈之下,日军只得派藤堂高虎率水军迎战。但陈璘同志实在是多才多艺,不但能抢,也能打,几次交锋下来,藤
堂高虎落荒而逃,再也不敢出来逞能(见璘舟师,惧不敢往来海中)。

躲不过也抢不过,日军叫苦不迭,特别是小西行长,因为三路日军中,他的处境最惨,加藤清正占据蔚山,岛津义
弘驻扎泗川,这两个地方离海很近,只要躲过陈璘,靠岸把粮食卸下来就能跑。

可是小西行长所处的顺天,不但离海远,而且水路复杂,千回百转,进去了就出不来,陈璘最喜欢在这里劫道,许
多日本船打死都不愿去。

\section[\thesection]{}

半个月下来,日军饿得半死不活,小西行长没辙了,竟然主动派人找到陈璘,希望他能让条道出来,而作为代价,
他提出了一个耸人听闻的交换条件----一千两百个人头。

这意思是,如果你放条生路给我走,我就留一千两百人给你,请功也好,杀头也罢,你自己看着办。

话说到这个地步,也是真没办法了。当然,陈璘并没有答应,因为他要的,绝不仅仅是一千两百人。

日军就此陷入绝境,但小西行长并不慌张,因为那个约定的日期,已经近在眼前。

十一月五日,只要等到那天,一切都将结束。

在期盼和忐忑之中,这一天终究还是到来了。

依照之前的约定,日军加藤清正、岛津义弘、小西行长三部开始有条不紊地收拾战利品,准备撤退。而对峙的明
军,却依然毫无动静,仍旧被蒙在鼓里。

如无意外,日军将携带其掠成果,背负着杀戮的血债,安然撤回日本。

然而意外发生了。

就在此前不久,日本五大老(丰臣秀吉五位托孤大臣)向明军派出使者,表示如果朝鲜派出王子作为人质,并每年交
纳贡米、虎皮、人参,日方出于怜悯,将会考虑撤军。

今时今日,还敢如此狂妄,似乎有点不近情理,但事实上,这是日军的一个策略。为了掩护即将到来的撤退,必须
麻痹敌军。

可是他们万万没有料到,这个所谓的计策,却起了完全相反的作用。

因为麻贵同志虽然姓麻,却很难被麻痹。毕竟在明朝政府混了几十年,什么阴谋诡计都见过了,日本人在这方面,
还处于小学生水平。

所以麻贵立即判定了日军的真实意图----逃跑。

此时是十一月七日,麻贵命令,全军动员,密切注意日军动向,随时准备出击。

十一月八日,驻扎在古今岛的陈璘接到密报,确认丰臣秀吉已经死亡,日军即将撤退。他随即下令,水军戒备,准
备作战。

明军知道,日军不知道明军知道。在千钧一发的局势中,战场迎来了最后的宁静。

无论如何,双方都已确定,生死成败,只在顷刻之间。

十天之后,最后摊牌。

万历二十六年(1598)十一月十八日,加藤清正突然自蔚山撤退。然而,出人意料的事情发生了----明军并未阻拦。

\section[\thesection]{}

随后,驻扎泗川的岛津义弘也率第五军撤退,明军仍然未动。

五大老一片欢腾,在他们看来,撤军行动十分成功,明军毫不知情。

然而接下来,一个消息打断了他们的欢呼----小西行长被拦住了。

作为脑筋最灵活的日军将领,小西行长的反应极快,获准撤退后,他立即带兵,日夜兼程赶赴海边,却看到了等待
已久的明军水师。

但小西行长并不惊慌,因为这一切早在他预料之中。

顺天离海较远,不利逃跑,而沿海地区水路复杂,易于封锁,如果明军不来,那才是怪事。

为了实现胜利大逃亡,他已想出了对策,并付诸实施,而到目前为止,事情进行得十分顺利,顺利脱身指日可待。

但事实上,五大老错了,小西行长也错了。

明军放任加藤清正和岛津义弘逃走,并非疏忽,而是一个圈套的开始。

在之前的十天里,麻贵对局势进行了认真的分析,他清醒地意识到,日军有意撤退,但凭借明军目前的兵力,是很
难全歼敌军的,恰恰相反,对方已有了充足的撤军准备,如果逼狗跳墙,后果将很难预料。

唯一的方法,就是逐个击破。

但日军是同时撤退的,明军兵力有限,鱼与熊掌不可兼得,如何做到这一点呢?

十一月四日,他终于找到了那个方法。

就在这一天,陈璘出海巡视,突然发现自顺天方向驶出一条日军小船,行踪隐蔽,速度极快。

要换在以往,陈璘会立即下令向此船开炮。

但这一次,他犹豫了,因为几十年战场经验告诉他,不能攻击这条船。

考虑片刻后,他派出了舰只跟踪此船,几个时辰之后,消息传回,他的估计得到了印证----这条船的目的地,是泗
川。

他立即将此时通报麻贵,双方的判断达成了惊人的一致:几天之内,日军将全军撤退,而那条小船,是小西行长派
出的,其唯一目的,是向岛津义弘求援。

这正是小西行长的对策,他知道,一旦撤退开始,靠海的加藤清正和岛津义弘必定能顺利溜号,而他地形不利,很
可能被堵住,到时只能找人帮忙。

加藤清正是老对头,不帮着明军打自己,就算不错了,是绝靠不住的。

只能指望岛津义弘了,他相信,关键时刻,这位二杆子是会拉兄弟一把的。

于是他派出小船通报此事,而结果也让他很满意,小船安全返回,并带来了岛津义弘的承诺。

后顾之忧解除,他终于放心了。

\section[\thesection]{}

然而就在此时,麻贵和陈璘已经制定出了最终的作战计划:

中路董一元、西路刘綎密切监视日军加藤清正及岛津义弘部,发现其撤军,立即上报,但不得擅自追击。

水军方面,陈璘部停止巡航,并撤去蔚山、泗川一带海域之水师,全军集结向顺天海域前进,堵住小西行长撤退的
海道。

放走加藤清正和岛津义弘,因为他们并不重要,只有小西行长,才是这场战争的胜负关键。

这是一个最佳的诱饵,在其诱惑之下,日军将逐个赶来,成为明军的完美猎物。

撤退、放行、堵截,一切按计划如期进行,双方都很满意,但胜利者终究只有一个,决定胜负的最后时刻已经到来。

十一月十八日,夜

小西行长没有看错人,岛津义弘不愧二杆子之名,虽然他已成功撤退到安全地带,但听说小西行长被围后,却依然
信守承诺,率第五军一万余人赶来救援。

但除了小西行长外,还有一个人也热切地期盼着他的到来----陈璘。

四天前,他召集全军,连夜赶到了顺天海域,经过仔细观察,他发现,从泗川到顺天,必须经过一条狭长的海道,
而这片海域的名字,叫做露梁海。

在露梁海的前方,只有两条水路,一条通往观音浦,另一条经猫岛,通往顺天。

他随即做出了如下部署:

副总兵邓子龙,率三千人,埋伏于露梁海北侧。

水军统制使李舜臣,率五千人,埋伏于露梁海南侧的观音浦。

而他自己则率领余下主力,隐蔽于附近海域。

当岛津义弘部队出现时,全军不得擅自行动,等待其部完全进入露梁海后,方可发动攻击。

攻击发起时,邓子龙部应以最快之速度,截断敌军后退之路,李舜臣部则由观音浦出动,袭击敌军之侧面,打乱敌
军之阵型。

以上两军完成攻击后,须坚守阵地,不惜任何代价,将岛津义弘部堵死于露梁海中,等待陈璘主力到来。

而那时,明军将发动最后的攻击,将侵略者彻底埋葬。

\section[\thesection]{}

一切就绪,李舜臣却发问了:邓子龙堵截后路,我守观音浦,猫岛何人驻守?

这是个很现实的问题,如岛津义弘熬过伏击,坚持向猫岛挺进,就能到达顺天,与小西行长成功会师,局势将一发
不可收拾。

然而陈璘告诉他,猫岛根本无须派兵驻守。

``岛津义弘是不会走这条路的,我肯定。''

在不安与等待中,十八日的夜晚到来。

此时的岛津义弘站在旗舰上,信心十足地向着目的地挺进。之前的泗川之战,虽然他只是侥幸捡个便宜,但毕竟是
胜了,又被人捧为名将,就真把自己当回事了。之所以跑来救小西行长,倒不是他俩关系多好,无非是二杆子精神
大爆发,别人不干,他偏干。

此外,他已认定,明军围困小西行长,必然放松外围的戒备,更想不到日军去而复返,此时进攻,必能一举击溃明
军。

在这个世界上,笨人的第一特征,就是自认为聪明。

事实印证了岛津义弘的猜想,明军以往严加防范的露梁海峡,竟然毫无动静,由一万五千余人组成的日军舰队,就
此大摇大摆地开了进去。

他们中间的大多数人都没能领到回航的船票。

日军的舰队规模很大,共有六百多条船,队列很长,当后军仍在陆续前进之时,前军的岛津义弘已依稀看到了前方
的猫岛。

但他永远不可能到达那里了,因为当最后一条船进入露梁海口的时候,等待已久的邓子龙发动了攻击。

邓子龙手下的这三千兵,大多是浙江人,跟随他从浙江前来此地,虽然名不见经传,却绝非寻常。在五十多年前,
这支队伍有一个更为响亮的名字----俞家军。

在当年那场艰苦卓绝的抗倭之战中,两位大明名将分别创建了专属于自己的军队:戚家军,以及俞家军。

俞大猷熟悉海战,是唯一堪与徐海对敌的明朝海军将领。而他所创建的俞家军,大都从渔民中选取,熟悉水性和流
向,善于驾船,并经过严格训练,多次与倭寇海盗交战,有丰富的战斗经验,堪称明朝最精锐的水军。

经过五十年的淬炼与更替,他们来到了朝鲜,露梁海。

接到进攻命令后,邓子龙部从埋伏处突然驶出,将日军归路堵死,并以十只战船为一组,向日军舰队发起多点突袭。

\section[\thesection]{}

这是一个致命的打击,由于日军队列过长,而且毫无防备,转瞬之间,后部上百条战船已被切成几段,虽然日军人
数占优,却陷入明军分割包围,动弹不得。

包围圈内的日军一片慌乱,他们纷纷拿起武器,准备和跳上船的明军肉搏,然而明军战舰却丝毫不动,保持着诡异
的平静。

日军的疑问没有持续太久,便听到了答案----可怕的轰鸣声。

明军的第二波攻击开始,不用跳帮,不用肉搏,因为在邓子龙的战舰上,装备着一种武器----虎蹲炮。这是一种大
型火炮,射程可达半里,虽然威力一般,炮弹飞个几百米就得掉水里,但近距离内打日军的铁皮木头船,还是绰绰
有余。

就这样,在炮轰、哀嚎、和惨叫声中,日舰队后军损失惨重,基本丧失了作战能力。

当炮声响起的时候,前军的岛津义弘立即意识到,中埋伏了。

但很快,他就显示出了惊人的镇定与沉着,并做出了正确的判断----继续前进。

后军已经深陷重围,敌军兵力不清,所以目前唯一的方法,就是攻击向前,与顺天的小西行长会师。只有这样,才
有反败为胜的可能。

在岛津义弘的指挥下,日军舰队抛弃了后军,不顾一切地向前挺进。

然而,他们没能走多远。

当岛津义弘军刚刚冲出露梁海时,便遭受了第二次致命的打击----李舜臣出现了。

被冷落三年后,李舜臣终于再次成为了水军统领,当他于三个月前上任时,迎接他的,却只有两千多老弱残兵和一
些破烂的船只,因为他的前任元均在战死的同时,还带走了许多水军舰船作为陪葬。

此时,明朝水军尚未到来,日军主帅藤堂高虎率领舰队横扫朝鲜海峡,无人可挡,而李舜臣,什么都没有。

九月十五日,藤堂高虎率四百余条战舰,闯入鸣梁海峡。

李舜臣得知消息后,即刻率少量龟船出战,确切地说,是十二条。这已经是他的全部家当。

四百对十二,于是几乎所有人都认定,虽然李舜臣是少有的水军天才,此战也必败无疑,除非奇迹发生。

但事实告诉我们,奇迹,正是由天才创造的。

战役结局证明,藤堂高虎的水军技术,也就能对付元均这类的废物,经过激战,李舜臣轻松获胜,并击沉四十余艘
敌舰,歼灭日水军三千余人,日军将领波多信时被击毙,藤堂高虎身负重伤,差点被生擒,日军大败,史称鸣梁海
之战。

\section[\thesection]{}

对李舜臣而言,这不过光荣的开始,而露梁海,将是传奇的结束。

当日军舰队出现在视野之中时,他毫不犹豫地下达了攻击令。

此时,岛津义弘的心中正充满期待,他已经看见了前方的猫岛,如此靠近,如此清晰,只要跨过此地,胜利仍将属
于自己。

然后,他就听见了炮声,从他的侧面。

在战场上,军队的侧翼是极其脆弱的。一旦被敌方袭击,很容易被拦腰截断,失去战斗能力,其作用类似于打群架
时被人脑后拍砖,是非常要命的一招。

很明显,龟船比砖头厉害得多。在李舜臣的统一指挥下,这些铁甲乌龟直插日军舰群,几乎不讲任何战术,肆无忌
惮地乱打乱撞。在这突然的打击下,日军指挥系统被彻底搅乱,混作一团,落海丧生者不计其数。

然而,就在这最为混乱的时刻,岛津义弘却并没有慌乱。

作为一位优秀的指挥官,他保持了清醒的意识,在攻击发起的那一刻,他已然确定,敌人来自侧翼。

而他的前方,仍然是一片坦途,很明显,明军并未在此设防。

那就继续前进吧,只要到达顺天,一切都将结束。

按照之前的计划,当邓子龙的第一声炮声响起时,陈璘启航出击。

出于隐蔽的需要,陈璘的军队驻扎在竹岛,这里离露梁海较远,需要行驶一段,才能到达会战地点。

而在此之前,岛津义弘将有足够的时间通过空虚的猫岛海域,成功登陆顺天。

然而陈璘并不着急,因为他知道,那看似无人防守的猫岛,是岛津义弘绝对无法逾越的。

拼死前行的日军舰队终于进入了猫岛海域,然而就在此时,奇怪的事情发生了。

在一片宁静之中,位列前列的三艘战舰突然发出巨响!船只受创起火,两艘被重伤,一艘沉没。

没有敌船,没有炮火,似乎也不是自爆,看着空无一人的水域,岛津义弘第一次对这个世界产生了怀疑----有鬼不
成?!

\section[\thesection]{}

这是一个值得纪念的时刻,在那片看似平静的海面下,一种可怕的武器正式登上历史舞台,它的名字,叫做水雷。

明代水雷,是以木箱为外壳,中间放置火药,根据海水浮力,填充重量不等的重物,以固定其位置,并保持漂浮于
海面之下,以便隐蔽及定位。

当然了,关于这东西,我也就了解这么多。相关细节,如引爆及防水问题本人一概不知,唯一能确定的,就是这玩
意确实能响,能用。

陈璘的自信,正是来源于此。

岛津义弘却依然是满脑浆糊,他的直觉告诉他,这是一个极为危险的地方,如果继续前进,就有全军覆灭的危险,
于是他下令,停止前进。

前行已无可能,绝望的日军只得掉头,向身后那个可怕的敌人发起最后的冲锋。

敌人的回归让李舜臣十分兴奋,他知道,最后的决战即将开始。

在乱军之中,李舜臣亲自擂鼓,率旗舰冲向日军舰群,这一刻,他已盼望了已久。

此时日军虽受重创,但主力尚存,李舜臣竟然孤军冲入敌阵,应该说,他很勇敢。但勇敢的另一个解释,就是愚蠢。

估计是打藤堂高虎之类的废物上了瘾,李舜臣压根就没把日军放在眼里,一路冲进了日军中军。然而岛津义弘用实
际行动证明,作为日本二杆子的优秀代表,他并不白给。

很快,身经百战的岛津水军便理清了头绪,组织五十余条战船,将李舜臣的旗舰围得严严实实,不断用火枪弓箭射
击,虽然龟船十分坚固,也实在扛不住这么个打法,船身多处起火,形势不妙。

眼看李舜臣就要落海喂鱼,陈璘赶到了。

我确信,这两个人之间的交情是很铁的,因为发现李舜臣被围之后,陈璘不等部队列阵,便义无反顾地冲了进去,
而此时他的身边,仅有四条战舰。

于是,他也被围住了。

此时,已是十九日清晨。

无论岛津义弘、陈璘、或是李舜臣,都没有料到,战局竟会如此复杂:明朝联军围住了日军,日军却又围住了明朝
两军主帅,仗打到这个份上,已经成了一团乱麻。

而第一个理出头绪的人,是岛津义弘。

在他的统一调配下,日军开始集中兵力,围攻陈璘和李舜臣的旗舰。

陈璘的处境比李舜臣还要惨,因为他的旗舰不是龟船,也没有铁刺铁钩,几名敢玩命的日军趁人不备,拼死跳了上
来,抽刀直奔陈璘而去。

事发突然,船上的所有人目瞪口呆,来不及做出任何反应,关键时刻,陈璘的儿子陈九经出场了。

\section[\thesection]{}

这位仁兄很是生猛,拼死扑了上去,用自己的身体挡刀,被砍得鲜血淋漓,巍然不动(血淋漓,犹不动)。

明军护卫这才反应过来,一拥而上,把那几名日军乱刀砍死。

惊出一头冷汗的陈璘没有丝毫喜悦,他很清楚,日军包围圈越来越小,跳上来的人会越来越多,援兵到来之前,如
果不玩一招狠的,下个被砍死的,必定是自己。

沉吟片刻后,他做出了一个决定。

很快,奇特的景象出现了,逐渐靠拢的日军惊奇地发现,陈璘的旗舰上竟然看不到任何士兵!船上空空荡荡,无人活
动,十分之安静。

这是十分诡异的一幕,但在头脑简单的日军士兵看来,答案十分简单:陈璘船上的人,已经全部阵亡。

于是他们毫无顾忌,纷纷跳了上去。

然而他们终究看到了明军,在即将着陆的时候。

其实明军一直都在,只不过他们趴在了甲板上。

为了给日军一个深刻的印象和教训,陈璘命令:所有明军一律伏身,并用盾牌盖住自己(挨牌而伏),手持长枪,仰
视上方,当看见从天而降的人时,立即对准目标----出枪。

伴随着凄厉惨叫声,无数士兵被扎成了人串,这一血腥的场景彻底吓住了日军,无人再敢靠近。

趁此机会,圈外的部分明军战舰冲了进来,与陈璘会师,企图攻破包围圈,但日军十分顽固,死战不退,双方陷入
僵持状态。

然而,就在这战斗最为激烈的时刻,陈璘的船上突然响起了鸣金声。

在日军思维中,鸣金,就是不准备打了,可如今大家都在海上,且你中有我,我中有你,没有收兵回营这一说,您
现在鸣金,算怎么回事?

而明军战船在收到这一信号后,却极为一致地停止了攻击,日军不明就里,加上之前吃过大亏,也不敢动,平静又
一次降临了战场。

这正是陈璘所期盼的,因为这一次,他并没有故弄玄虚,之所以鸣金,只因为他需要时间,去准备另一样秘密武器。

他得到了足够的时间。

随即,日军看到了另一幕奇景,无数后部带火的竹筒自明军舰上呼啸而出,重重地击打在自己的船上,所到之处爆
炸起火,浓烟四起,日军舰队陷入一片火海。

这种武器的名字,叫做火龙出水。

\section[\thesection]{}

虽然许多年后,面对拿火枪的英军,手持长矛,目光呆滞的清军几乎毫无抵抗之力,但很多人并不知道,几百年前
的明军,却有着先进的思维、创意,以及登峰造极的火器。

火龙出水,就是明代军事工业最为优秀的杰作。

该武器由竹筒或木筒制成,中间填充火药弹丸,后部装有火药引信,射程可达两百步,专门攻击对方舰船,是明军
水战的专用武器。点燃后尾部带火,在水上滑翔,故称为火龙出水。这也是人类军事史上最早的舰对舰导弹雏形。

什么新玩意都好,反正日军是经不起折腾了,陈璘和李舜臣趁机突围,开始组织追击。

至此,战场的主动权已完全操控在陈璘手中,然而接下来的事情,却出乎他的意料。

在猫岛设下水雷,在观音浦安置伏兵,正如陈璘计划的那样,日军的所有去路被一一切断,与顺天敌人会师的梦想
也彻底破灭,然而他依然疏漏了一点:失败后的敌人,将只有一个选择----撤退。

而撤退的唯一通道,是露梁海。

此时防守露梁海的,是邓子龙,他的手下,只有三千人。

岛津义弘已无任何幻想,他明白自己落入了圈套,此刻唯一的奢望,就是逃离此处。

在这最后的时刻,他用自己的实际行动诠释了穷寇莫追这个成语。遭受重创的日军舰队再次聚拢,不顾一切地向堵
截他们去路的邓子龙水师发动了近乎疯狂的进攻。

明军毕竟人少,在日军的拼死攻击下,防线渐渐不支,行将崩溃。

关键时刻,邓子龙出现了。

他虽然年过七十,却依然挺身而出,率领自己的旗舰,不顾一切地冲入日军船阵,因为这是唯一能够阻拦日军、争
取时间的方法。

邓子龙的战舰成功地吸引了日军的注意,在数十艘日舰的围攻下,邓子龙的船只很快起火燃烧,部下随即请示,希
望邓子龙放弃此船,转乘小艇,暂避他处。

然而邓子龙回答:

``此船即我所守之土,誓死不退!''

然后,他整装正容,在那艘燃烧的战舰上,坚持到人生的最后一刻。

坚守自己的岗位,无论何时、何地。在他看来,这是他应尽的职责。

从军四十余年,一贯如此。

\section[\thesection]{}

终结的决断

邓子龙战死了,他用自己的生命挡住了日军的退路。

在岛津义弘看来,失去将领的明军很快就会被击溃,并乖乖地让开道路。

但是他错了。

此时的明军已不再需要指挥,当他们亲眼目睹那悲壮的一幕,怒火被彻底引燃之时,勇气和愤怒已经成为了最为伟
大的统帅。

在复仇火焰的驱使下,邓子龙的浙兵发动了潮水般的逆袭,日军节节败退,被赶回了露梁海内。

在那里,他们又遇见了分别不久的老朋友:陈璘和李舜臣。

这下热闹了,陈璘军、李舜臣军,再加上退进来的岛津军和追击的邓子龙军,露梁海里布满战舰,可谓是人满为患。

岛津义弘军的末日终于来临,等候已久的陈璘和李舜臣对日舰发动了最后进攻,数百门舰炮猛烈轰鸣,无数日军不
是被炮弹当场炸死,就是跳海当饲料。在刺鼻的硫磺和血腥味中,伴随着燃烧的烈焰,蓝色的露梁海一片赤红。

这就是曾经横行海上,骁勇善战的岛津水军的最后一幕,也是古往今来侵略者的必然结局。

绝望的日军开始了最后的反扑,但已于事无补,在大炮的轰鸣声中,他们都将前往同一个世界。

然而就在最终胜利的时刻即将到来的时候,一个意外发生了。

在战斗中,李舜臣又一次身先士卒,考虑到之前他只有十二条破船就敢打日军四百条战舰,而今正值痛打落水狗,
不表现一把实在说不过去。

但就在他奋勇冲击的时候,一颗子弹飞来,击中了他的胸膛。

这是一件极为匪夷所思的事情,此时明朝联军占尽先机,日军已是强弩之末,一盘散沙,打一枪就得换个地方,基
本属于任人宰割型,行将崩溃。

敌军已被包围,兵力武器占优,士气十分振奋,残敌不堪一击,这就是当时的战况,且李舜臣乘坐龟船,四周都有
铁甲包裹,射击空隙有限,说难听点,就算站出去让人打,都未必能被击中。

然而李舜臣还是中弹了.

\section[\thesection]{}

在这世上,有些事情是说不准的,比如二战时的苏军大将瓦杜丁,自出道以来身经百战,什么恶仗、硬仗、找死仗
都打过。斯大林格勒挺过来了,库尔斯克打赢了,追得德军名将曼斯坦因到处跑,如此猛人,竟然在战役结束,到
地方检查工作的时候,遇上了一帮土匪,腿上挨了一冷枪。按说伤也不重,偏偏就没抢救过来,就这么死了。

李舜臣的情况大致如此。

啥也别说了,总之一句话,这就是命。

身负重伤的李舜臣明白,他的使命即将结束,但这场战役并未终结。

于是,在生命的最后时刻,他对身边的部将李莞留下了这样一句话:

``我就要死了,但现在战况紧急,不要透露我的死讯,请你接替我的位置,以我的名义,继续战斗下去。''

这也是他的最后遗言。

在战场上,唯一的衡量标准就是胜负,因为只有胜利者的故事,才能流传下来。

所以李舜臣依然是幸运的,他虽没能看到胜利的来临,但他的一切都将作为胜利者的传奇传扬万世,正如他所写过
的那首诗句:

全节终须报,成功岂可知?

平生心已定,此外有何辞!

节已报,心已定,便已成功,再有何辞?

伴随着李舜臣的逝去,日军迎来了自己的最后命运,在明朝联军的全力猛攻下,战斗变成了屠杀,日方四百余艘战
舰被击沉,一万余人阵亡,日军惨败。

但要说日军毫无亮点,那也是不客观的。要特别提出表扬的,就是岛津义弘同志,他用实际行动证明,自己的逃跑
本领可谓举世无双,在抛下无数垫背、送死的同胞后,他终于逃了出去,虽然此时他的身边,只剩下了几十余条破
船和几百名士兵。

万历二十六年(1598)十一月十九日中午,历时一天半的露梁海大战正式结束,日军精锐第五军全军覆没,史称露梁
海大捷。

露梁海大捷后,翘首期盼的小西行长部终于彻底崩溃,纷纷化整为零,四散奔逃,小西行长不落人后,率残部趁明
军不备,乘船偷渡出海,经过千辛万苦逃回日本,余部大部被歼。

至此,抗倭援朝战争正式结束,此战历时七年,最终,以中国军队的彻底胜利,以及日本军队的彻底失败而告终。

七年前,那杯由邪恶与野心酿成的苦酒,最终浇到丰臣秀吉的坟头上。

活该,死了也该。

正义终究战胜了邪恶,无论此时,或是三百四十年后,历史都用事实告诉了我们相同的道理:

无论何时何地,总会有那么几个不安分的侵略者,他们或许残暴,或许强大,或许看似不可战胜,但终将被埋葬。

\section[\thesection]{}

战争结束了,胜利也好,失败也罢,参战的主角们都有了各自的结局。

两年后(1600),超级``忍者''德川家康终于发作,集结兵力,准备欺负丰臣秀吉的孤儿寡妇,死硬派小西行长当即
联同石田三成等人,组成西军,出兵迎战。

但滑稽的是,出于对小西行长、石田三成的极度憎恨,作为丰臣秀吉的铁杆亲信,加藤清正、福岛正则等人当机立
断,放下与德川家康之间的敌我矛盾,毅然投入到轰轰烈烈的内部矛盾中去,加入东军,跟小西行长玩命。

而最搞笑的,莫过于岛津义弘,此人和丰臣秀吉关系本就不好,开战之初是德川家康的人,并奉命去帮助守城。结
果城里的人未接通报,以为他是敌人派来忽悠的,不但没有开门,还对他放了几枪。

换了别人,无非是回去找德川家康告一状,之后该干嘛还干嘛,可这位就不同了,二杆子精神再起,操着家伙连夜
投小西行长去也。

经过你来我往数个回合,这一大帮子人终于在日本关原碰上了,展开死磕,经过一天战斗,西军败退,小西行长战
败后逃走,后又被擒获斩首,岛津义弘还是一如既往地跑了路,后来托人求情捡了一条命。

丰臣秀吉创立的事业就此完结。

但历史的惩罚并未结束,十五年后(1615),战火再起,在大阪夏季战役中,德川家康攻克了丰臣家的最后据点大阪
城,丰臣秀吉的老婆孩子都死在城里,丰臣家族灭亡,断子绝孙。

我不是报应论者,但这一次,我信。

此后,德川家康统一日本,并建立了著名的德川幕府,他着力与明朝恢复友好关系,发展经济,颇有建树。

朝鲜失去了李舜臣,却迎来了和平,回复了平静的生活,为纪念那些为了朝鲜人民的安宁和自由而牺牲的明军将
士,朝鲜政府修建了大报坛,每年祭祀,以表示对明朝仗义相助的感激,并提醒后辈不忘报恩。

现在,大报坛已经消失了,为什么消失,我不知道。

明朝的大军得胜归来,万历并没有亏待他们,将领之中,麻贵升任右都督,陈璘和刘綎也升了官。

\section[\thesection]{}

当兵的也没白干,为表彰群众,据说万历从国库里拨出了八万两白银,作为对士兵的封赏,当然,具体到每个人的
头上,一层扒一层,外加还有陈璘这样的领导,能分到多少,那就不好说了。但无论如何,也算够意思了。

虽然在七年之中,曾有过无数的曲折,遇上许多的困难,付出了相当的代价,但这一切都是值得的。

因为打赢了。

所谓正义、邪恶、侵略、暴行,大多时候都是毫无意义的胡扯,衡量战争的唯一且永远的准则,就是胜利,或失败。

用黑暗的暴力维护了光明的正义,这正是明朝创立的不朽功勋。

这场战争的最后结局大致如此,十分清楚,但有趣的是,几百年后,历史对于这场战争的评价,却十分之不清楚。

具体说来是这样的:日本的史料表示,这是一场延续了战国光荣以及名将光辉的战争,虽然未必光彩(这一点,他们
是承认的)。

朝鲜(韩国)的史料则认为,这场战争之所以胜利,主要是因为李舜臣和朝鲜义军(无奈,政府军的表现实在太差),
至于其他方面的因素,当然是有的,但似乎也是比较次要的。

而明朝方面……,基本没什么动静。

现象是奇怪的,但原因是简单的,因为在明朝看来,这场战争,压根就不是什么大事。

这是千真万确的事实,所谓的抗倭援朝战争,在史学界实在不算个啥,也没听说哪位专家靠研究这事出了名,即使
在明代,它也只是万历三大征的一部分而已,史料也不算多,除了《万历三大征考》还算是马马虎虎外,许多细节
只能从日本和朝鲜史料中找。

说起来,也只能怪我国地大物博,什么事都有,什么人都出,就规模而言,这场战争确实不值一提,打了七年,从
头到尾,明军的总人数不过四万左右,直到最后一年,才勉强增兵至八万,且打两个月就收了场,架势并不算大。

而日本为了打这场仗,什么名将精兵之类的老本全都押上去了,十几万人拉到朝鲜,死光了再填,打到后来,国内
农民不够,竟然四处抓朝鲜人回去种田,实在是顶不住了。

朝鲜更不用说,被打得束手无策,奄奄一息,差点被人给灭了,国王都准备外出避难,苦难深重,自然印象深刻。

相比而言,日本是拼了老命,朝鲜是差点没命,而明朝却全然没有玩命的架势,派几万人出国,军费粮食自己掏腰
包,就把日本办挺了,事后连战争赔款都没要(估计日本也没钱给)。

什么叫强大?这就叫强大。

\section[\thesection]{}

在进行这场战争的同时,明朝还调兵十余万,围剿四川方向的杨应龙叛乱,在万历同志看来,这位叫杨应龙的土财
主(土司),比丰臣秀吉的威胁更大。

基于以上理由,在宣传方面,明朝也是相当落后。战争结束后,在日本,明明表现不咋样的加藤清正、岛津义弘都
被捧上了天,所谓``虎加藤''、``鬼石曼子''一波接一波的吹,从没消停过。

朝鲜方面,货真价实的李舜臣自不必说,死后被封公爵,几百年下来,能加的荣誉都加了,成为了家喻户晓的民族
英雄。

至于明朝,对相关人员的处理,大致是这样的:

战后,刘綎、陈璘任职都督同知(从一品),算是升了半级。当然,也不是白升的,几个月后,这二位仁兄就被调去
四川播州的穷山恶水,因为在那里,还有个杨应龙等着他们去收拾。

英勇献身的邓子龙也得到了封赏,他被追赐为都督佥事(从二品),并得到了一个世袭职位,给儿子找了个铁饭碗。

仅此而已。

但和李如松比起来,以上的几位就算不错了。这位仁兄智勇双全、能征善战,几乎以一己之力挽救了朝鲜战局,是
朝鲜战争中最为杰出的军事天才。

可这位盖世英雄,死后不但没人捧,还差点被口水淹死第二遍。

说到底,都是言官惹的祸。

明代是一个开明的朝代,言官可以任意发言,批评皇帝,弹劾大臣,用今天的话说,就是民主。

可是民主过了头,就有问题了,发展到万历年间,言官们已经是无所不骂,坏人要骂,好人也要骂,不干事的要
骂,干事的也要骂,且职位越高,权力越大,骂得就越响。

而李成梁十分符合这个条件,这位兄弟镇守边界数十年,权大势大,是最好的目标,外加他亏空贪污之类的事情也
没少干,下台之后自然不招人待见,弹章堆得和山一样高,说什么的都有。

李如松自然也未能幸免,加上他在朝鲜风光一时,功勋卓著,就成了连带打击对象。最恶心人的是御史丁应泰,不
但攻击他本人,连他的战绩也要骂,说平壤战役是小胜,日军死伤极少,碧蹄馆之战是大败,明军死伤极多。

这还不算,他居然检举朝鲜与日本串通,说李如松也有通倭嫌疑。

\section[\thesection]{}

要按照他的说法和算法,明军的士兵估计都是死后从坟里刨出来的(一共也就四、五万人),日军都是拿白鸽的和平
使者(死伤不多,就是要逃)。李如松应该算是双面间谍,明明和日军勾结,偏偏还把日军赶跑了。

这人不但无耻,还很无聊,弹劾一封接着一封,闹到最后,连不爱搭理人的万历也忍不住了,直接给他下了个革职
令让他滚蛋。

然而,从根本上讲,封赏过少,弹劾过多的责任者并不是丁应泰,更不是万历,因为按照明朝的惯例和规定,像抗
倭援朝这种规模的战役,带几万人出去打一场,封赏就这么少,弹劾就这么多,大家都习惯了。

所以真正的原因虽然可笑,却很真实:

对明朝而言,这实在不是个太大的事。

既然不是什么大事,自然就没人管,自己不管别人当然也不管,加上那些无聊的言官泼脏水,修明史的清代史官照
单全收,日本和朝鲜史料又站在自己的立场上各说各话,于是,对这场战争的评价,就变成了现在的这个样子:争
议、误解、谜团。

然而无论大小,历史上确实存在过这样一件事情:

四百多年前,有一群人为了摧垮贪欲和邪恶,远赴他乡,进行过一场伟大的战争,在这场惊心动魄的较量里,他们
中的许多人,为此献出了自己的一切。

所以我认为,我们应该知道这一切,知道有这样一场战争,有这样一群人,曾为了捍卫自由与正义,英勇奋战,毫
无畏惧。为了那些无比的智慧,无畏的勇气,以及无私的牺牲。

万历二十七年(1599)四月,征倭总兵麻贵率军凯旋归来,明神宗在午门接见了他。

在搞完大大小小不厌其烦的程序仪式后,明神宗下旨,当众宣读大明诏书,通传天下,宣告抗倭援朝之役就此结束。

这是一封诏书,也是一个预言,因为在这份长篇大论之中,有这样一句话:

义武奋扬,跳梁者,虽强必戮!

绝顶的官僚

在万历执政的前二十多年里,可谓是内忧不止,外患不断,他祖上留传下来的,也只能算是个烂摊子,而蒙古、宁
夏、朝鲜、四川,不是叛乱就是入侵,中间连口气都不喘,军费激增,国库难支。

可是二十年了,国家也没出什么大乱子,所有的困难,他都安然度过。

因为前十年,他有张居正,后十年,他有申时行。

\section[\thesection]{}

若评选明代三百年历史中最杰出的政治家,排行榜第一名非张居正莫属。在他当政的十年里,政治得以整顿,经济
得到恢复,明代头号政治家的称谓实至名归。

  但如果评选最杰出的官僚,结果就大不相同了,以张居正的实力,只能排第三。

  因为这两个行业是有区别的。

  从根本上讲,明代政治家和官僚是同一品种,大家都是在朝廷里混的,先装孙子再当爷爷,半斤对八两。但问
题在于,明代政治家是理想主义者,混出来后就要干事,要实现当年的抱负。

  而明代官僚是实用主义者,先保证自己的身份地位,能干就干,不能干就混。

  所以说,明代政治家都是官僚,官僚却未必都是政治家。两个行业的技术含量和评定指标各不相同,政治家要
能干,官僚要能混。

  张居正政务干得好,且老奸巨滑,工于心计,一路做到首辅,混得也还不错。但他死节不保,死后被抄全家,
差点被人刨出来示众,所以只能排第三。

明代三百年中,在这行里,真正达到登峰造极的水平,混到惊天地、泣鬼神的,当属张居正的老师,徐阶。

混迹朝廷四十多年,当过宰相培训班学员(庶吉士),骂过首辅(张璁),发配地方挂职(延平推官),好不容易回来,
靠山又没了(夏言),十几年被人又踩又坑,无怨无悔,看准时机,一锤定音,搞定(严嵩)。

上台之后,打击有威胁的人(高拱),提拔有希望的人(张居正),连皇帝也要看他的脸色,事情都安排好了,才安然
回家欢度晚年,活到了八十一岁,张居正死了他都没死,如此人精,排第一是众望所归。

而排第二的,就是张居正的亲信兼助手:申时行。

  相信很多人并不认同这个结论,因为在明代众多人物中,申时行并不是个引人瞩目的角色,但事实上,在官僚
这行里,他是一位身负绝学,超级能混的绝顶高手。

  无人知晓,只因隐藏于黑暗之中。

在成为绝顶官僚之前,申时行是一个来历不明的人,具体点讲,是身世不清,父母姓甚名谁,家族何地,史料上一
点儿没有,据说连户口都缺,基本属于黑户。

\section[\thesection]{}

  申时行是一个十分谨小慎微的人,平时有记日记的习惯。即使是微不足道的小事,如今天我和谁说了话,讲了
啥,他都要记下来,比如他留下的《召对录》,就是这一类型的著作。

  此外,他也喜欢写文章,并有文集流传后世。

  基于其钻牛角尖的精神,他的记载是研究明史的重要资料。然而奇怪的是,对于自己的身世,这位老兄却是只
字不提。

  这是一件比较奇怪的事,而我是一个好奇的人,于是,我查了这件事。

  遗憾的是,虽然我读过很多史书,也翻了很多资料,依然没能找到史料确凿的说法。

  确凿的定论没有,不确凿的传言倒有一个,而在我看来,这个传言可以解释以上的疑问。

  据说(注意前提)嘉靖十四年时,有一位姓申的富商到苏州游玩,遇上了一位女子,两人一见钟情,便住在了一
起。

  过了一段时间,女方怀孕了,并把孩子生了下来,这个孩子,就是后来的申时行。

  可是在当时,这个孩子不能随父亲姓申,因为申先生有老婆。

  当然了,在那万恶的旧社会,这似乎也不是什么违法行为,以申先生的家产,娶几个老婆也养得起,然而还有
一个更麻烦的问题----那位女子不是一般人,确切地说,是一个尼姑。

  所以,在百般无奈之下,这个见不得光的私生子被送给了别人。

  爹娘都没见过,就被别人领养,这么个身世,确实比较不幸。

  但不幸中的万幸是,这个别人,倒也并非普通人,而是当时的苏州知府徐尚珍。他很喜欢这个孩子,并给他取
了一个名字----徐时行。

  虽然当时徐知府已离职,但在苏州干过知府,只要不是海瑞,一般都不会穷。

  所以徐时行的童年非常幸福,从小就不缺钱花,丰衣足食,家教良好。而他本人悟性也很高、天资聪慧,二十
多岁就考上了举人,人生对他而言,顺利得不见一丝波澜。

  但惊涛骇浪终究还是来了。

  嘉靖四十一年(1562),徐时行二十八岁,即将上京参加会试,开始他一生的传奇。

  然而就在他动身前夜,徐尚珍找到了他,对他说了这样一句话:

  其实,你不是我的儿子。

  没等徐时行的嘴合上,他已把之前所有的一切都和盘托出,包括他的生父和生母。

  这是一个十分古怪的举动。

\section[\thesection]{}

按照现在的经验,但凡考试之前,即使平日怒目相向,这时家长也得说几句好话,天大的事情考完再说,徐知府偏
偏选择这个时候开口,实在让人费解。

然而我理解了。

就从现在开始吧,因为在你的前方,将有更多艰难的事情在等待着你,到那时,你唯一能依靠的人,只有你自己。

这是一个父亲,对即将走上人生道路的儿子的最后祝福。

徐时行沉默地上路了。我相信,他应该也是明白的,因为在那一年会试中,他是状元。

中了状元的徐时行回到了老家,真相已明,恩情犹在,所以他正式提出要求,希望能够归入徐家。

辛苦养育二十多年,而今状元及第,衣锦还乡,再认父母,收获的时候到了。

然而出乎所有人的意料,他的父亲拒绝了这个请求,希望他回归本家,认祖归宗。

很明显,在这位父亲的心中,只有付出,没有收获。

无奈之下,徐时行只得怀着无比的歉疚与感动,回到了申家。

天上终于掉馅饼了,状元竟然都有白捡的。虽说此时他的生父已经去世,但申家的人毫不犹豫地答应了他的请求,
敲锣打鼓,张灯结彩地把他迎进了家门。

从此,他的名字叫做申时行。

曲折的身世,幸福的童年,从他的养父身上,申时行获取了人生中的第一个重要经验,并由此奠定了他性格的主要
特点:

做人,要厚道。

然后当厚道的申时行进入朝廷后,才发现原来这里的大多数人都很不厚道。

在明代,只要进了翰林院,只要不犯什么严重的政治错误,几年之后,运气好的就能分配到中央各部熬资格,有才
的入阁当大学士,没才的也能混个侍郎、郎中,就算点背,派到了地方,官也升得极快,十几年下来,做个地方大
员也不难。

有鉴于此,每年的庶吉士都是各派政治势力极力拉拢的对象。申时行的同学里,但凡机灵点的,都已经找到了后
台,为锦绣前程做好准备。

申时行是状元,找他的人自然络绎不绝,可这位老兄却是巍然不动,谁拉都不去,每天埋头读书,毫不顾及将来的
仕途。同学们一致公认,申时行同志很老实,而从某个角度讲,所谓老实,就是傻。

然而事情的发展证明,老实人终究不吃亏。

\section[\thesection]{}

要知道,那几年朝廷是不好混的,先是徐阶斗严嵩,过几年,高拱上来斗徐阶,然后张居正又出来斗高拱,总而言
之是一塌糊涂。今天是七品言官,明天升五品郎中,后天没准就回家种田去了。

你方唱罢我登场,上台洗牌是家常便饭,世事无常,跟着谁都不靠谱,所以谁也不跟的申时行笑到了最后。当他的
同学纷纷投身朝廷拼杀的时候,他却始终呆在翰林院,先当修撰,再当左庶子。中间除了读书写文件外,还主持过
几次讲学(经筵),教过一个学生,叫做朱翊钧,又称万历。

俗语有云,长江后浪推前浪,前浪死在沙滩上。一晃十年过去,经过无数清洗,到万历元年,嘉靖四十一年的这拨
人,冲在前面的,基本上都废了。

就在此时,一个人站到了申时行的面前,对他说,跟着我走。

这一次,申时行不再沉默,他同意了。

因为这个人是张居正。

申时行很老实,但不傻。这十年里,他一直在观察,观察最强大的势力,最稳当的后台,现在,他终于等到了。

此后他跟随张居正,一路高歌猛进,几年内就升到了副部级礼部侍郎,万历五年(1577),他又当上了吏部侍郎,一
年后,他迎来了自己人生的第二个转折点。

万历六年(1578),张居正的爹死了,虽说他已经获准夺情,但也得回家埋老爹。为保证大权在握,他推举年仅四十
三岁的申时行进入内阁,任东阁大学士。

历经十几年的苦熬,申时行终于进入了大明帝国的最高决策层。

但是当他进入内阁后,他才发现,自己在这里只起一个作用----凑数。

因为内阁的首辅是张居正,这位仁兄不但能力强,脾气也大,平时飞扬跋扈,是不折不扣的猛人。

一般说来,在猛人的身边,只有两个选择,要么当敌人,要么当仆人。

申时行毫不犹豫地选择了后者,他很明白,像张居正这种狠角色,只喜欢一种人----听话的人。

申时行够意思,张居正也不含糊,三年之内,就把他提为吏部尚书兼建极殿大学士,少傅兼太子太傅(从一品)。

但在此时的内阁里,申时行还只是个小字辈,张居正且不说,他前头还有张四维、马自强、吕调阳,一个个排过
去,才能轮到他。距离那个最高的位置,依然是遥不可及。

申时行倒也无所谓,他已经等了二十年,不在乎再等十年。

\section[\thesection]{}

可他万万没有想到,不用等十年,一年都不用。

万历十年(1582)张居正死了。

树倒猢狲散。隐忍多年的张四维接班,开始反攻倒算,重新洗牌,局势对申时行很不利,因为地球人都知道他是张
居正的亲信。

在这关键时刻,申时行第一次展现了他无与伦比的``混功''。

作为内阁大学士,大家弹劾张居正,他不说话;皇帝下诏剥夺张居正的职务,他不说话;抄张居正的家,他也不说
话。

但不说话,不等于不管。

申时行是讲义气的,抄家抄出人命后,他立即上书,制止情况进一步恶化。还分了一套房子,十倾地,用来供养张
居正的家属。

此后,他又不动声色地四处找人做工作,最终避免了张先生被人从坟里刨出来示众。

张四维明知申时行不地道,偏偏拿他没办法。因为此人办事一向是滴水不漏,左右逢源,任何把柄都抓不到。

但既然已接任首辅,收拾个把人应该也不太难,在张四维看来,他有很多时间。

然而事与愿违,张首辅还没来得及下手,就得到了一个消息----他的父亲死了。

死了爹,就得丁忧回家,张四维不愿意。当然,不走倒也可以,夺情就行,但五年前张居正夺情的场景还历历在目。
考虑到自己的实力远不如张居正,且不想被人骂死,张四维毅然决定,回家蹲守。

三年后,又是一条好汉。

此时,老资格的吕调阳和马自强都走了,申时行奉命代理首辅,等张四维回来。

一晃两年半过去了,眼看张先生就要功德圆满,胜利出关,却突然病倒了。病了还不算,两个月后,竟然病死了。

上级都死光了,进入官场二十三年后,厚道的老好人申时行,终于超越了他的所有同学,走上了首辅的高位。

一个新的时代,将在他的手中开始。

取胜之道

就工作能力而言,申时行是十分卓越的,虽说比张居正还差那么一截,但在他的时代,却是最为杰出的牛人。

因为要当牛人,其实不难,只要比你牛的人死光了,你就是最牛的牛人。

就好比你上世纪三十年代和鲁迅见过面,给胡适鞠过躬,哪怕就是个半吊子,啥都不精,只要等有学问、知道你底
细的那拨人都死绝了,也能弄顶国学大师的帽子戴戴。

\section[\thesection]{}

更何况申时行所面对的局面,比张居正时要好得多:首先他是皇帝的老师,万历也十分欣赏这位新首辅;其次,他
很会做人,平时人缘也好,许多大臣都拥戴他;加上此时他位极人臣,当上了大领导,一切似乎都在他的掌握之中。

不过,只是似乎而已。

所谓朝廷,就是江湖。即使身居高位,扫平天下,也绝不会缺少对手。因为在这个地方,什么都会缺,就是不缺敌
人。

张四维死了,但一个更为强大的敌人,已经出现在他的面前。

而这个敌人,是万历一手造就的。

张居正死后,万历得到了彻底的解放。没人敢管他,也没人能管他,所有权力终于回到他的手中。他准备按自己的
意愿去管理这个帝国。

但在此之前,他还必须做一件事。

按照传统,打倒一个人是不够的,必须把他彻底搞臭,消除其一切影响,才算是善莫大焉。

于是,一场批判张居正的活动就此轰轰烈烈展开。

张居正在世的时候,吃亏最大的是言官。不是罢官,就是打屁股,日子很不好过,现在时移势易,第一个跳出来的
自然也就是这些人。

万历十二年(1584)三月,御史丁此吕首先发难,攻击张居正之子张嗣修当年科举中第,是走后门的关系户云云。

这是一次极端无聊的弹劾,因为张嗣修中第,已经是猴年马月的事,而张居正死后,他已被发配到边远山区充军。
都折腾到这份上了,还要追究考试问题,是典型的没事找事。

然而事情并非看上去那么简单,事实上,这是一个设计周密的阴谋。

丁此吕虽说没事干,却并非没脑子,他十分敏锐地察觉到,只要对张居正问题穷追猛打,就能得到皇帝的宠信,

这一举动还有另一个更阴险的企图:当年录取张嗣修的主考官,正是今天的首辅申时行。

也就是说,打击张嗣修,不但可以获取皇帝的宠信,还能顺道收拾申时行,把他拉下水,一箭双雕,十分狠毒。

血雨腥风就此而起。

申时行很快判断出了对方的意图,他立即上书为自己辩解,说考卷都是密封的,只有编号,没有姓名,根本无法舞
弊。

万历支持了他的老师,命令将丁此吕降职调任外地,大家都松了一口气。

然而这道谕令的下达,才是暴风雨的真正开端。

\section[\thesection]{}

明代的言官中,固然有杨继盛那样的孤胆英雄,但大多数情况下,都是团伙作案。一个成功言官的背后,总有一拨
言官。

丁此吕失败了,于是幕后黑手出场了,合计三双。

这三个人的名字,分别是李值、江东之,羊可立。在我看来,这三位仁兄是名副其实的``骂仗铁三角''。

之所以给予这个荣誉称号,是因为他们不但能骂,还很铁。

李、江、羊三人,都是万历五年(1577)的进士。原本倒也不熟,自从当了御史后,因为共同的兴趣和事业(骂人)走
到了一起,在战斗中建立了深厚的友谊,并成为了新一代的搅屎棍。

之所以说新一代,是因为在他们之前,也曾出过三个极能闹腾的人,即大名鼎鼎的刘台、赵用贤、吴中行。这三位
仁兄,当年曾把张居正老师折腾得只剩半条命,十分凑巧的是,他们都是隆庆(1571)五年的进士,算是老一代的铁
三角。

但这三个老同志都还算厚道人,大家都捧张居正,他们偏骂,这叫义愤。后来的三位,大家都不骂了,他们还骂,
这叫投机。

丁此吕的奏疏刚被打回来,李植就冲了上去,枪口直指内阁的申时行。还把管事的吏部尚书杨巍搭了上去,说这位
人事部长逢迎内阁,贬低言官。

话音没落,江东之和羊可立就上书附和,一群言官也跟着凑热闹,舆论顿时沸沸扬扬。

对于这些举动,申时行起先并不在意:丁此吕已经滚蛋了,你们去闹吧,还能咋地?

然而出人意料的事情发生了。几天以后,万历下达了第二道谕令,命令丁此吕留任,并免除应天主考高启愚(负责出
考题)的职务。

这是一个十分危险的政治信号。

其实申时行并不知道,对于张居正,万历的感觉不是恨,而是痛恨。这位曾经的张老师,不但是一个可恶的夺权
者,还是笼罩在他心头上的恐怖阴影。

支持张居正的,他就反对,反对张居正的,他就支持!无论何人、何时、何种动机。

这才是万历的真正心声,上次赶走丁此吕,不过是给申老师一个面子,现在面子都给过了,该怎么来,咱还怎么来。

申时行明白,大祸就要临头了:今天解决出考题的,明天收拾监考的,杀鸡儆猴的把戏并不新鲜。

\section[\thesection]{}

情况十分紧急,但在这关键时刻,申时行却表现出了让人不解的态度,他并不发文反驳,对于三位御史的攻击,保
持了耐人寻味的沉默。

几天之后,他终于上疏,却并非辨论文书,而是辞职信。

就在同一天,内阁大学士许国、吏部尚书杨巍同时提出辞呈,希望回家种田。

这招以退为进十分厉害,刑部尚书潘季驯、户部尚书王璘、左都御史赵锦等十余位部级领导纷纷上疏,挽留申时行。
万历同志也手忙脚乱,虽然他很想支持三位骂人干将,把张居正整顿到底,但为维护安定团结,拉人干活,只得再
次发出谕令,挽留申时行等人,不接受辞职。

这道谕令有两个意思,首先是安慰申时行,说这事我也不谈了,你也别走了,老实干活吧。

此外,是告诉江、羊、李三人,这事你们干得不错,深得我心(否则早就打屁股了),但到此为止,以后再说。

事情就此告一段落,然而之后的发展告诉了我们,这一切,只不过是热身运动。

问题的根源,在于``铁三角''。科场舞弊事件完结后,这三位拍对了马屁的仁兄都升了官:江东之升任光禄寺少
卿,李植任太仆寺少卿,羊可立为尚宝司少卿。

太仆寺少卿是管养马的,算是助理弼马温,正四品。光禄寺少卿管吃饭宴请,是个肥差,正五品。尚宝司少卿管公
章文件,是机要部门,从五品。

换句话说,这三个官各有各的好处,却并不大,可见万历同志心里有谱:给你们安排好工作,小事来帮忙,大事别
掺和。

这三位兄弟悟性不高,没明白其中的含义,给点颜色就准备开染坊。虽然职务不高,权力不大,却都很有追求,可
谓是手攥两块钱,心怀五百万,欢欣鼓舞之余,准备接着干。

而这一次,他们吸取了上次的教训,打算捏软柿子,将矛头对准了另一个目标----潘季驯。

可怜潘季驯同志,其实他并不是申时行的人。说到底,不过是个搞水利的技术员,高拱在时,他干,张居正在时,
他也干,是个标准的老好人,无非是看不过去,说了几句公道话,就成了打击对象。

话虽如此,但此人一向人缘不错,又属于特殊科技人才,还干着司法部部长(刑部尚书),不是那么容易搞定的。

可是李植只用了一封奏疏,就彻底终结了他。

\section[\thesection]{}

这封奏疏彻底证明了李先生的厚黑水平,非但绝口不提申时行,连潘技术员本人都不骂。只说了两件事----张居正
当政时,潘季驯和他关系亲密,经常走动,张居正死后抄家,他曾几次上书说情。

这就够了。

申时行的亲信,不要紧;个人问题,不要紧;张居正的同伙,就要命了。

没过多久,兢兢业业的潘师傅就被革去所有职务,从部长一踩到底,回家当了老百姓。

这件事干得实在太过龌龊,许多言官也看不下去了。御史董子行和李栋分别上书,为潘季驯求情,却被万历驳回,
还罚了一年工资。

有皇帝撑腰,``铁三角''越发肆无忌惮,把战火直接烧到了内阁的身上,而且下手也特别狠,明的暗的都来。先是
写匿名信,说大学士许国安排人手,准备修理李植、江东之。之后又明目张胆地弹劾申时行的亲信,不断发起挑衅。

部长垮台,首辅被整,闹到这个份上,已经是人人自危,鬼才知道下个倒霉的是谁。连江东之当年的好友,刑科给
事中刘尚志也憋不住了,站出来大吼一声:

``你们要把当年和张居正共事过的人全都赶走,才肯干休吗(尽行罢斥而后已乎)?!''

然而让人费解的是,在这片狂风骤雨之中,有一个人却始终保持着沉默。

面对漫天阴云,申时行十分之镇定,既不吵,也不闹,怡然自得。

这事要换在张居正头上,那可就了不得了。以这位仁兄的脾气,免不了先回骂两句,然后亲自上阵,罢官、打屁
股,搞批判,不搞臭搞倒誓不罢休。刘台、赵用贤等人,就是先进典型。

就能力与天赋而言,申时行不如张居正,但在这方面,他却远远地超越了张先生。

申首辅很清楚,张居正是一个不折不扣的政务天才。而像刘台、江东之这类人,除了嘴皮子利索,口水旺盛外,干
工作也就是个白痴水平。和他们去较真,那是要倒霉的,因为这帮人会把对手拉进他们的档次,并凭借自己在白痴
水平长期的工作经验,战胜敌人。

所以在他看来,李植、江东之这类人,不过是跳梁小丑,并无致命威胁,无须等待多久,他们就将露出破绽。

所谓宽宏大量,胸怀宽广之外,只因对手档次太低。

\section[\thesection]{}

然而``铁三角''似乎没有这个觉悟,万历十三年(1585)八月,他们再一次发动了进攻。

事情是这样的,为了给万历修建陵墓,申时行前往大峪山监督施工,本打算打地基,结果挖出了石头。

在今天看来,这实在不算个事,把石头弄走就行了。可在当时,这就是个掉脑袋的事。

皇帝的陵寝,都是精心挑选的风水宝地,要保证皇帝大人死后,也得躺得舒坦,竟然挑了这么块石头地,存心不让
皇上好好死,是何居心?

罪名有了,可申时行毕竟只是监工,要把他拉下水,必须要接着想办法。

经过一番打探,办法找到了:原来这块地是礼部尚书徐学谟挑的,这个人不但是申时行的亲家,还是同乡。很明
显,他选择这块破地,给皇上找麻烦,是有企图的,是用心不良的,是受到指使的。

只要咬死两人的关系,就能把申时行彻底拖下水。而这帮野心极大的人,也早已物色好了首辅的继任者,只要申时
行被弹劾下台,就立即推荐此人上台,并借此控制朝局,这就是他们的计划。

然而这个看似万无一失的计划,却有两个致命的破绽。

几天之后,三人同时上疏,弹劾陵墓用地选得极差,申时行玩忽职守,任用私人,言辞十分激烈。

在规模空前的攻击面前,申时行却毫不慌张,只是随意上了封奏疏说明情况,因为他知道,这帮人很快就要倒霉了。

一天之后,万历下文回复:

``阁臣(指申时行)是辅佐政务的,你们以为是风水先生吗(岂责以堪舆)!?''

怒火中烧的万历骂完之后,又下令三人罚俸半年,以观后效。

三个人被彻底打懵了,他们抓破脑袋,也想不明白这是怎么回事。

归根结底,还是信息工作没有到位。这几位仁兄晃来晃去,只知道找地的是徐学谟,却不知道拍板定位置的,是万
历。

皇帝大人好不容易亲自出手挑块地,却被他们骂得一无是处,不出口气实在说不过去。

不过还好,毕竟算是皇帝的人,只是罚了半年的工资,励精图治,改日再整。

可还没等这三位继续前进,背后却又挨了一枪。

\section[\thesection]{}

在此之前,为了确定申时行的接班人选,三个人很是费了一番脑筋,反复讨论,最终拍板----王锡爵。

这位王先生,之前也曾出过场。张居正夺情的时候,上门逼宫,差点把张大人搞得横刀自尽,是张居正的死对头,
加上他还是李植的老师,没有更适合的人选了。

看上去是那么回事,可惜有两点,他们不知道:其一,王锡爵是个很正派的人,他不喜欢张居正,却并非张居正的
敌人。

其二,王锡爵是嘉靖四十一年进士,考试前就认识了老乡申时行,会试,他考第一,申时行考第二,殿试,他考第
二,申时行第一。

没有调查研究,就没有发言权----毛泽东

基于以上两点,得知自己被推荐接替申时行之后,王锡爵递交了辞职信。

这是一封著名的辞职信,全称为《因事抗言求去疏》,并提出了辞职的具体理由:

老师不能管教学生,就该走人(当去)!

这下子全完了,这帮人虽说德行不好,但毕竟咬人在行,万历原打算教训他们一下后,该怎么样还怎么样。

可这仨太不争气,得罪了内阁、得罪了同僚,连自己的老师都反了水,再这么闹腾,没准自己都得搭进去,于是他
下令,江东之、李植、羊可立各降三级,发配外地。

家犬就这么变成了丧家犬,不动声色之间,申时行获得了最终的胜利。

和稀泥的艺术

对申时行而言,江东之这一类人实在是小菜一碟。在朝廷里呆了二十多年,徐阶、张居正这样的超级大腕他都应付
过去了,混功已达出神入化的地步,万历五年出山的这帮小喽罗自然不在话下。

混是一种生活技巧,除个别二杆子外,全世界人民基本都会混。因为混并不影响社会进步,人类发展,该混就混,
该干就干,只混不干的,叫做混混。

申时行不是混混,混只是他的手段,干才是他的目的。

一般说来,新官上任,总要烧三把火,搞点政绩,大干特干,然而综观申时行当政以来的种种表现,就会惊奇地发
现,他的大干,就是不干。他的作为,就是不作为。

申时行干的第一件事情,是废除张居正的考成法。

这是极为出人意料的一招,因为在很多人看来,申时行是张居正的嫡系,毫无理由反攻倒算。

\section[\thesection]{}

但申时行就这么干了,因为这样干,是正确的。

考成法,是张居正改革的主要内容,工作指标层层落实,完不成轻则罢官,重则坐牢,令各级官员威风丧胆。

在很长时间里,这种明代的打考勤,发挥了极大效用,有效提高了官员的工作效率,是张居正的得意之作。

但张先生并不知道,这种考成法,有一个十分严重的缺陷。

比如朝廷规定,户部今年要收一百万两税银,分配到浙江,是三十万,这事就会下派给户部浙江司郎中(正五品),
由其监督执行。

浙江司接到命令,就会督促浙江巡抚办理。巡抚大人就会去找浙江布政使,限期收齐。

浙江布政使当然不会闲着,立马召集各级知府,限期收齐。知府大人回去之后召集各级知县,限期收齐。

知县大人虽然官小,也不会自己动手,回衙门召集衙役,限期收齐。

最后干活的,就是衙役,他们就没办法了,只能一家一家上门收税。

明朝成立以来,大致都是这么个办法,就管理学而言,还算比较合理,搞了两百多年,也没出什么大问题。

考成法一出来,事情就麻烦了。

原先中央下达命令,地方执行,就算执行不了,也好商量。三年一考核,灾荒大,刁民多,今年收不齐,不要紧,
政策灵活掌握,明年努力,接着好好干。

考成法执行后,就不行了,给多少任务,你就得完成多少,短斤少两自己补上,补不上就下课受罚。

这下就要了命了,衙役收不齐,连累知县,知县收不齐,连累知府,知府又连累布政使,一层层追究责任,大家同
坐一条船,出了事谁也跑不掉。

与其自下而上垮台,不如自上而下压台。随着一声令下,各级官吏纷纷动员起来,不问理由,不问借口,必须完成
任务。

于是顺序又翻了过来,布政使压知府,知府压知县,知县压衙役,衙役……,就只能压老百姓了。

接下来的事情就简单了,上级压下级,下级压百姓。一般年景,也还能对付过去,要遇上个灾荒,那就惨了,衙役
还是照样上门,说家里遭灾,他点头,说家里死人,他还点头,点完头该交还得交。揭不开锅也好,全家死绝也
罢,收不上来官就没了,你说我收不收?

以上还算例行公事,到后来,事情越发恶劣。

\section[\thesection]{}

由于考成法业绩和官位挂钩,工作完成越多,越快,评定就越好,升官就越快。所以许多地方官员开始报虚数,狗
不拉屎的穷乡僻壤,也敢往大了报,反正自己也不吃亏。

可是朝廷不管那些,报了就得拿钱。于是挨家挨户地收,收不上来就逼,逼不出来就打,打急了就跑。而跑掉的这
些人,就叫流民。

流民,是明代中后期的一个严重问题。用今天的话说,就是社会不安定因素,这些人离开家乡,四处游荡,没有户
籍,没有住所,也不办暂住证,经常影响社会的安定团结。

到万历中期,流民数量已经十分惊人。连当时的北京市郊,都盘踞着大量流民。而且这帮人一般都不是什么老实巴
交的农民,偷个盗抢个劫之类的,都是家常便饭。朝廷隔三差五就要派兵来扫一次,十分难办。

而这些情况,是张居正始料未及的。

于是申时行毅然废除了考成法,并开辟了大量田地,安置各地的流民耕种,社会矛盾得以大大缓解。

废除考成法,是申时行执政的一次重要抉择。虽然是改革,却不用怎么费力,毕竟张居正是死人兼废人,没人帮他
出头,他的条令不废白不废。

但下一次,就没这么便宜的事了。

万历十八年(1590),总兵李联芳带兵在边界巡视的时候,遭遇埋伏,全军覆灭。下黑手的,是蒙古鞑靼部落的扯立
克。

事情闹大了,因为李联芳是明军高级将领,鞑靼部落把他干掉了,是对明朝政府的严重挑衅。所以消息传来,大臣
们个个摩拳擦掌,打算派兵去收拾这帮无事生非的家伙。

无论从哪个角度看,都是非打不可了,堂堂大明朝,被人打了不还手,当缩头乌龟,怎么也说不过去。而且这事闹
得皇帝都知道了,连他都觉得没面子,力主出兵。

老板发话,群众支持,战争已是势在必行,然而此时,申时行站了出来,对皇帝说:

``不能打。''

在中国历史上,但凡国家有事,地方被占了,人被杀了,朝廷总就是群情激奋,人人喊打,看上去个个都是民族英
雄,正义化身,然而其中别有奥秘:

临战之时,国仇家恨,慷慨激昂,大家都激动。在这个时候,跟着激动一把,可谓是毫无成本,反正仗也不用自己
打,还能落个名声,何乐而不为。

\section[\thesection]{}

主和就不同了,甭管真假,大家都喊打,你偏不喊,脱离群众,群众就会把你踩死。

所以主战者未必勇,主和者未必怯。

主和的申时行,就是一个勇敢的人。事实证明,他的主张十分正确。

因为那位下黑手的扯立克,并不是一般人,他的身份,是鞑靼的顺义王。

顺义王,是当年明朝给俺答的封号,这位扯立克就是俺答的继任者。但此人即不顺,也不义,好好的互市不干,整
天对外扩张,还打算联合蒙古、西藏各部落,搞个蒙古帝国出来和明朝对抗。

对这号人,打是应该的。但普鲁士伟大的军事家克劳塞维茨说过,战争是政治的继续,打仗说穿了,最终的目的就
是要对方听话,如果有别的方法能达到目的,何必要打呢?

申时行找到了这个方法。

他敏锐地发现,扯立克虽然是顺义王,但其属下却并非铁板一块。由各个部落组成,各有各的主张,大多数人和明
朝生意做得好好的,压根不想打仗,如果贸然开战,想打的打了,不想打的也打了,实在是得不偿失。分化瓦解才
是上策。

所以申时行反对。

当然,以申时行的水平,公开反对这种事,他是不会干的。夜深人静,独自起草,秘密上交,事情干得滴水不漏。

万历接到奏疏,认可了申时行的意见,同意暂不动兵,并命令他全权处理此事。

消息传开,一片哗然,但皇帝说不打,谁也没办法找皇帝算帐。申时行先生也是一脸无辜:我虽是朝廷首辅,但皇
帝不同意,我也没办法。

仗是不用打了,但这事还没完。申时行随即下令兵部尚书郑洛,在边界集结重兵,也不大举进攻,每天就在那里蹲
着。别的部落都不管,专打扯立克,而且还专挑他的运输车队下手,抢了就跑。

这种打法毫无成本,且收益率极高,明军乐此不疲,扯立克却是叫苦不迭,实在撑不下去了,只得率部躲得远远
的,就这样,不用大动干戈,不费一兵一卒,申时行轻而易举地解决了这个问题,恢复了边境的和平。

虽然张居正死后,朝局十分复杂,帮派林立,申时行却凭借着无人能敌的``混功'',应对自如,游刃有余。更为难
能可贵的是,他不但自己能混,还无私地帮助不能混的同志,比如万历。

\section[\thesection]{}

自从登基以来,万历一直在忙两件事,一是处理政务,二是搞臭张居正,从某种意义上讲,这两件事,其实是一件
事。

因为张居正实在太牛了,当了二十六年的官,十年的皇帝(实际如此),名气比皇帝还大,虽然人死了,茶还烫的冒
泡,所以不搞臭张居正,就搞不好政务。

但要干这件事,自己是无从动手的,必须找打手,万历很快发现,最好的打手,就是言官和大臣。

张居正时代,言官大臣都不吃香,被整得奄奄一息,现在万历决定,开闸,放狗。

事实上,这帮人的表现确实不错,如江东之、李植、羊可立等人,虽说下场不怎么样,但至少在工作期间,都尽到
了狗的本分。

看见张居正被穷追猛打,万历很高兴,看见申时行被牵连,万历也不悲伤,因为在他看来,这不过是轻微的副作
用,敲打一下申老师也好,免得他当首辅太久,再犯前任(张居正)的错误。

他解放言官大臣,指挥自若,是因为他认定,这些人将永远听从他的调遣。

然而他并不知道,自己犯下了一个多么可怕的错误。因为就骂人的水平而言,言官大臣和街头骂街大妈,只有一个
区别:大妈是业余的,言官大臣是职业的。

大妈骂完街后,还得回家洗衣做饭,言官大臣骂完这个,就会骂下一个。所以,当他们足够壮大之后,攻击的矛头
将不再是死去的张居正,或是活着的申时行,而是至高无上的皇帝。

对言官和大臣们而言,万历确实有被骂的理由。

自从万历十五年(1587)起,万历就不怎么上朝了,经常是``偶有微疾'',开始还真是``偶有'',后来就变成常
有,``微疾''也逐渐变成``头晕眼黑,力乏不兴'',总而言之,大臣们是越来越少见到他了。

必须说明的是,万历是不上朝,却并非不上班,事情还是要办,就好比说你早上起床,不想去单位,改在家里办
公,除了不打考勤,少见几个人外,也没什么不同,后世一说到这位仁兄,总是什么几十年不干活之类,这要么是
无意的误解,要么是有意的污蔑。

\section[\thesection]{}

在中国当皇帝,收益高,想要啥就有啥,但风险也大,屁股上坐的那个位置,只要是人就想要,但凡在位者,除了
个把弱智外,基本上都是怀疑主义者,见谁怀疑谁,今天这里搞阴谋,明天那里闹叛乱,日子过得那叫一个悬,几
天不看公文,没准刀就架在脖子上了。

万历自然也不例外,事实上,他是一个权力欲望极强,工于心计的政治老手,所有的人都只看到他不上朝的事实,
却无人察觉背后隐藏的奥秘:

在他之前,有许多皇帝每日上朝理政,费尽心力,日子过得极其辛苦,却依然是脑袋不保,而他几十年不上朝,谁
都不见,却依然能够控制群臣,你说这人厉不厉害?

但言官大臣是不管这些的,在他们的世界观里,皇帝不但要办事,还要上班,哪怕屁事没有,你也得坐在那,这才
叫皇帝。

万历自然不干,他不干的表现就是不上朝,言官大臣也不干,他们不干的表现就是不断上奏疏。此后的几十年里,
他们一直在干同样的事情。

万历十四年(1586)十月,这场长达三十余年的战争正式拉开序幕。

当时的万历,基本上还属于上朝族,只是偶尔罢工而已,就这样,也没躲过去。

第一个上书的,是礼部祠祭司主事卢洪春,按说第一个不该是他,因为这位仁兄主管的是祭祀,级别又低,平时也
不和皇帝见面。

但这一切并不妨碍他上书提意见,他之所以不满,不是皇帝不上朝,而是不祭祀。

卢洪春是一个很负责的人,发现皇帝不怎么来太庙,又听说近期经常消极怠工,便上书希望皇帝改正。

本来是个挺正常的事,却被他搞得不正常。因为这位卢先生除了研究礼仪外,还学过医,有学问在身上,不显实在
对不起自己,于是发挥专业特长,写就奇文一篇,送呈御览。

第二天,申时行奉命去见万历,刚进去,就听到了这样的一句话:

``卢洪春这厮!肆言惑众,沽名讪上,好生狂妄!着锦衣卫拿在午门前,着实打六十棍!革了职为民当差,永不叙用!''

以上言辞,系万历同志之原话,并无加工。

很久很久以前,这厮两个字就诞生了,在明代的许多小说话本中,也频频出现,其意思依照现场情况,有各种不同
的解释,从这家伙、这小子、到这混蛋,这王八蛋,不一而同。

但可以肯定的是,这两字不是好话,是市井之徒的常用语,皇帝大人脱口而出,那是真的急了眼了。

\section[\thesection]{}

这是可以理解的,因为卢洪春的那篇奏疏,你看你也急。

除了指责皇帝陛下不该缺席祭祀外,卢主事还替皇帝陛下担忧其危害:

``陛下春秋鼎盛,精神强固,头晕眼黑之疾,皆非今日所宜有。''

年纪轻轻就头晕眼黑,确实是不对的,确实应该注意,到此打住,也就罢了。

可是担忧完,卢先生就发挥医学特长:

``医家曰:气血虚弱,乃五劳七伤所致,肝虚则头晕目眩,肾虚则腰痛精泄。''

气血虚弱,肝虚肾虚,症状出来了,接着就是分析原因:

``以目前衽席之娱,而忘保身之术,其为患也深。''

最经典的就是这一句。

所谓衽席之娱,是指某方面的娱乐,相信大家都能理解,综合起来的意思是:

皇帝你之所以身体不好,在我看来,是因为过于喜欢某种娱乐,不知收敛保养,如此下去,问题非常严重。

说这句话的,不是万历他妈,不是他老婆,不是深更半夜交头接耳,天知地知,你知我知,而是一个管礼仪的六品
官,在大庭广众之下公开上书,且一言一语皆已千古流传。

再不收拾他,就真算白活了。

命令下达给了申时行,于是申时行为难了。

这位老油条十分清楚,如果按照万历的意思严惩卢洪春,言官们是不答应的;如果不处理,万历又不答应。

琢磨半天,想了个办法。

他连夜动笔,草拟了两道文书,第一道是代万历下的,严厉斥责卢洪春,并将其革职查办。第二道是代内阁下的,
上奏皇帝,希望能够宽恕卢洪春,就这么算了。

按照他的想法,两边都不得罪,两边都有交代。

事实证明,这是幻想。

首先发作的是万历。这位皇帝又不是傻子,一看就明白申时行耍两面派,立即下令,即刻动手打屁股,不得延误。
此外他还不怀好意地暗示,午门很大,多个人不嫌挤。

午门就是执行廷杖的地方,眼看自己要去垫背,申时行随即更改口风,把卢洪春拉出去结结实实地打了六十棍。

马蜂窝就这么捅破了。

言官们很惭愧。一个礼部的业余选手,都敢上书,勇于曝光皇帝的私生活,久经骂阵的专业人才竟然毫无动静,还
有没有职业道德?

于是大家群情激奋,以给事中杨廷相为先锋,十余名言官一拥而上,为卢洪春喊冤翻案。

\section[\thesection]{}

面对漫天的口水和奏疏,万历毫不退让,事实上,这是一个极端英明的抉择:一旦让步,从宽处理了卢洪春,那所
谓``喜欢某种娱乐,不注意身体''的黑锅,就算是背定了。

但驳回去一批,又来一批。言官们踊跃发言,热烈讨论,反正闲着也是闲着,不说白不说。

万历终于恼火了,他决定罚款,带头闹事的主犯罚一年工资,从犯八个月。

对言官而言,这个办法很有效果。

在明代,对付不同类别的官员,有不同的方法:要折腾地方官,一般都是降职。罚工资没用,因为这帮人计划外收
入多,工资基本不动,罚光了都没事。

言官就不同了,他们都是靠死工资的,没工资日子就没法过,一家老小只能去喝西北风,故十分害怕这一招。

于是风波终于平息,大家都消停了。

但这只是表面现象,对此,申时行有很深的认识。作为天字第一号混事的高手,他既不想得罪领导,又不想得罪同
事,为实现安定团结,几十年如一日地和稀泥,然而随着事件的进一步发展,他逐渐意识到,和稀泥的幸福生活长
不了。

因为万历的生活作风,是一天不如一天了。

事实上,卢洪春的猜测很可能是正确的,二十多岁的万历之所以不上朝,应该是沉迷于某种娱乐。否则实在很难解
释,整天在宫里呆着,到底有啥乐趣可言。

说起来,当年张居正管他也实在管得太紧。啥也不让干,吃个饭喝点酒都得看着。就好比高考学生拼死拼活熬了几
年,一朝拿到录取通知书,革命成功,自然就完全解放了。

万历同志在解放个人的同时,也解放了大家。火烧眉毛的事情(比如打仗,阴谋叛乱之类),看一看,批一批,其余
的事,能不管就不管,上朝的日子越来越少。

申时行很着急,但这事又不好公开讲,于是他灵机一动,连夜写就了一封奏疏。在我看来,这封文书的和稀泥技
术,已经达到了登峰造极的地步。

文章大意是这样的:

皇帝陛下,我听说您最近身体不好,经常头晕眼花(时作晕眩),对此我十分担心。我知道,您这是劳累所致啊! 由
于您经常熬夜工作,亲历亲为(一语双关,佩服),才会身体不好。为了国家,希望您能够清心寡欲,养气宁神(原文
用词),好好保重身体。

高山仰止,自惭形秽之感,油然而生。

\section[\thesection]{}

对于这封奏疏,万历还是很给了点面子。他召见了申时行,表示明白他的苦心,良药虽然苦口,却能治病,今后一
定注意。申时行备感欣慰,兴高采烈地走了。

但这只是错觉,因为在这个世界上,能够药到病除的药只有一种----毒药。

事实证明,万历确实不是一般人。因为一般人被人劝,多少还能改几天,他却是一点不改,每天继续加班加点,从
事自己热爱的娱乐。据说还变本加厉,找来了十几个小太监,陪着一起睡(同寝),也算是开辟了新品种。

找太监这一段,史料多有记载,准确性说不好,但有一点是肯定的,那就是万历同志依旧是我行我素,压根儿不给
大臣们面子。

既然不给脸面,那咱就有撕破脸的说法。

万历十七年十二月,明代,不,是中国历史上胆最大、气最足的奏疏问世了!其作者,是大理寺官员雒于仁。

雒于仁,字少泾,陕西泾阳人。纵观明清两代,陕西考试不大行,但人都比较实在。既不慷慨激昂,也不罗罗嗦
嗦,说一句是一句,天王老子也敢顶。比如后世的大贪污犯和珅,最得意的时候,上有皇帝撑腰,下有大臣抬轿。
什么纪晓岚、刘墉,全都服服帖帖,老老实实靠边站,所谓``智斗''之类,大都是后人胡编的,可谓一呼百应。而
唯一不应的,就是来自陕西的王杰。每次和珅说话,文武百官都夸,王杰偏要顶两句,足足恶心了和珅十几年,又
抓不到他的把柄,也只能是``厌之而不能去''(清史稿)。

雒于仁就属于这类人,想什么说什么,从不怕得罪人,而且他的这个习惯,还有家族传统:

雒于仁的父亲,叫做雒遵,当年曾是高拱的学生,干过吏科都给事中。冯保得势的时候,骂过冯保;张居正得势的
时候,骂过谭纶(张居正的亲信),为人一向高傲,平生只佩服一人,名叫海瑞。

有这么个父亲,雒于仁自然不是孬种。加上他家虽世代为官,却世代不捞钱,穷日子过惯了,光脚的不怕穿鞋的。
不怕罚工资,不怕降职,看不惯皇帝了,就要骂。随即一挥而就,写下奇文一篇,后世俗称为《酒色财气疏》。

\section[\thesection]{}

该文主旨明确,开篇即点明中心思想:

``陛下之恙,病在酒色财气者也,夫纵酒则溃胃,好色则耗精,贪财则乱神,尚气则损肝。''

这段话用今天的话讲,就是说皇上你确实有病,什么病呢?你喜欢喝酒,喜欢玩女人,喜欢捞钱,还喜欢动怒耍威
风,酒色财气样样俱全,自然就病了。

以上是全文的论点,接下来的篇幅,是论据,描述了万历同志在喝酒玩女人方面的具体体现,逐一论证以上四点的
真实性和可靠性,比较长,就不列举了。

综观此文,下笔之狠,骂法之全,真可谓是鬼哭狼嚎。就骂人的狠度和深度而言,雒于仁已经全面超越了海瑞前
辈,雒遵同志如果在天有灵,应该可以瞑目了。

更缺德的是,雒于仁的这封奏疏是十二月(农历)底送上去的,搞得万历自从收到这封奏疏,就开始骂,不停地骂,
没日没夜地骂,骂得新年都没过好。

骂过瘾后,就该办人了。

万历十八年(1590)正月初一,按照规矩,内阁首辅应该去宫里拜年。当然也不是真拜,到宫门口鞠个躬就算数。但
这一次,申时行刚准备走人,就被太监给叫住了。

此时,雒于仁的奏疏已经传遍内外,申先生自然知道怎么回事,不用言语就进了宫。看到了气急败坏的皇帝,双方
展开了一次别开生面的对话:(以下言语,皆出自申时行的原始记录)

万历:先生看过奏本(指雒于仁的那份),说朕酒色财气,试为朕评一评。

申时行:……(还没说话,即被打断)

万历:``他说朕好酒,谁人不饮酒?……又说朕好色,偏宠贵妃郑氏(即著名的郑贵妃),朕只因郑氏勤劳……何曾有偏?''

喘口气,接着说:

``他说朕贪财……朕为天子,富有四海之内,普天之下莫非王土,天下之财皆朕之财!又说朕尚气……勇即是气,朕岂不
知!人孰无气!''

这口气出完了,最后得出结论:

``先生将这奏本去票拟重处!''

申时行这才搭上话:

``此无知小臣误听道路之言……(说到此处,又被打断)''

万历大喝一声:

``他就是出位沽名!''

申时行傻眼了,他在朝廷混了几十年,从未见过这幅场景,皇帝大人一副吃人的模样,越说越激动,唾沫星子横
飞,这样下去,恐怕要出大事。

于是他闭上了嘴,开始紧张地思索对策。

\section[\thesection]{}

既不能让皇帝干掉雒于仁,也不能不让皇帝出气,琢磨片刻,稀泥和好了。

``他(指雒于仁)确实是为了出名(先打底),但陛下如果从重处罚他,却恰恰帮他成了名,反损皇上圣德啊!''

``如果皇上宽容,不和他去一般见识,皇上的圣德自然天下闻名(继续戴高帽)!''

在这堆稀泥面前,万历同志终于消了气:

``这也说得是,如果和他计较,倒不是损了朕的德行,而是损了朕的气度!''

上钩了,再加最后一句:

``皇上圣度如天地一般,何所不容!''(圆满收工)

万历沉默地点了点头。

话说到这,事情基本就算完了,申时行定定神,突然想起了另一件事,一件极为重要的事。

他决定趁此机会,解决此事。

然而他正准备开口,却又听见了一句怒斥:

``朕气他不过,必须重处!''

万历到底是年轻人,虽然被申时行和了一把稀泥,依然不肯干休,这会回过味来,又绕回去了。

这事还他娘没完了,申时行头疼不已,但再头疼事情总得解决,如果任由万历发作胡来,后果将不堪设想。

在这关键的时刻,申时行再次展现了他举世无双的混事本领,琢磨出了第二套和稀泥方案:

``陛下,此奏本(雒于仁)原本就是讹传,如果要重处雒于仁,必定会将此奏本传之四方,反而做了实话啊!''

利害关系说完,接下来该掏心窝了:

``其实原先我等都已知道此奏疏,却迟迟不见陛下发阁(内阁)惩处(学名:留中),我们几个内阁大学士在私底下都
互相感叹,陛下您胸襟宽容,实在是超越千古啊(马屁与说理相结合)。''

``所以以臣等愚见,陛下不用处置此事,奏疏还是照旧留存吧,如此陛下之宽容必定能留存史书,传之后世,千秋
万代都称颂陛下是尧舜之君,是大大的好事啊!''

据说拍马屁这个行当,最高境界是两句古诗,所谓``随风潜入夜,润物细无声'',在我看来,申时行做到了。

  但申先生还是低估了万历的二杆子性格,他话刚讲完,万历又是一声大吼:

``如何设法处他?只是气他不过!''

\section[\thesection]{}

好话说一堆,还这么个态度,那就不客气了:

``此本不可发出,也无他法处之,还望皇上宽恕,容臣等传谕该寺堂官(即大理寺高级官员),使之去任可也。''

这意思就是,老子不和稀泥了,明白告诉你,骂你的这篇文章不能发,也没办法处理,最多我去找他们领导,把这
人免职了事,你别再闹了,闹也没用。

很明显,万历虽然在气头上,却还是很识趣的,他清楚,目前形势下,自己不能把雒于仁怎么样,半天一言不发。
申时行明白,这是默认。

万历十八年的这场惊天风波就此了解,雒于仁骂得皇上一无是处,青史留名,却既没掉脑袋,也没有挨板子,拍拍
屁股就走人了。而气得半死的万历终于认定,言官就是混蛋,此后的几十年里,他都保持着相同的看法。

最大的赢家无疑是申时行,他保护了卢洪春、保护了雒于仁,安抚了言官大臣,也没有得罪皇帝,使两次危机成功
化解,无愧为和稀泥的绝顶高手。

自万历十一年执政以来,申时行经历了无数考验,无论是上司还是同僚,他都应付自如,七年间,上哄皇帝,下抚
大臣,即使有个把不识趣、不配合的,也能被他轻轻松松地解决掉,混得可谓如鱼得水。

然而正是这一天,万历十八年(1590)正月初一,在解决完最为棘手的雒于仁问题后,他的好运将彻底结束。

因为接下来,他说了这样一句话:

``臣等更有一事奏请。''

虽然雒于仁的事十分难办,但和申时行即将提出的这件事相比,只能说是微不足道。

他所讲的事情,影响了无数人的一生,以及大明王朝的国运,而这件事情,在历史上有个专用名词:``争国本''。

游戏的开始

在张居正管事的前十年,万历既不能执政,也不能管事,甚至喝酒胡闹都不行,但他还有一项基本的权力----娶老
婆。

万历六年(1578),经李太后挑选,张居正认可,十四岁的万历娶了老婆,并册立为皇后。

不过对万历而言,这不是个太愉快的事情,因为这个老婆是指认的,什么偶然邂逅,自由恋爱都谈不上,某月某
天,突然拉来一女的,无需吃饭看电影,就开始办手续,经过无数道繁琐程序仪式,然后正式宣告,从今以后,她
就是你的老婆了。

包办婚姻,纯粹的包办婚姻。

\section[\thesection]{}

虽然是凑合婚姻,但万历的运气还不错,因为他的这个老婆相当凑合。

万历皇后王氏,浙江人,属传统贤妻型,而且为人乖巧,定位明确,善于关键时刻抓关键人,进宫后皇帝都没怎么
搭理,先一心一意服侍皇帝他妈,早请示晚汇报,把老太太伺候好了,婆媳问题也就解决了。

此外她还是皇帝的办公室主任,由于后来万历不上朝,喜欢在家里办公,公文经常堆得到处都是,她都会不动声色
地加以整理,一旦万历找不着了,她能够立即说出公文放在何处,何时、由何人送入,在生活上,她对皇帝大人也
是关怀备至,是优秀的秘书老婆两用型人才。

这是一个似乎无可挑剔的老婆,除了一个方面----她生不出儿子。

古人有云:不孝有三,无后为大,虽说家里有一堆儿子,最后被丢到街上的也不在少数,但既然是古人云,大家就
只好人云亦云,生不出儿子,皇后也是白搭。于是万历九年(1581)的时候,在李太后的授意下,万历下达旨意:命
令各地选取女子,以备挑选。

其实算起来,万历六年两人结婚的时候,万历只有十四岁,到万历九年的时候,也才十七岁,连枪毙都没有资格,
就逼着要儿子,似乎有点不地道,但这是一般人的观念,皇帝不是一般人,观念自然也要超前,生儿子似乎也得比
一般人急。

但旨意传下去,被张居正挡了回来,并且表示,此令绝不可行。

不要误会,张先生的意思并非考虑民间疾苦,不可行,是行不通。

到底是首辅大人老谋深算,据说他刚看到这道旨意,便下断言:如按此令下达,决然无人可挑。

俗话说,一入候门深似海,何况是宫门,辛辛苦苦养大的女儿送进去,就好比黄金周的旅游景点,丢进人堆就找不
着了,谁也不乐意。那些出身名门、长相漂亮的自然不来,万一拉上来的都是些歪瓜裂枣,恶心了皇帝大人,这个
黑锅谁来背?

可是皇帝不能不生儿子,不能不找老婆,既要保证数量,也要确保质量,毕竟你要皇帝大人将就将就,似乎也是勉
为其难。

事情很难办,但在张居正大人的手中,就没有办不了的事,他脑筋一转,加了几个字:原文是挑选入宫,大笔一
挥,变成了挑选入宫册封嫔妃。

\section[\thesection]{}

事情就这么解决了,因为说到底,入不入宫,也是个成本问题,万一进了宫啥也混不上,几十年没人管,实在不太
值。在入宫前标明待遇,肯定级别,给人家个底线,自然就都来了。

这就是水平。

但连张居正都没想到,他苦心琢磨的这招,竟然还是没用上。

因为万历自己把这个问题解决了。

就在挑选嫔妃的圣旨下达后,一天,万历闲来无事,去给李太后请安,完事后,准备洗把脸,就叫人打盘水来。

水端来了,万历一边洗着手,一边四处打量,打量来,打量去,就打量上了这个端脸盆的宫女。

换在平常,这类人万历是一眼都不看的,现在不但看了,而且还越看越顺眼,顺眼了,就开始搭讪。

就搭讪的方式而言,皇帝和街头小痞子是没什么区别的,无非是你贵姓,哪里人等等。但差异在于,小痞子搭完
话,该干嘛还干嘛,皇帝就不同了。

几句话搭下来,万历感觉不错,于是乎头一热,就幸了。

皇帝非凡人,所以幸了之后的反应也不同于凡人,不用说什么一时冲动之类的话,拍拍屁股就走人了。不过万历还
算厚道,临走时,赏赐她一副首饰,这倒也未必是他有多大觉悟,而是宫里的规定:但凡临幸,必赐礼物。

因为遵守这个规定,他后悔了很多年。

就万历而言,这是一件小事,皇帝嘛,幸了就幸了,感情是谈不上的,事实上,此人姓甚名谁,他都未必记得。

这个宫女姓王,他很快就将牢牢记住。因为在不久之后,王宫女意外地发现,自己怀孕了。

这个消息很快就传到了万历那里,他非但不高兴,反而对此守口如瓶,绝口不提。

因为王宫女地位低,且并非什么沉鱼落雁之类的人物,一时兴起而已,万历不打算认这帐,能拖多久是多久。

但这位仁兄明显打错了算盘,上朝可以拖,政务可以拖,怀孕拖到最后,是要出人命的。

随着王宫女的肚子一天天大起来,知道这件事的人也一天天多起来,最后,太后知道了。

于是,她叫来了万历,向他询问此事。

万历的答复是沉默,他沉默的样子,很有几分流氓的风采。

\section[\thesection]{}

然而李太后对付此类人物,一向颇有心得。当年如高拱、张居正之类的老手都应付过去了,刚入行的新流氓万历自
然不在话下。既然不说话,就接着问。

装哑巴是行不通了,万历随口打哈哈,就说没印象了,打算死不认账。

万历之所以有持无恐,是因为这种事一般都是你知我知,现场没有证人,即使有证人,也不敢出来(偷窥皇帝,是要
命的)。

他这种穿上裤子就不认人的态度彻底激怒了李太后,于是,她找来了证人。

这个证人的名字,叫内起居注。

在古代文书中,起居注是皇帝日常言行的记录。比如今天干了多少活,去了多少地方,是第一手的史料来源。

但起居注记载的,只是皇帝的外在工作情况,是大家都能看见的,而大家看不见的那部分,就是内起居注。

内起居注记载的,是皇帝在后宫中的生活情况。比如去到哪里,和谁见面,干了些什么。当然,鉴于场所及皇帝工
作内容的特殊性,其实际记录者不是史官,而是太监。所谓外表很天真,内心很暴力,只要翻一翻内外两本起居
注,基本都能搞清楚。

由于具有生理优势,太监可以出入后宫,干这类事情也方便得多。皇帝到哪里,就跟到哪里(当然,不宜太近),皇
帝进去开始工作,太监在外面等着。等皇帝出来,就开始记录,某年某月某日,皇帝来到某后妃处,某时进,某时
出,特此记录存入档案。

皇帝工作,太监记录,这是后宫的优良传统,事实证明,这一规定是极其有效,且合理的。

因为后宫人太多,皇帝也不计数,如王宫女这样的邂逅,可谓比比皆是。实际上,皇帝乱搞并不重要,重要的,是
乱搞之后的结果。

如果宫女或后妃恰好怀孕,生下了孩子,这就是龙种,要是儿子,没准就是下一任皇帝,万一到时没有原始记录,
对不上号,那就麻烦了。

所以记录工作十分重要。

但这项工作,还有一个漏洞,因为事情发生的时候,只有皇帝、太监、后妃(宫女)三人在场。事后一旦有了孩子,
后妃自然一口咬定,是皇帝干的,而皇帝一般都不记得,是不是自己干的。

最终的确定证据,就是太监的记录。但问题在于,太监也是人,也可能被人收买,如果后妃玩花样,或是皇帝不认
账,太监也没有公信力。

\section[\thesection]{}

所以宫中规定,皇帝工作完毕,要送给当事人一件物品,而这件物品,就是证据。

李太后拿出了内起居注,翻到了那一页,交给了万历。

一切就此真相大白,万历只能低头认账。

万历十年(1582),上车补票的程序完成,王宫女的地位终于得到了确认,她挺着大肚子,接受了恭妃的封号。

两个月后,她不负众望生下了一个儿子,是为万历长子,取名朱常洛。

消息传来,举国欢腾,老太太高兴,大臣们也高兴,唯一不高兴的,就是万历。

因为他对这位恭妃,并没有太多感情。对这个意外出生的儿子,自然也谈不上喜欢。更何况,此时他已经有了德妃。

德妃,就是后世俗称的郑贵妃。北京大兴人,万历初年进宫,颇得皇帝喜爱。

在后来的许多记载中,这位郑贵妃被描述成一个相貌妖艳,阴狠毒辣的女人。但在我看来,相貌妖艳还有可能,阴
狠毒辣实在谈不上。在此后几十年的后宫斗争中,此人手段之拙劣,脑筋之愚蠢,反应之迟钝,实在令人发指。

综合史料分析,其智商水平,也就能到菜市场骂个街而已。

可是万历偏偏就喜欢这个女人,经常前去留宿。而郑妃的肚子也相当争气,万历十一年(1583)生了个女儿,虽然不
能接班,但万历很高兴,竟然破格提拔,把她升为了贵妃。

这是一个不详的先兆,因为在后宫中,贵妃的地位要高于其他妃嫔----包括生了儿子的恭妃。

而这位郑贵妃的个人素养也实在很成问题,当上了后妃领导后,除了皇后,谁都瞧不上,特别是恭妃,经常被她称
作老太婆。横行宫中,专横跋扈,十分好斗。

难能可贵的是,贵妃同志不但特别能战斗,还特别能生。万历十四年(1586),她终于生下了儿子,取名朱常洵。

这位朱常洵,就是后来的福王。按郑贵妃的想法,有万历当靠山,这孩子生出来,就是当皇帝的。但她做梦也想不
到,几十年后,自己这个宝贝儿子会死在屠刀之下。挥刀的人,名叫李自成。

但在当时,这个孩子的出生,确实让万历欣喜异常。他本来就不喜欢长子朱常洛,打算换人,现在替补来了,怎能
不高兴?

然而他很快就将发现,皇帝说话,不一定算数。

\section[\thesection]{}

吸取了以往一百多年里,自己的祖辈与言官大臣斗争的丰富经验。万历没敢过早暴露目标,绝口不提换人的事,只
是静静地等待时机成熟,再把生米煮成熟饭。

可还没等米下锅,人家就打上门来了,而且还不是言官。

万历十四年(1586)三月,内阁首辅申时行上奏:望陛下早立太子,以定国家之大计,固千秋之基业。

老狐狸就是老狐狸,自从郑贵妃生下朱常洵,申时行就意识到了隐藏的危险。他知道,自己的这个学生想干什么。

凭借多年的政治经验,他也很清楚,如果这么干了,迎面而来的,必定是史无前例的惊涛骇浪。从此,朝廷将永无
宁日。

于是他立即上书,希望万历早立长子。言下之意是,我知道你想干嘛,但这事不能干,你趁早断了这念头,早点洗
了睡吧。

其实申时行的本意,倒不是要干涉皇帝的私生活:立谁都好,又不是我儿子,与我何干?之所以提早打预防针,实在
是出于好心,告诉你这事干不成,早点收手,免得到时受苦。

可是他的好学生似乎打定主意,一定要吃苦,收到奏疏,只回复了一句话:

``长子年纪还小,再等个几年吧。''

学生如此不开窍,申时行只得叹息一声,扬长而去。

但这一次,申老师错了,他低估了对方的智商。事实上,万历十分清楚这封奏疏的隐含意义。只是在他看来,皇帝
毕竟是皇帝,大臣毕竟是大臣,能坚持到底,就是胜利。此即所谓,明知山有虎,偏向虎山行。

但一般说来,没事上山找老虎玩的,只有两种人:一种是打猎,一种是自尽。

话虽如此,万历倒也不打无把握之仗,在正式亮出匕首之前,他决定玩一个花招。

万历十四年(1586)三月,万历突然下达谕旨:郑贵妃劳苦功高,升任皇贵妃。

消息传来,真是粪坑里丢炸弹,分量十足.朝廷上下议论纷纷,群情激奋。

因为在后宫中,皇贵妃仅次于皇后,算第二把手。且历朝历代,能获此殊荣者少之又少(生下独子或在后宫服务多
年)。

按照这个标准,郑贵妃是没戏的。因为她入宫不长,且皇帝之前已有长子,没啥突出贡献,无论怎么算都轮不到她。

\section[\thesection]{}

万历突然来这一招,真可谓是煞费苦心。首先可以藉此提高郑贵妃的地位,子以母贵,母亲是皇贵妃,儿子的名分
也好办;其次还能借机试探群臣的反应。今天我提拔孩子他妈,你们同意了,后天我就敢提拔孩子。温水煮青蛙,
咱们慢慢来。

算盘打得很好,可惜只是掩耳盗铃。

要知道,在朝廷里混事的这帮人,个个都不简单:老百姓家的孩子,辛辛苦苦读几十年书,考得死去活来,进了朝
廷,再被踩个七荤八素,这才修成正果。生肖都是属狐狸的,嗅觉极其灵敏,擅长见风使舵,无事生非。皇帝玩的
这点小把戏,在他们面前也就是个笑话,傻子才看不出来。

更为难得的是,明朝的大臣们不但看得出来,还豁得出去。第一个出头的,是户部给事中姜应麟。

相对而言,这位仁兄还算文明,不说粗话,也不骂人,摆事实讲道理:

``皇帝陛下,听说您要封郑妃为皇贵妃,我认为这是不妥的。恭妃先生皇长子,郑妃生皇三子(中间还有一个,夭折
了),先来后到,恭妃应该先封。如果您主意已定,一定要封,也应该先封恭妃为贵妃,再封郑妃皇贵妃,这样才算
合适。''

``此外,我还认为,陛下应该尽早立皇长子为太子,这样天下方才能安定。''

万历再一次愤怒了,这可以理解,苦思冥想几天,好不容易想出个绝招,自以为得意,没想到人家不买账,还一言
点破自己的真实意图,实在太伤自尊。

为挽回面子,他随即下令,将姜应麟免职外放。

好戏就此开场。一天后,吏部员外郎沈璟上书,支持姜应麟,万历二话不说,撤了他的职。几天后,吏部给事中杨
廷相上书,支持姜应麟,沈璟,万历对其撤职处理。又几天后,刑部主事孙如法上书,支持姜应麟、沈璟、杨廷
相,万历同志不厌其烦,下令将其撤职发配。

在这场斗争中,明朝大臣们表现出了无畏的战斗精神:不怕降级,不怕撤职,不怕发配。个顶个地扛着炸药包往上
冲,前仆后继,人越闹越多,事越闹越大。中央的官不够用了,地方官也上书凑热闹,搞得一塌糊涂,乌烟瘴气。

然而事情终究还是办成了,虽然无数人反对,无数人骂仗,郑贵妃还是变成了郑皇贵妃。

\section[\thesection]{}

虽然争得天翻地覆,但该办的事还是办了。万历十四年三月,郑贵妃正式册封。

这件事情的成功解决给万历留下了这样一个印象:自己想办的事情,是能够办成的。

这是一个错误的判断。

然而此后,在册立太子的问题上,万历确实消停了----整整消停了四年多。当然,不闹事,不代表不挨骂。事实
上,在这四年里,言官们非常尽责。他们找到了新的突破口----皇帝不上朝,并以此为契机,在雒于仁等模范先锋
的带领下,继续奋勇前进。

但总体而言,小事不断,大事没有,安定团结的局面依旧。直到这历史性的一天:万历十八年(1590)正月初一。

解决雒于仁事件后,申时行再次揭开了盖子:

``臣等更有一事奏请。''

``皇长子今年已经九岁,朝廷内外都认为应册立为太子,希望陛下早日决定。''

在万历看来,这件事比雒于仁的酒色财气疏更头疼,于是他接过了申时行刚刚用过的铁锹,接着和稀泥:

``这个我自然知道,我没有嫡子(即皇后的儿子),长幼有序。其实郑贵妃也多次让我册立长子,但现在长子年纪还
小,身体也弱,等他身体强壮些后,我才放心啊。''

这段话说得很有水平,按照语文学来分析,大致有三层意思。

第一层先说自己没有嫡子,是说我只能立长子;然后又讲长幼有序,是说我不会插队,但说来说去,就是不说要立
谁;接着又把郑贵妃扯出来,搞此地无银三百两。

最后语气一转,得出结论:虽然我只能立长子、不会插队,老婆也没有干涉此事,但考虑到儿子太小,身体太差,
暂时还是别立了吧。

这招糊弄别人可能还行,对付申时行就有点滑稽了,和了几十年稀泥,哪排得上你小子?

于是申先生将计就计,说了这样一句话:

``皇长子已经九岁,应该出阁读书了,请陛下早日决定此事。''

这似乎是一件完全不相干的事情,但事实绝非如此,因为在明代,皇子出阁读书,就等于承认其为太子,申时行的
用意非常明显:既然你不愿意封他为太子,那让他出去读书总可以吧,形式不重要,内容才是关键。

\section[\thesection]{}

万历倒也不笨,他也不说不读书,只是强调人如果天资聪明,不读书也行。申时行马上反驳,说即使人再聪明,如
果没有人教导,也是不能成才的。

就这样,两位仁兄从继承人问题到教育问题,你来我往,互不相让,闹到最后,万历烦了:

``我都知道了,先生你回去吧!''

话说到这个份上,也只好回去了,申时行离开了宫殿,向自己家走去。

然而当他刚刚踏出宫门的时候,却听到了身后急促的脚步声。

申时行转身,看见了一个太监,他带来了皇帝的谕令:

``先不要走,我已经叫皇长子来了,先生你见一见吧。''

十几年后,当申时行在家撰写回忆录的时候,曾无数次提及这个不可思议的场景以及此后那奇特的一幕,终其一
生,他也未能猜透万历的企图。

申时行不敢怠慢,即刻回到了宫中,在那里,他看见了万历和他的两个儿子,皇长子朱常洛,以及皇三子朱常洵。

但给他留下最深刻印象的,却并非这两个皇子,而是此时万历的表情。没有愤怒,没有狡黠,只有安详与平和。

他指着皇长子,对申时行说:

``皇长子已经长大了,只是身体还有些弱。''

然后他又指着皇三子,说道:

``皇三子已经五岁了。''

接下来的,是一片沉默。

万历平静地看着申时行,一言不发。此时的他,不是一个酒色财气的昏庸之辈,不是一个暴跳如雷的使气之徒。他
是一个父亲,一个看着子女不断成长,无比欣慰的父亲。

申时行知道机会来了,于是他打破了沉默:

``皇长子年纪已经大了,应该出阁读书。''

万历的心意似乎仍未改变:

``我已经指派内侍教他读书。''

事到如今,只好豁出去了:

``皇上您在东宫的时候,才六岁,就已经读书了。皇长子此刻读书,已经晚了!''

万历的回答并不愤怒却让人哭笑不得:

``我五岁就已能读书!''

申时行知道,在他的一生中,可能再也找不到一个更好的机会,去劝服万历,于是他做出了一个惊人的举动。

他上前几步,未经许可,便径自走到了皇长子的面前,端详片刻,对万历由衷地说道:

``皇长子仪表非凡,必成大器,这是皇上的福分啊,希望陛下能够早定大计,朝廷幸甚!国家幸甚!''

万历十八年正月初一日,在愤怒、沟通、争执后,万历终于第一次露出了笑容。

\section[\thesection]{}

万历微笑地点点头,对申时行说道:

``这个我自然知道,其实郑贵妃也劝过我早立长子,以免外人猜疑,我没有嫡子,册立长子是迟早的事情啊。''

这句和缓的话,让申时行感到了温暖,儿子出来了,好话也说了,虽然也讲几句什么郑贵妃支持,没有嫡子之类的
屁话,但终究是表了态。

形势大好,然而接下来,申时行却一言不发,行礼之后便退出了大殿。

这正是他绝顶聪明之处,点到即止,见好就收,今天先定调,后面慢慢来。

但他无论如何也想不到,这次和谐的对话,不但史无前例,而且后无来者。``争国本''事件的严重性,将远远超出
他的预料,因为决定此事最终走向的,既不是万历,也不会是他。

谈话结束后,申时行回到了家中,开始满怀希望地等待万历的圣谕,安排皇长子出阁读书。

可是一天天过去了,希望变成了失望。到了月底,他也坐不住了,随即上疏,询问皇长子出阁读书的日期。这意思
是说,当初咱俩谈好的事,你得守信用,给个准信。

但是万历似乎突然失忆,啥反应都没有,申时行等了几天,一句话都没有等到。

既然如此,那就另出新招,几天后,内阁大学士王锡爵上书:

``陛下,其实我们不求您立刻册立太子,只是现在皇长子九岁,皇三子已五岁,应该出阁读书。''

不说立太子,只说要读书,而且还把皇三子一起拉上,由此而见,王锡爵也是个老狐狸。

万历那边却似乎是人死绝了,一点消息也没有,王锡爵等了两个月,石沉大海。

到了四月,包括申时行在内,大家都忍无可忍了,内阁四名大学士联名上疏,要求册立太子。

尝到甜头的万历故伎重演:无论你们说什么,我都不理,我是皇帝,你们能把我怎么样?

但他实在低估了手下的这帮老油条,对付油盐不进的人,他们一向都是有办法的。

几天后,万历同时收到了四份奏疏,分别是申时行、王锡爵、许国、王家屏四位内阁大学士的辞职报告。理由多种
多样,有说身体不好,有说事务繁忙,难以继任的,反正一句话,不干了。

自万历退居二线以来,国家事务基本全靠内阁,内阁一共就四个人,要是都走了,万历就得累死。

没办法,皇帝大人只好现身,找内阁的几位同志谈判,好说歹说,就差求饶了,并且当场表态,会在近期解决这一
问题。内阁的几位大人总算给了点面子,一番交头接耳之后,上报皇帝:病的还是病,忙的还是忙,但考虑到工作
需要,王家屏大学士愿意顾全大局,继续干活。

万历窃喜。

\section[\thesection]{}

因为这位兄弟的策略,叫拖一天是一天。拖到这帮老家伙都退了,皇三子也大了,到时木已成舟,不同意也得同意。
这次内阁算是上当了。

然而上当的人,只有他。

因为他从未想过这样一个问题:为什么留下来的,偏偏是王家屏呢?

王家屏,山西大同人,隆庆二年进士。简单地说,这是个不上道的人。

王家屏的科举成绩很好,被选为庶吉士,还编过《世宗实录》,应该说是很有前途的,可一直以来,他都没啥进步。
原因很简单,高拱当政的时候,他曾上书弹劾高拱的亲戚,高首辅派人找他谈话,让他给点面子,他说,不行。

张居正当政的时候,他搞非暴力不合作。照常上班,就是不靠拢上级,张居正刚病倒的时候,许多人都去祈福,表
示忠心,有人拉他一起去,他说,不去。

张居正死了,万历十二年,他进入内阁,成为大学士。此时的内阁,已经有了申时行、王锡爵、许国三个人,他排
第四。按规矩,这位甩尾巴的新人应该老实点,可他偏偏是个异类,每次内阁讨论问题,即使大家都同意,他觉得
不对,就反对。即使大家都反对,他觉得对,就同意。

他就这么在内阁里硬挺了六年,谁见了都怕,申时行拿他也没办法。更有甚者,写辞职信时,别人的理由都是身体
有病,工作太忙,他却别出一格,说是天下大旱,作为内阁成员,负有责任,应该辞职(久旱乞罢)。

把他留下来,就是折腾万历的。

几天后,礼部尚书于慎行上书,催促皇帝册立太子,语言比较激烈。万历也比较生气,罚了他三个月工资。

事情的发生,应该还算正常,不正常的,是事情的结局。

换在以往,申时行已经开始挥舞铁锹和稀泥了,先安慰皇帝,再安抚大臣,最后你好我好大家好,收工。

相比而言,王家屏要轻松得多,因为他只有一个意见----支持于慎行。

\section[\thesection]{}

工资还没扣,他就即刻上书,为于慎行辩解,说了一大通道理,把万历同志的脾气活活顶了回去。但更让人惊讶的
是,这一次,万历没有发火。

因为他发不了火,事情很清楚,内阁四个人,走了三个,留下来的这个,还是个二杆子,明摆着是要为难自己。而
且这位坚持战斗的王大人还说不得,再闹腾一次,没准就走人了,到时谁来收拾这个烂摊子?

可是光忍还不够,言官大臣赤膊上阵,内阁打黑枪,明里暗里都来,比逼宫还狠,不给个说法,是熬不过去了。

几天后,一个太监找到了王家屏,向他传达了皇帝的谕令:

``册立太子的事情,我准备明年办,不要再烦(扰)我了。''

王家屏顿时喜出望外,然而,这句话还没有讲完:

``如果还有人敢就此事上书,就到十五岁再说!''

朱常洛是万历十年出生的,万历发出谕令的时间是万历十八年,所以这句话的意思是说,如果你们再敢闹腾,这事
就六年后再办!

虽然不是无条件投降,但终究还是有了个说法,经过长达五年的斗争,大臣们胜利了----至少他们自己这样认为。

事情解决了,王家屏兴奋了,兴奋之余,就干了一件事。

他把皇帝的这道谕令告诉了礼部,而第一个获知消息的人,正是礼部尚书于慎行。

于慎行欣喜若狂,当即上书告诉皇帝:

``此事我刚刚知道,已经通报给朝廷众官员,要求他们耐心等候。''

万历气得差点吐了白沫。

因为万历给王家屏的,并不是正规的圣旨,而是托太监传达的口谕,看上去似乎没区别,但事实上,这是一个有深
刻政治用意的举动。

其实在古代,君无戏言这句话基本是胡扯,皇帝也是人,时不时编个瞎话,吹吹牛,也很正常,真正说了就要办
的,只有圣旨。白纸黑字写在上面,糊弄不过去。所以万历才派太监给王家屏传话,而他的用意很简单:这件事情
我心里有谱,但现在还不能办,先跟你通个气,以后遇事别跟我对着干,咱们慢慢来。

皇帝大人原本以为,王大学士好歹在朝廷混了几十年,这点觉悟应该还有,可没想到,这位一根筋的仁兄竟然把事
情捅了出去,密谈变成了公告,被逼上梁山了。

他当即派出太监,前去内阁质问王家屏,却得到了一个让他意想不到的答案。

\section[\thesection]{}

王家屏是这样辩解的:

``册立太子是大事,之前许多大臣都曾因上疏被罚,我一个人定不了,又被许多大臣误会,只好把陛下的旨意传达
出去,以消除大家的疑虑(以释众惑)。''

这番话的真正意思大致是这样的:我并非不知道你的用意,但现在我的压力也很大,许多人都在骂我,我也没办
法,只好把陛下拉出来背黑锅了。

虽然不上道,也是个老狐狸。

既然如此,就只好将错就错了,几天后,万历正式下发圣旨:

``关于册立皇长子为太子的事情,我已经定了,说话算数(诚待天下),等长子到了十岁,我自然会下旨,到时册立
出阁读书之类的事情一并解决,就不麻烦你们再催了。''

长子十岁,是万历十九年,也就是下一年,皇帝的意思很明确,我已经同意册立长子,你们也不用绕弯子,搞什么
出阁读书之类的把戏,让老子清净一年,明年就立了!

这下大家都高兴了,内阁的几位仁兄境况也突然大为改观,有病的病好了,忙的也不忙了,除王锡爵(母亲有病,回
家去了,真的)外,大家都回来了。

剩下来的,就是等了。一晃就到了万历二十年,春节过了,春天过了,都快要开西瓜了,万历那里一点消息都没有。

泱泱大国,以诚信为本,这就没意思了。

可是万历二十年毕竟还没过,之前已经约好,要是贸然上书催他,万一被认定毁约,推迟册立,违反合同的责任谁
都负担不起,而且皇上到底是皇上,你上疏说他耍赖,似乎也不太妥当。

一些脑子活的言官大臣就开始琢磨,既要敲打皇帝,又不能留把柄,想来想去,终于找到了一个完美的替代目
标----申时行。

没办法,申大人,谁让你是首辅呢?也只好让你去扛了。

很快,一封名为《论辅臣科臣疏》的奏疏送到了内阁,其主要内容,是弹劾申时行专权跋扈,压制言官,使得正确
意见得不到执行。

可怜,申首辅一辈子和稀泥,东挖砖西补墙,累得半死,临了还要被人玩一把,此文言辞尖锐,指东打西,指桑骂
槐,可谓是政治文本的典范。

文章作者,是南京礼部主事汤显祖,除此文外,他还写过另一部更有名的著作----牡丹亭。

\section[\thesection]{}

汤显祖,字义仍,江西临川人,上书这一年,他四十二岁,官居六品。

虽说四十多岁才混到六品,实在不算起眼。但此人绝非等闲之辈,早在三十年前,汤先生已天下闻名。

十三岁的时候,汤显祖就加入了泰州学派(也没个年龄限制),成为了王学的门人,跟着那帮``异端''四处闹腾,开
始出名。

二十一岁,他考中举人。七年后,到京城参加会试,运气不好,遇见了张居正。

之所以说运气不好,并非张居正讨厌他,恰恰相反,张首辅很赏识他,还让自己的儿子去和他交朋友。

这是件求之不得的好事,可问题在于,汤先生异端中毒太深,瞧不起张居正,摆了谱,表示拒不交友。

他既然敢跟张首辅摆谱,张首辅自然要摆他一道,考试落榜也是免不了的。三年后,他再次上京赶考,张首辅锲而
不舍,还是要儿子和他交朋友,算是不计前嫌。但汤先生依然不给面子,再次摆谱。首辅大人自然再摆他一道,又
一次落榜。

但汤先生不但有骨气,还有毅力,三年后再次赶考,这一次张首辅没有再阻拦他(死了),终于成功上榜。

由于之前两次跟张居正硬扛,汤先生此时的名声已经是如日中天。当朝的大人物张四维、申时行等人都想拉他,可
汤先生死活不搭理人家。

不搭理就有不搭理的去处,名声大噪的汤显祖被派到了南京,几番折腾,才到礼部混了个主事。

南京本来就没事干,南京的礼部更是闲得出奇,这反倒便宜了汤先生。闲暇之余开始写戏,并且颇有建树,日子过
得还算不错,直到万历十九年的这封上疏。

很明显,汤先生的政治高度比不上艺术高度,奏疏刚送上去,申时行还没说什么,万历就动手了。

对于这种杀鸡儆猴的把戏,皇帝大人一向比较警觉(他也常用这招),立马做出了反应,把汤显祖发配到边远地区(广
东徐闻)去当典史。

这是一次极其致命的打击,从此汤先生再也没能翻过身来。

万历这辈子罢过很多人的官,但这一次,是最为成功的。因为他只罢掉了一个六品主事,却换回一个明代最伟大的
戏曲家,赚大发了。

\section[\thesection]{}

二十八岁落榜后,汤显祖开始写戏。三十岁的时候,写出了《紫箫记》;三十八岁,写出了《紫钗记》。四十二岁
被赶到广东,七年后京察,又被狠狠地折腾了一回,索性回了老家。

来回倒腾几十年,一无所获。在极度苦闷之中,四十九岁的汤显祖回顾了自己戏剧化的一生,用悲凉而美艳的辞藻
写下了他所有的梦想和追求,是为《还魂记》,后人又称《牡丹亭》。

牡丹亭,全剧共十五出,描述了一个死而复生的爱情故事,(情节比较复杂,有兴趣自己去翻翻)。此剧音律流畅,
词曲优美,轰动一时,时人传诵:牡丹一出,西厢(《西厢记》)失色。此后传唱天下百余年,堪与之媲美者,唯有
孔尚任之《桃花扇》。

为官不济,为文不朽,是以无憾。

史赞:二百年来,一人而已。

总的说来,汤显祖的运气是不错的,因为更麻烦的事,他还没赶上。

汤先生上书两月之后,福建佥事李琯就开炮了,目标还是申时行。不过这次更狠,用词狠毒不说,还上升到政治高
度,一条条列下来,弹劾申时行十大罪,转瞬之间,申先生就成了天字第一号大恶人。

万历也不客气,再度发威,撤了李琯的职。

命令一下,申时行却并不高兴,反而唉声叹气,忧心忡忡。

因为到目前为止,虽然你一刀我一棍打个不停,但都是摸黑放枪,谁也不挑明。万历的合同也还有效,拖到年尾,
皇帝赖账就是理亏,到时再争,也是十拿九稳。

可万一下面这帮愤中愤老忍不住,玩命精神爆发,和皇帝公开死磕,事情就难办了。

俗语云:怕什么,就来什么。

工部主事张有德终于忍不住了,他愤然上书,要求皇帝早日册立太子。

等的就是你。

万历随即做出反应,先罚了张有德的工资,鉴于张有德撕毁合同,册立太子的事情推后一年办理。

这算是正中下怀,本来就不大想立,眼看合同到期,正为难呢,来这么个冤大头,不用白不用。册立的事情也就能
堂而皇之地往后拖了。

事实上,这是他的幻想。

因为在大臣们看来,这合同本来就不合理,忍气吞声大半年,那是给皇帝面子,早就一肚子苦水怨气没处泻,你敢
蹦出来,那好,咱们就来真格的!

\section[\thesection]{}

当然,万历也算是老运动员了。对此他早有准备,无非是来一群大臣瞎咋呼,先不理,闹得厉害再出来说几句话,
把事情熬过去,完事。

形势的发展和他的预料大致相同,张有德走人后,他的领导,工部尚书曾同亨就上书了,要求皇帝早日册立太子。

万历对此嗤之以鼻,他很清楚,这不过是个打头的,大部队在后。下面的程序他都能背出来,吵吵嚷嚷,草草收
场,实在毫无新鲜可言。

然而当下一封奏疏送上来的时候,他才知道,自己错了。

这封奏疏的署名人并不多,只有三个,分别是申时行、许国、王家屏。

但对万历而言,这是一个致命的打击。

因为之前无论群臣多么反对,内阁都是支持他的。即使以辞职回家相威胁,也从未公开与他为敌,是他的最后一道
屏障,现在竟然公开站出来和他对着干,此例一开,后果不堪设想。

特别是申时行,虽说身在内阁,时不时也说两句,但那都是做给人看的。平日里忙着和稀泥,帮着调节矛盾,是名
副其实的卧底兼间谍。

可这次,申时行连个消息都没透,就打了个措手不及,实在太不够意思,于是万历私下派出了太监,斥责申时行。
一问,把申时行也问糊涂了,因为这事他压根就不知道!

事情是这样的,这封奏疏是许国写的,写好后让王家屏署名,王兄自然不客气,提笔就签了名,而申时行的底细他
俩都清楚,这个老滑头死也不会签,于是许大人胆一壮,代申首辅签了名,拖下了水。

事已至此,申大人只能一脸无辜的表白:

``名字是别人代签的,我事先真不知道。''

事情解释了,太监也回去了,可申先生却开始琢磨了:万一太监传达不对怎么办?万一皇帝不信怎么办?万一皇帝再
激动一次,把事情搞砸怎么办?

想来想去,他终于决定,写一封密信。

这封密信的内容大致是说,我确实不知道上奏的事情,这事情皇上你不要急,自己拿主意就行。

客观地讲,申时行之所以说这句话,倒不一定是耍两面派,因为他很清楚皇帝的性格:

像万历这号人,属于死要面子活受罪,打死也不认错的。看上去非常随和,实际上极其固执,和他硬干,是没有什
么好处的。

\section[\thesection]{}

所以申时行的打算,是先稳住皇帝,再慢慢来。事实确如所料,万历收到奏疏后,十分高兴,当即回复:

``你的心意我已知道,册立的事情我已有旨意,你安心在家调养就是了。''

申时行总算松了口气,事情终于糊弄过去了。

但他做梦也想不到,他长达十年的和稀泥生涯,将就此结束----因为那封密信。

申时行的这封密信,属于机密公文,按常理,除了皇帝,别人是看不见的。

可是在几天后的一次例行公文处理中,万历将批好的文件转交内阁,结果不留神,把这封密信也放了进去。

这就好比拍好了照片存电脑,又把电脑拿出去给人修,是个要命的事。

文件转到内阁,这里是申时行的地盘,按说事情还能挽回。可问题在于申大人为避风头,当时还在请病假,负责工
作的许国也没留意,顺手就转给了礼部。

最后,它落在了礼部给事中罗大纮的手里。

罗大纮,江西吉水人。关于这个人,只用一句就能概括:一个称职的言官。

看到申时行的密信后,罗大纮非常愤怒,因为除了耍两面派外,申时行在文中还写了这样一句话:惟亲断亲裁,勿
因小臣妨大典。这句话说白了,就是你自己说了算,不要理会那些小臣。

我们是小臣,你是大臣?!

此时申时行已经发现了密信外泄,他十分紧张,立刻找到了罗大纮的领导,礼部科给事中胡汝宁,让他去找罗大纮
谈判。

可惜罗大纮先生不吃这一套,写了封奏疏,把这事给捅了出去,痛骂申时行两面派。

好戏就此开场,言官们义愤填膺。吏部给事中钟羽正、候先春随即上书,痛斥申时行,中书黄正宾等人也跟着凑热
闹,骂申时行老滑头。

眼看申首辅吃亏,万历当即出手,把罗大纮赶回家当了老百姓,还罚了上书言官的工资。

但事情闹到这个份上,已经无法收拾了。

经历过无数大风大浪的申时行,终究在阴沟里翻了船。自万历十年以来,他忍辱负重,上下协调,独撑大局,打落
门牙往肚里吞,至今已整整十年。

现在,他再也支撑不下去了。

万历十九年(1591)九月,申时行正式提出辞职,最终得到批准,回乡隐退。

大乱就此开始。

\section[\thesection]{}

申时行在的时候,大家都说朝廷很乱,等申时行走了,大家才知道,什么叫乱。

首辅走了,王锡爵不在,按顺序,应该是许国当首辅。可这位兄弟相当机灵,一看形势不对,写了封辞职信就跑了。

只剩王家屏了。

万历不喜欢王家屏,王家屏也知道皇帝不喜欢他,所以几乎在申时行走人的同时,他就提出辞职。

然而万历没有批,还把王家屏提为首辅。原因很简单,这么个烂摊子,现在内阁就这么个人,好歹就是他了。

内阁总算有个人了,但一个还不够,得再找几个。搭个班子,才好唱戏。说起来还是申时行够意思,早就料到有这
一天,所以在临走时,他向万历推荐了两个人:一个是时任吏部左侍郎赵志皋,另一个是原任礼部右侍郎张位。

这个人事安排十分有趣,因为这两个人兴趣不同,性格不同,出身不同,总而言之,就没一点共同语言,但事后证
明,就是这么个安排,居然撑了七八年,申先生的领导水平可见一斑。

班子定下来了,万历的安宁日子也到了头。因为归根结底,大臣们闹腾,还是因为册立太子的事情,申先生不过是
帮皇帝挡了子弹,现在申先生走了,皇帝陛下只能赤膊上阵。

万历二十年(1592)正月,真正的总攻开始了。

礼部给事中李献可首先发难,上书要求皇帝早日批准长子出阁读书,而且这位兄台十分机灵,半字不提册立的事,
全篇却都在催这事,半点把柄都不留,搞得皇帝陛下十分狼狈,一气之下,借口都不找了:

``册立已有旨意,这厮偏又来烦扰……好生可恶,降级调外任用!''

其实说起来,李献可不是什么大人物,这个处罚也不算太重。可万历万没想到,就这么个小人物,这么点小事儿,
他竟然没能办得了。

因为他的圣旨刚下发,就被王家屏给退了回来。

作为朝廷首辅,如果认为皇帝的旨意有问题,可以退回去,拒不执行,这种权力,叫做封还。

封还就封还吧,不办就不办吧,更可气的是,王首辅还振振有词:

这事我没错,是皇帝陛下错了!因为李献可没说册立的事,他只是说应该出阁读书,你应该采纳他的意见,即使不能
采纳,也不应该罚他,所以这事我不会办。

\section[\thesection]{}

真是要造反了,刚刚提了首辅,这白眼狼就下狠手。万历恨不得拿头撞墙,气急败坏之下,他放了王家屏的假,让
他回家休养去了。

万历的``幸福''生活从此拉开序幕。

几天后,礼部给事中钟羽正上疏,支持李献可,经典语言如下:

``李献可的奏疏,我是赞成的,请你把我一同降职吧(请与同谪)。''

万历满足了他的要求。

又几天后,礼部给事中舒弘绪上疏,发言如下:

``言官是可以处罚的,出阁读书是不能不办的。''

发配南京。

再几天后,户部给事中孟养浩上疏,支持李献可、钟羽正等人。相对而言,他的奏疏更有水平,虽然官很小(七
品),志气却大,总结了皇帝大人的种种错误,总计五条,还说了一句相当经典的话:

``皇帝陛下,您坐视皇长子失学,有辱宗社祖先!''

万历气疯了,当即下令,把善于总结的孟养浩同志革职处理,并拉到午门,打了一百杖。

暴风雨就是这样诞生的。

别人也就罢了,可惜孟先生偏偏是言官,干的是本职工作,平白被打实在有点冤。

于是大家都愤怒了。

请注意,这个大家是有数的,具体人员及最终处理结果如下所列:

内阁大学士赵志皋上疏,被训斥。

吏科右给事中陈尚象上疏,被革职为民。

御史邹德泳,户科都给事中丁懋逊、兵科都给事中张栋、刑科都给事中吴之佳、工科都给事中杨其休,礼科左给事
中叶初春,联名上疏抗议。万历大怒,将此六人降职发配。

万历终于做了一件了不起的事情。如果加上最初上疏的李献可,那么在短短的几天之内,他就免掉了十二位当朝官
员。这一伟大记录,就连后来的急性子崇祯皇帝也没打破。

事办到这份上,皇帝疯了,大臣也疯了。官服乌纱就跟白送的一样,铺天盖地到处乱扔,大不了就当老子这几十年
书白读了。拼个你死我活只为一句话:可以丢官,不能丢人!

在这一光辉思想的指导下,礼部员外郎董嗣成、御史贾名儒、御史陈禹谟再次上疏,支持李献可。万历即刻反击,
董嗣成免职,贾名儒发配,陈禹谟罚工资。

事情闹到这里,到底卷进来多少人,我也有点乱。但若以为就此打住,那实在是低估了明代官员的战斗力。

\section[\thesection]{}

几天后,礼部尚书李长春也上疏了。对这位高级官员,万历也没客气,狠狠地骂了他一顿,谁知没多久,吏部尚书
蔡国珍、侍郎杨时乔又上疏抗议,然而这一次,万历没有做出任何反应----实在骂不动了。

皇帝被搞得奄奄一息,王家屏也坐不住了,他终于出面调停,向皇帝认了错,并希望能够赦免群臣。

想法本是好的,方法却是错的。好不容易消停下去的万历,一看见这个老冤家,顿时恢复了战斗力,下书大骂:

``自你上任,大臣狂妄犯上,你是内阁大学士,不但不居中缓和矛盾,反而封还我的批示,故意激怒我!见我发怒,
你又说你有病在身,回家休养!国家事务如此众多,你在家躺着(高卧),心安吗!?既然你说有病,就别来了,回家养
病去吧!''

王家屏终于理解了申时行的痛苦,万历二十年(1592)三月,他连上八封奏疏,终于回了家。

这是一场实力不对等的较量,大臣的一句话,可能毫无作用,万历的一道圣旨,却足以改变任何人的命运。

然而万历失败了,面对那群前仆后继的人,他虽然竭尽全力,却依然失败了,因为权力并不能决定一切----当它面
对气节与尊严的时候。

王家屏走了,言官们暂时休息了。接班的赵志皋比较软,不说话,万历正打算消停几天,张位又冒出来了。

这位次辅再接再厉,接着闹,今天闹出阁讲学,明天就闹册立太子。每天变着法地折腾皇帝,万历同志终于顶不住
了。如此下去,不被逼死,也被憋死了。

必须想出对策。

考虑再三,他决定去找一个人,在他看来,只有这个人才能挽救一切。

万历二十一年(1593),王锡爵奉命来到京城,担任首辅。

王锡爵,字元驭,苏州太仓人。

嘉靖四十一年,他二十八岁,赴京赶考,遇见申时行,然后考了第一。

几天后参加殿试,又遇见了申时行,这次他考了第二。

据说他之所以在殿试输给申时行,不外乎两点,一是长得不够帅,二是说话不够滑。

帅不帅不好说,滑不滑是有定论的。

\section[\thesection]{}

自打进入朝廷,王锡爵就是块硬骨头。万历五年张居正夺情,大家上书闹,他跑到人家家里闹,逼得张居正大人差
点拔刀自尽。吴中行被打得奄奄一息,大家在场下吵,他跑到场上哭。

万历六年,张居正不守孝回京办公。大家都庆贺,他偏请假,说我家还有父母,实在没有时间工作,要回家尽孝,
张居正恨得直磨牙。

万历九年,张居正病重,大家都去祈福,他不屑一顾。

万历十年,张居正病逝,反攻倒算开始,抄家闹事翻案,人人都去踩一脚,这个时候,他说:

``张居正当政时,做的事情有错吗?!他虽为人不正,却对国家有功,你们怎能这样做呢?!''

万历十三年,他的学生李植想搞倒申时行,扶他上台,他痛斥对方,请求辞职。

三年后,他的儿子乡试考第一,有人怀疑作弊,他告诉儿子,不要参加会试,回家待业,十三年后他下了台,儿子
才去考试,会试第二,殿试第二。

他是一个经得起时间考验的人。

所以在万历看来,能收拾局面的,也只有王锡爵了。

王大人果然不负众望,到京城一转悠,就把情况摸清促了。随即开始工作,给皇帝上了一封密信。大意是说,目前
情况十分紧急,请您务必在万历二十一年册立太子,绝不能再拖延了,否则我就是再有能耐,也压制不了!

吸取了上次的教训,万历没敢再随便找人修电脑,专程派了个太监,送来了自己的回信。

可王锡爵刚打开信,就傻眼了。

信上的内容是这样的:

``看了你的奏疏,为你的忠诚感动!我去年确实说过,今年要举行册立大典,但是(注意此处),我昨天晚上读了祖训
(相当于皇帝的家规),突然发现里面有一句训示:立嫡不立长,我琢磨了一下,皇后现在年纪还不大,万一将来生
了儿子,怎么办呢?是封太子,还是封王?''

``如果封王,那就违背了祖训,如果封太子,那就有两个太子了,我想来想去,想了个办法,要不把我的三个儿子
一起封王,等过了几年,皇后没生儿子,到时候再册立长子也不迟。这事我琢磨好了,既不违背祖制,也能把事办
了,很好,你就这么办吧。''

阶级斗争又有新动向了,很明显,万历同志是很动了一番脑筋,觉得自己不够分量,把老祖宗都搬出来了,还玩了
个复杂的逻辑游戏,有相当的技术含量,现解析如下。

\section[\thesection]{}

按老规矩,要立嫡子(皇后的儿子),可是皇后又没生儿子,但皇后今天没有儿子,不代表将来没有。如果我立了长
子,嫡子生出来,不就违反政策了吗?但是皇后什么时候生儿子,我也不知道,与其就这么拖着,还不如把现在的三
个儿子一起封了了事,到时再不生儿子,就立太子。先封再立,总算对上对下都有了交代。

王锡爵初一琢磨,就觉得这事有点悬,但听起来似乎又只能这么办,思前想后,他也和了稀泥,拿出了两套方案。

方案一、让皇长子拜皇后为母亲,这样既是嫡子又是长子,问题就解决了。

方案二、按照皇帝的意思,三个儿子一起封王,到时再说。

附注:第二套方案,只有在万不得已的时候,才能使用。

上当了,彻底上当了。

清醒了一辈子的王大人,似乎终于糊涂了,他好像并不知道,自己已经跳入了一个陷阱。

事实上,万历的真正目标,不是皇长子,而是皇三子。

他喜欢郑贵妃,喜欢朱常洵,压根就没想过要立太子,搞三王并封,把皇长子、三子封了王,地位就平等了,然后
就是拖,拖到大家都不闹了,事情也就办成了。

至于所谓万不得已,采用第二方案,那也是句废话,万历同志这辈子,那是经常地万不得已。

总之,王锡爵算是上了贼船了。

万历立即选择了第二种方案,并命令王锡爵准备执行。

经过长时间的密谋和策划,万历二十一年(1593)正月二十六日,万历突然下发圣旨:

``我有三个儿子,长幼有序。但问题是,祖训说要立嫡子,所以等着皇后生子,一直没立太子,为妥善解决这一问
题,特将皇长子、皇三子、皇五子全部封王,将来有嫡子,就立嫡子,没嫡子,再立长子,事就这么定了,你们赶
紧去准备吧。''

圣旨发到礼部,当时就炸了锅。这么大的事情,事先竟没听到风声,实在太不正常,于是几位领导一合计,拿着谕
旨跑到内阁去问。

这下连内阁的赵志皋和张位也惊呆了,什么圣旨,什么三王并封,搞什么名堂!?

很明显,这事就是王锡爵办的。消息传出,举朝轰动,大家都认定,朝廷又出了个叛徒,而且还是主动投靠的。

\section[\thesection]{}

所有人都知道,万历已经很久不去找(幸)皇后了,生儿子压根就是没影的事。所谓三王并封,就是扯淡,大家都能
看出来,王锡爵你混了几十年,怎么看不出来?分明就是同谋,助纣为虐!

再说皇帝,你都说好了,今年就办,到时候了竟然又不认账。搞个什么三王并封,我们大家眼巴巴地盼着,又玩花
样,你当你耍猴子呢?!

两天之后,算帐的人就来了。

光禄寺丞朱维京第一个上书,连客套话都不说,开篇就骂:

``您先前说过,万历二十一年就册立太子,朝廷大臣都盼着,忽然又说要并封,等皇后生子。这种说法,祖上从来
就没有过!您不会是想愚弄天下人吧!''

把戏被戳破了,万历很生气,立即下令将朱维京革职充军。

一天后,刑部给事中王如坚又来了:

``十四年时,您说长子幼小,等个两三年;十八年时,您又说您没有嫡子,长幼有序,让我们不必担心;十九年
时,您说二十年就册立;二十年时,您又说二十一年举行;现在您竟然说不办了,改为分封,之前的话您不是都忘
了吧,以后您说的话,我们该信那一句?''

这话杀伤力实在太大,万历绷不住了,当即把王如坚免职充军。已经没用了,什么罚工资、降职、免职、充军,大
家都见识过了,还能吓唬谁?

最尴尬的,是礼部的头头脑脑们,皇帝下了圣旨,内阁又没有封还,按说是不能不办的。可是照现在这么个局势,
如果真要去办,没准自己就被大家给办了。想来想去,搞了个和稀泥方案:三王并封照办,但同时也举行册立太子
的仪式。

方案报上去,万历不干:三王并封,就为不立太子,还想把我绕回去不成?

既然给面子皇帝都不要,也就没啥说的了。礼部主事顾允成,工部主事岳元声,光禄寺丞王学曾等人继续上书,反
对三王并封,这次万历估计也烦了,理都不理,随他们去。

于是抗议的接着抗议,不理的照样不理,谁也奈何不了谁。

局面一直僵持不下,大家这才突然发觉,还漏了一个关键人物----王锡爵。

这事既然是王锡爵和皇帝干的,皇帝又不出头,也只能拿王锡爵开刀了。

\section[\thesection]{}

先是顾允成、张辅之等一群王锡爵的老乡上门,劝他认清形势,早日解决问题。然后是吏部主事顾宪成代表吏部全
体官员写信给王锡爵,明白无误地告诉他:现在情况很复杂,大家都反对你的三王并封,想糊弄过去是不行的,

王锡爵终于感受到了当年张居正的痛苦,不问青红皂白,就围上来群殴,没法讲道理,就差打上门来了。

当然,一点也没差,打上门的终究来了。

几天之后,礼部给事中史孟麟、工部主事岳元声一行五人,来到王锡爵办公的内阁,过来只干一件事:吵架。

刚开始的时候,气氛还算不错,史孟麟首先发言,就三王并封的合理性、程序性一一批驳,有理有节,有根有据。

事情到这儿,还算是有事说事,可接下来,就不行了。

因为王锡爵自己也知道,三王并封是个烂事,根本就没法辩,心里理亏,半天都不说话。对方一句句地问,他半句
都没答,憋了半天,终于忍不住了:

``你们到底想怎么样?''

岳元声即刻回答:

``请你立刻收回那道圣旨,别无商量!''

接着一句:

``皇上要问,就说是大臣们逼你这么干的!''

王锡爵气得不行,大声回复:

``那我就把你们的名字都写上去,怎么样?!''

这是一句威胁性极强的话。然而岳元声回答的声音却更大:

``那你就把我的名字写在最前面!充军也好,廷杖也好,你看着办!''

遇到这种不要命的二愣子,王锡爵也没办法,只好说了软话:

``请你们放心,虽然三王并封,但皇长子出阁的时候,礼仪是不一样的。''

首辅大人认输了,岳元声却不依不饶,跟上来就一句:

``那是礼部的事,不是你的事!''

谈话不欢而散,王锡爵虽然狼狈不堪,却也顶住了死不答应。因为虽然骂者众多,却还没有一个人能够找到他的死
穴。

这事看起来很简单,万历耍了个计谋,把王锡爵绕了进去,王大人背黑锅,哑巴吃黄连,有苦说不出。

事实上,那是不可能的,王锡爵先生,虽然人比较实诚,也是在官场打滚几十年的老油条,万历那点花花肠子,他
一清二楚,之所以同意三王并封,是将计就计。

他的真正动机是,先利用三王并封,把皇长子的地位固定下来,然后借机周旋,更进一步逼皇帝册立太子。

\section[\thesection]{}

在他看来,岳元声之流都是白颈乌鸦,整天吵吵嚷嚷,除了瞎咋呼,啥事也干不成。所以他任人笑骂,准备忍辱负
重,一朝翻身。

然而这个世界上,终究还是有聪明人的。

庶吉士李腾芳就算一个。

李腾芳,湖广湘潭人(今湖南湘潭)。从严格意义上讲,他还不是官,但这位仁兄人还没进朝廷,就有了朝廷的悟
性,只用一封信就揭破了王锡爵的秘密。

他的这封信,是当面交给王锡爵的,王大人本想打发这人走,可刚看几行字,就把他给拉住了:

``公欲暂承上意,巧借王封,转作册立!''

太深刻了,太尖锐了,于是王锡爵对他说:

``请你坐下来,好好谈一谈。''

李腾芳接下来的话,彻底打乱了王锡爵的部署:

``王大人,你的打算是对的。但请你想一想,封王之后,恐怕册立还要延后,你还能在朝廷呆多久?万一你退了,接
替你的人比你差,办不成这件事,负责任的人就是你!''

王锡爵沉默了,他终于意识到,自己的计划蕴含着极大的风险,但他仍然不打算改正这个错误。因为在这个计划
里,还有最后一道保险。

李腾芳走了,王锡爵没有松口,此后的十几天里,跑来吵架的人就没断过。但王大人心里有谱,打死也不说,直到
王就学上门的那一天。

王就学是王锡爵的门生,自己人当然不用客气,一进老师家门就哭,边哭还边说:

``这件事情(三王并封)大家都说是老师干的,如此下去,恐怕老师有灭门之祸啊!''

王锡爵却笑了:

``你放心吧,那都是外人乱说的。我的真实打算,都通过密奏交给了皇上,即使皇长子将来登基,看到这些文书,
也能明白我的心意。''

这就是王先生的保险,然而王就学没有笑,只说了一句话:

``老师,别人是不会体谅您的!一旦出了事,会追悔莫及啊!''

王锡爵打了个寒战,他终于发现,自己的思维中,有一个不可饶恕的漏洞:

如果将来册立失败,皇三子登基,看到了自己拥立长子的密奏,必然会收拾掉自己。

而如果皇长子登基,即使他知道密奏,也未必肯替自己出头。因为长子登基,本来就是理所当然,犯不着感谢谁,
到时,三王并封的黑锅只有他自己背。

所以结论是:无论谁胜利,他都将失败!

\section[\thesection]{}

明知是赔本的生意,还要做的人,叫做傻子。王锡爵不是傻子,自然不做。万历二十一年二月,他专程拜见了万
历,只提出了一个要求:撤回三王并封。

这下万历就不干了,好不容易把你拉上船,现在你要洗手不干,留下我一个人背黑锅,怎么够意思?

``你要收回此议,即无异于认错,如果你认错,我怎么办?我是皇帝,怎能被臣下挟持?''

话说得倒轻巧,可惜王大人不上当:你是皇帝,即使不认错,大家也不能把你怎么样,我是大臣,再跟着淌混水,
没准祖坟都能让人刨了。

所以无论皇帝大人连哄带蒙,王锡爵偏一口咬定----不干了。

死磨硬泡没办法,大臣不支持,内阁不支持,唯一的亲信跑路,万历只能收摊了。

几天后,他下达谕令:

``三王都不必封了,再等两三年,如果皇后再不生子,就册立长子。''

可是大臣们不依不饶,一点也不消停,接着起哄,因为大家都知道,皇帝陛下您多少年不去找皇后了,皇后怎么生
儿子,不想立就不想立,你装什么蒜?

万历又火了,先是辟谣,说今年已经见过皇后,夫妻关系不好,纯属谣传,同时又下令内阁,对敢于胡说八道的
人,一律严惩不怠。

这下子王锡爵为难了,皇帝那里他不敢再去凑热闹了,大臣他又得罪不起,想来想去,一声叹息:我也辞职吧。

说是这么说,可是皇帝死都不放,因为经历了几次风波之后,他已然明白,在手下这群疯子面前,一丝不挂十分危
险,身前必须有个挡子弹的,才好平安过日子。

于是王锡爵惨了,大臣轰他走,皇帝不让走,夹在中间受气,百般无奈之下,他决定拼一拼--找皇帝面谈。

可是皇帝大人虽然不上班,却似乎很忙,王锡爵请示了好几个月,始终不见回音。眼看要被唾沫淹死,王大人急眼
了,死磨硬泡招数全用上,终于,万历二十一年(1593)十一月,他见到了万历。

这是一次十分关键的会面,与会者只有两人,本来是天知地知,你知我知,但出于某种动机(估计是想保留证据),
事后王锡爵详细地记下了他们的每一句话。

\section[\thesection]{}

等了大半年,王锡爵已经毫无耐心:

``册立一事始终未定,大臣们议论纷纷,烦扰皇上(包括他自己),希望陛下早日决断,大臣自然无词。''

万历倒还想得开:

``我的主意早就定了,反正早晚都一样,人家说什么不碍事。''

不碍事?敢情挨骂的不是你。

可这话又不能明说,于是王大人兜了圈子:

``陛下的主意已定,我自然是知道的,但外人不知道内情,偏要大吵大嚷,我为皇上受此非议深感不忿,不知道您
有什么为难之处,要平白受这份闲气?''

球踢过来了,但万历不愧为老运动员,一脚传了回去:

``这些我都知道,我只担心,如果皇后再生儿子,该怎么办?''

王锡爵气蒙了,就为皇后生儿子的破事,搞了三王并封,闹腾了足足半年,到现在还拿出来当借口,还真是不要
脸,既然如此,就得罪了:

``陛下,您这话几年前说出来,还过得去,现在皇子都十三岁了,还要等到什么时候!从古至今即使百姓家的孩子,
十三岁都去读书了,何况还是皇子?!''

这已经是老子训儿子的口气了,但万历同志到底是久经考验,毫不动怒,只是淡淡地说:

``我知道了。''

王锡爵仍不甘心,继续劝说万历,但无论他讲啥,皇帝陛下却好比橡皮糖,全无反应,等王大人说得口干舌燥,气
喘吁吁,没打招呼就走人了,只留下王大人,痴痴地看着他离去的背影。

谈话是完了,但这事没完,王锡爵回家之后,实在是气不过,一怒之下,又写了一封胆大包天的奏疏。

因为这封奏疏的中心意思只有一个----威胁:

``皇上,此次召对(即谈话),虽是我君臣二人交谈,但此事不久后,天下必然知晓,若毫无结果,将被天下人群起
攻之,我即使粉身碎骨,全家死绝,也无济于事!''

这段话的意思是说,我和你谈过话,别以为大家都不知道,如果没给我一个结果,此事必将公之于天下,我完蛋
了,你也得下水!

这是硬的,还有软的:

``臣进入朝廷三十余年了,一向颇有名声,现在为了此事,被天下人责难,实在是痛心疾首啊!''

王锡爵是真没办法了,可万历却是王八吃秤砣,铁了心地对着干,当即写了封回信,训斥了王锡爵,并派人送到了
内阁。

按照常理,王大人看完信后,也只能苦笑,因为他虽为人刚正,却是个厚道人,从来不跟皇帝闹,可这一次,是个
例外。

\section[\thesection]{}

因为当太监送信到内阁的时候,内阁的张位恰好也在。这人就没那么老实了,是个喜欢惹事的家伙,王锡爵拆信的
时候,他也凑过来看。看完后,王锡爵倒没什么,他反而激动了。

这位仁兄二话不说,当即怂恿王锡爵,即刻上疏驳斥万历。有了张位的支持,王锡爵浑似喝了几瓶二锅头,胆也壮
了,针锋相对,写了封奏疏,把皇帝大人批驳得无地自容。

王锡爵没有想到,他的这一举动,却起到了意想不到的效果。

因为万历虽然顽固,却很机灵。他之所以敢和群臣对着干,无非是有内阁支持,现在王大人反水了,如果再闹下
去,恐怕事情就没法收拾,于是他终于下圣旨:万历二十二年春,皇长子出阁读书。

胜利在意想不到的时候来临了,王锡爵如释重负,虽然没有能够册立太子,但已出阁读书。无论如何,对内对外,
都可以交代了。

申时行没有办成的事情,王锡爵办成了,按说这也算是个政绩工程,王大人的位置应该更稳才是,然而事实并非如
此。

因为明代的大臣很执着,直来直往,说是册立,就必须册立。别说换名义,少个字都不行!所以出阁读书,并不能让
他们满意,朝廷里还是吵吵嚷嚷地闹个不停。

再加上另一件事,王锡爵就真是无路可走了。

因为万历二十一年(1593),恰好是京察年。

所谓京察,之前已介绍过,大致相当于干部考核,每六年京察一次,对象是全国五品以下官员(含五品),包括全国
所有的地方知府及下属、以及京城的京官。

虽然一般说来,明代的考察大都是糊弄事。但京察不同,因为管理京察的,是六部尚书之首的吏部尚书。收拾不了
内阁大学士,搞定几个五品官还是绰绰有余的。

所以每隔六年,大大小小的官员们就要胆战心惊一回。毕竟是来真格的,一旦京察被免官,就算彻底完蛋。

这还不算,最倒霉的是,如果运气不好,主持考核的是个死脑筋的家伙,找人说情都没用,那真叫玩的就是心跳。

万历二十一年(1593)的这次京察,就是一次结结实实的心跳时刻。因为主持者,是吏部尚书孙鑨和考功司郎中赵南
星。

\section[\thesection]{}

孙鑨倒没什么,可是赵南星先生,就真是个百年难得一遇的顽固型人物。

赵南星,字梦白,万历二年进士。早在张居正当政时期,他就显示了自己的刺头本色,一直对着干。张居正死后获
得提升,也不好好干,几年后就辞职回家了,据他自己说是身体不好,不想干了。

此人不贪钱,不好色,且认死理,此前不久才再次出山,和吏部尚书一起主持京察。

这么个人来干这么个事,很明显,就是来折腾人的。

果不其然,京察刚一开始,他就免了两个人的官,一个是都给事中王三余,另一个是文选司员外郎吕胤昌。

朝廷顿时一片恐慌。

因为这两个人的官虽不大,身份却很特殊:王三余是赵南星的亲家,吕胤昌是孙鑨的外甥。

拿自己的亲戚开刀,意思很明白:今年这关,你们谁也别想轻易过去。

官不聊生的日子就此开始,六部及地方上的一大批官员纷纷落马,哭天喊地,声震寰宇,连内阁大学士也未能幸免。
赵志皋的弟弟被赶回了家,王锡爵的几个铁杆亲信也糟了殃。

赵志皋是个老实人,也不怎么闹。王锡爵就不同了,他上门逼张居正的时候,赵南星也就是个小跟班,要说闹事,
你算老几?

很快,几个言官便上疏攻击吏部的人事安排,从中挑刺。赵南星自然不甘示弱,上疏反驳,争论了几天,皇帝最后
判定:吏部尚书孙鑨罚一年工资,吏部考功司郎中赵南星官降三级。

这个结果实在不值得惊讶,因为那段时间,皇帝大人正在和王锡爵合伙搞三王并封。

但王锡爵错了,因为赵南星先生,绝不是一个单纯的人。

事实上,他之所以被拉到前台,去搞这次京察,是因为在幕后,有个人在暗中操纵着一切。

这个人的名字,叫顾宪成。

关于这位仁兄的英雄事迹,后面还要详细介绍,这里就不多说了,但可以确定的是,万历二十一年的这次京察,是
在顾宪成的策划下,有预谋,有目的的政治攻击。关于这一点,连修明史的史官都看得清清楚楚(明史•顾宪成传)。

事实印证了这一点,前台刚刚下课,后台就出手了。一夜之间,左都御史李世达、礼部郎中于孔兼等人就冒了出
来,纷纷上疏攻击,王大人又一次成为了靶子。

\section[\thesection]{}

关键时刻,万历同志再次证明,他是讲义气的,而且也不傻。

奏疏送上去,他压根就没理,却发布了一道看似毫不相干的命令:吏部尚书孙鑨免职,吏部考功司郎中赵南星,削
职为民。

这条圣旨的意思是:别跟我玩花样,你们那点把戏我都明白,再闹,就连你们一起收拾。

应该说效果十分明显,很快,大家都不闹了。看上去,王锡爵赢了,实际上,他输了,且输得很惨。

因为孙鑨本就是个背黑锅的角色,官免了也就消停了。赵南星就不同了,硬顶王锡爵后,他名望大增,被誉为不畏
强暴,反抗强权的代表人物。虽然打包袱回了老家,却时常有人来拜访,每年都有上百道奏疏送到朝廷,推荐他出
来做官。而这位兄弟也不负众望,二十年后再度出山,闹出了更大的动静。

王锡爵就此完蛋,他虽然赢得了胜利,却输掉了名声,在很多人看来,残暴的王锡爵严酷镇压了开明的赵南星,压
制了正直与民意。

这是一件十分有趣的事情,因为这一切,都似曾相识。

十六年前,年轻官员王锡爵大摇大摆地迈进了张居正首辅的住所,慷慨激昂,大发议论后,扬长而去,然后名声大
噪。

十六年后,年轻官员赵南星向王锡爵首辅发起攻击,名满天下。

当年的王锡爵,就是现在的赵南星,现在的王锡爵,就是当年的张居正,很有趣。

有明一代,所谓的被压制者,未必真被压制,所谓的压制者,未必真能压制。

遍览明代史料,曾见直言犯上者无数,细细分析之后,方才发觉:犯上是一定的,直言是不一定的。因为在那些直
言背后,往往隐藏着不可告人的目的。

最后一根稻草

万历二十二年(1594)五月,王锡爵提出辞呈。

万历挽留了他很多次,但王锡爵坚持要走。

自进入朝廷以来,王锡爵严于律己,公正廉洁,几十年来如履薄冰,兢兢业业,终成大器。

万历二十一年,他受召回到朝廷担任首辅,二十二年离去,总共干了一年。

但这一年,就毁掉了他之前几十年累积的所有名声。

\section[\thesection]{}

虽然他忍辱负重,虽然他尽心竭力,努力维护国家运转,调节矛盾,甚至还完成了前任未能完成的事(出阁读书),
却再也无法支撑下去。

因为批评总是容易的,做事总是不容易的。

王锡爵的离去,标志着局势的进一步失控。从此以后,天下将不可收拾。

但没有人会料到,王大人辞职,将成为另一事件的导火线。和这件事相比,所谓的朝局纷争,册立太子,都不过是
小儿科而已。

因为首辅走了,日子却还得过,原本排第二的赵志皋应该接班,但这人实在太软,谁都敢欺负他,上到皇帝,下到
大臣,都觉得他压不住阵,于是皇帝下令,由大臣推荐首辅。

幕后人物顾宪成就此出马。

顾宪成,字叔时,江苏无锡人。万历四年参加乡试,考中第一名解元。三年后去考了进士,成绩平平,分配到户部
当了个主事。当官后,最不喜欢的人是张居正,平日怎么别扭怎么来。

比如张大人病重,大家都去上疏祷告,他不去,别人看他不上路,帮他署了名,他知道后不肯干休,非把自己的名
字划掉,那是相当执着。不过这也没什么,当时和张大人对着干的人多了去了,不缺他一个。

等到张居正死了,他就去了吏部,但也没升官,还接着当六品主事(正处级),这中间还请了三年假。

总之,这是个并不起眼的人。

万历二十一年京察时,孙鑨是吏部尚书(正二品),赵南星是考功司郎中(相当于司长,正五品),而顾宪成只是个考
功司员外郎(副手,从五品)。

万历八年进入朝廷,就当六品主事,混了十三年,才升了一级,实在有点说不过去。

但就是这么个说不过去的人,却是这场风暴的幕后操纵者(实左右之),不服都不行。

更为神奇的是,事情闹大了,孙鑨撤职了,赵南星回家了,连王首辅都辞职了,他却是巍然不动。非但不动,还升
了一级,当上了吏部文选司郎中。

之前说过,文选司负责官员人事选拔,是吏部第一肥差。根据史料的记载,顾宪成大致属于性格顽固,遇事不转弯
的人,如此个性,竟然能捞到这位置,实在有点不可思议。

不可思议的事情还在后面,当初孙鑨刚被免职的时候,吏部没有部长,王锡爵打算趁机换人,推荐自己的亲信罗万
化接班。顾宪成反对,推荐了右都御史陈有年。

最终结果:吏部尚书陈有年。

\section[\thesection]{}

你要知道,王锡爵大人此时的职务,是内阁首辅、建极殿大学士,领吏部尚书衔兼太子太保,从一品。而顾宪成,
是个刚提拔一年的五品郎中。

王锡爵的后面,有万历撑腰。顾宪成的后面,什么都看不见。

第一把手加第二把手,对付一个小小的司官,然而事实告诉我们,顾宪成赢了。

因为在顾宪成的背后,是一片深不可测的黑夜。

我认为,在那片黑暗中,隐藏着一股强大的力量。

很快,事实就将再次验证这一点。

当万历下令大臣推举入阁人选的时候,顾宪成先生又一次冒了出来,上疏推举人选。虽说这事的确归他管,但奇怪
的是,如此重大的政治决策,吏部的几位侍郎竟然毫无反应,尚书陈有年也对他言听计从。史料上翻来覆去,只有
他的光辉事迹,似乎吏部就他干活。

而当万历同志看到顾宪成推举的那个名字时,差点没把桌子掀了。

因为在顾宪成的名单上,第一个就是王家屏。

作为吏部官员,顾宪成明知这家伙曾把皇帝折腾得七荤八素,竟然还要推荐此人,明摆着就是跟皇帝过不去。

所以皇帝也忍无可忍了,终于打发顾宪成回了家。

明代的官员,虽然罢官容易,升官倒也不难,只要过个几年,时局一变,立马就能回到朝廷重新来过。而以顾宪成
之前的工作业绩和运动能量,东山再起不过是个时间问题。

可谁也没想到,顾先生这一走,就再也没回来。

虽然把这人开了,万历很有点快感,但由此酿成的后果,却是他死都想不到的。

自明开国以来,无论多大能耐,无论有何背景,包括那位天下第一神算刘伯温,如果下野之后没能重新上台,慢慢
地就边缘化了,然后走向同一结局----完蛋,从无例外。

例外,从顾宪成开始。

和赵南星一样,自从下野后,顾宪成名气暴涨。大家纷纷推举他再次出山,虽然没啥效果,也算捧了个人场。不久
之后,他的弟弟顾允成和同乡高攀龙也辞官回了家,三个人一合计,反正闲着也是闲着,就讲学吧。

这一讲就是三年,讲着讲着,人越来越多,于是有一天,顾宪成对高攀龙说了这样一句话:

``我们应该找个固定的讲习场所。''

\section[\thesection]{}

地方是有的,在无锡县城的东头,有一个宋代学者杨时讲过学的场地,但年久失修,又太破,实在没法用,所以这
事也就搁置了下来。

七年后,出钱的主终于找到了。常州知府欧阳东凤和顾宪成关系不错,听说此事,大笔一挥就给办了,拨出专款修
缮此地。此后,这里就成为了顾宪成等人的活动地点。

它的名字叫做东林书院,实事求是地讲,确实也就是个书院。但在此后的几十年中,它却焕发了不可思议的魔力,
成为了一种威力强大的信念,那些相信或接受的信众,历史上统称为东林党。

无数人的命运,大明天下的时局,都将由这个看似与朝廷毫无关系的地方,最终确定。

王锡爵回家去养老,顾宪成回家去讲学,王家屏自然也消停了,于是首辅的位置还是落到了赵志皋同志的身上。

这就真叫害死人了,因为赵志皋压根就不愿意干!

赵先生真是老资格了,隆庆二年中进士,先当翰林,再当京官,还去过地方。风风雨雨几十年,苦也吃了,罪也受
了,七十多岁才混到首辅,也没啥意思。

更为重要的是,他个性软弱,既不如申时行滑头,也不如王锡爵强硬。而明代的言官们大都不是什么善茬,一贯欺
软怕硬。一旦坐到这个位置上,别说解决册立太子之类的敏感问题,光是来找茬的,都够他喝一壶。

对此,赵先生十分清楚,所以他主动上疏,不愿意干,情愿回家养老。

可是万历是不会同意的。好不容易找来个堵枪眼的,你要走了,我怎么办?

无奈,赵志皋先生虽然廉颇老矣,不太能饭,但还是得死撑下去。

于是,自万历二十二年起,他开始了四年痛苦而漫长的首辅生涯。具体表现为,不想干,没法干,却又不能走。

说起来,他还是很敬业的。因为这几年正好是多事之秋,外面打日本,里面闹册立,搞得不可开交,赵大人外筹军
备,内搞协调,日夜加班忙碌,干得还不错。

可下面这帮大臣一点面子都不给,看他好欺负,就使劲欺负。宫里失火了有人骂他,天灾有人骂他,儿子惹事了有
人骂他,甚至没事,也有人骂他,说他就该走人(言志皋宜放)。

实在欺人太甚,老实人终于也发火了。

\section[\thesection]{}

王锡爵在的时候,平素说一不二,动辄训斥下属,除了三王并封这种惹众怒的事情外,谁也不敢多嘴骂他。到赵志
皋这儿,平易近人,待人和气,却老是挨骂,老先生一气之下,也骂人了:

``都是内阁首辅,势大权重的,你们就争相依附求取进步,势小权轻的,你们就争相攻击,博取名声!''

骂归骂,可下面这帮人实在啥觉悟也没有,还是喜欢拿老先生开涮。赵老头也真是倒霉,在这紧要关头,偏偏又出
了事。

事情出在兵部尚书石星的身上,如果你还记得,当时正值第一次抗倭援朝战争结束,双方谈判期间,石星最为信任
的大忽悠沈惟敬正处于巅峰期,谈判前景似乎很乐观,石大人便通报领导,说和平很有希望。

他的领导,就是赵志皋。赵大爷本来就不爱惹事,听了自然高兴,表示同意谈判。

结果大家都知道了,所谓和平,全是沈惟敬、小西行长等中日两方的职业骗子们通力协作,忽悠出来的。事情败露
后,沈惟敬杀头,石星坐牢。

按说这事赵先生最多也就是个领导责任,可言官们实在是道德败坏,总找软柿子捏,每次弹劾石星,都要把赵大人
稍带上。赵大人气得直喘气,要辞职,皇帝又不许。到万历二十六年,再撑不住了,索性回家养病休息,反正皇帝
也不管。

万历二十九年,赵大人死在了家里,不知是病死,还是老死。但我知道,他确实很累,因为直到他死的那天,辞职
都没有批下来,用今天的话说,他应该算是死在了工作岗位上。

赵志皋日子过得艰难,张位相对好点,因为他的脾气比较厉害,言官们没怎么敢拿他开刀。加上他是次辅,凡事没
必要太出头,有赵首辅挡在前面,日子过得也可以。

他唯一的问题,就是在抗倭援朝战争中,着力推荐了一个人。不但多次上疏保举,而且对其夸奖有加,说此人是不
世出之奇才,必定能够声名远播,班师凯旋。

这个人的名字,叫做杨镐。

\section[\thesection]{}

关于此人,我们之前已经说过了。从某个角度讲,他确实不负众望,虽然输了,还是输得声名远播,播到全国人民
都晓得,随即开始追究责任。大臣们开骂,骂得张位受不了,就上疏皇帝,说:

``大家都在骂我(群言交攻),但我是忠于国家的,且毫无愧疚,希望皇上体察(惟上矜察)。''

皇帝说:

``杨镐这个人,就是你暗中密奏,推荐给我的!(密揭屡荐)。我信了你,才会委派他做统帅,现在败仗打了,国威受
损,你还敢说自己毫无愧疚(犹云无愧)!?''

到这个份上,估计也没啥说的了,张位连辞职的资格都没有,就被皇帝免职,走的时候没有一个人帮他说话。

估计是受刺激太大,这位兄弟回家不久后就死了。

至万历二十九年,内阁的几位元老全部死光,一个看似微不足道的人,就此踏上这个舞台。

七年前,王锡爵辞职,朝廷推举阁臣,顾宪成推举了王家屏。但有一点必须说明:当时,顾先生推荐的,并非王家
屏一人,而是七个。

这七个人中,王家屏排第一,可是万历不买账,把顾宪成赶回了家。然而事实上,对顾先生的眼光,皇帝大人还是
有所认可的,至少认可排第四的那个。

南京礼部尚书沈一贯,第四。

沈一贯,字肩吾,隆庆二年进士。算起来,他应该是赵志皋的同班同学,不过他的成绩比赵大人要好得多,当了庶
吉士,后来又去翰林院,给皇帝讲过课。和之前几位类似,他跟张居正大人的关系也相当不好,不过他得罪张先生
的原因,是比较搞笑的。

事情经过是这样,有一天,沈教官给皇帝讲课,说着说着,突然发了个感慨,说自古以来,皇帝托孤,应该找个忠
心耿耿的人,如果找不到这种人,还不如多教育自己的子女,亲历亲为。

要知道,张居正同志的耳目是很多的,很快这话就传到了他的耳朵里,加上他的心胸又不算太宽广,所以张大人当
政期间,沈一贯是相当地萧条,从未受到重用。

相对于直言上疏、痛斥张居正,而落得同样下场的王锡爵等同志,我只能说,其实他不是故意的。

张居正死后,沈一贯才出头,历任吏部左侍郎、翰林院侍读学士,后来又去了南京当礼部尚书。

此人平素为人低调,看上去没有什么特点,然而,这只是表面现象而已。

顾宪成是朝廷的幕后影响者,万历是至高无上的统治者,两人势不两立。

所以一个既能被顾宪成推荐,又能被皇帝认可的人,是十分可怕的。

\section[\thesection]{}

万历二十二年(1594),沈一贯被任命为吏部尚书兼东阁大学士,进入了帝国的决策层。

很快,他就展示了他的异常之处,具体表现为,大家都欺负赵志皋,他不欺负。

赵首辅实在是个彻头彻尾的软柿子,无论大小官员,从他身边过,都禁不住要捏一把,而对赵大人尊敬有加的,只
有沈一贯(事皋甚恭)。

但沈一贯先生尊敬赵老头,绝非尊重老人,而是尊重领导,因为排第二的张位、排第三的陈于陛,他都很尊敬。

沈一贯就这样扎下了根,在此后的七年之中,赵志皋被骂得养了病,陈于陛被骂得辞了官,都没他什么事,他还曾
经联同次辅张位保举杨镐,据说还收了钱,可是杨镐事发,张位被弹劾免职,他竟安然无恙。

到万历二十九年(1601),死的死了,退的退了,只剩沈一贯,于是这个天字第一号大滑头终于成为了帝国的首辅。

凭借多年的混事技术,沈先生游刃有余,左推右挡,皇帝信任,大臣也给面子,地位相当稳固,然而在历史上,沈
一贯的名声一贯不佳,究其原因,就是他太过滑头。

因为从某种角度来讲,朝廷首辅就是背黑锅的,国家那么多事,总得找一个负责的,但沈先生全然没有这个概念,
能躲就躲能逃就逃,实在不太地道。

而当时朝廷的局势,却已走到了一个致命的关口。

万历二十九年,皇长子十九岁,虽然出阁读书,却依然不是太子,而且万历办事不厚道,对教自己儿子的讲官十分
刻薄,一般人家请个老师,都要小心伺候,从不拖欠教师工资,万历却连饭都不管,讲官去教他儿子,还得自己带
饭,实在太不像话。

相对而言,皇三子就真舒服得多了,要什么有什么,备受万历宠爱,娇生惯养,啥苦都没吃过,且大有夺取太子之
位的势头。

这些情况大家都看在眼里,外加郑贵妃又是个百年难得一见的蠢人,丝毫不知收敛,极为嚣张,可谓是人见人恨,
久而久之,一个父亲偏爱儿子的问题,就变成了恶毒地主婆欺负老实佃户的故事。

问题越来越严重,舆论越来越激烈,万历是躲一天算一天的主,偏偏又来了这么个首辅,要知道,大臣们不闹事,
不代表不敢闹事,一旦他们的怒火到达顶点,国家将陷入前所未有的骚乱。

然而动乱没有爆发,因为这个曾经搞倒申时行、王锡爵、王家屏等无数政治高手,看似永远无法解决的问题,竟然
被解决了。

而解决它的,就是为人极不地道,一贯滑头的沈一贯。

\section[\thesection]{}

说起来,这是个非常玄乎的事。

万历二十九年(1601)八月,沈一贯向皇帝上疏,要求册立太子,其大致内容是,皇长子年纪大了,应该册立太子,
正式成婚,到时有了孙子,您也能享子孙满堂的福啊。

无论怎么看,这都是一封内容平平的奏疏,立意不新颖,文采很一般,按照以往的惯例,最终的结局应该是被压在
文件堆下几年,再拉出去当柴禾烧。

可惊喜总是存在的,就在第二天,沈一贯收到了皇帝的回复:

``即日册立皇长子为太子!''

沈一贯当时就懵了。

这绝对不可能。

争了近二十年,无数猛人因此落马,无数官员丢官发配,皇帝都被折腾得半死不活,却死不松口。

然而现在,一切都解决了。事实摆在眼前,即日册立太子,非常清晰,非常明显。

沈一贯欣喜若狂,他随即派人出去,通报了这一消息,于是举朝轰动了,所有的人都欢呼雀跃,为这个等待了许久
的胜利。

``争国本''就此落下帷幕,这场万历年间最激烈复杂的政治事件,共逼退首辅四人,部级官员十余人、涉及中央及
地方官员人数三百多位,其中一百多人被罢官、解职、发配,闹腾得乌烟瘴气,还搞出了一个叫东林党的副产品,
几乎所有人都不相信,它会有解决的一天。

然而这件事情,却在最意想不到的时候,由最意想不到的人解决了,遭遇父亲冷落的朱常洛终于修成正果,荣登太
子。

但此事之中,仍然存在着一个最大的疑问:为什么那封上疏,能够破解这个残局?

我不知道沈一贯有没有想过这个问题,但我想了。

万历并不愚蠢,事实上,从之前的种种表现看,他是一个十分成熟的政治家,没有精神病史,心血来潮或是突发神
经,基本都可以排除,而且他的意图十分明显----立皇三子。

那么到底是什么原因,让他放弃了这个经历十余年的痛骂、折腾,却坚持不懈的企图?

\section[\thesection]{}

翻来覆去地审阅沈一贯的那封上疏,并综合此事发生前的种种迹象,我得出了结论:这是压死骆驼的最后一把稻草。

万历从来就不想立皇长子,这是毫无疑问的,但疑问在于,他知道希望很渺茫,也知道手底下这帮大臣都是死脑
筋,为何还要顶着漫天的口水和谩骂,用拖延战术硬扛十几年?

如果没有充分的把握,皇帝大人是不会吃这个苦的。

十几年来,他一直在等待两件事情的发生。然而这两件事他都没等到。

我曾经分析过,要让皇三子超越皇长子继位,修改出生证明之类的把戏自然是没用的,必须有一个理由,一个能够
说服所有人的理由,而这个答案只能是:立嫡不立长。

只有立嫡子,才能压过长子,并堵住所有人的嘴。

但皇三子就是皇三子,怎样可能变成嫡子呢?

事实上,是可能的,只要满足一个条件----郑贵妃当皇后。

只要郑贵妃当上皇后,皇后的儿子自然就是嫡子,皇三子继位也就顺理成章了。

可是皇后只有一个,所以要让郑贵妃当上皇后,只能靠等,等到王皇后死掉,或是等时机成熟,把她废掉,郑贵妃
就能顺利接位。

可惜这位王皇后身体很好,一直活到了万历四十八年(这一年万历驾崩),差点比万历自己活得还长,且她一向为人
本分厚道,又深得太后的喜爱,要废掉她,实在没有借口。

第一件事是等皇后,第二件事是等大臣。

这事就更没谱了,万历原本以为免掉一批人,发配一批人,再找个和自己紧密配合的首辅,软硬结合就能把事情解
决,没想到明代的大臣却是软硬都不吃,丢官发配的非但不害怕,反而很高兴,要知道,因为顶撞皇帝被赶回家,
那是光荣,知名度噌蹭地往上涨,值大发了。所以他越严厉,越有人往上冲,只求皇帝大人再狠一点,最好暴跳如
雷,这样名声会更大,效果会更好。

而首辅那边,虽然也有几个听话的,无奈都是些老油条,帮帮忙是可以的,跟您老人家下水是不可以的。好不容易
拉了个王锡爵下来,搞了三王并封,半路人家想明白了,又跑掉了。

至于王家屏那类人,真是想起来都能痛苦好几天,十几年磨下来,人换了不少,朝廷越来越闹,皇后身体越来越
好,万历同志焦头烂额,开始重新权衡利弊。

我相信,在他下定决心的过程中,有一件事情起到了关键的作用。

\section[\thesection]{}

此事发生的具体时间不详,但应该在万历十四年之后。

有一天,李太后和万历谈话,说起了皇长子,太后问:你为何不立他为太子?

万历漫不经心地答道:他是宫女的儿子。

太后大怒:你也是宫女的儿子!

这就是活该了,万历整天忙里忙外,却把母亲的出身给忘了,要知道这位李老太太,当年也就是个宫女,因为长得
漂亮才被隆庆选中,万历才当上了皇帝,如果宫女的儿子不能继位,那么万历兄是否应该引咎辞职呢?

万历当即冷汗直冒,跪地给老太太赔不是,好说歹说才糊弄过去。

这件事情,必定给他留下了极为深刻的印象。

皇后没指望,老太太反对,大臣不买账,说众叛亲离,丝毫也不过分。万历开始意识到,如果不顾一切,强行立皇
三子,他的地位都可能不保。

在自己的皇位和儿子的皇位面前,所有成熟的政治家都会做出同样的抉择。

决定政治动向的最终标准是利益,以及利益的平衡。

这是一条真理。

就这样,沈一贯捡了个大便宜,成就了册立太子的伟业,他的名声也如日中天,成为了朝廷大臣拥戴的对象。

可你要说他光捡便宜,不做贡献,那也是不对的,事实上,他确实做了一件了不起的事。

就在圣旨下达的第二天,万历反悔了,或许是不甘心十几年被人白喷了口水,或许是郑贵妃吹了枕头风,又找了借
口再次延期,看那意思是不打算办了。

但朝廷大臣们并没有看到这封推辞的诏书,因为沈一贯封还了。

这位一贯滑头的一贯兄,终于硬了一回,他把圣旨退了回去,还加上了这样一句话:

``万死不敢奉诏!''

沈一贯的态度,深深地震慑了万历,他意识到,自己已经无路可退

万历二十九年十月,皇帝陛下正式册立皇长子朱常洛为太子,``争国本''事件正式结束。

被压了十几年的朱常洛终于翻身,然而他的母亲,那位恭妃,却似乎永无出头之日。

\section[\thesection]{}

按说儿子当上太子,母亲至少也能封个贵妃,可万历压根就没提这件事,一直压着,直到万历三十四年,朱常洛的
儿子出世,她才被封为皇贵妃。

但皇贵妃和皇贵妃不一样,郑贵妃有排场,有派头,而王贵妃不但待遇差,连儿子来看他,都要请示皇帝,经批准
才能见面。

但几十年来,她没有多说过一句话,直到万历三十九年的那一天。

她已经病入膏肓,不久于人世,而朱常洛也获准去探望他,当那扇大门洞开时,她再次见到了自己的儿子。

二十九年前的那次偶遇,造就了她传奇的一生,从宫女到贵妃,再到未来的太后(死后追封)。

但是同时,这次偶遇也毁灭了她,因为万历同志很不地道,几十年如一日对她搞家庭冷暴力,既无恩宠,也无厚
待,生不如死。

然而她并不落寞,也无悔恨。

因为她看到了自己的儿子,已经长大成人的儿子。

青史留名的太后也好,籍籍无名的宫女也罢,都不重要。重要的是,作为一个母亲,在临终前看到了自己的儿子,
看到他经历千难万苦,终于平安成人,这就足够了。

所以,在这生命的最后一刻,她拉着儿子的衣角,微笑着说:

``儿长大如此,我死何恨。''

这里使用的是史料原文,因为感情,是无法翻译的。

还有,其实这句话,她是哭着说的,但我认为,当时的她,很高兴。

王宫女就此走完了她的一生,虽然她死后,万历还是一如既往地混账,竟然不予厚葬,经过当时的首辅叶向高反复
请求,才得到了一个谥号。

虽然她这一生,并没有什么可供传诵的事迹,但她已然知足。

在这个世界上,所有的爱都是为了相聚,只有母爱,是为了分离。

接受了母亲最后祝福的朱常洛还将继续走下去,在他成为帝国的统治者前,必须接受更为可怕的考验。

\section[\thesection]{}

梃击

朱常洛是个可怜人,具体表现为出身低,从小就不受人待见,身为皇子,别说胎教,连幼儿园都没上过,直到十二
岁才读书,算半个失学儿童。身为长子,却一直位置不稳,摇摇摆摆到了十九岁,才正式册立为太子。

读书的时候,老师不管饭,册立的时候,仪式都从简,混到这个份上,怎个惨字了得。

他还是个老实人,平时很少说话,也不闹事,待人也和气,很够意思,但凡对他好的,他都报恩。比如董其昌先
生,虽被称为明代最伟大的天才画家,但人品极坏,平日欺男霸女,鱼肉百姓,闹得当地百姓都受不了,但就是这
么个人,因为教过他几天,辞官后还特地召回,给予优厚待遇。

更为难得的是,对他不好的,他也不记仇,最典型的就是郑贵妃,这位妇女的档次属于街头大妈级,不但多事,而
且闹事,屡次跟他为难,朱常洛却不以为意,还多次替其开脱。

无论从哪个角度看,他都是一个不折不扣的好人。

但历史已经无数次证明,在皇权斗争中,好人最后的结局,就是废人。

虽然之前经历风风雨雨,终于当上太子,但帝国主义亡我之心不死,只要万历一天不死,他一天不登基,幕后的阴
谋将永不停息,直至将他彻底毁灭。

现实生活不是电影,坏人总是赢,好人经常输,而像朱常洛这种老好人,应该算是稳输不赢。

可是这一次,是个例外。

事实证明,万历二十九年,朱常洛被册立为太子,不过是万里长征走完了第一步,两年后,麻烦就来了。

这是一个很大的麻烦,大到国家动荡,皇帝惊恐,太子不安,连老滑头沈一贯都被迫下台。

但有趣的是,惹出麻烦的,既不是朱常洛,也不是郑贵妃,更不是万历,事实上,幕后黑手到底是谁,直至今日,
也无人知晓。

万历三十一年十一月,一篇文章在朝野之间开始流传,初始还是小范围内传抄,后来索性变成了大字报,民居市场
贴得到处都是,识字不识字都去看,短短十几天内朝廷人人皆知,连买菜的老大娘都知道了,在没有互联网和手机
短信的当年,传播速度可谓惊人。

之所以如此轰动,是因为这篇文章的内容,实在是太过火爆。

此文名叫《续忧危竑议》,全篇仅几百字,但在历史上,它却有一个诡异的名字----``妖书''。

在这份妖书中,没有议论,没有叙述,只有两个人的对话,一个人问,一个人答。问话者的姓名不详,而回答的那
个人,叫做郑福成。这个名字,也是文中唯一的主角。

\section[\thesection]{}

文章一开始,是两个人在谈事。一个说现在天下太平,郑福成当即反驳,说目前形势危急。因为皇帝虽然立了太
子,但那是迫于沈一贯的要求,情非得已,很快就会改立福王。

这在当年,就算是反动传单了,而且郑福成这个名字,也很有技术含量,郑贵妃、福王、成功三合一,可谓言简意
赅。

之所以被称为妖书,只说皇帝太子,似乎还不合格,于是内阁的两位大人,也一起下了水。

当时的内阁共有三人,沈一贯是首辅,另外两人是沈鲤和朱赓。妖书的作者别出心裁,挑选了沈一贯和朱赓,并让
他们友情客串,台词如下:

问:你怎么知道皇帝要改立福王呢?

郑福成答:你看他用朱赓,就明白了。朝中有这么多人,为什么一定要用朱赓呢?因为他姓朱,名赓,赓者,更也。
真正的意思,就是改日更立啊(佩服,佩服)。

这是整朱赓,还有沈一贯同志:

问:难道沈一贯不说话吗?

郑福成答:沈一贯这个人阴险狡诈,向来是有福独享,有难不当,是不会出头的。

闹到这个份上,作者还不甘心,要把妖书进行到底,最后还列出了朝廷中的几位高官,说他们都是改立的同党,是
大乱之源。

更为搞笑的是,这篇妖书的结尾,竟然还有作者署名!落款者分别是吏科都给事中项应祥,四川道御史乔应甲。

这充分说明,妖书作者实在不是什么良民,临了还要耍人一把,难能可贵的是,他还相当有版权意识,在这二位黑
锅的名下还特别注明,项应祥撰(相当于原著),乔应甲书(相当于执笔)。

这玩意一出来,大家都懵了。沈一贯当即上书,表示自己非常愤怒,希望找出幕后主使人,与他当面对质,同时他
还要求辞官,以示清白以及抗议。

而妖书上涉及的其他几位高级官员也纷纷上书,表示与此事无关,并要求辞职。

最倒霉的人是朱赓,或许是有人恶搞他,竟然把一份妖书放在了他的家门口。这位朱先生是个厚道人,吓得不行,
当即把这份妖书和自己的奏疏上呈皇帝,还一把鼻涕一把泪地哭诉,说我今年都快七十了,有如此恩宠已是意外,
也没啥别的追求,现在竟然被人诬陷,请陛下让我告老还乡。

\section[\thesection]{}

朝廷一片混乱,太子也吓得不行。他刚消停两年,就出这么个事,闹不好又得下去,整日坐卧不安,担惊受怕。

要说还是万历同志久经风雨,虽然愤怒,倒不怎么慌。先找太子去聊天,说我知道这不关你的事,好好在家读书,
别出门。

然后再发布谕令,安抚大臣,表示相信大家,不批准辞职,一个都别走。

稳定情绪后,就该破案了。像这种天字第一号政治案件,自然轮不上衙门捕快之类的角色,东厂锦衣卫倾巢而出,
成立专案组,没日没夜地查,翻天覆地地查。

万历原本以为,来这么几手,就能控制局势,然而这场风暴,却似乎越来越猛烈。

首先是太子,这位仁兄原本胆小,这下更是不得了,窝在家里哪里都不去,唯恐出事。而郑贵妃那边也不好受,毕
竟妖书针对的就是她,千夫所指,舆论压力太大,每日只能以泪洗面,不再出席任何公开活动。

内阁也消停了,沈一贯和朱赓吓得不行,都不敢去上班,呆在家里避风头。日常工作只有沈鲤干,经常累得半死。
大臣们也怕,因为所有人都知道,平时争个官位,抢个待遇的没啥,这个热闹却凑不得。虽说皇帝大人发话,安抚
大家不让辞职,可这没准是放长线钓大鱼,不准你走,到时候来个一锅端,那就麻烦大了。

总而言之,从上到下,一片人心惶惶。很多人都认定,在这件事情的背后,有很深的政治背景。

确实如此。

这是一件明代历史上著名的政治疑案,至今仍无答案,但从各种蛛丝马迹之中,真相却依稀可辨。

可以肯定的是,这件事情应该与郑贵妃无关,因为她虽然蠢,也想闹事,却没必要闹出这么大动静,把自己挤到风
口浪尖受罪,而太子也不会干这事,以他的性格,别人不来惹他就谢天谢地,求神拜佛了。

作案人既不是郑贵妃,也不是太子,但可以肯定的是,作案者,必定是受益者。

在当时的朝廷中,受益者不外乎两种,一种是精神受益者,大致包括看不惯郑贵妃欺压良民,路见不平也不吼,专
门暗地下黑手的人,写篇东西骂骂出口气。

这类人比较多,范围很大,也没法子查。

第二种是现实受益者。就当时的朝局而言,嫌疑人很少----只有两个。

\section[\thesection]{}

这两个人,一个是沈一贯,另一个是沈鲤。

这二位仁兄虽然是本家,但要说他们不共戴天,也不算夸张。

万历二十九年,沈一贯刚刚当首辅的时候,觉得内阁人太少,决定挑两个跑腿的,一个是朱赓,另一个是沈鲤。

朱赓是个老实人,高高兴兴地上班了,沈鲤却不买账,推辞了很多次,就是不来。沈一贯以为他高风亮节,也就没
提这事。

可两年之后,这位仁兄竟然又入阁了。沈一贯同志这才明白,沈鲤不是不想入阁,而是不买他的帐。因为这位本家
资历老,名望高,还给皇帝讲过课,关系很好,压根就看不起自己。

看不起自然就不合作,外加沈鲤也不是啥善人,两人在内阁里一向是势不两立。

而现在妖书案发,内阁三个人,偏偏就拉上了沈一贯和朱赓,毫无疑问,沈鲤是有嫌疑的。

这是我的看法,也是沈一贯的看法。

这位老油条在家呆了好几天,稳定情绪之后,突然发现这是一个绝佳的机会。

他随即恢复工作,以内阁首辅的身份亲自指挥东厂锦衣卫搜捕,而且还一反往日装孙子的常态,明目张胆对沈鲤的
亲信,礼部侍郎郭正域下手,把他的老乡、朋友、下属、仆人全都拉去审问。

在这个不寻常的行动背后,是一个不寻常的算盘:如果事情是沈鲤干的,那么应该反击,这叫报复,如果事情不是
沈鲤干的,那么也应该反击,这叫栽赃。

在这一光辉思想的指导下,斗争愈演愈烈,沈鲤的亲信被清算,他本人也未能幸免,锦衣卫派了几百人到他家,也
不进去,也不闹事,就是不走,搞得沈鲤门都出不去,十分狼狈。

但沈先生如果没两把刷子,是不敢跟首辅叫板的,先是朱常洛出来帮忙叫屈,又传话给东厂的领导,让他们不要乱
来,后来连万历都来了,直接下令不得骚扰沈鲤。

沈一贯碰了钉子,才明白这个冤家后台很硬,死拼是不行的,他随即转换策略,命令锦衣卫限期破案----抓住作案
人,不怕黑不了你。

可是破案谈何容易,妖书满街都是,传抄者无数,鬼才知道到底哪一张纸才是源头,十一月十日案发,查到二十
日,依然毫无进展。

东厂太监陈矩,锦衣卫都督王之桢急得直跳脚,如果还不破案,这官就算当到头了。

二十一日,案件告破。

\section[\thesection]{}

说起来,这起妖书案是相当的妖,案发莫名其妙不说,破案也破得莫名其妙。二十一日这天,先是锦衣卫衙门收到
一份匿名检举信,后又有群众举报,锦衣卫出动,这才逮住了那个所谓的真凶:皦生光。

皦生光先生是什么人呢?

答案是----什么人都不是。

这位仁兄既不是沈鲤的人,也不是沈一贯的人,他甚至根本就不是官员,而只是一个顺天府的秀才。

真凶到案,却没有人心大快,恰恰相反,刚刚抓到他的时候,朝廷一片哗然,大家都说锦衣卫和东厂太黑,抓不到
人了弄这么个人来背锅。

这种猜测很有道理,因为那封妖书,不是一个秀才能写得出来的。

那年头,群众参政议政积极性不高,把肚子混饱就行,谁当太子鬼才关心。更何况沈一贯和朱赓的关系,以及万历
迫不得已才同意立长子这些情况,地方官都未必知道,一个小秀才怎么可能清楚?

但细细一查,才发现这位仁兄倒还真有点来头。

原来皦生光先生除了是秀才外,还兼职干过诈骗犯。具体方法是欺负人家不识字,帮人写文章,里面总要带点忌
讳,不是用皇帝的避讳字,就是加点政治谣言。等人家用了,再上门勒索,说你要不给钱,我就跑去报官云云。

后来由于事情干得多了,秀才也被革了,发配到大同当老百姓,最近才又潜回北京。

可即便如此,也没啥大不了,归根结底,他也就是个普通混混,之所以被确定为重点嫌疑人,是因为他曾经敲诈过
一个叫郑国泰的人。

郑国泰,是郑贵妃的弟弟。

一个穷秀才,又怎么诈骗皇亲国戚呢?

按照锦衣卫的笔录,事情大致是这样的:有个人要去郑国泰家送礼,要找人写文章,偏偏这人不知底细,找到了皦
生光。皦秀才自然不客气,发挥特长,文章里夹了很多私货,一来二去,东西送进去了。

一般说来,以郑国泰的背景,普通的流氓是不敢惹的,可皦生光不是普通的流氓,胆贼大,竟然找上了门,要郑大
人给钱。至于此事的结局,说法就不同了,有的说郑国泰把皦生光打了一顿,赶出了门,也有的说郑国泰胆小,给
钱私了。

\section[\thesection]{}

但无论如何,皦秀才终究和此事搭上了边。有了这么个说法,事情就好办了,侦查工作随即开始,首先是搜查,家
里翻个底朝天,虽说没找到妖书,但发现了一批文稿,据笔迹核对(司法学名:文检),与妖书的初期版本相似(注
意,是相似)。

之后是走访当地群众,以皦秀才平日的言行,好话自然没有,加上这位兄弟又有前科,还进过号子,于是锦衣卫最
后定案:有罪。

案子虽然定了,但事情还没结。因为明朝的司法制度十分严格,处决人犯必须经过司法审讯。即便判了死罪,还得
由皇帝亲自进行死刑复核,这才能把人拉出去咔嚓一刀。

所以万历下令,鉴于案情重大,将此案送交三法司会审。

之前提过,三法司,即是明朝的三大司法机关:大理寺、都察院、刑部,大致相当于今天的司法部、监察部、最高
人民检察院、最高人民法院等若干部门。

三法司会审,是明代最高档次的审判,也是最为公平的审判。倒不是三法司这帮人有啥觉悟,只是因为参与部门
多,把每个人都搞定,比较难而已。例如当年的严世藩,人缘广,关系硬,都察院、大理寺都有人,偏偏刑部的几
个领导是徐阶的人,最后还是没躲过去。

相比而言,像皦秀才这种要钱没钱要权没权的人,死前能捞个三司会审,也就不错了,结案只是时间问题。

可是这起案件,远没有想象中那么简单。

一到三法司,皦秀才就不认账了。虽说之前他曾招供,说自己是仇恨郑国泰,故意写妖书报复,但那是在锦衣卫审
讯时的口供。锦衣卫是没有善男信女的,也不搞什么批评教育,政策攻心,除了打就是打,口供是怎么来的,大家
心里都有数。现在进了三法司,看见来了文明人,不打了,自然就翻了案。

更麻烦的是,沈一贯和朱赓也不认。

这二位明显是被妖书案整惨了,心有不甘,想借机会给沈鲤点苦头吃。上疏皇帝,说证词空泛,不可轻信,看那意
思,非要搞出个一二三才甘心。

所以在审讯前,他们找到了萧大亨,准备做手脚。

萧大亨,时任刑部尚书,是沈一贯的亲信,接到指令后心领神会,在审讯时故意诱供,让皦秀才说出幕后主使。

\section[\thesection]{}

可是皦秀才还真够意思,问来问去就一句话:

``无人主使!''

萧大亨没办法,毕竟是三法司会审,搞得太明显也不好,就给具体负责审案的下属,刑部主事王述古写了张条子,
还亲自塞进了他的袖口,字条大意是,把这件事情往郭正域、沈鲤身上推。

没想到王述古接到条子,看后却大声反问领导:

``案情不出自从犯人口里,却要出自袖中吗?!''

萧大亨狼狈不堪,再也不敢掺和这事。

沈鲤这边也没闲着,他知道沈一贯要闹事,早有防备:你有刑部帮忙,我有都察院撑腰。一声令下,都察院的御史
们随即开动,四下活动,灭火降温,准备冷处理此事。

其中一位御史实在过于激动,竟然在审案时,众目睽睽之下,对皦秀才大声疾呼:

``别牵连那么多人了,你就认了吧。''

审案审到这个份上,大家都是哭笑不得,要结案,结不了;不结案,又没个交代,皇帝、太子、贵妃、内阁,谁都
不能得罪。万一哪天皦秀才吃错了药,再把审案的诸位领导扯进去,那真是哭都没眼泪。

三法司的人急得不行,可急也没用,于是有些不地道的人就开始拿案件开涮。

比如有位审案御史,有一天突然神秘地对同事说,他已经确定,此案一定是皦秀才干的。

大家十分兴奋,认定他有内部消息,纷纷追问他是怎么知道的。

御史答:

``昨天晚上我做梦,观音菩萨告诉我,这事就是他干的。''

当即笑瘫一片。

没办法,就只能慢慢磨,开审休审,休审开审,周而复始,终于有一天,事情解决了。

皦生光也受不了了,天天审问,天天用刑,天天折腾,还不如死了好,所以他招供了:

``是我干的,你们拿我去结案吧。''

这个世界清净了。

万历三十二年(1604)四月,皦生光被押赴刑场,凌迟处死。

妖书案就此结束,虽说闹得天翻地覆,疑点重重,但有一点是肯定的,那就是:皦生光很冤枉。

因为别的且不谈,单说妖书上列出的那些官员,就皦秀才这点见识,别说认识,名字都记不全。找这么个人当替死
鬼,手真狠,心真黑。

妖书何人所写,目的何在,没人知道,似乎也没人想知道。

因为有些时候,真相其实一点也不重要。

\section[\thesection]{}

妖书案是结了,可轰轰烈烈的斗争又开始了。沈一贯被这案子整得半死不活,气得不行,卯足了劲要收拾沈鲤。挖
坑、上告、弹劾轮番上阵,可沈鲤同志很是强悍,怎么搞都没倒。反倒是沈一贯,由于闹得太过,加上树大招风,
竟然成为了言官们的新目标。骂他的人越来越多,后来竟然成了时尚(弹劾日众)。

沈一贯眼看形势不妙,只好回家躲起来,想要避避风头,没想到这风越刮越大,三年之间,弹劾他的奏疏堆起来足
有一人高,于是他再也顶不住了。

万历三十四年(1606),沈一贯请求辞职,得到批准。

有意思的是,这位仁兄走之前,竟然还提了一个要求:我走,沈鲤也要走。

恨人恨到这个份上,也不容易。

而更有意思的是,万历竟然答应了。

这是一个不寻常的举动,因为沈鲤很有能力,又是他的亲信。而沈一贯虽说人滑了点,办事还算能干,平时朝廷的
事全靠这两人办,万历竟然让他们全都走人,动机就一个字----烦。

自打登基以来,万历就没过几天清净日子。先被张居正压着,连大气都不敢出,等张居正一死,言官解放,吵架的
来了,天天闹腾。到生了儿子,又开始争国本,堂堂皇帝,竟然被迫就范。

现在太子也立了,某些人还不休息,跟着搞什么妖书案,打算混水摸鱼,手下这两人还借机斗来斗去,时不时还以
辞职相威胁,太过可恶。

既然如此,你们就都滚吧,有多远滚多远,让老子清净点!

沈一贯和沈鲤走了,内阁只剩下了朱赓。

这一年,朱赓七十二岁。

朱赓很可怜,他不但年纪大,而且老实,老实到他上任三天,就有言官上书骂他,首辅大人心态很好,统统不理。

可让他无法忍受的是,他不理大臣,皇帝也不理他。

内阁人少,一个七十多的老头起早贪黑熬夜,实在扛不住,所以朱赓多次上书,希望再找几个人入阁。

可是前后写了十几份报告,全都石沉大海,到后来,朱大人忍不住了,可怜七十多岁的老大爷,亲自跑到文华门求
见皇帝,等了半天,却还是吃了闭门羹。

\section[\thesection]{}

换在以前,皇帝虽然不上朝,但大臣还是要见的,特别是内阁那几个人,这样才能控制朝局。比如嘉靖,几十年不
上朝,但没事就找严嵩、徐阶聊天,后来索性做了邻居,住到了一起(西苑)。

但万历不同,他似乎是不想干了。在他看来,内阁一个人不要紧,没有人也不要紧,虽然朱首辅七十多了,也还活
着嘛。能用就用,累死了再说,没事就别见了,也不急这几天,会有人的,会见面的,再等等吧。

就这样,朱老头一边等一边干,一个人苦苦支撑,足足等了一年,既没见到助手,也没见过皇帝。

这一年里朱老头算被折腾惨了,上书国政,皇帝不理,上书辞职,皇帝也不理,到万历三十四年(1607),朱赓忍无
可忍,上书说自己有病,竟然就这么走了。

皇帝还是不理。

最后一个也走了。

内阁没人呆,首辅没人干,经过万历的不懈努力,朝廷终于达到了传说中的最高境界----千山鸟飞绝,万径人踪灭。

自明代开国以来,只有朱元璋在的时候,既无宰相,也无内阁,时隔多年,万历同志终于重现往日荣光。

而对于这一空前绝后的盛况,万历很是沉得住气,没人就没人,日子还不是照样过?

但很快,他就发现这日子没法过了。

因为内阁是联系大臣和皇帝的重要渠道,而且内阁有票拟权,所有的国家大事,都由其拟定处理意见,然后交由皇
帝审阅批准。所以即使皇帝不干活,国家也过得去。

朱元璋不用宰相和内阁,原因在于他是劳模,什么都能干。而万历先生连文件都懒得看,你要他去干首辅的活,那
就是白日做梦。

朝廷陷入了全面瘫痪,这么下去,眼看就要破产清盘,万历也急了,下令要大臣们推举内阁人选。

几番周折后,于慎行、叶向高、李廷机三人成功入阁,班子总算又搭起来了。

但这个内阁并没有首辅,因为万历特意空出了这个位置,准备留给一个熟人。

这个人就是王锡爵,虽说已经告老还乡,但忆往昔,峥嵘岁月稠。之前共背黑锅的革命友谊,给万历留下了深刻的
印象,所以他派出专人,去请王锡爵重新出山,并同时请教他一个问题。

王锡爵不出山。

由于此前被人坑过一次,加上都七十四岁了,王锡爵拒绝了万历的下水邀请,但毕竟是多年战友,还教过人家,所
以,他解答了万历的那个疑问。

\section[\thesection]{}

万历的问题是,言官太过凶悍,应该如何应付。

王锡爵的回答是,他们的奏疏你压根别理(一概留中),就当是鸟叫(禽鸟之音)!

我觉得,这句话十分之中肯。

此外,他还针对当时的朝廷,说了许多意见和看法,为万历提供了借鉴。

然后,他把这些内容写成了密疏,派人送给万历。

这是一封极为机密的信件,其内容如果被曝光,后果难以预料。

所以王锡爵很小心,不敢找邮局,派自己家人携带这封密信,并反复嘱托,让他务必亲手交到朝廷,绝不能流入任
何人的手中,也算是吸取之前申时行密疏走光的经验。

但他做梦也没想到,这一次,他的下场会比申时行还惨。

话说回来,这位送信的同志还是很敬业的,拿到信后立即出发,日夜兼程赶路,一路平安,直到遇见了一个人。

当时他已经走到了淮安,准备停下来歇脚,却听说有个人也在这里,于是他便去拜访了此人。

这个人的名字,叫做李三才。

李三才,字道甫,陕西临潼人,时任都察院右佥都御史,凤阳巡抚。

这个名字,今天走到街上,问十个人估计十个都不知道,但在当年,却是天下皆知。

关于此人的来历,只讲一点就够了:

二十年后,魏忠贤上台时,编了一本东林点将录,把所有跟自己作对的人按照水浒一百单八将称号,以实力排序,
而排在此书第一号的,就是托塔天王李三才。

总而言之,这是一个十分厉害的人物。

因为淮安正好归他管,这位送信人原本认识李三才,到了李大人的地头,就去找他叙旧。

两人久别重逢,聊着聊着,自然是要吃饭,吃着吃着,自然是要喝酒,喝着喝着,自然是要喝醉。

送信人心情很好,聊得开心,多喝了几杯,喝醉了。

李三才没有醉,事实上,他非常清醒,因为他一直盯着送信人随身携带的那口箱子。

在安置了送信人后,他打开了那个箱子,因为他知道,里面必定有封密信。

\section[\thesection]{}

得知信中内容之后,李三才大吃一惊,但和之前那位泄露申时行密疏的罗大纮不同,他并不打算公开此信,因为他
有更为复杂的政治动机。

手握着这封密信,李三才经过反复思考,终于决定:篡改此信件。

在他看来,篡改信件,更有利于达到自己的目的。

所谓篡改,其实就是重新写一封,再重新放进盒子里,让这人送过去,神不知鬼不觉。

可是再一细看,他就开始感叹:王锡爵真是个老狐狸。

古代没有加密电报,所以在传送机密信件时,往往信上设有暗号,两方约定,要么多写几个字,要么留下印记,以
防被人调包。

李三才手中拿着的,就是一封绝对无法更改的信,倒不是其中有什么密码,而是他发现,此信的写作者,是王时敏。

王时敏,是王锡爵的孙子,李三才之所以认定此信系他所写,是因为这位王时敏还有一个身份----著名书法家。

这是真没法了,明天人家就走了,王时敏的书法天下皆知,就自己这笔字,学都没法学,短短一夜时间,又练不出
来。

无奈之下,他只好退而求其次,抄录了信件全文,并把信件放了回去。

第二天,送信人走了,他还要急着把这封密信交给万历同志。

当万历收到此信时,绝不会想到,在他之前,已经有很多人知道了信件的内容,而其中之一,就是远在无锡的普通
老百姓顾宪成。

这件事可谓疑团密布,大体说来,有几个疑点:

送信人明知身负重任,为什么还敢主动去拜会李三才,而李三才又为何知道他随身带有密信,之后又要篡改密信呢?

这些问题,我可以回答。

送信人去找李三才,是因为李大人当年的老师,就是王锡爵。

非但如此,王锡爵还曾对人说,他最喜欢的学生,就是李三才。两人关系非常的好,所以这位送信人到了淮安,才
会去找李大人吃饭。

作为凤阳巡抚,李三才算是封疆大吏,而且他本身就是都察院的高级官员,对中央的政治动向十分关心,皇帝为什
么找王锡爵,找王锡爵干什么,他都一清二楚,唯一不清楚的,就是王锡爵的答复。

最关键的问题来了,既然李三才是王锡爵的学生,还算他的亲信,李三才同志为什么要背后一刀,痛下杀手呢?

因为在李三才的心中,有一个人,比王锡爵更加重要,为了这个人,他可以出卖自己的老师。

万历二年(1574),李三才考中了进士,经过初期培训,他分到户部,当上了主事,几年之后,另一个人考中进士,
也来到了户部当主事,这个人叫顾宪成。

\section[\thesection]{}

这之后他们之间发生了什么事情,史书上没有写,我也不知道,但是我惊奇地发现,当顾宪成和李三才在户部做主
事的时候,他们的上司竟然叫赵南星。

联想到这几位后来在朝廷里呼风唤雨的情景,我们有理由相信,在那些日子里,他们谈论的应该不仅仅是仁义道
德,君子之交,暗室密谋之类的把戏也没少玩。

李三才虽然是东林党,但道德水平明显一般,他出卖王老师,只是因为一个目的----利益。

只要细细分析一下,就能发现,李三才涂改信件的真正动机。

当时的政治形势看似明朗,实则复杂,新成立的这个三人内阁,可谓凶险重重,杀机无限。

李廷机倒还好说,这个人性格软弱,属于和平派,谁也不得罪,谁也不搭理,基本可以忽略。

于慎行就不同了,这人是朱赓推荐的,算是朱赓的人,而朱赓是沈一贯的人,沈一贯和王锡爵又是一路人,所以在
东林党的眼里,朱赓不是自己人。

剩下的叶向高,则是一个非同小可的人,此后一系列重大事件中,他起到了极为关键的作用,此人虽不是东林党,
却与其有着千丝万缕的联系,是个合格的地下党。

这么一摆,你就明白了,内阁三个人,一个好欺负,两个搞对立,遇到事情,必定会僵持不下。

僵持还算凑合,可要是王锡爵来了,和于慎行团结作战,东林党就没戏了。

虽然王锡爵的层次很高,公开表明自己不愿去,但东林党的同志明显不太相信,所以最好的办法,就是打开那封
信,看个究竟。

在那封信中,李三才虽然没有看到重新出山的许诺,却看到了毫无保留的支持,为免除后患,他决定篡改。

然而由于写字太差,没法改,但也不能就此算数,为了彻底消除王锡爵的威胁,他抄录并泄露了这封密信,而且特
意泄露给言官。

因为在信中,王锡爵说言官发言是鸟叫,那么言官就是鸟人了。鸟人折腾事,是从来不遗余力的。

接下来的事情可谓顺其自然,舆论大哗,言官们奋笔疾书,把吃奶的力气拿出来痛骂王锡爵,言辞极其愤怒,怎么
个愤怒法,举个例子你就知道了。

我曾翻阅过一位言官的奏疏,内容就不说了,单看名字,就很能提神醒脑--巨奸涂面丧心比私害国疏。

如此重压之下,王锡爵没有办法,只好在家静养,从此不问朝政,后来万历几次派人找他复出,他见都不见,连回
信都不写,估计是真的怕了。

事情的发展,就此进入了顾宪成的轨道。

\section[\thesection]{}

王锡爵走了,朝廷再也没有能担当首辅的人选,于是李廷机当上了首辅,这位兄弟不负众望,上任后不久就没顶住
骂,回家休养,谁叫也没用,基本算是罢工了。而异类于慎行也不争气,刚上任一年就死了,就这样,叶向高成为
了内阁的首辅,也是唯一的内阁大臣。

对手被铲除了,这是最好的结局。

必须说明的是,所谓李三才和顾宪成的勾结,并不是猜测,因为在史料翻阅中,我找到了顾宪成的一篇文章。

在文章中,有这样几句话:

``木偶兰溪、四明、婴儿山阴、新建而已,乃在遏娄江之出耳。''

``人亦知福清之得以晏然安于其位者,全赖娄江之不果出……密揭传自漕抚也,岂非社稷第一功哉?''

我看过之后,顿感毛骨悚然。

这是两句惊天动地的话,却不太容易看懂,要看懂这句话,必须解开几个密码。

第一句话中,木偶和婴儿不用翻译,关键在于新建、兰溪、四明、山阴、以及娄江五个词语。

这五个词,是五个地名,而在这里,则是暗指五个人。

新建,是指张位(新建人)、兰溪,是指赵志皋(兰溪人)、四明,是指沈一贯(四明人),山阴,是指朱赓(山阴人)。

所以前半句的意思是,赵志皋和沈一贯不过是木偶,张位和朱赓不过是婴儿!

而后半句中的娄江,是指王锡爵(娄江人)。

连接起来,我们就得到了这句话的真实含义:

赵志皋、沈一贯、张位、朱赓都不要紧,最为紧要的,是阻止王锡爵东山再起!

顾宪成,时任南直隶无锡县普通平民,而赵、张、沈、朱四人中,除张位外,其余三人都当过首辅,首辅者,宰相
也,一人之下,万人之上!

然而这个无锡的平民,却在自己的文章中,把这些不可一世的人物,称为木偶、婴儿。

而从文字语气中可以看出,他绝非单纯发泄,而是确有把握,似乎在他看来,除了王锡爵外,此类大人物都不值一
提。

一个普通老百姓能牛到这个份上,真可谓是前无古人后无来者。

\section[\thesection]{}

第二句话的玄机在于两个关键词语:福清和漕抚。

福清所指的,就是叶向高,而漕抚,则是李三才。

叶向高是福建福清人,李三才曾任漕运总督,把这两个词弄清楚后,我们就明白了这句话的意思:

``大家都知道叶向高能安心当首辅,是因为王锡爵不出山……密揭这事是李三才捅出来的,可谓是为社稷立下第一功!''

没有王法了。

一个平民,没有任何职务,远离京城上千里,但他说,内阁大臣都是木偶婴儿。而现在的朝廷第一号人物能够坐稳
位置,全都靠他的死党出力。

纵观二十四史,这种事情我没有听过,没有看过。

但现在我知道了,在看似杂乱无章的万历年间,在无休止的争斗和吵闹里,一股暗流正在涌动、在黑暗中集结,慢
慢地伸出手,操纵所有的一切。

疯子

王锡爵彻底消停了,万历三十六年,叶向高正式登上宝座,成为朝廷首辅,此后七年之中,他是内阁第一人,也是
唯一的人,史称``独相''。

时局似乎毫无变化,万历还是不上朝,内阁还是累得半死,大臣还是骂个不停,但事实真相并非如此。

在表象之下,政治势力出现了微妙的变化,新的已经来了,旧的赖着不走,为了各自利益,双方一直在苦苦地寻
觅,寻觅一个致对方于死地的机会。

终于,他们找到了那个最好、最合适的机会--太子。

太子最近过得还不错,自打妖书案后,他很是清净了几年,确切地说,是九年。

万历四十一年(1613),一个人写的一封报告,再次把太子拖下了水。

这个人叫王曰乾,时任锦衣卫百户,通俗点说,是个特务。

这位特务向皇帝上书,说他发现了一件非常离奇的事情:有三个人集会,剪了三个纸人,上面分别写着皇帝、皇太
后、皇太子的名字,然后在上面钉了七七四十九个铁钉(真是不容易)。钉了几天后,放火烧掉。

这是个复杂的过程,但用意很简单----诅咒,毕竟把钉子钉在纸人上,你要说是祈福,似乎也不太靠谱。

这也就罢了,更麻烦的是,这位特务还同时报告,说这事是一个太监指使的,偏偏这个太监,又是郑贵妃的太监。

\section[\thesection]{}

于是事情闹大了,奏疏送到皇帝那里,万历把桌子都给掀了,深更半夜睡不着觉,四下乱转,急得不行。太子知道
后,也是心急火燎,唯恐事情闹大,郑贵妃更是哭天喊地,说这事不是自己干的。

大家都急得团团转,内阁的叶向高却悄无声息,万历气完了,也想起这个人了,当即大骂:

``出了这么大的事,这人怎么不说话!?''(此变大事,宰相何无言)

此时,身边的太监递给他一件东西,很快万历就说了第二句话:

``这下没事了。''

这件东西,就是叶向高的奏疏,事情刚出,就送上来了。

奏疏的内容大致是这样的:

陛下,此事的原告(指王曰乾)和被告(指诅咒者)我都知道,全都是无赖混混,之前也曾闹过事,还被司法部门(刑
部)处理过,这件事情和以往的妖书案很相似,但妖书案是匿名,无人可查,现在原告被告都在,一审就知道,皇上
你不要声张就行了。

看完这段话,我的感觉是:这是个绝顶聪明的人。

叶向高的表面意思,是说这件事情,是非曲折且不论,但不宜闹大,只要你不说,我不说,把这件事情压下去,一
审就行。

这是一个不符合常理的抉择。因为叶向高,是东林党的人,而东林党,是支持太子的,现在太子被人诅咒,应该一
查到底,怎能就此打住呢?

事实上,叶向高是对的。

第二天,叶向高将王曰乾送交三法司审讯。

这是个让很多人疑惑的决定,这人一审,事情不就闹大了吗?

如果你这样想,说明你很单纯,因为就在他吩咐审讯的后一天,王曰乾同志就因不明原因,不明不白地死在了监牢
里,死因待查。

什么叫黑?这就叫黑。

而只要分析当时的局势,揭开几个疑点,你就会发现叶向高的真实动机:

首先,最大的疑问是:这件事情是不是郑贵妃干的,答案:无所谓。

自古以来,诅咒这类事数不胜数,说穿了就是想除掉一个人,又没胆跳出来,在家做几个假人,骂骂出出气,是纯
粹的阿Q精神。一般也就是老大妈干干(这事到今天还有人干,有多种形式,如``打小人''),而以郑贵妃的智商,正
好符合这个档次,说她真干,我倒也信。

但问题在于,她干没干并不重要,反正铁钉扎在假人上,也扎不死人,真正重要的是,这件事不能查,也不能有真
相。

\section[\thesection]{}

追查此事,似乎是一个太子向郑贵妃复仇的机会,但事实上,却是不折不扣的陷阱。

原因很简单,此时朱常洛已经是太子,只要没有什么大事,到时自然接班,而郑贵妃一哭二闹三上吊之类的招数,
闹了十几年,早没用了。

但如若将此事搞大,再惊动皇帝,无论结果如何,对太子只好坏处,没有好处。因为此时太子要做的,只有一件事
情----等待。

事实证明,叶向高的判断十分正确,种种迹象表明,告状的王曰乾和诅咒的那帮人关系紧密,此事很可能是一个精
心策划的阴谋,某些人(不一定是郑贵妃),为了某些目的,想把水搅浑,再混水摸鱼。

久经考验的叶向高同志识破了圈套,危机成功度过了。

但太子殿下一生中最残酷的考验即将到来,在两年之后。

万历四十三年(1615)五月初四日 黄昏

太子朱常洛正在慈庆宫中休息,万历二十九年他被封为太子,住到了这里,但他爹人品差,基础设施一应具缺,要
啥都不给,连身边的太监都是人家淘汰的,皇帝不待见,大臣自然也不买账,平时谁都不上门,十分冷清。

但这一天,一个特别的人已经走到他的门前,并将以一种特别的方式问候他。

他手持一根木棍,进入了慈庆宫。

此时,他与太子的距离,只有两道门

第一道门无人看守,他迈了过去。

在第二道门,他遇到了阻碍。

一般说来,重要国家机关的门口,都有荷枪实弹的士兵站岗,就算差一点的,也有几个保安,实在是打死都没人问
的,多少还有个老大爷。

明代也是如此,锦衣卫、东厂之类的自不必说,兵部吏部门前都有士兵看守,然而太子殿下的门口,没有士兵,也
没有保安,甚至连老大爷都没有。

只有两个老太监。

于是,他挥舞木棍,打了过去。

众所周知,太监的体能比平常人要差点(练过宝典除外),更何况是老太监。

很快,一个老太监被打伤,他越过了第二道门,向着目标前进。

目标,就在前方的不远处。

然而太监虽不能打,却很能喊,在尖利的呼叫声下,其他太监们终于出现了。

\section[\thesection]{}

接下来的事情还算顺理成章,这位仁兄拿的毕竟不是冲锋枪,而他本人不会变形,不会变身,也没能给我们更多惊
喜,在一群太监围攻下,终于束手就擒。

当时太子正在慈庆宫里,接到报告后并不惊慌,毕竟人抓住了,也没进来,他下令将此人送交宫廷守卫处理,在他
看来,这不过是个小事。

但接下来发生的一切,将远远超出他的想象。

人抓住了,自然要审,按照属地原则,哪里发案由哪里的衙门审,可是这个案子不同,皇宫里的案子,难道你让皇
帝审不成?

推来推去,终于确定,此案由巡城御史刘廷元负责审讯。

审了半天,刘御史却得出个让人啼笑皆非的结论----这人是个疯子。

因为无论他好说歹说,利诱威胁,这人的回答却是驴唇不对马嘴,压根就不对路,还时不时蹦出几句谁也听不懂的
话,算是个彻头彻尾的疯子。

于是几轮下来,刘御史也不审了,如果再审下去,他也得变成疯子。

但要说一点成就没有,那也不对,这位疯子交代,他叫张差,是蓟州人,至于其他情况,就一无所知了。

这个结果虽然不好,却很合适,因为既然是个疯子,自然就能干疯子的事,他闯进皇宫打人的事情就有解释了,没
有背景、没有指使,疯子嘛,也不认路,糊里糊涂到皇宫,糊里糊涂打了人,很好,很好。

不错,不错,这事要放在其他朝代,皇帝一压,大臣一捧,也就结了。

可惜,可惜,这是在明朝。

这事刚出,消息就传开了,街头巷尾人人议论,朝廷大臣们更不用说,每天说来说去就是这事,而大家的看法也很
一致:这事,就是郑贵妃干的。

所谓舆论,就是群众的议论,随着议论的人越来越多,这事也压不下去了,于是万历亲自出马,吩咐三法司会审此
案。

说是三法司,其实只有刑部,审讯的人档次也不算高,尚书侍郎都没来,只是两个郎中(正厅级)。

但这二位的水平,明显比刘御史要高,几番问下来,竟然把事情问清楚了。

侦办案件,必须找到案件的关键,而这个案子的关键,不是谁干了,而是为什么干,也就是所谓的:动机。

经过一番询问,张差说出了自己的动机:在此前不久,他家的柴草堆被人给烧了,他气不过,到地方衙门伸冤,地
方不管,他就到京城来上访,结果无意中闯入了宫里,心里害怕,就随手打人,如此而已。

如果用两个字来形容张差的说法,那就是扯淡。

\section[\thesection]{}

柴草被人烧了,就要到京城上访,这个说法充分说明了这样一点:张差即使不是个疯子,也是个傻子。

因为这实在不算个好理由,要换个人,怎么也得编一个房子烧光,恶霸鱼肉百姓的故事,大家才同情你。

况且到京城告状的人多了去了,有几个能进宫,宫里那么大,怎么偏偏就到了太子的寝宫,您还一个劲地往里闯?

对于这一点,审案的两位郎中心里自然有数,但领导意图他们更有数,这件事,只能往小了办。

这两位郎中的名字,分别是胡士相、岳骏声,之所以提出他们的名字,是因为这两个人,绝非等闲之辈。

于是在一番讨论之后,张差案件正式终结,犯人动机先不提,犯人结局是肯定的----死刑(也算杀人灭口)。

但要杀人,也得有个罪名,这自然难不倒二位仁兄,不愧是刑部的人,很有专业修养,从大明律里,找到这么一
条:宫殿射箭、放弹、投砖石伤人者,按律斩。

为什么伤人不用管,伤什么人也不用管,案件到此为止,就这么结案,大家都清净了。如此结案,也算难得糊涂,
事情的真相,将就此被彻底埋葬。

然而这个世界上,终究还是有不糊涂,也不愿意装糊涂的人。

五月十一日 刑部大牢

七天了,张差已经完全习惯了狱中的生活,目前境况,虽然和他预想的不同,但大体正常,装疯很有效,真相依然
隐藏在他的心里。

开饭时间到了,张差走到牢门前,等待着今天的饭菜。但他并不知道,有一双眼睛,正在黑暗中注视着他。

根据规定,虽然犯人已经招供,但刑部每天要派专人提审,以防翻供。

五月十一日,轮到王之寀。

王之寀,字心一,时任刑部主事。

主事,是刑部的低级官员,而这位王先生虽然官小,心眼却不小,他是一个坚定的阴谋论者,认定这个疯子的背
后,必定隐藏着某些秘密。

凑巧的是,他到牢房里的时候,正好遇上开饭,于是他没有出声,找到一个隐蔽的角落,静静地注视着那个疯子。

因为在吃饭的时候,一个人是很难伪装的。

之后一切都很正常,张差平静地领过饭,平静地准备吃饭。

然而王之寀已然确定,这是一个有问题的人。

因为他的身份是疯子,而一个疯子,是不会如此正常的。

\section[\thesection]{}

所以他立即站了出来,打断了正在吃饭的张差,并告诉看守,即刻开始审讯。

张差非常意外,但随即镇定下来,在他看来,这位不速之客和之前的那些大官,没有区别。

审讯开始,和以前一样,张差装疯卖傻,但他很快就惊奇地发现,眼前这人一言不发,只是静静地看着他。

他表演完毕后,现场又陷入了沉寂,然后,他听到了这样一句话:

``老实说,就给你饭吃,不说就饿死你。''(实招与饭,不招当饿死)

在我国百花齐放的刑讯逼供艺术中,这是一句相当搞笑的话,但凡审讯,一般先是民族大义、坦白从宽,之后才是
什么老虎凳、辣椒水。即使要利诱,也是升官发财,金钱美女之类。

而王主事的诱饵,只是一碗饭。

无论如何,是太小气了。

事实证明,张差确实是个相当不错的人,具体表现为头脑简单,思想朴素,在吃一碗饭和隐瞒真相、保住性命之
间,他毫不犹豫地选择了前者。

于是他低着头,说了这样一句话:

``我不敢说。''

不敢说的意思,不是不知道,也不是不说,而是知道了不方便说。

王之寀是个相当聪明的人,随即支走了所有的人,然后他手持那碗饭,听到了事实的真相:

``我叫张差,是蓟州人,小名张五儿,父亲已去世。''

``有一天,有两个熟人找到我,带我见了一个老公公(即太监),老公公对我说,你跟我去办件事,事成后给你几亩
地,保你衣食无忧。''

``于是我就跟他走,初四(即五月四日)到了京城,到了一所宅子里,遇见另一个老公公。''

``他对我说,你只管往里走,见到一个就打死一个,打死了,我们能救你。''

``然后他给我一根木棍,带我进了宫,我就往里走,打倒了一个公公,然后被抓住了。''

王之寀惊呆了。

他没有想到,外界的猜想竟然是真的,这的的确确,是一次策划已久的政治暗杀。

但他更没有想到的是,这起暗杀事件竟然办得如此愚蠢,眼前这位仁兄,虽说不是疯子,但说是傻子倒也没错,而
且既不是武林高手,也不是职业杀手,最多最多,也就是个彪悍的农民。

\section[\thesection]{}

作案过程也极其可笑,听起来,似乎是群众推荐,太监使用,顺手就带到京城,既没给美女,也没给钱,连星级宾
馆都没住,一点实惠没看到,就答应去打人,这种傻冒你上哪去找?

再说凶器,一般说来,刺杀大人物,应该要用高级玩意,当年荆轲刺秦,还找来把徐夫人的匕首,据说是一碰就
死,退一万步讲,就算是杀个老百姓,多少也得找把短刀,可这位兄弟进宫时,别说那些高级玩意,菜刀都没一
把,拿根木棍就打,算是怎么回事.

从头到尾,这事怎么看都不对劲,但毕竟情况问出来了,王之寀不敢怠慢,立即上报万历。

可是奏疏送上去后,却没有丝毫回音,皇帝陛下一点反应都没有。

但这早在王之寀的预料之中,他老人家早就抄好了副本,四处散发,本人也四处鼓捣,造舆论要求公开的审判。

他这一闹,另一个司法界大腕,大理寺丞王士昌跳出来了,也跟着一起嚷嚷,要三法司会审。

可万历依然毫无反应,这是可以理解的,要知道,人家当年可是经历过争国本的,上百号人一拥而上,那才是大世
面,这种小场面算个啥。

照此形势,这事很快就能平息下去,但皇帝陛下没有想到,他不出声,另一个人却跳了出来。

这个人,就是郑贵妃的弟弟郑国泰。

事情的起因,只是一封奏疏。

就在审讯笔录公开后的几天,司正陆大受上了一封奏疏,提出了几个疑问:

既然张差说有太监找他,那么这个太监是谁?他曾到京城,进过一栋房子,房子在哪里?有个太监和他说过话,这个
太监又是谁?

这倒也罢了,在文章的最后,他还扯了句无关痛痒的话,大意是,以前福王册封的时候,我曾上疏,希望提防奸邪
之人,今天果然应验了!

这话虽说有点指桑骂槐,但其实也没说什么,可是郑国泰先生偏偏就蹦了出来,写了封奏疏,为自己辩解。

这就是所谓对号入座,它形象地说明,郑国泰的智商指数,和他的姐姐基本属同一水准。

这还不算,在这封奏疏中,郑先生又留下了这样几句话:

有什么推翻太子的阴谋?又主使过什么事?收买亡命之徒是为了什么?……这些事我想都不敢想,更不敢说,也不忍听。

该举动生动地告诉我们,原来蠢字是这么写的。

\section[\thesection]{}

郑先生的脑筋实在愚昧到了相当可以的程度,这种货真价实的此地无银三百两,言官们自然不会放过,很快,工科
给事中何士晋就做出了反应,相当激烈的反应:

``谁说你推翻太子!谁说你主使!谁说你收买亡命之徒!你既辩解又招供,欲盖弥彰!''

郑国泰哑口无言,事情闹到这个地步,已经收不住了。

此时,几乎所有的人都认为,事实真相即将大白于天下,除了王之寀。

初审成功后,张差案得以重审,王之寀也很是得意了几天,然而不久之后,他才发现,自己忽视了一个很重要的问
题:

张差装疯非常拙劣,为碗饭就开口,为何之前的官员都没看出来呢?

思前想后,他得出了一个非常可怕的结论:他们是故意的。

第一个值得怀疑的,就是首先审讯张差的刘廷元,张差是疯子的说法,即源自于此,经过摸底分析,王之寀发现,
这位御史先生,是个不简单的角色。

此人虽然只是个巡城御史,却似乎与郑国泰有着紧密的联系,而此后复审的两位刑部郎中胡士相、岳骏声,跟他交
往也很密切。

这似乎不奇怪,虽然郑国泰比较蠢,实力还是有的,毕竟福王受宠,主动投靠的人也不少。

但很快他就发觉,事情远没有他想象的那么简单。

因为几天后,刑部决定重审案件,而主审官,正是那位曾认定刘廷元结论的郎中,胡士相。

胡士相,时任刑部山东司郎中,就级别而言,他是王之寀的领导,而在审案过程中,王主事惊奇地发现,胡郎中一
直闪烁其辞,咬定张差是真疯,迟迟不追究事件真相。

一切的一切,给了王之寀一个深刻的印象:在这所谓疯子的背后,隐藏着一股庞大的势力。

而刘廷元、胡士相,只不过是这股势力的冰山一角。

但让他疑惑不解的是,指使这些人的,似乎并不是郑国泰,虽然他们拼命掩盖真相,但郑先生在朝廷里人缘不好,
加上本人又比较蠢,要说他是后台老板,实在是抬举了。

那么这一切,到底是怎么回事呢?

王之寀的感觉是正确的,站在刘廷元、胡士相背后的那个影子,并不是郑国泰。

这个影子的名字,叫做沈一贯。

\section[\thesection]{}

就沈一贯的政绩而言,在史书中也就是个普通角色,但事实上,这位仁兄的历史地位十分重要,是明朝晚期研究的
重点人物。

因为这位兄弟的最大成就,并不是搞政治,而是搞组织。

我们有理由相信,在工作期间,除了日常政务外,他一直在干一件事----拉人。

怎么拉,拉了多少,这些都无从查证,但有一点我们是确定的,那就是这个组织的招人原则--浙江人。

沈一贯,是浙江四明人,在任人唯亲这点上,他和后来的同乡蒋介石异曲同工,于是在亲信的基础上,他建立了一
个老乡会。

这个老乡会,在后来的中国历史上,被称为浙党。

这就是沈一贯的另一面,他是朝廷的首辅,也是浙党的领袖。

应该说,这是一个明智的决定,因为你必须清楚地认识到这样一点:

在万历年间,一个没有后台(皇帝),没有亲信(死党)的首辅,是绝对坐不稳的。

所以沈一贯干了五年,叶向高干了七年,所以赵志皋被人践踏,朱赓无人理会。

当然,搞老乡会的绝不仅仅是沈一贯,除浙党外,还有山东人为主的齐党,湖广人(今湖北湖南)为主的楚党。

此即历史上著名的齐、楚、浙三党。

这是三个能量极大、战斗力极强的组织,因为组织的骨干成员,就是言官。

言官,包括六部给事中,以及都察院的御史,给事中可以干涉部领导的决策,和部长(尚书)平起平坐,对中央事务
有很大的影响。

而御史相当于特派员,不但可以上书弹劾,还经常下到各地视察,高级御史还能担任巡抚。

故此,三党的成员虽说都是些六七品的小官,拉出来都不起眼,却是相当的厉害。

必须说明的是,此前明代二百多年的历史中,虽然拉帮结派是家常便饭,但明目张胆地搞组织,并无先例,先例即
由此而来。

这是一个很有趣的谜团。

早不出来,晚不出来,为何偏偏此时出现?

而更有趣的是,三党之间并不敌对,也不斗争,反而和平互助,这实在是件不符合传统的事情。

存在即是合理,一件事情之所以发生,是因为它有发生的理由。

有一个理由让三党陆续成立,有一个理由让他们相安无事。是的,这个理由的名字,叫做东林党。

\section[\thesection]{}

无锡的顾宪成,只是一个平民,他所经营的,只是一个书院,但几乎所有人都知道,这个书院可以藐视当朝的首
辅,说他们是木偶、婴儿,这个书院可以阻挡大臣复起,改变皇帝任命。

大明天下,国家决策,都操纵在这个老百姓的手中。从古至今,如此牛的老百姓,我没有见过。

无论是在野的顾宪成、高攀龙、赵南星,还是在朝的李三才,叶向高,都不是省油的灯,东林党既有社会舆论,又
有朝廷重臣,要说它是纯道德组织,鬼才信,反正我不信。

连我都不信了,明朝朝廷那帮老奸巨滑的家伙怎么会信,于是,在这样一个足以影响朝廷,左右天下的对手面前,
他们害怕了。

要克服畏惧,最有效、最快捷的方法,就是找一个人来和你一起畏惧。

史云:明朝亡于党争。我云:党争,起于此时。

刘廷元、胡士相不是郑国泰的人,郑先生这种白痴是没有组织能力的,他们真正的身份,是浙党成员。

但疑问在于,沈一贯也拥立过太子,为何要在此事上支持郑国泰呢?

答案是,对人不对事。

沈一贯并不喜欢郑国泰,更不喜欢东林党,因为公愤。

所谓公愤,是他在当政时,顾宪成之类的人总在公事上跟他过不去,他很愤怒,故称公愤。

不过,他最不喜欢的那个人,却还不是东林党----叶向高,因为私仇,三十二年的私仇。

三十二年前(万历十一年1583)叶向高来到京城,参加会试。

叶向高,字进卿,福建福清人,嘉靖三十八年生人。

必须承认,他的运气很不好,刚刚出世,就经历了生死考验

因为在嘉靖三十八年,倭寇入侵福建,福清沦陷,确切地说,沦陷的那一天,正是叶向高的生日。

据说他的母亲为了躲避倭寇,躲在了麦草堆里,倭寇躲完了,孩子也生出来了,想起来实在不容易。

大难不死的叶向高,倒也没啥后福,为了躲避倭寇,一两岁就成了游击队,鬼子一进村,他就跟着母亲躲进山里,
我相信,几十年后,他的左右逢源,机智狡猾,就是在这打的底。

倭寇最猖獗的时候,很多人都丢弃了自己的孩子(累赘),独自逃命,也有人劝叶向高的母亲,然而她说:

``要死,就一起死。''

但他们终究活了下来,因为另一个伟大的明代人物----戚继光。

\section[\thesection]{}

考试

嘉靖四十一年(1562),戚继光发动横屿战役,攻克横屿,收复福清,并最终平息了倭患。

必须说明,当时的叶向高,不叫叶向高,只有一个小名,这个小名在今天看来不太文雅,就不介绍了。

向高这个名字,是他父亲取的,意思是一步一步,向高处走。

事实告诉我们,名字这个东西,有时候改一改,还是很有效的。

隆庆六年(1572),叶向高十四岁,中秀才。

万历七年(1579),叶向高二十一岁,中举人。

万历十一年(1583),叶向高二十五岁,第二次参加会试。考试结束,他的感觉非常好。

结果也验证了他的想法,他考中了第七十八名,成为进士。现在,在他的面前,只剩下最后一关----殿试。

殿试非常顺利,翰林院的考官对叶向高十分满意,决定把他的名次排为第一,远大前程正朝着叶向高招手。

然而,接下来的一切,却发生了出人意料的变化。

因为从此刻起,叶向高就与沈一贯结下了深仇大恨,虽然此前,他们从未见过。

要解释清楚的是,叶向高的第七十八名,并非全国七十八名,而是南卷第七十八名。

明代的进士,并不是全国统一录取,而是按照地域,分配名额,具体分为三个区域,南、北、中,录取比例各有不
同。

所谓南,就是淮河以南各省,比例为55\%。北,就是淮河以北,比例为35\%。而中,是指云贵川三省,以及凤阳,
比例为10\%

具体说来是这么个意思,好比朝廷今年要招一百个进士,那么分配到各地,就是南部五十五人,北部三十五人,中
部十人。这就意味着,如果你是南部人,在考试中考到了南部第五十六名,哪怕你成绩再好,文章写得比北部第一
名还好,你也没法录取。

而如果你是中部人,哪怕你文章写得再差,在南部只能排到几百名后,但只要能考到中部卷前十名,你就能当进士。

\section[\thesection]{}

这是一个历史悠久的规定,从二百多年前,朱元璋登基时,就开始执行了,起因是一件非常血腥的政治案件----南
北榜案件。这个案件是笔糊涂账,大体意思是一次考试,南方的举人考得很好,好到北方没几个能录取的,于是有
人不服气,说是考官舞弊,事情闹得很大,搞到老朱那里,他老人家是个实在人,也不争论啥,大笔一挥就干掉了
上百人。

可干完后,事情还得解决,因为实际情况是,当年的北方教学质量确实不如南方,你把人杀光了也没辙。无奈之
下,只好设定南北榜,谁都别争了,就看你生在哪里,南方算你倒霉,北方算你运气。

到明宣宗时期,事情又变了,因为云贵川一带算是南方,可在当年是蛮荒之地,别说读书,混碗饭吃都不容易,要
和南方江浙那拨人对着考,就算是绝户。于是皇帝下令,把此地列为中部,作为特区,而凤阳,因为是朱元璋的老
家,还特别穷,特事特办,也给列了进去。

当然了,这也是没办法的事,毕竟基础不同,底子不同,在考试上,你想一夜之间人类大同,那是不可能的,所以
现在这套理论还在用。我管这个,叫考试地理决定论。

这套理论很残酷,也很真实,主要是玩机率,看你在哪投胎。

比如你要是生在山东、江苏、湖北之类的地方,就真是阿弥陀佛了,这些地方经常盘踞着一群读书不要命的家伙,
据我所知,有些``乡镇中学''(地图上都找不到)的学生,高二就去高考(不记成绩),大都能考六百多分(七百五十分
满分),美其名曰:锻炼素质,明年上阵。

每念及此,不禁胆战心惊,跟这帮人做邻居的结果是:如果想上北大,六百多分,只是个起步价。

应该说,现在还是有所进步的,逼急还能玩点阴招,比如说……更改户口。

不幸的是,明代的叶向高先生没法玩这招,作为南卷的佼佼者,他有很多对手,其中的一个,叫做吴龙徴。

这位吴先生,也是福建人,但他比其他对手厉害得多,因为他的后台叫沈一贯。

按沈一贯的想法,这个人应该是第一,然后进入朝廷,成为他的帮手,可是叶向高的出现,却打乱了沈一贯的部署。

于是,沈一贯准备让叶向高落榜,至少也不能让他名列前茅。

而且他认定,自己能够做到这一点,因为他就是这次考试的主考官。

但是很可惜,他没有成功,因为一个更牛的人出面了。

\section[\thesection]{}

主考官固然大,可再大,也大不过首辅。

叶向高虽然没有关系,却有实力。文章写得实在太好,好到其他考官不服气,把这事捅给了申时行,申大人一看,
也高兴得不行,把沈一贯叫过去,说这是个人才,必定录取!

这回沈大人郁闷了,大老板出面了,要不给叶向高饭碗,自己的饭碗也难保,但他终究是不服气的,于是最终结果
如下:

叶向高,录取,名列二甲第十二名。

这是一个出乎很多人意料的结果,因为若要整人,大可把叶向高同志打发到三甲,就此了事,不给状元,却又给个
过得去的名次,实在让人费解。

告诉你,这里面学问大了。

叶向高黄了自己的算盘,自然是要教训的。但问题是,这人是申时行保的,申首辅也是个老狐狸,如果要敷衍他,
是没有好果子吃的,所以这个面子不但要给,还要给足。而二甲十二名,是最恰当的安排。

因为根据明代规定,一般说来,二甲十二名的成绩,可以保证入选庶吉士,进入翰林院,但这个名次离状元相当
远,也不会太风光,恶心下叶向高,的确是刚刚好。

但不管怎么说,叶向高还是顺顺当当地踏上了仕途。此后的一切都很顺利,直到十五年后。

万历二十六年(1598),就在这一年,叶向高的命运被彻底改变,因为他等到了一个千载难逢的机会。

此时皇长子朱常洛已经出阁读书,按照规定,应该配备讲官,人选由礼部确定。

众所周知,虽说朱常洛不受待见,但按目前形势,登基即位是迟早的事,只要拉住这个靠山,自然不愁前程。所以
消息一出,大家走关系拉亲戚,只求能混到这份差事。

叶向高走不走后门我不敢说,运气好是肯定的,因为决定人选的礼部侍郎郭正域,是他的老朋友。

名单定了,报到了内阁,内阁压住了,因为内阁里有沈一贯。

沈一贯是个比较一贯的人,十五年前那档子事,他一直记在心里,讲官这事是张位负责,但沈大人看到叶向高的名
字,便心急火燎跑去高声大呼:

``闽人岂可作讲官?!''

这句话是有来由的,在明代,福建一向被视为不开化地带,沈一贯拿地域问题说事,相当阴险。

\section[\thesection]{}

张位却不买账,他也不管你沈一贯和叶向高有什么恩怨,这人我看上了,就要用!

于是,在沈一贯的磨牙声中,叶向高正式上任。

叶讲官不负众望,充分发挥主观能动,在教书的同时,和太子建立了良好的私人关系。

根据种种史料反映,叶先生应该是个相当灵活的人,我们有理由相信,在教书育人的同时,他还广交了不少朋友,
比如顾宪成,比如赵南星。

老板有了,朋友有了,地位也有了,万事俱备,要登上拿最高的舞台,只欠一阵东风。

一年后,风来了,却是暴风。

万历二十七年(1601),首辅赵志皋回家了,虽然没死,也没退,但事情是不管了,张位也走了,内阁,只剩下了沈
一贯。

缺了人就要补,于是叶向高的机会又来了。

顾宪成是他的朋友,朱常洛是他的朋友,他所欠缺的,只是一个位置。

他被提名了,最终却未能入阁,因为内阁,只剩下了沈一贯。

麻烦远未结束,内阁首辅沈一贯大人终于可以报当年的一箭之仇了,不久后,叶向高被调出京城,到南京担任礼部
右侍郎。

南京礼部主要工作,除了养老就是养老,这就是四十岁的叶向高的新岗位,在这里,他还要呆很久。

很久是多久?十年。

这十年之中,朝廷里很热闹,册立太子、妖书案,搞得轰轰烈烈。而叶向高这边,却是太平无事。

整整十年,无人理,无人问,甚至也无人骂、无人整。

叶向高过得很太平,也过得很惨,惨就惨在连整他的人都没有。对于一个政治家而言,最痛苦的惩罚不是免职、不
是罢官,而是遗忘。

叶向高,已经被彻底遗忘了。

一个前程似锦的政治家,在政治生涯的黄金时刻,被冷漠地抛弃,对叶向高而言,这十年中的每一天,全都是痛苦
的挣扎。

但十余年之后,他将感谢沈一贯给予他的痛苦经历,要想在这个冷酷的地方生存下去,同党是不够的,后台也是不
够的,必须亲身经历残酷的考验和磨砺,才能在历史上写下自己的名字。

因为他并不是一个普通的首辅,在不久的未来,他将超越赵志皋、张位、甚至申时行、王锡爵。他的名字将比这些
人更为响亮夺目。

因为一个极为可怕的人,正在前方等待着他。而他,将是唯一能与之抗衡的人。这个人,叫做魏忠贤。

\section[\thesection]{}

万历三十五年(1607),沈一贯终于走了,年底,叶向高终于来了。

但沈一贯的一切,都留了下来,包括他的组织,他的势力,以及他的仇恨。

所以刘廷元、胡士相也好,疯子张差也罢,甚至这件事情是否真的发生过,根本就不要紧。

梃击,不过是一个傻子的愚蠢举动,并不重要,重要的是,通过这件事情,能够打倒什么,得到什么。

东林党的方针很明确,拥立朱常洛,并借梃击案打击对手,掌控政权。

所以浙党的方针是,平息梃击案,了结此事。

而王之寀,是一个找麻烦的人。

这才是梃击案件的真相。

对了,还忘了一件事:虽然没有迹象显示王之寀和东林党有直接联系,但此后东林党敌人列出的两大名单(点将录、
朋党录)中,他都名列前茅。

再审

王之寀并不简单,事实上,是很不简单。

当他发现自己的上司胡士相有问题时,并没有丝毫畏惧,因为他去找了另一个人----张问达。

张问达,字德允,时任刑部右侍郎,署部事。

所谓刑部右侍郎、署部事,换成今天的话说,就是刑部常务副部长。也就是说,他是胡士相的上司。

张问达的派系并不清晰,但清晰的是,对于胡士相和稀泥的做法,他非常不满。接到王之寀的报告后,他当即下
令,由刑部七位官员会审张差。

这是个有趣的组合,七人之中,既有胡士相,也有王之寀,可以听取双方意见,又不怕人捣鬼,而且七个人审讯,
可以少数服从多数。

想法没错,做法错了。因为张问达远远低估了浙党的实力。

在七个主审官中,胡士相并不孤单,大体说来,七人之中,支持胡士相,有三个人,支持王之寀的,有两个。

于是,审讯出现了戏剧化的场景。

张差恢复了理智,经历了王之寀的突审和反复,现在的张差,已经不再是个疯子,他看上去,十分平静。

主审官陆梦龙发问:

``你为什么认识路?''

这是个关键的问题,一个平民怎样来到京城,又怎样入宫,秘密就隐藏在答案背后。

顺便说明一下:陆梦龙,是王之寀派。

出乎所有人的意料,没有等待,没有反复,他们很快就听到了这个关键的答案:

``我是蓟州人,如果没有人指引,怎么进得去?''

此言一出,事情已然无可隐瞒。

再问:``谁指引你的?''

答:``庞老公,刘老公。''

完了,完了。

\section[\thesection]{}

虽然张差没有说出这两个人的名字,但大家的人心中,都已经有了确切的答案。

庞老公,叫做庞保,刘老公,叫做刘成。

大家之所以知道答案,是因为这两个人的身份很特殊----他们是郑贵妃的贴身太监。

陆梦龙呆住了,他知道答案,也曾经想过无数次,却没有想到,会如此轻易地得到。

就在他惊愕的那一瞬间,张差又说出了更让人吃惊的话:

``我认识他们三年了,他们还给过我一个金壶,一个银壶。''(予我金银壶各一)

陆梦龙这才明白,之前王之寀得到的口供也是假的,真相刚刚开始!

他立即厉声追问道:

``为什么(要给你)?!''

回答干净利落,三个字:

``打小爷!''

声音不大,如五雷轰顶。

因为所有人都知道,所谓小爷,就是太子爷朱常洛。

现场顿时大乱,公堂吵作一团,交头接耳,而此时,一件更诡异的事情发生了。

作为案件的主审官,胡士相突然拍案而起,大喝一声:

``不能再问了!''

这一下大家又懵了,张差招供,您激动啥?

但他的三位同党当即反应过来,立刻站起身,表示审讯不可继续,应立即结束。

七人之中,四对三,审讯只能终止。

但形势已不可逆转,王之寀、陆梦龙立即将案件情况报告给张问达,张侍郎十分震惊。

与此同时,张差的口供开始在朝廷内外流传,舆论大哗,很多人纷纷上书,要求严查此案。

郑贵妃慌了,天天跑到万历那里去哭,但此时,局势已无法挽回。

然而,此刻压力最大的人并不是她,而是张问达,作为案件的主办人,他很清楚,此案背后,是两股政治力量的死
磕,还搭上太子、贵妃、皇帝,没一个省油的灯。

案子如果审下去,审出郑贵妃来,就得罪了皇帝,可要不审,群众那里没法交代,还会得罪东林、太子,小小的刑
部右侍郎,这拨人里随便出来一个,就能把自己整死。

总而言之,不能审,又不能不审。

无奈之下,他抓耳挠腮,终于想出了一个绝妙的解决方案。

\section[\thesection]{}

在明代的司法审讯中,档次最高的就是三法司会审,但最隆重的,叫做十三司会审。

明代的六部,长官为尚书、侍郎,部下设司,长官为郎中、员外郎,一般说来是四个司,比如吏部、兵部、工部、
礼部都是四个司,分管四大业务,而刑部,却有十三个司。

这十三个司,分别是由明朝的十三个省命名,比如胡士相,就是山东司的郎中,审个案子,竟然把十三个司的郎中
全都找来,真是煞费苦心。

此即所谓集体负责制,也就是集体不负责,张问达先生水平的确高,看准了法不责众,不愿意独自背黑锅,毅然决
定把大家拉下水。

大家倒没意见,反正十三个人,人多好办事,打板子也轻点。

可到审讯那天,人们才真切地感受到,中国人是喜欢热闹的。

除了问话的十三位郎中外,王之寀还带了一批人来旁听,加上看热闹的,足有二十多人,人潮汹涌,搞得跟菜市场
一样。

这次张差真的疯了,估计是看到这么多人,心有点慌,主审官还没问,他就说了,还说得特别彻底,不但交代了庞
老公就是庞保,刘老公就是刘成,还爆出了一个惊人的内幕:

按张差的说法,他绝非一个人在战斗,还有同伙,包括所谓马三舅、李外父,姐夫孔道等人,是货真价实的团伙作
案。

精彩的还没完,在审讯的最后,张差一鼓作气,说出了此案中最大的秘密:红封教。

红封教,是个邪教,具体组织结构不详,据张差同志讲,组织头领有三十六号人,他作案,就是受此组织指使。

一般说来,凑齐了三十六个头领,就该去当强盗了,这话似乎太不靠谱,但经事后查证,确有其事,刑部官员们再
一查,就不敢查了,因为他们意外发现,红封教的起源地,就是郑贵妃的老家。

而据某些史料反映,郑贵妃和郑国泰,就是红封教的后台。这一点,我是相信的,因为和同时期的白莲教相比,这
个红封教发展多年,却发展到无人知晓,有如此成就,也就是郑贵妃这类脑袋缺根弦的人才干得出来。

张差确实实在,可这一来,就害苦了浙党的同胞们,审案时丑态百出,比如胡士相先生,负责做笔录,听着听着写
不下去了,就把笔一丢了事,还有几位浙党郎中,眼看这事越闹越大,竟然在堂上大呼一声:

``你自己认了吧,不要涉及无辜!''

\section[\thesection]{}

但总的说来,浙党还是比较识相的,眼看是烂摊子,索性不管了,同意如实上报。上报的同时,刑部还派出两拨
人,一拨去找那几位马三舅、李外父,孔道姐夫,另一拨去皇宫,找庞保、刘成。

于是郑贵妃又开始哭了,几十年来的保留剧目,屡试不爽,可这一次,万历却对她说:

``我帮不了你了。''

这是明摆着的,张差招供了,他的那帮外父、姐夫一落网,再加上你自己的太监,你还怎么跑?

但老婆出事,不管也是不行的,于是万历告诉郑贵妃,而今普天之下,只有一个人能救她,而这个人不是自己。

``唯有太子出面,方可了解此事。''

还有句更让人难受的话:

``这事我不管,你要亲自去求他。''

郑贵妃又哭了,但这次万历没有理她。

于是不可一世的郑贵妃收起了眼泪,来到了宿敌的寝宫。

事实证明,郑小姐装起孙子来,也是巾帼不让须眉,进去看到太子,一句不说就跪,太子也客气,马上回跪,双方
爬起来后,郑贵妃就开始哭,一边哭一边说,我真没想过要害你,那都是误会。

太子也不含糊,反应很快,一边做垂泪状(真哭是个技术活),一边说,我明白,这都是外人挑拨,事情是张差自己
干的,我不会误会。

然后他叫来了自己的贴身太监王安,让他当即拟文,表明自己的态度。随即,双方回顾了彼此间长达几十年的传统
友谊,表示今后要加强沟通,共同进步,事情就此圆满结束。

这是一段广为流传的史料,其主题意境是,郑贵妃很狡诈,朱常洛很老实,性格合理,叙述自然,所以我一直深信
不疑,直到我发现了另一段史料,一段截然不同的史料:

开头是相同的,郑贵妃去向万历哭诉,万历说自己没办法,但接下来,事情出现变化--他去找了王皇后。

这是一个很聪明的举动,因为皇后没有帮派,还有威望,找她商量是再合适不过了。

皇后的回答也直截了当:

``此事我也无法,必须找太子面谈。''

很快,老实太子来了,但他给出的,却是一个截然不同的答案:

``此事必有主谋!''

\section[\thesection]{}

这句话一出来,明神宗脸色就变了,郑贵妃更是激动异常,伸个指头出来,对天大呼:

``如果这事是我干的,我就全家死光!(奴家赤族)''

这句话说得实在太绝,于是皇帝也吼了一句:

``这是我的大事,你全家死光又如何?!(稀罕汝家)''

贵妃发火了,皇帝也发火了,但接下来的一句话,却浇灭了所有人的激情:

``我看,这件事情就是张差自己干的。''

说这句话的人,就是太子朱常洛。虽然几秒钟之前,他还曾信誓旦旦地要求追查幕后真凶。

于是大家都满意了,为彻底平息事端,万历四十三年(1615)五月二十八日,二十多年不上朝的万历先生终于露面了。
他召来了内阁大臣、文武百官,以及自己的太子,皇孙,当众训话,大致意思是:自己和太子关系很好,你们该干
嘛就干嘛,少来瞎搅和,此案是张差所为,把他干掉了事,就此定案,谁都别再折腾。太子的表现也很好,当众抒
发父子深情,给这出闹剧画上了圆满句号。

一天后,张差被凌迟处死,十几天后,庞保和刘成不明不白地死在了刑部大牢里,就杀人灭口而言,干得也还算相
当利落。

轰动天下的疯子袭击太子事件就此结束,史称明宫三大案之``梃击''。

梃击是一起复杂的政治案件,争议极大,有很多疑点,包括幕后主使人的真实身份。

因为郑贵妃要想刺杀太子,就算找不到绝顶高手,到天桥附近找个把卖狗皮膏药的,应该也不是问题,选来选去就
找了个张差,啥功夫没有,还养了他三年。这且不论,动手时连把菜刀都没有,拿根木棍闯进宫,就想打死太子,
相当无聊。

所以有些人认为,梃击案是朝廷某些党派所为,希望混水摸鱼,借机闹事,甚至有人推测此事与太子有关。因为这
事过于扯淡,郑贵妃不傻,绝不会这么干。

但我的看法是,这事是郑贵妃干的,因为她的智商,就是傻子水平。

对于梃击案,许多史书的评价大都千篇一律,郑贵妃狡猾,万历昏庸,太子老实,最后老实的太子在正义的东林官
员支持下,战胜了狡猾的郑贵妃。

这都是蒙人的。

仔细分析就会发现,郑贵妃是个蠢人,万历老奸巨滑,太子也相当会来事,而东林官员们,似乎也不是那么单纯。

\section[\thesection]{}

所以事实的真相应该是,一个蠢人办了件蠢事,被一群想挑事的人利用,结果被老滑头万历镇了下来,仅此而已。

之所以详细介绍此事,是因为我要告诉你:在接下来的叙述中,你将逐渐发现,许多你曾无比熟悉的人,其实十分
陌生,许多你曾坚信的事实,其实十分虚伪,而这,不过是个开头。

以上,就是万历同志执政四十余年的大致成就,具体说来,就是斗争、斗争、再斗争。

先斗倒张居正,再斗争国本、妖书、梃击,言官、大臣、首辅轮番上阵,一天到晚忙活这些事,几十年不上朝,国
家是不怎么管了,山东、山西、河南、江西及大江南北相继告灾,文书送上去,理都不理。而更滑稽的是,最大的
受害者不是老百姓,而是官员。

在万历年间,如果你考上进士,也别高兴,因为考上了,未必有官做。

一般说来,朝代晚期,总会出现大量贪官污吏,欺压百姓,摊派剥削,但我可以很负责地讲,万历年间这个问题很
不严重,因为压根就没官。

老子曾经说过,最好的国家,是老百姓不知道统治者是谁,从某个角度讲,万历同志做到了。

按照以往制度,六部给事中的名额,应该是五十余人,而都察院的名额,应该是一百余人。可到了万历三十五年,
六部给事中只有四个人,而且其中五个部没有都给事中,连个管事的都没有,都察院的十三道御史,竟然只剩下五
个人,干几十个人的活,累得要死。

更要命的是,都察院是监察机构,经常要到全国各地视察,五个人要巡全国十三个省,一年巡到头,连家都回不
去,其中最惨的一位兄弟,足足在外巡了六年,才找到个替死鬼,回了京城。

基层御史只有五个,高层御史却是一个都没有,左都御史、右都御史经常空缺,都察院考勤都没人管,来不来,干
不干,全都靠自觉。

最惨的,还是中央六部,当时的六部,部长副部长加起来,一共只有四个。礼部没有部长,户部只有一个副部长,
工部连副部长都没有,只有几个郎中死顶。

其实候补进士很多,想当官的人也多,可是万历同志就是不批,你能咋办? 最搞笑的是,即使万历批了,发了委任
状,你也当不了官。

\section[\thesection]{}

比如万历三十七年(1609),朝廷实在顶不住了,死磨硬泡,才让万历先生批了几百名官员的上任凭证。可是几个月
过去了,竟然无人上任,再一查才知道,凭证压根就没发。

因为根据规定,发放凭证的是吏部都给事中,可这个职位压根就没人,鬼来发证?

官员倒霉不说,还连累了犯人,到万历三十八年(1610),刑部大牢里已经关了上千名犯人,一直没人管,有些小偷
小摸的,审下来也就是个治安处罚,却被关了好几年,原因很简单,刑部长官退了,又没人接,这事自然无人理。

不过犯人还是应该感到幸运,毕竟管牢房伙食的人还在。

当官很难,辞官也难,你今天上完班,说明天我不干了,谁都不拦你,但要等你的辞职报告批下来,估计也得等个
几年。如果你等不及了,就这么走也行,没人追究你。

总而言之,万历的这个政府,基本属于无政府,如此看来,他应该属于无政府主义者,思想如此超前,着实不易。

一般说来,史料写到这段,总是奋笔疾书,痛斥万历昏庸腐朽,政府实效,人民生活在水深火热之中。

而在我看来,持这种看法的,不是装蒜,就是无知。

因为事实绝非如此。万历年间,恰恰是明代经济最发达的时期,所谓资本主义萌芽,正是兴盛于此。

而老百姓的生活,那真是滋润,想干什么就干什么,明初的时候,出去逛要村里开介绍信,未经许可乱转,抓住就
是充军。万历年间,别说介绍信,连户口(黄册)都不要了,你要有本事,跑到美国都没人管你。

至于日常活动,那就更不用说了,许多地方衙门里压根就没官,也没人收苛捐杂税,贪污受贿,许多农民涌入城市
打工,成为明代的农民工。

这帮人也很自由,今天给你干几天,明天给他干几天,雇主大都是江浙一带的老板,虽说也有些不厚道的老板拖欠
民工工资,但大体而言,还算是守规矩。

久而久之,城市的人越来越多,这些人就是所谓的市民,明代著名的市民文化由此而起,而最受广大市民欢迎的文
化读物,就是《金瓶梅》、三言等等。

\section[\thesection]{}

按照现在的说法,这些书籍大都含有封建糟粕,应该限制传播,至少也要写个此处划掉多少字之类的说明,但当时
连政府都没人管,哪有人理这个,什么足本善本满天飘,肆无忌惮。

穿衣服也没谱,朱元璋那时候,衣服的材料、颜色,都要按身份定,身份不到就不能穿,穿了就要打屁股,现在是
没人管了,想穿什么穿什么,还逐渐出现了性别混装,也就是男人穿女装,涂脂抹粉,搞女性化(不是太监),公然
招摇过市,还大受欢迎。

穿女装还好,而更耸人听闻的是,经常有些人(不是个把),什么都没穿,光着身子在市面上走来走去,即所谓裸奔。
刚奔的时候还有人喊,奔久了也就见怪不怪了。

至于思想,那更是没法说,由于王守仁的心学大量传播,特别是最为激进的泰州学派,狂得没边,什么孔子孟子,
三纲五常,那都是``放屁''、``假道学'',总而言之,打倒一切权威,藐视一切准则。

封建礼教也彻底废了,性解放潮流席卷全国,按照``二拍''的说法,女人离异再嫁,是再寻常不过的事情,青楼妓
院如雨后春笋,艳情小说极其流行,涌现了许多优秀作者和忠实读者群。今天流传下来的所谓明代艳情文学,大都
是那时的产物。

说到这个份上,我也无话可说了。

自然经济,这是纯粹的自然经济。

万历年间的真相大抵如此,一个政治纷乱,经济繁荣、文化灿烂、生机勃勃的世界。

然而这个世界,终究被毁灭了。

毁灭的起因,是一个人。这人的名字,叫李成梁。

不世之功臣

李成梁,是一个猛人,还不是一般的猛。

他出生于嘉靖五年(1526),世袭铁岭卫指挥佥事,算是高级军官,可到他这辈,混得相当差劲,家里能卖的都卖
了,非常穷,穷得连进京继承官职的路费都没有。

他本人也混得很差,直到四十岁,还是个穷秀才。后来找人借钱,好歹凑了个数(继承官职,是要行贿的),这才捞
到官位,还真不是一般的惨。

但此后,他便一发不可收拾。

当时的辽东很乱,虽然俺答部落改行做了生意,不抢了,但其他部落看俺答发了财,自己又没份,更不消停,一窝
蜂地来抢,什么插汉部、泰宁部、朵颜部、王杲部,乱得一塌糊涂,乱到十年之内,竟然有三位明朝大将战死。

然后李成梁来了,然后一切都解决了。

\section[\thesection]{}

打仗,实际上和打麻将差不多,排兵布阵,这叫洗牌,掷色子,就是开打,战况多变,就是不知道下一张摸什么
牌,而要想赢牌,一靠技术,二靠运气。

靠死运气,怎么打怎么赢,所谓福将。

靠死里打,怎么打怎么赢,所谓悍将。

李成梁,应该是福将加悍将。

隆庆四年(1570),李成梁到辽东接任总兵,却没人办交接手续,因为前任总兵王首道,是被蒙古人干掉的。

当时辽东的形势很乱,闹事的部落很多,要全列出来,估计得上百字,大致说来,闹得最凶的有如下几个:

蒙古方面:插汉部,首领土蛮。泰宁部,首领速巴亥。朵颜部,首领董狐狸。

女真方面:建州女真,王杲部。海西女真,叶赫部、哈达部,首领清佳努、孟格部禄。

这些名字很难记,也全都不用记,因为他们很快就会被李成梁干掉。

以上这些人中,最不消停的,是土蛮。他的部落最大,人最多,有十几万人,比较团结,具体表现为抢劫时大家一
起来,每次抢的时候,都是漫天烟尘,铺天盖地,明军看到就跑,压根无法抵挡。

所以李成梁来后,第一个要打的,就是这只出头鸟。

自从李大人出马后,土蛮就从没舒坦过。从万历元年起,李成梁大战五次,小战二十余次,基本算是年年打,月月
打。

总打仗不奇怪,奇怪的是,李成梁每次都打赢。

其实他的兵力很少,也就一两万人,之所以每战必胜,大致有两个原因:首先是技术问题,他属下的辽东铁骑,每
人配发三眼火铳,对方用刀,他用火枪,明明白白就欺负你。

其次是战术问题,李成梁不但骁勇善战,还喜欢玩阴招,对手来袭时,准备大堆财物,摆在外面,等蒙古人下马抢
东西,他就发动攻击。此外,他还不守合同,经常偷袭对手,靠这两大优势,十年之内,他累计斩杀敌军骑兵近五
万人,把土蛮折腾得奄奄一息。

看到这段史料,再回忆起他儿子李如松同志的信用问题,不禁感叹:家庭教育,是很重要的。

土蛮歇了,泰宁也很惨,被打得到处跑不说,万历十年(1582),连首领速巴孩都中了埋伏,被砍了脑袋。

蒙古休息了,女真精神了。

\section[\thesection]{}

女真,世代居住于明朝辽东一带,到万历年间,主要分为四个部落:海西女真、建州女真、黑龙江女真、东海女真。

黑龙江和东海的这两拨人,一直比较穷,吃饭都成问题,连抢劫的工具都没有,基本上可以忽略。

而最让人头疼的,是建州女真。

当时的建州女真,头领叫做王杲,这人用今天的话说,是个给脸不要脸的人。

他原本在这里当地主,后来势力大了,明朝封他当建州卫指挥使,官位不低,这人不满意,自封当了都督。

王杲的地盘靠近抚顺,明朝允许他和抚顺做生意,收入很高,这人不满意,诱杀了抚顺的守将,非要去抢一把。

因为他经常不满意,所以李成梁对他也不满意,万历元年(1573),找个机会打了一仗。

开始明军人少,王杲占了便宜,于是他又不满意了,拼命地追,追到后来,进了李成梁的口袋,又拼命跑,从建州
跑到海西,李将军也是个执着的人,从建州追到海西,王杲束手无策,只能投降。

投降后,属下大部被杀,他本人被送到京城,剐了。

但在乱军之中,有一个人跑了,这个人叫阿台,是王杲的儿子。十年后,祸患即由此而起。

建州女真完了,下一个要解决的,是海西女真。

海西女真中,第一个被解决的,是叶赫部。

应该承认一点:李成梁除掉叶赫部的方法,是相当无耻的。

万历十一年(1583),叶赫部首领,贝勒清佳努率两千余人来到开原,准备进行马市贸易。在这里,他们将用牲畜换
取自己所需的各种物资。

高兴而来,满载而归,过去无数次,他们都是这样做的。

然而这次不同。

当他们准备进入开原城时,守城明军拦住了他们,说:

``你们人太多了,不能全部入城。''

清佳努想了一下,回答:

``好的,我只带三百人进城。''

但当他入城后,才惊奇地发现,这里没有商人,没有小贩,没有拥挤的人流,只是一片空地。

然后,他听到了炮声。

炮声响起的同时,城外的李成梁下达了攻击令,数千名明军蜂拥而起,短短几分钟之内,清佳努和三百随从全部被
杀,城外的明军也很有效率,叶赫部只跑掉了四百四十人。

然后是哈达部。

\section[\thesection]{}

相对而言,哈达部人数少,也不怎么惹事,李成梁本来也没打算收拾他们。但不幸的是,哈达部有个孟格部禄,孟
格部禄又有个想法:和叶赫部联合。

这就有点问题了,因为李成梁先生的目标,并不是蒙古,甚至也不是女真,他选择敌人的唯一标准,就是强大。

强大,强大到足以威胁帝国的程度,就必须消灭。

本着这一指导原则,李成梁偷袭了哈达部,将部落主力歼灭,解决了这个问题。

自隆庆四年至万历十九年,在二十二年的时间里,李成梁把辽东变成了静土,并不干净,却很安静。

如果各部落团结,他就挑事,挑出矛盾后,就开始分类。听话的,就给胡萝卜吃;不听话的,就用大棒。多年来,
他作战上百次,大捷十余次,歼敌十多万人,年年立功受奖,年年升官发财,连戚继光都要靠边站,功绩彪炳,无
懈可击。

除了万历十一年的那一场战役。

万历十一年(1583),李成梁得到了一个消息:阿台出现了。

从战火中逃离的阿台,带着对明朝的刻骨仇恨,开始了他的二次创业。经过十年不懈的杀人抢劫,他成功地由小土
匪变成了大强盗,并建立了自己的营寨,继续与明朝对抗。

对付这种人,李成梁的办法有,且只有一个。

万历十一年(1583)二月,他自抚顺出兵,攻击阿台的营寨。

攻击没有想象中顺利,阿台非常顽强,李成梁竭尽全力,放火强攻全用上,竟然未能攻克,无奈之下,他找来了两
个帮手。

这两个帮手,实际上是帮他带路的向导,一个叫尼堪外兰,另一个,叫觉昌安。

这两位都是当地部落首领,所以李成梁希望他们出面,去找阿台谈判,签个合同把事情结了。当然了,遵不遵守合
同,那就另说了,先把人弄出来。

两个人就这么去了,但是,李成梁疏漏了一个重要的细节----动机。

同为建州女真,这两个人有着不同的动机,和不同的身份。

尼堪外兰是附近的城主,之所以帮助李成梁,是因为除掉阿台,他就能够获得利益。

而觉昌安跑过来,只是为了自己的孙女----阿台是他的孙女婿。

当两人来到城寨下时,不同的动机,终将导致不同的行为。

\section[\thesection]{}

觉昌安对尼堪外兰说,我进去劝降,你在外面等着,先不要动手。

尼堪外兰同意。

觉昌安进入城内,见到了阿台,开始游说。

很可惜,他的口才实在不怎么样,说得口干舌燥,阿台压根就没反应。

时间不断逝去,等在城外的尼堪外兰开始不耐烦了。

但他很明白,觉昌安还在里面,无论如何不能动手。

正在这个关键的时刻,李成梁的使者来了,只传达了一句话:

``为何还未解决?''

对李成梁而言,这只是个普通的催促。

但这句话,在尼堪外兰的脑海中,变成了命令。

他之所以跑来,不是为了觉昌安,更不是为阿台,只是为了利益和地盘,为了李成梁的支持。

于是,他打算用自己的方式去解决。

他走到城寨边,用高亢的声音,开始了自己的谈判:

``天朝大军已经到了,你们已经没有出路,太师(指李成梁)有令,若杀掉阿台者,就是此地之主!''

这是一个谎言。

所谓封官许愿,是尼堪外兰的创造,因为李成梁虽不守信用,但一个小小的营寨,打了就打了,还犯不着许愿开支
票。

但事实证明,人穷志短,空头支票,也是很有号召力的。

应该说,游牧民族是比较实诚的,喊完话后,没有思想斗争,没有激烈讨论,就有人操家伙奔阿台去了。

谁先砍的第一刀无人知晓,反正砍他的人是争先恐后,络绎不绝,最后被乱刀砍死,连觉昌安也未能幸免。

虽然城外的李成梁不知道怎么回事,但他知道该干什么,趁乱带兵杀了进去。

因为他不知道尼堪外兰的那个合同(估计知道了也没用),所以也就没有什么顾忌,办事也绝了点----城内共计两千
三百人,无一生还。

和觉昌安一起进城的,还有他的儿子塔克世,同样死在城里。

不过对于李成梁而言,这实在无关紧要,多死个把人无所谓,在他的战斗生涯中,这只是次微不足道的战斗,打扫
战场,捡完人头报功,回家睡觉。

尼堪外兰倒是高兴,虽然觉昌安是惨了点,毕竟讨好了李成梁,也算大功告成。

但在他们看不见的地方,有一个人已经点燃了火种,燎原冲天的烈焰,终将由此而起。他是觉昌安的孙子,他是塔
克世的儿子,他的名字,叫做努尔哈赤。

\section[\thesection]{}

万世之罪首

努尔哈赤很气愤----他应该气愤,他的祖父、父亲死了,而且死得很冤枉,看起来,李成梁害死了他的两位亲人,
实际上,是五个。

如果你还记得,觉昌安所以入城,是为了阿台的妻子,自己的孙女,当然,也就是努尔哈赤的堂姐,她也死在乱军
之中,这是第三个。

而阿台,自然就是努尔哈赤的堂姐夫,他是第四个,然而,他和努尔哈赤的关系,远比你想象得复杂得多。

嘉靖三十八年(1559),努尔哈赤生于赫图阿拉,他的祖父觉昌安和父亲塔克世都是女真世袭贵族,曾任建州左卫指
挥使。

滑稽的是,虽说家里成分很高,努尔哈赤的生活档次却很低,家里五兄弟,他排行老大,却很像小弟,从小就要帮
着干活,要啥没啥。

原因很简单,当时的女真部落,大都穷得掉渣,所谓女真贵族,虽说不掉渣,但也很穷,所以为了生计,小时候的
努尔哈赤曾到他的外祖父家暂住。

他的外祖父,就是我们的老朋友,王杲。

现在,先洗把脸,整理一下他们之间的关系:

努尔哈赤的母亲是王杲的女儿,也就是说,阿台是努尔哈赤的舅舅,但是阿台又娶了努尔哈赤的堂姐,所以他又是
努尔哈赤的堂姐夫,这还好,要换到努尔哈赤他爹塔克世这辈,就更乱了,因为阿台既是他的侄女婿,又是他的小
舅子。

乱是乱了点,考虑到当时女真族的生存状态,反正都是亲戚,也算将就了。

你应该能理解努尔哈赤有多悲痛了,在李成梁的屠刀之下,他失去了祖父觉昌安、外祖父王杲、父亲塔克世、堂姐
XX(对不起,没查到)以及舅舅阿台(兼堂姐夫)。

悲痛的努尔哈赤找到了明朝的官员,愤怒地质问道:

``我的祖父、父亲何故被害,给我一个说法!''

明朝的官员倒还比较客气,给了个说法:

``对不住,我们不是故意的,误会!''

很明显,这个说法不太有说服力,所以明朝官员还准备了一份礼物,以安抚努尔哈赤受伤的心灵。

这份礼物是三十份敕书,三十匹马、一份都督的任免状。

马和任免状大家都知道,我解释一下这敕书是个什么玩意。

所谓敕书,用今天的话说,就是贸易许可证。

\section[\thesection]{}

当时的女真部落,住在深山老林,除了狗熊啥都缺,过日子是过不下去了,要动粗,抢劫的经验又比不上蒙古,明
朝不愿开放互市,无奈之下,只好找到了这个折衷的方式,一道敕书,就能做一笔生意。三十分敕书,就是三十笔
生意。

明朝的意思很明白,人死了,给点补偿费,你走人吧。

客观地讲,这笔补偿费实在有点低,似乎无法平息努尔哈赤的愤怒。

然而他接受了。

他接受了所有的一切,回到了自己的家乡。

然后,他召集了族人,杀死了一头牛,举行了祭天仪式,拿出了祖上流传下来的十三副铠甲,宣布,起兵。

收了赔偿金再起兵,和收了钱不办事,似乎是异曲同工。但无论如何,努尔哈赤向着自己的未来迈出了第一步。这
一年,他二十五岁。

按照许多史料书籍的说法,下面将是努尔哈赤同志的光荣创业史,先起兵杀死尼堪外兰,然后统一建州女真,打败
海西女真最强的叶赫部落,至万历四十六年(1618),统一女真。

最后是基本类同的几句评价:非常光辉、非常励志、非常艰苦等等。

本人同意以上评语,却也要加上四个字:非常诡异。

据说努尔哈赤从小住在林子里,自己打猎、采集蘑菇,到市集上换东西,生活艰苦,所以意志坚定,渴了喝泉水,
饿了啃人参,所以身体强壮,天赋异禀,无师自通,所以极会打仗。

有以上几大优惠条件,所以十三副铠甲起兵,便不可收拾。

这绝不可能。

努尔哈赤起兵时,他的武器是弓箭,不是导弹,他带着十三副铠甲,不是十三件防弹衣,在当时众多的女真部落
中,他只不过是个小人物。

然而这个小人物,只用了三十多年,就统一了女真、建立了政权,占据了原本重兵集结的辽东,并正式向明朝挑战。

于是,我得出了一个结论:他得到了帮助。

而帮助他的这个人,就是李成梁。

我并不是阴谋论者,却惊奇地发现,无数的清代史料书籍中,都详细地描述了祖父觉昌安的惨死、李成梁的冷漠残
酷、努尔哈赤的无助,却不约而同地忽略了这样一个细节----努尔哈赤的祖父觉昌安,是李成梁的朋友、好朋友。

\section[\thesection]{}

据某些笔记的记载,努尔哈赤和李成梁之前很早就认识了,不但认识,努尔哈赤还给李成梁打过下手,他们之间,
还有一段极为神秘的纠葛。

据说努尔哈赤少年时,曾经因为闹事,被李成梁抓回来管教,不久之后,努尔哈赤被释放了,不是李成梁放的。

放走努尔哈赤的,是李成梁的老婆(小妾),而她放走努尔哈赤的理由也很简单----这人长得好(奇其貌,阴纵之出)。
至于他俩有无其他纠葛,我不知道,也不想知道。

相关的说法还有很多,什么努尔哈赤跟李成梁打过仗,一同到过京城,凡此种种,更不可思议的是,据说努尔哈赤
和李成梁还是亲家:努尔哈赤的弟弟,叫做舒尔哈齐,这位舒尔哈齐有个女儿,嫁给了李成梁的儿子李如柏,做妾。

而种种迹象表明,勇敢而悲痛的努尔哈赤,除了会打仗、身体好外,似乎还很会来事儿。他经常给李成梁送礼,东
西是一车车地拉,拍起马屁来,可谓``无所不用其极''(明史学者孟森语)。

所以,我们有理由认为,努尔哈赤和李成梁家族,有着某种不可告人的联系。

当你知道了这一点,再回头审视此前的几条记录,你就会发现,这个流传久远的故事的第二版本,以及隐藏其后的
真正秘密。

万历十一年(1583)二月,努尔哈赤祖父、父亲被误杀,努尔哈赤接受委任,管理部落。

万历十一年(1583)十二月,努尔哈赤部的死敌,海西女真中最强大的叶赫部贝勒清佳努被讨伐,所部两千余人全部
被杀,势力大减。

此后不久,努尔哈赤率兵攻打尼堪外兰,尼堪外兰自认有功,投奔李成梁,李成梁把他交给了努尔哈赤。

万历十五年(1587),海西女真哈达部孟格部禄联合叶赫,被李成梁发现,随即攻打,斩杀五百余人。

万历十六年(1588),叶赫部再度强大,李成梁再次出击,杀死清佳努的儿子那林脖罗,斩杀六百余人,叶赫部实力
大损,只得休养生息。

万历二十一年(1593),努尔哈赤终于统一建州女真,成为了女真最强大的部落。

万历二十一年(1593)九月,面对越来越强大的努尔哈赤,海西女真叶赫部联合哈达部、蒙古科尔沁部等九大部落,
组成联军,攻击努尔哈赤,失败,被杀四千余人,史称``古勒山之战''。

战后,努尔哈赤将叶赫部首领分尸,一半留存,一半交叶赫部。自此,叶赫部与爱新觉罗部不共戴天。据说其部落
首领于战败之时,曾放言如下:

``我叶赫部若只剩一女子,亦将倾覆之!''

叶赫部居住于那拉河畔,故又称叶赫那拉。

\section[\thesection]{}

这是几条似乎毫无关联的历史记载,其中某些之前还曾提过,但请你联系上下文再看一遍,因为秘密就隐藏其中。

如果你依然不得要领,那么我会给你一个提示----李成梁的习惯。

所谓习惯,是指一个人多年来不会轻易改变的行为方式,比如李成梁,他的习惯,是谁露头就打谁,谁强大就灭
谁,蒙古如此,叶赫部如此,哈达部也如此。

然而这个习惯,在努尔哈赤的身上,失效了。

整整十年,努尔哈赤从一个弱小部落逐渐强大,统一了建州女真,对如此庞然大物,李成梁却视而不见,海西女真
四分五裂,叶赫哈达部只是刚刚冒泡,就被他一顿猛打,压制下去。

这种举动,我认为可以用一个术语来形容----选择性失明。

更有意思的是,偶然之间,我还发现了一条这样的史料:万历二十年(1592)朝鲜战争爆发,李如松奉命出征,此
时,一个人自动请缨,要求入朝作战,保家卫国,支援李如松,当然了,这位仁兄我不说你也能猜到----努尔哈赤。

综上所述,我们可以得到这样一个结论:他们,是一伙的。

一切都从万历十一年的那场误会开始,劝降、误解、误杀,但接下来,真相被掩盖了。

等待着努尔哈赤的,并不是陌生、冷漠、孤独,而是交情、歉疚、庇护以及无私的帮助。

打击潜在的对手,给予发展的空间,得到的回应是,服从。

李成梁庇护努尔哈赤,和局势无关,只因为他认定,这是一个听话的亲信。

努尔哈赤主动请战,和明朝无关,只因为他认定,李氏家族是他的盟友。

而当若干年后尘埃落定,重整史料时,他们就会发现,一个得到敌人扶持、帮助的首领,是不太体面的。

所以掩盖和创造就开始了,所以几百年后,历史变成了现在的模样。

\section[\thesection]{}

李成梁做了件不公道的事情,他扶植了努尔哈赤,培养了明朝的敌人。

但公道地讲,他并不是故意的,更不是所谓的汉奸。

因为在他看来,所谓努尔哈赤,不过是一只柔弱的猫,给他吃穿,让他成长,最后成为一只温顺、听话的猫。

这只猫逐渐长大了,它的身躯变得强壮,叫声变得凄厉,脚掌长出了利爪,最后它亮出了獠牙。至此,我们终于知
道,它不是猫,而是老虎,它不是宠物,而是野兽。

但李成梁的观察能力,那真不是普通的差。

万历十九年(1591)李成梁退休,在此之前,他已打垮了蒙古、叶赫、哈达以及所有强大的部落,除了努尔哈赤。

非但不打,还除掉了他的对手,李成梁实在是个很够意思的人。

十年后,李成梁再次上任,此时的努尔哈赤已经统一了建州女真,极其壮大,但在李成梁看来,他似乎还是那只温
顺的猫,于是,他做出了一个错误的抉择----放弃六堡。

六堡,是明代在辽东一带的军事基地,是遏制女真的重要堡垒,也是辽东重镇抚顺、清河的唯一屏障,若丢失此
处,女真军队将纵横辽东、不可阻挡。

而此时的六堡,没有大兵压境,没有粮食饥荒,无论如何,都不应该、不需要、不能放弃。

然而李成梁放弃了。

万历三十四年(1606),李成梁正式放弃六堡,并迁走了这里的十余万居民,将此地拱手让给了努尔哈赤。

这是一个错误的抉择,也是一个无耻的抉择,李成梁将军不但丢失了战略重地,毁灭了十余万人的家园,还以此向
朝廷报功,所谓``招抚边民十余万'',实在不知世上有羞耻二字。

努尔哈赤毫无代价地占领六堡,明朝的繁荣、富饶,以及虚弱全部暴露在他的面前,那一刻,他终于看到了欲望,
和欲望实现的可能。

万历四十三年(1615),李成梁去世,年九十,不世之功臣,千秋之罪首。

建功一世,祸患千秋,万死不足恕其罪!

\section[\thesection]{}

几个月后,万历四十四年(1616),努尔哈赤在赫图阿拉建立政权,年号天命,史称后金,努尔哈赤称天命汗。这说
明他还是很给李成梁面子的,至少给了几个月的面子。

海西女真、叶赫部、哈达部,这些名词已不复存在,现在的女真,是唯一的女真,是努尔哈赤的女真,是拥有自己
文字(努尔哈赤找人造出来的)的女真,是拥有八旗制度,和精锐骑兵部队的女真。

辽东已经容不下努尔哈赤了,他从来不是一个老实本分的老百姓,也不是遵纪守法的好公民,当现有的财富和土地
无法满足他的欲望时,眼前这个富饶的大明帝国,将是他的唯一选择。

好了,面具不需要了,伪装也不需要了,唯一要做的,是抽出屠刀,肆无忌惮地砍杀他们的士兵,掳掠他们的百
姓,抢走他们的所有财富。

杀死士兵,可以得到装备马匹,掳掠百姓,可以获得奴隶,抢夺财富,可以强大金国。

当然了,这些话是不能明说的,因为一个强盗,杀人放火是不需要借口的,但对一群强盗而言,理由,是很有必要
的

万历四十六年(1618)正月,努尔哈赤在赫图阿拉,发出了战争的宣告:

``今岁(年),必征大明国!''

光叫口号是不够的,无论如何,还得找几个开战的理由。

四月,努尔哈赤找到了理由,七个。

此即所谓七大恨,在文中,努尔哈赤先生列举了七个明朝对不住他的地方,全文就不列了,但值得表扬的是,在挑
事方面,这篇文章,还真是下了点功夫。

祖父、父亲被杀,自然是要讲下的,李成梁的庇护,自然是不会提的,某些重大事件,也不能放过。比如边界问
题:擅自进入我方边界。经济问题:割了我们这边的粮食。外交问题:十名女真人在边界被害(这个理由好像很眼
熟)。

其中,最有意思的理由是:明朝偏袒叶赫、哈达部,对自己不公。

对于这句话,明朝有什么看法不好说,但被李成梁同志打残无数次的叶赫和哈达部,应该是有话要讲的。

这个七大恨,后来被包括袁崇焕在内的许多人驳斥过,凑热闹的事我就不干了。我只是认为,努尔哈赤先生有点多
余,想抢,抢就是了,想杀,杀就是了,何苦费那么大劲呢?

杀死一切敢于抵抗的人,抢走一切能够抢走的东西,占领一切能够占领的土地,目的十分明确。

抢掠,其实无须借口。

万历四十六年(1618)四月,努尔哈赤将他的马刀指向了第一个目标----抚顺。

\section[\thesection]{}

有一位古罗马的将领,在与日耳曼军队征战多年后,发出了这样的感叹:

他们不懂军事,却很彪悍,不懂权谋,却很狡猾。

这句简单的话,蕴藏着深厚的哲理。

很多人说过,最好的老师,不是特级教师,不是名牌学校,而是兴趣。

但我要告诉你,这个答案是错误的。

在这个世界上,最优秀的老师,是生存。

为了一块土地,为了一座房子,为了一块肉,为了在这个世界上多活一天,熟悉杀戮的技巧、掌握抢劫的诀窍,无
须催促、无须劝说,在每一天生与死的较量中,懂得生存,懂得如何去生存。

生存很困难,所以为了生存,必须更加狡诈、必须更加残暴。

所以在抚顺战役中,我们看到的,并不是纵横驰骋的游牧骑兵,光明正大的英勇冲锋,而是更为阴险狡诈的权谋诡
计。

万历四十六年(1618)四月十五日,努尔哈赤抵达抚顺近郊。

但他并没有发动进攻,却派人向城里散布了一个消息。

这个消息的内容是,明天,女真部落三千人,将携带大量财物来抚顺交易。

抚顺守将欣然应允,承诺打开城门,迎接商队的到来。

第二天(十五日)早晨,商队来了,抚顺打开了城门,百姓商贩走出城外,准备交易。

然后,满脸笑容的女真商队拿出了他们携带的唯一交易品----屠刀。

贸易随即变成了抢掠,商队变成了军队,很明显,女真人做无本生意的积极性要高得多。

努尔哈赤的军队再无须隐藏,精锐的八旗骑兵,在``商队''的帮助下,向抚顺城发动了进攻。

守城明军反应很快,开始组织抵抗,然而没过多久,抵抗就停止了,城内一片平静。

对于这个不同寻常的变化,努尔哈赤并不惊讶,因为这一切,都在他的计划之中。

很快,他就见到了计划中的那个关键棋子--李永芳。

李永芳,是抚顺城的守将之一,简单介绍下----是个叛徒。

他出卖抚顺城,所换来的,是副将的职称,和努尔哈赤的一个孙女。

抚顺失陷了,努尔哈赤抢到了所有能够抢到的财物、人口,明朝遭受了重大损失。

明军自然不肯干休,总兵张承胤率军追击努尔哈赤,却遭遇皇太极的伏兵,阵亡,全军覆没。

\section[\thesection]{}

抚顺战役,努尔哈赤掠夺了三十多万人口、牛马,获得了前所未有的财富,但这一切,只是个开始。

对努尔哈赤而言,继续抢下去,有很多的理由。

女真部落缺少日常用品,拿东西去换太麻烦,发展手工业不靠谱,抢来得最快。而更重要的是,当时的女真正在闹
灾荒,草地荒芜,野兽数量大量减少,这帮大爷又不耕地,粮食不够,搞得部落里怨声载道,矛盾激化。

所以继续抢,那是一举多得,既能够填补产业空白,又能解决吃饭问题,而且还能转嫁矛盾。

于是,万历四十六年(1618)七月,他再次出击,这次,他的目标是清河。

清河,就是今天的辽宁本溪,此地是通往辽阳、沈阳的必经之地,战略位置十分重要。

而清河的失陷过程也再次证明,努尔哈赤,实在是个狡猾狡猾的家伙。

七月初,他率军出征,却不打清河,反而跑到相反方向去闹腾,对外宣称是去打叶赫部,然后调转方向,攻击清河。

到了清河,也不开打,又是老把戏,先派奸细,打扮成商贩进了城,然后发动进攻,里应外合,清河人少势孤,守
军一万余人全军覆没。

之后的事情比较雷同,城内的十几万人口被努尔哈赤全数打包带走,有钱、有奴隶、有粮食,空白填补了,粮食保
证了,矛盾缓和了。

但他留下的,是一片彻底的白地,是无数被抢走口粮而饿死的平民,是无数家破人亡的惨剧,痛苦、无助。

无论什么角度、什么立场、什么观点、什么利益、什么目的、什么动机、什么想法、什么情感、什么理念、都应该
承认一点,至少一点:

这是抢掠,是自私、无情、带给无数人痛苦的抢掠。

征服的荣光背后,是无数的悲泣与哀嚎----本人语

会战

努尔哈赤是一位伟大的军事家,至少我是这样认为。

作为一名没有进过私塾,没有上过军校,没有受过系统军事训练的游牧民族首领,努尔哈赤懂得什么是战争,也懂
得如何赢得战争。他的战役指挥水平,已经达到了炉火纯青的地步。

在抚顺、清河以及之后一系列战役中,他表现出了惊人的军事天赋,无论是判断对方动向,选择战机、还是玩阴耍
诈,都可谓是无懈可击。

毫无疑问,他是这个时代最杰出的军事将领----在那两个人尚未出现之前。

\section[\thesection]{}

但对明朝而言,这位十分优秀的军事家,只是一名十分恶劣的强盗。不仅恶劣,而且残忍。

清河、抚顺战役结束后,抢够杀完的努尔哈赤非但没有歉意,不打收条,还做了一件极其无耻的事情。

他挑选了三百名当地平民,在抚顺关前,杀死了二百九十九人,只留下了一个。

他割下了这个人的耳朵,并让他带回一封信,以说明自己无端杀戮的理由:

``如果认为我做的不对,就约定时间作战!如果认为我做得对,你就送金银布帛吧,可以息事宁人!''

绑匪见得多了,但先撕票再勒索的绑匪,倒还真是第一次见。

明朝不是南宋,没有送礼的习惯。他们的方针,向来是不向劫匪妥协,何况是撕了肉票的劫匪。既然要打,那咱就
打真格的。

万历四十七年(1619)三月,经过长时间的准备,明军集结完毕,向赫图阿拉发起进攻。

明军共分东、西、南、北四路,由四位总兵率领,统帅及进攻路线如下:

东路指挥刘綎,自朝鲜进攻。

西路指挥官杜松,自抚顺进攻。

北路指挥官马林,自开原进攻。

南路指挥官李如柏,自清河进攻。

进攻的目标只有一个,赫图阿拉。

以上四路明军,共计十二万人,系由各地抽调而来,而这四位指挥官,也都大有来头。

李如柏的身份最高,他是李成梁的儿子,李如松的弟弟,但水平最低,你要说他不会打仗,比较冤枉,你要说他很
会打仗,比较扯淡。

马林的父亲,是马芳,这个人之前没提过,但很厉害,厉害到他的儿子马林,本来是个文人,都当上了总兵。至于
马先生的作战水平,相信你已经清楚。

这两路的基本情况如此,就指挥官来看,实在没什么戏。

但另外两路,就完全不同了。

东路指挥官刘綎,也是老熟人了。使六十多斤的大刀,还``轮转如飞'',先打日本,后扫西南,``万历三大征''打
了两大征,让他指挥东路,可谓志在必得。

但四路军中,最大的主力却并不是东路,最猛的将领也并不是刘綎。这两大殊荣,都属于西路军,以及它的指挥
官,杜松。

杜松,陕西榆林人,原任陕西参将,外号杜太师。

\section[\thesection]{}

前面提过,太师是朝廷的正一品职称,拿到这个头衔的,很少很少,除了张居正外,其他获得者一般都是死人、追
认。

但杜将军得到的这个头衔,确确实实是别人封的,只不过……不是朝廷。

他在镇守边界的时候,经常主动出击蒙古,极其生猛,前后共计百余战,无一败绩。蒙古人被他打怕了,求饶又没
用,听说明朝官员中太师最大,所以就叫他太师。

而杜将军不但勇猛过人,长相也过人,因为他常年冲锋肉搏,所以身上脸上到处都是伤疤,面目极其狰狞,据说让
人看着就不住地打哆嗦。

但这位刘綎都甘拜下风的猛人,这次前来上任,居然是带着镣铐来的,因为在不久之前,他刚犯了错误。

杜松虽然很猛,却有个毛病:小心眼。

所谓小心眼,一般是生气跟别人过不去,可是让人哭笑不得的是,杜松先生小心眼,总是跟自己过不去。

比如之前,他曾经跟人吵架,以武将的脾气,大不了一气之下动家伙砍人,可是杜兄一气之下,竟然出家当和尚了。

这实在是个奇怪的事,让人怎么都想不明白,可还没等别人想明白,杜松就想明白了,于是又还俗,继续干他的杀
人事业。

后来他升了官,到辽东当上了总兵,可是官升了,脾气一点没改,上阵打仗吃了亏(不算败仗),换了别人,无非写
了检讨,下次再来。

可这位兄弟不知那根筋不对,竟然要自杀,好歹被人拦住还是不消停,一把火把军需库给烧了,论罪被赶回了家,
这一次是重返故里。

虽说过了这么多年,经历了这么多事,但他的同事们惊奇地发现,这人一点没改,刚到沈阳(明军总营)报到,就开
始咋呼:

``我这次来,就是活捉努尔哈赤的,你们谁都别跟我抢!''

又不是什么好事,谁跟你抢?

事实也证明,这个光荣任务,没人跟他抢,连刘綎都不敢,于是最精锐的西路军,就成为了他的部属。

以上四路明军,共计十二万人,大致情况也就是这样,大明人多,林子太大,什么人都有,什么鸟都飞,混人、文
人、猛人,一应俱全。

说漏了,还有个鸟人----辽东经略杨镐。

杨镐,是一个出过场的人,说实话,我不太想让这人再出来,但可惜的是,我不是导演,没有换演员的权力。

作为一个无奈的旁观者,看着它的开幕和结束,除了叹息,只有叹息。

(长篇)明朝那些事儿-历史应该可以写得好看$[$1389$]$

参战明军由全国七省及朝鲜、叶赫部组成,并抽调得力将领指挥。全军共十二万人,号称四十七万,这是自土木堡
之变以来,明朝最大规模的军事行动。

要成事,需要十二万人,但要坏事,一个人就够了。

从这个角度讲,杨镐应该算是个很有成就的人。

自从朝鲜战败后,杨镐很是消停了一阵。但这个人虽不会搞军事,却会搞关系,加上他本人还比较老实,二十年
后,又当上了兵部左侍郎兼都察院右都御史。此外,他还加入了组织----浙党。

当时的朝廷首辅,是浙党的铁杆方从哲,浙党的首辅,自然要用浙党的将领,于是这个光荣的任务,就落在了杨镐
的身上。

虽然后来许多东林党拿杨镐说事,攻击方从哲,但公正地讲,在这件事上,方先生也是个冤大头。

我查了一下,杨镐兄的出生年月日不详,但他是万历八年(1580)的进士,考虑到他的智商和表现,二十岁之前考中
的可能性实在很小,三十而立、四十不惑都是有可能的。

如此算来,万历四十七年(1619)的时候,杨大爷至少也有六十多了。在当时的武将中,资历老、打过仗的,估计也
就他了。

方首辅没有选择的余地。

所以,这场战争的结局,也没有选择的余地。

万历四十七年(1619)二月二十一日,杨镐坐镇沈阳,宣布出兵。

下令后不久,回报:

今天下大雨,走不了。

走不了,那就休息吧。

这一休息就是四天,二月二十五日,杨镐说,今天出兵。

下令后不久,又回报:

辽东地区降雪,行军道路泥泞,请求延后。

几十年来,杨镐先生虽说打仗是不太行,做人倒还行,很少跟人红脸,对于合理化建议,他也比较接受,既然下大
雨延期他能接受,下大雪延期,似乎也没什么问题。

在这个世界上,好人不怕,坏人也不怕,就怕时好时坏、无端抽风的人。

杨镐偏偏就是个抽风的人,不知是那根筋有问题,突然发火了:

``国家养士,只为今日,若临机推阻,军法从事!''

完事还把尚方宝剑挂在门外,那意思是,谁敢再说话,来一个干一个。

窝囊了几十年,突然雄起,也算可喜可贺。

\section[\thesection]{}

然而接下来发生的一幕,就让杨先生雄不起来了。

按照惯例,出师之前,要搞个仪式,一般是找个叛徒、汉奸类的人物杀掉祭旗,然后再杀几头牲口祭天。

祭旗的时候,找了抚顺的一个逃兵,一刀下去,干掉了,可祭天的时候,却出了大问题。

事实证明,有时候,宰牲口比宰人要难得多,祭天的这头牛,不知是神牛下凡,还是杀牛刀太糙,反正是用刀捅、
用脚揣,折腾了好几次,才把这牛干掉。

封建社会,自然要搞点封建迷信,祭天的时候出了这事,大家都议论纷纷,然而杨镐先生却突然超越了时代,表现
出了不信鬼神的大无畏精神。他坚定地下达了命令:

出征!

然后,他就干了件蠢事,一件蠢得让人毛骨悚然的事。

在出征之前,杨镐将自己的出征时间、出征地点、进攻方向写成一封信,并托人送了出去,还反复叮嘱,必定要保
证送到。

收信人的名字,叫努尔哈赤。

对于他的这一举动,许多后人都难以理解,还有人认为,他有汉奸的嫌疑。

但我认为,以杨镐的智商,做出这样的事情,实在是不奇怪的。

在杨镐看来,自己手中有十二万大军,努尔哈赤下属的全部兵力,也只有六万,手下的杜松、刘綎,身经百战,经
验丰富,要对付山沟里的这帮游击队,毫无问题。

基于这种认识,杨镐认为,作为天朝大军,写这封信,是很有必要的。

在成功干掉一头牛,以及写信示威之后,四路大军正式出征,史称``萨尔浒之战'',就此拉开序幕。

但在序幕拉开之前,战役的结局,实际上已经注定。

因为几百年来几乎所有的人,都忽略了一个基本的问题:单凭这支明军,是无法消灭努尔哈赤的。

努尔哈赤的军队,虽然只有六万人,却身经百战,极其精锐,且以骑兵为主,明军就不同了,十二万人,来自五湖
四海,那真叫一个东拼西凑,除杜松、刘綎部外,战斗力相当不靠谱。

以指挥水平而论,就更没法说了,要知道,这努尔哈赤先生并不是山寨的土匪,当年跟着李成梁混饭吃,那是见过
大世面的,加上这位仁兄天赋异禀,极具军事才能,如果李如松还活着,估计还有一拼,以杜松、刘綎的能力,是
顶不住的。

实力,这才是失败的真相。

\section[\thesection]{}

杨镐的错误,并不是他干了什么,而是他什么也没干。

其实从他接手的那天起,失败就已注定。因为以当时明军的实力,要打赢是不容易的,加上他老人家,那就变成不
可能了。

可惜这位大爷对此毫无意识,还把军队分成了四部。

在这四支部队中,他把最精锐的六万余人交给了杜松,由其担任先锋。其余三部各两万人,围攻努尔哈赤。

这个想法,在理论上是很合理的,但在实践中,是很荒谬的。

按照杨镐的想法,仗是这么打的:努尔哈赤要呆在赫图阿拉,不许随便乱动,等到明朝四路大军压境,光荣会师,
战场上十二万对六万,(最好分配成两个对一个),也不要骑马,只能步战,然后决一死战,得胜回朝。

有这种脑子的人,只配去撞墙。

要知道,努尔哈赤先生的日常工作是游击队长,抢了就分,打了就跑,也从来不修碉堡炮楼,严防死守。

这就意味着,如果努尔哈赤集中兵力,杜松将不具备任何优势,再加上杜将军的脑筋向来缺根弦,和努尔哈赤这种
老狐狸演对手戏,必败无疑。。

而当努尔哈赤听到明军四路进军的消息后,只说了一句话:

``凭尔几路来,我只一路去。''

我仿佛看见,一出悲剧正上演,剧中没有喜悦。

二月二十八日,明军先锋杜松抵达抚顺近郊。

为了抢头功,他命令士兵日夜不停行军,但由于路上遭遇女真部队阻击,辎重落后,三月一日,他终于停下了脚
步,就地扎营。

他扎营的地点,叫做萨尔浒。

死战

此时的杜松,已经有点明白了,自他出征以来,大仗没有,小仗没完,今天放火明天偷袭,后勤也被切断,只能扎
营固守。

多年的战争经验告诉他,敌人就在眼前,随时可能发动进攻,情况非常不利,部下建议,应撤离此地。

但他并未撤退,却将手下六万人分为两部,分别驻守于吉林崖和萨尔浒。

杜松并未轻敌,事实上,他早已判定,隐藏在自己附近的,是女真军队的主力,且人数至少在两万以上。

以自己目前的兵力,攻击是不可能的,但防守还是不成问题的,所以没有撤退的必要。

应该说,他的判断是准确的,只有一点不同----埋伏在这里的,并不是女真部队的主力,而是全部。

\section[\thesection]{}

四路大军出发的时候,努尔哈赤已经明确,真正的主力,是杜松的西路军。所以他即刻动员全部兵力,向抚顺前
进,寻求决战。

当然,在决战之前,他还要玩点老把戏,摸哨、夜袭、偷粮食之类的活没少干,等到杜松不堪骚扰,在萨尔浒扎营
的时候,他已然是胜券在握。

接下来发生的一切,已无悬念。

三月二日,努尔哈赤发动八旗中的六旗,共计四万余人,猛攻明军萨尔浒大营,明军寡不敌众,全军覆没。

站在吉林崖大营的杜松,亲眼看到了萨尔浒的覆灭,他一言不发,穿上了自己的盔甲,集合了剩余的士兵,准备迎
接最后的战斗。

努尔哈赤再次发动了进攻,这一次,他带齐了八旗的全部兵力,向吉林崖发动了总攻。

面对绝对优势的敌人,杜松毫无畏惧,他率领明军拼死作战,激战直至夜晚,重创敌军。

然而实力就是实力,勇猛无畏的杜松终究还是战死了,和他一起阵亡的,还有上万名宁死不屈的士兵。

西路军就此全军覆没。

其实无论是决策错误,还是指挥错误,都已经不重要了,作为一名勇敢的将军,杜松已经尽到了自己的职责。

因为,他是战死的。

最先知道西路军覆没消息的,是马林。

此时他的位置,距离萨尔浒只有几十里。

作为一个文人,马林没有实践经验,但再没经验,也知道大祸就要临头。

关键时刻,马林体现出了惊人的理论天赋,他将所部两万余人分为三部,互相呼应,并且挖掘壕沟,加强防御,等
待着努尔哈赤的攻击。

无论从哪个角度讲,作为第一次上战场的将军,有如此表现,就算不错了。

可是不错是不够的。

一天之后,努尔哈赤发动了攻击。事实证明,马林的部署给他造成了相当大的麻烦,六万多人打了半天,一点进展
都没有,努尔哈赤没有办法,竟然带了一千亲兵上阵冲锋,才打开突破口。

但马林同志的表现也就到此为止了,毕竟他面对的,是三倍于他的敌人。而作为文人,他的观念也有点问题,最后
关头抛下了两个弟弟,自己先跑了。

北路马林军就此覆没。

西路军完了,北路军也完了,这个消息很快就传遍了辽东。

但东路的刘綎却对此毫不知情,因为他连路都没找到。

\section[\thesection]{}

刘綎的运气相当不好(或者说是相当好),由于他的行军道路比较偏,走后不久就迷了路,敌人没找着他,当然,他
也没找到敌人。

但这种摸黑的游戏没能持续多久。努尔哈赤已经擦掉了刀上的血迹,开始专心寻找刘綎。

三月初四,他找到了。

此时,刘綎的兵力只有一万余人,是努尔哈赤的四分之一。胜负未战已分。

然而还在山谷中转悠的刘綎并没有听到震耳的冲杀声,却等来了一个使者,杜松的使者。

使者的目的只有一个:传达杜松的命令,希望刘綎去与他会合。

此时,杜松已经死去,所以这个使者,是努尔哈赤派人假冒的。

但是刘綎并没有上当,他当即回绝了使者的要求。

不过他回绝的理由,确实有点搞笑:

``我是总兵,杜松也是总兵,他凭什么命令我!''

这下连假使者也急了,连说带比划,讲了一堆好话,刘綎才最终同意,前去与杜松会师。

然后,他依据指引,来到了一个叫阿布达里岗的地方,这里距离赫图阿拉只有几十里。

在这里,他看见了杜松的旗帜和军队。

但当这支军队冲入队列,发动攻击时,他才知道自己上当了。

寡不敌众、深陷重围,必败无疑,必死无疑。

但刘綎仍然镇定地拔出了刀,开始奋战。

之后的一切,史书上是这样介绍的:

阵乱,綎中流矢,伤左臂,又战

复伤右臂、犹鏖战不已,

内外断绝,面中一刀,截去半颊,犹左右冲突

手歼数十人而死。

用今天的话说,大致是这样:

阵乱了,刘綎中箭,左臂负伤,继续作战。

在战斗中,他的右臂也负伤了,依然继续奋战。

身陷重围无援,他的脸被刀砍掉了一半,依然继续奋战,左冲右杀。

最后,他杀死了数十人,战死。

这就是一个身陷绝境的将领的最后记录。

这是一段毫无感情,也无对话的文字,但在冷酷的文字背后,我听了刘綎最后的遗言和呼喊:

宁战而死,绝不投降!

\section[\thesection]{}

刘綎战死,东路军覆灭

现在,只剩下南路军了。

南路军的指挥官,是李如柏。

因为他的部队速度太慢,走了几天,才到达预定地点,此时其他三路军已经全军覆没。

于是在坐等一天之后,他终于率领南路军光荣回朝,除因跑得过快,自相践踏死了点人外,毫发无伤。

就军事才能而言,他是四人之中最差的一个,但他的运气却实在很好,竟然能够全身而退。

或许这一切,并不是因为运气。

因为许多人都依稀记得,他是李成梁的儿子,而且他还曾经娶过一个女子,可这位女子偏偏就是努尔哈赤的弟弟,
舒尔哈齐的女儿。

无论是运气太好还是太早知道,反正他是回来了。

但在战争,尤其是败仗中,活下来的人是可耻的,李如柏终究还是付出了代价。

回来后,他受到了言官的一致弹劾,而对于这样一个独自逃跑的人,所有人的态度都是一致的--鄙视。

偷生的李如柏终于受不了了,在这种生不如死的环境中,他选择了自尽,结束自己的生命。

萨尔浒大战就此结束,此战明军大败,死伤将领共计三百一十余人,士兵死伤四万五千八百七十余人,财物损失不
计其数。

消息传回京城,万历震怒了。

我说过,万历先生不是不管事,是不管小事,打了这么个烂仗,实在太过窝囊。

觉得窝囊了,自然要找人算帐,几路总兵都死光了,自然要找杨镐。

杨镐倒是相当镇定,毕竟他的关系搞得好,自他回来后,言官弹劾不绝于耳,但有老上级兼老同党方从哲保着,他
也不怎么慌。

可这事实在是太大了,皇帝下旨追查,言官拼命追打,特别是一个叫杨鹤的御史,三天两头上书,摆明了是玩命的
架势,那边努尔哈赤还相当配合,又攻陷了铁岭,几棍子抡下来,实在是扛不住了

不久后,他被逮捕,投入诏狱,经审讯判处死刑,数年后被斩首。

责任追究完了,但就在追究责任的时候,努尔哈赤也没歇着,还乘势攻下了全国比较大的城市--铁岭。

至此,辽东北部全部被努尔哈赤占领,明朝在辽东的根据地,只剩下了沈阳和辽阳。

看上去,局势十分危急,但事实上,是万分危急。

萨尔浒之战后,明军陷入了彻底的混乱,许多地方不见敌人,听到风声就跑,老百姓跑,当兵的也跑,个别缺德的
骑兵为了不打仗,竟然主动把马饿死。

而由于指挥系统被彻底打乱,朝廷的军饷几个月都无法发放,粮食也没有,对努尔哈赤而言,此地已经唾手可得。

但他终究没有得到,因为接替杨镐的人已经到任。他的名字,叫做熊廷弼。

\section[\thesection]{}

熊廷弼,是个不讨人喜欢的家伙。

熊廷弼,字飞白,江夏(今湖北武汉)人,自小聪明好学,乡试考中第一,三十岁就成为进士,当上了御史。

可此人脾气太坏,坏到见谁和谁过不去,坏到当了二十年的御史都没升官。

他还有个嗜好----骂人,且骂得很难听,后来连他都察院的同事都受不了,压根不搭理他,基本算是人见人厌。

但如果没有这个人见人厌的家伙,相信明朝差不多就可以收摊,下场休息去了。

万历四十七年(1619),萨尔浒大战后,在一片混乱之中,新任经略熊廷弼带着几个随从,进入了辽东。

他从京城出发的时候,开原还没有失陷,但当他到达辽东的时候,连铁岭都丢掉了。

等他到达辽阳的时候,才发现,明朝仅存的沈阳和辽阳,已几乎是一座空城。

他命令下属前往沈阳,稳定局势,叫来一个,竟然吓得直哭,打死都不敢去,再换一个,刚刚走出城,就跑回来
了,说打死也不敢再走。

于是熊廷弼说:

``我自己去。''

他从辽阳出发,一路走一路看,遇到逃跑的百姓,就劝他们回去,遇到逃跑的士兵,就收编他们,遇到逃跑的将
领,就抓起来。

就这样,到沈阳的时候,他已经集结了上万平民,数千名士兵,还有王捷、王文鼎等几位逃将。

安置了平民,整顿了士兵,就让人把逃将拉出去,杀头。

逃将求饶,说我们逃出来已经不容易了,何必要杀我们。

熊廷弼说:如果不杀你们,怎么对得起那些没有逃跑的人?

然后,他去见了李如桢。

李如桢是铁岭的守将,但后金军队进攻的时候,他却一直呆在沈阳。

不但一直呆在沈阳,铁岭被敌军攻击的时候,他连救兵都不派,坐视铁岭失守,让人十分费解,不知是反应迟钝,
还是另有密谋。

熊廷弼倒不打算研究这个问题,他只是找来这位仁兄,告诉他:你给我滚。

李如桢当时还是总兵,不是说免就能免的,可熊廷弼实在太过凶恶,李总兵当即就滚了,回去后又挨了熊廷弼的弹
劾,最后被关入监狱,判处死刑(后改充军)。

至此,一代名将李成梁的光荣世家彻底完结,除李如松外,都没啥好下场,连老家铁岭都被当年手下的小喽罗努尔
哈赤占据,可谓是干干净净、彻彻底底。

在当年的史料记载中,李成梁的事迹可谓数不胜数,和他同时期的戚继光,几乎完全被他的光芒所掩盖。

但几百年后,戚继光依然光耀史册,万人景仰,而李成梁,却几乎已不为人知。

我知道,历史只会夸耀那些值得夸耀的人。

当所有人都认为,熊廷弼的行动已告一段落时,他却又说了一句话:

``我要去抚顺。''

大家认为熊廷弼疯了。

当时的抚顺,已经落入努尔哈赤的手中,以目前的形势,带几个人去抚顺,无疑就是送死。

但熊廷弼说,努尔哈赤认定我不敢去,所以我现在去,反而是最安全的。

说是这么说,但敢不敢去,那是另外一码事。

熊廷弼去了,大家战战兢兢,他却毫不惊慌,优哉游哉地转了一圈。

\section[\thesection]{}

当所有人都胆战心惊的时候,他又下了个让人抓狂的命令:吹号角。

随行人员快要疯了,这就好比是孤身闯进山贼的山寨,再大喊抓贼,偷偷摸摸地来,你还大声喧哗,万一人家真的
冲出来,你怎么办?

但命令是必须执行的,人来了,号角吹了,后金军却一动不动。熊廷弼大摇大摆回了家。

几天后,努尔哈赤得知了事情的真相,非但不恼火发动进攻,反而派人堵住了抚顺进出的关口,严令死守,不得随
意出击。

努尔哈赤之所以表现如此低调,只是因为他和头号汉奸李永芳的一次对话。

当熊廷弼到来的消息传到后金时,李永芳急忙跑去找努尔哈赤,告诉他,这是个猛人。

努尔哈赤不以为然:辽东已经到了这个地步,这蛮子(后金对明朝将领的通称)就是再厉害,也只有一个人,如何挽
回危局?

李永芳回答:只要有他,就能挽回危局!

此后发生的一切,都证明了李永芳的判断,只用了短短几个月,熊廷弼就稳定了局势,此后他一反常态,除了防御
外,还组织了许多游击队,到后金占领地区进行骚扰,搞得对方疲于奔命,势头非常凶猛。

于是,努尔哈赤决定,暂时停止对明朝的进攻,休养生息,等待时机。

这个时机的期限,只有一年。

然而正是这关键的一年挽救了明朝。因为此时的朝廷,即将发生几件惊天动地的大事。

\section[\thesection]{}

在很多的史书中,万历中后期的历史基本上是这个样子:皇帝老休息,朝政无人管,大臣无事干。

前两头或许是正确的,但第三条是绝对不正确的。

隐藏在平静外表下的,是无比激烈的斗争。而斗争的主角,是东林党。

在许多人的印象中,东林是道德与正义的象征,一群胸怀理想的知识分子,为了同一个目标,走到一起来了。他们
怀揣着抱负参与政治,并曾一度掌控政权,却因为被邪恶的势力坑害,最终失败。

我认为,这是一个比较客观的说法。但是,很多人都忽略了一个问题,一个很有趣的问题:

一群只会读书的书呆子、知识分子,是如何掌控政权的呢?

正义和道德是值得景仰的,值得膜拜的,值得三拜九叩的,但是,正义和道德不能当饭吃,不能当衣服穿,更不可
能掌控政权。

因为掌控政权的唯一方式,就是斗争。

东林党的实力

道德文章固然有趣,却是无法解决问题的。

最先认识到这一点的人,应该是顾宪成。

在万历二十一年(1593)的那次京察中,吏部尚书孙鑨--撤职了,考功司郎中赵南星--回家了,首辅王锡爵--辞职
了,而这事幕后的始作俑者,从五品的小官,考功司员外郎顾宪成--升官了(吏部文选司郎中)。

升官了还不说,连他的上级,继任吏部尚书陈有年,也都是他老人家安排的,甚至后来回无锡当老百姓,他依然对
朝廷动向了如指掌。李三才偷看信件,王锡爵打道回府,朝廷的历任首辅,在他眼中不是木偶,就是婴儿。

这是一团迷雾,迷雾中的一切,似乎和他有关系,又似乎没有关系

拨开这团迷雾之后,我看到了一样东西----实力。

顾宪成的实力,来自于他的官职。

在吏部中,最大的是尚书(部长)、其次是侍郎(副部长),再往下就是四个司的郎中(司长),分别是文选司、验封司、
稽勋司、考功司。

但是,这四个司的地位是不同的,而其中最厉害的,是文选司和考功司,文选司负责人事任免,考功负责官员考
核,这两个司的官员向来无人敢惹,升官还是免职,发达还是破产,那就是一句话的事。

相对而言,验封司、稽勋司就一般了,一般到不用再介绍。

\section[\thesection]{}

有鉴于此,明代的吏部尚书和侍郎,大都由文选司和考功司的郎中接任。

而顾宪成先生的升迁顺序是:吏部考功司主事----考功司员外郎(副职)----文选司郎中。

这就意味着,那几年中,大明的所有官员(除少数高官),无论是升迁,还是考核,都要从顾宪成手底下过,即使不
过,也要打个招呼,就不打招呼,也得混个脸熟。

此外,我们有理由相信,顾宪成大人也是比较会来事的,因为一个不开窍的书呆子,是混不了多久的。

现在你应该明白了。

在这个世界上,实力和道德,经常是两码事。

东林之中,类似者还有很多,比如李三才。

李三才先生的职务,之前已经说过,是都察院佥都御史,巡抚凤阳,兼漕运总督。

都察院佥都御史多了去了,凤阳是个穷地方,不巡也罢,真正关键的职务,是最后那个。

自古以来,漕运就是经济运转的主要途径,基本算是坐地收钱,肥得没边,普天之下,唯一可以与之相比的,只有
盐政。

坐在这个位置上,要想不捞外快,一靠监督,二靠自觉。

很可惜,李三才不自觉,从种种史料分析,他很有钱,有钱得没个谱,请客吃饭,都是大手笔。

至于监督,那就更不用说了,这位李先生本人就是都察院的御史,自己去检举自己,估计他还没这个觉悟。

作为东林党的重量级人物,李三才在这方面的名声,那真是相当的大,大到几十年后,著名学者夏允彝到凤阳寻
访,还能听到相关事迹,最后还叹息一声,给了个结论----负才而守不洁。

列举以上两人,只是为了说明一点:

东林,是书院,但不仅仅是书院,是道德,但不仅仅是道德。它是一个有实力,有能力,有影响力、有斗争意识的
政治组织。

事实上,它的能量远远超出你的想象。

明白了这一点,你就会发现,那段看似平淡无奇的历史,每一分、每一秒,都是你死我活的争斗。

争斗的方式,是京察。

万历二十一年(1593),顾宪成失望地回家了,他虽费劲气力,却终究未能解决对手,京察失败。

但这一切,仅仅是个开始。

十二年后(万历三十三年),京察开始,主持者杨时乔,他的公开身份,是吏部左侍郎,他的另一个公开身份,是东
林党。

\section[\thesection]{}

当时的首辅,是浙党首领沈一贯,对于这位东林党下属,自然很不待见,于是,他决定换人。

沈一贯是朝廷首辅,杨时乔只是吏部二把手,然而意外发生了,虽然沈大人上窜小跳,连皇帝的工作都做了,却依
然毫无用处。杨侍郎该怎么来,还怎么来,几板斧抡下来,浙党、齐党、楚党、宣党……反正非东林党的,统统下
课,沈一贯拼了老命,才算保住几个亲信。

那么现在,请你再看一遍之前列举过的几条史料,玄机就在其中:

万历三十三年(1605),京察,沈一贯亲信以及三党干将被逐。

万历三十五年(1607),沈一贯退休回家。

同年,王锡爵的密信被李三才揭发,复出无望。

一年后,东林派叶向高成为首辅,开始执掌朝廷大权。

是的,这一切的一切,不是偶然。

而最终要获得的,正是权力。

权力已经在握,但还需要更进一步。

万历三十九年(1611),辛亥京察,主持人吏部尚书孙丕杨,东林党。

此时的首辅已经是叶向高了,东林党人遍布朝廷,对于那些非我族群而言,清理回家之类的待遇估计是免不了了。

然而一个人的掺和,彻底改变了这一切。这个人就是李三才。

此时的李三才已经升到了户部尚书,作为东林党的干将,他将进入内阁,更进一步。

算盘大致如此,可打起来就不是那么回事了。

听说李三才要入阁,朝廷顿时一片鸡飞狗跳,闹翻了天,主要原因在于李先生的底子不算干净,许多人对他有意见。

而更重要的是,这人实在太猛,太有能力。东林党已经如此强大,如果再让他入阁,三党的人估计就只能集体歇业
了。

于是,一场空前猛烈的反击开始。

明代的京察,按照地域,分为南察和北察,北察由尚书孙丕杨负责,而南察的主管者,是吏部侍郎史继楷,三党成
员,他选定的考察对象都是同一个类型----支持李三才的人。

很快,浙、楚、齐三党轮番上阵,对李三才发起了最后的攻击,他们的动机十分明确,明确到《明神宗实录》都写
了出来----``攻淮(李三才)则东林必救,可布一网打尽之局。

在集中火力打击之下,李三才没能顶住,回家养老去了。

(长篇)明朝那些事儿-历史应该可以写得好看$[$1400$]$

但就整体而言,此时的东林党依然占据着优势,叶向高执政,东林党掌权,非常强大,强大得似乎不可动摇。

然而就在此时,强大的东林党,犯了一个致命的错误。

一直以来,东林党的指导思想,是我很道德。强大之后,就变成了你不道德,工作方针,原先是党同伐异,强大之
后,就变成了非我族类,其心必异。

总而言之,不是我的同党,就是我的敌人。

这种只搞单边主义的混账做法,最终导致了一个混账的结果:

在东林党人的不懈努力下,齐、浙、楚三党终于抛弃了之前的成见,团结一致跟东林党死磕了。

他们的折腾,得到了立竿见影的回报:

万历四十二年(1614),叶向高退休回家。

万历四十五年(1617),京察开始,主持京察的,分别是吏部尚书郑继之、刑部尚书李志。

郑继之是楚党,李志是浙党。

有冤报冤,有仇报仇的时候到了,但凡是东林党,或者与东林党有关的人,二话不说,收包袱走人。这其中,还包
括那位揭发了梃击案真相的王之寀。

萨尔浒之战前,朝廷斗争情况大致如此,这场斗争的知名度相当小,但在历史上的地位相当重要。对明朝而言,其
重要程度,基本等于努尔哈赤+皇太极+李自成+张献忠。

因为这是一场延续了几十年的斗争,是一场决定明朝命运的斗争。

因为在不久之后,东林党将通过一个人的帮助,彻底击败浙、齐、楚三党。

然后,土崩瓦解的三党将在另一个人的指挥下,实现真正的融合,继续这场斗争,而那时,他们将有一个共同的名
字----阉党。

万历四十五年的京察,标志着东林党的没落,所谓东林党三大巨头,顾宪成已经死了,邹元标到处逛,赵南星家里
蹲。

两大干将也全部消停,叶向高提早退休,李三才回家养老。

此时的首辅,是浙党的方从哲,此时的朝廷,是三党的天下。对东林党而言,前途似乎一片黑暗。

但新生的机会终会到来,因为一个人的死去。

万历四十八年(1620)七月二十一日,万历不行了。

高拱、张居正、申时行、李成梁、东林党、朝鲜、倭寇、三大征、萨尔浒、资本主义萌芽、不上朝、太子、贵妃、
国本、打闷棍。

我只能说,他这辈子应该比较忙。

(长篇)明朝那些事儿-历史应该可以写得好看$[$1401$]$

关于这位兄弟的评论,我想了很久,很久,却是很久,很久,也想不出来。

你说他没干过好事吧,之前二十多年,似乎干得也不错,你说他软弱吧,他还搞了三大征,把日本鬼子赶回了老
家,你说他不理朝政吧,这几十年来哪件大事他不知道?

一个被张居正压迫过的人,一个勤于政务的人,一个被儿子问题纠缠了几十年的人,一个许多年不见大臣、不上班
的人,一个终生未出京城,生于深宫、死于深宫的人。

一个复杂得不能再复杂的人,一个简单得不能再简单的人。

于是,我最终懂得了这个人。

一个热血沸腾的青年,一个励精图治的君主,一个理想主义者,在经历残酷的斗争,无休止的吵闹,无数无效的抗
争,无数无奈的妥协后,最终理解了这个世界,理解了现实的真正意义,并最终成为了这个世界的牺牲品。

大致如此吧

明神宗朱翊钧,万历四十八年逝世,年五十八。

在这个残酷的世界面前,他还不够勇敢。

明光宗朱常洛

虽然几十年来,万历都不喜欢自己的长子朱常洛,但在生命的最后一刻,他终于做出了抉择,将皇位传给了这个久
经考验的儿子。

担惊受怕几十年的朱常洛终于熬出头了,万历四十八年(1620)八月一日,朱常洛正式登基,即后世所称之明光宗,
定年号为泰昌。

由于此时还是万历年间,按照惯例,要等老爹这一年过完,明年才能另起炉灶,用自己的年号。

可几乎所有的人都没有想到,这个年号,竟然没能用上。

因为朱常洛活了三十八年,明光宗却只能活一个月。

一个撑了三十八年,经历无数风雨险阻到达目标的人,却在一个月中意外死亡,是很不幸的。

导致死亡与不幸的罪魁祸首,是郑贵妃。

(长篇)明朝那些事儿-历史应该可以写得好看$[$1402$]$

红丸

应该说,朱常洛是个好孩子,至少比较厚道。

几十年来,他一直夹着尾巴做人,亲眼目睹了父亲的冷漠、朝廷的冷清,感受到了国家的凋敝,时局的危险。

他不愿意再忍受下去,于是,当政后的第一天,他用几道谕令显示了自己的决心。

大致说来,他是把他爹没办的事给办了,包括兑现白条----给辽东前线的士兵发工资,废除各地矿税,以及补充空
缺的官员。

这几件事情,办得很好,也很及时,特别是最后一条,把诸多被万历同志赶下岗的仁兄们拉了回来,实在是大快官
心,于是一时之间,光宗的人望到达了顶点,朝廷内外无不感恩戴德,兴高采烈。

但有一个人不高兴,非但不高兴,而且很害怕。

万历死后,郑贵妃终于明白,自己是多么的虚弱,今日之城内,已是敌人之天下。所谓贵妃,其实也不贵,如果明
光宗要对付她,贱卖的可能性是相当的大。

很快,一件事情就证明了她的判断。

考虑到万历死后不好办,之前郑贵妃软磨硬泡,让万历下了道遗嘱,讲明,一旦自己死后,郑贵妃必须进封皇后。

如此一来,等万历死后,她就成了太后,无论如何,铁饭碗是到手了。

明光宗看上去倒也老实,丝毫不赖帐,当即表示,如果父皇如此批示,那就照办吧。

但他同时表示,这是礼部的事,我批下去,让他们办吧。

按说皇帝批下来就没问题了,可是礼部侍郎孙如游不知怎么回事,非但不办,还写了个奏疏,从理论、辈分、名分
上论证了这件事,最后得出结论--不行。

光宗同志似乎也不生气,还把孙侍郎的奏疏压了下来,但封皇后这事再也没提。

郑贵妃明白了,这就是个托。

很明显,这位看上去很老实的人,实际上不怎么老实。既然如此,必须提前采取行动。

经过深思熟虑,她想出了一个计划,而这个计划的第一步,是一件礼物。

十天之后,她将这件礼物送给了朱常洛,朱常洛很高兴地收下了。

光宗皇帝的性命,就丢在了这份礼物上。

这份礼物,是八个美女。

对于常年在宫里坐牢,哪都不能去,啥也没有的朱常洛而言,这是一份丰厚的礼物,辛辛苦苦、畏畏缩缩了几十
年,终于可以放纵一下了。

古语有云:一口吃不成胖子,但朱常洛应该算是不同凡响,他几天就变成了瘦子,在史料上,含蓄的文言文是这样
描述的:

``是夜,连幸数人,圣容顿减。''

白天日理万机,晚上还要辛勤工作,身体吃不消,实在是件十分自然的事情。于是不久之后,朱常洛就病倒了。

这一天是万历四十八年(1620)八月十日。

计划的第二步即将开始,四天之后。

(长篇)明朝那些事儿-历史应该可以写得好看$[$1403$]$

万历四十八年(1620)八月十四日

皇帝的身体依然很差,身体差就该看医生,崔文升就此出了场。

崔文升,时任司礼监秉笔太监。前面曾经讲过,这是一个十分重要的职务,仅次于司礼监掌印太监。

可是这人来,并不是要给皇帝写遗嘱,而是看病,因为这位崔兄多才多能,除了能写外,还管着御药房,搞第二产
业。

后来的事情告诉我们,第二产业是不能随便乱搞的。

诊断之后,崔大夫胸有成竹,给病人开了一副药,并且乐观地表示,药到病除。

他开的这幅药,叫泻药。

一个夜晚辛勤工作,累垮了身体的人,怎么能服泻药呢?

所以后来很多史书都十分肯定地得出了结论:这是个``蒙古大夫''。

虽然我不在现场,也不懂医术,但我可以认定:崔文升的诊断,是正确的。

因为之前的史料中,有这样六个字:是夜,连幸数人。

这句话的意思大家应该知道,就不解释了,但大家也应该知道,要办到这件事情,难度是很大的。对光宗这种自幼
体弱的麻杆而言,基本就是个不可能的任务。

但是他完成了。

所以唯一的可能性是,他找了帮手,而这个帮手,就是药物。

是什么药物,大家心里也有数,我就不说了,这类药物在明代宫廷里,从来就是必备药,从明宪宗开始,到天天炼
丹的嘉靖,估计都没少用。明光宗初来乍到,用用还算正常。

可这位兄弟明显是用多了,加上身体一向不好,这才得了病。

在中医理论中,服用了这种药,是属于上火,所以用泻药清火,也还算对症下药。

应该说,崔文升是懂得医术的,可惜,是半桶水。

根据当时史料反映,这位仁兄下药的时候,有点用力过猛,手一哆嗦,下大了。

错误是明显的,后果是严重的,光宗同志服药之后,一晚上拉了几十次,原本身体就差,这下子更没戏了,第二天
就卧床不起,算彻底消停了。

蒙古的崔大夫看病经历大致如此,就这么看上去,似乎也就是个医疗事故。虽说没法私了,但毕竟大体上没错,也
没在人家身体里留把剪刀、手术刀之类的东西当纪念品,态度还算凑合。

可问题是,这事一冒出来,几乎所有的人都立刻断定,这是郑贵妃的阴谋。

(长篇)明朝那些事儿-历史应该可以写得好看$[$1404$]$

因为非常凑巧,这位下药的崔文升,当年曾经是郑贵妃的贴身太监。

这真是跳进黄河也洗不清,要看病,不找太医,偏找太监,找了个太监,偏偏又是郑贵妃的人,这太监下药,偏又
下猛了,说他没问题,实在有点困难。

对于这件事情,你说它不是郑贵妃的计划,我信,因为没准就这么巧;说它是郑贵妃的计划,我也信,因为虽说下
药这招十分拙劣,谁都知道是她干的,但以郑贵妃的智商,以及从前表现,这种蠢事,她是干得出来的。

无论动机如何,结果是肯定的,明光宗已经奄奄一息,一场惊天大变即将拉开序幕。

但这一切还不够,要达到目的,这些远远不够,即使那个人死去,也还是不够。

必须把控政权,把未来所有的一切,都牢牢抓在手中,才能确保自己的利益。

于是在开幕之前,郑贵妃找到了最后一个同盟者。

这位同盟者的名字,不太清楚。

目前可以肯定的是,她姓李,是太子的嫔妃。

当时太子的嫔妃有以下几种:大老婆叫太子妃,之后分别是才人、选侍、淑女等。

而这位姓李的女人,是选侍,所以在后来的史书中,她被称为``李选侍''。

李选侍应该是个美女,至少长得还不错,因为皇帝最喜欢她,而且皇帝的儿子,那个未来的天才木匠----朱由校,
也掌握在她的手中,正是因为这一点,郑贵妃找上了她。

就智商而言,李选侍还算不错(相对于郑贵妃),就人品而言,她和郑贵妃实在是相见恨晚,经过一番潜规则后,双
方达成协议,成为了同盟,为了不可告人的目的。

现在一切已经齐备,只等待着一个消息。

所有的行动,将在那一刻展开,所有的野心,将在那一刻实现。

小人物

目标就在眼前,一切都很顺利。

皇帝的身体越来越差,同党越来越多,帝国未来的继承人尽在掌握之中,在郑贵妃和李选侍看来,前方已是一片坦
途。

然而她们终究无法前进,因为一个微不足道的小人物。

明光宗即位后,最不高兴的是郑贵妃,最高兴的是东林党。

这是很正常的,从一开始,东林党就把筹码押在这位柔弱的太子身上,争国本、妖书案、梃击案,无论何时何地,
他们都坚定地站在这一边。

现在回报的时候终于到了。

(长篇)明朝那些事儿-历史应该可以写得好看$[$1405$]$

明光宗非常够意思,刚上任,就升了几个人的官,这些人包括刘一璟、韩旷、周嘉谟、邹元标、孙如游等等。

这几个人估计你不知道,其实也不用知道,只要你知道这几个人的职务,就能明白,这是一股多么强大的力量。

刘一璟、韩旷,是东阁大学士,内阁成员,周嘉谟是吏部尚书,邹元标是大理寺丞,孙如游是礼部侍郎。当然,他
们都是东林党。

在这群人中,有内阁大臣、人事部部长、法院院长,部级高官,然而,在后来那场你死我活的斗争中,他们只是配
角。真正力挽狂澜的人,是一个看似微不足道的小人物。

这个人的名字,叫做杨涟。

杨涟,字文孺,号大洪,湖广(湖北)应山人,万历三十五年(1607)进士,任常熟知县,后任户科给事中、兵科给事
中。

这是一份很普通的履历,因为这人非但当官晚,升得也不快,明光宗奄奄一息的时候,也才是个七品给事中。

但在这份普通履历的后面,是一个不普通的人。

上天总是不公平的,有些人天生就聪明,天生就牛,天生就是张居正、戚继光,而绝大多数平凡的人,天生就不聪
明,天生就不牛,天生就是二傻子,没有办法。

但上天依然是仁慈的,他给出了一条没有天赋,也能成功的道路。

对于大多数平凡的人而言,这是最好的道路,也是唯一的道路,它的名字,叫做纯粹。

纯粹的意思,就是专心致志、认真、一根筋、二杆子等等等等。

纯粹和执着,也是有区别的,所谓执着,就是不见棺材不掉泪,而纯粹,是见了棺材,也不掉泪。

纯粹的人,是这个世界上最可怕的人,他们的一生,往往只有一个目标,为了达到这个目标,他们可以不择手段,
不顾一切,他们无法被收买,无法被威逼,他们不要钱,不要女色,甚至不要权势和名声。

在他们的世界里,只有一个目标,以及坚定的决心和意志。

杨涟,就是一个纯粹的人。

(长篇)明朝那些事儿-历史应该可以写得好看$[$1406$]$

他幼年的事迹并不多,也没有什么砸水缸之类的壮举,但从小就为人光明磊落,还很讲干净,干净到当县令的时
候,廉政考核全国第一。此外,这位仁兄也是个不怕事的人,比如万历四十八年(1620),万历生病,半个月不吃
饭,杨涟听说了,也不跟上级打招呼,就跑去找首辅方从哲:

``皇上生病了,你应该去问安。''

方首辅胆子小,脾气也好,面对这位小人物,丝毫不敢怠慢:

``皇上一向忌讳这些问题,我只能去问宫里的内侍,也没消息。''

朝廷首辅对七品小官,面子是给足了,杨先生却不要这个面子,他先举了个例子,教育了首辅大人,又大声强调:

``你应该多去几次,事情自然就成了(自济)!''

末了,还给首辅大人下了个命令:

``这个时候,你应该住在内阁值班,不要到处走动!''

毫无惧色。

根据以上史料,以及他后来的表现,我们可以认定:在杨涟的心中,只有一个目标--为国尽忠,匡扶社稷。

事实上,在十几天前的那个夜晚,这位不起眼的小人物,就曾影响过这个帝国的命运。

万历四十八年(1620)七月二十一日,夜,乾清宫

万历就快撑不住了,在生命的最后时刻,他反省了自己一生的错误,却也犯下了一个十分严重的错误----没有召见
太子。

一般说来,皇帝死前,儿子应该在身边,除了看着老爹归西、嚎几声壮胆以外,还有一个重要意义----确认继位。

虽说太子的名分有了,但中国的事情一向难说,要不看着老爹走人,万一隔天突然冒出几份遗嘱、或是几个顾命大
臣,偏说老头子临死前改了主意,还找人搞了公证,这桩官司可怎么打?

但不知万历兄是忘了,还是故意的,反正没叫儿子进来。

太子偏偏是个老实孩子,明知老头子不行了,又怕人搞鬼,在宫殿外急得团团转,可就是不敢进去。

关键时刻,杨涟出现了。

在得知情况后,他当机立断,派人找到了一个极为重要的人物----王安。

王安,时任太子侍读太监,在明代的历史中,这是一个重量级人物。此后发生的一系列事件里,他都起着极为关键
的作用。

而在那个夜晚,杨涟只给王安带去了一句话,一句至关紧要的话:

``皇上已经病得很重了(疾甚),不召见太子,并不是他的本意。太子应该主动进宫问候(尝药视膳),等早上再回
去。''

这就是说,太子您之所以进宫,不是为了等你爹死,只是进去看看,早上再回去嘛。

对于这个说法,太子十分满意,马上就进了宫,问候父亲的病情。

当然,第二天早上,他没回去。

(长篇)明朝那些事儿-历史应该可以写得好看$[$1407$]$

朱常洛就此成为了皇帝,但杨涟并没有因此获得封赏,他依然是一个不起眼的给事中。不过,这对于杨先生而言,
实在是个无所谓的事。

他平静地回到暗处,继续注视着眼前的一切。他很清楚,真正的斗争刚刚开始。

事情正如他所料,蒙古崔大夫开了泻药,皇帝陛下拉得七荤八素,郑贵妃到处活动,李选侍经常串门。

当这一切被组合起来的时候,那个无比险恶的阴谋已然暴露无遗。

形势十分危急,不能再等待了。

杨涟决定采取行动,然而现实很残酷:他的朋友虽然多,却很弱小,他的敌人虽然少,却很强大。

周嘉谟、刘一璟、韩爌这拨人,级别固然很高,但毕竟刚上来,能量不大,而郑贵妃在宫里几十年,根基极深,一
手拉着李选侍,一手抓着皇长子,屁股还拼命往皇太后的位置上凑。

按照规定,她应该住进慈宁宫,可这女人脸皮相当厚,死赖在乾清宫不走,看样子是打算长住。

因为乾清宫是皇帝的寝宫,可以监视皇帝的一举一动,一旦光宗同志有啥三长两短,她必定是第一个采取行动的
人,那时,一切都将无可挽回。

而要阻止这一切,杨涟必须做到两件事情:首先,他要把郑贵妃赶出乾清宫;其次,他要把郑贵妃当太后的事情彻
底搅黄。

这就是说,先要逼郑老寡妇搬家,再把万历同志临死前封皇后的许诺当放屁,把郑贵妃翘首企盼的申请拿去垫桌脚。

杨涟先生的职务,是七品兵科给事中,不是皇帝。

事实上,连皇帝本人也办不了,光宗同志明明不喜欢郑贵妃,明明不想给她名分,也没法拍桌子让她滚。

这就是七品芝麻官杨涟的任务,一个绝对、绝对无法完成的任务。

但是他完成了,用一种匪夷所思的方式。

他的计划是,让郑贵妃自己搬出去,自己撤回当皇太后的申请。

这是一个看上去绝不可能的方案,却是唯一可能的方案。因为杨涟已经发现,眼前的这个庞然大物,有一个致命的
弱点,只要伸出手指,轻轻地点一下,就够了。

这个弱点有个名字,叫做郑养性。

郑养性,是郑贵妃哥哥郑国泰的儿子,郑国泰死后,他成为了郑贵妃在朝廷中的联系人,平日十分嚣张。

(长篇)明朝那些事儿-历史应该可以写得好看$[$1408$]$

然而杨涟决定,从这个人入手,因为经过细致的观察,他发现,这是一个外强中干,性格软弱的人。

万历四十八年(1620)八月十六日。杨涟直接找到了郑养性,和他一同前去的,还有周嘉谟等人。

一大帮子人上门,看架势很像逼宫,而事实上,确实是逼宫。

进门也不讲客套,周嘉谟开口就骂:

``你的姑母(指郑贵妃)把持后宫多年,之前争国本十几年,全都是因为她,现在竟然还要封皇太后,赖在乾清宫不
走,还给皇上奉送美女,到底有什么企图?!''

刚开始时,郑养性还不服气,偶尔回几句嘴,可这帮人都是职业选手,骂仗的业务十分精湛,说着说着,郑养性有
点扛不住了。

白脸唱完了,接下来是红脸:

``其实你的姑母应该也没别的意思,不过是想守个富贵,现在朝中的大臣都在这里,你要听我们的话,这事就包在
我们身上。''

红脸完了,又是唱白脸:

``要是不听我们的话,总想封太后,不会有人帮你,你总说没这想法,既然没这想法,就早避嫌疑!''

最狠的,是最后一句:

``如此下去,别说富贵,身家性命能否保得住,都未可知!''

郑养性彻底崩溃了。眼前的这些人,听到的这些话,已经打乱了他的思维。于是,他去找了郑贵妃。

其实就时局而言,郑贵妃依然占据着优势,她有同党,有帮手,如果赖着不走,谁也拿她没办法。什么富贵、性
命,这帮闹事的书呆子,也就能瞎嚷嚷几句而已。

然而关键时刻,郑贵妃不负白痴之名,再次显露她的蠢人本色,在慌乱的外甥面前,她也慌乱了。

经过权衡利弊,她终于做出了决定:搬出乾清宫,不再要求当皇太后。

至此,曾经叱诧风云的郑贵妃,正式退出了历史舞台,这位大妈费劲心机,折腾了三十多年,却啥也没折腾出来。
此后,她再也没能翻过身来。

这个看似无比强大的对手,就这样,被一个看似微不足道的人,轻而易举地解决了。

但在杨涟看来,这还不够,于是三天之后,他把目标对准了另一个人。

万历四十八年(1620)八月十九日,杨涟上书,痛斥皇帝。

(长篇)明朝那些事儿-历史应该可以写得好看$[$1409$]$

杨先生实在太纯粹,在他心中,江山社稷是第一位的,所以在他看来,郑大妈固然可恶,崔大夫固然可恨,但最该
谴责的,是皇帝。

明知美女不应该收,你还要收,明知春药不能多吃,你还要吃,明知有太医看病,你还要找太监,不是脑袋有病吧。

基于愤怒,他呈上了那封改变他命运的奏疏。

在这封奏疏里,他先谴责了蒙古大夫崔文升,说他啥也不懂就敢乱来,然后笔锋一转,对皇帝提出了尖锐的批
评----勤劳工作,不爱惜自己的身体。

必须说明的是,杨先生不是在拍马屁,他的态度是很认真的。

因为在文中,他先暗示皇帝大人忙的不是什么正经工作,然后痛骂崔文升,说他如何没有水平,不懂医术。最后再
转回来:就这么个人,但您还是吃他的药。这意思是说,崔大夫已经够没水平了,您比他还要差。

所以这奏疏刚送上去,内阁就放出话来,杨先生是没有好下场的。

三天后,这个预言得到了印证。

明光宗突然派人下令,召见几位大臣,这些人包括方从哲、周嘉谟、孙如游,当然,还有杨涟。此外,他还命令,
锦衣卫同时进宫,听候指示。

命令一下来,大家就认定,杨涟要完蛋了。

因为在这拨人里,方从哲是首辅,周嘉谟是吏部尚书,孙如游是礼部尚书,全都是部级干部,只有杨涟先生,是七
品给事中。

而且会见大臣的时候,召集锦衣卫,只有一种可能----收拾他。

由于之前的举动,杨涟知名度大增,大家钦佩他的人品,就去找方从哲,让他帮忙求个情。

方从哲倒也是个老好人,找到杨涟,告诉他,等会进宫的时候,你态度积极点,给皇上磕个头,认个错,这事就算
过去了。

但是杨涟的回答,差点没让他一口气背过去:

``死就死(死即死耳),我犯了什么错?!''

旁边的周嘉谟连忙打圆场:

``方先生(方从哲)是好意。''

可到杨先生这里,好意也不好使:

``知道是好意,怕我被人打死,要得了伤寒,几天不出汗,也就死了,死有什么可怕!但要我认错,绝无可能!''

就这样,杨涟雄赳赳气昂昂地进了宫,虽然他知道,前方等待着他的,将是锦衣卫的大棍。

可是他错了。

(长篇)明朝那些事儿-历史应该可以写得好看$[$1410$]$

那位躺在床上,病得奄奄一息的皇帝陛下非但没有发火,反而和颜悦色说了这样一句话

``国家的事情,全靠你们尽心为我分忧了。''

虽然称呼是复数,但他说这句话的时候,眼睛只看着杨涟。

这之后,他讲了许多事情,从儿子到老婆,再到郑贵妃,最后,他下达了两条命令:

一、 赶走崔文升。

二、 收回封郑贵妃为太后的谕令。

这意味着,皇帝陛下听从了杨涟的建议,毫无条件,毫无抱怨。

当然,对于他而言,这只是个顺理成章的安排。

但他绝不会想到,他这个无意间的举动,将对历史产生极重要的影响。

因为他并不知道,此时此刻,在他对面的那个人心中的想法。

从这一刻起,杨涟已下定了决心----以死相报。

一直以来,他都只是个小人物,虽然他很活跃,很有抱负,声望也很高,他终究只是小人物。

然而眼前的这个人,这个统治天下的皇帝,却毫无保留地尊重,并认可了自己的情感、抱负,以及纯粹。

所以他决定,以死相报,致死不休。

这种行为,不是愚忠,不是效命,甚至也不是报答。

它起源于一个无可争议,无可辩驳的真理:

士为知己者死。

这一天是万历四十八年(1620)八月二十二日,明光宗活在世上的时间,还有十天。

这是晚明历史上最神秘莫测的十天。一场更为狠毒的阴谋,即将上演。

八月二十三日

内阁大学士刘一璟、韩旷照常到内阁上班,在内阁里,他们遇见了一个人。

这个人的名字叫李可灼,时任鸿胪寺丞,他来这里的目的,是要进献``仙丹''。

此时首辅方从哲也在场,他对这玩意兴趣不大,毕竟皇帝刚吃错药,再乱来,这个黑锅就背不起了。

刘一璟和韩旷更是深恶痛绝,但也没怎么较真,直接把这人打发走了。

很明显,这是一件小事,而小事是不应该过多关注的。

但某些时候,这个理论是不可靠的。

两天后,八月二十五日

明光宗下旨,召见内阁大臣、六部尚书等朝廷重臣,此外,他特意叫上了杨涟。

对此,所有的人都很纳闷。

更让人纳闷的是,此后直至临终,他召开的每一次会议,都叫上杨涟,毫无理由,也毫无必要。或许是他的直觉告
诉他,这个叫杨涟的人,非常之重要。

他的直觉非常之准。

(长篇)明朝那些事儿-历史应该可以写得好看$[$1411$]$

此时的光宗,已经是奄奄一息,所以,几乎所有的大臣都认定,今天的会议,将要讨论的,是关乎国家社稷的重要
问题。

然而他们没有想到,这次内阁会议的议题,只有一个----老婆。

光宗同志的意思是,自己的后妃李选侍,现在只有一个女儿,伺候自己那么多年,太不容易,考虑给她升官,封皇
贵妃。

此外,他还把皇长子朱由校领了出来,告诉诸位大人,这孩子的母亲也没了,以后,就让李选侍照料他。

在场的所有人都目瞪口呆。

明明您都没几天蹦头了,趁着脑袋还管用,赶紧干点实事,拟份遗嘱,哪怕找口好棺材,总算有个准备。竟然还想
着老婆的名分,实在令人叹服。

在现场的人们看来,这是一个尊重妇女,至死不渝的模范丈夫

但是事实并非如此。

八月二十六日

出乎许多人的意料,明光宗再次下旨,召开内阁会议,与会人员包括内阁大臣及各部部长,当然还有杨涟。

会议与昨天一样,开得十分莫名其妙。这位皇帝陛下把人叫进来,竟然先拉一通家常,又把朱由校拉进来,说我儿
子年纪还小,你们要多照顾等等。

这么东拉西扯,足足扯了半个时辰(一个小时),皇上也扯累了,正当大家认为会议即将结束的时候,扯淡又开始了。

如昨天一样,光宗再次提出,要封李选侍为皇贵妃,大家这才明白,扯来扯去不就是这件事吗?

礼部尚书孙如游当即表示,如果您同意,那就办了吧(亦无不可)。

然而就在此时,一件令人震惊的事情发生了。

一个人突然闯了进来,公然打断了会议,并在皇帝、内阁、六部尚书的面前,拉走了皇长子朱由校。

这个人,就是李选侍。

所有人都懵了,没有人去阻拦,也没有人去制止。原因很简单,这位李选侍毕竟是皇帝的老婆,皇帝大人都不管,
谁去管。

而更让人难以置信的是,很快,他们就听见了严厉的斥责声,李选侍的斥责声,她斥责的,是皇帝的长子。

于是,一个空前绝后的场面出现了。

大明帝国未来的继承人,被一个女人公然拉走,当众责骂,而皇帝,首辅、各部尚书,全部毫无反应,放任这一切
的发生。

(长篇)明朝那些事儿-历史应该可以写得好看$[$1412$]$

所有的人静静地站在那里,听着那个女人的责骂,直到骂声结束为止。

然后,尚未成年的朱由校走了出来,他带着极不情愿的表情,走到了父亲的身边,说出了这样一句话:

``要封皇后!''

谜团就此解开,莫名其妙的会议,东拉西扯的交谈,终于有了一个明确的答案----胁迫。

开会是被胁迫的,闲扯是被胁迫的,一个奄奄一息的丈夫,一个年纪幼小的孩子,要不胁迫一把,实在有点说不过
去。

李选侍很有自信,因为她很清楚,这个软弱的丈夫不敢拒绝她的要求。

现在,她距离自己的皇后宝座,只差一步。

但是这一步,到死都没迈过去。

因为就在皇长子刚说出那四个字的时候,另一个声音随即响起:

``皇上要封皇贵妃,臣必定会尽快办理!''

说这句话的人,是礼部尚书孙如游。

李选侍太过天真了,和朝廷里这帮老油条比起来,她也就算个学龄前儿童。

孙尚书可谓聪明绝顶,一看情形不对,知道皇上顶不住了,果断出手,只用了一句话,就把皇后变成皇贵妃。

光宗同志也很机灵,马上连声回应:好,就这么办。

李小姐的皇后梦想就此断送,但她是不会放弃的,因为她很清楚,在自己的手中,还有一张王牌----皇长子。

只要那个奄奄一息的人彻底死去,一切都将尽在掌握。但她并不知道,此时,一双眼睛已经死死地盯住了她。

杨涟已经确定,眼前这个飞扬跋扈的女人,不久之后,将是一个十分可怕的敌人。而在此之前,必须做好准备。

八月二十九日

此前的三天里,光宗的身体丝毫不见好转,于是在这一天,他再次召见了首辅方从哲等朝廷重臣。

光宗同志这次很清醒,一上来就直奔主题:

寿木如何?寝地如何?

寿木就是棺材,寝地就是坟,这就算是交代后事了。

可是方从哲老先生不知是不是老了,有点犯糊涂,张口就是一大串,什么你爹的坟好、棺材好请你放心之类的话。

光宗同志估计也是哭笑不得,只好拿手指着自己,说了一句:

是我的(朕之寿宫)。

方首辅狼狈不堪,可还没等他缓过劲来,就听到了皇帝陛下的第二个问题:

``听说有个鸿胪寺的医官进献金丹,他在何处?''

明朝那些事儿-历史应该可以写得好看$[$1413$]$

对于这个问题,方从哲并未多想,便说出了自己的回答:

``这个人叫李可灼,他说自己有仙丹,我们没敢轻信。''

他实在应该多想想的。

因为金丹不等于仙丹,轻信不等于不信。

正是这个模棱两可的回答,导致了一个错误的判断:

``好吧,召他进来。''

于是,李可灼进入了大殿,他见到了皇帝,他为皇帝号脉,他为皇帝诊断,最后,他拿出了仙丹。

仙丹的名字,叫做红丸。

此时,是万历四十八年(1620)八月二十九日上午,明光宗服下了红丸。

他的感觉很好。

按照史书上的说法,吃了红丸后,浑身舒畅,且促进消化,增加食欲(思进饮膳)。

消息传来,宫外焦急等待的大臣们十分高兴,欢呼雀跃。

皇帝也很高兴,于是,几个时辰后,为巩固疗效,他再次服下了红丸。

下午,劳苦功高的李可灼离开了皇宫,在宫外,他遇见了等待在那里的内阁首辅方从哲。

方从哲对他说:

``你的药很有效,赏银五十两。''

李可灼高兴地走了,但他并没有领到这笔赏银。

方从哲以及当天参与会议的人都留下了,他们住在了内阁,因为他们相信,明天,身体好转的皇帝将再次召见他们。

六个时辰之后

凌晨,住在内阁的大臣们突然接到了太监传达的谕令:

即刻入宫觐见。

所有的人都明白,这意味着什么,但当他们尚未赶到的时候,就已得到了第二个消息----皇上驾崩了。

万历四十八年(1620)九月初一,明光宗在宫中逝世,享年三十九,享位一月。

皇帝死了,这十分正常,皇帝吃药,这也很正常,但吃药之后就死了,这就不正常了。

明宫三大案之``红丸案'',就此拉开序幕。

没有人知道,所谓的红丸,到底是什么药,也没有人知道,在死亡的背后,到底隐藏着什么样的阴谋。

明朝那些事儿-历史应该可以写得好看$[$1414$]$

此时向乾清宫赶去的人,包括内阁大臣、各部长官,共计十三人。在他们的心中,有着不同的想法和打算,因为皇
帝死了,官位、利益、权力,一切的一切都将改变。

只有一个人例外。

杨涟十分悲痛,因为那个赏识他的人,已经死了,而且死得不明不白。此时此刻,他只有一个念头。

查出案件的真相,找出幕后的黑手,揭露恶毒的阴谋,让正义得以实现,让死去的人得以瞑目。

这就是杨涟的决心。

但此时,杨涟即将面对的,却是一个更为复杂,更为棘手的问题。

虽然大家都住在内阁,同时听到消息,毕竟年纪不同,体力不同,比如内阁的几位大人,方从哲老先生都七十多
了,刘一璟、韩旷年纪也不小,反应慢点、到得晚点十分正常。

所以首先到达乾清宫的,只有六部的部长、都察院左都御史,当然还有杨涟。

这几个人已经知道了皇帝去世的消息,既然人死了,那就不用急了,就应该考虑尊重领导了,所以他们决定,等方
首辅到来再进去。

进不了宫,眼泪储备还不能用,而且大清早的,天都没亮,反正是等人,闲着也是闲着,于是,他们开始商讨善后
事宜。

继承皇位的,自然是皇长子朱由校了,但问题是,他的父亲死了,母亲也死了,而且年纪这么小,宫里没有人照
顾,怎么办呢?

于是,礼部尚书孙如游、吏部尚书周嘉谟、左都御史张问达提出:把朱由校交给李选侍。

这个观点得到了绝大多数人的支持,事实上,反对者只有一个。

然后,他们就听到了这个唯一反对者的声音:

``万万不可!''

其实就官职和资历而言,杨涟没有发言的资格,因为他此时他不过是个小小的七品给事中,说难听点,他压根就不
该呆在这里。

然而在场的所有人,都保持了沉默,静静地等待着他的发言,因为他是皇帝临死前指定的召见者,换句话说,他是
顾命大臣。

杨涟十分激动,他告诉所有的人,朱由校很幼稚,如果把他交给一个女人,特别是一个用心不良的女人,一旦被人
胁迫,后果将不堪设想。

这几句话,彻底唤起了在场朝廷重臣们的记忆,因为就在几天前,他们亲眼目睹了那个凶恶女人的狰狞面目。

他们同意了杨涟的意见。

但事实上,皇帝已经死了,未来的继承人,已在李选侍掌握之中。

所以,杨涟说出了他的计划:

``入宫之后,立刻寻找皇长子,找到之后,必须马上带出乾清宫,脱离李选侍的操纵,大事可成!''

十三位顾命大臣终于到齐了,在杨涟的带领下,他们走向了乾清宫。

一场你死我活的斗争即将开始。

明朝那些事儿-历史应该可以写得好看$[$1415$]$

战斗,从大门口开始

当十三位顾命大臣走到门口的时候,被拦住了。

拦住他们的,是几个太监。毫无疑问,这是李选侍的安排。

皇帝去世的时候,她就在宫内,作为一位智商高于郑贵妃的女性,她的直觉告诉她,即将到来的那些顾命大臣,将
彻底毁灭她的野心。

于是她决定,阻止他们入宫。

应该说,这个策略是成功的,太监把住大门,好说歹说就不让进,一帮老头加书呆子,不懂什么枪杆子里出政权的
深刻道理,只能干瞪眼。

幸好,里面还有一个敢玩命的:

``皇上已经驾崩,我们都是顾命大臣,奉命而来!你们是什么东西!竟敢阻拦!且皇长子即将继位,现情况不明,你们
关闭宫门,到底想干什么?!''

对付流氓加文盲,与其靠口,不如靠吼。

在杨涟的怒吼之下,吃硬不吃软的太监闪开了,顾命大臣们终于见到了已经歇气的皇上。

接下来是例行程序,猛哭猛磕头,哭完磕完,开始办正事。

大学士刘一璟首先发问:

``皇长子呢?他人在哪里?''

没人理他。

``快点交出来!''

还是没人理他。

李选侍清醒地意识到,她手中最重要的棋子,就是皇长子,只要控制住这个未来的继承人,她的一切愿望和野心,
都将得到满足。

这一招很绝,绝到杨涟都没办法,宫里这么大,怎么去找,一帮五六十岁的老头,哪有力气玩捉迷藏?

杨涟焦急万分,毕竟这不是家里,找不着就打地铺,明天接着找,如果今天没戏,明天李选侍一道圣旨下来,是死
是活都不知道!

必须找到,现在,马上,必须!

在这最为关键的时刻,一个太监走了过来,在大学士刘一璟的耳边,低声说出了两个字:

``暖阁。''

这个太监的名字,叫做王安。

王安,河北雄县人,四十多年前,他进入皇宫,那时,他的上司叫冯保。

二十六年前,他得到了新的任命,到一个谁也不愿意去的地方,陪一个谁也不愿意陪的人,这个人就是没人待见,
连名分都没有的皇长子朱常洛。

明朝那些事儿-历史应该可以写得好看$[$1416$]$

王安是个好人,至少是个识货的人,当朱常洛地位岌岌可危的时候,他坚定且始终站在了原地,无论是``争国
本'',还是``梃击''都竭尽全力,证明了他的忠诚。

朱常洛成为明光宗之后,他成为了司礼秉笔太监,掌控宫中大权。

这位仁兄最喜欢的人,是东林党,因为一直以来,东林党都是皇帝陛下的朋友。

而他最不喜欢的人,就是李选侍,因为这个女人经常欺负后宫的一位王才人,而这位王才人,恰好就是皇长子朱由
校的母亲。

此刻还不下烂药,更待何时?

刘一璟大怒,大吼一声:

``谁敢藏匿天子!''

可是吼完了,就没辙了,因为这毕竟是宫里,人躲在里面,你总不能破门而入去抢人吧。

所以最好的方法,是让李选侍心甘情愿地交人,然后送到门口,挥手致意。

这似乎绝不可能,但是王安说,这是可能的。随后,他进入了暖阁。

面对李选侍,王安体现出了一个卓越太监的素质,他虽没有抢人的体力,却有骗人的智力。

他对李选侍说,现在情况特殊,必须让皇长子出面,安排先皇的丧事,安抚大家的情绪,事情一完,人就能回来。

其实这谎扯得不圆,可是糊弄李选侍是够了。

她立即叫出了朱由校。

然而,就在她把人交给王安的那一瞬间,却突然醒悟了过来!她随即拉住了朱由校的衣服,死死拉住,不肯松手。

王安知道,动粗的时候到了,他决定欺负眼前这个耍赖的女人。因为太监虽说不男不女,可论力气,比李小姐还是
要大一些。

王安一把拉过朱由校,抱起就走,冲出了暖阁。当门外的顾命大臣们看见皇长子的那一刻,他们知道,自己胜利了。

于是,在先皇的尸体(估计还热着)旁,新任皇帝接受了顾命大臣们的齐声问候:万岁!

万岁喊完了,就该跑了。

在人家的地盘上,抢了人家的人,再不跑就是真是傻子了。

具体逃跑方法是,王安开路,刘一璟拉住朱由校的左手,英国公张维贤拉住朱由校的右手,包括方从哲在内的几个
老头走中间,杨涟断后。就这样,朱由校被这群活像绑匪(实际上也是)的朝廷大臣带了出去。

事情正如所料,当他们刚刚走出乾清宫的时候,背后便传来了李选侍尖利的叫喊声:

``哥儿(指朱由校),回来!''

\section[\thesection]{}

李大姐这嗓子太突然了,虽然没要人命,却把顾命大臣们吓了一跳,他们本来在乾清宫外准备了轿子,正在等轿夫
来把皇子抬走,听到声音后,脚一剁,不能再等了!

不等,就只能自己抬,情急之下,几位高干一拥而上,去抬轿子。

这四位高级轿夫分别是吏部尚书周嘉谟,给事中杨涟,内阁大学士刘一璟,英国公张维迎。

前面几位大家都熟,而最后这位张维迎,是最高世袭公爵,他的祖先,就是跟随明成祖朱棣靖难中阵亡的第一名将
张玉。

也就是说,四个人里除杨涟外,职务最低的是部长,我又查了下年龄,最年轻的杨涟,当时也已经四十八岁了,看
来人急眼了,还真敢拼命。

就这样,朱由校在这帮老干部的簇拥下,离开了乾清宫,他们的目标,是文华殿,只要到达那里,完成大礼,朱由
校就将成为新一代的皇帝。

而那时,李选侍的野心将彻底破灭。

当然,按照最俗套的电视剧逻辑,坏人们是不会甘心失败的,真实的历史也是如此。

毕竟老胳膊老腿,走不快,很快,大臣们就发现,他们被人追上了。

追赶他们的,是李选侍的太监。一个带头的二话不说,恶狠狠地拦住大臣,高声训斥:

``你们打算把皇长子带到哪里去?''

一边说,还一边动手去拉朱由校,很有点动手的意思。

对于这帮大臣而言,搞阴谋、骂骂人是长项,打架是弱项。于是,杨涟先生再次出场了。

他大骂了这个太监,并且鼓动朱由校:

``天下人都是你的臣子,何须害怕!''

一顿连骂带捧,把太监们都镇住了,领头的人见势不妙,就撤了。

这个被杨涟骂走的领头太监,名叫李进忠,是个不出名的人。但不久之后,他将更名改姓,改为另一个更有名的名
字----魏忠贤。

在杨涟的护卫下,朱由校终于来到了文华殿,在这里,他接受了群臣的朝拜,成为了新的皇帝,史称明熹宗。

明熹宗朱由校

这就算即位了,但问题在于,毕竟也是大明王朝,不是杂货铺,程序还要走,登基还得登。

有人建议,咱就今天办了得了,可是杨涟同志不同意,这位仁兄认定,既然要登基,就得找个良辰吉日,一查,那
就九月初六吧。

这是一个极为错误的决定。

\section[\thesection]{}

今天是九月初一,只要皇长子没登基,乾清宫依然是李选侍的天下,而且,她依然是受命照顾皇长子的人,对于她
而言,要翻盘,六天足够了。

然而杨涟本人,却没有意识到这一点。

就在他即将步入深渊的时候,一个人拉住了他,并且把一口唾沫吐在了他的脸上。

这个人的名字,叫做左光斗。

左光斗,字遗直,安徽桐城人。万历三十五年进士。现任都察院巡城御史,杨涟最忠实的战友,东林党最勇猛的战
士。

虽然他的职位很低,但他的见识很高,刚一出门,他就揪住了杨涟,对着他的脸,吐了口唾沫:

``到初六登基,今天才初一,如果有何变故,怎么收拾,怎么对得起先皇?!''

杨涟醒了,他终于明白,自己犯下了一个不可饶恕的错误。皇长子还在宫内,一旦李选侍掌握他,号令群臣,到时
必定死无葬身之地!

但事已至此,只能明天再说,毕竟天色已晚,皇宫不是招待所,杨大人不能留宿,无论如何,必须等到明天。

杨涟走了,李选侍的机会来了。

当天傍晚,朱由校再次来到乾清宫,他不能不来,因为他父亲的尸体还在这里。

可是他刚踏入乾清宫,就被李选侍扣住了,尸体没带走,还搭进去一个活人。

眼看顾命大臣们就要完蛋,王安又出马了。

这位太监可谓是智慧与狡诈的化身,当即挺身而出,去和李选侍交涉,按说被人抢过一次,总该长点记性,可是王
安先生几番忽悠下来,李选侍竟然又交出了朱由校。

这是个很难理解的事,要么是李小姐太弱智,要么是王太监太聪明,无论如何最终的结果是,李选侍失去了一个机
会,最后的机会。

因为第二天,杨涟将发起最为猛烈的进攻。

九月初二

吏部尚书周嘉谟和御史左光斗同时上书,要求李选侍搬出乾清宫。

这是一个十分聪明的战略,因为乾清宫是皇帝的寝宫,只要李选侍搬出去,她将无法制约皇帝,失去所有政治能量。

但要赶走李选侍,自己动手是不行的,毕竟这人还是后妃,拉拉扯扯成何体统?

经过商议,杨涟等人统一意见:让她自己走。

左光斗主动承担了这个艰巨的任务,为了彻底赶走这个女人,他连夜写出了一封奏疏,一封堪称恶毒无比的奏疏。

\section[\thesection]{}

文章大意是说,李小姐你不是皇后,也没人选你当皇后,所以你不能住乾清宫,而且这里也不需要你。

然后他进一步指出,朱由校才满十六岁,属于青春期少年,容易冲动,和你住在一起是不太合适的。

话说到这里,已经比较露骨了。

别慌,更露骨的还在后面。

在文章的最后,左光斗写出了一句画龙点睛的话:

``武氏之祸,再现于今,将来有不忍言者!''

所谓武氏,就是武则天,也就是说,左光斗先生担心,如此下去,武则天夺位的情形就会重演。

如果你认为这是一句非常过分的话,那你就错了,事实上,是非常非常过分,因为左光斗是读书人,有时候,读书
人比流氓还流氓。

希望你还记得,武则天原先是唐太宗的妃子,高宗是太宗的儿子,

后来,她又成了唐高宗的妃子。

现在,李选侍是明光宗的妃子,熹宗是光宗的儿子,后来……

所以左光斗先生的意思是,李选侍之所以住在乾清宫,是想趁机勾引她的儿子(名义上的)。

李选侍急了,这很正常,你看你也急,问题在于,你能咋办?

李选侍想出的主意,是叫左光斗来谈话。事实证明,这是个不折不扣的馊主意,因为左光斗的回答是这样的:

``我是御史,天子召见我才会去,你算是个什么东西(若辈何为者)?''

九月初三

左光斗的奏疏终于送到了皇帝的手中,可是皇帝的反应并不大,原因简单:他看不懂。

拜他父亲所赐,几十年来躲躲藏藏,提心吊胆,儿子的教育是一点没管,所以朱由校小朋友不怎么读书,却很喜欢
做木匠,常年钻研木工技巧。

幸好,他的身边还有王安。

王太监不负众望,添油加醋解说一番,略去儿童不宜的部分,最后得出结论:李选侍必须滚蛋。

朱由校决定,让她滚。

很快,李选侍得知了这个决定,她决定反击。

九月初四

李选侍反击的具体形式,是谈判。

她派出了一个使者,去找杨涟,希望这位钢铁战士会突然精神失常,放弃即将到手的胜利,相信她是一个善良、无
私的女人,并且慷慨大度的表示,你可以继续住在乾清宫,继续干涉朝政。

人不能愚蠢到这个程度。

但她可以。

\section[\thesection]{}

而她派出的那位使者,就是现在的李进忠,将来的魏忠贤。

这是两位不共戴天的死敌第一次正面交锋。

当然,当时的杨涟并没有把这位太监放在眼里,见面二话不说:

``她(指李选侍)何时移宫?''

李进忠十分客气:

``李选侍是先皇指定的养母,住在乾清宫,其实并没有什么问题。''

杨涟很不客气:

``你给我记好了,回去告诉李选侍,现在皇帝已经即位,让她立刻搬出来,如果乖乖听话,她的封号还能给她,如
果冥顽不灵,就等皇帝发落吧!''

最后还捎带一句:

``你也如此!''

李进忠沉默地走了,他很清楚,现在自己还不是对手,在机会到来之前,必须等待。

李选侍绝望了,但她并不甘心,在最后失败之前,她决心最后一搏,于是她去找了另一个人。

九月初五 登基前最后一日

按照程序规定,明天是皇帝正式登基的日期,但是李选侍却死不肯搬,摆明了要耍赖,于是,杨涟去找了首辅方从
哲,希望他能号召群臣,逼李选侍走人。

然而,方从哲的态度让他大吃一惊,这位之前表现积极的老头突然改了口风:

``让她迟点搬,也没事吧。(迟亦无害)''

杨涟愤怒了:

``明天是皇上登基的日子,难道要让他躲在东宫,把皇宫让给那个女人吗?!''

方从哲保持沉默。

李选侍终于聪明了一次,不能争取杨涟,就争取别人,比如说方从哲。

因为孤独的杨涟,是无能为力的。

但她错了,孤独的杨涟依然是强大的,因为在他的心中,始终都留存着一个信念:

当我只是个小人物的时候,你体谅我的激奋,接受我的意见,相信我的才能,将你的身后之事托付于我。

所以,我会竭尽全力,战斗至最后一息,绝不放弃。

因为你的信任,和尊重。

在这最后的一天里,杨涟不停地到内阁以及各部游说,告诉大家形势危急,必须立刻挺身而出,整整一天,即使遭
遇冷眼,被人讥讽,他依然不断地说着,不断地说着。

最终,许多人被他打动,并在他的率领下,来到了宫门前。

面对着阴森的皇宫,杨涟喊出了执着而响亮的宣言:

``今日,除非你杀掉我,若不移宫,宁死不离(死不去)!''

\section[\thesection]{}

由始至终,李选侍都是一个极为贪婪的女人,为达到目的,可以不择手段,不顾一切,虐待朱由校的母亲,逼迫皇
帝,责骂皇长子,只为她的野心和欲望。

但现在,她退缩了,她决定放弃。因为她已然发现,这个叫杨涟的人,是很勇敢的,敢于玉石俱焚、敢于同归于尽。

无奈地叹息之后,她退出了乾清宫,从此,她消失了,消失得无影无踪,她或许依然专横、撒泼,却已无人知晓,
因为,她已无关紧要。

随同她退出的,还有她的贴身太监们,时移势易,混口饭吃也不容易。

然而一位太监留了下来,他知道,自己的命运还未终结,因为他已经发现了一个新的目标--另一个女人。从这个女
人的身上,他将得到新的前途,以及新的名字。

强大,无比强大

万历四十八年(1620)九月初六,明熹宗朱由校在乾清宫正式登基,定年号为天启。

一个复杂无比,却又精彩绝伦的时代就此开始。

杨涟终于完成了他的使命,自万历四十八年(1620)八月二十二日起,在短短十五天之内,他无数次绝望,又无数次
奋起,召见、红丸、闯宫、抢人、拉拢、死磕,什么恶人、坏人都遇上了,什么阴招、狠招都用上了。

最终,他成功了。

据史料记载,在短短十余天里,他的头发已变成一片花白。

当天启皇帝朱由校坐在皇位上,看着这个为他的顺利即位费尽心血的人时,他知道,自己应该回报。

几日后,杨涟升任兵科都给事中,一年后,任太常少卿,同年,升任都察院佥都御史,后任左副都御史。短短一年
内,他从一个从七品的芝麻官,变成了从二品的部级官员。

当然,得到回报的,不仅是他。

东林党人赵南星,退休二十多年后,再度复出,任吏部尚书。

东林党人高攀龙,任光禄丞。后升任光禄少卿。

东林党人邹元标,任大理寺卿,后任刑部右侍郎,都察院左都御史。

东林党人孙慎行,升任礼部尚书。

东林党人左光斗,升任大理寺少卿,一年后,升任都察院左佥都御史。

以下还有若干官,若干人,篇幅过长,特此省略。

\section[\thesection]{}

小时候,老师告诉我,个人是渺小的,集体才是伟大的,现在,我相信了。

当皇帝的当皇帝,升官的升官,滚蛋的滚蛋,而那个曾经统治天下的人,却似乎已被彻底遗忘。

明光宗朱常洛,作为明代一位极具特点(短命)的皇帝,他的人生可以用四个字来形容----苦大仇深。

出生就不受人待见,母亲被冷遇,长大了,书读不上,太子立不了,基本算三不管,吃穿住行级别很低,低到连刺
杀他的人,都只是个普通农民,拿着根木棍,就敢往宫里闯。

好不容易熬到登基,还要被老婆胁迫,忍了几十年,放纵了一回,身体搞垮了,看医生,遇见了蒙古大夫,想治
病,就去吃仙丹,结果真成仙了。

更搞笑的是,许多历史书籍到他这里,大都只讲三大案,郑贵妃、李选侍,基本上没他什么事,原因很简单,他只
当了一个月皇帝。

在他死后,为了他的年号问题,大臣们展开了争论,因为万历四十八年七月,万历死了,八月,他就死了。而他的
年号泰昌,没来得急用。

问题来了,如果把万历四十八年(1620)当作泰昌元年,那是不行的,因为直到七月,他爹都还活着。

如果把第二年(1621)当作泰昌元年,那也是不行的,因为他去年八月,就已经死了。

这是一个无法解决的问题。

问题终究被解决了,凭借大臣们无比高超的和稀泥技巧,一个前无古人、后无来者的处理方案隆重出场:

万历四十八年(1620)一月到七月,为万历四十八年。八月,为泰昌元年。明年(1621),为天启元年。

这就是说,在这一年里,前七个月是他爹的,第二年是他儿子的,而他的年份,只有一个月。

原因很简单,他只当了一个月皇帝。

他很可怜,几十年来畏畏缩缩,活着没有待遇,死了没有年号,事实上,他人才刚死,就有一堆人在他尸体旁边你
死我活,抢儿子抢地方,忙得不亦乐乎。

原因很简单,他只当了一个月皇帝。

有人曾对我说,原来,历史很有趣。但我对他说,其实,历史很无趣。

因为在绝大多数情况下,历史没有正恶,只有成败。

左都御史、左副都御史、吏部尚书、刑部侍郎、大理寺丞等等等等,政权落入了东林党的手中。

它很强大,强大到无以复加的地步。对于这一现象,史称``众正盈朝''。

\section[\thesection]{}

按照某些史书的传统解释,从此,在东林党人的管理下,朝廷进入了一个公正、无私的阶段,许多贪婪的坏人被赶
走,许多善良的好人留下来。

对于这种说法,用两个字来评价,就是胡说。

用四个字来评价,就是胡说八道。

之前我曾经说过,东林党不是善男信女,现在,我再说一遍。

掌权之后,这帮兄弟干的第一件事,就是追查红丸案。

追查,是应该的,毕竟皇帝死得蹊跷,即使里面没有什么猫腻,但两位蒙古大夫,一个下了泻药,让他拉了几十
次,另一个送仙丹,让他飞了天,无论如何,也应该追究责任。

退一万步讲,就算你追究责任后还不过瘾,非要搞几个幕后黑手出来,郑贵妃、李选侍这几位重点嫌疑犯,名声
坏,又歇了菜,要打要杀,基本都没个跑。

可是现成的偏不找,找来找去,找了个老头----方从哲。

天启元年(1621),礼部尚书孙慎行上疏,攻击方从哲。大致意思是说,方从哲和郑贵妃有勾结,而且他还曾经赏赐
过李可灼,出事后,只把李可灼赶回了家,没有干掉,罪大恶极,应予严肃处理。

这就真是有点无聊恶搞了,之前说过,李可灼最初献药,还是方老头赶回去的,后来赏钱那是皇帝同意的,所谓红
丸到底是什么玩意,鬼才知道,稀里糊涂把人干掉,也不好。

所以无论从哪个角度看,方从哲都没错,而且此时东林党掌权,方老头识时务,也不打算呆了,准备回家养老去了。

可孙部长用自己的语言,完美地解释了强词夺理这个词的含义:

``从哲(方从哲)纵无弑之心,却有弑之罪,纵辞弑之名,难免弑之实。''

这意思是,你老兄即使没有干掉皇帝的心思,也有干掉皇帝的罪过,即使你退休走人,也躲不过去这事。

强词夺理还不算,还要赶尽杀绝:

``陛下宜急讨此贼,雪不共之仇!''

所谓此贼,不是李可灼,而是内阁首辅,他的顶头上司方从哲。

很明显,他很激动。

孙部长激动之后,都察院左都御史邹元标也激动了,跟着上书过了把瘾,不搞定方从哲,誓不罢休。

这是一件十分奇怪的事。

七十多岁的老头,都快走人了,为什么就是揪着不放呢?

因为他们有着一个不可告人的目的。

\section[\thesection]{}

郑贵妃不重要,李选侍不重要,甚至案件本身也不重要。之所以选中方从哲,把整人进行到底,真正的原因在于,
他是浙党。

只要打倒了方从哲,借追查案件,就能解决一大批人,将政权牢牢地抓在手中。

他们的目的达到了,不久之后,崔文升被发配南京,李可灼被判流放,而方从哲,也永远地离开了朝廷。

明宫三大案就此结束,东林党大获全胜。

局势越来越有利,天启元年(1621)十月,另一个重量级人物回来了。

这个人就是叶向高。

东林党之中,最勇猛的,是杨涟,最聪明的,就是这位仁兄了。而他担任的职务,是内阁首辅。

作为名闻天下的老滑头,他的到来,标志着东林党进入了全盛时期。

内忧已除,现在,必须解决外患。

因为他们还没来得及庆祝,就得知了这样一个消息----沈阳失陷。

沈阳是在熊廷弼走后,才失陷的。

熊廷弼驻守辽东以来,努尔哈赤十分消停,因为这位熊大人做人很粗,做事很细,防守滴水不漏,在他的管理下,
努尔哈赤成了游击队长,只能时不时去抢个劫,大事一件没干成。

出于对熊廷弼的畏惧和愤怒,努尔哈赤给他取了个外号:熊蛮子。

这是一个名副其实的外号,不但对敌人蛮,对自己人也蛮。

熊大人的个性前面说过了,彪悍异常,且一向不肯吃亏,擅长骂人,骂完努尔哈赤,还不过瘾,一来二去,连兵部
领导、朝廷言官也骂了。

这就不太好了,毕竟他还归兵部管,言官更不用说,平时只有骂人,没有被人骂,索性敞开了对骂,闹到最后,熊
大人只好走人。

接替熊廷弼的,是袁应泰。

在历史中,袁应泰是个评价很高的人物,为官廉洁,为人清正,为政精明,只有一个缺点,不会打仗。

这就没戏了。

他到任后,觉得熊廷弼很严厉,很不近人情,城外有那么多饥民(主要是蒙古人),为什么不放进来呢?就算不能打
仗,站在城楼上充数也不错嘛。

于是他打开城门,放人入城,亲自招降。

一个月后,努尔哈赤率兵进攻,沈阳守将贺世贤拼死抵抗,关键时刻,之前招安的蒙古饥民开始大肆破坏,攻击守
军,里应外合之下,沈阳陷落。贺世贤战死,七万守军全军覆没。

这一天,是天启元年(1621)三月十二日。

袁应泰没有时间后悔,因为他只多活了六天。

\section[\thesection]{}

攻陷沈阳后,后金军队立刻整队,赶往下一个目标----辽阳。

当年,辽阳的地位,大致相当于今天的沈阳,是辽东地区的经济、文化、军事中心,也是辽东的首府。此地历经整
修,壕沟围绕,防守严密,还有许多火炮,堪称辽东第一坚城。

守了三天。

战斗经过比较简单,袁应泰率三万军队出战,被努尔哈赤的六万骑兵击败,退回坚守,城内后金奸细放火破坏,大
乱,后金军乘虚而入,辽阳陷落。

袁应泰看见了城池的陷落,他非常镇定,从容穿好官服,佩带着宝剑,面向南方,自缢而死。他不是一个称职的大
明将领,却是一个称职的大明官员。

辽阳的丢失,标志着局势的彻底崩溃,标志着辽东成为了后金的势力范围,标志着从此,他们想去哪里,就去哪
里,想抢哪里,就抢哪里。

局势已经坏得不能再坏了,所以,不能用的人,也不能不用了。

天启元年(1621)七月,熊廷弼前往辽东。

在辽东,他遇见了王化贞。

他不喜欢这个人,从第一次见面开始。因为他发现,这人不买他的帐。

熊廷弼此时的职务是辽东经略,而王化贞是辽东巡抚。从级别上看,熊廷弼是王化贞的上级。

角色并不重要,关键在于会不会抢戏----小品演员陈佩斯

王化贞就是一个很会抢戏的人,因为他有后台,所以他不愿意听话。

关于这两个人的背景,有些历史书上的介绍大概如此:熊廷弼是东林党支持的,王化贞是阉党支持的。最终结局也
再次证明,东林党是多么地明智,阉党是多么地愚蠢。

胡扯

不是胡扯,就是装糊涂。

因为最原始的史料告诉我们,熊廷弼是湖广人,他是楚党的成员,而在大多数时间里,楚党是东林党的敌人。

至于王化贞,你说他跟阉党有关,倒也没错,可是他还有个老师,叫做叶向高。

天启元年的时候,阉党都靠边站,李进忠还在装孙子,连名字都没改,要靠这帮人,王化贞早被熊先生赶去看城门
了。

他之所以敢嚣张,敢不听话,只是因为他的老师,是朝廷首辅,朝中的第一号人物。

\section[\thesection]{}

熊廷弼是对的,所以他是东林党,或至少是东林党支持的,王化贞是错的,所以他是阉党,或至少是阉党赏识的。
大致如此。

我并非不能理解好事都归自己,坏事都归别人的逻辑,也并不反对,对某些坏人一棍子打死再踩上一只脚的行为,
我只是认为,做人,还是要厚道。

王化贞不听熊廷弼的话,很正常,因为他的兵,比熊廷弼的多。

当时明朝在辽东的剩余部队,大约有十五万,全都在王化贞的手中。而熊廷弼属下,只有五千人。

所以每次王化贞见熊廷弼时,压根就不听指挥,说一句顶一句,气得熊大人恨不能拿刀剁了他。

但事实上,王化贞是个很有能力的人。

王化贞,山东诸城人。万历四十一年进士。原先是财政部的一名处级干部(主事),后来不知怎么回事,竟然被调到
了辽东广宁(今辽宁北宁)。

此人极具才能,当年蒙古人闹得再凶,到他的地头,都不敢乱来。后来辽阳、沈阳失陷,人心一片慌乱,大家都往
关内跑,他偏不跑。

辽阳城里几万守军,城都丢了,广宁城内,只有几千人,还是个破城,他偏要守。

他非但不跑,还招集逃兵,整顿训练,居然搞出了上万人的队伍,此外,他多方联络,稳定人心,坚守孤城,稳定
了局势。所谓``提弱卒,守孤城,气不慑,时望赫然'',天下闻名,那也真是相当的牛。

熊廷弼也是牛人,但对于这位同族,他却十分不感冒,不仅因为牛人相轻,更重要的是,此牛非彼牛也。

很快,熊大人就发现,这位王巡抚跟自己,压根不是一个思路。

按他自己想法,应该修筑堡垒,严防死守,同时调集援兵,长期驻守。

可是王化贞却认定,应该主动进攻,去消灭努尔哈赤,他还说,只要有六万精兵,他就可以一举荡平。

熊廷弼觉得王化贞太疯,王化贞觉得熊廷弼太熊。

最后王化贞闭口了,他停止了争论,因为争论没有意义。

兵权在我手上,我想干嘛就干嘛,和你讨论,是给你个面子,你还当真了?

一切都按照王化贞的计划进行着,准备粮草,操练士兵,寻找内应,调集外援,忙得不亦乐乎。

忙活到一半,努尔哈赤来了。

天启二年(1622)正月十八日,努尔哈赤亲率大军,进攻广宁。

\section[\thesection]{}

之前半年,努尔哈赤听说熊廷弼来了,所以他不来。后来他听说,熊廷弼压根没有实权,所以他来了。

实践证明,王巡抚胆子很大,脑子却很小,面对努尔哈赤的进攻,他摆出了一个十分奇怪的阵型,先在三岔河布
阵,作为第一道防线,然后在西平堡设置第二道防线,其余兵力退至广宁城。

就兵力而言,王化贞大概是努尔哈赤的两倍,可大敌当前,他似乎不打算``一举荡平'',也不打算御敌于国门之
外,因为外围两道防线的总兵力也才三万人,是不可能挡住努尔哈赤的。

用最阴暗的心理去揣摸,这个阵型的唯一好处,是让外围防线的三万人和努尔哈赤死拼,拼完,努尔哈赤也就差不
多了。

事实确实如此,正月二十日,努尔哈赤率军进攻第一道防线三岔河,当天即破。

第二天,他来到了第二道防线西平堡,发动猛烈攻击,但这一次,他没有如愿。

因为西平堡守将罗一贯,是个比较一贯的人,努尔哈赤进攻,打回去,汉奸李永芳劝降,骂回去,整整一天,后金
军队毫无进展。

王化贞的反应还算快,他立即派出总兵刘渠、祁秉忠以及他的心腹爱将孙得功,分率三路大军,增援西平堡。

努尔哈赤最擅长的,就是围点打援。所以明军的救援,早在他意料之中。

但在他意料之外的,是明军的战斗力。

总兵刘渠、祁秉忠率军出战,两位司令十分勇猛,亲自上阵,竟然打得后金军队连连败退,于是,作为预备队的孙
得功上阵了。

按照原先的想法,孙得功上来,是为了加强力量,可没想到的是,这位兄弟刚上阵,却当即溃败,惊慌之余,孙大
将还高声喊了一嗓子:

``兵败了!兵败了!''

您都兵败了,那还打什么?

后金军随即大举攻击,明军大败,刘渠阵亡,祁秉忠负伤而死,孙得功逃走,所属数万明军全军覆没。

现在,在努尔哈赤面前的,是无助、毫无遮挡的西平堡。

罗一贯很清楚,他的城池已被团团包围,不会再有援兵,不会再有希望,对于胜利,他已无能为力。

但他仍然决定坚守,因为他认为,自己有这个责任。

正月二十二日,努尔哈赤集结所属五万人,发动总攻。

罗一贯率三千守军,拼死守城抵抗。

\section[\thesection]{}

双方激战一天,后金军以近二十倍的兵力优势,发起了无数次进攻,却无数次败退,败退在孤独却坚定的罗一贯眼
前。

明军凭借城堡大量杀伤敌军,后金损失惨重,毫无进展,只得围住城池,停止进攻。

但出乎他们意料的是,城头突然陷入了死一般的寂静,没有了呐喊,没有了杀声。

因为城内的士兵,已经放出了最后一支弓箭,发射了最后一发火炮。

在这最后的时刻,罗一贯站在城头,向着京城的方向,行叩拜礼,说出了他的遗言:

``臣力竭矣!''

然后,他自刎而死。

这是努尔哈赤自起兵以来,损失空前惨重的一战,据史料记载,和西平堡三千守军一同阵亡的,有近七千名后金军。

罗一贯尽到了自己的职责,王化贞也准备这样做。

得知西平堡失陷后,他连夜督促加强防守,并对逃回来的孙得功既往不咎,鼓励守城将士众志成城,击退后金军队。

然后,他就去睡觉了。

王化贞不是个怕事的人,当年辽阳失守,他无兵无将都敢坚守,现在手上有几万人,自然敢睡觉。

但还没等他睡着,就听见了随从的大叫:

``快跑!''

王化贞跑出卧房。

他看见无数百姓和士兵丢弃行李兵器,夺路而逃,原本安静祥和的广宁城,已是一片混乱,彻底的混乱。

而此时的城外,并没有努尔哈赤,也没有后金军,一个都没有。

这莫名其妙的一切,起源于两个月前的一个决定。

王化贞不是白痴,他很清楚努尔哈赤的实力,在那次谈话中,他之所以告诉熊廷弼,说六万人一举荡平,是因为他
已找到了努尔哈赤的弱点。

这个弱点,叫做李永芳。

李永芳是明朝叛将,算这一带的地头蛇,许多明军将领跟他都有交情,毕竟还是同胞兄弟,所以在王化贞看来,这
是一个可以争取的人。

于是,他派出了心腹孙得功,前往敌营,劝降李永芳。

几天后,孙得功回报,李永芳深明大义,表示愿意归顺,在进攻时作为内应。王化贞十分高兴。

两个月后,孙得功西平堡战败,惊慌之下,大喊``兵败'',导致兵败。

是的,你的猜测很正确,孙得功是故意的,他是个叛徒。

\section[\thesection]{}

孙得功去劝降李永芳,却被李永芳劝降,原因很简单,不是什么忠诚、爱国、民族、大同之类的话,只是他出价更
高。

为了招降李永芳,努尔哈赤送了一个孙女,一个驸马(额驸)的头衔,还有无数金银财宝,很明显,王化贞出不起这
个价。

努尔哈赤从来不做赔本买卖,他得到了极为丰厚的回报。

孙得功帮他搞垮了明朝的援军,但这还不够,这位誓把无耻进行到底的败类,决定送一份更大的礼物给努尔哈
赤----广宁城。

因为自信的王化贞,将城池的防守任务交给了他。

接下来的事顺理成章,从被窝里爬起来的王大人慌不择路,派人去找马,准备逃走,可是没想到,孙心腹实在太抠
门,连马都弄走了,搞得王大人只找到了几头骆驼,最后,他只能骑着骆驼跑路。

还好,那天晚上,孙心腹忙着带领叛军捣乱,没顾上逃跑的王巡抚,否则以他的觉悟,拿王大人的脑袋去找努尔哈
赤换个孙女,也是不奇怪的。

第二天,失意的王巡抚在逃走的路上,遇到了一个让他更为失意的人。

熊廷弼用实际行动证明,他不是一个慈悲的人,至少不会放过落水狗。当王巡抚痛哭流涕,反复检讨错误时,他用
一句话表示了他的同情:

``六万大军一举荡平?现在如何?''

王化贞倒还算认账,关键时刻,也不跟熊廷弼吵,只是提出,现在应派兵,坚守下一道防线----宁远。

这是一个十分明智的判断,可是熊大人得理不饶人,还没完了:``现在这个时候,谁肯帮你守城?晚了!赶紧掩护百
姓和士兵入关,就足够了!''

这句话的潜台词是,当初不听我的,现在我也不听你的。

事情到这份上,就没什么可说的了,作为丧家犬,王化贞没有发言权。于是,战局离开了王化贞的掌控,走上了熊
廷弼的轨道。

从王化贞到熊廷弼,从掌控到轨道,这是一个有趣的变化。变化的前后有很多不同点,也有一个共同点:都是错误
的。

虽然敌情十分紧急,城池空虚,但此时明军主力尚存,若坚定守住,估计也没什么问题。可是熊先生来了牛脾气,
不由分说,宁远也不守了,把辽东的几十万军民全部撤回关(山海关)内,放弃了所有据点。

\section[\thesection]{}

熊大人没有意识到,他已经做到了无数敌人、无数汉奸、无数叛徒想做却做不到的事情,因为事实上,他已放弃整
个辽东。

自明朝开国以来,稳固统治两百余年的辽东,就这么丢了。无论从哪个角度看,熊廷弼都没有理由、没有借口、没
有道理这样做。

但是他做了。

我认为,他是为了一口气。

当初不听我的话,现在看你怎么办?

就是这口气,最后要了他的命。

率领几十万军民,成功撤退的两位仁兄终于回京了,明朝政府对他们俩的处理,是相当一视同仁的----撤职查办。

无论谁对谁错,你们把朝廷在辽东的本钱丢得精光,还有脸回来?这个黑锅你们不背,谁背?当然,最后处理结果还
是略有不同,熊大人因为脾气不好,得罪人多,三年后(天启五年)就被干掉了。

相对而言,王大人由于关系硬,人缘好,又多活了七年,崇祯五年才正式注销户口。

对于此事,许多史书都说,王化贞死得该,熊廷弼死得冤。

前者我同意,后者,我保留意见。

事实上,直到王化贞逃走后的第三天,努尔哈赤才向广宁进发,他没有想到,明军竟然真的不战而逃,而且以他的
兵力,并不足以占据辽东。

然而当他到达广宁,接受孙得功投降之时,才发现,整个辽东,已经没有敌人。

因为慷慨的熊蛮子,已把这片广阔的土地毫无保留地交给了他。

白给的东西不能不要,于是在大肆抢掠之后,他率军向新的目标前进----山海关。

可是走到半路,他发现自己的算盘打错了。因为熊蛮子交给他的,不是辽东,而是一个空白的辽东。

为保证不让敌人抢走一粒粮,熊先生干得相当彻底,房子烧掉,水井埋掉,百姓撤走,基本上保证了千里无鸡鸣,
万里无人烟。

要这么玩,努尔哈赤先生就不干了,他辛苦奔波,最终的目的是为了抢东西,您把东西都搬走了,我还去干嘛?

而且从广宁到山海关,几百里路空无一人,很多坚固的据点都无人看守,别说抢劫,连打仗的机会都没有。

于是,当军队行进到一个明军据点附近时,努尔哈赤决定:无论这些地方有多广袤,无论这些据点有多重要,都不
要了,撤退。

努尔哈赤离开了这里,踏上了归途,但他不会想到,自己已经犯下了一个致命的错误。因为四年之后,他将再次回
到这里,并为争夺这个他曾轻易放弃的小地方,失去所有的一切。

这个他半途折返的地点,叫做宁远。

\section[\thesection]{}

堪与匹敌者,此人也

自万历四十六年,努尔哈赤起兵以来,短短三年时间,抚顺、铁岭、开原、辽阳、沈阳,直至整个辽东,全部陷落。

从杨镐、刘綎到袁应泰、王化贞、熊廷弼,不能打的完了,能打的也完了,熊人死了,牛人也死了。

辽东的局势,说差,那是不恰当的,应该说,是差得不能再差,差到官位摆在眼前,都没人要。

比如总兵,是明军的高级将领,全国不过二十人左右,用今天话说,是军区司令员。要想混到这个职务,不挤破头
是不大可能的。

一般说来,这个职务相当安全,平日也就是看看地图,指手划脚而已。然而这几年情况不同了,辽东打仗,明朝陆
续派去了十四位总兵,竟然全部阵亡,无一幸免。

总兵越来越少,而且还在不断减少,因为没人干,某些在任总兵甚至主动辞职,宁可回家种田,也不干这份工作。

但公认最差的职业,还不是总兵,是辽东经略。

总兵可以有几十个,辽东经略只有一个。总兵可以不干,辽东经略不能不干。

可是连傻子都知道,辽东都没了,人都撤回山海关了,没兵没地没百姓,还经略个啥?

大家不是傻子,大家都不去。

接替辽东经略的第一人选,是兵部尚书张鹤鸣,天启为了给他鼓劲,先升他为太子太保(从一品),又给他尚方宝
剑,还亲自送行。

张尚书没说的,屁股一拍,走了。

走是走了,只是走得有点慢,从京城到山海关,他走了十七天。

这条路线上星期我走过,坐车三个钟头。

张大人虽说没车,马总是有的,就两百多公里,爬也爬过去了。

这还不算,去了没多久,这位大人又说自己年老力衰,主动辞职回家了。

没种就没种,装什么蒜?

相比而言,接替他的宣府巡抚就好得多了。

这位巡抚大人接到任命后,连上三道公文,明白跟皇帝讲:我不去。

天启先生虽说是个木匠,也还有点脾气,马上下达谕令:不去,就滚(革职为民,永不叙用)。

不想去也好,不愿去也好,替死鬼总得有人当,于是,兵部侍郎王在晋出场了。

\section[\thesection]{}

王在晋,字明初,江苏太仓人。万历二十年进士。这位仁兄从没打过仗,之所以让他去,是因为他不能不去。

张尚书跑路的时候,他是兵部副部长,代理部长(署部事),换句话说,轮也轮到他了。

史书上对于这位仁兄的评价大都比较一致:什么废物、愚蠢,不一而同。

对此,我都同意,但我认为,他至少是个勇敢的人。

明知是黑锅,依然无怨无悔、义无反顾地去背,难道不勇敢吗?而他之所以失败,实在不是态度问题,而是能力问题。

因为他面对的敌人,是努尔哈赤。

努尔哈赤,明朝最可怕的敌人,战场应变极快,骑兵战术使用精湛,他的军事能力,可与大明历史上的任何一位名
将相媲美。

毫无疑问,他是这个时代最为强悍、最具天赋的军事将领,之一

他或许很好,很强大,却绝非没有对手。

事实上,他宿命的克星已然出现,就在他的眼前----不只一个。

王在晋到达辽东后,非常努力,非常勤奋,他日夜不停地勘查地形,考量兵力部署,经过几天几夜的刻苦专研,终
于想出了一个防御方案。

具体方案是这样的,王在晋认为,光守山海关是不够的,为了保证防御纵深,他决定再修一座新城,用来保卫山海
关,而这座新城就在山海关外八里的八里铺。

王在晋做事十分认真,他不但选好了位置,还拟好了预算,兵力等等,然后一并上交皇帝。

天启皇帝看后大为高兴,立即批复同意,还从国库中拨出了工程款。

应该说,王在晋的热情是值得肯定的,态度是值得尊重的,创意是值得鼓励的,而全盘的计划,是值得唾弃的。

光守山海关是不够的,因为一旦山海关被攻破,京城就将毫无防卫,唾手可得,虽说山海关沿线很坚固,很结实,
但毕竟是砖墙,不是高压电网,如果努尔哈赤玩一根筋,拼死往城墙上堆人,就是用嘴啃,估计也啃穿了。

在这一点上,王在晋的看法是正确的。

但这也是他唯一正确的地方,除此之外,都是胡闹。

哪里胡闹,我就不说了,等一会有人说。

总之,如按此方案执行,山海关破矣,京城丢矣,大明亡矣。

对于这一结果,王在晋不知道,天启自然也不知道,而更多的人,是知道了也不说。

就在一切几乎无可挽回的时候,一封群众来信,彻底改变了这个悲惨的命运。

\section[\thesection]{}

这封信是王在晋的部下写的,并通过朝廷渠道,直接送到了叶向高的手中,文章的主题思想只有一条:王在晋的方
案是错误的。

这下叶大人头疼了,他干政治是老手,干军事却是菜鸟,想来想去,这个主意拿不了,于是他跑去找皇帝。

可是皇帝大人除了做木匠是把好手,基本都是抓瞎,他也吃不准,于是,他又去找了另一个人。

天惊地动,力挽狂澜,由此开始。

``夫攻不足者守有余,度彼之才,恢复固未易言,令专任之,犹足以慎固封守。''

这句话,来自于一个人的传记。

这句话的大致意思是:以此人的才能,恢复失去的江山,未必容易,但如果信任他,将权力交给他,稳定固守现有
的国土,是可以的。

这是一个至高无上的评价。

因为这句话,出自于《明史》。说这句话的人,是清代的史官。

综合以上几点,我们可以认定,在清代,这是一句相当反动的话。

因为它的隐含意思是:

如果此人一直在任,大清是无法取得天下的。

在清朝统治下,捧着清朝饭碗,说这样的话,是要掉脑袋的。

可是他们说了,他们不但说了,还写了下来,并且流传千古,却没有一个人,因此受到任何惩罚。

因为他们所说的,是铁一般的事实,是清朝统治者无法否认的事实。与此同时,他们还用一种十分特殊的方式,表
达了对此人的崇敬。

在长达二百二十卷、记载近千人事迹的明史传记中,无数为后人熟知的英雄人物,都要和别人挤成一团。

而在这个人的传记里,只有他自己和他的子孙。

这个人不是徐达,徐达的传记里,有常遇春。

不是刘伯温,刘伯温的传记里,有宋濂、叶琛、章溢。

不是王守仁,王守仁的传记里,还搭配了他的门人冀元亨。

也不是张居正,张大人和他的老师徐阶、老对头高拱在一个传记里。

当然,更不是袁崇焕,袁将军住得相当挤,他的传记里,还有十个人。

这个人是孙承宗。

明末最伟大的战略家,努尔哈赤父子的克星,京城的保卫者,皇帝的老师,忠贞的爱国者。

举世无双,独一无二。

在获得上述头衔之前,他是一个不用功的学生,一个讨生活的教师,一个十六年都没有考上举人的落魄秀才。

\section[\thesection]{}

嘉靖四十二年(1563),孙承宗出生在北直隶保定府高阳(今河北省高阳县)。

生在这个地方,不是个好事。

作为明朝四大防御要地,蓟州防线的一部分,孙承宗基本是在前线长大的。

这个地方不好,或者说是太好,蒙古人强大的时候,经常来,女真人强大的时候,经常来,后来改叫金国,也常
来,来抢。

来一次,抢一次,打一次。

这实在不是个适合人类居住的地方,别的小孩都怕,可孙承宗不怕。

非但不怕,还过得特别滋润。

他喜欢战争,喜欢研究战争,从小,别人读四书,他读兵书。成人后,别人往内地跑,他往边境跑,不为别的,就
想看看边界。

万历六年(1578),保定府秀才孙承宗做出了一个决定----外出游学。这一年,他十六岁。在此后十余年的时间里,
孙秀才游历四方,努力向学,练就了一身保国的本领。

当然,这是史料里正式的说法。

实际上,这位仁兄在这十几年来,大都是游而不学,要知道,他当年之所以考秀才,不是为了报国,说到底,是混
口饭吃,游学?不用吃饭啊?

还好,孙秀才找到了一份比较好的工作----老师,从此,他开始在教育战线上奋斗,而且越奋斗越好,好到名声传
到了京城。

万历二十年(1592),在兵部某位官员的邀请下,孙秀才来到京城,成为了一位优秀的私人教师。

但是慢慢地,孙秀才有思想活动了,他发现,光教别人孩子是不够的,能找别人教自己的孩子,才是正道。

于是第二年(1593),他进入了国子监,刻苦读书,再一年后(1594),他终于考中了举人,这一年,他三十二岁。

一般说来,考上举人,要么去考进士,要么去混个官,可让人费解的是,孙举人却依然安心当他的老师,具体原因
无人知晓,估计他的工资比较高。

但事实证明,正是这个奇怪的决定,导致了他奇特的人生。

万历二十七年(1599),孙承宗的雇主奉命前往大同,就任大同巡抚。官不能丢,孩子的教育也不能丢,于是孙承宗
跟着去了。

我记得,在一次访谈节目中,有一名罪犯说过:无论搞多少次普法教育,都是没用的,只要让大家都去监狱住两
天,亲自实践,就不会再犯罪了。

我同意这个说法,孙承宗应该也同意。

\section[\thesection]{}

在那个地方,孙承宗发现了一个陌生而又熟悉的世界,拼死的厮杀,血腥的战场,智慧的角逐,勇气的考验。

战争,是这个世界上最神秘莫测,最飘忽不定,最残酷,最困难,最考验智商的游戏。在战场上,兵法没有用,规
则没有用,因为在这里,最好的兵法,就是实战,唯一的规则,就是没有规则。

大同的孙老师没有实践经验,也无法上阵杀敌。然而一件事情的发生却足以证实,他已经懂得了战争。

在明代,当兵是一份工作,是工作,就要拿工资,拿不到工资,自然要闹。一般人闹,无非是堵马路,喊几句,当
兵的闹,就不同了,手里有家伙,要闹就往死里闹,专用名词叫做``哗变''。

这种事,谁遇上谁倒霉,大同巡抚运气不好,偏赶上了。有一次工资发得迟了点,当兵的不干,加上有人挑拨,于
是大兵们二话不说,操刀就奔他家去了。

巡抚大人慌得不行,里外堵得严严实实,门都出不去,想来想去没办法,寻死的心都有了。

关键时刻,他的家庭教师孙承宗先生出马了。

孙老师倒也没说啥,看着面前怒气冲冲,刀光闪闪的壮丽景象,他只是平静地说:

``饷银非常充足,请大家逐个去外面领取,如有冒领者,格杀勿论。''

士兵一哄而散。

把复杂的问题弄简单,是一个优秀将领的基本素质。

孙承宗的镇定、从容、无畏表明,他有能力,用最合适的方法,处理最纷乱的局势,应对最凶恶的敌人。

大同,在长达五年的时间里,孙承宗看到了战争,理解了战争,懂得了战争,并最终掌握了战争。他的掌握,来自
他的天赋、理论以及每一次感悟。

辽东,大他三岁的努尔哈赤正在讨伐女真哈达部的路上,此时的他,已经是一位精通战争的将领,他的精通,来自
于砍杀、冲锋以及每一次拼死的冒险。

两个天赋异禀的人,以他们各自不同的方式,进入了战争这个神秘的领域,并获知了其中的奥秘。

二十年后,他们将相遇,以实践来检验他们的天才与成绩。

相遇

万历三十二年(1604),孙承宗向他的雇主告别,踏上了前往京城的道路。他的目标,是科举。这一年,他四十二岁。

\section[\thesection]{}

经过几十年的风风雨雨,秀才、落魄秀才,教师、优秀教师、举人、军事观察员,目睹战争的破坏、聆听无奈的哀
嚎、体会无助的痛苦,孙承宗最终确定了自己的道路。

他决定放弃稳定舒适的生活,他决定,以身许国。

于是在几十年半吊子生活之后,考场老将孙承宗打算认真地考一次。

这一认真,就有点过了。

放榜的那天,孙承宗得知了自己的考试名次----第二,全国第二。

换句话说,他是榜眼。

按照明朝规定,榜眼必定是庶吉士,必定是翰林,于是在上岗培训后,孙承宗进入翰林院,成为了一名正七品编修。

之前讲过,明代朝廷是讲出身的,除个别特例外,要想进入内阁,必须是翰林出身,否则,即使你工作再努力,能
力再突出,也是白搭。这是一个公认的潜规则。

但请特别注意,要入内阁,必须是翰林,是翰林,却未必能入内阁。

毕竟翰林院里不只一个人,什么学士、侍读学士、侍讲、修撰、检讨多了去了,内阁才几个人,还得排队等,前面
的人死一个才能上一个,实在不易。

孙承宗就是排队等的人之一,他的运气不好,等了足足十年,都没结果。

第十一年,机会来了。

万历四十二年(1614),孙承宗调任詹事府谕德。

这是一个小官,却有着远大的前程,因为它的主要职责是给太子讲课。

从此,孙承宗成为了太子朱常洛的老师,在前方等待着他的,是无比光明的未来。

光明了一个月。

万历四十八年(1620),即位仅一个月的明光宗朱常洛去世。

但对于孙承宗而言,这没有什么影响,因为他已经找到了一个新的学生----朱由校。

教完了爹再教儿子,真可谓是诲人不倦。

天启皇帝朱由校这辈子没读过什么书,就好做个木工,所以除木匠师傅外,他对其它老师极不感冒。

孙承宗是唯一的例外。

由于孙老师长期从事儿童(私塾)教育,对于木头型,愚笨型、死不用功型的小孩,一向都有点办法,所以几堂课教
下来,皇帝陛下立即喜欢上了孙老师,他从没有叫过孙承宗的名字,而代以一个固定的称谓:``吾师''。

这个称呼,皇帝陛下叫了整整七年,直到去世为止。

他始终保持对孙老师的信任,无论何人,以何种方式,挑拨、中伤,都无济于事。

我说的这个``何人'',是指魏忠贤。

\section[\thesection]{}

正因为关系紧,后台硬,孙老师的仕途走得很快,近似于飞,一年时间,他就从五品小官,升任兵部尚书,进入内
阁,成为东阁大学士。

所以,当那封打小报告的信送上来后,天启才会找到孙承宗,征询他的意见。

可孙承宗同志的回答,却出乎皇帝的意料:

``我也不知如何决断。''

幸好后面还有一句:

``让我去看看吧。''

天启二年(1622),兵部尚书兼东阁大学士孙承宗来到山海关。

孙承宗并不了解王在晋,但到山海关和八里铺转了一圈后,他对王大人便有了一个直观且清晰的判断----这人是个
白痴。

他随即找来了王在晋,开始了一段在历史上极其有名的谈话。

在谈话的开头,气氛是和谐的,孙承宗的语气非常客气:

``你的新城建成之后,是要把旧城的四万军队拉过来驻守吗?''

王在晋本以为孙大人是来找麻烦的,没想到如此友善,当即回答:

``不是的,我打算再调集四万人来守城。''

但王大人并不知道,孙先生是当过老师的人,对笨人从不一棍子打死,总是慢慢地折腾:

``照你这么说,方圆八里之内,就有八万守军了,是吗?''

王大人还没回过味来,高兴地答应了一声:

``是的,没错啊。''

于是,张老师算帐的时候到了:

``只有八里,竟然有八万守军?你把新城修在旧城前面,那旧城前面的地雷、绊马坑,你打算让我们自己人去趟吗?!''

``新城离旧城这么近,如果新城守得住,还要旧城干什么?!''

``如果新城守不住,四万守军败退到旧城城下,你是准备开门让他们进来,还是闭关守城,看着他们死绝?!''

王大人估计被打懵了,半天没言语,想了半天,才憋出来一句话:

``当然不能开门,但可以让他们从关外的三道关进来,此外,我还在山上建好了三座军寨,接应败退的部队。''

这么蠢的孩子,估计孙老师还没见过,所以他真的发火了:

``仗还没打,你就准备接应败军?不是让他们打败仗吗?而且我军可以进入军寨,敌军就不能进吗?现在局势如此危
急,不想着恢复国土,只想着躲在关内,京城永无宁日!''

王同学彻底无语了。

\section[\thesection]{}

事实证明,孙老师是对的,如果新关被攻破,旧关必定难保,因两地只隔八里,逃兵无路可逃,只能往关里跑,到
时逃兵当先锋,努尔哈赤当后队,不用打,靠挤,就能把门挤破。

这充分说明,想出此计划的王在晋,是个不折不扣的蠢货。但聪明的孙老师,似乎也不是什么善类,他没有帮助迟
钝生王在晋的耐心,当即给他的另一个学生----皇帝陛下写了封信,直接把王经略调往南京养老去了。

赶走王在晋后,孙承宗想起了那封信,便向身边人吩咐了这样一件事:

``把那个写信批驳王在晋的人叫来。''

很快,他就见到了那个打上级小报告的人,他与此人彻夜长谈,一见如故,感佩于这个人的才华、勇气和资质。

这是无争议的民族英雄孙承宗,与有争议的民族英雄袁崇焕的第一次见面。

孙承宗非常欣赏袁崇焕,他坚信,这是一个必将震撼天下的人物,虽然当时的袁先生,只不过是个正五品兵备佥事。

事实上,王在晋并不是袁崇焕的敌人,相反,他一直很喜欢袁崇焕,还对其信任有加,但袁崇焕仍然打了他的小报
告,且毫不犹豫。

对于这个疑问,袁崇焕的回答十分简单:

``因为他的判断是错的,八里铺不能守住山海关。''

于是孙承宗问出了第二个问题:

``你认为,应该选择哪里?''

袁崇焕回答,只有一个选择。

然后,他的手指向了那个唯一的地点----宁远。

宁远,即今辽宁兴城,位居辽西走廊中央,距山海关二百余里,是辽西的重要据点,位置非常险要。

虽然几乎所有的人都认为,宁远很重要,很险要,但几乎所有的人也都认为,坚守宁远,是一个愚蠢的决定。

因为当时的明朝,已经丢失了整个辽东,手中仅存的只有山海关,关外都是敌人,跑出二百多里,到敌人前方去开
辟根据地,主动深陷重围,让敌人围着打,这不是勇敢,是缺心眼。

我原先也不明白,后来我去了一趟宁远,明白了。

宁远是一座既不大,也不起眼的城市,但当我登上城楼,看到四周地形的时候,才终于确定,这是个注定让努尔哈
赤先生欲哭无泪的地方。

因为它的四周三面环山,还有一面,是海。

\section[\thesection]{}

说宁远是山区,其实也不夸张。它的东边是首山,西边是窟窿山,中间的道路很窄,是个典型的关门打狗地形,努
尔哈赤先生要从北面进攻这里,是很辛苦的。

当然了,有人会说,既然难走,那不走总行了吧。

很可惜,虽然走这里很让人恶心,但不恶心是不行的,因为辽东虽大,要进攻山海关,必须从这里走。

此路不通让人苦恼,再加个别无他路,就只能去撞墙了。

是的,还会有人说,辽东都丢了,这里只是孤城,努尔哈赤占有优势,兵力很强,动员个几万人把城团团围住,光
是围城,就能把人饿死。

这是一个理论上可行的方案,仅仅是理论。

如果努尔哈赤先生这样做了,那么我可以肯定,最先被拖垮的一定是他自己。

因为宁远最让人绝望的地方,并不是山,而是海。

明朝为征战辽东,在山东登州地区修建了仓库,如遇敌军围城,船队就能将粮食装备源源不断地送到沿海地区,当
然也包括宁远。

而努尔哈赤先生,只能眼睁睁地看着这一切的发生,要知道,他的军队里,没有海军这个兵种。

更为重要的是,距离宁远不远的地方,有个觉华岛,在岛上有明军的后勤仓库,可以随时支援宁远。

之所以把仓库建在岛上,原因很简单,明朝人都知道,后金没有海军,没有翅膀,飞不过来。

但有些事,是说不准的。

上个月,我从宁远坐船,前往觉华岛(现名菊花岛),才发现,原来所谓不远,也挺远,海上走了半个多钟头才到。

上岸之后,宁远就只能眺望了,于是,我问了当地人一个问题:你们离陆地这么远,生活用品用船运很麻烦吧。

他回答:我们也用汽车拉,不麻烦。

然后补充一句:冬天,海面会结冰。

我又问:这么宽的海面(我估算了一下,大概有近十公里),都能冻住吗?

他回答:一般情况下,冻不住

接着还是补充:去年,冻住了。

去年,是2007年,冬天很冷。

于是,我想起了三百八十一年前,发生在这里的那场惊天动地的战争,我知道,那一年的冬天,也很冷。

学生

孙承宗接受了袁崇焕的意见,他决定,在宁远筑城。

筑城的重任,他交给了袁崇焕。

但要准备即将到来的战争,这些还远远不够,还有很多事情要做。

孙承宗最先做的一件事,就是练兵。

\section[\thesection]{}

当时他手下的士兵,总数有七万多人,数字挺大,但也就是个数,一查才发现,有上万人压根没有,都是空额,工
资全让老领导们拿走了。

这是假人,留下来的真人也不顶用,很多兵都是老兵油子,领饷时带头冲,打仗时带头跑,特别是关内某些地方的
兵,据说逃跑时的速度,敌人骑马都赶不上。

对于这批人,孙承宗用一个字就都打发了:滚。

他遣散了上万名撤退先锋,因为他已经找到了一个极具战斗力的群体----难民。

难民,就是原本住得好好的人,突然被人赶走,地被占了,房子被烧,老婆孩子被杀,求生不得,求死不能。让这
样的人去参军打仗,是不需要动员的。

孙承宗从难民中挑选了七千人,编入了自己的军队,四年后,他们的仇恨将成为战胜敌人的力量。

除此之外,他还做了很多事,大致如下:

修复大城九,城堡四十五;练兵十一万,训练弓弩、火炮手五万;立军营十二、水营五、火营二、前锋后劲营八;
造甲胄、军事器械、弓矢、炮石、渠答(守城的擂石)、卤盾等数万具。另外,拓地四百里;招集辽人四十余万,训
练辽兵三万;屯田五千顷,岁入十五万两白银。

具体细节不知道,看起来确实很多。

应该说,孙承宗所做的这些工作非常重要,但绝不是最重要的。

十七世纪最重要的是什么?是人才。

天启二年(1622),孙承宗已经六十岁了,他很清楚,虽然他熟悉战争,精通战争,有着挽救危局的能力,但他毕竟
老了。

为了大明江山,为了百姓的安宁,为了报国的理想,做了一辈子老师的孙承宗决定,收下最后一个学生,并把自己
的谋略、战法、无畏的信念,以及永不放弃希望的勇气,全部传授给他。

他很欣慰,因为他已经找到了一个合适的人选----袁崇焕。

在他看来,袁崇焕虽然不是武将出身(进士),也没怎么打过仗,但这是一个具备卓越军事天赋的人,能够在复杂形
势下,作出正确的判断。

更重要的是,他有着战死沙场的决心。

因为战场之上,求生者死,求死者生。

\section[\thesection]{}

在之后的时间里,他着力培养袁崇焕,巡察带着他,练兵带着他,甚至机密决策也都让他参与。

当然,孙老师除了给袁同学开小灶外,还让他当了班干部。从宁前兵备副使、宁前道,再到人事部(吏部)的高级预
备干部(巡抚),只用了三年。

袁崇焕用实际行动证明,他是个不折不扣的优等生。三年里,他圆满完成了自己的工作,并熟练掌握了孙承宗传授
的所有技巧、战术与战略。

在这几年中,袁崇焕除学习外,主要的工作是修建宁远城,加强防御,然而有一天,他突然意识到了一个问题:

后金军以骑兵为主,擅长奔袭,行动迅猛,抢了就能跑,而明军以步兵为主,骑兵质量又不行,打到后来,只能坚
守城池,基本上是敌进我退,敌退我不追,这么下去,到哪儿才是个头?

是的,防守是不够的,仅凭城池、步兵坚守,是远远不够的。

彻底战胜敌人强大骑兵唯一方式,就是建立一支同样强大的骑兵。

所以,在孙老师的帮助下,他开始召集难民,仔细挑选,进行严格训练,只有最勇猛精锐,最苦大仇深的士兵,才
有参加这支军队的权力。

同时,他饲养优良马匹,大量制造明朝最先进的火器三眼神铳,配发到每个人的手中,并反复操练骑兵战法,冲刺
砍杀,一丝不苟。

因为他所需要的,是这样一支军队:无论面临绝境,或是深陷重围,这支军队都能够战斗到最后一刻,绝不投降。

他成功了。

他最终训练出了一支这样的军队,一支努尔哈赤、皇太极父子终其一生,直至明朝灭亡,也未能彻底战胜的军队。

在历史上,这支军队的名字,叫做关宁铁骑。

袁崇焕的成长,远远超出了孙承宗的预料,无论是练兵、防守、战术,都已无懈可击。虽然此时,他还只是个无名
小卒。

对这个学生,孙老师十分满意。

但他终究还是发现了袁崇焕的一个缺点,一个看似无足轻重的缺点,从一件看似无足轻重的小事上。

天启三年(1623),辽东巡抚阎鸣泰接到举报,说副总兵杜应魁冒领军饷。

要换在平时,这也不算是个事,但孙老师刚刚整顿过,有人竟然敢顶风作案,必须要严查。

于是他派出袁崇焕前去核实此事。

\section[\thesection]{}

袁崇焕很负责任,到地方后不眠不休,开始查账清人数,一算下来,没错,杜总兵确实贪污了,叫来谈话,杜总兵
也认了。

按规定,袁特派员的职责到此结束,就该回去报告情况了。

可是袁大人似乎太过积极,谈话刚刚结束,他竟然连个招呼都不打,当场就把杜总兵给砍了,被砍的时候,杜总兵
还在做痛哭流涕忏悔状。

事发太过突然,在场的人都傻了,等大家回过味来,杜总兵某些部下已经操家伙,准备奔着袁大人去了。

毕竟是朝廷命官,你又不是直属长官,啥命令没有,到地方就把人给砍了,算是怎么回事?

好在杜总兵只是副总兵,一把手还在,好说歹说,才把群众情绪安抚下去,袁特派员这才安然返回。

返回之后的第一个待遇,是孙承宗的一顿臭骂:

``杀人之前,竟然不请示!杀人之后,竟然不通报!士兵差点哗变,你也不报告!到现在为止,我还不知道,你到底杀
了什么人!以何理由要杀他!''

``据说你杀人的时候,只说是奉了上级的命令,如果你凭上级的命令就可以杀人,那还要尚方宝剑(皇帝特批孙承宗
一柄)干什么?!''

袁崇焕没有吱声。

就事情本身而言,并不大,却相当恶劣,既不是直系领导,又没有尚方宝剑,竟敢擅自杀人,实在太过嚣张。

但此刻人才难得,为了这么个事,把袁崇焕给办了,似乎也不现实,于是孙承宗把这件事压了下去,他希望袁崇焕
能从中吸取教训:意气用事,胡乱杀人,是绝对错误的。

事后证明,袁崇焕确实吸取了教训,当然,他的认识和孙老师的有所不同:不是领导,没有尚方宝剑,擅自杀人,
是不对的,那么是领导,有了尚方宝剑,再擅自杀人,就该是对的。

从某个角度讲,他这一辈子,就栽在这个认识上。

不过局部服从整体,杜总兵死了也就死了,无所谓,事实上,此时辽东的形势相当的好,宁远以及附近的松山、中
前所、中后所等据点已经连成了一片,著名的关宁防线(山海关----宁远)初步建成,驻守明军已达十一万人,粮食
可以供应三年以上,关外两百多公里土地重新落入明朝手中。

孙承宗修好了城池、整好了军队,找好了学生,恢复了国土,但这一切还不够。

要应对即将到来的敌人,单靠袁崇焕是不行的,必须再找几个得力的助手。

\section[\thesection]{}

助手

袁崇焕刚到宁远时,看到的是破墙破砖,一片荒芜,不禁感叹良多。

然而很快就有人告诉他,这是刚修过的,事实上,已有一位将领在此筑城,而且还筑了一年多。

修了一年多,就修成这个破样,袁崇焕十分恼火,于是他把这个人叫了过来,死骂了一顿。

没想到,这位仁兄全然没有之前被砍死的那位杜总兵的觉悟,非但不认错,竟然还跳起来,跟袁大人对骂,张口就
是老子打了多少年仗,你懂个屁之类的混话。

这就是当时的懒散游击将军,后来的辽东名将祖大寿的首次亮相。

祖大寿,是一个很有名的人,有名到连在他家干活的仆人祖宽都进了明史列传,然而这位名人本人的列传,却在清
史稿里,因为他最终还换了老板。

但奇怪的是,和有同样遭遇的吴某某、尚某某、耿某某比起来,他的名声相当好,说他是X奸的人,似乎也不多。原
因在于,几乎所有的人都认为,他已尽到了自己的本分。

祖大寿,字复宇,辽东宁远人,生在宁远,长在宁远,参军还在宁远。此人脾气暴躁,品性凶狠,好持刀砍人,并
凭借多年砍人之业绩,升官当上了游击,熊廷弼在的时候很赏识他。

后来熊廷弼走了,王化贞来了,也很赏识他,并且任命他为中军游击,镇守广宁城。

再后来,孙得功叛乱,王化贞逃跑了,关键时刻,祖大寿二话不说,也跑了。但他并没有跑回去,而是率领军队跑
到了觉华岛继续坚守。

坚守原则,却不吃眼前亏,从后来十几年中他干过的那些事来看,这是他贯彻始终的人生哲学。

对一个在阎王殿参观过好几次的人而言,袁崇焕这种进士出身,连仗都没打过的人,竟然还敢跑来抖威风,是纯粹
的找抽,不骂是不行的。

这场对骂的过程并不清楚,但结果是明确的,袁大人虽然没当过兵,脾气却比当兵的更坏,正如他的那句名
言:``你道本部院是个书生,本部院却是一个将首!''双方你来我往,几个回合下来,祖大寿认输了。

从此,他成为了袁崇焕的忠实部下,大明的优秀将领,后金骑兵不可逾越的铜墙铁壁。

祖大寿,袁崇焕的第一个助手。

\section[\thesection]{}

其实祖大寿这个名字,是很讨巧的,因为用当地口音,不留神就会读成祖大舅。为了不至于乱辈分,无论上级下
属,都只是称其职务,而不呼其姓名。

只有一个人,由始至终、坚定不移地称其为大舅,原因很简单,祖大寿确实是他的大舅。

这个人名叫吴三桂。

当时的吴三桂不过十一二岁,尚未成年,既然未成年,就不多说了。事实上,在当年,他的父亲吴襄,是一个比他
重要得多的人物。

吴襄,辽宁绥中人,祖籍江苏高邮,武举人。

其实按史料的说法,吴襄先生的祖上,本来是买卖人,从江苏跑到辽东,是来做生意的。可是到他这辈,估计是兵
荒马乱,生意不好做了,于是一咬牙,去考了武举,从此参加军队,迈上了丘八的道路。

由于吴先生素质高,有文化(至少识字吧),和兵营里的那些傻大粗不一样,祖大寿对其比较赏识,刻意提拔,还把
自己的妹妹嫁给了他。

吴襄没有辜负祖大寿的信任,在此后十余年的战斗中,他和他的儿子,将成为大明依靠的支柱。

吴襄,袁崇焕的第二个助手。

在逃到宁远之前,吴襄和祖大寿是王化贞的下属,在王化贞到来之前,他们是毛文龙的下属。

现在看来,毛文龙,似乎并不有名,也不重要,但在当时,他是个非常有名,且极其重要的人,至少比袁崇焕要重
要得多。

天启初年的袁崇焕,是宁前道,毛文龙,是皮岛总兵。

准确地说,袁崇焕,是宁前地区镇守者,朝廷四品文官。

而毛文龙,是左都督、朝廷一品武官、平辽将军、尚方宝剑的持有者、辽东地区最高级别军事指挥官。

换句话说,毛总兵比袁大人要大好几级,与毛文龙相比,袁崇焕只是一个微不足道的无名小卒,双方根本就不在同
一档次上。

因为毛总兵并不是一个普通的总兵。

明代总兵,是个统称,大致相当于司令员,但管几个省的,可以叫司令员,管一个县的,也可以叫司令员。比如,
那位吃空额贪污的杜应魁,人家也是个副总兵,但袁特派说砍,就把他砍了,眼睛都不眨,检讨都不写。

总而言之,明代总兵是分级别的,有分路总兵、协守总兵等等,而最高档次的, 是总镇总兵。

毛文龙,就是总镇总兵,事实上,他是大明在关外唯一的总镇级总兵。

\section[\thesection]{}

总镇总兵,用今天的话说,是大军区司令员,地位十分之高,一般都附带将军头衔(相当于荣誉称号,如平辽、破虏
等),极个别的还兼国防部长(兵部尚书)。

明朝全国的总镇总兵编制,有二十人,十四个死在关内,现存六人,毛文龙算一个。

但在这些幸存者之中,毛总兵是比较特别的,虽然他的级别很高,但他管的地盘很小----皮岛,也就是个岛。

皮岛,别名东江,位处鸭绿江口,位置险要,东西长十五里,南北宽十二里,毛总兵就驻扎在上面,是为毛岛主。

这是个很奇怪的事,一般说来,总镇总兵管辖的地方很大,不是省军区司令,也是地区军区司令,只有毛总兵,是
岛军区司令。

但没有人觉得奇怪,因为其他总兵的地盘,是接管的,毛总兵的地盘,是自己抢来的。

毛文龙,万历四年(1576)生人,浙江杭州人,童年的主要娱乐是四处蹭饭吃。

由于家里太穷,毛文龙吃不饱饭,自然上不起私塾,考不上进士。而就我找到的史料看,他似乎也不是斗狠的主,
打架撒泼的功夫也差点,不能考试,又不能闹腾,算是百无一用,比书生还差。

但要说他什么都没干,那也不对,为了谋生,他开始从事服务产业----算命。

算命是个技术活,就算真不懂,也要真能忽悠,于是毛文龙开始研究麻衣相术、测字、八卦等等。

但我们有理由相信,他在这方面的学问没学到家,给人家算了几十年的命,就没顾上给自己算一卦。不过,他在另
一方面的造诣,是绝对值得肯定的--兵法。

在平时只教语文,考试只考作文的我国古代,算命、兵法、天文这类学科都是杂学,且经常扎堆,还有一个莫名其
妙的统称----阴阳学。

而迫于生计,毛先生平时看的大都是这类杂书,所以他虽没上过私塾,却并非没读过书。据说他不但精通兵法理
论,还经常用于实践----聊天时用来吹牛。

就这么一路算,一路吹,混到了三十岁。

不知是哪一天,哪根弦不对,毛文龙突然决定,结束自己现在的生活,毅然北上寻找工作。

他一路到了辽东,遇见当时的巡抚王化贞,王化贞和他一见如故,认为他是优秀人才,当即命他为都司,进入军队
任职。

\section[\thesection]{}

这个世界上似乎没有这样的好事,没错,前面两句话是逗你们玩的。

毛文龙先生之所以痛下决心北上求职,是因为他的舅舅时来运转,当上了山东布政使,跟王化贞关系很好,并向王
巡抚推荐了自己的外甥。

王巡抚给了面子,帮毛文龙找了份工作,具体情况就是如此。

在王化贞看来,给安排工作,是挣了毛文龙舅舅的一个人情,但事实证明,办这件事,是挣了大明的一个人情。

毛文龙就这样到部队上班了,虽说只是个都司,但在地方而言,也算是高级干部了,至少能陪县领导吃饭,问题在
于,毛都司刚去的时候,不怎么吃得开,因为大家都知道他是关系户,都知道他没打过仗,所以,都瞧不起他。

直到那一天的到来----天启元年(1621)三月二十一日。

这一天,辽阳陷落,辽东经略袁应泰自尽,数万守军全军覆没,至此,广宁之外,明朝在辽东已无立足之地。难民
携家带口,士兵丢弃武器,大家纷纷向关内逃窜。

除了毛文龙。

毛文龙没有跑,但必须说明的是,他之所以不跑,不是道德有多高尚,而是实在跑不掉了。

由于辽阳失陷太快,毛先生反应不够快,没来得及跑,落在了后面,被后金军堵住,没辙了。

如果只有他一个人,化化妆,往脸上抹把土,没准还能顺过去。不幸的是,他的手下还有两百来号士兵。

带着这么群累赘,想溜,溜不掉;想打,打不过。明军忙着跑,后金军忙着追,敌人不管他,自己人也不管他。毛
文龙此时的处境,可以用一个词完美地概括----弃卒。

当众人一片哀鸣,认定走投无路之际,毛文龙找到了一条路----下海。

他找来了船只,将士兵们安全撤退到了海上。

然而很快,士兵们就发现,他们行进的方向不是广宁,更不是关外。

``我们去镇江。''毛文龙答。

于是大家都傻了。

所谓镇江,不是江苏镇江,而是辽东的镇江堡,此地位于鸭绿江入海口,与朝鲜隔江而立,战略位置十分重要,极
其坚固,易守难攻。

但大家之所以吃惊,不是由于它很重要,很坚固,而是因为它压根就不在明朝手里。

辽阳、沈阳失陷之前,这里就换地主了,早就成了后金的大后方,且有重兵驻守,这个时候去镇江堡,动机只有两
个:投敌,或是找死。

然而毛文龙说,我们既不投敌,也不寻死,我们的目的,是攻占镇江。

\section[\thesection]{}

很明显,这是在开玩笑,辽阳已经失陷了,没有人抵抗,没有人能够抵抗。大家的心中,有着共同且唯一的美好心
愿----逃命。

但是毛文龙又说,我没有开玩笑。

我们要从这里出发,横跨海峡,航行上千里,到达敌人重兵集结的坚固堡垒,凭借我们这支破落不堪、装备不齐、
刚刚一败涂地,只有几百人的队伍,去攻击装备精良、气焰嚣张、刚刚大获全胜的敌人,以寡敌众。

我们不逃命,我们要攻击,我们要彻底地击败他们,我们要收复镇江,收复原本属于我们的土地!

没有人再惊讶,也没有人再反对,因为很明显,这是一个合理的理由,一个足以让他们前去攻击镇江,义无反顾的
理由。

在夜幕的掩护下,毛文龙率军抵达了镇江堡。

事实证明,他或许是个冲动的人,但绝不是个愚蠢的人,如同预先彩排的一样,毛文龙发动了进攻,后金军队万万
想不到,在大后方竟然还会被人捅一刀,没有丝毫准备,黑灯瞎火的,也不知到底来了多少人,从哪里来,只能惊
慌失措,四散奔逃。

此战明军大胜,歼灭后金军千余人,阵斩守将佟养真,收复镇江堡周边百里地域,史称``镇江堡大捷''。

这是自努尔哈赤起兵以来,明朝在辽东最大,也是唯一的胜仗。

消息传来,王化贞十分高兴,当即任命毛文龙为副总兵,镇守镇江堡。

后金丢失镇江堡后,极为震惊,派出大队兵力,打算把毛文龙赶进海里喂鱼。

由于敌太众,我太寡,毛文龙丢失了镇江堡,被赶进了海里,但他没有喂鱼,却开始钓鱼----退守皮岛。

毕竟只是个岛,所以刚开始时,谁也没把他当回事,可不久之后,他就用实际行动,让努尔哈赤先生领会了痛苦的
真正含义。

自天启元年以来,毛文龙就没休息过,每年派若干人,出去若干天,干若干事,不是放火,就是打劫,搞得后金不
得安生。

更烦人的是,毛岛主本人实在狡猾无比,你没有准备,他就上岸踢你一脚,你集结兵力,设好埋伏,他又不来,就
如同耳边嗡嗡叫的蚊子,能把人活活折磨死。

后来努尔哈赤也烦了,估计毛岛主也只能打打游击,索性不搭理他,让他去闹,没想到,毛岛主又给了他一个意外
惊喜。

\section[\thesection]{}

天启三年(1623),就在后金军的眼皮底下,毛岛主突然出兵,一举攻占金州(今辽宁金州),而且占住就不走了,在
努尔哈赤的后院放了把大火。

努尔哈赤是真没法了,要派兵进剿,却是我进敌退,要登陆作战,又没有那个技术,要打海战,又没有海军,实在
头疼不已。

努尔哈赤是越来越头疼,毛岛主却越来越折腾,按电视剧里的说法,住孤岛上应该是个很惨的事,要啥啥没有,天
天坐在沙滩上啃椰子,眼巴巴盼着人来救。

可是毛文龙的孤岛生活过得相当充实,照史书上的说法,是``召集流民,集备军需,远近商贾纷至沓来,货物齐备
捐税丰厚。''

这就是说,毛岛主在岛上搞得很好,大家都不在陆地上混了,跟着跑来讨生活,岛上的商品经济也很发达,还能抽
税。

这还不算,毛岛主除了搞活内需外,还做进出口贸易,日本、朝鲜都有他的固定客商,据说连后金管辖区也有人和
他做生意,反正那鬼地方没海关,国家也不征税,所以毛岛主的收入相当多,据说每个月都有十几万两白银。

有钱,自然就有人了,在高薪的诱惑下,上岛当兵的越来越多,原本只有两百多,后来袁崇焕上岛清人数时,竟然
清出了三万人。

值得夸奖的是,在做副业的同时,毛岛主没有忘记本职工作,在之后的几年中,他创造了很多业绩,摘录如下:

(天启)三年,文龙占金州。

四年五月,文龙遣将沿鸭绿江越长白山,侵大清国东偏。

八月,遣兵从义州城西渡江,入岛中屯田。

五年六月,遣兵袭耀州之官屯寨。

六年五月,遣兵袭鞍山驿,越数日又遣兵袭撤尔河,攻城南。

乱打一气不说,竟然跑到人家地面上屯田种粮食,实在太嚣张了。

努尔哈赤先生如果不恨他,那是不正常的。

可是恨也白恨,科技跟不上,只能眼睁睁看着毛岛主胡乱闹腾。

拜毛文龙同志所赐,后金军队每次出去打仗的时候,很有一点惊弓之鸟的感觉,唯恐毛岛主在背后打黑枪,以至于
长久以来不能安心抢掠,工作精力和情绪受到极大影响,反响极其恶劣。

如此成就,自然无人敢管,朝廷哄着他,王化贞护着他,后来,王在晋接任了辽东经略,都得把他供起来。

毛文龙,袁崇焕的第三个帮助者,现在的上级、未来的敌人。

\section[\thesection]{}

天启三年(1623),袁崇焕正热火朝天地在宁远修城墙的时候,另一个人到达宁远。

这个人是孙承宗派来的,他的职责,是与袁崇焕一同守护宁远。这个人的名字叫满桂。

满桂,宣府人,蒙古族。很穷,很勇敢。

满桂同志应该算是个标准的打仗苗子,从小爱好打猎。长大参军了,就爱好打人,在军队中混了很多年,每次出去
打仗,都能砍死几个,可谓战功显赫,然而战功如此显赫,混到四十多岁,才是个百户。

倒不是有人打压他,实在是因为他太实在。

明朝规定,如果你砍死敌兵一人(要有首级),那么恭喜你,接下来你有两种选择,一、升官一级。二、得赏银五十
两。

每次满桂都选第二种,因为他很缺钱。

我不认为满桂很贪婪,事实上,他很老实。

因为他并不知道,选第二种的人,能拿钱,而选第一种的,既能拿权,也能拿钱。

就这么个混法,估计到死前,能混到个千户,就算老天开眼了。

然而数年之后一个人的失败,造就了他的成功,这个失败的人,是杨镐。

万历四十七年(1619),杨镐率四路大军,在萨尔浒全军覆没,光将领就死了三百多人,朝廷没人了,只能下令破格
提拔,满桂同志就此改头换面,当上了明军的高级将领----参将。

但真正改变他命运的,是另一个成功的人--孙承宗。

天启二年(1622),在巡边的路上,孙承宗遇见了满桂,对这位老兵油子极其欣赏(大奇之),高兴之余,就给他升
官,把他调到山海关,当上了副总兵,一年后,满桂被调往宁远,担任守将。

满桂是一个优秀的将领,他不但作战勇敢,而且经验丰富,还能搞外交。

当时的蒙古部落,已经成为后金军队的同盟,无论打劫打仗都跟着一起来,明军压力很大,而满桂的到来彻底改变
了这一切。

他利用自己的少数民族身份,对同胞进行了长时间耐心的劝说,对于不听劝说的,也进行了长时间耐心的攻打。很
快,大家就被他又打又拉的诚恳态度所感动,全都服气了(桂善操纵,诸部咸服)。

此外,他很擅长堆砖头,经常亲自监工砌墙,还很喜欢练兵,经常把手下的兵练得七荤八素。

\section[\thesection]{}

就这样,在满桂的不懈努力下,宁远由当初一座较大的废墟,变成了一座较大的城市(军民五万余家,屯种远至五十
里)。

而作为宁远地区的最高武官,他与袁崇焕的关系也相当好。

其实矛盾还是有的,但问题不大,至少当时不大。

必须说明一点,满桂当时的职务,是宁远总兵,而袁崇焕,是宁前道。就级别而言,满桂比袁崇焕要高,但明朝的
传统,是以文制武,所以在宁远,袁崇焕的地位要略高于满桂,高一点点。

而据史料记载,满桂是个不苟言笑,却极其自负的人。加上他本人是从小兵干起,平时干的都是砍人头的营生(一个
五十两),注重实践,最看不起的,就是那些空谈理论,没打过仗的文官,当然,这其中也包括袁崇焕。

但有趣的是,他和袁崇焕相处得还不错,并不是他比较大度,而是袁崇焕比较能忍。

袁大人是很有自知之明的。他很清楚,在辽东混的,大部分都是老兵油子,杀人放火的事情干惯了,在这些人看
来,自己这种文化人兼新兵蛋子,是没有发言权的。

所以他非常谦虚,非常能装孙子,还时常向老前辈们(如满桂)虚心请教,满桂们也心知肚明,知道他是孙承宗的
人,得罪不起,都给他几分面子。总之,大家混得都还不错。

满桂,袁崇焕的第四个帮助者,三年后的共经生死的战友,七年后置于死地的对手。

或许你觉得人已经够多了,可是孙承宗似乎不这么看,不久之后,他又送来了第五个人。

这个人,是他从刑场上救下来的,他的名字叫赵率教。

赵率教,陕西人,此人当官很早,万历中期就已经是参将了,履历平平,战功平平,资质平平,什么都平平。

表现一般不说,后来还吃了官司,工作都没了。后来也拜杨镐先生的福,武将死得太多没人补,他就自告奋勇,去
补了缺,在袁应泰的手下,混了个副总兵。

可是他的运气很不好,刚去没多久,辽阳就丢了,袁应泰自杀,他跑了。

情急之下,他投奔了王化贞,一年后,广宁失陷,王化贞跑了,他也跑了。

再后来,王在晋来了,他又投奔了王在晋。

由于几年之中,他到了好几个地方,到哪,哪就倒霉,且全无责任心,遇事就跑,遇麻烦就溜,至此,他终于成为
了明军之中有口皆碑的典型人物----反面典型。

\section[\thesection]{}

对此,赵率教没有说什么,也不能说什么。

然而不久后,赵率教突然找到了王在晋,主动提出了一个要求:

``我愿戴罪立功,率军收复失地。''

王在晋认为,自己一定是听错了,然而当他再次听到同样坚定的话时,他认定,赵率教同志可能是受了什么刺激。

因为在当时,失地这个概念,是比较宽泛的,明朝手中掌握的,只有山海关,往大了说,整个辽东都是失地,您要
去收复哪里?

赵率教回答:前屯。

前屯,就在宁远附近,是明军的重要据点。

在确定赵率教头脑清醒,没有寻死倾向之后,王在晋也说了实话:

``收复实地固然是好,但眼下无余兵。''

这就很实在了,我不是不想成全你,只是我也没法。

然而赵率教的回答彻底出乎了王大人的意料:

``无需派兵,我自己带人去即可。''

老子是辽东经略,手下都没几号人,你还有私人武装?于是好奇的王在晋提出了问题:

``你有多少人?''

赵率教答:

``三十八人。''

王在晋彻底郁闷了,眼下大敌当前,努尔哈赤随时可能打过来,士气如此低落,平时能战斗的,也都躲了,这位平
时特别能躲的,却突然站出来要战斗?

这都啥时候了,你开什么玩笑?还嫌不够乱?

于是一气之下,王在晋手一挥:你去吧!

这是一句气话,可他万没想到,这哥们真去了。

赵率教率领着他的家丁,三十八人,向前屯进发,去收复失地。

这是一个有明显自杀迹象的举动,几乎所有的人都认为,赵率教疯了。

但事实证明,赵先生没有疯,因为当他接近前屯,得知此地有敌军出现时,便停下了脚步。

``前方已有敌军,不可继续前进,收复此地即可。''

此地,就是他停下的地方,名叫中前所。

中前所,地处宁远近郊,大致位于今天的辽宁省绥中县附近,赵率教在此扎营,就地召集难民,设置营地,挑选精
壮充军,并组织屯田。

王在晋得知了这个消息,却只是轻蔑地笑了笑,他认为,在那片遍布敌军的土地上,赵率教很快会故伎重演,丢掉
一切再跑回来。

几个月后,孙承宗来到了这个原本应该空无一人的据点,却看见了广阔的农田、房屋,以及手持武器、训练有素的
士兵。

\section[\thesection]{}

在得知此前这里只有三十八人后,他找来了赵率教,问了他一个问题:

``现在这里有多少人?''

赵率教回答:

``民六万有余,士兵上万人。''

从三十八,到六万,面对这个让人难以置信的奇迹,孙承宗十分激动,他老人家原本是坐着马车来的,由于过于激
动,当即把车送给了赵率教,自己骑马回去了。

从此,他记住了这个人的名字。

就赵率教同志的表现来看,他是一个知道羞耻的人,知耻近乎勇,在经历了无数犹豫、困顿后,他开始用行动,去
证明自己的勇气。

可他刚证明到一半,就差点被人给砍了。

正当赵率教撩起袖子,准备大干一场的时候,兵部突然派人来找他,协助调查一件事情。

赵率教明白,这回算活到头了。

事情是这样的,当初赵率教在辽阳的时候,职务是副总兵,算是副司令员,掌管中军,这就意味着,当战争开始
时,手握军队主力的赵率教应全力作战,然而他逃了,并直接导致了作战失败。

换句话说,小兵可以跑,老百姓可以跑,但赵率教不能跑,也不应该跑,既然跑了,就要依法处理,根据明朝军
法,此类情形必死无疑。

但所谓必死无疑,还是有疑问的,特别是当有猛人求情的时候。

孙承宗听说此事后,当即去找了兵部尚书,告诉他,此人万不可杀,兵部尚书自然不敢得罪内阁大学士,索性做了
个人情,把赵率教先生放了。

孙承宗并不是一个仁慈的人,他之所以放赵率教一马,是因为他认定,这人活着比死了好。

而赵率教用实际行动,证明了孙承宗的判断,在不久后的那场大战中,他将起到至关重要的作用。

赵率教,袁崇焕的第五个帮助者。

惊变

天启元年(1620),孙承宗刚到辽东的时候,他所有的,只是山海关以及关外的八里地。

天启五年(1624),孙承宗巩固了山海关,收复了宁远,以及周边几百里土地。

在收复宁远之后,孙承宗决定再进一步,占据另一个城市--锦州。他认定,这是一个至关重要的地点。

但努尔哈赤似乎不这么看,锦州嘛,又小又穷,派兵守还要费粮食,谁要谁就拿去。

就这样,不费吹灰之力,孙承宗得到了锦州。

事后证明,自明朝军队进入锦州的那一刻起,努尔哈赤的悲惨命运便已注定。

因为至此,孙承宗终于完成了他一生中最伟大的杰作----关锦防线。

\section[\thesection]{}

所谓关锦防线,是指由山海关----宁远----锦州组成的防御体系,该防线全长四百余里,深入后金区域,沿线均有
明朝堡垒、据点,极为坚固。

历史告诉我们,再坚固的防线,也有被攻陷的一天。

历史还告诉我们,凡事总有例外,比如这条防线。

事实上,直到明朝灭亡,它也未被突破。此后长达十余年时间里,后金军队用手刨,用嘴啃,用牙咬,都毫无效
果,还搭上了努尔哈赤先生的一条老命。

这是一个科学、富有哲理而又使人绝望的防御体系,因为它基本上没有弱点。

锦州,辽东重镇,自古为入关要道,且地势险要,更重要的是,锦州城的一面,靠海。对于没有海军的后金而言,
这又是一个噩梦。

这就是说,只要海运充足,在大多数情况下,即使被围得水泄不通,锦州也是很难攻克的。

既然难打,能不能不打呢?

不能。

我的一位住在锦州的朋友告诉我,他要回家十分方便,因为从北京出发,开往东三省,在锦州停靠的火车,有十八
辆。

我顿时不寒而栗,这意味着,三百多年前的明朝,要前往辽东,除个别缺心眼爬山坡的人外,锦州是唯一的选择。

要想入关,必须攻克宁远,要攻克宁远,必须攻克锦州,要攻克锦州,攻克不了。

当然,有人会说,锦州不过是个据点,何必一定要攻陷?只要把锦州围起来,借个道过去,继续攻击宁远,不就行了
吗?

是的,按照这个逻辑,也不一定要攻陷宁远,只要把宁远围起来,借个道过去,继续攻击山海关,不也行吗?

这样看来,努尔哈赤实在太蠢了,这么简单的道理,为什么就没想到呢?我觉得,持有这种想法的人,应该去洗把
脸,清醒清醒。

假定你是努尔哈赤,带了几万兵,到了锦州,锦州没人打你,于是,你又到了宁远,宁远也没人打你,就这么一路
顺风到了山海关,准备发动攻击。

我相信,这个时候你会惊喜的发现,锦州和宁远的军队已经出现在你的后方,准备把你一锅端----除非这两地方的
守将是白痴。

现在你有大麻烦了,眼前是山海关,没准十天半月攻不下来,请屁股后面的军队别打你,估计人家不干,就算你横
下一条心,用头把城墙撞破,冲进了关内,抢到了东西,你也总得回去吧。

\section[\thesection]{}

如果你没长翅膀,你回去的路线应该是山海关----宁远----锦州……

看起来似乎比较艰难,不是吗?

这就是为什么曹操同志多年来不怕孙权,不怕刘备,偏偏就怕马腾、马超--这两位先生的地盘在他的后方。

这就是孙承宗的伟大成就,短短几年之间,他修建了若干据点,收复了若干失地,提拔了若干将领,训养了若干士
兵。

现在,在他手中的,是一条坚不可破的防线,一支精锐无比的军队,一群天赋异禀的卓越将领。但对于这一切,努
尔哈赤并不清楚,至少不十分清楚。

祖大寿、吴襄、满桂、赵率教、毛文龙以及袁崇焕,对努尔哈赤而言,这些名字毫无意义。

自万历四十六年起兵以来,明朝能打的将领,他都打了,杨镐、刘綎、杜松、王化贞、袁应泰,全都是手下败将,
无一例外,在他看来,新来的这拨人下场估计也差不多。

但他终将失败,败在这几个无名小卒的手中,并永远失去翻盘的机会。

话虽如此,努尔哈赤还是很有几把刷子的,他不了解目前的局势,却了解孙承宗的实力,很明显,这位督师大人比
熊廷弼还难对付,所以几年之内,他都没有发动大的进攻。

大的没有,小的还是有。

在后金的军队中,最优秀的将领无疑是努尔哈赤,但正如孙承宗一样,他的属下,也有很多相当厉害的猛人。

而在这些猛人里,最猛的,就是八大贝勒。

所谓八大贝勒,分别是指代善、阿敏、莽古尔泰、皇太极、阿济格、多尔衮、多铎、济尔哈朗。

在这八个人里,按照军功和资历,前四个大猛,故称四大贝勒,后四个小猛,故称四小贝勒。

其中最有名的,无疑是两个人,皇太极、多尔衮。

但最能打仗的,是三个人,除皇太极和多尔衮外,还有一个代善。

多尔衮年纪还小,就不说了,皇太极很有名,也不说了,这位代善,虽然年纪很大,且不出名,但很有必要说一说。

事实上,大贝勒代善是当时后金最为杰出的军事将领之一,此人非常勇猛,在与明朝作战时,经常身先士卒,且深
通兵法,擅长伏击,极其能打。

因为他很能打,所以努尔哈赤决定,挑选一个目标,由代善发动攻击,以试探孙承宗的虚实,而他选定的这个目
标,就是锦州。

\section[\thesection]{}

当代善率军来到锦州城下的时候,他才意识到,这是个结结实实的黑锅。

首先锦州非常坚固。在修城墙方面,孙承宗很有一套,城不但高,而且厚,光凭刀砍斧劈,那是没指望的,要想进
城,没有大炮是不行的。

大炮也是有的,不过不在城下,而在城头。

其实一直以来,明朝的火器水平相当高。万历三大征打日本的时候也很经用,后来之所以荒废,不是技术问题,而
是态度问题。

万历前期,皇帝陛下精神头足,什么事都愿意折腾,后来不想干了,天天躲着不上朝,下面也开始消极怠工。外加
火器工作危险性大,吃力不讨好,没准出个安全事故,是很麻烦的。

孙承宗不怕麻烦,他不但为部队添置三眼火铳等先进装备,还购置了许多大炮,尝试用火炮守城。而锦州,就是他
的试点城市。

虽然情况不妙,但代善不走寻常路,也不走回头路,依然一根筋,找人架云梯、冲车往城里冲。

此时的锦州守将,是赵率教。应该说,他的作战态度是很成问题的,面对着在城下张牙舞爪,极其激动的代善,他
却心平气和,毫不激动,时不时在城头转两圈,放几炮,城下便会迅速传来凄厉的惨叫声,在赔上若干架云梯,若
干条性命,却毫无所得的情况下,代善停止进攻。

虽然停止进攻,但代善还不大想走,他还打算再看两天。

可是孙承宗似乎是不欢迎参观的,代贝勒的屁股还没坐热,就得到一个可怕的消息,一支明军突然出现在自己的侧
翼。

这支部队是驻守前屯、松山的明军,听说客人来了,没赶上接风,特来送行。

在短暂慌乱之后,代善恢复了平静,作为一名经验丰富的将领,他有信心击退这支突袭部队。

可他刚带队发起反击,就看到自己屁股后面烟尘四起:城内的明军出动了。

这就算是腹背受敌了,但代善依然很平静,作为一名经验丰富的将领,他很有信心。

然后,很有信心的代善又得知了另一个消息----宁远、中前所等地的明军已经出动,正朝这边来,吃顿饭的功夫也
就到了。

但代善不愧是代善,作为一名经验丰富的将领,他非常自信,镇定地做出了一个英明的判断:快逃。

\section[\thesection]{}

可是来去自如只是一个幻想,很快代善就发现,自己已经陷入重围。明军毫不客气,一顿猛打,代善部伤亡十分惨
重。好在来的多是骑兵,机动力强,拼死往外冲,总算奔出了条活路,一口气跑上百里,直到遇见接他的二贝勒阿
敏,魂才算漂回来。

此战明军大胜,击溃后金军千余人,战后清点斩获首级六百多颗,努尔哈赤为他的试探付出了惨痛的代价。

在孙承宗督师辽东的几年里,双方很有点相敬如宾的意思,虽说时不时搞点小摩擦,但大仗没打过,孙承宗不动,
努尔哈赤不动。

可是孙承宗不动是可以的,努尔哈赤不动是不行的。

因为孙大人的任务是防守,只要不让敌人进关抢东西,他就算赢了。

努尔哈赤就不同了,他的任务是抢,虽说占了挺大一块地方,但人都跑光了,技术型人才不多,啥产业都没有。据
说有些地方,连铁锅都造不出来。孙承宗到辽东算出差,有补助,还有朝廷送物资,时不时还能回去休个假,努先
生完全是原生态,没人管没人疼,不抢怎么办?

必须抢,然而不能抢,因为有孙承宗。

作为世界超级大国,美国有一个非常有趣的形象代言人----山姆大叔。这位大叔的来历就不说了,他的具体特点是
面相端正,勤劳乐观,处事低调埋头苦干,属于那种不怎么言语,却特能干事的类型,是许多美国人争相效仿的楷
模。

孙承宗就是一个山姆大叔型的人物,当然,按年龄算,应该叫山姆大爷,这位仁兄相貌奇伟(画像为证),极富乐观
主义精神(大家都不干,他干),非常低调(从不出兵闹事),经常埋头苦干(参见前文孙承宗业绩清单)。

刚开始的时候,努尔哈赤压根瞧不起孙大爷,因为这个人到任后毫无动静,一点不折腾,什么一举荡平,光复辽
东,提都不提,别说出兵攻击,连挑衅斗殴都不来,实在没意思。

但慢慢地,他才发现,这是一个极其厉害的人。

就在短短几年内,明朝的领土以惊人的速度扩张,从关外的一亩三分地,到宁远,再到锦州,在不知不觉中,他已
收复了辽东近千里土地。

更为可怕的是,此人每走一步,都经过精心策划,步步为营稳扎稳打,趁你不注意,就刨你两亩地,每次都不多
占,但占住了就不走,几乎找不到任何弱点。

对于这种抬头望天,低头使坏的人,努尔哈赤是一点办法都没有,只能眼睁睁地看着对方大踏步的前进,自己大踏
步地后退,直到天启五年(1625)十月的那一天。

\section[\thesection]{}

幕后

这一天,努尔哈赤得到消息,孙承宗回京了。

他之所以回去,不是探亲,不是述职,也不是做检讨,而是彻底退休。

必须说明的是,他是主动提出退休的,却并不情愿,他不想走,却不能不走。

因为他曾无比依赖的强大组织东林党,被毁灭了。

关于东林党的覆灭,许多史书上的说法比较类似:一群有道德的君子,在无比黑暗的政治斗争中,输给了一群毫无
道德的小人,最终失败。

我认为,这个说法,那是相当的胡扯。

事实上,应该是一群精明的人,在无比黑暗的政治斗争中,输给了另一群更为精明的人,最终失败。

许多年来,东林党的失败之所以很难说清楚,是由于东林党的成功没说清楚。

而东林党的成功之所以没说清楚,是由于这个问题,很难说清楚。

这不是顺口溜,其实一直以来,在东林党的兴亡之中,都隐藏着一些不足为人道的玄机,很多人不知道,知道的人
不说。

凑巧的是,我是一个比较较真的人,对于某些很难说清楚的问题,不足为人道的玄机,有着很难说清楚,不足为人
道的兴趣。

于是,在查阅分析了许多史籍资料后,我得到了这样一个结论:东林党之所以成功,是因为强大,之所以失败,是
因为过于强大。

万历四十八年(1620),在杨涟、左光斗以及一系列东林党人的努力下,朱常洛顺利即位,成为了明光宗。

虽然这位仁兄命短,只活了一个月,但东林党人再接再厉,经历千辛万苦,又把他的儿子推了上去,并最终控制了
朝廷政权。

用正面的话说,这是正义战胜了邪恶,意志顽强,坚持到底。

用反面的话说,这是赌一把,运气好,找对了人,打对了架。

无论正面反面,几乎所有人都认为,东林党能够掌控天下,全靠明光宗死后那几天里,杨涟的拼死一搏,以及继任
皇帝的感恩图报。

这是一个重要的原因,但绝不是唯一重要的原因。因为在中国历史上,一般而言,只要皇帝说话,什么事都好办,
什么事都能办,可是明朝实在太不一般。

\section[\thesection]{}

明朝的皇帝,从来不是说了就算的,且不论张居正、刘瑾、魏忠贤之类的牛人,光是那帮六七品的小御史、给事
中,天天上书骂人,想干啥都不让,能把人活活烦死。

比如明武宗,就想出去转转,换换空气,麻烦马上就来,上百人跪在门口痛哭流涕,示威请愿,午觉都不让睡。闹
得你死我活,最后也没去成。

换句话说,皇帝大人连自己的事情都搞不定,你让他帮东林党控制朝政,那是不太现实的,充其量能帮个忙而已。

东林党掌控朝廷的真正原因在于,他们打败了朝廷中所有的对手,具体说,是齐、楚、浙三党。

众所周知,东林党中的许多成员是没有什么博爱精神的,经常耍二杆子性格,非我族类就是其心必异,什么人都敢
惹,搞了几十年斗争,仇人越来越多,特别是三党,前仆后继,前人退休,后人接班,一代代接茬上,斗得不亦乐
乎。

这两方的矛盾,那叫一个苦大仇深。什么争国本、妖书案、梃击案,只要是个机会,能借着打击对手,就绝不放
过,且从万历十几年就开始闹,真可谓是历史悠久。

就实力而言,东林党势头大,人多,占据优势,而三党迫于压力,形成了联盟,共同对付东林党,所以多年以来此
消彼长,什么京察、偷信,全往死里整。可由于双方实力差距不大,这么多年了,谁也没能整死谁。

万历末年,一个人来到了京城,不久之后,在极偶然的情况下,他加入了其中一方。

他加入的是东林党,于是,三党被整死了。

这是一个不折不扣的小人物,然而,正是这个小人物的到来,打破了几十年的僵局,这个人名叫汪文言。

如果你不了解这个人,那是正常的,如果你了解,那是不正常的。

甚至很多熟读明清历史的人,也只知道这个名字,而不清楚这个名字背后隐藏的东西。

因为这个人实在是太不起眼了。

事实上,为查这位仁兄的生平,我吃了很大苦头,翻了很多书,还专门去查了历史文献检索,竟然都没能摸清他的
底。

在几乎所有的史籍中,对于此人的描述都只有只言片语,应该说,这是奇怪的现象

对于一个在历史上有一定知名度的人而言,介绍如此之少,是很不正常的,但从某个角度讲,又是很正常的。

因为决定成败的关键人物,往往喜欢隐藏于幕后。

\section[\thesection]{}

汪文言,安徽人,不是进士,也不是举人,甚至不是秀才,他没有进过考场,没有当过官,只是个普通的老百姓。

对于这位老百姓,后世曾有一个评价:以布衣之身,操控天下。

汪布衣小时候情况如何不太清楚,从目前的材料看,是个很能混的人,他虽然不考科举,却还是当上了公务员----
县吏。

事实上,明代的公务员,并非都是政府官员,它分为两种:官与吏。

参加科举考试,考入政府成为公务员的,是官员。就算层次最低、底子最差的举人(比如海瑞),至少也能混个县教
育局长。

可问题在于,明朝的官员编制是很少的,按规定,一个县里有品级,吃皇粮的,只有知县(县长)、县丞(县政府办公
室主任)几个人而已。

而没有品级,也吃皇粮的,比如教谕(教育局长)、驿丞(县招待所所长),大都由举人担任,人数也不多。

在一个县里,只有以上人员算是国家公务员,换句话说,他们是领国家工资的。

然而一个县只靠这些人是不行的,县长大人日理万机,无论如何是忙不过来的,所以手下还要有跑腿的,偷奸耍滑
的,老实办事的,端茶倒水的。

这些被找来干活的人,就叫吏。

吏没有官职、没有编制,国家也不给他们发工资,所有收入和办公费用都由县里解决,换句话说,这帮人国家是不
管的。

虽然国家不管,没有正式身份,也不给钱,但这份职业还是相当热门,每年都有无数热血青年前来报考,没关系还
当不上,也着实吸引了许多杰出人才,比如阳谷县的都头武松同志,就是其中的优秀榜样。

这是因为在吏的手中,掌握着一件最为重要的东西----权力。

一般说来,县太爷都是上级派下来的,没有根基,也没有班底,而吏大都是地头蛇,熟悉业务,有权在手,熟门熟
路,擅长贪污受贿,黑吃黑,除去个把像海瑞那种软硬不吃的极品知县外,谁都拿这帮编外公务员没办法。

汪文言,就是编外公务员中,最狡猾,最会来事,最杰出的代表人物。

汪文言的官场生涯,是从监狱开始的,那时候,他是监狱的看守。

作为一名优秀的看守,他忠实履行了守护监狱,训斥犯人,收取贿赂、拿黑钱的职责。

\section[\thesection]{}

由于业务干得相当不错,在上级(收过钱的)和同僚(都是同伙)的一致推荐下,他进入了县衙,在新的岗位上继续开
展自己的光辉事业。

值得表扬的是,此人虽然长期和流氓地痞打交道,不光彩的事情也没少干,但为人还是很不错的,经常仗义疏财,
接济朋友。但凡认识他的,就算走投无路,只要找上门来,他都能帮人一把,江湖朋友纷纷前来蹭饭,被誉为当代
宋江。

就这样,汪文言名头越来越响,关系越来越野,越来越能办事,连知县搞不定的事情,都要找他帮忙。家里跟宋江
一样,经常宾客盈门,什么人都有,即有晁盖之类的江洋大盗,又有李逵之流的亡命之徒,上门的礼仪也差不多,
总是``叩头就拜'',酒足饭饱拿钱之后,就甘心做小弟,四处传扬汪先生的优秀品格。

在无数志愿宣传员的帮助下,汪先生逐渐威名远播,终于打出县城,走向全省,波及全国。

但无论如何,他依然只是一个县衙的小人物,直到有一天,他的名声传到了一个人的耳中。

这个人叫于与立,时任刑部郎中。

这位于郎中官职不算太高,但想法不低,经常四处串门拉关系,他听说汪文言的名声后,便主动找上门去,特聘汪
先生到京城,发挥特长,为他打探消息。

汪先生岂是县中物,毫不犹豫就答应了,准备到京城大展拳脚。

可几个月下来,汪文言发现,自己县里那套,在京城根本混不开。

因为汪先生一无学历,二无来历,档次太低,压根就没人搭理他。无奈之下,他只好出钱,去捐了个监生,不知找
了谁的门路,还混进了太学。

这可就真了不得了,汪先生当即拿出当年跑江湖的手段,上下打点,四面逢源,短短几月,上至六部官员,下到穷
学生,他都混熟了,没混熟的,也混个脸熟。

一时之间,汪文言从县里的风云人物,变成了京城的风云人物。

但这位风云人物,依然还是个小人物。

因为真正掌控这个国家权力中枢的重要人物,是不会搭理他的,无论是东林党的君子,还是三党的小人,都看不上
这位江湖人士。

但他终究找到了一位可靠的朋友,并在他的帮助下,成功进入了这片禁区。

这位不计较出身的朋友,名叫王安。

\section[\thesection]{}

要论出身,在朝廷里比汪文言还低的,估计也只有太监了,所以这两人交流起来,也没什么心理障碍。

当时的王安,并非什么了不得的人物,虽说是太子朱常洛的贴身太监,可这位太子也不吃香,要什么没什么,老爹
万历又不待见,所以王安同志混得相当不行,没人去搭理他。

但汪文言恰恰相反,鞍前马后帮他办事,要钱给钱,要东西给东西,除了女人,什么都给了。

王安很喜欢汪文言。

当然,汪文言先生不是人道主义者,也不是慈善家,他之所以结交王安,只是想赌一把。

一年后,他赌赢了。

在万历四十八年(1620)七月二十一日的那个夜晚,当杨涟秘密找到王安,通报老头子即将走人的消息时,还有第三
个人在场----汪文言。

杨涟说,皇上已经不行了,太子应立即入宫继位,以防有变。

王安说,目前情形不明,没有皇上的谕令,如果擅自入宫,凶多吉少。

杨涟说,皇上已经昏迷,不会再有谕令,时间紧急,绝不能再等!

王安说,事关重大,再等等。

僵持不下时,汪文言用自己几十年官海沉浮的经验,做出了一个判断。

他对王安说:杨御史是对的,不能再等待,必须立即入宫。

一直以来,王安对汪文言都极为信任,于是他同意了,并带领朱常洛,在未经许可的情况下进入了皇宫,成功即位。

这件事不但加深了王安对汪文言的信任,还让东林党人第一次认清了这个编外公务员,江湖混混的实力。

继杨涟之后,东林党的几位领导,大学士刘一璟、韩旷、尚书周嘉谟、御史左光斗等人,都和汪文言拉上了关系。

就这样,汪文言加深了与东林党的联系,并最终成为了东林党的一员----瞎子都看得出,新皇帝要即位了,东林党
要发达了。

但当他真正踏入政治中枢的时候,才发现,局势远不像他想象的那么乐观。

当时明光宗已经去世,虽说新皇帝也是东林党捧上去的,但三党势力依然很大,以首辅方从哲为首的浙党、以山东
人给事中亓诗教为首的齐党、和以湖广人官应震、吴亮嗣为首的楚党,个个都不是省油的灯。

三党的核心,是浙党,此党的创始人前任首辅沈一贯,一贯善于拉帮结派,后来的接班人,现任首辅方从哲充分发
扬了这一精神,几十年下来,朝廷内外,浙党遍布。

\section[\thesection]{}

齐党和楚党也不简单,这两个党派的创始人和成员基本都是言官,不是给事中,就是御史,看上去级别不高,能量
却不小,类似于今天的媒体舆论,动不动就上书弹劾兴风作浪。

三党分工配合,通力协作,极不好惹,东林党虽有皇帝在手,明里暗里斗过几次,也没能搞定。

关键时刻,汪文言出场。

在仔细分析了敌我形势后,汪文言判定,以目前东林党的实力,就算和对方死拼,也只能死,没得拼。

而最关键的问题在于,东林党的这帮大爷都是进士出身,个个都牛得不行,进了朝廷就人五人六,谁都瞧不上谁,
看你不顺眼也不客套,恨不得操板砖上去就拍。

汪文言认为,这是不对的,为了适应新的斗争形势,必须转变观念。

由于汪先生之前在基层工作,从端茶倒水提包拍马开始,一直相当低调,相当能忍,所以在他看来,这个世界上没
有永远的敌人,也没有永远的朋友,只要会来事,朋友和敌人,是可以相互转化的。

秉持着这一理念,他拟定了一个计划,并开始寻找一个恰当的人选。很快,他就找到了这个人----梅之焕。

梅之焕,字彬父,万历三十二年进士,选为庶吉士。后任吏科给事中。

此人出身名门,文武双全,十几岁的时候,有一次朝廷阅兵,他骑匹马,没打招呼,稀里糊涂就跑了进去,又稀里
糊涂地要走。阅兵的人不干,告诉他你要不露一手,今天就别想走。梅之焕二话不说,拿起弓就射,九发九中,射
完啥也不说,摆了个特别酷的动作,就走人了(长揖上马而去)。

除上述优点外,这人还特有正义感,东厂坑人,他就骂东厂,沈一贯结党,他就骂沈一贯,是个相当强硬的人。

但汪文言之所以找到这位仁兄,不是因为他会射箭,很正直,而是因为他的籍贯。

梅之焕,是湖广人,具体地说,是湖北麻城人。

明代官场里,最重要的两大关系,就是师生、老乡。一个地方出来的,都到京城来混饭吃,老乡关系一攀,就是兄
弟了。所以自打进入朝廷,梅之焕认识的,大都是楚党成员。

可这人偏偏是个东林党。

有着坚定的东林党背景,又与楚党有着密切的联系,很好,这正是那个计划所需要的人。

\section[\thesection]{}

汪文言认为,遇到敌人,直接硬干是不对的,在操起板砖之前,应该先让他自己绊一跤。

三党是不好下手的,只要找到一个突破口,把三党变成两党,就好下手了。

在仔细衡量利弊后,他选择了楚党。

因为在不久之前,发生过这样一件事情。

虽然张居正大人已经死去多年,却依然被人怀念,于是朝中有人提议,要把这位大人从坟里再掘出来,修理一顿。

这个建议的提出,充分说明朝廷里有一大帮吃饱了没事干,且心理极其阴暗变态的王八蛋,按说是没什么人理的,
可不巧的是,提议的人,是浙党的成员。

这下就热闹了,许多东林党人闻讯后,纷纷赶来骂仗,痛斥三党,支持张居正。

说句实话,当年反对张居正的时候,东林党也没少掺合,之所以跑来伸张正义,无非是为了反对而反对,提议是什
么并不重要,只要是三党提出的,就是错的,对人不对事,不必当真。

梅之焕也进来插了句话,且相当不客气:

``如果江陵(指张居正)还在,你们这些无耻小人还敢这样吗?''

话音刚落,就有人接连上书,表示同意,但让所有人都出乎意料的是,支持他的人,并不是东林党,而是官应震。

官应震,是楚党的首领,他之所以支持梅之焕,除了两人是老乡,关系不错外,还有一个十分重要的原因:死去的
张居正先生,是湖广人。

这件事情让汪文言认识到,所谓三党,并不是铁板一块,只要动动手脚,就能将其彻底摧毁。

所以,他找到了梅之焕,拉拢了官应震,开始搞小动作。

至于他搞了什么小动作,我确实很想讲讲,可惜史书没写,我也不知道,只好省略,反正结论是三党被搞垮了。

此后的事情,我此前已经讲过了,方从哲被迫退休,东林党人全面掌权,杨涟升任左副都御史,赵南星任吏部尚
书,高攀龙任光禄丞,邹元标任左都御史等等。

之所以让你再看一遍,是要告诉你,在这几个成功男人的背后,是一个沉默的男人。

这就是东林党成功的全部奥秘,很明显,不太符合其一贯正面光辉的形象,所以如果有所隐晦,似乎可以理解。

东林党的成功之路到此结束,同学们,现在我们来讲下一课:东林党的失败之路。

\section[\thesection]{}

在我看来,东林党之所以失败,是因为自大、狂妄,以及嚣张,不是一个,而是一群。

如果要在这群人中寻找一个失败的代表,那这个人一定不是杨涟,也不是左光斗,而是赵南星。

虽然前两个人很有名,但要论东林党内的资历跟地位,他们和赵先生压根就没法比。

关于赵南星先生的简历,之前已经介绍过了,从东林党创始人顾宪成时代开始,他就是东林党的领导,原先干人
事,回家呆了二十多年,人老心不老,又回来干人事。

一直以来,东林党的最高领导人(或者叫精神领袖),是三个人,他们分别是顾宪成、邹元标以及赵南星。

顾宪成已经死了,天启二年,邹元标也退休了,现在只剩下了赵南星。

赵先生不但在东林党内有着至高无上的地位,他在政府里,也占据着最牛的职务----吏部尚书。一手抓东林党,一
手抓人事权,换句话说,赵南星就是朝廷的实际掌控者

但失败之根源,正是此人。

天启三年(1623),是一个很特殊的年份,因为这一年,是京察年。

所谓京察年,也就是折腾年。六年一次,上级考核各级官吏,有冤报冤,有仇报仇,万历年间的几次京察,每年搞
得不亦乐乎,今年也不例外。

按照规定,主持折腾工作的,是吏部尚书,也就是说,是赵南星。

赵南星是个很负责的人,经过仔细考察,列出了第一批名单,从朝廷滚蛋的名单,包括以下四人:亓诗教、官应震、
吴亮嗣、赵兴邦。

如果你记性好,应该记得这几位倒霉蛋的身份,亓诗教,齐党首领,赵兴邦,浙党骨干、官应震、吴亮嗣,楚党首
领。

此时的朝政局势,大致是这样的,东林党大权在握,三党一盘散沙,已经成了落水狗。

很明显,虽然这几位兄弟已经很惨了,但赵先生并不干休,他一定要痛打落水狗。

这是一个很过分的行为,不但要挤掉他们的政治地位,还要挤掉他们的饭碗,实在太不厚道。

更不厚道的是,就在不久之前,楚党还曾是东林党的同盟,帮助他们掌控政权,结果官应震大人连屁股都没坐热,
就被轰走了。

这就意味着,汪文言先生连哄带骗,好不容易建立的牢固同盟,就此彻底崩塌。

\section[\thesection]{}

赵大人在把他们扫地出门的同时,也不忘给这四位下岗人员一个响亮的称号----四凶。为此,他还写了一篇评论文
章《四凶论》,以示纪念。

跟着这四位一起走人的,还有若干人,他们都有着共同的身份:三党成员、落水狗。

此处不留爷,自有留爷处,既然赵大人不给饭吃,就只好另找饭馆开饭了。

就在此时,一个人站在他们面前,体贴地对他们说,在这世界上,赵南星并不是唯一的饭馆老板。

据史料记载,这个人言语温和,面目慈祥,是个亲切的胖老头。

现在,让我们隆重介绍:明代太监中的极品,宦官制度的终极产物,让刘瑾、王振等先辈汗颜的后来者,比万岁只
差一千岁的杰出坏人、恶棍、流氓地痞的综合体----魏忠贤。

魏忠贤,北直隶(今河北)肃宁县人,曾用名魏进忠。

对于魏公公的出身,历史上一直有两种说法。一种说,他的父母都是贫苦农民;另一种说,他的父母都是街头玩杂
耍的。

说法是不同的,结果是一样的,因为无论农民或杂耍,都是穷人。

家里穷,自然就没钱给他读书,不读书,自然就不识字,也没法考取功名,升官发财,小孩不上学,父母又不管,
只能整天在街上闲逛。

就这样,少年魏忠贤成为了失学儿童、文盲、社会无业游荡人员。

但这样的悲惨遭遇,丝毫没有影响魏忠贤的心情,因为他压根儿不觉得自己很惨。

混混的幸福

多年前,我曾研习过社会学,并从中发现了这样一条原理:社会垃圾(俗称混混),是从来不会自卑的。

虽然在别人眼中,他们是当之无愧的人渣、败类、计划生育的败笔,但在他们自己看来,能成为一个混混,是极其
光荣且值得骄傲的。

因为他们从不认为自己在混,对于这些人而言,打架、斗殴、闹事,都是美好生活的一部分,抢小孩的棒棒糖和完
成一座建筑工程,都是人生意义的自我实现,没有任何区别。

做了一件坏事,却绝不会后悔愧疚,并为之感到无比光辉与自豪的人,才是一个合格的坏人,一个纯粹的坏人,一
个坏得掉渣的坏人。

魏忠贤,就是这样一个坏人。

\section[\thesection]{}

根据史料记载,少年魏忠贤应该是个非常开朗的人,虽然他没钱上学,没法读书,没有工作,却从不唉声叹气,相
当乐观。

面对一没钱、二没前途的不利局面,魏忠贤不等不靠,毅然走上社会,大玩特玩,并在实际生活中确定了自己的人
生性格(市井一无赖尔)。

他虽然是个文盲,却能言善辩(目不识丁,言辞犀利),没读过书,却无师自通(性多狡诈),更为难得的是,他虽然
身无分文,却胸怀万贯,具体表现为明明吃饭的钱都没有,还敢跑去赌博(家无分文而一掷百万),赌输后没钱给,
被打得生活不能自理,依然无怨无悔,下次再来。

混到这个份上,可算是登峰造极了。

然而混混魏忠贤,也是有家庭的,至少曾经有过。

在他十几岁的时候,家里就给他娶了老婆,后来还生了个女儿,一家人过得还不错。

但为了快乐的混混生活,魏忠贤坚定地抛弃了家庭,在他尚未成为太监之前,四处寻花问柳,城中的大小妓院,都
留下了他的足迹,家里仅有的一点钱财,也被他用光用尽。

被债主逼上门的魏忠贤,终于幡然悔悟,经过仔细反省,他发现,原来自己并非一无所有----还有个女儿。

于是,他义无反顾地卖掉了自己的女儿,以极其坚定的决心和勇气,为了还清赌债。

能干出这种事情的人,也就不是人了,魏忠贤的老婆受不了了,离家出走改嫁了。应该说,这个决定很正确,因为
按当时情形看,下一个被卖的,很可能是她。

原本只有家,现在连家都没了,卖无可卖的魏忠贤再次陷入了困境。

被债主逼上门的魏忠贤,再次幡然悔悟,经过再次反省,他再次发现,原来自己并非一无所有,事实上,还多了件
东西。

只要丢掉这件东西,就能找一份好工作----太监。

这并非魏忠贤的个人想法,事实上在当地,这是许多人的共识。

魏忠贤所在的直隶省河间府,一向盛产太监,由于此地距离京城很近,且比较穷,从来都是宫中太监的主要产地,
并形成了固定产业,也算是当地创收的一种主要方式。 混混都混不下去,人生失败到这个程度,必须豁出去了。

经过短期的激烈思想斗争,魏忠贤树立了当太监的远大理想,然而当他决心在太监的大道上奋勇前进的时候,才惊
奇地发现,原来要当一名太监,是很难的。

\section[\thesection]{}

一直以来,在人们的心目中,做太监,是迫于无奈,是没办法的办法。

现在,我要严肃地告诉你,这种观点是错误的。太监,是一份工作,极其热门的工作,而想成为一名太监,是很难
的。

事实上,太监这个职业之所以出现,只是因为一个极其简单的原因----宫里只有女人。

由于老婆太多,忙不过来,为保证皇帝陛下不戴绿帽子(这是很有可能的),宫里不能进男人。可问题是,宫里太
大,上千人吃喝拉撒,重活累活得有人干,女的干不了,男的不能进,只好不男不女了。

换句话说,太监其实就是进城干活的劳工。唯一不同的是,他们的工作地点,是皇宫。

既然是劳工,就有用工指标,毕竟太监也有个新陈代谢,老太监死了,新太监才能进,也就是说,每年录取太监比
例相当低。

有多低呢?我统计了一下,大致是百分之十到百分之十五,而且哪年招还说不准,今年要不缺人,就不招。

对于有志于踏入这一热门行业,成为合格太监的众多有志青年而言,这是一个十分残酷的事实,因为这意味着,在
一百个符合条件(割了)的人中,只有十到十五人,能够成为光荣的太监。

事实上,自明代中期,每年都有上千名符合条件(割过了),却没法入宫的太监(候选)在京城等着。

要知道,万一切了,又当不了太监,那就惨了。虽说太监很吃香,但归根结底,吃香的只是太监的工资收入,不是
太监本人。对于这类``割了''的人,人民群众是相当鄙视的。

所以众多未能成功入选的太监候选人,既不能入宫,也不能回家,只能在京城混。后来混得人越来越多,严重影响
京城社会治安的稳定,为此,明朝政府曾颁布法令:未经允许,不得擅自阉割。

我一直相信,世事皆有可能。

太监之所以如此热门,除了能够找工作,混饭吃外,还有一个重要的原因----权力。

公正地讲,明代是一个公正的朝代。任何一个平凡的人,哪怕是八辈贫农,全家只有一条裤子,只要出个能读书
的,就能当官,就能进入朝廷,最终掌控无数人的命运。

唯一的问题在于,这条道路虽然公正,却不平坦。

\section[\thesection]{}

因为平凡的人是很多的,且大都不安分,要想金榜题名,考中进士,走着上去是不行的,一般都得踩着上去----踩
着那些被你淘汰的人。

明朝的进士,三年考一次,每次录取名额平均一百多人,也就是说,平均每年能进入朝廷,看见皇帝大人尊容的,
只有三四十人。而决定所有人命运的,只是那张白纸,和几道考试的题目。

同一张纸,同一道题目,同一个地方。

不同的人,不同的脑袋,不同的手。

能否出人头地,只能靠你自己,靠你那非凡的智力、领悟力,以及你那必定能够超越常人,必定与众不同的信念。

所以我一直认为,科举制度,是一种杰出、科学的人才选拔制度,它杜绝了自东汉以来,腐败不堪的门阀制度,最
大地保证了人才的选拔,虽然它并不完美,却亦无可取代。

当然,事情到这里,还没有结束,因为当你考上了进士,脱离了科举体系,就会发现,自己进入了另一个全新的体
系----文官体系。

在那个体系中,你只是微不足道的一员,还要熬资历,干工作,斗智斗勇,经过几十年的磨砺之后,你才能成为精
英中的精英,并具备足够的智商和经验,领导这个伟大的国度继续前进。

这就是于谦、李贤、徐阶、张居正、申时行等人的成功之道,也是必经之道。虽然他们都具有优异的天赋,坚韧的
性格,坎坷的经历,但要想名流千古,这是无法逃避的代价。

在那条通往最高宝座的道路上,只有最优秀,最聪明,最有天分的人,才能到达终点

但许多人不知道,有些不那么聪明,不那么优秀,不那么有天分的人,也能走到终点。

因为在通往终点的方向,有一条捷径,这条捷径,就是太监之路。

太监不需要饱读诗书,不需要层层培训,不需要处理政务,不需要苦苦挣扎,他们能够跨过所有文官体系的痛苦经
历,直接获取成功,只需要讨好一人--皇帝。

皇帝就是老板,就是CEO,就是一言九鼎,总而言之,是说了就算的人物。

而太监,就在皇帝的身边,所以只要哄好皇帝,太监就能得到权力,以及他想得到的一切。

这就是有明一代,无数的人志愿成为太监的全部、真实的原因。

但现在摆在无业游民魏忠贤面前的,不仅仅是个录取名额比例,还有一个更为基本的难点--阉割问题。

\section[\thesection]{}

魏忠贤当政以后,对自己以前的历史万般遮掩,特别是他怎么当上太监,怎么进宫这一段,是绝口不提,搞得云里
雾里,捉摸不透。

但这种行为,就好比骂自己的儿子是王八蛋一样,最终只能自取其辱。

他当年的死党,后来的死敌刘若愚太监告诉我们,魏公公不愿提及发家史,是因为违背了太监成长的正常程序----
他是自宫的。

我一直坚信,东方不败是这个世界上最伟大、最杰出,也最有可能的自宫者。

这绝不仅仅因为他的自宫,绝无混饭吃、找工作的目的,而是为了中华武学的发展。

真正的原因在于,当我考证了太监阉割的全过程后,才不禁由衷感叹,自宫不仅需要勇气,没准还真得要点功夫。

很多人不知道,其实阉割是个技术工作,想一想就明白了,从人身上割点东西下来,还是重要部位,稍有不慎,命
就没了。

所以很多年以来,干这行的都是家族产业,代代相传,以割人为业,其中水平最高的,还能承包官方业务,获得官
方认证。

一般这种档次的,不但技术高,能达到庖丁解人的地步,快速切除,还有配套医治伤口,消毒处理,很有服务意识。

所有说,东方不败能在完全外行的情况下,完成这一复杂的手术,且毫无后遗症(至少我没看出来),没有几十年的
内功修养,估计是白扯。

魏忠贤不是武林高手(不算电影电视),要他自我解决,实在勉为其难,于是只好寻到上述专业机构,找人帮忙。

可到地方一问,才知道人家服务好,收费也高,割一个得四五两银子,我估算了一下,合人民币大概是三四千块。

这可就为难魏公公了,身上要有这么多钱,早拿去赌博翻本,哪犯得着干这个?

割还是不割,这不是一个问题,问题是,没钱。

但现实摆在眼前,不找工作是不行了,魏公公心一横----自己动手,前程无忧。

果不其然,业余的赶不上专业的,手术的后遗症十分严重,出血不止,幸亏好心人路过,帮他止了血。

成功自宫后,魏忠贤跑去报名,可刚到报名处,问清楚录取条件,当时就晕了。

\section[\thesection]{}

事情是这样的,宫里招太监,是有年龄要求的,因为小孩进宫好管,也好教,可是魏忠贤同志自己扳指头一算,今
年芳龄已近二十。

这可要了命了,年龄是硬指标,跟你一起入宫的,都是几岁的孩子,哪个太监师傅愿意带你这么个五大三粗的小伙
子,纯粹浪费粮食。

魏忠贤急了,可急也没用,招聘规定是公开的,你不去问,还能怪谁?

可事到如今,割也割了,又没法找回来,想再当混混,没指望了,要知道,混混虽然很混,也瞧不起人妖。

宫进不去,家回不去,魏公公就此开始了他的流浪生涯,具体情况他本人不说,所以我也没法同情他,但据说是过
得很惨,到后来,只能以讨饭为生,偶尔也打打杂工。

万历十六年(1588),穷困至极的魏忠贤来到了一户人家的府上,在这里,他找到了一份佣人的工作。

他的命运就此改变。一般说来,寻常人家找佣人,是不会找阉人的,魏忠贤之所以成功应聘,是因为这户人家的主
人,也是个阉人。

这个人的名字,叫孙暹,是宫中的太监,准确地说是太监首领,他的职务,是司礼监秉笔太监。这个职务,是帮助
皇帝批改奏章的,前面说过很多次,就不多说了。

魏忠贤很珍惜这个工作机会,他起早贪黑,日干夜干,终于有一天,孙暹找他谈话,说是看在他比较老实的份上,
愿意保举他进宫。

万历十七年(1589),在经历了无数波折之后,魏忠贤终于圆了他的梦,进宫当了一名太监。

不好意思,纠正一下,是火者。

实际上,包括魏忠贤在内的所有新阉人,在刚入宫的时候,只是宦官,并不是太监,某些人甚至一辈子也不是太监。

因为太监,是很难当上的。

宫里,能被称为``太监''的,都是宦官的最高领导,太监以下,是少监,少监以下,是监丞,监丞以下,还有长随、
当差。

当差以下,就是火者了。

那么魏火者的主要工作是什么呢?大致包括以下几项:扫地、打水、洗马桶、开大门等等。

很明显,这不是一份很有前途的工作,而且进宫这年,魏忠贤已经二十一岁了,所以在相当长的时间里,魏忠贤很
不受人待见。

一晃十几年过去了,魏忠贤没有任何成就,也没有任何名头,因为他的年龄比同期入宫的太监大,经常被人呼来喝
去,人送外号``魏傻子''。

但这一切,全都是假象。

\section[\thesection]{}

据调查(本人调查),最装牛的傻人,与人接触时,一般不会被识破。

而最装傻的牛人,在与人接触时,一辈子都不会被识破。

魏忠贤就是后者的杰出代表。

许多人评价魏忠贤时,总是一把鼻涕一把泪,说大明江山,太祖皇帝,怎么就被这么个文盲、傻子给废掉了。

持有这种观点的人,才是傻子。

能在明朝当官,且进入权力核心的这拨人,基本都是高智商的,加上官场沉浮,混了那么多年,生人一来,打量几
眼,就能把这人摸得差不多,在他们面前耍花招,那就是自取其辱。

而在他们的眼中,魏忠贤是一个标准的老实人,年纪大,傻不拉矶的,每天都呵呵笑,长相忠厚老实,人家让他干
啥就干啥,欺负他,占他便宜,他都毫不在意,所以从明代,直到今天,很多人认定,这人就是个傻子,能混成后
来那样,全凭运气。

这充分说明,魏公公实在是威力无穷,在忽悠了明代的无数老狐狸后,还继续忽悠着现代群众。

在我看来,魏忠贤固然是个文盲,却是一个有天赋的文盲,他的这种天赋,叫做伪装。

一般人在骗人的时候,都知道自己在骗人,而据史料分析,魏公公骗人时,不知道自己在骗人,他骗人的态度,是
极其真诚的。

在宫里的十几年里,他就用这种天赋,骗过了无数老滑头,并暗中结交了很多朋友,其中一个叫做魏朝。

这位魏朝,也是宫里的太监,对魏忠贤十分欣赏,还帮他找了份工作。这份工作的名字,叫做典膳。

所谓典膳,就是后宫管伙食的,听起来似乎不怎么样,除了混吃混喝,没啥油水。

管伙食固然没什么,可关键在于管谁的伙食。

魏公公的服务对象,恰好就是后宫的王才人。这位王才人的名头虽然不响,但他儿子的名气很大----朱由校。

正是在那里,魏忠贤第一次遇见了决定他未来命运的两位关键人物----朱常洛父子。

虽然见到了大人物,但魏忠贤的命运仍无丝毫改变,因为王才人身边有很多太监,他不过是极其普通的一个,平时
连跟主子说话的机会都没有。

而且此时朱常洛还只是太子,且地位十分不稳,随时可能被拿下,所以他老婆王才人混得也不好,还经常被另一位
老婆李选侍欺负。

\section[\thesection]{}

这么一来,魏忠贤自然也混得差,到万历四十七年(1619),魏忠贤进宫二十周年纪念之际,他混到了人生的最低
点:由于王才人去世,他失业了。

失业后的魏忠贤无计可施,只能回到宫里,当了一个仓库保管员。

但被命运挑选的人,注定是不会漏网的,在经过无数极为复杂的人事更替,误打误撞后,魏忠贤竟然摇身一变,又
成了李选侍的太监。

正是在这个女人的手下,魏忠贤第一次露出了他的狰狞面目。

这位入宫三十年,已五十多岁的老太监突然焕发了青春,他不等不靠,主动接近李选侍,拍马擦鞋,无所不用其
极,最终成为了李选侍的心腹。

因为在他看来,这个掌握帝国未来继承人(朱由校),且和他一样精明、自私、无耻的女人,将大有作为。

万历四十八年(1620),魏忠贤的机会到了。

这一年七月,明神宗死了,明光宗即位,李选侍成了候选皇后,朱由校也成了后备皇帝。

可是好景不长,只过了一个月,明光宗又死了,李选侍成了寡妇。

当李寡妇不知所措之时,魏忠贤及时站了出来,开导了李寡妇,告诉她,其实你无需失望,因为一个更大的机会,
就在你的眼前:只要紧紧抓住年幼的朱由校,成为幕后的操纵者,你得到的,将不仅仅是皇后甚至太后的头衔,而
是整个天下。

这是一个很好的想法,可惜绝非独创,朝廷里文官集团的老滑头们,也明白这一点。

于是在东林党人的奋力拼杀下,朱由校又被抢了回去,李选侍就此彻底歇菜,魏忠贤虽然左蹦右跳,反应活跃,最
终也没逃脱下岗的命运。

正是在这次斗争中,魏忠贤认识了他宿命中的对手,杨涟。

杨涟,是一个让魏忠贤寒毛直竖的人物。

两人第一次相遇,是在抢人的路上。杨涟抢走朱由校,魏忠贤去反抢,结果被骂了回来,哆嗦了半天。

第二次相遇,是他奉命去威胁杨涟,结果被杨涟威胁了,杨大人还告诉他,再敢作对,就连你一块收拾。

魏忠贤相当识趣,掉头就走,从此以后,再不敢惹这人。

总而言之,在魏忠贤的眼中,杨涟是个不贪财,不好色,不怕事,几乎没有任何弱点,还特能折腾的人,而要对付
这种人,李选侍是不够分量的,必须寻找一个新的主人。

\section[\thesection]{}

然而很遗憾,在当时的宫里,比李选侍还狠的,只有东林党,就算魏太监想进,估计人家也不肯收。

看起来是差不多了,毕竟魏公公都五十多了,你要告诉他,别灰心,不过从头再来,估计他能跟你玩命。

但拯救他的人,终究还是出现了。

许多人都知道,天启皇帝朱由校是很喜欢东林党的,也很够意思,继位一个月,就封了很多人,要官给官,要房子
给房子。

但许多人不知道,他第一个封的并不是东林党,继位后第十天,他就封了一个女人,封号``奉圣夫人''

这个女人姓客,原名客印月,史称``客氏''

客,是一个非常特别的姓氏,估计这辈子,你也很难遇上一个姓客的,而这位客小姐,那就更特别了,可谓五百年
难得一遇的极品。

进宫之前,客印月是北直隶保定府村民候二的老婆,相貌极其妖艳,且极其早熟,啥时候结婚没人知道,反正十八
岁就生了儿子。

她的命运就此彻底改变。因为就在同一年,宫里的王才人生出了朱由校。

按照惯例,必须挑选合适的乳母去喂养朱由校,经过层层选拔,客印月战胜众多竞争对手,成功入宫。

刚进宫时,客印月极为勤奋,随叫随到,两年后,她的丈夫不幸病逝,但客印月表现了充分的职业道德,依然兢兢
业业完成工作,在宫里混得相当不错。

但很快,宫里的人就发现,这是一个有问题的女人。

有群众反映,客印月常缺勤出宫,行踪诡异,经常出入各种娱乐场所,后经调查,客印月有生活作风问题,时常借
机外出幽会。

作为宫中的乳母,如此行径,结论是清晰的,情节是严重的,但处罚是没有的。有人议论,没人告发。

因为这个看似普通的乳母,一点也不普通。

按说乳母这份活,也就是个临时工,孩子长大了就得走人,该干嘛干嘛去,可是客小姐是个例外,朱由校断奶,她
没走,朱由校长大了,她也没走,朱由校十六岁,当了皇帝,她还是没走。

根据明朝规定,皇子长到六岁,乳母必须出宫,但客印月偏偏不走,硬是多混了十多年,也没人管,因为皇帝不让
她走。

\section[\thesection]{}

不但不让走,还封了个``奉圣夫人'',这位夫人的架子还很大,在宫中可以乘坐轿子,还有专人负责接送。要知
道,内阁大学士刘一璟,二品大员,都六十多了,在朝廷混了一辈子,进出皇宫也得步行。

非但如此,逢年过节,皇帝还要亲自前往祝贺,请她吃饭。夏天,给她搭棚子,送冰块;冬天给她挖坑,烧炭取暖。
宫里给她分了房子,宫外也有房子,还是黄金地段,就在今天北京的正义路上,步行至天安门,只需十分钟,极具
升值潜力。

她家还有几百个仆人伺候,皇宫随意出入,想住哪里就住哪里,想怎么住就怎么住。

所谓客小姐,说破天也就是个保姆,如此得势嚣张,实在很不对劲。

一年之后,这位保姆干出了一件更不对劲的事情。

天启二年(1622),明熹宗朱由校结婚了,皇帝嘛,娶个老婆很正常,谁也没话说。

可是客阿姨(三十五了)不高兴了,突然跳了出来,说了一些不着边际的话,用史籍《明季北略》的话说,是``客氏
不悦''。

皇帝结婚,保姆不悦,这是一个相当无厘头的举动。更无厘头的是,朱由校同志非但没有``不悦'',还亲自跑到保
姆家,说了半天好话,并当即表示,今后我临幸的事情,就交给你负责了,你安排哪个妃子,我就上哪过夜,绝对
服从指挥。

这也太过分了,很多人都极其不满,说你一个保姆,老是赖在宫里,还敢插手后宫,某些胆大的大臣先后上疏,要
求客氏出宫。这事说起来,确实不大光彩,皇帝大人迫于舆论压力,就只好同意了。

但在客氏出宫当天,人刚出门,熹宗就立刻传谕内阁,说了这样一段话:今日出宫,午膳至晚未进,暮思至晚,痛
心不已,着时进宫奉慰,外廷不得烦激。

这段话的意思是客氏今天出宫,我中午饭到现在都没吃,整天都在想念她,非常痛心。还是让她回来安慰我吧,你
们这些大臣不要再烦我了!

傻子都知道了,这两个人之间,必定存在着一种十分特殊的关系。

对此,后半生竭力揭批魏忠贤,猛挖其人性污点的刘若愚同志曾在著作中,说过这样一句话:

倏出倏入,人多讶之,道路流传,讹言不一,尚有非臣子之所忍言者

这句话的意思是,经常进进出出,许多人都惊讶,也有很多谣言,那些谣言,做臣子的是不忍心提的。

此言非同小可。

\section[\thesection]{}

所谓臣子不忍心提,那是瞎扯,不敢提倒是真的。

朱由校的母亲王才人死得很早,他爹当了几十年太子,自己命都难保,这一代人的事都搞不定,哪有时间关心下一
代。所以朱由校基本算是客氏养大的。十几年朝夕相处,而且客氏又是``妖艳美貌,品行淫荡'',要有点什么瓜田
李下,鸡鸣狗盗,似乎也能理解。

就年龄而言,客氏比朱由校大十八岁,按说不该引发猜想,可惜明代皇帝在这方面,是有前科的。比如成化年间的
明宪宗同志,他的保姆万贵妃,就比他大十九岁,后来还名正言顺地搬被子住到一起。就年龄差距而言,客氏也技
不如人,没能打破万保姆的记录,如此看来,传点绯闻,实在比较正常。

当然,这两人之间到底有没有猫腻,谁都不知道,知道也不能写,但可以肯定的是,皇帝陛下对于这位保姆,是十
分器重的。

客氏就是这么个人物,皇帝捧,大臣让,就连当时的东厂提督太监和内阁大臣都要给她几分面子。

对年过半百的魏忠贤而言,这个女人,是他成功的唯一机会,也是最后的机会。于是,他下定决心,排除万难,一
定要争取这个人。而争取这个人的最好方法,就是让她成为自己的老婆。

你没有看错,我没有写错,事实就是如此。

虽然魏忠贤是个太监,但他是可以找老婆的。

作为古代宫廷的传统,太监找老婆,有着悠久的历史,事实上,还有专用名词----对食。

对食,就是大家一起吃饭,但在宫里,你要跟人对食,人家不一定肯。

历代宫廷里,有很多宫女,平时不能出宫,且没啥事干,且不能嫁人,长夜漫漫寂寞难耐,闲着也是闲着,许多人
就在宫中找对象,可是宫里除皇帝外,又没男人,找来找去,长得像男人的,只有太监。

没办法,就这么着吧。

虽说太监不算男人,但毕竟不是女人,反正有名无实,大家一起过日子,说说话,也就凑合了。

这种现象,即所谓对食。自明朝开国以来,就是后宫里的经典剧目,经常上演,一般皇帝也不怎么管,但要遇到凶
恶型的,还是相当危险。比如明成祖朱棣,据说被他看见,当头就是一刀,眼睛都不眨。

\section[\thesection]{}

到明神宗这代,开始还管管,后来他都不上朝,自然就不管了。

但魏忠贤要跟客氏``对食'',还有一个极大的障碍:客氏已经有对象了。

其实对食,和谈恋爱也差不多,也有第三者插足,路边野花四处踩,寻死觅活等俗套剧情,但这一次,情况有点特
殊。

因为客氏的那位对食,恰好就是魏朝。

之前我说过了,魏朝是魏忠贤的老朋友,还帮他介绍过工作,关系相当好,所谓``朋友妻,不可欺'',实在是个问
题。

但魏忠贤先生又一次用事实证明了他的无耻,面对朋友的老婆,二话不说,光膀子就上,毫无心理障碍。

但人民群众都知道,要找对象,那是要条件的,客氏就不用说了,皇帝的乳母,宫里的红人,不到四十,``妖艳美
貌,品行淫荡'',而魏朝是王安的下属,任职乾清宫管事太监,还管兵仗局,是太监里的成功人士,可谓门当户对。

相比而言,魏忠贤就寒掺多了,就一管仓库的,靠山也倒了,要挖墙脚,希望相当渺茫。

但魏忠贤没有妄自菲薄,因为他有一个魏朝没有的优点:胆儿大。

作为曾经的赌徒,魏忠贤胆子相当大,相当敢赌。表现在客氏身上,就是敢花钱,明明没多少钱,还敢拼命花,不
但拍客氏马屁,花言巧语,还经常给她送名贵时尚礼物,类似今天送法国化妆品,高级香水,相当有杀伤力。

这还不算,他隔三差五请客氏吃饭。吃饭的档次是``六十肴一席,费至五百金''。翻译成白话就是,一桌六十个
菜,要花五百两银子。

五百两银子,大约是人民币四万多,就一顿饭,没落太监魏忠贤的消费水平大抵如此。

人穷不要紧,只要胆子大,这就是魏忠贤公公的人生准则。其实这一招到今天,也还能用,比如你家不富裕,就六
十万,但你要敢拿这六十万去买个戒指求婚,没准真能蒙个把人回来。

外加魏太监不识字,看上去傻乎乎的,老实得不行,实在是宫中女性的不二选择,于是,在短短半年内,客氏就把
老情人丢到脑后,接受了这位第三者。

然而在另外一本史籍中,事情的真相并非如此。

\section[\thesection]{}

几年后,一个叫宋起凤的人跟随父亲到了京城。因为他家和宫里太监关系不错,所以经常进宫转悠,在这里他看到
很多,也听到了很多。

几十年后,他把自己当年的见闻写成了一本书,取名《稗说》。

所谓稗,就是野草。宋起凤先生的意思是,他的这本书,是野路子,您看了爱信不信,就当图个乐,他不在乎。

但就史料价值而言,这本书是相当靠谱的。因为宋起凤不是东林党,不是阉党,不存在立场问题,加上他在宫里混
的时间长,许多事是亲身经历,没有必要胡说八道。

这位公正的宋先生,在他的野草书里,告诉我们这样一句话:

``魏虽腐余,势未尽,又挟房中术以媚,得客欢。''

这句话,通俗点说就是,魏忠贤虽然割了,但没割干净。后半句儿童不宜,我不解释。

按此说法,有这个优势,魏忠贤要抢魏朝的老婆,那简直是一定的。

能说话,敢花钱,加上还有太监所不及的特长,魏忠贤顺利地打败了魏朝,成为了客氏的新对食。

说穿了,对食就是谈恋爱,谈恋爱是讲规则的,你情我愿,谈崩了,女朋友没了,回头再找就是了。

但魏朝比较惨,他找不到第二个女朋友。

因为魏忠贤是个无赖,无赖从来不讲规则,他不但要抢魏朝的女朋友,还要他的命。天启元年(1620),在客氏的配
合下,魏朝被免职发配,并在发配的路上被暗杀。

魏忠贤之所以能够除掉魏朝,是因为王安。

作为三朝元老太监,王安已经走到了人生的顶点,现在的皇帝,乃至于皇帝他爹,都是他扶上去的,加上东林党都
是他的好兄弟,那真是天下无敌,比东方不败猛了去了。可是王安也有一个致命的弱点--喜欢高帽子。

高帽子,就是拍马屁。所谓``千穿万穿,马屁不穿'',真可谓是至理名言,无论这人多聪明,多精明,只要找得
准,拍得狠,都不堪一击。

自盘古开天辟地以来,我们就知道,马屁,是有声音的。但魏忠贤的马屁,打破了这个俗套,达到马屁的最高境
界----无声之屁。

\section[\thesection]{}

每次见王安,魏忠贤从不主动吹捧,也不说话,只是磕头,王安不叫他,他就不去,王安不问他,他就不说话。王
安跟他说话,他不多说,态度谦恭点到即止。

他不来虚的,尽搞实在的,逢年过节送东西,还是猛送,礼物一车车往家里拉。于是当魏朝和魏忠贤发生争斗的时
候,他全力支持了魏忠贤,赶走了魏朝。但他并不知道,魏忠贤的目标并不是魏朝,而是他自己。

此时的魏忠贤已经站在了门槛上,只要再走一步,他就能获取至高无上的权力。但是王安,就站在他的面前。必须
铲除此人,才能继续前进。

跟之前对付魏朝一样,魏忠贤毫无思想障碍,朋友是可以出卖的,上级自然可以出卖,作为一个无赖、混混、人
渣,无时无刻,他始终牢记自己的本性。

可是怎么办呢?

王安不是魏朝,这人不但地位高,资格老,跟皇帝关系好,路子也猛,东林党的杨涟、左光斗都经常去他家串门。

要除掉他,似乎绝无可能。

但是魏忠贤办到了,用一种匪夷所思的方式。

天启元年(1620),司礼监掌印太监卢受因为犯了事,被罢免了。在当时,卢受虽然地位高,势力却不大,所以这事
并不起眼。王安,正是栽在了这件并不起眼的事情上。

前面讲过,在太监里面,最牛的是司礼太监,包括掌印太监一人,秉笔太监若干人。

作为司礼监的最高领导,按照惯例,如职位空缺,应该由秉笔太监接任。在当时而言,就是王安接任。

必须说明,虽然王安始终是太监的实际领导,但他并不是掌印太监,具体原因无人知晓。可能是这位仁兄知道枪打
出头鸟,所以死不出头,想找人去顶缸。

但这次不同了,卢受出事后,最有资历的就剩下他,只能自己干了。

但魏忠贤不想让他干,因为这个位置太过重要,要让王安坐上去,自己要出头,只能等下辈子了。

可是事实如此,生米做成了熟饭,魏忠贤无计可施。

王安也是这么想的,他打点好一切,并接受了任命。按照以往的惯例,写了一封给皇帝的上疏。主要意思无非是我
无才无能,干不了,希望皇上另找贤能之类的话。

接受任命后,再写这些,似乎比较虚伪,但这也是没办法,在我们这个有着光荣传统的地方,成功是不能得意的,
得意是不能让人看见的。

几天后,他得到了皇帝的回复:同意,换人。

\section[\thesection]{}

王安自幼入宫,从倒马桶干起,熬到了司礼监,一向是现实主义者,从不相信什么神话。但这次,他亲眼看见了神
话。

写这封奏疏,无非是跟皇帝客气客气,皇帝也客气客气,然后该干嘛干嘛,突然来这么一杠子,实在出人意料。但
更出人意料的是,没过多久,他就被勒令退休,彻底赶出了朝廷。而那个他亲手捧起的朱由校,竟然毫无反应。

魏忠贤,确实是一个聪明绝顶的人,在苦思冥想后,他终于找到了这个不是机会的机会:你要走,我批准,实在是
再自然不过。

但这个创意的先决条件是,皇帝必须批准,这是有难度的。因为皇帝大人虽说喜欢当木工,也没啥文化,但要他下
手坑捧过他的王公公,实在需要一个理由。

魏忠贤帮他找到了这个理由:客氏。

乳母、保姆、外加还可能有一腿,凭如此关系,要他去办掉王公公,应该够了。

王安失去了官职,就此退出政治舞台,凄惨离去。此时他才明白,几十年的宦海沉浮,尔虞我诈的权谋,扶植过两
位皇帝的功勋,都抵不上一个保姆。

心灰意冷的他打算回去养老,却未能如愿。因为一个人下定决心,要斩草除根,这人不是魏忠贤。

以前曾有个人问我,在整死岳飞的那几个人里,谁最坏?

我不假思索地回答,当然是秦桧。

于是此人脸上带着欠揍的表情,微笑着对我说,不对,是秦桧他老婆。

我想了一下,对他说:你是对的。

我想起了当年读过的那段记载,秦桧想杀岳飞,却拿不定主意干不干,于是他的老婆,李清照的表亲王氏告诉他,
一定要干,必须要干,不干不行,于是他干了。

魏忠贤的情况大致如此,这位仁兄虽不认朋友,倒还认领导,想来想去,对老婆客氏说,算了吧。

然后,客氏对他说了这样几句话:

``移宫时,对外传递消息,说李选侍挟持太子的,是王安,东林党来抢人,把太子拉走的,是王安;和东林党串
通,逼李选侍迁出乾清宫的,还是王安。此人非杀不可!''

说这句话的时候,她的表情十分严肃,态度十分认真。

女人比男人更凶残,信乎。

\section[\thesection]{}

魏忠贤听从了老婆的指示,他决定杀掉王安。

这事很难办,皇帝大人比魏忠贤厚道,他固然不用王安,却绝不会下旨杀他。

但在魏忠贤那里,就不难办了。因为接替王安,担任司礼监掌印太监的,是他的心腹王体乾,而他自己,是司礼监
秉笔太监兼东厂提督太监,大权在握,想怎么折腾都行,反正皇帝大人每天都做木匠,也不大管。

很快,王安就在做苦工的时候,发生了意外,夜里突然就死掉了,后来报了个自然死亡,也就结了。

至此,魏忠贤通过不懈的无耻和卑劣,终于掌握了东厂的控制权,成为了最大的特务。皇帝的往来公文,都要经过
他的审阅,才能通过,最少也是一言八鼎了。然而,每次有公文送到时,他都不看,因为他不识字。

在文盲这一点上,魏忠贤是认账且诚实的,但他并没有因此耽误国家大事,总是把公文带回家,给他的狗头军师们
研究,有用的用,没用的擦屁股垫桌脚,做到物尽其用。

入宫三十多年后,魏忠贤终于走到了人生的高峰。

但还不是顶峰。

战胜了魏朝,除掉了王安,搞定了皇帝,但这还不够,要想成为这个国家的真正统治者,必须面对下一个,也是最
后一个敌人----东林党。

于是,在成为东厂提督太监后不久,魏忠贤经过仔细思考、精心准备,对东林党发动攻击。

具体行动包括,派人联系东林党的要人,包括刘一璟、周嘉谟、杨涟等人,表示自己刚上来,许多事情还望多多关
照,并多次附送礼物。

此外,他还在公开场合,赞扬东林党的某些干将,兴奋之情溢于言表。

更让人感动的是,他多次在皇帝面前进言,说东林党的赵南星是国家难得的人才,工作努力认真,值得信赖,还曾
派自己的亲信上门拜访,表达敬意。

除去遭遇车祸失忆,意外中风等不可抗力因素,魏忠贤突然变好的可能性,大致是0\%,所以结论是,这些举动都是
伪装。在假象的背后,隐藏着不可告人的秘密。

这个不可告人的秘密就是:魏忠贤想跟东林党做朋友。

有必要再申明一次,这句话我没有写错。

其实我们这个国家的历史,一向是比较复杂的。所谓你中有我,我中有你,能凑合就凑合,能糊弄就糊弄。向上追
溯,真正执着到底,绝不罢休的,估计只有山顶洞人。

魏忠贤并不例外,他虽然不识字,却很识相。

\section[\thesection]{}

他非常清楚,东林党这帮人不但手握重权,且都是读书人,其实手握重拳并不可怕,书呆子才可怕。

自古以来,读书人大致分为两种,一种叫文人,另一种叫书生。文人是``文人相轻'',具体特点为比较无耻外加自
卑。你好,他偏说坏;你行,他偏说不行;胆子还小,平时骂骂咧咧,遇上动真格的,又把头缩回去,实在是相当
之扯淡。

而书生的主要特点,是``书生意气'',表现为二杆子加一根筋。好就是好,不好就是不好,认死理,平时不惹事,
事来了不怕死。关键时刻敢于玩命,文弱书生变身钢铁战士,不用找电话亭,不用换衣服,眨眼就行。

当年的读书人,还算比较靠谱,所以在东林党里,这两种人都有,后者占绝大多数,形象代言人就是杨涟,咬住就
不撒手,相当头疼。

这种死脑筋,敢于乱来的人,对于见机行事、欺软怕硬的无赖魏忠贤而言,实在是天然的克星。

所以魏忠贤死乞白赖地要巴结东林党,他实在是不想得罪这帮人。这世道,大家都不容易,混碗饭吃嘛,我又不想
当皇帝,最多也就是个成功太监,你们之前跟王安合作愉快,现在我来了,不过是换个人,有啥不同的。

对于魏忠贤的善意表示,东林党的反应是这样的:上门的礼物,全部退回去,上门拜访的,赶走。

最不给面子的,是赵南星。

在东林党人中,魏忠贤最喜欢赵南星,因为赵南星和他是老乡,容易上道,所以他多次拜见,还人前人后,逢人便
夸赵老乡如何如何好。

可是赵老乡非但不领情,拒不见面。有一次,还当着很多人的面,针对魏老乡的举动,说了这样一句话:宜各努力
为善。

联系前后关系,这句话的隐含意思是,各自干好各自的事就行了,别动歪心思,没事少烦我。

魏忠贤就不明白了,王安你们都能合作,为什么不肯跟我合作呢?

其实东林党之所以不肯和魏忠贤合作,不是因为魏忠贤是文盲,不是因为他是无赖,只是因为,他不是王安。

没有办法,书生都是认死理的。虽然从本质和生理结构上讲,王安和魏忠贤实在没啥区别,都是太监,都是司礼
监,都管公文,但东林党一向是做熟不如做生,对人不对事,像魏忠贤这种无赖出身,行为卑劣的社会垃圾,他们
是极其鄙视的。

\section[\thesection]{}

应该说,这种思想是值得尊重的,值得敬佩的,却是绝对错误的。因为他们并不知道,政治的最高技巧,不是你死
我活,而是妥协。

魏忠贤愤怒了,他的愤怒是有道理的,不仅是因为东林党拒绝合作,更重要的是,他感觉自己被鄙视了。

这个世上的人分很多类,魏忠贤属于江湖类,这种人从小混社会,狐朋狗友一大串,老婆可以不要,女儿可以不
要,只有面子,是不能不要的。东林党的蔑视,给他那污浊不堪的心灵以极大的震撼,他痛定思痛,幡然悔悟,毅
然做出了一个决定:

既然不给脸,那就撕破脸吧!

但魏公公很快就发现,要想撕破脸,一点也不容易。

因为他是文盲。

解决魏朝、王安,只要手够狠,心够黑就行,但东林党不同,这些人都是知识分子,至少也是个进士,擅长朝廷斗
争,这恰好是魏公公的弱项。

在朝廷里干仗,动刀动枪是不行的,一般都是骂人打笔仗,技术含量相当之高,多用典故成语,保证把你祖宗骂绝
也没一脏字,对于字都不识的魏公公而言,要他干这活,实在有点勉为其难。

为了适应新形势下的斗争,不至于被人骂死还哈哈笑,魏公公决定找几个助手,俗称走狗。

最早加入,也最重要的两个走狗,分别是顾秉谦与魏广微。

顾秉谦,万历二十三年(1595)进士,坏人。

此人翰林出身,学识过人,无耻也过人,无耻到魏忠贤没找他,他就自己上门去了。

当时他的职务是礼部尚书,都七十一了,按说干几年就该退休,但这孙子偏偏人老心不老,想更进一步,大臣又瞧
不上他,索性投了太监。

改变门庭倒也无所谓,这人最无耻的地方在于,他干过这样一件事:

有一次为了升官,顾秉谦先生不顾自己七十高龄,带着儿子登门拜访魏忠贤,说了这样一段话:

``我希望认您做父亲,但又怕您觉得我年纪大,不愿意,索性让我的儿子给您做孙子吧!''

顾秉谦,嘉靖二十九年(1550)生,魏忠贤,隆庆二年(1568)出生。顾秉谦比魏忠贤大十八岁。

无耻,无语。

\section[\thesection]{}

魏广微,万历三十二年(1604)进士,可好可坏的人。

魏广微的父亲,叫做魏允贞,魏允贞有一个最好的朋友,叫做赵南星。

万历年间,魏允贞曾当过侍郎。他和赵南星的关系很好,两人曾有八拜之交,用今天话说,是拜过把子的把兄弟。

魏广微的仕途比较顺利,考中翰林,然后步步高升,天启年间,就当上了礼部侍郎。

按说这个速度不算慢,可魏先生是个十分有上进心的人,为了实现跨越性发展,他找到了魏忠贤。

魏公公自然求之不得,仅过两年,就给他提级别,从副部长升到部长,并让他进入内阁,当上了大学士。

值得表扬的是,魏广微同志有了新朋友,也不忘老朋友。上任之后,第一件事就去拜会父亲当年的老战友赵南星。

但赵南星没有见他,让他滚蛋的同时,送给了他四个字:``见泉无子!''

魏广微之父魏允贞,字见泉。

这是一句相当狠毒的话,你说我爹没有儿子,那我算啥?魏广微十分气愤。

气愤归气愤,他还是第二次上门,要求见赵南星。

赵南星还是没见他。

接下来,魏广微做出了一个出人意料的举动,他又去了。

魏先生不愧为名门之后,涵养很好,当年刘备请诸葛亮出山卖命,也就三次,魏广微不要赵大人卖命,吃顿饭聊聊
天就好。

但赵南星还是拒而不见。

面对着紧闭的大门,魏广微怒不可遏,立誓,与赵南星势不两立。

魏广微之所以愤怒,见不见面倒是其次,关键在于赵南星坏了规矩。

当时的赵南星,是吏部尚书,人事部部长,魏广微却是礼部尚书,东阁大学士。虽说两人都是部长,但魏广微是内
阁成员,相当于副总理,按规矩,赵部长还得叫他领导。但魏大学士不计较,亲自登门,还三次,您都不见,实在
有点太不像话。

就这样,这个可好可坏的人,在赵南星的无私帮助下,变成了一个彻底的坏人。

除了这两人外,魏忠贤的党羽还有很多,如冯铨、施凤来、崔呈秀、许显纯等等,后人统称为:五虎、五彪、十狗、
十孩儿,光这四拨人加起来,就已有三十个。

\section[\thesection]{}

这还是小儿科,魏公公的手下,还有二十孩儿、四十猴孙、五百义孙,作为一个太监,如此多子多孙,实在是有福
气。

我曾打算帮这帮太监子孙亮亮相,搞个简介,起码列个名,但看到``五百义孙''之类的字眼时,顿时失去了勇气。

其实东林党在拉山头、搞团体等方面,也很有水平。可和魏公公比起来,那就差得多了。

因为东林党的入伙标准较高,且渠道有限:要么是同乡(乡党),同事(同科进士),要么是座主(师生关系),除个别
有特长者外(如汪文言),必须是高级知识分子(进士或翰林),还要身家清白,没有案底(贪污受贿)。

而魏公公就开放得多了,他本来就是无赖、文盲,还兼职人贩子(卖掉女儿),要找个比他素质还低的人,那是比较
难的。

所以他收人的时候,非常注意团结。所谓英雄莫问出处,富贵不思来由,阿猫阿狗无所谓,能干活就行,他手下这
帮人也还相当知趣,纷纷用``虎''、``彪''、``狗''、``猴''自居,甭管是何禽兽,反正不是人类。

这帮妖魔鬼怪构成很复杂,有太监、特务、六部官员、地方官、武将,涉及各个阶层,各个行业,百花齐放。

虽然他们来自不同领域,但有一点是相同的:他们都是经过精挑细选,纯度极高的人渣。

比如前面提到的四位仁兄,即很有代表性:

崔呈秀,原本是一贪污犯,收了人家的钱,被检举丢了官,才投奔魏公公。

施凤来,混迹朝廷十余年,毫无工作能力,唯一的长处是替人写碑文。

许显纯,武进士出身,锦衣卫首领,残忍至极,喜欢刑讯逼供,并有独特习惯:杀死犯人后,将其喉骨挑出,作为
凭证,或作纪念。

但相对而言,以上三位还不够份,要论王八蛋程度,还是冯铨先生技高一筹。这位仁兄全靠贪污起家,并主动承担
陷害杨涟、左光斗等人的任务,唯恐坏事干得不够多,更让人称奇的是,后来这人还主动投降了清朝,成为了不知
名的汉奸。

短短一生之中,竟能集贪官、阉党、汉奸于一体,如此无廉耻,如此无人格,说他是禽兽,那真是侮辱了禽兽。

综上所述,魏忠贤手下这帮人,在工作和生活中,有着这样一个特点:

什么都干,就是不干好事,什么都要,就是不要脸。

\section[\thesection]{}

其实阉党之中的大多数人,都曾是三党的成员,在彻底出卖自己的灵魂和躯体,加入这个温馨的集体,成为毫无廉
耻的禽兽之前,他们曾经也是人。

多年以前,当他们刚踏入朝廷的时候,都曾品行端正满怀理想,立志以身许国,匡扶天下,公正地对待每一个人,
谨言慎行,并最终成为一个青史留名的伟人。

但他们终究倒下了,在残酷的斗争、仕途的磨砺、党争的失败面前,他们失去了最后的勇气和尊严,并最终屈服,
屈服于触手可及的钱财、权位和利益。

魏忠贤明白,坚持理想的东林党,是绝不可能跟他合作的,要想继续好吃好喝混下去,就必须解决这些人,现在,
他准备摊牌了。

但想挑事,总得有个由头,东林党这帮人都是道德先生,也不怎么收黑钱,想找茬整顿他们,是有相当难度的。

考虑再三之后,魏忠贤找到了一个看似完美的突破口----汪文言。

作为东林党的智囊,汪文言起着极其关键的作用,左推右挡来回忽悠,拥立了皇帝,搞垮了三党,人送外号``天下
第一布衣''。

但在魏忠贤看来,这位布衣有个弱点:他没有功名,不能做官,只能算是地下党。对这个人下手,即不会太显眼,
又能打垮东林党的支柱,实在是一举两得。

所以在王安死后,魏忠贤当即指使顺天府府丞绍辅忠,弹劾汪文言。

要整汪文言,是比较容易的。这人本就是个老油条,除东林党外,跟三党也很熟。后来三党垮了,他跟阉党中的许
多人关系也很铁,经常来回倒腾事儿,收人钱财,替人消灾,底子实在太不干净。

更重要的是,他的老东家王安倒了,靠山没了,自然好收拾。

事实恰如所料,汪文言一弹就倒,监生的头衔没收,还被命令马上收拾包裹滚蛋。

汪文言相当听话,也不闹,乖乖地走人了,可他还没走多远,京城里又来了人,从半道上把他请了回去----坐牢。

赶走汪文言,是不够的,魏忠贤希望,能把这个神通广大又神秘莫测的人一棍子打死,于是他指使御史弹了汪先生
第二下,把他直接弹进了牢房里。

魏忠贤终于满意了,行动进行极其顺利,汪文言已成为阶下囚,一切都已准备妥当,下面……

下面没有了。

\section[\thesection]{}

因为不久之后,汪文言就出狱了。

此时的魏忠贤是东厂提督太监、掌控司礼监、党羽遍布天下,而汪先生是个没有功名,没有身份,失去靠山的犯人。
并且魏公公很不喜欢汪文言,很想把他打翻在地,再踏上一只脚,这看上去,似乎是件十分容易的事情。毕竟连汪
文言的后台王安,都死在了魏忠贤的手中。

无论如何,他都不应该、不可能出狱。然而他就是出狱了。

他到底是怎么出狱的,我不知道,反正是出来了,成功自救,魏公公也毫无反应,王安都没有办到的事情,他办到
了。

而且这位仁兄出狱之后,名声更大,赵南星、左光斗、杨涟都亲自前来拜会慰问,上门的人络绎不绝,用以往革命
电影里的一句话:坐牢还坐出好来了。

更出人意料的是,不久之后,朝廷首辅叶向高主动找到了他,并任命他为内阁中书。

所谓内阁中书,大致相当于国务院办公厅主任,是个极为重要的职务。汪文言先生连举人都没考过,竟然捞到这个
位置,实在耸人听闻。

而对这个严重违背常规的任命,魏公公竟然沉默是金,什么话都不说。因为他已经意识到,自己还没有足够的实
力,去战胜这个神通广大的人。

于是,魏忠贤停止了行动,他知道,要打破目前的僵局,必须继续等待。

此后的三年里,悄无声息之中,他不断排挤东林党,安插自己的亲信,投靠他的人越来越多,他的党羽越来越庞
大,实力越来越强,但他仍在沉默中等待。

因为他已看清,这个看似强大的东林党,实际上非常脆弱,吏部尚书赵南星不可怕,佥都御史左光斗不可怕,甚至
首辅叶向高,也只是一个软弱的盟友。

真正强大的,只有这个连举人都考不上,地位卑微,却机智过人,狡猾到底的汪文言,要解决东林党,必须除掉这
个人,没有任何捷径。

这是一件非常冒险的事,魏忠贤不喜欢冒险,所以他选择等待。

但事情的发展,超出了所有人的预料,包括魏忠贤在内。

天启四年(1624)吏科给事中阮大铖上书,弹劾汪文言、左光斗互相勾结,祸乱朝政。

热闹就此开始,阉党纷纷加入,趁机攻击东林党,左光斗也不甘示弱,参与论战,朝廷上下,口水滔滔,汪文言被
免职,连首辅叶向高也申请辞职,乱得不可开交。

但讽刺的是,对于这件事,魏忠贤事先可能并不知道。

\section[\thesection]{}

这事之所以闹起来,无非是因为吏科都给事中退了,位置空出来,阮大铖想要进步,就开始四处活动,拉关系。

偏偏东林党不吃这套,人事部长赵南星听说这事后,索性直接让他滚出朝廷,连给事中都不给干。阮大铖知道后,
十分愤怒,决定告左光斗的黑状。

这是句看上去前言不搭后语的话,赵南星让他滚,关左光斗何事?

原因在于,左光斗是阮大铖的老乡,当年阮大铖进京,就是左光斗抬举的。所以现在他升不了官,就要找左光斗的
麻烦。

看起来,这个说法仍然比较乱,不过跟``因为生在荆楚之地,所以就叫萌萌''之类的逻辑相比,这种想法还算正常。

这位逻辑``还算正常''的阮大铖先生,真算是奇人。可以多说几句。后来他加入了阉党,跟着魏忠贤混,混砸了又
跑到南京,跟着南明混,南明混砸了,他又加入满清,在满清军营里,他演出了人生中最精彩,最无耻的一幕。

作为投降的汉奸,他毫无羞耻之心,还经常和满清将领说话。白天说完,晚上接着说,说得人家受不了,对他说:
您口才真好,可我们明天早起还要打仗,早点洗了睡吧。

此后不久,他因急于抢功跑得太快,猝死于军中。

但在当时,阮大铖先生这个以怨报德的黑状,只是导火索。真正让魏公公极为愤怒,痛下杀手的,是另一件事。准
确地说,是另一笔钱。

其实一直以来,魏公公虽和东林党势不两立,却只有公愤,并无私仇。但几乎就在阮大铖上书的同一时刻,魏公公
得到消息,他的一笔生意黄了,就黄在东林党的手上。

这笔生意值四万两银子,和他做生意的人,叫熊廷弼。

希望大家还记得这兄弟,自从回京后,他已经被关了两年多了,由于情节严重,上到皇帝下到刑部,倾向性意见相
当一致----杀。

事到如今,只能开展自救了,熊廷弼开始积极活动,找人疏通关系,希望能送点钱,救回这条命。

七转八转,他终于找到了一位叫做汪文言的救星,据说此人神通广大,手到擒来。

汪文言答应了,开始活动,他七转八转,找到了一个能办事的人----魏忠贤。

当然,鉴于魏忠贤同志对他极度痛恨,干这件事的时候,他没有露面,而是找人代理。

\section[\thesection]{}

魏忠贤接到消息,欣然同意,并开出了价码----四万两,熊廷弼不死。

汪文言非常高兴,立刻回复了熊廷弼,告诉他这个好消息,以及所需银子的数量(很可能不是四万两,毕竟中间人也
要收费)

以汪文言的秉性,拿中介费是一定的,拿多少是不一定的,但这次,他一文钱也没拿到。因为熊廷弼拿不出四万两。
拿不出钱来,事情没法办,也就没了下文。

但魏忠贤不知是手头紧,还是办事认真负责,发现这事没消息了,就好了奇,派人去查。七转八转,终于发现那个
托他办事的人,竟然是汪文言!

过分了,实在过分了,魏忠贤感受到了出离的愤怒:和我作对也就罢了,竟然还要托我办事,吃我的中介费!

拿不到钱,又被人耍了一把的魏忠贤国仇家恨顿时涌上心头,当即派人把汪文言抓了起来。

汪文言入狱了,但这只是开始,魏忠贤的最终目标,是通过他,把东林党人拉下水。

但事实再一次证明,冲动是魔鬼。一时冲动的魏公公惊奇地发现,他又撞见鬼了,汪文言入狱后,审来审去毫无进
展,别说杨涟、左光斗,就连汪先生自己也在牢里过得相当滋润。

之所以出现如此怪象,除汪先生自己特别能战斗外,另一个人的加入,也起了极大的作用。

这个人名叫黄尊素,时任都察院监察御史。

这是一个很有名的人,知道他的人比较多,但他还有个更有名的儿子----黄宗羲。如果连黄宗羲都不知道,应该回
家多读点书。

在以书生为主的东林党里,黄尊素是个异类。此人深谋远虑,凡事三思而行,擅长权谋,与汪文言并称为东林党两
大智囊。

得知汪文言被抓后,许多东林党人都很愤怒,但也就是发发牢骚,真正做出反应的,只有两个人,其中一个,就是
黄尊素。

他敏锐地感觉到,魏忠贤要动手了。

抓汪文言只是个开头,很快,这场战火就将延伸到东林党的身上。到时一切都迟了。

于是,他连夜找到了锦衣卫刘侨。

刘侨,时任锦衣卫镇抚司指挥使,管理诏狱,汪文言就在他地盘坐牢。

这人品格还算正派,所以黄尊素专程找到他,疏通关系。

黄尊素表示,人你照抓照关,但万万不能牵涉到其他人,比如左光斗、杨涟等等。

刘侨答应了。

\section[\thesection]{}

刘侨是个聪明人,他明白黄尊素的意思。便照此意思吩咐审讯工作,所以汪文言在牢里满口胡话,也没人找他麻烦。

而另一个察觉魏忠贤企图的人,是叶向高。

叶向高毕竟是见过世面的,几十年朝廷混下来,一看就明白。即刻上书表示汪文言是自己任命的,如果此人有问
题,就是自己责任,与他人无关,特请退休回家养老。

叶首辅不愧为老狐狸,他明知道,朝廷是不会让自己走的,却偏要以退为进,给魏忠贤施加压力,让他无法轻举妄
动。

看到对方摆出如此架势,魏忠贤退缩了。

太冲动了,时候还没到。

在这个回合里,东林党获得了暂时的胜利,却将迎来永远的失败。

抓汪文言时,魏忠贤并没有获胜的把握,但到了天启四年(1624)五月,连东林党都不再怀疑自己注定失败的命运。

因为魏公公实在太能拉人了。

几年之间,所谓``众正盈朝''已然变成了``众兽盈朝''。魏公公手下那些飞禽走兽已经遍布朝廷,王体乾掌控了司
礼监,顾秉谦、魏广微进入内阁,许显纯、田尔耕控制锦衣卫。六部里,只有吏部部长赵南星还苦苦支撑,其余各
部到处都是阉党,甚至管纪检监察的都察院六科,都成为了阉党的天下。

对于这一转变,大多数书上的解释是世风日下,人心不古,道德沦丧,品质败坏等等等等。

其实原因很简单,就一句话:实在。

魏忠贤能拉人,因为他实在。

你要人家给你卖命,拿碗白饭对他说,此去路远,多吃一点,那是没有效果的。毕竟千里迢迢,不要脸面,没有廉
耻来投个太监,不见点干货,心理很难平衡。

在这一点上,魏公公表现得很好,但凡投奔他的,要钱给钱,要官给官,真金实银,不打白条。

相比而言,东林党的竞争力实在太差,什么都不给还难进,实在有点难度过高。

如果有人让你选择如下两个选项:坚持操守,坚定信念和理想,一生默默无闻,家徒四壁,为国为民,辛劳一生。

或是放弃原则,泯灭良心,少奋斗几十年,青云直上,升官发财,好吃好喝,享乐一生。

嗟乎!大阉之乱,以缙绅之身而不改其志者,四海之大,有几人欤?

----《五人墓碑记》

不用回答,我们都知道答案。

\section[\thesection]{}

很久以前,我曾经看过一部电影,电影里的黑社会老大在向他的手下训话,他说,昨天晚上他做了一个梦,梦见这
个世界上没有黑社会了。

因为这个世界上的人,都变成了黑社会。

这句话在魏忠贤那里,已不再是梦想。

他不问出身,不问品格,将朝廷大权赋予所有和他一样卑劣无耻的人。

而这些靠跪地磕头、自认孙子才掌握大权的人,自然没有什么造福人民的想法,受尽屈辱才得到的荣华富贵,不屈
辱一下老百姓,怎么对得起自己呢?

在这种良好愿望的驱使下,某些匪夷所思的事情,开始陆续发生。比如某县有位富翁,闲来无事杀了个人,知县秉
公执法,判了死刑。这位仁兄不想死,就找到一位阉党官员,希望能够拿钱买条命。

很快他就得到了答复:一万两。

这位财主同意了,此外他还提出了一个要求:希望杀掉那位判他死刑的知县,因为这位县太爷太过公正,实在让他
不爽。

要说还是阉党的同志们实在,收钱之后立马放人,并当即捏造了罪名,把那位知县干掉了。

无辜的被害者,正直的七品知县,司法、正义,全加在一起,也就一万两。

事实上,这个价码还偏高。

搞到后来,除封官许愿外,魏忠贤还开发了新业务:卖官!有些史料还告诉我们,当时的官职都是明码标价,买个知
县,大致是两三千两,要买知府,五六千两也就够了。

如此看来,那位草菅人命的财主,还真是不会算帐。索性找到魏公公,花一半钱买个知府,直接当那知县的上级,
找个由头把他干掉,还能省五千两,亏了,真亏了。

自开朝以来,大明最黑暗的时刻,终于到来!

我们想干什么就干什么,我们想怎么干就怎么干,为了获取权力和财富,所付出的尊严和代价,要从那些更为弱小
的人身上加倍掠夺。蹂躏、欺凌、劫掠,不用顾忌,不用考虑,我们可以为所欲为!

因为在这个时代,没有人能阻止我们,没有人敢阻止我们!

道统

几年来,杨涟一直在看。

他看见那个无恶不作的太监,抢走了朋友的情人,杀死了朋友,坑死了上司,却掌握了天下的大权,无需偿命,没
有报应。

那个叫天理的东西,似乎并不存在。

\section[\thesection]{}

他看见,一个无比强大的敌人,已经出现在自己的面前。

在明代历史上,从来不缺重量级的坏人,比如刘瑾,比如严嵩,但刘瑾多少还读点书,知道做事要守规矩,至少有
个底线,所以他明知李东阳和他作对,也没动手杀人。严嵩虽说杀了夏言,至少还善待自己的老婆。

而魏忠贤,是一个文盲,逼走老婆,卖掉女儿,他没原则,没底线,阴险狡诈,不择手段,已达到了无耻无极限的
境界。他绝了后,也空了前。

当杨涟回过神来,他才发现,自己身边,已是空无一人,那些当年的敌人、甚至朋友、同僚都已抛弃良知,投入了
这个人的怀抱。在利益的面前,良知实在太过脆弱。

但他依然留在原地,一动不动,因为他依然坚持着一样东西----道统。

所谓道统,是一种规则,一种秩序,是这个国家几千年来历经苦难挫折依旧前行的动力。

杨涟和道统已经认识很多年了。

小时候,道统告诉他,你要努力读书,研习圣人之道,将来报效国家。

当知县时,道统告诉他,你要为官清廉,不能贪污,不能拿不该拿的钱,要造福百姓。

京城,皇帝病危,野心家蠢蠢欲动,道统告诉他,国家危亡,你要挺身而出,即使你没有义务,没有帮手。

一直以来,杨涟对道统的话都深信不疑,他照做了,并获得了成功:

是你让我相信,一个普通的平民子弟,也能够通过自己的努力,坚持不懈,成就一番事业,成为千古留名的人物。
  

你让我相信,即使身居高位,尊容加身,也不应滥用自己的权力,去欺凌那些依旧弱小的人。

你让我相信,一个人活在这世上,不能只是为了自己。他应该清正廉洁,严于律己,坚守那条无数先贤走过的道
路,继续走下去。

但是现在,我有一个疑问:

魏忠贤是一个不信道统的人,他无恶不作,肆无忌惮,没有任何原则,但他依然成为了胜利者,越来越多的人放弃
了道统,投奔了他,只是因为他封官给钱,如同送白菜。

我的朋友越来越少,敌人越来越多,在这条道路上,我已是孤身一人,

道统说:是的,这条道路很艰苦,门槛高,规矩多,清廉自律,家徒四壁,还要立志为民请命,一生报效国家,实
在太难。

那我为何还要继续走下去呢?

\section[\thesection]{}

因为这是一条正确的道路,几千年来,一直有人走在这条孤独的道路上,无论经过多少折磨,他们始终相信规则,
相信每个人都有着自己的尊严和价值,相信这个世界上,存在着公理与正义,相信千年之下,正气必定长存。

是的,我明白了,现在轮到我了,我会坚守我的信念,我将对抗那个强大的敌人,战斗至最后一息,即使孤身一人。

好吧,杨涟,现在我来问你,最后一个问题:

为了你的道统,牺牲你的一切,可以吗?

可以

杨涟

天启四年(1624)六月,左副都御史杨涟写就上疏,弹劾东厂提督太监魏忠贤二十四大罪。

在这篇青史留名的檄文中,杨涟历数了魏忠贤的种种罪恶,从排除异己、陷害忠良、图谋不轨、杀害无辜,可谓世
间万象,无所不包,且真实可信,字字见血。

由此看来,魏忠贤确实是人才,短短几年里,跨行业、跨品种,坏事干得面面俱到,着实不易。

这是杨涟的最后反击,与其说是反击,不如说是愤怒。因为连他自己都很清楚,此时的朝廷,从内阁到六部,都已
是魏忠贤的爪牙。按照常理,这封奏疏只要送上去,必定会落入阉党之手,到时只能是废纸一张。

杨涟虽然正直,却并非没有心眼,为了应对不利局面,他想出了两个办法。

他写完这封奏疏后,并没有遵守程序,把它送到内阁,而是随身携带,等待着第二天的到来。

因为在这一天,皇帝大人将上朝议事,那时,杨涟将拿出这封奏疏,亲口揭露魏忠贤的罪恶。

在清晨的薄雾中,杨涟怀揣着奏疏,前去上朝,此时除极个别人外,无人知道他的计划,和他即将要做的事。

然而当他来到大殿前的时候,却得到一个让人哭笑不得的消息:皇帝下令,今天不办公(免朝)。

紧绷的神经顿时松弛了下来,杨涟明白,这场生死决战又延迟了一天。

只能明天再来了。

但就在他准备打道回府之际,却突然意识到一个问题,于是他改变了主意。

杨涟走到了会极门,按照惯例,将这封奏疏交给了负责递文书的官员。

在交出文书的那一刻,杨涟已然确定,不久之后,这份奏疏就会放在魏忠贤的文案上。

之所以做此选择,是因为他别无选择。

\section[\thesection]{}

杨涟是一个做事认真谨慎的人,他知道,虽然此事知情者很少,但难保不出个把叛徒,万一事情曝光,以魏公公的
品行,派个把东厂特务把自己黑掉,也不是不可能的。

不能再等了,不管魏忠贤何时看到,会不会在上面吐唾沫,都不能再等了。

第一个办法失败了,杨涟没能绕开魏忠贤,直接上书。事实上,这封奏疏确实落到了魏忠贤的手中。

魏忠贤知道这封奏疏是告他的,但不知是怎么告的,因为他不识字。所以,他找人读给他听

但当这位无恶不作、肆无忌惮的大太监听到一半时,便打断了朗读,不是歇斯底里的愤怒,而是面无人色的恐惧。

魏忠贤害怕了,这位不可一世,手握大权的魏公公,竟然害怕了。

据史料的记载,此时的魏公公面无人色,两手不由自主颤抖,并且半天沉默不语。

他已经不是四年前那个站在杨涟面前,被骂得狗血淋头,哆哆嗦嗦的老太监了。现在他掌握了内阁,掌握了六部,
甚至还掌握了特务,他一度以为,天下再无敌手。

但当杨涟再次站在他面前的时候,他才明白,纵使这个人孤立无援、身无长物,他却依然畏惧这个人,深入骨髓的
畏惧。

极度的恐慌彻底搅乱了魏忠贤的神经,他的脑海中只剩下一个念头:绝对不能让这封奏疏传到皇帝的手中!

奏疏倒还好说,魏公公一句话,说压就压了,反正皇帝也不管。但问题是,杨涟是左副都御史,朝廷高级官员,只
要皇帝上朝,他就能够见到皇帝,揭露所有一切。

怎么办呢?魏忠贤冥思苦想了很久,终于想出了一个没办法的办法:不让皇帝上朝。

在接下来的三天里,皇帝都没有上朝。

但这个办法实在有点蠢,因为天启皇帝到底是年轻人,到第四天,就不干了,偏要去上朝。

魏忠贤头疼不已,但皇帝大人说要上朝,不让他去又不行,迫于无奈,竟然找了上百个太监,把皇帝大人围了起
来,到大殿转了一圈,权当是给大家一个交代。

此外,他还特意派人事先说明,不允许任何人发言。

总之,他的对策是,先避风头,把这件事压下去,以后再跟杨涟算帐。

\section[\thesection]{}

得知皇帝三天没有上朝,且目睹了那场滑稽游行的杨涟并不吃惊,事情的发展,早在他意料之中。

因为当他的第一步计划失败,被迫送出那份奏疏的时候,他就想好了第二个对策。

虽然魏忠贤压住了杨涟的奏疏,但让他惊奇的是,这封文书竟然长了翅膀,没过几天,朝廷上下,除了皇帝没看
过,大家基本是人手一份,还有个把缺心眼的,把词编成了歌,四处去唱,搞得魏公公没脸出门。

杨涟充分发挥了东林党的优良传统,不坐地等待上级批复,就以讲学传道为主要途径,把魏忠贤的恶劣事迹广泛传
播,并在短短几天之内,达到了妇孺皆知的效果。

比如当时国子监里的几百号人,看到这封奏疏后,欢呼雀跃,连书都不读了,每天就抄这份二十四大罪,抄到手
软,并广泛散发。

吃过魏公公苦头的人民大众自不用说,大家一拥而上,反复传抄,当众朗诵,成为最流行的手抄本。据说最风光的
时候,连抄书的纸都缺了货。

左光斗是少数几个事先的知情者之一,此时自然不甘人后,联同朝廷里剩余的东林党官员共同上书,斥责魏忠贤。
甚至某些退休在家的老先生,也来凑了把热闹。于是几天之内,全国各地弹劾魏忠贤的公文纸纷至沓来,堆积如
山,足够把魏忠贤埋了再立个碑。

眼看革命形势一片大好,许多原先是阉党的同志也坐不住了,唯恐局势变化自己垫背,一些人纷纷倒戈,掉头就骂
魏公公,搞得魏忠贤极其狼狈。

事实证明,广大人民群众对魏忠贤的愤怒之情,就如同那滔滔江水,延绵不绝。搞得连深宫之中的皇帝,都听说了
这件事,专门找魏忠贤来问话,到了这个地步,事情已经瞒不住了。

杨涟没有想到,自己的义愤之举,竟然会产生如此重大的影响,在他看来,照此形势发展,大事必成,忠贤必死。

然而有一个人,不同意杨涟的看法。

在写奏疏之前,为保证一击必中,杨涟曾跟东林党的几位重要人物,如赵南星、左光斗通过气,但有一个人,他没
有通知,这个人是叶向高。

由始至终,叶向高都是东林党的盟友,且身居首辅,是压制魏忠贤的最后力量,但杨先生就是不告诉他,偏不买他
的帐。

因为叶向高曾不只一次对杨涟表达过如下观点:

对付魏忠贤,是不能硬来的。

\section[\thesection]{}

叶向高认为,魏忠贤根基深厚,身居高位,且内有奶妈(客氏),外有特务(东厂),以东林党目前的力量,是无法扳
倒的。

杨涟认为,叶向高的言论,是典型的投降主义精神。

魏忠贤再强大,也不过是个太监。他手下的那帮人,无非是乌合之众,只要能够集中力量,击倒魏忠贤,就能将阉
党这帮人渣一网打尽,维持社会秩序、世界和平。

更何况,自古以来,邪不胜正。

邪恶是必定失败的!基于这一基本判断,杨涟相信,自己是正确的,魏忠贤终究会被摧毁。

历史已经无数次证明,邪不胜正是靠谱的,但杨涟不明白,这个命题有个前提条件----时间。

其实在大多数时间里,除去超人、蝙蝠侠等不可抗力出来维护正义外,邪是经常胜正的。所谓好人、善人、老实人
常常被整得凄惨无比,比如于谦、岳飞等等,都是死后多少年才翻身平反。

只有岁月的沧桑,才能淘尽一切污浊,扫清人们眼帘上的遮盖与灰尘,看到那些殉道者无比璀璨的光芒,历千年而
不灭。

杨涟,下一个殉道者。

很不幸,叶向高的话虽然不中听,却是对的。以东林党目前的实力,要干掉魏忠贤,是毫无胜算的。

但决定他们必定失败宿命的,不是奶妈,也不是特务,而是皇帝。

杨涟并不傻,他知道大臣靠不住,太监靠不住,所以他把所有的希望,都寄托在皇帝身上。希望皇帝陛下雷霆大
怒,最好把魏公公五马分尸再拉出去喂狗。

可惜,杨涟同志寄予厚望的天启皇帝,是靠不住的。

自有皇帝以来,牛皇帝有之,熊皇帝有之,不牛不熊的皇帝也有之,而天启皇帝比较特别:他是木匠。

身为一名优秀的木匠,明熹宗有着良好的职业素养,他经常摆弄宫里建筑。具体表现为在他当政的几年里,宫里经
常搞工程,工程的设计单位、施工、监理、检验,全部由皇帝大人自己承担。

更为奇特的是,工程的目的也很简单,修好了,就拆,拆完了,再修,以达到拆拆修修无穷尽之目的。总之,搞来
搞去,只为图个乐。

\section[\thesection]{}

这是大工程,小玩意天启同志也搞过。据史料记载,他曾经造过一种木制模型,有山有水有人,据说木人身后有机
关控制,还能动起来,纯手工制作,比起今天的遥控玩具有过之而无不及。

为检验自己的实力,天启还曾把自己的作品放到市场上去卖,据称能卖近千两银子,合人民币几十万。要换在今
天,这兄弟就不干皇帝,也早发了。

可是,他偏偏就是皇帝。

大明有无数木匠,但只有一个皇帝,无论是皇帝跑去做木匠,还是木匠跑来做皇帝,都是彻底地抓瞎。

当然,许多书上说这位皇帝是低能儿,从来不管政务,不懂政治,那也是不对的,虽然他把权力交给了魏忠贤,也
不看文件,不理朝廷,但他心里是很有数的。

比如魏公公,看准了皇帝不想管事,就爱干木匠,每次有重要事情奏报,他都专挑朱木匠干得最起劲的时候去,朱
木匠自然不高兴,把手一挥:我要你们是干什么的?

这句话在手,魏公公自然欢天喜地,任意妄为。

但在这句话后,朱木匠总会加上一句:好好干,莫欺我!

这句话的表面意思是,你不要骗我,但隐含意思是,我知道,你可能会骗我。

事实上,对魏忠贤的种种恶行,木匠多少还知道点,但在他看来,无论这人多好,只要对他坏,就是坏人;无论这
人多坏,只要对他好,就是好人。

基于这一观点,他对魏忠贤有着极深的信任,就算不信任他,也没有必要干掉他。

叶向高正是认识到这一点,才认定,单凭这封奏疏,是无法解决魏忠贤的。

而东林党里的另一位明白人黄尊素,事发后也问过这样一个问题:

``清君侧者必有内援,杨公有乎?''

这意思是,你要搞定皇帝身边的人,必须要有内应,当然没内应也行,像当年猛人朱棣,带几万人跟皇帝死磕,一
直打到京城,想杀谁杀谁。

杨涟没有,所以不行。

但他依然充满自信,因为奏疏在社会上引起的强烈反响和广大声势让他相信:真理和正义是站在他这边的。

但是实力,并不在他的一边。

奏疏送上后的第五天,事情开始脱离杨涟的轨道,走上了叶向高预言的道路。

底线

焦头烂额的魏忠贤几乎绝望了,面对如潮水涌来的攻击,他束手无策,无奈之下,他只能跑去求内阁大臣,东林党
人韩旷,希望他手下留情。

韩旷给他的答复是:没有答复。

\section[\thesection]{}

这位东林党内除叶向高外的最高级别干部,对于魏公公的请求,毫无回应,别说赞成,连拒绝都没有。

如此的态度让魏忠贤深信,如果不久之后自己被拉出去干掉,往尸体上吐唾沫的人群行列中,此人应该排在头几名。

与韩旷不同,叶向高倒还比较温柔。他曾表示,对魏忠贤无须赶尽杀绝,能让他消停下来,洗手不干,也就罢了。

这个观点后来被许多的史书引用,来说明叶向高那卑劣的投降主义和悲观主义思想,甚至还有些人把叶先生列入了
阉党的行列。

凡持此种观点者,皆为站着说话不腰疼、啃着馒头看窝头之流。

因为就当时局势而言,叶向高说无须赶尽杀绝,那只是客气客气的,实际上,压根就无法赶尽杀绝。

事情的下一步发展完美地印证了这一点。

在被无情地拒绝后,魏忠贤丢掉了所有的幻想,他终于明白,对于自己的胡作非为,东林党人是无法容忍,也无法
接纳的。

正邪不能共存,那么好吧,我将把所有的一切,都拉入黑暗之中。

魏忠贤立即找到了另一个人,一个能够改变一切的人。

在皇帝的面前,魏忠贤表现得相当悲痛,一进去就哭,一边哭一边说:

``现在外面有人要害我,而且还要害皇上,我无法承担重任,请皇上免去我的职务吧。''

这种混淆是非,拉皇帝下水的伎俩,虽然并不高明,却比较实用,是魏公公的必备招数。

面对着痛哭流涕的魏忠贤,天启皇帝只说了一句话,就打乱了魏公公的所有部署:

``听说有人弹劾你,是怎么回事?''

听到这句话时,魏忠贤知道,完蛋了。他压住杨涟的奏疏,煞费苦心封锁消息,这木匠还是知道了。

对于朱木匠,魏忠贤还是比较了解的,虽不管事,绝不白痴,事到如今不说真话是不行了。

于是他承认了奏疏的存在,并顺道沉重地控诉了对方的污蔑。

但皇帝陛下似乎不太关心魏公公的痛苦,只说了一句话:``奏疏在哪里,拿来给我!''

这句话再次把魏公公推入了深渊。因为在那封奏疏上,杨涟列举了很多内容,比如迫害后宫嫔妃,甚至害死怀有身
孕的妃子,以及私自操练兵马(内操),图谋不轨等等。

\section[\thesection]{}

贪污受贿,皇帝可以不管,坑皇帝的老婆,抢皇帝的座位,皇帝就生气了。

更何况这些事,他确实也干过,只要皇帝知道,一查就一个准。

奏疏拿来了,就在魏忠贤的意志即将崩溃的时候,他听到了皇帝陛下的指示:``读给我听。''

魏忠贤笑了。

因为他刚刚想起一件很重要的事----皇帝陛下,是不大识字的。

如果说皇帝陛下的文化程度和魏公公差不多,似乎很残酷,但却是事实,天启之所以成长为准文盲(认字不多),归
根结底,还是万历惹的祸。

万历几十年不立太子,太子几十年不安心,自己都搞不定,哪顾得上儿子,儿子都顾不上,哪顾得上儿子读书,就
这么折腾来折腾去,把天启折腾成了木匠。

所以现在,他并没有自己看,而是找了个人,读给他听。

魏忠贤看到了那个读奏疏的人,他确定,东林党必将死无葬身之地。这个朗读者,是司礼监掌印太监,他的死党,
王体乾。

就这样,杨涟的二十四条大罪,在王太监的口里缩了水,为不让皇帝大人担心,有关他老婆和他个人安危的,都省
略了,而魏公公一些过于恶心人的行为,出于善意,也不读了。

所以一篇文章读下来,皇帝大人相当疑惑,听起来魏公公为人还不错,为何群众如此愤怒?

但这也无所谓,反正也没什么大事,老子还要干木匠呢,就这么着吧。

于是他对魏忠贤说,你接着干吧,没啥大事。

魏忠贤彻底解脱了。

正如叶向高所说的那样,正义和道德是打不倒魏忠贤的,能让这位无赖屈服的,只有实力。而唯一拥有这种实力的
人,只有皇帝。

现在皇帝表明了态度,事件的结局,已无悬念。

天启四年(1624)十月,看清虚实的魏忠贤,终于举起了屠刀。

同月,在毫无预兆的情况下,皇帝下旨,训斥吏部尚书赵南星结党营私,此后皇帝又先后下文,批评杨涟、左光斗、
高攀龙等人,最后索性给他们搞了个总结,一顿猛踩,矛头直指东林党。

可以肯定的是,皇帝大人对此是不大清楚的,他老人家本不识字,且忙做木匠,考虑到情况比较特殊,为保证及时
有力迫害忠良,魏公公越级包办了所有圣旨。

大势已去,一切已然无可挽回。

\section[\thesection]{}

同月,心灰意冷的赵南星、杨涟、左光斗纷纷提出辞职,回了老家。东林党就此土崩瓦解。

只剩下一个人----叶向高。

叶向高很冷静,由始至终,他都极其低调,魏忠贤倒霉时,他不去踩,魏忠贤得意时,他不辞职,因为他知道,自
己将是东林党最后的希望

必须忍耐下去,等待反攻的时机。

但是,他错误地估计了一点----魏忠贤的身份。

魏忠贤是一个无赖,无赖没有原则,他不是刘瑾,不会留着李东阳给自己刨坟。

几天之后,叶向高的住宅迎来了一群不速之太监,每天在叶向高门口大吵大嚷,不让睡觉,无奈之下,叶向高只得
辞职回家。

两天后,内阁大学士韩旷辞职,魏忠贤的非亲生儿子顾秉谦接任首辅,至此,内阁彻底沦陷。

东林党失败了,败得心灰意冷,按照以往的惯例,被赶出朝廷的人,唯一的选择是在家养老。

但这一次,魏公公给他们提供了第二个选择----赶尽杀绝。

因为魏公公不是政治家,他是无赖流氓,政治家搞人,搞倒搞臭也就罢了,无赖流氓搞人,都是搞死为止。杀死那
些毫无抵抗能力的人,这就是魏忠贤的品格。

但要办到这一点,是有难度的。

大明毕竟是法制社会,要干掉某些人,必须要罪名,至少要个借口,但魏公公查遍了杨涟等人的记录,作风问题、
经济问题,都是统统的没有。

东林党用实际行动证明了这样一点:他们或许狭隘、或许偏激,却不贪污,不受贿,不仗势欺民,他们的所有举
动,都是为了百姓的生计,为了这个国家的未来。

什么生计、未来,魏公公是不关心的,他关心的是,如何合理地把东林党人整死:抓来打死不行,东林党人都有知
名度,社会压力太大,抓来死打套取口供,估计也不行,这帮人是出了名的硬骨头,攻坚难度太大。

于是,另一个人进入了魏忠贤的视线,他相信,从此人的身上,他将顺利地打开突破口。

虽然在牢里,但汪文言仍然清楚地感觉到,世界变了,刘侨走了,魏忠贤的忠实龟孙,五彪之一的许显纯接替了他
的位置,原先好吃好喝,现在没吃没喝,审讯次数越来越多,态度越来越差。

但他并不知道,地狱之门才刚刚打开。

\section[\thesection]{}

魏忠贤明白,东林党的人品是清白的,把柄是没有的,但这位汪文言是个例外,这人自打进朝廷以来,有钱就拿,
有利就贪,东林党熟,阉党也熟,牛鬼蛇神全不耽误,谈不上什么原则。只要从他身上获取杨涟等人贪污的口供,
就能彻底摧毁东林党。

面对左右逢源、投机取巧的汪文言,这似乎不是什么难事。

天启五年(1625),许显纯接受魏忠贤的指示,审讯汪文言。

史料反映,许显纯很可能是个心理比较变态的人,他不但喜欢割取犯人的喉骨,还想出了许多花样繁多的酷刑,比
如用铁钩扎穿琵琶骨,把人吊起来,或是用沾着盐水的铁刷去刷犯人,皮肤会随着惨叫声一同脱落。所谓审讯,就
是赤裸裸的折磨。

第一次审讯后,汪文言已经是遍体鳞伤,半死不活。

但许显纯并不甘休,之后他又进行了第二次、第三次审讯,十几次审下来,审到他都体力不支,依然乐此不疲。

因为无论他怎么殴打、侮辱、拷问汪文言,逼他交代东林党的罪行,这个不起眼的小人物始终重复一句话: ``不知
道。''

无论拷打多少次,折磨多少回,穷凶极恶的质问,丧心病狂的酷刑,这就是他唯一的回答。

当汪文言的侄子买通了看守,在牢中看到不成人形的汪文言时,禁不住痛哭流涕。

然而汪文言用镇定地语气对他说:

``不要哭,我必死,却并不怕死!''

许显纯急眼了,在众多的龟孙之中,魏公公把如此重要的任务交给他,实在是莫大的信任,为不让太监爷爷失望,
他必须继续拷打。

终于有一天,在拷打中,奄奄一息的汪文言用微弱的声音对许显纯说:``你要我承认什么,说吧,我承认就是了。''

许显纯欣喜万分,说道:``只要你说杨涟收取贿赂,作口供为证,就放了你。''

在短暂的沉默之后,一个微弱却坚定的声音响起:

``这世上,没有贪赃的杨涟。''

六年前,他之所以加入东林党,不是为了正义,是为了混饭吃。

混社会的游民,油滑的县吏,唯利是图,狡猾透顶的官僚汪文言,为了在这丑恶的世界上生存下去,他的一生,都
在虚伪、圆滑、欺骗中度过,他的每次选择,都是为了利益,都是妥协的产物。

但在这人生的最后时刻,他做出了最后的抉择:面对黑暗,绝不妥协。

付出生命,亦在所不惜。

\section[\thesection]{}

许显纯无计可施,所以他决定,用一种更不要脸的方式解决问题----伪造口供。

在这个问题上,许显纯再次显示了他的变态心理,他一边拷打汪文言,一边在他的眼前伪造证词,意思很明白:我
就在你的面前,伪造你的口供,你又能怎么样呢?

但当他洋洋得意地伪造供词的时候,对面阴暗的角落里,那个遍体鳞伤,奄奄一息的人发出了声音。

无畏的东林党人汪文言,用尽他最后的力气,向这个黑暗的世界,迸发出愤怒的控诉:

``不要乱写,就算我死了,也要与你对质!

这是他留在世间的最后一句话。

这句话告诉我们,追逐权位,利益至上的老油条汪文言,经历几十年官场沉浮、尔虞我诈之后,拒绝了诱惑,选择
了理想,并最终成为了一个正直无私的人。

血书

许显纯怕了,他怕汪文言的诅咒,于是,他找到了一个解决方法:杀死汪文言。

死后对质还在其次,如果让他活着对质,下一步计划将无法进行。

天启五年(1625)四月,汪文言被害于狱中,他始终没有屈服。

同月,魏忠贤的第二步计划开始,杨涟、左光斗、魏大中等东林党人被逮捕,他们的罪名是受贿,而行贿者是已经
处决的熊廷弼。

受贿的证据自然是汪文言的那份所谓口供,在这份无耻的文书中,杨涟被认定受贿两万两,左光斗等人也人人有份。

审讯开始了,作为最主要的对象,杨涟被首先提审。

许显纯拿出了那份伪造的证词,问:

``熊廷弼是如何行贿的?''

杨涟答:

``辽阳失陷前,我就曾上书弹劾此人,他战败后,我怎会帮他出狱?文书尚在可以对质。''

许显纯无语。

很明显,许锦衣卫背地耍阴招有水平,当面胡扯还差点,既然无法在沉默中发言,只能在沉默中变态:

``用刑!''

下面是杨涟的反应:

``用什么刑?有死而已!''

许显纯想让他死,但他必须找到死的理由。

拷打如期进行,拷打规律是每五天一次,打到不能打为止,杨涟的下颌脱落,牙齿打掉,却依旧无一字供词。

于是许显纯用上了钢刷,几次下来,杨涟体无完肤,史料有云:``皮肉碎裂如丝''。

然``骂不绝口'',死不低头。

\section[\thesection]{}

在一次严酷的拷打后,杨涟回到监房,写下了《告岳武穆疏》。

在这封文书中,杨涟没有无助的报怨,也没有愤怒的咒骂,他说:

``此行定知不测,自受已是甘心。''

他说:

``涟一身一家其何足道,而国家大体大势所伤实多。''

昏暗的牢房中,惨无人道的迫害,无法形容的痛苦,死亡边缘的挣扎,却没有仇恨,没有愤懑。

只有坦然,从容,以天下为己任。

在无数次的尝试失败后,许显纯终于认识到,要让这个人低头认罪,是绝不可能的。

栽赃不管用的时候,暗杀就上场了。

魏忠贤很清楚,杨涟是极为可怕的对手,是绝对不能放走的。无论如何,必须将他杀死,且不可走漏风声。

许显纯接到了指令,他信心十足地表示,杨涟将死在他的监狱里,悄无声息,他的冤屈和酷刑将永无人知晓。

事实确实如此,朝廷内外只知道杨涟有经济问题,被弄进去了,所谓拷打、折磨,闻所未闻。

对于这一点,杨涟自己也很清楚,他可以死,但不想死得不明不白。

所以,在暗无天日的监房中,杨涟用被打得几近残废的手,颤抖地写下了两千字的绝笔遗书。在遗书中,他写下了
事情的真相,以及自己坎坷的一生。

遗书写完了,却没用,因为送不出去。

为保证杨涟死得不清不楚,许显纯加派人手,经常检查杨涟的牢房,如无意外,这封绝笔最终会落入许显纯手中,
成为灶台的燃料。

于是,杨涟将这封绝笔交给了同批入狱的东林党人顾大章。

顾大章接受了,但他也没办法,因为他是东林重犯,如果杨涟被杀,他必难逃一死。且此封绝笔太过重要,如若窝
藏必是重犯,推来推去,谁都不敢收。

更麻烦的是,看守查狱的时候,发现了这封绝笔,顾大章已别无选择。

他面对监狱的看守,坦然告诉他所有的一切,然后从容等待结局。

短暂的沉寂后,他看见那位看守面无表情地收起绝笔,平静地告诉他:这封绝笔,绝不会落到魏忠贤的手中。

这封绝笔开始被藏在牢中关帝像的后面,此后被埋在牢房的的墙角下,杨涟被杀后,那位看守将其取出,并最终公
告于天下。

无论何时何地,正义终究是存在的。

明朝历代皇帝

明朝:1368--l644,共277年,历十六帝,朱姓,建都:南京,成祖移至北京。

太祖,朱元璋,1368年-1398年,洪武

惠帝,朱允炆,1399年-1402年,建文

成祖,朱 棣,1403年-1424年,永乐

仁宗,朱高炽,1425年-1425年,洪熙

宣宗,朱瞻基,1426年-1435年,宣德

英宗,朱祁镇,1436年-1449年,正统

代宗,朱祁钰,1450年-1457年,景泰

英宗,朱祁镇,1457年-1464年,天顺

宪宗,朱见深,1465年-1487年,成化

孝宗,朱佑樘,1488年-1505年,弘治

武宗,朱厚照,1506年-1521年,正德

世宗,朱厚熜,1522年-1566年,嘉靖

穆宗,朱载垕,1567年-1572年,隆庆

神宗,朱翊钧,1573年-1620年,万历

光宗,朱常洛,1620年-1620年,泰昌

熹宗,朱由校,1621年-1627年,天启

思宗,朱由检,1628年-1644年,崇祯

明末

安宗简皇帝,朱由崧,年号,弘光(1644--1645)。神宗显皇帝(万历)之孙,福王朱常洵之子,崇祯皇帝之堂弟。

潞王,朱常淓,监国(1645.6--1645.7)。神宗显皇帝(万历)之弟。

绍宗襄皇帝,朱聿键,年号,隆武(1645--1646)。唐王朱桱八世孙,崇祯皇帝族叔祖。

鲁王,朱以海,监国(1645--1653)。鲁王朱檀九世孙,崇祯皇帝族叔。

唐王,朱聿粤,年号,绍武(1646)。朱聿键之弟。

永历皇帝,朱由榔,年号,永历(1646--1661)。桂王之子,崇祯皇帝堂兄。



16位 不包括南明 一共16位!! 也有说十七的!

明太祖朱元璋

明太祖朱元璋,字国瑞,明朝的开国皇帝。

明惠帝朱允炆

明惠帝朱允炆,明太祖朱元璋的长孙,太子朱标的长子。

明成祖朱棣

成祖朱棣是历史上争议颇大的一位帝王,他立有不世之功,创造了明初盛世,但他好大喜功,多疑好杀,手上沾满
了鲜血。是功大于过,还是过大于功,只有众人自己把握了。

明仁宗朱高炽

明仁宗朱高炽,成祖朱棣的长子,生于洪武十一年,生母徐皇后。早在洪武时期,朱高炽就被立为燕王世子,而且
由于它的儒雅与仁爱深得皇祖的喜爱。美中不足的是朱高炽身体肥胖,行动不便,总要两个内侍搀扶才能行动,而
且也总是跌跌撞撞,因此对于一生嗜武的成祖来讲,他并不喜欢这个儿子

宣宗朱瞻基

宣宗朱瞻基,洪熙皇帝朱高炽的长子,出生于洪武三十一年,在朱瞻基出生的那天晚上,他的皇祖当时还是燕王的
朱棣曾经作了一个梦,他梦见太祖皇帝将一个大圭赐给了他,在古代,大圭象征着权力,朱元璋将大圭赐给他,正
说明要将江山送给他

明英宗朱祁镇

明英宗朱祁镇,宣宗皇帝的长子,他的一生充满了传奇色彩,宣德年间,宣宗朱瞻基的正宫胡皇后举止得体,贤良
温淑,是一位不可多得好皇后,宣宗还有一位贵妃,姓孙,这位孙贵妃深的宣宗皇帝的喜爱,唯一遗憾是没有得到
皇后的宝座,于是孙贵妃绞尽脑汁总想挤掉胡皇后而自立。机会终于来了,宣宗皇帝的子嗣一直不旺胡皇后没能为
宣宗生下一个皇子,孙贵妃虽然也没能生子,但他想出了一条偷梁换栋的计策,他派人在宫中四处打探看哪位宫女
被皇帝临幸后怀有了身孕,于是将找到的宫女藏在秘室之中,与外界隔绝,派专人送饭、照看。然后买通御医,对
外号称怀孕,并伪装了许多怀孕的迹象。

明代宗朱祁钰

明代宗朱祁钰,宣宗皇帝的次子。谈起景泰帝,大家不免想起任用于谦北京保卫战,夺门之变英宗复辟等一些传奇
经历

明宪宗朱见深

明宪宗朱见深,原名朱见浚,英宗土木堡之变后,两岁的朱见浚就被孙太后立为太子,谁知命运与他开了个玩笑,
他的叔叔景泰帝即位后,随着政局的逐渐稳定,就开始考虑如何废掉他,而用自己的儿子取而代之。

明孝宗朱佑樘

明孝宗朱佑樘,宪宗皇帝之子,生母纪氏。

孝宗皇帝有着多难的童年,她的母亲是在万贵妃的阴影下偷偷生下了他,而且直到六岁

才敢站出来与宪宗相会。当时的宪宗皇帝正苦于没有子嗣,突然一个皇子从天而降非常

高兴,并马上册立为太子,正位东宫。

明武宗朱厚照

明武宗,朱厚照,孝宗长子,生母张皇后。两岁被立为皇太子。由于孝宗一生只宠爱张皇后,而张皇后只为孝宗生
了两个儿子,次子又早夭,因此武宗自小就被视为掌上明珠,而且少年的武宗非常聪明,老师教他的东西总是能很
快学会,按理说他应该成为一个很好的皇帝,但就是因为周围的太监,毁了这个年轻的孩子

明世宗朱厚骢

明世宗朱厚骢,年号嘉靖,明代第11位皇帝。嘉靖皇帝并不是武宗的儿子,也不是孝宗的孙子。由于武宗荒淫,没
有留下子嗣,孝宗则是单传,只有武宗一个孩子,因此,孝宗一脉到了武宗驾崩也就断了香火,皇位继承人就要从
最近支的皇族中选出。孝宗的弟弟,成化皇帝的次子兴王朱佑沅被确定为最近支的皇室,当时兴王已经去世,因此
就又他唯一的儿子朱厚骢来继承皇位,他就是后来的嘉靖皇帝。

明穆宗朱载后

明穆宗朱载后,年号,隆庆,世宗皇帝的三子,三十岁登基,是为明朝的第十二位皇帝

明神宗朱翊钧

明神宗朱翊钧,穆宗皇帝长子,6岁立为太子,10岁即皇帝位,年号万历。

明光宗朱常洛

明光宗朱常洛,年号泰昌,明代第14位皇帝,在位一个月

明熹宗朱由校

明熹宗朱由校,明代第十五位皇帝,在位七年,年号天启

明思宗朱由检

明思宗,朱由检,年号:崇祯,明代第十六位皇帝,光宗朱常洛之子,熹宗朱由校之弟

\section[\thesection]{}

天启五年(1625)七月,许显纯开始了谋杀。

不能留下证据,所以不能刀砍,不能剑刺,不能有明显的皮外伤。

于是许显纯用铜锤砸杨涟的胸膛,几乎砸断了他的所有肋骨。

然而杨涟没有死。

他随即用上了监狱里最著名的杀人技巧----布袋压身。

所谓布袋压身,是监狱里杀人的不二法门,专门用来处理那些不好杀,却又不能不杀的犯人。具体操作程序是:找
到一只布袋,里面装满土,晚上趁犯人睡觉时压在他身上。按照清代桐城派著名学者方苞的说法(当年曾经蹲过黑
牢),基本上是晚上压住,天亮就死,品质有保障。

然而杨涟还是没死,每晚在他身上压布袋,就当是盖被子,白天拍土又站起来。

口供问不出来倒也罢了,居然连人都干不掉,许显纯快疯了。

于是这个疯狂的人,使用了丧心病狂的手段。

他派人把铁钉钉入了杨涟的耳朵。

具体的操作方法,我不知道。我只知道,这不是人能干出来的事情。

铁钉入耳的杨涟依然没有死,但例外不会再发生了,毫无人性的折磨、耳内的铁钉已经重创了杨涟,他的神智开始
模糊。

杨涟知道,自己活不了多久了,于是他咬破手指,对这个世界,写下了最后的血书。

此时的杨涟已处于濒死状态,他没有力气将血书交给顾大章,在那个寂静无声的黑夜里,凭借着顽强的意志,他拖
着伤残的身体,用颤抖的双手,将血书藏在了枕头里。结束吧,杨涟微笑着,等待着最后的结局。

许显纯来了,用人间的言语来形容他的卑劣与无耻,已经力不从心了。

看着眼前这个有着顽强信念,和坚韧生命力的人,许显纯真的害怕了,敲碎他全身的肋骨,他没有死,用土袋压,
他没有死,用钉子钉进耳朵,也没有死。

无比恐惧的许显纯决定,使用最后,也是最残忍的一招。

天启五年(1625)七月二十四日 夜。

许显纯把一根大铁钉,钉入了杨涟的头顶。这一次,奇迹没有再次出现,杨涟当场死亡,年五十四。

伟大的殉道者,就此走完了他光辉的一生!

杨涟希望,他的血书能够在他死后清理遗物时,被亲属发现。

然而这注定是个破灭的梦想,因为这一点,魏忠贤也想到了。

\section[\thesection]{}

为消灭证据,他下令对杨涟的所有遗物进行仔细检查,绝不能遗漏。

很明显,杨涟藏得不好,在检查中,一位看守轻易地发现了这封血书。

他十分高兴,打算把血书拿去请赏。

但当他看完这封血迹斑斑的遗言后,便改变了主意。

他藏起了血书,把它带回了家,他的妻子知道后,非常恐慌,让他交出去。

牢头并不理会,只是紧握着那份血书,一边痛哭,一边重复着这样一句话:

``我要留着它,将来,它会赎清我的罪过。''

三年后,当真相大白时,他拿出了这份血书,并昭示天下,如下:

仁义一生,死于诏狱,难言不得死所,何憾于天,何怨于人?唯我身副宪臣,曾受顾命,孔子云:托孤寄命,临大节
而不可夺。持此一念终可见先帝于在天,对二祖十宗于皇天后土,天下万世矣!

大笑大笑还大笑,刀砍东风,于我何有哉!

他不知道自己还能活多久,不知道死后何人知晓,不知道能否平反,也不知道这份血书能否被人看见。

毫无指望,只有彻底的孤独和无助。

这就是阴森恐怖的牢房里,肋骨尽碎的杨涟,在最为绝望的时刻,写下的文字,每一个字,都闪烁着希望和光芒。

拷打、折磨,毫无人性的酷刑,制服了他的身体,却没有征服他的意志。无论何时,他都坚持着自己的信念,那个
他写在绝笔中的信念,那个崇高、光辉、唯一的信念:

涟即身无完骨,尸供蛆蚁,原所甘心

但愿国家强固,圣德刚明,海内长享太平之福。

此痴愚念头,至死不改。

有人曾质问我,遍读史书如你,所见皆为帝王将相之家谱,有何意义?

千年之下,可有一人,不求家财万贯,不求出将入相,不求青史留名,唯以天下、以国家、以百姓为任,甘受屈
辱,甘受折磨,视死如归?

我答:曾有一人,不求钱财,不求富贵,不求青史留名,有慨然雄浑之气,万刃加身不改之志。

杨涟,千年之下,终究不朽!

老师

左光斗只比杨涟多活了一天。

身为都察院高级长官,左光斗也是许显纯拷打的重点对象,杨涟挨过的酷刑,左光斗一样都没少。

而他的态度,也和杨涟一样,绝不退让,绝不屈服。

虽然被打得随时可能断气,左光斗却毫不在乎,死不低头。

他不在乎,有人在乎。

\section[\thesection]{}

先是左光斗家里的老乡们开始凑钱,打算把人弄出来,至少保住条命。无效不退款后,他的家属和学生就准备进去
探监,至少再见个面。

但这个要求也被拒绝了。

最后,他的一位学生费尽浑身解数,才买通了一位看守,进入了监牢。

他换上了破衣烂衫,化装成捡垃圾的,在黑不隆冬的诏狱里摸了半天,才摸到了左光斗的牢房。

左光斗是坐着的,因为他的腿已经被打没了(筋骨尽脱)。面对自己学生的到访,他没有表现出任何惊讶,因为他根
本不知道----脸已被烙铁烙坏,连眼睛都睁不开。

他的学生被惊呆了,于是他跪了下来,抱住老师,失声痛哭。

左光斗听到了哭声,他醒了过来,没有惊喜,没有哀叹,只有愤怒,出离的愤怒:

``蠢人!这是什么地方,你竟然敢来!(此何地也,而汝前来)国家已经到了这个地步,我死就死了,你却如此轻率,
万一出了事,将来国家的事情谁来管!?''

学生呆住了,呆若木鸡。

左光斗的愤怒似乎越发激烈,他摸索着地上的镣铐,做出投掷的动作,并说出了最后的话:

``你还不走?!再不走,无需奸人动手,我自己杀了你(扑杀汝)!''

面对着世界上最温暖的威胁,学生眼含着热泪,快步退了出去。

临死前,左光斗用自己的行动,给这名学生上了最后一课:一个人应该坚持信念,至死也不动摇。

天启五年(1625)七月二十五日,左光斗在牢中遇害,年五十一。

二十年后 扬州

南京兵部尚书,内阁大学士,南明政权的头号重臣史可法,站在城头眺望城外的清军,时为南明弘光元年(1645)二
月。

雪很大,史可法却一直站在外面,安排部署,他的部下几次劝他进屋躲雪,他的回复总是同一句话:``我不能对不
起我的老师,我不能对不起我的老师!(愧于吾师)''

史可法最终做到了,他的行为,足以让他的老师为之自豪。

左光斗死后,同批入狱的东林党人魏大中、袁化中,周朝瑞先后被害。

活着的人,只剩下顾大章。

\section[\thesection]{}

顾大章,时任礼部郎中,算是正厅级干部,在这六人里就官职而言并不算大,但他还是有来头的,他的老师就是叶
向高,加上平时活动比较积极,所以这次也被当作要犯抓了进来。

抓进来六个,其他五个都死了,他还活着,不是他地位高,只是因为他曾经担任过一个特殊的官职----刑部主事。

刑部主事,大致相当于司法部的一个处长,但凑巧的是,他这个部门恰好就是管监狱的,所谓刑部天牢、锦衣诏狱
的看守,原先都是他的部下。

现在老上级进去了,遇到了老下级,这就好比是路上遇到劫道的,一看,原来你是我小学时候的同学,还一起罚过
站,这就不好下手了。咬咬牙,哥们你过去吧,这单生意我不做了,下次注意点,别再到我的营业区域里转悠。

外加顾主事平时为人厚道,对牢头看守们都很照顾,所以他刚进去的时候,看守都向他行礼,对他非常客气,点头
哈腰,除了人渣许显纯例行拷打外,基本没吃什么亏。

但其他人被杀后,他的处境就危险了,毕竟一共六个,五个都死了,留你一个似乎不太像话。更重要的是,这些惨
无人道的严刑拷打,是不能让人知道的,要是让他出狱,笔杆子一挥全国人民都知道了,舆论压力比较大。

事实上,许显纯和魏忠贤确实打算把顾大章干掉,且越快越好。顾大章去阎王那里伸冤的日子已经不远了。

然而这个世界上,意外的事情总是经常发生的。

一般说来,管牢房的人交际都比较广泛。特别是天牢、诏狱这种高档次监狱,进来的除了窦娥、忠良外,大都有点
水平,或是特殊技能,江洋大盗之类的牛人也不少见。

我们有理由相信,顾大章认识一些这样的人。

因为就在九月初,处死他的决议刚刚通过,监狱看守就知道了。但是这位看守没有把消息告诉顾大章,却通知了另
一个人。

这个人的姓名不详,人称燕大侠,也在诏狱里混,但既不是犯人,也不是看守,每天就混在里面,据说还是主动混
进来的,几个月了都没人管。

他怎么进来的,不得而知,为什么没人管,不太清楚,但他之所以进来,只是为了救顾大章。为什么要救顾大章,
也不太清楚,反正他是进来了。

得知处决消息,他并不慌张,只是找到报信的看守,问了他一个问题:``我给你钱,能缓几天吗?''

\section[\thesection]{}

看守问:``几天?''

燕大侠答:``五天。''

看守答:``可以。''

五天之后,看守跑来找燕大侠:``我已尽力,五日已满,今晚无法再保证顾大章的安全,怎么办?''

燕大侠并不紧张:``今晚定有转机。''

看守认为,燕大侠在做梦,他笑着走了。

几个时辰之后,他接到了命令,将顾大章押往刑部。

还没等他缓过神来,许显纯又来了。

许显纯急匆匆跑来,把顾大章从牢里提出来,声色俱厉地说了句话:``你几天以后,还是要回来的!''

然后,他又急匆匆地走了。

顾大章很高兴。

作为官场老手,他很理解许显纯这句话的隐含意义----自己即将脱离诏狱,而许显纯无能为力。

因为所谓锦衣卫、东厂,都是特务机关,并非司法机构。这件案子被转交刑部,公开审判,就意味着许显纯们搞不
定了。

很明显,他们受到了压力。

但为什么搞不定,又是什么压力,他不知道。

这是个相当诡异的问题:魏公公权倾天下,连最能搞关系的汪文言都整死了,然而燕大侠横空出世,又把事情解决
了,实在让人难以理解。

顾大章不知道答案,看守不知道答案,许显纯也未必知道。

燕大侠知道,可是他没告诉我,所以我也不知道。

之前我曾介绍过许多此类幕后密谋,对于这种鬼才知道的玩意,我的态度是,不知道就说不知道,绝不猜。

我倒是想猜,因为这种暗箱操作,还是能猜的。如当年太史公司马迁先生,就很能猜的,秦始皇死后,李斯和赵高
密谋干掉太子,他老人家并不在场,上百年前的事,天知地知你知我知,对话都能猜出来。过了几千年,也没人说
他猜得不对,毕竟事情后来就是那么干的。

可这件事实在太过复杂,许显纯没招,魏公公不管(或是管不了),他们商量的时候也没叫我去,实在是不敢乱猜。

无论事实真相如何,反正顾大章是出来了。在经历几十天痛苦的折磨后,他终于走出了地狱。

按说到了刑部,就是顾大人的天下了,可实情并非如此。因为刑部尚书李养正也投了阉党,部长大人尚且如此,顾
大人就没辙了。

\section[\thesection]{}

天启五年(1625)九月十二日,刑部会审。

李养正果然不负其阉党之名,一上来就喝斥顾大章,让他老实交代。更为搞笑的是,他手里拿的罪状,就是许显纯
交给他的,一字都没改,底下的顾大章都能背出来,李尚书读错了,顾大人时不时还提他两句。

审讯的过程也很简单,李尚书要顾大章承认,顾大章不承认,并说出了不承认的理由:``我不能代死去的人,承受
你们的诬陷。''

李尚书沉默了,他知道这位曾经的下属是冤枉的,但他依然做出了判决:杨涟、左光斗、顾大章等六人,因收受贿
赂,结交疆臣,处以斩刑。

这是一份相当无聊的判决,因为判决书里的六个人,有五个已经挂了,实际上是把顾大章先生拉出来单练,先在诏
狱里一顿猛打,打完再到刑部,说明打你的合法理由。

形势急转直下,燕大侠也慌了手脚,一天夜里,他找到顾大章,告诉他情况不妙。

然而出乎意料的是,顾大章并不惊慌,恰恰相反,他用平静的口吻,向燕大侠揭示了一个秘密----出狱的秘密。

第二天,在刑部大堂上,顾大章公开了这个秘密。

顾大章招供了,他供述的内容,包括如下几点,杨涟的死因,左光斗的死因,许显纯的刑罚操作方法,绝笔、无人
性的折磨,无耻的谋杀。

刑部知道了,朝廷知道了,全天下人都知道了。

魏忠贤不明白,许显纯不明白,甚至燕大侠也不明白,顾大章之所以忍辱负重,活到今天,不是心存侥幸,不是投
机取巧。

他早就想死了,和其他五位舍生取义的同志一起,光荣地死去,但他不能死。

当杨涟把绝笔交给他的那一刻,他的生命就不再属于他自己,他知道自己有义务活下去,有义务把这里发生的一
切,把邪恶的丑陋,正义的光辉,告诉世上所有的人。

所以他隐忍、等待,直至出狱,不为偷生,只为永存。正如那天夜里,他对燕大侠所说的话:

``我要把凶手的姓名传播于天下(播之天下),等到来日世道清明,他们一个都跑不掉(断无遗种)!''

``吾目暝矣。''

这才是他最终的目的。

他做到了,是以今日之我们,可得知当年之一切。

一天之后,他用残废的手(三个指头已被打掉)写下了自己的遗书,并于当晚自缢而死。

杨涟,当日你交付于我之重任,我已完成。

``吾目暝矣。''

\section[\thesection]{}

至此,杨涟、左光斗、魏大中、袁化中、周朝瑞、顾大章六人全部遇害,史称``六君子之狱''。

就算是最恶俗的电视剧,演到这里,坏人也该休息了。

但魏忠贤实在是个超一流的反派,他还列出了另一张杀人名单。

在这份名单上,有七个人的名字,分别是高攀龙、李应升、黄遵素、周宗建,缪昌期、周起元、周顺昌。

这七位仁兄地位说高不高,就是平时骂魏公公时狠了点,但魏公公一口咬死,要把他们组团送到阎王那里去。

六君子都搞定了,搞个七君子不成问题。

春风得意、无往不胜的魏公公认为,他已经天下无敌了,可以把事情做绝做尽。

魏忠贤错了。

在一部相当胡扯的香港电影中,某大师曾反复说过句不太胡扯的话:凡事太尽,缘分必定早尽。

刚开始的时候,事情是很顺利的,东林党的人势力没有,气节还是有的,不走也不逃,坐在家里等人来抓,李应升、
周宗建,缪昌期、周起元等四人相继被捕,上路的时候还特高兴。

因为在他们看来,坚持信念,被魏忠贤抓走,是光辉的荣誉。

高攀龙更厉害,抓他的东厂特务还没来,他就上路了----自尽。

在被捕前的那个夜晚,他整理衣冠,向北叩首,然后投水自杀。死前留有遗书一封,有言如下:可死,不可辱。

在这七个人中,高攀龙是都察院左都御史,李应升、周宗建、黄尊素都是御史,缪昌期是翰林院谕德,周起元是应
天巡抚,说起来,不太起眼的,就数周顺昌了。

这位周先生曾吏部员外郎,论资历、权势,都是小字辈,但事态变化,正是由他而起。

周顺昌,字景文,万历四十一年进士,嫉恶如仇。

说起周兄,还有个哭笑不得的故事,当初他在外地当官,有一次人家请他看戏,开始挺高兴,结果看到一半,突然
怒发冲冠,众目睽睽之下跳上舞台,抓住演员一顿暴打,打完就走。

这位演员之所以被打,只是因为那天,他演的是秦桧。听说当年演白毛女的时候,通常是演着演着,下面突来一
枪,把黄世仁同志干掉,看来是有历史传统的。

连几百年前的秦桧都不放过,现成的魏忠贤当然没问题。

\section[\thesection]{}

其实最初名单上只有六个人,压根就没有周顺昌,他之所以成为候补,是因为当初魏大中过境时,他把魏先生请到
家里,好吃好喝,还结了亲家,东厂特务想赶他走,结果他说: ``你不知道世上有不怕死的人吗?!回去告诉魏忠
贤,我叫周顺昌,只管找我!''

后来东厂抓周起元的时候,他又站出来大骂魏忠贤,于是魏公公不高兴了,就派人去抓他。

周顺昌是南直隶吴县人,也就是今天的江苏苏州,周顺昌为人清廉,家里很穷,还很讲义气,经常给人帮忙,在当
地名声很好。

东厂特务估计不太了解这个情况,又觉得苏州人文绉绉的,好欺负,所以一到地方就搞潜规则,要周顺昌家给钱,
还公开扬言,如果不给,就在半道把周顺昌给黑了。

可惜周顺昌是真没钱,他本人也看得开,同样扬言:一文钱不给,能咋样?

但是人民群众不干了,他们开始凑钱,有些贫困家庭把衣服都当了,只求东厂高抬贵手。

这次带队抓人的东厂特务,名叫文之炳,可谓是王八蛋中的王八蛋,得寸进尺,竟然加价,要了还要。

这就过于扯淡了,但为了周顺昌的安全,大家忍了。

第二天,为抗议逮捕周顺昌,苏州举行罢市活动。

要换个明白人,看到这个苗头,就该跑路,可这帮特务实在太过嚣张(或是太傻),一点不消停,还招摇过市欺负老
百姓,为不连累周顺昌,大家又忍了。

一天后,苏州市民涌上街头,为周顺昌送行,整整十几万人,差点把县衙挤垮,巡抚毛一鹭吓得不行,表示有话好
好说。有人随即劝他,众怒难犯,不要抓周顺昌,上奏疏说句公道话。

毛一鹭胆子比较小,得罪群众是不敢的,得罪魏忠贤自然也不敢,想来想去,一声都不敢出。

所谓干柴烈火,大致就是这个样子,十几万人气势汹汹,就等一把火。

于是文之炳先生挺身而出了,他大喊一声:

``东厂逮人,鼠辈敢尔?''

火点燃了。

勒索、收钱不办事、欺负老百姓,十几万人站在眼前,还敢威胁人民群众,人蠢到这个份上,就无须再忍了。

短暂的平静后,一个人走到了人群的前列,面对文之炳,问出了一个问题: ``东厂逮人,是魏忠贤(魏监)的命令
吗?''

\section[\thesection]{}

问话的人,是一个当时寂寂无名,后来名垂青史的人,他叫颜佩韦。

颜佩韦是一个平民,一个无权无势的平民,所以当文特务确定他的身份后,顿时勃然大怒:

``割了你的舌头!东厂的命令又怎么样?''

他穿着官服,手持武器,他认为,手无寸铁的老百姓颜佩韦会害怕,会退缩。

然而,这是个错误的判断。

颜佩韦振臂而起:``我还以为是天子下令,原来是东厂的走狗!''

然后他抓住眼前这个卑劣无耻、飞扬跋扈的特务,拳打脚踢,发泄心中的怒火。

文之炳被打蒙了,但其他特务反应很快,纷纷拔刀,准备上来砍死这个胆大包天的人。

然而接下来,他们看见了让他们恐惧一生的景象,十几万个胆大包天的人,已向他们冲来。

这些此前沉默不语,任人宰割的羔羊,已经变成了恶狼,纷纷一拥而上,逮住就是一顿暴打。由于人太多,只有离
得近的能踩上几脚,距离远的就脱鞋,看准了就往里砸(提示:时人好穿木屐)。

东厂的人疯了,平时大爷当惯了,高官看到他们都打哆嗦,这帮平民竟敢反抗,由于反差太大,许多人思想没转过
弯来,半天还在发愣。

但他们不愧训练有素,在现实面前,迅速地完成了思想斗争,并认清了自己的逃跑路线,四散奔逃,有的跑进民
宅,有的跳进厕所,有位身手好的,还跳到房梁上。

说实话,我认为跳到房梁上的人,脑筋有点问题,人民群众又不是野生动物,你以为他们不会爬树?

对于这种缺心眼的人,群众们使用了更为简洁的方法,一顿猛揣,连房梁都揣动了,直接把那人摇了下来,一顿群
殴,当场毙命。

相对而言,另一位东厂特务就惨得多了,他是被人踹倒的,还没反应过来,又是一顿猛踩,被踩死了,连肇事者都
找不着。

值得夸奖的是,苏州的市民们除了有血性外,也很讲策略。所有特务都被抓住暴打,但除个别人外,都没打死----
半死。这样既出了气,又不至于连累周顺昌。

打完了特务,群众还不满意,又跑去找巡抚毛一鹭算帐。其实毛巡抚比较冤枉,他不过是执行命令,胆子又小,吓
得魂不附体,只能躲进粪坑里,等到地方官出来说情,稳定秩序,才把浑身臭气的毛巡抚捞出来。

\section[\thesection]{}

这件事件中,东厂特务被打得晕头转向,许多人被打残,还留下了极深的心理创伤。据说有些人回京后,一辈子都
只敢躲在小黑屋里,怕光怕声,活像得了狂犬病。

气是出够了,事也闹大了。

东厂抓人,人没抓到还被打死几个,魏公公如此窝囊,实在耸人听闻,几百年来都没出过这事。

按说接下来就该是腥风血雨,可十几天过去,别说反攻倒算,连句话都没有。

因为魏公公也吓坏了。

事发后,魏忠贤得知事态严重,当时就慌了,马上把首辅顾秉谦抓来一顿痛骂,说他本不想抓人,听了你的馊主
意,才去干的,闹到这个地步,怎么办?

魏忠贤的意思很明白,他不喜欢这个黑锅,希望顾秉谦帮他背。但顾大人岂是等闲之辈,只磕头不说话,回去就养
病,索性不来了。

魏公公无计可施,想来想去,只好下令,把周顺昌押到京城,参与群众一概不问。

说是这么说,过了几天,顾秉谦看风声过了,又跳了出来,说要追究此事。

还没等他动手,就有人自首了。

自首的,是当天带头的五个人,他们主动找到巡抚毛一鹭,告诉他,事情就是自己干的,与旁人无关,不要株连无
辜。

这五个人的名字是:颜佩韦、杨念如、沈扬、周文元、马杰。

五人中,周文元是周顺昌的轿夫,其余四人并未见过周顺昌,与他也无任何关系。

几天后,周顺昌被押解到京,被许显纯严刑拷打,不屈而死。

几月后,周顺昌的灵柩送回苏州安葬,群情激奋,为平息事端,毛一鹭决定处决五人。

处斩之日,五人神态自若。

沈扬说:无憾!

马杰大笑:``吾等为魏奸阉党所害,未必不千载留名,去,去!''

颜佩韦大笑: ``列位请便,学生去了!''

遂英勇就义。

五人死后,明代著名文人张傅感其忠义,挥笔写就一文,是为《五人墓碑记》,四百年余后,被编入中华人民共和
国中学语文课本。

嗟夫!大阉之乱,以缙绅之身而不改其志者,四海之大,有几人欤?

而五人生于编伍之间,素不闻诗书之训,激昂大义,蹈死不顾。

----《五人墓碑记》

颜佩韦和马杰是商人,沈扬是贸易行中间人,周文元是轿夫,杨念如是卖布的

不要以为渺小的,就没有力量;不要以为卑微的,就没有尊严。

弱者和强者之间唯一的差别,只在信念是否坚定。

\section[\thesection]{}

犹豫的人

这五位平民英雄的壮举直接导致了两个后果:一、魏忠贤害怕了,他以及他的阉党,受到了极大的震动,用历史书
上的话说,是为粉碎阉党集团奠定了群众基础。

相比而言,第二个结果有点歪打正着:七君子里最后的幸存者黄尊素,逃过了一劫。

东林党两大智囊之一的黄尊素之所以能幸免,倒不是他足智多谋,把事情都搞定了,也不是魏忠贤怕事,不敢抓
他,只是因为连颜佩韦等人都不知道,那天被他们打的人里,有几位兄弟是无辜的。

其实民变发生当天,抓周顺昌的特务和群众对峙时,有一批人恰好正经过苏州,这批人恰好也是特务----抓黄尊素
的特务。黄尊素是浙江余姚人,要到余姚,自然要经过苏州,于是就赶上了。

实在有点冤枉,这帮人既没捞钱,也没勒索,无非是过个路,可由于群众过于激动,过于能打,见到东厂装束的人
就干,就把他们顺道也干了。

要说还是特务,那反应真是快,看见一群人朝自己冲过来,虽说不知怎么回事,立马就闪人了,被逼急了就往河里
跳,总算是逃过了一劫。

可从河里出来后一摸,坏了,驾帖丢了。

所谓驾帖,大致相当于身份证加逮捕证,照眼下这情景,要是没有驾帖就跑去,能活着回来是不太正常的。想来想
去,也就不去了。

于是黄尊素纳闷了,他早就得到消息,在家等人来抓,结果等十几天,人影都没有。

但黄尊素是个聪明人,聪明人明白一个道理----覆巢之下,岂有完卵。

躲是躲不过去的,大家都死了,一个人怎能独活呢?

于是他自己穿上了囚服,到衙门去报到,几个月后,他被许显纯拷打至死。

在黄尊素走前,叫来了自己的家人,向他们告别。

大家都很悲痛,只有一个人例外。

他的儿子黄宗羲镇定地说道:``父亲若一去不归,儿子来日自当报仇!''

一年之后,他用比较残忍的方式,实现了自己的诺言。

黄尊素死了,东林党覆灭,``六君子''、``七君子''全部殉难,无一幸免,天下再无人与魏忠贤争锋。

\section[\thesection]{}

纵观东林党的失败过程,其斗争策略,就是毫无策略,除了愤怒,还是愤怒,输得那真叫彻底,局势基本是一边
倒,朝廷是魏公公的,皇帝听魏公公的,似乎毫无胜利的机会。

事实上,机会还是有的,一个。

在东林党里,有一个特殊的人,此人既有皇帝的信任,又有足以扳倒魏忠贤的实力----孙承宗。

在得知杨涟被抓后,孙承宗非常愤怒,当即决定弹劾魏忠贤。

但他想了一下,便改变了主意。

孙承宗很狡猾,他明白上书是毫无作用的,他不会再犯杨涟的错误,决定使用另一个方法。

天启四年(1625)十一月,孙承宗开始向京城进发,他此行的目的,是去找皇帝上访告状。

对一般人而言,这是不可能的,因为朱木匠天天干木匠活,不大见人,还有魏管家帮他闭门谢客,想见他老人家一
面,实在难如登天。

但孙承宗不存在这个问题,打小他就教朱木匠读书,虽说没啥效果(认字不多),但两人感情很好,魏公公几次想挑
事,想干掉孙承宗,朱木匠都笑而不答,从不理会,因为他很清楚魏公公的目的。

他并不傻,这种借刀杀人的小把戏,是不会上当的。

于是魏忠贤慌了,他很清楚,孙承宗极不简单,不但狡猾大大的,和皇帝关系铁,还手握兵权,如果让他进京打小
报告,那就真没戏了,就算没告倒,只要带兵进京来个武斗,凭自己手下这帮废物,是没指望的。

魏忠贤正心慌,魏广微又来凑热闹了,这位仁兄不知从哪得到的小道消息,说孙承宗带了几万人,打算进京修理魏
公公。

为说明事态的严重性,他还打了个生动的比方:一旦让孙大人进了京,魏公公立马就成粉了(公立齑粉矣)。

魏公公疯了,二话不说,马上跑到皇帝那里,苦苦哀求,不要让孙承宗进京,当然他的理由很正当:孙承宗带兵进
京是要干掉皇帝,身为忠臣,必须阻止此种不道德的行为。

但出乎他意料的是,皇帝大人毫不慌张,他还安慰魏公公,孙老师靠得住,就算带兵,也不会拿自己开刀的。

这个判断充分说明,皇帝大人非但不傻,还相当地幽默,魏公公被涮得一点脾气都没有。话说完,皇帝还要做木
匠,就让魏公公走人,可是魏公公不走。

他知道,今天要不讨个说法,等孙老师进京,没准就真成粉末了。所以他开始哭,且哭出了花样----``绕床痛哭''。

\section[\thesection]{}

也就是说,魏公公赖在皇帝的床边,不停地哭。皇帝在床头,他就哭到床头,皇帝到床尾,他就哭到床尾,孜孜不
倦,锲而不舍。

皇帝也是人,也要睡觉,哭来哭去,真没法了,只好发话:``那就让他回去吧。''

有了这句话,魏忠贤胆壮了,他随即命人去关外传令,让孙承宗回去。

然而不久之后,有人告诉了他一个消息,于是他又下达了第二道命令:``孙承宗若入九门,即刻逮捕!''

那个消息的内容是,孙承宗没有带兵。

孙承宗确实没有带兵,他只想上访,不想造反。

所以魏忠贤改变了主意,他希望孙承宗违抗命令,大胆反抗来到京城,并最终落入他的圈套。

事实上,这是很有可能的,鉴于地球人都知道,魏公公一向惯于假传圣旨,所以愤怒的孙承宗必定会拒绝这个无理
的命令,进入九门,光荣被捕。

然而他整整等了一夜,也没有看到这一幕。

孙承宗十分愤怒,他急匆匆地赶到了通州,却接到让他返回的命令。他的愤怒到达了顶点,于是他没有丝毫犹
豫----返回了。

孙承宗实在聪明绝顶,虽然他知道魏忠贤有假传圣旨的习惯,但这道让他返回的谕令,却不可能是假的。因为魏忠
贤知道他和皇帝的关系,他见皇帝,就跟到邻居家串门一样,说来就来了,胡说八道是没用的。

然而现在他收到了谕令,这就代表着皇帝听从了魏忠贤的忽悠,如果继续前进,后果不堪设想,所以跑路是最好的
选择。

现在摆在他面前的,有两个选择:一,回去睡觉,老老实实呆着。

二,索性带兵进京,干他娘一票,解决问题。

孙承宗是一个几乎毫无缺陷的人,政治上面很会来事,谁也动不了,军事上稳扎稳打,眼光独到,且一贯小心谨
慎,老谋深算,所以多年来,他都是魏忠贤和努尔哈赤最为害怕的敌人。

但在这一刻,他暴露出了自己人生中的最大弱点----犹豫。

孙承宗是典型的谋略型统帅,他的处事习惯是如无把握,绝不应战,所以他到辽东几年,收复无数失地,却很少打
仗。

而眼前的这一仗,他没有必胜的把握,所以他放弃。

无论这个决定正确与否,东林党已再无回天之力。

\section[\thesection]{}

三十年前,面对黑暗污浊的现实,意志坚定的吏部员外郎顾宪成相信,对的终究是对的,错的终究是错的。于是他
决心,建立一个合理的秩序,维护世上的公义,使那些身居高位者,不能随意践踏他人,让那些平凡的人,有生存
的权利。

为了这个理想,他励精图治,忍辱负重,从那个小小的书院开始,经历几十年起起落落,坚持道统,至死不渝。在
他的身后,有无数的追随者杀身成仁。

然而杀身固然成仁,却不能成事。

以天下为己任的东林党,终究再无回天之力。

其实我并不喜欢东林党,因为这些人都是书呆子,自命清高,还空谈阔论,缺乏实干能力。

小时候,历史老师讲到东林党时,曾说道:东林党人并不是进步的象征,因为他们都是封建士大夫。

我曾问:何谓封建士大夫?

老师答:封建士大夫,就是封建社会里,局限、落后,腐朽的势力,而他们的精神,绝不代表历史的发展方向。

多年以后,我亲手翻开历史,看到了另一个真相。

所谓封建士大夫,如王安石、如张居正、如杨涟、如林则徐。

所谓封建士大夫精神,就是没落,守旧,不懂变通,不识时务,给脸不要脸,瞧不起劳动人民,自命清高,即使一
穷二白,被误解,污蔑,依然坚持原则、坚持信念、坚持以天下为己任的人。

他们坚信自己的一生与众不同,高高在上,无论对方反不反感。

坚信自己生来就有责任和义务,去关怀与自己毫不相干的人,无论对方接不接受。

坚信国家危亡之际,必须挺身而出,去捍卫那些自己不认识,或许永远不会认识的芸芸众生,并为之奋斗一生,无
论对方是否知道,是否理解。

坚信无论经过多少黑暗与苦难,那传说了无数次,忽悠了无数回,却始终未见的太平盛世,终会到来。

遗弃

孙承宗失望而归,他没有能够拯救东林党,只能拯救辽东。

魏忠贤曾经想把孙老师一同干掉,可他反复游说,皇帝就是不松口,还曾经表示,如果孙老师出了事,就唯你是问。

魏公公只好放弃了,但让孙老师呆在辽东,手里握着十几万人,实在有点睡不安稳,就开始拿辽东战局说事,还找
了几十个言官,日夜不停告黑状。

孙承宗撑不下去了。

天启五年(1625)十月,他提出了辞呈。

\section[\thesection]{}

可是他提了N次,也没得到批准。

倒不是魏忠贤不想他走,是他实在走不了,因为没人愿意接班。

按魏忠贤的意思,接替辽东经略的人,应该是高第。

高第,万历十七年进士,是个相当厉害的人。

明代的官员,如果没有经济问题,进士出身,十几年下来,至少也能混个四品。而高先生的厉害之处在于,他混了
整整三十三年,熬死两个皇帝,连作风问题都没有,到天启三年(1623),也才当了个兵部侍郎,非常人所能及。

更厉害的是,高先生只当了一年副部长,第二年就退休了。

魏忠贤本不想用这人,但算来算去,兵部混过的,阉党里也只有他了。于是二话不说,把他找来,说,我要提你的
官,去当辽东经略。

高先生一贯胆小,但这次也胆大了,当即回复:不干,死都不干。

为说明他死都不干的决心,他当众给魏忠贤下跪,往死了磕头(叩头岂免):我都这把老骨头了,就让我在家养老吧。
魏忠贤觉得很空虚。

费了那么多精神,给钱给官,就拉来这么个废物。所以他气愤了:必须去!混吃等死不可能了,高第擦干眼泪,打起
精神,到辽东赴任了。

在辽东,高第用实际行动证实,他既胆小,也很无耻。

到地方后,高先生立即上了第一封奏疏:弹劾孙承宗,罪名:吃空额。

经过孙承宗的整顿,当时辽东部队,已达十余万人,对此高第是有数的,但这位兄弟睁眼说瞎话,说他数下来,只
有五万人。其余那几万人的工资,都是孙承宗领了。

对此严重指控,孙承宗欣然表示,他没有任何异议。

他同时提议,今后的军饷,就按五万人发放。

这就意味着,每到发工资时,除五万人外,辽东的其余几万苦大兵就要拿着刀,奔高经略要钱。

高第终于明白,为什么东林党都倒了,孙承宗还没倒,要论狡猾,他才刚起步。但高先生的劣根性根深蒂固,整人
不成,又开始整地方。

他一直认为,把防线延伸到锦州、宁远,是不明智的行为,害得经略大人暴露在辽东如此危险的地方,有家都回不
去,于心何忍?

还不如放弃整个辽东,退守到山海关,就算失去纵深阵地,就算敌人攻破关卡,至少自己是有时间跑路的。

\section[\thesection]{}

他不但这么想,也这么干。

天启五年(1625)十一月,高第下令,撤退。

撤退的地方包括锦州、松山、杏山、宁远、右屯、塔山、大小凌河,总之关外的一切据点,全部撤走。

撤退的物资包括:军队、平民、枪械、粮食,以及所有能搬走的物件。

他想回家,且不想再来。

但老百姓不想走,他们的家就在这里,他们已经失去很多,这是他们仅存的希望。

但他们没有选择,因为高先生说了,必须要走,``家毁田亡,嚎哭震天'',也得走。

高第逃走的时候,并没有追兵,但他逃走的动作实在太过逼真,跑得飞快,看到司令跑路,小兵自然也跑,孙承宗
积累了几年的军事物资、军粮随即丢弃一空。

数年辛苦努力,收复四百余里江山,十余万军队,几百个据点,就这样毁于一旦。

希望已经断绝,东林党垮了,孙承宗走了,所谓关宁防线,已名存实亡,时局已无希望,很快,努尔哈赤的铁蹄,
就会毫不费力地踩到这片土地上。没有人想抵抗,也没有人能抵抗,跑路,是唯一的选择。有一个人没有跑。

他看着四散奔逃的人群,无法控制的混乱,说出了这样的话: ``我是宁前道,必与宁前共存亡!我绝不入关,就算
只我一人,也要守在此处(独卧孤城),迎战敌人!''宁前道者,文官袁崇焕。

袁崇焕

若夫以一身之言动、进退、生死,关系国家之安危、民族之隆替者,于古未始有之。有之,则袁督师其人也。

----梁启超

关于袁崇焕的籍贯,是有纠纷的。他的祖父是广东东莞人,后来去了广西滕县,这就有点麻烦,名人就是资源,就
要猛抢,东莞说他是东莞人,滕县说他是藤县人,争到今天都没消停。但无论是东莞,还是滕县,当年都不是啥好
地方。

明代的进士不少,但广东和广西的很少,据统计,70\%以上都是江西、福建、浙江人。特别是广西,明代二百多
年,一个状元都没出过。

袁崇焕就在广西读书,且自幼读书,因为他家是做生意的,那年头做生意的没地位,要想出人头地,只有读书。就
智商而言,袁崇焕是不低的,他二十三岁参加广西省统一考试,中了举人,当时他很得意,写了好几首诗庆祝,以
才子自居。

一年后他才知道,自己还差得很远。

\section[\thesection]{}

袁崇焕去北京考进士了,不久之后,他就回来了。

三年后,他又去了,不久之后,又回来了。

三年后,他又去了,不久之后,又回来了。

以上句式重复四遍,就是袁崇焕同学的考试成绩。

从二十三岁,一直考到三十五岁,考了四次,四次落榜。

万历四十七年(1619),袁崇焕终于考上了进士,他的运气很好。他的运气确实很好,因为他的名次,是三甲第四十
名。明代的进士录取名额,大致是一百多人,是按成绩高低录取的,排到三甲第四十名,说明他差点没考上。

关于这一点,我曾去国子监的进士题名碑上看过,在袁崇焕的那科石碑上,我找了很久,才在相当靠下的位置(按名
次,由上往下排),找到他的名字。

在当时,考成这样,前途就算是交代了,因为在他之前,但凡建功立业、匡扶社稷,如徐阶、张居正、孙承宗等
人,不是一甲榜眼,就是探花,最次也是个二甲庶吉士。

所谓出将入相,名留史册,对位于三甲中下层的袁崇焕同志而言,是一个梦想。

当然,如同许多成功人士(参见朱重八、张居正)一样,袁崇焕小的时候,也有许多征兆,预示他将来必定有大出息。
比如他放学回家,路过土地庙,当即精神抖擞,开始教育土地公:土地公,为何不去守辽东?!

虽然我很少跟野史较真,但这个野史的胡说八道程度,是相当可以的。

袁崇焕是万历十二年(1584)生人,据称此事发生于他少年时期,往海了算,二十八岁时说了这话,也才万历四十
年,努尔哈赤先生是万历四十六年才跟明朝干仗的,按此推算,袁崇焕不但深谋远虑,还可能会预知未来。

话虽如此,但这种事总有人信,总有人讲,忽悠个上千年都不成问题。

比如那位著名的预言家查诺丹马斯,几百年前说世纪末全体人类都要完蛋,传了几百年,相关书籍、预言一大堆,
无数人信,搞得政府还公开辟谣。

我曾研习欧洲史,对这位老骗子,倒还算比较了解,几百年后不去管它,当年他曾给法兰西国王查理二世算
命,说:国王您身体真是好,能活到九十岁。

查理二世很高兴,后来挂了,时年二十四岁。

\section[\thesection]{}

总之,就当时而言,袁崇焕肯定是个人才(全国能考前一百名,自然是个人才),但相比而言,不算特别显眼的人才。

接下来的事充分说明了这点,由于太不起眼,吏部分配工作的时候,竟然把这位仁兄给漏了,说是没有空闲职位,
让他再等一年。

于是袁崇焕在家待业一年,万历四十八年(1620),他终于得到了人生中的第一个职务:福建邵武知县。

邵武,今天还叫邵武,位于福建西北,在武夷山旁边,换句话说,是山区。

在这个山区县城,袁崇焕干得很起劲,很积极,丰功伟绩倒说不上,但他曾经爬上房梁,帮老百姓救火,作为一个
县太爷,无论如何,这都是不容易的。

至于其他光辉业绩,就不得而知了,毕竟是个县城,要干出什么惊天动地的好事,很难。

天启二年(1622),袁崇焕接到命令,三年任职期满,要去北京述职。

改变命运的时刻到来了。

明代的官员考核制度,是十分严格的,京城的就不说了,京察六年一次,每次都掉层皮。即使是外面天高皇帝远的
县太爷,无论是偏远山区,还是茫茫沙漠,只要你还活着,轮到你了,就得到本省布政使那里报到,然后由布政使
组团,大家一起上路,去北京接受考核。

考核结果分五档,好的晋升,一般的留任,差点的调走,没用的退休,乱来的滚蛋。

袁崇焕的成绩大致是前两档,按常理,他最好的结局应该是回福建,升一级,到地级市接着干慢慢熬。

但袁崇焕的运气实在是好得没了边,他不但升了官,还是京官。因为一个人看中了他。

这个人的名字叫侯恂,时任都察院御史,东林党人。

侯恂是个不出名的人,级别也低,但很擅长看人,是骡子是马,都不用拉出来,看一眼就明白。

当他第一次看到袁崇焕的时候,就认定此人非同寻常,必可大用,这一点,袁崇焕自己都未必知道。

更重要的是,他的职务虽不高,却是御史,可以直接向皇帝上书。所以他随即写了封奏疏,说我发现了个人才,叫
袁崇焕,希望把他留用。

当时正值东林党当政,皇帝大人还管管事,看到奏疏,顺手就给批了。

几天后,袁崇焕接到通知,他不用再回福建当知县了,从今天起,他的职务是,兵部职方司主事,六品。

\section[\thesection]{}

顺便说句,提拔了袁崇焕的这位无名侯恂,有个著名的儿子,叫做侯方域,如果不知道此人,可以去翻翻《桃花
扇》。

接下来的事情十分有名,各种史料上都有记载:兵部职方司主事袁崇焕突然失踪,大家都很着急,四处寻找,后来
才知道,刚上任的袁主事去山海关考察了。

这件事有部分是真的,袁崇焕确实去了山海关。但猫腻在于,袁大人失踪绝不是什么大事,也没那么多人找他。当
时广宁刚刚失陷,皇帝拉着叶向高的衣服,急得直哭,乱得不行,袁主事无非是个处级干部,鬼才管他去哪。

袁崇焕回来了,并用一句话概括了他之后十余年的命运: ``予我兵马钱粮,我一人足守此!''

在当时说这句话,胆必须很壮,因为当时大家认定,辽东必然丢掉,山海关迟早失守,而万恶的朝廷正四处寻找背
黑锅的替死鬼往那里送,守辽东相当于判死刑,闯辽东相当于闯刑场。这时候放话,是典型的没事找死。

事情确实如此,袁崇焕刚刚放话,就升官了。因为朝廷听说了袁崇焕的话,大为高兴,把他提为正五品山东按察司
佥事,山海关监军,以表彰他勇于背黑锅的勇敢精神。

大家听到这个消息,不管认识的,还是不认识的,都纷纷来为袁崇焕送行,有的还带上了自己的子女,以达到深刻
的教育意义:看到了吧,这人就要上刑场了,看你还敢胡乱说话!

在一片哀叹声中,袁崇焕高高兴兴地走了,几个月后,他遇到了上司王在晋,告了他的黑状,又几个月后,他见到
了孙承宗。

且慢,且慢,在见到这两个人之前,他还遇见了另一个人,而这次会面是绝不能忽略的。

因为在会面中,袁崇焕确定了一个秘诀,四年后,努尔哈赤就败在了这个秘诀之上。

离开京城之前,袁崇焕去拜见了熊廷弼。

熊廷弼当时刚回来,还没进号子,袁崇焕上门的时候,他并未在意。在他看来,这位袁处长,不过是前往辽东挨踹
的另一个菜鸟。

所以他问: ``你去辽东,有什么办法吗?(操何策以往)''

袁崇焕思考片刻,回答: ``主守,后战。''

熊廷弼跳了起来,他兴奋异常,因为他知道,眼前的这个人已经找到了制胜的道路。

\section[\thesection]{}

所谓主守后战,就是先守再攻,说白了就是先让人打,再打人。

这是句十分简单的话。

真理往往都很简单。

正如毛泽东同志那句著名的军事格言:打得赢就打,打不赢就走。很简单,很管用。

一直以来,明朝的将领们绞尽脑汁,挖坑,造枪,练兵,修碉堡,只求能挡住后金军前进的步伐。

其实要战胜天下无双的努尔哈赤和他那可怕的骑兵,只要这四个字。这四个字他们并非不知道,只是不想知道。

作为大明天朝的将领,对付辽东地区的小小后金,即使丢了铁岭、丢了沈阳、辽阳,哪怕辽东都丢干净,也要打。

所以就算萨尔浒死十万人,沈阳死六万人,也要攻。

这不是智力问题,而是态度问题。

后金军队不过是抢东西的强盗,努尔哈赤是强盗头,对付这类货色,怎么能当缩头乌龟呢?

然而袁崇焕明白,按努尔哈赤的实力和级别,就算是强盗,也是巨盗。

他还明白,缩头的,并非一定是乌龟,毒蛇在攻击之前,也要收脖子。

后金骑兵很强大,强大到明朝骑兵已经无法与之对阵,努尔哈赤很聪明,聪明到这个世上已无几人可与之抗衡。

抱持着此种理念,袁崇焕来到辽东,接受了孙承宗的教导。在那里,他掌握胜利的手段,寻找胜利的帮手,坚定胜
利的信念。而与此同时,局势也在一步步好转,袁崇焕相信,在孙承宗的指挥下,他终将看到辽东的光复。

然而这一切注定都是幻想。

天启五年(1625)十月,他所信赖和依靠的孙承宗走了。

走时,袁崇焕前去送行,失声痛哭,然而孙承宗只能说:事已至此,我已无能为力。

然而高第来了,很快,他就看见高大人丢弃了几年来,他为之奋斗的一切,土地、防线、军队、平民,毫不吝惜,
只为保住自己的性命。

袁崇焕不撤退,虽然他只是个无名小卒,无足轻重,但他有保国的志向,制胜的方法,以及坚定的决心。

在过去的几年里,我一直这里,默默学习,默默进步,直到有一天,我看到了胜利的希望。

所以我不会撤退,即使你们全都逃走,我也绝不撤退。 ``我一人足守此!''

``独卧孤城,以当虏耳!''

现在,履行诺言的时候到了。

\section[\thesection]{}

但这个诺言注定是很难兑现的,因为两个月后,他获知了一个可怕的消息。

天启六年(1626)正月十四日,努尔哈赤来了,带着全部家当来了。根据史料分析,当时后金的全部兵力,如果加上
老头、小孩、残疾人,大致在十万左右,而真正的精锐部队,约有六七万人。

努尔哈赤的军队,人数共计六万人,号称二十万。

按某些军事专家的说法,这是当时世界上最为强大的骑兵部队,对于这个说法,我认为比较正确。

理由十分简单:对他们而言,战争是一种乐趣。

由于处于半开化状态,也不在乎什么诗书礼仪,传统道德,工作单位,打小就骑马,骁勇无畏,说打就打绝不含
糊,更绝的是,家属也大力支持:

据史料记载,后金骑兵出去拼命前,家里人从不痛哭流涕,悲哀送行,也不报怨政府,老老少少都高兴得不行,跟
过节似的,千言万语化作一句话,多抢点东西回来!坦白地讲,我很能理解这种心情,啥产业结构都没有,又不大会
种地,做生意也不在行,不抢怎么办?所以他们来了,带着抢掠的意图、锋锐的马刀和胜利的把握。

努尔哈赤是很有把握的,此前,他已等待了四年,自孙承宗到任时起。

一个卓越的战略家,从不会轻易冒险,努尔哈赤符合这个条件,他知道孙承宗的可怕,所以从不敢惹这人,但是现
在孙承宗走了。

当年秦桧把岳飞坑死了,多少还议了和,签了合同,现在魏忠贤把孙承宗整走,却是毫无附加值,还附送了许多礼
物,礼单包括锦州、松山、杏山、右屯、塔山、大小凌河以及关外的所有据点。

这一年,努尔哈赤六十七岁,就目前史料看,没有老年痴呆的迹象,他还有梦想,梦想抢掠更多的人口、牲畜、土
地,壮大自己的子民

公正地讲,站在他的立场上,这一切都无可厚非。

孙承宗走了,明军撤退了,眼前已是无人之地,很明显,他们已经失去了抵抗的勇气。

进军吧,进军到前所未至的地方,取得前所未有的胜利,无人可挡!

一切都很顺利,后金军毫不费力地占领了大大小小的据点,没有付出任何代价,直到正月二十三日那一天。

天启六年(1626)正月二十三日,努尔哈赤抵达了宁远城郊,惊奇地发现,这座城市竟然有士兵驻守,于是他派出了
使者。

\section[\thesection]{}

他毫不掩饰自己的得意,写出了如下的话: ``我带二十万人前来攻城,必破此城!如果你们投降,我给你们官做。''

在这封信中,他没有提及守将袁崇焕的姓名,要么是他不知道这个人,要么是他知道,却觉得此人不值一提。

总之在他看来,袁崇焕还是方崇焕都不重要,这座城市很快就会投降,并成为努尔哈赤旅游团路经的又一个观光景
点。

三天之后,他会永远记住袁崇焕这个名字。

他原以为要等一天,然而下午,城内的无名小卒袁崇焕就递来了回信:``这里原本就是你不要的地方,我既然恢
复,就应当死守,怎么能够投降呢?''

然后是幽默感:``你说有二十万人,我知道是假的,只有十三万而已,不过我也不嫌少!''

胜利之路

努尔哈赤决定,要把眼前这座不听话的城市,以及那个敢调侃他的无名小卒彻底灭掉。

他相信自己能够做到这一点,因为他已确知,这是一座孤城,在它的前方和后方,没有任何援军,也不会有援军,
而在城中抵挡的,只是一名不听招呼的将领,和一万多孤立无援的明军。

六年前,在萨尔浒,他用四万多人,击溃了明朝最为精锐的十二万军队,连在朝鲜打得日本人屁滚尿流的名将刘
綎,也死在了他的手上。

现在,他率六万精锐军队,一路所向披靡,来到了这座小城,面对着仅一万多人的守军,和一个叫袁崇焕的无名小
卒。

胜负毫无悬念。

对于这一点,无论是努尔哈赤以及他手下的四大贝勒,还是明朝的高第、甚至孙承宗,都持相同观点。

我们的同志在困难的时候,要看到成绩,要看到光明,要提高我们的勇气。 ----毛泽东

袁崇焕是相信光明的,因为在他的手中,有四种制胜的武器。

第一种武器叫死守,简单说来就是死不出城,任你怎么打,就不出去,死也死在城里。

虽然这个战略比较怂,但很有效,你有六万人,我只有一万人,凭什么出去让你打?有种你打进来,我就认输。

他的第二种武器,叫红夷大炮。

\section[\thesection]{}

大炮,是明朝的看家本领,当年打日本的时候,就全靠这玩意,把上万鬼子送上天,杀人还兼带毁尸功能,实在是
驱赶害虫的不二利器。

但这招在努尔哈赤身上,就不大中用了,因为日军的主力是步兵,而后金都是骑兵,速度极快,以明代大炮的射速
和质量,没打几炮马刀就招呼过来了。

袁崇焕清楚这一点,但他依然用上了大炮--进口大炮。

红夷大炮,也叫红衣大炮,纯进口产品,国外生产,国外组装。

我并非瞧不起国货,但就大炮而言,还是外国的好。其实明代的大炮也还凑合,在小型手炮上面(小佛郎机),还有
一定技术优势,但像大将军炮这种大型火炮,就出问题了。

这是一个无法攻克的技术问题----炸膛。

大家要知道,当时的火炮,想把炮弹打出去,就要装火药,炮弹越重,火药越多,如果火药装少了,没准炮弹刚出
炮膛就掉地上了,最大杀伤力也就是砸人脚,可要是装多了,由于炮管是一个比较封闭的空间,就会内部爆炸,即
炸膛。

用哲学观点讲,这是一个把炸药填入炮膛,却只允许其冲击力向一个方向(前方)前进的二律背反悖论。

这个问题到底怎么解决,我不知道,袁崇焕应该也不知道,但外国人知道,他们造出了不炸膛的大炮,并几经辗
转,落在了葡萄牙人的手里。

至于这炮到底是哪产的,史料有不同说法。有的说是荷兰,有的说是英国,罗尔斯罗伊斯还是飞利浦,都无所谓,
好用就行。

据说这批火炮共有三十门,经葡萄牙倒爷的手,卖给了明朝。拿回来试演,当场就炸膛了一门(绝不能迷信外国
货),剩下的倒还能用,经袁崇焕请求,十门炮调到宁远,剩下的留在京城装样子。

这十门大炮里,有一门终将和努尔哈赤结下不解之缘。

为保证大炮好用,袁崇焕还专门找来了一个叫孙元化的人。按照惯例,买进口货,都要配发中文说明书,何况是大
炮。葡萄牙人很够意思,虽说是二道贩子,没有说明书,但可以搞培训,就专门找了几个中国人,集中教学,而孙
元化就是葡萄牙教导班的优秀学员。

\section[\thesection]{}

袁崇焕的第三种武器,叫做坚壁清野。

为了保证不让敌人抢走一粒粮,喝到一滴水,袁崇焕命令,烧毁城外的一切房屋、草料,将所有居民转入城内。此
外,他还干了一件此前所有努尔哈赤的对手都没有干过的事----清除内奸。

努尔哈赤是个比较喜欢耍阴招的人,对派奸细里应外合很有兴趣,此前的抚顺、铁岭、辽阳、沈阳、广宁都是这么
拿下的。

努尔哈赤不了解袁崇焕,袁崇焕却很了解努尔哈赤,他早摸透了这招,便组织了除奸队,挨家挨户查找外来人口,
遇到奸细立马干掉,并且派民兵在城内站岗,预防奸细破坏。

死守、大炮、坚壁清野,但这还不够,远远不够,努尔哈赤手下的六万精兵,已经把宁远团团围住,突围是没有希
望的,死守是没有援兵的,即使击溃敌人,他们还会再来,又能支撑多久呢?

所以最终将他带上胜利之路的,是最后一种武器。

这件武器,从一道命令开始。

布置外防务后,袁崇焕叫来下属,让他立即到山海关,找到高第,向他请求一件事。

这位部下清楚,这是去讨援兵,但他也很迷茫,高先生跑得比兔子都快,才把兵撤回去,怎么可能派兵呢?``此行必
定无果,援兵是不会来的。''

袁崇焕镇定地回答:``我要你去,不是讨援兵的。''

``请你转告高大人,我不要他的援兵,只希望他做一件事。''``如发现任何自宁远逃回的士兵或将领,格杀勿论!''

这件武器的名字,叫做决心。

我没有朝廷的支持,我没有老师的指导,我没有上级的援兵,我没有胜利的把握,我没有幸存的希望。

但是,我有一个坚定的信念。

我不会后退,我会坚守在这里,战斗到最后一个人,即使同归于尽,也绝不后退。

这就是我的决心。

正月二十四日的那一天,战争即将开始之前,袁崇焕召集了他的所有部下,在一片惊愕声中,向他们跪拜。

他坦白地告诉所有人,不会有援兵,不会有帮手,宁远已经被彻底抛弃。

但是我不想放弃,我将坚守在这里,直到最后一刻。

然后他咬破中指写下血书,郑重地立下了这个誓言。

我不知道士兵们的反应,但我知道,在那场战斗中,在所有坚守城池的人身上,只有勇气、坚定和无畏,没有懦弱。

\section[\thesection]{}

天启六年正月二十四日晨,努尔哈赤带着轻蔑的神情,发动了进攻的命令,声势浩大的精锐后金军随即涌向孤独的
宁远城。

必须说明,后金军攻城,不是光膀子去的,他们也很清楚,骑着马是冲不上城墙的,事实上,他们有一套相当完整
的战术系统,大致有三拨人。

每逢攻击时,后金军的前锋,都由一种特别的兵种担任----楯兵。所有的楯兵都推着楯车。所谓楯车,是一种木
车,在厚木板的前面裹上几层厚牛皮,泼上水,由于木板和牛皮都相当皮实,明军的火器和弓箭无法射破,这是第
一拨人。

第二拨是弓箭手,躲在楯车后面,以斜四十五度角向天上射箭(射程很远),甭管射不射得中,射完就走人。

最后一拨就是骑兵,等前面都忙活完了,距离也就近了,冲出去砍人效果相当好。

无数明军就是这样被击败的,火器不管用,骑兵砍不过人家,只好就此覆灭。

这次的流程大致相同,无数的楯兵推着木车,向着城下挺进,他们相信,城中的明军和以往没有区别,火器和弓箭
将在牛皮面前屈服。

然而牛皮破了。

架着云梯的后金军躲在木板和牛皮的后面,等待靠近城墙的时刻,但他们等到的,只是晴天的霹雳声,以及从天而
降的不明物体。

值得庆祝的是,他们中的许多人还是俯瞰到了宁远城的全貌----在半空中。

宁远城头的红夷大炮,以可怕的巨响,喷射着灿烂的火焰,把无数的后金军,他们破碎的楯车,以及无数张牛皮,
都送上了天空----然后是地府。

关于红夷大炮的效果,史书中的形容相当贴切且耸人听闻:``至处遍地开花,尽皆糜烂''。

当第一声炮响的时候,袁崇焕不在城头,他正在接见外国朋友----朝鲜翻译韩瑗。

巨响吓坏了朝鲜同志,他惊恐地看着袁崇焕,却只见到一张笑脸,以及轻松的三个字: ``贼至矣!''

几个月前,当袁崇焕决心抵抗之时,就已安排了防守体系,总兵满桂守东城,参将祖大寿守南城,副将朱辅守西
城,副总兵朱梅守北城,袁崇焕坐镇中楼,居高指挥。

四人之中,以满桂和祖大寿的能力最强,他们守护的东城和南城,也最为坚固。

\section[\thesection]{}

后金军是很顽强的,在经历了重大打击后,他们毫不放弃,踩着前辈的尸体,继续向城池挺进。

他们选择的主攻方向,是西南面。

这个选择不是太好,因为西边的守将是朱辅,南边的守将是祖大寿,所以守护西南面的,是朱辅和祖大寿。

更麻烦的是,后金军刚踏着同志们的尸体冲到了城墙边,就陷入了一个奇怪的境地。

攻城的方法,大抵是一方架云梯,拼命往上爬,一方扔石头,拼命不让人往上爬,只要皮厚硬头皮,冲上去就赢了。
可是这次不同,城下的后金军惊奇地发现,除顶头挨炮外,他们的左侧、右侧、甚至后方都有连绵不断的炮火袭
击,可谓全方位、全立体,无处躲闪,痛不欲生。

这个痛不欲生的问题,曾让我百思不得其解,后来我去了一趟兴城(今宁远),又查了几张地图,解了。

简单地讲,这是一个建筑学问题。

要说清这个问题,应该画几个图,可惜我画得太差,不好拿出来丢人,只好用汉字代替了,看懂就行。

大家知道,一般的城池,是``口''字型,四四方方,一方爬,一方不让爬,比较厚道。

更猛一点的设计,是``凹''字型,敌军进攻此类城池时,如进入凹口,就会受到左中右三个方向的攻击,相当难受。

这种设计常见于大城的内城,比如北京的午门,西安古城墙的瓮城,就是这个造型。

或者是城内有点兵,没法拉出去打,又不甘心挨打的,也这么修城,杀点敌人好过把瘾。

但我查过资料兼实地观查之后,才知道,创意是没有止境的。

宁远的城墙,大致是个``山''字。

也就是说,在城墙的外面,伸出去一道城楼,在这座城楼上派兵驻守,会有很多好处,比如敌人刚进入山字的两个
入口时,就打他们的侧翼,敌人完全进入后,就打他们的屁股。如果敌人还没有进来,在城头上架门炮,可以提前
把他们送上天。

此外,这个设计还有个好处,敌人冲过来的时候,有这个玩意,可以把敌人分流成两截,分开打。

当然疑问也是有的,比如把城楼修得如此靠前,几面受敌,如果敌人集中攻打城楼,该怎么办呢?

答案:随便打,无所谓。

\section[\thesection]{}

因为这座城楼伸出去,就是让人打的。而且我查了一下,这座城楼可能是实心的,下面没有通道,士兵调遣都在城
头上进行,也就是说,即使你把城楼拆了,还得接着啃城墙,压根就进不了城。

我不知道这城楼是谁设计的,只觉得这人比较狠。

除地面外,后金军承受了来自前、后、左、右、上(天上)五个方向的打击,他们能够得到的唯一遮挡,就是同伴的
尸体,所以片刻之间,已经尸横遍野,血流成河。

然而进攻者没有退缩,无功而返,努尔哈赤的面子且不管,啥都没弄到,回去怎么跟老婆孩子交代?

在残酷的现实面前,后金军终于爆发了。

虽然不断有战友飞上天空,但他们在尸体的掩护下,终究还是来到了城下,开始架云梯。

然而炮火实在太猛,天上还不断掉石头,弓箭火枪不停地打,刚架上去,就被推下来,几次三番,他们爬墙的积极
性受到了沉重的打击,于是决定改变策略----钻洞。

具体施工方法是,在头上盖牛皮木板,用大斧、刀剑对着城墙猛劈,最终的工程目的,是把城墙凿穿。

这是一个难度很大的工程,头顶上经常高空抛物不说,还缺乏重型施工机械,就凭人刨,那真是相当之困难。

但后金军用施工成绩证明,他们之前的一切胜利,都不是侥幸取得的。

在寒冷的正月,后金挖墙队顶着炮火,凭借刀劈手刨,竟然把坚固的城墙挖出了几个大洞,按照史料的说法,是
``凿墙缺二丈者三四处'',也就是说,二丈左右的缺口,挖出了三四个。

明军毫无反应。不是没反应,而是没办法反应,因为城头的大炮是有射程的,敌人若贴近城墙,就会进入射击死
角,炮火是打不着的,而火枪、弓箭都无法穿透后金军的牛皮,只能眼睁睁地看着对方紧张施工,毫无办法。

就古代城墙而言,凿开两丈大的洞,就算是致命伤了,一般都能塌掉,但奇怪的是,洞凿开了,城墙却始终不垮。

原因在于天冷,很冷。

按史料分析,当时的温度大致在零下几十度,城墙的地基被冰冻住,所以不管怎么凿,就是垮不下来。

\section[\thesection]{}

但袁崇焕很着急,因为指望老天爷,毕竟是不靠谱的,按照这个工程进度,没过多久,城墙就会被彻底凿塌,六万
人涌进来,说啥都没用了。

当务之急,要干掉城下的那帮牛皮护身的工兵,然而大炮打不着,火枪没有用,如之奈何?

关键时刻,群众的智慧发挥了最为重要的作用。

城墙即将被攻破之际,城头上的明军突然想出了一个反击的方法。

这个方法有如下步骤,先找来一张棉被,铺上稻草,并在里面裹上火药,拿火点燃,扔到城下。

棉被、稻草加上火药,无论是材料,还是操作方法,都是平淡无奇的,但是效果,是非常恐怖的。

几年前,我曾找来少量材料,亲手试验过一次,这次实验的直接结果是,我再没有试过第二次,因为其燃烧的速度
和猛烈程度,只能用可怕两个字形容。(特别提示,该实验相当危险,切勿轻易尝试,切勿模仿,特此声明。)

明军把棉被卷起来,点上火,扔下去,转瞬间,壮观的一幕出现了。

沾满了火药的棉被开始剧烈燃烧,开始四处飘散,漂到哪里,就烧到哪里,只要沾上,就会陷入火海,即使就地翻
滚,也毫无作用。

在冰天雪地的严寒中,伴随着恐怖的大炮轰鸣声,一道火海包围了宁远城,把无数的后金军送入了地狱,英勇的后
金工程队全军覆没。

这种临时发明的武器,就是鼎鼎大名的``万人敌'',从此,它被载入史册,并成为世界上最早的燃烧瓶的雏形。

战斗,直至最后一人

眼前的一切,都超出了努尔哈赤的想象,以及心理承受程度。

万历十二年(1584),他二十五岁,以十三副盔甲起兵,最终杀掉了仇人尼堪外兰,而那一年,袁崇焕才刚刚出生。
他跟随过李成梁,打败过杨镐,杀掉了刘綎、杜松,吓走了王化贞,当他完成这些丰功伟业,名声大振的时候,袁
崇焕只是个四品文官,无名小卒。

之前几乎每一次战役,他都以少打多,以弱胜强,然而现在他带着前所未有的强大兵力,势不可挡之气魄,进攻兵
力只有自己六分之一的小人物袁崇焕,输了。

战无不胜,攻无不克,小本起家的天命大汗是不会输的,也是不能输的,即使伤亡惨重,即使血流成河,用尸体
堆,也要堆上城头!所以,观察片刻之后,他决定改变攻击的方向--南城。

\section[\thesection]{}

这个决定充分证明,努尔哈赤同志是一位相当合格的指挥官。

他认为,南城就快顶不住了。

南城守将祖大寿同意这个观点。

就实力而言,如果后金军全力攻击城池一面,明军即使有大炮,也盖不住对方人多,失守只是个时间问题。

好在此前后金军缺心眼,好好的城墙不去,偏要往夹脚里跑,西边打,南边也打,被打了个乱七八糟,现在,他们
终于觉醒了。

知错就改的后金军转换方向,向南城涌去。

我到宁远时,曾围着宁远城墙走了一圈,没掐表,但至少得半小时,宁远城里就一万多人,分摊到四个城头,也就
两千多人。以每面城墙一公里长计算,每米守兵大致是两人。

这是最乐观的估算。

所以根据数学测算,面对六万人的拼死攻击,明军是抵挡不住的。

事情发展与数学模型差不多,初期惊喜之后,后金军终于呈现出了可怕的战斗力,鉴于上面经常扔``万人敌``,墙
就不去凿了,改爬云梯。

冲过来的路上,被大炮轰死一批,冲到城脚,被烧死一批,爬墙,被弓箭、火枪射死一批。

没被轰死、烧死,射死的,接着爬。

与此同时,后金军开始组织弓箭队,对城头射箭,提供火力支援。

在这种拼死的猛攻下,明军开始大量伤亡,南城守军损失达三分之一以上,许多后金军爬上城墙,与明军肉搏,形
势十分危急。

祖大寿战败前,袁崇焕赶到了。

袁崇焕并不在城头,他所处的位置,在宁远城正中心的高楼。这个地方,我曾经去过,登上这座高楼,可以清晰地
看到四城的战况。

袁崇焕率军赶到南城,在那里,他投入了最后的预备队。长久以来的训练终于显现了效果,在强敌面前,明军毫无
畏惧,与后金军死战,把爬上城头的人赶了回去。

与此同时,为遏制后金军的攻势,明军采用了新战略----火攻。

明军开始大量使用火具,除大炮、万人敌、火枪外,火球甚至火把,但凡是能点燃的,就往城下扔。

这个战略是有道理的,你要知道,这是冬天,而冬天时,后金士兵是有几件棉衣的。

战争是智慧的源泉,很快,更缺德的武器出现了,不知是谁提议,拉出了几条长铁索,用火烧红,甩到城下用来攻
击爬墙的后金士兵。

\section[\thesection]{}

于是壮丽的一幕出现了,在北风呼啸中,几条红色的锁链在南城飘扬,它甩向哪里,惨叫就出现在哪里。

在熊熊的烈火之中,后金的攻势被遏制了,尸体堆满宁远城下,却始终未能前进一步,直至黄昏。

至此,宁远战役已进行一天,后金军伤亡惨重,死伤达一千余人,却只换来了几块城砖。

然而战斗并没有结束。

愤怒至极的努尔哈赤下达了一个出人意料的命令:夜战。

夜战并不是后金的优势,但仗打到这个份上,缩头就跑,就是一个严肃的面子问题,努尔哈赤认定,敌人城池受
损,兵力已经到达极限,只要再攻一次,宁远城就会彻底崩塌。

在领导的召唤下,后金士兵举着火把,开始了夜间的进攻。正如努尔哈赤所料,他很快就等到了崩溃的消息,后金
军的崩溃。

几次拼死进攻后,后金的士兵们终于发现,他们确实在逐渐逼近胜利----用一种最为残酷的方法:

攻击无果,伤亡很大,尸体越来越多,越来越厚,如果他们全都死光,是可以踩着尸体爬上去的。

沉默久了,就会爆发,爆发久了,就会崩溃,在又一轮的火烧、炮轰、箭射后,后金军终于违背了命令,全部后撤。

正月二十四日深夜,无奈的努尔哈赤接受了这个事实,他压抑住心中怒火,准备明天再来。

但他不知道的是,如果他不放弃进攻,第二天历史将会彻底改变。

袁崇焕也已顶不住了,他已经投入了所有的预备队,连他自己也亲自上阵,左手还负了伤,如果努尔哈赤豁出去再
干一次,后果将不堪设想。

努尔哈赤放弃了,他坚持了,所以他守住了宁远。

而下一个问题是,能否击溃后金,守住宁远。

从当天后金军的表现看,这个问题的答案是肯定的----不能。

没有帮助,没有援军,修了几年的坚城,只用一天,就被打成半成品,敌人战斗力太过强悍,很明显,如果后金军
豁出去,在这里待上几月,就是用手刨也刨下来了。对于这个答案,袁崇焕的心里是有数的。

于是,他来到了最后一个问题:既然必定失守,还守不守?他决定坚守下去,即使全军覆没,毫无希望,也要坚持到
底,坚持到最后一个人。

军队应该具有一往无前的精神,它要压倒一切敌人,而决不被敌人所屈服。不论在任何艰难困苦的场合,只要还有
一个人,这个人就要继续战斗下去。

----毛泽东

\section[\thesection]{}

袁崇焕很清楚,明天城池或许失守,或许不失守,但终究是要失守的。以努尔哈赤的操行成绩,接踵而来的,必定
是杀戮和死亡。然而袁崇焕不打算放弃,因为他是一个没有援军、没有粮食、没有理想、没有希望,依然能够坚持
下去的人。

四十二岁年前,袁崇焕出生于穷乡僻壤,一直以来,他都很平凡,平凡的中了秀才,平凡的中了举人,平凡的落
榜,平凡的再次赶考,平凡的再次落榜,平凡的最终上榜。

然后是平凡的知县,平凡的处级干部,平凡的四品文官,平凡的学生,直至他违抗命令,孤身一人,面对那个不可
一世、强大无比的对手。

四十年平凡的生活,不断的磨砺,沉默的进步,坚定的信念,无比的决心:只为一天的不朽。

正月二十五日

以前有个人对我说过这样一句话:

只要你不放弃自己,上天就不会放弃你。

绝境中的袁崇焕,在沉思中等来了正月二十五日的清晨,他终究没有放弃。

于是,他等来了奇迹。

天启六年(1626)正月二十五日,改变历史的一天。

努尔哈赤怀着满腔的愤怒,发动了新的进攻。他认为,经过前一天的攻击,宁远已近崩溃,只要最后一击,胜利触
手可得。

然而他想不到的是,战斗是以一种不可思议的形式开始的。

第一轮进攻被火炮打退后,他看见勇猛的后金士兵们怂了。无论将领们怒吼,还是威胁,以往工作积极性极高的后
金军竟然不买账了,任你怎么说,就是不冲。

这是可以理解的,大家出来打仗,说到底是想抢点东西,发发小财,现在人家炮架上了,打死上千人,尸体都堆在
那儿,还要往上冲,你当我们白内障看不见啊。

勇敢,也是要有点智商的。

努尔哈赤是很地道的,为了消除士兵们的恐惧心理,他毅然决定,停止进攻,把尸体捞回来先。

为一了百了,他还特事特办,在城外开办了简易火葬场,什么遗体告别,追悼会都省了,但凡抢回来的尸体,往里
一丢了事。

烧完,接着打。

努尔哈赤已近乎疯狂了,现在他所要的,并不是宁远,也不是辽东,而是脸面,起兵三十年,纵横天下无人可敌,
竟然攻不下一座孤城,太丢人了,实在太丢人了。

\section[\thesection]{}

所以他发誓,无论如何,一定要争回这个面子。

不想丢人,就只能丢命。

面对蜂拥而上的后金军,袁崇焕的策略还是老一套----大炮。

要说这外国货还是靠谱,顶在城头上轰了一天,非但没有炸膛,还越打越有劲,东一炮``尽皆糜烂'',西一炮``尽
皆糜烂'',相当皮实。

但是意外还是有的,具体说来是一起安全事故。

很多古装电视剧里,大炮发射大致是这么个过程:一人站在大炮后,拿一火把点引线,引线点燃后轰一声,炮口一
圈白烟,远处一片黑烟,这炮就算打出去了。

可以肯定的是,如按此方式发射红夷大炮,必死无疑。

我认为,葡萄牙人之所以卖了大炮还要教打炮,绝不仅是服务意识强,说到底,是怕出事。

由于红夷大炮的威力太大,在大炮轰击时,炮尾炸药爆炸时,会产生巨大的后座力,巨大到震死人不成问题,所以
每次发射时,都要从炮签出一条引线,人躲得远远的,拿火点燃再打出去。

经过孙元化的培训,城头的明军大都熟悉规程,严格按安全规定办事,然而在二十五日这一天,由于城头忙不过
来,一位通判也上去凑热闹,一手拿线,一手举火,就站在炮尾处点火,结果被当场震死。

但除去这起安全事故外,整体情况还算正常,大炮不停地轰,后金军不停地死,然后是抢尸体,抢完再烧,烧完再
打,打完再死,死完再抢、再烧,死死烧烧无穷尽也。

直至那历史性的一炮。

到底是哪一炮,谁都说不清,但可以肯定的是,在那寒冷的一天,漫天的炮火轰鸣声中,有一炮射向了城下,伴随
着一片惊叫和哀嚎,命中了一个目标。

这个目标到底是谁,至今不得要领,但可以肯定是相当重要的,因为一个不重要的人,不会坐在黄帐子里(并及黄龙
幕),也不会让大家如此悲痛(嚎哭奔去)。

对于此人身份,有多种说法,明朝这边,说是努尔哈赤,清朝那边,是压根不提。

这也不奇怪,如果战无不胜的努尔哈赤,在一座孤城面前,对阵一个无名小卒,被一颗无名炮弹重伤,实在太不体
面,换我,我也不说。

\section[\thesection]{}

于是接下来,袁崇焕看到了让他百思不得其解的景象,冲了两天的后金军退却了,退到了五里之外。

很明显,坐在黄帐子里的那人,是个大人物,但按照后金的道德标准,死个把领导也不是什么大事,这实在是件相
当奇怪的事情。

第二天,当袁崇焕站在城头的时候,他终于确信,自己已经创造了奇迹。

后金军仍然在攻城,攻势比前两天更为猛烈,但长期的军事经验告诉袁崇焕,这是撤退的前兆。

几个时辰之后,后金军开始总退却。

当然努尔哈赤是不会甘心的,所以在临走之前,他把所有的怒火发泄到了宁远城边的觉华岛上,那里还驻扎着几千
明军,以及上万名无辜的百姓。

那一年的冬天很冷,原本相隔几十里的大海,结上了厚厚的冰,失落的后金军踏着冰层,向岛上发动猛攻,毫无遮
挡的明军全军覆没,此外,士兵屠杀了岛上所有的百姓(逢人立碎),以显示努尔哈赤的雄才大略,并向世间证明,
努尔哈赤先生并不是无能的,他至少还能杀害手无寸铁的平民。

宁远之战就此结束,率领全部主力,拼死攻击的名将努尔哈赤,最终败给了仅有一万多人,驻守孤城的袁崇焕,铩
羽而归。

此战后金损失极为惨重,虽然按照后金的统计,仅伤亡将领两人,士兵五百人,但很明显,这是个相当谦虚的数字。

数学应用题1:十门大炮轰六万人,轰了两天半,每炮每天只轰二十炮(最保守的数字),问:总共轰多少炮?

答:以两天计算,至少四百炮。

数学应用题2:后金军总共伤亡五百人,以明军攻击数计算,平均每炮轰死多少人?

答:以五百除以四百,平均每炮轰死1.25人。

参考史料:``红夷大炮者,周而不停,每炮所中,糜烂数十尺,断无生理。''

综合由应用题1、应用题2及参考资料,得出结论如下:每一个后金士兵,都有高厚度的装甲保护,是不折不扣的钢
铁战士。

扯淡就此结束,根据保守统计,在宁远战役中,后金军伤亡的人数,大致在四千人以上,损失大量攻城车辆、兵器。

这是自万历四十六年以来,后金军的第一次总退却,战无不胜的努尔哈赤终于迎来了他人生的第一次战败。

\section[\thesection]{}

或许直到最后,他也没弄明白,到底是谁击败了他,那座孤独的宁远城,那几门外国进口的大炮,还是那一万多陷
入绝境的明军。

他不知道,他的真正对手,是一种信念。

即使绝望,毫无生机,永不放弃。

在那座孤独的城市里,有一个叫袁崇焕的人,在过去的几十年中,一直坚守着这样的信念。

他不知道,也永远不会知道了。

因为七个月后,他就翘辫子了。

天启六年(1626)八月十一日,征战半生的努尔哈赤终于逝世了。

他的死因,有很多说法,有说是被炮弹打坏的,也有的说是病死的,但无论是病死还是打死,都跟袁崇焕有着莫大
的关系。

挨炮就不说了,那么大一铁陀子,外加各类散弹,穿几个窟窿不说,再加上破伤风,这人就废定了。

就算他没挨炮,精神上也受到了严重的损害,有点心理障碍十分正常,外加努先生自打出道以来,从没吃过亏,败
在无名小卒的手上,实在太丢面子,就这么憋屈死,也是很有可能的。

在这一点上,袁崇焕也做出了很大贡献,在击退努尔哈赤后,他立即派出了使者,给努老先生送去了一封信,内容
如下:``你横行天下这么久,今天竟然败在我的手里,应该是天命吧!''

努尔哈赤很有礼貌,还派人回了礼,表示下次再跟你小子算帐(约期再战)。

至于努先生的内心活动,用他自己的话说,是这样的:``我自二十五岁起兵以来,攻无不克,战无不胜,小小的宁
远,竟然攻不下来,这是命啊!''

说完不久就死了。

一代枭雄努尔哈赤死了,对于这个人的评价,众多纷纭,有些人说他代表了先进的,进步的势力,冲击了腐败的明
朝,为历史的发展做出了贡献云云。

我才疏学浅,不敢说通晓古今,但基本道理还是懂的,遍览他的一生,我没有看到进步、发展、只看到了抢掠、杀
戮和破坏。

我不清楚什么伟大的历史意义,我只明白,他的马队所到之处,没有先进生产力,没有国民生产指数,没有经济贸
易,只有尸横遍野、残屋破瓦,农田变成荒地,平民成为奴隶。

我不知道什么必定取代的新兴霸业,我只知道,说这种话的人,应该自己到后金军的马刀下面亲身体验。

马刀下的冤魂和马鞍上的得意,没有丝毫区别,所有的生命,都是平等的,任何人都没有无故剥夺的权力。

\section[\thesection]{}

皇太极

失败的努尔哈赤悲愤了几个月后,终于笑了----含笑九泉。

老头笑着走了,有些人就笑不出来了----比如他的几个儿子。

当时,具备继承资格的人,有八个。

这八个人分别是四大贝勒:代善、阿敏、莽古尔泰、皇太极;

四小贝勒:阿济格、多尔衮、济尔哈朗、多铎。

位置只有一个。

拜许多``秘史''类电视剧所赐,这个连史学研究者都未必重视的问题,竟然妇孺皆知,且说法众多,什么努尔哈赤
讨厌皇太极,喜欢多尔衮,皇太极使坏,干掉了多尔衮他妈,抢了多尔衮的汗位等等等等。

以上讲法,在菜市场等地遇熟人时随便说说,是可以的,正式场合,就别扯了。

事实上,打努尔哈赤含笑那天起,汗位就已注定,它只属于一个人----皇太极。

因为除这位仁兄外,别人都有问题。

努尔哈赤确实很喜欢多尔衮,可是问题在于,多尔衮同志当时还是小屁孩,游牧民族比较实在,谁更能打、更能
抢,谁就是老大,要搞任人唯亲,广大后金人民是不答应的。

四小贝勒里的其他三人,那更别提了,年龄小不说,老头还不待见,以上四人可以全部淘汰。

而四大贝勒里,阿敏是努尔哈赤的侄子,没资格,排除;莽古尔泰比较蠢,性情暴躁,排除,能排上号的,只有代
善和皇太极。

但是代善也有问题----生活作风,这个问题还相当麻烦,因为据说和他传绯闻的,是努尔哈赤的后妃。

代善是聪明人,有这个前科,汗位是不敢指望了,他相当宽容地表示,自己就不争这个位置了,让皇太极干吧。

于是,在众人的一致推举下,天启六年(1626)九月初一,皇太极登基。

在后金将领中,论军事天赋,能与袁崇焕相比的,只有三个人:努尔哈赤、代善、皇太极(多尔衮比较小,不算)。

但要论政治水平,能摆上台面的,只有皇太极。因为一个月后,他做了一件努尔哈赤绝不可能做到的事。

天启六年(1626)十月,袁崇焕代表团来到了后金首都沈阳,他们此来的目的是吊丧,同时祝贺皇太极上任。

在很多书籍里,宁远战役后的袁崇焕是很悲惨的,战绩无人认可,也没有封赏,所有的功劳都被魏忠贤抢走,孤苦
伶仃,悲惨世界。

\section[\thesection]{}

可以肯定的是,这些说法是未经史籍,也未经大脑的,因为就在宁远胜利后的几天,袁崇焕就得到了皇帝的表扬,
兵部尚书王永光跟袁崇焕不大对劲,也大发感慨:八年来贼始一挫,乃知中国有人矣!

总之,捷报传来,全国欢腾,唯一不欢腾的人,就是高第。这位兄弟实在太不争气,所以连阉党都不保他,被干净
利落地革职赶回了家。

除口头表扬外,明朝也相当实在,正月底打胜,2月初就提了,先是都察院右佥都御史,一个月后又加辽东巡抚,然
后是兵部右侍郎,两个月内就到了副部级。

部下们也没有白干,满桂、赵率教、左辅、朱梅、祖大寿都升了官,连他的孙承宗老师也论功行赏了。

当然,领导的功劳是少不了的,比如魏忠贤公公,顾秉谦大人等等,虽说没去打仗,但整日忙着阴人,也是很辛苦
的。

无论如何,袁崇焕出头了,虽说他是孙承宗的学生,东林党的成员,但边界得有人守吧,所以阉党不难为他,反正
好人坏人都不管他,任他在那倒腾。

几个月后,得知努尔哈赤死讯后,他派出了代表团。

这就倒腾大了。

在明朝看来,后金就是以努尔哈赤为首的强盗团伙,压根不是政权,堂堂天朝怎么能和团伙头目谈判呢?所以多年以
来,都是只打不谈。但问题是,打来打去都没个结果,正好这次把团伙头目憋屈死了,趁机去谈谈,也没坏处。

当然,作为一名文官出身的将领,袁崇焕还有点政治头脑,谈判之前,先请示了皇帝,才敢开路。

憋死(打伤致死)了人家老爹,还派人来吊丧,是很不地道的,如此行径,是让人难以忍受的。

然而皇太极忍了。他不但忍了,还作出了出人意料的回应。

他用最高标准接待了袁崇焕的使者,好吃好喝招待,还搞了个阅兵式,让他们玩了一个多月,走的时候还送了几匹
马、几十只羊,并热情地向自己杀父仇人的使者微笑挥手告别。这意味着,一个比努尔哈赤更为可怕的敌人出现了。

懂得暴力的人,是强壮的,懂得克制暴力的人,才是强大的。

在下次战争到来之前,必须和平,这就是皇太极的真实想法。

\section[\thesection]{}

袁崇焕也并非善类,对于这次谈判,他在给皇帝的报告中,做出了充分的解释:``奴死之耗,与奴子情形,我已备
得,尚复何求?''

这句话的意思是,努尔哈赤的死讯,他儿子的情况,我都知道了,还有什么要求呢?谈来谈去,就谈出了这么个玩意。

谈判还是继续,到第二年(天启七年)正月,皇太极又派人来了。

可这人明显不上道,谈判书上还附了一篇文章----当年他爹写的七大恨。

但你要说皇太极有多恨,似乎也说不上,因为,就在七大恨后面,他还列上了谈判的条件,比如金银财宝,比如土
地等等。

也就是想多要点东西嘛,还把死去的老爷子搬出来,实在辛苦。

袁崇焕是很幽默的,他在回信中,很有耐心地逐条批驳了努尔哈赤的著作,同时表示,拒绝你的一切要求。这意思
是,虽然你爸憋屈死了,我表示同情,但谈归谈,死人我也不买账。

过了一月,皇太极又来信了,这哥们明显是玩上瘾了,他竟把袁崇焕批驳七大恨的理由,又逐条批驳了一次,当然
正事他也没忘了谈,这次他的胃口小了点,要的东西也减了半。

文字游戏玩玩是可以的,但具体工作还要干,在这一点上,皇太极同志的表现相当不错,就在给袁崇焕送信的同
时,他发动了新的进攻,目标是朝鲜。

天启七年(1627)正月初八,阿敏出兵朝鲜,朝军的表现相当稳定,依然是一如以往地不经打,一个月后平壤就失陷
了,再过一个月,朝鲜国王就签了结盟书,表示愿意服从后金。

朝鲜失陷,明朝是不高兴的,但不高兴也没办法,今天不同往日了,家里比较困难,实在没法拉兄弟一把,失陷,
就失陷了吧。

一边谈判,一边干这种事,实在太过分了,所以在来往的文书中,袁崇焕愤怒地谴责了对方的行径,痛斥皇太极没
有谈判的诚意。

话这么说,袁崇焕也没闲着,他也很忙,忙着砌砖头。自打宁远之战结束后,他就开始修墙了,打坏的重砌,没坏
的加固。他还把几万民工直接拉到锦州,抢工期抓进度,短短几个月,锦州再度成为坚城。

\section[\thesection]{}

此外,他还重新占领了之前放弃的大凌河、前屯、中后所、中右所,修筑堡垒,全面恢复关宁防线。

光修墙是不够的,为把皇太极彻底恶心死,他大量召集农民,只要来人就分地,一文钱都不要,白送,开始大规模
屯田,积累军粮。

一边谈判,一边干这种事,实在太过分了,所以在来往的文书中,皇太极愤怒地谴责了对方的行径,痛斥袁崇焕没
有谈判的诚意。

到了天启七年(1627)五月,老头子的身后事办完了,朝鲜打下来了,锦州修起来了,防线都恢复了,屯田差不多
了,双方都满意了。

打吧。

天启七年(1627)五月六日,皇太极率六万大军,自沈阳出发,进攻锦州,``宁锦大战''就此揭开序幕。

此时出战,并非皇太极的本意,老头子才挂了几个月,遗产分割、追悼会刚刚搞完,朝鲜又打了仗,实在不是进攻
的好时候,但没办法,不打不行----家里闹灾荒了。天启七年,辽东受了天灾,袁崇焕和皇太极都遭了灾,紧缺粮
食。

为解决粮食问题,袁崇焕决定,去关内调粮,补充军需。

为解决粮食问题,皇太极决定,去关内抢粮,补充军需。

没办法,吃不上饭啊,又没处调粮食,眼看着要闹事,与其闹腾我不如闹你们,索性就带他们去抢吧。

对于皇太极的这个打算,袁崇焕是有思想准备的,所以他擦亮了大炮,备齐了炮弹,静静等待着后金抢粮队到来。

宁远之战后,袁崇焕顺风顺水,官也升了,权也大了,声势如日中天,威信很高,属下十分服气。

不服气的人也是有的,比如满桂。

其实满桂和袁崇焕的关系是不错的,他之所以不服气,是因为另一个人----赵率教。

在宁远之战时,赵率教驻守前屯,打得最激烈的时候,满桂感觉要撑不住了,就派人给赵率教传令,让他赶紧派人
增援。可赵率教不去。

因为你吃不消,我也吃不消,一共这么多人,你的兵比我还多,谁增援谁?所以不去。

当时情况危急,满桂倒也没有计较,仗打完了,想起这茬了,回头要跟赵率教算帐。

于是袁崇焕出场了,现在他是辽东巡抚,遇到这种事情,自然是要和稀泥的。但是他万没有想到,这把稀泥非但没
有和成,还把自己给和进去了。

\section[\thesection]{}

因为满桂根本不买账,非但不肯了事,还把袁崇焕拉下了水,说他拉偏架。

原因在于,宁远之战前,满桂是宁远总兵,袁崇焕,是宁前道。满桂的级别比袁崇焕高,但根据以文制武惯例,袁
崇焕的地位要略高于满桂。

战后,满桂升到了右都督,袁崇焕升到兵部侍郎兼辽东巡抚,按级别,袁崇焕依然不如满桂,但论地位,他依然比
满桂高。

这就相当麻烦了,要知道,满桂光打仗就打了二三十年,他砍人头攒钱(一个五十两)的时候,袁举人还在考进士,
且他级别一直比袁崇焕高,现在又是一品武官,你个三品文官,我服从管理就不错了,瞎搅和什么?

外加他又是蒙古人,为人比较直爽,毫不虚伪,说打,操家伙就上。至于袁崇焕,他本人曾自我介绍过:``你道本
部院是个书生,本部院却是个将首!''

于是来来往往,火花四射,袁崇焕随即表示,满桂才堪大用,希望朝廷加以重用(随你怎么用,不要在这儿用)。

满桂气得不行,又干不过袁崇焕(巡抚有实权),就告到了袁崇焕的上司,新任辽东督师王之臣那里。

王之臣也是文官,所以也和稀泥,表示满桂也是个人才,你们都消停吧,都在关外为国效力。

按说和稀泥也就行了,但王督师似乎不甘寂寞,顺道还训袁崇焕几句,于是袁大人也火了,当即上书表示自己很
累,要退休(乞休)。王督师顿时火冒三丈,也上了奏疏,说自己要引退(引避)。

问题闹大了,朝廷亲自出马,使出了杀手锏----还是和稀泥。

但朝廷毕竟是朝廷,这把稀泥的质量十分之高。

先是下了封文书,给两人上了堂历史课,说此前经抚不和(指熊廷弼和王化贞),丢掉很多地方,你们要吸取教训,
不要再闹了。

然后表示,你们两个都是人才,都不要走,但为防你们两个在一起会互相死磕,特划定范围,王之臣管关内,袁崇
焕管关外,有功一起赏,有黑锅也一起背,舒坦了吧!

命令下来后,袁崇焕和王之臣都相当识趣,当即做出反应,表示愿意留任,并且同意满桂留任,继续共同工作。

不久之后,袁崇焕任命满桂镇守山海关,风波就此平息----至少他自己这样认为。

然而这件小事,最终也影响了他的命运。

\section[\thesection]{}

但不管有什么后遗症,至少在当时,形势是很好的,一片大好,

满桂守山海关,袁崇焕守宁远、锦州,所有的堡垒都已修复完毕,所有的城墙都已加固,弹药充足,粮草齐备,剩
下的只有一件事----张开怀抱等你。

五月十一日,皇太极一头扎进了怀抱。

他的六万大军分为三路,中路由他亲率,左路指挥莽古尔泰,右路指挥代善、阿敏,于同日在锦州城下会师,完成
合围。

消息传到宁远城的时候,袁崇焕慌张了。他虽然做好了准备,预料到了进攻,却没有料到,会来得这么快。

锦州城的守将是赵率教。

袁崇焕尚且没有准备,赵率教就不用说了,看城下黑压压一片,实在有点心虚,思考片刻后,他镇定下来,派两个
人爬出城墙(不能开门),去找皇太极谈判。

这两个人的到来把皇太极彻底搞迷糊了,老子兵都到城下了,你要么就打,要么投降,谈什么判?

但愿意谈判,也不是坏事,他随即写了封回信,希望赵率教早日出城投降,奔向光明。

使者拿着书信回去了,皇太极就此开始了等待,下午没信,晚上没信,到了第二天,还是没信。

于是他向城头瞭望,看到明军在抢修防御工事。

这场战役中,赵率教是比较无辜的,其实他压根就不是锦州守将,只不过是恰好呆在那里,等守将到任,就该走人
了,没想到皇太极来得太突然,没来得及走,被围在锦州了。四下一打量,官最大的也就是自己了,无可奈何,锦
州守将赵率教就此出场。

但细一分析,问题来了,辽东兵力总共有十多万,山海关有五万人,宁远有四万人,锦州只有一两万,兵力不足且
不说,连出门求援的人都还没到宁远,怎么能开打呢?

所以他决定,派人出城谈判,跟皇太极玩太极。

皇太极果然名不副实,对太极一窍不通,白等了一天,到五月十三日,想明白了,攻城。

六万后金军集结完毕,锣鼓喧天,鞭炮齐鸣,军旗招展,人山人海,等待着皇太极的指令。

皇太极沉默片刻,终于下达了指令:停止进攻。

\section[\thesection]{}

皇太极是一个不折不扣的好汉,好汉是不吃眼前亏的。

面对着城头黑洞洞的大炮,他决定,暂不进攻--谈判。

他主动派出使者,要求城内守军投降,第一次没人理他,第二次也没人理,到第三批使者的时候,赵率教估计是烦
得不行,就站到城头,对准下面一声大吼: ``要打就打,光说不顶用!(可攻不可说也)''

皇太极知道,忽悠是不行了,只能硬拼,后金军随即蜂拥而上,攻击城池。

但宁远战役的后遗症实在太过严重,后金军看见大炮就眼晕,没敢玩命,冲了几次就退了,任上级骂遍三代亲属,
就是不动。

皇太极急了,于是他坐了下来,写了一封劝降信,派人送到城门口,被射死了,又写一封,再让人去送,没人送。
无奈之下,他派人把这封劝降信射进了城里,毫无回音。

傻子都明白,你压根就攻不下来,你攻不下来,我干嘛投降?

但皇太极似乎不明白这个道理,第二天,他又派了几批使者到锦州城谈判,皇天不负有心人,终于有了回应,守军
说,你要谈判,使者是不算数的,必须派使臣来,才算正规。

皇太极欣喜若狂,连忙选了两个人,准备进城谈判。

可是这两位仁兄走到门口,原本说好开门的,偏偏不开,向上喊话,又没人答应,总而言之无人理会,只好打转回
家。

皇太极很愤怒,因为他被人涮了,但问题是,涮了他,他也没办法。

皇太极度过了失望的一天,而即将到来的第二天,却会让他绝望。

清晨,正当皇太极准备动员军队攻城的时候,城内的使者来了,不但来了,还解释了昨天没开门的原因:不是我们
不热情,实在天色太晚,不方便开门,您多见谅,今天白天再派人来,我们一定接待。

皇太极很高兴,又派出了使臣,可是到了城下,明军依然不给开门。

这批使臣还比较负责,赖在城下就不走了,于是过了一会,赵率教又出来喊了一嗓子:``你们退兵吧,我大明给赏
钱!(自有赏)''

就在皇太极被弄得几乎精神失常,气急败坏的时候,城内突然又派出了使者,表示谈可以,但不能到城里,愿意到
皇太极的大营去谈判。差点被整疯的皇太极接待了使者,并且写下了一封十分有趣的书信。

\section[\thesection]{}

这封书信并不是劝降信,而是挑战信,他在信中表示,你们龟缩在城里,不是好汉,有种就出来打,你们出一千
人,我这里只出十个人,谁打赢了,谁就算胜。你要是敢,咱们就打,要是不敢,就献出城内的所有财物,我就退
兵。

所谓一千人打不过十个人,比如一千个手无寸铁的傻子打不过十个拿机枪的特种兵,一千个平民打不过十个超人,
都是很可能的。

在这点上,皇太极体现出游牧民族的狡猾,联系到他爹喜欢玩阴的,这个提议的真正目的,不过是引明军出战。

但书信送入城后,却迟迟没有反应,连平时出来吼一嗓子的赵率教也没有踪影,无人搭理。

究其原因,还是招数太低级,这种摆明从《三国演义》上抄来的所谓激将法(《三国演义》是后金将领的标准兵书,
人手一本),只有在《三国演义》上才能用。

皇太极崩溃了,要么就打,要么就谈,要谈又不给开门,送信你又不回,你他娘到底想怎么样?

其实赵率教是有苦衷的,他本不想耍皇太极玩,可是无奈,谁让你来这么早,搞得老子走也走不掉,投降又说不过
去,只好等援兵了,可是空等实在不太像话,闲来无事谈谈判,当作消遣仅此而已。

正月十六日,消遣结束,因为就在这一天,援兵到达锦州。

得到锦州被围的消息后,袁崇焕十分焦急,他随即调派兵力,由满桂率领,前往锦州会战。援军的数量很少,只有
一万人。

六年前,在辽阳战役中,守将袁应泰以五万明军,列队城外,与数量少于自己的后金军决战,结果一塌糊涂,连自
己都搭了进去。

六年后,满桂带一万人,去锦州打六万后金军。

他毫无畏惧,因为他所率领的,是辽东最为精锐的部队----关宁铁骑。

经过几年不懈的努力,这支由辽人为主的骑兵训练有素,并配备精良的多管火器,作战极为勇猛,具有极强的冲击
力,成为明末最强悍的武装力量。

在满桂带领下,关宁铁骑日夜兼程,于十六日抵达塔山附近的笊篱山。

按照战前的部署,援军应赶到锦州附近,判明形势发动突袭,击破包围。然而这个构想被无情地打破了,因为就在
那天,一位后金将领正在笊篱山巡视----莽古尔泰。

\section[\thesection]{}

这次偶遇完全打乱了双方的计划,片刻惊讶后,满桂率先发动冲锋。

后金军毫无提防,前锋被击溃,莽古尔泰虽说比较蠢,打仗还算凑合,很快反应过来,倚仗人多,发动了反击,你
来我往几个回合,不打了。

因为大家都很忙,莽古尔泰来巡视,差不多也该回去了,满桂来解围,但按目前形势,自己不被围进去就算不错,
所以在短暂接触后,双方撤退,各回各家。

几乎就在满桂受挫的同一时刻,袁崇焕使出了新的招数。他写好了一封信,并派人秘密送往锦州城,交给赵率教。
然而不幸的是,这封信被后金军半路截获,并送到了皇太极的手中。

信的内容,让皇太极极为震惊:``锦州被围,但我已调集水师援军以及山海关、宣府等地军队,全部至宁远集结,
蒙古援军也即将到来,合计七万余人,耐心等待,必可里应外合,击破包围。''

至此,皇太极终于知道了袁崇焕的战略,确切地说,是诡计。

锦州被围,援军就这么多,所以只能忽悠,但辽东总共就这么多人,大家心知肚明,所以忽悠必须从外地着手,什
么宣府兵、蒙古兵等等,你说多少就多少,在这点上,袁崇焕干得相当好,因为皇太极信了。

五月十七日,他更改了部署。

三分之一的后金军撤除包围,在外城驻防,因为据``可靠情报'',来自全国四面八方(蒙古、宣府等)的援军,过几
天就到。

六万人都没戏,剩下这四万就可以休息了,在明军的大炮面前,后金军除了尸体,没有任何收获。

第二天,皇太极再次停止了进攻。

他又写了封信,用箭射入锦州,再次劝降。

对于他的这一举动,我也无语,明知不可能的事,还要几次三番去做,且乐此不疲,到底什么心态,实在难以理解。

估计城内的赵率教也被他搞烦了,原本还出来骂几嗓子,现在也不动弹了,连忽悠都懒得忽悠他。

五月十九日,皇太极确信,自己上当了。

很明显,除了三天前和莽古尔泰交战的那拨人外,再也没有任何援兵。

但问题是,锦州还是攻不下来,即使皇太极写信写到手软,射箭射到眼花,还是攻不下来。

\section[\thesection]{}

这样的失败是不能被接受的,所以皇太极决定,改变计划,攻击第二目标。但在此之前,他打算再试一次。

五月二十日,后金军发动了最后的猛攻。

在这几天里,日程是大致相同的,进攻,大炮,点火,轰隆,死人,撤走,抬尸体,火化,再进攻,再大炮,再点
火,再轰隆,再死人,以此类推。

五月二十五日,皇太极再也无法忍受,使出最后的杀手锏----撤退。但他的撤退相当有特点,因为他撤退的方向,
不是向后,而是向前。

他决定越过锦州,前往宁远,因为宁远,就是他的第二攻击目标。

经过审慎的思考,皇太极正确地认识到,自己面对的,是一条严密的防线,锦州不过是这条防线上的一点。

所有的防线,都有核心,要彻底攻破它,必须找到这个核心----宁远。

只要攻破宁远,就能彻底切断锦州与关内的联系,明军将永远地失去辽东

皇太极决定孤注一掷,派遣少量兵力监视锦州,率大队人马直扑宁远,他坚信,自己将在那里迎来辉煌的胜利。

宁远决战

五月二十八日,皇太极抵达宁远。

一年前,他的父亲在这里倒下,现在,他将在这里再次站立起来----反正他自己是这么想的。

然而当他靠近宁远城的时候,却看见了一幕奇特的场景。 按照惯例,进攻是这样开始的,明军守在城头,架设大
炮,后金军架好营帐,准备云梯、弓箭,然后开始攻城。但这一次,他看到的,是整齐的明军----站在城外。

总兵孙祖寿率军,驻守西门,满桂、祖大寿率军,驻守西门,其余兵力驻守南、北方向。宁远守军共三万五千余
人,位列城外,准备迎战。

现在的袁崇焕,是一个很有自信的人,他相信,凭借自己的实力,可以击败纵横天下的后金骑兵,不用龟缩城内,
不用固守城池,击败他们,就在他们的面前,用他们自己的方式!

皇太极的神经被彻底搞乱了,这个阵势已经超越了他的理解能力,于是他下达命令暂停进攻,等等看看先。

看了半天,他明白了----这是挑衅,随即发出了怒吼:``当年皇考太祖(努尔哈赤)攻击宁远,没有攻克,今天我打
锦州,又没攻克,现在敌人在外布阵,如果还不能胜,我国威何存?!''

\section[\thesection]{}

皇太极认为,不打太没面子,必须且一定要打,但有人认为,不能打。

所谓有人,是指大贝勒代善、二贝勒阿敏、三贝勒莽古尔泰。换句话说,四大贝勒里,三个都不同意。

虽说皇太极是拍板的,但毕竟是少数派,双方陷入僵持。于是皇太极说,你们都回去吧,我再考虑考虑。三个人撤
了,然而没过多久,他们就听见了进攻的号角。

对这三位大哥级人物,皇太极还是给面子的:至少把他们忽悠走了再动手。

一向只敢躲在城里打炮的明军,竟然站出来单干,实在太嚣张了,他再也无法遏制自己的愤怒,率全军发动了总攻。

很多时候,愤怒者往往是弱者。

三位贝勒毫无提防,事已至此,只能跟着冲了。

但当他们冲到城边时,才终于发现,明军敢来单干,是有原因的。

皇太极发动进攻,是打过算盘的,骑兵作战,明军不是后金军的对手,放弃拿手的大炮,偏要打马战,不占这个便
宜实在不好意思。

袁崇焕之所以摆这个阵势,是因为他认定,关宁铁骑的战斗力,足以与后金骑兵抗衡,但更重要的是,他也没说不
用大炮。

皇太极认为,当双方骑兵交战时,城头的大炮是无法发射的,因为那样可能误伤自己的军队。

袁崇焕知道这一点,但他认为,大炮是可以发射的,具体使用方法是,双方骑兵展开厮杀时,用大炮轰后金的后继
部队。

换句话说,就是引诱皇太极的骑兵进攻,等上钩的人差不多了,就用大炮攻击他们的后队,截断增援,始终保持人
多打人少。

在大炮的轰鸣声中,满桂率领骑兵,向蜂拥前来的后金军发动了冲锋。

一直以来,在后金军的眼里,明军骑兵很好欺负,一打就散,一散就跑,一跑就死,很明显,眼前的这帮对手也是
如此。

但自第一次交锋开始时起,自信就变成了绝望。

首先,这帮人使用的不是马刀,而是铁制大棒,抡起来呼呼作响,撞上就皮开肉绽,更可怕的是,这种大棒还能发
射火器,打着打着冷不丁就开枪,实在太过缺德。

而且这帮人的精神状态明显不正常,跟打了鸡血似的,一点不害怕,且战斗力极强,见人就往死里打,身中数箭数
刀,依然死战不退。

\section[\thesection]{}

在这群恐怖的对手面前,战无不胜的后金军,终于体验到了一种前所未有的经历----崩溃。

当后金军如潮水般涌来的时候,满桂知道,胜利的时刻到了。

关宁铁骑是一群不太正常的人,他们和以往的明军骑兵不同,不但是因为他们经过长期训练,且装备先进武器三眼
火铳(即当枪打,又当棒使),更为重要的原因在于,他们是既得利益者。

根据袁崇焕的原则``以辽人守辽土'',关宁铁骑的主要成员都是辽东人,因为根据以往长期实践,外地人到辽东打
仗,一般都没什么积极性,爱打不打,反正丢了就丢了,正好回老家。

而对于关宁铁骑来说,他们已经无家可归,这里就是他们唯一的家。

但最终决定他们拼命精神的,是袁崇焕的第二条原则:``以辽土养辽人''。

和当年的李成梁一样,袁崇焕很明白,要人卖命,就要给人好处。在这一点上,他毫不含糊,只要打仗就给军饷,
此外还分地,打回来的地都能分,反正是抢来的,也没谁去管,爱怎么分怎么分。更有甚者,据说每次打仗,抢回
来的战利品,他都敢分,没给朝廷报帐。

这么一算就明白了,拼死打仗,往光明了说,是保卫家园,保卫大明江山,往黑了说,打仗有工资拿,有土地分,
还能分战利品。

国仇家恨外加工资外快,要不拼命,实在没有天理。

所以每次打仗的时候,关宁铁骑都格外激动,所谓保家卫国,对他们而言,绝不是一个空洞的口号,因为踩在脚底
下那块土,没准就是他自己的家和地(地契为证)。

所以这场战斗的结局也就不难预料了,关宁铁骑如同疯子一般冲入后金骑兵队,大砍大杀,时不时还射两枪,威慑
力极大,后金军损失惨重,只能收缩等待后续部队。

而与此同时,城头的大炮开始怒吼,伴随着后金军后队的惨叫声,宣告着残酷的事实:他们的攻击已经失败。

皇太极并没有气馁,死人嘛,很正常的事情,死光拉倒,把城攻下来就行。

在他的指挥下,后金军略加整顿,向宁远城发起更猛烈的进攻。战斗持续到中午,在关宁铁骑的强大冲击力下,后
金军损失极大,却依然没有退却。

然而就在此时,皇太极得知了一个让他震惊的消息。

\section[\thesection]{}

锦州出事了

自五月十二日进攻开始,就一直呆在城里不露头的赵率教终于出现了,他没有出来喊话,而是带着一群人,冲进了
锦州城边的后金大营,一阵乱砍乱杀之后,又冲了出来,回到了城中。

这招实在太狠,城下的后金军做梦都想不到,城里这帮人竟然还敢冲出来,以致于人家砍完、杀完、跑完了,看着
眼前的尸体,还以为是在做梦。

当赵率教看见城下的后金军绕开锦州,前往宁远那一刻起,他就知道,战役的结局已经注定。

宁远的骑兵和大炮,将彻底打碎皇太极的梦想,这是毫无疑问的,而对城下的这些留守人员,是可以趁机打几下
的,当然,要等他们的主力走远点。

这次进攻导致后金军伤亡近五百人,更重要的是,它让皇太极认识到,锦州不是安全的后方,那个死不出头的赵率
教可能随时出头,将自己置于死地。

他打算放弃了,但按照以往的习惯,临走前,他还要再试一把。

后金军对宁远发动了最猛烈也是最后一次进攻,凭借着坚强的意志,尽管未能攻破关宁铁骑,部分后金军依然冲到
了宁远城边。

然后,他们看到了一道沟,很深的沟。

挖这条沟的,是袁崇焕手下的一支特殊部队--车营。

车营,是为应对后金的骑兵冲击组建的战斗团体,由步兵和战车组成,作战时推出战车,挖掘战壕,阻挡骑兵冲
击,并使用火枪和弓箭反击,攻击说不上,防守是没问题的。

没戏了,毕竟马不是坦克,开不过去,在被赶过来的关宁铁骑一顿猛打后,后金军彻底放弃,退出了战斗。

五月二十九日,皇太极离开宁远,向锦州撤退。

宁远之战,明军方面,出城迎战的满桂身中数箭(没死),他和将领尤世威的坐骑也被射死。

但在后金方面,死得就不只是马了,其伤亡极为惨重,贝勒济尔哈朗重伤,大贝勒代善的两个儿子萨哈廉和瓦克达
重伤,将领觉罗拜山、备御巴希战死,仅仅一天,后金损失高达四千余人。

皇太极走了,他原本以为能超越他的父亲,攻克这座不起眼的城市,然而事实是,上一次,他爹还在墙上刨了几个
洞,这一次,他连城墙都没摸着。

回去吧,皇太极同志,宁远是无法攻克的,回家消停几年再来。

\section[\thesection]{}

偏不消停

皇太极并不较真,但这次例外,因为他刚刚上任,面子实在是丢大了,没点业绩,将来如何服众呢?

所以在回家的路上,他又有了一个想法,攻击锦州。

这是一个将大败变成惨败的想法。

五月三十日,皇太极到达锦州,再次合围。

他整肃队伍派出骑兵,击鼓、鸣号,呐喊示威,可就是不打。

非但不打,他还把大营设在离城五里外的地方。五里,是明军大炮的最远射程。

就这样,白天派人去城边吼,晚上躲在营帐发抖,一连五天,天天如此。

六月四日,皇太极决定,发动进攻。

进攻的重点是锦州南城,后金军动用大量云梯,冒死攻城。

接下来的事情我不大想讲了,因为皇太极是个很烦人的家伙,啥新意都没有,攻城的程序,从他爹开始,一直到
他,这么多年,都没什么长进,后金军一批批上,一批批死,又一批批火化,毫无进展。

赵率教这边也差不多,他虽然进攻不大行,打防守还是不成问题的,守着城池,用大炮,看准人多的地方就轰,按
照程序操作,十分轻松。

而且趁着后金军撤走的这几天,赵率教还在城边修了几条壕沟,以保证后金军在进攻时,能在这里停上一会,为大
炮提供固定的打击地点。

战斗继续着,确切地说,不是战斗,而是屠杀。

后金军根本没法靠近城墙,每到沟边,就有定点爆破,不是被轰上天,就是被打下沟,尸横遍野。不过客观地讲,
赵率教挖这几条沟也方便了后金军,人打死就直接进了沟,管杀,也管埋。

就这样,高效率的定点爆破进行了半日,后金军伤亡极大,按赵率教的报告,打死不下三千,打伤不计其数。

明军的伤亡人数不明,但很有可能是零,因为在整个战斗中,后金军最远才到壕沟(包括沟里),以弓箭的射程,要
打死城头明军,似乎可能性不大。

打仗也是要计算成本的,这次战役,皇太极带上了全部家当,而他的全部家当,也就七万多人,按一天损失三千人
的打法,他还能打二十多天。

这生意不能再做了。

六月五日,皇太极撤军,算是彻底撤了。

\section[\thesection]{}

第二天,他率军路过大凌河城,此处空无一人,于是皇太极下令----拆了。

泄愤需要,理解万岁。

战役至此结束,五月十一日至六月五日,在长达二十余天中,后金与大明在锦州、宁远一线展开大战,最终以后金
惨败告终,史称``宁锦大捷''。

在这场战役中,后金军伤亡极大。据保守估计,应该在一万人左右,多名牛录战死,退回沈阳。

该结果充分说明,明朝只要自己不捣腾自己,后金是没戏的。

六月六日,就在皇太极撤退的第二天,袁崇焕向朝廷报捷:``十年来尽天下之兵,未尝敢于奴战,合马交锋,今始
一刀一枪拼命,不知有夷之凶狠剽悍……诸军愤恨此贼,一战挫之。''

天启皇帝回应:``十年之积弱,今日一旦挫其狂峰!''

皇帝很高兴,大臣很高兴,整个朝廷,包括魏忠贤在内,都很高兴。

现在是天启七年(1627)六月,很明显,形势还是一片大好。

天启七年(1627)七月初一,兵部侍郎、辽东巡抚袁崇焕提出,身体有病,辞职。

一般说来,辞职的原因只有一个:如果不辞职,会遇到比辞职更倒霉的事。

袁崇焕的情况更复杂一点,首先是有人告他,且告得比较狠。

宁锦大捷后几天,御史李应荐上书,弹劾袁崇焕,说他在战役中,不援助锦州,是作战不积极的表现,还用了个专
用名词----``暮气''。

``暮气''大致就是晚上的气,跟没气也差不了多少,用这个词损人,足见中华文化之博大精深。

如果你觉得这个弹劾太扯淡,那说明你还没见过世面。明代的言官,从没有想不到,也没有做不到,只有想不想
做,啥理由都能找,啥人物都敢碰,相比以往的张居正、李如松等等,这只是小儿科。

此外,不服气应该也是他辞职的原因之一。

宁锦大战后,论功行赏,最大的功劳自然是魏忠贤,头功;其次是监军太监;再其次是太监(什么都没干的);再再
其次是阉党大臣,如顾秉谦、崔呈秀等等等等;再再再其次,是魏忠贤的从孙(时年四岁,学龄前儿童),封侯爵。

袁崇焕的奖励是:升一级,赏银三十两。

如果是个老实人,也就罢了,袁崇焕的性格,要让他服气,那是梦想。

\section[\thesection]{}

而最重要,也最关键的原因在于,再干下去,就没意思了。

说到底,要想干出点成绩,自己努力是不够的,还得有人罩着,按此标准,袁崇焕只能算个体户。

许多书上说,袁崇焕之所以离职,是因为他是东林党,所以阉党容不下他,把他赶走了。

这个说法有部分不是胡扯,也就是说,有部分是胡扯,袁崇焕虽然职务不低,但在东林党里,实在是个不起眼的角
色,也没什么影响力,既不是首犯,也不算从犯,你要明白,阉党也是人,事情也多,也没功夫见人就灭,像袁崇
焕这类人物,睁只眼闭只眼就过了。

但干不下去也是实情,袁崇焕的档案实在太黑,比如,他中进士时,录取他的人是韩旷(东林党大学士),提拔他的
人是侯恂(东林党御史),培养他的人是孙承宗(模范东林党),如此背景,没抓起来就算是奇迹了,虽说他本人比较
乖巧,但要魏公公买他的帐,也不太现实。

基于以上原因,他提出辞职,基于同样原因,他的辞职被批准。

死了上万人,折腾几十天,连块砖头都没挖到的皇太极永远不会想到,袁崇焕就这么失败了,败在一个连大字都不
识的人妖手里。

妖风

魏忠贤已经是名副其实的人妖了,不是人,而是妖。

解决掉东林党,没有敌人了,就开始四处闹腾刮妖风了。

最先刮出来的,是那个妇孺皆知的称号----九千岁,但事实上,这只是个简称,全称是``九千九百岁爷爷''。

阉党的贵孙们尽力了,由于天生缺少部件和职位的稀缺性,魏人妖当不上万岁,所以只能九千九百了,用数学的角
度讲,应该算极限接近。

除称号外,魏公公丝毫不放松对自己的要求,还有个很牛的官衔,就不列出来了,因为我算了一下,总计两百多
字,全写出来比较麻烦。

光有称号和官衔是不够的,人也得实在点,吃穿住行,还得买房子。

简单点说,除了不穿龙袍,魏公公的待遇和皇帝基本是一样的,至于房子,魏公公也不怎么挑,只是比较执着----
看中了就要。

而且他还有个不好的习惯:只要,不怎么买。

比如参政米万钟,在北京郊区有套房子(园林别墅),魏忠贤看中了,象征性地出了个价,要买,米万钟不卖。

魏忠贤同意了,他免了米万钟的官职,直接占了他的房子,一分钱都没花。

\section[\thesection]{}

在强买强卖这个问题上,魏忠贤是讲究平等的,无论平民百姓还是皇亲国戚,全都一视同仁。如某位权贵有座大院
子,魏忠贤想要,人家没给,魏忠贤随即编了个罪名,把他绕了进去,还打了几十棍。

除了自己住的地方外,魏忠贤也没忘了家乡。他的老家河北肃宁,一向很穷,以出太监闻名,现在终于也露了脸。
为了让肃宁人民时刻感受到魏公公的光辉,他专门拨款(朝廷出),重新整修了肃宁城,一个小县城,挖了几条护城
河,还修了三十座敌楼,城楼十二栋,大炮就安了上百门,实在有够夸张。

问题在于,魏公公不忘家乡,却忘了老乡,肃宁的穷光蛋们还是穷光蛋,除了隔三差五被拉去砌墙,生活质量没啥
改善。

肃宁是个县城,且战略地位极其不重要,修得跟碉堡似的,这么穷的地方,请人来抢人家都未必来,搞得南来北往
的强盗们哭笑不得

搞笑的是,十几年后,后金军入侵河北,经过这里,本来没打算抢肃宁,但这城墙修得实在太好,忍不住好奇心,
就攻了一下,想打进去看看里面有多少钱。而更搞笑的是,肃宁太过坚固,任他们死攻活攻,竟然没能够攻进去(进
了白进)。

这件事告诉我们,一个人,即使是魏公公这样的人,如果下定决心要做点事,也是可以做成的。

吃喝不愁了,有房子了,光宗耀祖了,官位称号都有了,还缺吗?

还缺。

自古以来,人类追求的东西不外乎以下几种:金钱、权力、地位,这些魏忠贤全都有了。但最重要的那件东西,他
并没有得到。

那是无数帝王将相梦寐以求,却终究梦断的奢望----入圣。

成为圣贤,成为像老子、孔子、孟子一样的人,为万民景仰,为青史称颂!

问题是,魏公公不识字,也写不出《论语》、《道德经》之类的玩意,现在还镇得住,再过个几十年就没辙了。

为保证长治久安,数百年如一日地当圣人,魏忠贤干了这样几件事:

第一件是修书,虽然他不识字,但他的龟孙还是比较在行的,经过仔细钻研,一本专著随即出版发行,名为《三朝
要典》。

这是一本很有趣的书,在这本书里,讲了三个故事。

\section[\thesection]{}

第一个故事叫梃击,讲述疯子张差误闯宫廷,被王之寀诱供,以达到东林党不可告人的目的。

第二个故事叫红丸,说的是明光宗体弱多病,服用营养品``红丸'',后因体弱死去,无辜的医生李可灼被诬陷。

第三个故事移宫,是最让人气愤的,一群以杨涟为首的东林党人恶霸,趁皇帝死去,闯入宫中,欺负弱小,赶走了
善良的寡妇李选侍。

为弘扬正义,澄清事实,特作本书,由于瞎编时间短,作者水平有限,有错漏之处,敬请指正。

从这本书里,我看到了愤怒,很多人的愤怒,浙党、楚党、方从哲,以及所有政治斗争的失败者,还有那个拉住轿
子,被杨涟喝斥的小人物李进忠。

为圆满完成对东林党人的总清算,除此书外,魏忠贤还弄出了一份别出心裁的名单----东林点将录。

几年前,为了抓住伊拉克的头头们,美军特制了一副扑克牌,把人都印在上面,抓人之余还能打牌,创意备受称赞。

但和几百年前的魏公公比起来,美军就差的太远了,他的敌人们统统按照水浒传一百单八将归类编印成册,每个人
都有对应外号,读来琅琅上口,而且按牌数算,美军只有一副扑克,只能打斗地主,魏公公能做两副打拖拉机。

这份东林点将录的内容相当精彩,排第一的托塔天王,是南京户部尚书李三才,第二男主角及时雨宋江,由大学士
叶向高扮演。

戏中其余主角,以排名为序,不分姓氏笔画:

玉麒麟卢俊义----吏部尚书赵南星饰演

入云龙公孙胜----左都御史高攀龙饰演

智多星吴用----左谕德缪昌期饰演

鉴于以下一百余人中没有路人甲、宋兵乙之流,全部有名有姓有外号有官职,篇幅太长,故省略。

值得一提的是,在之前斗争中给魏人妖留下深刻印象的杨涟和左光斗,都得到了重要的角色,其中杨涟扮演的,是
大刀关胜,而左光斗,是豹子头林冲。

当然了,创意并不是魏公公首创的,灵感爆发的撰写者是王绍徽,时任吏部尚书,这位王尚书并非等闲之辈,据说
他虽然惟命是从,毫无道德,人品低劣,但相当女性化,长相柔美,还特别喜欢给人起外号,所以江湖上的朋友给
他也取了个响亮的外号--王媳妇。

\section[\thesection]{}

王媳妇向来尊重长辈,特别是对魏公公,他知道自己的公公不识字,写得太复杂看不懂,但《水浒》还是听过的,
所以想了这么个招。

魏公公很高兴,因为他终于看到了一本自己能够看懂的书,兴奋之余,他跑去找皇帝,展示这个文化成果。

可是当皇帝拿到这份东林点将录的时候,却问出了一个足以让魏公公跳河的问题:``什么是《水浒》?''

魏公公热泪盈眶了,他终于遇到了知音:在这世上,要找到一个文化比他还低的人,是太不容易了。

本着扫除文盲的决心和责任,魏文盲对朱文盲详细解说了水浒的意义和内容。皇帝满意了,他翻开首页,看到了托
塔天王李三才,随即问了第二个让魏公公崩溃的问题:``谁是托塔天王?''

如此朋友实在难寻,有生以来,魏公公第一次有机会展示自己的学问,他马上将自己听来的托塔天王晁盖的故事和
盘托出,从生平、入行当强盗、智取生辰纲,梁山结义等等,娓娓道来

然而他还没有讲完,皇帝大人就用一声大喝打断了他:``好!托塔天王,有勇有谋!''

讲坏话竟然讲出这个效果,那一刻,魏忠贤觉得自己的人生非常失败。

他闭上了嘴,收回了这本书,再没有提过,至于他回去后有没有找王媳妇算帐,就不知道了。

除著书立言外,魏公公成为圣贤的另一个标志,是修祠堂。

所谓祠堂,是用来祭奠祖先的,换句话说,供在里面的都是死人,而魏公公是唯一一个供在里面,却又活着的人。

修祠这个事,是浙江巡抚潘汝桢先弄出来的,为表尊重,他把魏公公的祠堂修在西湖边上,住在他旁边的也是位名
人----岳飞(岳庙)。

这个由头一出来,就不得了了,全国各地只要有点钱的,就修祠堂,据说袁崇焕同志也干过这活。

为显示对魏公公的尊重,祠堂选址还专挑黄金地段,比如凤阳的祠堂,就修在朱元璋祖宗皇陵的旁边。南京的祠
堂,竟然修在了朱元璋的坟头,重八兄在天有灵,知道一个死太监竟敢跟自己抢地盘,说不定会把棺材啃穿。

\section[\thesection]{}

但最猛的还是江西,江西巡抚杨邦宪要修祠堂,唯恐地段不好,竟然把朱圣贤(朱熹)的祠堂给砸了,然后在遗址上
重建,以表明不破不立的决心。

书写完了,祠堂修了,魏人妖当圣人的日子不远了,各种妖魔鬼怪就跳出来了。

最能闹腾的,是国子监监生陆万龄,他公然提出,要在国子监里给魏忠贤修祠堂。他还说,当年孔子写了《春
秋》,现在魏公公写了《三朝要典》,孔子是圣贤,所以魏公公也应该是圣贤。

无耻的人读过书后,往往会变得更加无耻。

由于这个人的恶心程度超越了人类的极限,搞得跟魏忠贤关系不错的一位国子监司业(副校长)也受不了了,表示无
法忍受,辞职走人。

面对如此光辉的荣誉,魏忠贤的内心没有一丝不安,他很高兴,也希望大家都高兴。

但这实在有点难,因为他并不是圣贤,而是死太监,是无恶不作、无耻至极的死太监。要想普天同庆,万民敬仰,
只能到梦里忽悠自己了。

捧他的人越多,骂他的人也就越多,朝廷不给骂,就在民间骂,传到魏公公耳朵里,魏公公很不高兴。

可是国家这么大,人这么多,背后骂你两句,你能如何?

魏公公说,我能。他自信的来源,就是特务。

作为东厂提督太监,魏忠贤对阴人一向很有心得,在他的领导下,东厂特务遍布全国,四下刺探。

比如在江西,有一个人到书店买书,看到《三朝要典》,就拿起来看,觉得不爽,就说了两句。

结果旁边一人突然爆起,跑过来揪住他,说自己是特务,要把他抓走,好在那人地头熟,找朋友说了几句话,又送
了点钱,总算没出事。

这个故事虽然悲剧开头,好歹喜剧结尾,下一个故事既不是悲剧,也不是喜剧,而是恐怖电影。

这个故事是我十多年前读古书时看到的,一直到今天,都没能忘记。

故事发生在一个深夜,四周无人,四个人在密室(或是地下室)交谈,大家兴致很高,边喝边谈,慢慢地,有一个人
喝多了。

酒壮胆,这位胆大的仁兄就开始骂魏忠贤,越骂越起劲,然而奇怪的是,旁边的三个人竟然沉默了,一言不发,在
密室里,静静地听着他开骂。

突然,门被人踢破了,几个人在夜色中冲了进来,把那位骂人的兄弟抓走,却没有为难那三个旁听者(请注意这句
话)。

\section[\thesection]{}

这意味着,在那天夜里,这几人的门外,有人在耐心地倾听着里面的声音。

他们不但听清了屋内的谈话,还分清了每个发言的人,以及他说话的内容。

这倒没什么,当年朱重八也干过这种事。

但最为可怕的是,这几个人,只是小人物,不是大臣,不是权贵,只是小人物。

深夜里,趴在不知名的小人物家门口,认真仔细地听着每一句话,随时准备破门而入。

周厉王的时候,但凡说他坏话的,都要被干掉,所以人们在路上遇到,只能使个眼色,不敢说话,时人称为暴政。

然而魏公公说,在家说我坏话,就以为我不知道吗,幼稚。

周厉王实行政策后没几年,百姓渐渐不满,没过几年,他就被赶到山里去了。

魏公公搞了几年,什么事都没有。

严嵩在的时候,严党不可一世,也拿徐阶没办法;张居正在的时候,内有冯保,外有爪牙,依然有言官跟他捣乱,
魏公公当政时期,这个世界很清净。

因为他搞定了所有人,包括皇帝在内。

除了皇帝,他可以干掉任何人。包括皇帝的儿子和老婆。

事实上,他也搞到了皇帝的头上。

对于天启皇帝,魏忠贤是很有好感的,这人文化比他还低,干活比他还懒,业务比他还差,如此难得的废柴,哪里
去找?

所以魏忠贤认定,在自己的这块自留地上,只能有这根废柴,任何敢于长出来的野草,都必须被连根铲除。

所谓野草,就是皇帝的儿子。

天启皇帝虽然素质差点,但生儿子还是有两把刷子的,到天启六年,他已经先后生了三个儿子。

一个都没有活下来。

天启三年十月,皇后生下一子,早产,夭折。

十余天后,慧妃生下第二子,母子平安,皇帝大喜,大赦天下,九个月后,夭折。

天启五年十月,容妃生子,八个月后,夭折

我相信,明代坐月子的水平就算比不上今天,也差不到哪去,搞出这么个百分百死亡率,要归功于魏忠贤同志的艰
苦努力。

比如第一个皇子,由于是皇后生的,大肚子时直接下手似乎有点麻烦,但要等她生下来,估计更麻烦,经过反复思
考后,魏忠贤使用了一个独特的方法,除掉这个孩子。

\section[\thesection]{}

我确信,该方法的专利不属于魏忠贤(多半是客氏),因为只有女人,才能想出如此专业,如此匪夷所思的解决方案。

按某些史料的说法,事情是这样的,皇后腰痛,要找人治,魏公公随即体贴地推荐了一个人帮她按摩,这个人在按
摩时使用了一种奇特的手法,伤了胎儿,并直接导致皇后早产,是名副其实的无痛``人''流。

如此杀人不见血之神功,实在让人叹为观止,如果这一招数流传下来,

无数药厂、医院估计就要关门大吉了。

这件事情虽然流得相当利索,但传得相当快,没过多久,宫廷内外都知道了,以至于杨涟在写那封魏忠贤二十四大
罪时,把这条也列进去。

但皇帝不知道,估计就算知道,也不信。

此后,皇帝大人的两个儿子,虽然平安出生,但几个月后就都去见列祖列宗了。

可惜,关于这两起死亡事件,没有证据显示跟魏公公有关,充其量只是嫌疑犯。问题在于,他是唯一的嫌疑犯,所
以只能委屈他,反正他身上的烂帐多了去了,也不在乎这一件。

除了皇帝的儿子外,皇帝的老婆也没能保住。

比如裕妃,原本很受皇帝宠信,但由于怀了孕,魏忠贤决定整整她,联合客氏,把她发配到冷宫。

更恶劣的是,他还调走了裕妃身边的宫女,让她单独在宫里进行生存训练,连水都没给,最后终于饥渴而死。此
外,慧妃、容妃、甚至皇后,只要是皇帝宠信的,能生儿子的,全部都挨过整。

魏忠贤的努力,最终换来了胜利的成果:登基六年的天启皇帝,虽然竭尽全力,身心健康,依然毫无收获。

魏忠贤的动机很简单,他并不想当皇帝,只是害怕生出了太子,长大后比他爹聪明,不受自己控制,就不好混了。

这个算盘没有打错,毕竟皇帝大人才二十二岁,还有很多时间,再享个十几年的福,让他生儿子也不迟。

更何况从大臣到太监,一切都在控制之中,即使新皇帝即位,也是自己说了算,世间已没有敌人了。

天启六年(1626),情况大抵如此。

但事实上,这两个假设都是错误的,首先,皇帝大人今年确实只有二十二岁,不过历史记载,他临终时,也只有二
十三岁。

其次,魏公公是有敌人的,和以往不同的是,这个敌人虽不起眼,却将置他于死地。

\section[\thesection]{}

我知道,所有的场景,荒唐的,奇异的,不可理解的,都在上天的眼里,六年前,他送来了一个女人,把魏忠贤送
上了至高无上的宝座,创造了传奇。

现在,他决定终结这个传奇,把那个当年的无赖打回原形,而承担这个任务的,也是一个女人。

这个女人叫做张嫣。

就在六年前,当客氏和魏忠贤打得火热,太监事业蒸蒸日上的时候,十五岁的张嫣进入了皇宫。

作为河南选送的后妃人选,她受到了皇帝的召见。

面试结果十分之好,张嫣年级很小,却很漂亮,皇帝很喜欢,并记下了她的名字。

而当客氏见到她时,却感受到了一种极致的惊恐,她的直觉告诉她,她所苦心经营的一切,都将毁在这个女孩的手
上。于是她去向皇帝哭诉,执意反对,要把这个小女孩送回去。

一贯对他言听计从的皇帝,第一次违背了奶妈的意愿,无论客氏哭天抢地,置若罔闻。

非但如此,十几天后,他竟然把这个女孩封了皇后,史称懿安皇后。

客氏是个相当精明的人,她认为,这个女孩太过漂亮,会影响她在皇帝心中的地位,但是她错了。

这个女孩不但漂亮,而且精明,她不但抢走了皇帝的宠信,还将夺走她所有的一切。

虽然张皇后才十五,但她的心智年龄应该是五十多,自打入宫起,就开始跟客氏干仗,且丝毫无惧,时常还把魏公
公拉进宫来骂几句,完全不把魏大人当外人,九千岁恨得咬牙切齿,没有办法。

到天启三年(1623),张皇后怀孕了,客氏无计可施,让人按摩时做了人工流产。

这件事情让客氏高兴了很久,然而她想不到的是,短暂的得意换来的,将是永远的毁灭。

在失去孩子的那一天,张皇后发誓,客氏和魏忠贤将为此付出惨重的代价。

双方矛盾开始激化,由一本书开始。

此后不久的一天,皇帝来到了张皇后的寝宫,发现她正在看书,于是发问:``你在看什么书?''

``《赵高传》。''

皇后这样回答。

皇帝没有说话,他虽然不知道托塔天王,却知道赵高。

很快,魏忠贤就知道了这件事,他十分愤怒,决定反击。

\section[\thesection]{}

第二天,皇帝在宫里闲逛的时候,意外发现了几个素未谋面的生人,大惊失色,立刻召集侍卫,经过搜查,这些人
的身上都带有武器。

此事非同小可,相关嫌疑人立即被送往东厂,进行严密审查。

这是魏忠贤的诡计,他在宫中埋伏士兵,伪装成刺客,故意被皇帝发现,而这些刺客必定会被送到东厂审问,在东
厂里,刺客们一定会坦白从宽,说出指使人,想坑谁,就坑谁。

魏忠贤想坑的人,叫做张国纪----张皇后的父亲。

这是一条相当毒辣的计策,泰山也好,岳父也罢,扯上这个罪名,上火星也跑不掉。

然而就在他准备实施这个计划时,一个人出面阻止了他。

这个人表示,即使死,他也绝不同意这种诬陷行为。

不过这位仁兄并不是什么善人,他就是魏忠贤的忠实走狗,司礼监掌印太监王体乾。

只用一句话,他就说服了魏忠贤:``皇上凡事都不怎么管,但对兄弟老婆是很好的,你要是告状,有个三长两短,
我们就没命了!''

魏忠贤到底是老江湖,立刻打消主意,为了信息安全,他干掉了那几个被他安排扮演刺客的兄弟。

皇后是干不倒了,那就一心一意跟着皇帝混吧。

可是皇帝已经混不下去了。

天启七年(1627)八月,天启皇帝病危。

病危,自然不是勤于政务,估计是做木匠太过操劳,也算是倒在了工作岗位上。

魏忠贤很伤心,真的很伤心,他很明白,如果皇帝大人就此挂掉,以后就难办了。

拜自己所赐,皇帝的几个儿子都被干掉了,所以垂帘听政、欺负小孩之类的把戏没法玩了,而皇位继承者,将是天
启皇帝的弟弟。

明光宗虽然只当了一个月皇帝,但生儿子的能力却相当了得,足足有七个。

不过很可惜,七个儿子活到现在的只剩两个,一个是天启皇帝朱由校。

而另一个,是信王朱由检,当时十七岁,他后来的称呼,叫做崇祯。

对于朱由检,魏忠贤并不了解,但他明白,十七岁的人,如果不是天启这样的极品,要想控制,难度是很大的。

废柴难得,所以当务之急,必须保住皇帝的命。

他随即公告天下,为皇帝寻找名医偏方,兵部尚书霍维华不负众望,仅用了几天,就找到了一个药方。

他说,用此药方,有起死回生之效。

出于好奇,我找到了这个药方。

\section[\thesection]{}

药名:仙方灵露饮,配方如下:

优良小米少许,加入木筒蒸煮,木筒底部镂空,安放金瓶一个,边煮边加水,煮好的米汁流入银瓶,煮到一定时
间,换新米再煮,直到银瓶满了为止。

金瓶中的液体,就是灵露,据说有长寿之功效。

事实证明,灵露确实是有效果的,天启皇帝服用后,感觉很好,连吃几天后,却又不吃了----病情加重,吃不下去。

其实对此药物,我也有所了解,按以上配方及制作方法,该灵露还有个更为通俗的称呼----米汤。

用米汤,去抢救一个生命垂危,即将歇菜的人,这充分反映了魏公公大无畏的人道主义精神。

真是蠢到家了。

皇帝大人喝下了米汤,然后依然头都不回地朝黄泉路上一路狂奔,拉都拉不住。

痛定思痛,魏忠贤决定放弃自己的医学事业,转向专业行当----阴谋。

当皇帝将死未死之时,他找到了第一号心腹崔呈秀,问他,大事可行否?

狡猾透顶的崔呈秀自然知道是什么大事,于是他立刻做出了反应----沉默。

魏忠贤再问,崔呈秀再沉默,直到魏大人生气了,他才发了句话:我怕有人闹事。

直到现在,魏忠贤才明白,自己收进来的,都是些胆小怕死的货,都靠不住,只能靠自己了。

他找到客氏,经过仔细商议,决定从宫外找几个孕妇进宫当宫女,等皇帝走人,就搞个狸猫换太子,说是皇帝的遗
腹子。反正宫里的事是他说了算,他说是,就是,不是也是。

为万无一失,他还找到了张皇后,托人告诉她,我找好了孕妇,等到那个谁死了,就生下来直接当你的儿子,接着
做皇帝,你挂个名就能当太后,不用受累。这是文明的说法,流氓的讲法自然也有,比如宫里的事我管,你要不听
话,皇帝死后怎么样就不好说了。

皇后回答:如听从你的话,必死,不听你的话,也必死,同样是死,还不如不听,死后可以见祖宗在天之灵!

说完,她就跑去找皇帝,报告此事。

按常理,这种事情,只要让皇帝知道了,是必定完蛋的。

然而当皇后见到奄奄一息的皇帝,对他说出这件事时,皇帝陛下却只说了三个字:我知道。

\section[\thesection]{}

魏忠贤并不怕皇后打小报告,在发出威胁之前,他就已经找到了皇帝,本着对社稷人民负责的态度,准备给皇后贡
献一个儿子,以保证后继有人。

皇帝非常高兴。

这很正常,皇帝大人智商本不好使,加上病得稀里糊涂,脑袋也就只剩一团浆糊了。

所以魏忠贤相信,自己的目的一定能够实现。

但他终究还是犯了一个错误,和当年东林党人一样的错误:低估女人。

今天的张皇后,就是当年的客氏,且有过之而无不及。她不但有心眼,而且很有耐心,经过和皇帝长达几个时辰的
长谈,她终于让这个人相信,传位给弟弟,才是最好的选择。

很快,住在信王府里的朱由检得到消息,皇帝要召见他。在当时的朝廷里,朱由检这个名字的意义,就是没有意义。

朱由检,生于万历三十八年,自打出生以来,一直悄无声息,什么梃击、红丸、移宫、三党、东林党、六君子,统
统没有关系。他一直很低调,从不发表意见,当然,也没人征求他的意见。但他是个明白人,至少他明白,此时此
刻召他觐见,是个什么意思。

就快断气的皇帝哥哥没有丝毫客套,一见面就拉住了弟弟的手,说了这样一句话:``来,吾弟当为尧舜。''

尧舜是什么人,大家应该知道。

朱由检惊呆了,像这种事,多少要开个会,大家探讨探讨,现在一点思想准备都没有,突然收这么大份礼,怎么好
意思呢?而且他一贯知道,自己的这位哥哥比较迟钝,没准是魏忠贤设的圈套,所以,他随即做出了答复``臣死
罪!''意思是,我不敢答应。

这一天,是天启七年(1627)八月十一日。

皇帝已经撑不了多久,他决心,把自己的皇位传给眼前的这个人,但这一切,眼前的人并不知道,他只知道,这可
能是个圈套,非常危险,绝不能答应。

两个人陷入了沉默。

在这关键时刻,一个人从屏风后面站了出来,打破了僵局,并粉碎了魏忠贤的梦想。

张皇后对跪在地上的朱由检说,事情紧急,不可推辞。朱由检顿时明白,这件事情是靠谱的,他马上答应了。

八月二十二日,足足玩了七年的木匠朱由校驾崩,年二十三。就在那一天,得知噩耗的魏忠贤没有发丧,他立即封
锁了消息。

\section[\thesection]{}

疑惑

魏忠贤的意图很明显,在彻底控制政局前,绝不能出现下一个继任者。

但就在那天,他见到了匆匆闯进宫的英国公张维迎:``你进宫干什么?''

``皇上驾崩了,你不知道?''

``谁告诉你的?''

``皇后。''

魏忠贤确信,女人是不能得罪的。

皇帝刚刚驾崩,皇后就发布了遗诏,召集英国公张维迎入宫。

在朝廷里,唯一不怕魏忠贤的,也只有张维迎了,这位仁兄是世袭公爵,无数人来了又走了,他还在那里。

张维迎接到的第一个使命,就是迎接信王即位。

事已至此,魏忠贤明白,没法再海选了,十七岁的朱由检,好歹就是他了。

他随即见风使舵,派出亲信太监前去迎接。

朱由检终于进宫了,战战兢兢地进来了。

按照以往程序,要先读遗诏,然后是劝进三次。所谓劝进,就是如果继任者不愿意当皇帝,必须劝他当。

之所以劝进三次,是因为继任者必须不愿当皇帝,必须劝三次,才当。

虽然这种礼仪相当无聊,但上千年流传下来,也就图个乐吧。

和无数先辈一样,朱由检苦苦推辞了三次,才勉为其难地答应做皇帝。

接受了群臣的朝拜后,张皇后走到他的面前,在他的耳边,对他说出了诚挚的话语:``不要吃宫里的东西(勿食宫中
食)!''

这就是新皇帝上任后,听到的第一句祝词。

他会意地点了点头。

事实上,张皇后有点杞人忧天,因为皇帝大人早有准备:他是有备而来的。照某些史料的说法,他登基的时候,随
身带着干粮(大饼),就藏在袖子里。

天启七年(1627)八月二十四日,朱由检举行登基大典,正式即位。

在登基前,他收到了一份文书,上面有四个拟好的年号,供他选择:

明代每个皇帝,只有一个年号,就好比开店,得取个好名字,才好往下干,所以选择时,必须谦虚谨慎。

第一个年号是兴福,朱由检说不好;

第二个是咸嘉,朱由检也说不好;

第三个是乾圣,朱由检还说不好;

最后一个是崇祯。

朱由检说,就这个吧。

\section[\thesection]{}

自1368年第一任老板朱元璋开店以来,明朝这家公司已经开了二百五十九年,换过十几个店名,而崇祯,将是它最
后的名字。

和以往许多皇帝一样,入宫后的第一个夜晚,崇祯没有睡着。他点着蜡烛,坐了整整一夜,不是因为兴奋,而是恐
惧,极度的恐惧。

因为他很清楚,在这座宫里,所有的人都是魏忠贤的爪牙,他随时都可能被人干掉。

每个经过他身边的人,都可能是谋杀者,他不认识任何人,也不了解任何人,在空旷而阴森的宫殿里,没有任何地
方是安全的。

于是那天夜里,他坐在烛火旁,想出了一个办法,度过这惊险的一夜。

他拦住了一个经过的太监,对他说:``你等一等。''

太监停住了,崇祯顺手取走了对方腰间的剑,说道: ``好剑,让我看看。''

但他并没有看,而是直接放在了桌上,并当即宣布,奖赏这名太监。

太监很高兴,也很纳闷,然后,他听到了一个让他更纳闷的命令:``召集所有的侍卫和太监,到这里来!''

当所有人来到宫中的时候,他们看到了丰盛的酒菜,并被告知,为犒劳他们的辛苦,今天晚上就呆在这里,皇帝请
吃饭。

人多的地方总是安全的。

第一天度过了,然后是第二天、第三天,崇祯静静地等待着,他知道,魏忠贤绝不会放过他。

但事实上,魏忠贤不想杀掉崇祯,他只想控制这个人。

而要控制他,就必须掌握他的弱点。所谓不怕你清正廉洁,就怕你没有爱好,魏忠贤相信,崇祯是人,只要是人,
就有弱点。

几天后,他给皇帝送上了一份厚礼。

这份礼物是四个女人,确切地说,是四个漂亮的女人。

男人的弱点,往往是女人,这就是魏忠贤的心得。

这个理论是比较准确的,但对皇帝,就要打折扣了,毕竟皇帝大人君临天下,要什么女人都行,送给他还未必肯要。

对此,魏忠贤相当醒目,所以他在送进女人的同时,还附送了副产品----迷魂香。

所谓迷魂香,是香料的一种,据说男人接触迷魂香后,会性欲大增,看老母牛都是双眼皮。就此而言,魏公公是很
体贴消费者的,管送还管销。

但他万万想不到,这套近乎完美的营销策略,却毫无市场效果。据内线报告,崇祯压根就没动过那几个女人。

因为四名女子入宫的那一天,崇祯对她们进行了仔细的搜查,找到了那颗隐藏在腰带里的药丸。

\section[\thesection]{}

在许多的史书中,崇祯皇帝应该是这么个形象:很勤奋,很努力,就是人比较傻,死干死干往死了干,干死也白干。

这是一种为达到不可告人目的,用心险恶的说法,

真正的崇祯,是这样的人:敏感、镇定、冷静、聪明绝顶。

其实魏忠贤对崇祯的印象很好。天启执政时,崇祯对他就很客气,见面就喊``厂公''(东厂),称兄道弟,相当激
动,魏忠贤觉得,这个人相当够意思。

经过长期观察,魏忠贤发现,崇祯是不拘小节的人,衣冠不整,不见人,不拉帮结派,完全搞不清状况。

这样的一个人,似乎没什么可担心的。

然而魏忠贤并不这样看。

几十年混社会的经验告诉他,越是低调的敌人,就越危险。

为证实自己的猜想,他决定使用一个方法。

天启七年(1627)九月初一,魏忠贤突然上书,提出自己年老体弱,希望辞去东厂提督的职务,回家养老。

皇帝已死,靠山没了,主动辞职,这样的机会,真正的敌人是不会放过的。

就在当天,他得到了回复。

崇祯亲自召见了他,并告诉了他一个秘密。

他对魏忠贤说,天启皇帝在临死前,曾对自己交代遗言:要想江山稳固,长治久安,必须信任两个人,一个是张皇
后,另一个,就是魏忠贤。

崇祯说,这句话,他从来不曾忘记过,所以,魏公公的辞呈,我绝不接受。

魏忠贤非常感动,他没有想到,崇祯竟然如此坦诚,如此和善,如此靠谱。

就在那天,魏忠贤打消了图谋不轨的念头,既然这是一个听招呼的人,就没有必要撕破脸。

崇祯没有撒谎,天启确实对他说过那句话,他也确实没有忘记,只是每当他想起这句话时,都禁不住冷笑。

天启认为,崇祯是他的弟弟,一个听话的弟弟;而崇祯认为,天启是他的哥哥,一个白痴的哥哥。

虽然只比天启小六岁,但从个性到智商,崇祯都要高出一截,魏忠贤是什么东西,他是很清楚的。

而他对魏公公的情感,也是很明确的----干掉这个死人妖,把他千刀万剐,掘坟刨尸!

每当看到这个不知羞耻的太监耀武扬威,鱼肉天下的时候,他就会产生极度的厌恶感,没有治国的能力,没有艰辛
的努力,却占据了权位,以及无上的荣耀。

\section[\thesection]{}

一切应该恢复正常了。

他不过是皇帝的一条狗,有皇帝罩着,谁也动不了他。现在皇帝换人了,没人再管这条狗,却依然动不了他。

因为这条狗,已经变成了狼。

崇祯很精明,他知道眼前的这个敌人有多么强大。

除自己外,他搞定了朝廷里所有的人,从大臣到侍卫,都是他的爪牙,身边没有盟友,没有亲信,没有人可以信
任,他将独自面对狼群。

如果冒然动手,被撕成碎片的,只有自己。

所以要对付这个人,必须有点耐心,不用着急,游戏才刚刚开始。

目标,最合适的对象

魏忠贤开始相信,崇祯是他的新朋友。

于是,天启七年(1627)九月初三,另一个人提出了辞呈。

这个人是魏忠贤的老搭档客氏。

她不能不辞职,因为她的工作是奶妈。

这份工作相当辛苦,从万历年间开始,历经三朝,从天启出生一直到结婚、生子,她都是奶妈。

现在喂奶的对象死了,想当奶妈也没辙了。

当然,她不想走,但做做样子总是要的,更何况魏姘头已经探过路了,崇祯是不会同意辞职的。

一天后,她得到了答复----同意。

这一招彻底打乱了魏忠贤的神经,既然不同意我辞职,为什么同意客氏呢?

崇祯的理由很无辜,她是先皇的奶妈,现在先皇死了,我也用不着,应该回去了吧。其实我也不好意思,前任刚死
就去赶人,但这是她提出来的,我也没办法啊。

于是在宫里混了二十多年的客大妈终于走到了终点,她穿着丧服,离开了皇宫,走的时候还烧掉了一些东西:包括
天启皇帝小时候的胎发、手脚指甲等,以示留念。

魏忠贤身边最得力的助手走了,这引起了他极大的恐慌,他开始怀疑,崇祯是一只披着羊皮的狼,正逐渐将自己推
入深渊。

还不晚,现在还有反击的机会。

但皇帝毕竟是皇帝,能不翻脸就不要翻脸,所以动手之前,必须证实这个判断。

第二天(九月初四),司礼监掌印太监王体乾提出辞职。这是一道精心设计的题目。

客氏被赶走,还可能是误会,毕竟她没有理由留下来,又是自己提出来的。而王体乾是魏忠贤的死党,对于这点,
魏忠贤知道,崇祯也知道。换句话说,如果崇祯同意,魏忠贤将彻底了解对方的真实意图。

那时,他将毫不犹豫地采取行动。

\section[\thesection]{}

一天后,他得到了回复----拒绝。

崇祯当即婉拒了王体乾的辞职申请,表示朝廷重臣,不能够随意退休。魏忠贤终于再次放心了,很明显,皇帝并不
打算动手。

这一天是天启七年(1627)九月初七。

两个月后,是十一月初七,地点,北直隶河间府阜城县。那天深夜,在那间阴森的小屋里,魏忠贤独自躺在床上,
在寒风中回想着过去,是的,致命的错误,就是这个判断。

王体乾没有退休,事实上,这对王太监而言,并非一件好事。

而刚舒坦下来的魏公公却惊奇地发现,事情发展变得越发扑朔迷离,九月十五日,皇帝突然下发旨意奖赏太监,而
这些太监,大都是阉党成员。

他还没来得及高兴,就在第二天,又传来了一个惊人的消息,都察院副都御史杨所修上疏弹劾。

杨所修弹劾的并不是魏忠贤,而是四个人,分别是兵部尚书崔呈秀,太仆寺少卿陈殷,巡抚朱童蒙,工部尚书李养
德。

这四个人的唯一共同点是,都是阉党,都是骨干,都很无耻。

虽然四个人贪污受贿,无恶不作,把柄满街都是,杨所修却分毫没有提及,事实上,他弹劾的理由相当特别----不
孝。

经杨所修考证,这四个人的父母都去世了,但都未回家守孝,全部``夺情''了,不合孝道。

这是一个很合理的理由,当年的张居正就被这件事搞得半死不活,拿出来整这四号小鱼小虾,很有意思。

魏忠贤感到了前所未有的恐惧,因为这四个人都是他的心腹,特别是崔呈秀,是他的头号死党,很明显,矛头是对
着他来的。

让人难以理解的是,自从杨涟、左光斗死后,朝廷就没人敢骂阉党,杨所修跟自己并无过节,现在突然跳出来,必
定有人主使。

而敢于主使者,只有一个人选。

然而接下来的事情,却让魏忠贤陷入了更深的疑惑。一天后,皇帝做出了批复,痛斥杨所修,说他是``率性轻
诋'',意思是随便乱骂人,

经过仔细观察,魏忠贤发现,杨所修上疏很可能并非皇帝指使,而从皇帝的表现来看,似乎事前也不知道,总之,
这只是个偶发事件。

\section[\thesection]{}

但当事人还是比较机灵的,弹劾当天,崔呈秀等人就提出了辞职,表示自己确实违反规定,崇祯安慰一番后,同意
几人回家,但出人意料的是,他坚决留下了一个人----崔呈秀。

事情解决了,几天后,另一个人却让这件事变得更为诡异。九月二十四日,国子监副校长朱三俊突然发难,弹劾自
己的学生,国子监监生陆万龄。

这位陆万龄,之前曾介绍过,是国子监的知名人物,什么在国子监里建生祠,魏忠贤应该与孔子并列之类的屁话,
都是他说的,连校长都被他气走了。

被弹劾并不是怪事,奇怪的是,弹劾刚送上去,就批了,皇帝命令,立即逮捕审问。

魏忠贤得到消息极为惊恐,毕竟陆万龄算是他的粉丝,但他到底是老江湖,当即进宫,对皇帝表示,陆万龄是个败
类,应该依法处理。

皇帝对魏忠贤的态度非常满意,夸奖了他两句,表示此事到此为止。

处理完此事后,魏忠贤拖着一身的疲惫回到了家,但他并不知道,这只是个开头。

第二天(九月二十五日),他又得知了另一个消息----一个好消息。

他的铁杆,江西巡抚杨邦宪向皇帝上书,夸奖魏忠贤,并且殷切期望,能为魏公公再修座祠堂。

魏忠贤都快崩溃了,这是什么时候,老子都快完蛋了,这帮孙子还在拍马屁,他立即向皇帝上书,说修生祠是不对
的,自己是反对的,希望一律停止。

皇帝的态度出乎意料。崇祯表示,如果没修的,就不修了,但已经批准的,不修也不好,还是接着修吧,没事。

魏忠贤并不幼稚,他很清楚,这不过是皇帝的权宜之计,故作姿态而已。

但接下来皇帝的一系列行动,却让他开始怀疑自己的看法。

几天后,崇祯下令,赐给魏忠贤的侄子魏良卿免死铁券。免死铁券这件东西,之前我是介绍过的,用法很简单,不
管犯了多大的罪,统统地免死,但有一点我忘了讲,有一种罪状,这张铁券是不能免的--谋逆。

没等魏忠贤上门感谢,崇祯又下令了,从九月底一直下令到十月初,半个多月里,封赏了无数人,不是升官,就是
封荫职(给儿子的),受赏者全部都是阉党,从魏忠贤到崔呈秀,连已经死掉的老阉党魏广微都没放过,人死了就追
认,升到太师职务才罢手。

魏忠贤终于放弃了最后的警惕,他确信,崇祯是一个好人。

\section[\thesection]{}

经过一个多月的考察,魏忠贤判定,崇祯不喜欢自己,也无法控制,但作为一个成熟的政治家,只要自己老老实实
不碍事,不挡路,崇祯没必要跟自己玩命。

这个推理比较合理,却不正确。如魏忠贤之前所料,崇祯是有弱点的,他确实有一样十分渴求的东西,不是女人,
而是权力。

要获得至高无上的权力,成为君临天下的皇帝,必须除掉魏忠贤。

青蛙遇到热水,会很快地跳出去,所以煮熟它的最好方法,是用温水。

杨所修的弹劾,以及国子监副校长的弹劾,并不是他安排的,在他的剧本里,只有封赏、安慰,和时有时无的压力。
他的目的是制造迷雾,彻底混乱敌人的神经。

经过一个多月的你来我往,紧张局势终于缓和下来,至少看上去如此。

在这片寂静中,崇祯准备着进攻。

几天后,寂静被打破了,打破它的人不是崇祯。

吏科给事中陈尔翼突然上疏,大骂杨所修,公然为崔呈秀辩护,而且还上纲上线,说这是东林余党干的,希望皇帝
严查。

和杨所修的那封上疏一样,此时上疏者,必定有幕后黑手的指使。

和上次一样,敢于主使者,只有一个人选--魏忠贤。

也和上次一样,真正的主使者,并不是魏忠贤。

杨所修上疏攻击的时候,崇祯很惊讶,陈尔翼上疏反击的时候,魏忠贤也很惊讶,因为他事先并不知道。

作为一个政治新手,崇祯表现出了极强的政治天赋,几十年的老江湖魏公公被他耍得团团转。但他并不知道,在这
场游戏中,被耍的人,还包括他自己。

看上去事情是这样的:杨所修在崇祯的指使下,借攻击崔呈秀来弹劾魏忠贤,而陈尔翼受魏忠贤的指派,为崔呈秀
辩护发动反击。

然而事情的真相,远比想象中复杂得多:杨所修和陈尔翼上疏开战,确实是有幕后黑手的,但既不是魏忠贤,也不
是崇祯。

杨所修的指使者,叫陈尔翼,而陈尔翼的指使者,叫杨所修。

\section[\thesection]{}

如果你不明白,我们可以从头解释一下这个复杂的圈套:

诡计是这样开始的,有一天,右副都御史杨所修经过对时局的分析,做出了一个肯定的判断:崇祯必定会除掉阉党。

看透了崇祯的伪装后,他决定早做打算。顺便说一句,他并不是东林党,而是阉党,但并非骨干。

为及早解脱自己,他找到了当年的同事,吏科给事中陈尔翼。

两人商议的结果是,由杨所修出面,弹劾崔呈秀。

这是条极端狡诈的计谋,是人类智商极致的体现:

弹劾崔呈秀,可以给崇祯留下一个深刻的印象,认定自己不是阉党,即使将来秋后算帐,也绝轮不到自己头上。

但既然认定崇祯要除掉阉党,要提前立功,为什么不干脆弹劾魏忠贤呢?

原因很简单,如果崇祯未必能干得过魏忠贤,到时回头清算,自己也跑不了,而且魏忠贤毕竟是阉党首领,如果首
领倒掉,就会全部清盘,彻查阉党,必定会搞到自己头上。

崔呈秀是阉党的重要人物,攻击他,可以赢得崇祯的信任,也不会得罪魏忠贤,还能把阉党以往的所有黑锅都让他
背上,精彩,真精彩。

为了大家,崔先生,你就背了吧。

这个近乎完美的计划,几乎得到了一个近乎完美的结局。几乎得到,就是没有得到。

因为计划的进行过程中,出现了纰漏:他们忽略了一个人----崔呈秀。

杨所修、陈尔翼千算万算,却算漏了崔呈秀本人,能成为阉党的头号人物,崔大人绝非善类,这把戏能骗过魏忠
贤,却骗不了崔呈秀。

弹劾发生的当天,他就看穿了这个诡计,他意识到,大祸即将临头。但他只用了几天时间,就十分从容地解决了这
个问题。

他派人找到了杨所修,大骂了对方一顿,最后说,如果你不尽快了结此事,就派人查你。

大家同坐一条船,谁的屁股都不干净,敢玩阴的,大家就一起完蛋!

这句话相当有效,杨所修当即表示,愿意再次上疏,为崔呈秀辩解。

问题是,他已经骂过了,再上疏辩护,实在有点婊子的感觉,所以,这个当婊子的任务,就交给了陈尔翼。

问题是,原先把崔呈秀推出来,就是让他背锅的,现在把他拉出来,就必须填个人进去,杨所修不行,魏忠贤不
行,崇祯更不行,实在很难办。

但陈尔翼不愧是老牌给事中,活人找不到,找到了死人。

他把所有的责任,都推到了所谓``东林余孽''的身上,如此一来,杨所修是无知的,崔呈秀是无辜的,世界又和平
了。

\section[\thesection]{}

倒腾来,又倒腾去,崔呈秀没错,杨所修没错,陈尔翼当然也没错,所有的错误,都是东林党搞的,就这样,球踢
到了崇祯的身上。

但最有水平的,还是崇祯,面对陈尔翼的奏疏,他只说了几句话,就把球踢到天上: ``大臣之间的问题,先帝(指
天启)已经搞清楚了,我刚上台(朕初御极),这些事情不太清楚,也不打算深究,你们不许多事!''

结果非常圆满,崔呈秀同志洗清了嫌疑,杨所修和陈尔翼虽说没有收获,也没有损失,完美落幕。

但事情的发展,却出现了意想不到的变化。

天启七年(1627)十月十五日,云南监察御史杨维垣上疏,弹劾崔呈秀贪权弄私,十恶不赦!

在这封文书中,杨维垣表现出极强的正义感,他愤怒地质问阉党,谴责了崔呈秀的恶行。

杨维垣是阉党。

说起来大家的智商都不低,杨所修的创意不但属于他,也属于无数无耻的阉党同仁们,反正干了也没损失,不干白
不干,白干谁不干?

形势非常明显,崔呈秀已经成为众矢之的,对于立志搞掉阉党的崇祯而言,这是最好的机会。

但崇祯没有动手。他不但没有动手,还骂了杨维垣,说他轻率发言。

事实上,他确实不打算动手,虽然他明知现在解决崔呈秀,不但轻而易举,还能有效打击阉党,但他就是不动手。

因为他的直觉告诉他,在杨维垣的这封奏疏背后,隐藏着不可告人的秘密。

很快,他的直觉得到了证实。

几天后,杨维垣再次上疏,弹劾崔呈秀。

这是一个怪异的举动,皇帝都发了话,依然豁出去硬干,行动极其反常。

而反常的原因,就在他的奏疏里。 在这封奏疏里,他不但攻击崔呈秀,还捧了一个人----魏忠贤。

照他的说法,长期以来,崔呈秀没给魏忠贤帮忙,净添乱,是不折不扣的罪魁祸首。

崇祯的判断很正确,在杨维垣的背后,是魏忠贤的身影。从杨所修的事情中,魏忠贤得到了启示:全身而退绝无可
能,要想平安过关,必须给崇祯一个交代。

\section[\thesection]{}

所以他指使杨维垣上书,把责任推给崔呈秀,虽然一直以来,崔呈秀都帮了很多忙,还是他的干儿子。

没办法,关键时刻,老子自己都保不住,儿子你就算了吧。

但崇祯是不会上当的,在这场残酷的斗争中,目标只有一个,不需要俘虏,也不接受投降。

夜半歌声

真正的机会到来了。

十月二十三日,工部主事陆澄源上书,弹劾崔呈秀,以及魏忠贤。

崇祯决定,开始行动。

因为他知道,这个叫陆澄源的人并不是阉党分子,此人职位很小,但名气很大,具体表现为东林党当政,不理东林
党,阉党上台,不理阉党,是公认的混不吝,软硬都不吃,他老人家动手,就是真要玩命了。

接下来的是例行程序,崇祯照例批评,崔呈秀照例提出辞职。

但这一次,崇祯批了,勒令崔呈秀立即滚蛋回家。

崔呈秀哭了,这下终于完蛋了。

魏忠贤笑了,这下终于过关了。

丢了个儿子,保住了命,这笔交易相当划算。

但很快,他就知道自己错了。

两天后,兵部主事钱元悫上书,痛斥崔呈秀,说崔呈秀竟然还能在朝廷里混这么久,就是因为魏忠贤。

然后他又开始痛斥魏忠贤,说魏忠贤竟然还能在朝廷里混这么久,就是因为皇帝。

不知钱主事是否过于激动,竟然还稍带了皇帝,但更令人惊讶的是,这封奏疏送上去的时候,皇帝竟然全无反应。

几天后,刑部员外郎史躬盛上疏,再次弹劾魏忠贤,在这封奏疏里,他痛责魏忠贤,为表达自己的愤怒,还用上了
排比句。

魏忠贤终于明白,自己上当了,然而为时已晚。

说到底,还是读书太少,魏文盲并不清楚,朝廷斗争从来只有单项选择,不是你死,就是我活。

天启皇帝死的那天,他的人生就只剩下一个选择----谋逆。

他曾胜券在握,只要趁崇祯立足未稳,及早动手,一切将尽在掌握。

然而,那个和善、亲切的崇祯告诉他,自己将继承兄长的遗愿,重用他,信任他,太阳照常升起。

于是他相信了。

所以他完蛋了。

现在反击已不可能,从他抛弃崔呈秀的那一刻开始,他就失去了所有的威信,一个不够意思的领导,绝不会有够意
思的员工。

阉党就此土崩瓦解,他的党羽纷纷辞职,干儿子、干孙子跟他划清界线,机灵点的,都在家写奏疏,反省自己,痛
骂魏公公,告别过去,迎接美好的明天。

面对铺天盖地而来的狂风暴雨,魏忠贤决定,使出自己的最后一招。

\section[\thesection]{}

当年他曾用过这一招,效果很好。

这招的名字,叫做哭。

在崇祯面前,魏忠贤嚎啕大哭,失声痛哭,哭得死去活来。

崇祯开始还安慰几句,等魏公公哭到悲凉处,只是不断叹气。

眼见哭入佳境,效果明显,魏公公收起眼泪,撤了。

哭,特别是无中生有的哭,是一项历史悠久的高难度技术。当年严嵩就凭这一招,哭倒了夏言,最后将其办挺。他
也曾凭这一招,扭转了局势,干掉了杨涟。

魏公公相信,凭借自己声情并茂的表演,一定能够感动崇祯。

崇祯确实很感动。

他没有想到,一个人竟然可以恶心到这个程度,都六十的人了,几乎毫无廉耻,眼泪鼻涕说下就下,不要脸,真不
要脸。

到现在,朝廷内外,就算是扫地的老头,都知道崇祯要动手了。但他就不动手,他还在等一样东西。

其实朝廷斗争,就是街头打架斗殴,但斗争的手段和程序比较特别。拿砖头硬干是没办法的,手持西瓜刀杀入敌阵
也不是不行的,必须遵守其自身规律,在开打之前,要先放风声,讲明老子是哪帮哪派,要修理谁,能争取的争
取,不能争取的死磕,才能动手。

崇祯放出了风声,他在等待群臣的响应。

可是群臣不响应。

截至十月底,敢公开上书弹劾魏忠贤的人只有两三个,这一事实说明,经过魏公公几年来的言传身教,大多数的人
已经没种了。

没办法,这年头混饭吃不易,等形势明朗点,我们一定出来落井下石。

然而崇祯终究等来了一个有种的人。

十月二十六日,一位国子监的学生对他的同学,说了这样一句话:``虎狼在前,朝廷竟然无人敢于反抗!我虽一介平
民,愿与之决死,虽死无撼!''

第二天,国子监监生钱嘉征上书弹劾魏忠贤十大罪。

钱嘉征虽然只是学生,但文笔相当不错,内容极狠,态度极硬,把魏忠贤骂得狗血淋头,引起极大反响。

魏忠贤得到消息,十分惊慌,立即进宫面见崇祯。

遗憾,他没有玩出新意,还是老一套,进去就哭,哭的痛不欲生,感觉差不多了,就收了神功,准备回家。

就在此时,崇祯叫住了他: ``等一等。''

他找来一个太监,交给他一份文书,说:``读。''

\section[\thesection]{}

就这样,魏忠贤亲耳听到了这封要命的文书,每一个字都清清楚楚。

他痛苦地抬起头,却只看到了一双冷酷的眼睛和嘲弄的眼神。

那一刻,他的威望、自信、以及抵抗的决心,终于彻底崩溃。

精神近乎失常的魏忠贤离开了宫殿,但他没有回家,而是去了另一个地方,在那里,还有一个人,能挽救所有的一
切。

魏忠贤去找的人,叫做徐应元。

徐应元的身份,是太监,不同的是,十几年前,他就是崇祯的太监。事到如今,只能求他了。

徐应元是很够意思的,他客气地接待了魏忠贤,并给他指出了一条明路:立即辞职,退休回家,可以保全身家性命。

魏忠贤思前想后,认了。

立即回家,找人写辞职信,当然,临走前,他没有忘记感谢徐应元对他的帮助。

徐应元之所以帮助魏忠贤,是想让他死得更快。

和魏忠贤一样,大多数太监的习惯是见风使舵,落井下石。

一直以来,崇祯都希望,魏忠贤能自动走人(真心实意),毕竟阉党根基太深,这样最省事。

在徐应元的帮助下,第二天,魏忠贤提出辞职了,这次他很真诚。

同日,崇祯批准了魏忠贤的辞呈,一代巨监就此落马。

落马的那天,魏忠贤很高兴。因为他认为,自己已经放弃了争权,无论如何,崇祯都不会也没有必要赶尽杀绝。

一年前,东林党人也是这样认为的。

应该说,魏忠贤的生活是很不错的,混了这么多年,有钱有房有车,啥都不缺了。特别是他家的房子,就在现在北
京的东厂胡同,二环里,黄金地段,交通便利,我常去附近的社科院近代史所开会,曾去看过,园林假山、深宅大
院,上千平米,相当气派,但据说这只是当年他家的角落,最多也就六分之一。

从河北肃宁的一个小流氓,混到这个份上,也就差不多了,好歹有个留京指标。

但这个指标的有效期,也只有三天了。

天启七年(1627)十一月一日,崇祯下令,魏忠贤劳苦功高,另有重用----即日出发,去凤阳看坟。

得到消息的魏忠贤非常沮丧,但他不知道,崇祯也很沮丧。

崇祯是想干掉魏忠贤的,但无论如何,魏公公总算是三朝老监,前任刚死两个月,就干掉他实在不好意思。

但接下来发生的事情,却改变了他的决定。

\section[\thesection]{}

当他宣布赶走魏忠贤的时候,有一个人站了出来,反对他的决定,而这个人,是他做梦都想不到的。

或许是收了钱,或许是说了情,反正徐应元是站出来了,公然为魏忠贤辩护,希望皇帝给他个面子。

面对这个伺候了自己十几年,一向忠心耿耿的老太监,崇祯毫不犹豫地做出了抉择:``奴才!敢与奸臣相通,打一百
棍,发南京!''

太监不是人啊。

顺便说一句,在明代,奴才是朝廷大多数太监的专用蔑呼,而在清代,奴才是朝廷大多数人的尊称(关系不好还不能
叫,只能称臣,所谓做奴才而不可得)。

这件事情让崇祯意识到,魏忠贤是不会消停的。而下一件事使他明白,魏忠贤是非杀不可的。

确定无法挽回,魏公公准备上路了,足足准备了三天。在这三天里,他只干了一件事----打包。

既然荣华于我如浮云,那就只要富贵吧。

但这是一项相当艰苦的工作,几百个仆人干了六天,清出四十大车,然后光荣上路,前呼后拥,随行的,还有一千
名隶属于他本人的骑兵护卫。

就算是轻度弱智的白痴,都知道现在是个什么状况,大难当头,竟然如此嚣张,真是活腻了。

魏忠贤没有活腻,他活不到九千九百岁,一百岁还是要追求的。

事实上,这个大张旗鼓的阵势,是他最后的诡计。

这个诡计的来由是历史。

历史告诉我们,战国的时候,秦军大将王翦出兵时,一边行军一边给秦王打报告,要官要钱,贪得无厌,有人问
他,他说,我军权在手,只有这样,才能让秦王放心。

此后,这一招被包括萧何在内的广大仁人志士(识相点的)使用,魏忠贤用这招,说明他虽不识字,却还是懂得历史
的。

可惜,是略懂。

魏公公的用意是,自己已经无权无势,只求回家过几天舒坦日子,这么大排场,只是想告诉崇祯老爷,俺不争了,
打算好好过日子。

然而,他犯了一个错误----没学过历史唯物主义。

\section[\thesection]{}

所谓历史唯物主义的要点,就是所有的历史事件,都要根据当时的历史环境来考虑。

王翦的招数能够奏效,是因为他手中有权,换句话说,他的行为,实际上是跟秦王签合同,我只要钱要官,帮你打
江山,绝不动你的权。

此时的魏忠贤,已经无权无官,凭什么签合同?

所以崇祯很愤怒,他要把魏忠贤余下的都拿走,他的钱,还有他的命。

魏忠贤倒没有这个觉悟,他依然得意洋洋地出发了。

但聪明人还是有的,比如他的心腹太监李永贞,就曾对他说,低调,低调点好。

魏忠贤回答:若要杀我,何须今日?

今日之前,还无须杀你。

魏忠贤出发后的第三天,崇祯传令兵部,发出了逮捕令。

这一天是十一月六日,魏忠贤所在的地点,是直隶河间府阜城县。

护卫簇拥的魏公公终于明白了自己的处境,几天来,他在京城的内线不断向他传递着好消息:他的亲信,包括五虎、
五彪纷纷落马,老朋友王体乾退了,连费尽心思拉下水的徐应元也被发配去守陵,翻身已无指望。

就在他情绪最为低落的时候,京城的快马又告诉他一个最新的消息:皇帝已经派人追上来了。

威严的九千九百岁大人当场就晕了过去。

追上来,然后呢?逮捕,入狱,定罪,斩首?还是挨剐?

天色已晚,无论如何,先找个地方住吧,活过今天再说。

魏忠贤进入了眼前的这座小县城:他人生中的最后一站。阜城县是个很小的县城,上千人一拥而入,挤满了所有的
客店,当然,魏忠贤住的客店,是其中最好的。

为保证九千岁的人有地方住,许多住店的客人都被赶了出去,虽然天气很冷,但这无关紧要,毕竟他们都是无关紧
要的人。在这些人中,有个姓白的书生,来自京城。

所谓最好的客店,也不过是几间破屋而已,屋内没有辉煌的灯光,十一月的天气非常的冷,无情的北风穿透房屋,
发出凄冷的呼啸声。

在黑暗和寒冷中,伟大的,无与伦比的,不可一世的九千九百岁蜷缩在那张简陋的床上,回忆着过往的一切。

隆庆年间出生的无业游民,文盲,万历年间进宫的小杂役,天启年间的东厂提督,朝廷的掌控者,无数孙子的爷
爷,生祠的主人,堪与孔子相比的圣人。

到而今,只剩破屋、冷床,孤身一人。

荒谬,究竟是自己,还是这个世界?

四十年间,不过一场梦幻。

不如死了吧。

此时,他的窗外,站立着那名姓白的书生。

在这个寒冷的夜晚,没有月光,在黑暗和风声中,书生开始吟唱。

\section[\thesection]{}

夜半,歌起

在史料中,这首歌的名字叫做《桂枝儿》,但它还有一个更贴切的名字----五更断魂曲。

曲分五段,从一更唱到五更:

一更,愁起

听初更,鼓正敲,心儿懊恼。

想当初,开夜宴,何等奢豪。

进羊羔,斟美酒,笙歌聒噪.

如今寂廖荒店里,只好醉村醪。

又怕酒淡愁浓也,怎把愁肠扫?

二更,凄凉

二更时,展转愁,梦儿难就。

想当初,睡牙床,锦绣衾稠。

如今芦为帷,土为坑,寒风入牖。

壁穿寒月冷,檐浅夜蛩愁。

可怜满枕凄凉也,重起绕房走。

三更,飘零

夜将中,鼓咚咚,更锣三下。

梦才成,又惊觉,无限嗟呀。

想当初,势顷朝,谁人不敬?

九卿称晚辈,宰相为私衙。

如今势去时衰也,零落如飘草。

四更,无望

城楼上,敲四鼓,星移斗转。

思量起,当日里,蟒玉朝天。

如今别龙楼,辞凤阁,凄凄孤馆。

鸡声茅店里,月影草桥烟。

真个目断长途也,一望一回远。

五更,荒凉

闹攘攘,人催起,五更天气。

正寒冬,风凛冽,霜拂征衣。

更何人,效殷勤,寒温彼此。

随行的是寒月影,吆喝的是马声嘶。

似这般荒凉也,真个不如死!

五更已到,曲终,断魂。

多年后,史学家计六奇在他的书中记下了这个夜晚发生的一切,但这一段,在后来的史学研究中,是有争议的,就
史学研究而言,如此诡异的景象,实在不像历史。

但我相信,在那个夜晚,我们所知的一切是真实的。

因为历史除了正襟危坐,一丝不苟外,有时也喜欢开开玩笑,算算总账。

至于那位姓白的书生,据说是河间府的秀才,之前为图嘴痛快,说了魏忠贤几句坏话,被人告发前途尽墨,于是编
曲一首,等候于此不计旧恶,帮其送终。

但在那天夜里,魏忠贤听到的,不是这首曲子,而是他的一生。

想当初,开夜宴,何等奢豪。想当初,势顷朝,谁人不敬?

如今寂廖荒店里,只好醉村醪,如今势去时衰也,零落如飘草。

\section[\thesection]{}

魏忠贤是不相信天道的。当无赖时,他强迫母亲改嫁,卖掉女儿,当太监时,他抢夺朋友的情人,出卖自己的恩人。

九千九百岁时,他泯灭一切人性,把铁钉钉入杨涟的脑门,把东林党赶尽杀绝。

他没有信仰,没有畏惧,没有顾忌。

然而天道是存在的,四十年后,他把魏忠贤送到了阜城县的这所破屋里。

这里距离魏公公的老家肃宁,只有几十里。四十年前,他经过这里,踏上了前往京城的路。现在,他回来了,即将
失去所有的一切。

我认为,这是一种别开生面的折腾,因为得到后再失去,远比一无所有要痛苦得多。

魏公公费尽心力,在成功的路上一路狂奔,最终却发现,是他娘的折返跑。

似这般荒凉也,真个不如死!

真个不如死啊!

那就死吧。

魏忠贤找到了布带,搭在了房梁上,伸进自己的脖子,离开了这个世界。

天道有常,或因人势而迟,然终不误。

落水狗

第二天早上,魏忠贤的心腹李朝钦醒来,发现魏忠贤已死,绝望之中,自缢而亡。

在魏忠贤的一千多陪同人员,几千朝廷死党里,他是唯一陪死的人。

得知魏忠贤的死讯后,一千多名护卫马上行动起来,瓜分了魏公公的财产,四散奔逃而去。

魏公公死了,但这场大戏才刚刚开始。

别看今天闹得欢,当心将来拉清单

----小兵张嘎

清单上的第一个人,自然是客氏。

虽然她已经离宫,但崇祯下令,把她又拎了进来。

进来后先审,但客氏为人极其阴毒,且以耍泼闻名,问什么都骂回去。

于是换人,换了个太监审,而且和魏忠贤有仇(估计是专门找来的),由于不算男人,也就谈不上不打女人,加上没
文化,不会吵架,二话不说就往死里猛打。

客氏实在是个不折不扣的软货,一打就服,害死后妃,让皇后流产,找孕妇入宫冒充皇子,出主意害人等等,统统
交代,只求别打。

但那位太监似乎心理有点问题,坦白交代还打,直到奄奄一息才罢休。

口供报上来,崇祯十分震惊,下令将客氏送往浣衣局做苦工。

当然了,这只是个说法,客氏刚进浣衣局,还没分配工作,就被乱棍打死,跟那位被她关入冷宫,活活渴死的后妃
相比,这种死法没准还算痛快点。

客氏死后,她的儿子被处斩,全家被发配。

按身份排,下一个应该是崔呈秀。

但是这位兄弟实在太过自觉,自觉到死得比魏公公还要早。

得知魏忠贤走人的消息后,崔呈秀下令,准备一桌酒菜,开饭。

\section[\thesection]{}

吃饭的方式很特别,和韦小宝一样,他把自己大小老婆都拉出来,搞了个聚餐,还摆上了多年来四处搜刮的古玩财
宝。

然后一边吃,一边拿起他的瓶瓶罐罐(古董),砸。

吃一口,砸一个,吃完,砸完,就开始哭。

哭好,就上吊。

按日期推算,这一天,魏忠贤正在前往阜城县的路上。

兄弟先走一步。

消息传到京城,崇祯非常气愤,老子没让你死,你就敢死?

随即批示: ``虽死尚有余辜!论罪!''

经过刑部商议,崔呈秀应该斩首。

虽然人已死了,不要紧,有办法。

于是刚死不久的崔呈秀又被挖了出来,被斩首示众,怎么杀是个能力问题,杀不杀是个态度问题。

接下来是抄家,无恶不作的崔呈秀,终于为人民做了件有意义的事,由于他多年来勤奋地贪污受贿,存了很多钱,
除动产外,还有不动产,光房子就有几千间,等同于替国家攒钱,免去了政府很多麻烦。

作为名单上的第三号人物,崔呈秀受到了高标准的接待,以此为基准,一号魏忠贤和二号客氏,接待标准应参照处
理。

所以,魏忠贤和客氏被翻了出来,客氏的尸体斩首,所谓死无全尸。

魏忠贤惨点,按崇祯的处理意见,挖出来后剐了,死后凌迟,割了几千刀。

这件事情的实际意义是有限的,最多也就是魏公公进了地府,小鬼认不出他,但教育意义是巨大的,在残缺的尸体
面前,明代有史以来最大,最邪恶的政治团体阉党,终于彻底崩盘。

接下来的场景,是可以作为喜剧素材的。

魏忠贤得势的时候,无数人前来投奔,上至六部尚书,大学士,下到地方知府知县,能拉上关系,就是千恩万谢。

现在而今眼目下,没办法了,能撤就撤,不能撤就推,比如蓟辽总督阎鸣泰,有一项绝技----修生祠,据我统计,
他修的生祠有十余个,遍布京城一带,有的还修到了关外,估计是打算让皇太极也体验一下魏公公的伟大光辉。

凭借此绝活,当年很是风光,现在麻烦了,追查阉党,头一个就查生祠,谁让修的,谁出的钱,生祠上都刻着,跑
都跑不掉。

\section[\thesection]{}

为证明自己的清白,阎总督上疏,进行了耐心的说明,虽说生祠很多,但还是可以解释的,如保定的生祠,是顺天
巡抚刘诏修的,通州的生祠,是御史梁梦环修的,这些人都是我的下级,作为上级领导,责任是有的,监督不够是
有的,检讨是可以的,撤职坐牢是不可以的。

但最逗的还是那位国子监的陆万龄同学,本来是一穷孩子,卖力捧魏公公,希望能够混碗饭吃,当年也是风光一
时,连国子监的几位校长都争相支持他,陆先生本人也颇为得意。

然而学校领导毕竟水平高,魏公公刚走,就翻脸了,立马上疏,表示国子监本与魏忠贤势不两立,出了陆万龄这种
败类,实在是教育界的耻辱,将他立即开除出校。

据统计,自天启七年(1627)十一月至次年二月,几个月里,朝廷的公文数量增加了数倍,各地奏疏纷至沓来,堪称
数十年未有之盛况。

这些奏疏字迹相当工整,包装相当精美,内容相当扯淡:上来就痛骂魏忠贤,痛骂阉党,顺便检举某些同事的无耻
行径,最后总结:他们的行为让我很愤怒,跟我不相干。

心中千言万语化为一句话:我不是阉党,皇帝大人,您就把我们当个屁放了吧。

效果很明显,魏忠贤倒台一个月里,崇祯毫无动静,除客氏崔呈秀外,大家过得都还不错。

事实上,当时的朝廷,大学士、六部尚书、都察院乃至于全国各级地方机构,都由阉党掌握,所谓法不责众,大家
都有份,你能把大家都拉下水吗?把我们都抓了,找谁帮你干活?

所以,在阉党同志们看来,该怎么干还怎么干,该怎么活还怎么活。

这个看法在大多数人的身上,是管用的。

而崇祯,属于少数派。

一直以来,崇祯处理问题的理念比较简单,就四个字----斩草除根。所谓法不责众,在他那里是不成问题的,因为
他的祖宗有处理这种问题的经验。

比如朱元璋,胡惟庸案件,报上来同党一万人,杀,两万人,杀杀,三万人,杀杀杀。无非多说几个杀字,不费劲。

时代进步了,社会文明了,道理还一样。

六部尚书是阉党,就撤尚书,侍郎是阉党,就撤侍郎,一半人是阉党,就撤一半,全是,就全撤,大明没了你们就
不转吗?这年头,看门的狗难找,想当官的人有的是,谁怕谁!

\section[\thesection]{}

值得一提的是,虽然上述奏疏内容雷同,但崇祯的态度是很认真的,他不但看了,而且还保存下来。

很简单,真没事的人是不会写这些东西的,原本找不着阉党,照着奏疏抓人,贼准。

十一月底,准备工作就绪,正式动手。最先处理的,是魏忠贤的家属,比如他侄子魏良卿,屁都不懂的蠢人,也封
到公爵了(宁国公),还有客氏的儿子候国兴(锦衣卫都指挥使),统统拉出去剁了。

接下来,是他的亲信太监,毕竟大家生理结构相似,且狼狈为奸,算半亲戚,优先处理。

这拨人总共有四个,分别是司礼监掌印太监王体乾,秉笔太监李永贞、李朝钦、刘若愚。

作为头等罪犯,这四位按说都该杀头,可到最后,却只死了两个,杀了一个。

第一个死的是李朝钦,他是跟着魏忠贤上吊的,并非他杀,算自杀。

唯一被他杀的,是李永贞。其实这位兄弟相当机灵,早在九月底,魏公公尚且得意的时候,他就嗅出了风声,连班
都不上了,开始在家修碉堡,把院子封得严严实实,只留小洞送饭,每天窝在里面,打死也不出头。

坚持到底,就是胜利。

李永贞没有看到胜利的一天,到了十月底,他听说魏忠贤走人了,顿时大喜,就把墙拆了,出来放风。

刚高兴几天,又听到消息,皇帝要收拾魏公公了,慌了,再修碉堡也没用了。于是他使出了绝招----行贿。

当然,行贿崇祯是不管用的,他拿出十余万两银子(以当时市价,合人民币六千万至八千万),送给了崇祯身边的贴
身太监,包括徐应元和王体乾。

这两人都收了。

不久后,他得到消息,徐应元被崇祯免了,而王体乾把他卖了。

在名列死亡名单的这四位死太监中,最神秘的,莫过于王体乾了。

此人是魏忠贤的铁杆,害死王安,迫害东林党,都有他忙碌的身影,是阉党的首脑人物。

但奇怪的是,当我翻阅几百年前那份阉党的最终定罪结果时,却惊奇地发现,以他的丰功劣迹,竟然只排七等(共有
八等),罪名是谄附拥戴,连罚款都没交,就给放了。

伺候崇祯十几年的徐应元,光说了几句话,定罪比他还高(五等),这个看上去很难理解的现象,有一个简单的答
案:王体乾叛变了。

\section[\thesection]{}

据史料分析,王体乾可能很早就``起义''了,所以一直以来,崇祯对魏忠贤的心理活动、斗争策略都了如指掌,当
了这么久卧底,也该歇歇了。

所以他钱照收,状照告,第二天就汇报了崇祯,李永贞得知后,决定逃跑。

跑吧,大明天下,还能跑去非洲不成?

十几天后,他被抓捕归案。

进了号子,李太监还不安分,打算自杀,他很有勇气地自杀了四次,却很蹊跷地四次都没死成,最后还是被拉到刑
场,一刀了断。

名单上最后一位,就是刘若愚了。

这位仁兄,应该是最有死相的,早年加入阉党,一直是心腹,坏事全干过,不是卧底,不是叛徒,坦白交代,主动
退赃之类的法定情节一点没有,不死是不可能的。可他没死。因为刘若愚虽然罪大恶极,但这个人有个特点:能写。

在此之前,阉党的大部分文件,全部出于他手,换句话说,他算是个技术人员,而且他知道很多情况,所以崇祯把
他留了下来,写交代材料。刘太监很敬业,圆满地完成了这个任务,他所写的《酌中志》,成为后代研究魏忠贤的
最重要史料。

只要仔细阅读水浒传,就会发现,梁山好汉们招安后,宋江死了,最能打的李逵死了,最聪明的吴用也死了,活下
来的,大都是身上有门手艺的,比如神医安道全之流。

以上事实清楚地告诉我们,平时学一门技术是多么的重要。

处理完人妖后,接下来的就是人渣了,主要是``五虎''和``五彪''。

五虎是文臣,分别是(排名分先后):兵部尚书崔呈秀、原兵部尚书田吉、工部尚书吴淳夫、太常寺卿倪文焕、副都
御史李燮龙。

五彪是武官,分别是:左都督田尔耕、锦衣卫指挥许显纯、都督同知崔应元、右都督孙云鹤、锦衣卫佥事杨寰。

关于这十个人,就不多说了,其光辉事迹,不胜枚举,比如田尔耕,是迫害``六君子''的主谋,并杀害了左光斗等
人,而许显纯大人,曾亲自把钉子钉进杨涟脑门。用今天的话说,足够枪毙几个来回。

因为此十人一贯为非作歹,民愤极大,崇祯下令,将其逮捕,送交司法部门处理。

经刑部、都察院调查,并详细会审,结果如下:崔呈秀已死,不再追究,其他九人中,田尔耕、许显纯曾参与调查
杨涟、左光斗等人的罪行,结果过失致人死亡,入狱,剩余七人免官为民,就此结案。

\section[\thesection]{}

这份判决只能用一个词来形容----恬不知耻

崇祯很不满意,随即下令,再审。

皇帝表态,不敢怠慢,经过再次认真细致的审讯,重新定罪如下:以上十人,除崔呈秀已死外,田尔耕、许显纯因
为过失致人死亡,判处死缓,关入监狱,其余七人全部充军,充军地点是离其住处最近的卫所。

鉴于有群众反应,以上几人有贪污罪行,为显示威严,震慑罪犯,同时处以大额罚款,分别是倪文焕五千两,吴淳
夫三千两,李燮龙、田吉各一千两。结案。

报上去后,崇祯怒了。

拿钉子钉耳朵,打碎全身肋骨,是过失致人死亡,贪了这么多年,只罚五千、三千,你以为老子好哄是吧。

更奇怪的是,案子都判了,有些当事人根本就没到案,比如田吉,每天还出去遛弯,十分逍遥。

其实案子审成这样,是再正常不过的事了。

审讯此案的,是刑部尚书苏茂相、都察院左都御史曹思诚。

苏茂相是阉党,曹思诚也是阉党

让阉党审阉党,确实难为他了。

愤怒之余,崇祯换人了,他把查处阉党的任务交给了吏部尚书王永光。

可王永光比前两位更逗,命令下来他死都不去,说自己能力有限,无法承担任务。

因为王永光同志虽然不是阉党,也不想得罪阉党。

按苏茂相、曹思诚、王永光以及无数阉党们的想法,形势是很好的,朝廷内外都是阉党,案子没人敢审,对五虎、
五彪的处理,可以慢慢拖,实在不行,就判田尔耕和许显纯死刑,其他的人能放就放,不能放,判个充军也就差不
多了。

没错,司法部长、监察部长、人事部长都不审,那就只有皇帝审了。

几天后,崇祯直接宣布了对五虎五彪的裁定,相比前两次裁决,比较简单:

田吉,杀! 吴淳夫,杀! 倪文焕,杀! 田尔耕,杀! 许显纯,杀! 崔应元,杀! 孙云鹤,杀! 杨寰,杀! 李燮龙,杀!

崔呈秀,已死,挖出来,戳尸!

以上十人,全部抄家!没收全部财产!

什么致人死亡,什么入狱,什么充军,还他娘就近,什么追赃五千两,都去死吧!

曹思诚、苏茂相这帮等阉党本来还有点想法,打算说两句,才发现,原来崇祯还没说完。

``左都御史曹思诚,阉党,免职查办!''

``刑部尚书苏茂相,免职!''

跟我玩,玩死你们!

随即,崇祯下令,由乔允升接任刑部尚书,大学士韩旷、钱龙锡主办此案,务必追查到底,宁可抓错,不可放过。

\section[\thesection]{}

挑出上面这几个人办事,也算煞费苦心,乔允升和阉党向来势不两立,韩旷这种老牌东林党,不往死里整,实在对
不起自己。

扫荡,一个不留!

几天过去,经过清查,内阁上报了阉党名单,共计五十多人,成果极其丰硕。

然而这一次,崇祯先更为愤怒,他当即召集内阁,严厉训斥:人还不够数,老实点!

大臣们都很诧异,都五十多了,还不够吗?

既然皇上说不够,那就再捞几个吧。

第二天,内阁又送上了一份名单,这次是六十几个,该满意了吧。

这次皇帝大人没有废话,一拍桌子:人数不对,再敢糊弄我,以抗旨论处!

崇祯是正确的,内阁的这几位仁兄,确实糊弄了他。

虽然他们跟阉党都有仇,且皇帝支持,但阉党人数太多,毕竟是个得罪人的事,阉党也好,东林党也罢,不过混碗
饭吃,何必呢?

不管了,接着糊弄:``我们是外臣,宫内的人事并不清楚。''

崇祯冷笑:``我看不是不知道,是怕得罪人吧!(特畏任怨耳)''

怪事,崇祯初来乍到,他怎么知道人数不对呢?

崇祯帮他们解开了这个迷题。

他派人抬出了几个包裹,扔到阁臣面前,说: ``看看吧。''

打开包裹的那一刻,大臣们明白,这次赖都赖不掉了。

包裹里的,是无数封跟魏忠贤勾搭的奏疏,很明显,崇祯不但看过,还数过。

混不过去,只能玩命干了。

就这样,自天启七年(1627)十二月,一直到崇祯元年(1628)三月,足足折腾了四个月,阉党终于被彻底整趴下了。

最后的名单,共计二百六十一人,分为八等。

特等奖得主两人,魏忠贤,客氏,罪名:首逆,处理:凌迟。

一等奖得主六人,以崔呈秀为首,罪名:首逆同谋,处理:斩首。

二等奖得主十九人,罪名:结交近侍,处理:秋后处决。

三等奖得主十一人,罪名:结交近侍次等,处理:流放

此外,还有四等奖得主(逆孽军犯)三十五人,五等奖得主(谄附拥戴军犯)十六人,六等奖得主(交结近侍又次等)一
百二十八人,七等奖得主(祠颂)四十四人,各获得充军、有期徒刑、免职等奖励。

以上得奖结果,由大明北京市公证员朱由检同志公证,有效。

\section[\thesection]{}

对此名单,许多史书都颇有微辞,说是人没抓够,放跑了某些阉党,讲这种话的人,脑袋是有问题的。

我算了一下,当时朝廷的编制,六部只有一个部长,两个副部长(兵部有四个),每个部有四个司(刑部和户部有十三
个),每个司司长(郎中)一人,副司长(员外郎)一人,处长(主事)两人。

还有大衙门都察院,加上各地御史,才一百五十人,其余部门人数更少,总共(没算地方政府)大致不会超过八百人。

人就这么多,一下子刨走两百六十多,还不算多?

其实人家也是有苦衷的,毕竟魏公公当政,不说几句好话,是混不过去的,现在换了领导,承认了错误,也就拉倒
了吧。

然而崇祯不肯拉倒,不只他不肯,某些人也不肯。

这个某些人,是指负责定案的人。

大家在朝廷里,平时你来我往,难免有点过节,现在笔在手上,说你是阉党,你就是阉党,大好挖坑机会,不整一
下,难免有点说不过去。

比如大学士韩旷,清查阉党毫不积极,整人倒是毫不含糊,骂过魏公公的,不一定不是阉党,骂过他的,就一定是
阉党,写进去!

更搞笑的是,由于人多文书多,某些兄弟被摆了乌龙,明明当年骂的是张居正,竟然被记成了东林党,两笔下去就
成了阉党,只能认倒霉。

此外,在这份名单上,还有几位有趣的人物。比如那位要在国子监里给魏公公立牌坊的陆万龄同学,屁官都不是,
估计连魏忠贤都没见过,由于风头太大,竟然被订为二等,跟五虎五彪一起,被拉出去砍了。

那位第一个上疏弹劾魏公公的杨维垣,由于举报有功,被定为三等,拉去充军。

而在案中扮演了滑稽角色的陈尔翼、杨所修,也没能跑掉,根据情节,本来没他们什么事,鉴于其双簧演得太过精
彩,由皇帝特批六等奖,判处有期徒刑,免官为民。

复仇

总体说来,这份名单虽然有点问题,但是相当凑合,弘扬了正气,恶整了恶人,虽然没有不冤枉一个好人,也没有
放过大多数坏人,史称``钦定逆案''。

其实崇祯和魏忠贤无仇,办案子,无非是魏公公挡道,皇帝看不顺眼,干掉了。

\section[\thesection]{}

但某些人就不同了。干掉是不够的,死了的人挫骨扬灰,活着的人赶尽杀绝,才算够本!

黄宗羲就是某些人中的优秀代表。

作为``七君子''中黄遵素的长子,黄宗羲可谓天赋异禀,不但精通儒学,还懂得算术、天文。据说天上飞的,地上
跑的,没有他不知道的,被称为三百年来学术之集大成者,与顾炎武、王夫之并称。

更让人无语的是,黄宗羲还懂得经济学,他经过研究发现,每次农业税法调整,无论是两税法还是一条鞭法,无论
动机如何善良,最终都导致税收增加,农民负担加重,换句话说,不管怎么变,最终都是加。

这一原理后被社科院教授秦晖总结,命名为``黄宗羲定律'',中华人民共和国国务院经过调研,采纳这一定律,于
2006年彻底废除了农业税,打破了这个怪圈。

善莫大焉。

但这四个字放在当时的黄宗羲身上,是不大恰当的,因为他既不善良,也不大度。

当时恰好朝廷审讯许显纯,要找人作证,就找来了黄宗羲。

事情就是这么闹起来的。

许显纯此人,说是死有余辜,还真是有余辜,拿锤子砸人的肋骨,用钉子钉人耳朵,钉人的脑袋,六君子、七君
子,大都死在他的手中,为人恶毒,且有心理变态的倾向。

此人向来冷酷无情,没人敢惹,杨涟如此强硬,许先生毫不怯场,敢啃硬骨头,亲自上阵,很有几分硬汉色彩。

但让人失望的是,轮到这位变态硬汉入狱,当场就怂了,立即展现出了只会打人,不会被人打的特长。他全然没有
之前杨涟的骨气,别说拿钉子顶脑门,给他几巴掌,立马就晕,真是窝囊死了。

值得庆幸的是,崇祯的监狱还比较文明,至少比许显纯在的时候文明,打是打,但锤子、钉子之类的东西是不用
的,照此情形,审完后一刀了事,算是便宜了他。但便宜不是那么容易找到的。

审讯开始,先传许显纯,以及同案犯``五彪''之一的崔应元,然后传黄宗羲。

黄宗羲上堂,看见仇人倒不生气,表现得相当平静,回话,作证,整套程序走完,人不走。

大家很奇怪,都看着他。

别急,先不走,好戏刚刚开场。

黄宗羲来的时候,除了他那张作证的嘴外,还带了一件东西----锥子。

\section[\thesection]{}

审讯完毕,他二话不说,操起锥子,就奔许显纯来了。

这一刻,许显纯表现出了难得的单纯,他不知道审案期间拿锥子能有啥用,只是呆呆地看着急奔过来的黄宗羲,等
待着他的答案。

答案是一声惨叫。

黄宗羲终于露出了狰狞面目,手持锥子,疯狂地朝许显纯身上戳,而许显纯也不愧孬种本色,当场求饶,并满地打
滚,开始放声惨叫。

许先生之所以大叫,是有如意算盘的:这里毕竟是刑部大堂,众目睽睽之下,难道你们都能看着他殴打犯人吗?答案
是能。

无论是主审官还是陪审人员,没有一个人动手,也没有人上前阻拦,大家都饶有兴致的看着眼前的这一幕,黄宗羲
不停地扎,许显纯不停地喊,就如同电视剧里最老套的台词:你喊吧,就是喊破喉咙也不会有人来救你!

因为所有人都记得,这个人曾经把钢钉扎进杨涟的耳朵和脑门,那时,没有人阻止他。

但形势开始变化了,许显纯的声音越来越小,鲜血横流,黄宗羲却越扎越起劲,如此下去,许先生被扎死,黄宗羲
是过瘾了,黑锅得大家背。

于是许显纯被拉走,黄宗羲被拉开,他的锥子也被没收。

审完了,仇报了,气出了,该消停了。

黄宗羲却不这么认为,他转头,又奔着崔应元去了。

其实这次审讯,崔应元是陪审,无奈碰上了黄恶棍,虽然没挨锥子,却被一顿拳打脚踢,鼻青脸肿。

到此境地,主审官终于认定,应该把黄宗羲赶走了,就派人上前把他拉开,但黄宗羲打上了瘾,被人拉走之前,竟
然抓住了崔应元的胡子,活生生地拔了下来!

当年在狱中狂施暴行的许显纯,终于尝到了暴行的滋味,等待着他的,是最后的一刀。

什么样的屠夫,最终也只是懦夫。

如许显纯等人,都是钦定名单要死的,而那些没死的,似乎还不如死了的好。

比如阉党骨干,太仆寺少卿曹钦程,好不容易捡了条命,回家养老,结果所到之处,都是口水(民争唾其面),实在
呆不下去,跑到异地他乡买了个房子住,结果被人打听出来,又是一顿猛打,赶走了。

还有老牌阉党顾秉谦,家乡人对他的感情可谓深厚,魏忠贤刚倒台,人民群众就冲进家门,烧光了他家,顾秉谦跑
到外地,没人肯接待他,最后在唾骂声中死去。

\section[\thesection]{}

而那些名单上没有,却又应该死的,也没有逃过去。比如黄宗羲,他痛殴许显纯后,又派人找到了当年杀死他父亲
的两个看守,把他们干掉了。

大明是法制社会,但凡干掉某人,要么有司法部门批准,要么偿命,但黄宗羲自己找人干了这俩看守,似乎也没人
管,真是没王法了。

黄宗羲这么一闹,接下来就热闹了,所谓``六君子''、``七君子'',都是有儿子的。先是魏大中的儿子魏学濂上
书,要为父亲魏大中伸冤,然后是杨涟的儿子杨之易上书,为父亲杨涟伸冤,几天后,周顺昌的儿子周茂兰又上
书,为父亲周顺昌伸冤。

顺便说一句,以上这几位的上书,所用的并非笔墨,而是一种特别的材料----血。

这也是有讲究的,自古以来,但凡奇冤都写血书,不用似乎不够分量。

但崇祯同志就不干了,拿上来都是血迹斑斑的东西,实在有点发怵,随即下令:你们的冤情我都知道,但上奏的文
书是用墨写的,用血写不合规范,今后严禁再写血书。

但他还是讲道理的,崇祯二年(1629)九月,他下令,为殉难的东林党人恢复名誉,追授官职,并加封谥号。

杨涟得到的谥号,是忠烈,以此二字,足以慨其一生。

至此,为祸七年之久的阉党之乱终于落下帷幕,大明有史以来最强大,最邪恶的势力就此倒台。纵使它曾骄横一
时,纵使它曾不可一世。

迟来的正义依然是正义。

在这个世界上,所谓神灵、天命,对魏忠贤而言,都是放屁,在他的身上,只有一样东西----迷信。

不信道德,不信仁义,不信报应,不信邪不胜正。

迷信自己,迷信力量,迷信权威,迷信可以为所欲为,迷信将取得永远的胜利。

而在遍览史书十余载后,我信了,至少信一样东西----天道。

自然界从诞生的那刻起,就有了永恒的规律,春天成长,冬天凋谢,周而复始。

人世间也一样,从它的起始,到它的灭亡,规则恒久不变,是为天道。

在史书中无数的尸山血河、生生死死背后,我看到了它,它始终在那里,静静地注视着我们,无论兴衰更替,无论
岁月流逝。

它告诉我,在这个污秽、混乱、肮脏的世界上,公道和正义终究是存在的。

天道有常,从它的起始,到它的灭亡,恒久不变。

\section[\thesection]{}

复起

崇祯是一个很有想法的人,很想有番作为,但当他真正站在权力的顶峰时,却没有看到风景,只有一片废墟。

史书有云:明之亡,亡于天启。也有史书云:实亡于万历。还有史书云:始亡于嘉靖。

应该说,这几句话都是有道理的,经过他哥哥、他爷爷、他爷爷的爷爷几番折腾,已经差不多了,加上又蹦出来个
九千岁人妖,里外一顿猛捶,大明公司就剩一口气了。

朝廷纷争不断,朝政无人理会,边疆烽火连天,百姓民不聊生,干柴已备,只差一把火。

救火员崇祯登场。

他浇的第一盆水,叫做袁崇焕。

崇祯是很喜欢袁崇焕的,因为他起用袁崇焕的时间,是天启七年 (1627) 十一月十九日。

此时,魏忠贤刚死十三天,尸体都还没烂。

几天后,在老家东莞数星星的袁崇焕接到了复起任职通知,大吃一惊。

吃惊的不是复起,而是职务。

袁崇焕当时的身份是平民,按惯例,复起也得有个级别,先干个主事(处级),过段时间再提,比较合理。

然而他接受的第一个职务,是都察院右都御史,兵部左侍郎。

兵部右侍郎,是兵部副部长,都察院右都御史,是二品正部级,也就是说,在一天之内,布衣袁崇焕就变成了正部
级副部长。

袁部长明显没缓过劲来,在家呆了几个月,啥事都没干,却又等来了第二道任职令。

这一次,他的职务变成了兵部尚书,督师蓟辽。

明代有史以来,最不可思议的任职令诞生了。

因为兵部尚书,督师蓟辽,是一个很大的官,很大。

所谓兵部尚书就是国防部部长,很牛,但最牛的官职,是后四个字----督师蓟辽。

我之前曾经说过,明代的地方官,最大的是布政使、按察使和指挥使,为防互相扯皮,由中央下派特派员统一管
理,即为巡抚。

鉴于后期经营不善,巡抚只管一个地方,也摆不平,就派高级特派员管理巡抚,即为总督。

到了天启崇祯,局势太乱,连总督都搞不定了,就派特级特派员,比总督还大,即为督师。

\section[\thesection]{}

换句话说,督师是明代除皇帝外,管辖地方权力最大的官员。

而要当巡抚、总督、督师的条件,也是不同的。

要当巡抚,至少混到都察院佥都御史(四品正厅级)或是六部侍郎(副部级),才有资格。

而担任总督的,一般都是都察院都御史(二品部级),或是六部尚书(部长)。

明代最高级别的干部,就是部级,所以能当上督师的,只剩下一种人----内阁大学士。

比如之前的孙承宗,后来的杨嗣昌,都是大学士督师。

袁崇焕例外。

就在几个月前,他还只是袁百姓,几月后,他就成了袁尚书,还破格当上了督师,而袁督师的管辖范围包括蓟州、
辽东、登州、天津、莱州等地,换句话说,袁督师手下,有五六个巡抚。

任职令同时告知,立刻启程,赶到京城,皇帝急着见你。

崇祯确实急着见袁崇焕,因为此时的辽东,已经出现了一个更为强大的敌人。

自从被袁崇焕打跑后,皇太极始终很消停,他没有继续用兵,却开始了不同寻常的举动。

皇太极和他老爹不同,从某种角度讲,努尔哈赤算半个野蛮人,打仗,占了地方就杀,不杀的拉回来做奴隶,给贵
族当畜牲使,在后金当官的汉人,只能埋头干活,不能骑马,不能养牲口,活着还好,要是死了,老婆就得没收,
送到贵族家当奴隶。

相比而言,皇太极很文明,他尊重汉族习惯,不乱杀人,讲信用,特别是对汉族前来投奔的官员,那是相当的客
气,还经常赏赐财物。

总而言之,他很温和。

温和文明的皇太极,是一个比野蛮挥刀的努尔哈赤更为可怕的敌人。

张牙舞爪的人,往往是脆弱的,因为真正强大的人,是自信的,自信就会温和,温和就会坚定。

无需暴力,无需杀戮,因为温和,才是最高层次的暴力。

在皇太极的政策指引下,后金领地逐渐安定,经济开始发展稳固,而某些在明朝混不下去的人,也开始跑去讨生
活,这其中最典型的人物,就是范文程。

每次说到这个人,我都要呸一口,呸。

呸完了,接着说。

说起汉奸,全国人民就会马上想起吴三桂,但客观地讲,吴三桂当汉奸还算情况所迫,范文程就不同了,他是自动
前去投奔,出卖自己同胞的,属于汉奸的最原始,最无耻形态。

他原本是个举人(另说是秀才),因为在大明混得不好,就投了皇太极,在此后几十年的汉奸生涯中,他起了极坏的
作用,更讽刺的是,据说他还有个光荣的嫡系祖先----范仲淹。

\section[\thesection]{}

想当年,范仲淹同志在宋朝艰苦奋斗,抗击西夏,如在天有灵,估计是要改家谱的。不过自古以来,爷爷是好汉,
孙子哭着喊着偏要当汉奸的,实在太多,古代有古代的汉奸,现在有现代的汉奸,此所谓汉奸恒久远,遗臭永流传。

在范文程的帮助下,皇太极建立了朝廷(完全仿照明朝),开始组建国家机器,进行奴隶制改造,为进入封建社会而
努力。

要对付这个可怕的敌人,必须立刻采取行动。

在紫禁城里的平台,怀着憧憬和希望,皇帝陛下第一次见到了袁崇焕。

这是一次十分重要的召见,史称平台召对。

他们见面的那一天,是崇祯元年(1628)七月十四日。

顺便说一句,由于本人数学不好,在我以上叙述的所有史实中,日期都是依照原始史料,使用阴历。而如果我没记
错的话,阴历七月十四日,是鬼节。

七月十四,鬼门大开,阴风四起。

那天有没有鬼出来我不知道,但当天的这场谈话,确实比较鬼。

谈话开始,崇祯先客套,狠狠地夸奖袁崇焕,把袁督师说得心潮澎湃,此起彼伏,于是,袁督师激动地说出了下面
的话: ``计五年,全辽可复。''

这句话的意思是,五年时间,我就能恢复辽东,彻底解决皇太极。

这下吹大发了。

百年之后的清朝史官们,在经过时间的磨砺和洗礼后,选出了此时此刻,唯一能够挽救危局的人,并给予了公正的
评价。

但这个人不是袁崇焕,而是孙承宗。

翻阅了上千万字的明代史料后,我认为,这个判断是客观的。

袁崇焕是一个优秀的战术实施者,一个坚定的战斗执行者,但他并不是一个卓越的战略制定者。

而从他此后的表现看,他也不是一个能正确认识自己的人。

所有的悲剧,即由此言而起。

崇祯很兴奋,兴奋得连声夸奖袁崇焕,说你只要给我好好干,我也不吝惜赏赐,旁边大臣也猛添柴火,欢呼雀跃,
气氛如此热烈,以至于皇帝陛下决定,休会。

但脑袋清醒的人还是有的,比如兵科给事中许誉卿。

他抱着学习的态度,找到了袁崇焕,向他讨教如何五年平辽。

照许先生的想法,袁督师的计划应该非常严密。

然而袁崇焕的回答只有四个字:聊慰上意!

\section[\thesection]{}

翻译过来就是,随口说说,安慰皇上的。

差点拿笔做笔记的许誉卿当时就傻了。

他立刻小声(怕旁边人听见)地对袁崇焕说: ``上英明,岂可浪对?异日按期责功,奈何?''

这句话意思是,皇上固然不懂业务,但是比较较真,现在忽悠他,到时候他按日期验收工作,你怎么办?

袁督师的反应,史书上用了四个字:怃然自失。

没事,牛吹过了,就往回拉。

于是,当崇祯第二次出场的时候,袁督师就开始提要求了。

首先是钱粮,要求户部支持,武器装备,要求工部支持。然后是人事,用兵、选将,吏部、兵部不得干涉,全力支
持。最后是言官,我在外打仗,言官唧唧喳喳难免,不要让他们烦我。

以上要求全部得到了满足,立即。

崇祯是个很认真的人,他马上召集六部尚书,开了现场办公会,逐个落实,保证兑现。

会议就此结束,双方各致问候,散伙。

在这场召对中,崇祯是很真诚的,袁崇焕是很不真诚的,因为当时的辽东局势已成定论,后金连衙门都修起来了,
能够守住就算不错,你看崇祯兄才刚二十,又不懂业务,就敢糊弄他,是很不厚道的。

就这样,袁崇焕胸怀五年平辽的口号,在崇祯期望的目光中,走向了辽东。

可他刚走到半路,就有人告诉他,你不用去了,去了也没兵。

就在他被皇帝召见的十天后,宁远发生了兵变。

兵变的原因,是不发工资。

我曾翻阅过明代户部记录,惊奇地发现,明朝的财政制度,是非常奇特的,因为几乎所有的地方政府,竟然都没有
行政拨款。也就是说,地方办公经费,除老少边穷地区外,朝廷是不管的,自己去挣,挣得多就多花,挣得少就少
花,挣不到就滚蛋。

而明朝财政收入的百分之八十,都用在了同一个地方----军费。

什么军饷、粮草、衣物,打赢了有赏钱,打输了有补偿,打死了有安家费,再加上个别不地道的人吃空额,扣奖
金,几乎每年都不够用。

宁远的情况大致如此,由于财政困难,已经连续四个月没有发工资。

要知道,明朝拖欠军饷和拖欠工钱是不一样的,不给工资,最多就去衙门告你,让你吃官司,不给军饷,就让你吃
大刀。

\section[\thesection]{}

最先吃苦头的,是辽东巡抚毕自肃,兵变发生时,他正在衙门审案,还没反应过来,就被绑成了粽子,关进了牢
房,和他一起被抓的,还有宁远总兵朱梅。

抓起来就一件事,要钱,可惜的是,翻遍巡抚衙门,竟然一文钱没有。

其实毕自肃同志,确实是个很自肃的人,为发饷的事情,几次找户部要钱,讽刺的是,户部尚书的名字叫做毕自
严,是他的哥哥,关系铁到这个份上,都没要到钱,可见是真没办法了。

但苦大兵不管这个,干活就得发工钱,不发工钱就干你,毕大人最先遭殃,被打得遍体鳞伤,奄奄一息,关键时刻
部下赶到,说你们把他打死也没用,不如把人留着,我去筹钱。

就这样,兵变弄成了绑票,东拼西凑,找来两万银子,当兵的不干,又要闹事,无奈之下,巡抚衙门主动出面,以
政府做担保,找人借了五万两银子(要算利息),补了部分工资,这才把人弄出来。

毕自肃确实是个好人,出来后没找打他的人,反而跟自己过不去,觉得闹到这个局势,有很大的领导责任,但他实
在太过实诚,为负责任,竟然自杀了。

毕巡抚是个老实人,袁督师就不同了,听说兵变消息,勃然大怒:竟敢闹事,反了你们了!

立刻马不停蹄往地方赶,到了宁远,衙门都不进,直接就奔军营。

此时的军营,已彻底失去控制,军官都不敢进,进去就打,闹得不行,袁崇焕进去了,大家都安静了。

所谓闹事,也是有欺软怕硬这一说的。

袁崇焕首先宣读了皇帝的谕令,让大家散会,回营休息,然后他找到几个心腹,只问了一个问题:``谁带头闹的?''

回答: ``杨正朝,张思顺。''

那就好办了,先抓这两个。

两个人抓来,袁崇焕又只问了一个问题:想死,还是想活。

不过是讨点钱,犯不着跟自己过不去,想活。

想活可以,当叛徒就行。

很快,在两人的帮助下,袁崇焕找到了参与叛乱的其余十几个乱党,对这些人,就没有问题,也没有政策了,全部
杀头。

领头的没有了,自然就不闹了,接下来的,是追究领导责任。

负有直接责任的中军部将吴国琦,杀头,其余相关将领,免职的免职,查办的查办,这其中还包括后来把李自成打
得满世界乱逃的左良玉。

\section[\thesection]{}

兵变就此平息,但问题没有解决,毕竟物质基础决定上层建筑,老不发工资,玉皇大帝也镇不住。

袁崇焕直接找到崇祯,开口就要八十万。

八十万两白银,折合崇祯时期米价,大致是人民币六亿多。

袁崇焕真敢要,崇祯也真敢给,马上批示户部尚书毕自严,照办。

毕自严回复,不办。

崇祯大发雷霆,毕自严雷打不动,说来说去就一句话,没钱。

毕尚书不怕事,也不怕死,他的弟弟死都没能发出军饷,你袁崇焕算老几?

事实确实如此,我查了一下,当时明朝每年的收入,大致是四百万两,而明朝一年的军费,竟然是五百万两!如此下
去,必定破产。

明朝,其实就是公司,公司没钱要破产,明朝没钱就完蛋,而军费的激增,应归功于努尔哈赤父子这十几年的抢掠
带折腾,所谓明亡清兴的必然结局,不过如此。

虽说经济紧张,但崇祯还是满足了袁崇焕的要求,只是打了个折----三十万两。

钱搞定了,接下来是搞人。首先是辽东巡抚,毕巡抚死后,这个位置一直没人坐,袁崇焕说,干脆别派了,撤了这
个职务拉倒。

崇祯同意了。

然后袁崇焕又说,登州、莱州两地(归他管)干脆也不要巡抚了,都撤了吧。

崇祯又同意了。

最后袁崇焕还说,为方便调遣,特推荐三人:赵率教、何可纲、祖大寿(他的铁杆),赵率教为山海关总兵,何可纲
为宁远总兵,原任总兵满桂、麻登云(非铁杆),另行任用。

崇祯还是同意了。

值得一提的是,在请示任用这三个人的时候,袁崇焕曾经说过一句话: ``臣选此三人,愿与此三人共始终,若到期
无果,愿杀此三人,然后自动请死。''

此后的事情证明,这个誓言是比较准的,到期无果,三人互相残杀,他却未能请死。

至此,袁崇焕人也有了,钱也有了,蓟辽之内,已无人可与抗衡。

不,不,还是有一个。

近十年来,历任蓟辽总督,无论是袁应泰、熊廷弼,王化贞,都没有管过他,也管不了他。

``孤处天涯,为国效命,曲直生死,惟君命是从。''

臣左都督,挂将军印领尚方宝剑,总兵皮岛毛文龙泣血上疏。

\section[\thesection]{}

决定

袁崇焕想杀掉毛文龙。这个念头啥时候蹦出来的,实在无法考证,反正不是一天两天了,而杀人动机,只有四个
字:看不顺眼。

当然,也有些人说,袁崇焕要杀掉毛文龙,是要为投敌做准备,其实这个说法并不新鲜,三百多年前袁崇焕快死那
阵,京城里都这么说。

但事实上,这是个相当无聊的讲法,因为根据清朝《满文老档》的记载,毛文龙曾经跟皇太极通过信,说要投敌,
连进攻路线都商量好了,要这么说,袁崇焕还算是为国除害了。

鉴于清朝有乱改史料的习惯,再加上毛文龙一贯的表现,其真实性是值得商榷。

袁崇焕之所以决定干掉毛文龙,只是因为毛文龙不太听话。

毛文龙所在的皮岛,位于后金的后方,要传命令过去,要么穿越敌军阵地,要么坐船,如果不是什么惊天剧变,谁
也不想费这个事。

躲在岛上,长期没人管,交通基本靠走,通讯基本靠吼,想听话也听不了,所以不太听话。

更重要的是,毛文龙在皮岛,还是很有点作用的,他位于后金后方,经常派游击队骚扰皇太极,出来弄他一下,又
不真打,实在比较恶心。被皇太极视为心腹大患。

但这个人也是有问题的,毛总兵驻守皮岛八年,做得最成功的不是军事,而是经济,皮岛也就是个岛,竟然被他做
成了经济开发区,招商引资,无数的客商蜂拥而至,大大小小的走私船都从他那儿过,收钱就放行,他还参股。

打仗倒也真打,每年都去,就是次数少点----六次,大多数时间,是在岛上列队示威,或者派人去后金那边摸个岗
哨,打个闷棍之类。

但总体而言,毛文龙还是不错的,一人孤悬海外,把生意做得这么大,还牵制了皇太极,虽说打仗不太积极,但以
他的兵力,能固守就及格了。

鉴于以上原因,历代总督、巡抚都是睁只眼闭只眼,放他过去了。但袁崇焕是不闭眼的,他的眼里,连粒沙子都不
容。

几年前,当他只是个四品宁前道的时候,就敢不经请示杀副总兵,现在的袁督师手握重权,小小的皮岛总兵算老几?

更恶劣的是,毛文龙有严重的经济问题,八年多账目不清,还从不接受检查,且虚报战功,也不听招呼,实在是罪
大恶极,必须干掉!

其实毛总兵是有苦衷的,说我捞钱,确是事实,那也是没办法,就这么个荒岛,要不弄点钱,谁跟你干?说我虚报战
功,也是事实,但这年头,不打仗的都吹牛,打仗的都虚报,多报点成绩也正常,都照程序走,混个屁啊?

\section[\thesection]{}

我曾查阅明代户部资料及相关史料,毛文龙手下的人数,大致在四万多人左右,按户部拨出的军饷,是铁定不够用
的,换句话说,毛总兵做生意赚的钱,很多都贴进了军饷,很够意思。

可惜对袁崇焕同志而言,这些都没有意义,在这件事上,他是纯粹的对人不对事。

大难即将临头的毛总兵依然天真无邪,直到他得知了那个消息。

崇祯二年(1629)四月,蓟辽督师袁崇焕下令:凡运往东江之物资船队,必须先开到宁远觉华岛,然后再运往东江。

接到命令后,毛文龙当场晕菜,大呼:``此乃拦喉切我一刀,必定立死!''

只是换个地方起运,为什么立死呢?

因为毛总兵的船队是有猫腻的,不但里面夹杂私货,还要顺道带商船上岛,袁督师改道,就是断了他的财路,只能
散伙。

他立即向皇帝上疏,连声诉苦,说自己混不下去了,连哭带吓唬,得到的,却只是皇帝的几个字:从长计议。

从长计议?怎么从长,喝西北风?

在他最困难的时候,一个最不可能帮助他的人帮助了他。

穷得发慌的毛文龙突然收到了十万两军饷,这笔钱是袁崇焕特批的。

拿钱的那一刻,毛文龙终于明白了袁崇焕的用意:拿我的钱,就得听我的话。

也好,先拿着,到时再慢慢谈。

然而袁崇焕的真实用意是:拿我的钱,就要你的命!

说起来,毛文龙算是老江湖了,混了好几十年,还是吃了没文化的亏,要论耍心眼,实在不如袁崇焕。

他做梦也想不到,很久以前,袁督师就打算干掉他。

早在崇祯元年(1628)七月,袁崇焕在京城的时候,曾找到大学士钱龙锡,对他说过这样一句话:``(毛文龙)可用则
用之,不可用则杀之。''

这还不算,杀的方法都想好了: ``入其军,斩其帅!''

后来他给皇帝的奏疏上,也明明白白写着:``去年(崇祯元年)十二月,臣安排已定,文龙有死无生矣!''

\section[\thesection]{}

``安排已定'',那还谈个屁

但谈还是要谈,因为毛总兵手下毕竟还有几万人,占据要地,如果把他咔嚓了,他的部下起来跟自己死磕,那就大
大不妙了。

所以袁崇焕决定,先哄哄他。

他先补发了十万两军饷,然后又在毛总兵最困难的时候,送去了许多粮食和慰问品,并写信致问候。

毛文龙终于上当了,他十分感激,终于离开了皮岛老巢,亲自前往宁远,拜会袁崇焕。

机会来了。

在几万重兵的注视下,毛文龙进入了宁远城。

他拜会了袁崇焕,并受到了热情的接待,双方把酒言欢,然后……

然后他安然无恙地走了。

袁崇焕确实想杀掉毛文龙,但绝不是在宁远。

这个问题,有点脑子的人就能想明白,如果在宁远把他干掉了,他手下那几万人,要么作鸟兽散,要么索性反出去
当土匪,或是投敌,到时这烂摊子怎么收?

所以在临走时,袁崇焕对毛文龙说,过一个月,我要去你的地盘阅兵,到时再叙。

因为解决问题的最好方法,就是在他自己的地盘上干掉他。

崇祯二年(1629)五月二十九日,袁崇焕的船队抵达双岛。

双岛距离皮岛很近,是毛文龙的防区,五月三十日,毛文龙到达双岛,与袁崇焕会面。

六月初一夜晚,袁崇焕来到毛文龙的营房,和他进行了谈话,双方都很客气,互相勉励,表示时局艰难,要共同努
力,度过难关。

这是两人三次谈话中的第一次。

既然在自己的地盘,自然要威风点,毛文龙带来了三千多士兵,在岛上列队,准备迎接袁崇焕的检阅。

六月初三,列队完毕,袁崇焕上岛,开始检阅。

出乎意料的是,毛文龙显得很紧张,几十年的战场经验告诉他,这天可能要出事,所以在整个检阅过程中,他的身
边都站满了拿刀的侍卫。

然而袁崇焕显得很轻松,他的护卫不多,却谈笑自若,搞得毛文龙相当不好意思。

或许是袁崇焕的诚意感动了毛文龙,他赶走了护卫,就在当天深夜,来到了袁督师的营帐,和他谈话。

这是他们三次谈话中的第二次。

第二天,和睦的气氛终于到达了顶点,一整天都在吃吃喝喝中度过,夜晚,好戏终于开场。

\section[\thesection]{}

毛文龙来到袁崇焕的营帐,开始了人生中的最后一次谈话。

一般说来,两人密谈,内容是不会外泄的,好比秦朝赵高和李斯的密谋,要想知道,只能靠猜。

我不在场,也不猜,却知道这次谈话的内容,因为袁崇焕告诉了我。

一个月后,在给皇帝的奏疏中,袁崇焕详细记录了在这个杀戮前的夜晚,他和毛文龙所说的每句话。

袁崇焕说:``你在边疆这么久,实在太劳累了,还是你老家杭州西湖好。''

毛文龙说:``我也这么想,只是奴(指后金)尚在。''

袁崇焕说:``会有人来替你的。''

毛文龙说:``此处谁能代得?''

袁崇焕没有回答这个问题,接着说:``我此来劳军,你手下兵士每人赏银一两,布一匹,米一石,按人头发放。''

毛文龙说:``我这里有三千五百人,明天就去领赏。''

讨论了一些细节问题后,谈话正式结束。

毛文龙的命运就此结束。

他不知道,这个夜晚的这次谈话,是他最后救命的机会,而所有的秘密,就藏在这份看似毫不起眼的记录里。

现在,让我来翻译一下这份记录:

在谈话的开始,袁崇焕说杭州西湖好,解释:毛文龙你回老家吧,只要你把权力乖乖让出来,可以不杀你。

毛文龙说工作任务重,不能走,解释:我在这儿很舒坦,不想走。

袁崇焕说,可以找人替你,解释:这里不是缺了你不行,大把人可以代替你。

毛文龙说,此处谁代得,解释:都是我的人,谁能替我!

这算是谈崩了,接下来的,是袁崇焕的最后一次尝试。袁崇焕说,按人头发放赏赐,解释:把你的家底亮出来,到
底有多少人,老实交代。

毛文龙说,这里的三千五百人,明天领赏,解释:知道你想查我家底,就是不告诉你!

谈不拢,杀吧。

六月五日

袁崇焕在山上设置了大帐,准备在那里召见毛文龙。

然后他走到路边,等待着毛文龙的到来。

毛文龙列队完毕,准备上山。

袁崇焕拦住了他,说,不用这么多人,带上你的亲信将领就行了。

毛文龙表示同意,带着随从跟着袁崇焕上了山。

在上山的路上,袁崇焕突然停住脚步,对着毛文龙身旁的将校们,说了这样一句话:``你们在边疆为国效力,每月
的粮饷只有一斛,实在太辛苦了,请受我一拜!''

\section[\thesection]{}

袁督师如此客气,大家受宠若惊,纷纷回拜,所以,在一片忙乱之中,许多人都没有听懂他的下一句话:``你们只
要为国家效力,今后不用怕无粮饷。''

这句话的意思是,就算你们的毛总兵死了,只要继续干,就有饭吃。

一路走,一路聊,袁崇焕很和气,毛文龙很高兴,气氛很好,直到进入营帐的那一刻。

``毛文龙!本部院与你谈了三日,只道你回头是迟也不迟,哪晓得你狼子野心总是欺诳,目中无本部院,国法岂能容
你!''

面对袁崇焕严厉的训斥,毛文龙却依旧满脸堆笑----还没反应过来。

太突然了,事情怎么能这样发展呢?

袁崇焕到底有备而来,毛总兵脑袋还在运算之中,他就抛出了重量级的武器----十二大罪。

这十二大罪包括钱粮不受管辖、冒功、撒泼无礼、走私、干海盗、好色、给魏忠贤立碑、未能收复辽东土地等等。

这十二大罪的提出,证明袁崇焕同志的挖坑功夫,还差得太远。

类似这种材料公文,骂的是人是鬼不要紧,有没有事实也不要紧,贵在找得准,打得狠,比如杨涟参魏忠贤的二十
四大罪,就是该类型公文的典范。

但袁崇焕给毛文龙栽的这十二条,实在不太高明,所谓冒功、无礼、好色,只要是人就干过,实在摆不上台。而最
有趣的,莫过于给魏忠贤立碑,要知道,当年袁巡抚也干过这出,他曾向朝廷上书,建议在宁远给魏忠贤修生祠,
可惜由于提早下课,没能实现。

这些都是扯淡,其实说来说去就两个字:办你。

文龙兄尚在晕菜之际,袁督师已经派人脱了他的官服,绑起来了。

绑成粽子的毛文龙终于清醒过来,大喊一声:``文龙无罪!''

敢喊这句话,是有底的,毕竟是自己的地盘,几千人就等在外边,且身为一品武官,总镇总兵,除皇帝外,无人敢
杀。

但袁崇焕敢,他敢杀毛文龙,有两个原因

第一个原因:他是袁崇焕,四品文官就敢杀副总兵的袁崇焕。

第二个原因是一件东西,他拿了出来给毛文龙看。

当看到这件东西时,毛文龙终于服软了,这玩意他并不陌生,事实上再熟悉不过了,因为他自己也有一件----尚方
宝剑。

活到头了。

\section[\thesection]{}

虽说文龙兄手里也有一把尚方宝剑,可惜那是天启皇帝给的,所谓尚方宝剑,是皇帝的象征,不是死皇帝的象征,
人都死了,把死人送给你的宝剑拿出来,吓唬鬼还行,跟现任皇帝的剑死磕,只能是找死了。

手持尚方宝剑的袁崇焕,此刻终于说出了他的心声和名言:``你道本部院是个书生,本部院是朝廷的将首!''

毛文龙明白,今天这关不低头是过不去了,马上开始装孙子:``文龙自知死罪,只求督师恩赦。''统帅认怂了,属
下自然不凑热闹,毛文龙的部将毫无反抗,当即跪倒求饶,只求别把自己搭进去。

其实事情到此为止,教训教训毛文龙,也就凑合了。

然而袁崇焕很执着。

局势尽在掌握,胜利就在眼前,这一切的一切冲昏了他的头脑,让他说出了下面的话:``今日杀了毛文龙,本督师
若不能恢复全辽,愿试尚方宝剑偿命!''

这话很准。

然后他面向京城的方向请旨跪拜,将毛文龙拉出营帐,斩首。

辽东的重量级风云人物毛文龙,就此结束了他传奇的一生。

可惜毛总兵并不知道,他是可以不死的,因为袁崇焕根本就杀不了他,只要他向袁崇焕索要一样东西。

这件东西,就是皇帝的旨意。

在古往今来的戏台、电视剧里,尚方宝剑都是个很牛的东西,扛着到处走,想杀谁就杀谁。

这种观点,基本上是京剧票友的水平。别的朝代且不说,在明朝,所谓尚方宝剑,说起来是代天子执法,但大多数
时,也就做个样子,表示皇帝信任我,给我这么个东西,可以狐假虎威一下,算是特别赏赐。

一般情况下,真凭这玩意去砍人的,是少之又少,最多就是砍点中低级别的阿猫阿狗,敢杀朝廷一品大员的,也只
有袁崇焕这种二杆子。

换句话说,袁崇焕要干掉毛文龙,必须有皇帝的旨意,问题在于,毛文龙同志当官多年,肯定也知道这一点,他为
什么不提出来呢?

对于这个疑问,我曾百思不得其解,经过仔细分析材料,我才发现,原来毛文龙同志之所以认栽,只是出于一个偶
然的误会:

因为当袁崇焕拿出尚方宝剑,威胁要杀掉毛文龙的时候,曾说过这样一句话,正是这句话,断送了毛文龙的所有期
望。

\section[\thesection]{}

他说:我五年平辽,全凭法度,今天不杀你,如何惩戒后人?皇上给我尚方宝剑,就是为此!

这是句相当忽悠人的话,特别是最后一句,皇上给我尚方宝剑,就是为此。

为此----到底为什么?

所谓为此,就是为了维护纪律,也就是客气客气的话,没有特指,因为皇帝并未下令,用此剑杀死毛文龙。

但在毛文龙听来,为此,就是皇帝发话,让袁同志拿着家伙,今天上岛来砍自己,所以他没有反抗。

换句话说,毛文龙同志之所以束手待毙,是因为他的语法没学好,没搞清主谓宾的指代关系,弄错了行情。

从小混社会,有丰富江湖经验的毛总兵就这么被稀里糊涂地干掉了。这就是小时候不好好读书的恶果。

人干掉了,接下来的是擦屁股程序。

首先是安慰大家,我只杀毛文龙,首恶必办,胁从不问。然后是发钱,袁崇焕随身带着十万两(约六千多万人民
币),全都发了,只是这种先杀人,再分钱的方式,实在太像强盗打劫。

而最后,也最重要的一步,是安抚。

毛文龙手下这几万人,基本都是他的亲信,要保证这些人不跑,也不散伙,袁崇焕很是花了一番心思,先是换了一
批将领,安插自己的亲信,然后又任命毛文龙的儿子毛承禄当部将,这意思是,我虽然杀了你爹,但那是公事,跟
你没有关系,照用你,别再闹事。

几大棒加胡萝卜下去,效果很好,没人闹,也没人反,该干啥还干啥,袁崇焕很高兴。

毛文龙就这么死了,似乎什么都没有改变。

但后果是有的,且非常非常非常非常严重。

最高兴的是皇太极,他可以放心了,因为毛文龙所控制的区域,除皮岛外,还有金州、旅顺等地区,而毛总兵人品
虽不咋样,但才能出众,此人一死,这些地盘就算没人管了,他可以放心大胆地进攻京城。

而自信的袁督师认定,他的善后工作非常出色。但他不知道的是,在那群被他安抚的毛文龙部下里,有这样三个
人,他们的名字分别是尚可喜、耿仲明、孔有德。

这三位仁兄就不用多介绍了,都是各类``辫子戏''里的老熟人了,前两位先是造反,折腾明朝,后来又跟着吴三桂
造反,折腾清朝,史称``三藩''。

\section[\thesection]{}

而最后这位孔有德更是个极品,他是清朝仅有的两名汉人封王者之一(另一个是吴三桂)。现在北京有个地名叫公主
坟,据说里面埋的就是孔有德的女儿。

当汉奸能当出这么大成就,实在是因为他的汉奸当得非常彻底。后来镇守桂林时,遇到了明末第一名将李定国,被
打得满地找牙,气不过,竟然自焚了。清朝认为这兄弟很够意思,就追认了个王。

这三位仁兄原先都是山东的矿工,觉得挣钱没够,就改行当了海盗,后来转正成了毛文龙的部将。事实证明,这三
个人只有毛文龙能镇住,因为两年后,他们就都反了。

事实还证明,他们是很有点水平的,后来当汉奸时很能打仗,为大清的统一事业做出了卓越贡献。再提一句,那位
被袁督师提拔的毛文龙之子毛承禄后来也反了,不过运气差点,没当上汉奸,就被剁了。

所谓文龙该死,结果大致如此。

但跟上述结果相比,下面这个才是最为致命的。

到底是朝廷里混过的,杀死毛文龙后,袁崇焕立刻意识到,这事办大了。

所以他立即上书,向皇帝请罪,说这事我办错了,以我的权力,不应该杀死毛文龙,请追究我的责任,等待皇帝处
分。

袁崇焕认识错误的态度很诚恳,方法却不对。如果要追究责任,处分、撤职、充军都是不够的,唯一能够摆平此事
的方法,就是杀人偿命。

杀人的必备程序

在明朝,杀一个人很难吗?

答案是不难,拍黑砖、打闷棍、路上遇到劫道的,手脚利落的,也就一根烟功夫。

但要合法地杀掉一个人,很难。

因为大明是法制社会,彻头彻尾的法制社会。

这绝不是开玩笑,只要熟读以下攻略,就算你在明朝犯了死罪,要想不死,也是可能的。

比如你在明朝犯了法(杀了人),就要定罪,运气要是不好,定了个死罪,就要杀头。

但暂时别慌,只要你没干造反之类的特种行当,不会马上被推出去杀掉,一般都是秋后处决。有人会问,秋后处决
不一样是处决吗?不过是多活两天而已。

确实是多活了,但只要你方式得当,就不只是多活两天,事实上,据记载,最高记录是二十多年。

之所以出现这种奇怪的现象,是因为要处决一个人,必须经过复核,而在明朝,复核的人不是地方政府,也不是最
高法院大理寺,甚至不是刑部部长。

唯一拥有复核权的人,是皇帝。

\section[\thesection]{}

这句话的意思是,无论你在哪里犯罪,市区、县城乃至边远山区,无论你犯的是什么罪,杀人、放火或是砸人家窗
户,且无论你是张三、李四、王二麻子,还是王侯将相,只要你犯了死罪,除特殊情况外,都得层层报批,县城报
省城,省城报刑部,刑部报皇帝,皇帝批准,才能把你干掉。

自古以来,人命关天。

批准的方式是打勾,每年刑部的官员,会把判刑定罪的人写成名单,让皇帝去勾,勾一个杀一个。

但问题是,如果你的名字在名单上,无非也就让皇帝大人受累勾一笔,秋后就拉出去砍了,怎么可能活二十多年不
死呢?

不死攻略一:

死缓二十多年的奇迹,起源于皇帝大人的某种独特习惯,要知道,皇帝大人在勾人的时候,并不是全勾,每张纸
上,他只勾一部分,经常会留几个。

此即所谓君临天下,慈悲为怀,皇帝大人是神龙转世,犯不着跟你们平头百姓计较,少杀几个没关系。

但要把你的性命寄托在皇帝大人打勾上,实在太悬,万一那天他心情欠佳,全勾了,你也没辙。

所以要保证活下来,我们必须另想办法。

不死攻略二:

相对而言,攻略二的生存机率要高得多。当然,成本也高得多。

攻略二同样起源于皇帝大人的某种习惯--日理万机。

要打通攻略二,靠运气是没戏的。你必须买通一个人,但这个人不是地方官员(能买通早就买了),也不是刑部(人太
多,你买不起),更加不是皇帝(你试试看)。

而是太监。

皇帝大人从来不清理办公桌,也不整理公文的,每次死刑名单送上来,都是往桌上一放,打完勾再换一张,毕竟我
国幅员辽阔,犯罪分子一点不缺,动不动几十张勾决名单,今天勾不完,放在桌上等着明天批。

但是皇帝们绝不会想到,明天勾的那张名单,并不是今天眼前的这张。

玄机就在这里,既然皇帝只管打勾,名字太多,又记不住,索性就把下面名单挪到上面去,让没出钱的难兄难弟们
先死,等过段时间,看着关系户的那张名单又上来了,就再往下放,周而复始,皇帝不批,就不能杀,就在牢里住
着,反正管吃管住,每年全家人进牢过个年,吃顿团圆饭,不亦乐乎。

\section[\thesection]{}

而能干这件事的,只有皇帝身边的太监,而且这事没啥风险,也就是把公文换个位置,又没拿走,皇帝发现也没话
说。

但这件事也不容易。因为能翻皇帝公文的,大都是司礼监,能混到司礼监的,都不是凡人,很难攀上关系,且收费
也很贵,就算买通了,万一哪天他忘了,或是下去了,该杀还是得杀。

无论费多大功夫,能保住命,还是值得的。

不过需要说明的是,以上攻略不适用于某些特殊人物,比如崇祯,工作干劲极大,喜欢打勾,一勾全勾完,且记性
极好,又比较讨厌太监,遇到这种皇帝,就别再指望了。

综上所述,在明代,要合法干掉一个人,是很难的。

之所以说这么多,得出这个结论,只是要告诉你,袁崇焕的行为,有多么严重。

杀个老百姓,都要皇帝复核,握有重兵,关系国家安危的一品武官毛文龙,就这么被袁崇焕杀了,连个报告都没有。

仅此一条,即可处死袁崇焕。

更重要的是,此时已有传言,说袁崇焕杀死毛文龙,是与皇太极配合投敌,因为他做了皇太极想做而做不到的事。

这种说法是比较扯的,整个辽东都在袁崇焕的手中,他要投敌,打开关宁防线就行,毛文龙只能在岛上看着。

事情闹到这步,只能说他实在太有个性了。

在朝廷里,太有个性的人注定是混不长的。

但袁崇焕做梦也没想到,他等来的,却是一份嘉奖。

崇祯二年(1629)六月十八日,崇祯下令,痛斥毛文龙专横跋扈,目无军法,称赞袁崇焕处理及时,没有防卫过当,
加以奖励。

这份旨意说明了崇祯对袁崇焕的完全推崇和信任,以及对毛文龙的完全唾弃。

他是这样说的,不是这样想的。

按照史料的说法,听说此事后,崇祯``惊惶不已''。

惊惶是肯定的,好不容易找了个人收拾残局,结果这人一上来,啥都没整,就先干掉了帮自己撑了八年的毛总兵,
脑袋进水了不成?

但崇祯同志不愧为政治家,关键时刻义无反顾地装了孙子:人你杀了,就是骂你,他也活不了,索性骂他几句,说
他死得该再吐上几口唾沫,没问题。

袁崇焕非常高兴,杀人还杀出好了,很是欢欣鼓舞了几天,但他并不清楚,他可以越权,可以妄为,却必须满足一
个条件。

\section[\thesection]{}

这个条件的名字,叫做办事。

要当督师,可以,要取消巡抚,可以,辽东你说了算,可以,杀掉毛文龙,也可以,但前提条件是,你得办事,五
年平辽,只要平了,什么都好办,平不了嘛,就办你。

袁崇焕很清楚这点,但毕竟还有五年,鬼知道五年后什么样,慢慢来。

但两个月后,一个人的一次举动,彻底改变了他的命运,顺便说一句,这人不是故意的。

崇祯二年(1629)十月,皇太极准备进攻。

虽然之前曾被袁崇焕暴打一顿,狼狈而归,但现实是严峻的,上次抢回来的东西,都用得差不多,又没有再生产能
力,不抢不行啊。

可问题是,关宁防线实在太硬,连他爹算在内,都去了两次了,连块砖头都没能敲回来。

皇太极进攻的消息,袁崇焕听到过风声,一点不慌。

北京,背靠太行山脉和燕山山脉,通往辽东的唯一大道就是山海关,把这道口子一堵,鬼都进不来,所以袁崇焕很
安心。

关卡是死的,人是活的。

冥思苦想的皇太极终于想出了通过关宁防线的唯一方法----不通过关宁防线。

中国这么大,不一定非要从辽东去,飞不了,却可以绕路。辽东没法走,那就绕吧,绕到蒙古,从那儿进去,没辙
了吧。

就这样,皇太极率十万军队(包括蒙古部落),发动了这次决定袁崇焕命运的进攻。

这是一次载入军事史册的突袭,皇太极充分展现了他的军事才华,率军以不怕跑路的精神,跑了半个多月,从辽东
跑到辽西,再到蒙古。

蒙古边界没有坚城,没有大炮,皇太极十分轻松地跨过长城,在地图上画个半圆后,于十月底到达明朝重镇遵化。

遵化位于北京西北面,距离仅两百多公里,一旦失守,北京将无险可守。

袁崇焕终于清醒了,但大错已经酿成,当务之急,是派人挡住皇太极。

估计是欺负皇太极上了瘾,袁崇焕没有亲自上阵,他把这个光荣的任务交给了赵率教。

皇太极同志带了十万人,全部家当,以极为认真的态度来抢东西,竟然只派个手下,率这么点人(估计不到一万)来
挡,太瞧不起人了。

\section[\thesection]{}

赵率教不愧名将之名,得令后率军连赶三天三夜,于十一月三日到达遵化,很不容易。

十一月四日,出去打了一仗,死了。

对于赵率教的死,许多史料上说,他是被冷箭射死,部下由于失去指挥,导致崩溃,全军覆没。

但我认为赵率教死不死,不是概率问题,是个时间问题,就那么点人,要对抗十万大军,就算手下全变成赵率教,
估计也挡不住。

赵率教阵亡,十一月五日,遵化失陷。

占领遵化后,后金军按照惯例,搞了次屠城,火光冲天,鬼哭狼嚎,再讲一下,不知是为了留个纪念,还是觉得风
水好,清军入关后,把遵化当成了清朝皇帝的坟地,包括所谓``千古一帝''的康熙、乾隆以及``名垂青史''的慈禧
太后,都埋在这里。

几具有名的尸体躺在无数具无名的尸体上,所谓之霸业,如此而已。

最后说几句,到了民国时期,土匪出身的孙殿英又跑到遵化,挖了清朝的祖坟,据说把乾隆、慈禧等一干伟大人物
的尸体乱踩一通,着实是死不瞑目。当然,由于此事干得不地道,除个别人(冯玉祥)说他是革命行为外,大家都
骂,又当然,骂归骂,从坟里掏出来的宝贝,什么乾隆的宝剑,慈禧的玉枕头(据说是宋美龄拿了),还是收归收。

几百年折腾来,折腾去,也就那么回事。

但遵化怎么样,对当时的袁崇焕而言,已经不重要了。

十一月五日,得知消息的袁督师明白,必须出马了。随即亲率大军,前去迎战皇太极。

十一月十日,当他到达京城近郊,刚松口气的时候,却得知了一个意外的消息。

原任兵部尚书王洽被捕了,而接替的他的人,是孙承宗。

王洽刚上任不久就下台,实在是运气太差,突然遇上这么一出,打也打不过,守也守不住,只好撤职,一般说来,
老板开除员工,也就罢了,但崇祯老板比较牛,撤职之后又把他给砍了。

关键时刻,崇祯决定,请孙承宗出马,任内阁大学士、兵部尚书。

在这场史称``己巳之变''的战争中,这是崇祯做出的最英明,也是唯一英明的决定。

此时的袁崇焕已经到达遵化附近的蓟州,等待着皇太极的到来,因为根据后金军之前的动向看,这里将是他的下一
个目标。

这是个错误的判断。

\section[\thesection]{}

皇太极绕开蓟州,继续朝京城挺进。

情况万分紧急,因为从种种迹象看,他的最终目的就是京城。

但袁崇焕不这么看,他始终认为,皇太极就是个抢劫的,兜圈子也好,绕路也罢,抢一把就走,京城并无危险。

其实孙承宗也这样认为,但毕竟是十万人的抢劫团伙,所以他立即下令,袁崇焕应立即率部,赶到京郊昌平、三河
一带布防,阻击皇太极。

到此为止,事情都很正常。

接下来发生的一切,都很不正常。

袁崇焕知道了孙承宗的部署,却并未执行,当年的学生,今天的袁督师,已无需服从老师的意见。

他召集军队,开始了一种极为诡异的行动方式。

十一月十一日,袁崇焕率军对皇太极发动追击。

说错了,是只追不击。

皇太极绕过蓟州,开始北京近郊旅游,三河、香河、顺义一路过去,所到之处都抢劫留念。袁崇焕一直跟着他,抢
到哪里就跟到哪里。

就这样,袁崇焕几万人,皇太极十万人,共十多万人在北京周围转悠,从十一日到十五日,五天一仗没打。

袁崇焕在这五天里的表现,是有争议的,争议了几百年,到今天都没消停。

争议的核心只有一个:他到底想干什么?

大敌当前,既不全力进攻,也不部署防守,为什么?

当时人民群众的看法比较一致:袁崇焕是叛徒。

不攻也不守,跟着人家兜圈子,不是叛徒是什么?

更重要的是,皇太极在这五天里没闲着,四处抢劫,抢了又没人做主,郊区居民异常愤怒,都骂袁崇焕。

朝廷的许多高级官员也很愤怒,也骂袁崇焕,因为他们也被抢了(北京城土地紧张,园林别墅都在郊区)

民不聊生,官也不聊生,叛徒的名头算是背定了。

所以每当翻阅这段史料时,我总会寻找一样东西----动机。

叛徒是不对的,要叛变不用等到今天,他手下的关宁军是战斗力最强的部队,将领全都是他的人,只要学习吴三桂
同志,把关一交,事情就算结了。

失误也不对,凭他的智商和水平,跟着敌人兜圈之类的蠢事,也还干不出来。

所以我很费解,费解他的举动为何如此奇怪,直到我想起了在这三年前他对熊廷弼说过的四个字,才终于恍然大悟。

``主守,后战。''

\section[\thesection]{}

致命漏洞

袁崇焕很清楚,以战斗力而言,如果与后金军野战,就算是最精锐的关宁铁骑,也只能略占上风,要想彻底击败皇
太极,必须用老方法:凭坚城,用大炮。

而这里,唯一的坚城,就是北京。

为实现这一战略构想,必须故意示弱,引诱皇太极前往北京,然后以京城为依托,发动反击。

鉴于袁崇焕同志已经死了,也没时间告诉我他的想法,但事情的发展印证了这一切。

十一月十六日,当皇太极终于掉头,冲向北京时,袁崇焕当即下令,向北京进发。

袁崇焕坚信,到达京城之时,即是胜利到来之日。

但事实上,命令下发的那天,他的死期已然注定。

因为在计划中,他忽视了一个十分不起眼,却又至关重要的漏洞。

一直以来,袁崇焕的固定战法都是坚守城池,杀伤敌军,待敌疲惫再奋勇出击,从宁远到锦州,屡试不爽。

所以这次也一样,将敌军引至城下,诱其攻坚,待其受挫后,全力进攻,可获全胜。

很完美,很高明,如此完美高明的计划,大明最伟大的战略家,城里的孙承宗先生竟然没想到。

孙承宗想到了。

他坚持在北京外围迎敌,不想诱敌深入,不想大获全胜,并不是他愚蠢,而是因为他不但知道袁崇焕的计划,还知
道这个计划的致命漏洞。

这个漏洞,可以用五个字来概括:这里是北京。

无论理论还是实战,这个计划都无懈可击,之前宁远的胜利已经证明,它是行得通的。

但是这一次,它注定会失败,因为这里是北京。

宁远也好,锦州也罢,都是小城市,里面当兵的比老百姓还多,且位居前线,都是袁督师说了算,让守就守,让撤
就撤,不用讨论,不用测评。

但在京城里,说话算数的人只有一个,且绝不会是袁崇焕。

袁督师这辈子什么都懂,就是不懂政治。皇上坐在京城里,看着敌军跑来跑去,就在眼皮子底下转悠,觉都睡不
好,把你叫来护驾,结果你也跑来跑去,就是不动手,把皇帝当猴耍,现在连招呼也没打,就突然冲到北京城下,
到底想干什么?!

洞悉这一切的人,只有孙承宗。

所以谦虚的老师设置了那个无比保守,却也是唯一可行的计划。

骄傲的学生拒绝了这个计划,他认为,自己已经超越了所有的人。

\section[\thesection]{}

就在袁崇焕率军到达北京的那一天,孙承宗派出了使者。

这位使者前往袁崇焕的军营,只说了一段话:皇上十分赏识你,我也相信你的忠诚,但是你杀掉了毛文龙,现在又
把军队驻扎在城外,很多人都怀疑你,希望你尽力为国效力,若有差错,后果不堪设想。

虽然在史料上,这段话是使者说的,但很明显,这是一个老师,对他学生的最后告诫。

孙承宗的判断一如既往,很准。

袁崇焕到北京的那一天,是十一月十七日,很巧,他刚到不久,另一个人就到了----皇太极。

跳进黄河都洗不清了。

我曾查过当时的布阵方位,皇太极的军队在北城,而袁崇焕在南城的广渠门,虽说比较远,但你刚来,人家就到,
实在太像带路的,要人民群众不怀疑你,实在很难。

更重要的是,明朝有规定,边防军队,未经皇帝允许,不得驻扎于北京城下。但袁崇焕同志实在很有想法,谁都没
请示,就到了南城。

到这份上,如果还不怀疑袁崇焕,就不算正常了。

京城里大多数人很正常,所以上到朝廷,下到卖菜的,全都认定,袁崇焕有问题。

唯一不正常的,是崇祯。他没有骂袁崇焕,只是下令袁崇焕进城,他要亲自召见。

召见的地点是平台,一年前,袁崇焕在这里,得到了一切。现在,他将在这里,失去一切。

其实袁崇焕本人是有思想准备的,一年过去,寸土未复不说,还让皇太极打到了城下,实在有点说不过去,皇帝召
见,大事不妙。

如果是叛徒,是不会去的,然而他不是叛徒,所以他去了。跟他一起进去的,还有三个人,分别是总兵满桂、黑云
龙、祖大寿。

祖大寿是袁崇焕的心腹,而满桂跟袁崇焕有矛盾,黑云龙是他的部下。

此前我曾一度纳闷,见袁崇焕,为什么要拉这三个人进去,后来才明白,其中大有奥妙。

袁崇焕的政治感觉相当好,预感今天要挨整,所以进去时脱掉了官服,穿着布衣,戴黑帽子以示低调。

然而接下来的事情却是他做梦都想不到的。

崇祯没有发火,没有训斥,只是做了一个动作:

他解下了自己身上的大衣,披到了袁崇焕的身上。

袁督师目瞪口呆。

\section[\thesection]{}

一年多啥也没干,敌人都打到城下了,竟然还这么客气,实在太够意思了。

在以往众多的史料中,对崇祯同志都有个统一的评价:急躁。然而这件事情充分证明,崇祯,是一个成熟、卓越的
政治家。

一年前开会,要钱给钱,要粮给粮,看谁顺眼就提谁(比如祖大寿),看谁不顺眼就换谁(比如满桂),无所谓,只要
把活干好。

一年了,寸土未复,干掉了牵制后金的毛文龙,皇太极来了,也不玩命打,跟他在城边兜圈子,严重违反治安规
定,擅自带兵进驻城下,还是那句话,你到底想干什么?

在这种情况下,只要是个人,就要解决袁崇焕了。

崇祯不是人,他是皇帝,一个有着非凡忍耐力,和政治判断的皇帝。

以他的脾气,换在以往,早就把袁崇焕给剁了,现在情况紧急,必须装孙子。所以自打袁崇焕进来,他一直都很客
气,除了脱衣服,就是说好话,你如何辛苦,如何忠心,我如何高兴等。

其实千言万语就一句话:你的工作干得很不好,我很不高兴,但是现在不能收拾你。

到这个份上,还能如此克制,实在难得,如果要给崇祯同志的表现打分的话,应该是十分。

而袁崇焕同志之后的表现,应该是负分。

说的事情没有做到,做的事情不应该做,又让皇帝大人吃那么多苦头,却得到了这样的嘉奖,袁崇焕受宠若惊。

所谓受宠若惊,是受宠后自己吃惊,他接下来的举动,却让别人吃惊。

在感谢皇帝大人的恩典后,袁崇焕开始了一场让无数人匪夷所思许多年的演说:

他首先描述了敌情,按照他的说法,敌军异常强大,且倾尽全力,准备拿下北京,把皇帝陛下赶出去,连继位的日
子都定好了,很难抵挡。这段话是彻头彻尾的胡说,且是故意的胡说,皇帝大人不懂业务,或许还会乱想,袁崇焕
是专业人士,明知皇太极是穷的没办法,才来抢一把的,抢完了人家即回去了,竟然还要蒙领导,实在太不像话了。

但问题的关键在于,为什么?

袁崇焕的这一表现,被当时以及后来的许多人认定,他是跟皇太极勾结的叛徒。

从经济学的观点来看,这是不太可能的。所谓勾结,总得有个理由,换句话说,有个价钱,但问题是,当年皇太极
同志,可是很穷的。

\section[\thesection]{}

要知道,皇太极之所以来抢,是因为家里没钱,没钱,怎么跟人勾结呢?

虽说此前也有李永芳、范文程之类的人前去投奔,但事实上,也都并非什么大人物。比如李永芳,只是个地区总
兵,而且就这么个小人物,努尔哈赤同志都送了一个孙女,一个驸马的(额驸)头衔,还有无数金银财宝,才算把他
套住。

范文程更不用说,大明混不下去,到后金混饭吃的,只是一个举人而已,皇太极都给个大学士,让他当主力参谋。

李永芳投降的时候,是地区副总兵,四品武官,努尔哈赤就搭进去一个孙女,按照这个标准,如果要买通明代最大
地方官,总管辽东、天津、登州、莱州、蓟州五个巡抚的袁崇焕,估计他就算把女儿、孙女全部打包送过去,估计
也是白搭。

至于分地盘,就更不用说了,皇太极手里的地方,也就那么大,要分都拿不出手,谁跟你干?

当然,如果你非要较真,说他们俩一见如故,不要钱和地盘,老子也豁出去跟你干,我也没办法。

所以从经济学的角度讲,只要袁崇焕智商正常,是不会当叛徒的。

他糊弄皇帝的唯一原因,是两个字----心虚。

没法不心虚,跟皇帝吹了牛,说五年平辽,不到一年,人家就带兵来平你了。之前干掉了毛总兵,在北京城下又跟
人兜圈,不经许可冲到城下,这事干得实在太糙。不把敌人说得狠点,不把任务描述得艰巨点,怎么混过去?

可他万万没想到,这一糊弄,就糊弄过了。

皇帝当场傻眼不说,大臣们都吓得不行,户部尚书毕自严的舌头伸了出来,半天都没收回去。客观地讲,袁督师干
了一件相当缺德的事,但精彩的表演还没完,等大家惊讶完后,他又说了这样一句话。

我始终认为,这句话让他最终送了命。 ``我的士兵连日征战,希望能够进城修整。''

这孩子没救了。

在明朝,边防军队未经许可进驻城下,基本就算造反,竟然还要兵马入城休息,实在太嚣张了。

当然,这个要求是有前科的。之前不久,满桂在城外与后金军大战,中途曾经进入德胜门瓮城休息,按袁崇焕的想
法,他的地位比满桂高,满桂能进瓮城,他也能进。

举动如此可疑,大家本来就猜忌你,还要带兵入城,辽东人参吃多了。所以崇祯立即做出了答复:不行。

\section[\thesection]{}

袁督师倒也不依不饶:那我自己进城。

答复:不行。

会议就此结束。

这一天是崇祯二年(1629)十一月二十三日,根据种种迹象显示,崇祯判定,袁崇焕不可再用。

但除掉此人,还需要时间,至少七天。

幕后人物

袁崇焕的宿命已经注定。

但他的悲剧,不在于他最后被杀,而是他直到被杀,也不知道为什么。

事实上,致他于死地的那几条罪状里,有一条是很滑稽的。这条滑稽的罪状,来源于三天前的一次偶然事件。三天
前,是十一月二十日。

在这一天,皇太极率军发动了进攻。

这是自于谦保卫战后,京城发生的最大规模的战斗,皇太极以南北对进战术,分别进攻北城的德胜门和南城的广渠
门。

为保证不白来,皇太极下了血本,北路军五万余人,由他亲率,随同攻击的包括大贝勒代善,济尔哈朗等,而守卫
北城的,是满桂。

南路军也不白给,共四万人,三贝勒莽古尔泰带队,还包括后来辫子戏里的主要角色多尔衮、多铎,守在这里的,
就是袁崇焕。

战斗同时开始。

袁崇焕率所部九千余人,在城外列阵迎敌。

莽古尔泰虽然比较蠢,但算术还是会的,四万对九千,往前冲就是了。但战术还是要讲的,他先率军先冲袁崇焕的
左翼,冲不动,退了。过了一会,又率军冲击明军右翼,还是冲不动,又退了。

估计是自尊心受到了伤害,第三次,他率领全部主力,直接扑袁崇焕。

后果很严重。

袁崇焕带来的,是明军最精锐的部队----关宁铁骑。

而且据某些史料讲,包括祖大寿、吴襄在内的一干猛人,都在这支部队里。几乎就在莽古尔泰冲锋的同时,袁崇焕
发动了反冲锋。

此战无需介绍战术,因为基本没有战术,双方骑兵对冲,谁更能砍,谁就能赢。

战斗过程极其惨烈,四小贝勒之一的阿济格的坐骑被射死,他身中数箭,差点当场完蛋,莽古尔泰本人被击伤。

袁崇焕也很悬,为鼓励士兵,他亲自上阵参加冲锋。据史书记载,他左冲右突如入无人之境,身中数箭,竟然毫发
无伤,有如神助。

同样身中数箭,阿济格被射得奄奄一息,袁督师还能继续奋斗,秘诀在于四个字----``重甲难透''。

\section[\thesection]{}

这四个字的意思是,袁督师身上的盔甲厚,箭射到他身上,一点事都没有。

在关宁铁骑的攻击下,后金军开始败退。但八旗军的战斗力相当强悍,加上莽古尔泰脑子不好用,还有几把力气,
再次集结部队,发动了第二次冲锋。

死磕的力量是很大的,袁督师的中军被冲散,他在乱军之中被人围攻,差点被剁。好在部下反应快,帮他格了几刀
(格之获免),才从鬼门关爬出来。

稳住阵脚后,关宁军开始反击,然后又是你打过来,我打过去,一直折腾了八个钟头,直到晚上六点,莽古尔泰终
于支持不住,败退,没来得及跑的,都被赶进了护城河。

广渠门之战结束,后金累计伤亡一千余人,明军大胜。

南城胜利之际,北城的满桂正在苦苦支撑。

进攻德胜门的军队,包括皇太极的亲军主力,战斗力非常强,满桂先派部将迎战,没一会就被打回来。关键时刻,
满桂同志表现出了高昂的革命斗志,亲自上阵,并指挥城头炮兵开炮支援。

在他的光辉榜样映照下,城下明军勇猛作战,城上明军勇猛开炮,后金军死伤惨重。但不知城头上的哪位仁兄,点
炮的时候太过勇猛,一哆嗦偏了准头,一炮直奔满桂同志,当场就把他撂倒,遍体负伤,好在捡了条命,被人护着
回去养伤了。

主帅虽然撤走,但在大炮的掩护下,明军依然奋战不已,付出重大伤亡后,皇太极被迫撤退,德胜门之战就此结束。

这一天对袁崇焕而言,是很光荣的,他凭借自己的精兵良将,在京城打败了实力强劲的八旗军。

更重要的是,同一天出战的满桂,是他的死敌,当着皇帝的面,一个打出去,一个抬回来,实在很有面子。

可是他想不到,满桂同志的这笔帐,最终会算到他的身上,因为在那天战役结束时,一个流言开始在京城流传:开
炮打伤满桂的,就是袁崇焕。

这个说法是不可信的,因为满桂在德胜门作战,而袁崇焕在广渠门,今天在北京,要跑个来回,估计都要一个钟
头,无论如何,袁崇焕都是过不去的。

\section[\thesection]{}

但袁督师背这个黑锅,也不是全无道理,他跟满桂从宁远就开始干仗,后来硬把满总兵挤回关内,从来就不待见这
人,现在满桂受伤了,算在他头上也不奇怪。

从毛文龙开始,到满桂,再到崇祯,袁崇焕一步步将自己逼入绝境,虽然他自己并不知晓。

袁崇焕,广西藤县人,自``蛮夷之地''而起,奋发读书,然资质平平,四次落第,以三甲侥幸登科,后赴辽东,得
孙承宗赏识,于辽东溃败之时,以独军守孤城,屹然不倒,先后击溃努尔哈赤、皇太极父子,护卫辽东。

后受阉党所迫离职,蒙崇祯器重再起,然性格跋扈,调离满桂,安插亲信,以尚方宝剑杀毛文龙,奉调守京,不顾
大局,擅自驻防于城下,致京郊怨声四起,后不惜性命,与皇太极苦战,大破敌军,不顾生死,身先士卒。

我想,差不多了。

最终命运揭晓之前,袁崇焕的表现大致如此。

他并不是一个天赋异禀的人,经过努力和奋斗,还有难得的机遇(比如孙承宗),才最终站上历史的舞台。

他并不完美,不守规章,不讲原则,想怎么干就怎么干,私心很重,听话的就提,不听话的就整(或杀)。

而某些所谓``专家''的所谓``力挽狂澜'',基本就是扯淡。关于这个问题,我曾在社科院明史学会的例会上,跟明
史专家讨论过多次。客观地讲,以他的战略眼光(跟着皇太极绕京城跑圈)和实际表现(擅杀毛文龙),守城出战确属
上乘,让他继续镇守辽东,还能闹出什么事来也难说,所谓挽救危局,随便讲几句吧。

袁崇焕绝不是叛徒,也绝不是一个关键性人物,他存在与否,并不能决定明朝的兴衰成败。换句话说,以他的才
能,无论怎么折腾,该怎么样还怎么样。

对于这个悲剧性的结论,我不知道袁崇焕是否知道,他的一生丰富多彩,困守孤城,决死拼杀、遭人排挤、纵横驰
骋、身处绝境,人家遇不上的事,他大都遇上了。但无论何时、何地,得意、失意,他一直在努力,他坚信,自己
的努力终将改变一切。

他始终没有放弃过

崇祯二年(1629)十一月二十七日,京城九门换防,一切准备就绪。

最终的结局已经注定,无需改变,也无法改变。

就在这天,坚定的袁崇焕开始了自己人生中的最后一战----左安门之战。

\section[\thesection]{}

袁崇焕列队于城外。

因为不能入城,只能背城布阵。背对着冰冷的墙砖,在京城凛冽的寒风中,他面对皇太极,展开了波澜壮阔人生的
最后一幕。

后金军用潮水般的进攻,证明了自己还想进北京抢一把的美好愿景,但关宁铁骑用倒在他们面前的无数尸体证明,
你们不行。

双方在左安门外持续激战,经过长达五个多小时的拉锯,皇太极终于支持不住,再次败退。左安门之战,以明军获
胜告终。

结束了,都结束了。

一个将军最好的归宿,就是在最后一场战役中,被最后一颗子弹打死。 --巴顿

我原先认为,说这句话的人,应该是吃饱了撑的外加精神失常,现在我明白了,他是对的。

崇祯二年(1629)十二月一日,袁崇焕得到指示,皇帝召见立即进城。

召见的理由是议饷,换句话说就是发工资。

命令还说,部将祖大寿一同觐见。

从古到今,领工资这种事都是跑着去的。袁崇焕二话不说,马上往城里跑,所以他忽略了如下问题:既然是议饷,
为什么要拉上祖大寿?

跑到城下,却没人迎接,也不给开城门,等了半天,丢下来个筐子,让袁督师蹲进去,拉上来。

这种入城法虽说比较寒掺,但好歹是进去了,在城内守军的指引下,他来到了平台。

满桂和黑云龙也来了,正等待着他。

在这个曾带给他无比荣誉和光辉的地方,他第三次见到了崇祯。

第一次来,崇祯很客气,对他言听计从,说什么是什么,要什么给什么。第二次来,还是很客气,十一月份了,城
头风大(我曾试过),二话不说就脱衣服,很够意思。

第三次来,崇祯很直接,他看着袁崇焕,以低沉的声音,问了他三个问题:

一、 你为什么要杀毛文龙。

二、 敌军为何能长驱直入,进犯北京。

三、 你为什么要打伤满桂。

袁崇焕没有回答。

对于他的这一反应,许多史书上说,是没能反应过来,所以没说话。事实上,他就算反应过来,也很难回答。

比如毛文龙同志,实在是不听话外加不顺眼,才剁了的,要跟崇祯明说,估计是不行的。再比如敌军为何长驱直
入,这就说来话长了,最好拿张地图来,画几笔,解释一下战术构思,最后再顺便介绍自己的作战特点。

\section[\thesection]{}

至于最后满桂问题,对袁督师而言,是很有点无厘头的,因为他确实不知道这事。

总而言之,这三个问题下来,袁督师就傻了。

对于袁督师的沉默,崇祯更为愤怒,他当即命令满桂脱下衣服,展示伤疤。

其实袁崇焕是比较莫名其妙的,说得好好的,你脱衣服干嘛?又不是我打的,关我屁事。

但崇祯就不这么想了,袁崇焕不出声,他就当是默认了,随即下令,脱去袁崇焕的官服,投入大牢。

这是一个让在场所有人都很惊讶的举动,虽然有些人已经知道,崇祯今天要整袁崇焕,但万万没想到,这哥们竟然
玩大了,当场就把人给拿下。更重要的是,袁崇焕手握兵权,是城外明军总指挥,敌人还在城外呢,你把他办了,
谁来指挥?

所以内阁大学士成基命、户部尚书毕自严马上提出反对,说了一堆话:大致意思是,敌人还在,不能冲动,冲动是
魔鬼。

但崇祯实在是个四头牛都拉不回来的人物,老子抓了就不放,袁崇焕军由祖大寿率领,明军总指挥由满桂担任,就
这么定了!

现在你应该明白,为什么两次平台召见,除袁崇焕外,还要叫上满桂、黑云龙和祖大寿。

祖大寿是袁崇焕的心腹,只要他在场,就不怕袁军哗变,而满桂是袁崇焕的死敌,抓了袁崇焕,可以马上接班,如
此心计,令人胆寒。

综观崇祯的表现,断言如下:但凡说他蠢的,真蠢。

但这个滴水不漏的安排,还是漏水。

袁崇焕被抓的时候,祖大寿看上去并不吃惊。

他没有大声喧哗,也没有高调抗议,甚至连句话都没说。毕竟抓了袁崇焕后,崇祯就马上发了话,此事与其他人无
关,该干什么还干什么。

但史书依然记下了他的反常举动----发抖,出门的时候迈错步等等。

对于这一迹象,大家都认为很正常----领导被抓了,抖几抖没什么。

只有一个人发现了其中的玄妙。

这个人叫余大成,时任兵部职方司郎中。

祖大寿刚走,他就找到了兵部尚书梁廷栋,对他说:``敌军兵临城下,辽军若无主帅,必有大乱!''

梁廷栋毫不在意:``有祖大寿在,断不至此!''

余大成答:``作乱者必是此人!''

\section[\thesection]{}

梁廷栋没搭理余大成,回头进了内阁。

在梁部长看来,余大成说了个笑话。于是,他就把这个笑话讲给了同在内阁里的大学士周延儒。这个笑话讲给一般
人听,也就是笑一笑,但周大学士不是一般人。

周延儒,字玉绳,常州人,万历四十一年进士。周延儒同志的名气,是很大的。十几年前我第一次翻明史的时候,
曾专门去翻他的列传,没有翻到。后几经查找才发现,这位仁兄被归入了特别列传----奸臣传。

奸臣还不好说,奸是肯定的。此人天资聪明,所谓万历四十一年进士,那是谦虚的说法。事实上,他是那一年的状
元,不但考试第一,连面试(殿试)也第一。

听到这句话,嗅觉敏锐的周延儒立即起身,问: ``余大成在哪里?''

余大成找来了,接着问:``你认为祖大寿会反吗?''

余大成回答: ``必反。''

``几天?''

``三天之内。''

周延儒立即指示梁廷栋,密切注意辽军动向,异常立即报告。

第一天,十二月二日,无事。

第二天,十二月三日,无事。

第三天,十二月四日,出事。

祖大寿未经批示,于当日凌晨率领辽军撤离北京,他没有投敌,临走时留下话,说要回宁远。

回宁远,也就是反了。皇帝十分震惊,关宁铁骑是精锐主力,敌人还在,要都跑了,摊子怎么收拾?

周延儒很镇定,他立即叫来了余大成,带他去见皇帝谈话。

皇帝问:祖大寿率军出走,怎么办?

余大成答:袁崇焕被抓,祖大寿心中畏惧,不会投敌。

皇帝再问:怎么让他回来?

余大成答:只有一件东西,能把他拉回来。

这件东西,就是袁崇焕的手谕。

好办,马上派人去牢里,找袁督师写信。

袁督师不写。

可以理解,被人当场把官服收了,关进了号子,有意见难免,加上袁督师本非善男信女,任你说,就不写。

急眼了,内阁大学士,外加六部尚书,搞了个探监团,全跑到监狱去,轮流劝说,口水乱飞。袁督师还是不肯,还
说出了不肯的理由:``我不是不写,只是写了没用,祖大寿听我的话,是因为我是督师,现我已入狱,他必定不肯
就范。''

\section[\thesection]{}

这话糊弄崇祯还行,余大成是懂业务的:什么你是督师,他才听你的话,那崇祯还是皇上呢,他不也跑了吗?

但这话说破,就没意思了,所以余大成同志换了个讲法,先捧了捧袁崇焕,然后从民族大义方面,对袁崇焕进行了
深刻的教育,说到最后,袁督师欣然拍板,马上就写。

拿到信后,崇祯即刻派人,没日没夜地去追,但祖大寿实在跑得太快,追上的时候,人都到锦州了。

事实证明,袁督师就算改行去卖油条,说话也是算数的。祖大寿看见书信(还没见人),就当即大哭失声,二话不说
就带领部队回了北京。

局势暂时稳定,一天后,再度逆转。

十二月十七日,皇太极再度发起攻击。

这次他选择的目标,是永定门。

估计是转了一圈,没抢到多少实在玩意,所以皇太极决定,玩一把大的,他集结了所有兵力,猛攻永定门。

明军于城下列阵,由满桂指挥,总兵力约四万,迎战后金。

战役的结果再次证明,古代游牧民族在玩命方面,是有优越性的。

经过整日激战,明军付出重大伤亡,主将满桂战死,但后金军也损失惨重,未能攻破城门,全军撤退。

四年前,籍籍无名的四品文官袁崇焕,站在那座叫宁远的孤城里,面对着只知道攒钱的满桂、当过逃兵的赵率教、
消极怠工的祖大寿,说:``独卧孤城,以当虏耳!''

在绝境之中,他们始终相信,坚定的信念,必将战胜强大的敌人。

之后,他们战胜了努尔哈赤,战胜了皇太极,再之后,是反目、排挤、阵亡、定罪、叛逃。

赵率教死了,袁崇焕坐牢了,满桂指认袁崇焕后,也死了,祖大寿终将走上那条不归之路。

共患难者,不可共安乐,世上的事情,大致都是如此吧。

密谋

永定门之战后,一直没捞到硬货的皇太极终于退兵了----不是真退。

他派兵占据了遵化、滦城、永平、迁安,并指派四大贝勒之一的阿敏镇守,以此为据点,等待时机再次发动进攻。

战局已经坏到不能再坏的地步,虽然外地勤王的军队已达二十多万,鉴于满桂这样的猛人也战死了,谁都不敢轻举
妄动,朝廷跟关外已基本失去联系,辽东如何,山海关如何,鬼才知道,京城人心惶惶,形势极度危险。

然后,真正的拯救者出现了。

\section[\thesection]{}

半个月前,草民孙承宗受召进入京城,皇帝对他说:``从今天起,你就是大学士,这是上级对你的信任。''然后皇
帝又说,``既然你是孙大学士了,现在出发去通州,敌人马上到。''

对于这种平时不待见,临时拉来背锅的欠揍行为,孙承宗没多说什么,在他看来,这是义务。

但要说上级一点不支持,也不对,孙草民进京的时候,身边只有一个人,他去通州迎敌的时候,朝廷还是给了孙大
学士一些人。一些人的数量是,二十七个。

孙大学士就带着二十七个人,从京城冲了出来,前往通州。当时的通州已经是前线了,后金军到处劫掠,杀人放火
兼干车匪路霸,孙大学士路上就干了好几仗,还死了五个人,到达通州的时候,只剩二十二个。

通州是有兵的,但不到一万人,且人心惶惶,总兵杨国栋本来打算跑路了,孙承宗把他拉住,硬拽上城楼,巡视一
周,说明白不走,才把大家稳住。

通州稳定后,作为内阁大学士兼兵部尚书,孙承宗开始协调各路军队,组织作战。

以级别而言,孙大学士是总指挥,但具体实施起来,却啥也不是。

且不说其他地区的勤王军,就连嫡系袁崇焕都不听招呼,孙承宗说,你别绕来绕去,在通州布防,把人挡回去就是
了,偏不听,协调来协调去,终于把皇太极协调到北京城下。

然后又是噼里啪啦一阵乱打,袁督师进牢房,皇太极也没真走,占着四座城池,随时准备再来。京城附近的二十多
万明军,也是看着人多,压根没人出头,关宁铁骑也不可靠,祖大寿都逃过一次了,难保他不逃第二次。

据说孙承宗是个水命,所以当救火队员实在再适合不过了。他先找祖大寿。

祖大寿是个比较难缠的人,且向来嚣张跋扈,除了袁崇焕,谁的面子都不给。

但孙承宗是例外,用今天的话说,当年袁督师都是给他提包的,老领导的老领导,就是领导的平方。

孙大学士说:袁督师已经进去了,你要继续为国效力。

祖大寿说,袁督师都进去了,我不知哪天也得进去,还效力个屁。

孙承宗说:就是因为袁督师进去了,你才别闹腾,赶紧给皇帝写检讨,就说你要立功,为袁督师赎罪。

\section[\thesection]{}

祖大寿同意了,立即给皇帝写信。

这边糊弄完了,孙承宗马上再去找皇帝,说祖大寿已经认错了,希望能再有个机会,继续为国效命。

话刚说完,祖大寿的信就到了,皇帝大人非常高兴,当即回复,祖大寿同志放心去干,对你的举动,本人完全支持。

虽然之前他也曾对袁崇焕说过这句话,但这次他做到了,两年后祖大寿在大凌河与皇太极作战,被人抓了,后来投
降又放回来,崇祯问都没问,还接着用。如此铁杆,就是孙承宗糊弄出来的。

孙承宗搞定了祖大寿,又去找马世龙。

马世龙也是辽东系将领,跟祖大寿关系很好,当时拿着袁崇焕的信去追祖大寿的,即是此人。这人的性格跟祖大寿
很类似,极其强横,唯一的不同是,他连袁崇焕的面子都不给,此前有个兵部侍郎刘之纶,带兵出去跟皇太极死
磕,命令他带兵救援,结果直到刘侍郎战死,马世龙都没有来。

但是孙大学士仍然例外,什么关宁铁骑、关宁防线,还有这帮认人不认组织的武将,都是当年他弄出来的,能压得
住阵的,也只有他。

但手下出去找了几天,都没找到这人,因为马世龙的部队在西边被后金军隔开,没消息。

但孙承宗是有办法的,他出了点钱,找了几个人当敢死队,拿着他的手书,直接冲过后金防线,找到了马世龙。

老领导就是老领导,看到孙承宗的信,马世龙当即表示,服从指挥,立即前来会师。

至此,孙承宗终于集结了辽东系最强的两支军队,他的下一个目标是:击溃入侵者。

皇太极退出关外,并派重兵驻守遵化、永平四城,作为后金驻关内办事处,下次来抢东西也好有个照应。

这种未经许可的经营行为,自然是要禁止的,崇祯三年(1630)二月,孙承宗集结辽东军,发起进攻。

得知孙承宗进攻的消息时,皇太极并不在意,按年份算,这一年,孙承宗都六十八了,又精瘦,风吹都要摆几摆,
看着且没几天蹦头了,实在不值得在意。

结果如下:

第一天,孙承宗进攻栾城,一天,打下来了。

第二天,进攻迁安,一天,打下来了。

第三天,皇太极坐不住了,他派出了援兵。

\section[\thesection]{}

带领援兵的,是皇太极的大哥,四大贝勒之一的阿敏。

阿敏是皇太极的大哥,在四大贝勒里,是很能打的。派他去,显示了皇太极对孙承宗的重视,但我始终怀疑,皇太
极跟阿敏是有点矛盾的。

因为战斗结果实在是惨不忍睹。

阿敏带了五千多人到了遵化,正赶上孙承宗进攻,但他刚到,看了看阵势,就跑路了。

孙承宗并没有派兵攻城,他只是在城下,摆上了所有的大炮。

战斗过程十分无聊,孙承宗对炮兵的使用已经炉火纯青,几十炮打完,城墙就轰塌了,阿敏还算机灵,早就跑到了
最后一个据点----永平。

如果就这么跑回去,实在太不像话,所以阿敏在永平城下摆出了阵势,要跟孙承宗决战。

决战的过程就不说了,直接说结果吧,因为从开战起,胜负已无悬念,孙承宗对战场的操控,已经到了炉火纯青的
地步,大炮轰完后,骑兵再去砍,真正实现了无缝对接。

阿敏久经沙场,但在孙老头面前,军事技术还是小学生水平,连一天都没撑住,白天开打下午就跑了,死伤四千余
人,连他自己都负了重伤,差点没能回去。

就这样,皇太极固守的关内四城全部失守,整个过程只用五天。

消息传到京城,崇祯激动了,他二话不说,立马跑到祖庙向先辈汇报,并认定,从今以后,就靠孙承宗了。

事情就这样结束了,自崇祯二年十一月起,皇太极率军进入关内,威胁北京,沿途烧杀抢掠,所过之地实行屠城,
尸横遍野,史称``己巳之变''。

在这场战争中,无辜百姓被杀戮,经济受到严重破坏,包括满桂在内的几位总兵阵亡,袁崇焕下狱,明朝元气大伤。

但一切已经过去,对于崇祯而言,明天比昨天更重要。

当然,在处理明天的问题前,必须先处理昨天的问题。

这个问题的名字叫做袁崇焕。

对话

怎么处理袁崇焕,这是个问题。

其实崇祯并不想杀袁崇焕。

十二月一日,逮捕袁崇焕的那天,崇祯给了个说法----解职听堪。

这四个字的意思是,先把职务免了,再看着办。

看着办,也就是说可以不办。

\section[\thesection]{}

事实上,当时帮袁崇焕说话的人很多,看情形关几天没准就放了,将来说不定还能复职。

但九个月后,崇祯改变了主意,他已下定决心,处死袁崇焕。

为什么?

对于这一变化,许多人的解释,都来源于一个故事。

故事是这样的:

崇祯二年(1629)十一月二十八日,在北京城外无计可施的皇太极,决定玩个阴招。

他派人找来了前几天抓住的两个太监,并把他们安排到了一个特定的营帐里,派专人看守。

晚上,夜深人静之时,在太监的隔壁营帐,住进了两个人,这两个人用人类能够听见的声音(至少太监能听见),说
了一个秘密。

秘密的内容是袁崇焕已经和皇太极达成了密约,过几天,皇太极攻击北京,就能直接进城。

这两个太监不负众望,听见了这个秘密,第二天,皇太极又派人把他们给送了回去。

他们回去之后,就找到了相关部门,把这件事给说了,崇祯大怒,认定袁崇焕是个叛徒,最终把他给办了。

故事讲完了。

这是个相当智慧且相当胡扯的故事。

二十年前,我刚上小学二年级的时候,曾相信过这个故事,后来我长大了,就不信了。

但把话说绝了,似乎不太好,所以我更正一下:如果当事人全都是小学二年级水平,故事里的诡计是可以成功的。

因为这个故事实在太过幼稚。

首先,你要明白,崇祯不是小学二年级学生,他是一个老练成熟的政治家,也是大明的最高领导。

三年前,满朝都是阉党,他啥都没说,只凭自己,就摆平了无法无天的魏忠贤;两年前,袁崇焕不经许可,干掉了
毛文龙,他还是啥都没说。

明朝的言官很有职业道德,喜欢告状,自打袁崇焕上任,他的检举信就没停过,说得有鼻子有眼,某些问题可能还
是真的,他仍然没说。

敌军兵临城下,大家都骂袁崇焕是叛徒,他脱掉自己的衣服,给袁崇焕披上,打死他都没说。

所以最后,他听到了两个从敌营里跑出来的太监的话,终于说了:杀掉袁崇焕。

无语,彻底的无语。

我曾十分好奇,这个让人无语的故事到底是怎么来的。

经过比对记载此事的几十种史料,我确定,这个故事最早出现的地方,是清军入关后,由清朝史官编撰的《清太宗
实录》。

明白了。

\section[\thesection]{}

记得当年我第一次去看清朝入关前的原始史料,曾经比较烦,因为按照常规,这些由几百年前的人记录的资料,是
比较难懂的,而且基本都是满文,我虽认识几个,但要看懂,估计是很难的。

结果大吃一惊。

我看懂了,至少明白这份资料说些什么,且毫不费力,因为在我翻开的那本史料里,有很多绣像。

所谓绣像,用今天的话说就是插图,且画工很好,很详细,打仗、谈事都画出来,是个人就能看明白。

后来我又翻过满洲实录,也有很多插图,比如宁远之战、锦州之战,都画得相当好。

这是个比较奇怪的现象,古代的插图本图书很多,比如金瓶梅、西游记等等,但通常来讲,类似政治文书、历史记
录之类的玩意,为示庄重,是没有插图的,从司马迁、班固,到修明史的张廷玉,二十五史,统统地没有。顺便说
句,如果哪位仁兄能够找到司马迁版原始插图史记,或是班固版插图汉书,记得通知我,多少钱我都收。

疑惑了很久后,我终于找到了答案----文化。

后金是游牧民族,文化比较落后,虽说时不时也有范文程之类的文化人跑过去,但终究是差点,汉字且不说,满文
都是刚造出来的,认识的人实在太少。

但这么多年,都干过些什么事,必须要记,开个会、谈个话之类的,一个个传达太费劲,写成文字印出去,许多人
又看不懂,所以就搞插图版,认字的看字,不认字的就当连环画看,都能明白。

而在军事作战上,这点就更为明显了。

努尔哈赤、皇太极以及后来的多尔衮,都是卓越的军事家,能征善战,但基本都是野路子练出来的,属于实干派。
在这方面,明朝大致相反,孙承宗袁崇焕都是考试考出来的,属于理论派。

打仗这个行当,和打架有点类似,被人拍几砖头,下次就知道该拿菜刀还是板砖,朝哪下手更狠,老是当观众,很
难有技术上的进步。所以在战场上,卷袖子猛干的实干派往往比读兵书的理论派混得开。

但马克思同志告诉我们,理论一旦与实践结合,就会产生巨大的能量,成功范例如孙承宗等,都是旷世名将。

皇太极等人及时意识到了自己工作中的不足,于是他们摆事实,找差距,决定普及理论。

\section[\thesection]{}

在明朝找人来教,估计是不行了,所以教育的主要方法,是读兵书。反正兵书也不是违禁品,找人去明朝采购回
来,每人发一本,慢慢看。

工作进行得十分顺利,托人到关内去买,但采购员到地方,就傻眼了。

因为从古至今,兵书很多,什么太公兵法、孙子兵法、六韬三略且不说,光是明代,兵书就有上百种,是出版行业
的一支生力军。

面对困难,皇太极们没有气馁,他们经过仔细研讨比较,终于确定了最终的兵法教材,并大量采购,保证发到每个
高级将领手中。

此后无论是行军还是打仗,后金军的高级将领们都带着这本指定兵法教材,早晚阅读。

这本书的名字,叫做《三国演义》。

其实没必要吃惊,毕竟孙子兵法之类的书,确实比较深奥,到京城街上拉个人回来,都未必会读。要让天天骑马打
仗的人读,实在勉为其难,当时《三国演义》里的语言,大致就相当于是白话文了,方便理解,而且我相信,这本
书很容易引起后金将领们的共鸣----有插图。

没错,答案就在这本书中。

所谓反间计的故事,如不知来源,可参考《三国演义》之蒋干中计,综合上述资料,以皇太极们的文化背景,能编
出这么个故事,差不多了。

但更关键的,是下一个问题--为什么要编这个故事。

这个问题困惑了我三年,一次偶然的机会,让我找到了答案----我的答案。

我认定,这是一个阴谋,一个蓄谋已久且极其高明的阴谋。

关于此阴谋的来龙去脉,鉴于本人为此思考了很久,所以我决定,歇口气,等会再讲。

其实改变崇祯主意的,并不是那个幼稚的反间计,而是一次谈话。

这次谈话发生在一年前,谈话的两个人,分别是内阁大学士钱龙锡,和刚刚上任的蓟辽督师袁崇焕。

谈话内容如下:

钱龙锡:平辽方略如何?

袁崇焕:东江、关宁而已。

钱龙锡:东江何解?

袁崇焕:毛文龙者,可用则用之,不可用则除之。

翻译一下,意思大致是这样的:钱龙锡问,你上任后准备怎么干。袁崇焕答,安顿东江和关宁两个地方。钱龙锡又
问:为什么要安顿东江。

袁崇焕答:东江的毛文龙,能用就用,不能用就杀了他。

\section[\thesection]{}

按说这是两人密谈,偏偏就被记入了史料,实在是莫名其妙。而且这份谈话记录看上去似乎也没啥,钱龙锡问袁崇
焕的打算,袁崇焕说准备收拾毛文龙,仅此而已。

但杀死袁崇焕的,就是这份谈话记录。

崇祯二年(1629)十二月七日,御史高捷上疏,弹劾钱龙锡与袁崇焕互相勾结,一番争论之后,钱龙锡被迫辞职。

著名史学家孟森曾说过,明朝有两大祸患:第一是太监,其次是言官。

我认为,这句话是错的。言官应该排在太监的前面,如太监是流氓,言官就是流氓2.0版本--文化流氓。

鉴于明代政治风气实在太过开明,且为了保持政治平衡,打朱元璋起,皇帝就不怎么管这帮人。结果脾气越惯越
大,有事说事,没事说人,逮谁骂谁,见谁踩谁(包括皇帝),到了崇祯,基本已经形成了有组织,有系统的流氓集
团,许多事情就坏在他们的手里。

在这件事上,他们表现得非常积极,此后连续半年,关于袁崇焕同志叛变、投敌乃至于生活作风等多方面问题的黑
材料源源不断,一个比一个狠(许多后人认定所谓袁崇焕投敌卖国的铁证,即源自于此)。

就这么骂了半年,终于出来个更狠的。

崇祯三年(1630)八月,山东御史史范上疏,弹劾钱龙锡收受袁崇焕贿赂几万两,连钱放在哪里,都说得一清二楚。

太阴险了。

在明代,收点黑钱,捞点外快,基本属于内部问题,不算啥事,但这封奏疏却截然不同。

因为他说,送钱的人是袁崇焕。这钱就算是阎王送的,都没问题,惟独不能是袁崇焕。

因为袁崇焕是边帅,而钱龙锡是内阁大臣。按照明朝规定,如果边帅勾结近臣,必死无疑(有谋反嫌疑)。

十天后,崇祯开会,决定,处死袁崇焕。

崇祯二年(1629)十二月袁崇焕入狱,一群人围着骂了八个月,终于,骂死了。

事情就是这样吗?

不是

在那群看似漫无目的,毫无组织的言官背后,是一双黑手,更正一下,是两双。

这两双手的主人,一个叫温体仁,一个叫周延儒。

周延儒同志前面已经介绍过了,这里讲一下温体仁同志的简历:男,浙江湖州人,字长卿,万历二十六年进士。

\section[\thesection]{}

这两人后面还要讲,这里就不多说了,对这二位有兴趣的,可以去翻翻明史。顺提一下,很好找,直接翻奸臣传,
周延儒同志就在严嵩的后面,接下来就是温体仁。

应该说,袁崇焕从``听堪'',变成了``听斩'',基本上就是这二位的功劳。但这件事情,最有讽刺意味的,也就在
这里。

因为温体仁和周延儒,其实跟袁崇焕没仇,且压根儿就没想干掉袁崇焕。

他们真正想要除掉的人,是钱龙锡。

有点糊涂了吧,慢慢来。

一直以来,温体仁和周延儒都想解决钱龙锡,可是钱龙锡为人谨慎,势力很大,要铲除他非常困难。十分凑巧,他
跟袁崇焕的关系很好,这次恰好袁崇焕又出了事,所以只要把袁崇焕的事情扯大,用他的罪名,把钱龙锡拉下水,
就能达到目的。

袁崇焕之所以被杀,不是因为他自己,而是因为钱龙锡,钱龙锡之所以出事,不是因为他自己,而是因为袁崇焕。

幕后操纵,言官上疏,骂声一片,只是为了一个政治目的。

接下来要解开的迷题是,他们为什么要除掉钱龙锡。

有人认为,这是一个复仇的问题。是由于党争引起的,周延儒和温体仁都是阉党,因为被整,所以借此事打击东林
党,报仇雪恨。

我认为,这是一个历史基本功问题,是由于史料读得太少引起的。

周延儒和温体仁绝不是阉党,虽然他们并非什么好鸟,但这一点我是可以帮他们二位担保的。事实上,阉党要有他
们这样的人才,估计也倒不了。

崇祯元年(1628),就在崇祯大张旗鼓猛捶阉党的时候,温体仁光荣提任礼部尚书,周延儒荣升礼部侍郎。堂堂阉
党,如此顶风作案,公然与严惩阉党的皇帝勾结获得提升,令人发指。

在攻击袁崇焕的人中,确实有阉党,但这件事情的幕后策划者,却绝非同类,当一切的伪装去除后,真正的动机始
终只有俩字----权力。

内阁的权力很大,位置却太少,要把自己挤上去,只有把别人挤下来。事实上,他们确实达到了目的,由于袁崇焕
的事太大,钱龙锡当即提出辞职,而跟钱龙锡关系很好的大学士成基命几个月后也下课,温体仁入阁,成为了大学
士。

而袁崇焕,只是一个无辜的牺牲品。

\section[\thesection]{}

崇祯三年(1630)八月十六日,崇祯在平台召开会议----第四次会议。

第一次,他提拔袁崇焕,袁崇焕很高兴;第二次,他脱衣服给袁崇焕,袁崇焕很感动;第三次,他抓了袁崇焕,袁
崇焕很意外;第四次,他要杀掉袁崇焕,袁崇焕不在。

袁崇焕虽没办法与会(坐牢中),却毫无妨碍会议的盛况,参加会议的各单位有内阁、六部、都察院、大理寺、通政
司、五府、六科、锦衣卫等等,连翰林院都来凑了人数。

人到齐了,崇祯开始发言,发言的内容,是列举袁崇焕的罪状。主要包括给钱给人给官,啥都没干,且杀掉毛文
龙,放纵敌人长驱而入,消极出战等等。

讲完了,问:``三法司如何定罪?''

没人吱声。

弄这么多人来,说这么多,还问什么意见,想怎么办就怎么办吧。

于是,崇祯说出了他的裁决:

依律,凌迟。

现场鸦雀无声。

袁崇焕的命运就这样确定了。

他是冤枉的。

在场的所有人,都是凶手。

温体仁、周延儒未必想干掉袁崇焕,崇祯未必不知道袁崇焕是冤枉的,袁崇焕未必知道自己为什么会死。

但他就是死了。

很滑稽,历史有时候就是这么滑稽。

袁崇焕被押赴西市,行刑。

或许到人生的最后一刻,他都不知道自己为什么会死,他永远也不会知道,在这个世界上,有着许多或明或暗的规
则,必须适应,必须放弃原则,背离良知,和光同尘,否则,无论你有多么伟大的抱负,多么光辉的理想,都终将
被湮灭。

袁崇焕是不知道和光同尘的,由始至终,他都是一个不上道的人。他有才能,有抱负,有个性,施展自己的才能,
实现自己的抱负,彰显自己的个性,如此而已。

那天,袁崇焕走出牢房,前往刑场,沿途民众围观,骂声不绝。

他最后一次看着这个他曾为之奉献一切的国家,以及那些他用生命护卫,却谩骂指责他的平民。

倾尽心力,呕心沥血,只换来了这个结果。

我经常在想,那时候的袁崇焕,到底在想些什么。

他应该很绝望,很失落,因为他不知道,什么时候他的冤屈才能被洗刷,他的抱负才能被了解,或许永远也没有那
一天,他的全部努力,最终也许只是遗臭万年的骂名。

\section[\thesection]{}

然而就在行刑台上,他念出了自己的遗言:

一生事业总成空,

半世功名在梦中。

死后不愁无勇将,

忠魂依旧守辽东。

这是一个被误解、被冤枉、且即将被千刀万剐的人,在人生的最后时刻留下的诗句。

所以我知道了,在那一刻,他没有绝望,没有失落,没有委屈,在他的心中,只有两个字----坚持。

一直以来,几乎所有的人都告诉我,袁崇焕的一生是一个悲剧。

事实并非如此。

因为在我看来,他这一生,至少做到了一件事,一件很多人无法做到的事----坚持。

蛮荒之地的苦读书生,福建的县令,京城的小小主事,坚守孤城的宁远道,威震天下的蓟辽督师,逮捕入狱的将
领,背负冤屈死去的囚犯。

无论得意,失意,起或是落,始终坚持。

或许不能改变什么,或许并不是扭转乾坤的关键人物,或许所作所为并无意义,但他依然坚定地,毫无退缩地坚持
下来。

直到生命的最后一刻,他也没有放弃。

阴谋

袁崇焕是一个折腾了我很久的人。

围绕这位仁兄的是是非非,叛徒也罢,英雄也好,几百年吵下来,毫无消停迹象

但一直以来,对袁崇焕这个人,我都感到很纳闷。因为就历史学而言,历史人物的分类大致分为三级:

第一级:关键人物,对历史发展产生过转折性影响的,归于此类。

典型代表:张居正。如果没这人,就没有张居正改革,万历同志幼小的心灵没准能茁壮成长,明朝也没准会早日完
蛋,总而言之,都没准。再比如秦桧,也是关键人物,他要不干掉岳飞,不跟金朝和谈,后来怎么样,也很难说。
总而言之,是能给历史改道的人。

第二级:重要人物,对历史产生重大影响的,归于此类。

典型代表:戚继光。没有戚继光,东南沿海的倭寇很难平息。但此级人物与一级人物的区别在于,就算没有戚继
光,倭寇也会平息,无非是个时间问题。换句话说,这类人没法改道,只能在道上一路狂奔。

第三级:鸡肋人物,但凡史书留名,又不属于上述两类人物的,皆归于此类。

典型代表:太多,就不扯了,这类人基本都有点用,但不用似乎也没问题,属路人甲乙丙丁型。

袁崇焕,是第二级。

\section[\thesection]{}

明末是一个特别乱的年代。朱氏公司已经走到悬崖边,就快掉下去了,还有人往下踹(比如皇太极之流),也有人往
上拉(比如崇祯,杨嗣昌)。出场人物很多,但大都是二、三级人物,折腾来折腾去,还是亡了。

一级人物也有,只有一个。

只有这个人,拥有改变宿命的能力----我说过了,是孙承宗。关宁防线的构建者,袁崇焕、祖大寿、赵率教、满桂
的提拔者,收拾烂摊子,收复关内四城,赶走皇太极的护卫者。

从头到尾,由始至终,都是他在忙活。

其实二级人物袁崇焕和一级人物孙承宗之间的差距并不大,他有坚定的决心,顽强的意志,卓越的战斗能力,只差
一样东西----战略眼光。

他不知道为什么不能随便杀总兵,为什么不能把皇太极放进来打,为什么自己会成为党争的牺牲品。

所以他一辈子,也只能做个二级人物。

好了,现在最关键的时刻到了:

为什么一个二级人物,会引起这么大的争议呢?不是民族英雄,就是卖国贼。

卖国贼肯定不是。所谓指认袁崇焕是卖国贼的资料,大都出自当时言官们的奏疏,要么是家在郊区,被皇太极烧
了;要么是跟着温体仁、周延儒混,至少也是看袁崇焕不顺眼。这帮人搞材料,那是很有一套的,什么黑写什么,
偶尔几份流传在外,留到今天,还被当成宝贝。

其实这种黑材料,如果想看,可以找我。外面找不到的,我这里基本都有,什么政治问题、经济问题、生活作风问
题,应有尽有,编本袁崇焕黑材料全集,绰绰有余。

至于民族英雄,似乎也有点悬,毕竟他老人家太有个性,干过些不地道的事,就水平而言,也不如孙老师,实在有
点勉为其难。

所以一直以来,我都在思考这个问题,从未间断,因为我隐约感到,在所谓民族英雄与卖国贼之争的背后,隐藏着
不为人知的秘密。

直到有一天,我找到了这个秘密的答案:阴谋。

\section[\thesection]{}

那一天,我跟几位史学家聊天,偶尔有人说起,据某些史料及考证,其实弘光皇帝(朱由崧,南明南京政权皇帝)跟
崇祯比较类似,也是相当勤政,卖命干没结果。

这位弘光同志,在史书上,从来就是皇帝的反面教材,吃喝嫖赌无一不精,所以我很奇怪,问:``若果真如此,为
何这么多年,他都是反面形象?''

答: ``因为他是清朝灭掉的。''

都解开了。

崇祯很勤政,崇祯并非亡国之君,弘光很昏庸,弘光活该倒霉,几百年来,我们都这样认为。

但我们之所以一直这样认为,只是因为有人这样告诉我们。

之所以有人这样告诉我们,是因为他们希望我们这样认为。

在那一刻,我脑海中的谜团终于解开,所有看似毫不相关的线索,全都连成了一线。

崇祯不该死,因为他是被李自成灭掉的,所以李自成在清朝所修明史里面的分类,是流寇。

而我依稀记得,清军入关时,他们的口号并非建立大清,而是为崇祯报仇,所以崇祯应该是正义的。

弘光之所以该死,因为他是被清军灭掉的,大清王朝所剿灭的对象,必须邪恶,所以,弘光应该是邪恶的。

在百花缭乱的历史评论背后,还是只有两字--利益。

但凡能争取大明百姓支持的,都要利用,但凡是大清除掉的,都是敌人。只为了同一个目的----维护大清利益,稳
固大清统治。

掌握这把钥匙,就能解开袁崇焕事件的所有疑团。

其实袁崇焕之所以成为几百年都在风口浪尖上转悠,只是因为一个意外事件的发生。

由于清军入关时,打出了替崇祯皇帝报仇的口号,所以清朝对这位皇帝的被害,曾表示极度的同情,对邪恶的李自
成、张献忠等人,则表示极度的唾弃(具体表现,可参阅明史流寇传)。

因此,对于崇祯皇帝,清朝的评价相当之高,后来顺治还跑到崇祯坟上哭了一场,据说还叫了几声大哥,且每次都
以兄弟相称,很够哥们,但到康乾时期,日子过安稳了,发现不对劲了。

因为崇祯说到底,也是大明公司的最后一任董事长,说崇祯如何好,如何死得憋屈,说到最后,就会出现一个悖论:

既然崇祯这么好,为什么还要接受大清的统治呢?

所以要搞点绯闻丑闻之类的玩意,把人搞臭才行。

但要直接泼污水,是不行的,毕竟夸也夸了,哭也哭了,连兄弟都认了,转头再来这么一出,太没水准。

要解决这件事,绝不能挥大锤猛敲,只能用软刀子背后捅人。最好的软刀子,就是袁崇焕。

\section[\thesection]{}

阴谋的来龙去脉大致如上,如果你不明白,答案如下:要诋毁崇祯,无需谩骂,无需污蔑,只需要夸奖一个人----
袁崇焕。

因为袁崇焕是被崇祯干掉的,所以只要死命地捧袁崇焕,把他说成千古伟人,而如此伟人,竟然被崇祯干掉了,所
谓自毁长城,不费吹灰之力,就能把崇祯与历史上宋高宗(杀岳飞)之流归为同类。

当然了,安抚大明百姓的工作还是要做,所以该夸崇祯的,还是得夸,只是夸的内容要改一改,要着力宣传他很勤
政,很认真,很执着,至于精明能干之类的,可以忽略忽略。总而言之,一定要表现人物的急躁、冲动,想干却没
干成的形象。

而要树立这个形象,就必须借用袁崇焕。

之后的事情就顺理成章了,把袁崇焕树立为英雄,没有缺点,战无不胜,只要有他在,就有大明江山,再适当渲染
气氛,编实录,顺便弄个反间计故事,然后,在戏剧的最高潮,伟大的英雄袁崇焕----

被崇祯杀掉了。

多么愚蠢,多么自寻死路,多么无可救药。

就这样,在袁崇焕的叹息声中,崇祯的形象出现了:

一个很有想法,很有能力,却没有脑子,没有运气,没有耐心,活活被憋死的皇帝。

最后,打出主题语:

如此皇帝,大明怎能不亡?

收工。

袁崇焕就这样变成了明朝的对立面,由于他被捧得太高,所以但凡跟他作对的(特别是崇祯),都成了反面人物。

肯定了袁崇焕,就是否定了崇祯,否定了明朝,清朝弄到这么好的挡箭牌,自然豁出去用,所以几百年下来,跟袁
督师过不去的人也很多,争来争去,一直争到今天。

说到底,这就是个套。

几百年来,崇祯和袁崇焕,还有无数的人,都在这个套子里,被翻来覆去,纷争、吵闹,自己却浑然不知。所以,
应该戳破它。

当然,这一切只是我的看法,不能保证皆为真理,却可确定绝非谬误。

其实无论是前世的纷争,还是后代的阴谋,对袁崇焕本人而言,都毫无意义。他竭尽全力,立下战功,成为了英
雄,却背负着叛徒的罪名死去。

很多人曾问我,对袁崇焕,是喜欢,还是憎恶。

\section[\thesection]{}

对我而言,这是个没有意义的问题,因为我坚信历史的判断和评价,一切的缺陷和荣耀,都将在永恒的时间面前,
展现自己的面目,没有伪装,没有掩饰。

所以我竭尽所能,去描述一个真实的袁崇焕:并非天才,并非优等生,却运气极好,受人栽培,意志坚定,却又性
格急躁,同舟共济,却又难以容人,一个极其单纯,却又极其复杂的人。

在这世上,只要是人,都复杂,不复杂的,都不是人。

袁崇焕很复杂,他极英明,也极愚蠢,曾经正确,也曾经错误。其实他被争议,并不是他的错,因为他本就如此,
他很简单的时候,我们以为他很复杂,他很复杂的时候,我们以为他很简单。

事实上,无论叛徒,或是英雄,他都从未变过,变的,只是我们自己。

越过几百年的烟云,我看到的袁崇焕,并没有那么复杂,他只是一个普通的人,在那个风云际会的时代,抱持着自
己的理想,坚持到底。

即使这理想永远无法实现,即使这注定是个悲剧的结尾,即使到人生的最后一刻,也永不放弃。

有时候,我会想起这个人,想起他传奇的一生,他的光荣,他的遗憾。

有时候,我看见他站在我的面前,对我说:

我这一生,从没有放弃。

抽签

对袁崇焕而言,一切都结束了,但对崇祯而言,生活还要继续,明天,又是新的一天,当然,未必会更好。

他亲手除掉了有史以来最庞大、最邪恶的阉党,却惊奇地发现,另一个更强大的敌人,已经站立在他的面前。

这是一个看不见的敌人。

崇祯上台不久,就发现了一件奇怪的事:他是皇帝,大家也认这个皇帝,交代下去的事,却总是干不成,工作效率
极其低下。因为自登基以来,所有的大臣都在干同一件事--吵架。

今天你告我,明天我告你,瞎折腾,开始崇祯还以为这是某些阉党的反扑,但时间长了才发现,这是纯粹的、无组
织、无纪律的吵架。

一夜之间,朝廷就变了,正事没人干,尽吵,且极其复杂。当年朝廷斗争,虽说残酷,好歹还分个东林党,阉党,
带头的也是魏忠贤、杨涟之类的大腕,而今不同了,党争标准极低,只要是个人,哪怕是六部里的一个主事处长,
都敢拉帮结伙,逮谁骂谁,搞得崇祯摸不着头脑:是谁弄出来这帮龟孙?

就是他自己。

\section[\thesection]{}

这一切乱象的源头,来源自一年前崇祯同志的一个错误决定。

解决魏忠贤后,崇祯认为,除恶必须务尽,矫枉必须过正,干人必须彻底,所以开始拉清单,整阉党,但凡跟魏忠
贤有关系的,拍马屁的,站过队的,统统滚他娘的。

这是一个极其不地道的举动。大家到朝廷来,无非是混,谁当朝就跟谁混,说几句好话,服软低头,也就是混碗饭
吃。像杨涟那样的英雄人物,我们都是身不能至,心向往之,起码在精神上支持他,现在反攻倒算,打工一族,何
苦呢?

但崇祯同志偏要把事做绝,砸掉打工仔的饭碗,那就没办法了。大家都往死里整,当年你说我是阉党,整顿我,没
事,过两年我上来,不玩死你不算好汉。

特别是东林党,那真不是善人,逮谁灭谁,不听话的,有意见的,就打成阉党,啥事都干不成。

比如天启七年(1627),除掉魏忠贤后,崇祯打算重建内阁,挑了十几个人候选,官员就开始骂,这个有问题,那个
是特务,搞得崇祯很头疼,选谁都有人骂,都得罪人,抓狂不已。

在难题面前,崇祯体现出了天才政治家的本色,闭门几天,想出了一个中国政治史上前所未有的绝招。只要用这
招,无论选谁,大家都服气,且毫无怨言----枚卜。

天启七年(1627)十二月,在崇祯的亲自主持下,枚卜大典召开。

就读音而言,枚卜和没谱是很像的,实际上,效果也差不多,因为所谓枚卜,用今天的话说,就是抓阄。

具体方法是,把候选人的名字写在字条上,放进金瓶,然后摇一摇,再拿夹子夹,夹到的上岗,没夹到下课,完事。

内阁大学士,大致相当于内阁成员,首辅大学士就是总理,其他大学士就是副总理,是大明帝国除皇帝外的最高领
导----抓阄抓出来的。

有人曾告诉我,论资排辈是个好政策,我不信,现在我认为,抓阄也是个好政策,你最好相信。

抓阄抓出来的,谁也没话说,且防止走后台,告黑状、搞关系等等,好歹就是一抓,都能服气,实为中华传统厚黑
学、稀泥学之瑰宝。

\section[\thesection]{}

崇祯同志的首任内阁就此抓齐,总共九人,除之前已经在位的三个,后面六个全是抓的,包括后来被袁崇焕拖下水
的钱龙锡同志,也是这次抓出来的。

这是明朝有史以来最庞大的内阁之一,具体都是谁就不说了,因为没过一年,除钱龙锡外,基本都下课了。

下课的原因不外务以下几种:被骂走,被挤走,被赶走,自己走。

不是不想干,实在是环境太恶劣,明朝这帮大臣都不省油,个个开足马力,谁当政,就把谁往死里骂。特别是言
官,人送外号``抹布'':干净送别人,肮脏留自己,贴切。

但归根结底,还是这帮孙子欠教育,内阁大臣又比较软,好好说话,就是不听,首任内阁刚成立,就一拥而上,弹
来骂去,当即干挺五个。

这下皇帝也不干了,你们把人赶走,是痛快了,老子找谁干活?

所以崇祯元年(1628)十一月,崇祯决定,再抓几个。

吏部随即列出候选名单,准备抓阄。

在这份名单上,有十一个人,按说抓阄这事没谱,能不能入阁全看运气,但这一次,几乎所有的人都认定,有一个
人,必定能够入阁。

这个人的名字,叫做钱谦益。

《三国演义》到了八十回后,猛人基本都死绝了,稍微有点名的,也就是姜维、刘禅之类的杂鱼。明末倒也凑合,
还算名人辈出,特别是干仗的武将,什么袁崇焕、皇太极、张献忠、李自成,知名度都高。

文臣方面就差多了,到了明末,特别是崇祯年间,十几年里,文臣无数,光内阁大臣就换了五十个,都是肉包子打
狗。就算研究历史的,估计也不认识,而其中唯一的例外,就是钱谦益。

钱谦益,字受之,苏州常熟人,万历三十六年进士,名人,超级名人。

钱谦益之所以有名,很大原因在于,他有个更有名的老婆----柳如是。

关于这个人的是是非非,以后再说,至少在当时,他就很有名了。

因为他不但饱读诗书,才华横溢,且是东林党的领导。阉党倒台,东林上台,理所应当,朝廷里从上到下,基本都
是东林党,现在领导要入阁,就是探囊取物。

所以连钱谦益自己都认为,抓阄只是程序问题,入阁只是时间问题,洗个澡,换件衣服,就准备换单位上班了。

可这世上,越是看上去没事的事,就越容易出事。

\section[\thesection]{}

作弊

钱谦益入内阁,一般说来是没有对手的,而他最终没有入阁,是因为遇上了非一般的对手。

在崇祯十余年的统治中,总共用过五十个内阁大臣,鉴于皇帝难伺候,下属不好管,大部分都只干了几个月,就光
荣下岗。

只有两个人,能够延续始终,把革命进行到底,这两个人,一个是周延儒,一个是温体仁。

虽然二位兄弟在历史上的名声差点(奸臣传),但要论业务能力和智商,实在无与伦比。

不幸的是,钱谦益的对手,就是这两位。

之所以要整钱谦益,不是因为他们也在吏部候选名单上,实际上,他们连海选都没入,第一轮干部考察就被刷下来
了。

海选都没进,为什么要坑决赛选手呢?

因为实在太不像话了。

海选的时候,钱谦益的职务是礼部右侍郎,而周延儒是礼部左侍郎,温体仁是礼部尚书。

同一个部门,副部长入阁,部长连决赛都没进,岂有此理。

所以两个岂有此理的人,希望讨一个公道。

在后世的史书里,出于某种目的,温体仁和周延儒的归类都是奸臣,也就是坏人。但仔细分析,就会发现,至少在
当时,这两位坏人,都是弱势群体。在当时的朝廷,东林党势力极大,内阁和六部,大都是东林派,所以钱谦益基
本上算是个没人敢惹的狠角色。

但温部长和周副部长认为,让钱副部长就这么上去,实在太不公平,必须闹一闹。

于是,他们决定整理钱谦益的黑材料,经过不懈努力,他们找到了一个破绽,七年前的破绽。

七年前(天启元年)

作为浙江乡试的主考官,钱谦益来到浙江监考,考试、选拔、出榜,考试顺利完成。

几天后,他回到了北京,又几天后,礼部给事中顾其中上疏弹劾钱谦益,罪名,作弊。

批判应试教育的人曾说,今日之高考,即是古代之进士科举,罪大恶极。

我觉得这句话是不恰当的,因为客观地讲,高考上榜的人,换到明代,最多就是秀才,举人可以想想,进士可以做
梦。

\section[\thesection]{}

明代考完,如果没有意外,基本能有官做,且至少是处级(举人除外),高考考完,大学毕业,如果没有意外,且运
气好点,基本能有工作。

明代的进士考试,每三年一次,每次录取名额,大概是一百五十多人,现在高考,每年两次,每次录取名额……

所以总体说来,明代的进士考试,大致相当于今天的高考+公务员考试+高级公务员选拔。

只要考中,学历有了,工作有了,连级别都有了,如此好事,自然挤破头,怕挤破头,就要读书,读不过,就要作
弊。

鉴于科举关系重大,明代规定,但凡作弊查实,是要掉脑袋的。但由于作弊前景太过美妙,所以作弊者层出不穷,
作弊招数也推陈出新。由低到高,大致分为四种。

最初级的作弊方式,是夹带,所以明朝规定,进入考场时,每人只能携带笔墨,进考场就把门一锁,吃喝拉撒都在
里面,考完才给开门。

为适应新形势的需要,同学们开动脑筋,比如把毛笔凿空,里面塞上小抄,或是在砚台里面夹藏,更牛一点的,就
找人在考场外看准地方,把答案绑在石头上扔进去,据说射箭进去的也有。面对新局面,朝廷规定,毛笔只能用空
心笔杆,砚台不能太厚,考场内要派人巡逻等等。

这是基本技术,更高级一点的,是第二种方法:枪手代考,明朝的同学们趁着照相技术尚未发明,四处找人代考。
当然朝廷不是吃素的,在准考证上,还加上了体貌特征描述,比如面白,无须,高个等等。

以上两项技术,都是常用技术,且好用,为广大人民群众喜闻乐见,所以流传至今,且发扬光大。今日之大学,继
承前辈遗志者,大有人在。

但真正有钱,有办法的,用的是第三种方法--买考题。

考试最重要的,就是考题,只要知道考题,不愁考不上,所以出题的考官,都是重点对象。

但问题是,明代规定,知情人员如果卖题,基本是先下岗再处理。轻则坐牢,重则杀头,风险太大,而且明朝为了
防止作弊,还额外规定,所有获知考题人员,必须住进考场,无论如何,不许外出。

所以在明朝,卖考题的生意是不好做的。

虽然买不到考题,但天无绝人之路,有权有势的同学们还有最后一招杀手锏,此招一出,必定上榜----买考官。不
过,这些考官并不是出题的考官,而是改题的考官。

\section[\thesection]{}

是的,知不知道题目并不重要,就算你交白卷,只要能搞定改题的人,就能金榜题名。但问题是,给钱固然容易,
那么多卷子,怎么对上号呢?

最原始的方法,是认名字,毕竟跟高考不同,考试的人就那么多,看到名字就录取。

魔高一尺,道高一丈,从此以后,试卷开始封名,实行匿名批改。

但作弊的同学们是不会甘心失败的,有的做记号,有的故意在考卷里增大字体,只为对改卷的考官说一句话:我就
是给钱的那个!

这几招相当地有效,且难以禁止,送进去不少人,面对新形势朝廷不等不靠,经过仔细钻研,想出了一个绝妙的对
策。

具体方法是,所有的考卷收齐后,密封姓名,不直接交给考官,而是转给一个特别的人。

这个人并非官员,他收到考卷后,只干一件事--抄。

所有的考卷,都由他重新抄写,然后送给考官批改,全程由人监督。

这招实在是狠,因为所有的考卷,是统一笔迹,统一形式,考官根本无从判断,且毫不影响考试成绩,可谓万无一
失。

综上所述,作弊与反作弊的斗争是长期的,艰苦的,没有尽头的,同学们为了前途,虽屡战屡败,但屡败屡战,到
明代,斗争达到了高潮。

高潮,就发生在天启元年的浙江。

在这次科举考试中,监考程序非常严密,并实行了统一抄写制度,按说是不会有问题的。

但偏偏就出了问题。

因为有人破解了统一抄写制度。

虽然笔迹相同,试卷相同,但这个方法,依然有漏洞,依然可以作弊。

作弊的具体方法是,考生事前与考官预定密码,比如一首唐诗,或是几个字,故意写在试卷的开头,或是结尾,这
样即使格式与字迹改变,依然能够辨别出考卷作者。

在这次考试中,有一个叫钱千秋的人,买到了密码。

密码是七个字----一朝平步上青云。按照约定,他只要将这七个字,写在每段话的末尾,就能平步青云,金榜题名。

事情非常顺利,考试结束,钱千秋录取。

这位钱同志也相当守规矩,录取之后,乖乖地给了钱,按说事情就该结了。

可是意外发生了。

\section[\thesection]{}

因为这种事情,一个人是做不成的,必须是团伙作案,既然是团伙,就要分赃,既然分赃,就可能不匀,既然不
匀,就可能闹事,既然闹事,就必定出事。

钱千秋同志的情况如上,由于卖密码给他的那帮人分赃不匀,某些心态不好的同志就把大家都给告了,于是事情败
露,捅到了北京。

但这件事情说起来,跟钱谦益的关系似乎并不大,虽然他是考官,并没有直接证据证实,他就是卖密码的人,最多
也就背个领导责任。

不巧的是,当时,他有一个仇人。

这个仇人的名字,叫做韩敬,而滑稽的是,他所以跟钱谦益结仇,也是因为作弊。

十年前,举人钱谦益从家乡出发,前往北京参加会试,而韩敬,是他同科的同学。

在考场上,他们并未相识,但考试结束时,就认识了,以一种极为有趣的方式。

跟其他人不同,在考试成绩出来前,钱谦益就准备好当状元了,因为他作弊了。

但他作弊的方式,既不是夹带,也不是买考官,甚至不是买密码,而是作弊中的最高技巧----买朝廷。

买考题、买考官都太小儿科了,既然横竖要买,还不如直接买通朝廷,让组织考试的人,给自己定个状元,直接到
位,省得麻烦。

所以在此之前,他已经通过熟人,买通了宫里能说得上话的几个太监,找好了主考官,考完后专门找出他的卷子,
给个状元了事。

当然,办这种事,成本非常巨大。据说钱同志花了两万两白银,按今天的人民币算,大致是一千二百万。

能出得起这个价钱,还要作弊,可见作弊之诚意。

两万白银,买个官也行了,钱谦益出这个价,就是奔着状元名头去的,但他万没想到,还有个比他更有诚意的。

在考试前,韩敬也很自信,因为他也出了钱,且打了包票,必中状元。

可是卷子交上去后,他却得到了一个让人震惊的消息----他的卷子被淘汰了。

淘汰是正常的,要真有水平,就不用出钱了。

可问题是,人找了钱出了,怎么能收钱不办事呢?

韩敬在朝廷里是有关系的,于是连夜找人去查,才知道他的运气不好。偏偏改他卷子的人,是没收过钱的,看完卷
子就怒了,觉得如此胡说八道的人,怎么还能考试,就判了落榜。

落榜不要紧,找回来再改成上榜就行。

\section[\thesection]{}

韩敬同学毕竟手眼通天,找到了其他考官,帮他找卷子重新改。

可是找来找去,竟然没找到。后来才知道,因为那位考官太讨厌他的卷子,直接就给扔废纸堆里了,翻了半天垃
圾,才算把卷子给淘回来。

按常理,事已至此,重新改个上榜进士,也就差不多了,但韩敬同学对名次的感情实在太深,非要把自己的卷子改
成第一名。

但名次已经排定,且排名都是出了钱的(比如钱谦益),你要排第一,别人怎么办?

关键时刻,韩敬使出了绝招----加钱。

钱谦益找太监,出两万两,他找大太监,加价四万两,跟我斗,加死你!

四万两,大致是两千四百万人民币,出这个价钱,买个状元,无语。

更无语的,是钱谦益,出了这么多钱,都打了水飘,好在太监办事还比较地道,虽然没有状元,也给了个探花(第三
名)。

花这么多钱,买个状元,并不是吃饱了撑的。要知道,状元不光能当官,还能名垂青史。自古以来,状元都是最高
荣誉,且按规定,每次科举的录取者,都刻在石碑上,放在国子监里供后代瞻仰(现在还有),状元的名字就在首
位,几万两买个名垂青史,值了。

但钱谦益同志是不值的,虽说也是探花,但花了这么多钱,只买了个次品,心理极不平衡,跟韩敬同学就此结下梁
子。

韩敬是幸运的,也是不幸的,他虽然加了钱,买到了状元,却并不知道得罪钱谦益的后果。

因为钱同学虽然钱不够多,关系不够硬,却很能混。进朝廷后没多久就交了几个朋友,分别叫做孙承宗、叶向高、
杨涟、左光斗。

概括成一句话,他投了东林党。

万历末年,东林党是很有点能量的,而钱谦益也并不是个很大方的人,所以没过几年搞京察的时候,韩敬同志就因
为业绩不好,被整走了。

背负血海深仇的韩敬同志,终于等到了现在的机会,他大肆宣扬,应该追究钱谦益的责任。

但是说来说去,毕竟只是领导责任,经过朝廷审查,钱千秋免去举人头衔,充军,主考官(包括钱谦益)罚三个月工
资。

七年之后。

在周延儒和温体仁眼前的,并不是一起无足轻重的陈年旧案,而是一个千载难逢的机会。

\section[\thesection]{}

在很多史书里,这都是一段催人泪下的段落,强大且无耻的温体仁和周延儒,组成了恶毒的同盟,坑害了无辜弱小
的钱谦益。

我觉得,这个说法,如果倒转过来,是比较符合事实的。

首先,温体仁和周延儒无不无耻,还不好讲;钱谦益无辜,肯定不是。

温体仁之所以要整钱谦益,是个心态问题。

他是当年内阁首辅沈一贯的门生,钱谦益刚入伙的时候,他就是老江湖了,在朝廷里混迹多年,威信很高,而且他
还是礼部部长,专管钱谦益,居然还被抢了先,实在郁闷。周延儒则不同,他是真吃亏了,且吃的就是钱谦益的亏。

其实原本推选入阁名单时,排在第一的,应该是周延儒,因为他状元出身,且受皇帝信任,但钱谦益感觉此人威胁
太大,怕干不过他,就下了黑手,派人找到吏部尚书王永光,做了工作,把周延儒挤了。

其次,在当时朝廷里,强大的那个,应该是钱谦益。他是东林党领袖,一呼百应,从上到下,都是他的人,温体仁
周延儒基本算是孤军奋战。

当时的真实情况大致如此。

形势很严峻,但同志们很勇敢,在共同的敌人面前,温体仁、周延儒擦干眼泪,决定跟钱谦益玩命。

周延儒问温体仁,打算怎么干。

温体仁说,直接上疏弹劾钱谦益。

周延儒问,然后呢?

温体仁说,没有然后。

周延儒很生气,因为他认为,温体仁在拿他开涮,一封奏疏怎么可能干倒钱谦益呢?

温体仁没有回答。

周延儒告诉温体仁,先找几个人通通气,做些工作,搞好战前准备,别急着上疏。

第二天,温体仁上疏了。

就文笔而言,这封奏疏非常一般,主要内容是弹劾钱谦益主使作弊。也没玩什么写血书,沐浴更衣之类的花样,也
没做工作,没找人,递上去就完了。

然后他告诉周延儒,必胜无疑。

周延儒认为,温体仁是疯了。

辩论

事情的发展,跟周延儒想得差不多,朝廷上下一片哗然,崇祯也震惊了,决定召开御前会议,辩论此事。

辩论议题:浙江作弊案,钱谦益有无责任。

辩论双方:

正方,没有责任,辩论队成员:钱谦益、内阁大学士李标、钱龙锡、刑部尚书乔允升,吏部尚书王永光……(以下省略)

反方,有责任,辩论队成员:温体仁、周延儒(以下无省略)。

\section[\thesection]{}

崇祯元年(1628)十一月六日,辩论开始。

所有的人,包括周延儒在内,都认定温体仁必败无疑。

奇迹,就是所有人都认定不可能发生,却终究发生的事。

这场惊天逆转,从皇帝的提问开始:``你说钱谦益受贿,是真的吗?''

温体仁回答:是真的。

于是崇祯又问钱谦益: ``温体仁说的话,是真的吗?''

钱谦益回答:不是。

辩论陈词就此结束,吵架开始。

温体仁先声夺人,说,钱千秋逃了,此案未结。

钱谦益说:查了,有案卷为证。

温体仁说:没有结案。

钱谦益说:结了。

刑部尚书乔允升出场。

乔允升说:结案了,有案卷。

温体仁吃了秤砣:没有结案。

吏部尚书王永光出场。

王永光说:结案了,我亲眼看过。

礼部给事中章允儒出场

章允儒说:结案了,我曾看过口供。

温体仁很顽强:没有结案!

崇祯做第一次案件总结:``都别废话了,把案卷拿来看!''

休会,休息十分钟。

再次开场,崇祯问王永光:刑部案卷在哪里?

王永光说:我不知道,章允儒知道。

章允儒出场,回答:现在没有,原来看过。

温体仁骂:王永光和章允儒是同伙,结党营私!

章允儒回骂:当年魏忠贤在位时,驱除忠良,也说结党营私!

崇祯大骂:胡说!殿前说话,竟敢如此胡扯!抓起来!

这句话的对象,是章允儒。

章允儒被抓走后,辩论继续。

温体仁发言:推举钱谦益,是结党营私!

吏部尚书王永光发言:推举内阁人选,出于公心,没有结党。

内阁大臣钱龙锡发言:没有结党。

内阁大臣李标发言:没有结党。

崇祯总结陈词:推举这样的人(指钱谦益),还说出于公心!

二次休会

再次开场,钱龙锡发言:钱谦益应离职,听候处理。

崇祯发言:我让你们推举人才,竟然推举这样的恶人,今后不如不推。

温体仁发言:满朝都是钱谦益的人,我很孤立,恨我的人很多,希望皇上让我告老还乡。

崇祯发言:你为国效力,不用走。

辩论结束,反方,温体仁获胜,逆转,就此完成。

史料记载大致如此,看似平淡,实则暗藏玄机。

这是一个圈套,是温体仁设计的完美圈套。

\section[\thesection]{}

这个圈套分三个阶段,共三招。

第一招,开始辩论时,无论对方说什么,咬定,没有结案。

这个举动毫不明智,许多人被激怒,出来跟他对骂指责他。然而这正是温体仁的目的。

很快,奇迹就发生了,章允儒被抓走,崇祯的天平向温体仁倾斜。

接下来,温体仁开始实施第二步----挑衅。

他直接攻击内阁,攻击所有大臣,说他们结党营私。

于是大家都怒了,纷纷出场,驳斥温体仁。

这也是温体仁的目的。

至此,崇祯认定,钱谦益与作弊案有关,应予罢免。

第三阶段开始,内阁的诸位大人终于意识到,今天输定了,所以主动提出,让钱谦益走人,温体仁同志随即使出最
后一招----辞职。

当然,他是不会辞职的,但走到这一步,摆摆姿态还是需要的。

三招用完,大功告成。

温体仁没有魔法,这个世界上也没有奇迹,他之所以肯定他必定能胜,是因为他知道一个秘密,崇祯心底的秘密。

这个秘密的名字,叫做结党。

温体仁老谋深算,他知道,即使朝廷里的所有人,都跟他对立,只要皇帝支持,就必胜无疑,而皇帝最不喜欢的事
情,就是结党。

崇祯登基以来,干掉了阉党,扶植了东林党,却没能消停,朝廷党争不断,干什么什么都不成,所以最恨结党。

换句话说,钱谦益有无作弊,并不重要,只要把他打成结党,就必定完蛋。

事实上,钱谦益确实是东林党的领袖,所以在辩论时,务必不断挑事,耍流氓,吸引更多的人来骂自己,都无所谓。

因为最后的决断者,只有一个。

当崇祯看到这一切时,他必定会认为,钱谦益的势力太大,结党营私,绝不可留。

这就是温体仁的诡计,事实证明,他成功了。

通过这个圈套,他骗过了崇祯,除掉了钱谦益,所有的人都被他蒙在鼓里,至少他自己这样认为。

但事实可能并非如此,这场辩论的背后,真正的胜利者,是另一个人----崇祯。

其实温体仁的计谋,崇祯未必不知道,但他之所以如此配合,是因为这是一个千载难逢的机会。

当时的朝廷,东林党实力很强,从内阁到言官,都是东林党,虽说就工作业绩而言,比阉党要强得多,但归根结
底,也是个威胁,如此下去再不管,就管不住了。

现在既然温体仁跳出来,主动背上黑锅,索性就用他一把,敲打一下,提提醒,换几个人,阿猫阿狗都行,只要不
是东林党,让你们明白,都是给老子打工的,老实干活!

\section[\thesection]{}

当然明白人也不是没有,比如黄宗羲,就是这么想的,还写进了书里。

但搞倒了钱谦益,对温体仁而言,是纯粹的损人不利已,因为他老兄太过讨嫌,没人推举他,闹腾了半天,还是消
停了。

消停了一年,机会来了,机会的名字,叫袁崇焕。

画了一个圈,终于回到了原点。

之后的事,之前都讲了,袁督师很不幸,指挥出了点问题,本来没事,偏偏和钱龙锡拉上关系,就这么七搞八搞,
自己进去了,钱龙锡也下了水。

在很多人眼里,崇祯初年是很乱的,钱谦益、袁崇焕、钱龙锡、作弊、通敌、下课。

现在你应该明白,其实一点不乱,事实的真相就是这么简单,只有两个字----利益,周延儒的利益,温体仁的利
益,以及崇祯的利益。

钱谦益、袁崇焕、还有钱龙锡,都是利益的牺牲品。

而这个推论,有一个最好的例证。袁崇焕被杀掉后,钱龙锡按规定,也该干掉,死刑批了,连刑场都备好,家人都
准备收尸了,崇祯突然下令:不杀了。

关于这件事,许多史书上都说,崇祯皇帝突然觉悟。

我觉得,持这种观点的人,确实应该去觉悟一下,其实意思很明白,教训教训你,跟你开个玩笑,临上刑场再拉下
来,很有教育意义。

周延儒和温体仁终究还是成功了,崇祯三年(1630)二月,周延儒顺利入阁,几个月后,温体仁入阁。

温体仁入阁,是周延儒推荐的,因为崇祯最喜欢的,就是周延儒。但周兄还是很讲义气,毕竟当年全靠温兄在前面
踩雷,差点被口水淹死,才有了今天的局面,拉兄弟一把,是应该的。

其实就能力而言,周延儒和温体仁都是能人,如果就这么干下去,也是不错的,毕竟他们都是恶人,且手下并非善
茬,换个人,估计压不住阵。

但所谓患难兄弟,基本都有规律,拉兄弟一把后,就该踹兄弟一脚了。

最先开踹的,是温体仁。

钱龙锡被皇帝赦免后,第一个上门问候的,不是东林党,而是周延儒。

周兄此来的目的,是邀功。什么皇上原本很生气,很愤怒,很想干掉你,但是关键时刻,我挺身而出,在皇帝面前
帮你说了很多好话,你才终于脱险云云。

这种先挖坑,再拉人,既做婊子,又立牌坊的行为,虽很无聊,却很有效,钱龙锡很感动,千恩万谢。

\section[\thesection]{}

周延儒走了,第二个上门问候的来了,温体仁。

温体仁的目的,大致也是邀功,然而意外发生了。

因为钱龙锡同志刚从鬼门关回来,且经周延儒忽悠,异常激动,温兄还没开口,钱龙锡就如同连珠炮般,把监狱风
云,脱离苦海等前因后果全盘托出。

特别讲到皇帝愤怒,周延儒挺身而出,力挽狂澜时,钱龙锡同志极为感激,眼泪哗哗地流着。

温体仁安静地听完,说了句话。

这句话彻底止住了钱龙锡的眼泪:``据我所知,其实皇上不怎么气愤。''

啥?不气愤?不气愤你邀什么功?混蛋!

所以钱龙锡气愤了。类似这种事情,自然有人去传,周延儒知道后,也很气愤----我拉你,你踹我?

温体仁这个人,史书上的评价,大都是八个字:表面温和,深不可测。

其实他跟周延儒的区别不大,只有一点:如果周延儒是坏人,他是更坏的坏人。

对他而言,敌人的名字是经常换的,之前是钱谦益,之后是周延儒。

所以在搞倒周延儒这件事上,他是个很坚定,很有毅力的人。

不久之后,他就等到了机会,因为周延儒犯了一个与钱谦益同样的错误----作弊。

崇祯四年,周延儒担任主考官,有一个考生跟他家有关系,就找到他,想走走后门,周考官很大方,给了个第一名。

应该说,对此类案件,崇祯一向是相当痛恨的,更巧的是,这事温体仁知道了,找了个人写黑材料,准备下点猛
药,让周延儒下课。不幸的是,周延儒比钱谦益狡猾得多。听到风声,不慌不忙地做了一件事,把问题搞定了,充
分反映了他的厚黑学水平。

他把这位考生的卷子,交给了崇祯。

应该说,这位作弊的同学还是有点水平的,崇祯看后,十分高兴,连连说好,周延儒趁机添把火,说打算把这份卷
子评为第一,皇帝认为没有问题,就批了。

皇帝都过了,再找麻烦,就是找抽了,所以这事也就过了。

但温体仁这关,终究是过不去的。

崇祯年间的十七年里,一共用了五十个内阁大臣,特别是内阁首辅,基本只能干几个月,任期超过两年的,只有两
个人。

第二名,周延儒,任期三年。

第一名,温体仁,任期八年。

温首辅能混这么久,只靠两个字,特别。

特别能战斗,特别能折腾。

\section[\thesection]{}

在此后的一年里,温体仁无怨无悔、锲而不舍地折腾着,他不断地找人黑周延儒,但皇帝实在很喜欢周首辅,虽屡
败屡战,却屡战屡败,直到一年后,他知道了一句话。就是这句话,最终搞定了千言万语都搞不定的周延儒。

全文如下: ``余有回天之力,今上是羲皇上人。''

前半句很好懂,意思是我的能量很大。后半句很不好懂,却很要命。

今上,是指崇祯,所谓羲皇上人,具体是谁很难讲,反正是原始社会的某位皇帝,属于七十二帝之一,就不扯了,
而他的主要特点,是不管事。

翻译过来,意思是,我的能量很大,皇上不管事。

这句话是周延儒说的,是跟别人聊天时说的,说时旁边还有人。

温体仁把这件事翻了出来,并找到了证人。

啥也别说了,下课吧。

周延儒终于走了,十年后,他还会再回来,不过,这未必是件好事。

朝廷就此进入温体仁时代。

按照传统观点,这是一个极其黑暗的时代,在无能的温体仁的带领下,明朝终于走向了不归路。

我的观点不太传统,因为我看到的史料告诉我,这并非事实。

温体仁能够当八年的内阁首辅,只有一个原因----他能够当八年的内阁首辅。

作为内阁首辅,温体仁具备以下条件:首先,他很精明强干。据说一件事情报上来,别人还在琢磨,他就想明白
了,而且能很快做出反应。其次,他熟悉政务,而且效率极高,还善于整人(所以善于管人)。

最后,他不是个好人。当然,对朝廷官员而言,这一点在某些时候,绝对不是缺点。

估计很多人都想不到,这位温体仁还是个清官,不折不扣的清官,做了八年首辅,家里还穷得叮当响,从来不受
贿,不贪污。

相对而言,流芳千古的钱谦益先生,就有点区别了,除了家产外,也很能挣钱(怎么来的就别说了),经常出没红灯
区。六十多岁了,还娶了柳如是。明朝亡时,说要跳河殉国,脚趾头都还没下去,就缩了回来,说水冷,不跳了,
就投降了清朝。清朝官员前来拜访,看过他家后,发出了同样的感叹:你家真有钱。

\section[\thesection]{}

温体仁未必是奸臣,钱谦益未必是好人。不需要惊讶,历史往往跟你所想的并不一样。英雄可以写成懦夫,能臣可
以写成奸臣,史实并不重要,重要的是,谁来写。

温体仁的上任,对崇祯而言,不算是件坏事。就人品而言,他确实很卑劣,很无耻,且工于心计,城府极深,但要
镇住朝廷那帮大臣,也只能靠他了。

应该说,崇祯是有点想法的,毕竟他手中的,不是烂摊子,而是一个烂得不能再烂的摊子。边关战乱,民不聊生,
政治腐败,朝廷混乱,如此下去,只能收摊。崇祯同志一直很担心,如果在他手里收摊,将来下去了,没脸见当年
摆摊的朱重八(后来他用一个比较简单的方法办到了)。

所以执政以来,他干了几件事,希望力挽狂澜。

第一件事,就是肃贪。

到崇祯时期,官员已经相当腐败,收钱办事,就算是好人了。对此,崇祯非常地不满,决心肃贪。

问题在于,明朝官场,经过二百多年的磨砺,越来越光,越来越滑。潜规则、明规则,基本已经形成一套行之有效
的规章,大家都在里边混,就谈不上什么贪不贪了,所谓天下皆贪,即是天下无贪。

当然,偶尔也有个把人,是要突破规则,冒冒头的。

比如户部给事中韩一良,就是典型代表。

当崇祯下令整顿吏治时,他慷慨上书,直言污秽,而且还说得很详细,什么考试作弊内幕,买官卖官内幕,提成、
陋规等等。为到达警醒世人的目的,他还坦白,自己身为言官,几个月之内,已经推掉了几百两银子的红包。

崇祯感动了,这都什么年月了,还有这样的人啊,感动之余,他决定在平台召开会议,召见韩一良及朝廷百官,并
当众嘉奖提升。

皇帝很激动,后果很严重。

因为韩一良同志本非好鸟,也没有与贪污犯罪死磕到底的决心,只是打算骂几句出出气,没想到皇帝大人反应如此
强烈,无奈,事都干了,只能硬着头皮去。

在平台,崇祯让人读了韩一良的奏疏,并交给百官传阅,大为赞赏,并叫出韩一良,提升他为都察院右佥都御史。

原本只是七品,一转眼,就成了四品。

我研读历史,曾总结出一条恒久不变的规律----世上的事,从没有白给的.

\section[\thesection]{}

韩一良同志还没高兴完,就听到了这样一句话:``此文甚好,希望科臣(指韩一良)能指出几个贪污的人,由皇帝惩
处,以示惩戒。''

说话的人,是吏部尚书王永光。

王永光很不爽,自打听到这封奏疏,他就不爽了,因为他是吏部尚书,管理人事,说朝廷贪污成风,也就是说他管
得不太好,所以他决定教训韩一良同志。

这下韩御史抓瞎了,因为他没法开口。

自古以来,所谓集体负责,就是不负责,所以批评集体,就是不批评。韩御史本意,也就是批评集体,反正没有具
体对象,没人冒头反驳,可以过过嘴瘾。

现在一定要你说出来,是谁贪污,是谁受贿,就不好玩了。

但崇祯似乎很有兴趣,当即把韩一良叫了出来,让他指名道姓。韩一良想了半天,说,现在不能讲。

崇祯说,现在讲。

韩一良说,我写这封奏疏,都是泛指,不知道名字。

崇祯怒了:你一个名字都不知道,竟然能写这封奏疏,胡扯!五天之内,把名字报来!

事儿大了,照这么搞,别说升官,能保住官就不错,韩一良回去了,在家抓狂了五天,憋得脸通红,终于憋出了一
份奏疏。

很明显,韩一良是下了功夫的,因为在这份奏疏里,他依然没有说出名字,却列出了几种人的贪污行径,并希望有
关部门严查。当然,他也知道,这样是不过了关的,就列出了几个人----已经被处理过的人。

反正处理过了,骂绝祖宗十八代,也不要紧。

这封极为滑头的奏疏送上去后,崇祯没说什么,只是下令在平台召集群臣,再次开会。

刚开始的时候,气氛是很和谐的,崇祯同志对韩一良说,你文章里提到的那几个人,都已经处理了,就不必再提了。

然后,他又很和气地提到韩一良的奏疏,比如他曾经拒绝红包,达几百两之多的优秀事迹。

戏演完了,说正事:``是谁送钱给你的!说!''

韩一良同志懵了,但优秀的自律精神鼓舞了他,秉承着打死也不说的思想,到底也没说。

崇祯也很干脆,既然你不说,就不要干了,走人吧。

韩一良同志的升官事迹就此结束,御史没捞到,给事中丢了,回家。

然而最伤心的,并不是他,是崇祯。

\section[\thesection]{}

他不知道,自己如此坦白,如此真诚,如此想干点事,怎么连句实话都换不到呢?

这个问题,没人能回答

但要说他啥事都没干成,也不对,事实上,崇祯二年(1629),他就干过一件大事,且相当成功。

这年四月,刑部给事中刘懋上疏,请求清理驿站。

所谓驿站,就是招待所,著名的伟大的政治家、军事家、哲学家王守仁先生,就曾经当过招待所的所长。

当然,王守仁同志干过的职务很多,这是最差的一个。因为在明代,驿站所长虽说是公务员,论级别,还不到九
品,算是不入流,还要负责接待沿途官员,可谓人见人欺。所以一直以来,驿站都没人管。

但到崇祯这段,驿站不管都不行了。因为明代规定,驿站接待中央各级官员,由地方代管。

这句话不好理解,说白了,就是驿站管各级官员吃喝拉撒睡,但费用自负。

因为明代地方政府,并没有办公经费,必须自行解决,所以驿站看起来,级别不高,也没人管。

但驿站还是有油水的,因为毕竟是官方招待所,上面来个人没法接待,追究到底,还是地方官吃亏,所以每年地方
花在驿站上的钱,数额也很多。

而且驿站还有个优势,不但有钱,且有政策--摊派。

只要有接待任务,就有名目,就能逼老百姓,上面来个人,招待所所长自然不会自己出钱请人吃饭,就找老百姓
摊,你家有钱,就出钱,没钱?无所谓,你们要相信,只要是人,就有用处,什么挑夫、轿夫,都可以干。

其实根据规定,过往官员,如要使用驿站,必须是公务,且出示堪合(介绍信),否则,不得随便使用。

也就是说说。

到崇祯年间,驿站基本上就成了车站,按说堪合用完了,就要上交,但这事也没人管,所以许多人用了,都自己收
起来,时不时出去旅游,都用一用,更缺德的,还把这玩意当礼物,送给亲朋好友,让大家都捞点实惠。

鉴于驿站好处如此之多,所以但凡过路官员,无论何等妖魔鬼怪,都是能住就住,不住也宰点钱,既不住也不宰
的,至少也得找几个人抬轿子,顺便送一程。

比如我国古代最伟大的地理学家徐霞客,云游各地(驿站),拿着堪合四处转悠,绝对没少用。

\section[\thesection]{}

刘懋建议,整顿驿站,不但可以节省成本,还能减轻地方负担。但问题是,怎么整顿。

刘懋的方法很简单,一个字----裁。

裁减驿站,开除富余人员,减开支,严管介绍信,非紧急不得使用。按照他的说法,只要执行这项措施,朝廷一年
能省几十万两白银,且地方负担能大大减轻。

崇祯很高兴,同意了,并且雷厉风行地执行了。

一年之后,上报执行成果,裁减驿站二百余处,全国各省累计减少经费八十万两,成绩显著。

不久之后,刘懋就滚蛋了。

这世上,有很多事情,看上去是好事,实际上不是,比如这件事。

刘懋同志干这件事,基本是``损人不利己''。国家没有好处,地方经费节省了,也省不到老百姓头上,地方吃驿站
的那帮人又吃了亏,要跟他拼命,闹来闹去折腾一年,啥都没有,只能走人。

崇祯同志很扫兴,好不容易干了件事,又干成这副熊样,好在没有造成严重后果,反正驿站有没有无所谓,就这么
着吧。

事实上,如果他知道刘懋改革的另一个后果,估计就不会让他走了,他会把刘懋留下来,然后,砍成两截。

因为汇报裁减业绩的人,少报了一件事:之所以减掉了八十余万两白银的经费,是因为裁掉驿站的同时,还裁掉了
上万名驿卒。

崇祯二年(1629),按照规定,银川驿站被撤销,驿卒们统统走人。

一个驿卒无奈地离开了,这里已无容身之所。为了养活自己,他决定,去另找一份工作,一份更有前途的工作。

这个驿卒的名字,叫做李自成。

换句话说,崇祯上台以后,是很想干事的。但有的事,干了也白干,有的事,干了不如不干,朝廷就是这么个朝
廷,大臣就是这帮大臣,没法干。

所以他很失落,很伤心,但更伤心的事,还在后头。

因为上面这些事,最多是不能干,但下面的事情,是不能不干。

崇祯四年(1631),辽东总兵祖大寿急报:被围。

他被围的地方,叫做大凌河。

一年前,孙承宗接替了袁崇焕的位置,成为蓟辽总督。虽然老头已经七十多了,但实在肯靠谱,上任不久,就再次
巡视辽东,转了一圈,回来给崇祯打了个报告。

\section[\thesection]{}

报告的主要内容是,关锦防线非常稳固,但锦州深入敌前,孤城难守,建议在锦州附近的大凌河筑城,扩大地盘,
稳固锦州。

这个报告体现了孙承宗同志卓越的战略思想。七年前,他稳固山海关,恢复了宁远,稳固宁远,恢复了锦州,现
在,他稳固锦州,是打算恢复广宁,照这么个搞法,估计是想稳固沈阳,恢复赫图阿拉,把皇太极赶进河里。

想法好,做得也很好,被派去砌城的,是总兵祖大寿、副总兵何可纲。

在袁崇焕死前,曾向朝廷举荐过三个人,分别是赵率教、祖大寿、何可纲。

他在举荐三人时,曾说过:``臣选此三人,愿与此三人共始终,若到期无果,愿杀此三人,然后自动请死。''

袁崇焕的意思是,我选了这几个人,工作任务要是完不成,我就先自相残杀,然后自杀。

这句话比较准,却也不太准。

因为袁崇焕还没死,赵率教就先死了。袁崇焕死的时候,祖大寿也没死,逃了。

现在,只剩下了祖大寿和何可纲,他们不会自杀,却将兑现这个诺言的最后一部分----自相残杀。

投降

带了一万多人,祖大寿跟何可纲去砌砖头了,砌到一半,皇太极来了。

皇太极之所以来,也是不能不来,因为当他发现明军在大凌河筑城时,就明白,孙老头又使坏了。

如果让明军在大凌河站住脚,锦州稳固,照孙承宗的风格,接下来必定是蚕食,慢慢地磨,今天占你十亩地,站住
了,明天再来,还是十亩,玩死你。

所以,他亲率大军,前往大凌河,准备拆迁。

但祖大寿辛苦半年多,自然不让拆,早早收工,把人都撤了回来,准备当钉子户。

然而,当皇太极气喘吁吁地赶到大凌河城下时,却又不动手了。

他只是远远地扎营,然后在城下开始挖沟。

皇太极很卖力,在城下呆了一个多月,也不开打,只是围城挖沟,挖沟围城,经过不懈努力,竟然沿着大凌河城挖
了个圈,此外,他还很有诚意地找来木头,围城修了一圈栅栏。

如此用功,只因害怕。

鉴于此前他在宁远、锦州吃过大亏,看见城头的大炮就哆嗦,所以决定,不攻城,只围城,等围得差不多了,再攻。

\section[\thesection]{}

对于这一举动,祖大寿嗤之以鼻,并不害怕,事实上,得知围城后,他还派人在城头喊话:``我军粮草充足,足以
支撑两年,你奈我何?''

皇太极听到了,并不生气,想了个很绝的回答,又派了个人去回话:``那就困你三年!''

所谓粮食支撑两年,自然是吹牛的,几天倒还成,而且祖大寿当时手下的部队,有一万多人,虽然皇太极的兵力是
两万多,但以他的水平,守半个月没问题。

更重要的是,他还有个指望----援军。

大凌河被围的消息传来后,孙承宗立刻开始组织援军,先派了几拨小部队,由吴襄带头,往大凌河奔。据说后来的
著名人物吴三桂也在部队里。

可惜,这支部队刚到松山,就被打回去了。

皇太极早有准备,因为他的部队,攻城不在行,打野战没问题,反正这破楼拆定了,来几拨打几拨!

孙承宗也很硬,这城楼修定了,就是用人挤,也要挤进去!

崇祯四年(1631),最大规模的援军出发了。

这支援军由大将张春率领,共四万余人,奔袭大凌河,列阵迎敌。

大客户上门,皇太极自然亲自迎接,到阵前一看,傻眼了。

统帅张春是个不怎么出名,却有点水平的人,他千里迢迢赶到大凌河,却摆出防守的阵势,收缩兵力,广建营寨,
然后架起大炮,等皇太极来打。

因为就双方军事实力而言,跟皇太极玩骑兵对砍,基本等于自杀。摆好阵势,准备大炮,还能打几天。

这是个极为英明的抉择,可惜,还不够。

战斗开始,皇太极派出精锐骑兵,以左右对进战术,攻击张春军两翼。

但张春同志很有水平,阵势摆的很好,大炮打得很准,几轮下来,后金军队损失惨重。

在战场上,英明是不够的,决定战争胜负的,是实力。

进攻失败后,皇太极拿出了他的实力----大炮。

由于之前被大炮打得太惨,皇太极决定,开发新技术,造大炮。

经过刻苦偷学,后金军造出了自己的大炮,共三十门,虽说质量如何不能保证,至少能响。

\section[\thesection]{}

所以当巨大的轰鸣声从后金军队中传出时,张春竟然产生错觉,认为是自己的大炮炸膛,还派人去查,但残酷的事
实告诉他,敌人已经马刀换炮了。

但张春认定,无论如何,都要顶住,他亲自上阵督战,希望稳住阵脚。

这个愿望落空了。

为保证此战必胜,张春来的时候,还带上了一员猛将----吴襄。按原先的想法,吴将军是本地人,跟皇太极也打了
不少仗,熟悉情况。

应该说,这个说法是很对的,吴襄到底了解情况,一看仗打成这样,立马就跑了。

这种搞法极其恶心,并直接导致了张春的溃败。

明朝四万援军就此覆灭,而城内的祖大寿,基本可以绝望了。

但绝望的祖大寿不打算放弃,他决定突围。

突围的地点,选在南城,据他观察,南城敌人最为薄弱。

按祖大寿的想法,能突出去最好,突不出去就回来,也就是试试。但他万没想到,这一试,竟然解决了一个贝勒。

几天后,祖大寿发动突围,与后金军发生激战。

围困南城的,是皇太极的哥哥莽古尔泰,此人属于大脑很稀缺,四肢很发达类型,故被称为后金第一猛将(粗人代名
词),但这次,他遇上了更猛的祖大寿。

战斗非常激烈,祖大寿不愧为名将,带着城里的兵(并非关宁军)往死里冲,重创城南军队。

莽古尔泰感觉不对,便向皇太极请求援兵,但出乎意料的是,援兵竟然迟迟不到,莽古尔泰只能亲自督阵,用上所
部全部兵力,才挡住了祖大寿的突围,损失极为惨重。

莽古尔泰在四大贝勒里,排行第三(皇太极第四),被弟弟忽悠了,实在是气不过,所以他立即找到皇太极,说自己
损失过重,要求换防。

但皇太极压根不搭理他,莽古尔泰气不过,就把刀抽了出来,要砍皇太极,幸好被人拦住,才没出事。

搞笑的是,莽古尔泰同志回去后,居然怂了,且越想越怕,连夜都跑到皇太极那里承认错误。

皇太极倒也干脆,直接绑了关进牢房,不久后莽古尔泰就死了,死因不明。

这已经不是皇太极第一次耍诈了,他老人家虽然靠兄弟上台,却很信不过兄弟,按照他的想法,四大贝勒是没有必
要的,只要一个就够了。

为达到这一目的,每到打硬仗时,他都故意安排兄弟上阵,所谓``打死敌人除外患,打死自己除内乱''。

\section[\thesection]{}

比如崇祯三年,他听说孙承宗出兵关内四城,明知敌人很猛,就派二贝勒阿敏出征,被打了个稀里哗啦回来,趁机
撤了兄弟的职。

这次也差不多,如此说来,他大概还差祖大寿个人情。

但祖大寿的情况并未改变,他依然出不去,援军依然没法来,他依然不投降。

皇太极想招降祖大寿,很想,所以他费劲心机,先是往城里射箭,夹带信件,可是祖大寿的习惯很不好,总不回。

打了个把月,回信了。

这也是迫不得已,当初被围的时候,实在太过突然,按照明朝规定,军事部队执行任务时,身边只带三天干粮,现
在都三十天了,吃什么?

吃人。

大凌河城里,除了一万多军队外,还有两万多民工,几千匹马。

还好,没有粮食,吃马也能活,过了几十天,马吃完了。

没办法,只能吃人了。

当兵的开始吃民工,而且很有组织性,今天吃几个,就杀几个,挑好人,组织起来杀掉,分吃。

杀掉的人除了肉吃完外,连骨头都没剩,收起来当柴禾烧,用人骨烤人肉,真正是物尽其用。

就是这样,也没有投降。

但祖大寿已经到极限了,这样下去,没被后金军打死,也被城里的兵给吃了。所以他开始跟皇太极联系。联系的话
题很简单,两个字----投降。

皇太极知道城里很困难,很缺粮食,但他并不知道,祖大寿很坚韧。

祖大寿根本不想投降,他只是拖延时间,等待援军,但时间越来越长,援军却越来越少,于是,经过审慎地思考,
祖大寿做出了一个抉择,脱离苦海的抉择。

他与皇太极的使者进行了会谈,表示愿意投降。

崇祯四年(1631),祖大寿召集众将,宣布决定,投降。

所有的人都赞成,只有一个人反对----何可纲。

袁崇焕没有看错人,何可纲是一个靠得住的人,他严辞拒绝了祖大寿的提议,即使饿死,绝不投降!

袁崇焕也没有说错,他的魔咒最终应验了。

大家都投降,你不投降,就只有杀了你了。

祖大寿用行动,完成了袁崇焕诺言的最后部分:自相残杀。

他命令将拒不投降的何可纲推出城外,斩首示众。

\section[\thesection]{}

何可纲死前,并不惊慌,也不愤怒,只有鄙视,对叛徒祖大寿的鄙视。或许在他看来,这是最后的解脱,他终究没
有辜负袁崇焕的期望。

但他并不知道,坚持到底的人,并不只他一个,坚持的方式,除死外,还有其它方式,比死更痛苦的方式。

杀死何可纲后,祖大寿出城投降。

对于祖大寿同志,皇太极显示了最高程度的敬意,比对兄弟还客气,带着所有高级官员出营迎接,连跪拜礼都免
了,拉进大营后,管吃管喝,吃完喝完又送土特产,安排休息。

祖大寿很感动,随即提出,希望为后金立功,并拟出了一个方案:

锦州的守将,都是自己的手下,虽然现在有巡抚丘禾嘉坐镇,但只要能潜入城内,召集部下,就能杀掉丘禾嘉,攻
陷锦州。

皇太极同意了他的方案,给祖大寿凑了几百人,假装大凌河逃兵,护送他进入锦州,并派出多尔衮率领军队,隐藏
在锦州附近,等待祖大寿的信号。

信号是炮声,按照约定,祖大寿如顺利入城,应于十一月二日放炮,第二天动手,杀掉丘禾嘉,如一切顺利,就鸣
炮通知城外后金军,里应外合,攻克锦州。

两天后,在皇太极的注视下,祖大寿率领随从,出发前往锦州。

事情非常顺利,十一月一日,在后金军的暗中护送下,祖大寿顺利入城。

从某个角度看,皇太极是个生意人。

其实他并不相信祖大寿,所以劝降又放走,还客客气气地请客送礼,只是希望得到更大的回报。

十一月二日,当他听到锦州城内传来炮声时,他终于放心了,祖大寿传出入城信号,这次生意不会亏本了。

但是第二天,他没有听到炮声,很明显,祖大寿还没有动手。

第三天,也没有炮声。

就在他极度怀疑之刻,却收到了祖大寿的密信。

这封信是祖大寿从城中送出的,大致内容是说,由于出发仓促,且锦州军队很多,身边的人又少,暂时无法动手,
过两天再说。

既然如此,就多等两天。

两天,没信。

又两天,还没信。

到第三个两天,终于有信了。

皇太极又收到了祖大寿的信,写得相当客气,首先感谢皇太极同志的耐心等待,然后诉苦,说锦州城内防布森严,
难以动手,希望皇太极继续等着,估计到来年,就能办这事了。

被人涮了。

\section[\thesection]{}

其实从开始,祖大寿就没打算投降,堂堂大明总兵,怎么能投降呢?

但不投降就出不去,所以他决定,投个降,先出去。

但是何可纲反对。

此时,祖大寿有两种选择,第一,当着大家告诉何可纲,我们不是投降,是忽悠皇太极的,等出去后,我们就找个
机会跑路,回家洗了睡。

但这么干,难保不被人举报,保密起见最好别讲。且何可纲本是个二杆子,要死就死,投降就投降,投什么假降?

第二;杀了他。

只能这样。

于是何可纲死去了,祖大寿活下来,为了同一个目标。

事实上,祖大寿回到锦州后,啥都没干,就说自己跑回来了,继续一心一意地镇守锦州,坚决打击皇太极。

但刚涮完人家,就不认账,实在太过缺德,所以他在十一月二日的时候,还是按约定放了几炮,就当是给皇太极同
志留个纪念,说声拜拜。

至于送信解释情况,说自己暂时无法下手,倒也并非客气,实在是没办法,因为他的许多部下和亲属,还在皇太极
那边,自己跑了,还不客气客气,就扯淡了。所以这几封信的意思也很明确,就是说我虽然骗了你,但你也消消
气,别把事情做绝,将来没准还能合作。

当然,关于这件事,也有争议说祖大寿同志不是诈降,是真降,只不过回锦州后人手不足没法下手,所以才没干。
这种说法是不太靠谱的,因为很快,他就接受了锦州防务,镇守锦州,要多少人手有多少人手,也没干。

袁崇焕终究没有看错人。

但这件事情最奇特的地方,既不是祖大寿忽悠,也不是皇太极被忽悠,而是崇祯。

锦州守将,巡抚丘禾嘉是一个极其谨慎的人,虽然祖大寿没说实话,但他已多方查证,确认了祖大寿的投降,并且
写成了报告,上报崇祯。奇怪的是,报告送上去了,崇祯也看了,却没有任何反应,压根就没理这事,依然委任祖
大寿镇守锦州。

在这世上混,大家都不容易,睁只眼闭只眼算了吧。

最倒霉的反倒是孙承宗。他开始砌墙的时候,很多人就不服气,现在墙没砌好,就给人拆了,还收拾了施工队,于
是又是一片口水铺天盖地而来,孙承宗比较识趣,一个月后就辞职走人了。

\section[\thesection]{}

历经三朝风云,关宁防线的构架者,袁崇焕、祖大寿的提拔者,忠诚的爱国者,力挽狂澜的伟大战略家孙承宗,结
束了。

但这并不是他的终点,七年之后,他将在另一个舞台上,演出他人生最辉煌的一刻,以最壮烈的方式。

意外的意外

大凌河失陷了,皇太极走了,孙承宗也走了,这就是崇祯四年大凌河之战的结果。

但还有一个结果,是很多人并不知道,也没有料到的。

而这个结果的出现,和袁崇焕同志有莫大的关系。

袁崇焕杀掉毛文龙后,皮岛的局势很稳定,过了一年,就开始闹事。

闹事的根本原因,还是毛文龙,因为这位兄弟太有才能,以致于他在岛上的时候,大肆招兵,不但招汉人,还招满
人。

毕竟不管汉人满人,都认钱,而且满人作战勇猛,更好用,加上毛文龙会忽悠,越招越多,许多关外的人还专程坐
船来参军,到最后竟然有上千人。

但毛文龙死后,继任的人能力差点,没法控制局面,就兵变了,先是士兵互砍,然后是将领互砍,最后总兵黄龙专
程带兵上岛,才算把事镇住。

但这件事一闹,许多人都不想在岛上呆了。其中有两个人,这两个人是孔有德和耿仲明。

但到底去哪里,还是个问题,这二位仁兄都是山东人,原先还是矿工,出来闯关东,现在闯不下去,一合计,还是
回老家。

当然,回去挖矿是不能的,既然是兵油子,还是当兵合算,找来找去,听说登莱巡抚孙元化那里缺人,就去了。

孙元化,明代伟大的科学家,徐光启的学友,特长是炸药学、弹道学,简而言之,是搞大炮的。

据说这人不但精通物理、化学,还懂葡萄牙语,当年还上过葡萄牙火炮培训班,属于放炮专家。

当时他正跟葡萄牙人搞科学试验(造大炮),手下缺人,孔有德带人跑过来,十分之高兴,当即就把人给收编了。

其实孙先生虽说致力于科学研究,也曾打过仗,之前还曾当过宁远副使,给袁崇焕答打过工,也见过世面。可惜,
知识分子就是知识分子。

他并不知道,所谓孔有德、耿仲明,属于有奶便是娘型,是典型的兵油子,给钱就开工,不给钱就打老板,招这么
俩员工,只好认倒霉。

\section[\thesection]{}

其实刚开始的时候,这两位矿工兄弟还是很听话的,也服管,估计换了老板,也想好好干两天。

然而意外发生了。

祖大寿在大凌河筑城,被人围攻,朝廷四处调援兵,孙元化归孙承宗管,孙承宗找他要兵,他就把孔有德派去了。

孔有德很听命,立马就出发,前去拯救祖大寿。

走到半路,意外的意外发生了。

因为此时已经是十月份(阴历),天开始下雪,孔有德估计是走得急了点,不知是粮食没带够,还是当兵的想开小
灶,反正是几个人私自到老百姓家打猎,把人家里的鸡给吃了。

吃完了,被人发现了。

吃了就吃了吧,并非什么大事,大不了赔几只。

可问题是,当地的老百姓比较彪悍,且没说赔鸡,把人抓住以后,先修理了一顿,打得很惨。

消息传上去,当即炸锅,孔有德怒了,这还了得,后金军老子都没怕过,怕老百姓?二话不说,索性抢你娘的。

问题是,抢完了怎么办,毕竟大明是法制社会,犯了法,是要杀头的,所以孔有德破罐子破摔,反了。

孔有德同志原本是挖矿的,也没什么政治目标,更不打算替天行道,但既然反了,替天抢一把还是要的。

他带领部队,开始沿路抢劫。

此时,得到消息的孙元化急得不行,连忙找来山东巡抚余大成商量对策,谈来谈去,谈出一个结果--招安。

想出这么个招,原因在于他们认定,孔有德的反叛是出于误会,只要把他拉回来,安慰安慰,没准再给几只鸡,就
能解决问题。

更重要的是,这件事如果追究起来,黑锅就背定了,趁着现在事情还不大,瞒报情况拉人回来,还能保住官位,所
以不能动武,只能招安。

事实证明,瞒报注定是要穿帮的。

孙元化派出使者,找到孔有德,告诉他,赶紧归队投降,否则就什么什么。

孔有德很害怕,当即表示愿意投降,前往登州接受整编。

孙元化很满意,坐在城里等着孔有德,几天后,孔有德顺利到达登州,干的第一件事,就是攻城。

孙元化同志毕竟是知识分子,他并不知道,像孔有德这种兵油子,本没有道德观念,算是无赖,而能镇得住他的,
也只有更无赖的无赖,比如毛文龙。

\section[\thesection]{}

而孙专家最多也就是个技术员,对孔有德而言,不欺负是白不欺负。

还好守军反应快,立即出城迎敌。

但就战斗力而言,双方差距实在太大,登州城里的部队,平时最多也就打打土匪,跟从皮岛来的孔有德相比,只能
算仪仗。

所以没过多久,部队就被孔有德军击溃,退回城内。

虽然失利,但大体还算不错,因为登州城有大炮,据城坚守,应该没有问题。

可惜孙元化同志疏忽了极为重要的一点----他忘记了一个人:耿仲明。

耿仲明还在城内,作为孔有德的铁杆、老乡、战友兼同事,如果不拉兄弟一把,是不地道的。

耿仲明很地道,所以他连夜打开了城门,放孔有德进城,登州沦陷了。

孙元化很有点骨气,听说叛军入城,就准备自杀,但手慢了点,导致自杀未遂,被俘。

孔有德到底是混社会的,讲点江湖道义,没有杀孙元化,只是把他扣作人质,同时,他又致信山东巡抚余大成,要
求和谈。

好在余大成还比较清醒,知道事情闹大了,当即上报朝廷,登州失陷。

崇祯大怒,搞这么大的事,现在才来汇报,干什么吃的!他马上下令,免去孙元化,余大成的职务,委派谢涟为信任
登莱巡抚,接替孙元化,平定叛乱。

很快,孔有德也得知了这个消息,他明白,只能一条路走到黑了。

但他对孙元化似乎很有感情,到这份上,都没动他一根指头,竟然给放了。

但他做梦都没想到,自己难得干了件好事,也能把孙专家害死。

因为这事从头到尾,孙专家的责任太大,所以孙元化千里迢迢投奔朝廷后,就被朝廷逮了,送到京城,审讯完毕,
竟然判了死刑,拉出去砍了。

现在的孔有德很麻烦,他虽然占据了登州,但也就是个县城,且还在明朝腹地,上天没路,下地没门,渡海没船,
基本是歇菜了。

但非常难得,孔有德同志很乐观,他非但没有走,还干起了大买卖,找来了当年的同事李九成、耿仲明,陈友时,
还拉上毛文龙的儿子毛承禄,并广泛招募各地犯罪分子,扩编军队。

更搞笑的是,他们还组织政府,开始封官,封到一半,发现没有官印,还专门抓了几个刻印章的,帮他们刻印,很
有点过日子的意思。

\section[\thesection]{}

当然,他们在百忙之中,没有忘记自己的主业----抢劫,原先只抢个把县,现在牛了,统筹抢劫,分兵几路,从登
州开始,沿着山东半岛去抢,搞得民不聊生。

崇祯决定,解决这个问题。

但新任巡抚谢涟刚到任,就发现,在围剿孔有德之前,他必须先突围。

孔有德同志手下这帮兵,打后金军,只能算是凑合,但打关内这帮人,实在是绰绰有余,谢涟到达莱州之后,就被
围了。

但孔有德攻城的水平明显是差点,双方陷入僵持,你进不来,我出不去。

朝廷倒真急眼了,听说新到的巡抚又被围住,立即增兵,两万多人,直奔莱州。

孔有德听说朝廷援兵到了,也不含糊,加班加点地攻城,现炒现卖,拉出了登州城里的大炮,猛轰城头,竟然轰死
了新到任的山东巡抚(谢涟是登莱巡抚)。

谢涟虽说打仗没谱,还是比较硬的,死撑,等援兵来。他等来的不是援兵,而是一个做梦也想不到的消息。

围城的孔有德派出了使者,交给他一封信,信中表示,希望谢大人开恩,愿意投降。

听明白了,不是要谢大人投降,而是要谢大人接受投降。

这是个比较搞笑的事,深陷重围还没投降,包围的人倒要投降了,鬼才信。

谢涟信了,因为形势摆在眼前,朝廷援兵即刻就到,孔有德是聪明人,投降是他仅存选择。

他决定亲自出城,接受投降。

谢大人到底还是知识分子,他不知道,孔有德同志虽然是个聪明人,却是个聪明的坏人,从他反叛那天起,就没打
算回头。

时候到了,孔有德张灯结彩,锣鼓喧天,亲自在城门迎接。谢巡抚很受感动,带着几个随从出城受降。

为示庄重,他还去找莱州总兵,让他一起出城。

总兵不去。

不但不去,还劝谢巡抚,最好别去。

跟谢涟不同,这位总兵,是从基层干起来的,比较了解兵油子的特点,认定有诈,坚持不去。

保住莱州,就此一举。

接下来的过程很有戏剧性,谢涟出城后,受到了孔有德的热情接待,手下纷纷上前,亲密地围住了谢巡抚,把他直
接拉倒了大营。

一进去,就变脸了。

孔有德的打算是,先把谢巡抚绑起来,当作人质,然后又把随同的一个知府拉到城下,逼他传话,让里面的人投降。

\section[\thesection]{}

这位知府表示配合,到城下,让喊话,就真喊了:

``我死后,你们要好好守城!(汝等固守)''

按常规,此时发生的事情,应该是贼兵极其愤怒,残忍地杀害了知府大人。

但事情并非如此,因为知府大人固然有种,但更有种的,是那位不肯出城的总兵。

他听说巡抚被人劫了,知府在下面喊话,二话不说,就让人装炮弹,看准敌人密集地区,开炮。

敌人的密集地,也就是知府大人所在地,几炮打下去,叛军死伤惨重,知府大人也在其中,壮烈捐躯。

虽然巡抚够傻,好在知府够硬,总兵够狠,莱州终究守住。

但孔有德还是溜了,赶在援军到来之前。

这么闹下去,就没完了,崇祯随即下令,出狠招,调兵。

照目前情况看,要收拾这帮人,随便找人没有效果,要整,就必须恶整。

所以,他调来了两个猛人。

第一个,新任山东巡抚朱大典,浙江金华人,文官出身,但此人性格坚毅,饱读兵书,很有军事才能。

但更猛的,是第二个。

此时的山东半岛,基本算孔有德主管,巡抚的工作,他基本都干,想怎么来怎么来,看样子是打算定居了。

而且此时他的手下,已经有四五万人,且很有战斗经验,对付一般部队,绰绰有余。

所以派来打他的,是特种部队。

崇祯五年(1632)七月,明军先锋抵达莱州近郊,与孔有德军相遇,大败之。

孔有德很不服气,决定亲自出马,在沙河附近布下阵势,迎战明军。

他迎战的,是明军先锋。明军先锋,是关宁铁骑,统领关宁铁骑的,是吴三桂。

猛胜朱大典者,吴三桂也。

虽然按年龄推算,此时的吴三桂,还不到二十,但已经很猛,只要开战就往前冲,连他爹都没法管,对付孔有德之
流,是比较合适的。

战斗的进程可以用一个词形容----杀鸡焉用牛刀。

关宁铁骑的战斗力,已经讲过了,这么多年来,能跟皇太极打几场的,也就这支部队。

而孔有德的军队,虽然也在辽东转悠,但基本算是游击队,逢年过节跟毛文龙出来打黑枪,实在没法比。

反映在战斗力上,效果非常明显。

\section[\thesection]{}

孔有德的军队一触即溃,被吴三桂赶着跑了几十里,死了近万人,才算成功逃走。

原本孔有德的战术,是围城打援,围着莱州,援军来一个打一个。

但这批援军实在太狠,别说打援,城都别围了,立马就撤。

莱州成功解围,但吴三桂的使命并未结束,他接下来的目标,是登州。

被彻底打怕的孔有德退回登州,在那里,他纠集了耿仲明、李九成、毛承禄的所有军力,共计三万余人固守城池,
他坚信,必定能够守住。

其实朱大典也这么想,倒不是孔有德那三万人太多,而是因为登州城太厚。

登州,是明代重要的军事基地,往宁远、锦州送粮食,大都由此地起航,所以防御极其坚固。

更要命的是,后来孙元化来了,这位兄弟是搞大炮的,所以他修城墙的时候,是按炮弹破坏力来算。

换句话说,平常的城墙,也就能抗凿子凿,而登州的城墙,是能扛大炮的,抗击打能力很强。

更麻烦的是,孙巡抚是搞理科的,比较较真,把城墙修得贼厚且不说,还充分利用了地形,把登州城扩建到海边,
还专门开了个门,即使在城内支持不住,只要打开此门,就能立刻乘船溜号,万无一失。

所以朱大典很担心,凭借目前手中的兵力,如果要硬攻,没准一年半载还打不下来。

按朱大典的想法,这是一场持久战,所以他筹集了三个月的粮食,准备在登州城过年。

到了登州,就后悔了,不用三个月,三天就行。

孔有德到底还是文化低,对于登州城的技术含量,完全无知。听说明军到来,跟耿仲明一商量,认为如果龟缩城
内,太过认怂,索性出城迎战,以示顽抗到底之决心。

这个决心,只维持了一天。

率军出城作战的,是跟孔有德共同叛乱的李九成,他威风凛凛地列队出城,摆好阵势,随即,就被干掉了。

明军出战的,依然是关宁铁骑,来去如风,管你什么阵势不阵势,就怕你没出来,出来就好办,骑兵反复冲锋,见
人就打,叛军四散奔逃,鉴于李九成站在队伍最前面(最威风),所以最快被干掉,没跑掉的全数被歼。

\section[\thesection]{}

此时城里的叛军,还有上万人,但孔有德明显对手下缺乏信心,晚上找耿仲明,毛承禄谈话,经过短时间磋商,决
定跑路。说跑就跑,三个人带着部分手下、家属、沿路抢劫成果,连夜坐船,从海边跑了。

按孔有德的想法,跑他个冷不防,这里这帮傻人不知道,还能顶会,为自己争取跑路时间。

然而意外发生了,他过高估计了自己手下的道德水准,毕竟谁都不傻,孔有德刚跑,消息就传了出去,而类似孔有
德这类黑社会团伙,只要打掉领头的,剩下的人用扫把都能干掉。

于是还没等城外明军动手,城里就先乱了,登州城门洞开,逃跑的逃跑,投降的投降,跳海的跳海,朱大典随即率
军进城,收复登州。

事情算是结了,但孔有德这帮人在山东乱搞了半年,不抓回来修理修理太不像话,所以将领们纷纷提议,要率军追
击孔有德。

但朱大典没有同意。

不同意出兵,是因为不需要出兵。

逃到海上的孔有德很得意,虽说登州丢了,但半年来东西也没少抢,地主当不成,还能当财主。

得意到半路,遇上个人,消停了。

他遇上的这个人,名叫黄龙。

孔有德跟黄龙算是老熟人,因为黄龙曾经当过皮岛总兵,还管过孔有德。

孔有德怕的人比较少,而黄龙就属于少数派之一,孔有德之所以投孙元化,就是因为黄龙太厉害,在他手下太难混。

在最不想见人的地方,最不想见人的时候,遇上了最不想见的人,孔有德很伤心。

老领导黄龙见到了老部下孔有德,倒也没客气,上去就打,孔先生当即被打懵,部下伤亡过半,连他的亲人都没幸
免(他抢劫是带家属的),纷纷堕海而亡。

但最不幸的还不是他,而是毛承禄。

这位仁兄先是老爹(毛文龙)被杀,朝廷给了个官,也不好好干,被孔有德拉下水搞叛乱,落到这般地步,而关键时
刻,孔有德不负众望,毅然抛弃了这位老上级的公子,把他丢给了黄龙。

而孔有德和耿仲明不愧干过海盗,虽说打海战差点,但逃命还凑合,拼死杀出血路,保住了性命。

毛承禄就不行了,被抓住后送到了京城,被人千刀万剐。

黄龙的战役基本上彻底摧毁了叛军,孔有德和耿仲明逃上岸的时候,已经是光杆司令了。山东叛乱就此结束。

这次叛乱历时半年,破坏很大,而最关键的是,叛乱造成了两个极为重要的结果--足以影响历史的结果。

\section[\thesection]{}

第一个是坏结果:鉴于生意赔得太大,既没钱,也没人了,回本都回不了。孔有德、耿仲明经过短时间思想斗争,
决定去当汉奸,投靠皇太极。

其实这两个人投降,倒也没什么,关键在于他们曾在孙元化手下混过,对火炮技术比较了解,且由于一贯打劫,却
在海上被人给劫了,很是气愤,不顾知识产权,无私地把技术转让给了皇太极。从此火炮部队成为了后金的固定组
成部分,虽说孔有德、耿仲明文化不高,学得不地道,造出来的大炮准头也差点,但好歹是弄出来了。

更重要的是,由于他们辛苦折腾半年,弄回来的本钱,连同家属,都被明军赶进海里喂鱼,亏了老本,所以全心全
意给后金打工,向明朝复仇。

一年后,他们找到了复仇的机会。

除锦州、宁远外,明朝在关外的重要据点,大都是海岛,这些海岛有重兵驻守,时不时出来打个游击,是后金的心
腹大患,其中实力最强的守岛人,叫做尚可喜。之前我说过,孔有德、耿仲明、尚可喜是山东老乡,且全都是挖矿
的,现在孔有德决定改行挖人,劝降尚可喜。

一边是国家利益,民族大义,一边是老乡、老同事,尚可喜毫不为难地做出了抉择----当汉奸。

当英雄很累,当汉奸很轻松。

第二个是好结果,经过这件事,崇祯清楚地认识到,关内的军队,是很废的,关外的军队,是很强的,所以有什么
麻烦事,可以找关外军队解决(比如打农民军)。

偶然的偶然

山东的叛乱是个麻烦事,但要看跟谁比,要跟西北比,就不算个事。

据说朱元璋当年建都的时候,曾经找人算过一卦,大致内容跟现在做生意的差不多,比如这笔生意能做多少年,有
什么忌讳等等。

据说那位算卦的半仙想了很久,说了八个字:

始于东南,终于西北。

朱元璋建都南京,就是东南,按照这句话的指示,最后收拾他的人,是从西北过来的。

这句话看起来很玄,实际上倒未必。这位半仙懂不懂算卦我不知道,但他肯定是懂历史的,自古以来,中原政权完
蛋,自己把自己折腾死的除外,大多数外来的什么匈奴、蒙古,都在西北一带。

\section[\thesection]{}

但就崇祯而言,肯定是不信的。因为对明朝威胁最大的,是后金。而后金的位置是东北,就算是被灭了,也是始于
东南,终于东北。

但事实告诉我们,算卦这种事,有时是很准的。

西北很早就有人闹事了,但原先并不大,最多就是几十个人,抢个商铺,拿几把菜刀,闹完后上山当匪,杀掉的最
高官员,也就是个知县,如果混得好,没准将来还能招安,当正规军。

到崇祯元年,事情闹大了。

整个陕西、甘肃一带,民变四起,杀掉知县,只能算起步了。个别地方还干掉了巡抚,而且杀完抢完且不散伙,经
常到处流窜,到哪抢哪。

这种团伙,史书上称之为流贼。

流贼的特点是,四处跑,抢完就走,打一枪换个地方。组织性不强。昨天抢完,今天就走,可以,昨天被抢,今天
加入抢别人,也可以。成员流动性很大,但都有固定领导团队。

当时的西北,类似这种团队有很多,优秀的团队管理者也很多。但久而久之,问题出现了,由于成员流动性太大,
且没有固定办公场所,团伙成员文化又低,天天跟着混,时间长了,很难分清谁是谁。

为妥善解决这个问题,团队首领们想出了一个绝招----取外号。

所以在崇祯元年,陕西巡抚呈交皇帝的报告上,有如下称呼:飞天虎、飞山虎、混天王、王和尚、黑杀神、大红狼、
小红狼、一丈青、上天龙、过天星。

全是外号。

取这样的外号,是很符合实际需要的。毕竟团队成员文化比较低,你要取个左将军、右都督之类的称号,他也不知
道是啥意思,而且这种外号,大都是神魔鬼怪,叫起来相当威风。

至于这上面提到的诸位神魔到底是谁,别问我,我也不知道。

鉴于该行当风险很大,且从业者很多,要是运气不好,刚入行,把外号取好就被干掉,也很正常。而且许多外号由
于过于响亮,使用率很高,经常是几个人共用一个外号,要搞清楚谁是谁,实在很难。

无论叫什么,姓甚名谁,其实都无所谓,你只需要知道,当时的西北,已经不可收拾。

按一般史书的说法,这种情况之所以出现,是因为明朝末年,朝廷腐败,经济萧条,贪官污吏,苛捐杂税数不胜
数,民不聊生,于是铤而走险。

这种说法,就是传说中的套话,虽说不是废话,也差不多。

因为事实并非如此。

\section[\thesection]{}

很多人并不知道,明朝末年的民间经济并没有萧条,比如东南沿海,经济实在太好,开生意做买卖,相当红火,大
家齐心协力,正在搞资本主义萌芽,萧什么条?

赋税也没多少,以往两百多年,官田的赋税,只有百分之十,民间地主的赋税,最多也就收百分之二十。后来开征
三饷也才到百分之四十。当然,个把地主恶霸除外。

西北之所以涌出这么多英雄好汉,只是因为崇祯运气不好,遇到了一件东西。

中庸有云:国之将兴,必有祯祥,国之将亡,必有妖孽。

其实遇到妖孽,倒也没什么,毕竟还有实体,实在不行,找人灭了它。

崇祯遇上的,叫做灾荒。

翻开史书,你会不禁感叹,崇祯同志的运气实在太差:崇祯元年,陕西旱灾。崇祯二年,陕西旱灾,崇祯三年,陕
西旱灾,崇祯四年,陕西旱灾…………

灾荒之后,没有粮食吃,就是饥荒。

没有粮食吃,就吃人。

对受灾的人而言,吃人,并非童话。

据说当时西北各地的小孩,是不能四处乱跑的,如果没看住,跑了出去,基本就算没了。

注意,不是失踪,是没了。

失踪的意思,是被拐卖了,没了的意思,是被吃了。

据说,当时还有人肉市场,具体干什么买卖,看名字就知道。

说这么多,只是想说,这并不是童话,也不是神话,而是真话。

既然有灾荒,朝廷为什么不赈灾呢?

答案很简单,没钱。

此前有个经济学家对我说,明朝灭亡的真正原因,是没钱。我表示同意,财政赤字太多,挣得没有花的多,最后垮
台。

但他看了看我,说:我说的没钱,不是没有收入,是没钱。

有什么区别吗?

然后,他讲了一个小时,再然后,我翻了一个月的经济学,明白了区别。

我很想从头到尾,把我明白的事情告诉你们。但如果这样做,我会很累,你们也会很累,所以我决定,用几句话,
把这个问题说清楚。

明朝灭亡,并非是简单的政治问题,事实上,这是世界经济史上的一个重要案例。

所谓没钱,是没有白银。

明朝,是当时世界上最先进的国家之一,到崇祯接班的时候,商品经济已经十分发达,而商品经济十分发达的标
志,就是货币。

明朝的货币,是白银。

\section[\thesection]{}

简单地说,没钱的意思,就是没有白银,没有白银,无论你有多少经济计划,有多少财政报表,都是胡扯淡。

举个例子,陕西受灾,朝廷估算,要赈灾,必须一百万两白银,但是就算你把皇帝的圣旨拿到陕西,也换不来一两
银子,因为没有白银,所以无法赈灾。

好了,下一个问题,为什么没有白银。

先纠正一下,不是没有白银,而是白银不够。

为什么白银不够?

这是个很复杂的经济学问题,我不太想讲,估计人也不太想听。但不讲似乎也不行,简单说两句。

用大家都能明白的话说,就是白银有限,朝廷用掉了一两白银,未必能挣回来一两,加上我国人民,素来以勤俭节
约闻名,许多人拿到真金白银,不喜欢花,要么存在家里,要么溶掉,做几个香炉、人像之类的,还能美化环境,
所以市场的白银越来越少。

更重要的是,明朝的商品经济实在太过发达,经济越发达,需要的白银就越多,可是白银就那么多,所以到最后,
白银就不够用了。这种现象,在经济学上有一个通称----通货紧缩。

我知道,有人会提出这样的问题--为什么不用纸币?

很好,如果你提出这个问题,说明你很聪明。

但我要告诉你,在你之前的六百多年,有人问过这个问题。这个人的名字,叫朱元璋。六百多年前,他就想到了这
个问题,所以开始发行纸币。

在经济学中,有这样一句谚语:棍棒打不垮经济理论。

这句话的通俗意思是,无论你多牛,都要照规矩来。

朱元璋就是牛人,也要按规矩来。虽然他发行了纸币,一千、一万都印过,可惜的是,几百年来,大家还是认白
银,就不认纸币,再牛都没用。

这个问题到此为止,多余的话就不说了,你只要知道,崇祯同志是想赈灾的,之所以赈灾不成,是因为没有钱,之
所以没有钱,是因为没有白银,之所以没有白银……

当然,之所以西北先闹起来,除去天灾、银祸外,还有点地方特色。西北一带,向来比较缺水,比较穷困,比较没
人理,外加地方官比较扯淡,所以这个地方的人,过得比较苦。

生活艰苦,饭都没处吃,自然没条件读书。

没条件读书,自然考不上功名,考不上功名,自然没官做。没官做,也得找事做。

而西北一带人,最主要的工作,就是当兵。

\section[\thesection]{}

生活艰苦,民风自然彪悍,当兵是最合适的工作。

除了当兵之外,还有一份更为合适的工作----驿站。

驿站虽说比较小,但好歹是官办的,也算是吃皇粮的,而且各省都有拨款,搞点潜规则,多少能捞点油水,养活自
己,是不成问题的。据统计,光是甘肃陕西,就有几万人指着驿站过日子。

崇祯二年(1629),驿站没了。

之前我说过,被裁掉了,裁掉它的,是一个叫做刘懋的好人。

崇祯同志的运气实在太差,灾荒、钱慌、又夺了人家的饭碗,如果不闹,就不正常了。

他不是故意的。

所有的一切,都是偶然。偶然的灾荒,偶然裁掉驿站,偶然的地点。如果其中任意一个偶然没有发生,也许就不会
有最后的灭亡。

可惜,全都偶然了。

我曾经百思不得其解,因此我认定,在这些偶然的背后,隐藏着一个必然,一个真正的,决定性的原因。就是这个
原因,导致了明朝的灭亡。

我想了很久,终于想出了这个最终的原因,四个字----气数已尽。

这个世界上的一切,大致都是有期限的。一个人能红两年,很可能是偶然的,能红十年,就是有道行的,能红二十
年,那是刘德华。

公司也一样,能开两年,很正常,能开二十年,不太正常,能开两百年的,自己去数。

封建王朝跟公司差不多,只开个几年就卷铺盖的,也不少。最多也不过三百年,明朝开了二百多年,够意思了。

抚战

当然,崇祯是不会这样想的,无论如何,他都要撑下去,否则将来到地下,没脸见开铺的朱元璋。

所以他派出了杨鹤。

杨鹤,湖广武陵人(湖南常德),时任都察院左副都御史。经朝廷一致推荐,杨鹤被任命为兵部侍郎,三边总督,接
替之前总督武之望的职务。

工作交接十分简单。应该说,基本不用交接,因为杨鹤到任的时候,武之望已经死了。不是他杀,是自杀。

武总督是个很有责任感的人,鉴于西北民变太多,估计回去也没什么好果子吃,索性自杀。

而杨鹤之所以接替这个职务,是因为一次偶然的谈话。

杨鹤是一个进步比较慢的人,在朝廷里混三十多年,才当上佥都御史,混成这样,全靠他那张嘴。

\section[\thesection]{}

皇帝喜欢魏忠贤,他骂魏忠贤;皇帝讨厌熊廷弼,他为熊廷弼辩护。想什么说什么,几起几落,该怎么来还怎么来。

崇祯元年,他被重新委任为御史,当时民变四起,大家都在商议对策。

有一次,几个人聚到一起,聊天。聊的就是这个,杨鹤就在其中。

杨鹤是都察院的,这事跟他本无关系,他之所以掺和进来,还是两个字----嘴欠。

反正是吹牛,不用动真格的,就瞎聊。这个说要打,那么说要杀,如此热闹,杨鹤终于忍不住了,他说,不能打,
也不能杀。

然后他提出了自己的理论----元气说。

在他看来,造反的人,说到底,也还是老百姓。如果杀人太多,就是损伤元气,国家现在比较困难,应该培养元
气,不能乱杀。

几句话,就把大家彻底说懵了,对于他的观点,大家有着相同的评价----胡说八道。

不杀人,怎么平乱?

这是一个不为绝大多数人接受的理论,不要紧,有一个人接受就行。

不久之后,崇祯知道了这个理论,十分高兴,召见了杨鹤。

好事一件接着一件。很快吏部主动提出,鉴于杨鹤同志的理论很有使用价值,正好前任三边总督武之望死了,正式
提名杨鹤同志升任该职务。

杨鹤不想去。原因很简单,本来就是吹吹牛的,压根不会打仗,去了干啥?被人打?

但是牛都吹了,外加吏部支持,皇帝支持,如此重任在肩,咬咬牙就去了。

可是杨同志不知道,吏部之所以支持他,是因为讨厌。皇帝之所以支持他,是因为省事。

和杨鹤不同,吏部的同志们都是见过世面的。知道平乱是要砍人的,砍人是要死人的,死人是要流血的。杨鹤这套
把戏,也只能忽悠人,为达到前事不忘、后事之师的效果,让后来的无数白痴书呆子明白,乱讲话要倒霉,才着力
推荐他去。

死在那边最好,就算不死,也能脱层人皮

相比而言,崇祯的用心是比较善良的。他之所以喜欢杨鹤,是因为杨鹤提出了很好的理论----省钱的理论。

不花钱,不杀人,不用军饷,不用调兵,就能平息叛乱,太省了。

就算是忽悠人的,最多把杨鹤拉回来砍了,很省成本,如此生意,不做白不做。

\section[\thesection]{}

就这样,一脑袋浆糊的杨鹤去陕西上任,至少在当时,他的自我感觉很好。

杨鹤理论之中,最核心的一条,叫做和气。

用他自己的话说,杀人是伤和气的。所以能救活一个,就是一个,毕竟参加民变的,原先就是民,

这个理论,一年前,应该是对的

杨鹤同志到任后,就发现不对了。

有一次,农民军进攻县城,被击退,抓住了几个俘虏,由杨鹤审问。但还没问,杨鹤就发现一件极为诡异的事----
他似乎见过这几个人

确实见过,阅兵的时候见过。

没错,这几个人曾经站在阅兵的队伍里,曾经是他的部下。

强,弱,之间

农民军的战斗力很强吗?

对于这个疑问,最好的答案,应该是个反问----农民军的战斗力怎么会强呢?

在中国历史上,造反这类活,从来都是被动式。闲着没事干,但凡有口饭吃,是不会有人造反的,成本高,门槛也
高。但遗憾的是,造反这份工作,除了成本、门槛高外,技术含量还高。

要知道,明朝参加这项活动的,主要是农民。农民的基本工作,是种地,基本工具,是锄头。

而阻止他们参与这项活动的,是明军士兵。士兵的基本工作,是杀人,基本工具,是刀剑。

所以在明末大多数情况下,几百个农民军跟几百个明军对战,是不太可能发生的。据史料记载,大部分情况,是几
万农民军,战胜了几百明军,或是几百农民军,搞定十几个看衙门的捕快。而更大多数情况,是几千明军追着几万、
甚至十几万农民军跑。

没办法,毕竟打仗是个技术活。圣贤曾经说过,把武器交给没有受过训练的民众,让他们去打仗,就是让他们送死。

没有训练,没有武器,没有兵法,没有指挥,就没有胜利。但杨鹤先生惊奇地发现,他面对的情况,是完全不同的。

西北的民军里,除了业余造反的以外,还有很多专业造反的人士----明军,而且数量很多。

他们精通战术,作战狡猾,懂得明军的弱点,非常难以对付,且数量是越来越多,民变越来越大。

出现此类情况,归根结底,原因就两个字----没钱。

之前我说过,朝廷没有钱。没有钱的结果,除了没钱赈灾外,还没钱发军饷。

\section[\thesection]{}

据统计,当时全国的部队,大致有上百万人,而能够按时领军饷的,只有辽东军区的十余万人。

而且就连辽东军,也不能保证按时发工资,拖几个月,也是经常的事。袁崇焕同志就曾经处理过相关事务。

辽东是前线,尚且如此,其他地方就别提了。西北一带,既然不是前线,自然没钱。有的人几年都没拿到工资,穷
得叮当响,据说连武器都卖了,只求换顿饭吃。没钱赈灾,老百姓吃苦,也没辙,没钱发饷,当兵的吃苦,就有辙
了。

兜里没钱,手里有刀,怎么办?

凉拌,抢!

情况就是如此,官兵越来越少,民军越来越多,局势越来越撑不住。

杨鹤面对的形势大致如此,大家都明白,就他不明白,等他明白了,跑也跑不掉了。如果换个会打仗的,能用兵
的,多少还能撑几天,但杨鹤同志的主要特长,是招抚理论,这就比较麻烦了。据说当时朝廷里,有些人开玩笑,
说杨鹤如果能撑一年,就倒着爬出去。

就当时的情况看,这位仁兄爬出去的可能性,大致是零。杨鹤同志的下岗日期,指日可待。

一年后,杨鹤向崇祯呈交了名单,在这份名单上,有这样十几个名字:神一魁、王左桂、王嘉胤、红狼、小红狼、
点灯子、过天星、独头虎……(以下略去XX字)

以上人等,全部归降。

这些人是干嘛地,看名字就能猜到,但这些人什么分量,估计你就不知道了。

在当时的起义军中,最能打的,就是神一魁。此人具体情况不详,但应该受过军事训练,作战十分强悍,属于带头
大哥级人物。

王左桂、王嘉胤,如果你不知道,那不怪你。对这二位兄弟,只提几句话就够了:当时,在王左桂的手下,有个小
头目,叫做李自成。王嘉胤营门口站岗的,叫做张献忠。

至于后面那几位,就不说了,说了也没人知道,你只要明白,他们都是当时一等一的牛人,随便一个摆出来,都能
搅得天翻地覆。

都投降了。

除这些人之外,当时陕西、甘肃境内几乎所有的农民军,都投降了。

他们投降的对象,就是那个一脑袋浆糊,啥也不懂,不会打仗的杨鹤。

奇迹就这样发生了,发生在所有人的眼前。

杨鹤不懂兵法,不熟军事,但他有一样别人没有的武器----诚意。

\section[\thesection]{}

作为一个不折不扣的好人,杨先生很有诚意地寻找叛军,很有诚意地进行谈判,很有诚意地劝说投降,最后,他的
诚意得到了回报。

事实证明,农民军之所以造反,并不是吃饱了撑的,只是因为吃不饱。现在既然朝廷肯原谅他们,给他们饭吃,自
然愿意投降,毕竟造反这事,要经常出差,东跑西跑风险太大。

而对于杨总督,他们也是很客气的,很有点宋江喜迎招安的意思。

比如神一魁投降,约好地点,杨鹤打开城门,派出群众代表,热烈欢迎。众多民军头目大部到场,在杨总督的率领
下,前往关帝庙,在关老爷面前,宣誓投降(关老爷靠得住)。

虽然此前双方素未谋面(可能在往城下射箭时看过几眼),但双方都表现出了相当的热情。特别是杨总督,获得了民
军的一致推崇,他们赶走了杨鹤的轿夫,坚持一定要亲自把他抬到总督府,并以此为荣。

一时间,西北喜讯接连,朝廷奔走相告,杨鹤跟各民军领袖的关系也相当好,逢年过节,还互相送礼,致以节日的
问候。

局面大好,大好。有效期,半年。

杨鹤同志读过很多书,干过许多工作,明白很多道理,但是他并不知道,从招抚的第一天开始,他就已经失败了。
因为有一个问题,他始终没弄明白。

正是这个问题,注定了他的悲惨结局。

这个问题是,他们为什么要造反?

答案是:为了活下去。

怎样才能活下去呢?

有钱,有粮食。

要说明这个问题,可以用一个三段论:

造反,是因为没钱、没粮食;投降,是因为有钱,有粮食。

杨鹤有钱,有粮食吗?

没有。

所以停止投降,继续造反。

在招降之前,杨鹤曾经认为,只要民军肯投降,事情就结束了,可是投降之后,他才明白,事情才刚开始。

光是神一魁的部队,就有三万多人,这么多人,怎么安置?

招来当兵,就别扯了,连自己手下那点人的军饷都解决不了,招来这些人,喝西北风?

赶回家种地,似乎也是白扯,年年灾荒,要能回家种地,谁还造反?

对于这个悖论,崇祯同志是知道的,也想了办法。

\section[\thesection]{}

他先找了几万两银子,安排发放。然后又从自己的私房钱(内库)里,拿出了十万两,交给杨鹤,让他拿去花。

应该说,这一招还是很有效果的,民军们拿到钱,确实消停了相当长的时间。

具体是多长呢?

我前面说过了,半年。

半年,把钱都花完了,自然就不投降了,该怎么着还怎么着,继续反!

为了活下去。

猛人出场

崇祯四年(1631),领了半年工资后,神一魁再次反叛,西北群起响应,而且这次阵势更大,合计有三十多万人。

搞到这个地步,朝廷极为不满,许多大臣纷纷上告。

杨鹤很委屈,他本来就不是武将。之所以跑来办这事,实在是被人弄来的,原来是吹吹牛而已,你偏认真。来了之
后,都没闲着,天天忙活这事,钱花完了,人家又反了,我有什么办法?

崇祯更委屈,原本看你吹得挺好,觉得你能办事,才把你派过去。这么信任你,你招降了人,我立马就给你十几万
两银子,连老子的私房钱都拿出来了,你把钱花完了,这帮人又反了,十万两都打了水飘,你干什么吃的?

杨鹤委屈,就写信给崇祯,说我本不想干,你硬要我干,我要招抚,也是没有办法。

崇祯委屈,就写了封命令:锦衣卫,把杨鹤抓起来。

崇祯四年(1631)九月,杨鹤被捕,后发配袁州。

鉴于杨鹤的黑锅实在太重,由始至终,朝廷没人替他说话。

例外总是有的。

命令传出后,一个山海关的参政主动上书,要求替杨鹤承担处罚。

如此黑锅都敢背,是不正常的,但这个人帮杨鹤背锅,就是再正常不过了。

这位参政,是杨鹤的儿子,叫做杨嗣昌。

崇祯没有理睬,杨鹤先生的命运未能改变,依然去了袁州。

帮父亲背锅,看起来,是一件微不足道的小事,却导致了两个重大后果。

从这份奏疏上,崇祯看到了一个忠于父亲的人。按照当时的逻辑,忠臣,必定就是孝子,所以他记住了杨嗣昌的名
字。他认定,此人将来必可大用。

而杨嗣昌背黑锅不成,父亲被发配了,对他而言,莫过于奇耻。从此,他牢牢记住了那些降而复叛的人,此仇,不
共戴天。

杨鹤离开了,但这场大戏刚刚开幕,真正的猛人,即将出场。

\section[\thesection]{}

一年前,招抚失败后,民军首领王左桂派出起义军,进攻军事重镇韩城,韩城派人去找杨鹤,告急。

杨鹤很急,因为他的政策是招抚,手中实在没有兵,但到这节骨眼上,就是自己拿菜刀,也不能不去了。

但他终究没有掌握菜刀技术,无奈,他想起了一个人。这个人的手上也没有兵,但杨鹤相信,这个人是有办法的。

第一个猛人登场,他的名字,叫做洪承畴。

洪承畴接到了求援的命令,从某种程度上说,这是个相当扯淡的命令,你是总督都没办法,我怎么办?

但他并未抱怨,召集了自己的下人和亲兵,并就地招募了一些人,踏上了前往韩城的道路。

这是文官、陕西参政洪承畴的第一次出征,这年,他三十七岁。

洪承畴,字彦演,号亨九。福建南安人。

根据记载,此人的家世,可谓显赫一时:

曾祖父洪以诜,字德谦,中宪大夫,太傅兼太子太师、武央殿大学士。

曾祖母林氏、一品夫人。

祖父洪有秩,资政大夫、兵部尚书兼都察院右副都御史。

祖母戴氏,夫人。

有这么一份简历,基本就可以吃闲饭了。

可惜,洪承畴没能吃闲饭,事实上,他连饭都吃不上。

因为所有的这些简历,都是后来封的,换句话说,是他挣回来的。

洪承畴出生时,他的父亲因为家境贫寒,外出打工去了,他的母亲虽然穷,却比较有文化,从小就教他读书写字。
洪承畴很聪明,据说7岁就能背三字经,这是很了不起的。比如说我,27的时候,还只能背人之初,性本善。

万历四十三年(1615年),洪承畴23岁,参加全省统考(乡试),他的成绩很好,全省第19名。

第二年,他到北京参加全国统考,成绩更好,全国第17名,二甲。

然后分配工作,他被分配到刑部。

这个结果对他而言,是比较倒霉的。

原因我说过,在明代,要想将来入阁当大学士,必须当庶吉士,进翰林院。以洪承畴的成绩,应该能进,可是偏就
没进。

此后的十几年,洪承畴混得还可以,当上了刑部郎中,又被外放地方,当了参政。

参政这个官,说大不大说小不小,通常是混到最后,光荣地退休。

\section[\thesection]{}

没考上翰林的进士,混饭吃的小参政,到历史留名,骂声不绝,余音绕柱的大人物,只是因为,他外放的地方,是
陕西。

刚去陕西的时候,洪承畴带了很多书,

所以洪承畴带兵去救韩城的时候,只是一个书生,他没有打过仗,也没有杀过人。

据说在世界上,有这样一种人,他们天生就会打仗,天生就会杀人。

这是事实,不是据说。

洪承畴是一个真正的天才,军事天才,他带着临时拼起来的家丁、仆人、伙夫,就这么上了战场,却没有丝毫的胆
怯。

面对优势敌军,他凭借卓越的指挥,轻易击败了起义军,斩杀五百余人,解围韩城。

在洪承畴人生中,有过无数次战役,有过无数个强大的对手,最重要的,是这一次。

这个微不足道的胜利,让洪承畴明白,他是多么的强大,强大到可以力挽狂澜,可以改变无数人的命运。

他要凭借着自己的努力,挽救这个末落的王朝,创造太平的盛世。

讽刺的是,他最终做到了,却是以一种他做梦也未曾想到的方式。

洪承畴是一个务实的人,具体表现在,他正确地意识到,杨鹤是一个蠢货。

招抚是没有用的,钱是不够用的,唯一有用的方式,是镇压。

来陕西上任之前,洪承畴带来了很多书。三十年以来,书,是他仅有的寄托。

战后,他丢掉了书,做出了一个新的抉择----开战。

奇迹就是这样发生的,此后的两个月里,洪承畴率领这支纯粹的杂牌部队,连战连胜,民军闻之色变,望风而逃。

在历史上,他的这支军队,有一个专门的称呼----``洪兵''。

洪承畴是文官,杨鹤也是文官,这是两个人的共同点,也是他们唯一的共同点。

对待民军,杨鹤是很客气的,投降前,他好言好语招抚,投降后,他好吃好喝招待。

而洪承畴的态度有点差别。投降前,他说,如果不投降,就杀掉你们;投降后,他说,你们投降了,所以杀掉你们。

对于这件事情,我始终很疑惑,读圣贤书,就读出这么个觉悟?

自古以来,杀人放火之类的事,从来没断过,但公认最无耻的事,就是``杀降'',人家都投降了,你还要干掉他,
太过缺德。

但更让我疑惑的是,这种缺德事,洪承畴同志非但干了,还经常干。

比如那位曾经围过韩城,被洪承畴打跑的王左桂,后来也投降了。洪承畴听说后,决定请他吃饭。

还没吃完,一群人冲进来,把王左桂剁了。

我始终觉得,这事干得相当龌龊,就算动手,起码也得等人家吃完饭。

\section[\thesection]{}

落在他手上的民军头领,不是抵挡到底被杀,就是不抵抗投降被杀。总之,无论抵抗到底,还是不抵抗到底,都得
被杀。

但事实告诉我们,在某些时候,这种方法是有效的,至少对某些人很有效。

这个某些人,是指张献忠之类的人。

关于张献忠的具体情况,这里先不讲;关于他后来有没有在四川干过那些事,也不讲;只讲一个问题----投降的次
数。

我曾经在图书馆翻过半个月的史料,查询张献忠先生投降的相关问题,我知道他是经常投降的,但我不知道,他能
经常到这个份上。

简单地说,他的投降次数,用一只手,是数不过来的,两只手都未必,而且他投降的频率也很高。有一次,从投降
到再反,只用了十几天。

这是难能可贵的。一般说来,投降之后,也得履行个程序,吃个饭,洗个澡,找个地方定居,以上工作全部忙完,
至少也得个把月。但张先生效率之高,速度之快,实在令人咂舌。

相比而言,李自成就好得多了。虽然他也投降,但还是很有几分硬气的,说不投降,就不投降,属于硬汉型人物。

大体而言,当时许多民军的行为程序是,起兵、作战、被官军包围,投降,走出包围圈,拿起武器,继续作战。此
类表演,基本都是固定节目,数不胜数。很快,你就会看到两个典型案例。

洪承畴跟杨鹤不同,他是一个现实主义者。在他看来,要彻底扭转形势,不能招抚,不能受降,只有一个办法----
赶尽杀绝。

这种方式的效果相当明显,短短几个月内,西北局势开始稳定,各路民军纷纷受挫,首领被杀。

他的优异表现得到了很多人的关注,包括崇祯。对他而言,高升是迟早的事。但他毕竟太年轻,资历太浅,还要继
续等。两个月后,一件事情的发生,缩短了洪承畴的等待时间。

崇祯四年(1631),估计是有心脏病,或是胆囊炎,起义军进攻延绥巡抚镇守城池的时候,这位巡抚大人竟然被活活
吓死。

没胆的人死了,就让有胆的人上,洪承畴接替了他的位置。

进步是没有止境的,又过了两个月,他的顶头上司杨鹤被抓了,总督的位置空了出来。没人能顶替,也没人愿意顶
替,除了洪承畴。

\section[\thesection]{}

崇祯四年(1631)十月,洪承畴正式接任三边总督。

噩梦开始了。

当时的起义军,已经遍布西北,人数有几十万。虽说其中许多都是凑人数的,某些部队还携家带口,什么八十老
母,几岁小孩都带上,但看起来,确实相当吓人。

比如宁夏总兵贺虎臣,有一次听说起义军到境内观光,立即带了两千精兵,准备出战。到地方后,他看到了起义军
的前锋队伍。

然而他没有动手,就在那里看着,静静地看着,看了会,就走了。

因为他始终没有看到这支队伍的尾巴。

这是一列长队,从前到后,长几十里

对这样的起义军,看看就行了,真要动手,就傻了。

问题在于,当时的西北,到处都是这样的队伍,穿街过巷,比游行还壮观,见着就发怵。

然后,洪承畴来了

在这个世界上,洪承畴害怕的东西,大致还不多。

在给朝廷的报告里,他天才地解决了这个问题:

西北民变,人数虽多,但大都是胁从,且老幼俱在,并无战力,真正精壮之人,十之一二而已,击其首,即可大破
之。这意思是,虽然闹事的人多,但真正能打仗的,十个人里面,最多也就一两个,把这几个人干掉,事情就结了。

实践证明,他的理论非常正确,所谓几十万义军,真正能打仗的,也就几万人而已。

而这几万人中,最强悍的,是三个人:王左桂、王嘉胤、神一魁。只要除掉这三个人,大局必定。

这三个人中,王左桂已经被杀掉了,所以下一个目标,是王嘉胤。

然而就在此时,洪承畴得知了一个惊人的消息----王嘉胤死了。

王嘉胤是被杀的,杀掉他的人,是他的部下。

他的部下之所以要杀他,实在是被人逼得没办法。

逼他们的人,叫做曹文诏。

第二个猛人

对曹文诏这个人,洪承畴曾经有过一个评价:世间良将,天下无双。

曹文诏,山西大同人,和洪承畴不一样,他没有履历,没读过书,没有背景,出人头地之前,他只是个小兵。

十年前,他在一个人的手下当兵,跟着此人去了辽东。这个人叫做熊廷弼。

九年前,广宁兵败,明军溃败,他没有逃跑,而是坚持留了下来,见到了他第二个上司----孙承宗。

六年前,孙承宗走了,他还是留了下来,此时,他已经当上了游击,而他的新上司,就是袁崇焕。

\section[\thesection]{}

两年前,他跟着袁崇焕到了京城,守护北京,结果袁崇焕被抓,他依然留了下来。

一年前,他跟随孙承宗前往遵化,在那里,他奋勇作战,击退后金贝勒阿敏,并最终收复关内四城。

然后,他来到了西北。

对于这个人,我想就没必要多说了,从熊廷弼、孙承宗到袁崇焕,他都跟过,从努尔哈赤、皇太极到阿敏,他都打
过。

什么世面都见过,什么牛人都跟过,现在把他调回来,打农民军。

而且他不是一个人回来的,跟着他回来的,还有一千人。

这一千人,是他的老部下,他们隶属于一支特殊的部队----关宁铁骑。

关宁铁骑,是明朝最精锐的特种部队,但人数并不多,大致在六千人左右,其中一半,在祖大寿的手中,曹文诏带
回来的,只是六分之一。而他的对手王嘉胤,手下的民军主力,在三万人左右。

王嘉胤什么来历,说法很多,靠谱的不多,但在当时那一拨人里,他是很牛的。之前我说过,在他手下,有个叫张
献忠的小喽罗。顺便再说句,后来威震天下、被称为``闯王''的高迎祥(李自成是闯王2.0版本),都是他的人,给他
打工。

而且这人很难得,很有点组织才能,连个县都没占住,就开始搞政府机构。但最搞笑的是,他还大胆地搞了机构改
革,突破常规,明朝有的,他有;明朝没有的,他也有,不但有六部都察院,还有宰相。

当然,对于这些,曹文诏是没有兴趣的,到任后一个月,他就动手了。

按通常的说法,他率数倍于民军的官兵,以压倒性的优势,发动了进攻。

但事实是有点区别的,王嘉胤的兵力前面说过,是三万人,而曹文诏带去的人,是三千。

估计王嘉胤原先没在部队混过,也不大知道曹文诏何许人也,对曹总兵的来访,他倒不是很紧张,毕竟就三千人,
还能咋样。

王嘉胤认为,就算曹文诏再强,就算他手下有关宁铁骑,但毕竟是十个打一个,无论如何,都是不会输的。所以他
摆好了阵势,准备迎敌。

他太单纯了。

要知道,打了十几年仗,换了三任领导,从努尔哈赤打到皇太极,还能混到现在,光凭勇猛,十条命都是不够的。

\section[\thesection]{}

曹文诏之所以出名,不是因为勇猛,而是因为耍诈。

此人身经百战,通晓兵法,到地方后,压根没动手,先断了王嘉胤的粮道。王嘉胤慌了,要坚守,没有粮食,要突
围,又没法冲出去。

就这样,王嘉胤冲了两个月,终于,在他即将放弃时,奇迹出现了。

曹文诏的包围圈,竟然出现了漏洞,王嘉胤终于找到机会,冲出重围。

王嘉胤感觉很幸运,虽说被困了两个月,但好歹还是出来了。换个地方,还能接着干。

可惜他并不知道,曹文诏是一个没有漏洞的人,他所有的失误,都是故意的。

把人围起来,然后死磕,是可以的,但是损失太大,最好的方法,是把他们放出来,然后一路追着打。

在这个思想的指导下,王嘉胤逃了出来,逃出来后,就后悔了。因为从他逃出来那天起,曹文诏就跟在他屁股后
面,紧追不放,追上就是一顿猛捶,五天之内打了五仗,王嘉胤一败涂地。

更可气的是,曹文诏似乎不打算一次把他玩死,每次打完就撤,等你跑远点,下次再打,反正他的部队是骑兵。对
此,王嘉胤极为郁闷。

其实曹文诏也很郁闷,谁让你有三万人,我只有三千,只能慢慢打。

打了两个月,王嘉胤崩溃了,王嘉胤的部下也崩溃了。在某个混乱的夜晚,王嘉胤被部下杀死,部分投降了曹文诏。

王左桂死了,王嘉胤也死了,剩下的,还有神一魁。

在所有的起义军中,最能打的,最能坚持的,就是神一魁。

为了彻底铲除这个心腹之患,洪承畴决定,跟曹文诏合作。

所谓合作,就是客气客气。就官职而言,洪承畴是总督,曹文诏是总兵,洪承畴是进士,曹文诏是老粗。基本上,
洪承畴怎么说,曹文诏就怎么做,相当听话。

几年后的那场悲剧,即源自于此。

其实这个时候,神一魁已经挂了,真正掌控军权的,是四个人:红军友、李都司、杜三、杨老柴。

虽说头头死了,但势头一点没消停,光主力部队,就有五万人,聚集在甘肃镇原,准备进攻平凉。

所以洪承畴决定,一次性彻底解决问题。

除曹文诏之外,他还调来了王承恩、贺虎臣等人,基本上西北最能打的几个总兵,都到齐了。

到齐了,就是群殴。群殴之后,民军撑不住了,决定向庆阳撤退。

\section[\thesection]{}

想法是好的,可惜做不了。特别是曹文诏,由于他率领的关宁铁骑,每人都有两匹马,骑累一匹就换一匹,机动性
极强,民军往哪跑,他就等在哪。跑来跑去,没能跑出去。

经过两个月的僵持,双方终于在镇原附近的西濠决战,史称西濠之战。

整个战役的过程,大致相当于一堂生动的骑兵训练课。刚开打,还没缓过劲,曹文诏就率军冲入了敌军,乱砍乱
杀,大砍大杀,基本上是怎么砍怎么有。

砍完了,退回来,歇会,歇完了,再冲进去,接着砍。所谓如入无人之境,大致就是这个状态。

民军的阵脚大乱,与此同时,洪承畴派出了他的主力洪兵,连同贺虎臣的宁夏兵,王承恩的甘肃兵,发动总攻,敌
军就此彻底崩溃。

此战,民军损失近万人,首领杜三、杨老柴被生擒(曹文诏抓的)。

残余部队全部逃散。

通常状态下,都打残了,也就拉倒了。

洪承畴不肯拉倒,打残是不够的,打死是必须的。

神一魁的四个头领,抓了两个,还剩两个----红军友、李都司。

这个艰巨的任务,由曹文诏接手,他率领自己的两千骑兵,开始了追击。接下来,是曹文诏的表演时间。

面对曹文诏的追击,几万军队几乎无法抵抗,连战连败,死伤近万,主要原因,还是曹文诏太猛。

曹总兵是见过大世面的,最猛的八旗军他都没怕过,打半业余的民军,自然没问题。每次进攻,他都带头冲锋,打
得民军头目胆战心惊,时人有云:``军中有一曹,西贼闻之心胆摇''。

这种说法是客观的,却是不全面的,因为曹总兵不但玩硬的,还玩阴的。

在追击的路上,曹文诏的手下报告,他们抓住了一个叫李宫用的敌军将领,

按日常惯例,处理方法都是拉出去砍了,但曹文诏想了想,对手下说,放了这个人。

此后的事情,用史书上的话说,``文诏乃纵反间,绐其党,杀红军友。''

这句话的意思是,曹文诏放走了这个人,并利用他使了个反间计,忽悠了他的同党,杀掉了四大首领中的红军友。

其实我也很想告诉你,这个反间计到底怎么使的,只是我查了很多史料,也没查个明白,

有一点是肯定的,对民军而言,曹文诏,是最为恐惧的敌人。

\section[\thesection]{}

人恐惧了,就会逃跑,逃无可逃,就不逃了。

神一魁剩下的,只有李都司了。

他很恐惧,所以他逃跑,但残酷的事实告诉他,继续跑,是没有前途的。

所以他决定,不跑了,回头,决战曹文诏!

等等,再想想

想明白了,不跑了,回头,伏击曹文诏!

没办法,对付这样的猛人,还是伏击比较靠谱。

他们伏击的地点,叫做南原。

为保证圈套成功,他们围住了附近的一群明军,吸引曹文诏前来救援。

曹文诏来了,但在这里,他看到了敌军上千名骑兵,二话不说就追。

追到了南原,穿进了圈套,伏兵四起。

应该说,伏兵还是有点作用的,受到突然袭击,曹文诏的部队被打乱,曹文诏被冲散。

李都司估计是读过史书的,至少看过淝水之战,他当即派人在军中大喊:曹文诏已死!

很快,就喊成了口号,鉴于曹文诏不知被冲到哪去了,所以这个谣言很有点用,明军开始动摇。

然后,曹文诏就开始辟谣了,不用话筒,用长矛。

精彩表演开始,按史书上的说法,是``持矛左右突,匹马萦万众中。诸军望见''。

拿着长矛,左冲右突,单枪匹马在万军之中,如入无人之境,然后,大家都看见了他。

遇上这么个人,谣言是不管用了,伏击也别扯了,所以最后的结果,只能是``大败,僵尸蔽野''。

数过来,这应该是第二次大败了。但对于洪承畴和曹文诏而言,还没完。

残余部队的残余继续逃跑,曹文诏继续追击,然后是大败、复大败,又复大败。一路败到平凉,李都司终于不用败
了,洪承畴杀掉了他。神一魁的四大头领,最终无人幸免。

但到这份上,曹总兵还没消停,他继续追击残敌,竟然追到了甘肃、宁夏,连续几战,把残敌赶尽杀绝,至此,神
一魁的势力彻底退出历史舞台。西北之内,反军所剩无几。

王左桂、王嘉胤,神一魁,崇祯元年的三大民军领袖,就此结束。他们的戏份,在这个舞台上,他们注定只是个配
角。

主角

配角死光了,但龙套并没死,因为活不下去的人,终究还是活不下去,头头死了,就另找活路。

秉持这个原则,王左桂、王嘉胤、神一魁的残部,以及所有无法活下去的人,为了生存,继续战斗。

\section[\thesection]{}

但鉴于陕西、甘肃打得太狠,他们跑到了山西。

虽说是半业余组织,但吃了这么大的亏,总结总结经验是应该的。于是,在王嘉胤部将王自用的号召下,所有剩下
来的民军领袖,聚集在一块,开了个会。

会议的内容,是检讨教训,互相学习,互相促进,顺便再选领导。

其实也不用选,一般这种事,都是论资排辈。经过群众推举,王自用以资历最多,工龄最长,顺利当选新任头头。

鉴于曹文诏、洪承畴之类猛人的出现,大家共同认为,必须团结起来,协同作战。

当时去开会的,共有三十六支部队,史称``三十六营''。

跟以往一样,这三十六位头目,有三十六个外号,大致如下:紫金梁、闯王、八大王、曹操、闯塌天、闯将、扫地
王、黑煞神……

就外号水平而言,跟水浒传还没在一个档次上,梁山好汉们的文化程度,估计是够格的,什么急先锋、拼命三郎、
花和尚,都是现代的流行用语,相比而言,扫地王之类的外号,实在让人不知所谓。

而且就人数而言,也差点,水浒好汉们,总共是一百单八个,这次只有三十六个,也就够个天罡。

但在某一点上,他们跟梁山好汉是很相似的,不可思议的相似。

你应该还记得,梁山好汉排队时,排在第一的,并不是及时雨宋江,而是托塔天王晁盖。然而晁盖并不是真正的主
角,因为后来他被人给挂了。

这次的三十六位老大也一样,排在第一的紫金梁,就是王自用,他是当时的首领,后来倒没被人挂,自己挂了。

真正的主角,是后面的五位,外号你要不知道,那就对个号吧:

闯王----高迎祥

八大王----张献忠

曹操----罗汝才

闯塌天----刘国能

最后,是最牛的一位,闯将----李自成

这是极为有趣的五个人,他们性格不同,关系不同,有的是上下级,有的是战友,有的是老乡,为了生存,揭竿而
起。

然而在此后的十几年里,他们终将因为各自的原因,选择各自的道路,或互相猜忌,或者互相排挤,互相残杀,直
至人生的终点。

终点太远了,从起点说起吧。

\section[\thesection]{}

开完这次会后,各位老大纷纷表示,要统一思想,集中力量,共同行动。

这次开会的起义军,总兵力,近二十万人,开完后就分开了。

分开去打仗。

他们兵分几路,开始向山西各地进军。

崇祯得知,立即下令山西巡抚,全力围剿。

当时的山西巡抚,是个水货。

这位仁兄调兵倒很有一套,听说敌人来了,马上四处拉人,陕西、甘肃、宁夏的兵都被他拉了过来,光是总兵,就
有三个。但这人有个毛病,喜欢排兵布阵,把人调来调去,指挥乱七八糟,还没等他布出个形状,几路民军连续攻
克多地,闹得天翻地覆。

于是崇祯恼火了,他决定换人,换一个能让这三十六位首领做噩梦的人----曹文诏。曹文诏算是出头了。原先在辽
东系,也就是个游击,荣归故里后,短短一年时间,就升了副总兵,现在是总兵。

山西总兵,大致相当于军区司令员,但按崇祯的意思,这个总兵,大致相当于总司令,因为根据命令,所有追剿
军,都要服从曹文诏的指挥。

对于这个安排,三十六位头头是有准备的,所以他们决定,以太原一带为基地,协同合作,集中优势兵力,击溃曹
文诏。

崇祯六年(1633),曹文诏正式上任,积极备战,准备进攻。

大战即将开幕,但在开幕之前,这场戏又挤上来一个人。对这个人,曹文诏是比较熟悉的,因为在到西北之前,他
经常见到这个人。

此人之所以上场,是被崇祯临时硬塞进来。一般说来,但凡在历史舞台上混的,除个别猛人外(如朱元璋),艺术生
涯都比较短,混个几年就得下场。

但这位仁兄,上场的时间实在很长,曹文诏下去了,他没下去,明朝亡了,他都没下去,直到死在场上,都是主角。

隆重介绍,第三个猛人----左良玉。

就知名度而言,左良玉是比较高的,在很大程度上,他要感谢孔尚任,因为这位仁兄把他写进了自己的戏里(《桃花
扇》),虽然不是啥正面角色,但好歹是露了脸。

左良玉,字昆山,无学历,文盲。

左良玉的身世,是非常秘密的,秘密到连他自己都不知道。从小父母双亡,由叔父抚养长大,就这么个出身,你让
他饱读诗书,就是拿他开涮。

没书读,也得找工作,长大以后,左良玉去当了兵,小兵。

他的成长经历,跟曹文诏类似,但他混得比曹文诏好,到崇祯元年的时候,就已经混到了都司。

顺便说一句,他之所以混得好,跟个人努力关系不大,只是因为一个偶然的机会。

\section[\thesection]{}

天启年间,他还是个小兵时,有一次机缘巧合,遇到了一个人。

当时的左良玉,实在没啥特点,谁都瞧不上,但这个人算是例外。看见左良玉后,惊为天人,说他很好,将来很强
大,就说了几句话,建议朝廷给他提了个游击。

这位慧眼识才的仁兄,叫做侯恂,希望你还记得他,因为天启二年,他还曾经提拔过另一个人--袁崇焕。

按侯恂的说法,左良玉是个难得的人才,很快就会出人头地。

但事情跟他所说的,似乎还是有点差距,左良玉一直到崇祯元年,还是个小人物。

但不负侯恂所望,左良玉终究还是出名了,只是出名的方式,比较特别。

这事之前也提过,崇祯元年,宁远兵变,巡抚毕自肃自尽,袁崇焕来收拾残局,收拾来收拾去,就把左良玉给收拾
了。

当兵的没拿到工资,才兵变,左良玉有工资,自然不参加,但手下的兵哗变,他负领导责任,就这么被赶回了家。

回家呆了几天,又回来了。

袁崇焕死后,孙承宗又把他召了回来,去打关内四城,就是在那里,他开始崭露头角,和曹文诏并肩作战,收复了
遵化。

恰好,这段时间侯恂也混得不错,顺道给他提了副将,从此顺风顺水。

客观地讲,左良玉同志的进步,基本上是靠侯恂的。但后来的事情告诉我们,侯恂是个眼光很准的人。袁崇焕,他
没有看错;左良玉,也没有。

根据史料记载,左良玉身材很高,作战很猛,且足智多谋。虽说没文化,但很懂兵法,每次打仗都给人下套挖坑,
此外,他个人的战斗技术也相当厉害。

除作战外,左良玉还有点个人技术,他使用的兵器,不是长矛,而是弓箭。据说百发百中,而且左右手都能射箭,
速度极快。

到山西后,果然不同凡响。

先在涉县打了一仗,大败之,然后在辉县打了一仗,大败之,最后到了武安,被大败之。

这是个比较奇怪的事,当时左良玉的手下,有七八千人,竟然被农民军全歼,他自己带着几个手下好不容易才跑回
来,实在很没有名将风采。

不过不要紧,就算名将,也有发挥失常的时候,何况还有个不会发挥失常的名将。

\section[\thesection]{}

曹文诏的发挥从未失常,对于皇帝的信任,他很感动。

猛人被感动,反映在行动上,就是猛打,猛杀。

崇祯六年(1633)二月,文诏开始攻击。

他追击的敌人,有二十万,而他的兵力,是三千人。

无须怀疑,你没有看错,这就是曹文诏所有,且仅有的兵力。

他的追击之旅,第一站是霍州。在这里,他遇上了自己的第一个对手----上天龙。

上天龙究竟是谁,就别问了。我只知道,他是死在曹文诏手下的第一个首领。

上天龙手下,有上万人,摆好阵势,曹文诏率军冲锋。

这位兄弟抵抗的时间,也就是那一冲的瞬间----一冲就垮。

垮得实在太快,所以头头也没来得及跑,就被曹文诏杀了。

他的第二站,是孟县。

孟县,离太原没多远,在这里等待着他的,是混世王。

混世王这个外号,是很有点哲学意味的,毕竟在世上,也就是个混。但曹文诏用实际行动生动地告诉他,混是容易
的,混成王是很难的。

双方在孟县相遇,混世王的兵力,大致是曹文诏的六倍。六十倍都没用。

曹文诏毫无费力,就击溃了混世王,混世王想跑,没跑掉,被曹文诏斩杀

当时的太原,算是民军的天下,因为这里是三十六营首领,紫金梁王自用的老巢。此外,如闯王高迎祥、闯将李自
成等猛人,也都在那一带混。

曹文诏来后,就没法混了。

在他到任几个月后,史书上出现了这样的记载,``五台、盂县、定襄、寿阳贼尽平。''

曹文诏实在太猛,他连续作战,连续获胜,先后击溃十几支民军,但凡跟他作战的,基本都撑不过一天。此后,他
又在太谷、范村、榆社连续发起攻击,``贼几消尽。''

  其实打到这个份上,就算够意思了,但曹文诏是个比较较真的人,非要干到底,因为那个最终的目标,就在他
的眼前----紫金梁。

曹文诏是明白人,他知道,就凭对方这二十多万人,即使站在那里不动,让他砍,三千人,也得砍上十天半月。

所以最快,最方便的办法,就是干掉紫金梁。

为实现这个目标,他发动了连续攻击,关于这段时间的经历,史书上的记载,大致是时间、地名、斩杀人数----曹
文诏斩杀的人数。

短短十五天内,曹文诏率军七战七胜,打得紫金梁到处乱跑。先到泽州、再到润城、沁水,每到一地,最多一天,
曹文诏就到,到了就打,打了就胜。

紫金梁原本的想法,是集中兵力,跟曹文诏死磕。

死磕未必能行,死是肯定的。

\section[\thesection]{}

一个月,紫金梁的兵力已经损失了近三分之一,这么下去,实在赔不起了。

于是他做出决定,分兵。

紫金梁现在的想法是,曹文诏再猛,也没法分身,分兵之后,就看运气了,谁运气不好,被逮着,命苦不能怨政府。

就这么办了,紫金梁分工。他去榆社,老回回(三十六营之一)去武乡,过天星(三十六营之一)去高泽。

关于结局,史书上记载如下:``文诏皆击败''

到底怎么办到,我到今天也没弄明白。

但紫金梁、八大王们明白了,混到今天,再不躲就没命了。

曹文诏是山西总兵,山西是没法呆了,往外跑。

跑路的方向,有两个,一个是直隶(河北),另一个是河南。

紫金梁去了河南,至少在那里,他还是比较安全的。

这个想法再次被证明,是错误的。因为曹文诏同志是很负责的,别说中国河南,就算欧洲的荷兰,估计照去。在曹
文诏的追击下,紫金梁王自用吃了大亏,好不容易跑到河南济源,终于解脱了。

人死了,就解脱了。

所幸,他还算是善终,在被曹文诏干掉之前,就病死了。崇祯六年五月,紫金梁死去了,三十六营联盟宣告结束。
紫金梁结束了他的使命,接替他的,将是一个更为强大的人。

合谋

当然,对当时的起义军而言,这并不重要,重要的是,曹文诏还在追。

紫金梁死后,曹文诏继续攻击。在林县,他遇上了滚地龙率领的民军主力,一晚上功夫,全灭敌军,杀死滚地龙。
此后又攻下济源,在那里,他杀死了三十六营的重要头领老回回。

  洪承畴在陕西,陕西消停了,曹文诏在山西,山西也消停了。虽然河南也不安全,但对于众位头领而言,能去
的地方,也只有河南了,具体的地点,是河南怀庆。

河南怀庆,位于河南北部,此地靠近山西五台山地区,地段很好,想打就打,不想打就钻山沟,是个好地方。

于是,崇祯六年(1633)六月,山西、陕西的民军基本消失----全跑去河南了。

河南的日子还算凑合,虽说曹文诏经常进来打几圈,但时不时还能围个县城,杀个把知县,混得还算凑合。到崇祯
六年六月,来这里的民军,已经有十几万人。

\section[\thesection]{}

但好日子终究到头了,因为另一个猛人,来到了河南----左良玉。

三年前,孙承宗收复关内四城的时候,最能打的两个,就是左良玉和曹文诏。

就军事天赋而言,两人水平相当,也有人说,左良玉还要厉害点,之所以打仗成绩不好,说到底还是个人员素质问
题。

曹文诏率领的,是关宁铁骑,所谓天下第一强军,战斗力极强,打起来也顺手。

但左良玉估计是跟袁崇焕关系不好,来的时候,没有分到关宁铁骑(大多数在祖大寿的手上),只能在当地招兵。

这就比较麻烦了,倒不是说当地人不能打仗,关键在于,参加民军闹事的,大都也是当地人。老乡见老乡,两眼泪
汪汪,都是苦人家,闭只眼就过去了,官军也好,民军也罢,都是混饭吃,何必呢?

而这一次,左良玉得到了一支和以往不同的军队--昌平兵。

明代的军队,就战斗力而言,一般是北方比南方强。北方的军队,最能打的,自然是辽东军。问题在于,辽东军成
本太高,给钱不说,还要给地,相对而言,昌平兵性价比很高,而且就在京城附近,也好招。

带着这拨人,左良玉终于翻身了,他连续出击,屡战屡胜,先后斩杀敌军上万人,追着敌军到处跑。

到崇祯六年(1633)九月,不再跑了

民军主力被他赶到了河南武安,估计是跑得太辛苦,大家跑到这里,突然想到了一个问题:我们有十几万人,还跑
什么?就在这里,跟左良玉死磕。

这是一个极为错误的抉择。

敌人不跑了,左良玉也不跑了,他开始安静下来,不发动进攻,也不撤退。

对左良玉的反常举动,民军首领们很纳闷,但鉴于左总兵向来彪悍,他们一致决定等几天,看这位仁兄到底想干什
么。

左良玉想干的事情,就是等几天。

他虽然很猛,也很明白,凭自己这点兵力,追着在屁股后面踹几脚还可以,真卷袖子上去跟人拼命,是万万不能地。

在对手的配合下,左良玉安心地等了半个月,终于等来了要等的人。

根据崇祯的统一调派,山西总兵曹文诏、京营总兵王朴、总兵汤九州以及河南本地军队,日夜兼程,于九月底抵达
武安,完成合围。

对首领们而言,现在醒悟,已经太晚了。

\section[\thesection]{}

下面,我们介绍这个包围圈里的诸位英雄。据史料记载,除了知名人物高迎祥、张献忠、罗汝才、李自成外,还有
若干历史人物,如薛仁贵、刘备(都是外号)以及某些新面孔:比如鞋底光(一直没想明白这外号啥意思,估计是说他
跑得快),逼上路(这个外号很有觉悟)、一块云(估计原先干过诗人)、三只手(这个……);某些死人,比如混世王、上
天龙……(应该之前已经被曹文诏干掉了)。

大抵而言,所有你知道,或是不知道的,都在这个圈里。

对诸位首领而言,崇祯六年的冬天应该是过不去了。因为除被围外,他们即将迎来另一个相当可怕的消息。按规
定,但凡跨省调动,应该指认一名前线总指挥,根据级别,这个包围圈的最高指挥者,必定是曹文诏。

当然,如果真是曹文诏管这摊子事,历史估计就要改写了,因为以他老人家的脾气,逮住这么个机会,诸位首领连
全尸都捞不着。

可是,不是曹文诏。

因为一个偶然的事件。

崇祯六年九月,曹文诏被调离,赴大同任总兵。

关于这次任命,许多史书上都用了一个词来形容----自毁长城。

打得好好的,偏要调走,纯粹是找抽。

而这笔帐,大都算到了御史刘令誉的头上。

因为据史料记载,曹文诏当年在山西的时候,跟刘御史住隔壁,曹总兵书读的少,估计也不大讲礼貌,欺负了刘御
史,两人结了梁子。

后来刘御史到河南巡视,曹总兵跟他聊天,聊着聊着不对劲了,又开始吵,刘御史可能吃了点亏,回去就记住了,
告了一黑状,把曹文诏告倒了,经崇祯批准,调到大同。

史料是对的,说法是不对的。

因为按照明代编制,山西总兵和大同总兵,算是同一级别,而且崇祯对曹文诏极为信任,别说一状,一百状都告不
倒。真正的答案,在半年后揭晓。

崇祯七年(1634)初,皇太极率军进攻大同。

崇祯是个很苦的孩子,上任时年纪轻轻,小心翼翼地装了两年孙子,干掉了死太监,才算正式掌权,掌权之后,手
下那帮大臣又斗来斗去,好不容易干了几件事(比如裁掉驿站),又干出来个李自成。辛辛苦苦十几年,最后还是没
辙。

\section[\thesection]{}

史料告诉我们,崇祯很勤奋,他每天只睡几个小时,天天上朝,自己和老婆穿的衣服都打着补丁,也不好色(估计没
时间),兢兢业业这么多年,没享受权利,尽承担义务。这样的皇帝,给谁谁都不干。

很可怜。

可怜的崇祯同志之所以要把曹文诏调到大同,是因为他没有办法。

家里的事要管,外面的事也得管,毕竟手底下能打仗的人就这么多,要有两个曹文诏,这事就结了。

对于皇太极的这次进攻,崇祯是有准备的,但当进攻开始的时候,才发现准备不足。

皇太极进攻的兵力,大致在八万人左右,打宁远没指望,但打大同还是靠谱的。

自进攻发起之日,一个月内,大同防线全面击破,各地纷纷失守,曹文诏虽然自己很猛,盖不住手下太弱,几乎毫
无还手之力。

击破周边地区后,皇太极开始集结重兵,攻击大同。

大同是军事重镇,一旦失陷,后果不堪设想。就兵力对比而言,曹文诏手下只有两万多人,而主力关宁铁骑,只有
一千多人,失陷只是时间问题。

于是崇祯也玩命了,在他的调派下,吴襄率关宁铁骑主力,日夜兼程赶往大同,参与会战。

曹文诏也确实厉害,硬扛了十几天,等来了援兵。

皇太极眼看没指望,抢了点东西也就撤了。

崇祯七年(1634)的风波就此平息,手忙脚乱,终究是搞定了。

但曹文诏同志就惨了,虽然他保住了大同,但作为最高指挥官,责任是跑不掉的,好在朝廷里有人帮他说几句话,
才捞了个戴罪立功。

但皇太极这次进攻,导致的最严重后果,既不是抢了多少东西,杀了多少人,也不是让曹总兵被黑锅,而是那个包
围圈的彻底失败。

其实在崇祯十七年的统治中,有很多次,他都有机会将民军彻底抹杀。

这是第一次。

事实证明,那个包围圈相当结实,众位头领人多势众,从九月被围时起,就开始突围,突了两个月,也没突出去。
到十一月,连他们自己都认定,完蛋的日子不远了。

当时已是冬季,天气非常地冷,几万人被围在里面,没吃没喝,没进没退,打也打不过,跑也跑不掉。

然而不要紧,还有压箱底的绝技,只要使出此招,强敌即可灰飞烟灭----投降。

\section[\thesection]{}

当然了,投降是暂时的,先投降,放下武器,等出了圈,拿起武器,咱再接着干。

但你要知道,投降也是有难度的。

为顺利投降,他们凑了很多钱,找到了京城总兵王朴,向他行贿。

没有办法,因为你要投降,还要看人家接不接受你投降。为了共同的目标,适当搞搞关系,也是应该的

而且按很多人的想法,首领们应该是很穷的,总兵应该是很富的,事实上,这句话倒过来说,也还恰当。比如后来
的张献忠,在谷城投降后,行贿都行到了朝廷里,上到大学士、下到知县,都收过他的钱。

人不认人,钱认人,这个道理,很通用。

问题在于,参与包围的人那么多,为什么偏偏行贿王朴呢?

这是一个关键问题,而这个问题的答案充分说明,诸位头领的脑袋,是很好使的。

只能行贿王朴,没有别的选择。

因为王朴同志,是京城来的。在包围圈的全部将领中,他是最单纯的,最没见过世面。

王朴同志虽然来自京城,见惯大场面,但西北的场面,实在是没有见过,而在这群头领面前,他也实在比较单纯。

他知道,打仗有两种结果,投降就投降,不投降就打死,却不知道还有第三种----假投降。

他也不知道,在这个包围圈里的诸位头领,都有投降的经历,且人均好几次,某些层次高点的,如张献忠,那都是
投降的专业人士。

再加上无知单纯的王总兵,也有点不单纯,还是收了头领们的钱,他还算比较地道,收钱就办事,

崇祯六年(1634)十一月十八日,首领们派了代表,去找王朴(钱已经送过了),表示自己的投降诚意,希望大家从此
放下屠刀(当然,主要是你们),立地成佛。

王朴非常高兴,他的打算是完美的,受降,自己发点财,还能立功受奖,善莫大焉。他随即下令,接受投降,并催
促众首领早日集结队伍,交出武器。

当然他并没有撤除包围,那种蠢事他还是干不出来的。但既然投降了,就是内部矛盾了,没必要兴师动众,可以原
地休息,要相信同志。

你要说王朴没有丝毫提防,那也不对,他限令头头们十日之内,必须全部缴械投降。

不用十天,四天就够了。

\section[\thesection]{}

二十四日,十余万民军突破王朴的防线,冲出了包围圈。

大祸就此酿成。

鉴于所有的军队都在搞包围,河南基本是没什么兵,所以诸位头领打得相当顺手,很是逍遥了几天。

也就几天。

十二月三日,左良玉就追来了。

包围圈被破后,崇祯极为恼火,据说连桌子都踹了,当即下令处罚王朴,并严令各部追击。

左良玉跑得最快。之所以最快,倒不是他责任心有多强,只是按照行政划分,河南是他的防区,如果闹起来,他是
要背黑锅的。

摆在面前的局势,是非常麻烦的,十几万民军涌入河南,遍地开花,压根没法收拾。

左良玉收拾了,他收拾了河南境内的所有民军----只用了二十天

实践证明,左总兵是不世出的卓越猛人,他率领几千士兵,连续出击,在信阳、叶县等地先后击溃大量民军,肃清
了所有民军,从头至尾,二十天。

左良玉同志工作成绩如此突出,除了黑锅的压力,以及他本人的努力外,还有一个更为重要的原因:他所肃清的,
只是河南境内的民军,那些头领的主力,已经跑了。

跑到湖广了,具体地点,是湖广的郧阳(今湖北郧阳)。

我认为,他们跑到这个地方,是经过慎重考虑的。

跟河南接壤的几个省份,陕西是不能去的,洪承畴在那里蹲着,而且这人专杀投降的,去了也没前途。

山西也不能去,虽说曹文诏调走了,但几年来,广大头领们基本被打出了恐曹症,到了山西地界,就开始发怵,不
到万不得已,也不要去。

那就去湖广吧。

最早进去的是高迎祥和李自成,且去的时候,随身带着几万人。郧阳巡抚当时就晕菜了,因为郧阳属于山区,平时
都没什么人跑来,也没什么兵,这回大发了,一来,就来几万人,且都是闹事的。各州各县接连失陷,完全没办
法,只好连夜给皇帝写信,说敌人太多,我反正是没办法了,伸长脖子,等着您给一刀。

这段日子,对高迎祥和李自成而言,是比较滋润的,没有洪承畴,没有曹文诏,没有左良玉,在他们看来,郧阳是
山区,估摸着也没什么猛人,自然放心大胆。

这个看法是错误的。

事实上,这里是有猛人的,第四个猛人。

\section[\thesection]{}

说起来这位猛人所以出山,还要拜高迎祥同志所赐,他要不闹,估计这人还出不来。

但值得庆幸的是,在此人正式露面之前,高迎祥和李自成就跑了。

具体跑到哪里,就不知道了,反正是几个省乱转悠,看准了就打一把,其余头领也差不离,搞得中原各省翻天覆
地,连四川也未能幸免。

事情闹到这个地步,只能用狠招了。

崇祯七年,崇祯正式下令,设置一个新职务。

明代有史以来最大的地方官,就此登场。

在此之前,明代最大的地方官,就是袁崇焕,他当蓟辽督师时,能管五个地区。

光荣的记录被打破了,因为这个新职位,能管五个省。

这个职务,在历史中的称谓,叫做五省总督。包括山西、陕西、河南、湖广、四川,权力极大,也没什么管辖范
围,反正只要是流贼出没的地方,都归他管。

职位有了,还要有人来当,按照当时的将领资历,能当这个职务的,只有两个选择:A:洪承畴,B:曹文诏。

答案是C,两者皆不是

任职者,叫做陈奇瑜。

陈奇瑜,万历四十四年进士,历任都察院御史、给事中,后外放陕西任职。

在陕西,他的职务是右参政,而左参政,是我们的老朋友洪承畴。

但为什么要选他干这份工作,实在是个让人费解的事。就资历而言,他跟洪承畴差不多,而且进步也慢点,崇祯四
年的时候,洪承畴已经是三边总督了,他直到一年后,才干到延绥巡抚,给洪承畴打工。

就战绩而言,他跟曹文诏也没法比。无论如何,都不应该是他,但无论如何,偏就是他了。

所以对于这个任命,许多人都有异议,认定陈奇瑜有背景,走了后门。

但事实上,陈奇瑜并非等闲之辈。

崇祯五年的时候,由于民军进入山西,主力部队都去了山西,陕西基本是没人管,兵力极少。

兵力虽少,民变却不少,据统计,陕西的民军,至少有三万多人。

这三万多人,大都在陈奇瑜的防区,而他的手下,只有两千多人。

一年后,这三万多人都没了----全打光了。

因为陈奇瑜,是一个近似猛人的猛人。

作为大刀都扛不起来的文官,陈奇瑜同志有一种独特的本领----统筹。

\section[\thesection]{}

他是一个典型的参谋型军官,善于谋划、组织,而当时的民军,只能到处流窜,基本无组织,有组织打无组织,一
打一个准。

凭借着突出的工作成绩,陈奇瑜获得了崇祯的赏识,从给洪总督打工,变成洪总督给他打工。

对于领导的提拔,陈奇瑜是很感动的,也很卖力,准备收拾烂摊子。

这是一个涉及五个省,几十万人的烂摊子,基本上,已经算是烂到底了,没法收拾。

陈奇瑜到任后,第一个命令,是开会。各省的总督、总兵,反正是头衔上带个总字的,都叫来了。

然后就是分配任务,你去哪里,打谁,他去哪里,打谁,打好了,如何如何,打不好,如何如何,一五一十都讲明
白,完事了,散会。

散会后,就开打。

崇祯七年(1634)二月,陈奇瑜上任,干了四个月,打了二十三仗。

全部获胜。

陈奇瑜以无与伦比组织和策划能力告诉我们,所谓胜利,是可以算出来的。

多算胜,少算不胜,而况于无算乎?

----孙子兵法

陈总督最让人惊讶的地方,倒不是他打了多少胜仗,而在于,他打这些胜仗的目的。

打多少仗,杀多少人,都不是最终目的,最终的目的是,再打一仗,把所有人都杀光。

而要实现这个目标,他必须把所有的首领和民军,都赶到一个地方,并在那里,把他们全都送进地府。

他选中的这个地方,叫做车厢峡。

车厢峡位于陕西南部,长几十里,据说原先曾被当作栈道,地势极为险要。

所谓险要,不是易守难攻,而是易攻难守。

此地被群山环绕,通道极其狭窄,据说站在两边的悬崖上,往下扔石头,一扔一个准。

更要命的是,车厢峡的构造比较简单,只有一个进口,一个出口,没有其他小路,从出口走到进口,要好几天。这
就意味着,如果你进了里面,要么回头,要么一条路走到黑,没有中场休息。

几万民军,就进了这条路。

这几万民军,是民军的主力,据说里面还有李自成和张献忠。

为什么走这条路,没有解释,反正进去之后,苦头就大了去了。

陈奇瑜的部队堵住了后路,还站在两边的悬崖上,往下射箭、扔石头,没事还放把火玩,玩了十几天,彻底玩残了。

想跑是跑不掉的,想打也打不着,众头领毫无办法,全军覆没就在眼前,实在熬不住了。

使用杀手锏的时候到了。

\section[\thesection]{}

我说过,他们的杀手锏,就是投降,准确地说,是诈降。

没条件,谁投降啊?

----春节晚会某小品

很有道理,很现实,但在这里,应该加上两个字:

没条件,谁让你投降啊?

所以在投降之前,必须先送钱,就如同上次送给王朴那样。

于是头领们凑了点钱,送给了陈奇瑜。

然而陈奇瑜没有收。

崇祯没看错人,陈奇瑜同志确实是靠得住的,他没有收钱。

麻烦了,不收钱,我们怎么安心投降,不,是诈降呢?

但事实证明,头领们的智商是很高的,他们随即使出了从古至今,百试不爽的绝招----买通左右。

陈奇瑜觉悟很高,可是扛不住手下人的觉悟不高,收了钱后,就开始猛劝,说敌人愿意投降,就让他们投降,何乐
不为?

陈奇瑜没有同意。

陈奇瑜并不是王朴,事实上,他对这帮头领,那是相当了解,原先当延绥巡抚时,都是老朋友,知道他们狡猾狡猾
地,所以没怎么信。

我之前曾经说过,陈奇瑜是一个近似猛人的猛人。

所谓近似猛人的猛人,就是非猛人

他跟真正的猛人相比,有一个致命的弱点。

拿破仑输掉滑铁卢战役后,有人曾说,他之所以输,是因为缺少一个人----贝尔蒂埃。

贝尔蒂埃是拿破仑的参谋长,原先是测绘员,此人极善策划,参谋能力极强,但凡打仗,只要他在,基本都打赢
了,当时,他不在滑铁卢。

但最后,有人补充了一句:如果只有他(贝尔蒂埃)在,但凡打仗,基本都是要输的。

陈奇瑜的弱点,就是参谋。

和贝尔蒂埃一样,陈总督是个典型的参谋型军官,他很会参谋,很能参谋,然而参来参去,把自己弄残了。

军队之中,可以没有参谋,不能没有司令,因为在战场上,最关键的素质,不是参谋,而是决断。

陈奇瑜同志只会参谋,不会决断。

面对手下的劝说和胜利的诱惑,他妥协了。

陈奇瑜接受了投降,在他的安排下,近五万民军走出了车厢峡。

其实陈奇瑜也很为难,既要他们投降,又不能让他们诈降,要找人看着,但如果人太多,会引起对方疑虑。为了两
全其美,他动脑筋,想出了一个绝妙的方法:每一百降军,找一个人看着,监督行动。

注意,是一个人,看守一百个人。

想出这个法子,只能说他的脑袋坏掉了。

\section[\thesection]{}

跟上次不同,这次张献忠毫不拖拖拉拉,很有工作效率,走出车厢峡,到了开阔地,连安抚金都没拿,反了。我很
同情那些看守一百个人的人。

事情到这里,就算是彻底扯淡了,崇祯极为愤怒,朝廷极为震惊,陈奇瑜极为内疚,最终罢官了事。

了事?那是没可能的。

各路头领纷纷焕发生机,四处出战,河南、陕西、宁夏、甘肃、山西,烽烟四起。

估计是历经考验,外加焕发第二次生命的激动,民军的战斗力越来越强,原本是被追着跑,现在个把能打的,都敢
追着官兵跑。比如陕西著名悍将贺人龙,原本是去打李自成,结果被李自成打得落花流水,还围了起来,足足四十
多天,断其粮食劝他投降,搞得贺总兵差点去啃树皮,差点没撑过来。

到崇祯八年(1635),中原和西北,基本是全乱了,这么下去,不用等清兵入关,大明可以直接关门。

好在崇祯同志脑子转得快,随即派出了王牌--洪承畴。

在当时,能干这活的,也就洪承畴了,这个人是彻头彻尾的实用主义者,手狠且心黑,对于当前时局,他的指导思
想只有一字----杀。

杀光了,就没事了。

就任五省总督之后,他开始组织围剿,卓有成效,短短几个月,民军主力又被他赶到了河南,各地民变纷纷平息。

接下来的程序,应该是类似的,民军被逼到某个地方,被包围,然后被逼无奈,被迫诈降。

所谓事不过三,玩了朝廷两把,就够意思了,再玩第三把,是不可能的。

洪承畴已经磨好刀,等待投降的诸位头领,这一次,他不会让历史重演。

是的,历史是不会重演的。

这次被逼进河南的民军,算是空前规模,光是大大小小的首领,就有上百人,张献忠、李自成、高迎祥、罗汝才、
刘国能等大腕级人物,都在其中。民军的总人数,更是达到了创纪录的三十万。

为了把这群人一网打尽,崇祯也下了血本,他调集了近十万大军,包括左良玉的昌平兵,曹文诏的关宁铁骑、洪承
畴的洪兵,总而言之,全国的特种部队,基本全部到齐。

但凡某个朝代,到了最后时刻,战斗力都相当之差,但明朝似乎是个例外。几十年前,几万人就能把十几万日军打
得落花流水,几十年后,虽说差点,但还算凑合。

和以往一样,面对官军的追击,民军节节败退,到崇祯八年(1635),他们被压缩到洛阳附近,即将陷入重围,历史
即将重演。

但终究没有重演。

因为在最关键的时刻,他们开了个会。

\section[\thesection]{}

开会的地点,在河南荥阳,故史称``荥阳大会''。

这是一次极为关键的会议,一次改变了无数人命运的会议。

参与会议者,包括所有你曾经听说过,或者你从未听说过,或者从未存在过的著名头领。用史书上的说法,是``十
三家''和``七十二营''。

家和营都是数量单位,但具体有多少人,实在不好讲。某些家,如高迎祥,有六七万人,某些营,兴许是皮包公
司,只有几个人,都很难讲,但加起来,不会少于二十五万人。

当然,开会的人也多,十三加上七十二,就算每户只出个把代表,也有近百人。

简而言之,这是一次空前的大会,人多的大会。

根据史料留下的会议记录,会议是这样开始的,曹汝才先说话,讲述当前形势。

形势就别讲了,虽说诸位头领文化都低,还是比较明白事情的,敌人都快打上来了,还讲个屁?

有人随即插话,提出意见,一个字----逃。

此人认为,敌人来势很猛,最好是快跑,早跑,跑到山区,保命。

在场的人,大都赞成这个意见。

然后,一人大喝而起:``怯懦诸辈!''

说话的人,是张献忠。

张献忠,陕西延安府人,万历三十四年出生。

历史上,张献忠是一个有争议的人,夸他的人实在不多,骂他的人实在不少。

反映在他的个人简历上,非常明显。

但凡这种大人物,建功立业之后,总会有人来整理其少年时期的材料,而张献忠先生比较特殊,他少年时期的材
料,似乎太多了点。

就成分而言,有人说,他家世代务农;有人说,他家是从商的;也有人说,他是世家后代;还有人说,他是读书出
身;最后有人说,他给政府打工,当过捕快。

鉴于说法很多,传说很多,我就不多说了,简单讲下,这几种说法的最后结果:

务农说:务农不成,歉收,去从军了。

从商说:从商不成,亏本,去从军了。

世家说:世家破落,没钱,去从军了。

读书说:读书没谱,落第,去当兵了。

打工说:没有前途,气愤,去当兵了。

史料太多,说法太多,但所有的史料都说,他是一个不成功的人。

\section[\thesection]{}

无论是务农、读书、从商、世家、打工,就算假设全都干过,可以确定的是,都没干好。

为什么没干好,没人知道,估计是运气差了点,最后只能去从军。

从军在当时,并非什么优秀职业,武将都没地位,何况苦大兵。

当兵,无非是拿饷。可是当年当兵,基本没有饷拿,经常拖欠工资,拖上好几个月,日子过得比较艰苦。

但奇怪的是,张献忠不太艰苦。据史料记载,他的小日子过得比较红火,有吃有喝,相当滋润。家里还很有点积蓄。

这是个奇怪的现象,而唯一的解释,就是他有计划外收入。

而更奇怪的是,他还经常被人讹,特别是邻居,经常到他家借钱,借了还不还,他很气愤,去找人要,人家不给,
他没辙。这是更为奇怪的一幕,作为手上有武器的人,还被人讹,只能说明,这些计划外收入,都是合法外收入。

据说,张献忠先生除了当兵之外,还顺便干点零活,打点散工,具体包括强盗、打劫等等。

这种兼职行为,应该是比较危险的,常在河边走,毕竟要湿鞋。张献忠同志终于被揭发了,他被关进监狱,经过审
判,可能是平时兼职干得太多,判了个死刑。

关键时刻,一位总兵偶尔遇见了他,觉得他是个人才,就求了个情,把他给放了。

应该说这位总兵的感觉,还是比较准的,张献忠确实是个人才,造反的人才。

据说平时在军队里,张献忠先生打仗、兼职之余,经常还发些议论,说几句名人名言,比如``燕雀安知鸿鹄之
志'',``王侯将相,宁有种乎''等等。

而他最终走上造反道路,是在崇祯三年(1630),那时,王嘉胤造反,路过他家乡,张献忠就带了一帮人,加入了队
伍。

张献忠起义的过程,是比较平和的,没人逼他去修长城,他似乎也没掉队,至于爹妈死光,毫无生路等情况,跟他
都没关系,而且在此之前,他还是吃皇粮的,实在没法诉苦。

所以这个人造反的动机,是比较值得怀疑的。

参加起义军后,张献忠的表现还凑合,跟着王嘉胤到处跑,打仗比较勇猛,打了一年,投降了。

因为杨鹤来了,大把大把给钱,投降是个潮流,张献忠紧跟时代潮流,也投了降。

当然,后来他花完钱后,顺应潮流,又反了。

\section[\thesection]{}

此后的事情,只要是大事,他基本有份。三十六营开会、打进山西、打进河南、被人包围、向王朴诈降、又被人包
围、向陈奇瑜诈降,反正能数得出来的事,他都干过。

但在这帮头领里,他依然是个小人物,总跟着别人混,直至这次会议。

他驳斥了许多人想逃走的想法,是很有种的,但除了有种外,就啥都没有了。因为敌人就在眼前,你要说不逃,也
得想个辙。然而张献忠没辙。

于是,另一个人说话了,一个有辙的人: ``一夫犹奋,况十万众乎!官兵无能为也!''

李自成如是说。

李自成,陕西米脂人,万历三十四年生人。

比较凑巧的是,李自成跟张献忠,是同一年生的。

而且这两人的身世,都比较搞不清楚,但李自成相对而言,比较简单。

根据史料的说法,他家世代都是养马的。在明代,养马是个固定职业,还能赚点钱,起码混口饭吃,生活水准,大
致是个小康。

所以李自成是读过书的,他从小就进了私塾,但据说成绩不好,很不受老师重视,觉得这孩子没啥出息。直到有一
天。

这天,老师请大家吃饭,吃螃蟹。

当然,老师的饭没那么容易吃,吃螃蟹前,让大家先根据螃蟹写首诗,才能开吃。

李自成想了想,写了出来。

老师看过大家的诗,看一首,评一首,看到他写的诗,没有说话。

因为在这首诗里,有这样一句话:一身甲胄任横行。

这位老师是何许人也,实在没处找,但可以肯定的是,他是一个比较厉害的人物,因为在短暂犹豫之后,他说出了
一个准确的预言:你将来必成大器,但始终是乱臣贼子,不得善终!

但李自成同学的大器之路,似乎并不顺利,吃过饭不久,他就退学了,因为他的父亲去世了。

没有经济基础,就没有上层建筑,李自成决定,先去打基础,但问题是,他家并不是农民,也没地,种地估计是瞎
扯,所以他唯一能够选择的,就是给人打工。

这段时间,应该是李自成比较郁闷的时期。因为他年纪小,父亲又死了,经常被人欺负,有些地主让他干了活,还
不给钱,万般无奈之下,他托了个关系,去驿站上班了。

\section[\thesection]{}

李自成的职务是驿卒,我说过,驿站大致相当于招待所,驿卒就是招待所服务员,但李自成日常服务的,并不是
人,而是马。

由于世代养马,所以李自成对马,是比较有心得的,他后来习惯于用骑兵作战,乃至于能在山海关跟吴三桂的关宁
铁骑打出个平手,估计都是拜此所赐。

李自成在驿站干得很好,相比张献忠,他是个比较本分的人,只想混碗饭吃。

崇祯二年,饭碗没了。我说过很多次,是刘懋同志建议,全给裁掉了。

刘懋认为,驿站纰漏太多,浪费朝廷资源,李自成认为,去你娘的。

你横竖有饭吃,没事干了,来砸我的饭碗。

但李自成还没有揭竿而起的勇气,他回了家,希望打短工过日子。

我也说过很多次,从崇祯元年,到崇祯六年,西北灾荒。

都被他赶上了,灾荒时期,收成不好,没人种地,自然没有短工的活路。此时,李自成听说,有一个人正在附近招
人,去了的人都有饭吃。

他带着几个人去了,果然有饭吃。

这位招聘的人,叫做王左桂。

王左桂是干什么的,之前也说过了,作为与王嘉胤齐名的义军领袖,他比较有实力。

当时王左桂的手下,有几千人,分为八队,他觉得李自成是个有料的人,就让他当了八队的队长。

这是李自成担任的第一个职务,也是最小的职务,而他的外号,也由此而生----八队闯将。

一年后,王左桂做出了一个决定,他要攻打韩城。

他之所以要打这里,是经过慎重考虑的,因为韩城的防守兵力很少,而且当时的总督杨鹤,没有多少兵力可以增
援,攻打这里,可谓万无一失。

判断是正确的,正如之前所说的,杨鹤确实没有兵,但他有一个手下,叫洪承畴。

这次战役的结果是,洪承畴一举成名,王左桂一举完蛋,后来投降了,再后来,被杀降。

王左桂死掉了,他的许多部属都投降了,但李自成没有,他带着自己的人,又去投奔了不沾泥。

不沾泥是个外号,他的真名,叫做张存孟(也有说叫张存猛)。但孟也好,猛也罢,这人实在是个比较无足轻重的角
色,到了一年后,他也投降了。

然而李自成没有投降,他又去投了另一个人。这一次,他的眼光很准,因为他的新上司,就是闯王高迎祥。

\section[\thesection]{}

这是极其有趣的一件事,王左桂投降了,李自成不投降,不沾泥投降了,他也没投降。

虽说李自成也曾经投降过,比如被王朴包围,被陈奇瑜包围等等,但大体而言,他是没怎么投降的。

这说明,李自成不是痞子,他是有骨气的。

相比而言,张献忠的表现实在不好。

他投降的次数实在太多,投降的时机实在太巧,每次都是打不过,或是眼看打不过了,就投降,等缓过一口气,立
马就翻脸不认人,接着干,很有点兵油子的感觉。

史料记载,张献忠的长相,是比较魁梧的,他身材高大,面色发黄(所以有个外号叫黄虎),看上去非常威风。

而李自成就差得多了,他的身材不高,长得也比较抱歉,据说不太起眼(后来老婆跑路了估计与此有关),但他很讲
义气,很讲原则,且从不贪小便宜。

历史告诉我们,痞子就算混一辈子,也还是痞子,滑头,最后只能滑自己。长得帅,不能当饭吃。

成大器者的唯一要诀,是能吃亏。

吃亏就是占便宜,原先我不信,后来我信,相当靠谱。

李自成很能吃亏,所以开会的时候,别人不说,他说。

第八队队长,不起眼的下属,四处寻找出路的孤独者,这是他传奇的开始。

他说,一个人敢拼命,也能活命,何况我们有十几万人,不要怕!

大家都很激动,他们认识到,李自成是对的,到这个份上,只能拼了。

但问题在于,他们已经被重重包围,在河南呆下去,死路,去陕西,还是死路,去山西,依然是死路,哪里还有路?

有的,还有一条。

李自成以他卓越的战略眼光,和无畏的勇气,指出那条唯一道路。

他说,我们去攻打大明的都城,那里很容易打。

他不是在开玩笑。

当然,这个所谓的都城,并不是北京,事实上,明代的都城有三个。

北京,是北都,南京,是南都,还有一个中都,是凤阳。

打北京,估计路上就被人干挺了,打南京,也是白扯,但打凤阳,是有把握的。

凤阳,位于南直隶(今属安徽),这个地方之所以被当作都城,只是因为它是朱元璋的老家。事实上,这里唯一与皇
室有关的东西,就是监狱(宗室监狱,专关皇亲国戚),除此以外,实在没啥可说,不是穷,也不是非常穷,而是非
常非常穷。

\section[\thesection]{}

但凤阳虽然穷,还特喜欢摆谱,毕竟老朱家的坟就在这,逢年过节,还喜欢搞个花灯游行,反正是自己关起门来
乐,警卫都没多少。

这样的地方,真是不打白不打。而且进攻这里,可以吸引朝廷注意,扩大起义军的影响。

话是这么说,但是毕竟洪承畴已经围上来了,有人去打凤阳,就得有人去挡洪承畴,这么多头领,谁都不想吃亏。

所以会议时间很长,讨论来讨论去,大家都想去打凤阳,最后,他们终于在艰苦的斗争中成长起来,领悟了政治的
真谛,想出了一个只有绝顶政治家,才能想出的绝招----抓阄。

抓到谁就是谁,谁也别争,谁也别抢,自己服气,大家服气。抓出来的结果,是兵分三路,一路往山西,一路往湖
广,一路往凤阳。

但这个结果,是有点问题的,因为我查了一下,抓到去凤阳的,恰好是张献忠、高迎祥、李自成。

没话说了。但凡是没办法了,才抓阄,但有的时候,抓阄都没办法。

真没办法。

抓到好阄的一干人等,向凤阳进发了,几天之后,他们将震惊天下。

在洪承畴眼里,所谓民军,都是群没脑子的白痴,但一位哲人告诉我们,老把别人当白痴的人,自己才是白痴。

检讨

很巧,民军抵达凤阳的时候,是元宵节。

根据惯例,这一天凤阳城内要放花灯,许多人都涌出来看热闹,防守十分松懈。

就这样,数万人在夜色的掩护下,连大门都没开,就大摇大摆地进了凤阳城。

慢着,似乎还漏了点什么----大门都没开,怎么能够进去?

答:走进去。

因为凤阳根本就没有城墙。

凤阳所以没有城墙,是因为修了城墙,就会破坏凤阳皇陵的风水。

就这样,连墙都没爬,他们顺利地进入了凤阳,进入了老朱的龙兴地。

接下来的事情,是比较顺理成章的,据史料记载,带军进入凤阳的,是张献忠。

如果是李自成,估计是比较文明的,可是张献忠先生,是很难指望的。

之后的事情,大致介绍一下,守卫凤阳的几千人全军覆没,几万多间民房,连同各衙门单位,全部被毁。

除了这些之外,许多保护单位也被烧得干净,其中最重要的单位,就是朱元璋同志的祖坟。

\section[\thesection]{}

看好了,不是朱元璋的坟(还在南京),是朱元璋祖宗的坟。

虽说朱五一(希望还记得这名字)同志也是穷苦出身,但张献忠明显缺乏同情心,不但烧了他的坟,还把朱元璋同志
的故居(皇觉寺)也给烧了。

此外,张献忠还很有品牌意识,就在朱元璋的祖坟上,树了个旗帜,大书六个大字:``古元真龙皇帝''

就这样,张献忠在朱元璋的祖坟上逍遥了三天,大吃大喝,然后逍遥而去。

事大了。

从古至今,在骂人的话里,总有这么一句:掘你家祖坟。但一般来讲,若然不想玩命,真去挖人祖坟的,也没多少。

而皇帝的祖坟,更有点讲究,通俗说法叫做龙脉,一旦被人挖断,不但死人受累,活人也受罪,是重点保护对象。

在中国以往的朝代里,除前朝被人断子绝孙外,接班的也不怎么挖人祖坟,毕竟太缺德。

真被人刨了祖坟的,也不是没有,比如民国的孙殿英,当然他是个人行为,图个发财,而且当时清朝也亡了,龙脉
还有没有,似乎也难说。

朝代还在,祖坟就被人刨了的,只有明朝。

所以崇祯听到消息后,差点晕了过去。

以崇祯的脾气,但凡惹了他的,都没有好下场。崇祯二年,皇太极打到北京城下,还没怎么着,他就把兵部尚书给
砍了,现在祖坟都被人刨了,那还了得。

但醒过来之后,他却做出了一个让人意外的决定----做检讨。

请注意,不是让人做检讨,而是自己做检讨。

皇帝也是人,是人就会犯错误,如皇帝犯错误,实在没法交代,就得做检讨。这篇检讨,在历史上的专用名词,叫
做``罪己诏''。

崇祯八年(1635)十月二十八日,崇祯下罪己诏,公开表示,皇陵被烧,是他的责任,民变四起,是他的责任,用人
不当,也是他的责任,总而言之,全部都是他的责任。

这是一个相当奇异的举动,因为崇祯同志是受害者,张献忠并非他请来的,受害者写检讨,似乎让人难以理解。

其实不难理解,几句话就明白了。

根据惯例,但凡出了事,总要有人负责,县里出事,知县负责,府里出事,知府负责,省里出事,巡抚负责。

现在皇帝的祖坟出了事,谁负责?

只有皇帝负责。

\section[\thesection]{}

对崇祯而言,所谓龙脉,未必当真。要知道,当年朱元璋先生的父母死了,都没地方埋,是拿着木板到处走,才找
到块地埋的,要说龙脉,只要朱元璋自己的坟没被人给掘了,就没有大问题。

但祖宗的祖宗的坟被掘了,毕竟影响太大,必须解决。

解决的方法,只能是自己做检讨。

事实证明,这是一个相当高明的方法。自从皇帝的祖坟被掘了后,上到洪承畴,下到小军官,人心惶惶,唯恐这事
拿自己开刀,据说左良玉连遗书都写了,就等着拉去砍了,既然皇帝做了检讨,大家都放心了,可以干活了。

当然,皇帝背了大锅,小锅也要有人背,凤阳巡抚和巡按被干掉,此事到此为止。

崇祯如此大度,并非他脾气好,但凡是个人,刨了他的祖坟,都能跟你玩命,更何况是皇帝。

但没办法,毕竟手下就这些人,要把洪承畴、左良玉都干掉了,谁来干活?

对于这一点,洪承畴、左良玉是很清楚的,为保证脑袋明天还在脖子上,他们开始全力追击起义军。

说追击,是比较勉强的,因为民军的数量,大致有三十万,而官军,总共才四万人。就算把一个人掰开两个用,也
没法搞定。

好在,还有一个以一当十的人,曹文诏。

为保证能给崇祯同志个交代,崇祯八年六月,曹文诏奉命出发,追击民军。曹文诏的攻击目标,是十几万民军,而
他的手下,只有三千人。

自打开战起,曹文诏就始终以少打多,几千人追几万人,是家常便饭。

但上山的次数多了,终究会遇到老虎的。

曹文诏率领骑兵,一口气追了几百里,把民军打得落花流水,斩杀数千人。

但自古以来,人多打人少,不是没有道理的。

跑了几百里后,终于醒过来了,三千人而已,跑得这么快,这么远,至于吗?

于是一合计,集结精锐兵力三万多人,回头,准备跟曹文诏决战。

崇祯四年起,曹文诏跟民军打过无数仗,从来没输过,胆子特大,冲得特猛,一猛子就扎了进去。

进去了就再没出来。

民军已走投无路,这次他们没打算逃跑,只打算死拼。而曹文诏由于太过激动,只带了先锋一千多人,就跑过来了。

三万个死拼的人,对一千个激动的人,用现在的编制换算,基本相当于一个人打一个排,能完成这个任务的,估计
只有兰博。

\section[\thesection]{}

曹文诏不是兰博,但他实在也很猛,带着骑兵冲了十几次,所至之处,死伤遍地,从早上一直打到下午,斩杀敌军
几千人。

眼看快到晚上,杀得差不多了,曹文诏准备走人。

这并非玩笑,曹总兵是骑马来的,就算打不赢,也能跑得赢。

在混乱的包围圈中,他集结兵力,发动突击,很快就突出了缺口,准备回家洗澡睡觉。

当时场面相当混乱,谁都没认出谁,在民军看来,跑几个也没关系,所以也不大有人去管这个缺口。

但关键时刻,出情况了。

曹文诏骑马经过大批民军时,有一个小兵正好被俘,又正好看见了曹文诏,就喊了一句: ``将军救我!''

当时的环境,应该是很吵的,有多少人听见很难说,但很不巧,有一个最不该听见的人,听见了。

这个人是民军的一个头目,而在不久之前,他曾在曹文诏的部队里干过。作为一个敬业的人,他立即对旁人大喊:
``这就是曹总兵!''

既然是曹总兵,那就别想跑了。

民军集结千人,群拥而上围攻曹文诏。

曹文诏麻烦了,此时,他的手下已经被打散,跟随在他身边的,只有几个随从。

必死无疑。

必死无疑的曹文诏,在他人生的最后时刻,诠释了勇敢的意义。

面对上千人的围堵,他单枪匹马,左冲右突,亲手斩杀数十人,来回冲杀,无人可挡。没人上前挑战,所有的人只
是围着他,杀退一层,再来一层。

曹文诏是猛人,猛人同样是人,包围的人越来越多,他的伤势越来越重,于是,在即将力竭之时,他抽出了自己的
刀。在所有人的注视下,他举刀自尽。

曹文诏就这样死了,直到生命的最后一刻,他依然很勇敢。无论如何,一个勇敢的人,都是值得敬佩的。

崇祯极其悲痛,立即下令追认曹文诏为太子太保,开追悼会,发抚恤金,料理后事等等。

从某个角度讲,曹文诏算是解脱了,崇祯还得接着受苦,毕竟那几十万人还在闹腾,这个烂摊子,必须收拾。所
以,曹文诏死后不久,崇祯派出了另一个人。

当时的局势,已经是不能再坏了,凤阳被烧了,曹文诏被杀了,皇帝也做了检讨,原先被追着四处跑的民军,终于
到达了风光的顶点。

\section[\thesection]{}

据史料记载,当时的将领,包括左良玉、洪承畴在内,都是畏畏缩缩,遇上人了,能不打就不打,非打不可,也就
是碰一碰,只求把人赶走,别在自己防区里转悠,就算万事大吉。

对此,诸位头领大概也是明白的,经常带着大队人马转来转去,有一次,高迎祥带着十几万人进河南,左良玉得到
消息,带人去看了看,啥都没说就回来了。照这么下去,估计高迎祥就算进京城,大家也只能看看了。

然而一切都变化了,从那个人到任时开始。

对这个人,崇祯给予了充分的信任,给了一个绝后而不空前的职务----五省总督。

这个职务,此前只有陈奇瑜和洪承畴干过,但这人上来,并非是接班的,事实上,他是另起炉灶,其管辖范围包括
江北、河南、湖广、四川、山东。

当时全国,总共只有十三个省,洪承畴管五个,他管五个,用崇祯的说法是:洪承畴督师西北,你去督师东南,天
下必平!这个人就是之前说过的第四个猛人,他叫卢象升

对大多数人而言,卢象升是个很陌生的名字,但在当时,这是一个相当知名的名字,而在高迎祥、李自成的嘴里,
这人有个专用称呼:卢阎王。

就长相而言,这个比喻是不太恰当的,因为所有见过卢象升的人,第一印象基本相同:这是个读书人。

卢象升,字建斗,江苏宜兴人。明代的江苏,算是个风水宝地,到明末,西北打得乌烟瘴气,国家都快亡了,这边
的日子还是相当滋润,雇工的雇工,看戏的看戏。

鉴于生活条件优越,所以读书人多,文人多,诗人也多,钱谦益就是其中的优秀代表。

但除此外,这里也产猛人----卢象升。

所谓猛人,是不恰当的,事实上,他是猛人中的猛人。

但在十几年前,他跟这个称呼,基本是八杆子打不着,那时,他的头衔,是卢主事。

天启二年(1622),江苏宜兴的举人卢象升考中了进士,当时吏部领导挑中了他,让他在户部当主事。

据史料说,卢主事长得很白,人也很和气,所以人缘混得很好,没过两年,就提了员外郎,只用了三年时间,又提
了知府。

到崇祯二年,卢象升已经是五品正厅级干部了,就提拔速度而言,相当于直升飞机,而且卢知府人品确实很好,从
来没有黑钱收入,群众反应很好。

总之,卢知府的前途是很光明的,生活是很平静的,日子是很惬意的,直到崇祯二年。

\section[\thesection]{}

这年是比较闹腾的,基本都是大事,比如皇太极打了进来,比如袁崇焕被杀死,当然,也有小事,比如卢象升带了
一万多人,跑到了北京城下。

当时北京城下的援兵很多,有十几路,卢象升这路并不起眼,却是最有趣的一路,因为压根没人叫他来。

卢象升是文官,平时也没兵,但他听说京城危急,情急之下,自己招了一万多人,就跑过来了。

明末的官员,是比较有特点的,最大的特点,就是推卸责任,能不承担的,绝不承担,能承担的,也不承担,算是
彻头彻尾的王八蛋。

卢象升负责任,起码他知道,领了工资,就该办事。

但遗憾(或者是万幸)的是,卢象升同志没能打上仗,他在城下呆了一个多月,后金军就走了。

当然,这未必是件坏事,因为以他当时的实力,要真跟人碰上,十有八九是个死。

但这无所事事的一个月,却永远地改变了卢象升的命运,因为这段时间里,他亲眼目睹,一个叫袁崇焕的统帅,如
何在一夜之间,变成了囚犯。

这件事情,最终影响了他的一生,并让他在九年之后,做出了那个关键性的抉择。

朝廷的特点,一向是能用就使劲用,既然卢知府这么积极,干脆就让他改了行。

崇祯三年,卢象升提任参政,专门负责练兵。

当时最能打仗、最狠的兵,除辽东,就是西北,这两个地方的人相当彪悍,战斗力很强,敢于玩命,就算打到最后
一个人,也不投降,是明朝主要的兵源产地。卢象升练兵的地方是北直隶,就单兵作战能力而言,算是二流。

然而事实证明,只有二流的头头,没有二流的兵。

明朝的精锐部队,大都有自己的名字。比如袁崇焕的兵,叫做关宁铁骑,洪承畴的兵,叫做洪兵,而卢象升的兵,
叫天雄军。

就战斗力而言,明末的军队中,最强的,当属关宁铁骑,天雄军的战斗力,大致排在第三(第二还没出场),比洪兵
强。

据高迎祥和李自成讲,他们最怕的明军,就是天雄军。

比如关宁铁骑,虽然战斗力强,但都是骑兵,冲来冲去,死活好歹都是一下子,但天雄军就不同了,比膏药还讨
厌,贴上就不掉,极其顽固,只要碰上了,就打到底,不脱层皮没法跑。

\section[\thesection]{}

天雄军的士兵,大都来自大名、广平当地,并没有什么特别,之所以如此强悍,只是因为卢象升的一个诀窍。

两百多年后,有一个人使用了他的诀窍,组建了一支极为强悍的部队,这个人的名字,叫做曾国藩。

没错,这个诀窍的名字,叫做关系。

和曾国藩的湘军一样,卢象升的天雄军,大都是有关系的,同乡、同学、兄弟、父子,反正大家都是熟人,随便死
个人,能愤怒一堆人,很有战斗力。但这种关系队伍,还有个问题,那就是冲锋的时候,一个人冲,就会有很多人
跟着冲,但逃跑的时候,有一个人跑,大家也会一起跑。

比如曾国藩同志,有次开战,就遇到这种事,站在后面督战,还划了条线,说越过此线斩,结果开打不久,就有人
跑路,且一跑全跑,绕着线跑,追都没追上,气得投了河。

卢象升没有这个困惑,因为每次开战,他都站在最前面。

事实上,卢先生被称为卢阎王,不是因为他很能练兵,而是因为他很能杀人----亲手杀人。

之前我说过,卢象升长得很白,但我忘了说,他的手很黑。

卢象升是个很有天赋的人。据史料记载,他天生神力,射箭水平极高,长得虽然文明,动作却很粗野,每次作战
时,都拿着大刀追在最前面,赶得对方鸡飞狗跳。

他最早崭露头角,是一次激烈的战斗。

崇祯六年,山西流寇进入防区,卢象升奉命出击,对方情况不详,以骑兵为主力,战斗力很强,人数多达两万。

卢象升只有两千人,刚开战,身边人还没反应过来,他就一头扎进了敌营。

他的这一举动,搞得对方也摸不着头脑,被他砍死了几个人后,才猛然醒悟,开始围攻他。

卢象升的大刀水平估计相当好,敌人只能围住,无法近身,万般无奈,开始玩阴的,砍他的马鞍(刃及鞍)。

马鞍被干掉了,卢象升掉下了马,然后,他站了起来,操起大刀,接着打(步战)。

接下来的事情,就比较骇人听闻了,卢象升就这么操着大刀,带着自己的手下,把对方赶到了悬崖边。

没办法了,只能放冷箭。

\section[\thesection]{}

敌人的箭法相当厉害,一箭射中了卢象升的额头,又一箭,射死了卢象升的随从。

这两箭的意思大致是,你他娘别欺人太甚,逼急了跟你玩命。

这两箭的结果大致是,卢象升开始玩命了。而且他玩命的水平,明显要高一筹。

他提着大刀,越砍越有劲,几近疯狂(战益疾)。这下对方被彻底整懵了,感觉玩命都玩不过他,只好乖乖撤退,以
后再没敢到他的地界闹事。

虽然卢象升的水平很高,但在当时,他还不怎么出名,也没机会出头,然而帮助他进步的人出现了,这人的名字叫
做高迎祥。

崇祯七年,高迎祥等人跑出了包围圈,就进了郧阳,郧阳被折腾得够呛,巡抚也下了课,这事说过了。

但这件事,对卢象升而言,有着决定性的意义,因为接替郧阳巡抚的人,就是他。

如果高迎祥知道这件事情的后果,估计是死都不会去打郧阳的。

卢象升是个聪明人,聪明在他很明白,凭借目前的兵力,要把民军彻底解决,是绝不可能的。

作为五省总督(后来变成七省),他手下能够作战的精锐兵力,竟然只有五万人,但在这几省地界上转来转去的诸位
头领,随便拉出来一个,都有好几万人,总计几十万,还满世界转悠,没处去找。

但他更明白,彻底解决民军的头领,是绝对可能的。

民军虽然人多势众,但大都是文盲,全靠打头的领队,只要把打头的干掉,立马就变良民。

而在所有的头头里,最有号召力,最能带队的,就是闯王。

强调,现在的闯王是高迎祥,不是李自成。

在所有的头领中,高迎祥是个奇特的人,他的奇特之处,就是他一点也不奇特。

明末的这帮头领,都是比较特别的,用今天的话说,就是很有个性。

但凡古代干这行的,基本是两种人,吃不上饭的,和混不下去的。文化修养,大都谈不上,所以做事一般都不守规
矩,想怎么来就怎么来,军队也是一样,今天是这帮人,没准明天就换人了,指望他们严守纪律,按时出操,没谱。

但高迎祥是个特例,他没什么个性,平时不苟言笑,打赢了那样,打输了还那样。

\section[\thesection]{}

许多头领打仗,明天究竟怎么走,不管,也懒得管,打到哪算哪。

高迎祥的行军路线,都是经过精心设计的,并表明路标,引导部队行进。更吓人的是,高迎祥的部队,是有统一制
服的----铠甲。

一般说来,盔甲这种玩意,只有官军才用(费用比较高,民军装备不起),大部都是皮甲,而高迎祥部队的盔甲,是
铁甲。

所谓重甲骑兵,就是这个意思,更吓人的是,他的骑兵,每人都有两三匹马,日夜换乘,一天可以跑几百里,善于
奔袭作战。

就这么个人,连洪承畴这种杀人不眨眼的角色,看见他都发怵。打了好几次,竟然是个平手。

所以一直以来,高迎祥都被朝廷列为头号劲敌。

卢象升准备解决这个人。

当然,他很明白,光凭他手下的天雄军,是很难做到的,所以,他上书皇帝,几经周折,要来了一个特殊的人。这
个人的名字,叫做祖宽。

祖宽,不是祖大寿的亲戚,具体点讲,他是祖大寿的佣人。但祖大寿同志实在太过厉害,一个佣人跟着他混了几
年,也混出来了,还当上了宁远参将。

其实对于祖宽,卢象升并不了解,他最了解的,是祖宽手下的三千部队----关宁铁骑。

作为祖大寿的亲信,祖宽掌管三千关宁军,卢象升明白,要战胜高迎祥,必须把这个人拉过来,必须借用这股力量。
现在,他终于成功了,他认定,高迎祥的死期已然不远。

此时的高迎祥,正在为攻打汝宁做准备,还没完事,祖宽就来了。

高迎祥到底是有点水平,他从没见过祖宽,但看架势,似乎比较难搞,毅然决定跑路。

但他之所以跑路,不是为逃命,而是为了进攻。

高迎祥的战略思想十分清晰,敌人弱小,就迎战,敌人强大,就先跑路,多凑几个人,人多了再打。

一年前,曹文诏就是被这种战法报销的。

这一次,他的目的地,是陕州,在这里,有两个人正等待着他----李自成、张献忠。

民军最豪华的阵容,也就这样了,高迎祥集结兵力,等待着祖宽的到来。

以现有的兵力,高闯王坚信,如果祖宽来了,就回不去了。

祖宽果然来了,也果然没有回去,因为高迎祥、李自成、张献忠又跑路了。

高迎祥的这次选择,是极为英明的,因为祖宽过来的时候,队伍里多了个人----左良玉。

\section[\thesection]{}

高迎祥的这套策略,对付像王朴那样的白痴,估计还是有点用的,但祖宽这种老兵油子,那就没招了,他立马看穿
了这个诡计,拉上了左良玉,一起去找高迎祥算帐。

接下来是张献忠先生的受难时间。

其实这事跟张献忠本没有关系,只是高迎祥让他过来帮忙,顺道挣点外快,可惜不巧的是,碰上了硬通货。

跑路的时候,根据惯例,为保证都能跑掉,是分头跑的,高迎祥、李自成是一拨,张献忠是另一拨。

所以官军的追击路线,也是两拨,左良玉一拨,祖宽一拨。

不幸的是,祖宽分到的,就是张献忠。

我说过,祖宽手下的,是关宁铁骑,跑得很快,所以他只用了一个晚上,就追上了张献忠,大破之。

张献忠逃跑了,他率领部队,连夜前行,一天一夜,跑到了九皋山。

安全了,终于安全了。然后,他就看到了祖宽。

估计是等了很久,关宁军很有精神,全军突击,大砍大杀,张献忠主力死伤几千人,拼死跑了出去。

又是一路狂奔,奔了几百里,张献忠相信,无论如何,起码暂时是安全了。

然后,祖宽又出现了。我说过,他的速度很快。

此后的结果,是非常壮观的,用史书的话说----伏尸二十余里。

张献忠出离愤怒了,而这一次,他做出了违反常规的决定,比较有种,回头跟祖宽决战。

是的,上面这句话是不靠谱的,张献忠先生从来不会违反常规,他之所以回头跟祖宽决战,因为在逃跑的路上,遇
上了两个人----李自成、高迎祥。

人多了,胆就壮了,张献忠集结数万大军,在龙门设下埋伏,等待祖宽的到来。

\section[\thesection]{}

张献忠的这个埋伏,难度很大,因为祖宽太猛,手下全是关宁铁骑,久经沙场,``发一声喊,伏兵四起''之类的场
景,估计吓不住,就算用几万人围住,要冲出来,也就几分钟时间。

面对困境,张献忠同志展现了水平,他决定,攻击中间。利用突袭,把敌军一分为二,分而击破,这是唯一的方法。

单就质量而言,他的手下实在比较一般,但正如一位名人所说,有数量,就有质量,他集结了十倍于祖宽的兵力,
开始等待。

不出所料,祖宽出现了,依然不出所料,他没有丝毫防备,带领所有的兵力,进入了埋伏圈。

张献忠不出所料地发动了攻击,数万大军发动突袭,不出所料地把关宁军冲成了两截。

接下来,就是出乎意料的事了。

他惊奇地发现,虽然自己的人数占绝对优势,虽然自己出现得相当突然,但从这些被包围的敌人脸上,他看不到任
何慌张。

其实张先生这一招,用在大多数官军身上,是很有效果的,对关宁军,是无效的。这帮人在辽东,主要且唯一的工
作,就是打仗,见惯大场面,所谓伏兵,无非是出来的地方偏点,时间突然点,队伍分成两截,照打,有啥区别?

特别是祖宽,伏兵出现后,他非但没往前跑,反而亲自断后,就地组织反击,而他手下的关宁军,似乎也没有想跑
的意思,左冲右突,大砍大杀,战斗从早上开始,一直打到晚上,伏兵打成了败兵,进攻打成了防守,眼看再打下
去就要歇菜,撒腿就跑。

前后三战,张献忠损失极为惨重,死伤无数,被打出了毛病,据说听到卢象升、祖宽的名字就打哆嗦。

河南不能呆了,他率领军队,转战安徽。

相比而言,高迎祥、李自成的遭遇,可以用八个字来形容----只有更惨,没有最惨。

高迎祥第一次遇见卢象升,是在汝阳城外。

据史料记载,当时他的手下,有近二十万人,光是营帐,就有数百里(连营百里),浩浩荡荡,准备攻城,看起来相
当吓人。

而他的对手,赶来救援的卢象升,只有一万多人。

其实一直以来,官军能够打败民军,原因在于官军骑马,而民军只能撒脚丫跑。

但高迎祥是个例外,我说过,他的军队,是重甲骑兵,而且每人有两匹马,机动性极强,而卢象升手下能跟他打两
把的,只有关宁铁骑,且就一两千人。

更麻烦的是,当卢象升到达汝阳的时候,军需官告诉他,没粮食。没粮食的意思,就是没饭吃,没饭吃的意思,就
是没法打仗。

一般说来,军中断粮一天,军队就会失去一半战斗力,断粮两天以上,全军必定崩溃。

卢象升的军队断粮三天,没有一个逃兵。

这个看似没有可能的奇迹,之所以成为可能,只是因为卢象升的一个举动----他也断粮。

他非但不吃饭,连水都不喝(水浆不入口),此即所谓身先士卒。所以结果也很明显----得将士心,同仇敌忾。

其实很多时候,群众是好说话的,因为他们所需要的并非粮食,而是公平。

\section[\thesection]{}

公平的卢象升,是个很聪明的人,经过几天的观察,他敏锐地发现,高迎祥的部队虽然强悍,但是比较松散,选择
合适的突破点,还是可以打一打的。

卢象升选择的突破点,是城西,鉴于自己步兵太多,骑兵太少,硬冲过去就是找死,他想到了一个办法。

一千多年前,诸葛亮同志鉴于实在干不过魏国的骑兵(蜀国以步兵为主),想到了同样的方法。

没错,对付骑兵,成本最低,老少咸宜的方式,就是弓箭,确切地说,是弩。

诸葛亮用的,叫做连弩,卢象升用的,史料上说,是强弩,具体工艺结构不太清楚,但确实比较强,因为历史告诉
我们,高迎祥的重甲骑兵,在开战后仅仅几个小时里,就得到了如下结果----强弩杀贼千余人。

其实城西的部队被击破,死一千多人,对高迎祥而言,并不是啥大事,毕竟他的总兵力,有几十万人之多,但他的
军阵中,有一个致命的弱点,导致了汝阳之战的失败。

这个弱点,就是人太多。

几十万人,连营百里,而据卢象升给皇帝的报告,高迎祥的主力骑兵,有五六万人,其余的大都是步兵以及部队家
属。

步兵倒还好说,家属就麻烦了,这拨人没有作战能力,又大多属于多事型,就爱瞎咋呼,看到城西战败,便不遗余
力地四处奔走,大声疾呼,什么敌人很多,即将完蛋之类。而最终的结果,就是真的完蛋了。

汝阳之战结束,高迎祥的几十万大军就此土崩瓦解,纷纷四散逃命,但高迎祥实在有点军事水平,及时布置后卫,
阻挡卢象升的追击。

其实卢象升也没打算追击,一万人去追二十万人,脑子有问题。

但今天不追不等于明天也不追,卢象升看准机会,跟踪追击,在确山再次击败高迎祥,杀敌军数千人。

卢象升的亮相就此谢幕,自崇祯八年五月至十一月,他率绝对劣势兵力,先后十余战,每战必胜,斩杀敌军总计三
万余人,彻底扭转了战略局势。

当然,高迎祥并不这么想,他依然认为,失败只是偶然,他所有的兵力,是卢象升的几十倍,战略的主动权,依然
在他的手中,今年灭不了你,那就明年。

这个想法,让他最终只活到了明年。

\section[\thesection]{}

十一月过去了,接下来的一个月,是很平静的,卢象升没有动,高迎祥也没有动,原因非常简单----过年。

无论造反也好,镇压也罢,都是工作,工作就是工作,遇到法定假日,该休息还是得休息。

休息一个月,崇祯九年正月,接着来。

最先行动的,是卢象升,他行动的具体方式,是开会。

开会内容,自然是布置作战计划,研究作战策略,讨论作战方案。相对而言,高迎祥的行动要简单得多,只有两个
字----开打。

从心底里,卢象升是瞧不上高迎祥的,毕竟是草寇,没读过书,没考过试,没有文化,再怎么闹腾,也就是个草
寇,所以对于高迎祥的动向,卢象升是很有把握的:要么到河南开荒,要么去山西刨土,或者去湖广钻山沟,还有
什么出息?

为此,他做了充分的准备,还找到了洪承畴,表示一旦高迎祥跑到西北五省,自己马上跑过去一起打。

然而高迎祥的举动,却是他做梦都想不到的。

闯王同志之所以叫闯王,就是因为敢闯,所以这一次,他决定攻击一个卢象升绝对想不到的地方----南京。当然,
在刚开始的时候,这个举动并不明显,他会合张献忠,从河南出发,先打庐州,打了几天,撤走。

接下来,他开始攻击和州,攻陷。

攻陷和州后,他开始攻击江浦,江浦距离南京,只有几十公里。

如果你有印象的话,就会发现,两百多年前,曾经有人以几乎完全相同的路线,发起了攻击,并最终取得天下----
朱元璋。

高迎祥同志估计是读过朱重八创业史的,所以连进攻路线,都几乎一模一样,可惜他不知道,真正的成功者,是无
法复制的。

朝廷大为震惊,南京兵部尚书立即调集重兵,对高迎祥发动反攻击,经过几天激战,高迎祥退出江浦。

退是退了,偏偏没走。

他集结几十万人,开始攻打滁州。

至此可以断定,他应该读过朱重八传记,因为几百年前,朱元璋就是从和州出发,攻占滁州,然后从滁州出发,攻
下了南京。

滁州只是个地级市,人不多,兵也不多,而攻击者,包括李自成、张献忠等十几位头领,三十万人,战斗力最强,
最能打的民军,大致都来了。

所有的头领,所有的士兵,都由高迎祥指挥。

高闯王终于爬上了人生山峰的顶点。

他决定,进攻滁州,继续向前迈步。

山峰的顶点,再迈一步,就是悬崖。

\section[\thesection]{}

惨败

但至少在当时,形势非常乐观,滁州城内的兵力还不到万人,几十万人围着打,无论如何,是没问题的。

几天后,他得知卢象升率领援军,赶到了。

但他依然不怵,因为卢象升的援兵,也只有两万多人。此前虽说吃过卢阎王的亏,但现在手上有三十万人,平均十
五个人打一个,就算用脚算,也能算明白了。

卢象升率领总兵祖宽、游击罗岱,向滁州城外的高迎祥发动了进攻。

双方会战的地点,是城东五里桥。

在讲述这场战役之前,有必要介绍一下滁州的地形,在滁州城东,有一条很宽的河流,水流十分汹涌。

我再重复一遍,河流很宽,水流很汹涌。

这场会战的序幕,是由祖宽开始的,关宁铁骑担任先锋,冲入敌阵,发动了进攻。

战斗早上开始,下午结束。

下午结束的时候,那条很宽,水流很汹涌的河流,已经断流了,断流的原因,史料说法如下----积尸填沟委堑,滁
水为不流。

通俗点的说法,就是尸体填满了河道,水流不动。

尸体大部分的来源,是高迎祥的部下,在经历近七年的光辉创业后,他终于等来了自己最惨痛的溃败。

关宁铁骑实在太猛,面对城东两万民军,如入无人之境,乱砍乱杀。

高迎祥很聪明,他立即反应过来,调集手下主力骑兵,准备发动反击,毕竟有三十万人,只要集结反攻,必定反败
为胜。

红楼梦里的同志们曾告诉我们这样一句话:大有大的难处。

高迎祥的缺点,就是他优点----人太多。

人多,嘴杂,外加刚打败仗,通讯不畅,也没有高音喇叭喊话,乱军之中,谁也摸不清怎么回事,所以高闯王折腾
了半天,也没能集中自己的部队。

但高闯王还是很灵活的,眼看兵败如山倒,撒腿就往外跑,他相信,自己很快就能脱离困境。

这是很正确的,因为根据以往经验,官军都是拿工资的,而拿工资的人,有一个最大的特点----拿多少钱,干多少
事。无论是洪承畴,还是左良玉,只要把闹事的赶出自己管辖范围就算数了,没人较真。所谓跟踪追击这类活动,
应该属于加班行为,但朝廷历来没有发加班费的习惯,所以向来是不怎么追的,追个几里,意思到了,也就撤了。

\section[\thesection]{}

但是这一次,情况发生了变化。

我说过,卢象升是一个好人,一个负责任的官员。这一点反映在战斗上,就是认死理,凡是都往死了办。按照这个
处事原则,他追了很远----五十里。

之前我还说过,卢象升的外号,是卢阎王,虽然长得很白,但手很黑,无论是民军,还是民军家属,只要被他追
上,统统都格杀勿论,五十里之内,民军尸横遍野,保守估计,高迎祥的损失,大致在五万人以上。

追到五十里外,停住了。

不追,不是因为不想追,也不是不能追,而是不必追。

摆脱了追击的高迎祥很高兴,现在的局势并不算坏,三年前,他被打得只剩下几千人,逃到湖广郧阳,避避风头,
二十天后出山,又是一条好汉,何况手上有几十万人乎?

但安徽终究是呆不下去了,他转变方向,向寿山进发,准备在那里渡过黄河,去河南打工。

黄河岸边,他就遇到了明军总兵刘泽清。

刘泽清用大刀告诉他,此路不通。刘泽清并非猛人,并非大人物,也没多少兵,但是,他有渡口。

他就堵在河对岸,封锁渡口,烧毁船只,高迎祥只能看看,掉头回了安徽。

无所谓,到哪儿都是混。

但在回头的路上,他又遇见了祖大乐。祖大乐也是辽东系的著名将领,遇上了自然没话说,又是一顿打,高迎祥再
次夜奔。

好不容易奔到开封,又遇见了陈永福。

陈永福是个当时没名,后来有名的人,五年后,他坚守城池,把一个人变成了独眼龙----独眼李自成。

这种人,自然不白给,在著名地点朱仙镇跟高迎祥干了一仗,大败了高迎祥。

高迎祥终于发现,事情不大对劲了,自己似乎掉进了圈套。

他的感觉,是非常正确的。

得知高迎祥攻击滁州时,卢象升曾极为惊慌,但惊慌之后,他萌生了一个计划----彻底消灭高迎祥的计划。

高迎祥的想法,是非常高明的,学习朱重八同志,突袭南直隶,威胁南京,但遗憾的是,他忽略了一个重要的问
题----他没有在这里混过。

没有混过的意思,就是人头不熟,地方不熟,什么都不熟。

所以这个计划的关键在于,绝不能让高迎祥离开,把他困在此地,就必死无疑。

\section[\thesection]{}

刘泽清挡住了他的去路,祖大乐把他赶到了开封,陈永福又把他赶走,但这一切,只是序幕,最终的目的地,叫做
七顶山。

七顶山,位于河南南阳附近,被祖大乐与陈永福击败后,高迎祥逃到了这里,就在这里,

他看到了一个等候已久的熟人----卢象升。当然,除了卢象升外,还有其余一干人等,比如祖大乐、祖宽、陈永福
等等。

此时的高迎祥,手下还有近十万人,就兵力而言,大致是卢象升的两倍,更关键的是,他的主力重甲骑兵,依然还
有三万多人。

然而战争的结果,却让人大跌眼镜,号称``第一强寇''的高迎祥,竟然毫无还手之力,主力基本被全歼,仅带着上
千号人夺路而逃。

这是一个比较难以理解的事,最好的答案,似乎还是四个字----气数已尽。

十几万士兵、下属打得干干净净,兵器、家当丢得一干二净,高迎祥同志这么多年,折腾一圈,从穷光蛋,又变成
了穷光蛋,基本算是白奋斗了,应该说,他很倒霉。

但我个人认为,有个人比他更倒霉----李自成。

这个世界上,还有什么事情,比变成光杆司令更倒霉呢?有的,比如,变成光棍司令。

李自成的麻烦在于,他的老婆给他戴了绿帽子。

这位给李自成送帽子的老婆,叫邢氏,虽然不能肯定李自成有多少老婆,但这个老婆,是比较牛的。

按史料的说法,这位老婆基本不算家庭妇女,估计也不是抢来的,相当之强悍,打仗杀人毫无含糊,更难得的是,
她还很有智谋,帮李自成管账,据说私房钱都管。在管账的时候,她见到了高杰。

高杰,米脂人,李自成的老乡。据说打小时候就认识,后来李自成造反,他毫不犹豫,搭伙一起干,从崇祯二年开
始,同生共死,是不折不扣的铁哥们。

铁哥们,也是会生锈的。

李自成第一次怀疑高杰,是因为一件偶然的事。

崇祯七年八月,时任五省总督陈奇瑜,派出参将贺人龙进攻李自成。

贺人龙是个相当猛的人,此人战斗力极强,且杀人如麻,每次上战场,都要带头冲锋,被称为贺疯子。

贺疯子气势汹汹地到了地方,看到了李自成,打了一仗,非但没打赢,还被人给围住了,

且一围就是两个月。

\section[\thesection]{}

但李自成并不想杀掉贺人龙,因为贺人龙是他的老乡,而且他正在锻炼队伍阶段,需要人才,就写了封信,让高杰
送过去,希望贺人龙投降。

这个想法是比较幼稚的,贺人龙同志说到底是吃皇粮的,有稳定的工作,要他跟着李自成同志四处乱跑,基本等于
胡扯,所以信送过去后,毫无回音,说拿去擦屁股也有可能。

按说这事跟高杰没关系,贺人龙投不投降,是他自己的事,可是意外发生了。

去送信的使者,从贺人龙那里回来后,没有直接去找李自成,而是找了高杰。

这算是个事吗?

在这个世界上,很多事,说是事,就是事,说不是,就不是。

而李自成明显是个喜欢把简单的问题搞复杂的人,加上贺人龙同志守城很厉害,他打了两个月,连根毛都没拔下
来,所以他开始怀疑,贺人龙和高杰,有不同寻常的关系,就把高杰撤了回来。

无论是铁哥们,还是钛哥们,在利益面前,都是一脚蹬。

对李自成同志的行为,高杰相当不爽,但这事说到底,还是高杰的责任。

因为他回来之后,就跟邢氏勾搭上了。

到底是谁勾搭谁,什么时候勾搭上的,基本算是无从考证,但史料上说,是因为高杰长得很帅,而邢氏是管账的,
高杰经常跑去报销,加上邢氏的立场又不太坚定,一来二去,就勾搭上了。

关于这件事情的严重性,高杰同志是有体会的,在回顾了和李自成十几年的交情、几年的战斗友谊,以及偷人老婆
的内疚后,他决定,投奔官军。

当然,他是比较够意思的,临走时,把邢氏也带走了。

对李自成而言,这是一个极为沉重的打击,老婆跑了,除面子问题外,更为严重的是,他的很多秘密,老婆都知道
(估计包括私房钱的位置)。

除了老婆损失外,还有人才损失。

在当时李自成的部下里,最能打仗的,就是高杰,此人极具天赋,投奔了官军后,就一直打,打到老主顾李自成都
歇菜了,他还在继续战斗。

高杰投降的对象,是洪承畴,洪总督突然接到天上掉下来的馅饼,自然高兴异常,立刻派兵出击,连续击败李自
成,斩杀万人。

总而言之,对各位头领而言,崇祯九年算是个流年,老婆跑了,手下跑了,跑来跑去,就剩下自己了。

\section[\thesection]{}

对高迎祥而言,更是如此。

老婆跑了,再找一个就是,十几万大军都跑光了,就只能钻山沟了。

所以高闯王毅然决定,跑进郧阳山区。

两年前,就是在那里,被打得只剩半条命的高迎祥捡了条命,东山再起。

卢象升闻讯,立刻找到祖宽和祖大乐,吩咐他们,立即率军出发,追击高迎祥。

祖宽回答:不干。

卢象升无语。

之所以无语,因为他们从来就没干过。

关宁铁骑

很久以前,我以为所谓战争,大都是你死我活,上了战场,管你七大姑八大姨,都往死里打,特别是明末,但凡开
打,就当不共戴天,不共戴地,不共戴地球,打死了算。

后研读历史多年,方才知道,以上皆为忽悠是也。

按史料的说法,当时的作战场景大致如下:

比如一支官军跟民军相遇,先不动手,喊话,喊来喊去,就开始聊天,聊得差不多,民军就开始丢东西,比如牲
口,粮食等等,然后就退,等退得差不多了,官军就上前,捡东西,捡得差不多,就回家睡觉,然后打个报告给朝
廷,说歼敌多少多少,请求赏赐云云。

应该肯定的是,在当时,有这种行为的官军,是占绝大多数,认认真真打仗的,只占极少数,所谓``抛生口,弃辎
重,即纵之去''。

现象也好理解,当时闹事的,大都是西北一带人,而当兵的,也大都是关中人。双方语言相通,说起来都是老乡,
反正给政府干活,政府也不发工资(欠饷),即使发了工资,都没必要玩命,这么打仗,非但能领工资,还能捞点外
快,最后回去了还能领赏,非常有利于创收。在史料中,这种战斗方式有个专用名词:打活仗。

因为活仗好打,且经济效益丰富,所以大家都喜欢打,打来打去,敌人越打越多,局势越来越恶化,直到关宁铁骑
的到来。

其实关宁铁骑的人数没多少,我算了一下,入关作战的加起来,也就五千来人,卢象升、洪承畴手下最能打的,基
本就是这些人,最厉害的几位头领,都是被他们打下去的。

之所以能打,有两个原因,首先,这帮人在辽东作战,战斗经验丰富,而且装备很好,每人均配有三眼火铳,且擅
长使用突袭战术,冲入敌阵,势不可挡。

\section[\thesection]{}

而第二个原因,相当地搞笑,却又相当地真实。

我说过,每次打仗时,民军都要喊话,所谓喊话,无非就是谈条件,我给你多少钱,你就放我走,谈妥了就撤,谈
不妥再打。

但每次遇到关宁铁骑,喊话都是没用的,经常是话没喊完,就冲过来了,完全不受收买,忠于职守。

我此前曾以为,如此尽忠职守,是因为他们很有职业道德,后来看的书多了才明白,这是个误会。套用史料上的
话,是``边军无通言语,逢贼即杀'',意思是,辽东军听不懂西北方言,喊话也听不懂,所以见了就砍。

所以我一直认为,多学点语言,是会用得着的。

高迎祥就是吃了语言的亏,估计是屡次喊话没成,也没机会表达自己的诚意,所以被人穷追猛打了几个月,也没接
上头。

在众多的民军中,高迎祥的部队,算是战斗力最强的,手下骑兵,每人两匹马,身穿重甲,也算是山寨版的关宁铁
骑。虽说战斗力还是差点,但山寨版有山寨版的优势,比如……钻山沟。

高迎祥钻了郧阳山区,祖宽是不钻的,因为他的部队,大部都是骑兵,且待遇优厚,工资高,要让他们爬山,实在
太过困难,卢象升协调了一个多月,也没办法。

照这个搞法,估计过几个月,闯王同志带着山寨铁骑出来闹腾,也就是个时间问题。

在这最为危急的时刻,更危急的事情发生了。

崇祯九年(1636)四月,当卢象升同志正在费尽口水劝人进山时,辽东的皇太极做出了一个重大的决定----建国。

皇太极建都于沈阳,定国号为清,定年号为崇德。

这一举动表明,皇太极同志正式单飞,另立分店,准备单干。

通常来讲,新店开张,隔壁左右都要送点花圈花篮之类的贺礼,很明显,明朝没有这个打算,也没这个预算。

不要紧,不送,就自己去抢。

崇祯九年(1636)六月,清军发起进攻。这次进攻的规模很大,人数有十万人,统兵将领是当时清军第一猛将阿济
格,此人擅长骑兵突击,非常勇猛。

难得的是,他不但勇猛,脑子也很好用,关宁防线他是不去碰的,此次进关,他选择的路线,是喜峰口。

\section[\thesection]{}

此后的战斗没有悬念,明朝的主力部队,要么在关宁防线,要么在关内,所以阿济格的抢掠之旅相当顺利,连续突
破明军防线,只用了半个月,就打到了顺义(今北京市顺义区)。

我认为,阿济格是个很能吃苦的人,具体表现为不怕跑路,不怕麻烦,到了北京城下,没敢进去,就开始围着北京
跑圈,从顺义跑到了怀柔(今北京怀柔区),又从怀柔跑到了密云(今北京密云区),据说还去了趟西山(今北京西
山),圆满完成了画圈任务。

当然,他也没白跑。据统计,此次率军入侵,共攻克城池十二座,抢掠人口数十万,金银不计其数。

鉴于明朝主力无法赶到,只能坚壁清野,所以阿济格在北京呆了很长时间,而且,他还是个很有点幽默感的人,据
说他抢完走人时,还立了块牌子,上写四个字----各官免送!

我始终认为,王朝也好,帝国也罢,说穿了,就是个银行,这边收钱,那边付钱,总而言之,拆东墙,补西墙。不
补不行,几百年里,跑来拆墙的人实在太多,国家治不好,老百姓闹事,国防搞不好,强盗来闹事,折腾了这边,
再去折腾那边,边拆边补,边补边拆。

但国家也好,银行也罢,都怕一件事----银行术语,叫做挤兑,政治术语,叫内忧外患,街头大妈术语,叫东墙西
墙一起拆。

明朝大致就是这么个状况,客观地看,如果只有李自成、张献忠闹事,是能搞定的,如果只有清军入侵,也是能搞
定的,偏偏这两边都闹,就搞不定了。

于是一个月后,卢象升得知了一个惊人的消息,他被调离前线,等待他的新岗位,是宣大总督。

对于这个任命,无数后人为之捶腿、顿足、吐唾沫,说什么眼看内患即将消停,卢象升却走了,以至于局势失去控
制,崇祯昏庸等等等等。

在我看来,这个任命,无非是挖了东墙的砖,往西墙上补,不补不行,如此而已。

卢象升走了,两年后,他将在新的岗位上,完成人生最壮烈的一幕。

接班

听说卢象升离开的消息后,高迎祥非常高兴,因为他很清楚,像卢阎王这样的猛人,不是量产货,他擦亮眼睛,等
待着下一个对手的出现。

\section[\thesection]{}

他等来的接班人,叫做王家桢

王家桢,直隶人,时任兵部侍郎,此人口才极佳,善读兵法,出谋划策,滔滔不绝。行了,直说吧,这是个废柴。

他之所以被派来干这活,实在是因为嘴太贱,太喜欢谈兵法,太引人注目,最终得到了这份光荣的工作。

但王总督对自己的实力还是很明白的,刚到不久就上书皇帝,说自己身体比较弱,当五省总督太过勉为其难,干巡
抚就成。

崇祯还是很体贴的,让他改行当了河南巡抚。

但王巡抚刚上任没几天,就遇上了一件千载难逢的倒霉事。

这件倒霉事,叫做兵变,兵变并不少见,之所以说是千载难逢,是因为参与兵变的,是王巡抚的家丁。

连家丁都兵变,实在难能可贵,连崇祯同志都哭笑不得,直接把他赶回家卖红薯。

有这样的好同志来当总督,高迎祥的好日子就此开张,没过多久,他就出了山区,先到河南,拉起了几万人的队
伍,连战连胜,此后又转战陕西,气势逼人,洪承畴拿他都没办法。

四大猛人里,曹文诏死了,洪承畴没辙,左良玉固守,高迎祥最怕的卢象升,又去了辽东,现在而今眼前目下,高
闯王可谓天下无敌。

然后,第五位猛人出场了。

在这人出场前,高先生跟四大猛人打了近七年,越打越多,越打越风光,从几千打到几万、几十万,基本是没治了。
当时朝廷上下一致认为,隔几天跟他打一仗,能让他消停会,就不错了。至于消灭他,大致是个梦想。

在这人出场后,梦想变成了现实。

他没有用七年,连七个月都没用。事实上,直到崇祯九年(1636)三月,他才出山,只用了四个月,就搞定了高迎祥。

在历代史料里,每到某王朝即将歇业的时候,经常看到这样一句话,XX死而X亡矣。

前面的XX,一般是指某猛人的名字,后面的X,是朝代的名字,这句话的意思是,某猛人,是某王朝最后的希望,某
猛人死了,某王朝也就消停了。

在明代完形填空里,这句话全文如下:

传庭死,而明亡矣。

传庭者,孙传庭也。

孙传庭

孙传庭是个相当奇怪的人,因为在杀死高迎祥之前,他从未带过兵,从未打过仗,过去三十多年里,他从事的主要
工作,是人事干部。

\section[\thesection]{}

孙传庭,字伯雅,山西代县人,万历四十七年进士。在崇祯九年之前,历任永城、商丘知县,吏部主事。

其实他的运气不错。我查了查,万历四十七年的进士,到天启初年,竟然就当上了吏部郎中,人事部正厅级干部,
专管表彰奖励。

六部之中,吏部最大,而按照惯例,吏部尚书,一般都是从吏部郎中里挑选的。孙传庭万历二十一年(1593)出生,
照这个算法,他当郎中的时候,还不到三十岁,年轻就是资本,照这个状态,就算从此不干,光是熬,都能熬到尚
书。

然而没过两年,孙传庭退休了,提前三十年退休。

他丢弃了所有的前途和官位,毅然回到了家乡,因为他看不顺眼一个人----魏忠贤。

看不顺眼的人,很多,愿意辞官的,不多。

崇祯元年,魏忠贤被办挺了,无论在朝还是在野,包括当年给魏大人鞠躬、提鞋的人,都跳出来对准尸体踩几脚,
骂几句,图个前程。

但孙传庭依然毫无动静,没有人来找他,他也不去找人,只是平静地在老家呆着,生活十分平静。

八年后,他打破了平静,主动前往京城,请求复职。

出发之前,他说出了自己复出的动机:``待天下平定之日,即当返乡归隐。''

朝廷很够意思,这人没打招呼就跑了,也没点组织原则,十年之后又跑回来,依然让他官复原职,考虑到他原先老
干人事工作,就让他回了吏部,接着搞人事考核。

对他而言,这份工作的意思,大致就是混吃等死,但他没有提出异议,平静地接受,然后,平静地等待。

一年后,机会出现了,在陕西。

当时的陕西巡抚,是个非常仁义的人,具体表现为每次在城墙上观战,都不睁眼。据他自己说,是不忍心看,但大
多数人认为,他是不敢,这号人在和平时期,估计还能混混,这年头,就只能下岗。

巡抚这个职务,是个肥缺,平时想上任是要走后门的,但陕西巡抚,算是把脑袋别在裤腰带上混饭吃的,没准哪天
就被张某某、高某某剁了,躲都没处躲,孙传庭就此光荣上任,因为主动申请的人,只有他一个。

孙传庭出发之前,皇帝召见了他。

\section[\thesection]{}

对于孙巡抚的勇敢,崇祯非常欣赏,于是给了孙传庭六万两白银,作为军费。

除此之外,一无所有。

按崇祯的说法,国家比较困难,经费比较紧张,也就这么多了,你揣着走吧,省着点用。

当年杨鹤拿了崇祯十万两私房钱,招抚民军,也就用了几个月,孙传庭拿着六万两,也就打个水漂。

但人和人是不一样的。

自古以来,要人办事,就得给钱,如果没钱,也行,给政策。

孙传庭很干脆,他不要钱,只要政策,自己筹饷,自己干活,朝廷别管,反正干好了是你的,干不好我也跑不掉。

就这样,孙传庭拿着六万两白银,来到了陕西。

当时陕西本地的军队,战斗力很差,按照当时物价,六万两白银,大致只够一万人半年的军饷,最能打的将领,如
曹变蛟(曹文诏的侄子)、左光先、祖宽,要么在洪承畴手下,要么跟着卢象升。总之,孙传庭算是个三无人员:无
钱、无兵、无将。

但凡这种情况,若想咸鱼翻身,大都要经过卧薪尝胆、励精图治、艰苦奋斗、奋发图强等过程,至少也得个两三
年,才闪亮登场,大破敌军。

孙传庭上任的准确时间,是崇祯九年(1636)三月,他全歼高迎祥的时间,是崇祯九年(1636)七月。

从开始,到结束,从一无所有,到所向披靡,我说过,四个月。他到底是怎么完成的,到今天,也没想明白。

子午谷

此时的高迎祥,已经来到陕西。

他之所以来陕西,是因为此时的陕西比较好混。

虽说洪承畴一直都在陕西,而他手下的洪兵也相当厉害,但他最近正在陕北对付另一位老冤家李自成。不知是李自
成让他来帮忙,还是听说陕西巡抚比较软,高迎祥义无反顾地来了,单程。

自古以来,从下至上,要想进入陕西,必先经过汉中,所以当年刘备占据四川,要攻击曹操的长安,必占据汉中,
此后诸葛亮六次北伐,都经过汉中出祁山作战。

高迎祥也不例外,但在进军汉中的路上,有一支队伍挡住了他。

率领这支队伍的,是孙传庭。

对于孙传庭,高迎祥并不熟悉,也不在乎,而且这支队伍只有万把人,似乎也不难打,他随即率领军队发起攻击,
打了几次,损失上千人,没打动。

兵力占据优势,但多年的战斗经验告诉高迎祥,这是一支比较邪门的军队,不能再打了,他决定绕道。

他的直觉非常正确,那支镇守汉中,只有万把人的部队,在历史上,却有一个专门的称呼----秦军。

\section[\thesection]{}

之前我说过,明末的军队,战斗力最强的,是关宁铁骑,排第三的,是天雄军,排在第二的,是秦军。

关宁铁骑强悍,因为机动,天雄军善战,因为团结,而秦兵的战斗力,因为个性。

我曾查阅明代兵部资料,惊奇地发现,秦兵的主力,大都来自同一个地方----陕西榆林。

榆林,是个非常奇特的地方,据说每次打仗的时候,压根不用动员,只要喊两嗓子,无论男女老幼,抄起家伙就
上,而且说砍就砍,绝无废话。

因为这里只有士兵,没有平民。

榆林,明朝九边之一,自打朱元璋时起,就不怎么种地,传统职业就是当兵。平时街坊四邻聊天,说的也不是今年
种了多少地,收了多少粮食,大都是打了哪些地方,砍了多少人头(按人头收费)。几百年下来,形成独特个性,具
体表现为,进攻时,就算只有一个,都敢冲锋,撤退时,就算只剩一个,都不投降。

而且这里的人跟民军相当有缘分。听说民军来了,就算只是路过,都极其兴奋,冲出去就打,男女老幼齐上阵,估
计是当兵的人多,什么张大叔李大伯,上次就死在民军手里,喊一嗓子,能动员一群亲戚,后来李自成攻打榆林,
全城百姓包括大妈大爷在内,都没一个投降,就凭这个县,足足跟李自成死磕了八天,实在太过强悍。

孙传庭的兵,大致就是这些人。所以高迎祥没办法,是很正常的。

但高迎祥同志是要面子的,来都来了,还让我空手回去?无论如何,都要闯进去。

人有的时候,不能太执着。

执着的高迎祥经过深刻思考,多方查找,终于想到了一个方法。

他找到了一条隐蔽的小路,从这条小路,可以绕开汉中,直逼西安,只要计划成功,他就能一举攻克西安,占领陕
西,大功告成。

一千多年前,有两个人在几乎相同的地方,陷入了相同的困局,他们都发现了这条路。一个人说,由此地进攻,必
可大获全胜;另一个人说,若设伏于此,必定全军覆没!

没错,这两个人,一个叫诸葛亮,一个叫魏延,而他们发现的这条小路,叫做子午谷。

\section[\thesection]{}

至于结局,地球人(看过《三国演义》$<$演义上说的,别当真,看看就行)的地球人)都知道,魏延想打,诸葛亮不让
打,最后司马懿跳出来说,就知道你不敢打。

对于这个故事,许多人都说,诸葛亮过于谨慎,要按照魏延的搞法,早就打到长安了(魏延自己也这么说)。

而在高迎祥的故事里,只有魏延,没有诸葛亮。

所以一千年后,他在同样的地方,做出了不同的选择----出兵子午谷。

崇祯九年(1636)七月,高迎祥率领全部主力,冲入了子午谷,从这里,他将迅速到达西安。

但他不知道,这条路还通往另一个地点----地狱。

子午谷之所以是小路,是因为很小,对高迎祥而言,这句话绝对不是废话。

由于道路狭窄,而且天降大雨,他的几万大军,走了好几天,才走了一半,人困马乏,物资损失严重。

但高迎祥毫不沮丧,因为他相信,这个出乎许多人意料的举动,几天之后,必将震惊天下。

许多人确实没料到,但许多人里,并不包括孙传庭。

七月十六日,经过艰苦行军,高迎祥终于到达黑水峪,只要通过这里,前方就是坦途。

然后,满怀憧憬的高迎祥,看见了满怀愤怒的孙传庭。愤怒是可以理解的,因为他已经在这里,等了十五天。

孙传庭的军事嗅觉极为敏锐,从高迎祥停止进攻的那一刻,他就意识到,这兄弟要玩花样了。

而他唯一可能的选择,只有子午谷。

所以在撤离汉中,在子午谷的黑水峪耐心等待,因为他知道,艰苦跋涉之后,出现在他面前的高迎祥,是十分脆弱
的。

总攻随即开始,就人数对比而言,高迎祥的手下,大约在五万人以上,孙传庭兵力无法考证,估计在两万人左右,
狭路相逢。

无论是高迎祥,还是孙传庭,都很清楚,玩命的时刻到了。

生命的最后时刻,高迎祥展现了他令人生畏的战斗力,虽然极为疲劳,但他依然率军发动多次突击,三次击破孙传
庭的包围圈。

但他终归没能跑掉,原因很简单,这是一条小路。

在小路里打仗,就好比在胡同里打架,就算拿着青龙偃月刀,都没有板砖好使,而且道路太窄,没法跑开,所以他
每次冲出去,没过多久,又被围住。

孙传庭的部队也着实厉害,抗击打能力极强,每次被冲垮,没过多久就又聚拢,充分发挥榆林的优良传统,作战到
底,毫不退让。

以死相拼,死不后退,激战四天。

孙传庭取得了最后的胜利。

\section[\thesection]{}

崇祯九年(1636)七月二十日,负伤的高迎祥在山洞中被俘,与他一同被俘的,还有他的心腹将领刘哲、黄龙,他的
几万大军,已在此前彻底崩溃。

纵横世间七年的闯王高迎祥,就此结束了他的一生,在过去的七年中,他曾驰骋西北,扫荡中原,但终究未能成功。
毫无疑问,他是一个了不起的人物,然后终究到此为止。科学点的说法,是运气不好,迷信点的说法,这就是命。

高迎祥被捕的消息传到京城时,崇祯皇帝没信,不是不信,是不敢信,等人到了面前,才信。处死高迎祥的那一
刻,崇祯开始相信,自己能力挽狂澜。

最后的帅才

高迎祥被杀了,对崇祯而言,是利好消息,而对某些头领而言,似乎也不利空。

高迎祥死后,许多头领纷纷投降,比如蝎子块、冲破天等等,原先跟着高闯王干,闯王都没闯过去,自己也就消停
了。

但有某些人,是比较高兴的,比如张献忠。

张献忠跟高迎祥似乎有点矛盾,原先曾跟着打凤阳,但后来分出去单干,也不在一个地界混,算是竞争关系,高迎
祥死后,论兵力,他就是老大。

还有一个人,虽然很悲伤,却很实惠。

一直以来,李自成都跟着高迎祥干,高迎祥的外号,叫做闯王,而李自成,是闯将。据某些史料上说,李自成是高
迎祥的外甥,这话估计不怎么靠谱,但关系很铁,那是肯定的。

高迎祥的死,给了李自成两样东西。

第一样是头衔,从此,闯王这个名字,只属于李自成。第二样是兵力,高迎祥的残部,由他的部将率领,投奔了李
自成。

在这个风云变幻的乱世,离去者,是上天抛弃的,留存者,是上天眷顾的。

对张献忠和李自成而言,他们的天下之路,才刚迈出第一步。

第一步,是个坑。

我说过,对民军头领而言,崇祯九年(1636)是个流年,卢象升来了,打得乱七八糟,好不容易跑进山区,人都调走
了,又来了个孙传庭,还干掉了高迎祥。

按说坏事都到头了,可是事实告诉我们,所谓流年,是一流到底,绝不半流而废。

\section[\thesection]{}

一个比孙传庭更可怕的对手,即将出现在他们的面前,与之前的洪承畴、曹文诏、卢象升不同,他并非一个能够上
阵杀敌的将领。

他是统帅。

崇祯九年(1636),阿济格率领大军打进来时,崇祯非常紧张,但最紧张的人并不是他,而是张凤翼。

张凤翼,时任兵部尚书,他之所以紧张,是因为按惯例,如果京城(包括郊区)被袭,皇帝会不高兴,皇帝不高兴,
就要拿人撒气,具体地说,就是他。

更要命的是,崇祯老板撒气的途径,是追究责任,具体地说,是杀人,比如七年前,皇太极打到京城,兵部尚书王
洽就被干掉了,按照这个传统,他是跑不掉的。

但张部长还算识相,眼看局面没法收拾,就打了个报告,说清军入侵,是我的责任,我想戴罪立功,到前方去,希
望批准。

崇祯当即同意,打发他去了前线。

但张尚书到前线后,似乎也没去拼命,每天只干一件事----吃药。他吃的,是毒药。

这是一种比较特别的毒药,吃了不会马上死,必须坚持吃,每天吃,饭前饭后吃,锲而不舍地吃,才能吃死。

对于张尚书的举动,我曾十分疑惑,想死解腰带就行了,实在不行操把菜刀,费那么大劲干甚?

过了好几年,才想明白,高,水平真高。

如果自杀,按当时的状况,算是畏罪,死了没准抚恤金都没有,但要上阵杀敌,似乎又没那个胆,索性慢性自杀,
就当自然死亡了,还算是牺牲在工作岗位上,该享受的待遇,一点不少,老狐狸。

这兄弟不但死得慢,算得也准,清军九月初退兵,他九月初就死,连一天都没耽误。

他死了,也就拉倒了,可是崇祯同志不能拉倒,必须继续招工。

但榜样在前面,岗位风险太高,说了半天,也没人肯干。

左右为难之际,崇祯想到了一个人。

这个人很孝顺,曾三次上书,请求让自己代替父亲受罚,那是在他决心处罚杨鹤的时候。

他还清楚地记得这个人的名字----杨嗣昌。

杨嗣昌,字文弱,湖广武陵人,万历三十八年进士。

崇祯见到杨嗣昌时,很忧虑。

局势实在太差,民军闹得太凶,清军打得太狠,两头夹攻,东一榔头西一棒,实在难于应付,如此下去,亡国是迟
早的事,怎么办?

杨嗣昌只说了一句,一句就够了: ``大明若亡,必亡于流贼!''

\section[\thesection]{}

如果你仔细想想,就会发现这句话实在准得离谱。

按照杨嗣昌的说法,清军或许很强,但短时间内,并没有太大威胁,但如果不尽快解决民军,大明必定崩溃。

简单地说,就是先解决内部矛盾,再解决外部矛盾。

为了实现这个意图,杨嗣昌还提出了一个计划,这个计划在历史上的名字,是八个字:四正六隅,十面张网。

四正,包括湖广、河南、陕西、凤阳,六隅,是指山东、山西、应天、江西、四川、延绥。简单地说,这个优秀计
划的大致内容,是一部垃圾电影的名字----十面埋伏。

它的大致意思是,全国范围内,设置十个战区,四个主要,六个次要,只要发现民军出现,各地将联合围剿。简而
言之,就是划定管辖范围,在谁的地方出事,就让谁去管,出事的主管,没出事的协管。

听完杨嗣昌的计划,崇祯只说了一句话: ``我用你太晚了!''

对于这句话,朝廷的许多大臣都认为,是彻彻底底的胡扯,无论是杨嗣昌,还是他的那个什么十面埋伏,都是空口
白说,毫无价值,在他们看来,杨嗣昌同志将是第三个被干掉的兵部尚书。

然而他们错了,如果说在当时的世界上,还有一个人能够拯救危局,那么这个人,只能是杨嗣昌。

两年后,只剩十八个人的李自成,和束手投降的张献忠,可以充分说明这一点。

所有的转变,都从这一刻开始,魏忠贤、清军入侵、民变四起,朝廷争斗,紧张,痛苦,毫无生机,但始终未曾放
弃。

或许崇祯本人并不知道,经过长达八年暗无天日的努力,他即将迎来大明的曙光。

放他去!

崇祯死前,曾说过这样一句话:诸臣误我!

对于这句话,大多数人认为,是在推卸责任。

但考证完崇祯年间的朝政,我认为,这句话比较正确。确切地说,给崇祯打工的这帮大臣,除部分人外,大多数可
以分为两种,一种叫混蛋,一种叫混帐。

这个世界上,有两种人最痛苦,第一种是身居高位者,第二种是身居底层者,第一种人很少,第二种人很多。第一
种人叫崇祯,第二种人叫百姓。

\section[\thesection]{}

而最幸福的,就是中间那拨人,主要工作,叫做欺上瞒下,具体特点是,除了好事,什么都办,除了脸,什么都要。

崇祯每天打交道的,就是这拨人,比如崇祯三年(1630)西北灾荒,派下去十万石粮食赈灾,从京城出发的时候,就
只剩下五万,到地方,还剩两万,分到下面,只剩一万,实际领到的,是五千。

这事估计是办得太恶心了,崇祯也知道了,极为愤怒,亲自查办。

案情查明:先动手的,是户部官员,东西领下来,不管好坏,先拦腰切一刀,然后到了地方,巡抚先来一下,知府
后来一下,剩下的都发到乡绅手里,美其名曰代发,代着代着就代没了。

结合该案,综合明代史料,崇祯时期的官员,比较符合如下规律:脸皮的厚度,跟级别职务,大致成反比例增长。

这是比较合理的,位高权重的,几十年下来,有身份,也要面子,具体办事的就不同了,树不要皮,必死无疑,人
不要脸,天下无敌,好欺负的,就往死了欺负,能捞钱的,就往死了捞,啥名节、脸面,都顾不上,捞点实惠才是
最实在的,正如马克思所说,资本的积累,那是血淋淋的。

而且这拨人,还有个特点,什么青史留名、国家社稷,那都太遥远了,跟他们讲道理,促膝谈心都是没用的,用今
天的话说,就是吃硬不吃软。教育没有用的,骂也没有用,往脸上吐唾沫都没用,相对而言,比较合适的方式是,
把唾沫吐到眼里,再说上一句:孙子,我能治你!

比如当年追查阉党,就那么几个人,研究来研究去,连亲手干掉杨涟的许显纯,都研究成过失杀人,撤职了事,还
是崇祯亲自上阵,才把这人干掉。

再比如这事,案发后,崇祯非常生气,下令严查,查到户部,户部研究半天,拉出来几个人,说是失职,给撤了,
准备结案。

崇祯生气了,重装上阵,找出来几个主犯,杀了,剩下的,充军。

总之,崇祯年间的朝廷,是比较混账的,而带头混账的,是温体仁。

温体仁这个人,历史上的评价不高:奸臣,彻头彻尾的奸臣。

我之前说过,温体仁是个很有能力的人,精明强干,博闻强记,善于处理政务。

所以综合起来,温体仁先生的最终评价应该是,一个很有能力、精明强干、博闻强记、善于处理政务的彻头彻尾的
奸臣。

\section[\thesection]{}

温体仁,是个很复杂的坏人,复杂在无论你怎么看,都会发现,这是个真正的好人。

在工作中,温体仁是个很勤劳的人。据史料记载,他兢兢业业,每天从早干到晚,很能工作,别人几年干不了的
事,他几天就能搞定。

在生活中,他是廉政典范。据说他当首辅时,给他送礼的人从门口排到街上,等几天,他一个都不见,所有的礼品
都退,退不了的就扔,比海瑞还海瑞。

在处理与同事间的关系上,他非常谦虚,从不说别人坏话,而且很能听取他人意见。比如有个叫文震孟的人,是他
的晚辈,刚入内阁,他却非常尊敬,遇事都要找来商量,一点架子没有。

综上所诉,温体仁同志在过去的几年里,在工作上、生活上严格要求自己,团结同事,评定应为优秀。

那么接下来,我们就温体仁同志的评定问题,进行鉴定:在工作中,他反映敏捷,很有能力,但历史告诉我们,要
成为一个青史留名的坏人,没有能力,是不行的。

在生活上,他严格要求自己,不受贿赂,是因为他的仇人太多,要被人抓住把柄,是很麻烦的。

在跟同事相处时,他确实很和善,比如对文震孟,相当地客气,但原因在于,文震孟是崇祯的老师,后台很硬,而
且当时他正在挖坑,等着文老师跳下去。

如果纵观温体仁的经历,可以发现,他有个历史悠久的习惯----整人。

崇祯二年(1629),他跟周延儒合谋,整垮了钱龙锡,进了内阁,过了几年,他又整垮了周延儒,当了首辅,又过了
两年,他整垮了前途远大的文震孟,维护了自己的地位。

而且他整人的方式相当地高明。比如文震孟有个亲信,因为犯了事,要受处分。顺便说句,这人的事比较大,按情
节,至少也是撤职。

文震孟和皇帝关系好,名声很好,势力很大,且刚进内阁,对温体仁而言,是头号眼中钉,但面对如此难得的整人
机会,他毅然放弃了,非但没有落井下石,反而帮忙找了人,只给了个降职处分,很够意思,文震孟很感激。

大坑就是这样挖成的。

\section[\thesection]{}

温体仁很清楚,崇祯是个眼睛不揉沙子的人,处分官员,是只有更重,没有最重,如果从轻处理,皇帝大人是不会
答应的,肯定会加重,而文震孟同志比较正直,脾气也大,肯定要跟皇帝死磕,下场是比较明显的。

事情如他所料,皇帝大人听说后,非常震怒,把那人直接撤职,赶回家种田了,而文震孟不愧硬汉本色,跟皇帝吵
了好几天,加上温体仁煽风点火,竟然也被免了。

其实这些倒无所谓,在道上混的,整个把人,搞点阴谋,也没什么,这种事,当年张居正也没少办,之所以是奸
臣,是因为他不办事。

崇祯登基以来,干过很多事,平乱、抗金、整顿,忙完这边又忙那边,而温体仁上台以来,就只干一件事----个人
进步。

为了个人进步,他很精明,坑了钱龙锡,坑了周延儒,坑了文震孟,坑了所有的挡路或可能挡路的人。

为了个人进步,他除了精明外,有时还很傻--装傻。

有一次,崇祯把他找来,有件事情要问他的看法,温体仁当即回答:不知道。

崇祯随即追问,为何不知道。

温体仁回答:臣本愚笨(原话),只望皇上圣裁。

为了个人进步,他很团结同志,很合群,为了整倒钱龙锡,他拉拢了周延儒,两人齐心合力,还把钱谦益同志送回
了家。

当然,为了个人进步,他有时也不合群,很孤独,比如他对老朋友周延儒下手时,很干脆,没有丝毫犹豫,整人太
多,多年家里鬼都不上门,还经常跟崇祯说,我不结党,所以孤独。

明明很阴险,很狡猾,很恶心人,还动不动就说我很耿直,我很愚蠢,很能促进食欲。

能人,兼职奸人,最奸的能人,是奸人,最能的奸人,还是奸人。

鉴定完毕。

在当时朝廷里,只是混过几年的,大致都知道温体仁同志的本性,换句话说,都知道他是个奸人。

可是知道没用,因为温体仁先生是个能干的奸人,而且深得皇帝信任,谁都告不倒,时人有云:崇祯遭瘟(温)。而
且他本人心黑手狠兼皮厚,在朝廷混了多年,就快修炼成妖了,实在无人可比。

俗语有云,占着茅坑,不拉屎。客观地说,在内阁大臣的位置上,温体仁的行为并不符合这句话,确切地说,他占
着茅坑,只拉屎。

外敌入侵,内乱不止,诚此危急存亡之秋,温体仁同志孜孜不倦,为了自己而奋斗,整人、挖坑,忙得不亦乐乎,
如果让他继续折腾,大明可以提前关门

但不知是气数未尽,还是坟里的朱重八发威,天下无敌的温体仁,终究还是等来了敌人----一个他曾战胜过的敌人。

\section[\thesection]{}

自打辩论会上掉进温体仁的大坑,被赶回家,钱谦益已经在家呆了八年。八年里,除了看人种地(他是地主),主要
的娱乐,就是写诗。

这些诗大都收入他的文集,可以找来看看,心理效果明显,心情好时看,可以抑郁,心情不好时看,可以去自杀。

诗的主要意思,基本比较雷同,什么我很后悔,我要归隐,我白活了,我没意思,反正一句话,我这一辈子,是走
了黑道。

毕竟家里蹲了七八年,有点怨气很是正常,但钱谦益同志还是说错了,他走的黑道,还没有黑到头。

崇祯十年(1637),在家看人种地的钱谦益突然听说,有一个叫张汉儒的当地师爷,写了份状子告他。

要知道,钱大人虽说在上面混得很差,但到地方,还是比较恶霸的,小小师爷闹事,容易摆平。

然而没过几天,他就迎来了几位从京城来的客人----几位来抓他的客人。

在被押解的路上,钱谦益才搞明白,原来那位师爷的状子,是告御状。

这个世上,但凡有人的地方,就有斗争,但凡斗争,就有谱,包括政治斗争。一般说来,把对手弄到偏远山区,回
家养老,也就够本了,没必要赶尽杀绝,但这事,也因人而异,比如温体仁,就是个没谱的人。要么是他太过得
意,或者太恨钱谦益,总之他没打算按着谱走,某天突然心血来潮,想起在那遥远的江南,还有个没被整死的钱谦
益。

没整死,就往死里整。

但他毕竟位高权重,如果要自己动手,传出去实在太丢面子,而且容易留下把柄,所以他决定,借刀杀人。

他借到的刀,就是张汉儒。

之所以找到张汉儒,因为这人是个衙门师爷,小人物,无论如何,跟内阁首辅,都是扯不上关系的,而且张师爷长
期在法律界工作,对拍黑砖之类的工作非常熟悉,且乐此不疲。

果然,接到工作指示后,张师爷连夜工作,写出了一份状子。

所谓小人物,在写状子这点上,是不恰当的。当年大人物杨涟告魏忠贤,总共二十四条大罪,而张师爷告钱谦益的
罪状,有五十八条。

\section[\thesection]{}

这五十八条罪状,堪称经典之作,包括贪污、受贿、走私、通敌、玩权、结党,总而言之,只要你能想到的罪状,
他都写了。

但钱谦益倒没怎么慌,因为这份状子写得实在太过扯淡,都赶回家当老百姓了,还贪污个甚?玩权、掌控朝政,基本
就是胡话,崇祯这么精明的人,是不会信的。

可是他到北京,就真慌了,因为他在朝廷的朋友告诉他,他的罪状,皇帝已经批了,即将定罪。

其实钱谦益同志应该有点思想准备,要明白,温体仁是首辅,所有的公文,都是他票拟的,底下送上来,他签个
字,皇帝都未必看,要收拾你小子,小菜。

钱谦益不愧是当过东林党领导的,虽然回家消停几年,威望依然很大,他被抓过来,很多人出面,什么给事中、郎
中、尚书,包括大学士,都帮他说话,说他很冤枉,情节很曲折。

全无作用,皇帝知道了,也没理。

因为温体仁要的,就是这个效果。

八年前,兵强马壮的钱谦益,没能干过势单力孤的温体仁,是因为温体仁同志精通心理学。

他很清楚,说话人再多都没用,说了能算的只有崇祯,而崇祯最讨厌的事情,就是拉帮结派,帮忙的人越多,就越
坏事。都八年了,钱大人还没明白这个道理,实在毫无长进。

所以外面越是起哄,皇帝就越不买账,钱谦益同志的脑袋,就离鬼头刀越来越近。

温体仁已做好庆祝准备,等待着钱谦益被杀的那一天。

对此,钱谦益颇有共识,他虽在牢里,消息很灵通,感觉事情不太对劲,就亲自写了几封信,托人直接交给皇帝,
为自己辩解。

但结果很不幸,皇帝大人压根没看,很明显,他对钱谦益同志,是比较厌恶的。

钱谦益终于走到了绝路,帮忙没用,辩解没用,找皇帝都没用,找什么人似乎都没用了。

等着他的,只有喀嚓一刀。

有句俗语:万事留一线,将来好见面。这句俗语,用比较通俗的话说,就是没必要逼人太甚。

被逼得太甚的钱谦益,在阴暗的牢房里,终于使出了杀手锏。

关于钱谦益同志,之前介绍的时候,漏了一点,这位仁兄除了是东林党的头头外,还有个关系----他中进士的时
候,录取他的老师,叫做孙承宗。

\section[\thesection]{}

孙承宗同志,大家都很熟悉了,很有本事,除了能打仗外,也能搞关系,魏忠贤在的时候,都拿他没办法。

但问题是,孙承宗已经退休好几年了,说话也不好使,让他出面,估计也很麻烦。

钱谦益并没有幻想,他所以找到孙承宗,只是希望孙老师帮他找另一个人,这个人的名字叫做曹化淳。

曹化淳,是知名人士,我依稀记得,在金庸的小说《碧血剑》里,他是个死跑龙套的,且跑过好几回。

但在崇祯十年的时候,他是司礼监秉笔太监,崇祯的亲信。

在当时,能跟温体仁较劲的,也就只有他了。

但问题是,这位太监同志跟温体仁无仇,钱谦益也并非他的亲戚,犯不上较这个劲。

但钱谦益认定,这个人,能帮他的忙,救他的命。

凭什么呢?就凭十年前,他曾经写过一篇文章。

其实这篇文章,跟曹化淳并没有丝毫关系,但钱谦益相信,看着这篇文章的份上,曹化淳是会帮忙的。

因为这篇文章是王安的墓志铭。

我讲过,很久以前,魏忠贤是王安的亲信,但我没有讲过,当时王安的亲信,还有一个曹化淳。

这似乎是个比较复杂的关系。大致是这么回事。

钱谦益去找曹化淳帮忙,因为他曾经帮王安写过墓志铭,而曹化淳是王安的亲信,所以看在死人的面子上,多少要
帮点忙。外加他的老师孙承宗,面子比较广,托他出面,还有点活人的面子,死人活人双管齐下,务必成功。

成功了。

曹化淳得知消息,非常吃惊,加上这人跟着王安,还有点良心,感觉温体仁太过分,就答应帮个忙。

当然,找完了人还得听消息,钱谦益找了个人,天天去朝廷找人打听情况,连续找了三天,都没人理会,毫无消
息,第四天,他得到准确的口信:可安心矣。

可安心矣的意思,就是这事已经搞定,收拾行李,准备出狱。钱谦益也是这么理解的,他相信曹化淳已经解决了一
切。

曹化淳原本也这么认为,他上下活动,估计再过几天,事情就结了。

可是偏就没有结。

因为温体仁又来了。

温首辅以为钱谦益必死,没想到过了几天,竟然连曹化淳都折腾进来了,这样下去,事情就黄了,既然干了,就干
到底,所以他决定,连曹化淳一起整。

\section[\thesection]{}

他先散布消息,说钱谦益跟曹化淳合伙,然后还找了个证人,让他出面,指证钱谦益给曹化淳行贿,最后为万无一
失,他还请了假。

每次但凡要整人时,温体仁就会请假,回家呆着,这意思是在我请假期间,发生的任何事情,我既不知道,也不在
场,事完了,拍拍屁股再去上班。

其实对温体仁而言,钱谦益死,还是不死,都没多大关系,反正就政治地位而言,钱地主已经是个死龙套。

可做可不做的好事,最好做,可做可不做的坏事,最好不做。可惜,温体仁同志没有这个觉悟。

在他看来,钱谦益已经是个平民,而袒护钱谦益的曹化淳,不过是个司礼太监,作为内阁首辅,要办这两个人,是
很容易的。可惜他不知道,曹化淳这个人的复杂程度,远远超出他的想象。

因为曹化淳非但是太监,还有特务背景,他原本在东厂干过,到司礼监后,跟现任东厂提督太监王之心是哥们,关
系很铁。

而今温大人竟拿他开刀,实在是搞错了码头。曹公公勃然大怒,立刻跑到东厂,找到王之心,商量对策,毕竟温体
仁老奸巨猾,无懈可击,要彻底搞倒他,必须想个办法。

商量半天,办法有了。

先去找皇帝,主动报告此事,说事情很复杂,后果很严重,于是皇帝大人也震惊了,下令严查,事情闹大了。接下
来,就是去抓人,温体仁是没法抓的,但张汉儒一干人等,随便抓,抓回来,就直接丢进东厂。

据说东厂的刑罚,总共有上百种,花样繁多,能够让人恨自己生出来,比什么测谎仪好用多了,所以但凡丢进这里
的人,都很诚实。

张汉儒之流,似乎也不是什么钢铁战士。按史料的说法,进来的头天晚上,曹化淳去审了一次,就审出来了,除了
交代本人作案情况外,连幕后主使温体仁先生的诸多言行,也一起交代了。

曹化淳拿到口供,立马就奔了崇祯,崇祯看过之后,沉默了很久,然后,他说了四个字: ``体仁有党!''

这四个字的意思,用江湖术语解释:温体仁,是有门派的。

崇祯是不喜欢门派的,作为武林盟主,任何门派他都不喜欢,像温体仁这种人见狗嫌的家伙,虽然讨厌,但用着放
心。

然而这件事清楚地告诉他,温体仁同志也有门派,虽然门派比较小,但再小都是门派。

然后,他拿来了一封奏疏。

\section[\thesection]{}

这封奏疏是温体仁的辞职信,按照他的传统,为了彻底表示自己的清白,他写了这封文书,说自己身体不好,估计
也帮不了皇帝了,希望让自己回家养老。

类似这种客气信件,崇祯也会客气客气,写几句挽留的话,然后该怎么干还怎么干。

然而这一次,在这封奏疏上,他只写了三个字。

奏疏送到温体仁家时,他正在吃饭,他停了下来,等待着以往听过许多次的客套话。

然而这一次,他只听到了三个字----放他去。

放他去的意思,大致有以下几种:滚、快滚、从哪里来,滚哪里去。

据说当时就晕了过去。

温体仁终于倒了,这位聪明绝顶的仁兄,从顶上摔了下来,他落寞地回了家,第二年,死在家乡。

明代最后的一位权奸,就此落幕,确实,最后一个。

天才的计划

温体仁下台,最受益的人,应该是杨嗣昌。我查了一下,他崇祯十年(1637)三月当兵部尚书,温体仁是六月走人
的,按照温先生的脾气,像杨嗣昌这种牛人,不踩下去,是不大可能的。

温体仁走了,杨嗣昌来了,不久之后,他就将进入内阁,实践自己天才的计划。

按照杨嗣昌的计划,要实现十面张网,现在的人是不够的,必须再增兵十二万,

要增兵,就得给钱,按杨嗣昌的算法,必须增加饷银二百八十万两以上。

这个计划极为冒险,因为这笔钱杨嗣昌是不出的,崇祯也是不出的,唯一的来源,只能是找老百姓要,具体说来,
就是加租。比如原先你一年交一百多斤粮食,全家还能丰衣足食,张献忠、李自成打过来的时候,你可能会出门看
热闹,然后回家吃饭。然后官府告诉你,加租,每年交两百斤,结果全家只能吃糠,再打过来的时候,你就会出
门,帮李自成叫声好,让他们往死里打,帮你出口气。

再后来,官府告诉你,再加租,每年交四百斤,结果全家连糠都没法吃,不用人家打上门,你就会打好包袱,出门
去找闯王同志。

为了搞定西北民变,崇祯已经加过几次租了,如果再加,后果不堪设想。所以很多大臣坚决反对。

但是崇祯仍旧同意了,因为他相信,杨嗣昌的计划,能够挽救危局。

\section[\thesection]{}

最后,杨嗣昌说,要实现这个计划,我必须用一个人。

崇祯同意了。

杨嗣昌推举的这个人,叫熊文灿。

熊文灿,贵州永宁卫人。万历三十五年进士。历任礼部主事、布政使、两广总督。

杨嗣昌之所以推举熊文灿,只是因为一个误会。

不久前,两广总督的熊文灿得知了这样一个消息,崇祯的一名亲信太监来到广东探访,干啥不知道,虽说来意不
明,但对这种特派员之类的人物,熊总督心里是有数的,专程请过来吃饭。

既然是吃饭,就要喝酒,吃饱喝足,再送点礼,这位太监也很上道,非常高兴,一来二去,也就熟了。既然是熟
人,也就好说话了,双方无话不谈,从国内形势到国际风云,什么都说,但只有一件事,熊总督始终没有套出来。

你到底来干什么的?

几天后,这位太监要走,熊总督决定再请他吃顿饭,最后套口风。这顿饭吃得很满意,双方临别,喝得也多,喝着
喝着,就开始说起民变的事。

熊总督估计是喝多了,外加豪气干云,当时拍着桌子大喝一声: ``诸臣误国,如果我去,怎么会让他们闹到如此地
步(令鼠辈至是哉)!''

他万没想到,有个人比他还激动。

太监立即站了起来,他流露出多年卧底终于找到同志的表情,热烈地握住了熊总督的手,说出了熊总督套了很多
天,都没有套出来的话:``我到这里来,就是来考察你的!回去我就禀报皇上,让您去平乱,除了你,谁还能扫清流
贼!(非公不足办此贼)''

酒醒了。

熊总督到底是多年的老官僚,听到这话,当时酒就醒了,脑筋急速运转后,凭借二十余年的功底,立即提出了五
难,四不可。所谓五难,四不可,大致就是九个条件,也就是说,只有满足了这些条件,熊总督才能勉为其难地上
任。

大致说来,这就是一篇公文,就算让专职秘书写,也得写个一天两天,熊总督转眼就能完工,实在用心良苦。

然而太监也并非凡人,只用一句话,就打碎了熊总督的如意算盘: ``你放心,这些我回去都会禀报皇帝,但如果皇
帝都答应,你就别推辞了。''

就这样,熊总督的一片报国之心穿越上千里路,来到了京城。

崇祯知道了,杨嗣昌也知道了,在那遥远的南方,有一个叫熊文灿的忠义之士,愿意为国付出一切。

\section[\thesection]{}

当然了,熊总督的那些条件,自然不在话下,关键时刻,有人肯上,就难能可贵了,怎么能够吝惜条件呢?

所以在这关键时刻,杨嗣昌提出了熊文灿,而崇祯也欣然同意了,他们都相信,他能圆满实现这个天才的计划。

于是,远在千里之外的熊总督接到了调令,他即将前往中原,接替无能的前任总督王家桢。

熊文灿原先的辖区,是广东、广西两个省,而他现在的辖区,包括河南、山西、陕西、湖广、四川五省,按说,他
应该很高兴,高兴得一头撞死。

两广总督,虽说管的都是不发达地区,盗贼也多,但好歹图个平安,也没人来闹,现在这五个省,动辄就是几十万
人武装大游行,且都是巨寇、猛寇,没准哪天就被抓走,实在比较刺激。但既然来了,再跟皇帝说,其实我是忽悠
您的,那天是喝多了,估计也不行,想来想去,只能硬着头皮上了。

后世有很多人,对熊先生相当不屑,说他没有能力,没有气魄,但在我看来,熊总督并没有那么不堪。他自幼读
书,当过地方官,也到过京城,还出过海(出使琉球),见过大世面,总体而言,他只有两样东西不会----这也不
会,那也不会。

虽说熊文灿能力比较差,比较怕事,比较没有打过仗,但他能够升到两广总督,竟然是靠一项军功。

这项军功的具体内容是,他搞定了一个许多人都无法搞定的人,此人的名字,叫郑芝龙。

郑芝龙,是福建一带的著名海盗,有个著名的儿子----郑成功。

熊总督招降郑芝龙后,又用郑芝龙干掉了其他海盗,成功搞定福建沿海,最终搞定自己,获得提升。

但熊总督长年以来的表现有目共睹,骗得了上级,骗不了群众,所以他去上任的时候,许多人都认定,熊总督是壮
官一去不复返了。

崇祯十年(1637)十月,熊文灿正式来到湖广上任,迎接他的,是下属左良玉。

刚开始的时候,左良玉对熊总督还比较客气(没摸清底细),过了几天,发现熊总督黔``熊''技穷,除了天天开会,
啥本事都没有,索性就消失了。没办法,像熊总督这种熊人,左总兵是看不上的。

\section[\thesection]{}

熊总督也急了,他本不想来,来了,将领又不听使唤,自己手下的兵力,加起来还不到一万人,又要完成业绩,无
奈之下,只好使出老招数----招抚。

当时在他的辖区里,最大的两股民军,分别是张献忠和刘国能。其中张献忠有九万人,刘国能有五万。

熊文灿决定招抚这两个人。

虽然在朝廷混得还行,但论江湖经验,跟张献忠、刘国能比,熊总督还是很傻很天真,他不知道这二位的投降史,
也不了解黑道的规矩,更何况,他的兵还不到人家的十分之一,要想招降,是很困难的。

但熊总督最头疼的问题,还不是上面这些,他首先要解决的,是另一个问题----发通知。

因为张献忠和刘国能从事特殊行业,平时也没住在村里,以熊总督的情报系统,要找到这两个人,似乎很难,情急
之下,为了表示自己的诚意,熊总督派了几百个人,以今日张贴医治性病广告之决心,在村头乡尾四处贴告示,以
告知朝廷招安之诚意。对此,左良玉嗤之以鼻,连杨嗣昌听说后,也只能苦笑。

总之,在当时,熊总督在大家的眼里,大约是个笑话,笑完了,就该滚蛋了。

然而这个笑话,却以一种无人可以预料的方式,继续了下去。

过了不久,熊总督就得到消息,民军的同志们找来了。

先找上门的是张献忠,他表示,自己虽然兵强马壮,但是很想投降,很想为国效力,但鉴于投降程序很麻烦,所以
需要准备几天。

这是鬼话。

类似这种话,张献忠说的次数,估计他自己都数不清,这也是张头领看熊总督是生人,专程忽悠一把,要换了洪承
畴、卢象升等一干熟人,拉出去就剁了。

但张献忠派人上门,除了逗人玩,还有客观原因。

自打崇祯九年围剿风暴以来,经济形势是一天不如一天,高迎祥垮台了,众多头领环境都不好,随时可能破产裁
员,包括李自成在内。

高迎祥死后,孙传庭就放出了话,只要搞定了李自成,他就退休回家。

李自成在陕北对付洪承畴,已经很吃力了,又来了这么个冤家,两下夹攻,连吃败仗,没办法,陕西没法呆了,只
好掉头进了四川。

偏偏年景太差,又赶上杨嗣昌开始搞十面埋伏工程,只能接着往前跑,前有追兵,后有堵截,实在没办法,只能以
掩耳盗铃之势窝在原地,动弹不得。

\section[\thesection]{}

环境如此,张献忠混得也差,留个后路是必要的,所以找到了熊总督,当然,投降是不会的,先谈条件,过几年实
在不行了,再投降。

但他万没想到,过几天,他就会乖乖投降。

因为几天后,一个消息传来,刘国能投降了。

刘国能,外号闯塌天,在当时的诸位头领中,他大概能排到前五名,是个相当棘手的人物。

他得知熊总督招降的消息后,也找上门来,表示自己虽兵强马壮,但是很想投降,鉴于投降程序很麻烦,需要准备
几天。

其实刘国能同志的台词,跟张献忠的差不多,不同之处在于,他准备了几天,就真的投降了。

崇祯十年(1637)十一月,刘国能率五六万大军,向仅有一万人的熊文灿投降,服从改编。

小时候,我读《水浒传》的时候,曾经相当厌恶宋江,觉得他替天行道,开始造反,很是英雄,最后却又接受招
安,去征讨方腊,很是狗熊,同样的一个人,怎么前后差别那么大呢?

后来我才明白,造反的宋江,和招安的宋江,始终是同一个人。

为什么要造反?

造反,就是为了招安。

当年的宋江,原本是给政府干活,而且还有职务,根据水浒的说法,日子过得很不错,除了拿工资,还勾结黑社会
(如晁盖等人),吃点外快,还经常结交江湖兄弟,给钱从不小气(宋江:你当及时雨的名号是白给的?),只是一时手
快,在被检举之前,干掉了自己的小妾,所以才被迫流落造反。

刘国能的情况比较类似,跟张献忠不一样,他原本是读过书的,据说还有个秀才的功名,但后来不知一时冲动,还
是懵懂无知,竟然造了反,好在运气不错,这么多年没被干掉,还混得不错。

但造反这活,混得不错是不够的,毕竟工作不太稳定,危险性大,刘国能又是个比较孝顺的人,希望在家孝敬父
母,所以趁此机会,准备投降,换个工作。

刘国能这一投降,就把张献忠吓懵了:投降,还有抢生意的?

眼看问题严重,他立即派出使者,去找熊文灿,表示近期就投降了。

但是熊总督也硬气了,没有盛情挽留,反而表示,关于投降的问题,还要研究研究,才确实是否接受。

原本投降是供不应求,现在成了供大于求,卖方市场变成买方市场,麻烦了。

\section[\thesection]{}

但张献忠不愧是在朝廷里混过的,非常机灵,立刻转变思路,决定,送礼。

而且张献忠明智地意识到,熊总督的道行很深(两广总督是个肥差),单是送钱估计没戏,所以他专程找了几件古董
玉器(反正是抢来的),派人送了过去,只求一件事,让我投降。

捞钱之余还有政绩,如此好事,对熊总督而言,不干就不是人。他马上接受了投降,并且命令张献忠等人就地安置。

张献忠投降的时候,手下有七八万人,而他的驻地,在谷城(今湖北谷城)。

消息传来,崇祯极为高兴,认定熊总督是旷世奇才,大加赞赏。

杨嗣昌也很高兴,高兴之余,他提出了一个想法。

客观地讲,这是个比较阴险的想法,以致于后来很多人认为,如果照这个想法办了,天下就消停了。

这个想法的具体内容,是让张献忠在投降之前,办一件事----打李自成。

这就好比黑帮团伙,每逢拉人入伙的时候,都要让新人干点缺德事,比如砍人放火之类,专用术语,叫沾点血,今
后才好一起干。

但崇祯同志实在很讲道义,他表示人家刚来投降,就让人干这种事,似乎有点过分,所以也就这么算了。

对崇祯的信任,谷城的张献忠先生如果毫无感动,那也是很正常的。作为投降专业户,他所要考虑的,是什么时候
再造反,以及造反之后,什么时候再投降。

实际情况,似乎也是如此,崇祯十一年(1638)十月,张献忠同志已经难能可贵地投降了十个月,很明显,他也不打
算打破自己以往的投降记录,开始私下联系,蠢蠢欲动。而以熊总督的觉悟,估计只有张先生的砍刀砍到他的枕头
上,才能反应过来。

然而,就在以往场景即将重播之际,一个消息,彻底地打乱了张献忠的计划。

三个月前,陕西的李自成呆不下去,跑到了四川,刚到四川时,李自成过得还可以,后来洪承畴调集重兵围剿,他
就退往山区,双方僵持不下,李自成瞅了个空,又跑回了陕西。

以往每次李自成跑路的时候,洪承畴都礼送出境,送出去就行,确保他别回来,并不多送,但这次李自成发现,洪
承畴开始讲礼貌了。

\section[\thesection]{}

李自成从四川出来的时候,屁股后面跟着一群送行的人,比如关宁军的主要将领祖大弼、左光先以及曹文诏的侄子
曹变蛟等等。

而且这帮人很有诚意,一直跟在后面,且玩命地打,比如曹变蛟,带着三千骑兵,跟了二十多天,连衣服都没换(未
卸甲),连续击败李自成,直接把人赶出了陕西。

洪承畴之所以如此卖力,是因为挨了骂。按照防区划定,陕西归孙传庭管,四川归洪承畴管,照孙传庭的想法,李
自成进了四川,就别让他再出去了,可是洪承畴不知怎么回事,竟然又让李自成跑了。

孙传庭自然不干,认定是洪承畴玩花样,让自己背黑锅,气得不行,就告了一状。

这一状相当狠,崇祯极为愤怒,马上就批了个处分,那意思是,你想干就好点干,不想干我就干你,搞得洪承畴连
觉都没法睡,连夜开会,准备跟李自成玩真的。

对方突然下猛招,李自成没有思想准备,连陕西都没呆住,只能往外跑了。

一路往西北跑,跑了几天几夜,到了甘肃,终于没人追了。但过了几天,李自成才明白,不追是有理由的。

在明代,西北是比较荒凉的,陕西的情况还凑合,再往外跑,基本就没人了,所以压根没必要追,让他自己饿死就
行。

洪承畴的想法大致如此,事情也正如他所料,李自成混得实在太惨,没人、没粮,一个多月,损失竟然过半,已经
穷途末路。

然而出乎洪承畴意料的是,没过几天,李自成竟然穿越严密封锁,又回来了----从他的眼皮底下。

据说这件事情吓得洪大人几天没睡着觉,毕竟刚刚作过检查,还出这么大的事,随即写信,向崇祯请罪。但崇祯的
领导水平实在是高,一句话都没说,只是让他戴罪立功。

感动得眼泪汪汪的洪大人决心,从行动来报答领导的信任,马上找到孙传庭,要跟他通力合作,彻底解决李自成。

孙传庭很够意思,啥也不说了,立即调兵,发动了总攻。在一个月里,跟李自成打了四仗。

四仗之后,李自成只剩一千人。

只剩一千人的李自成,躲进了汉中的深山老林。

\section[\thesection]{}

原本几万精锐手下,被打得只剩一个零头,甚至连他最可靠的亲信祁总管,也带着人当了叛徒,在山沟里受冻的李
自成,感到了刺骨的寒意。

如果是张献忠,到这个时候,估计早就投降洗了睡了,但李自成依然不投降,他依然坚定。

但再坚定,都要解决问题,李自成明白,老呆在山里,终究是不行的,必须走出去。

经过分析,他正确地认识到,四川是不能去了,陕西也不能去了,要想有所成就,唯一的目的地,是河南。

河南有人口,有灾荒,加上还有几个从前的老战友,所以,这是李自成最好的,也是唯一的选择。

而从汉中到河南,必须经过南原。

南原,位于潼关附近,是此去必经之路,为了交通安全,李自成在出发前,进行了长期侦查,摸清地形,为了麻痹
敌人,他在山区蹲了一个多月,直到所有官军撤走,才正式上路。

一路上,李自成相当机灵,数次避过官军,但终究有惊无险地到了南原。

南原是他的最后一站,只要通过这里,他的命运就将彻底改变。

一个月前,当李自成只剩一千余人,躲进山里的时候,孙传庭认为,这是歼灭李自成的最好时机,必须立刻进山围
剿,至少也要围困。

然而洪承畴反对,他认为既不要围剿,也不用围困。

孙传庭很愤怒,他判定,李自成必定会再次出山,而且他的进攻方向,一定是河南。

这一次,洪承畴没有反对,他说,确实如此。

既然确实如此,为什么不全力围剿呢?

因为最好的围剿地点,是潼关南原,无论他从何处出发,那里是他的必经之路。

所以当李自成全军进入南原之后,他才发现,自己落入了陷阱。

据史料记载,为了伏击李自成,孙传庭集结了三万以上的兵力,每隔数十里,就埋伏一群人,山沟、丛林,只要能
塞人的地方,都塞满。

如此架势,别说突围,就算是挤,估计都挤不出去。

所以从战斗一开始,就毫无悬念,蜂拥而上的明军开始猛攻,挨了闷棍后,李自成开始突围,往附近的山里跑,然
而跑进去才发现,明军比他进来得还早,于是又往外跑,跑了一天,没能跑出去。

李自成部余下的一千多人,是他的精锐亲军,九年来,南征北战,无论是四川、陕西、钻山沟,绕树林,都坚定不
移地跟着走。

到了南原,就再也走不动了。

\section[\thesection]{}

虽然经过拼死厮杀,终究没能突围出去,从白天打到晚上,一千个人,只剩下了十八个。

李自成也是十八个人之一,他趁着夜色,率领部将刘宗敏,逃出了包围圈,他的手下全军覆没,老婆孩子全部被俘。

在一片黑暗中,孤独的李自成逃入了商洛山,在那里,他将开始艰难的等待。

至此,西北民变基本平息,几位著名头领,基本都被按平,要么灭了,要么投降,没灭也没降的,似乎也很悲哀,
毕竟连被没灭的价值都没有,是很郁闷的。

张献忠老实了,现在经济形势这么差,工作不好找,如果再去造反,吃饭都成问题,所以他收回了自己的再就业计
划,开始踏踏实实当个地主。(谷城基本归他管)

消停了。

民变基本平息,朝廷基本安定,要走的走了,要杀的杀了,要招安的也招安了,经过长达十年的混乱,大明终于等
来了曙光。

对目前的情况,崇祯很高兴,他忙活了十年,终于得到了喘息的机会,他曾对大臣说,再用十年,必将社稷兴盛,
天下太平。

十年?

一年都没有。

看到光明的崇祯并不知道,他看到的,并不是曙光,而是回光,回光返照。

其实问题很简单

几乎就在李自成全军覆没的同时,一件事情的发生,再次改变了大明帝国的命运。

崇祯十一年(1638),皇太极决定,进攻明朝,清军兵分两路,多尔衮率左翼军,岳托率右翼军,越过长城,发动猛
攻。

应该说,为了这次进攻,皇太极是很费心思的,他不去打关宁防线(也是实在打不过来),居然绕了个大圈,跑到了
密云。

密云的守军很少,但几乎没人认为,清军会从这里进攻,因为这里山多,且险,要从这里过来,要爬很多山,而且
很难爬,要爬很久。从这里打进来,那是绝无可能。

据说经常卖假古董的人,最喜欢听到的话,就是某位很懂行的顾客,很自信地表示,古董的某某特征,是绝对仿不
出来的。

皇太极有没有卖过古董,那是无从考证,但他选择的地方,就是这里,他的战术非常简单,就是爬山。

\section[\thesection]{}

清军到这里后,开始爬山,确实很多山,很难爬,足足爬了三天。

但终究是爬过来了。

清军爬过来的时候,蓟辽总督吴阿衡正在喝酒,且喝大了,脑袋比较晕,清军都到密云了,他才明白过来。

人喝醉之后,有两个后果,一、头疼,二、胆子大。

这两个后果,吴总督都有,最终后果是,头疼的吴总督,胆大无比,带着几千人,就奔着清军去了。

喝醉的人,要是一打一,仗着抗击打能力,还有点胜算,但要是群殴,也就只能被殴,没过多长时间,吴总督就被
殴死了,清军突破长城防线,全线入侵,形势万分危急。

密云距离北京,今天坐车,如果没堵车,大致是两个钟头,当年骑马,如果没堵马,估计也就一两天。

离京城一两天,也就是离崇祯一两天,所以消息传到京城,大家都很恐慌,只有几个人不慌,其中之一,就是崇祯。

崇祯之所以不慌,是因为六个月前,他就知道清军会入侵,而且连入侵的时间,他都知道得一清二楚。

六个月前,有一个人将攻击的时间,方式都告诉了他,这个人并非间谍,也不是卧底,他的名字,叫皇太极。

半年前的一天,杨嗣昌曾在私下场合对崇祯说了一个故事,这个故事比较长,所以千言万语化为一句话:

在东汉,开国皇帝汉光武帝刘秀,跟匈奴议和了。

这个故事的意思很明白,就是让崇祯去跟清朝和谈。

客观地讲,这是唯一的方法。

就军事实力而言,当时的清朝虽然军队人数不多(最大兵力二十万),但战斗力相当强(某些西方军事学家跟着凑热
闹,说是十七世纪最强的骑兵),明朝的军队人数,大致在六十万到八十万左右,但能打仗的(辽东系、洪兵、秦
兵),也就是二十多万,要真拉开了打,估计也不太行。

好在地形靠谱,守着几个山口,清军也打不过来,所以按照常理,是能够维持的。

但要命的是后院起火,出了李自成等一干猛人,只能整天拆东墙补西墙,所以杨嗣昌建议,跟清朝和谈,先解决内
部矛盾。

其实杨嗣昌的故事,还有下半段:刘秀跟匈奴和谈,搞定内部后,没过多少年,就派汉军出塞,把匈奴打得落荒而
逃。

所谓秋后算账,虽然杨嗣昌没讲,但崇祯明白,所以他决定,先忍一口气,跟清朝和谈,搞定国内问题先。

当时知道这件事情的,只有三个人,包括崇祯、杨嗣昌、太监高起潜。

\section[\thesection]{}

为保证万无一失,和谈使者是不能派的,杨嗣昌不知去哪里寻摸来个算命的,跑到皇太极那边,说要谈判。

皇太极的态度相当好,说愿意和谈,而且表示,如果和谈成功,就马上率军撤回原地。

当然,这位老兄一向不白给,末了还说了一句,如果和谈不成功,我就打过去,具体时间,是在今年的秋天。

崇祯愿意和谈,因为这是没有办法的办法。

过了几个月,在他的暗中指使下,杨嗣昌正式提出,建议与清朝和谈。

此后的事情,打死他都想不到。

建议提出后,按史料的说法,赞成的人很少,反对的人很多,事实上,是只有人反对,没有人赞成。

最先蹦出来的,是六部的几个官员,骂了杨嗣昌,然后是一拨言官,说杨嗣昌卖国,应该拉出去千刀万剐,全家死
光光。

但把这件事最终搅黄的,是最后出场的人,一个人----黄道周。

黄道周同志的简历,我就不多说了,这位仁兄后来有个外号,叫``黄圣人'',后来跟清军死战到底,堪称名副其实,

黄圣人当着皇帝的面,直接跟杨嗣昌搞辩论,一通天理人欲,先把杨嗣昌说晕,然后发挥特长(他的专业是理学),
从理论角度证明,杨嗣昌主张议和,是天理难容,违背人伦等等。

说了半天,杨嗣昌基本没有还手之力,崇祯虽然气不过,但黄先生理论基础太扎实,也没办法,等辩论完了,也不
宣布结果,当场就下了令,黄道周连降六级,到外地去搞地方建设。

皇帝大人虽然出了气,但和谈是绝不可能了,杨嗣昌再也没提,大家都能等,皇太极例外,他在关外等了几个月,
眼看没了消息,认定是被忽悠,就又打了进来。

对当时的崇祯而言,和谈是最好的出路,其实问题很简单,当年汉高祖如此英雄,还得往匈奴送人和亲,皇太极从
来没要过人,无非是要点钱,弄点干货,也就完事了。

但如此简单的问题,之所以搞得这么复杂,如此多人反对,其实只是因为一件东西----心态。

\section[\thesection]{}

天朝上邦

我曾研习过交通史(中外交往),惊奇地发现,国家与国家之间的关系,和人其实差不多,穷了,就瞧不起你,打
你,富了,就给你面子,听话。

比如美国,说谁是流氓谁就是流氓,说打谁就打谁,盟友遍布天下,时不时还搞个会盟,弄个盟军,朋友遍天下,
全世界人民都羡慕。

但这事你要真信了,那就傻了,要知道,那都是拿钱砸出来的,听话,就是友好邻邦,就给美元,给援助,很人
道,不听话,就是流氓国家,给导弹,很暴力。

而且山姆大叔是真有钱,导弹那是贵,一百万美元一个,照扔,一扔就几十个,心眼太实在,我估摸着,要全换成
手榴弹,从飞机上往下扔,也能扔个把月。

归根结底,就是两个字,实力。

谁有实力,谁就是大爷,没实力,就是孙子,美国有实力,其实也就一百多年,趁着英国老大爷跟德国老大爷干
仗,奋发图强,终成超级大爷。

相比而言,中国当大爷的时间,实在是比较长,自打汉朝起,基本就是世界先进国家,虽然中途闹腾过,后来唐朝
时又起来了,也是全世界人民羡慕,往死了派留学生,相对而言,欧洲除了罗马帝国挺得比较久,大部分时间,都
是一帮封建社会的职业文盲砍来砍去,直到明朝中期,都是世界领先。

鉴于时间太久,心态难免有点问题,比如后来英国工业革命,开始当大爷了,就派使者到中国,见到乾隆。本意大
致是要跟中国通商。

然后,乾隆同志对他们说,回去给你们乔治(当时的英国国王)带个信,就说你的孝心我知道了,你的贡品我收到了
(战舰模型),我天朝应有尽有,你就不要再费心了,给我送这些东西,是比较耽误事的,你们那里是蛮荒之地,生
活很困难,好好种地,我这里东西很多,赏点给你,回家好好用吧。

几十年后,在蛮荒之地种地的英国农民们,驾驶着战舰打了进来。

这种毛病由来已久,毕竟牛了太多年,近的朝鲜、越南、日本且不说,最远的,能打到中亚、西伯利亚,自古以
来,就是天朝上邦,四方来拜,外国使臣来访,表面上好吃好喝招待着,临走还捎堆东西,说天朝物产丰富,什么
都有,只管拿,背地里说人家是蛮夷,没文化,落后,看你可怜,给你几个赏钱。

牛的时候,怎么干都行,等到不牛了,还想怎么干都行,那就不行了。

\section[\thesection]{}

明朝的官员思维,大致就是如此,就军事实力而言,谈判是最好的选择,然而没有人选择。

这种行为,说得好听点,叫坚持原则,说得不好听,叫不识时务,明朝最后妥协的机会,就这样被一群不识时务的
人拒绝了。

十年前,我读到这里的时候,曾经很讨厌黄道周,讨厌这个固执、不识时务的人,我始终认为,他的决策是完全错
误的。

直到我知道了黄道周的结局:

七年后,当清军入关时,在家赋闲的黄道周再次出山,辅佐唐王。

唐王的地盘,大致在福建一带,他是个比较有追求的人,很想打回老家,可惜他有个不太有追求的下属----郑芝龙。

郑芝龙的打算,是混,无论清朝明朝,自己混好就行,唐王打算北伐,郑芝龙说你想去就去,反正我不去。

唐王所有的兵力,都在郑芝龙的手里,所以说了一年多,只打雷没下雨。

这时黄道周站出来,他说,战亦亡,不战亦亡,与其坐而待毙,何如出关迎敌。

唐王很高兴,说你去北伐吧,然后他说,我没有兵给你。

黄道周说,不用,我自己招兵。

然后他回到了家,找到了老乡、同学、学生,招来了一千多人,

大部分人都是百姓。

隆武元年(唐王年号,1645),黄道周出师北伐,他的军队没有经验,从未上过战场,甚至没有武器,他们拥有的最
大杀伤力武器,叫做锄头、扁担。所以这支军队在历史上的名字,叫做``扁担军''。

黄道周的妻子随同出征,她召集了许多妇女,一同前往作战,这支部队连扁担都没有,史称``夫人军''。

就算是最白痴的白痴,也能明白,这是自寻死路。

然而黄道周坚定地向前进发,明知必死无疑。正如当年他拒绝和谈,绝不妥协。

三个月后,他在江西婺源遭遇清军,打了这支队伍的第一仗,也是最后一仗。

结果毫无悬念,武器的批判没能代替批判的武器,黄道周全军覆没。黄道周被俘,被送到了南京,无数人轮番出面
劝他投降,他严辞拒绝。

三个月后,他在南京就义,死后衣中留有血书,内容共十六字:纲常万古,节义千秋,天地知我,家人无忧。

落款:大明孤臣黄道周

正如当年的他,不识时务,绝不妥协。

\section[\thesection]{}

有人曾对我说,文明的灭绝是正常的,因为麻烦太多,天灾人祸、内斗外斗,所以四大文明灭了三个,只有中国文
明流传至今,实在太不容易。

我想想,似乎确实如此,往近了说,从鸦片战争起,全世界强国(连不强的都来凑热闹)欺负我们,连打带抢带烧带
杀,还摊上个``量中华之物力''配合人家乱搞的慈禧,打是打不过,搞发展搞不了(洋务),同化也同不了(人家也有
文明),软不行,硬也不行,识时务的看法,是亡定了。

然而我们终究没有亡,挺过英法联军,挺过甲午战争,挺过八国联军,挺过抗日,终究没有亡。

因为总有那么一群不识时务的人,无论时局形势如何,无论敌人有多强大,无论希望多么渺茫,坚持,绝不妥协。

所以我想说的是,当年的这场辩论,或许决定了大明的未来,或许黄道周并不明智,或许妥协能够挽回危局,但不
妥协的人,应该得到尊重。

面对冷酷的世间、无奈的场景,遇事妥协,不坚持到底,是大多数人、大多数时间的选择,因为妥协,退让很现
实,很有好处。

但我认为,在人的一生中,至少有那么一两件事,应该不妥协,至少一两件。因为不妥协、坚持虽然不现实,很没
好处,却是正确的。

人,是要有一点精神的,至少有一点。

卢象升的选择

明朝的道路就此确定,不妥协,不退让。

相应的结果也很确定,皇太极带着兵,再次攻入关内,开始抢掠。

这次入关的,可谓豪华阵容,清朝最能打的几个,包括阿济格、多尔衮、多铎、岳托,全都来了,只用三天,就打
到密云,京城再度戒严。

要对付猛人,只能靠猛人,崇祯随即调祖大寿进京,同时,他还命令陕西的孙传庭、山东的刘泽清进京拉兄弟一
把,总之,最能打仗的人,他基本都调来了。

但问题在于,祖大寿、孙传庭这类人,虽然能力很强,但有个问题----不大服管。特别是祖大寿,自从袁崇焕死
后,他基本上就算是脱离了组织,谁当总督,都不敢管他,当然,他也不服管。

对这种无组织、无纪律的行为,崇祯很愤怒,后果不严重,毕竟能打的就这几个,你要把他办了,自己提着长矛上
阵?

但不管终究是不行的,崇祯决定,找一个人,当前敌总指挥。

这个人必须有能力强,战功多,威望高,威到祖大寿等猛人服气,且就在京城附近,说用就能用。

满足以上条件的唯一答案,是卢象升。

\section[\thesection]{}

崇祯十一年(1638),卢象升到京城赴任。

他赶到京城,本来想马上找皇帝报到,然而同僚打量他后,问:你想干嘛?

之所以有此一问,是因为这位仁兄来的时候,父亲刚刚去世,尚在奔丧,所以没穿制服,披麻戴孝,还穿着草鞋。
如果这身行头进宫,皇帝坐正中间,他跪下磕头,旁边站一堆人,实在太像灵堂。

换了身衣服,见到了崇祯,崇祯问,现在而今,怎么办?

卢象升看了看旁边的两个人,只说了一句话:主战!

站在他身边的这两人,分别是杨嗣昌、高起潜。

这个举动的意思是,知道你们玩猫腻,就这么着!

据说当时杨嗣昌的脸都气白了。

崇祯倒很机灵,马上出来打圆场,说和谈的事,那都是谣传,是路边社,压根没事。

卢象升说,那好,我即刻上阵。

第二天,卢象升赴前线就任,就在这一天,他收到了崇祯送来的战马、武器。

其实崇祯送来这些东西,只是看他远道而来,意思意思。

然而卢象升感动了,他说,以死报国!

就如同九年前,没有命令,无人知晓,他依然率军保卫京城。

他始终是个单纯的人!

几天后,卢象升得知,清军已经逼近通州,威胁京城。

当时他的手下,只有三万多人,大致是清军的一半,而且此次出战的,都是清军主力,要真死磕,估计是要休息
的,所以大多数识时务的明军将领都很消停,能不动就不动。

然而卢象升不识时务,他分析形势后,决心出战。

卢象升虽然单纯,但不蠢,他明白,要打,白天是干不动的,只能晚上摸黑去,夜袭。

在那个漆黑的夜晚,士兵出发前,他下达了一条名垂青史的军令:刀必见血!人必带伤!马必喘汗!违者斩!

趁着夜色,卢象升向着清军营帐,发起了进攻。

进攻非常顺利,清军果然没有提防,损失惨重,正当战况顺利进行之时,卢象升突然发现了一个严重的问题。

他的后军没有了。

按照约定,前军进攻之后,后军应尽快跟上,然而他等了很久,也没有看到后军,虽然现在还能打,但毕竟是趁人
不备,打了一闷棍,等人家醒过来,就不好办了,无奈之下,只能率前军撤退。

卢象升决定夜袭时,高起潜就在现场。

\section[\thesection]{}

作为监军太监,高起潜并没有表示强烈反对,他只是说,路途遥远,很难成功,卢象升坚持,他也就不说了。

但这人不但人阴(太监),人品也阴,暗地里调走了卢象升的部队,搞得卢总督白忙活半天。

差点把命搭上的卢象升气急败坏,知道是高起潜搞事,极为愤怒,立马去找了杨嗣昌。

这个举动充分说明,卢总督虽然单纯,脑袋还很好使,他知道高起潜是皇帝身边的太监,且文化低,没法讲道理,
要讲理,只能找杨嗣昌。

在杨嗣昌看来,卢象升是个死脑筋,没开窍,所以见面的时候,他就给卢象升上了堂思想教育课,告诉他,议和是
权宜之计,是伟大的,是光荣的。

卢象升只说了一句话,就让杨嗣昌闭上了嘴。

这句话也告诉我们,单纯的卢象升,有时似乎也不单纯。``我手领尚方宝剑,身负重任,如果议和,当年袁崇焕的
命运,就要轮到我的头上!''

袁崇焕这辈子最失败的地方,就是不讲政治,相比而言,卢象升很有进步。

九年前,他在北京城下,亲眼看到了袁崇焕的下场。那一幕,在他的心里,种下了难以磨灭的印象。他很清楚,如
果议和,再被朝里那帮言官扯几句,汉奸叛徒的罪名,绝对是没个跑。

与其死在刑场,不如死在战场,他下定了决心。

杨嗣昌也急了,当即大喝一声:你要这么说,就用尚方宝剑杀我!

卢象升毫不示弱:要杀也是杀我,关你何事?如今,只求拼死报国!

杨嗣昌沉默了,他明白,这是卢象升的最后选择。

卢象升想报国,但比较恶搞的是,崇祯不让。

事实上,卢象升对形势的分析是很准确的,因为夜袭失败,朝廷里那帮吃饱了没事干的言官正准备弹劾他,汉奸、
内奸之类的说法也开始流传,如果他同意和谈,估计早就被拉出去一刀了。

更麻烦的是,崇祯也生气了,因为卢象升上任以来,清军依然嚣张,多处城池被攻陷,打算换个人用用。

此时,一位名叫刘宇亮的人站了出来,说,我去。

刘宇亮,时任内阁首辅,朝廷重臣,国难如此,实在看不下去,极为激动,所以站了出来。

崇祯非常高兴,大大地夸奖了刘大人几句。

等皇帝大人高兴完了,刘大人终于说出了话的下半句:我去,阅兵。

\section[\thesection]{}

崇祯感觉很抑郁,好不容易站出来,搞得这么激动,竟然是涮我玩的?

其实这也不怪刘首辅,毕竟他从没打过仗,偶尔激动,以身报国,激动完了,回家睡觉,误会而已。

但崇祯生气了,生气的结果就是,他决定让刘首辅激动到底,一定要他去督师。

关键时刻,杨嗣昌出面了。

杨嗣昌之所以出头,并非是他跟刘首辅有什么交情,实在是刘首辅太差,太没水平,让这号人去带兵,他自己死了
倒没啥,可惜了兵。

所以他向皇帝建议,刘首辅就让他回去吧。目前在京城里,能当督师的,只有一个人。

崇祯知道这个人是谁,但他不想用。

杨嗣昌坚持,这是唯一人选。

崇祯最终同意了。

三天后,卢象升再次上任。

此时,清军的气势已经达到顶点,接连攻克城池,形势非常危急。

然而卢象升没有行动,他依然按兵未动。

因为此时他的手下,只有五千人,杨嗣昌讲道理,高起潜却不讲,阴人阴到底,调走了大部主力,留下的只有这些
人。

打,只能是死路一条,卢象升很犹豫。

就在这时,他得知了一个消息----高阳失陷了。

高阳,位处直隶(今河北),是个小县城,没兵,也没钱,然而这个县城的失陷,却震惊了所有的人。

因为有个退休干部,就住在县城里,他的名字叫孙承宗。

他培养出了袁崇焕,构建了关宁防线,阻挡了清军几十年,熬得努尔哈赤(包括皇太极)都挂了,也没能啃动。无论
怎么看,都够意思了。

心血、才华、战略、人才,这位举世无双的天才,已经奉献了所有的一切,然而,他终将把报国之誓言,进行到人
生的最后时刻。

清军进攻的时候,孙承宗七十六岁,城内并没有守军,也没有将领,更没有粮草,弹丸之地,不堪一击。

很明显,清军知道谁住在这里,所以他们并没有进攻,派出使者,耐心劝降,做对方的思想工作,对于这位超级牛
人,可谓是给足了面子。

而孙承宗的态度,是这样的,清军到来的当天,他就带着全家二十多口人,上了城墙,开始坚守。

在其感召之下,城中数千百姓,无一人逃亡,准备迎敌。

\section[\thesection]{}

每次看到这里,我都会想起黄道周,想起后来的卢象升,想起这帮顽固不化的人,正如电影集结号里,在得知战友
战死的消息后,男主角叹息一声的那句台词:老八区教导队出来的,有一个算一个,都他妈死心眼。

黄道周和孙承宗应该不是教导队出来的,但确实是死心眼。

这种死心眼,在历史中的专用称谓,叫做----气节。

失望的清军发动了进攻,在坚守几天后,高阳失守,孙承宗被俘。

对于这位俘虏,清军给予了很高的礼遇,希望他能投降,当然,他们自己也知道,这基本上是不可能的。

所以在被拒绝之后,他们毫无意外,只是开始商量,该如何处置此人。

按照寻常的规矩,应该是推出去杀掉,成全对方的忠义,比如文天祥等等,都是这么办的。

然而清军对于这位折磨了他们几十年的老对手,似乎崇拜到了极点,所以他们决定,给予他自尽的权利。

孙承宗接受了敌人的敬意,他整顿衣着,向北方叩头,然后,自尽而死。

这就是气节。

消息很快流传开来,举国悲痛。

崇祯十一年(1638)十二月二十日,听说此事的卢象升,终于下定了决心。

此前,他曾多次下令,希望高起潜部向他靠拢,合兵与清军作战,但高起潜毫不理会。而从杨嗣昌那里,他得知,
自己将无法再得到任何支援。他的粮草已极度缺乏,兵力仅有五千,几近弹尽粮绝。

而清军的主力,就在他的驻地前方,兵力是他的十倍,锋芒正锐。

弄清眼前形势的卢象升,走出了大营。

和孙承宗一样,他向着北方,行叩拜礼。

然后,他召集所有的部下,对他们说了这样一番话:

我作战多年,身经几十战,无一败绩,今日弹尽粮绝,敌众我寡,

而我决心已定,明日出战,愿战着随,愿走者留,但求以死报国,不求生还!

十二月二十一日,卢象升率五千人,向前进发,所部皆从,无一人留守。

出发的时候,卢象升身穿孝服,这意味着,他没有打算活着回来。

前进至巨鹿时,遭遇清军主力部队,作战开始。

清军的人数,至今尚不清楚,根据史料推断,至少在三万以上,包围了卢象升部。

面对强敌,卢象升毫无畏惧。他列阵迎敌,与清军展开死战,双方从早上,一直打到下午,战况极为惨烈,卢象升
率部反复冲击,左冲右突,清军损失极大。

在这天临近夜晚的时候,卢象升明白,败局已定了。

\section[\thesection]{}

他的火炮、箭矢已经全部用尽,所部人马所剩无几。

但他依然挥舞马刀,继续战斗,为了他最后的选择。

然后,清朝官员编写的史料告诉我们,他非常顽强,他身中四箭、三刀,依然奋战。他也很勇敢,自己一人,杀死
了几十名清兵,

但他还是死了,负伤力竭而死,尽忠报国而死。

相信很多人并不知道,卢象升虽然位高权重,却很年轻,死时,才刚满四十岁。

他死的时候,身边的一名亲兵为了保住他的尸首,伏在了他的身上,身中二十四箭而死。

他所部数千人,除极少数外,全部战死。

我再重复一遍,这就是气节。

在明末的诸位将领中,卢象升是个很特殊的人,他虽率军于乱世,却不扰民、不贪污,廉洁自律,坚持原则,从不
妥协。

中庸有云:国有道,不变塞焉,国无道,至死不变。

无论这个世界多么混乱,坚持自己的信念。

我钦佩这样的人。

幽默

记得不久前,我去央视对话节目做访谈,台下有问观众站起来,说,之前一直喜欢看你的书,但最近却发现了个问
题。

什么问题?

之前喜欢看,是因为你写的历史很幽默,很乐观,但最近发现你越来越不对劲,怎么会越来越惨呢?

是啊,说句心里话,我也没想到会这样,应该改变一下,这么写,比如崇祯没有杀袁崇焕,皇太极继位的时候,心
脏病突发死了,接班的多尔衮也没蹦几天,就被孝庄干掉了,然后孤儿寡母在辽东过上了安定的生活。李自成进入
山林后,没过几天,由于水土不服,也都过去了。

然后,伟大的大明朝终于千秋万代,崇祯和他的子孙们从此过着幸福的生活。

是的,现在我要告诉你的是,历史的真相。

历史从来就不幽默,也不乐观,而且在目前可知的范围内,都没有什么大团圆结局。

所谓历史,就是过去的事,它的残酷之处在于,无论你哀嚎、悲伤、痛苦、流泪、落寞、追悔,它都无法改变。

它不是观点,也不是议题,它是事实,既成事实,拉到医院急救都没办法的事实。

我感觉自己还是个比较实诚的人,所以在结局即将到来之前,我想,我应该跟您交个底,客观地讲,无论什么朝代
的史书,包括明朝在内,都不会让你觉得轻松愉快,一直以来,幽默的并不是历史,只是我而已。

\section[\thesection]{}

虽然结局未必愉快,历史的讲述终将继续,正如历史本身那样,但本着为人民服务的精神,我将延续特长,接着幽
默下去,不保证你不难受,至少高兴点。

忽悠

正如以往,清军没有长期驻守的打算,抢了东西就跑了,回去怎么分不知道,但被抢的明朝,那就惨了。

首先是将领,卢象升战死,孙传庭、洪承畴全都到了辽东,准备防守清军,我说过,这是拆了东墙补西墙,没办
法,不拆房子就塌了。

其次是兵力,能打仗的兵,无论是洪兵,还是秦兵,都调到辽东了。

所以最后的结果是,东墙补上了,西墙塌了。

说起忽悠这个词,近几年极为流行,有一次我跟人聊天,说起这个词,突然想起若有一天,此词冲出东北,走向世
界,用英文该怎么解释,随即有人发言,应该是cheat(欺骗)。

我想了一下,觉得似乎对,但不应该这么简单,毕竟如此传神的词,应该有一个传神的翻译,苦思冥想之后,我找
到了一个比较恰当的翻译:here and there

回想过去十几年,自打学习英语以来,我曾翻译过不下两篇英语文章,虽然字数较少(三百字左右),但回望短暂的
翻译生活,我认为这个词是最为恰当的。

这个词语的灵感,主要来自于熊文灿先生。作为一个没有兵力,没有经验的高级官员,他主要的武器,就是先找这
里,再找那里,属于纯忽悠型。

但值得夸奖的是,他的忽悠是很有效果的,在福建的时候,手下只有几个兵,对面有一群海盗,二话不说,先找到
了郑芝龙,死乞白赖地隔三差五去找人家(所以后来有的官员弹劾他,说他是求贼),请客送礼,反复招安,终于招
来了郑芝龙。

虽然后来证明,郑大人是不大可靠的,但在当时,是绝对够用了,后来他借助郑大人的力量,杀掉了不肯投降的海
盗刘香,平定了海乱。

这种空手道的生意,估计熊大人是做上瘾了,所以到中原上任的时候,他也玩了同一套把戏,先here招降了刘国
能,再用刘国能,there招降了张献忠,here and there,无本生意,非常高明。

但这种生意有个问题,因为熊大人本人并无任何实力,只要here不行,或者there不行,他就不行了。

\section[\thesection]{}

张献忠就是个不行的人,按照他的习惯,投降的时候,就要想好几时再造反,所以刚开始,他就不肯缴械,当然,
这也有个说法,之所以不肯缴械,是因为他认为自己罪孽深重,要留着自己这几杆枪,为朝廷效力。

熊文灿倒是很高兴,表扬了好几次,后来他果真缺兵,去找张献忠要几千人帮忙,张献忠又说还没安顿好,先休整
几天。

张献忠住的地方,就在今天襄樊的谷城地区,他老人家在此,基本就是县长了,想干什么就干什么,每天都要去县
城里转一圈,算是视察,他手下的兵也没消停,每天都要刻苦操练。

与此同时,张县长也开始意识到,自己以前的行为是有错误的,比如,每次打仗的时候,都用蛮力,很少动脑子,
且部队文化太低,没有读过兵法。为了加强理论教育,保证将来再造反的时候,有相当的理论基础,他找来了一个
叫做潘独鳌的秀才,给他当军师。

这位潘独鳌到底何许人也,待查,估计是个吴用型的人物,应该是几次举人没考上,又想干点事,就开始全心全意
地给张献忠干活,具体说就是教书,每天晚上,在张县长的统一带领下,大大小小的头目们跑去听课,课程有好几
门,比如孙子兵法等等。学习完后,张县长还要大家写出学习心得,结合实际(比如再次造反后,该怎么打仗),分
析讨论,学习气氛非常浓烈。

但他所干过最猖狂的事,还是下面这件事。

崇祯十二年(1639)年初的一天,谷城县令阮之钿接到报告,说谷城来了个人,正在和张献忠见面。

阮县令的职责是监视张献忠,加上他还比较尽责,就派了个人去打探看看到底是谁来了,谈了些什么。

没过多久,那人就回来了,他说谈了些什么,就不太知道了,但来的那个人,他认出来了。

谁?

李自成。

阮知县差点晕过去。

按照常理,自从一年前被打垮后,李自成应该躲在山沟里艰苦朴素,怎么会出来呢?还这么大摇大摆地见张献忠。

让人难以想象,这个来访者确实是李自成,他是来找张献忠要援助的。

更让人想不到的是,李自成就这么在谷城呆了几天,都没人管,又大摇大摆地走了。

其实不是没人管,是没法管。

\section[\thesection]{}

张献忠之所以嚣张,是因为他手下还有几万人,而熊大人,我说过,他的主要能力,就是这里、那里的忽悠,要真
拿刀收拾张县长,就没辙了。

而且更麻烦的是,他还收了张献忠的钱。

在明末农民起义的许多头领里,张头领是个异类,异就异在他不太像绿林好汉,反而很像官僚。

比如他在投降后,就马上马不停蹄地开始送礼,从熊文灿开始,每个月都要去孝敬几趟,而且他还喜欢串门,联络
感情,连远在京城的诸位大人,他也没忘了,经常派人去送点孝敬,所以每次有什么事,他都知道得比较早。

此外,张县长还很讲礼数,据某些史料讲,他去见上级官员时,还行下跪礼,且非常周到,具有如此天赋,竟然干
了这个,实在选错了行。

古语有云,司马昭之心,路人皆知,而张县长的心,似乎也差不多了,从上到下,都知道他要反,只不过迟早而
已,比如左良玉,曾多次上书,要求解决张献忠,还有阮知县,找熊文灿讲了几次,熊大人没理他,结果气得阮大
人回家自尽了。

总之,无论谁说张献忠要反,熊文灿都表示,这是没可能的,张献忠绝不会反。

对此,许多史料都奋笔疾书,说熊大人是白痴,是智商有问题。

我觉得这么说,是典型的人身攻击,熊大人连忽悠都能玩,绝非白痴。他之所以始终不相信张献忠会反,是因为他
不能相信。

我相信,此时此刻,熊文灿的脑海里,经常出现这样一番对话,对话的时间,是两年前,熊大人刚刚接到调令,在
以找死的觉悟准备赴任之前。

对话的地点,是庐山。对话的人,是个和尚,叫做空隐。

熊文灿找到空隐,似乎是想算卦,然而还没等他说话,空隐和尚就先说了:``你错了!(公误矣)''

怎么个错法呢?

``你估量估量,你有能搞定流贼的士兵吗?(自度所将兵足制贼死命乎)

``不能。''

``有能够指挥大局,独当一面的将领吗?(有可属大事、当一面、不烦指挥而定者乎)

``没有。''

按照上下文的关系,下一句话应该是:

那你还干个屁啊!

\section[\thesection]{}

但空隐毕竟是文明人,用了比较委婉的说法(似乎也没太委婉): ``你两样都无,上面(指皇帝)又这么器重你,一旦
你搞不定,要杀头的!''

熊文灿比较昏,等了半天,才想出一句话:``招抚可以吗?''

然而空隐回答: ``我料定你一定会招抚,但是请你记住,海贼不同流贼,你一定要慎重!''

这段对话虽然比较玄乎,但出自正统史料,并非杂谈笔记,所以可信度相当高,空隐提到的所谓海贼,指的就是郑
芝龙,而流贼,就不用多说了。

他的意思很明确,熊大人你能招降海上的,却未必能招降地上的,可问题是,熊大人只有忽悠的能耐,就算海陆空
一起来,他也只能招抚。外加他还收了张献忠的钱,无论如何,死撑都要撑下去。

死撑的结果,就是撑死。

张献忠之所以投降,不过是避避风头,现在风头过去,赶巧清军入侵,孙传庭和洪承畴两大巨头都到辽东,千载难
逢,决不能错过。

于是,崇祯十二年(1639)五月,正当崇祯兄收拾清军入侵残局的时候,张献忠再次反叛,攻占谷城。

谷城县令阮之钿真是好样的,虽然他此前服毒自尽,没有死成,又抢救过来了,但事到临头,很有点士大夫精神,
张献忠的军队攻入县城,大家都跑了,他不跑,非但不跑,就坐在家里等着,让他投降,不降,杀身成仁。

很明显,张献忠起兵,是有着充分准备的,因为他第一个目标,并非四周的州县,而是曹操。

以曹操作为外号,对罗汝才而言,是比较贴切的,作为明末三大头领之一,他很有点水平,作战极狡猾,部下精
锐,所以张献忠在起兵之前,先要拉上他。

罗汝才效率很高,张献忠刚反,他就反,并与张献忠会师,准备在新的工作岗位上继续奋斗。

顺道说一句,张献忠同志在离开谷城前,干的最后一件事,是贴布告,布告的内容,是一张名单,包括这几年他送
出去的贿赂,金额,以及受贿人的名字,全部一清二楚,诏告天下。

不该收的,终究要还。

我没有看到那份布告,估计熊文灿同志的名字,应该名列前茅。但此时此刻,受贿是个小问题,渎职才是大问题。

\section[\thesection]{}

熊文灿还算反应快,而且他很幸运,因为当时世上,能与张献忠、罗汝才匹敌的人,不会超过五个,而在他的手
下,就有一个。

在众多头领中,左良玉最讨厌,也最喜欢的,就是张献忠。他讨厌张献忠,是因为这个人太闹腾,他喜欢张献忠,
是因为这个人虽然闹腾,却比较好打,他能当上总兵,基本就是靠打张献忠,且无论张头领状态如何,心情好坏,
只要遇到他,就是必败无疑。

所以左总兵毅然决定,虽说熊大人很蠢,但看在张献忠份上,还是要去打打。

几天后,左良玉率军,与张献忠、罗汝才在襄阳附近遭遇,双方发生激战,惨败----左良玉。

所谓惨败,意思是,左良玉带着很多人去,只带着很少人跑回来。之所以失败,是因为他太过嚣张,瞧不上张献
忠,结果被人打了埋伏。

这次失败还导致了两个后果,一、由于左良玉跑得太过狼狈,丢了自己的官印,当年这玩意丢了,是没法补办的,
所以不会刻公章的左总兵很郁闷。

二, 熊文灿把官丢了,纵横忽海几十年,终于把自己忽了下去。

一个月后,崇祯下令,免去熊文灿的职务,找了个人代替他,将其逮捕入狱,一年后,斩首。

代替熊文灿的人,是杨嗣昌,逮捕熊文灿的人,是杨嗣昌,如果你还记得,当年推举熊文灿的人,是杨嗣昌。

从头到尾,左转左转左转左转,结果就是个圈,他知道,事到如今,他只剩下一个选择。

崇祯十二年(1639)九月,杨嗣昌出征。

明朝有史以来,所有出征的将领中,派头最大的,估计就是他了,当时他的职务,是东阁大学士,给他送行的,是
皇帝本人,还跟他喝了好几杯,才送他上路。

崇祯是个很容易激动的人,激动到十几年里,能换几十个内阁大学士,此外,他的疑心很重,很难相信人。

而他唯一相信,且始终相信的人,只有杨嗣昌。在他看来,这个人可信,且可靠。

可信的人,未必可靠。

对于崇祯的厚爱,杨嗣昌很感动,据史料说,他当时就哭了,且哭得很伤心,很动容,表示一定完成任务,不辜负
领导的期望。

当然,光哭是不够的,哭完之后,他还向崇祯要了两样东西,一样给自己的:尚方宝剑,另一样是给左良玉的:平
贼将军印。

\section[\thesection]{}

然后,杨嗣昌离开了京城,离开了崇祯的视线,此一去,即是永别。

崇祯十二年(1639)十月,杨嗣昌到达襄阳,第一件事,是开会。与会人员包括总督以及所有高级将领。杨嗣昌还反
复交代,大家都要来,要开一次团结的大会。

人都来了,会议开始,杨嗣昌的第一句话是,逮捕熊文灿,押送回京,立即执行。

然后,他拿出了尚方宝剑。

明白?这是个批斗会。

总督处理了,接下来是各级军官,但凡没打好的,半路跑的,一个个拉出来单练,要么杀头,要么撤职,至少也是
处分,当然,有一个人除外----左良玉。

左良玉很慌张,因为他的罪过很大,败得太惨,按杨大人的标准,估计直接就拉出去了。

但杨嗣昌始终没有修理他,直到所有的人都处理完毕,他才叫了左良玉的名字,说,有样东西要送给你。

左良玉很激动,因为杨嗣昌答应给他的,是平贼将军印。

在明代,将军这个称呼,并非职务,也不是级别,大致相当于荣誉称号,应该说,是最高荣誉,有明一代,武将能
被称为将军的,不会超过五十个人。

对左良玉而言,意义更为重大,因为之前他把总兵印丢了,这种丢公章的事,是比较丢人的,而且麻烦,公文调兵
都没办法,现在有了将军印,实在是雪中送火锅,太够意思。

杨嗣昌绝顶聪明,要按照左良玉的战绩,就算砍了,也很正常,但他很明白,现在手下能打仗的,也就这位仁兄,
所以必须笼络。先用大棒砸别人,再用胡萝卜喂他,恩威并施,自然服气。

效果确实很好,左良玉当即表示,愿意跟着杨大人,水里水里去,火里火里去,干到底。

对于杨嗣昌的到来,张献忠相当紧张,紧张到杨大人刚来,他就跑了。

因为他知道,熊文灿只会忽悠,但杨嗣昌是玩真格的,事业刚刚起步,玩不起。

张献忠对局势有足够的判断,对实力有足够的认识,可惜,跑得不足够快。

他虽然很拼命地跑,但没能跑过左良玉,心情激动的左大人热情高涨,一路狂奔,终于在四川截住了张献忠。

\section[\thesection]{}

战斗结果说明,如果面对面死打,张献忠是打不过的,短短一天之内,张献忠就惨败,败得一塌糊涂,死伤近万
人,老婆孩子,连带那位叫做潘独鳌的军师,都给抓了,由于败得太惨,跑得太快,张献忠连随身武器都丢了(大
刀),这些东西被左良玉全部打包带走,送给了杨嗣昌。

消息传来,万众欢腾,杨嗣昌极为高兴,当即命令左良玉,立即跟踪追击,彻底消灭张献忠。

左良玉依然积极,马上率军,尾随攻击张献忠。

局势大好。

士为知己者死

十几天后,左大人报告,没能追上,张献忠跑了。

杨嗣昌大怒,都打到这份上了,竟然还让人跑了,干什么吃的,怎么回事?

左良玉回复:有病。

按左大人的说法,是因为他进入四川后,水土不服,结果染了病,无力追赶,导致张献忠跑掉。

但按某些小道消息的说法,事情是这样的,在追击过程中,张献忠派人找到左良玉,说你别追我了,让我跑,结果
左良玉被说服了,就让他跑了。

这种说法的可能性,在杨嗣昌看来,基本是零,毕竟左良玉跟张献忠是老对头,而且左大人刚封了将军,正在兴头
上,残兵败将,拿啥收买左良玉?无论如何,不会干这种事。

然而事实就是这样。

左良玉很得意,张献忠很落魄,左良玉很有钱,张献忠很穷,然而张献忠确实收买了左良玉,没花一分钱。

他只是托人,对左良玉说了一句话。

这句话的大意是,你之所以受重用,是因为有我,如果没有我,你还能如此得意吗?

所谓养寇自保,自古以来都是至理名言,一旦把敌人打光了,就要收拾自己人,左良玉虽说是文盲,但这个道理也
还懂。

然而就凭这句话,要说服左良玉,是绝无可能的,毕竟在社会上混了这么多年,一句话就想蒙混过关,纯胡扯。

左良玉放过张献忠,是因为他自己有事。

因为一直以来,左良玉都有个问题----廉政问题。文官的廉政问题,一般都是贪污受贿,而他的廉政问题,是抢劫。

按史料的说法,左良玉的军队纪律比较差,据说比某些头领还要差,每到一地都放开抢,当兵的捞够了,他自己也
没少捞,跟强盗头子没啥区别。

对他的上述举动,言官多次弹劾,朝廷心里有数,杨嗣昌有数,包括他自己也有数,现在是乱,如果要和平了,追
究法律责任,他第一个就得蹲号子。

所以,他放跑了张献忠。

这下杨嗣昌惨了,好不容易找到个机会,又没了,无奈之下,他只能自己带兵,进入四川,围剿张献忠。

\section[\thesection]{}

自打追缴张献忠开始,杨嗣昌就没舒坦过。

要知道,张献忠他老人家,原本就是打游击的,而且在四川一带混过,地头很熟,四川本来地形又复杂,这里有个
山,那里有个洞,经常追到半路,人就没了,杨大人只能满头大汗,坐下来看地图。

就这么追了大半年,毫无结果,据张献忠自己讲,杨嗣昌跟着他跑,离他最近的时候,也有三天的路,得意之余,
有一天,他随口印出一首诗。

这是一首诗,一首打油诗,一首至今尚在的打油诗(估计很多人都听过),打油诗都能流传千古,可见其不凡功力,
其文如下:

前有邵巡抚,常来团转舞。

后有廖参军,不战随我行。

好个杨阁部,离我三尺路。

文采是说不上了,意义比较深刻,所谓邵巡抚,是指四川巡抚邵捷春,廖参军,是指监军廖大亨。据张献忠同志观
察,这二位一个是经常来转转,一个是经常跟着他走,只有杨嗣昌死追,可是没追上。

这首诗告诉我们,杨嗣昌很孤独。

所有的人,都在应付差事,出工不出力,在黑暗中坚持前行的人,只有他而已。

在史书上,杨嗣昌是很嚣张的,闹腾这么多年,骂他的口水,如滔滔江水,延绵不绝,然而无论怎么弹劾,就是不
倒。就算他明明干错了事,却依然支持他,哪怕打了败仗,别人都受处分,他还能升官。

当年我曾很不理解,现在我很理解。

他只是信任这个人,彻底地相信他,相信他能力挽狂澜,即使事实告诉他,这或许只能是个梦想。

毕竟在这个冷酷的世界上,能够彻底地相信一个人,是幸运的。

崇祯并没有看错人,杨嗣昌终将回报他的信任,用他的忠诚、努力,和生命。

崇祯十三年(1640)十二月,跟着张献忠转圈的杨嗣昌得到了一个令他惊讶消息:张献忠失踪。

对张献忠的失踪,杨嗣昌非常关心,多方查找,其实如张头领永远失踪,那也倒好,但考虑到他突遭意外(比如被外
星人绑走)的几率不大,为防止他在某地突然出现,必须尽快找到这人,妥善处理。

张献忠去向哪里,杨嗣昌是没有把握,四川、河南、陕西、湖广,反正中国大,能藏人的地方多,钻到山沟里就没
影,鬼才知道。

\section[\thesection]{}

但张献忠不会去哪里,他还有把握,比如京城、比如襄阳。

京城就不必说了,路远坑深,要找死,也不会这么个死法。而襄阳,是杨嗣昌的大本营,重兵集结,无论如何,绝
不可能。

下次再有人跟你说,某某事情绝无可能,建议你给他两下,把他打醒。

张献忠正在去襄阳的路上。

对张献忠而言,去襄阳是比较靠谱的,首先,杨嗣昌总跟着他跑,兵力比较空虚,其次,他的老婆孩子都关在襄
阳,更重要的是,在襄阳,有一个人,可以置杨嗣昌于死地。

为了达到这个目的,他创造了跑路的新纪律,据说一晚上跑了三百多里,先锋部队就到了,但人数不多----十二个。

虽然襄阳的兵力很少,但十二个人估计还是打不下来的,张献忠虽然没文凭,但有常识,这种事情他是不会做的。

所以这十二个人的身份,并不是他的部下,而是杨嗣昌的传令兵。

他们穿着官军的衣服,趁夜混入了城,以后的故事,跟特洛伊木马计差不多,趁着夜半无人,出来放火(打是打不过
的),城里就此一片浆糊,闹腾到天明,张献忠到了。

他攻下了襄阳,找到了自己的老婆孩子,就开始找那个能让杨嗣昌死的人。

找半天,找到了,这个人叫朱翊铭。

朱翊铭,襄王,万历皇帝的名字,是朱翊钧,光看名字就知道,他跟万历兄是同辈的,换句话说,他算是崇祯皇帝
的爷爷。

但这位仁兄实在没有骨气,明明是皇帝的爷爷,见到了张献忠,竟然大喊:千岁爷爷饶命。

很诡异的是,张献忠同志非常和气,他礼貌地把襄王同志扶起来,让他坐好。

襄王很惊慌,他说,我的财宝都在这里,任你搬用,别客气。

张献忠笑了,他说,你有办法让我不搬吗?

襄王想想也是,于是他又说,那你想要什么?

张献忠又笑了:我要向你借一样东西。

什么东西?

脑袋。

在杀死襄王的时,张献忠说:如果没有你的脑袋,杨嗣昌是死不了的。

此时的杨嗣昌,刚得知张献忠进入湖广,正心急火燎地往回赶,赶到半路,消息出来,出事了,襄阳被攻陷,襄王
被杀。

此后的事情,按很多史料的说法,杨嗣昌非常惶恐,觉得崇祯不会饶他,害怕被追究领导责任,畏罪自杀。

我个人认为,这种说法很无聊。

\section[\thesection]{}

如果是畏罪,按照杨嗣昌同志这些年的工作状况,败仗次数,阵亡人数,估计砍几个来回,都够了,他无需畏惧,
只需要歉疚。

真实的状况是,很久以前,杨嗣昌就身患重病,据说连路都走不了,吃不下饭,睡不着觉,按照今天的标准,估计
早就住进高干病房吊瓶了。

然而他依然坚持,不能行走,就骑马,吃不下,就少吃或不吃,矢志不移地追击张献忠。我重复一遍,这并非畏
惧,而是责任。

许多年来,无论时局如何动荡,无论事态如何发展,无论旁人如何谩骂,弹劾,始终支持,保护,相信,相信我能
挽回一切。

山崩地裂,不可动摇,人言可畏,不能移志,此即知己。

士为知己者死。

所以当他得知襄王被杀时,他非常愧疚,愧疚于自己没有能够尽到责任,没有能够报答一个知己的信任。

一个身患重病的人,是经不起歉疚的,所以几天之后,他就死了,病重而亡。

他终究没能完成自己的承诺。

他做得或许不够好,却已足够多。

对于杨嗣昌的死,大致有两种态度,一种是当时的,一种是后来的,这两种态度,都可以用一个字来形容----活该。

当时的人认为,这样的一个人长期被皇帝信任,实在很不爽,应该死。

后来的人认为,他是刽子手,罪大恶极,应该死。

无论是当时的,还是后来的,我都不管,我只知道,我所看到的。

我所看到的,是一个人,在绝境之中,真诚,无条件信任另一个人,而那个人终究没有辜负他的信任。

选择,没有选择

杨嗣昌死了,崇祯很悲痛,连他爷爷辈的亲戚(襄王)死了,他都没这么悲痛,非但没追究责任,还追认了一品头
衔,抚恤金养老金,一个都没少。知己死了,没法以死相报,以钱相报总是应该的。

其实和崇祯比起来,杨嗣昌是幸运的,死人虽说告别社会,但毕竟就此解脱,彻底拉倒。

而崇祯是不能拉倒的,因为他还要解决另一个问题,一个更麻烦的问题。

崇祯十三年(1340),崇祯正忙着收拾张献忠的时候,皇太极出兵了。

虽然此前他曾多次出兵,但这一次很不寻常。

因为他的目标,是锦州。

\section[\thesection]{}

自打几次到关宁防线挖砖头未果,皇太极就再也没动过锦州的心思,估计是十几年前被袁崇焕打得太狠,打出了恐
x症,到锦州城下就打哆嗦。

所以每次他进攻的时候,都要不远万里,跑路、爬山、爬长城,实在太过辛苦,久而久之,搏命精神终于爆发,决
定去打锦州。

但实践证明,孙承宗确实举世无双,他设计的这条防线,历经近二十年,他本人都死了,依然在孜孜不倦地折腾皇
太极。

皇太极同志派兵打了几次,毫无结果,最后终于怒了,决定全军上阵。

同年四月,他发动所部兵力,包括多尔衮、多铎、阿济格,甚至连尚可喜、孔有德的汉奸部队,都调了出来,同
时,还专门造了上百门大炮,对锦州发动了总攻。

守锦州的,是祖大寿

事情的发展告诉皇太极,当年他放走祖大寿,是比较不明智的。因为这位仁兄明显没有念他的旧情,还很能干,被
围了近三个月,觉得势头危险,才向朝廷求援。

而且据说祖大寿的求援书,相当地强悍,非但没喊救命,还说敌军围城,若援军前来,要小心敌人陷阱,不要轻敌
冒进,我还撑得住,七八月没问题

但崇祯实在够意思,别说七八月,连七八天都没想让他等,他当即开会,商量对策。

开会的问题主要是两个,一、要不要去,二、派谁去。

第一个问题很快解决,一定要去。

就军事实力而言,清军的战斗力,要强于明军,辽东能撑二十多年,全靠关宁防线,如果丢了,很没戏了。

第二个问题,也没什么疑问,卢象升死了,杨嗣昌快死了。

只有洪承畴。

问题解决了,办事。

崇祯十三年(1640)五月,洪承畴出兵了。

得知他出兵后,皇太极就懵了。

打了这么多年,按说皇太极同志是不会懵的,但这次实在例外,因为他虽然料定对方会来,却没有想到,会来得这
么多。

洪承畴的部队,总计人数,大致在十三万左右。属下将领,包括吴三桂、白广恩等,参与作战部队除本部洪兵外,
还有关宁铁骑一部,总之,最能打的,他基本都调来了。

本来是想玩玩,对方却来玩命,实在太敞亮了。

\section[\thesection]{}

考虑到对方的战斗能力和兵力,皇太极随即下令,继续围困锦州,不得主动出战,等待敌军进攻。

但是接下来的事情,却让他很晕。

因为洪承畴来后,看上去没有打仗的打算,安营、扎寨,每天按时吃饭,睡觉,再吃饭,再睡觉,再不就是朝城里
(锦州)喊喊话,兄弟挺住等等。

晕过之后,他才想明白,这是战术。

洪承畴的打算很简单,他判定,如果真刀真枪拼命,要打败清军,是很困难的,所以最好的方法,就是守在这里,
慢慢地耗,把对方耗走了,完事大吉。

这是个老谋深算的计划,也是最好的计划。对这一招,皇太极也没办法,要走吧,人都拉来了,路费都没着落,就
这么回去,太丢人。

但要留在这里,对方又不跟你开仗,只能耗着。

耗着就耗着吧,总好过回家困觉。

局势就此陷入僵持,清军在祖大寿外面,洪承畴在清军外面,双方就隔几十里地,就不打。

当然,清军也没完全闲着,硬攻不行,就开始挖地道,据说里三层、外三层,赛过搞网络的,密密麻麻。

但事实告诉我们,祖大寿,那真是非一般的顽强,而且他还打了埋伏,之前跟朝廷说,他可以守八个月,实际满打
满算,他守了两年。

就这样,从崇祯十三年(1640)五月到崇祯十四年(1641)五月,双方对峙一年。

六月底,开战了。

洪承畴突然打破平静,出兵,向松山攻击挺进。

这个举动大大出乎清军的意料,清军总指挥多尔衮(皇太极回家)没有提防,十万人突然扑过来,被打了个措手不
及,战败。

消息传来,皇太极晕了,一年都没动静,忽然来这么一下,你打鸡血了不成?

多年的作战经验告诉他,决战的时刻即将到来,于是他立即上马,率领所有军队,前往松山。

但是,有个问题。

当时皇太极,正在流鼻血。

一般说来,流鼻血,不算是个问题,拿张手纸塞着,也还凑合。

但皇太极的这个鼻血,据说相当之诡异,流量大,还没个停,连续流了好几天,都没办法。

但军情紧急,在家养着,估计是没辙了,于是皇太极不顾流鼻血,带病工作,骑着马,一边流鼻血,一边就这么去
了。

\section[\thesection]{}

让人难以理解的是,他没有找东西塞鼻孔,却拿了个碗,就放在鼻子下面,一边骑马一边接着,连续两天两夜赶到
松山,据说到地方时,接了几十碗。

反正我是到今天都没想明白,拿这碗干什么用的。

会战地点,松山,双方亮出底牌。

清军,总兵力(包括孔有德等杂牌)共计十二万,洪承畴,总兵力共计十三万,双方大致相等。

清军主将,包括多尔衮、多铎、济尔哈朗等精锐将领,除个把人外,都很能打。

洪承畴方面,八部总兵主将,除吴三桂外,基本都不能打。

至于战斗力,就不多说了,清军的战斗力,大致和关宁铁骑差不多,按照这个比率,自己去想。

换句话说,要摊开了打,洪承畴必败无疑。

但洪承畴,就是洪承畴。

崇祯十四年(1641) 七月二十八日,洪承畴突然发动攻击,率明军抢占制高点乳锋山,夺得先机。

他十分得意,此时他的军中的一个武官对他说了一件事:占据高地固然有利,但我军粮少,要提防清军抄袭后路。

然而洪承畴似乎兴奋过度,把那个人训了一顿,说:我干这行十几年,还需要你提醒?

大多数历史学者认为,这句话,就是他失败的最终原因。

因为就战略而言,固守是最好的方法,进攻是最差的选择,而更麻烦的是,当时的洪承畴,在进攻之前,只带了三
天的粮食。

无论如何,只带三天的粮食,是绝对不够的。

所以结论是,一贯英明的洪承畴,犯了一个愚蠢的错误,最终导致了战败。

我原本认为,这个结论很对,洪承畴很蠢,起码这次很蠢。

后来我想了想,才发现,洪承畴不蠢,起码这次不蠢。在他看似荒谬的行动背后,隐藏着一个极为精明的打算。

其实洪承畴并不想进攻,他很清楚,进攻极为危险,但他没有办法。

因为有个人一直在催他,这个人的名字叫陈新甲,时任兵部尚书,而这位陈尚书的外号,叫小杨嗣昌。

杨嗣昌同志的特点,是风风火火,玩命了干,能得这个外号,可见陈大人也不白给。

自打洪承畴打持久战,他就不断催促出战,要洪督师赶紧解决问题,是打是不打,多少给个交代。

但洪承畴之所以出战,不仅因为陈尚书唠叨,像他这样的老油条,是不会怕唐僧的。

\section[\thesection]{}

他之所以决定出战,最根本的原因,就是两个字----没钱。

我查过资料,明末时期的军饷,以十万人计,吃喝拉撒外加工资、奖金,至少在三十万两白银以上。

要在平时,这也是个大数,赶巧李自成、张献忠都在闹腾,要是洪承畴再耗个几年,崇祯同志的裤子,估计都要当
出去。

所以不打不行。

但洪承畴不愧为名将,所以在出发前,他想出了一个绝招:只带三天粮食。

要还没明白,我就解释一遍:

带上三天粮食出征,如果遇上好机会,就猛打一闷棍,打完就跑,也不怕对手断后路。

如果没有机会,看情形不妙,立马就能跑,而且回来还能说,是粮食不够了,才跑回来的,对上面有了个交代,又
不怕追究政治责任,真是比猴还精。

精过头,就是蠢

如果换了别人,这个主意没准也就成了,可惜,他的对手是皇太极。

皇太极不愧老牌军事家,刚到松山,还在擦鼻血,看了几眼,就发现了这个破绽。

八月二十日,就在洪承畴出发的第二天,他派遣将领突袭洪军后路,占领锦州笔架山粮道。

``欲战,则力不支;欲守,则粮已竭。''洪承畴彻底休息了。

当然,当然,在彻底休息前,洪承畴还有一个选择----突围。

毕竟他手里还有十几万人,要真玩命,还能试试。

于是他找来了手下的八大总兵,告诉他们事态紧急,必须通力合作,然后,他细致分配了工作,从哪里出发,到哪
里会合,一切安排妥当,散会。

我忘了说,在这八个总兵里,有一个人,叫做王朴。

第二天,突围开始。

按照洪承畴的计划,突围应该是很有秩序的,包括谁进攻,谁佯攻,谁殿后,大家排好队,慢慢来

可还没等洪承畴同志喊一二三,两个人就先跑了。

那两个先跑的人,一个是王朴。

如果没有重名,这位王朴兄,应该就是八年前,在黄河边上收钱,放走诸位头领的总兵同志。

照此看来,他还是有进步的,八年前,收钱让别人跑,现在撒腿就跑,也没想着找皇太极同志拿钱,实在难得。

而另一位带头逃跑的,史料记载有点争议,但大多数人认为,是吴三桂。

\section[\thesection]{}

无论如何,反正是散了,彻底散了,全军溃败,无法收拾,十余万人土崩瓦解,被人杀的,被踩死的,不计其数,
损失五万多人。

洪承畴还算是镇定,关键时刻,找到了曹变蛟、丘民仰,还聚了上万人,占据松山城,准备伺机撤退。

可是皇太极很不识相,非要解决洪承畴,开始围城,劝降。

洪承畴拒不投降,派使者向京城求救。

可他足足等了半年,也没有等来救兵,他很纳闷,为什么呢?

因为他糊涂了,就算用脚趾头想,也能明白,援兵是绝不会到的。

要知道,他老人家来,就是救援锦州的,能带的部队都带了,可现在他也被人围住,再去哪里找人救他?

其实洪承畴同志不知道,皇帝陛下也在等,不过他等的,不救兵,而是洪承畴的死亡通知书。

按史料的说法,洪承畴同志被围之后不久,京城这边追悼会什么的都准备好了,家属慰问,发放抚恤,追认光荣,
基本上程序都走了,就等着洪兄弟为国捐躯。

其实洪承畴原本也这么盘算来着,死顶,没法顶了,就捐躯。做梦都没想到,他连捐躯都没捐成。

崇祯十五年(1642)二月二十日,在这个值得纪念的日子,松山副将夏承德与清军密约,打开了城门,洪承畴被俘。

几个月后,无计可施的祖大寿终于投降,这次,他是真的投降了。

自崇祯十三年(1641)至崇祯十五年(1643),明朝和清朝在松山、锦州一带会战,以明军失利告终,史称``松锦大
战''。

除宁远外,辽东全境陷落,从此,明朝在关外,已无可战。

消息传到北京,照例,崇祯很悲痛,虽然这几年他经常悲痛,但这次,他尤其激动,连续几天都泪流满面,因为他
又失去了一位好同志----洪承畴。

按目击者的说法,洪承畴同志被抓之后,非常坚强,表示啥也别说了,给我一刀就行,后来英勇就义,眼睛都没
眨,很勇敢,很义气。

所以崇祯很是感动,他亲自主持了洪承畴同志的追悼会,还给他修了坛(明朝最高规格葬礼),以表彰他英勇就义的
精神。

洪承畴没有就义,他投降了。

当然,他刚被俘的时候,还是比较坚持原则的,没有投降,结果过了几天,由于平时没有注意批评和自我批评,关
键时刻没能挺住,还是投降了。

\section[\thesection]{}

至于他投降后的种种传奇,就不说了,可以直接跳过,说说他的结局。

清朝统一中原时,洪承畴由于立下大功,干了很多工作,有很大的贡献,被委以重任,担任要职。

清朝统一中原后,洪承畴由于立下大功,干了很多工作,有很大的贡献,被剥夺一切官职,光荣退休。

后来他死了,死后追封爵位,三等阿达哈哈番,这是满语,汉语翻译过来,是三等轻车都尉。

如果你不清楚清朝爵位制度,我可以解释,高级爵位分为公、候、伯、子、男五级,每个爵位,又分一到三等,一
等为最高。

男爵再往下一等,就是轻车都尉,三等轻车都尉,是轻车都尉中的最低等。我查了一下,大致是个从三品级别。

我记得洪承畴活着给明朝打工时,就是从一品太子太保,死了变从三品,有性格。

后来又过了几十年,乾隆发话,要编本书,叫做贰臣传。

所谓贰臣,通俗点说,就是叛徒,洪承畴同志以其光辉业绩,入选叛徒甲等。

在此之前,似乎就是乾隆同志,还曾发话,说抗清而死的黄道周,堪称圣人,说史可法是英雄,要给他立碑塑像。

我又想起了陈佩斯那个经典小品里的台词:叛徒,神气什么!

好像还是这个小品,另一句话是:你说我当时要是咬咬牙,不就挺过来了吗?

絮絮叨叨说这几句,只是想说:一、历史证明,叛徒是没有好下场的。同志瞧不起的人,敌人也瞧不起。

二、黄道周挺过来了,我敬佩,卢象升挺过来了,我景仰,洪承畴没挺过来,我鄙视,但理解。

咬牙挺过来,是不容易的。

所以,我不接受,但我理解。

气数

现在的崇祯,基本已经焦了,里面打得一塌糊涂,外面打得糊涂一塌,没法混了。

但他还是要撑下去,直到撑死,因为最能折腾他的那位仁兄还没出场。

据说打崇祯十二年起,崇祯同志经常做梦,梦见有一个人,在他的手上,写了一个字----有。

这是个很奇怪的梦,而且还不止一次,所以他把这个梦告诉文武大臣,让他们帮忙解释。

大家听说,都说很好,说很吉利,我想了想,有道理,因为有,总比没有好。

然而有一个人却大惊失色,这个人叫王承恩,是崇祯的贴身太监。

\section[\thesection]{}

散朝后,他找到了崇祯,对他说出了这个梦境的真实意义,可怕的寓意----大明将亡。

按照王承恩的解释,这个有,实际上是两个字。上面,是大字少一撇,下面,是明字少半边。

所以这个字的意思,就是大明,要少一半。

崇祯不信,不敢信,大明江山,自打朱重八起,二百多年,难道要毁在自己手上?

个人认为,崇祯同志过于忧虑了,因为毁不毁,这事不由他。

但这个梦实在比较准,我查了一下,他做梦的时间,大致就是那个毁他江山的人,出现的时间。

崇祯十二年(1639),一个人从深山中走出。

他的随从很少,很单薄,且很不起眼,无论是张献忠,还是皇太极,他都望尘莫及。但命中注定,他才是最终改变
一切的人,五年之后。

这人我不说,你也知道是李自成。

李自成在山里蹲了一年多,干过什么,没人知道,只知道他出来之后,进步很快。

一年多时间,他又有了几千人,占了几个县城。

但就全国而言,他实在排不上,有时经济困难,还得找张献忠拉兄弟一把。

鉴于生计困难,崇祯十三年(1640)初,他率军进入河南,新年新气象,他准备到那里碰碰运气。

通常来讲,这个想法没啥搞头,因为之前他经常全国到处出差,河南也是出差地之一,跑老跑去,没什么意外惊喜。

但这次不一样。

崇祯十三年(1640),河南大旱。

这场大旱,史料上说,是两百多年未遇之大旱,河南的景象,借用古人的话:白骨露於野,千里无鸡鸣。

大旱也好,没有鸡叫也罢,没有牛,没有猪都罢,有一样东西,是终究不会罢的----征税。

不征税,就没钱打张献忠,没钱防皇太极,必须征。

这么个环境,让人不造反,真的很难。

至于结局,不用想也知道,劳苦大众,固然劳苦,也是大众,劳苦久了,大众就要闹事,就要不交税,不纳粮,于
是接下来,就是那句著名的口号:

吃他娘,穿他娘,开了大门迎闯王,闯王来时不纳粮。

之前我说什么来着?气数。

没错,就是气数。

其实气数这玩意,说穿了,就是个使用年限,好比饼干,只能保质三天,你偏三年后吃,就只能拉肚子。

\section[\thesection]{}

又好比房子,只能住三十年,你偏要住四十年,就只能住危房,没准哪天上厕所的时候,被埋进去。

什么东西,都有使用年限,比如大米,比如王朝,比如帝国。

不同的是,大米的年限看得见,王朝的年限看不见。

看不见,却依然存在。

对于气数,崇祯是不信的,开始不信。

等到崇祯十四年,怕什么来什么,后院起火,前院也起火,卢象升死了,辽东败了,中原乱了,信了。

在一次检讨会上,他紧绷了十四年的神经,终于崩溃了。

他嚎啕大哭,一边哭,一边说:我登基十四年,饱经忧患,国家事情多,灾荒多,没有粮食,竟然人吃人,流寇四
起,这都是我失德所致啊,这都是我的错啊。

他不停地哭,不停地哭。

我同情他。

大臣们似乎也很同情,纷纷发言,说这不是您的错。

但不是皇帝的错,是谁的错呢?

气数。

几乎所有的人,众口一词,说出了这两字。

崇祯终于认了,他承认这是气数。但他终究是不甘心的:``就算是气数,人力也可补救,这么多年了,补救何用?''

然后接着大哭。

崇祯大哭的时候,李自成正在前进,在属于他的气数上,大踏步地前进。

在河南,他毫不费力地招募了十几万人,只用了两年时间,就占领了河南全境,所向披靡,先后杀死陕西总督傅宗
龙、汪乔年,以及我们的老熟人福王朱常洵。

鉴于崇祯同志的倒霉史,已经太长,鉴于他受的苦,实在太多,鉴于不想有人说我拿崇祯同志混事,还鉴于我比较
乐观,不太喜欢落井下石,所以,我决定简单点,至少保证你不至于看得太过郁闷。

李自成同志依然在前进,一年后,他进入陕西,击败了明朝的最后一位猛人孙传庭,占领西安。明军就此再无还手
之力。

崇祯十六年(1643),李自成在西安,集结所有兵力,准备向京城出发,他将终结这已延续二百七十多年的帝国。

在出发前,他发出了一道檄文,文中有八个字:

嗟尔明朝,气数已尽。

嗟尔,明朝

对于上述八个字,崇祯应该是认账的,因为不认账不行。

\section[\thesection]{}

上台以前,憋足了劲要干掉那个死人妖,死人妖干掉了,又出来党争,后金入侵,看准了袁崇焕,要他出来上岗,
一顿折腾,后金没能折腾回去,袁督师倒给折腾没了,本想着卧薪尝胆,忍几年,搞好国内经济建设,再去收复大
好河山,结果出了天灾,又出来若干人等造反。

调兵,干掉若干人等,若干人等被干掉,又出来了若干更狠的人(比如张献忠、李自成),再调兵,把若干更狠的
人,又打下去,投降的投降,跑的跑,正准备一鼓作气……

清军打进来了。

好吧,那就去打清军,全部主力调到辽东,打个一年半载,好不容易把人熬走,后院又起火了,投降的不投降,跑
进去的又跑出来。

很巧,又是灾荒,大荒,没法活,于是大家跟着一起造反。

这种编剧思路,很类似于早些年的经典电视剧《渴望》,按当时编剧思路,就是找个弱女子,什么坏事、孬事、恶
心人到死的事,都让她碰上,整体流程大致是,一棍子打过来,挺住,再一棍子打过来,继续挺住,挺到最后,就
好人一生平安了。

崇祯的故事就是这样,他挨棍子的数量,估计比渴望女主角要多得多,抗击打能力更强,但不同的是,他的故事没
有一个好的结局。

因为他的故事,是真实的,而真实的东西,往往都很残酷。

崇祯并非一个温和的人,他很急躁,很用力,用今天的话说,叫用力过猛,但那个烂摊子,不用力过猛,只能收摊。

崇祯很节俭,他的衣服、袜子,都打了补丁,请注意,打补丁的,并不一定很节俭,往往很浪费,比如后来清朝的
道光同志,衣服破了,让人去打了个补丁,五十两白银,这哥们全然是败家的,还说特便宜。

而崇祯的补丁,是他找老婆打的,免费。

此外,崇祯还有个特点:走路慢,因为走得快,里面的破衣服就会飘出来----节俭是节俭,脸面还是要的。

他工作很努力,每天白天上朝,晚上加班,据史料记载,大致要干七八个时辰(十四到十六个小时),累得半死不
活,第二天接着干。

简单地说,崇祯同志干的,是这样一份工作,没有工作范围,没有工作界限,什么都要管,每天上班,不是跟人吵
架(言官),就是看人吵架(党争),穿得破烂,吃得也少,跟老婆困觉较少,只睡五六小时,时不时还有噩耗传来,
什么北边打过来,西边打过去,祖坟被人烧了,部将被人杀了,东西被人抢了等等。

这工作,谁干?

\section[\thesection]{}

最不幸的是,崇祯同志以上所有的不幸,都无法换来一个幸福的结局----他的努力,终究失败。

但比最不幸更不幸的是(简称最最不幸),崇祯知道这点。

知道结局(注:悲剧),也无法改变,却依然要继续,这就是人生的最大悲哀。

史料告诉我们,崇祯同志应该知道自己的结局,他多次谈到命数,气数,经常对人哀叹:大明天下,奈何亡于朕手!

然而他依然尽心尽力、全力以赴、日以继夜、夜以继日、勤勤恳恳、任劳任怨、不到长城心不死,撞了南墙不回
头,往死了干,直到最后结局到来,依然没有放弃,直到兵临城下的那一天,依然没有放弃。

一个了不起的人。

结局到来的具体过程,就没必要细说了,我说过,我是个有幽默感的人,很明显,至少对于崇祯而言,这段并不幽
默。

我还说过,我是个不喜欢写废话的人,同样,对崇祯而言,这段是废话。

当然,对李自成同志而言,这段很幽默,也不是废话,他从陕西出发,只用了三个月时间,就到了北京。

三月十七日,李自成的军队到达西直门(他从西边来),开始攻城。

崇祯同志有句名言,诸臣误我,还有一句,是文臣人人可杀,三月十七日,事实证明,这两句话很正确。

内阁大臣拿不出主意,连话都没几句,且不说了,守城的诸位亲信,什么兵部尚书、吏部侍郎,压根就没抵抗,全
部打开城门投降。

当天,外城失陷,第二天,内城失陷

崇祯住在紫禁城,就是今天的故宫,故宫有多大,去过的地球人都知道。

这里,就是他的最后归宿。

三月十八日的夜晚

在这个夜晚,发生了很多事,都是后事。

其实后事处理起来,也很简单,就几句话,后妃上吊,儿子跑掉(对于后患,大多数人都不留),料理完了,身边还
有个女儿。

这个女儿,叫做长平公主,关于她的前世今生,金庸同志已经说过了,虽然相关内容(包括后来跟韦小宝同志的际
遇),百分之九十以上都是胡扯,但有一点是正确的,他确实砍断了女儿的手臂。

这个举动在历史上非常有名,实际情况,却比许多人想象中复杂得多,但无论如何,原因很简单,他不希望这个女
儿落入敌人的手中,遭受更大的侮辱。

不是残忍,而是慈爱。

我知道,许多人永远无法理解,那是因为,他们永远无需去理解。

处理完一切后,崇祯决定,去做最后一件事----自尽。

自尽,是一件比较有勇气的事,按照某位哲学家的说法,你敢死,还不敢活吗?没种。

但现实是残酷的,而今这个世界,要活下去,比死需要更大的勇气。

但崇祯的死,并非懦弱,而是一种态度,负责任的态度。

我说过,所谓王朝,跟公司单位差不多,单位出了事,领导要负责任,降级、扣工资、辞退,当然,也包括自尽。

崇祯决定自尽,他打算用这种方式,表达他的如下观点:

一、 绝不妥协。

二、 绝不当俘虏。

三、 尊严

于是,在那天夜里,崇祯登上了煤山(今天叫做景山),陪在他身边的,还有一个叫做王承恩的太监。

就这样吗?

就这样吧

他留下了最后的遗言:诸臣误朕,朕死,无面目见祖宗,自去冠冕以发覆面,任贼分尸,勿伤百姓一人。

所有的一切,都结束了。

他走向了那颗树。

应该结束了。

按照惯例,每个人的讲述结束时,会有一句结束语,而当这个王朝结束的时候,也会有一句话,最后一句话。

是的,这句话我已经写过了,不是昨天,也不是前天,而是几年以前,在我的第一本书里,朱元璋登基那一段的最
后,有一句话,就是那句,几年前,我就写好了。

还记得吗?

所有的王朝,他的开始,正如他的结束,所以才有了这句结束语,没错,就是下面这句:

走上了这条路,就不能再回头。

\section[\thesection]{}

从理论上说……

结束了

结束了吗?

结束了。

真的结束了吗?

没有。

从理论上说,文章结束了,但从实践上说,还没有。

废话

其实历史和小说不一样,因为历史的答案,所有人都知道,崇祯同志终究是要死的,而且肯定是吊死,他不会撞
墙,不会抹脖子,不会喝敌敌畏,总而言之,我不说,你们都知道。

所以结局应该是固定的,没有支线。

但是,我的结局,并不是这个。换句话说,我的文章,有两个结局,这只是第一个。

我读了十五年历史,尊重历史,所以这篇文章从头至尾,不能说无一字无来历,但大多数,都是有出处的。我不敢
瞎编。

所以第二个结局,也是真实的,只不过比较奇特,它一直在我的脑海里,最后,我决定把这个比较奇特的结局写出
来。

大头都写完了,俺歇几天,还要整理文稿,争取一两个星期后,就能完成并发出来。

写了这么久,我写得很辛苦,诸位也看得辛苦,两星期后再见,保重。

我去困觉了。

\section[\thesection]{}

徐宏祖出生的时候,是万历十五年。

在这个特定的年龄出生,真是缘分,但外面的世界,跟徐宏祖并没有多大关系,他的老家在江阴,山清水秀,不用
搞政治,也不怕被人砍,比较清净。

当然,清净归清净,在那年头,要想出人头地,青史留名,只有一条路----考试(似乎今天也是)。

徐宏祖不想考试,不想出人头地,不想青史留名,他只想玩。

按史籍说,是从小就玩,且玩得比较狠,比较特别,不扔沙包,不滚铁环,只是四处瞎转悠,遇到山就爬,遇到河
就下,人极小,胆子极大。

此外,他极其讨厌考试,长大后,让他去考科举,死都不去。该情节,放在现在,大致相当于抗拒高考。这号人,
当年跟今天的下场,估计是差不多,被拉回家打一半死不活,绝无幸免。

然而徐宏祖的父母没有打他,非但没有打他,还告诉他,你要想玩,就玩吧,做自己喜欢做的事情就行。

这种看似惊世骇俗的思想,似乎很不合理,但对徐家人而言,很合理。

对了,应该介绍一下徐宏祖同志的家世,虽然他的父母,并非什么大人物,也没名气,但他有一位祖先,还算是很
有名的,当然,不是好名。

在徐宏祖出生前九十年,徐家的一位先辈进京赶考,路上遇到了一位同伴,叫做唐寅,又叫唐伯虎。

没错,他就是徐经。

后来的事情,之前讲过,据说是徐经作弊,结果拉上了唐伯虎,大家一起完蛋,进士没考上,连举人都没了,所以
徐经同志痛定思痛,对坑害了无数人(主要是他)的科举制度深恶痛绝,教育子孙,要与这个万恶的制度决裂,爱考
不考,去他娘的。

对这段百年恩怨,徐宏祖是否了解,不清楚,但他会用,那是肯定的。更重要的是,徐家虽说没有级别,还有点
钱,所以他决定,索性不考了,出去旅游。

刚开始,他旅游的范围,主要是江浙一带,比如紫金山、太湖、普陀山等等。后来愈发勇猛,又去了雁荡山、九华
山、黄山、武夷山、庐山等等。

但这里,存在着一个问题----钱。

\section[\thesection]{}

旅行家和大侠的区别在于,旅行家是要花钱的,列一下,大致包括以下费用:交通费、住宿费、导游费、餐饮费、
门票费,如果地方不地道,还有个挨宰费。

我说过,徐家是有钱的,但只是有点钱,没有很多钱,大约也就是个中产阶级。按今天的标准,一年去旅游一次,
也就够了,但徐宏祖的旅行日程是:一年休息一次。

他除了年底回家照顾父母外,一年到头都在外面,但就这么个搞法,他家竟然还过得去。

原因很简单,比如交通费,他不坐火车、也不坐汽车(想坐也没),少数骑马,多靠步行(骑马爬山试试)。

住宿费,基本不需要,徐宏祖去的地方,当年大都没有人去,别说三星级,连孙二娘的黑店都没有,树林里、悬崖
上,打个地铺,也就睡了。

餐饮费,也没有,我考察过,徐宏祖同志去的地方,也没什么餐馆,每次他出发的时候,都是带着干粮,而且他很
扛饿,据说能扛七八天,至于喝水,山里面,那都是矿泉水。

门票费也是不用了,当年谁要能在徐宏祖同志去的地方,设个点收门票,那只能说明,他比徐宏祖还牛,该收。

挨宰费是没有的,但挨宰是可能的,且比较敞亮,从没有暗地加价坑钱,都是拿刀,明着来抢。要知道,没门票的
地方,固然没有奸商,却很可能有强盗。

据本人考证,徐宏祖最大的花销,是导游费用。作为一个旅行家,徐宏祖很清楚,什么都能省,这笔钱是不能省
的,否则走到半山腰,给你挖个坑,让你钻个洞,那就休息了。

就这样,家境并不十分富裕的徐宏祖,穿着俭朴的衣服,没有随从,没有护卫,带着干粮,独自前往名山大川,风
餐露宿,不怕吃苦,不怕挨饿,一年只回一次家,只为攀登。

从俗世的角度,徐宏祖是个怪人,这人不考功名,不求做官,不成家立业,按很多人的说法,是毁了。

我知道,很多人还会说,这种生活荒谬,是不符合常规的,是不正常的,是缺根弦的,是精神有问题的。

我认为,说这些话的人,是吃饱了,撑的,人只活一辈子,如何生活,都是自己的事,自己这辈子浑浑噩噩地没活
好,厚着脸皮还来指责别人,有多远,就去滚多远。

\section[\thesection]{}

徐宏祖旅行的唯一阻力,是他的父母。他的父亲去世较早,只剩他的母亲无人照料。圣人曾经教导我们:父母在,
不远游。

所以在出发前,徐宏祖总是很犹豫,然而他的母亲找到他,对他说了这样一番话:``男儿志在四方,当往天地间一
展胸怀!''

就这样,徐宏祖开始了他伟大的历程。

他二十岁离家,穿着布衣,没有政府支持,没有朋友帮助,独自一人,游历天下二十余年,他去过的地方,包括湖
广、四川、辽东、西北,简单地说,全国十三省,全部走遍。

他爬过的山,包括泰山、华山、衡山、嵩山、终南山、峨眉山,简单地说,你听过的,他都去过,你没听过的,他
也去过。

此外,黄河、长江、洞庭湖、鄱阳湖,金沙江、汉江,几乎所有江河湖泊,全部游历。

在游历的过程中,他曾三次遭遇强盗,被劫去财物,身负刀伤,还由于走进大山,无法找到出路,数次断粮,几乎
饿死。最悬的一次,是在西南。

当时,他前往云贵一带,结果走到半路,突然发现交通中断,住处被当地土著围着,过了几天,外面又来了明军,
又开始围,围了几天,就开始打,打了几天,就开始乱。徐宏祖好歹是见过世面的,跑得快,总算顺利脱身。

在旅行的过程中,他还开始记笔记,每天的经历,他都详细记录下来,鉴于他本人除姓名外,还有个号,叫做霞
客,所以后来,他的这本笔记,就被称为《徐霞客游记》。

崇祯九年(1636),五十岁的徐宏祖决定,再次出游,这也是他的最后一次出游,虽然他自己没有想到。

正当他考虑出游方向的时候,一个和尚找到了他。

这个和尚的法号,叫做静闻,家住南京,他十分虔诚,非常崇敬鸡足山迦叶寺的菩萨,还曾刺破手指,血写过一本
法华经。

鸡足山在云南。

当时的云南鸡足山,算是蛮荒之地,啥也不通,要去,只能走着去。

很明显,静闻是个明白人,他知道自己要一个人去,估计到半路就歇了,必须找一个同伴。

徐宏祖的名气,在当时已经很大了,所以他专门找上门来,要跟他一起走。

\section[\thesection]{}

对徐宏祖而言,去哪里,倒是个无所谓的事,就答应了他,两个人一起出发了。

他们的路线是这样的,先从南直隶出发,过湖广,到广西,进入四川,最后到达云贵。

不用到达云贵,因为到湖广,就出事了。

走到湖广湘江(今湖南),没法走了,两人坐船准备渡江。

渡到一半,遇上了强盗。

对徐宏祖而言,从事这种职业的人,他已经遇到好几次了,但静闻大师,应该是第一次。此后的具体细节不太清
楚,反正徐宏祖赶跑了强盗,但静闻在这场风波中受了伤,加上他的体质较弱,刚撑到广西,就圆寂了。

徐宏祖停了下来,办理静闻的后事。

由于路上遭遇强盗,此时,徐宏祖的路费已经不足了,如果继续往前走,后果难以预料。

所以当地人劝他,放弃前进念头,回家。

徐宏祖跟静闻,是素不相识的,说到底,也就是个伴,各有各的想法,静闻没打算写游记,徐宏祖也没打算去礼
佛,实在没有什么交情。而且我还查过,他此前去过鸡足山,这次旅行对他而言,并没有太大的意义。

然而他说,我要继续前进,去鸡足山。

当地人问:为什么要去。

徐宏祖答:我答应了他,要带他去鸡足山。

可是,他已经去世了。

我带着他的骨灰去。答应他的事情,我要帮他做到。

徐宏祖出发了,为了一个逝去者的愿望,为了实现自己的承诺,虽然这个逝去者,他并不熟悉。

旅程很艰苦,没有路费的徐宏祖背着静闻的骨灰,没有任何资助,他只能住在荒野,靠野菜干粮充饥,为了能够继
续前行,他还当掉了自己所能当掉的东西,只是为了一个承诺。

就这样,他按照原定路线,带着静闻,翻阅了广西十万大山,然后进入四川,越过峨眉山,沿着岷江,到达甘孜松
潘。

渡过金沙江,渡过澜沧江,经过丽江、经过西双版纳,到达鸡足山。

迦叶寺里,他解开了背上的包裹,拿出了静闻的骨灰。

到了。

我们到了。

他郑重地把骨灰埋在了迦叶寺里,在这里,他兑现了承诺。

然后,他应该回家了。

但他没有。

\section[\thesection]{}

从某个角度讲,这是上天对他的恩赐,因为这将是他的最后一次旅途,能走多远,就走多远吧。

他离开鸡足山,又继续前行,行进半年,翻越了昆仑山,又行进半年,进入藏区,游历几个月后,踏上归途。

回去没多久,就病了。

喜欢锻炼的人,身体应该比较好,天天锻炼的人(比如运动员),就不一定好,旅游也是如此。

估计是长年劳累,徐宏祖终究是病倒了,没能再次出行。崇祯十四年(1641),病重逝世,年五十四。

他所留下的笔记,据说总共有两百多万字,可惜没有保留下来,剩余的部分,大约几十万字,被后人编成《徐霞客
游记》。

在这本书里,记载了祖国山川的详细情况,涉及地理、水利、地貌等情况,被誉为十七世纪最伟大的地理学著作,
翻译成几十国语言,流传世界。

好的,总结应该出来了,这是一个伟大的地理学家的故事,他为了研究地理,四处游历,为地理学的发展做出了突
出贡献,是中华民族的骄傲。

是这样吗?

不是的

其实讲述这人的故事,只想探讨一个问题,他为何要这样做。

没有资助,没有承认(至少生前没有),没有利益,没有前途,放弃一切,用一生的时间,只是为了游历?

究竟为了什么?

我很疑惑,很不解,于是我想起另一个故事。

新西兰登山家希拉里,在登上珠穆朗玛峰后,经常被记者问一个问题:你为什么要爬?

他总不回答,于是记者总问,终于有一次,他答出了一个让所有人都无法再问的答案:因为它(指珠峰),就在那里!

因为它就在那里。

其实这个世上很多事,本不需要理由,之所以需要理由,是因为很多人喜欢找抽,抽久了,就需要理由了。

正如徐霞客临终前,所说的那句话:``汉代的张骞,唐代的玄奘,元代的耶律楚材,他们都曾游历天下,然而,他
们都接受了皇帝的命令,受命前往四方。''

``我只是个平民,没有受命,只是穿着布衣,拿着拐杖,穿着草鞋,凭借自己,游历天下,故虽死,无憾。''

说完了。

我要讲的那样东西,就在这个故事里。

我相信,很多人会问,你讲了什么?

用如此之多的篇幅,讲述一个王朝的兴起和衰落,在终结的时候,却说了这样一个故事,你到底想说什么?

我重复一遍,我要讲的那样东西,就在这个故事里,已经讲完了。

所以后面的话,是讲给那些不明白的人,明白的人,就不用继续看。

此前,我讲过很多东西,很多兴衰起落、很多王侯将相、很多无奈更替,很多风云变幻,但这件东西,我个人认
为,是最重要的。

因为我要告诉你,所谓千秋霸业,万古流芳,以及一切的一切,只是粪土。先变成粪,再变成土。

现在你不明白,将来你会明白,将来不明白,就再等将来,如果一辈子都不明白,也行。

而最后讲述的这件东西,它超越上述的一切,至少在我看来。

但这件东西,我想了很久,也无法用准确的语言,或是词句来表达,用最欠揍的话说,是只可意会,不可言传。

然而我终究是不欠揍的,在遍阅群书,却无从开口之后,我终于从一本不起眼,且无甚价值的读物上,找到了这句
适合的话。

这是一本台历,一本放在我面前,不知过了多久,却从未翻过,早已过期的台历。

我知道,是上天把这本台历放在了我的桌前,它看着几年来我每天的努力,始终的坚持,它静静地,耐心地等待着
终结。

它等待着,在即将结束的那一天,我将翻开这本陪伴我始终,却始终未曾翻开的台历,在上面,有着最后的答案。

我翻开了它,在这本台历上,写着一句连名人是谁都没说明白的名人名言。

是的,这就是我想说的,这就是我想通过徐霞客所表达的,足以藐视所有王侯将相,最完美的结束语:

成功只有一个----按照自己的方式,去度过人生。

(全文完)

后 记

本来没想写,但还是写一个吧,毕竟那么多字都写了。

记得前段时间,去央视面对面访谈,主持人问我,书写完的时候,你有什么感觉?

其实这个问题,我曾经问过我自己很多次,高兴、兴奋、沮丧,什么都有可能。

但当这刻来到的时候,我只感觉,没有感觉。

不是矫情。

怎么说呢,因为我始终觉得写这玩意,是个小得没法再小的事。然而很快有人告诉我,你的书在畅销排行榜蹲了几
天,几月,几年,然后是几十万册、几百万册,直到某天,某位仁兄很是激动地对我说,改革开放三十年,这本书
发行量,可以排进前十五名。

有意思吗?说实话,有点意思。

雷打不动的,还有媒体,报纸、期刊、杂志、电视台,从时尚到社会,从休闲到时局,从中央到地方,从中国到外
国,借用某位同志的话,连宠物杂志都上门找你。平均一天几个访问,问的问题,也大致雷同,翻来覆去,总也是
那么几个问题,每天都要背几遍,像我这么乏味的人,谁愿意跟我聊,那都是交差,我明白。

外型土得掉渣,也硬拽上若干电视讲坛,讲一些相当通俗,相当大众,相当是人就能听明白的所谓历史(类似故事
会),当然,该问的还得问下去,还讲的可能还得讲下去。

这个没意思,没意思,也得接着混。

我始终觉得,我是个很平凡的人,扔人堆里就找不着,放在通缉令上,估计都没人能记住,到现在还这么觉得,今
天被人记住了,明天就会被人忘记,今天很多人知道,明天就不知道,所以所谓后记,所谓感想,所谓获奖感言之
类的无聊的,乱扯的,自欺欺人的,胡说八道的,都休息吧。

那么接下来,说点有必要说的话。

首先,是感谢,非常之感谢。

记得马未都同志有次对我说,这世上很多人都有不喜欢你的理由。因为你成名太早,成名太盛,太过年轻,人家不
喜欢你,那是有道理的,所以无论人家怎么讨厌你,怎么逗你,你都得理解,应该理解。

我觉得这句话很有道理,所以一直以来,我都无所谓。

但让我感动的是,广大人民群众应该还是喜欢我的,一直以来,我都得到了许多朋友的帮助,没有你们,我撑不到
今天,谢谢你们,非常真诚地谢谢你们。

谢谢。

然后是心得,如果要问我,有个什么成功心得,处世原则,我觉得,只有一点,老实做人,勤奋写书,无它。

几年来,我每天都写,没有一天敢于疏忽,不惹事,不闹事,即使所谓盛名之下,我也从未懈怠,有人让我写文章
推荐商品,推荐什么就送什么,还有的希望我做点广告,费用可以到六位数,顺手就挣。

我没有理会。因为我不是商人。

出版商亲自算给我听,由于我坚持把未出版部分免费发表,因此每年带来的版税损失,可以达到七位数,这还不包
括盗版,以及各种未经许可的文本。

我依然坚持,因为我相信,这是个自由的时代,每个人有看与不看的自由,也有买和不买的自由,任何人都不应该
被强迫。

这是我的处世原则,我始终坚持,或许很多人认为这么干很吃亏,但结果,相信你已经看到。

好的,还有历史,既然写了历史,还要说说对历史的看法。

就剩几句了,虚的就算了,来点实在的吧。

很多人问,为什么看历史,很多人回答,以史为鉴。

现在我来告诉你,以史为鉴,是不可能的。

因为我发现,其实历史没有变化,技术变了,衣服变了,饮食变了,这都是外壳,里面什么都没变化,还是几千年
前那一套,转来转去,该犯的错误还是要犯,该杀的人还是要杀,岳飞会死,袁崇焕会死,再过一千年,还是会死。

所有发生的,是因为它有发生的理由,能超越历史的人,才叫以史为鉴,然而我们终究不能超越,因为我们自己的
欲望和弱点。

所有的错误,我们都知道,然而终究改不掉。

能改的,叫做缺点,不能改的,叫做弱点。

顺便说下,能超越历史的人,还是有的,我们管这种人,叫做圣人。

以上的话,能看懂的,就看懂了,没看懂的,就当是说疯话。

最后,说说我自己的想法。

因为看得历史比较多,所以我这个人比较有历史感,当然,这是文明的说法,粗点讲,就是悲观。

这并非开玩笑,我本人虽然经常幽默幽默,但对很多事情都很悲观,因为我经常看历史(就好比很多人看电视剧一
样),不同的是,我看到的那些古文中,只有悲剧结局,无一例外。

每一个人,他的飞黄腾达和他的没落,对他本人而言,是几十年,而对我而言,只有几页,前一页他很牛,后一页
就怂了。

王朝也是如此。

真没意思,没意思透了。

但我坚持幽默,是因为我明白,无论这个世界有多绝望,你自己都要充满希望

人生并非如某些人所说,很短暂,事实上,有时候,它很漫长,特别是对苦难中的人,漫长得想死。

但我坚持,无论有多绝望,无论有多悲哀,每天早上起来,都要对自己说,这个世界很好,很强大。

这句话,不是在满怀希望光明时说的,很绝望、很无助,很痛苦,很迷茫的时候,说这句话。

要坚信,你是一个勇敢的人。

因为你还活着,活着,就要继续前进。

曾经有人问我,你怎么了解那么多你不应该了解的东西,你怎么会有那么多六七十岁的人才有的感受。我说我不知
道。跟我一起排话剧的田沁鑫导演说,我是上辈子看了太多书,憋屈死了,这辈子来写。

我没话说。

还会不会写?应该会,感觉还能写,还写得出来,毕竟还很年轻,离退休尚早,尚能饭

继续写之前,先歇歇,累得慌。

是的,这个世界还是很有趣的。

最后送一首食指的诗给大家,我所要跟大家讲的,大致就在其中了吧。

当蜘蛛网无情地查封了我的炉台

当灰烬的余烟叹息着贫困的悲哀

我依然固执地铺平失望的灰烬

用美丽的雪花写下:相信未来



当我的紫葡萄化为深秋的露水

当我的鲜花依偎在别人的情怀

我依然固执地用凝霜的枯藤

在凄凉的大地上写下:相信未来



我要用手指那涌向天边的排浪

我要用手掌那托住太阳的大海

摇曳着曙光那枝温暖漂亮的笔杆

用孩子的笔体写下:相信未来



我之所以坚定地相信未来

是我相信未来人们的眼睛

她有拨开历史风尘的睫毛

她有看透岁月篇章的瞳孔



不管人们对于我们腐烂的皮肉

那些迷途的惆怅、失败的苦痛

是寄予感动的热泪、深切的同情

还是给以轻蔑的微笑、辛辣的嘲讽



我坚信人们对于我们的脊骨

那无数次的探索、迷途、失败和成功

一定会给予热情、客观、公正的评定

是的,我焦急地等待着他们的评定



朋友,坚定地相信未来吧

相信不屈不挠的努力

相信战胜死亡的年轻

相信未来、热爱生命



二十多岁写,写完还是二十多岁,有趣。

是的,这个世界还是很有趣的

无需害怕

无需绝望

要相信自己



迈陂塘•冬寒有感

过匆匆,世间繁华,转眼几番风雨。随生随灭逐天意,你我空得辛苦。自来去,生与死,皆是赤身无所取。任醇酒
满樽,美人在怀,都作尘与土。

逍遥游,岂能此生虚度?志气当传万古。丈夫直行如日月,计较沉浮几许?君且住,君不见,公道自在人心处。谁共
此语?念易水萧萧,燕然漠漠,有英雄无数。

汪泰恒赋词,取十年前旧词。


\end{document}



