% Demonstration of pgf-umlsd.sty, a convenient set of macros for drawing
% UML sequence diagrams. Written by Xu Yuan <xuyuan.cn@gmail.com> from
% Southeast University, China.

\documentclass{article}

\usepackage{fontspec,xunicode,xltxtra}
% ��վʹ��Georgia
% \setmainfont[Mapping=tex-text]{Georgia}
% \setmainfont[Mapping=tex-text]{Garamond}
% \setmainfont[Mapping=tex-text]{Linux Libertine}
% \setmainfont[Mapping=tex-text,Ligatures=Common]{Adobe Garamond Pro}
\setmainfont[Mapping=tex-text,Ligatures=Common]{Myriad Pro}

% \setmainfont[Ligatures=Required,Mapping=tex-text]{Palatino Linotype}
% \setsansfont[Ligatures=Required,Mapping=tex-text]{Myriad Pro}
% \setsansfont[Mapping=tex-text]{Arial}
\setsansfont[Mapping=tex-text,Numbers=Uppercase]{Myriad Pro}
\setmonofont[Mapping=tex-text]{Courier New}

\usepackage{calc}
\usepackage[paper=a6paper,landscape]{geometry}

\pagestyle{empty}

\usepackage{tikz}
\usetikzlibrary{arrows,shadows} % for pgf-umlsd
\usepackage[underline=false,rounded corners=false]{pgf-umlsd}

\begin{document}

\begin{figure}
  \centering
  \begin{sequencediagram}
    \newthread{ss}{A}
    % \newinst{ps}{PhysicsServer}
    % \newinst[1]{sense}{SenseServer}
    \newthread[red]{ctr}{B}
    
    % \begin{sdloop}[green!20]{Run Loop}
      \mess{ctr}{StartCycle}{ss}
      % \begin{call}{ss}{Update()}{ps}{}
      %   \prelevel
      %   \begin{callself}{ctr}{SenseAgent()}{}
      %     \begin{call}[3]{ctr}{Read}{sense}{}
      %     \end{call}
      %   \end{callself}
      %   \prelevel\prelevel\prelevel\prelevel
      %   \setthreadbias{west}
      %   \begin{call}{ps}{PrePhysicsUpdate()}{sense}{}
      %   \end{call}
      %   \setthreadbias{center}
      %   \begin{callself}{ps}{Update()}{}
      %   \end{callself}
      %   \begin{call}{ps}{PostPhysicsUpdate()}{sense}{}
      %   \end{call}
      % \end{call}
      % \mess{ss}{EndCycle}{ctr}
      % \begin{callself}{ctr}{ActAgent()}{}
      %   \begin{call}{ctr}{Write}{sense}{}
      %   \end{call}
      % \end{callself}
    % \end{sdloop}

  \end{sequencediagram}
  \caption{Example of a sequence with parallel activities.}
\end{figure}


\end{document}
